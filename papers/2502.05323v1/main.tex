%%
%% This is file `sample-sigconf-authordraft.tex',
%% generated with the docstrip utility.
%%
%% The original source files were:
%%
%% samples.dtx  (with options: `all,proceedings,bibtex,authordraft')
%% 
%% IMPORTANT NOTICE:
%% 
%% For the copyright see the source file.
%% 
%% Any modified versions of this file must be renamed
%% with new filenames distinct from sample-sigconf-authordraft.tex.
%% 
%% For distribution of the original source see the terms
%% for copying and modification in the file samples.dtx.
%% 
%% This generated file may be distributed as long as the
%% original source files, as listed above, are part of the
%% same distribution. (The sources need not necessarily be
%% in the same archive or directory.)
%%
%%
%% Commands for TeXCount
%TC:macro \cite [option:text,text]
%TC:macro \citep [option:text,text]
%TC:macro \citet [option:text,text]
%TC:envir table 0 1
%TC:envir table* 0 1
%TC:envir tabular [ignore] word
%TC:envir displaymath 0 word
%TC:envir math 0 word
%TC:envir comment 0 0
%%
%%
%% The first command in your LaTeX source must be the \documentclass
%% command.
%%
%% For submission and review of your manuscript please change the
%% command to \documentclass[manuscript, screen, review]{acmart}.
%%
%% When submitting camera ready or to TAPS, please change the command
%% to \documentclass[sigconf]{acmart} or whichever template is required
%% for your publication.
%%
%%
\documentclass[sigconf]{acmart}
\usepackage{multirow}
% \usepackage[normalem]{ulem}
\usepackage{soul}


%%
%% \BibTeX command to typeset BibTeX logo in the docs
\AtBeginDocument{%
  \providecommand\BibTeX{{%
    Bib\TeX}}}

%%RR version CLEAN commands
\newcommand{\ladd}[1]{#1}
\newcommand{\lrem}[1]{}

%%RR version EDITS commands
% \newcommand{\ladd}[1]{\textcolor{blue}{#1}}
%\newcommand{\lrem}[1]{\textcolor{red}{\st{#1}}}

%% Rights management information.  This information is sent to you
%% when you complete the rights form.  These commands have SAMPLE
%% values in them; it is your responsibility as an author to replace
%% the commands and values with those provided to you when you
%% complete the rights form.


%%
%% Submission ID.
%% Use this when submitting an article to a sponsored event. You'll
%% receive a unique submission ID from the organizers
%% of the event, and this ID should be used as the parameter to this command.
%%\acmSubmissionID{123-A56-BU3}

%%
%% For managing citations, it is recommended to use bibliography
%% files in BibTeX format.
%%
%% You can then either use BibTeX with the ACM-Reference-Format style,
%% or BibLaTeX with the acmnumeric or acmauthoryear sytles, that include
%% support for advanced citation of software artefact from the
%% biblatex-software package, also separately available on CTAN.
%%
%% Look at the sample-*-biblatex.tex files for templates showcasing
%% the biblatex styles.
%%

%%
%% The majority of ACM publications use numbered citations and
%% references.  The command \citestyle{authoryear} switches to the
%% "author year" style.
%%
%% If you are preparing content for an event
%% sponsored by ACM SIGGRAPH, you must use the "author year" style of
%% citations and references.
%% Uncommenting
%% the next command will enable that style.
%%\citestyle{acmauthoryear}

\copyrightyear{2025}
\acmYear{2025}
\setcopyright{cc}
\setcctype{by}
\acmConference[CHI '25]{CHI Conference on Human Factors in Computing Systems}{April 26-May 1, 2025}{Yokohama, Japan}
\acmBooktitle{CHI Conference on Human Factors in Computing Systems (CHI '25), April 26-May 1, 2025, Yokohama, Japan}\acmDOI{10.1145/3706598.3713508}
\acmISBN{979-8-4007-1394-1/25/04}

%%
%% end of the preamble, start of the body of the document source.
\begin{document}

%%
%% The "title" command has an optional parameter,
%% allowing the author to define a "short title" to be used in page headers.
\title{Supporting Contraceptive Decision-Making in the Intermediated Pharmacy Setting in Kenya}

%%
%% The "author" command and its associated commands are used to define
%% the authors and their affiliations.
%% Of note is the shared affiliation of the first two authors, and the
%% "authornote" and "authornotemark" commands
%% used to denote shared contribution to the research.
\author{Lisa Orii}
\email{lisaorii@cs.washington.edu}
\orcid{0000-0002-6490-4776}
\affiliation{%
  % \department{Computer Science \& Engineering}
  \institution{University of Washington}
  \city{Seattle}
  \state{Washington}
  \country{USA}
}
\author{Elizabeth K Harrington}
\email{harri@uw.edu}
\orcid{0000-0002-8196-069X}
\affiliation{%
  % \department{Computer Science \& Engineering}
  \institution{University of Washington}
  \city{Seattle}
  \state{Washington}
  \country{USA}
}
\author{Serah Gitome}
\email{sgitome@kemri.go.ke}
\orcid{0009-0002-3371-4213}
\affiliation{%
  \institution{Kenya Medical Research Institute}
  \city{Nairobi}
  \country{Kenya}
}
\author{Nelson Kiprotich Cheruiyot}
\email{cheruiyotnelsonk@gmail.com}
\orcid{0009-0008-5436-3102}
\affiliation{%
  \institution{Independent design consultant}
  \city{Nairobi}
  \country{Kenya}
}
\author{Elizabeth Anne Bukusi}
\email{director@kemri.go.ke}
\orcid{0000-0002-2031-2808}
\affiliation{%
  \institution{Kenya Medical Research Institute}
  \city{Nairobi}
  \country{Kenya}
}
\author{Sandy Cheng}
\email{sandyc01@cs.washington.edu}
\orcid{0009-0004-4360-539X}
\affiliation{%
  \institution{University of Washington}
  \city{Seattle}
  \state{Washington}
  \country{USA}
}
\author{Ariel Fu}
\email{arielfu@cs.washington.edu}
\orcid{0009-0003-4480-537X}
\affiliation{%
  \institution{University of Washington}
  \city{Seattle}
  \state{Washington}
  \country{USA}
}
\author{Khushi Khandelwal}
\email{khushik@cs.washington.edu}
\orcid{0009-0008-1634-1332}
\affiliation{%
  \institution{University of Washington}
  \city{Seattle}
  \state{Washington}
  \country{USA}
}
\author{Shrimayee Narasimhan}
\email{shrima.narasimhan@gmail.com}
\orcid{0009-0009-6886-0799}
\affiliation{%
  \institution{University of Washington}
  \city{Seattle}
  \state{Washington}
  \country{USA}
}
\author{Richard Anderson}
\email{anderson@cs.washington.edu}
\orcid{0000-0002-7283-7219}
\affiliation{%
  \institution{University of Washington}
  \city{Seattle}
  \state{Washington}
  \country{USA}
}

%%
%% By default, the full list of authors will be used in the page
%% headers. Often, this list is too long, and will overlap
%% other information printed in the page headers. This command allows
%% the author to define a more concise list
%% of authors' names for this purpose.
\renewcommand{\shortauthors}{Orii et al.}

%%
%% The abstract is a short summary of the work to be presented in the
%% article.
\begin{abstract}
Adolescent girls and young women (AGYW) in sub-Saharan Africa face unique barriers to contraceptive access and lack AGYW-centered contraceptive decision-support resources. To empower AGYW to make informed choices and improve reproductive health outcomes, we developed a tablet-based application to provide contraceptive education and decision-making support in the pharmacy setting - a key source of contraceptive services for AGYW - in Kenya. We conducted workshops with AGYW and pharmacy providers in Kenya to gather app feedback and understand how to integrate the intervention into the pharmacy setting. Our analysis highlights how intermediated interactions - a multiuser, cooperative effort to enable technology use and information access - could inform a successful contraceptive intervention in Kenya. The potential strengths of intermediation in our setting inform implications for technological health interventions in intermediated scenarios in \lrem{LMICs}\ladd{low- and middle-income countries}, including challenges and opportunities for extending impact to different populations and integrating technology into resource-constrained healthcare settings.
\end{abstract}

%%
%% The code below is generated by the tool at http://dl.acm.org/ccs.cfm.
%% Please copy and paste the code instead of the example below.
%%
\begin{CCSXML}
<ccs2012>
   <concept>
       <concept_id>10003120.10003121.10011748</concept_id>
       <concept_desc>Human-centered computing~Empirical studies in HCI</concept_desc>
       <concept_significance>500</concept_significance>
       </concept>
   <concept>
       <concept_id>10003456.10010927.10010930.10010933</concept_id>
       <concept_desc>Social and professional topics~Adolescents</concept_desc>
       <concept_significance>500</concept_significance>
       </concept>
 </ccs2012>
\end{CCSXML}

\ccsdesc[500]{Human-centered computing~Empirical studies in HCI}
\ccsdesc[500]{Social and professional topics~Adolescents}

%%
%% Keywords. The author(s) should pick words that accurately describe
%% the work being presented. Separate the keywords with commas.
\keywords{HCI4D, intermediation, health, contraception}

%%
%% This command processes the author and affiliation and title
%% information and builds the first part of the formatted document.
\maketitle

\section{Introduction}\label{sec:Intro} 


Novel view synthesis offers a fundamental approach to visualizing complex scenes by generating new perspectives from existing imagery. 
This has many potential applications, including virtual reality, movie production and architectural visualization \cite{Tewari2022NeuRendSTAR}. 
An emerging alternative to the common RGB sensors are event cameras, which are  
 bio-inspired visual sensors recording events, i.e.~asynchronous per-pixel signals of changes in brightness or color intensity. 

Event streams have very high temporal resolution and are inherently sparse, as they only happen when changes in the scene are observed. 
Due to their working principle, event cameras bring several advantages, especially in challenging cases: they excel at handling high-speed motions 
and have a substantially higher dynamic range of the supported signal measurements than conventional RGB cameras. 
Moreover, they have lower power consumption and require varied storage volumes for captured data that are often smaller than those required for synchronous RGB cameras \cite{Millerdurai_3DV2024, Gallego2022}. 

The ability to handle high-speed motions is crucial in static scenes as well,  particularly with handheld moving cameras, as it helps avoid the common problem of motion blur. It is, therefore, not surprising that event-based novel view synthesis has gained attention, although color values are not directly observed.
Notably, because of the substantial difference between the formats, RGB- and event-based approaches require fundamentally different design choices. %

The first solutions to event-based novel view synthesis introduced in the literature demonstrate promising results \cite{eventnerf, enerf} and outperform non-event-based alternatives for novel view synthesis in many challenging scenarios. 
Among them, EventNeRF \cite{eventnerf} enables novel-view synthesis in the RGB space by assuming events associated with three color channels as inputs. 
Due to its NeRF-based architecture \cite{nerf}, it can handle single objects with complete observations from roughly equal distances to the camera. 
It furthermore has limitations in training and rendering speed: 
the MLP used to represent the scene requires long training time and can only handle very limited scene extents or otherwise rendering quality will deteriorate. 
Hence, the quality of synthesized novel views will degrade for larger scenes. %

We present Event-3DGS (E-3DGS), i.e.,~a new method for novel-view synthesis from event streams using 3D Gaussians~\cite{3dgs} 
demonstrating fast reconstruction and rendering as well as handling of unbounded scenes. 
The technical contributions of this paper are as follows: 
\begin{itemize}
\item With E-3DGS, we introduce the first approach for novel view synthesis from a color event camera that combines 3D Gaussians with event-based supervision. 
\item We present frustum-based initialization, adaptive event windows, isotropic 3D Gaussian regularization and 3D camera pose refinement, and demonstrate that high-quality results can be obtained. %

\item Finally, we introduce new synthetic and real event datasets for large scenes to the community to study novel view synthesis in this new problem setting. 
\end{itemize}
Our experiments demonstrate systematically superior results compared to EventNeRF \cite{eventnerf} and other baselines. 
The source code and dataset of E-3DGS are released\footnote{\url{https://4dqv.mpi-inf.mpg.de/E3DGS/}}. 





%!TEX root = 2024_auv_mola_drl6dof_main.tex
%%%%%%%%%%%%%%%%%%%%%%%%%%%%%%%%%%%%%%%%%%%%%%%%%%%%%%%%%%%%%%%%%%%%%%%
\section{Background}
\label{sec:background}

%%%%%%%%%%%%%%%%%%%%%%%%%%%%%%%%%%%%%%%%%%%%%%%%%%
%%% Reinforcement learning
\subsection{Deep Reinforcement learning}

\ac{drl} \cite{RichardSutton20} is a method that aims to train an agent's policy $\pi$ to map states into actions by interacting with the environment. This is achieved by maximizing a numerical reward signal and using a \ac{mdp} framework to regulate the interaction between the \ac{rl} agent’s policy and the environment. At each time step, the agent observes a state $\bm{s}$, takes an action $\bm{a}$, and upon transitioning to the next state, receives a reward $r$. Once the episode (i.e., process) is complete, the accumulated reward is calculated as the sum of all time steps rewards in that episode.

\ac{drl} methods can be model-based or model-free. Model-based methods use a model to predict the next state and reward, while model-free methods learn solely from experiencing the unmodeled and unknown consequences of an action. While learning from trial and error may result in less efficient learning, model-free methods have the advantage when a model is unavailable or inaccurate.

\begin{figure}[t!]
\centering
\includegraphics[width=0.45\textwidth]{figures/reward_vs_step.pdf}%
\caption{Average reward per episode over a moving window of 100 episodes obtained by the TQC, SAC, and TD3 algorithms during a $2.5\times10^6$ step training, equivalent to 3125 episodes.}
\label{fig:rewards}
\end{figure}

%%%%%%%%%%%%%%%%%%%%%%%%%%%%%%%%%%%%%%%%%%%%%%%%%%
%% 6DOF Error Computation
\subsection{\ac{6dof} Error Computation}

The position errors are determined by the difference between the current position $(x, y, z)$ and the goal position $(x_d, y_d, z_d)$ following the North-East-Down (NED) convention, computed as
%%%
% Keep to remove space between equations and paragraph
%%%
\begin{equation}
    e_x(t) = x^t - x_d^t,\; e_y(t) = y^t - y_d^t,\; e_z(t) = z^t - z_d^t.
\label{eq:errors}
\end{equation}

To compute the error in attitude, we will evaluate the difference between the current orientation and the goal attitude, both with respect to the fixed world frame. This involves representing both poses as rotation matrices ($\bm{R}\in SO(3)$) and converting their difference to exponential coordinates $[\bm{{e_\theta}}]\in so(3)$ through the matrix logarithm:
%%%
% Keep to remove space between equations and paragraph
%%%
\begin{equation}
     [\bm{{e_\theta}}(t)] = \log(\bm{R}(t)^T \cdot \bm{R}_d)
\end{equation}

Then, the skew-symmetric matrix $[\bm{{e_\theta}}(t)]$ is converted into its vector representation $\bm{{e_\theta}}(t) \in \mathbb{R}^3$, where its entries correspond to the element-wise error for the attitude, defined as
%%%
% Keep to remove space between equations and paragraph
%%%
\begin{equation}
    \begin{bmatrix} \theta_{x}^t & \theta_{y}^t & \theta_{z}^t \end{bmatrix} = \bm{{e_\theta}}(t).
    \label{eq:attitude_error}
\end{equation}

Furthermore, to provide a single metric for attitude error evaluation, we compute $\theta^t$ based on the axis-angle representation for $\bm{{e_\theta}}(t)$, as described in \eqref{eq:theta_error}. By using this metric, we obtain a global evaluation of orientation, which aligns the controller's performance with practical manual navigation comparisons.
%%%
% Keep to remove space between equations and paragraph
%%%
\begin{equation}
    \theta^t = ||\bm{{e_\theta}}(t)||
    \label{eq:theta_error}
\end{equation}

\section{Related Work}

\subsection{Intermediation in LMICs}


In LMICs, where access to technology or digital literacy is limited, intermediated interactions emerge as a common practice to enable technology access and use for a vast number of people. According to Sambasivan et al. ~\cite{10.1145/1753326.1753718}, intermediation is accomplished when a digitally skilled user enables information seeking and use for a ``beneficiary-user” for whom technology is inaccessible due to non-literacy, lack of technology-operation skills, or financial constraints. Parikh and Ghosh ~\cite{1626204} articulate an understanding of intermediated tasks that identifies how technology availability, existing power relationships between users, and technical literacy can impact the ways in which interactions manifest. Prior work emphasizes the critical role of human relations or ``human infrastructure” in intermediated interactions and their potential for being ``more robust and pervasive than technology networks” ~\cite{10.1145/2369220.2369258} because they can overcome access and use constraints and can influence information penetration ~\cite{10.1145/2909609.2909664, 10.1145/1753326.1753610}. The human agent facilitating intermediation affords the flexibility to accommodate unexpected needs and behaviors ~\cite{10.1145/2737856.2738023}. However, human factors of intermediaries, such as their background and motivations, can influence intermediation ~\cite{10.1145/3449118, 10.1145/3313831.3376465, 10.1145/3411764.3445410}.

Intermediaries can support new tasks in addition to continuing their primary tasks. In Tanzania, mobile money agents, whose main purpose was to assist community members with mobile money services\lrem{, also supported} \ladd{began to support} feature phone users in using a \lrem{new }newly introduced mobile app ~\cite{10.1145/3613904.3642099}. In India, local mobile shops assisted in the dissemination of health education videos in addition to continuing their usual business ~\cite{10.1145/2909609.2909655}. Regardless of the tasks they support, intermediaries often contribute more than their technical literacy - they also facilitate information flow. In Digital Green ~\cite{4937388} and Projecting Health~\cite{10.1145/2737856.2738023} in India, mediators facilitated video-based education on domain-specific topics and discussions that furthered the audience’s understanding of the video content. In Bangladesh, small businesses, such as pharmacies and grocery shops, bridged the information gap between extremely impoverished people and low-cost healthcare services, connecting clients to affordable healthcare ~\cite{10.1145/3449118}. In these examples and beyond, intermediaries have been effective not only in supporting technology access and use, but also in disseminating information and encouraging people to engage with information ~\cite{10.1145/2369220.2369253}. 


\subsection{Interventions for Person-Centered Contraceptive Education and Decision-Making Support} 

There is a growing body of work in contraceptive education and decision-support that focuses on improving preference-sensitive decision-making, meaning that method choice can depend on a variety of factors such as perception of method safety, effectiveness, and side effects ~\cite{madden2015role, marshall2016young}. Research on preference-sensitive shared decision-making with a provider during contraceptive counseling demonstrates that women and service providers differ in their preferences for and prioritization of contraceptive characteristics ~\cite{weisberg2013women}. Such work has informed the development of person-centered contraceptive decision-support tools such as My Birth Control, a web-based tool in the U.S. used before counseling to improve women’s experiences of contraceptive counseling and support them in choosing contraceptive methods that fit their needs and preferences ~\cite{dehlendorf2019cluster}. The app includes educational models and surveys to indicate preferences for method characteristics. My Birth Control increased women’s knowledge of methods, reduced decision conflict, encouraged providers to acknowledge clients’ preferences and values, and clients appeared to be more confident in their method preferences ~\cite{holt2020patient, dehlendorf2019mixed}. Our app takes inspiration from My Birth Control’s person-centered focus. 

\ladd{Person-centeredness is also evident in HCI work that explores opportunities for technology to support information seeking of sexual and reproductive health in light of the challenges and needs faced by specific populations. Dewan et al. ~\cite{10.1145/3613904.3641934} offer guidance on technology development for teenagers seeking information on reproductive health given teenagers’ information seeking practices and struggles with finding appropriate information sources. Patel et al. ~\cite{10.1145/3679318.3685380} propose technology designs that facilitate safe and appropriate sexual and reproductive care for LGBTQ+ people with uteruses, considering their experiences with in-person care.}

\lrem{However, person-centered contraceptive decision-support tools}\ladd{However, aforementioned work is situated in high-income countries. Technologies that support person-centered contraceptive decision-support} have yet to be a focus for contraceptive interventions in LMICs. There are, however, interventions such as text messaging, phone calls, or voice messaging that are designed to facilitate contraceptive education and behavior change. While these include interventions tailored to a specific population, they are intended to engage users in health-related communications, increase contraceptive knowledge, or promote contraceptive uptake and continuation, which differs from supporting person-centered decision-making ~\cite{harrington2019mhealth, mccarthy2018development, 10.1145/2702123.2702124}. A plausible explanation for the lack of person-centered contraceptive decision-making tools in LMICs may be the paucity of data on contraceptive preferences and priorities of women in these countries. \lrem{In sub-Saharan Africa, where our work takes place, t}\ladd{T}here is even less data on what adolescents consider important to their contraceptive decisions, though there are studies that demonstrate adolescents’ concerns about future fertility and side effects ~\cite{velonjara2018motherhood, ochako2015barriers, gueye2015belief}. \ladd{While there are examples of HCI-driven interventions aimed to provide adolescence related sexual and reproductive health guidance in LMICs ~\cite{10.1145/3411764.3445694, wang2022artificial}, these interventions assume that adolescents have access to personal mobile devices that can be used privately, which is not common for adolescents in many LMICs ~\cite{madonsela2023development}.}\lrem{ While there are interventions focused on contraceptive information provision}\lrem{\mbox{ ~\cite{feroz2021using}}}{\lrem{, a} \ladd{Moreover, a}pproaches to support decision-making in community-based settings, such as pharmacies where many adolescents access contraception ~\cite{gonsalves2023pharmacies, gonsalves2020mixed, radovich2018meets}, are limited ~\cite{dev2019acceptability}. Our work \ladd{aims to improve the quality and person-centeredness of contraceptive care available to AGYW by tailoring our intervention to AGYW’s needs, values, and preferences.}\lrem{ aims to reach AGYW in a setting where they access contraceptive services to improve contraceptive care.}
\section{Methods}

This study builds on formative work aimed to provide decision-making support for AGYW seeking contraception in pharmacies in Kenya. Our team previously explored factors that influence contraceptive preferences and decision-making for AGYW in Kisumu, Kenya, assessed the relative importance of contraceptive method and service delivery characteristics, and examined the quality of contraceptive care in pharmacies in western Kenya. Rigorous qualitative and qualitative research informed our intervention design and provided an understanding of contraceptive care for AGYW in Kenya seeking contraceptive services in the pharmacy setting.

We designed and developed \lrem{an app}\ladd{the Mara Divas app} to be placed in the pharmacy setting, a key access point for contraceptive services for Kenyan AGYW, to provide contraceptive education and decision-support tailored to AGYW's needs and preferences. We conducted co-design workshops with AGYW and pharmacy staff in Kisumu county, Kenya to understand how to integrate the app into the pharmacy setting and to gather app feedback. Feedback from the workshops inspired revisions to the app, which were made in between workshops and then after. \ladd{This research was approved by the Kenya Medical Research Institute Scientific Ethics Review Unit and the University of Washington Human Subjects Division. We obtained a research permit from the Kenya National Commission for Science, Technology, and Innovation.}\lrem{This study was approved by the IRB in Kenya and at our university after undergoing rigorous review.} 

\subsection{Setting}
\ladd{This work takes place in Kisumu county which is located in western Kenya. Its largest population center is the city of Kisumu, which is the third largest city in Kenya. In 2022 in Kisumu county, 9.2\% of AGYW aged 15-19 years had ever experienced a live birth and 11.1\% had ever experienced pregnancy ~\cite{DHS2023}. Luo is the primary ethnic group of Kisumu county where DhoLuo, Kiswahili (the national language), and English are spoken. Public health clinics offer most contraceptive methods for free or for a small fee. Pharmacies usually offer condoms, emergency contraceptives, depot medroxyprogesterone acetate (injectables, also known as ``depo”), and combined oral contraceptive pills. Some pharmacies also place contraceptive implants on site, but most refer patients to nearby clinics for implants or intrauterine devices (IUD/IUCD).}

\subsection{App Design and Development}
Considering that AGYW have low levels of knowledge regarding side effects of contraception and hold strong beliefs about contraceptive harms ~\cite{harrington2021spoiled, ochako2015barriers, sedlander2018they}, we did not develop an app that only recommended the ``right” contraceptive method. We were motivated to provide information on various contraceptive options regarding topics that are important to AGYW, including bleeding patterns, privacy of contraceptive use, and influence on future fertility ~\cite{harrington2023adaptation, harrington2021spoiled} and demystify side effects. We were inspired by the person-centeredness of My Birth Control, a U.S.-based decision support tool that assists women with contraceptive method selection ~\cite{dehlendorf2019cluster, holt2020patient, dehlendorf2019mixed}. \ladd{We named our app the Mara Divas app, in which ``Mara” means ``my own” in DhoLuo, the local language of Kisumu county, Kenya.}

The \ladd{Mara Divas} app is \lrem{intended}\ladd{designed} to be used in a private counseling room in the pharmacy setting. \ladd{The room will provide a tablet with the app installed and a pair of headphones. The intended users of the app are AGYW seeking contraceptive services in the pharmacy setting. As such, when we use the term ``user” henceforth, we refer to the AGYW using the app.} When the AGYW approaches the pharmacist, the pharmacist would take them to a private counseling room to use the app for 5-20 minutes. Lastly, the pharmacist would communicate directly with the AGYW about their choice and provide either the desired method or a referral. 


\begin{figure*}[hbt!]
  \centering
\includegraphics[width=1.0\linewidth]{figures/app.png}
\caption{Mara Divas app configuration. Enlarged images of each screen are in Appendix \ref{big_images}. (a) Main screen with questions and topics that are of most interest and importance to AGYW. (b) Short explanations of each contraceptive method and where it is administered on a female body. (c) Information presented with text, audio, and videos (provider and peer). (d) Survey where users can indicate preferences for method characteristics. (e) Recommendations for contraceptive methods based on survey results.}
\Description{Five app screens, described from left to right. (a) Main screen with nine questions and topics that are of most interest and importance to AGYW. Top of the screen has three language toggle buttons labeled ``Kiswahili,” ``Dholuo,” and ``English” where ``English” is grayed out to indicate its selection. These buttons are presented in all of the screens explained after. Top of the screen also includes a ``Your Favorites” button with a thumbs-up sign preceding the text, which is also in screens (b) and (e). The screen is filled with buttons, labeled in order from top to button, ``What are my options?” ``What will happen to my period?” ``How long does the method work?” `1`What is my chance of getting pregnant?” ``Can I keep it private?” ``What if I’m ready to have a baby?” and ``Take our quiz and find your method!” Each button is accompanied by a representative image of that topic and an audio button. Two buttons labeled ``3 things you need to know about the E-pill (P\ladd{-}2)” and ``Preventing HIV and STIs” is also exhibited. (b) Top of the page is titled ``What are my options?” A female body in the center of the screen with contraceptive method icons around it. On the left side of the body are icons and accompanying labels of pills (daily pills), emergency pill (e-pill, P\ladd{-}2), IUCD (coil), and on the right side are injection (depo), implant, condom, and female condom. Below each label is a button ``Favorite it!” with a thumbs-up sign preceding the text. The injection (depo) is selected, indicated by a different color. A pop-up at the bottom of the screen shows a description of the injection (depo) with an audio button and a real image of the injection. (c) Top of the page is titled ``What will happen to my period?” Below the title are seven contraceptive methods with icons and labels, as described in screen (b). Implant is selected, indicated by a different color. Middle of the screen displays a text description of the method and an audio button. Below this is two videos - one of a provider, one of an AGYW peer - placed side by side. Below the videos is a button labeled ``WHY?” with a magnifying glass icon preceding the text (d) Top of the page is titled ``Quiz.” Multiple choice quiz/survey questions displayed are ``How do you feel about changes to your periods?” ``If you had to guess, when do you think you might want a pregnancy?” ``How long do you want your method to work?” and ``How important is it to you to keep your method private from your parents or partner?” Selections for each question are indicated by a different color. A scroll bar indicates continuation of the quiz. (e) Top of the page is titled ``Recommendations” and below it, ``Here are some recommendations that might be right for you!” Recommendations displayed are titled ``Based on how you feel about light, irregular periods, these methods might be a good choice for you:” and ``Based on how you feel about the possibility of your periods stopping, these methods might be a good choice for you:” with icons and labels of the depo and implant. Below each method label is a button labeled ``Learn More” and ``Favorite it!” with a thumbs-up sign preceding the text. Below the buttons are short relevant explanations for other recommendations. The bottom of the screen has a button labeled ``END SESSION.” A scroll bar indicates continuation of the quiz.}
\label{fig:app}
\end{figure*}

We built an Android app using the Flutter \ladd{3.19.5} platform\ladd{~\cite{flutter} that supports Android 5.0 and all subsequent versions. Our app targets} \lrem{targeting }Samsung Galaxy \ladd{Tab A8} tablets \ladd{with a 10.5-inch display}. We chose tablets because the pharmacy setting provides a large enough space for private table usage and the contents on a tablet would be easier to interact with than on a phone. We chose Android devices because of their low cost and AGYW’s familiarity with Android. Flutter was selected as an app development framework because it is a mature mobile app platform and provides a professional app appearance. 



%Fig 1: App configuration. (a) Main screen with questions and topics that are of most interest and importance to AGYW. (b) Screen presenting short explanations of each contraceptive method and where they are placed on a female body. (c) Screen with text, audio, and videos (provider and peer). (d) Survey to indicate preferences for method characteristics. (e) Recommendations for contraceptive methods based on survey results.

The content of the app was carefully crafted by a clinical reproductive health expert who has extensive research experience in Kenya. The app’s main screen presents sections with topics that are of most interest to AGYW (Figure \ref{fig:app} a), as learned through formative work. Each section leads the user to a screen with topic-specific information for seven contraceptive methods (Figure \ref{fig:app} b, c): male condom, female condom, daily contraceptive pills, implant, injection (depo), IUCD, and emergency contraceptive (e-pill). For each method, we also made a screen that briefly summarized information about that method. Users can also ``favorite” methods that they like and want to return to for review. 

Once all sections are visited, users complete a survey to indicate preferences for method characteristics (Figure \ref{fig:app} d). The survey is composed of five multiple choice questions. Although five questions are limited in understanding the users’ contraceptive preferences, we restricted the number of questions because AGYW dislike lengthy questioning ~\cite{gonsalves2020pharmacists}. Based on survey responses, the app recommends contraceptive methods by the user’s method characteristic preference (Figure \ref{fig:app} e). %For example, if the user indicated that it was extremely important to keep their method private, the recommendation would state, ``Based on how important it is to you to keep your method private, the following methods might be a good choice for you:” with methods that align with the user’s preference around privacy. 

We prepared the \ladd{Mara Divas} app in DhoLuo, Kiswahili (the national language), and English, which are languages spoken in Kisumu county, Kenya, where the workshops took place and the app is planned to be piloted. A language toggle tool is available on all screens so that users can engage with content in their preferred language. To reach AGYW with limited language literacy, we included audio and brief video components (Figure \ref{fig:app} c). Text is accompanied by a spoken version of the text which was recorded by study staff, indicated by an audio button next to the text. We included two types of videos: videos where providers share their expertise on contraceptive topics, and videos where AGYW share their experiences with using contraceptives. These videos were narrated by service providers and AGYW actors in Kisumu. We incorporated four provider videos and three peer videos. Videos were between one to three minutes long.

\subsection{Co-design Workshops}
We conducted co-design workshops to understand how to integrate the intervention into the pharmacy setting and to gather potential app \lrem{users}\ladd{stakeholders, namely AGYW and pharmacy staff,} with varied preferences to share feedback on the app and ideate new designs or content. 

\subsubsection{Study Design}
\lrem{The workshops took place in Kisumu county, Kenya. Luo is the primary ethnic group of Kisumu county where DhoLuo, Kiswahili, and English are spoken. Public health clinics offer most contraceptive methods for free or for a small fee. Pharmacies usually offer condoms, emergency contraceptives, depot medroxyprogesterone acetate (injectables, also known as ``depo”), and combined oral contraceptive pills. Some pharmacies also place contraceptive implants on site, but most refer patients to nearby clinics for implants or intrauterine devices (IUD/IUCD).} 

We conducted four workshops - three with AGYW aged 15-24 years with pharmacy-based contraceptive care experience and one with pharmacy staff - to allow for iteration in app prototype development between workshops. Workshops were conducted between April - May 2024. Workshops were led by four facilitators, all who spoke English and Kiswahili and two who also spoke DhoLuo. Facilitators were also involved in the creation and revision of the study design. Workshop activities were mainly led by two facilitators with human-centered design experience (third and fourth authors). The first author was also present for the first three workshops. 

Each workshop began with a distribution of informed consent forms, followed by an explanation of consent guidelines by study staff and a pre-workshop questionnaire administered on a tablet by study staff. After the pre-workshop questionnaire, participants re-grouped to review the background and objectives of the study, and confidentiality and consent guidelines. We then proceeded to conduct activities where participants engaged in discussion and produced artifacts (e.g., post-it notes) and facilitators took field notes. Workshops lasted between 9AM and 5PM, including \lrem{a }\ladd{one 90-minute} lunch break and \ladd{two 30-minute} tea breaks \ladd{in between workshop activities}. Numerous small revisions to the app were made in response to feedback in between workshops and after completion of all workshops. 

\subsubsection{Participant Recruitment}
AGYW participants were recruited using community-based strategies to optimize the diversity of perspectives and experiences in the three AGYW workshops by age, education, contraceptive experience and experience with mobile apps. Purposive sampling ~\cite{tongco2007purposive} was used to recruit AGYW from peer mobilizers in the community and in-person in the pharmacy settings. Snowball sampling ~\cite{parker2019snowball} was also used, by which study staff asked interested participants to recommend peers for inclusion. Female study staff who spoke DhoLuo, Kiswahili, and English provided an introduction to the study and potential participants who met the inclusion criteria were assessed for their availability for workshop dates. AGYW under 18 years of age and not emancipated minors were encouraged to speak to a trusted adult about participating in the study. Eligible AGYW met the following criteria: 1) cisgender female; 2) age 15-24 years; 3) has accessed pharmacy-based contraception in the last three months; 4) not currently pregnant; 5) speaks DhoLuo, Kiswahili, and/or English; 6) feels comfortable reading DhoLuo, Kiswahili, and/or English; and 7) willing to sign informed consent. \ladd{This work targets AGYW aged 15-24 years because we determined that this age group is representative of a population that will most likely benefit from our app. We also note that most AGYW under the age of 15 are not sexually active, as only 8\% of Kenyan AGYW aged 15-24 years experienced first sexual intercourse before the age of 15 ~\cite{kenyadhs}. We also wanted to minimize the risks to a vulnerable population by working with slightly older AGYW.}

Pharmacy staff participants were recruited from pharmacies in Kisumu and periurban areas to balance the gender and age of pharmacy staff in the pharmacy staff workshop. Eligible pharmacists met the following criteria: 1) at least 18 years of age; 2) currently working in a pharmacy providing condoms, emergency contraception, combined oral contraceptive pills, and depot medroxyprogesterone acetate (hormonal contraceptive injection); 3) currently working in a pharmacy that has a counseling room/private space; 4) speaks Kiswahili and/or English; 5) feels comfortable reading DhoLuo, Kiswahili, and/or English; and 6) willing to sign informed consent. 

We recruited 33 AGYW and 10 pharmacy staff (6 female). \ladd{We conducted four workshops:}\lrem{We had} one workshop of 12 AGYW aged 15-17 years\footnote{\lrem{There was one 19-year-old in the workshop with a younger age group but}\ladd{All except one participant in the younger age group workshop was aged 15-17 years and so} we present the \lrem{two }\ladd{AGYW} workshops as having \ladd{two} distinct age groups.}\ladd{,} \lrem{and} two workshops with 10 and 11 AGYW aged 18-24 years\ladd{, and one workshop of 10 pharmacy staff}. The sample size was determined through the review of relevant literature in the health domain using similar methods ~\cite{hunter2021designing, wilkinson2022developing, harrington2021spoiled} and guidelines in qualitative research ~\cite{sandelowski1995sample}. AGYW ranged in age from 15-24 years \lrem{(mean=19.24, SD=2.75)}\ladd{and the median age was 19 (IQR 17-21).}\lrem{ and p} \ladd{P}harmacy staff ranged in age from 32-48 years \lrem{(mean=37.10, SD=6.87)}\ladd{and the median age was 33.5 (IQR 32-40)}. \ladd{A summary of participant characteristics for the three AGYW workshops and one pharmacy staff workshop are in Tables \ref{tab:AGYW-participants} and \ref{tab:pharm-participants}, respectively.} AGYW participants and \lrem{pharmacist}\ladd{pharmacy staff} participants were compensated with 1000 KSH and 2000 KSH (approximately 8 USD and 15 USD), respectively, for their participation. \ladd{The compensation was determined based on the judgment of the Kenyan ethics review board of what was appropriate and not coercive.}

\begin{table}[h!]
\caption{Participant characteristics of AGYW workshops.}
\Description{Table with characteristics and workshop ID for 3 AGYW workshops. Characteristics are in one column with median age (IQR), education level, and mobile phone use. Corresponding values are in the other columns where each column represents one workshop.}
\label{tab:AGYW-participants}
% \normalsize
\resizebox{\columnwidth}{!} {
\begin{tabular}{|p{4.4cm}||p{1.6cm}|p{1.6cm}|p{1.6cm}|}
\hline
\multicolumn{1}{|c||}{} &
\multicolumn{3}{c|}{\textbf{Workshop ID}} \\
\multicolumn{1}{|c||}{\textbf{Characteristic}} &
\multicolumn{1}{c}{\textbf{AGYW-1}} &
\multicolumn{1}{c}{\textbf{AGYW-2}} &
\multicolumn{1}{c|}{\textbf{AGYW-3}} \\
\hline
Total Participants, n & 10 & 12 & 11 \\
\hline
Median Age (IQR) & 20.5 (19.25-21.75) & 17 (15.75-17) & 21 (19-23) \\
\hline
Education Level, n &  &  & \\ 
\hspace{3mm}Primary school, not complete & 0 & 1 & 0 \\ 
\hspace{3mm}Primary school, complete & 1 & 3 & 1 \\
\hspace{3mm}Secondary school, not complete & 1 & 7 & 1\\
\hspace{3mm}Secondary school, complete & 3 & 1 & 7\\
\hspace{3mm}Post-secondary school & 5 & 0 & 2 \\ 
\hline
Mobile Phone Use, n &  &  & \\ 
\hspace{3mm}Has access to a mobile phone & 10 & 5 & 11 \\ 
\hspace{3mm}Owns a personal mobile phone & 10 & 4 & 11 \\
\hspace{3mm}Has access to a smartphone & 9 & 2 & 6 \\
\hline
\end{tabular}
}
\end{table}

\begin{table}[h!]
\caption{Participant characteristics of pharmacy staff workshop.}
\Description{Table with characteristics and workshop ID for pharmacy staff workshop. Characteristics are in one column with median age (IQR), education level, and years of professional experience. Corresponding values are in the other columns where each column represents one workshop.}
\label{tab:pharm-participants}
% \normalsize
\resizebox{\columnwidth}{!} {
\begin{tabular}{|p{7cm}||p{4cm}|}
\hline
\multicolumn{1}{|c||}{} &
\multicolumn{1}{c|}{\textbf{Workshop ID}} \\
\multicolumn{1}{|c||}{\textbf{Characteristic}} & \multicolumn{1}{c|}{\textbf{Pharmacy-1}} \\
\hline
Total Participants, n& 10 (6 female) \\
\hline
Median Age (IQR) & 33.5 (32-40) \\
\hline
%Highest Education Level &  \\ 
%\hspace{3mm}Bachelor's Degree & 1/10 \\ 
%\hspace{3mm}Diploma & 9/10 \\
Licensure, n & \\
\hspace{3mm} Pharmacy owner & 1\\
\hspace{3mm} Pharmacy technician & 8\\
\hspace{3mm} Other & 1\\
\hline
Years of Working in the Pharmacy, n &   \\ 
\hspace{3mm}5 or more years & 9 \\ 
\hspace{3mm}1-2 years & 1 \\
\hline
Median Score for Confidence in Advising AGYW (IQR) &   \\ 
\hspace{3mm}On family planning options & 5 (4-5) \\ 
\hspace{3mm}On how to use family planning methods & 5 (4-5) \\
\hspace{3mm}On family planning side effects & 5 (4.25-5) \\
\hline
\end{tabular}
}
\end{table}


\subsubsection{Data Collection}
Data was collected through pre-workshop questionnaires, artifacts (e.g., post-it notes), \ladd{written} field notes \ladd{in English, and audio-recordings. The primary data sources were artifacts and written notes that were produced for each workshop and all its activities, as the workshops were centered around small or large group activities that involved creating visual representations of ideas and concepts.}\lrem{, and audio-recordings of} \ladd{Large group interviews were audio-recorded as a secondary source of data to retain individual participant quotations and perspectives.} \lrem{ facilitated app engagement, app feedback, and group interviews.}\ladd{When participants were engaged in small-group activities or speaking one-on-one with a facilitator, it was impractical to rely on audio-recordings, so written note-taking was used.} Activities were conducted in the order they are presented in this section. \ladd{The flow of activities is described in Figure \ref{fig:flow}.}



\begin{figure*}[hbt!]
  \centering
\includegraphics[width=1.0\linewidth]{figures/flow.png}
\caption{Flow of workshop activities.}
\Description{Flow of workshop activities from left to right: pre-workshop questionnaire (1 hour), journey mapping (1-2 hours), free app engagement (20 mins), usability testing (1 hour), facilitated app engagement (1 hour), app feedback (1 hour), group interview (1-2 hours).}
\label{fig:flow}
\end{figure*}


\paragraph{Pre-workshop questionnaire}
Each participant individually completed a pre-workshop questionnaire with study staff. For AGYW, we collected 1) demographic information; 2) mobile phone and app use; 3) relationships and sexual history; and 4) reproductive history and contraceptive experience and desires. In this paper, we present data on aspects of AGYW’s demographic \lrem{information}\ladd{characteristics including age and education levels}, mobile phone \lrem{and app use}\ladd{use}, and contraceptive experiences. Other data will be presented in a future paper. For pharmacy staff, we collected 1) demographic information; and 2) confidence in advising AGYW on contraception. \ladd{For confidence in advising AGYW on contraception, we used a 5-point Likert scale, with 1 being not at all confident and 5 being extremely confident.} The pre-workshop questionnaire is included in Appendix \ref{question}. 

\paragraph{Journey mapping}
Facilitators led participants to create a journey map to visualize the entire flow of a pharmacy visit experience. We focused on ``pain points” - moments that create discomfort, embarrassment, or inconvenience - and moments where there is the potential for delight. Journey mapping lasted between one to two hours. 

For the AGYW workshop\ladd{s}, we applied different methods of journey mapping. In some workshops, participants used post-it notes to write their actions, thoughts, and goals that they have before, during, and after pharmacy visits (Figure \ref{fig:workshop} a). In other workshops, participants conducted an interactive role-play depicting a pharmacy visit, in which one participant acted as the pharmacist, another acted as an AGYW, and observing participants commented on their experiences during the role play (Figure \ref{fig:workshop} b). The purpose was to understand counseling from AGYW’s perspective.

In the pharmacy staff workshop, one participant acted as a pharmacist and one member of the study team acted as an AGYW to role-play a contraceptive counseling scenario (Figure \ref{fig:workshop} c). The purpose was to understand how a pharmacy staff would conduct counseling. After role-play, participants discussed how their experiences differed or aligned with what was presented.

\paragraph{Free app engagement}
Facilitators introduced the \ladd{Mara Divas} app to the participants, explaining its objective and the workshop’s goal to gather app feedback. Each participant was given a tablet and headphones. Participants were given unstructured time to engage with the app to gain familiarity. Workshop facilitators answered questions and documented real-time observations and feedback. Free app engagement \ladd{was conducted in both AGYW workshops and the pharmacy staff workshop. This activity} lasted approximately 20 minutes.

\paragraph{Usability testing}
In AGYW workshops, participants individually conducted a series of tasks on the app. The purpose was to test how easy the app was used by AGYW to inform changes for future design iterations. Each participant was given three tasks to complete (e.g., watch a video in DhoLuo of a peer talking about how to keep family planning methods private) and a facilitator observed and took notes as the participant attempted the tasks. We prepared 12 tasks total. \ladd{Usability testing was only conducted in the AGYW workshop}\lrem{We did not conduct usability testing in the pharmacy staff workshop} because pharmacy staff were not target users \ladd{of the app}. Usability testing lasted approximately one hour.

\paragraph{Facilitated app engagement}
As a group, participants discussed specific aspects or sections of the app. Participants were asked to discuss whether and how their opinions on contraception changed after engaging with the app, whether the content was helpful for addressing AGYW concerns, and how the app could influence their decision-making. In the pharmacy staff workshop, participants \ladd{also} discussed whether and how the app’s content differed from their training or experience. All participants also discussed the language toggle tool, audio, and video and compared their experiences and preferences with reading, listening, and watching. Discussions were audio-recorded. \ladd{Facilitated app engagement was conducted in both AGYW workshops and the pharmacy staff workshop.} This activity lasted approximately one hour.

\paragraph{App Feedback}
In a group, facilitators elicited structured positive feedback, criticisms, questions, and ideas regarding the app. We used the “I Like, I Wish, What If” method ~\cite{Dam_Siang_2023} because we thought this would be a helpful way to engage in participatory ideation for participants who do not have much experience with giving constructive critique. %Facilitators probed feedback on the app’s content, language toggle tool, audio, and video. 
Participants and/or facilitators recorded the feedback on post-it notes (Figure \ref{fig:workshop} d). Discussions were audio-recorded. \ladd{App feedback was conducted in both AGYW workshops and the pharmacy staff workshop.} This activity lasted approximately one hour.


\paragraph{Group interview}
Facilitators conducted a semi-structured group interview to discuss how participants’ experiences in the pharmacy would change after exploring the app. Participants discussed their concerns with app usage in the pharmacy setting, the pharmacist’s role in supporting AGYW with and after using the app, how to combine counseling with app engagement, the app’s usefulness to the pharmacy staff, and how contraceptive access can be made easier for AGYW. Discussions were audio-recorded. \ladd{The group interview was conducted in both AGYW workshops and the pharmacy staff workshop.} This activity lasted approximately one to two hours. 

\begin{figure*}[hbt!]
  \centering
\includegraphics[width=1.0\linewidth]{figures/workshop.jpeg}
\caption{Workshop activities. Enlarged images are in Appendix \ref{big_images}. (a) AGYW journey mapping where AGYW and facilitators wrote actions, thoughts, goals, and emotions before a pharmacy visit on post-it notes. (b) AGYW journey mapping with role-play, led by the facilitator. (c) Pharmacy staff journey mapping with role-play. (d) App feedback documented by AGYW on post-it notes using the ``I Like, I Wish, What If” method.}
\Description{Four images of workshop activities, described from left to right. (a) Poster paper with "BEFORE PHARMACY" at the top. Below, a post-it note labeled "ACTION/THOUGHT" and next to it multiple post-it notes with actions and thoughts AGYW have before visiting the pharmacy setting. Below is a post-it note labeled "GOAL" and next to it post-it notes of goals that AGYW have for visiting the pharmacy setting. Below are three post-it notes placed vertically, each with a different facial expression drawn on it, where one is happy, one is neutral, and one is sad. Post-it notes are placed next to it and connected with a string to indicate the emotional path. On each post-it note is a sticker with the same three facial expressions. (b) A girl wearing a sign labeled "AGYW" with a drawing of a girl is facing five girls (workshop participants) and a woman (workshop facilitator). (c) A sign on the wall that says "AMUA PHARMACY". A man wearing a sign labeled "Pharmacist Amua Pharmacy" with a drawing of a man is talking to a woman with a sign labeled "AGYW" and a drawing of a girl. A paper labeled ``AMUA pharmacy” is behind them, to illustrate that the characters are in a pharmacy. (d) Poster paper with "I LIKE" at the top with post-it notes of comments on what participants liked about the app, "I WISH" with post-it notes with comments on  what participants wished the app had, and "IDEAS" with post-it notes with comments on ideas participants had with improving the app.}
\label{fig:workshop}
\end{figure*}



%Figure 2. Workshop activities. (a) AGYW journey mapping where AGYW and facilitators wrote actions, thoughts, goals, and emotions before a pharmacy visit on post-it notes. (b) AGYW journey mapping with AGYW role-playing a pharmacy visit, led by the facilitator. (c) Pharmacy staff journey mapping with pharmacy staff and study staff role-playing a counseling scenario. (d) AGYW providing app feedback on post-it notes using the ``I Like, I Wish, What If” method.

\subsection{Data Analysis}


Data sources included detailed written notes \ladd{in English} recorded by a designated staff member during the workshops, selective transcriptions of workshop audio-recordings, and workshop artifacts (e.g., arranged post-it notes from journey maps and app feedback, photos). \ladd{The primary data sources were artifacts and written notes that were produced for all workshop activities and the secondary data source was audio-recordings from large group workshop activities.} %\ladd{The primary data sources were artifacts and written notes that were produced for each workshop, as the workshops were centered around small or large group activities that involved creating visual representations of ideas and concepts. Large group interview workshop activities were audio-recorded as a secondary source of data, in order to retain individual participant quotations and perspectives. When participants were engaged in small-group activities or speaking one-on-one with a facilitator, it was impractical to rely on audio-recordings, so written note-taking was used.}

Multimedia data from the workshops were analyzed using two processes in tandem: a collaborative analysis by the larger design team (including the first, second, third, and fourth authors), and an in-depth thematic analysis by the first author. Team-based collaborative analysis took place in real time immediately after each workshop and at weekly design team meetings \ladd{that} continued after workshop completion. The collaborative team was made up of Kenyan study staff and researchers from Kenyan and U.S. institutions, with expertise in human-centered design, computer science\ladd{/HCI}, and clinical reproductive health. In the meetings immediately after and between workshops, the team discussed, compared, and contrasted observations within and among workshops. The main goal of these discussions was to identify app design updates to be made for the next workshop, using key insights generated from workshop activities such as free app engagement, usability testing, facilitated app engagement, and app feedback. After data collection was complete, the collaborative design team conducted analytic virtual meetings to consolidate data sources and begin to interpret the data holistically. The team recollected and discussed workshop interactions, observations, and feedback from participants that was particularly insightful, emerged frequently, or was surprising, and cataloged the iterative changes to the app. The Kenyan team members contributed additional expertise in the local context and languages, which enriched and contextualized the data.

Concurrent with the collaborative process, the first author qualitatively analyzed the \lrem{data}\ladd{artifacts and the written notes} with a specific focus on AGYW interactions with the tablet-based app and perceptions of the role of the pharmacist with the app. Using an inductive approach, the first author independently reviewed the \lrem{data}\ladd{the artifacts and the written notes} from the journey mapping, with a focus on key actions and emotions that AGYW and pharmacists expressed, and created an initial matrix of concepts highlighting AGYW concerns, desires, and decision-making processes when accessing contraception in the pharmacy setting. These concepts contextualized participant responses generated from app-specific workshop activities\lrem{Transcripts from the workshops, alongside text from w} (i.e., free and facilitated app engagement, app feedback, group interview), which were coded using an inductive approach guided by the intermediation model. Using thematic analysis ~\cite{Braun_Clarke_2022}, the codes were consolidated into an initial set of themes and iteratively refined. Themes were then discussed between the first and last author to build consensus and develop cross-cutting insights across the workshops. \ladd{After identifying themes, the first author reviewed the audible audio-recordings to select representative participant quotations that speak to the themes and concepts and transcribed and translated them in English with support from the Kenyan study staff.} 




\subsubsection{Ethical Considerations}
Although minors were involved in this study, parental permission for study involvement was waived with approval from the \lrem{IRBs}\ladd{ethics review boards} in Kenya and the U.S. This decision was made through reflection on the reality that parental involvement in sensitive sexual and reproductive health research among adolescents may be harmful for adolescent research participants. To prepare to address conflicts arising from parents who have concerns about their child’s participation in the study, we encouraged minors to consult with a trusted adult prior to giving their assent. Participants were given a contact number so that parents may call to learn more or to arrange to talk with the study team. 

\section{Findings}
We report qualitative findings here from the focus group and diary study interviews and surveys. We first draw an overview of the participants' existing writing habits and envisioned use of Rambler from the focus group, then answer each research question with themes identified from thematic analysis. We end this section with a list of design suggestions made by participants.
\subsection{Focus Group - Participants' Envisioned Usage Scenarios}
\subsubsection{Participants' Writing Habits}
Participants' motivations for writing varied from emotionally significant moments, unique experiences, to academic or work-related writings. They talked about their writing on mobile phones, which was handy for capturing random thoughts (P3, P5, P6). They liked to utilize specialized tools to aid in various writing stages such as idea storage (P8), topic management (P5), logical restructuring (P5), and grammar refinement (P9). They also explored tools to effectively segment and organize multiple writing projects (P2).

Some participants had experience with dictation and mentioned their use of it to multitask or avoid typing (P1). There was a perceived text input efficiency (P5) and benefit in language practice (P6). Some found capturing spoken content disruptive or awkward (P2, P3, P4), with concerns regarding accuracy and responsiveness in noisy environments (P10). Participants who were familiar with LLM tools heavily relied on them for various tasks, including ideation (P1, P2), providing initial ideas (P1), finding words (P2), revising tone (P4), improving grammar (P9), generating specific writing characteristics (P2), learning concepts (P1), practicing writing (P8), improving skills (P6), and translating text (P6).

\subsubsection{Envisioned Use Cases and Scenarios of Rambler}

The participants envisioned using Rambler to assist with schoolwork (P5), learning skills (P6), capturing conversation information (P2), generating functional content (such as emails) (P5), diary writing (P4, P6), screenplay creation (P11, P12), and scripting episodes for podcasts (P10).
Moreover, they liked the flexibility of using Rambler on any device, and envisioned utilizing it while walking on the road (P7), lounging in bed (P2, P4, P6), or relaxing outdoors in a park (P10). This flexibility in usage locations improves the acceptance of Rambler, catering to users' diverse writing preferences and habits.

\subsection{How do creative/academic writers use a speech-based writing tool in real life? }
In the subsequent diary study, participants generally performed their envisioned writing scenarios except for a few topic changes due to technical or logistic constraints. Table~\ref{tab:users} in the Appendix summarized what each participant wrote, in what environment, and on what device. We can see that the majority of writing tasks were done on mobile phones (56.8\%), primarily on a desk or in bed with only a few occasions during walking. Some used the computer (43.2\%) for a larger display and convenient operation. The surveys asked how they distributed their time for writing. The answers showed that 42\% of the writing tasks were completed in one go, while the rest of the tasks were done in a ``distributed'' manner. One task was done after several sessions, once per day or half a day based on time or location in their routine. 

Like the lab study findings of Rambler~\cite{lin2024rambler}, in the diary study we also observed the two distinct writing strategies: 1) \textit{outline first}---users create an outline through dictation and then expand on it; and 2) \textit{free-speaking}---users dictate detailed content spontaneously before editing it. The key difference between these two patterns lies in whether users formulate their narratives before or after dictation. From the logged content, we noticed that the former strategy tended to be used in academic or communicative writings with an external audience, while the latter strategy was more used in personal or reflective writings for oneself. 

\subsubsection{Outline Expansion for Academic or Communicative Writing.}

When participants have clear ideas about what they are going to write, they choose to dictate key points into each Ramble and expand those points with AI-powered semantic functions. Such writing patterns often occur when participants craft logically structured writings and have envisioned the structure, though not necessarily the details, before writing (P2, P3, P9, P14). For example, when writing a responding letter, P2 began by dictating five paragraphs that highlighted different key points to create an outline (see Figure~\ref{fig:str1}): they mention ``\textit{this is a letter I want to respond to his previous letter so I already have the structure in my mind which point I want to reply.}'' After creating the outline, each Ramble was processed through a custom magic prompt to expand the content, such as \textit{``expand this outline to a full-text paragraph''} and \textit{``add an example at the beginning or the end.''} By doing this step, the LLM adds more detail to the outline, and the length of each paragraph goes from one sentence to about five sentences. P2 reflected on this experience, stating, \textit{``I usually don't have enough time to reorganize that into a like complete article. But every letter I sent to my boyfriend is like complete article, so I think it's content I can use to finish the task.''}

Aside from relying on AI-generated content, some participants enhanced their writing by dictating more to supplement the content added by Magic Prompt or by re-organizing the content inspired by AI-generated content.
For instance, when P3 prepared a presentation script to update instructors about a project, they began by outlining their ideas based on existing slides. They then used Magic Prompt to refine the script. In the process, the AI-generated content often sparked memories of additional details, enabling P3 to elaborate on their initial outline. As P3 noted,\textit{``I adjusted the slides in the process, correspondingly, I also adjusted the content of my scripts.''}
It is common that when participants want to add some content to the end of a Ramble, they usually create a new Ramble and merge it with the previous one instead of respeaking to add content.

\begin{figure}
    \Description{This figure illustrates a structured workflow for content development and editing, where ideas progress from outlining to expansion, refinement, reordering, and manual merging to form a cohesive final document.}
    \centering
    \includegraphics[width=0.95\linewidth]{strategy1.png}
    \caption{Participant P2 wrote a letter by dictating an outline first and expanding from it. The numbers represent the order of actions. Blue ones are generated by dictation, and green ones are generated by macro revision.}
    \label{fig:str1}
\end{figure}

\subsubsection{Capturing Loose Thoughts in Personal and Reflective Writing.}

As P5 said, \textit{``I usually spoke a big chunk to Rambler at the beginning.''} Instead of dictating in a typing-like way, more articles were created by speaking everything in one go at the beginning, especially when participants knew the details to be described, such as writing diaries (P4), emotional notes (P6), and comments about a movie (P14). As shown in Figure~\ref{fig:str2}, P4 depicted the day's itinerary in detail. Then, the Ramble with a long paragraph is split into three short Rambles by Semantic Split. Each Ramble marks an event in the diary. For each Ramble, P4 used Magic Prompt to improve grammar, add more details, and change the tone to be more chill and easy to understand. Finally, P4 used Semantic Merge to combine two related Rambles to finish the draft.

Since users do not have a predefined outline when writing, subsequent restructuring is more common than the previous strategies. P5 reports that Semantic Split and Merge can meet their expectations in restructuring: \textit{``Semantic Split made the initial chunk paragraph into a few smaller paragraphs, and the qualities of the smaller paragraphs normally met my satisfaction.''} P5 also found that Semantic Split helped create paragraphs that started with key points, clearing up long paragraphs: \textit{``It is helpful for breaking text into manageable paragraphs with clear topic sentences''}. P12 believed that Semantic Merge helped merge similar paragraphs and encouraged them to rethink the structure so that they had a clearer direction for writing: \textit{``Semantic Merge is the main reason why I use Rambler because I need to sort out my thoughts in a logical way when brainstorming, rather than simply placing words with STT, which doesn’t help me save time.''}

After reorganization, Magic Prompt is widely used to adjust the writing style and tone. For example, P5 asks Rambler to create formal, grammatically correct writing by eliminating pauses and expressions while preserving the original meaning: \textit{``Improve writing to be formal. and correct my grammar but keep my meaning''}. To improve the performance of Magic Prompt, users gave several demands in one prompt and provide the context: \textit{``Make the tone sounds more chill and use words easy to understand it’s my diary don’t be too serious.''} (P2) When they are not satisfied with the results, they will iterate on the prompts until satisfaction, such as \textit{``make it more intense''} and \textit{``Not that intense. Make it sound like he is scared. But more anxious.''} (P8) Aside from macro-revisions, manual editing is generally used to polish the wording, such as short phrases or punctuation: \textit{``When I knew the exact word(s) I wanted to add, I would add it by typing.''} (P9)

\begin{figure}
    \Description{The figure illustrates how larger ideas are broken down into focused segments and later recombined for improved organization and tone adjustments.}
    \centering
    \includegraphics[width=0.95\linewidth]{strategy2.png}
    \caption{Participant P4 wrote an experience sharing by speaking detailed content and reorganizing it. The numbers represent the order of actions. Blue ones are generated by dictation, and green ones are generated by macro revision.}
    \label{fig:str2}
\end{figure}

\subsection{What new affordances does writing with speech have and how could AI help?}

We summarized the following findings about the affordances of writing with speech, as a new way of writing that people perceive and feel in their experience of using Rambler, including how it facilitated emotional expression, boosted writing productivity, and improved their self-efficacy. 

\subsubsection{Speaking as a Natural and Emotional Expression Channel and a Communicative Act.}
The inherent quality of using speech mirrors human natural conversation, thus offering users an uninterrupted channel for authentic emotional recording and introspection. 
P3 likened the nature of writing with speech to an undisturbed telephone conversation:\textit{ ``Doing speech-based writing is like talking on the phone and would not be interfered with.''} 
P5 highlighted the emotional resonance enabled by dictation: using speech to write about emotional events would capture the emotions so vividly and authentically for them that upon reviewing the transcript, they could feel how they felt at that moment.
This unfiltered self-expression makes speech-based writing suitable for personal diary. 
As P2 explained, \textit{``I could be the real me when using dictation to write a diary.''}
P6 noted, \textit{``It's just like you speak to another person, but that person actually doesn't exist and you can see whatever you want to see to it.''}

Furthermore, the act of expressing emotions through dictation serves as a vehicle for organizing one's psychological landscape and fostering introspection, like a meditative experience.
As P4 describes \textit{``using Rambler feels like meditation or self-reflection''}.
Speaking is also associated with collaborating with a teammate, facilitating communicative production.
P4 highlighted that writing with speech was like chatting with a close person, expanding the branches of thoughts together. P13 also noted that the process of writing with speech was like talking with someone by phone, whose purpose was expressing their thoughts.
Considering this property, one participant decided to use Rambler to write a screenplay and found it highly satisfactory. P12 said, \textit{``Rambler directly captured my very simple words, the output generated by dictation surprised me, since the format and literary style was very close to screenplay. In screenplay, you don't need to write in a sophisticated way. So I realized that a screenplay can be simply completed, and I just finished this one (screenplay) in a very cozy sitting posture on the sofa.''}


\vspace{-2mm}
\subsubsection{Speaking Boosts Productivity and LLM Prevents Overthinking.}

The efficiency of dictation
often lead to a surprising experience with the substantial volume of text generated by talking. P14 was notably impressed by the productivity of using dictation
after merely speaking for 1 to 2 minutes, which produced a few hundred words.
They felt it lowered the pressure of writing tasks. P8 echoed a similar experience after a few days of usage and stated that it became easier to output several hundreds of words in one go compared to typing every single sentence on the keyboard.


Sometimes writers overthink a detail and get stuck as they fixate on word choice from the beginning. Perfectionism can block individuals’ inspiration and hinder their creative thinking. Using Rambler, participants felt that the LLM features nudged them to wisely assign their attention and energy towards deliberating the overall content and structure of their work.
P12 noted that they could save energy to think about new ideas with the assistance of LLMs in crafting the wording details. The fluent experience of ideation with the aid of LLM helped prevent individuals from overthinking and reduced their perfectionism in the writing process.
As P8 stated, \textit{``Writing with speech is beneficial, especially with preventing me from overthinking sentences, helping me go with the flow, and leaving some of the more complex sentence building to AI.''}

\vspace{2mm}
\subsubsection{LLM Features Foster Self-efficacy.}~\label{efficacy}
The LLM features for polishing the oral spoken draft into written formats brought confidence to users in their writing process. P4 commented that LLMs could assist formal writing in good quality since it was able to rephrase their spoken tone to a suitable style.
P14 also expressed appreciation for this aspect: \textit{``Since LLM could polish the wording in a beautiful style, I didn't need to worry about whether my speaking was correct or not, instead, it helped to remove the errors in my original speaking at the beginning, it effectively avoided the errors.''}

Moreover,
participants prompted LLMs to adopt specific literary styles, and improved their language learning by observing the polished outcomes. 
P14 noted personal growth in language proficiency through this feature. As the LLM polished their originally simple wording into a higher quality, they learned new writing styles and skills from the change and realized that it could be a good opportunity to improve English writing skills.
P2 was amazed after employing an original English diary writing style by prompting the LLM:
\textit{``Rambler helped me to make my diary look more like an English native speaker, which impressed me a lot. I felt surprised by the change since it is highly like the style in a book about interesting English diaries I read before, it's a way to enhance my writing skill and make my writing close to the original English literary style.''}

\subsection{What factors affect the user acceptance of writing with speech?}
Nine out of twelve participants (P2, P3, P4, P5, P6, P9, P13, P14, P15) reported that they were able to get used to writing with speech in the short period of the diary study, while three (P8, P10, P12) expressed reservation. As shown in Figure~\ref{fig:useracc}, the median of the score for ``comfort of use'' is 5 (out of 7) for all three tasks submitted over time, while the deviation of the scores decreases after the first use. This shows positive user acceptance of this new approach for writing, as well as some initial learning curve to get used to it.

\subsubsection{Productivity Gain.} Although we could not measure task completion time effectively in this study due to its in-the-wild nature, we did ask for an estimation of time they spent on a writing task in each survey. Their answers showed, 66.7\% of the 500-word writing tasks were completed in 10 to 30 minutes, 15.4\% of them needed 30 to 60 minutes, 10.3\% over an hour, 7.7\% within 10 minutes.
Participants felt a sense of achievement after completing a task with Rambler and expressed positive surprises in gaining trust in the technology. For P14, because writing was a heavy task that brought a huge mental burden, it was previously hard for them to complete an article in traditional ways (either typing or handwriting), and so they gave up having a writing routine.
In the focus group discussion, they initially expressed doubts about whether speech could facilitate writing and thought dictation technology employed in writing was an ambiguous concept to them. Yet in the post-study interview, they shared a big sense of surprise from the productivity gain in writing with speech.
They expressed willingness to restart a writing routine in the future via dictation. 

\subsubsection{Effective help for organizing thoughts.} Another reason for participants to adopt this technology was that the LLM features could greatly reduce their effort by merging several vague ideas into one. P12 said in the focus group that using traditional writing to illustrate details of ideas from a vague concept could always block their writing. 
In the post-study interview, they shared their pleasant experiences of inputting their vague ideas and prompting the LLM to merge them into a coherent paragraph. They felt that the LLM features could effectively and efficiently boost clarity at the ideation stage.

\begin{figure}
    \Description{A boxplot illustrating how participants felt comfortable during three rounds of writing tasks (1st, 2nd, and 3rd).}
    \centering
    \includegraphics[width=0.9\linewidth]{user_acceptance.png}
    \caption{Participants' user acceptance during three rounds of writing tasks using a 7-point Likert scale.}
    \label{fig:useracc}
\end{figure}

\subsubsection{Remaining Challenges.}
In the meantime, there remain some challenges in adopting this new writing paradigm.
One challenge the participants faced was that they felt distracted during the process of speaking while thinking about the next sentence (P8). 
Although Rambler leveraged LLM to correct recognition errors and disfluencies, detailed editing is still necessary at times.
P3 expressed the inconvenience of having to modify a typo in the middle of a sentence. Unlike typing on keyboards, it is nearly impossible to direct the cursor to a certain place via LLM prompting.

Last but not least, several participants raised the issue of being unable to multitask during speech composition. For instance, participants felt that they could not use Rambler to write something that requires them to research online at the same time. P2 said, \textit{``Using Rambler on phone is not convenient for reading and doing research meanwhile.''}
P3 also mentioned the inconvenience of how Rambler currently cannot support multitasking in different tabs or apps.

\subsection{User-Suggested Improvement for Design}
\subsubsection{Context Awareness} Participants made suggestions of reusing their Custom Magic Prompts. As P9 stated, ``\textit{It would be better if Rambler could save my prompts' history, so I could quickly access them again if I needed.}'' P2 said, ``\textit{It would be more useful if the prompt can be applied in all the Rambles simultaneously on the purpose of efficiently knowing the context of my article and adjusting the writing tone of the whole article in one go.}''

\subsubsection{Personalized Conversational Support} Given the nature of speaking as a communicative act, integrating a virtually personalized assistant in writing with speech could represent a strategic enhancement. P14 said, ``\textit{It could encourage users to have more inspirations to write, if there is a chatbot or assistant knowing the context on Rambler, since the behavior of having conversation or communication with them can enhance people to think forward.}''
P2 attempted that prompting an assistant-like suggestion but did not get the expected result, what they did was giving an identity to the LLM, and expected the feedback from LLM to generate more human-like information for personal content. Similarly, P9 had the same feeling as well in the creation process: \textit{“It could be very useful to me if I could personalize a chatbot on Rambler”.}

\subsubsection{Alternative Information Storage and Display} Some participants felt it would be good to keep their original audio mapped to the transcripts for error prevention (P9, P15). Others also wanted to be able to see the text and outline side-by-side so that they could read the main points of each paragraph at a glance during composition (P15). Participants also requested easier ways to distinguish the Rambles, such as by adding a title for each. P9 said, ``\emph{It would be better if there's a title for each Ramble ... I had to re-read them every time if I wanted to remind myself of the main point of each paragraph. The process could be annoying and consumed my patience.}''
\section{Discussion and Conclusion}
\label{sec:discussion}


\textbf{Conclusion.} In this paper, we propose LRM to utilize diffusion models for step-level reward modeling, based on the insights that diffusion models possess text-image alignment abilities and can perceive noisy latent images across different timesteps. To facilitate the training of LRM, the MPCF strategy is introduced to address the inconsistent preference issue in LRM's training data. We further propose LPO, a method that employs LRM for step-level preference optimization, operating entirely within the latent space. LPO not only significantly reduces training time but also delivers remarkable performance improvements across various evaluation dimensions, highlighting the effectiveness of employing the diffusion model itself to guide its preference optimization. We hope our findings can open new avenues for research in preference optimization for diffusion models and contribute to advancing the field of visual generation.

\textbf{Limitations and Future Work.} (1) The experiments in this work are conducted on UNet-based models and the DDPM scheduling method. Further research is needed to adapt these findings to larger DiT-based models \cite{sd3} and flow matching methods \cite{flow_match}. (2) The Pick-a-Pic dataset mainly contains images generated by SD1.5 and SDXL, which generally exhibit low image quality. Introducing higher-quality images is expected to enhance the generalization of the LRM. (3) As a step-level reward model, the LRM can be easily applied to reward fine-tuning methods \cite{alignprop, draft}, avoiding lengthy inference chain backpropagation and significantly accelerating the training speed. (4) The LRM can also extend the best-of-N approach to a step-level version, enabling exploration and selection at each step of image generation, thereby achieving inference-time optimization similar to GPT-o1 \cite{gpt_o1}.
\section{Conclusion}
We present live monitoring and mid-run interventions for multi-agent systems. We demonstrate that monitors based on simple statistical measures can effectively predict future agent failures, and these failures can be prevented by restarting the communication channel. Experiments across multiple environments and models show consistent gains of up to 17.4\%-20\% in system performance, with an addition in inference-time compute.
Our work also introduces \ourenv{}, a new environment for studying multi-agent cooperation.



% Grouping authors' names or e-mail addresses, or providing an ``e-mail
% alias,'' as shown below, is not acceptable:
% \begin{verbatim}
%   \author{Brooke Aster, David Mehldau}
%   \email{dave,judy,steve@university.edu}
%   \email{firstname.lastname@phillips.org}
% \end{verbatim}


% If your author list is lengthy, you must define a shortened version of
% the list of authors to be used in the page headers, to prevent
% overlapping text. The following command should be placed just after
% the last \verb|\author{}| definition:
% \begin{verbatim}
%   \renewcommand{\shortauthors}{McCartney, et al.}
% \end{verbatim}






% \begin{table}
%   \caption{Frequency of Special Characters}
%   \label{tab:freq}
%   \begin{tabular}{ccl}
%     \toprule
%     Non-English or Math&Frequency&Comments\\
%     \midrule
%     \O & 1 in 1,000& For Swedish names\\
%     $\pi$ & 1 in 5& Common in math\\
%     \$ & 4 in 5 & Used in business\\
%     $\Psi^2_1$ & 1 in 40,000& Unexplained usage\\
%   \bottomrule
% \end{tabular}
% \end{table}

% To set a wider table, which takes up the whole width of the page's
% live area, use the environment \textbf{table*} to enclose the table's
% contents and the table caption.  As with a single-column table, this
% wide table will ``float'' to a location deemed more
% desirable. Immediately following this sentence is the point at which
% Table~\ref{tab:commands} is included in the input file; again, it is
% instructive to compare the placement of the table here with the table
% in the printed output of this document.

% \begin{table*}
%   \caption{Some Typical Commands}
%   \label{tab:commands}
%   \begin{tabular}{ccl}
%     \toprule
%     Command &A Number & Comments\\
%     \midrule
%     \texttt{{\char'134}author} & 100& Author \\
%     \texttt{{\char'134}table}& 300 & For tables\\
%     \texttt{{\char'134}table*}& 400& For wider tables\\
%     \bottomrule
%   \end{tabular}
% \end{table*}

% Always use midrule to separate table header rows from data rows, and
% use it only for this purpose. This enables assistive technologies to
% recognise table headers and support their users in navigating tables
% more easily.


% \section{Acknowledgments}

% This section has a special environment:
% \begin{verbatim}
%   \begin{acks}
%   ...
%   \end{acks}
% \end{verbatim}
% so that the information contained therein can be more easily collected
% during the article metadata extraction phase, and to ensure
% consistency in the spelling of the section heading.

% Authors should not prepare this section as a numbered or unnumbered {\verb|\section|}; please use the ``{\verb|acks|}'' environment.

% \section{Appendices}

% \begin{verbatim}
%   \appendix
% \end{verbatim}


%%
%% The acknowledgments section is defined using the "acks" environment
%% (and NOT an unnumbered section). This ensures the proper
%% identification of the section in the article metadata, and the
%% consistent spelling of the heading.
\begin{acks}
Research reported in this publication was funded by the Eunice Kennedy Shriver Institute for Child Health and Human Development (K12HD001264) and the University of Washington Global Innovation Fund. The authors acknowledge Merceline Awuor and Cellestine Aoko for their major contributions to recruitment, data collection, and essential administrative support. We also thank Sandy Cheng, Ariel Fu, Khushi Khandelwal, Shrimayee Narasimhan, and Shobhit Srivastava for their contributions to the development of the Mara Divas app. 
\end{acks}

%%
%% The next two lines define the bibliography style to be used, and
%% the bibliography file.

\bibliographystyle{ACM-Reference-Format}
\bibliography{references}



%%
%% If your work has an appendix, this is the place to put it.
% \section
% \appendix

\appendix
\appendix
% \setcounter{table}{0}
% \renewcommand*{\thetable}{\arabic{table}}
% \renewcommand*{\thefigure}{\arabic{figure}}
\section{Related algorithms and metric caculation}
\label{app:related_algo_metric}

\subsection{Performance-energy Consistency} In this paper, performance-energy consistency refers to the consistency between the results evaluated using an energy model and those evaluated using real-world metrics for the same sample. Specifically, the consistency requires that good samples are assigned low energy, while poor samples are assigned high energy. Performance-energy consistency measures the proportion of element pairs that maintain the same relative order in both permutations \( X \) and \( Y \), where \( X \) and \( Y \) represent the index arrays obtained by sorting the original energy values \( \mathbf{E} = (E_1, E_2, \dots, E_N) \) and performance metric values \( \mathbf{P} = (P_1, P_2, \dots, P_N) \), respectively, in ascending order. In this paper, the energy values are calculated by energy model $E_\theta(x_0)$ for samples $\x_0$. The performance metric values are calculated as the L2 distance between the generated samples $\x_0$ and the ground truth under the given condition.

Let \( X = (X_1, X_2, \dots, X_N) \) and \( Y = (Y_1, Y_2, \dots, Y_N) \) be the index arrays obtained by sorting the original energy values \( \mathbf{E} = (E_1, E_2, \dots, E_N) \) and performance metric values \( \mathbf{P} = (P_1, P_2, \dots, P_N) \), respectively, in ascending order. Specifically, \( X_i \) is the rank of the \( i \)-th sample in the sorted energy values \( \mathbf{E} \), and \( Y_i \) is the rank of the \( i \)-th sample in the sorted performance metric values \( \mathbf{P} \).

\textbf{Consistency Definition:}
The \textbf{consistency} is defined as the proportion of consistent pairs \( (i, j) \) where \( i < j \) and the relative order of \( i \) and \( j \) in \( X \) is the same as in \( Y \). Specifically:
\[
\text{Consistency} = \frac{1}{\binom{N}{2}} \sum_{i=1}^{N-1} \sum_{j=i+1}^{N} \mathbb{I}\left( (X_i < X_j \land Y_i < Y_j) \lor (X_i > X_j \land Y_i > Y_j) \right),
\]
where:
\begin{itemize}
    \item \( \binom{N}{2} = \frac{N(N-1)}{2} \) is the total number of pairs \( (i, j) \) with \( i < j \),
    \item \( \mathbb{I}[\cdot] \) is the indicator function, which evaluates to 1 if the condition inside the brackets holds (i.e., the relative order is consistent), and 0 otherwise.
\end{itemize}
\subsection{Adversarial sampling}
During the sampling process, energy optimization often gets trapped in local minima or incorrect global minima, making it difficult to escape and hindering the sampling of high-quality samples.
\subsection{Negative Sample Generation} Negative samples are generated by introducing noise into the positive sample \( x_0 \). In the Maze and Sudoku experiments, permutation noise is applied to the channel dimension to induce significant changes in the solution. Other noise types can be used, as this remains a hyperparameter choice. Specifically, we first randomly sample two scalars \( p_1 \) and \( p_2 \) from a uniform distribution in the interval \( [0, 1] \), i.e., \( p_1, p_2 \sim \text{Uniform}(0, 1) \) ($p_1<p_2$). Then, for each channel position of the positive sample \( x_0 \), we swap the channel positions with probabilities \( p_1 \) and \( p_2 \), resulting in \( x_0^{-} \) and \( x_0^{--} \), such that the L2 distance between \( x_0^{-} \) and \( x_0 \) is smaller than the L2 distance between \( x_0^{--} \) and \( x_0 \). For other noise types, such as Gaussian noise, we normalize the L2 norm of the noise and apply noise at different scales to ensure that the L2 distance from \( x_0^{-} \) to \( x_0 \) is smaller than the L2 distance from \( x_0^{--} \) to \( x_0 \).


\subsection{Linear-regression algorithm} Given three points \((x_1, y_1)\), \((x_2, y_2)\), and \((x_3, y_3)\), we wish to fit a line of the form ~\cite{lane2003introduction}:

\[
y = kx + b
\]
The mean of the \(x\)-coordinates and the mean of the \(y\)-coordinates are:
\[
\bar{x} = \frac{1}{3}(x_1 + x_2 + x_3), \quad \bar{y} = \frac{1}{3}(y_1 + y_2 + y_3)
\]
The slope \(k\) of the best-fit line is given by the formula:

\[
k = \frac{\sum_{i=1}^{3} (x_i - \bar{x})(y_i - \bar{y})}{\sum_{i=1}^{3} (x_i - \bar{x})^2}
\]
This formula represents the least-squares solution for the slope.
Once the slope \(k\) is determined, the intercept \(b\) can be calculated as:
\[
b = \bar{y} - k\bar{x}
\]
The equation of the best-fit line is:
\[
\hat{y} = kx + b
\]
\section{Details of experiments}
\label{app:Exp_detail}
\subsection{Detais of Sudoku experiments}
\label{app:Exp_sudoku}
For Sudoku experiment, the dataset, model architecture, and training configurations are adopted from \citet{du2024learning}. We mainly use solving success rate to evaluate different models. Model backbone and training configurations can be found in Fig. \ref{fig:sudoku_ebm} and Table \ref{tab:sudoku_exp_detail}, respectively. All the exploration hyperparameters $c$ are set as 100 for Sudoku task.
\begin{figure}[H]
\begin{minipage}{0.9\textwidth}
\centering
\small
\begin{tabular}{c}
    \toprule
    3x3 Conv2D, 384 \\
    \midrule
    Resblock 384 \\
    \midrule
    Resblock 384 \\
    \midrule
    Resblock 384 \\
    \midrule
    Resblock 384 \\
    \midrule
    Resblock 384 \\
    \midrule
    Resblock 384 \\
    \midrule
    3x3 Conv2D, 9 \\ 
    \bottomrule
\end{tabular}
\caption{The model architecture for \proj on Sudoku task. The energy value is computed using the L2 norm of the final predicted output similar to \citet{du2023reduce}, while the output is directly used as noise prediction for the diffusion baseline.}
\label{fig:sudoku_ebm}
\end{minipage}
\end{figure}
\begin{table}[ht]
  \begin{center}
    \caption{\textbf{Details of  training for Sudoku task}. }
    \vskip -0.15in
    \label{tab:2d_model_architecture}
    \begin{tabular}{l|c} % <-- Alignments: 1st column left, 2nd middle and 3rd right, with vertical lines in between
    \multicolumn{2}{l}{}\\
      \hline
       \multicolumn{1}{l|}{Training configurations } & \multicolumn{1}{l}{}\\
      \hline
      Number of training steps & 100000  \\
      Training batch size & 64 \\
      Learning rate & 0.0001 \\
      Diffusion steps & 10 \\
      Inner loop optimization steps & 20 \\
      Denoising loss type & MSE \\
      Optimizer & Adam \\
        \hline
    \end{tabular}
      \label{tab:sudoku_exp_detail}
  \end{center}
\end{table}
\subsection{Details of Maze experiments}
\label{app:Exp_maze}
The details of maze experiments and model backbone are provided in Table \ref{tab:maze_exp_detail} and Fig. \ref{fig:maze_ebm}, respectively. The key metric, the maze-solving success rate is defined as the proportion of model-generated paths that have no breakpoints, do not overlap with walls, and begin and end at the start and target points, respectively. Maze datasets are generated by \citet{ivanitskiy2023configurable}, and detailed hyperparameter configurations are in Table \ref{tab:maze_exp_detail}. All the exploration hyperparameters $c$ are set as 100 for Maze task.
\begin{figure}[H]
\begin{minipage}{0.9\textwidth}
\centering
\small
\begin{tabular}{c}
    \toprule
    3x3 Conv2D, 384 \\
    \midrule
    Resblock 384 \\
    \midrule
    Resblock 384 \\
    \midrule
    Resblock 384 \\
    \midrule
    Resblock 384 \\
    \midrule
    Resblock 384 \\
    \midrule
    Resblock 384 \\
    \midrule
    3x3 Conv2D, 9 \\ 
    \bottomrule
\end{tabular}
\caption{The model architecture for \proj on Maze task. The energy value is computed using the L2 norm of the final predicted output similar to \citet{du2023reduce}, while the output is directly used as noise prediction for the diffusion baseline.}
\label{fig:maze_ebm}
\end{minipage}
\end{figure}
\begin{table}[ht]
  \begin{center}
    \caption{\textbf{Details of Maze dataset, training}. }
    \vskip -0.15in
    \label{tab:2d_model_architecture}
    \begin{tabular}{l|c} % <-- Alignments: 1st column left, 2nd middle and 3rd right, with vertical lines in between
    \multicolumn{2}{l}{}\\
      \hline
      \multicolumn{1}{l|}{Dataset:} & \multicolumn{1}{l}{}\\ 
      \hline
      Size of training dataset with grid size 4 & 10219   \\
      Size of training dataset with grid size 5 & 9394   \\
      Size of training dataset with grid size 6 & 10295  \\
      Minimum length of solution path & 5 \\
      Algorithm to generate the maze & DFS \\
      Size of test dataset with grid size 6 & 837   \\
      Size of test dataset with grid size 8 & 888   \\
      Size of test dataset with grid size 10 & 948   \\
      Size of test dataset with grid size 12 & 960   \\
      Size of test dataset with grid size 15 & 975   \\
      Size of test dataset with grid size 20 & 978   \\
      Size of test dataset with grid size 30 & 994   \\
      \hline
       \multicolumn{1}{l|}{Training configurations } & \multicolumn{1}{l}{}\\
      \hline
      Number of training steps & 200000  \\
      Training batch size & 64 \\
      Learning rate & 0.0001 \\
      Diffusion steps & 10 \\
      Inner loop optimization steps & 20 \\
      Denoising loss type & MSE + MAE \\
      Optimizer & Adam \\
        \hline
    \end{tabular}
      \label{tab:maze_exp_detail}
  \end{center}
\end{table}

\section{Performance sensitivity to hyperparameters}
\label{app:hyperparameters_sensitivity}

% inner loop opt steps, mcts noise scale(original model, mixed trained model hMCTS & Random search) more visualizations?
In this subsection, we analyze the impact of several hyperparameters on the experimental results. As shown in Table \ref{tab:maze_noise_scale}, the influence of different noise scales on the performance of various methods is presented. The hMCTS denoising and random search require a relatively larger noise scale to better expand the search space and improve final performance, while the diffusion model with naive inference performs best with a smaller noise scale. As demonstrated in Table \ref{tab:maze_inner_loop_opt} and Fig. \ref{fig:maze_opt_step}, the effect of varying inner-loop optimization steps on the results is also analyzed. It can be observed that performance improves gradually with an increasing number of steps, and after 5 steps, the performance stabilizes and the improvement slows down. Therefore, we chose 5 inner-loop optimization steps for the Maze experiments in this paper.
\begin{figure}[h!]
\vskip 0.2in
\begin{center}
\centerline{\includegraphics[width=0.55\textwidth]{fig/maze_optimization_steps_vs_values.pdf}}
\caption{Visualization of success rate across different number of inner-loop optimization steps on Maze with grid size $\mathbf{15\times15}$. }
\label{fig:maze_opt_step}
\end{center}
\vskip -0.2in
\end{figure}
\begin{table}[ht]
\caption{Success rate across the different number of inner-loop optimization step on Maze with grid size \textbf{15}. }
\label{tab:maze_inner_loop_opt}
\vskip 0.15in
\begin{center}
\resizebox{0.85\textwidth}{!}{ % Resize the table to fit within a single column
\begin{tabular}{l|cccccccccc}
\toprule
 &\multicolumn{10}{c}{\textbf{Number of optimization step}} \\
\cmidrule(lr){2-11} 
\textbf{Methods}                & 1               & 2               & 3               & 4               & 5               & 6               & 7               & 8               & 9 &10        \\
\midrule
T-SCEND tr. (ours), Naive inference & 0.0000 & 0.1562 & 0.2109 & 0.2734 & 0.2812 & 0.2734 & 0.2812 & 0.2969 & 0.2969 & 0.2969\\
\bottomrule
\end{tabular}
}
\end{center}
\vskip -0.1in
\end{table}
\begin{table}[ht]
\caption{Success rate across different noise scales on Maze with grid size \textbf{15}. }
\label{tab:maze_noise_scale}
\vskip 0.15in
\begin{center}
\resizebox{1\textwidth}{!}{ % Resize the table to fit within a single column
\begin{tabular}{l|cccccccccc}
\toprule
 &\multicolumn{10}{c}{\textbf{Noise scale}} \\
\cmidrule(lr){2-11} 
\textbf{Methods}                & 0.1               & 0.2               & 0.3               & 0.4               & 0.5               & 0.6               & 0.7               & 0.8               & 0.9 &1.0        \\
\midrule
T-SCEND tr. (ours), hMCTS denoising (energy)               & 0.3828 & 0.4375 & 0.5312 & 0.6094 & 0.6562 & 0.6953 & 0.7031 & 0.7344 & 0.7734 & 0.7969 \\
T-SCEND tr. (ours), naive inference                    & 0.3125 & 0.2656 & 0.2578 & 0.2344 & 0.2422 & 0.2656 & 0.2578 & 0.2422 & 0.2500 & 0.2500 \\
T-SCEND tr. (ours), Random search(energy)      & 0.3906 & 0.4453 & 0.5312 & 0.5703 & 0.5938 & 0.6328 & 0.6641 & 0.6719 & 0.6797 & 0.6562 \\
\bottomrule
\end{tabular}
}
\end{center}
\vskip -0.1in
\end{table}
\section{Additional results}
\label{app:additional_results}
\begin{figure}[h!]
\vskip 0.2in
\begin{center}
\centerline{\includegraphics[width=0.6\textwidth]{fig/maze_success_rate_vs_ts.pdf}}
\caption{Visualization of Success rate across different MCTS start step $t_s$. }
\label{fig:maze_success_rate_vs_ts}
\end{center}
\vskip -0.2in
\end{figure}
The parameter \( t_s \) controls the proportion of the total inference budget allocated to MCTS denoising. When \( t_s = 9 \), it means only MCTS denoising is used, while \( t_s = 0 \) means only best-of-N random search is employed. For \( 0 < t_s < 9 \), hMCTS denoising is applied. As shown in Table \ref{tab:maze_mcts_start_step} and Fig. \ref{fig:maze_success_rate_vs_ts}, there is a noticeable peak in model performance as \( t_s \) varies.
\begin{table}[h!]
\caption{Success rate of hMCTS denoising on Maze with grid size \textbf{15} across different MCTS start steps. }
\label{tab:maze_mcts_start_step}
\vskip 0.15in
\begin{center}
\resizebox{\textwidth}{!}{ % Resize the table to fit within a single column
\begin{tabular}{l|cccccccccc}
\toprule
\cmidrule(lr){2-11} 
\textbf{Methods} & 0               & 1               & 2               & 3               & 4               & 5               & 6               & 7               & 8               & 9         \\
\midrule
Original, hMCTS denoising (energy)      & 0.0781 & 0.0703 & 0.0859& 0.0781 & 0.1250& 0.1484& 0.1250 & 0.0781 & 0.0625 & 0.0703\\
T-SCEND tr. (ours), hMCTS denoising (energy)   & 0.6562 & 0.6094& 0.6641 & 0.7969 & 0.7969 & 0.6406& 0.4922 & 0.4922 & 0.4609 & 0.4453 \\
\bottomrule
\end{tabular}
}
\end{center}
\vskip -0.1in
\end{table}

\begin{table}[h!]
\caption{Success rate of Random search for different training methods on Maze with grid size \textbf{15} and Sudoku harder dataset guided with ground truth accuracy. Untrained, Random search (gt) represents use ground truth to guide the random search.  Here, $L=N$. Bold font denotes the best model. }
\label{tab:maze_diffus_baseline_diversity}
\vskip 0.15in
\begin{center}
\resizebox{\textwidth}{!}{ % Resize the table to fit within a single column
\begin{tabular}{l|cccccc|ccccccc}
\toprule
\multicolumn{1}{c|}{} & \multicolumn{6}{c}{\textbf{Maze success rate}} & \multicolumn{7}{c}{\textbf{Sudoku success rate}}\\ 
\cmidrule(lr){2-14} 
\textbf{Methods} & \textbf{$N$=1} & \textbf{$N$=11} & \textbf{$N$=21} & \textbf{$N$=41} & \textbf{$N$=81} & \textbf{$N$=161} &\textbf{$N$=1} & \textbf{$N$=11} & \textbf{$N$=21} & \textbf{$N$=41} & \textbf{$N$=81} & \textbf{$N$=161} & \textbf{$N$=321} \\
\midrule
Untrained, Random search (gt) & 0.0000 & 0.0000 & 0.0000 & 0.0000 & 0.0000 & 0.0000 & 0.0000 & 0.0000 & 0.0000 & 0.0000 & 0.0000 & 0.0000 & 0.0000 \\ 
Original, Random search (gt) & 0.0625 & 0.1250 & 0.1094 & 0.1328 & 0.1719 & 0.1719 & 0.0859 & 0.1641 & 0.2188 & 0.2344 & 0.2422 & 0.2656 & 0.2969 \\
DDPM, Random search (gt) & 0.0312&0.1094&0.1587&0.1746&0.2031&0.2422& 0.0000          & 0.0000          & 0.0000          & 0.0000          & 0.0000          & 0.0000          & 0.0156 \\
T-SCEND tr. w/o LRNCL, Random search (gt) & \textbf{0.2500} & \textbf{0.5078} & \textbf{0.5938} & \textbf{0.6562} & \textbf{0.7109} & \textbf{0.7422} & \textbf{0.1094} & \textbf{0.2578} & \textbf{0.2969} & \textbf{0.3438} & \textbf{0.3750} & \textbf{0.3828} & \textbf{0.4219} \\ 
\bottomrule
\end{tabular}
}
\end{center}
\vskip -0.1in
\end{table}
\begin{table}[h!]
\caption{Success rate and element-wise accuracy of Random search for different training methods on Sudoku harder dataset guided with ground truth accuracy. Here, $L=N$. Bold font denotes the best model. }
\label{tab:sudoku_diffus_baseline_ddpm}
\vskip 0.15in
\begin{center}
\resizebox{1\textwidth}{!}{ % Resize the table to fit within a single column
\begin{tabular}{l|ccccccc|ccccccc}
\toprule
& \multicolumn{7}{c|}{\textbf{Success rate}} & \multicolumn{7}{c}{\textbf{Element-wise} accuracy}\\
\cmidrule(lr){2-15} 
Methods &\textbf{$N$=1} & \textbf{$N$=11} & \textbf{$N$=21} & \textbf{$N$=41} & \textbf{$N$=81} & \textbf{$N$=161} & \textbf{$N$=321} &
\textbf{$N$=1} & \textbf{$N$=11} & \textbf{$N$=21} & \textbf{$N$=41} & \textbf{$N$=81} & \textbf{$N$=161} & \textbf{$N$=321} \\
\midrule
DDPM, Random search, GT accuracy guided     & 0.0000          & 0.0000          & 0.0000          & 0.0000          & 0.0000          & 0.0000          & 0.0156          & 0.5071          & 0.6089          & 0.6316          & 0.6492          & 0.6691          & 0.6881          & 0.6999          \\
Original, Random search, GT accuracy guided & 0.0781          & 0.1641          & 0.2188          & 0.2344          & 0.2422          & 0.2656          & 0.2812          & \textbf{0.6650} & 0.7731          & 0.7952          & 0.8036          & 0.8217          & 0.8347          & 0.8491          \\
T-SCEND tr. w/o LRNCL, Random search, GT accuracy guided            & \textbf{0.1094} & \textbf{0.2578} & \textbf{0.2969} & \textbf{0.3438} & \textbf{0.3750} & \textbf{0.3828} & \textbf{0.4219} & 0.6442 & \textbf{0.7855} & \textbf{0.8096} & \textbf{0.8317} & \textbf{0.8466} & \textbf{0.8628} & \textbf{0.8854} \\ 
\bottomrule
\end{tabular}
}
\end{center}
\vskip -0.1in
\end{table}





% \begin{table}[]
% \caption{Success rate of mixed inference on Maze with grid size \textbf{15} across different MCTS start steps. Bold font denotes the best model.}
% \label{tab:maze_mcts_start_step}
% \begin{tabular}{lllllllllll}
% \cline{1-1}
% \multicolumn{1}{|l|}{Method} & \multicolumn{10}{c}{MCTS start step}                                                                                                                                              \\
% \multicolumn{1}{|l|}{}                        & 0               & 1               & 2               & 3               & 4               & 5               & 6               & 7               & 8               & 9               \\ \cline{1-1}
% Mixed inference, KL \& LRNCL ,Energy guided   & 0.6562 ± 0.4750 & 0.6094 ± 0.4879 & 0.6641 ± 0.4723 & 0.7969 ± 0.4023 & 0.7969 ± 0.4023 & 0.6406 ± 0.4798 & 0.4922 ± 0.4999 & 0.4922 ± 0.4999 & 0.4609 ± 0.4985 & 0.4453 ± 0.4970 \\
% Mixed inference, Original, Energy guided      & 0.0781 ± 0.2684 & 0.0703 ± 0.2557 & 0.0859 ± 0.2803 & 0.0781 ± 0.2684 & 0.1250 ± 0.3307 & 0.1484 ± 0.3555 & 0.1250 ± 0.3307 & 0.0781 ± 0.2684 & 0.0625 ± 0.2421 & 0.0703 ± 0.2557
% \end{tabular}
% \end{table}
\section{Limitations and future work}
\label{app:limit_future} 
Our inference framework primarily relies on MCTS, which presents two key limitations: (1) limited compatibility with parallel computing, and (2) challenges in effectively evaluating node quality during the early stages of denoising. Future work could explore integrating alternative search strategies, such as those proposed by \citet{wu2024inference}. Additionally, to enhance performance-energy consistency, we introduce linear-regression negative contrastive learning, which enforces a linear relationship between energy and the distance to real samples. Further investigation is needed to assess the broader implications of this constraint and explore alternative regularization approaches. Lastly, while our current implementation utilizes Gaussian noise for branching, other diffusion-based branching mechanisms remain an open area for exploration.
\section{Visualization of results}
\label{app:vis_results}
\subsection{Visualization of Maze experiments}
\label{app:maze_vis}
This section presents visualizations of the training in Fig. \ref{fig:maze_training_vis}, test Maze data in Fig. \ref{fig:maze_test_vis}, and samples generated by different methods in Fig. \ref{fig:maze_samples_diff}. In the visuals, black pixels denote walls, green represents the starting point, red represents the goal point, blue marks the solved path, and white represents the feasible area. All visualizations are based on a few representative samples. The results from the training and test sets clearly show that the tasks in the test set are notably more challenging than those in the training set. Visual comparisons of samples generated by different methods reveal that the originally trained model, regardless of the inference strategy, performs consistently worse than \proj.
\begin{figure}[tb]
\vskip 0.2in
\begin{center}
\centerline{\includegraphics[width=0.8\textwidth]{fig/maze_plot_multi_grid_size_appendix_train.pdf}}
\caption{Visualization of training maze dataset. }
\label{fig:maze_training_vis}
\end{center}
\vskip -0.2in
\end{figure}

\begin{figure}[ht]
\vskip 0.2in
\begin{center}
\centerline{\includegraphics[width=0.8\textwidth]{fig/maze_plot_multi_grid_size_appendix_test.pdf}}
\caption{Visualization of test maze dataset, where the blue paths are ground-truth solutions.}
\label{fig:maze_test_vis}
\end{center}
\vskip -0.2in
\end{figure}

\begin{figure}[ht]
\vskip 0.2in
\begin{center}
\centerline{\includegraphics[width=0.8\textwidth]{fig/maze_plot_diff.pdf}}
\caption{Visualization of samples generated by different training and inference methods.}
\label{fig:maze_samples_diff}
\end{center}
\vskip -0.2in
\end{figure}
\subsection{Visualization of Sudoku experiments}
\label{app:sudoku_vis}

\begin{figure}[ht]
% \vskip 0.2in
\begin{center}
\centerline{\includegraphics[width=0.7\textwidth]{fig/sudoku_train_test_samples.pdf}}
\caption{Visualization of training and test Sudoku dataset.}
\label{fig:sudoku_training_test_vis}
\end{center}
% \vskip -0.2in
\end{figure}

\begin{figure}[ht]

\vskip 0.2in
\begin{center}
\centerline{\includegraphics[width=0.7\textwidth]{fig/sudoku_plot_diff.pdf}}
\caption{Visualization of samples generated by different training and inference methods.}
\label{fig:sudoku_samples_diff}
\end{center}
\vskip -0.2in
\end{figure}

This section presents visualizations of the training and test Sudoku data in Fig.~\ref{fig:sudoku_training_test_vis}, and representative samples generated by different methods in Fig.~\ref{fig:sudoku_samples_diff}. In the
visuals, black numbers denote the condition, green numbers represent correct predictions, and red numbers represent wrong predictions. All visualizations are derived from a few representative samples. The comparison between the training and test sets clearly indicates that the tasks in the test set are significantly more difficult than those in the training set. When comparing the samples generated by different methods, it is evident that the originally trained model, regardless of the inference strategy, consistently underperforms compared to \proj.

% training dataset, landscape visualization, solution of different models
% \section{Analysis of failure case}
% \tao{TODO}
% \jiashu{TODO}
% \label{app:failure_analysis} 
% training dataset, test dataset, solutions of different models

\end{document}
\endinput
%%
%% End of file `sample-sigconf-authordraft.tex'.
