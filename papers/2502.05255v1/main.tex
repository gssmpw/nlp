\documentclass[12pt, twoside]{article}
\usepackage[utf8]{inputenc}
\usepackage{setspace}
\usepackage{fancyhdr}
\usepackage[margin=1in, headheight=15pt]{geometry}
\frenchspacing
\pagestyle{fancy}
\fancyhf{}
\fancyhead[OL]{\shorttitle}
\fancyhead[R]{\thepage}
\fancyhead[EL]{\surname}
\renewcommand{\headrulewidth}{0.5pt}
\usepackage[dvipsnames]{xcolor}
\usepackage{graphics}
\usepackage{tikz}
\usetikzlibrary{calc}
\usetikzlibrary{decorations.pathreplacing}
\usetikzlibrary{arrows.meta}
\tikzset{every picture/.style={/utils/exec={\sffamily}}}
\usepackage{amsmath,amsfonts,amsthm}
\usepackage{mathrsfs}
\setlength\parindent{20pt}
\usepackage{cancel}
\usepackage[normalem]{ulem}
\usepackage{siunitx}
\usepackage{float}
\usepackage{enumitem}
\usepackage{colortbl}
\usepackage{tabularx}
\usepackage{todonotes}
\newcolumntype{L}[2]{>{\hsize=#1\hsize\columncolor{#2}\raggedright\arraybackslash}X}%
\newcolumntype{R}[2]{>{\hsize=#1\hsize\columncolor{#2}\raggedleft\arraybackslash}X}%
\newcolumntype{C}[2]{>{\hsize=#1\hsize\columncolor{#2}\centering\arraybackslash}X}%
\usepackage{wrapfig}
\usepackage{transparent}
\usepackage{booktabs}
\usepackage{subcaption}
\usepackage[font=footnotesize,labelfont=bf]{caption}
\usepackage{titlesec}
\titleformat*{\section}{\large\bfseries}
\titleformat*{\subsection}{\normalsize\bfseries}
\usepackage[nottoc,numbib]{tocbibind}
\usepackage{tabularx}

\usepackage{natbib}
\setlength{\bibsep}{0.2pt}
\bibliographystyle{apsr}
\setcitestyle{aysep={}}
\newcommand\posscite[1]{\citeauthor{#1}\textcolor{gray}{'s} (\citeyear{#1})}
\usepackage{hyperref}
\renewcommand{\sectionautorefname}{Section}
\renewcommand{\subsectionautorefname}{Section}
\renewcommand{\subsubsectionautorefname}{Section}
\renewcommand{\appendixautorefname}{Section}
\hypersetup{colorlinks=true, citecolor=gray, linkcolor=gray, urlcolor=gray}
\makeatletter
\renewcommand*{\@fnsymbol}[1]{\ensuremath{\ifcase#1\or*\or\dagger\else\@arabic{\numexpr#1-2\relax}\fi}}
\makeatother


\newcommand{\mytitle}{Incivility and Contentiousness Spillover between COVID-19 and Climate Science Engagement}
\newcommand{\shorttitle}{Incivility and Contentiousness Spillover between COVID-19 and Climate Science}
\newcommand{\acknowledgements}{Corresponding authors: Hasti Narimanzadeh (hasti.narimanzadeh@aalto.fi), Arash Badie-Modiri (arash.badie-modiri@aalto.fi), and Ted Hsuan Yun Chen (ted.hsuanyun.chen@gmail.com). We thank Xun Cao, Ali Faqeeh, Boyoon Lee, Nanna Lauritz Sch\"onhage, Keith Smith, Tuomas Yl\"a-Anttila, and participants at the 2023 Environmental Politics and Governance Conference for providing help at various points in the project. We thank Altmetric for providing API access to its data through the \href{https://www.altmetric.com/our-audience/researchers/research-access/}{Researcher Data Access Program}. This research is funded in part by the Helsinki Institute for Social Sciences and Humanities and the Helsinki Institute of Sustainability Science. We also acknowledge the computational resources provided by the Aalto Science-IT project.}
\author{Hasti Narimanzadeh\thanks{Equal contribution ordered by reverse seniority.} \textsuperscript{,}\thanks{Department of Computer Science, Aalto University, Finland} \and Arash Badie-Modiri\footnotemark[2] \textsuperscript{,}\footnotemark[3] \textsuperscript{,}\thanks{Department of Network and Data Science, Central European University, Austria} \and Iuliia Smirnova\thanks{Faculty of Social Sciences, University of Helsinki, Finland} \and Ted Hsuan Yun Chen\thanks{Department of Environmental Science and Policy, George Mason University, USA}}
\newcommand{\surname}{Narimanzadeh \& Badie-Modiri et al.}
\date{\today}

\title{\mytitle\thanks{\acknowledgements}}
\begin{document}
\pagenumbering{roman}
\singlespacing

\maketitle\thispagestyle{empty}
\begin{abstract}

    \noindent Affective polarization and its accompanying cleavage-based sorting drives incivility and contentiousness around climate change and other science-related issues. Looking at the COVID-19 period, we study cross-domain spillover of incivility and contentiousness in public engagements with climate change and climate science on Twitter and Reddit. We find strong evidence of the signatures of affective polarization surrounding COVID-19 spilling into the climate change domain. Across different social media systems, COVID-19 content is associated with incivility and contentiousness in climate discussions. These patterns of increased antagonism were responsive to pandemic events that made the link between science and public policy more salient. We also show that the observed spillover activated along pre-pandemic political cleavages, specifically anti-internationalist populist beliefs, that linked climate policy opposition to vaccine hesitancy. Our findings highlight the dangers of entrenched cross-domain polarization manifesting as spillover of antagonistic behavior.

        \vspace{0.5cm}

    \noindent \textbf{Keywords:} affective polarization; climate change; COVID-19; social media; science communication
    
\end{abstract}

\newpage
\tableofcontents

\newpage
\onehalfspacing
\pagenumbering{arabic}
\setcounter{page}{1}


\section{Introduction}
There is a virtual consensus in the scientific community that climate change is caused by human activity \citep{myers2021consensus,cook2013quantifying}, and an overwhelming majority of scientists across disciplines agree that fundamental changes are needed to address the climate crisis \citep{dablander2024climate}.
Despite this, a non-trivial minority of the public remains firmly skeptical of climate science and opposed to stronger climate policies \citep{leiserowitz2023climate}. Notably, they are often vocal about their skepticism and opposition on social media platforms like Twitter \citep{falkenberg2022growing}. 

Evidence-based debates over findings are the cornerstone of the scientific endeavor, but unsubstantiated claims that run counter to the scientific consensus and questions about the integrity of scientists are linked to the delegitimization of science and lower trust for scientists among the public \citep{leiserowitz2013climategate}, as is the presence of disagreements and incivility in media coverage about science \citep{chinn2022can}. Just as important, research into the public's antagonistic engagement with science-related issues finds extensive evidence of a host of additional deleterious phenomena that have been associated with increased social and political polarization, including motivated reasoning \citep{druckman2019evidence}, partisan sorting \citep{chen2021polarization}, and the politicization of science \citep{druckman2017crisis}.

In this study, we explore public engagement with climate change and climate science on social media, focusing on how manifestations of affective polarization over the climate issue -- animosity toward members of the political outgroup \citep{iyengar2018strengthening} -- are affected by spillover from politically divisive events in other domains of science. We examine patterns of incivility and contentiousness, which are harmful outcomes of affective polarization \citep{mason2015disrespectfully}. Incivility, which ranges from incensed discussions to outright rude critiques and name-calling \citep{anderson2014nasty}, intensifies polarization \citep{brundidge2024clinching, chen2024disagreement}. It is associated with contributing to the dehumanization of outgroup members \citep{moore2020exaggerated} and the overall decrease in the quality of the information environment by, e.g., reducing individuals' trust \citep{mutz2005new} and willingness to engage \citep{ng2020toward}. Public contentiousness over scientific findings is further problematic in the context of climate change as it runs counter to the scientific community's overwhelming agreement regarding the anthropogenic sources of climate change. 



More generally, spillover across different societal issues is itself evidence of polarization, which has implications for science communication and policymaking, as it indicates social or political cleavages are becoming more entrenched \citep{baldassarri2008partisans, kozlowski2021issue}. To study spillover, which is both a mechanism and outcome of polarization, we look at the COVID-19 period, during which the severity of the pandemic increased the visibility and importance of communication between the general public and the scientific community. This period saw high levels of politicization and polarization toward COVID-19 \citep{green2020elusive, hart2020politicization}, which likely contributed to the diverging trust in science among different segments of the population \citep{radrizzani2023both, kerr2021political}. The impact of the COVID-19 shock therefore provides us an opportunity to examine important patterns in how the public engages with science across different issues, as defining and novel events from the pandemic can reveal evidence of spillover from COVID-19 to climate science discussions. 

We explore the following questions about how cross-domain spillover leads to greater polarization in how the public engages with science. How do large scale shocks to the role of science in society change how the public engages with scientific issues across different domains? Do negative aspects of COVID-19 science spill over to public engagement with climate science? What are the conceptual pathways through which these spillovers travel?

There is a growing tendency for issues to become linked through their association with partisan positions, whereby individuals' partisanship predict their attitudes toward issues, even to the extent that their positions toward different issues become predictive of one another \citep{kozlowski2021issue}. With the constant spotlight on the role of science in policymaking during the pandemic, we expect public tendencies surrounding COVID-19 to spill into how the public engages with climate science. Specifically, given the salience of polarized elite political cues surrounding COVID-19 \citep{green2020elusive}, we expect increased incivility in climate discussions that also involve COVID-19 topics. We also expect the combination of diverging trust toward the institution of science and the increasing inclination for individuals to do their ``own research'' \citep{chinn2023support} to lead to greater contentiousness in public debates about climate science issues. 

We approach our research questions by studying antagonistic public engagement with climate change-related content on social media, including those that explicitly engage with published climate science literature. We focus on Twitter, which, at least until 2022, was an important space for public contestations over social and political issues \citep{theocharis2015conceptualization}. Moreover, within the scientific community, there had been a generally shared notion that Twitter is the primary social media platform for high salience science communication \citep[][]{hammer2021social}, which makes it an interesting case for questions about the science-public interface \citep{walter2019scientific}. We also examine Reddit to supplement our findings. We show that the presence of COVID-19 topics is strongly predictive of climate change conversations taking a turn toward increased incivility and contentiousness, that these patterns respond to landmark events in the COVID-19 domain, and that incivility spillover tends to activate along existing sources of political polarization.


\section{Results}
Our examination focuses on references to COVID-19 in climate change discussions and dynamic trends in various facets of public climate change discussions during the 134-week period of February~2019--August~2021. We collected a set of 38.6 million tweets that contain keywords related to climate change. For 311,000 of these tweets, we further collected the entirety of their ensuing conversations. We also collected the same data for tweets directly referencing climate science publications through hyperlinks, resulting in 265,000 tweets and 42,000 conversations. 
Finally, we obtained all 2.1 million comments on Reddit that mention the phrase ``climate change'' during this period \citep{socialgrep2022reddit}. Across all three data sets, we observe a sharp decrease in climate-related posts at the onset of the pandemic, which corroborates existing findings on the ``finite pool of worry'' \citep{smirnov2022covid} and ``finite pool of attention'' hypotheses \citep{sisco2023examining}. 

We study the impact of cross-domain spillover between public science issues in two main ways. First, we consider how concepts from one domain are used in another by looking at patterns of incivility and contentiousness when COVID-19 enters the climate change narrative. By examining this intersection, we are able to uncover shifts in polarization that arise when global crises intertwine with wider scientific debates. Second, we look at how antagonistic communication patterns in climate change and climate science align with important COVID-19 events over time. 




\subsection{Incivility Spills Over between COVID-19 and Climate Change}\label{sec:toxicity-polarization}

To assess whether there is spillover between the two public science domains of COVID-19 and climate change, we compared the incivility of all pandemic period (March 20, 2020 onward) climate posts based on whether they contained COVID-19 content. We also looked at climate posts that specifically invoked Anthony Fauci, who, as then-Chief Medical Advisor to the President of the U.S., polarized COVID-19 discussions by being emblematic of how the institution of science impacts politics \citep{chen2023anti}. To measure incivility, we used toxicity detection from Perspective API \citep{jigsaw2017perspective}, which provides the probability a post is ``a rude, disrespectful, or unreasonable comment that is likely to make you leave a discussion.'' This operationalization maps directly onto the disengaging effects of incivility that we want to capture. From examples of tweets identified to be toxic (\autoref{tab:toxic-tweets}), we see that both COVID-19 and Fauci as topics capture polarized, vitriolic debates on the science involved.

\begin{table}[!tb]
    \centering\footnotesize\renewcommand{\arraystretch}{1.2}
    \begin{tabular}{p{0.08\textwidth} p{0.85\textwidth}} \toprule
$P$(toxic) & Tweet Text \\ \midrule
0.944 &
        Are people really that stupid? Do they really believe this crap? Climate change is a bunch of bullshit so is Dr. Fauci. It's just pathetic. And this figures just every now and then they're gonna make people lock down and do the mask bullshit again. Weak minded fucking sheep. [pouting-face emoji]\\

& [Link to a clip of Anthony Fauci's appearance on NBC during which the interviewer states that there will be more pandemics in the future because of climate change.]
\\ 





    0.654 & When you find out climate change scientists are all like Dr.~Fauci. Phoney flakes who don’t know shit from shinola. \\  


\bottomrule
    \end{tabular}
    \caption{Examples of uncivil climate tweets that reference Anthony Fauci. Tweet texts have been edited for typos and formatting.}\label{tab:toxic-tweets}
\end{table}


Across all three data sets -- general climate Twitter, climate science Twitter, and climate Reddit -- we find evidence of incivility spillover between climate change and COVID-19. Estimates from linear probability models of toxicity in climate posts show that posts invoking COVID-19 and Fauci have a significantly higher probability of being toxic, with the exception of climate science tweets referencing only COVID-19 (\autoref{fig:toxicity}.A).\footnote{Our results are robust to estimates from fractional logistic regression models \citep{papke2008panel}.} Among general climate tweets, these results hold when account fixed effects are included, meaning the observed pattern of COVID-19 posts being more toxic is not simply due to different individuals choosing different things to talk about, but is attributable to an individual-level behavioral tendency to be more toxic when discussing COVID-19.\footnote{We do not fit account fixed effects models for the scientific tweets data because there is not enough within-user variation in our predictors, nor for the Reddit data because it does not contain user information.}

Our results are robust to controlling for whether the post mentions international organizations or figures who are targets of conspiracy theories (a point we return to in \autoref{sec:international}); whether the post is the first post of a conversation, which has a higher chance of being a news or scientific article posted by an institutional account; and whether the posting account was active before the COVID-19 period because we expect Twitter to have attracted new users during the pandemic who are more uncivil than existing ones.

\begin{figure}[!t]
    \centering
    \includegraphics[width=1\textwidth]{climate_submission_figures/toxicity_coef_plots.pdf}
    \includegraphics[width=1\textwidth]{climate_submission_figures/toxicity_trend_day.pdf}
    \caption{Presence of COVID-19 content and incivility of climate posts. Panel A shows coefficient estimates from linear probability model of toxicity in climate posts across three social media systems. Panel B shows coefficient estimates from Cox proportional hazards models of toxicity onset in climate conversations on Twitter. Panel C shows temporal trends in daily climate tweet toxicity by difference in content. Panel~C's inset shows the temporal trends for climate science tweets and Reddit comments.}
    \label{fig:toxicity}
\end{figure}



Next, we leverage the sequential structure of conversations to show that general climate tweets that include references to COVID-19 tend to incite greater toxicity in their replies. From Cox proportional hazards models of toxicity onset in general climate and climate science conversations, we find that the number of replies until one exceeds 0.5 toxicity probability is significantly lower in general climate conversation that begin with a tweet mentioning COVID-19 (\autoref{fig:toxicity}.B). At any given time, replies to these conversations are 9\% more likely to devolve into incivility compared to conversations that do not begin with COVID-19 content. This pattern however does not hold for climate science conversations. Further, corroborating results from our post-level models, we find that toxicity onset in both general climate and climate science conversations are likely to be concurrent to mentions of COVID-19 or Fauci.


Finally, we find that the tendency for users to invoke COVID-19 in more toxic posts aligns with important pandemic events (\autoref{fig:toxicity}.C). Here, we estimated toxicity trends for climate posts with and without COVID-19 content in all three data sets using the BEAST time series method \citep{zhao2019detecting}, a trend and change point estimation method designed for noisy and cyclical data, such as social media patterns \citep{reuning2022facebook}. First, as in the linear models, the presence of COVID-19 content is strongly predictive of toxicity in general climate discussions on both Twitter and Reddit, but not in posts that explicitly link to climate science. Second, increases in the tendency for invoking COVID-19 in toxic climate discussions clearly correspond to times when the link between COVID-19 science and public policy became more salient, such as the lead up to the 2020 U.S. presidential elections and the onset of the Delta variant and its associated policy responses in early summer 2021. Third, COVID-19 climate tweets initially start at approximately 0.1 toxicity probability, which is the same as general climate tweets, but quickly spike within the first three months. This temporal pattern is indicative of partisan sorting, whereby new issues enter the public consciousness as nonpartisan but quickly become divisive and politically charged as a result of elite-led polarization, which was strong during the COVID-19 period \citep{green2020elusive}. 


\subsection{Climate Conversations Related to COVID-19 Are More Contentious}\label{sec:increasing-divisiveness}
\begin{figure}[!t]
    \centering
    \includegraphics[width= 0.33334\textwidth]{climate_submission_figures/disagreement_coef_plots.pdf}%
    \includegraphics[width= 0.66666\textwidth]{climate_submission_figures/disagreement_trend_week.pdf}
    \caption{Presence of COVID-19 content and probability of disagreement between climate conversation participants. Panel A shows coefficient estimates from linear probability models of disagreement in different types of climate conversations. Panel B shows temporal trends in weekly average climate conversation disagreement probability by whether the first tweet mentions COVID-19.}
    \label{fig:disagreement}
\end{figure}

Next, we consider how the contentiousness of Twitter conversations about climate change and those that specifically engaged with climate science publications varies by whether they include COVID-19 content. We measure contentiousness in conversations by looking at the level of disagreement between sequential pairs of messages in a conversation tree. To estimate message-pair disagreement, we fine-tuned the DistilBERT language representation machine learning model \citep{sanh2019distilbert} on a data set of annotated comment-reply pairs \citep{pougue2021debagreement} to return the estimated probability that a post is in disagreement with what it is responding to. 

On the whole, we find that climate conversations that engage with COVID-19 topics are significantly more contentious than those that do not. First, we fit linear probability models of response disagreement in all message-pairs to see whether linking climate change to COVID-19 resulted in higher levels of disagreement and whether individuals invoked COVID-19 to disagree about climate change (\autoref{fig:disagreement}.A). We find evidence for both in conversations about climate science publications, as COVID-19 content in root and response tweets were both associated with higher levels of response disagreement. \autoref{tab:contentious-convos} shows examples of both types of contentiousness spillover. In general climate conversations, we only find evidence for the use of COVID-19 content to disagree with others.

For our BEAST time series analysis (\autoref{fig:disagreement}.B), which focuses on whether the conversation is related to COVID-19 (measured by whether the root tweet contains COVID-19 keywords), we calculated conversation-level disagreement by averaging over all message-pairs in the conversation. We first find that disagreement with climate change posts immediately dropped with COVID-19 onset -- a pattern also observed with the incivility of climate posts -- which indirectly substantiates the ``finite pool of attention'' hypothesis that the pandemic drew the public's attention away from the climate issue \citep{sisco2023examining}. At the same time, however, climate conversations that relate to COVID-19 grew sharply in disagreement levels, and by the end of our observation period in August 2021, the level of disagreement in climate conversations had rebounded to above pre-pandemic highs. This pattern was even stronger in conversations directly engaging with climate science publications, with COVID-19 topics immediately exceeding pre-pandemic levels of disagreement.

\begin{table}[!tb]
    \centering\footnotesize\renewcommand{\arraystretch}{1.2}

    \begin{tabular}{p{0.11\textwidth} p{0.025\textwidth} p{0.78\textwidth}} \toprule
$P$(disagree) & \multicolumn{2}{p{0.82\textwidth}}{Tweet-Reply Pairs} \\ \midrule
\multicolumn{3}{p{0.9\textwidth}}{\textbf{Disagreement with climate tweet that references COVID-19}}\\
0.801 & \multicolumn{2}{p{0.82\textwidth}}{Excited collaboration with 43 authors from 27 institutes shows COVID-19 caused the largest decrease of CO2 emissions since 1900. Larger than that of WWII.}\\
& \multicolumn{2}{p{0.82\textwidth}}{[Link to a \textit{Nature Communications} article.]}  \\ 

& $\hookrightarrow$ & Let's not fool ourselves with these numbers. In Asia from January to September 2020 the average temperature was 2.30 degrees Celsius. And that's with a baseline of 1841. If you go back to a 1750 baseline, temperatures are even higher.\\\midrule

\multicolumn{3}{p{0.9\textwidth}}{\textbf{Invoking COVID-19 to disagree with climate tweet}}\\

0.918 & \multicolumn{2}{p{0.82\textwidth}}{China's iron \& steel industry is a major source of air pollution. In 2015 it brought in tighter standards. From 2014 to 2018 (only 4 years!): Particulate matter (PM) fell by 47\%. SO2 fell by 42\%. Change can happen quickly.}\\

& \multicolumn{2}{p{0.82\textwidth}}{[Link to a \textit{Nature Sustainability} article.]} \\ 

& $\hookrightarrow$ &Do not believe any of the CCP's ``figures''. You are being Wuhaned.
\\ 
\bottomrule
    \end{tabular}
    \caption{Examples of high disagreement responses in climate science conversations that reference COVID-19. Tweet texts have been edited for typos and formatting.}\label{tab:contentious-convos}
\end{table}


\subsection{COVID-19 Incivility Spillover Activates along Pre-existing Populist Beliefs}\label{sec:international}

We have thus far demonstrated that incivility and contentiousness spilled over between the COVID-19 and climate change domains. We now show evidence that this kind of spillover occurred in part because pre-existing populist anti-internationalist sentiments and conspiracy theories provide links between climate policy opposition and COVID-19 vaccine hesitancy.

Since the 21\textsuperscript{st} century populist surge \citep{mudde2019far}, there has been a trend of decreasing support for international organizations and globalist policies in most countries, a sentiment manifesting most strongly among groups adversely affected by a globalized economy \citep{bearce2019popular}. Specific to climate policy, with the growing political salience of the climate issue, far-right populist parties have started taking an anti-internationalist opposition toward international climate governance \citep{lockwood2018right,schworer2024climate}. At the extreme, these sentiments devolve into conspiracy theories about international elites trying to exert global control \citep{castanho2017elite} -- such as the example in \autoref{tab:org-tweets} about George Soros and the United Nations.

Similar patterns exist in the public health domain, where conspiracy beliefs, including those about international elites, tend to align with anti-vaccination attitudes \citep{hornsey2018psychological}. While these were fringe positions before the COVID-19 period, they became more mainstream during the pandemic. On social media, anti-vaccination and anti-internationalist conspiracy theories became rampant \citep{darius2021disinformed, erokhin2022covid}. Offline, anti-vaccine individuals in the U.K. often simultaneously held pseudo-scientific, populist, and conspiracy beliefs \citep{holford2023psychological}, and a sizable segment of the U.S. population negatively viewed the World Health Organization~(WHO) during the pandemic \citep{bayram2021trusts}.

Did the COVID-19 pandemic elevate existing populist sentiments against climate science and policies? Based on prior research showing that cognitively coherent beliefs reinforce one another \citep{taber2006motivated}, we expect overlapping antagonistic populist beliefs toward international elites and scientists in the realms of public health and climate change to have served as a pathway for spillover between the two global crises, leading to greater antagonism toward international organizations in climate discussions during the pandemic.

\begin{table}[!t]
    \centering\footnotesize\renewcommand{\arraystretch}{1.2}
    \begin{tabular}{p{0.48\textwidth} p{0.48\textwidth}} \toprule
        Pre-COVID &         Post-COVID \\
        \midrule
        Climate change ``scientists'' are not stupid. They are on the biggest ever gravy train. Politicians are the stupid ones for allowing them and UN bodies to rip us off. All part of the Soros funded NWO [New World Order].
        &
        We won't be in deep shit because of non-existent climate change. We'll be in deep shit because dumb fucks like these people are in charge, and are pushing a hoax for more total control. Klaus Schwab%
        , Fauci, Gates, etc.
        \\ \bottomrule

    \end{tabular}
    \caption{Examples of climate tweets with high probability of toxicity (above 0.75) that mention international organizations before and after the onset of COVID-19 pandemic. Tweet texts have been edited for typos and formatting.}
    \label{tab:org-tweets}
\end{table}

To explore this possibility, we examined the temporal dynamics, before and after the pandemic's onset, of toxicity differences between climate tweets that referenced international organizations and those that do not.
Because anti-internationalist sentiments lead to greater incivility in climate discussions, if the anti-internationalist sentiments activated by COVID-19 spilled into the climate domain, there should be greater toxicity in climate change discussions about international organizations compared to the baseline, even in tweets not directly concerning the pandemic. As the onset of the pandemic was exogenous to climate change in the short term, observed temporal differences can be attributed to the COVID-19 shock.

\begin{figure}[!t]
    \centering
    \includegraphics[width=0.5\textwidth]{climate_submission_figures/io_seconddiff.pdf}
    \caption{Quarterly difference in toxicity between tweets containing references to international organizations and those that do not, compared to the baseline quarter (2019 Q4). The lighter confidence intervals are computed from standard errors clustered at the Twitter account level.
    }
    \label{fig:io-toxicity}
\end{figure}

Looking at second differences in a linear probability model of tweet toxicity \citep{mize2019best}, we find a significant increase in toxicity toward international organizations in climate discussions during the COVID-19 period starting Q1 2020, when compared to the baseline period of Q4 2019 (\autoref{fig:io-toxicity}). The difference in toxicity probability before the pandemic, on the other hand, was relatively stable. The highest spike occurred in Q2 2020, immediately following the WHO declaring COVID-19 to be a global pandemic, and not until Q3 2021 did the difference fall to pre-pandemic levels. These findings provide clear evidence that the nature of climate discussions about international organizations changed with the pandemic.


\section{Discussion}

While the COVID-19 pandemic and its public salience has subsided, legacies of pandemic era politics continue today. Science being contested as part of the political process, which was apparent with the political vitriol surrounding Anthony Fauci, continues with politically-motivated appointments to science-related posts following government transitions \citep{tanne2024trump}. Trust in scientists remains relatively low at levels similar to the pandemic period \citep{pew2024public}, and recent research shows worrying signs of behavioral spillover between COVID-19 and post-pandemic vaccination attitudes \citep{lunz2024covid}. With climate change remaining ever as important, it is imperative to better understand how spillover among concurrent crises might lead to system-wide entrenchment of affectively polarized attitudes.

In this study, we showed that affective polarization surrounding COVID-19 spilled over to how the public engages with climate change and climate science on social media. References to COVID-19, especially those surrounding Anthony Fauci, were strongly associated with incivility and contentiousness in climate change discussions. With some exceptions, COVID-19 content both invited and were used for antagonistic engagement with the climate issue. These relationships were responsive to landmark pandemic era events, as incivility and contentiousness increased when the role of science in policymaking was made more salient by COVID-19 policies. We further found evidence that this spillover occurred in part through existing sources of affective polarization. Populist sentiments against international organizations and associated elites provided a link between climate policy opposition and COVID-19 vaccine hesitancy, which manifested in higher incivility of climate discussions even when COVID-19 was not referenced.

By demonstrating the existence of affective polarization spillover and how it activated along existing political cleavages, our study adds to the literature on how COVID-19 impacted the public's engagement with the climate issue, which has thus far focused on how concurrent crises can lead to shifts in their relative issue saliency \citep{smirnov2022covid,sisco2023examining} and the potential for policymaking synergies between them \citep{bergquist2023politics}. More broadly beyond COVID-19 and climate change specifically, our findings show that conceptual synergies can lead to harmful antagonistic behavior spilling over across different domains, especially for public policy issues that are strongly informed by science, such as public health and environmental governance, highlighting the dangers of entrenched affective polarization on how the public engages with science.


\section{Data and Methods}
All code and data required for reproducing our results will be made publicly available at \url{https://doi.org/10.5281/zenodo.14834207}.

\subsection{Data Collection}
We used several data sets of online social media posts published between February 1, 2019 and August 26, 2021, spanning a total of 134 weeks. We subsetted all posts to English ones using the language estimate returned by Perspective API. We collected all tweet data using the full archive search endpoint from Twitter’s v2 API suite during October 2021--June 2022. Additional Twitter data collection details are in \autoref{app:twitter-collect}.

\paragraph{General climate tweets} 38.6 million tweets from 5.6 million unique users published in the data collection window containing at least one climate change related keyword or hashtag.

\paragraph{General climate conversations} Full reply tree of 311,000 conversations (stratified random sample of approximately 2300 per week) from the \emph{General climate tweets} data set that have at least one reply and two unique participants.

\paragraph{Climate scientific tweets} 265,000 tweets from 81,000 unique users containing a reference (hyperlink) to a scientific publication containing climate science related keywords in the publication title. This data set was curated using Altmetric's API made available through its Researcher Data Access Program.

\paragraph{Climate scientific conversations} Full reply tree of all 42,000 conversations from the \emph{Climate scientific tweets} data set that have at least one reply and two unique participants.


\paragraph{Reddit climate change comments} 2.1 million Reddit comments mentioning the keyword ``climate change''. This data set was provided by \citet{socialgrep2022reddit}. Preprocessing details are in \autoref{app:reddit-collect}.

\subsection{Measurement}
\subsubsection{Topic Classification}  Based on the presence or absence of specific keywords, all posts were assigned binary labels for each of the following categories: 1) For COVID-19, we used keywords and hashtags generally used to refer to the virus or pandemic and its interventions (e.g., vaccines, social distancing); 2) for Anthony Fauci, we only used his surname, as it is distinct enough to capture relevant references to him and discriminate against false positives; 3) for international organizations, we used a list of keywords that include international entities (i.e., international organizations and public figures). The full keyword lists are in \autoref{app:topic-keywords}.


\subsubsection{Toxicity} Our climate posts were labeled with a probability of exhibiting toxicity by the Perspective API provided by Google \citep{wulczyn2017ex,jigsaw2017perspective}. The model defines toxicity to be ``a rude, disrespectful, or unreasonable comment that is likely to make you leave a discussion'' \citep{mitchell2019model}. Based on the Unified Toxic Content Classification multilingual architecture \citep{lees2022new}, this model is trained on data from online forums such as comments from Wikipedia talk pages \citep{wulczyn2017ex} and New York Times, labeled by three to ten crowdsourced human annotators. Each training sample is labeled with a toxicity probability based on the percentage of annotators identifying the text to be toxic.

\subsubsection{Disagreement} To understand the level of disagreement among participants, we fine-tuned a 3-class classifier based on the DistilBERT language representation model \citep{sanh2019distilbert} using the DEBAGREEMENT comment-reply agreement and disagreement data set \citep{pougue2021debagreement}, a set of 43 thousand comment-reply pairs from political and climate subreddits, each labeled as either disagreement, neutral, or agreement. Because we are only concerned with the presence of disagreement, constructing a binary classifier based on the three class probabilities results in $f_1$ scores of 0.806 and 0.716 for disagreement and lack of disagreement (i.e.,~neutral or agreement) respectively. Performance details of our fine-tuned DistilBERT model are in \autoref{app:distilbert}.

We used this model with pairs of tweet-replies from our Twitter conversation data, where consecutive tweets from the same author were combined to form a single post, as they are artifacts of Twitter's maximum character limit.


\subsection{Linear Probability Model and Fractional Logistic Regression Model}

We fitted all models in \texttt{R} \citep{r2024} with the \texttt{fixest} package \citep{berge2018efficient}. Confidence intervals were obtained using the \texttt{marginaleffects} package \citep{arelbundock2024how}.

\paragraph{Climate Post Toxicity}
We fitted separate linear probability models of COVID-19 period (March 20, 2020--August 26, 2021) climate post toxicity for all three of our data sets -- general climate tweets, climate science tweets, and climate Reddit posts. In all three models, we included day fixed effects. We additionally included account fixed effects for the general climate tweets data set in a separate model. In all models, we clustered the standard errors at the day level. Fitting all models using fractional logistic regression \citep{papke2008panel} yielded substantively similar results, which we show in \autoref{app:fraclogit}.

\paragraph{Climate Conversation Disagreement}
We fitted separate linear probability models of COVID-19 period response disagreement for both of our climate Twitter data sets. In both models, we included day fixed effects, and clustered the standard errors at the day and conversation levels. Results from fractional logistic regression models are substantively similar, which we show in \autoref{app:fraclogit}.

\paragraph{Climate Post Toxicity across Time and International Organization Content}
We fitted a model for tweet toxicity using our general climate Twitter data set for our entire observation period (February 1, 2019--August 26, 2021). In the model, we interacted the international organization keyword term with every quarter in the data (Q1 2019--Q3 2021), then compared the coefficient of the international organization term from each quarter to the pre-pandemic baseline (Q4 2019). We computed standard errors both with and without account clustering. To ensure the increased incivility during the pandemic is not directly due to COVID-19 tweets, we removed them from this analysis, but our results are robust to whether these tweets are included, which we show in \autoref{app:io-robust}.

\subsection{Cox Proportional Hazards Model}
To understand the effect of mentions of COVID-19 and other related topics on the onset of toxicity in the course of climate conversations, we constructed an ensemble of Cox proportional hazard models in \texttt{R} \citep{r2024} using the \texttt{survival} package \citep{therneau2024survival-package}. Specifically, we modeled the first occurrence of a toxic reply in a Twitter thread, or consecutive chain of tweets and replies, from the COVID-19 period. The analysis is based on a resampling method. In each realization, one thread from each conversation tree is selected at random to ensure tweets closer to the root of the conversation are not over-represented in the analysis, as they appear in more threads than tweets closer to the leaves.

In our analysis we considered occurrences of COVID-19 keywords and international organizations keywords on the root tweet as a property of the conversation, and occurrences these keywords, along with Fauci keywords, in subsequent replies were treated as time-dependent covariates. We did not include mentions of Fauci in the root tweet of conversations because this was an extremely rare occurrence, especially in the scientific data set. 

Full details of how we implemented the Cox proportional hazards model are in \autoref{app:coxph}. We show that our results are robust to alternative specifications of what constitutes an English conversation in \autoref{app:coxph-robust}.


\subsection{Time Series Analysis}
To analyze temporal trends in toxicity and disagreement, we used the Bayesian Estimator of Abrupt change, Seasonal change, and Trend (BEAST) model \citep{zhao2019detecting}, which is implemented in in \texttt{R} \citep{r2024} as the \texttt{Rbeast} package.

BEAST is a time-series decomposition algorithm designed to analyze nonlinear temporal dynamics across multiple timescales while taking into account seasonal behavior (e.g.,~weekly trends) without manually specifying the model. It does so by simultaneously estimating the temporal trends and cyclical patterns that make up an observed time series, and the change points in both of these dynamic processes. 
Specifically, it decomposes the observed time series into a trend that is linear between change points (i.e., a series of piecewise linear trends) and a cyclical pattern that follows a harmonic function whose parameters are constant between change points. 
Instead of opting for a single ``best'' model, BEAST uses Bayesian model averaging of multiple competing models to obtain the point and uncertainty estimates of model parameters -- the trend, the harmonic function's parameters, and the number and location of change points.

\paragraph{Toxicity Time Series}
In our toxicity analysis, our time series are at the day level. We specified the cyclical pattern to be every seven days, given our expectations about Twitter activity patterns (e.g., there to be less activity on the weekends). When estimating change points, we specified there to be a maximum of fifteen change points (which was never reached), and that any two change points must be at least four weeks apart.

\paragraph{Disagreement Time Series}
In our disagreement analysis, our time series are at the week level because we had fewer conversations than we had posts. Given our observations are already aggregated to the week level, we specified no cyclical pattern. When estimating change points, we specified there to be a maximum of fifteen change points (which was never reached), and that any two change points must be at least four weeks apart.

 




\clearpage
\singlespacing\footnotesize
\section*{CRediT Author Statement} 
\textbf{Hasti Narimanzadeh:} Conceptualization, Methodology, Software, Validation, Formal Analysis, Investigation, Data Curation, Writing - Original Draft. 
\textbf{Arash Badie-Modiri:} Conceptualization, Methodology, Software, Validation, Formal Analysis, Investigation, Data Curation, Writing - Original Draft, Visualization. 
\textbf{Iuliia Smirnova:} Conceptualization, Validation, Investigation, Data Curation. 
\textbf{Ted Hsuan Yun Chen:} Conceptualization, Methodology, Validation, Formal Analysis, Investigation, Writing - Original Draft, Visualization, Supervision, Funding Acquisition.

\bibliography{citations}


\clearpage
\appendix
\normalsize
\onehalfspacing
\newcounter{sisection}
\setcounter{sisection}{1}

\renewcommand{\thesection}{S{\arabic{sisection}}}
\renewcommand{\thefigure}{\thesection.\arabic{figure}}
\renewcommand{\thetable}{\thesection.\arabic{table}}


\counterwithin{figure}{section}
\counterwithin{table}{section}

\section{Twitter Data Collection}\label{app:twitter-collect}
We used Twitter's Academic API to collect our Twitter data. To collect the general climate tweets, we queried the API for tweets with hashtags and keywords from \autoref{tab:keyword-categories} for the February 1, 2019--August 26, 2021 period (missed data from February 4, 2021). We subsetted the data to only English tweets using Perspective API's language estimate. We also excluded tweets containing ``political climate'' and ``Percip:''. The latter is frequently used by bots that automatically tweet regional weather forecast. This yielded 38.6 million tweets.

To identify all tweets that reference (through hyperlinks) climate science publications, we first used Altmetric's Explorer API to filter scientific publications by whether their titles contained one of our keywords (\autoref{tab:keyword-categories}), then used Altmetric's Details Page API to obtain IDs of all tweets that referenced these publications. We used Twitter's Academic API to collect these tweets, which we subset to only English ones. This yielded 265,000 tweets.

\begin{table}[!htb]
    \centering\renewcommand{\arraystretch}{1.5}\footnotesize
    \begin{tabularx}{\textwidth}{r X} \hline\hline
        Type & String  \\ \hline

        Hashtags & climatehoax, globalwarming, climateneutrality, climatecrisis, climatebrawl, climaterisk, chooseforward, climateemergency, climatestrike, climatechange, climatefriday, climatescience, actonclimate, climatehysteria, climatestrikeonline, climatetwitter, climatejustice, climate, fridaysforfuture, fridays4future, schoolstrike4climate, facetheclimateemergency, climatetwitter, climatetech, globalwarminghoax, gretathunberg, parisagreement, nomoreemptypromises \\
        Keywords & climate, global warming, greenhouse gas, greenhouse emission, paris agreement \\
        \hline
    \end{tabularx}
    \caption{Hashtags and keywords used for filtering climate change tweets.}
    \label{tab:collection}
\end{table}

We then used Twitter's API to collect the conversation trees stemming from our climate and climate science tweets. Specifically, for all root tweets that have at least one reply from a different user, we obtained all replies to the root tweet and all replies to replies, infinitely deep. For the general climate change data, we collected a stratified random sample of approximately 2300 conversations per week, and the entire set of climate science conversations.

\autoref{tab:dataset-users} shows the distribution of the number of users that were active before and after the COVID-19 onset.





\begin{table}[htb!]
\caption{User counts by their activity before and during the pandemic.}\label{tab:dataset-users}
    \centering\footnotesize\renewcommand{\arraystretch}{1.5}
    \begin{tabular}{r | c c}\hline\hline
        User Subset & Climate General & Climate Science\\ 
        \hline
        Only before the pandemic & 2.0M & 28K \\
        Only during the pandemic & 2.2M & 38K \\
        During the entire period & 1.3M & 14K \\ \hline
        Total users & 5.6M & 81K \\ \hline
    \end{tabular}
\end{table}












\clearpage
\stepcounter{sisection}
\section{Reddit Data Collection}\label{app:reddit-collect}
For our Reddit analysis, we used the Reddit Climate Change Dataset provided by \citet{socialgrep2022reddit}. This data set consists of approximately 4.6 million Reddit comments posted before September 2022 that contain ``climate change''. We removed comments posted outside our time window and non-English comments. We also removed comments containing phrases indicating that they were written by bots. Specifically, we filtered the comments on a set of common phrases used by bots on Reddit to perform automatic tasks, e.g.~``I am a bot'' or ``This action was performed automatically'' (\autoref{tab:reddit-bot-phrases}), as well as excluding all comments from the subredditsummarybot because it mostly consist of automatic submissions. Our final Reddit data set contains 2.1 million comments.




\begin{table}[!htb]
    \centering\footnotesize\renewcommand{\arraystretch}{1.5}
    \caption{Phrases used for filtering automatic Reddit comments.}
    \label{tab:reddit-bot-phrases}
    \begin{tabular}{l}\hline\hline
         Filter phrase  \\
         \hline
         I am a bot \\
         I'm a bot \\
         This action was performed automatically \\
         This message was posted by a bot. \\
         This book has been suggested \\
         This comment was left automatically (by a bot). \\
         You can summon this bot any time in \\
         I detect haikus. And sometimes, successfully. \\ \hline
    \end{tabular}
\end{table}

\clearpage
\stepcounter{sisection}
\section{Topic Classification Keywords}\label{app:topic-keywords}

Based on the presence or absence of specific keywords, we labeled all posts as containing content from the following categories: 1) COVID-19, 2) Anthony Fauci, and 3) international organizations. \autoref{tab:keyword-categories} contains our keywords.


To lower the probability of false positives, Twitter handles and the keyword ``gates'' were only considered if surrounded by word boundaries. Keywords shown in all caps were searched case sensitively and surrounded with word boundaries. The keyword ``WHO'' was specifically searched case sensitively and surrounded with word boundaries, and only if less than 50\% of the tweet was using Latin alphabet, or if at least 70\% of those Latin alphabet letters were uppercase. This condition was set to avoid tweets simply containing the interrogative ``who'' written in all caps, while still allowing tweets from languages that do not use the Latin alphabet. All other keywords were searched case insensitively while disregarding word boundaries, allowing us to capture them in hashtags or with affixes.

\begin{table}[!htb]
    \centering\renewcommand{\arraystretch}{1.5}\footnotesize
    \begin{tabularx}{\textwidth}{r X} \hline\hline
        Category & Keywords  \\ \hline

        COVID-19 & coronavirus, corona virus, covid,
        covid19, covid-19, mask, pandemic,
        lockdown, wuhan, sars-cov-2, sarscov2,
        flatten the curve, flattening the curve, flatteningthecurve,
        flattenthecurve, hand sanitizer, handsanitizer,
        social distancing, socialdistancing, work from home,
        workfromhome, working from home, workingfromhome,
        mrna, vax, vaccin, pfizer,
        the jab, jabbed, jabbing,
        moderna, astrazeneca, biontech, sinovac,
        sinopharm, johnson \& johnson, johnson\&johnson \\
        Anthony Fauci & fauci\\

        International Organizations & WHO, @who,
        w.h.o, world health org, WEF, @wef,
        w.e.f, world economic forum, Davos,
        UN, @un, u.n, united nations,
        gates, billgates, @billgates, @gatesfoundation,
        soros, @georgesoros \\
        \hline
    \end{tabularx}
    \caption{Keywords used for determining mentions of different topics in tweet texts}
    \label{tab:keyword-categories}
\end{table}




\clearpage
\stepcounter{sisection}
\section{DistilBERT Model of Disagreement}\label{app:distilbert}
To measure contentiousness, we fine-tuned a 3-class classifier of message-pair disagreement based on the DistilBERT language representation model \citep{sanh2019distilbert} with manually-labeled comment and replies from the BlackLivesMatter, Brexit, climate, democrats, and Republican subreddits \citep{pougue2021debagreement}.

Our fine-tuned model labels a comment-reply pair as either disagreement, neutral or agreement. \autoref{tab:distilbert-performance} shows the performance statistics of the model measured on a random 10\% test split of the DEBAGREEMENT data that was not included in the training. The fine-tuned DistilBERT model achieved an overall accuracy of 66.5\% on the test set. 
Because we are only concerned capturing the presence of disagreement, we constructed a binary classifier based on the three class probabilities results by combining neutral and agreement labels. This returned $f_1$ scores of 0.806 and 0.716 for disagreement and lack of disagreement (i.e.,~neutral or agreement) respectively.


\begin{table}[ht!]
    \centering\footnotesize\renewcommand{\arraystretch}{1.5}
    \caption{Performance characteristics of the 3-class disagreement classification model.}
    \begin{tabular}{r | c c c c} \hline\hline
        Label        & Precision & Recall & $f_1$ score & Support \\\hline
        Disagreement & 0.707     & 0.726  & 0.716       & 1717    \\
        Neutral      & 0.539     & 0.505  & 0.522       & 1109    \\
        Agreement    & 0.689     & 0.701  & 0.695       & 1464    \\ \hline
    \end{tabular}
    \label{tab:distilbert-performance}
\end{table}




 
 








\clearpage
\stepcounter{sisection}
\section{Cox Proportional Hazards Model Details}\label{app:coxph}
We modelled the time to onset of toxicity in Twitter climate conversations using the Cox proportional hazards model, focusing on whether this varied by the COVID-19 variables we are interested in.

\subsection{Model Specification}
We analyzed the general climate conversations and climate science conversations separately. For both sets of analysis, we used all English conversations March 20, 2020 onward from the respective data sets. 

In the models, we included a mixture of both thread-level covariates, i.e., occurrences of certain keywords in the root tweet, and reply depth-level covariates, i.e., occurrences of these keywords in reply tweets. Specifically, we included covariates for whether the root tweet contains COVID-19 keywords and international organization keywords, and covariates for whether specific replies contain COVID-19, Fauci, and international organization keywords.

Finally, to prevent violations of the proportional hazards assumption from these models, we stratified our models by whether the conversation's root tweet was already toxic.

\subsection{Data Construction}
In its raw form, a conversation is composed of replies to the root tweet, and replies to these replies, up to an infinite depth. At any point when there are multiple replies to the same tweet, the reply tree branches into different threads. To prepare this conversation data for the Cox proportion hazards model, we take the following steps.

\begin{itemize}[noitemsep]
    \item Each thread, or branch of the conversation tree defined by its unique terminal tweet, were treated as separate entities that we observe over time.
    \item Threads started by the automated account @wikipediachain, which document random walks through Wikipedia links, were removed.
    \item Tweets with toxicity probability that are not able to be estimated by Perspective API\footnote{Tweets in languages not supported by the Perspective API (i.e., those outside of Arabic, Chinese, Czech, Dutch, English, French, German, Hindi (including when written with Latin alphabet), Indonesian, Italian, Japanese, Korean, Polish, Portuguese, Russian, Spanish and Swedish), and tweets containing only non-textual material (e.g., images, videos, or links). Perspective API returned NA values for 3.4\% of general climate tweets and 6.5\% of climate science tweets.} and all subsequent replies in its thread are removed, because we are unable to assess how they affect subsequent replies.
    \item Consecutive tweets from the same author, which generally appear because of Twitter character limits, were combined into a single tweet.
    \item Each tweet was classified as toxic if its toxicity probability was at least 0.5.
    \item For each thread, tweets after the first occurrence of a toxic tweet were dropped, because we are modelling toxicity onset.
    \item All root tweets were removed, because we are modelling reply toxicity.
\end{itemize}

These steps yielded a data set with thread-reply depth observations. Each observation has thread-level covariates (e.g., COVID-19 content in root tweet) and reply depth-level, i.e., ``time-varying'', covariates (e.g., COVID-19 content in reply tweet).



\subsection{Estimation}
As threads from the same conversation are not independent from one another due to sharing tweets higher up in the conversation (i.e., closer to the root tweet), we based our analysis on a resampling method. In each realization, we randomly sampled one thread from each conversation. This means that no two threads in the same realization come from the same conversation, and therefore do not shared any tweets. This resampling eliminated the dependence between chains from the same conversation, and also ensured that the proportional hazard assumption is not violated due to presence of extremely large conversation.

Using the \texttt{survival} package \citep{therneau2024survival-package} in \texttt{R} \citep{r2024}, we fitted 20,000 resampled realizations of our data. In our results, we report the mean coefficient estimates as the point estimate and the 2.5\textsuperscript{th} and 97.5\textsuperscript{th} percentile values as the confidence interval.



\subsection{Proportional Hazards Assumption Diagnostics}
We assessed whether our models satisfy the proportional hazards assumption using the Schoenfeld test \citep{therneau2000modeling}. We conducted the test for all fitted models and report the proportion of realizations that pass the test in \autoref{tab:schoenfeld}.

\begin{table}[ht!]
    \centering\footnotesize\renewcommand{\arraystretch}{1.2}
    \caption{The proportion of bootstraps satisfying Cox proportional hazard model assumptions by not failing the Schoenfeld test with $p > 0.05$.}\label{tab:schoenfeld}
    \begin{tabular}{r | c c} \hline\hline
    & \multicolumn{2}{c}{Proportion of Bootstraps Not Failing Schoenfeld Test} \\ 
            & All Tweets & Root Scientific Tweets  \\
        \hline
        COVID-19 root           & 0.99 & 0.99 \\
        COVID-19 keywords       & 0.76 & 1.00 \\
        Fauci keywords          & 0.99 & 1.00 \\
        International root      & 0.98 & 1.00 \\
        International keywords  & 1.00 & 0.95 \\
        \hline
    \end{tabular}
\end{table}

\clearpage
\stepcounter{sisection}
\section{Fractional Logistic Regression for Toxicity and Disagreement}\label{app:fraclogit}
Linear probability models often provide an adequate approximation of non-linear models, especially when the range of outcome probabilities are not extreme. Still, we fit additional fractional logistic regression models \citep{papke2008panel} of toxicity probability and disagreement probability with the same model specifications as what we used for our main analysis (\autoref{fig:robust-fraclogit}) to show that our results are robust to accounting for the bounded nature of probabilities.

\begin{figure}[!htb]
    \centering
    \includegraphics[width=\linewidth]{climate_submission_figures/fraclogit_coef_plots.pdf}
    \caption{Robustness check of main results using fractional logistic regression models. Panel A shows the average marginal effect estimates for toxicity probability, corresponding to results from \autoref{fig:toxicity}.A. Panel B shows the average marginal effect estimates for disagreement probability, corresponding to \autoref{fig:disagreement}.A.}
    \label{fig:robust-fraclogit}
\end{figure}

\clearpage
\stepcounter{sisection}
\section{Cox Proportion Hazards Model without Subsetting to English Replies}\label{app:coxph-robust}

Our standard approach to data preprocessing is to subset to only English posts. For our Twitter conversation analyses using the Cox proportional hazards model, this means removing the first appearance of a non-English tweet in a thread then truncating the rest of the thread, as we cannot assess how the non-English replies affected subsequent replies. We show that our findings regarding COVID-19 content in conversations are robust to an alternative specification, where we subset to conversations beginning with an English root tweet, do not limit replies to only English ones. 

\begin{figure}[!htb]
    \centering
    \includegraphics[width = 0.5\textwidth]{climate_submission_figures/coxph_en_convos.pdf}
    \caption{Robustness check of main Cox proportional hazards model results using full replies instead of only English ones, corresponding to \autoref{fig:toxicity}.B.}
    \label{fig:coxph-robust}
\end{figure}

\clearpage
\stepcounter{sisection}
\section{COVID-19 Tweets in the Anti-internationalist Analysis}\label{app:io-robust}
When assessing the impact of COVID-19 on the relationship between anti-internationalist sentiment and toxicity in climate discussions, we examined only tweets that do not directly engage with COVID-19 to ensure our results are not driven purely by the toxicity of COVID-19 content. Still, we show that our results are robust to the inclusion of COVID-19 tweets (\autoref{fig:io-toxicity-covid}).

\begin{figure}[!htb]
    \centering
    \includegraphics[width = 0.5\textwidth]{climate_submission_figures/io_seconddiff_covid.pdf}
    \caption{Robustness check of toxicity probability of international organization tweets over time when including tweets with overlapping COVID-19 content, corresponding to \autoref{fig:io-toxicity}.}
    \label{fig:io-toxicity-covid}
\end{figure}


\end{document}
