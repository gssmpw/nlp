%\thispagestyle{empty}


\section{The Proposed Lorecast Methodology}
\label{sec:Methods}


\subsection{Overview}

The goal of Lorecast is to take natural language prompts as input and produce performance and power forecasts for the circuit corresponding to the prompts. An overview of the Lorecast methodology is provided in Figure~\ref{fig:workfolw1}. %We use a natural language description of the design as input and obtain the predicted performance and power results of the described design as output.
 It consists of two phases. 
 Phase I is LLM-based Verilog code generation
 and Phase II is performance/power forecasting according to the generated Verilog code.
Although LLM-based Verilog code generation has been studied in prior research, our approach differs significantly in a crucial aspect: functional correctness is far less critical in our case. In conventional approaches~\cite{liu2023VerilogEval,lu2024RTLLM}, functional correctness is essential because the generated Verilog code is typically intended for synthesis. By contrast, in our methodology, functional errors usually have very small impact to performance and power forecasting.
Functional correctness remains a significant challenge for LLM-based Verilog generation and is far from being well solved. Our innovative use of LLM-based Verilog generation largely bypasses this challenge. As such, we can focus on syntax correctness, which is much more achievable. Additionally, using Verilog code as an intermediate representation enables significantly better forecasting accuracy compared to direct performance and power predictions using LLMs.
 
 
 %In the first phase, we generate Verilog code using our LLM-based self-correction code generation framework. In the second phase, we apply a machine learning model to make predictions based on the generated Verilog code.

\iffalse
\begin{figure*}[!htp]
    \centering
    \includegraphics[width=1\linewidth]{figure/workfolw3.pdf}
    \caption{Overview of the Proposed Framework.}
    \label{workfolw1}
\end{figure*}
\begin{figure*}[h]
    \centering
    \includegraphics[width=1\linewidth]{figure/workfolw4.pdf}
    \caption{Overview of the Proposed Framework.}
    \label{workfolw1}
\end{figure*}
\fi

% \begin{figure*}[hbt]
%     \centering
%     \includegraphics[width=1\linewidth]{figure/workfolw8.pdf}
%     \caption{Overview of the Proposed Framework.}
%     \label{fig:workfolw1}
% \end{figure*}
