% This must be in the first five lines to tell arXiv to use pdfLaTeX, which is strongly recommended.
\pdfoutput=1
% In particular, the hyperref package requires pdfLaTeX in order to break URLs across lines.

\documentclass[11pt]{article}

% Change "review" to "final" to generate the final (sometimes called camera-ready) version.
% Change to "preprint" to generate a non-anonymous version with page numbers.
% \usepackage[review]{acl}
\usepackage[preprint]{acl}


% Standard package includes
\usepackage{times}
\usepackage{latexsym}
\usepackage{amsmath}
\usepackage{adjustbox}
\usepackage{enumitem}



% For proper rendering and hyphenation of words containing Latin characters (including in bib files)
\usepackage[T1]{fontenc}
% For Vietnamese characters
% \usepackage[T5]{fontenc}
% See https://www.latex-project.org/help/documentation/encguide.pdf for other character sets

% This assumes your files are encoded as UTF8
\usepackage[utf8]{inputenc}

% This is not strictly necessary, and may be commented out,
% but it will improve the layout of the manuscript,
% and will typically save some space.
\usepackage{microtype}

% This is also not strictly necessary, and may be commented out.
% However, it will improve the aesthetics of text in
% the typewriter font.
\usepackage{inconsolata}

%Including images in your LaTeX document requires adding
%additional package(s)
\usepackage{graphicx}
% ------------ Self-imported packages --------------
\usepackage{booktabs} % for \toprule \bottomrule
\usepackage{tabularx} % for tables
\usepackage{multirow} % for \multirow
\usepackage{hyperref}
\usepackage{caption}
\usepackage{xcolor} % for text background color
\usepackage{soul}
\usepackage{subfigure} % for \subfigure

\definecolor{myred}{RGB}{248, 206, 204}
\definecolor{mygreen}{RGB}{213, 232, 212}

\newcommand{\hlred}[1]{\sethlcolor{myred}\hl{#1}}
\newcommand{\hlgreen}[1]{\sethlcolor{mygreen}\hl{#1}}

% \newcommand{\m}{\textsc{UNBench}}
\newcommand{\m}{UNBench}


% --------------------------------------------------
% If the title and author information does not fit in the area allocated, uncomment the following
%
%\setlength\titlebox{<dim>}
%
% and set <dim> to something 5cm or larger.

\title{Benchmarking LLMs for Political Science: A United Nations Perspective}

% \title{{\m}: A United Nations Benchmark for LLM in Political Science}

% \title{UNBench: A United Nations Benchmark for LLM in Political Science}
% \title{Evaluating Large Language Models on Real-World Role-Play: A United Nations Benchmark Dataset}
% Topic: Role-play, multi agents, simulation

% Yueqing Liang, Liangwei Yang, Chen Wang, Congying Xia, Rui Meng, Xiongxiao Xu, Haoran Wang, Ali Payani, Kai Shu

\author{
 \textbf{Yueqing Liang\textsuperscript{1}},
 \textbf{Liangwei Yang\textsuperscript{2}},
 \textbf{Chen Wang\textsuperscript{3}},
 \textbf{Congying Xia\textsuperscript{4}}, 
 \textbf{Rui Meng\textsuperscript{2}}, \\
 \textbf{Xiongxiao Xu\textsuperscript{1}},
 \textbf{Haoran Wang\textsuperscript{6}},
 \textbf{Ali Payani\textsuperscript{5}},
 \textbf{Kai Shu\textsuperscript{6}}
\\
 \textsuperscript{1}Illinois Institute of Technology
 \textsuperscript{2}Salesforce
\\
 \textsuperscript{3}University of Illinois at Chicago
 \textsuperscript{4}Meta
 \textsuperscript{5}Cisco
 \textsuperscript{6}Emory University
\\
\texttt{\{yliang40, xxu85\}@hawk.iit.edu}\\
\texttt{\{liangwei.yang, ruimeng\}@salesforce.com}\\
\texttt{cwang266@uic.edu}, 
\texttt{congyingxia@meta.com}, 
\texttt{apayani@cisco.com}\\
\texttt{\{haorang.wang, kai.shu\}@emory.edu}
}




\begin{document}
\maketitle
\begin{abstract}
Large Language Models (LLMs) have achieved significant advances in natural language processing, yet their potential for high-stake political decision-making remains largely unexplored. This paper addresses the gap by focusing on the application of LLMs to the United Nations (UN) decision-making process, where the stakes are particularly high and political decisions can have far-reaching consequences. We introduce a novel dataset comprising publicly available UN Security Council (UNSC) records from 1994 to 2024, including draft resolutions, voting records, and diplomatic speeches. Using this dataset, we propose the United Nations Benchmark (UNBench), the first comprehensive benchmark designed to evaluate LLMs across four interconnected political science tasks: co-penholder judgment, representative voting simulation, draft adoption prediction, and representative statement generation. These tasks span the three stages of the UN decision-making process—drafting, voting, and discussing—and aim to assess LLMs' ability to understand and simulate political dynamics. Our experimental analysis demonstrates the potential and challenges of applying LLMs in this domain, providing insights into their strengths and limitations in political science. This work contributes to the growing intersection of AI and political science, opening new avenues for research and practical applications in global governance. The UNBench can be accessed at: \url{https://github.com/yueqingliang1/UNBench}.

\end{abstract}


\section{Introduction}

Large Language Models (LLMs) such as GPT-4~\cite{openai2023gpt4}, Llama~\cite{dubey2024llama}, and DeepSeek~\cite{liu2024deepseek} have achieved unprecedented proficiency in language tasks and are increasingly under development tailored for different domains~\cite{cheng2023adapting}. Yet, their adaptation to high-stakes political decision-making remains underexplored—particularly in scenarios where model outputs could influence real-world governance. Political science demands capabilities beyond semantic understanding: predicting coalition dynamics, interpreting ambiguous diplomatic language, and navigating the tension between national interests and global norms. These challenges unfold within the United Nations (UN), where a single draft resolution, once adopted, becomes binding international law under Chapter VII of the UN Charter, with cascading impacts on global security, trade, and human rights (e.g., Resolution 1973’s no-fly zone over Libya in 2011).
% Understanding how LLMs process such critical political artifacts is therefore both a technical frontier and a societal imperative.
Studying the application of LLMs in political science represents both a technical frontier and a critical societal challenge.


% There is no existing work studying draft resolutions
% UNSC draft resolutions are not merely political statements, they reshape realities. An adopted resolution can authorize military interventions (Resolution 678, 1990), impose sanctions crippling national economies (Resolution 1718 on North Korea, 2006), or redefine global priorities (Resolution 2341 on critical infrastructure protection, 2017). 
% However,  the draft resolutions are political Schrödinger’s cats: until voted upon, they simultaneously embody hope for consensus and threats of veto-driven collapse. 
% Therefore, stakeholders like governments need to anticipate veto threats to avoid wasted diplomatic capital, and multinational corporations require early warnings to adjust operations (e.g., sanctions compliance). Despite their significance, UNSC's draft resolutions remain a data desert. Existing UN corpora [6,7] focus on sanitized final texts, akin to studying wars solely through peace treaties while ignoring battlefield maneuvers. This bias cripples AI’s utility for stakeholders who need to anticipate rather than postdict political outcomes.

The scope of UN resolutions is extensive, extending far beyond political statements. Adopted resolutions can authorize military interventions (such as Resolution 678 in 1990), impose sanctions that cripple national economies (e.g., Resolution 1718 on North Korea in 2006), or redefine global priorities (e.g., Resolution 2341 on critical infrastructure protection in 2017). 
By analyzing the open-access data from draft resolutions and meeting records, we can explore how current LLMs understand the critical issues facing the international community and assess their ability to interpret bilateral and multilateral relations. This extends LLM's application toward political science, enhancing the analysis of international policies and diplomacy.

\begin{figure}[t]
    \centering
    % linewidth
    % textwidth
    \includegraphics[width=1.0\linewidth]{figures/intro.pdf}
    % \captionsetup{justification=centering}
    \caption{Three Key Stages of the United Nations Decision-Making Process}
    \label{fig:scenario}
\end{figure}

% There is no unified benchmark containing multiple classical political science functionalities
% While recent advances in LLMs have sparked interest in their application to political science, existing benchmarks remain fragmented. As outlined in prior work (e.g., political-llm.org), classical political science functionalities—including \textit{predictive tasks}, \textit{generative tasks}, \textit{simulation}, \textit{causal inference}, and \textit{social impact} analysis—demand distinct capabilities. However, current LLM evaluations (e.g., MMLU, BIG-Bench) narrowly focus on isolated tasks, such as sentiment analysis of political texts or simple vote prediction. This fragmentation overlooks the interconnected nature of real-world political decision-making, particularly in high-stakes multilateral settings like the United Nations Security Council (UNSC). 

Currently, there is no dedicated benchmark designed specifically for LLM applications in political science. Existing benchmarks (e.g., MMLU~\cite{hendryckstest2021,hendrycks2021ethics}, BIG-Bench~\cite{suzgun2022challenging}) with related tasks remain fragmented and their designs may not adequately reflect LLMs' understanding of political science.
Such fragmented evaluation overlooks the interconnected nature of real-world political decision-making, particularly in high-stake multilateral settings like the UN. Political science encompasses complex, inter-connected tasks~\cite{li2024political}. Figure~\ref{fig:scenario} shows the stages of each UN draft resolution. It consists of three stages. 
(1) Drafting: The creation of the resolution's text, involving the collaboration among member states.
(2) Voting: The process in which the resolution is formally approved or rejected by voting from the UN members.
(3) Discussing: Each member states the rationality of the voting. 
Different tasks occur in different stages and also inter-connected with each other.


% To bridge these gaps, we introduce \textbf{the United Nations Benchmark (UNBench)}, the first comprehensive benchmark to evaluate LLMs' ability across four distinct yet interconnected political science tasks:
% (1) \textbf{\textit{Predictive task}}: input a draft resolution to predict its passage probability, requiring analysis of historical voting patterns and geopolitical alignments. (2) \textbf{\textit{Voting simulation}}: instruct an LLM to act as a national agent (e.g., "As the U.S. representative") and output voting decisions (['In favour', 'Against', etc.]), testing contextual understanding of national interests. (3) \textbf{\textit{Co-author selection}}: Given anonymized draft content, identify optimal co-author nations, simulating coalition-building strategies in multilateral diplomacy. (4) \textbf{\textit{Generative task}}: Generate country-specific speeches justifying voting positions, evaluating persuasive language generation under political constraints.
% Our benchmark is built from publicly available UNSC official records (1994-2024), comprising [] draft resolutions, [] voting records, and [] diplomatic speeches which are extracted from [] meeting records. 

To fill the benchmark gap in political science, we introduce \textbf{the United Nations Benchmark (UNBench)}, the first comprehensive benchmark to evaluate LLMs' ability across four distinct yet interconnected political science tasks of different stages:
(a) \textbf{Co-Penholder Judgement}: Given anonymized draft content, identify optimal co-author nations, simulating coalition-building strategies in multilateral diplomacy.
(b) \textbf{Representatives Voting Simulation}: Instruct an LLM to act as a national agent (e.g., "As the U.S. representative") and output voting decisions (['In favour', 'Against', etc.]), testing contextual understanding of national interests.
(c) \textbf{Draft Adoption Prediction}: Input a draft resolution to predict its passage probability, requiring analysis of historical voting patterns and geopolitical alignments.
(d) \textbf{Representative Statement Generation}: Generate country-specific speeches justifying voting positions, evaluating persuasive language generation under political constraints.
The tasks are designed from different UN stages, varying across predictive and generative tasks.
Our benchmark is built from publicly available UNSC official records (1994-2024), comprising draft resolutions, voting records, and diplomatic speeches which are extracted from meeting records. In summary, our work makes the following contributions:

% \begin{itemize}
%     \item \textbf{First to introduce UNSC draft resolutions to LLM research}: We present the first systematic study of how LLMs perform in the United Nations Security Council (UNSC) context.
%     \item \textbf{First dataset focused on UNSC draft resolutions}: Existing UN-related datasets contain only adopted resolutions, neglecting the more complex and informative draft resolution phase. Our dataset fills this gap, capturing the nuances of diplomatic negotiation, veto power, and alliance formation.
%     \item \textbf{Comprehensive political AI benchmark}: Compared to prior political benchmarks, UNSC-Bench integrates multiple classical political science functionalities—including predictive modeling, voting simulation, coalition-building analysis, and persuasive text generation—into a unified evaluation suite for LLMs.
% \end{itemize}

\begin{itemize}
    \item Systematically curated and processed United Nations data from 1994 to 2024, including draft resolutions, voting records, and diplomatic speeches, to provide a new and comprehensive dataset that facilitates LLM applications in political science.
    \item Introduce the first comprehensive benchmark in political science, designed to assess LLM performance across various real-world tasks in different stages of the UN process.
    \item Conduct extensive experimental analysis on the dataset and benchmark, demonstrating the effectiveness and limitations of current LLMs in handling complex political tasks.
\end{itemize}

% The paper will first introduce how we construct UNSC-Bench, then explore the dataset, then report the performances of various LLMs for each task. 

% The emergence of large language models (LLMs) like GPT-4 \cite{openai2023gpt4} and PaLM \cite{chowdhery2022palm} has marked a significant shift in Natural Language Processing (NLP), enabling advanced applications that extend beyond traditional tasks like machine translation \cite{liu2020translation} and question-answering \cite{brown2020language, raffel2020exploring}. A particularly promising application is role-playing, where LLMs simulate diverse characters or personas with distinct knowledge, perspectives, and interactive styles \cite{shanahan2023role}. This capability enhances the model’s interactivity, usability, and contextual richness, making it more adaptable for complex real-world applications \cite{wang2023interaction}.

% While role-playing has primarily focused on individual/personal-level roles, such as simulating historical figures or domain-specific experts, the simulation of organization/group-level roles remains underexplored. Group-level roles, such as representing institutions or collectives, pose unique challenges. These include modeling complex dynamics like collaboration, negotiation, and conflict resolution, as well as maintaining temporal consistency and relational reasoning across multiple stakeholders. Addressing these challenges requires datasets and evaluation frameworks that reflect the intricacies of collective behavior and decision-making.

% However, despite this potential, current datasets and benchmarks for evaluating role-playing in LLMs have several limitations. Many of these datasets are either synthetic or derived from simplified, fictional interactions, often limited to monolingual contexts like English or Chinese. Such datasets lack the depth, realism, and linguistic diversity necessary to evaluate LLMs in high-stakes, real-world settings where decisions have lasting consequences and require nuanced ethical and strategic reasoning.




% To address these gaps, we introduce \textbf{UNRoles}, 
% the first comprehensive dataset designed to evaluate LLM role-playing capabilities at the group level. Derived from authentic UNSC records, this multilingual dataset spans several decades of historical interactions, capturing the decision-making processes of member states across six official UN languages: English, French, Spanish, Russian, Arabic, and Chinese. The UNSC setting provides an unparalleled context for testing LLMs on challenges such as high-stakes decision-making, temporal and policy consistency, relational reasoning, and multilingual role adaptation.
% a unified benchmark that is designed to comprehensively evaluate LLM role-playing across both \textbf{persona-based}(fine-grained) and \textbf{character-based}(coarse-grained) scenarios. By leveraging data from two major United Nations (UN) bodies—the Security Council (UNSC) and the General Assembly (UNGA)—this benchmark offers a unique opportunity to assess LLMs in high-stakes, real-world, and multilingual settings. The UNSC and UNGA datasets complement each other by targeting distinct aspects of role-play: the UNSC dataset emphasizes fine-grained, character-specific roles, while the UNGA dataset facilitates evaluations of coarse-grained persona abstractions.

% To address these gaps, we introduce the United Nations Security Council (UNSC) Benchmark, a multilingual, real-world dataset designed to evaluate LLM role-playing capabilities in the context of international diplomacy. Derived from authentic UNSC records, this dataset spans several decades of historical interactions, capturing the decision-making processes of UNSC member states in six official UN languages: English, French, Spanish, Russian, Arabic, and Chinese. This setting provides an unparalleled opportunity to assess LLMs on challenges that go beyond simple role-play, including high-stakes decision-making, temporal consistency, relational reasoning, and multilingual role adaptation.



% The UNRoles Benchmark is structured around key tasks that test distinct aspects of LLM role-play in complex scenarios:
% (1) Co-Drafter Judgment: Evaluate the model’s relational reasoning by asking it to identify potential co-drafters for resolutions based on historical alliances and diplomatic relations.
% (2) Voting reaction: Assesses decision-making consistency by predicting each country’s stance on resolutions, simulating the ethical and strategic trade-offs inherent to high-stakes international policy.
% (3) Speech simulation: Tests the model’s ability to generate formal, contextually appropriate language that aligns with each country’s diplomatic stance and rhetorical style.
% This benchmark is uniquely positioned to advance LLM capabilities in global, context-sensitive role-play applications. By grounding the evaluation in real-world data, historical depth, and multilingual diversity, the UNRoles Benchmark offers a robust framework for testing LLMs in areas that are underexplored by existing benchmarks. The insights gained from this benchmark could extend to various high-stakes domains, from business strategy and policy-making to cross-cultural communication and international relations.

% In summary, the UNRoles benchmark fills a critical gap in role-playing research by:
% \begin{itemize}
%     \item providing a dataset based on real-world records from international diplomacy, allowing for evaluations in high-stakes, authentic settings.
%     \item supporting multilingual role-play, with data in six UN languages to test models' ability to maintain role fidelity across languages.
%     \item including temporal depth, with records spanning multiple decades, which allows for assessments of long-term role consistency and adaptability.
% \end{itemize}

% The emergence of large language models (LLMs) like GPT-4 \cite{openai2023gpt4} and PaLM \cite{chowdhery2022palm} has marked a significant shift in the field of Natural Language Processing (NLP). These models, known for their extensive capabilities in generating human-like text, have redefined NLP from focusing on traditional tasks such as machine translation \cite{liu2020translation} and question-answering \cite{brown2020language, raffel2020exploring} to enabling more sophisticated, contextually rich tasks at an agent-like level. Among these advancements, one notable application of LLMs is in role-playing, where the model simulates diverse characters or personas with distinct styles and contextual depth \cite{shanahan2023role}. This function not only enhances the model’s usability and interaction experience but also makes it more familiar and relatable to users \cite{wang2023interaction}.

% Despite the potential of role-playing applications, current datasets for evaluating this capability are often limited in realism, language variety, and scope. Many datasets are either synthetic or derived from simple scripted interactions, providing a narrow view of role-play capabilities and limited to English or Chinese. This paper addresses these gaps by introducing a multilingual, real-world role-play dataset derived from the United Nations Security Council (UNSC). With authentic, complex, and multilingual data, this dataset serves as a robust benchmark to test LLMs' ability to engage in role-play across diverse languages and contexts.


\section{Background}

\subsection{LLMs in Political Science}
% 简单介绍5个可以用LLM做的传统political tasks,说明我们的benchmark是一个unified的benchmark for political science
Existing benchmark datasets have played a critical role in advancing Large Language Models (LLMs) in political science tasks. Datasets such as OpinionQA~\cite{santurkar2023whose}, PerSenT~\cite{bastan2020author}, and GermEval-2017~\cite{wojatzki2017germeval} evaluate LLMs on classifying sentiments or identifying topics within political texts. These benchmarks primarily emphasize static text understanding. BillSum~\cite{kornilova2019billsum} and CaseLaw~\cite{shu2024lawllm} specialize in summarization or analysis of legislative documents. Datasets like PolitiFact~\cite{shu2020fakenewsnet}, GossipCop~\cite{grover2022public}, and Weibo~\cite{jin2017multimodal} focus on detecting misinformation. Election prediction and voting behavior datasets, such as U.S. Senate Returns 2020~\cite{DVN/ER9XTV_2022} and State Precinct-Level Returns 2018~\cite{DVN/ZFXEJU_2022} evaluate models' capabilities in statistical pattern recognition and forecasting. These tasks often rely on structured data and focus on predictive performance related to electoral outcomes. 

While these benchmarks have advanced single-task evaluation, they isolate tasks and fail to capture the interconnectedness of real-world political scenarios. They also often focus on static or structured data, neglecting the dynamic nature of high-stakes political decision-making. 


\subsection{United Nations-Related Datasets}
% 介绍其他benchmark的场景,说我们是第一个把联合国引入LLM的benchmark。
% 之前有political science的dataset,但是它们要么没有text,要么只关注adopted的resolution,没有人collect draft resolutions从起草到通过的整个决策过程,然而这个过程是更有价值的。

The United Nations has long been a focal institution for global politics, attracting extensive study in political science ~\cite{bailey2017estimating, voeten2013data}. While various publicly available datasets shed partial light on UN processes, they typically focus on limited aspects of the organization. Harvard Dataverse UN Voting Dataset~\cite{DVN/LEJUQZ_2009} compiles pairwise country voting statistics, offering quantitative insights but lacking textual data such as draft resolutions or debate transcripts. UNSCR.com~\cite{unscr2025} collects mainly \textit{adopted} Security Council resolutions, providing topic labels and related resolutions but minimal coverage of \textit{draft} content. UNSCdeb8~\cite{unscdeb8_zenodo} includes verbatim debate transcripts from 2010 to 2017, capturing real-time deliberation but omitting links to draft texts or voting outcomes. Lastly, the UN Parallel Corpus~\cite{un_parallel_corpus} supplies multilingual final resolutions and meeting records (1994–2014), yet lacks draft-stage materials and detailed metadata to trace evolving negotiations.

Despite aiding research on voting patterns or resolution content, these datasets do not capture the \emph{entire draft-to-adoption pipeline}, missing out on the high-stakes negotiations and coalition-building that characterize UN decision-making. Moreover, none have been transformed into a benchmark specifically for LLM evaluation, limiting their utility for advanced NLP tasks. Our dataset addresses these shortcomings by compiling \emph{draft resolutions, verbatim debates, and voting records} from 1994–2024, thereby offering the first UN LLM benchmark to encompass the full trajectory from initial drafting to final resolution adoption.


\section{The United Nations Benchmark}
\label{Sec:Method}

\begin{figure*}[t]
\centering
\includegraphics[width=\linewidth]{figures/major.pdf}
\caption{The proposed \textbf{{\m}}. It consists of $4$ tasks extracted from different stages of a UN draft.}
\label{fig:major}
\end{figure*}


% A United Nations Security Council (UNSC) draft resolution undergoes three distinct stages before being either adopted or rejected. The \textit{first stage} is the drafting phase, where member states negotiate the resolution’s content and seek co-penholders. The \textit{second stage} involves the voting process, where permanent and non-permanent members cast their votes, ultimately determining the resolution's outcome. The \textit{third stage} is the discussion phase, where representatives justify their votes, articulate national interests, and engage in diplomatic discourse. From this process, we extract four tasks designed to evaluate different capabilities of LLMs. In this section, we systematically introduce each stage and its corresponding tasks, followed by a detailed explanation of our data collection and dataset construction process.
We build UNBench from United Nations Security Council resolutions, which undergoes three distinct stages (Drafting, Voting, Discussing) before being either adopted or rejected. From this process, we extract four tasks designed to evaluate different capabilities of LLMs in political science. In this section, we systematically introduce each stage and its corresponding tasks, followed by a detailed explanation of our data collection and dataset construction process.

\subsection{Benchmark Notation Definition}


To formalize our benchmark, we introduce the following notations:

\begin{itemize}
    \item \(\mathcal{R} = \{r_{1}, r_{2}, \dots, r_{N}\}\): The set of all \textbf{draft resolutions}. Each resolution \(r_{i}\) contains proposed actions, mandates, and contextual details (e.g., sponsoring countries).
    
    \item \(\mathcal{C} = \{c_{1}, c_{2}, \dots, c_{M}\}\): The set of all \textbf{UN members}, encompassing both permanent and non-permanent members.
    
    \item For each resolution \(r_{i} \in \mathcal{R}\), \(\mathcal{C}_{\mathrm{candidate}}(r_{i}) \subseteq \mathcal{C}\) denotes the \textbf{candidate co-penholder} countries—those likely to sponsor or support \(r_{i}\) during the \emph{drafting} process.
    
    \item \(\mathcal{V}(r_{i}) = \{v_{i,1}, v_{i,2}, \dots, v_{i,15}\}\): The \textbf{votes} of the 15 Security Council members on \(r_{i}\). Each vote \(v_{i,j}\) is one of \(\{\texttt{[In Favour]}, \texttt{[Against]}, \texttt{[Abstention]}\}\).
    
    \item \(\text{Result}(r_{i}) \in \{\textsc{Adopted}, \textsc{NotAdopted}\}\): The \textbf{decision} on \(r_{i}\). It is \(\textsc{NotAdopted}\) if it fails to secure the necessary majority or is vetoed by a permanent member.
    
    \item \(\mathcal{S}(r_{i}, c_{j})\): The \textbf{official statement} (speech) of country \(c_{j}\) regarding \(r_{i}\), delivered during the \emph{discussion} stage. This typically includes the country’s rationale, policy concerns, and diplomatic stance.
\end{itemize}




\subsection{Stage 1: Drafting}
\label{sec:drafting}

Drafting is the initial phase in the lifecycle of a resolution. Typically, one or more countries—often referred to as \emph{penholders}—take the lead in preparing a draft text, outlining the resolution’s objectives, scope, and operative clauses. The draft is then refined through closed-door consultations, circulated informally among Council members. A unique feature of drafting is the practice of \emph{co-penholdership}, wherein multiple countries jointly sponsor or “own” the resolution from its inception. We design a task focusing on \emph{identifying the most suitable co-penholder}.

\paragraph{Task 1: Co-Penholder Judgement.}  
Formally, let \(r_{i}\) be a draft resolution authored by country \(c_{a}\). We sample \(\mathcal{C}_{\mathrm{candidate}}(r_{i}) \subseteq \mathcal{C}\) as a set of co-penholders candidates, each representing a different country. The LLM is prompted to assume the \emph{role of the author country} \(c_{a}\), given the text of \(r_{i}\), and asked to choose exactly one co-penholder from the set \(\mathcal{C}_{\mathrm{candidate}}(r_{i})\). In practice, we vary the number of candidates from 2 to 5, making this task a multi-choice setup with a single correct answer.

Co-penholdership reflects shared strategic interests, diplomatic partnerships, or specialized expertise on the issue at hand. Identifying a suitable co-penholder requires the model to:

\begin{itemize}
    \item \textbf{Comprehend Contextual Information:} Understand the resolution’s key themes (e.g., conflict prevention, sanctions regime, peacekeeping mandates), and recognize which policy domains (e.g., human rights, climate, nuclear disarmament) are relevant. This tests the model’s ability to integrate textual comprehension of policy content with broader geopolitical and diplomatic reasoning.

    \item \textbf{Infer Diplomatic Alignments:} Analyze historical or implied alignments and identify country pairings likely to co-sponsor a resolution. This evaluates whether the model can correlate textual cues with knowledge of past collaborations or alliances, and navigate multi-choice questions where the differences between candidate countries may be subtle or context-dependent.

    \item \textbf{Reason About Multilateral Cooperation:} Weigh factors such as a candidate country’s veto power (if permanent), geopolitical priorities, and regional interests to recommend a co-penholder that maximizes the resolution’s likelihood of success.
    
\end{itemize}


Hence, Task~1 offers a focused measure of the model’s capacity to perform political and textual reasoning in a controlled, multi-choice format, laying the groundwork for subsequent stages involving voting and post-vote deliberation.


\subsection{Stage 2: Voting}
\label{sec:voting}

In the second stage of the resolution lifecycle, each of the 15 Council members casts a vote to determine whether a draft resolution is \emph{adopted} or \emph{rejected}. Permanent members wield veto power, meaning a single ``\texttt{Against}'' vote from any of the five permanent countries can block the resolution, regardless of overall support. Non-permanent members, on the other hand, primarily influence the outcome through collective consensus and persuasive negotiation. Based on this voting mechanism, we define two tasks that capture different facets of decision-making at this stage.


\paragraph{Task 2: Representatives Voting Simulation.}
Formally, for a draft resolution \(r_{i}\), let \(\mathcal{V}(r_{i}) = \{v_{i,1}, \ldots, v_{i,15}\}\) denote the votes cast by each of the 15 Council members. In this task, the LLM is given the content of \(r_{i}\) and prompted to \emph{assume the role of a specific representative} \(c_{j}\) (where \(c_{j} \in \mathcal{C}\)) to determine how that country would vote on \(r_{i}\). Each vote \(v_{i,j}\) must be one of \(\{\texttt{[In Favour]}, \texttt{[Against]}, \texttt{[Abstention]}\}\).

\paragraph{Objective and Challenges.}
Effective voting simulation requires the model to:
\begin{itemize}
    \item \textbf{Comprehend the Resolution:} Interpret the text of \(r_{i}\) in light of its subject matter (e.g., conflict prevention, sanctions).
    \item \textbf{Incorporate National Interests:} Weigh a representative country’s known priorities and geopolitical alignments (e.g., historical alliances, regional blocs).
    \item \textbf{Account for Veto Power:} Recognize whether \(c_{j}\) is a permanent member with veto ability.
\end{itemize}
These dimensions test not only the model’s reading comprehension but also its aptitude for political and strategic reasoning, reflecting real-world complexities in UNSC negotiations.

\paragraph{Task 3: Draft Adoption Prediction.}
Once the votes \(\mathcal{V}(r_{i})\) are cast, the resolution is \emph{adopted} if it secures the necessary majority (i.e., at least nine \texttt{[In Favour]} votes) and no permanent member exercises a veto. In this task, the LLM receives the text of \(r_{i}\) and \emph{predict the final outcome}, denoted
\[
    \mathrm{Result}(r_{i}) \;\in\;\{\textsc{Adopted}, \textsc{NotAdopted}\}.
\]
Unlike Task~2, which focuses on individual country votes, this task tests whether the model can account for the collective dynamics of all 15 Council members. Key factors include but are not limited to overall council sentiment, potential veto threats, historical precedents, etc.
% \begin{itemize}
%     \item \textbf{Overall Council Sentiment:} Does the resolution align with the majority view?
%     \item \textbf{Potential Veto Threats:} Could a permanent member’s opposition override the majority?
%     \item \textbf{Historical Precedents:} Have similar resolutions been vetoed or supported in the past?
% \end{itemize}
Accurate adoption prediction thus demands higher-level inference about the distribution of possible votes, the interplay of veto power, and the delicate balancing of geopolitical interests. Together, Tasks~2 and~3 provide complementary perspectives on an LLM’s capacity to model real-world decision-making under complex international relations.


\subsection{Stage 3: Discussing}
\label{sec:discussing}

Once the voting concludes, each member typically delivers a statement clarifying the vote and articulating national positions or broader policy perspectives. These statements reveal the rationale behind each country’s stance—whether \emph{In Favour}, \emph{Against}, or \emph{Abstention}. Since these statements are given in an open discussion format, countries may engage in debates, directly addressing or countering the arguments made by other members. This final discussion phase can shape diplomatic narratives surrounding the resolution’s implications and signal future policy directions. We design a task that evaluates an LLM’s ability to generate representative statements aligned with national interests, voting behavior, and diplomatic discourse norms.

\paragraph{Task 4: Representative Statement Generation}
Formally, for a draft resolution \(r_{i}\), let \(\mathcal{S}(r_{i}, c_{j})\) denote the official statement made by country \(c_{j}\). In this task, the LLM receives the text of \(r_{i}\) alongside contextual details, including the outcome of the vote, each country's voting decision, and any prior statements made in the discussion (if available, in the order they were delivered). The model is then asked to \emph{generate the statement} that \(c_{j}\) would deliver. This statement should reflect:

\begin{itemize}
    \item \textbf{National Interests and Policies:} How does \(c_{j}\)’s geopolitical position shape its response to the resolution (e.g., security concerns, regional dynamics)?
    \item \textbf{Vote Justification:} If \(c_{j}\) voted \texttt{[In Favour]}, \texttt{[Against]}, or \texttt{[Abstention]}, the statement should provide a coherent rationale for the decision.
    \item \textbf{Diplomatic Tone and Style:} UNSC discourse follows a formal, measured tone. The model should generate text that aligns with the conventions of diplomatic statements.
\end{itemize}

By prompting the model to produce \emph{country-specific} statements, Task~4 evaluates higher-level language generation skills in a multi-faceted political context. The ability to incorporate historical alliances, policy priorities, and rhetorical conventions into coherent and persuasive statements indicates an advanced understanding of both textual composition and global political dynamics.



\section{Potential Applications}
\label{sec:potential_app }

The {\m} offers significant value to both LLM researchers and stakeholders in international governance, enabling practical applications and advancing research in geopolitical AI. Below, we outline potential use cases for each group.

\paragraph{For LLM Researchers:}
The {\m} provides a rich testbed for advancing research in LLMs, particularly in the context of geopolitical reasoning and time-series analysis: (1) \textbf{\textit{Geopolitical Reasoning}}: The tasks in the benchmark span a wide range of capabilities, from alliance identification (Task~1) to issue-specific voting prediction (Task~2), offering researchers a comprehensive framework for evaluating and improving LLMs’ understanding of international relations. (2) \textbf{\textit{Temporal Analysis}}: With data spanning 30 years, the benchmark enables time-series tasks such as predicting trends in diplomatic behavior, forecasting shifts in international alliances, or analyzing the impact of historical events (e.g., the end of the Cold War) on UNSC dynamics. For instance, researchers could use the dataset to predict how emerging global issues (e.g., climate change) will influence future resolutions. (3) \textbf{\textit{Fine-Grained Prediction}}: The benchmark’s focus on multi-choice and generative tasks challenges researchers to develop models that balance precision and creativity. For example, improving ROUGE scores in Task~4 could lead to breakthroughs in generating protocol-compliant diplomatic text. (4) \textbf{\textit{Bias and Fairness Analysis}}: The dataset provides an opportunity to study and mitigate biases in LLMs’ geopolitical reasoning, ensuring that models do not perpetuate stereotypes or oversimplify complex international dynamics.

By addressing these research challenges, the {\m} can drive advancements in LLMs’ ability, contributing to more reliable and ethical AI systems for international governance.

% By addressing these research challenges, the {\m} can drive advancements in LLMs’ ability to reason about real-world geopolitical scenarios, ultimately contributing to more reliable and ethical AI systems for international governance.



\vspace{-5pt}
\paragraph{For UN Stakeholders:}
The ability to predict and analyze UNSC decision-making using LLMs also has implications for diplomats, policymakers, and international organizations: (1) \textbf{\textit{Draft Adoption Forecasting}}: By predicting whether a draft resolution will be adopted (Task~3), stakeholders can proactively adjust negotiation strategies, allocate resources more effectively, and build coalitions to maximize the likelihood of success. 
% For example, knowing that a climate resolution is unlikely to pass could prompt earlier lobbying efforts or revisions to the draft. 
(2) \textbf{\textit{Voting Behavior Simulation}}: Simulating country-specific voting behavior (Task~2) allows stakeholders to anticipate the positions of key nations, identify potential allies or opponents, and tailor diplomatic outreach accordingly. This could be particularly useful for smaller nations or NGOs seeking to navigate complex geopolitical landscapes. 
% (3) \textbf{\textit{Strategic Partnership Identification}}: Identifying optimal co-penholders (Task~1) can streamline the resolution drafting process, ensuring that proposals are backed by influential and aligned nations. This reduces the risk of deadlock and enhances the efficiency of diplomatic negotiations. 
% (4) \textbf{\textit{Diplomatic Language Assistance}}: Generating country-specific statements (Task~4) can assist diplomats in crafting rhetorically aligned and protocol-compliant texts, saving time and reducing the risk of miscommunication.

These applications demonstrate how the {\m} can serve as a decision-support tool, enabling stakeholders to navigate the complexities of international governance with greater foresight.



% \subsection{Dataset Construction}
% Our dataset $\mathcal{D}$ is constructed from UNSC meeting records, draft resolutions, and voting histories from 1994 to 2024, containing rich metadata for each draft resolution and country’s interaction. 

% The main goal of our work is to compile a unified and extensive collection of UNSC's decision making process for draft resolutions, which can evaluate multiple capabilities of LLMs. To achieve this, we need to collect multi-perspective data from UNSC's website. The United Nations maintains their official records in their digital library\cite{}, 

% Challenges when constructing the benchmark: 
% 数据收集:(1) draft,voting,meeting的数据关系复杂。draft,voting,meeting在UN的官方数据库中是相对独立的三个板块,我们需要根据某些关键ID把它们link起来。如图\ref{fig:major}底部所示,我们先获得draft resolutions,然后根据它们对应的resolution ID(adopted)找到对应的voting records,再根据voting record中的meeting record ID找到对应的meeting records……(2)缺失值。尽管UN官方的数据库中记录了全面的draft,voting,和meeting数据,但是它们都存在缺失值。缺失值主要有两种,metadata中关键字的缺失/空值/格式不一致,以及对应链接的缺失……(3)由于联合国的官方文件span a long time,所以爬数据时网站格式不统一……
% 数据转化:联合国的官方文件全都是以pdf的形式存在的,然而,目前的LLM api还无法接收pdf形式的输入,所以需要提取文件的pdf,把它们转化成plain text。对于unstructured的political pdf parsing,很容易出现错乱。我们尝试了……(举一些常见的python pdf packages),最终采用基于LLM的LlamaParse……
% 数据处理:(1)unadopted draft没有label,需要自己从网站上draft resolution的notes中提取,并和adopted的resolution ID进行比对…… (2)国家的表示形式不统一……



% \subsection{Data Splits}


% \subsection{Evaluation}



\begin{table}[!htp]
\centering
\begin{tabularx}{\linewidth}{llr}
\toprule
\textbf{Task} & \textbf{Statistic} & \textbf{Value} \\
\midrule
\multirow{4}{*}{Task 1} 
    & \# Drafts                 & 1,300  \\
    & \# Unique Draft Authors   & 209    \\
    & Avg. \# Authors per Draft  & 7      \\
    & \# Total Instances         & 355,126\\
\midrule
\multirow{5}{*}{Task 2} 
    & \# Drafts                 & 1,162  \\
    & \# Total Instances         & 17,430 \\
    & \# \texttt{[In Favour]}    & 17,020 \\
    & \# \texttt{[Against]}      & 16     \\
    & \# \texttt{[Abstention]}   & 391    \\
\midrule
\multirow{3}{*}{Task 3} 
    & \# Drafts                 & 1,978  \\
    & \# \texttt{[Adopted]}      & 1,880  \\
    & \# \texttt{[NotAdopted]}   & 98     \\
\midrule
\multirow{4}{*}{Task 4} 
    & \# Meetings (Drafts)         & 1,752  \\
    & \# Statements              & 7,394  \\
    & \# Countries               & 204    \\
    & Avg. \# Tokens per Statement & 450  \\
\bottomrule
\end{tabularx}
\caption{Statistics for our {\m}.}
\label{tab:dataset_stats}
\end{table}






\begin{table*}[!htp]\centering
\begin{tabular}{lp{1.25cm}p{1.25cm}p{1.25cm}p{1.25cm}p{1.25cm}p{1.25cm}p{1.25cm}p{1.25cm}}
\toprule
&\multicolumn{2}{c}{\textbf{task 1}} &\multicolumn{2}{c}{\textbf{task 2}} &\multicolumn{2}{c}{\textbf{task 3}} &\multicolumn{2}{c}{\textbf{task 4}} \\\cmidrule{2-9}
\textbf{Model} & (1/2) & (1/5) &Bal. ACC &PR AUC &Bal. ACC &Mac. F1 &ROUGE &Cosine Sim. \\\midrule
BERT &0.011 &0.010 &0.537 &0.396 &0.333 &0.328 &/ &/ \\

DeBERTa &0.010 &0.011 &0.500 &0.527 &0.333 &0.328 &/ &/ \\

Llama-3.2-1B &0.581 &0.269 &0.546 &0.185 &0.320 &0.326 &0.033 &0.329 \\

Llama-3.2-3B &0.578 &0.297 &0.597 &0.385 &0.597 &\textbf{0.402} &0.041 &0.290 \\

Llama-3.1-8B &0.665 &0.379 &0.530 &0.168 &0.357 &0.359 &0.039 &0.355 \\

Mistral-7B &0.563 &0.281 &0.426 &0.268 &0.529 &0.140 &0.194 &0.575 \\

GPT-4o &\textbf{0.726} &\textbf{0.464} &\textbf{0.823} &\textbf{0.696} &\textbf{0.677} &\underline{0.363} & 0.199 &\underline{0.619} \\

Qwen2.5-7B & 0.642 & 0.293 & 0.699 & 0.375 & 0.578 & 0.241 & \underline{0.201} & \textbf{0.623} \\

DeepSeek-V3 &\underline{0.695} &\underline{0.422} &\underline{0.724} &\underline{0.655} &\underline{0.668} &0.351 &\textbf{0.207} &\textbf{0.623} \\

\bottomrule
\end{tabular}
\caption{Our {\m} contains four tasks. For each task, we choose two metrics to show. (1/k) means choosing 1 from k choices, Bal. ACC is balance accuracy, PR AUC is precision-recall AUC. The best results for each metric are highlighted in \textbf{bold}, while the second-best results are \underline{underlined}. More results could be found in Appendix~\ref{sec:appendix_results}. }
\label{tab:exp_major}
\end{table*}




\section{Experiments}
\label{Sec:Exp}

Having introduced the design of {\m}, we now turn to an empirical evaluation of various models on our dataset. In this section, we first present summary statistics of {\m}, then detail the experimental setups and metrics used to assess model performance across Tasks~1--4.


\subsection{Dataset Statistics}
Our {\m} covers a broad range of draft resolutions, voting records, and meeting transcripts, providing diverse scenarios for evaluating multiple LLM capabilities. As shown in Table~\ref{tab:dataset_stats}, Task~1 features 1{,}300 draft resolutions with a total of 355{,}126 instances—reflecting a multi-choice setup where each instance corresponds to an author country selecting a co-penholder. Task~2 contains 17{,}430 instances of individual votes for each country that participating in the voting, while Task~3 comprises 1{,}978 draft resolutions with both \texttt{[Adopted]} and \texttt{[NotAdopted]} labels. Finally, Task~4 includes 7{,}394 statements from 1{,}752 UNSC meetings, testing the ability to generate coherent speeches that align with national stances. 
Due to space constraints, we provide the detailed construction process of {\m} in Appendix~\ref{sec:appendix_data_construct}.



\subsection{Experimental Setup}

\paragraph{Models.}
Tasks 1, 2, and 3 are classification-oriented. We compare two \emph{traditional text classification models} (BERT~\cite{devlin2018bert} and DeBERTa~\cite{he2020deberta}) against several \emph{instruction-tuned LLMs}: \textit{Llama-3.2-1B-Instruct}~\cite{dubey2024llama}, \textit{Llama-3.2-3B-Instruct}, \textit{Llama-3.1-8B-Instruct}, \textit{Mistral-7B-Instruct}~\cite{jiang2023mistral}, \textit{DeepSeek-V3}~\cite{liu2024deepseek}, \textit{Qwen2.5-7B-Instruct}~\cite{yang2024qwen2}, and \textit{GPT-4o} via the Azure API. BERT and DeBERTa are fine-tuned for three epochs with a learning rate of \(5\times10^{-5}\). Llama models run on an 8\(\times\)A6000 GPU server, while Mistral and DeepSeek are accessed through the TogetherAI platform. Since Task~4 (representatives statement generation) is inherently a generative task, we only evaluate LLMs on it, setting a temperature of 0.0 for consistent comparisons and adjusting maximum output lengths per task.

\paragraph{Settings.}
For each classification-oriented task, we employ a time-based train/test split. Specifically, we reserve half of the samples from the less frequent labels (according to chronological order) as the test set, ensuring the training set remains balanced and temporally earlier. This protocol simulates real-world scenarios where future events must be predicted from past data.

\paragraph{Metrics.}
Task~1 is a multi-choice question with \(k\) ranging from 2 to 5. We calculate accuracy by checking whether the model identifies the single valid co-penholder. Tasks~2 and~3 are classification problems (multi-class and binary, respectively), so we report metrics robust to class imbalance, including F1-score, balanced accuracy (Bal. ACC), and PR AUC. Task~4 is evaluated using text-generation metrics (ROUGE) and semantic similarity (Sentence-BERT~\cite{reimers2019sentence}) to measure how closely the generated statements match ground truth in style and content. For brevity, we present only two primary metrics per task in Table~\ref{tab:exp_major}, with detailed breakdowns available in Appendix~\ref{sec:appendix}.




% Success in Co-Penholder Judgement suggests an LLM can go beyond simple textual analysis to demonstrate an understanding of political nuance. This foundational drafting step sets the stage for the more complex tasks of \emph{voting} and \emph{discussion} that follow, making it a crucial baseline indicator of diplomatic reasoning in the UNSC context.

\subsection{Task 1: Co-Penholder Judgement}

\begin{figure}[t]
    \centering
    % linewidth
    % textwidth
    \includegraphics[width=1.0\linewidth]{figures/task1.pdf}
    % \captionsetup{justification=centering}
    \caption{Models performance in Task 1 by varying the number of choices. }
    \label{fig:task1}
\end{figure}

% deepseek
This task evaluates LLMs' ability to \textit{\textbf{identify strategic geopolitical alliances}} by selecting co-penholders for UNSC draft resolutions, requiring nuanced understanding of international relations and procedural norms. GPT-4o (0.726) and DeepSeek-V3 (0.695) dominate, demonstrating superior contextual reasoning and geopolitical knowledge. Smaller LLMs (e.g., Llama-3.2-1B: 0.581) lag significantly, while traditional models (BERT: 0.011) fail entirely, underscoring the necessity of LLM-scale architectures for complex political inference. In addition, we vary the number of candidate choices (2–5) to test models’ robustness under increasing complexity. As shown in Figure~\ref{fig:task1}, all models exhibit declining accuracy as choices increase, with GPT-4o maintaining dominance across all levels. The widening performance gaps as choices increase highlight the divergent capacities of LLMs to resolve ambiguity, validating that large, modern LLMs excel at synthesizing latent political knowledge, while smaller or traditional models lack the representational capacity for such nuanced reasoning.

% chatgpt 
% This task evaluates whether a model can interpret diplomatic alignments and select the single correct co-penholder in a multi-choice format. As shown in Table~\ref{tab:exp_major}, fine-tuned BERT and DeBERTa achieve near-zero accuracy, suggesting that classical text encoders struggle with this multi-faceted geopolitical reasoning. In contrast, instruction-tuned LLMs obtain significantly higher scores, underscoring their ability to combine textual comprehension with basic foreign-policy knowledge. GPT-4o ranks first (\(0.726\) for 1/2 and \(0.464\) for 1/5), followed by DeepSeek-V3 (\(0.695\) and \(0.422\)). We also observe a modest gap between smaller Llama variants and the best-performing models, indicating that higher capacity and specialized instruction tuning may be crucial for capturing the nuanced country-level alignments needed for co-penholdership decisions.

% deepseek
\subsection{Task 2: Representatives Voting Simulation}
Focused on simulating country-specific voting behavior, this task tests models’ ability to \textit{\textbf{infer issue-specific voting patterns}} by analyzing how nations prioritize resolution content and contextual geopolitical dynamics. Unlike Task 1, which evaluates proactive alliance-building, Task 2 emphasizes reactive decision-making based on the interplay of national interests, ideological alignment, and external pressures. GPT-4o achieves the highest performance (0.823 Bal. ACC, 0.696 PR AUC), demonstrating a strong ability to model nuanced trade-offs. DeepSeek-V3 (0.724 Bal. ACC, 0.655 PR AUC) and Llama-3.2-3B (0.597 Bal. ACC) show moderate success but struggle with ambiguous cases, while Mistral-7B (0.426 Bal. ACC) performs poorly, reflecting its inability to systematically weigh competing factors. The results highlight LLMs’ potential to simulate diplomatic behavior but reveal significant variance in their capacity to reason about issue-specific voting dynamics.


% chatgpt 
% \subsection{Task 2: Representatives Voting Simulation}
% Task~2 focuses on simulating how a given country would vote on a draft resolution, thereby testing a model’s capacity to integrate the resolution’s content with the national interests of that country. Balanced accuracy and PR AUC highlight a model’s performance under class imbalance (rare \texttt{[Against]} votes). GPT-4o achieves the highest scores (\(0.823\) for balanced accuracy, \(0.696\) for PR AUC), followed by DeepSeek-V3 (\(0.724\) and \(0.655\)). Qwen2.5-7B also performs competitively (\(0.699\) and \(0.375\)), suggesting it can capture some aspects of diplomatic nuance. Overall, these results show that top-tier LLMs can account for veto considerations, historical alliances, and policy preferences, though the gap between them and simpler baselines (e.g., DeBERTa) remains substantial.

\subsection{Task 3: Draft Adoption Prediction}
Whereas Task 2 centers on individual votes, Task 3 measures \textit{\textbf{document-level outcome prediction}}, requiring holistic reasoning about all 15 Council members to predict whether a resolution is eventually \texttt{[Adopted]} or \texttt{[NotAdopted]}. GPT-4o shows the best Bal. ACC (\(0.677\)) and a competitive macro-F1 (\(0.363\)), while Llama-3.2-3B surpasses others in macro-F1 (\(0.402\)) but has lower Bal. ACC (\(0.597\)). The divergence between Bal. ACC and F1 metrics reveals the challenge of modeling adoption mechanics, where understanding Council-wide political dynamics, potential veto threats, and support coalitions play roles.
% These divergences suggest different trade-offs in how models handle true positives versus balanced coverage across classes. Overall, strong performance on Task~3 demands an understanding of Council-wide political dynamics, potential veto threats, and support coalitions. Traditional fine-tuned encoders (BERT/DeBERTa) remain inadequate here, reflecting the complexity of global multi-stakeholder negotiations.

% deepseek
\subsection{Task 4: Representatives Statement Generation}
This task evaluates LLMs’ ability to generate \textit{\textbf{style-sensitive diplomatic statements}} that align with country-specific rhetoric and protocol. Qwen2.5-7B and DeepSeek-V3 tie for semantic fidelity (0.623 Cosine), demonstrating strong alignment with the intended meaning and tone of diplomatic statements. DeepSeek-V3 also leads in lexical overlap (0.207 ROUGE), suggesting better adherence to precise terminological requirements. Mistral-7B achieves high Cosine similarity (0.575) but modest ROUGE (0.194), indicating strength in paraphrasing and conceptual alignment rather than verbatim replication. All models underperform in ROUGE, exposing limitations in precise terminological alignment—a critical requirement for diplomatic drafting. This highlights the unresolved challenge of balancing creativity and protocol adherence in LLM-generated diplomatic text, particularly in capturing the formal and nuanced language of international diplomacy.


% chatgpt
% \subsection{Task 4: Representatives Statement Generation}
% In Task~4, models generate a diplomatic statement explaining why a country voted a particular way, testing advanced text-generation capabilities and alignment with a state’s geopolitical stance. We measure performance with ROUGE (content overlap) and cosine similarity (semantic closeness). DeepSeek-V3 attains the highest ROUGE (\(0.207\)) and ties with Qwen2.5-7B for the best cosine score (\(0.623\)), indicating strong capability to produce both on-topic and contextually aligned statements. GPT-4o (\(0.619\) cosine) and Mistral-7B (\(0.575\)) also perform well, though GPT-4o’s ROUGE (\(0.199\)) slightly trails behind. By contrast, Llama variants yield lower semantic similarity, suggesting that robust rhetorical generation may require more specialized training or larger model capacity.

% \subsection{Cross-Task Summary}
\paragraph{Cross-Task Summary.}
Each task targets a distinct facet of UNSC decision-making. Task~1 primarily tests \textit{\textbf{textual and geopolitical reasoning}} in a multi-choice format, Task~2 and Task~3 emphasize \textit{\textbf{political prediction capabilities}} (from simulating individual votes to forecasting final outcomes), and Task~4 stresses \textit{\textbf{diplomatic language generation}}, requiring alignment with formal protocols and country-specific rhetoric. Performance gaps across tasks and models highlight both the promise and complexity of applying LLMs to real-world international governance, reinforcing the need for dedicated benchmarks as {\m}. 




\section{Conclusions}
This paper introduces UNBench, the first comprehensive benchmark for evaluating LLMs' capabilities in political science through UN Security Council records (1994-2024). By designing four interconnected tasks spanning the complete UN resolution lifecycle—co-penholder judgement, representatives voting simulation, draft adoption prediction, and representative statement generation—we provide a more authentic framework for assessing LLMs' understanding of complex diplomatic dynamics. Our work not only addresses the current gap in LLM evaluation frameworks but also establishes a foundation for future research at the intersection of artificial intelligence and international relations, demonstrating how LLMs could potentially assist in analyzing global governance processes. More applications and detailed analyses can be found in Appendix~\ref{sec:appendix}.

% \section{Conclusions}
% This paper introduces UNBench, the first comprehensive benchmark for evaluating LLMs' capabilities in political science through UN Security Council records (1994-2024). By designing four interconnected tasks spanning the complete UN resolution lifecycle—co-penholder judgement, representatives voting simulation, draft adoption prediction, and representative statement generation—we provide a more authentic framework for assessing LLMs' understanding of complex diplomatic dynamics. Our work not only addresses the current gap in LLM evaluation frameworks but also establishes a foundation for future research at the intersection of artificial intelligence and international relations. For LLM researchers, UNBench enables advancements in geopolitical reasoning, time-series analysis, and protocol-compliant text generation, offering opportunities to study biases, improve fine-grained prediction, and explore temporal trends in diplomatic behavior. Ultimately, this benchmark demonstrates how LLMs could potentially assist in analyzing and navigating global governance processes, paving the way for more reliable and ethical AI applications in international relations.


\section{Limitations}
Despite UNBench's contributions, several limitations should be noted. Our dataset is restricted to UN Security Council records from 1994-2024, which may not fully represent broader international relations dynamics. The benchmark currently focuses on English-language documents, potentially missing nuances in diplomatic communications in other UN languages. Additionally, as political landscapes evolve rapidly, historical patterns in our training data may not accurately reflect current diplomatic dynamics. Finally, there exists potential data contamination since LLMs may have been pre-trained on publicly available UN documents. 
The original UN Security Council records are publicly available, and their interpretation and official authority remain with the United Nations.
In this paper, generative AI tools (ChatGPT, Grammarly) are used to fix grammar bugs and typos.
As a benchmark for political science, we do not foresee any potential risks that need to be disclosed. 




\bibliography{arxiv}

\newpage
\appendix
\label{sec:appendix}

\section{Data Analysis}
This section introduces our data analysis on the collected datasets from United Nation.

\subsection{Subjects Analysis}

\subsubsection{Author-subject Co-occurrence Analysis}

\begin{figure*}[!htp]
\centering
\includegraphics[width=0.9\linewidth]{figures/da_heatmap.pdf}
\caption{\textbf{Author-Subjects Relationships.} This figure shows the co-occurrence matrix of the top 15 authors and subjects. Each cell represents the number of times an author has written about a topic. The darker the cell, the more the author has written about the topic. 
}
\label{fig:author-subject}
\end{figure*}

The author-subjects co-occurrence matrix is shown in Figure~\ref{fig:author-subject}, which reveals distinct patterns in UN Security Council draft resolution authorship. Permanent members of the Security Council—particularly the United States, United Kingdom, and France—demonstrate high engagement across most subject areas, with notably strong involvement in peacekeeping operations and humanitarian assistance. This pattern underscores their important role in global governance while reflecting Western powers' emphasis on human rights and humanitarian interventions. 
From a thematic perspective, peacekeeping operations, humanitarian assistance, and human rights in armed conflicts emerge as the most prominent subjects across member states, forming an interconnected core of Security Council priorities. This pattern suggests a holistic approach to global security challenges, where military peacekeeping efforts are consistently coupled with humanitarian considerations. The strong co-occurrence between sanctions-related topics and peacekeeping operations indicates that the Council frequently employs a dual strategy of enforcement and intervention. Notably, the emergence of terrorism and counter-terrorism as significant themes reflects the evolving nature of global security threats. The data also reveals that technical cooperation and peacebuilding subjects often appear alongside humanitarian assistance, suggesting a long-term approach to crisis resolution that extends beyond immediate security concerns.


\subsubsection{Controversial Subjects in UN}

\begin{figure*}[!hp]
\centering
\includegraphics[width=0.9\linewidth]{figures/da_top30_not-all-yes_subjects.pdf}
\caption{This figure shows the top 30 subjects that at least one country did not vote 'Yes' on.}
\label{fig:no_subjects}
\end{figure*}

The Top-30 subjects that at least one country did not vote 'Yes' is shown in Figure~\ref{fig:no_subjects}.
It reveals that enforcement-related topics—particularly sanctions, peacekeeping operations, and humanitarian assistance—generate the most disagreement in Security Council voting. 
The high frequency of non-affirmative votes on these subjects suggests persistent tensions between international intervention and national sovereignty. Additionally, the presence of international criminal courts and human rights matters among frequently contested subjects highlights the ongoing challenges in balancing international justice with state sovereignty concerns.


\subsubsection{Trend in UN Resolutions}

\begin{figure}[!hp]
\centering
\includegraphics[width=\linewidth]{figures/da_top10_topics_per_5_year_period.pdf}
\caption{This figure shows the top 10 subjects per 5-year period from 1994 to 2024.}
\label{fig:top_10_subjects}
\end{figure}


\begin{figure}[t]
    \centering
    \includegraphics[width=\linewidth]{figures/da_topic_duration_distribution.pdf}
    % \captionsetup{justification=centering}
    \caption{Distribution of the duration of each subject. We can observe that most subjects last for 1 to 5 years, while a few last for more than 30 years.}
    \label{fig:duration}
\end{figure}


The trends in UN resolution topics are also changing over time. The Top-10 subjects per 5-year period from 1994 to 2024 are shown in Figure~\ref{fig:top_10_subjects}. Besides, we also show the distribution of the duration of each subject in Figure~\ref{fig:duration}. Most subjects last for 1 to 5 years, while a few last for more than 30 years.
The two figures reveal two key patterns in the United Nations' focus on global issues. First, certain topics have consistently appeared over the years, indicating the UN’s ongoing attention to these issues. Topics such as international peace and security, human rights, and conflict resolution have remained at the forefront of UN resolutions, reflecting the organization's continuous efforts to address global stability, protect human rights, and resolve conflicts. These persistent topics suggest a sustained, long-term commitment to addressing the most pressing and enduring global challenges. On the other hand, there are topics that have emerged briefly and faded over time, often in response to specific events or crises. For example, resolutions related to regional conflicts or emergency sanctions have been intermittently, typically tied to short-lived geopolitical developments such as military interventions or economic sanctions. These topics highlight the UN's responsive nature, focusing on immediate crises that do not necessarily remain central once the situation is resolved. This pattern underscores the dynamic balance between the UN’s long-standing priorities and its flexibility in addressing emerging global issues as they arise. 
Our benchmark collects all resolutions from this period, making it both challenging and comprehensive, capturing the full scope of the UN's evolving focus on global governance.


\subsection{National-Wised Analysis}

\subsubsection{Voting Frequency}
\begin{figure}[htp]
\centering
\includegraphics[width=\linewidth]{figures/da_top30_vote_participation.pdf}
\caption{The top 30 countries that participated in voting the most.}
\label{fig:vote_frequency}
\end{figure}

Figure~\ref{fig:vote_frequency} shows the vote frequency of Top-30 countries. It reveals a clear dominance by the five permanent members of the UN Security Council (UNSC), namely China, France, Russia, the United Kingdom, and the United States. These countries consistently hold the highest number of votes, reflecting their influential roles in shaping international decisions and maintaining global security. In addition to the permanent members, other non-permanent members such as Japan, Brazil, Argentina, Germany, and Nigeria also appear prominently on the list, highlighting their significant involvement in global governance. Their high voting frequencies may reflect their strategic interests, regional influence, and active participation in international diplomacy. The presence of these countries, along with the permanent members, underscores the UN Security Council's complex decision-making process, where both major powers and key regional players contribute to shaping resolutions and global policies.

\begin{figure}[t]
    \centering
    \includegraphics[width=\linewidth]{figures/da_vote_participation_distribution.pdf}
    % \captionsetup{justification=centering}
    \caption{Distribution of the number of votes each country participated}
    \label{fig:resolution frequency}
\end{figure}



The distribution of the number of votes each country participated in, as shown in Figure~\ref{fig:resolution frequency}, reveals that most countries have participated in fewer than 250 votes. This suggests that the majority of countries engage in voting on a limited number of resolutions, likely reflecting their geopolitical priorities and areas of influence within the UN. While some countries may focus on specific regional or issue-based resolutions, others may be more passive in their participation, contributing to fewer votes overall. The relatively small number of countries with more than 250 votes highlights the most active players in UN decision-making, likely including key international powers and nations with significant stakes in global governance. This distribution underscores the varied levels of involvement in the UN’s voting process, with certain countries playing a consistently active role while others engage more selectively.



% \subsubsection{Nation Characteristics revealed from UN Resolution}

% Table~\ref{tab:top_not_vote_yes_countries} and Table~\ref{tab:top_receive_not_yes_authors} provide insights into the voting behavior of countries within the UN. Russia and China are the most frequent rejecters, with 126 and 92 instances of not voting "Yes," reflecting their selective stance on resolutions based on national and geopolitical interests. In contrast, the United States has only 14 instances of rejection, indicating a higher alignment with the majority of resolutions. The top authors of drafts receiving rejections, such as the United States, the United Kingdom, and France, are frequently involved in proposing resolutions that face opposition, which could be due to their leadership roles in security and international affairs. Smaller countries like Germany, Japan, and Italy are less often the authors of contested drafts, suggesting they take a more reactive role in the UN's decision-making. Overall, these patterns highlight how major powers frequently lead resolutions that spark divergence, while countries like Russia and China maintain more selective voting behaviors.

% \begin{table}[!hp]\centering
% \begin{tabular}{p{4cm}r}\toprule
% \textbf{Country} &\textbf{\# Vote that not 'Y'} \\\midrule
% RUSSIAN FEDERATION &126 \\
% CHINA &92 \\
% UNITED STATES &14 \\
% BRAZIL &11 \\
% VENEZUELA &11 \\
% PAKISTAN &8 \\
% FRANCE &8 \\
% EGYPT &8 \\
% ALGERIA &8 \\
% KENYA &7 \\
% \bottomrule
% \end{tabular}
% \caption{This table shows the top 10 countries that did not vote 'Yes' the most.}
% \label{tab:top_not_vote_yes_countries}
% \end{table}


% \begin{table}[!hp]\centering
% \begin{tabular}{p{4cm}r}\toprule
% \textbf{Country} &\textbf{Count} \\\midrule
% United States &240 \\
% United Kingdom &176 \\
% France &157 \\
% Germany &77 \\
% Japan &46 \\
% Italy &41 \\
% Spain &38 \\
% Sweden &37 \\
% Belgium &34 \\
% Portugal &32 \\
% \bottomrule
% \end{tabular}
% \caption{This table presents the top 10 countries that are the most common authors of drafts that received at least one not 'Yes' vote, along with the corresponding count of such drafts.}
% \label{tab:top_receive_not_yes_authors}
% \end{table}


\begin{table}[htp]\centering
\resizebox{0.8\linewidth}{!}{
\begin{tabular}{lrr}\toprule
\textbf{Author} &\textbf{Rejection} &\textbf{Count} \\\midrule
United States &RUSSIAN &68 \\
United States &CHINA &52 \\
United Kingdom &RUSSIAN &47 \\
France &RUSSIAN &45 \\
France &CHINA &39 \\
United Kingdom &CHINA &39 \\
Germany &RUSSIAN &22 \\
Germany &CHINA &21 \\
Japan &CHINA &15 \\
Italy &CHINA &13 \\
\bottomrule
\end{tabular}}
\caption{This table shows the top 10 pairs of authors and countries that vote not 'Yes' the most. The ``Rejection'' means receiving either a 'No' or 'Abstention' vote. The 'Count' column represents the number of times the author's draft was not voted 'Yes' by the country.}
\label{tab:pair_author_rejection}
\end{table}

\subsubsection{Country Relationships revealed within UN Resolutions}

% \paragraph{Relationship Between Authoring Countries and Countries Casting Not ‘Yes’ Votes. }
% Analyzing the relationship between countries that author UN draft resolutions and those that cast not ‘Yes’ votes (i.e., votes against or abstentions) provides insight into international diplomatic dynamics and geopolitical alliances.


Table~\ref{tab:pair_author_rejection} presents the top 10 pairs of authors and countries that most frequently voted "No" or abstained on draft resolutions. The data reveals that the United States has the highest number of rejections, with Russia rejecting 68 times and China 52 times. This indicates a consistent divide between the U.S. and these two powers, likely reflecting ongoing geopolitical tensions. Similarly, the United Kingdom and France also show high rejection rates from both Russia and China, suggesting a shared stance among Western powers in opposition to certain resolutions proposed by these countries. On the other hand, countries like Germany, Japan, and Italy appear less frequently in the table, with rejections ranging from 13 to 22 times. These smaller states seem to align more often with the major powers but still demonstrate some differences, particularly with China and Russia. Overall, the table highlights significant diplomatic rifts between the Western powers and Russia/China, with frequent rejections indicating key areas of contention in international relations.

\begin{table}[htp]\centering
\resizebox{0.8\linewidth}{!}{
\begin{tabular}{p{5cm}r}\toprule
\textbf{Country Pair} &\textbf{Count} \\\midrule
(FRANCE, UK) &1,153 \\
(UK, US) &1,147 \\
(FRANCE, US) &1,142 \\
(CHINA, UK) &1,068 \\
(CHINA, FRANCE) &1,064 \\
(CHINA, US) &1,058 \\
(RUSSIAN, UK) &1,035 \\
(FRANCE, RUSSIAN) &1,031 \\
(RUSSIAN, US) &1,024 \\
(CHINA, RUSSIAN) &1,013 \\
\bottomrule
\end{tabular}}
\caption{This table shows the top 10 pairs of countries that voted 'Yes' together the most. 'US' and 'UK' are the abbreviations for 'UNITED STATES' and 'UNITED KINGDOM'. The 'Count' column represents the number of times the two countries voted 'Yes' together.
}
\label{tab:paired_yes}
\end{table}


Table~\ref{tab:paired_yes} and Table~\ref{tab:paired_no} provide insights into the voting patterns of country pairs within the United Nations, highlighting both strong collaboration and significant divergence in voting behaviors. Table~\ref{tab:paired_yes} shows that certain country pairs, such as France and the United Kingdom (1,153 joint "Yes" votes), United Kingdom and the United States (1,147 joint "Yes" votes), and France and the United States (1,142 joint "Yes" votes), consistently align on many resolutions, reflecting their close diplomatic and strategic cooperation. In contrast, China and the Western powers also exhibit frequent collaboration, with the China-United Kingdom, China-France, and China-United States pairs each voting together over 1,000 times, indicating areas of common interest despite occasional political differences. On the other hand, Table~\ref{tab:paired_no} reveals country pairs that most often did not vote "Yes" together. The China-Russia pair stands out with 69 instances of disagreement, indicating that the two countries share some geopolitical interests. Other pairs, such as Algeria-China (8 times) and Russia-South Africa (6 times), also display divergent voting patterns, reflecting how national and regional interests can influence voting behavior at the UN. These tables underscore the complex dynamics of international diplomacy, where countries may cooperate on certain issues while diverging on others based on their specific interests and priorities.

\begin{table}[htp]\centering
\resizebox{0.8\linewidth}{!}{
\begin{tabular}{p{5cm}r}\toprule
\textbf{Country Pair} &\textbf{Count} \\\midrule
(CHINA, RUSSIAN) &69 \\
(ALGERIA, CHINA) &8 \\
(ALGERIA, RUSSIAN) &7 \\
(RUSSIAN, SOUTH AFRICA) &6 \\
(GABON, RUSSIAN) &6 \\
(CHINA, GABON) &6 \\
(RUSSIAN, VENEZUELA &6 \\
(KENYA, RUSSIAN) &5 \\
(EGYPT, RUSSIAN) &5 \\
(CHINA, INDIA) &5 \\
\bottomrule
\end{tabular}}
\caption{This table shows the top 10 pairs of countries that did not vote 'Yes' together the most. The 'Count' column represents the number of times the two countries did not vote 'Yes' together.
}
\label{tab:paired_no}
\end{table}

% \section{Potential Applications}
% \label{sec:appendix_app}

% The {\m} offers significant value to both stakeholders in international governance and LLM researchers, enabling practical applications and advancing research in geopolitical AI. Below, we outline potential use cases for each group.  

% \subsection{For UN Stakeholders}  
% The ability to predict and analyze UNSC decision-making using LLMs has immediate implications for diplomats, policymakers, and international organizations: (1) \textbf{Draft Adoption Forecasting}: By predicting whether a draft resolution will be adopted (Task~3), stakeholders can proactively adjust negotiation strategies, allocate resources more effectively, and build coalitions to maximize the likelihood of success. For example, knowing that a climate resolution is unlikely to pass could prompt earlier lobbying efforts or revisions to the draft. (2) \textbf{Voting Behavior Simulation}: Simulating country-specific voting behavior (Task~2) allows stakeholders to anticipate the positions of key nations, identify potential allies or opponents, and tailor diplomatic outreach accordingly. This could be particularly useful for smaller nations or NGOs seeking to navigate complex geopolitical landscapes. (3) \textbf{Strategic Partnership Identification}: Identifying optimal co-penholders (Task~1) can streamline the resolution drafting process, ensuring that proposals are backed by influential and aligned nations. This reduces the risk of deadlock and enhances the efficiency of diplomatic negotiations. (4) \textbf{Diplomatic Language Assistance}: Generating country-specific statements (Task~4) can assist diplomats in crafting rhetorically aligned and protocol-compliant texts, saving time and reducing the risk of miscommunication.  

% These applications demonstrate how the {\m} can serve as a decision-support tool, enabling stakeholders to navigate the complexities of international governance with greater foresight and precision.  

% \subsection{For LLM Researchers}  
% The {\m} also provides a rich testbed for advancing research in LLMs, particularly in the context of geopolitical reasoning and time-series analysis: (1) \textbf{Geopolitical Reasoning}: The tasks in the benchmark span a wide range of capabilities, from alliance identification (Task~1) to issue-specific voting prediction (Task~2), offering researchers a comprehensive framework for evaluating and improving LLMs’ understanding of international relations. (2) \textbf{Temporal Analysis}: With data spanning 30 years, the benchmark enables time-series tasks such as predicting trends in diplomatic behavior, forecasting shifts in international alliances, or analyzing the impact of historical events (e.g., the end of the Cold War) on UNSC dynamics. For instance, researchers could use the dataset to predict how emerging global issues (e.g., climate change) will influence future resolutions. (3) \textbf{Fine-Grained Prediction}: The benchmark’s focus on multi-choice and generative tasks challenges researchers to develop models that balance precision and creativity. For example, improving ROUGE scores in Task~4 could lead to breakthroughs in generating protocol-compliant diplomatic text. (4) \textbf{Bias and Fairness Analysis}: The dataset provides an opportunity to study and mitigate biases in LLMs’ geopolitical reasoning, ensuring that models do not perpetuate stereotypes or oversimplify complex international dynamics.  

% By addressing these research challenges, the {\m} can drive advancements in LLMs’ ability to reason about real-world geopolitical scenarios, ultimately contributing to more reliable and ethical AI systems for international governance.  


\subsection{Dataset Construction}
\label{sec:appendix_data_construct}

Our dataset \(\mathcal{D}\) is constructed from United Nations Security Council (UNSC) meeting records, draft resolutions, and voting histories spanning the years 1994 to 2024. The resulting corpus not only includes the textual content of each draft resolution but also contextual metadata such as voting outcomes, sponsoring nations, meeting transcripts, and the temporal sequence of events.

The overarching goal of this work is to provide a \emph{unified and extensive} collection of the decision-making process, thereby enabling evaluation of multiple LLM capabilities in a single benchmark. To achieve this, we collect multi-perspective data from the official website and digital library\cite{UNSC_digital_lib}, which archive draft resolutions, voting records, and meeting minutes. Below, we highlight key challenges and our corresponding strategies in constructing \(\mathcal{D}\) in the different stages of our benchmark construction.




\paragraph{Data Collection.}
In the data collection stage, we have three challenges: 
(1) \textit{\textbf{Fragmented Records.}} Draft resolutions, voting logs, and meeting transcripts reside in separate sections of the UN database. We utilize shared identifiers (e.g., resolution numbers, meeting record IDs) to \emph{link} these sources. As illustrated schematically in Figure~\ref{fig:major}, we first retrieve all draft resolutions, then query corresponding voting records by resolution ID (when applicable), and finally map meeting transcripts via the meeting record ID.
(2) \textit{\textbf{Missing or Incomplete Metadata.}} Despite the UN’s comprehensive record-keeping, certain entries contain missing fields (e.g., sponsor lists), inconsistent data formats, or broken links. We mitigate these issues by cross-referencing multiple UN repositories, manually curating ambiguous entries, and applying standardized naming conventions for country references.
(3) \textit{\textbf{Historical Document Diversity.}} The official document formats and website structures vary considerably across decades, complicating automated crawling and parsing. We address this by implementing adaptive web-scraping scripts that detect layout differences and by performing iterative quality checks to ensure data consistency.



\paragraph{Data Conversion.}
UN documents are primarily stored in PDF format, making direct ingestion by current LLMs infeasible. We therefore extract and convert the content into plain text. Early attempts using generic Python PDF libraries yielded mixed accuracy due to the unstructured, domain-specific nature of political documents. We applied a \emph{LLM-based} parser (LlamaParse\cite{llamaparse}) to handle complex formatting (e.g., multi-column layouts, footnotes, multilingual text).
% After experimenting with multiple tools (e.g., PyPDF2, PDFMiner), we ultimately applied a \emph{LLM-based} parser (LlamaParse\cite{llamaparse}) to handle complex formatting (e.g., multi-column layouts, footnotes, multilingual text).


\begin{table}[tp]\centering
\resizebox{\columnwidth}{!}{
\begin{tabular}{lrrrrr}\toprule
&\textbf{2 choices} &\textbf{3 choices} &\textbf{4 choices} &\textbf{5 choices} \\\midrule
Llama-3.2-1B &0.581 &0.394 &0.312 &0.269 \\
Llama-3.2-3B &0.578 &0.393 &0.328 &0.297 \\
Llama-3.1-8B &0.665 &0.507 &0.408 &0.379 \\
GPT-4o &0.726 &0.613 &0.511 &0.464 \\
DeepSeek-V3 &0.695 &0.555 &0.443 &0.422 \\
Mistral-7B &0.563 &0.407 &0.335 &0.281 \\
Qwen2.5-7B &0.642 &0.478 &0.353 &0.293 \\
\bottomrule
\end{tabular}}
\caption{Comprehensive results for Task 1. }
\label{tab:appendix_task1}
\end{table}




\paragraph{Data Processing.}
(1) \textit{\textbf{Labeling Adopted vs.\ Unadopted Drafts.}} Some drafts never become official resolutions (i.e., “unadopted”), lacking a formal resolution ID. We thus inspect the official notes in each draft’s record and cross-verify with the final resolution index to categorize them correctly.
(2) \textit{\textbf{Country Name Normalization.}} Different records refer to the same country with variations (e.g., “United Kingdom” vs.\ “Kingdom”). We employ a standardized dictionary to unify references to the same country entity across all entries.
(3) \textit{\textbf{Metadata Alignment.}} For each draft \(r_i\), we compile relevant information—author/sponsor countries, date, issue category, voting breakdown, and meeting transcripts—into a structured format compatible with modern NLP frameworks.


Through these steps, UNBench incorporates the \emph{entire} lifecycle of each UNSC draft resolution, from initial sponsorship and negotiations to final votes and discussions, ensuring comprehensive coverage for our benchmark tasks.





\section{Detailed Results of UNBench}
\label{sec:appendix_results}

In this section, we present the detailed results of the UNBench benchmark, evaluating multiple models across four distinct tasks. The tables provide comprehensive performance metrics for each model on various tasks, including accuracy, precision, recall, AUC, F1 score, and other relevant evaluation metrics.


\begin{table}[tp]\centering
\resizebox{\columnwidth}{!}{
\begin{tabular}{lrrrrr}\toprule
&\textbf{ROUGE} &\textbf{Jaccard} &\textbf{TF-IDF} &\textbf{SentBERT} \\\midrule
Llama-3.2-1B &0.0328 &0.0304 &0.3666 &0.3293 \\
Llama-3.2-3B &0.0407 &0.0341 &0.4287 &0.2902 \\
Llama-3.1-8B &0.0394 &0.0363 &0.4021 &0.3553 \\
GPT-4o &0.1985 &0.1837 &0.7958 &0.6188 \\
DeepSeek-V3 &0.2069 &0.1876 &0.8012 &0.6225 \\
Mistral-7B &0.1935 &0.1688 &0.7522 &0.5750 \\
Qwen2.5-7B &0.2008 &0.1761 &0.7842 &0.6229 \\
\bottomrule
\end{tabular}}
\caption{Comprehensive results for Task 4. Similarity of IT-IDF and SentBert are calculated by cosine similarity.}
\label{tab:appendix_task4}
\end{table}


Task 1 (Table~\ref{tab:appendix_task1}) evaluates the models' performance across multiple-choice tasks with varying numbers of choices (2 to 5). The results indicate that GPT-4o outperforms the other models across all choice levels, particularly excelling in the 2-choice and 3-choice tasks, where it maintains the highest scores in terms of accuracy. Models like Llama-3.1-8B and DeepSeek-V3 also show competitive results, especially for more complex tasks (4 and 5 choices), though they trail behind GPT-4o.

\begin{table*}[htbp]\centering
\resizebox{0.8\linewidth}{!}{
\begin{tabular}{lrrrrrrrrrr}\toprule
&\textbf{Accuracy} &\textbf{AUC} &\textbf{Bal. ACC} &\textbf{Precision} &\textbf{Recall} &\textbf{F1} &\textbf{PR\_AUC} &\textbf{MCC} &\textbf{G-Mean} \\\midrule
Llama-3.2-1B &0.898 &0.497 &0.320 &0.332 &0.320 &0.326 &0.334 &0.006 &0.464 \\
Llama-3.2-3B &0.523 &0.597 &0.597 &0.520 &0.597 &0.402 &0.956 &0.087 &0.597 \\
Llama-3.1-8B &0.917 &0.532 &0.357 &0.360 &0.357 &0.359 &0.338 &0.079 &0.502 \\
GPT-4o &0.922 &0.731 &0.677 &0.400 &0.677 &0.363 &0.343 &0.162 &0.729 \\
DeepSeek-V3 &0.931 &0.720 &0.668 &0.464 &0.668 &0.351 &0.343 &0.151 &0.718 \\
Mistral-7B &0.557 &0.593 &0.426 &0.345 &0.426 &0.268 &0.341 &0.100 &0.569 \\
Qwen2.5-7B &0.935 &0.719 &0.699 &0.373 &0.699 &0.375 &0.344 &0.141 &0.719 \\
\bottomrule
\end{tabular}}
\caption{Comprehensive results for Task 2. }
\label{tab:appendix_task2}
\end{table*}

Task 2 (Table~\ref{tab:appendix_task2}) presents a set of metrics evaluating model performance on binary classification tasks. Here, Qwen2.5-7B leads with the highest accuracy (0.935) and AUC (0.719), indicating its strong ability to differentiate between classes. However, GPT-4o shows superior performance in other metrics such as F1 (0.686) and G-Mean (0.807), making it the most balanced model for this task. DeepSeek-V3 also performs strongly across multiple metrics, especially in precision (0.828), suggesting it excels in tasks where false positives need to be minimized.




\begin{table*}[htp]\centering
\resizebox{0.8\linewidth}{!}{
\begin{tabular}{lrrrrrrrrrr}\toprule
&\textbf{Accuracy} &\textbf{AUC} &\textbf{Bal. ACC} &\textbf{Precision} &\textbf{Recall} &\textbf{F1} &\textbf{PR\_AUC} &\textbf{MCC} &\textbf{G-Mean} \\\midrule
Llama-3.2-1B &0.815 &0.546 &0.546 &0.083 &0.245 &0.124 &0.185 &0.057 &0.456 \\
Llama-3.2-3B &0.523 &0.597 &0.597 &0.073 &0.679 &0.132 &0.385 &0.087 &0.591 \\
Llama-3.1-8B &0.935 &0.530 &0.530 &0.211 &0.076 &0.111 &0.168 &0.098 &0.273 \\
GPT-4o &0.968 &0.823 &0.823 &0.714 &0.660 &0.686 &0.696 &0.670 &0.807 \\
DeepSeek-V3 &0.966 &0.724 &0.724 &0.828 &0.453 &0.585 &0.655 &0.597 &0.671 \\
Mistral-7B &0.867 &0.529 &0.529 &0.084 &0.151 &0.108 &0.140 &0.044 &0.370 \\
Qwen2.5-7B &0.926 &0.578 &0.578 &0.250 &0.189 &0.215 &0.241 &0.179 &0.427 \\
\bottomrule
\end{tabular}}
\caption{Comprehensive results for Task 3. }
\label{tab:appendix_task3}
\end{table*}

Task 3 (Table~\ref{tab:appendix_task3}) focuses on multi-class classification tasks, where GPT-4o again stands out with the highest accuracy (0.968) and balanced performance across other metrics like recall, F1, and G-Mean. DeepSeek-V3 shows strong performance in precision (0.828) and recall (0.453), which may indicate a more specialized capability in identifying specific class instances.




Task 4 (Table~\ref{tab:appendix_task4}) assesses the models' ability to generate meaningful representations and comparisons between text using various similarity measures like ROUGE, Jaccard, and cosine similarity. DeepSeek-V3 performs best across multiple metrics, particularly in cosine similarity (0.8012 with TF-IDF and 0.6225 with SentBERT), demonstrating its strength in textual similarity and comparison tasks. GPT-4o also shows strong performance, particularly with cosine similarity (0.7958 with TF-IDF and 0.6188 with SentBERT).

Overall, the results demonstrate the competitive nature of current models, with GPT-4o leading in several tasks due to its balanced performance across various metrics. However, other models like DeepSeek-V3 and Qwen2.5-7B also show strong results in specific areas, such as precision and text similarity. These findings highlight the strengths and limitations of each model, offering valuable insights for selecting the most suitable model for specific tasks within the UNBench framework.


\end{document}
