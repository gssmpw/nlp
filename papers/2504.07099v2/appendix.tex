\section{Appendix}
\label{sec:app}

\subsection{Preliminary}
\label{sec:app-pre}
\noindent\textbf{Fourier Transform}.
The Fourier Transform~\citep{nussbaumer1982fast}  serves as a cornerstone mathematical tool for dissecting the frequency composition of signals.  It decomposes a time-domain signal into its underlying sinusoidal components, thereby revealing the frequencies present and their corresponding magnitudes. The Continuous-Time Fourier Transform (CTFT) for a continuous-time signal $x(t)$ is defined as:
\begin{equation}
\begin{aligned}
X(f) = \int_{-\infty}^{\infty} x(t)e^{-j2\pi ft} dt, \quad j = \sqrt{-1}
\end{aligned}
\end{equation}
where $X(f)$ is the continuous frequency spectrum. $x(t)$ is the continuous-time signal. $f$ is the continuous frequency variable.
$j$ is the imaginary unit ($\sqrt{-1}$). The inverse CTFT reconstructs the time-domain signal from its frequency representation:
\begin{equation}
\begin{aligned}
x(t) = \int_{-\infty}^{\infty} X(f)e^{j2\pi ft} df, \quad j = \sqrt{-1}
\end{aligned}
\end{equation}


For discrete-time signals, the Discrete Fourier Transform (DFT) provides a discrete frequency representation. The DFT of a sequence $x[n]$ of length $N$ is defined as:
\begin{equation}
\begin{aligned}
    X[k] = \sum_{n=0}^{N-1} x[n] e^{-j\frac{2\pi}{N}kn}, \quad j = \sqrt{-1}
\end{aligned}
\end{equation}
where $X[k]$ denotes the discrete frequency spectrum, and $k$ represents the discrete frequency index. The original time-domain signal can be reconstructed via the Inverse Discrete Fourier Transform (IDFT):
\begin{equation}
\begin{aligned}
x[n] = \frac{1}{N} \sum_{k=0}^{N-1} X[k] e^{j\frac{2\pi}{N}kn}
\end{aligned}
\end{equation}

Computationally, a direct DFT calculation exhibits $\mathcal{O}(N^2)$ complexity. The Fast Fourier Transform (FFT) algorithm offers a remarkable improvement, reducing this complexity to $\mathcal{O}(N \log N)$. This substantial efficiency gain makes the FFT an indispensable tool in various domains, including signal processing, telecommunications, and image analysis, where swift frequency-domain analysis is paramount. The FFT achieves its efficiency through a recursive decomposition of the DFT computation, exploiting inherent symmetries in the trigonometric functions involved.



\noindent\textbf{Laplace Transform}.
The Laplace transform~\citep{schiff2013laplace} is a powerful integral transform widely used in the analysis of linear time-invariant (LTI) systems. It converts a function of time, often representing a system's input or output signal, into a function of a complex frequency variable, $s$. This transformation simplifies the analysis of differential equations, frequently encountered in describing LTI system behavior. The key benefit is the conversion of differential equations into algebraic equations, which are significantly easier to solve.
The unilateral (one-sided) Laplace transform of a function $f(t)$, defined for $t \ge 0$, is given by:
\begin{equation}
\label{eq:laplace}
F(s) = \mathcal{L}\{f(t)\} = \int_0^\infty f(t)e^{-st} dt
\end{equation}

where $F(s)$ represents the Laplace transform of $f(t)$. $s$ is a complex frequency variable, often expressed as $s = \sigma + j\omega$, with $\sigma$ representing the real part (related to exponential decay) and $\omega$ the imaginary part (related to frequency). The integral's lower limit of 0 reflects the typical application to causal signals (signals that are zero for $t < 0$).


The inverse Laplace transform recovers the original time-domain function $f(t)$ from its Laplace transform $F(s)$:
\begin{equation}
f(t) = \mathcal{L}^{-1}\{F(s)\} = \frac{1}{2\pi j} \int_{\sigma - j\infty}^{\sigma + j\infty} F(s)e^{st} ds
\end{equation}
This integral is a complex line integral along a vertical line in the complex $s$-plane, with $\sigma$ chosen to ensure convergence.  The inverse transform is frequently computed using tables of Laplace transforms and techniques like partial fraction decomposition.  


\noindent\textbf{Wavelet Transform}. The wavelet transform~\citep{meyer1989wavelets} is a mathematical tool used in signal processing that allows for the decomposition of a signal in the time-frequency domain. Unlike the Fourier transform, the wavelet transform can reveal both the frequency characteristics of a signal and the time distribution of these frequency components, making it particularly effective for analyzing non-stationary signals. According to the computation methods, wavelet transforms can be divided into the continuous wavelet transform and discrete wavelet transform.

The continuous wavelet transform (CWT) computes wavelet coefficients by analyzing a signal across different frequencies and time positions, providing a detailed energy distribution at the expense of high computational cost. The following equation shows the CWT, which decomposes a continuous signal $f(t)$ into frequency components localized in time:
\begin{equation}
\begin{aligned}
    F(\tau, s) = \frac{1}{\sqrt{\vert s \vert}} \int_{-\infty}^{\infty} f(t) \psi^* \left( \frac{t - \tau}{s} \right) dt
\end{aligned}
\end{equation}
where $\tau$ is the translation parameter, determining the position of the wavelet along the time axis, and $s$ is the scale parameter, which controls the width of the wavelet and thus affects the frequency resolution. This continuous transform enables the examination of the signal's frequency components in a time-localized manner, making it suitable for analyzing non-stationary signals.

The discrete wavelet transform (DWT) performs multi-scale decomposition of a signal, separating it into different frequency bands while maintaining both frequency and time localization. Unlike the CWT, the DWT is computationally efficient and suitable for digital signal processing. The following equation illustrates the DWT, which decomposes a discrete signal $f[t_m]$ into wavelet coefficients $D(a, b)$ based on discrete scaling and translation steps:
\begin{equation}
\begin{aligned}
    D(a, b) = \frac{1}{\sqrt{b}} \sum_{m=0}^{p-1} f[t_m] \psi \left( \frac{t_m - a}{b} \right)
\end{aligned}
\end{equation}
where $a$ is the translation parameter, controlling the position of the wavelet on the time axis, and $b$ is the scale parameter, determining the width of the wavelet, which affects the frequency resolution. This transformation enables the decomposition of the signal $f[t_m]$ into components at various scales and positions, facilitating multi-resolution analysis.


\vspace{-0.1in}
\begin{figure}[htb]
    \centering
    \setlength{\belowcaptionskip}{-0.5cm}  % Adjust this to reduce spacing
    \includegraphics[width=0.50\textwidth]{figures/framework_survey}
    \caption{Representative studies with time evolving}
    \label{figure_time}
    % \vspace{-0.3cm}  % Adjust this to further reduce spacing
\end{figure}


\subsection{References and Methods}
Constrained by page limitations, certain references and technical details have been necessarily omitted. Our review primarily encompasses publications from leading venues such as ICML, NeurIPS, TPAMI, and arXiv. We intend to further enrich this analysis with a comprehensive incorporation of the latest advancements in the field. 





% \subsection{Discussion}
% % 信号分类及依据,时域,空域,频域,小波域,拉普拉斯域
% Signals can be classified and analyzed based on their domain representation.  The time domain directly captures signal amplitude variations over time.  The spatial domain represents signals as functions of spatial coordinates, useful for images and other spatially distributed data. Transforming to the frequency domain, using techniques like Fourier transforms, reveals the constituent frequencies of a signal.  The wavelet domain provides a multi-resolution analysis, decomposing signals into different frequency bands with varying time resolutions. Finally, the Laplace domain, often used in systems analysis, represents signals using the Laplace transform, facilitating the study of system stability and response. Each domain offers unique insights into signal characteristics, making the choice of domain crucial for effective analysis and processing.

% %为什么需要transform domain

% The need for transform domain analysis stems from the limitations of solely relying on time or spatial domain representations.  While these domains directly reflect signal amplitude variations over time or space, they often obscure underlying patterns and structures.  Transforming signals into domains like the frequency, wavelet, or Laplace domain reveals hidden information crucial for various applications.  For instance, frequency domain analysis highlights the dominant frequencies in a signal, beneficial for noise reduction and feature extraction. Wavelet transforms offer multi-resolution analysis, revealing both frequency and time localization, ideal for analyzing non-stationary signals. The Laplace domain simplifies the analysis of linear systems, providing insights into stability and response characteristics. In essence, transform domains provide alternative perspectives that enhance signal understanding, improve feature extraction, and facilitate more efficient processing.


% %结论和发现
% % 与高频频道(索引较大的框)相比,低频频道(索引较小的框)的选择频率更高。 这表明对于视觉推理任务而言,低频通道通常比高频通道更具信息性
% % 
% % F-principle(Frequency Principle,频率原则)是指在训练深度神经网络的过程中,网络倾向于优先学习低频信息,然后逐渐学习高频信息的现象。


% Analysis reveals a preference for lower-frequency channels (smaller index bins) over higher-frequency channels (larger index bins) in visual reasoning tasks, evidenced by their significantly higher selection frequency. This finding supports the Frequency Principle (F-principle), which describes the tendency of deep neural networks to prioritize learning low-frequency information before progressively acquiring higher-frequency details during training. The dominance of low-frequency channels suggests their greater informational value for this specific task.









