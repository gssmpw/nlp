\section{Future Directions}
\label{sec:future}
Future advancements in frequency-domain learning hinge on several key directions, particularly in time series analysis.

\noindent\textbf{First}, expanding beyond traditional transforms to incorporate advanced techniques like empirical mode decomposition (EMD), Hilbert-Huang transform (HHT), or other emerging methods can uncover more nuanced and richer feature representations in time series analysis, paving the way for deeper insights into complex signals.

\noindent\textbf{Second}, developing adaptive learning algorithms is crucial. These algorithms should dynamically adjust to the varying frequency characteristics of different datasets, ensuring robust performance across diverse applications in time series.

\noindent\textbf{Third}, further exploration of the synergy between deep learning and frequency-domain methods is needed. Integrating deep learning architectures with frequency-based feature extraction could significantly improve predictive accuracy and model interpretability in the time series domain.

\noindent\textbf{Finally}, incorporating domain-specific knowledge into frequency-domain models will enhance their performance in specific applications like time series prediction, leading to more impactful results across various fields.


% In this section, we aim to explore the future directions on improving learning in frequency domain:
% \textbf{(1)} Exploring Advanced Transform Techniques: Exploring and integrating advanced transform techniques beyond Fourier, Laplace, and wavelet transforms, such as empirical mode decomposition (EMD)~\citep{rilling2003empirical,zeiler2010empirical,rehman2010multivariate,lei2013review}, Hilbert-Huang transform (HHT)~\citep{huang2014hilbert,datig2004performance,peng2005improved,de2022survey}, or other emerging methods, to further enhance machine learning models. \textbf{(2)} Investigating Adaptive Frequency Learning: Developing adaptive learning algorithms~\citep{yucelen2012low,righetti2006dynamic,chen2001frequency} that dynamically adjust to the frequency characteristics of data, enabling models to effectively adapt to varying signal properties. \textbf{(3)} Investigating LLMs in Frequency Domain: Investigating the potential of LLMs to enhance frequency-domain analysis~\citep{sahoo2019application,rahman2013efficient} is another future direction, potentially leading to improved performance in tasks such as signal processing and time series forecasting. 
% \textbf{(4)} Incorporating Domain Knowledge: Incorporating domain-specific knowledge~\citep{ruppert2019disentangling,maree2015addressing,kejriwal2019domain,wang2024recent} into frequency domain learning models to improve interpretability, generalization, and performance on specific applications.
% \textbf{(5)} Hybrid Transform Approaches: Investigating the potential benefits of combining multiple transform methods or developing hybrid transform approaches to capture diverse aspects of data representations efficiently.
% \textbf{(6)} Adaptive Frequency Learning: Developing adaptive learning algorithms that dynamically adjust to the frequency characteristics of data, enabling models to effectively adapt to varying signal properties.
% \textbf{(7)} Incorporating Domain Knowledge: Incorporating domain-specific knowledge into frequency domain learning models to improve interpretability, generalization, and performance on specific applications.
% \textbf{(8)} Deep Learning Integration: Investigating the integration of deep learning models with frequency domain learning techniques to harness the strengths of both paradigms for improved predictive performance and feature extraction.
% \textbf{(9)} Ethical and Fairness Considerations: Considering ethical implications and fairness aspects in the application of frequency domain learning techniques to ensure responsible and unbiased model development and deployment.
% \textbf{(10)} Benchmarking and Evaluation: Establishing standardized benchmarks and evaluation metrics for frequency domain learning methods to facilitate comparative studies, reproducibility, and advancements in the field.
% By exploring these avenues and addressing the associated challenges, researchers can advance the field of Learning in the Frequency Domain and unlock new opportunities for improving machine learning models across various domains.



% 12/30
% Challeges and furture challenges (promising directions)




% \section{References and Methods}
% Constrained by page limitations, certain references and technical details have been necessarily omitted. Our review primarily encompasses publications from leading venues such as ICML, NeurIPS, TPAMI, and arXiv. We intend to further enrich this analysis with a comprehensive incorporation of the latest advancements in the field. 