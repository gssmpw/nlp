% \section{Domain Transform Methods}

% This section provides a concise overview of three fundamental transform methods: Fourier Transform, Laplace Transform, and Wavelet Transform.

% \noindent \textbf{Fourier Transform}. Fourier transform~\cite{nussbaumer1982fast} is a fundamental tool for analyzing the frequency content of signals by decomposing a time-domain signal into sinusoidal components, revealing the frequencies and their magnitudes. 
% And the process of Fourier Transform is shown as follows:
% \begin{equation}
% \begin{aligned}
% X(f) = \int_{-\infty}^{\infty} x(t)e^{-j2\pi ft} dt, \quad j = \sqrt{-1}
% \end{aligned}
% \end{equation}
% where $X(f)$ is the continuous frequency spectrum, $x(t)$ is the continuous-time signal, and $f$ is the continuous frequency variable.


% The Continuous-Time Fourier Transform (CTFT) provides the continuous frequency spectrum for continuous-time signals, while the Discrete Fourier Transform (DFT) gives a discrete frequency representation for discrete-time signals. For computational efficiency, the Fast Fourier Transform (FFT) reduces the direct DFT's $\mathcal{O}(N^2)$ complexity to $\mathcal{O}(N \log N)$, making it essential in fields like signal processing and telecommunications. The FFT achieves this speedup through recursive decomposition, leveraging symmetries in trigonometric calculations to facilitate rapid frequency-domain analysis, which is critical in applications requiring real-time data processing.



% \noindent \textbf{Laplace Transform}. Laplace transform~\cite{schiff2013laplace} is a crucial tool for analyzing linear time-invariant (LTI) systems, converting time-domain functions into functions of a complex frequency variable $s$. The unilateral (one-sided) Laplace transform of a function $f(t)$, defined for $t \ge 0$, is shown:

% \begin{equation} % \label{eq:laplace}
%     F(s) = \mathcal{L}\{f(t)\} = \int_0^\infty f(t)e^{-st} dt
% \end{equation}
% where $F(s)$ represents the Laplace transform of $f(t)$. $s$ is a complex frequency variable, often expressed as $s = \sigma + j\omega$, with $\sigma$ representing the real part (related to exponential decay) and $\omega$ the imaginary part (related to frequency). The integral's lower limit of 0 reflects the typical application to causal signals (signals that are zero for $t < 0$).




% \noindent \textbf{Wavelet Transform}. The wavelet transform~\cite{meyer1989wavelets} is a mathematical tool used in signal processing that allows for the decomposition of a signal in the time-frequency domain. The following equation shows the CWT, which decomposes a continuous signal $f(t)$ into frequency components localized in time:
% \begin{equation}
% \begin{aligned}
%     F(\tau, s) = \frac{1}{\sqrt{\vert s \vert}} \int_{-\infty}^{\infty} f(t) \psi^* \left( \frac{t - \tau}{s} \right) dt
% \end{aligned}
% \end{equation}
% where $\tau$ is the translation parameter, determining the position of the wavelet along the time axis, and $s$ is the scale parameter, which controls the width of the wavelet and thus affects the frequency resolution.

% Unlike the Fourier transform, the wavelet transform can reveal both the frequency characteristics of a signal and the time distribution of these frequency components, making it particularly effective for analyzing non-stationary signals. According to the computation methods, wavelet transforms can be divided into the continuous wavelet transform (CWT) and discrete wavelet transform (DWT). The CWT computes wavelet coefficients by analyzing a signal across different frequencies and time positions, providing a detailed energy distribution at the expense of high computational cost. The DWT performs multi-scale decomposition of a signal, separating it into different frequency bands while maintaining both frequency and time localization. Unlike the CWT, the DWT is computationally efficient and suitable for digital signal processing. 


\section{Problem Definition}
\label{sec:prob}
The input consists of a long-term time series $\mathcal{X} = (x_1, \dots, x_L) \in \mathbb{R}^{L \times V}$, where $L$ is the historical window length and $V$ is the number of variables. The corresponding ground truth for the prediction is $\mathcal{Y} = (x_{L+1}, \dots, x_{L+H}) \in \mathbb{R}^{H \times V}$, with $H$ representing the prediction horizon.

\noindent \textbf{Frequency Transform.} To more effectively capture periodic patterns inherent in time series data, numerous studies have employed transformations that convert the data into the frequency domain. Formally, we denote the frequency domain transformation by a generic operator $\text{FT}(\cdot)$, defined as follows:
\begin{equation}
\begin{aligned}
    \mathbf{X}' = \text{FT}(\mathcal{X})
\end{aligned}
\end{equation}
The primary objective of learning in the frequency domain is to capture periodic information in time series while preserving temporal dependencies. We provide a pipeline for time series analysis through frequency transformation in Figure~\ref{fig:intro}. 