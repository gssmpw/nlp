%%%% ijcai25.tex

\typeout{IJCAI--25 Instructions for Authors}

% These are the instructions for authors for IJCAI-25.

\documentclass{article}
\pdfpagewidth=8.5in
\pdfpageheight=11in

% The file ijcai25.sty is a copy from ijcai22.sty
% The file ijcai22.sty is NOT the same as previous years'

\usepackage{ijcai25}
\newcommand{\citep}[1]{\citeauthor{#1} [\citeyear{#1}]}


% Use the postscript times font!
\usepackage{times}
\usepackage{soul}
\usepackage{url}
% \usepackage[hidelinks]{hyperref}
\usepackage[utf8]{inputenc}
\usepackage[small]{caption}
\usepackage{graphicx}
\usepackage{amsthm}
\usepackage{booktabs}
\usepackage{algorithm}
\usepackage{algorithmic}
\usepackage[switch]{lineno}
\usepackage{tabularx}

\usepackage{float}
\usepackage{amsmath,amsfonts}
\usepackage{algorithmic}
\usepackage{times}
\usepackage{latexsym}
\usepackage{xtab,booktabs}
\usepackage{tabularx}
\usepackage{multirow}

\usepackage[T1]{fontenc}
\usepackage{tikz}
\usepackage{xcolor} 
\usepackage[colorlinks=true,     
            linkcolor=black,     
            citecolor=black,      
            urlcolor=blue      
           ]{hyperref}

\usepackage{forest}
\usetikzlibrary{graphs}
\usepackage[utf8]{inputenc}
\usepackage{booktabs}
\usepackage{ragged2e}
% \usetikzlibrary{trees}
\usepackage{microtype}
\usepackage{multirow}
\usepackage{makecell}
\usepackage{url}
\usepackage{colortbl}       % the color of table
\usepackage{threeparttable}
\usepackage{enumitem}

\definecolor{hidden-draw}{RGB}{251,239,214}
\definecolor{hidden-orange}{RGB}{218, 97, 91}
\definecolor{lightred}{RGB}{220,92,96}
\definecolor{deepblue}{RGB}{125,174,224}
\definecolor{lightpurp}{RGB}{179,149,189}
\definecolor{lightpurple}{RGB}{130, 132, 131}
\definecolor{lightgray}{gray}{0.9}

\definecolor{hiddenc1}{RGB}{59, 118, 122}
\definecolor{hiddenc2}{RGB}{69,105,144}
\definecolor{hiddenc3}{RGB}{130,130,170}

\definecolor{hid-vae}{RGB}{251,239,214}
\definecolor{hid-gnn}{RGB}{179,149,189}
\definecolor{hid-trans}{RGB}{122, 199,226}
\definecolor{hid-dm}{RGB}{225, 225, 255}
\definecolor{hid-llm}{RGB}{84,190,170}
\definecolor{hid-ssl}{RGB}{176,217,146}
\definecolor{hid-dms}{RGB}{238, 144, 59}
\newcommand{\hx}[1]{{\bf\color{cyan}[{\sc WHX:} #1]}}
\newcommand{\zz}[1]{{\bf\color{red}[{\sc SZZ:} #1]}}


% Comment out this line in the camera-ready submission
% \linenumbers

\urlstyle{same}

% the following package is optional:
%\usepackage{latexsym}

% See https://www.overleaf.com/learn/latex/theorems_and_proofs
% for a nice explanation of how to define new theorems, but keep
% in mind that the amsthm package is already included in this
% template and that you must *not* alter the styling.
\newtheorem{example}{Example}
\newtheorem{theorem}{Theorem}
\newcommand{\upcite}[1]{\textsuperscript{\textsuperscript{\cite{#1}}}}

\pdfinfo{
/TemplateVersion (IJCAI.2025.0)
}


\title{Beyond the Time Domain: Recent Advances on Frequency\\ Transforms in Time Series Analysis}


\author{
Qianru Zhang$^1$\thanks{Equal contribution. $^\dagger$Corresponding author.}\and
Peng Yang$^{1,*}$\and
Honggang Wen$^{1,*}$\and
Xinzhu Li$^{1,*}$
\and
Haixin Wang$^{2,*}$\and\\
Fang Sun$^{2,*}$
\and
Zezheng Song$^3$
\and
Zhichen Lai$^4$
\and
Rui Ma$^6$
\and
Ruihua Han$^{1}$
\and\\
Tailin Wu$^5$
\and
Siu-Ming Yiu$^{1,\dagger}$
\and
Yizhou Sun$^2$
\and
Hongzhi Yin$^{7,\dagger}$\\
\affiliations
$^1$The University of Hong Kong (HKU),
$^2$University of California, Los Angeles (UCLA),
$^3$University of Maryland, College Park (UMCP),
$^4$Aalborg University,
$^5$Westlake University,
$^7$The University of Queensland (UQ),
$^6$Microsoft
}





\begin{document}

\maketitle

\begin{abstract}
    The field of time series analysis has seen significant progress, yet traditional methods predominantly operate in temporal or spatial domains, overlooking the potential of frequency-based representations. This survey addresses this gap by providing the first comprehensive review of frequency transform techniques-Fourier, Laplace, and Wavelet Transforms-in time series. We systematically explore their applications, strengths, and limitations, offering a comprehensive review and an up-to-date pipeline of recent advancements. By highlighting their transformative potential in time series applications including finance, molecular, weather, etc. This survey serves as a foundational resource for researchers, bridging theoretical insights with practical implementations. A curated GitHub repository further supports reproducibility and future research.
\end{abstract}

\section{Introduction}


\begin{figure}[t]
\centering
\includegraphics[width=0.6\columnwidth]{figures/evaluation_desiderata_V5.pdf}
\vspace{-0.5cm}
\caption{\systemName is a platform for conducting realistic evaluations of code LLMs, collecting human preferences of coding models with real users, real tasks, and in realistic environments, aimed at addressing the limitations of existing evaluations.
}
\label{fig:motivation}
\end{figure}

\begin{figure*}[t]
\centering
\includegraphics[width=\textwidth]{figures/system_design_v2.png}
\caption{We introduce \systemName, a VSCode extension to collect human preferences of code directly in a developer's IDE. \systemName enables developers to use code completions from various models. The system comprises a) the interface in the user's IDE which presents paired completions to users (left), b) a sampling strategy that picks model pairs to reduce latency (right, top), and c) a prompting scheme that allows diverse LLMs to perform code completions with high fidelity.
Users can select between the top completion (green box) using \texttt{tab} or the bottom completion (blue box) using \texttt{shift+tab}.}
\label{fig:overview}
\end{figure*}

As model capabilities improve, large language models (LLMs) are increasingly integrated into user environments and workflows.
For example, software developers code with AI in integrated developer environments (IDEs)~\citep{peng2023impact}, doctors rely on notes generated through ambient listening~\citep{oberst2024science}, and lawyers consider case evidence identified by electronic discovery systems~\citep{yang2024beyond}.
Increasing deployment of models in productivity tools demands evaluation that more closely reflects real-world circumstances~\citep{hutchinson2022evaluation, saxon2024benchmarks, kapoor2024ai}.
While newer benchmarks and live platforms incorporate human feedback to capture real-world usage, they almost exclusively focus on evaluating LLMs in chat conversations~\citep{zheng2023judging,dubois2023alpacafarm,chiang2024chatbot, kirk2024the}.
Model evaluation must move beyond chat-based interactions and into specialized user environments.



 

In this work, we focus on evaluating LLM-based coding assistants. 
Despite the popularity of these tools---millions of developers use Github Copilot~\citep{Copilot}---existing
evaluations of the coding capabilities of new models exhibit multiple limitations (Figure~\ref{fig:motivation}, bottom).
Traditional ML benchmarks evaluate LLM capabilities by measuring how well a model can complete static, interview-style coding tasks~\citep{chen2021evaluating,austin2021program,jain2024livecodebench, white2024livebench} and lack \emph{real users}. 
User studies recruit real users to evaluate the effectiveness of LLMs as coding assistants, but are often limited to simple programming tasks as opposed to \emph{real tasks}~\citep{vaithilingam2022expectation,ross2023programmer, mozannar2024realhumaneval}.
Recent efforts to collect human feedback such as Chatbot Arena~\citep{chiang2024chatbot} are still removed from a \emph{realistic environment}, resulting in users and data that deviate from typical software development processes.
We introduce \systemName to address these limitations (Figure~\ref{fig:motivation}, top), and we describe our three main contributions below.


\textbf{We deploy \systemName in-the-wild to collect human preferences on code.} 
\systemName is a Visual Studio Code extension, collecting preferences directly in a developer's IDE within their actual workflow (Figure~\ref{fig:overview}).
\systemName provides developers with code completions, akin to the type of support provided by Github Copilot~\citep{Copilot}. 
Over the past 3 months, \systemName has served over~\completions suggestions from 10 state-of-the-art LLMs, 
gathering \sampleCount~votes from \userCount~users.
To collect user preferences,
\systemName presents a novel interface that shows users paired code completions from two different LLMs, which are determined based on a sampling strategy that aims to 
mitigate latency while preserving coverage across model comparisons.
Additionally, we devise a prompting scheme that allows a diverse set of models to perform code completions with high fidelity.
See Section~\ref{sec:system} and Section~\ref{sec:deployment} for details about system design and deployment respectively.



\textbf{We construct a leaderboard of user preferences and find notable differences from existing static benchmarks and human preference leaderboards.}
In general, we observe that smaller models seem to overperform in static benchmarks compared to our leaderboard, while performance among larger models is mixed (Section~\ref{sec:leaderboard_calculation}).
We attribute these differences to the fact that \systemName is exposed to users and tasks that differ drastically from code evaluations in the past. 
Our data spans 103 programming languages and 24 natural languages as well as a variety of real-world applications and code structures, while static benchmarks tend to focus on a specific programming and natural language and task (e.g. coding competition problems).
Additionally, while all of \systemName interactions contain code contexts and the majority involve infilling tasks, a much smaller fraction of Chatbot Arena's coding tasks contain code context, with infilling tasks appearing even more rarely. 
We analyze our data in depth in Section~\ref{subsec:comparison}.



\textbf{We derive new insights into user preferences of code by analyzing \systemName's diverse and distinct data distribution.}
We compare user preferences across different stratifications of input data (e.g., common versus rare languages) and observe which affect observed preferences most (Section~\ref{sec:analysis}).
For example, while user preferences stay relatively consistent across various programming languages, they differ drastically between different task categories (e.g. frontend/backend versus algorithm design).
We also observe variations in user preference due to different features related to code structure 
(e.g., context length and completion patterns).
We open-source \systemName and release a curated subset of code contexts.
Altogether, our results highlight the necessity of model evaluation in realistic and domain-specific settings.





% 
\tikzstyle{my-box}=[
    rectangle,
    draw=hidden-draw,
    rounded corners,
    align=left,
    text opacity=1,
    minimum height=1.5em,
    minimum width=5em,
    inner sep=2pt,
    fill opacity=.8,
    line width=0.8pt,
]
\tikzstyle{leaf-head}=[my-box, minimum height=1.5em,
    draw=hidden-orange, % 调颜色
    % fill=hidden-draw,  % 调颜色
    text=black, font=\normalsize,
    inner xsep=2pt,
    inner ysep=4pt,
    line width=0.8pt,
]
\tikzstyle{leaf-task}=[my-box, minimum height=2.5em,
    draw=hidden-orange, % 调颜色
    % fill=hidden-draw,  % 调颜色
    text=black, font=\normalsize,
    inner xsep=2pt,
    inner ysep=4pt,
    line width=0.8pt,
]
\tikzstyle{leaf-taska}=[my-box, minimum height=2.5em,
    draw=hidden-orange, % 调颜色
    % fill=hidden-draw,  % 调颜色
    text=black, font=\normalsize,
    inner xsep=2pt,
    inner ysep=4pt,
    line width=0.8pt,
]
\tikzstyle{modelnode-task1}=[my-box, minimum height=1.5em,
    draw=hidden-orange, % 调颜色
    fill=hidden-draw,  % 调颜色
    text=black, font=\normalsize,
    inner xsep=2pt,
    inner ysep=4pt,
    line width=0.8pt,
]
\tikzstyle{leaf-task10}=[my-box, minimum height=1.0em,
    draw=hidden-orange, % 调颜色
    % fill=gray!30,  % 调颜色
    text=black, font=\normalsize,
    inner xsep=2pt,
    inner ysep=4pt,
    line width=0.6pt,
]
\tikzstyle{modelnode-task6}=[my-box, minimum height=1.5em,
    draw=hidden-orange, % 调颜色
    % fill=red!25,  % 调颜色
    text=black, font=\normalsize,
    inner xsep=2pt,
    inner ysep=4pt,
    line width=0.8pt,
]
\tikzstyle{modelnode-task7}=[my-box, minimum height=1.5em,
    draw=hidden-orange, % 调颜色
    % fill=red!25,  % 调颜色
    text=black, font=\normalsize,
    inner xsep=2pt,
    inner ysep=4pt,
    line width=0.8pt,
]
\tikzstyle{modelnode-task8}=[my-box, minimum height=1.5em,
    draw=hidden-orange, % 调颜色
    % fill=red!25,  % 调颜色
    text=black, font=\normalsize,
    inner xsep=2pt,
    inner ysep=4pt,
    line width=0.8pt,
]
\tikzstyle{modelnode-task9}=[my-box, minimum height=1.5em,
    draw=hidden-orange, % 调颜色
    % fill=red!25,  % 调颜色
    text=black, font=\normalsize,
    inner xsep=2pt,
    inner ysep=4pt,
    line width=0.8pt,
]
\tikzstyle{leaf-paradigms}=[my-box, minimum height=2.5em,
    draw=hidden-orange, % 调颜色
    % fill=hiddenc2,  % 调颜色
    text=black, font=\normalsize,
    inner xsep=2pt,
    inner ysep=4pt,
    line width=0.8pt,
]
\tikzstyle{leaf-others}=[my-box, minimum height=2.5em,
    %fill=hidden-pink!80,
    draw=hidden-orange, % 调颜色
    % fill=hiddenc3,  % 调颜色
    text=black, font=\normalsize,
    inner xsep=2pt,
    inner ysep=4pt,
    line width=0.8pt,
]
\tikzstyle{leaf-other}=[my-box, minimum height=2.5em,
    %fill=hidden-pink!80,
    draw=orange!80, % 调颜色
    fill=orange!15,  % 调颜色
    text=black, font=\normalsize,
    inner xsep=2pt,
    inner ysep=4pt,
    line width=0.8pt,
]
\tikzstyle{modelnode-task}=[my-box, minimum height=1.5em,
    draw=black, % 调颜色
    % fill=red!25,  % 调颜色
    text=black, font=\normalsize,
    inner xsep=2pt,
    inner ysep=4pt,
    line width=0.8pt,
]
\tikzstyle{modelnode-paradigms}=[my-box, minimum height=1.5em,
    draw=black, % 调颜色
    % fill=blue!15,  % 调颜色
    text=black, font=\normalsize,
    inner xsep=2pt,
    inner ysep=4pt,
    line width=0.8pt,
]
\tikzstyle{modelnode-others}=[my-box, minimum height=1.5em,
    draw=black, % 调颜色
    % fill=green!15,  % 调颜色
    text=black, font=\normalsize,
    inner xsep=2pt,
    inner ysep=4pt,
    line width=0.8pt,
]
\tikzstyle{modelnode-other}=[my-box, minimum height=1.5em,
    draw=black, % 调颜色
    % fill=orange!15,  % 调颜色
    text=black, font=\normalsize,
    inner xsep=2pt,
    inner ysep=4pt,
    line width=0.8pt,
]
\begin{figure*}[!th]
    \centering
    \resizebox{1\textwidth}{!}{
        \begin{forest}
            for tree={
                grow=east,
                reversed=true,
                anchor=base west,
                parent anchor=east,
                child anchor=west,
                base=left,
                font=\normalsize,
                rectangle,
                draw=hidden-draw,
                rounded corners,
                align=center,
                minimum width=1em,
                edge+={darkgray, line width=1pt},
                s sep=3pt,
                inner xsep=0pt,
                inner ysep=3pt,
                line width=0.8pt,
                ver/.style={rotate=90, child anchor=north, parent anchor=south, anchor=center},
                edge path={
                    \noexpand\path [draw, \forestoption{edge}]
                    (!u.parent anchor) -- ++(5pt,0) |- (.child anchor)\forestoption{edge label};
                },
            },
            [% 中括号之间不要换行!!!
                Learning in Frequency Domain,leaf-head,ver
                % [
                %      Low-level Vision \\(\textbf{Sec.~\ref{sec:low-level-vision}}),leaf-task,text width=9em
                %     [
                %         Convolution-Based, leaf-task10, text width=8.5em
                %         [\cite{chang2000adaptive,chappelier2006oriented,veena2016least}{, }\\WT\cite{chappelier2006oriented}{, }FPA\cite{chen2024large}\\GFD\cite{wang2017virtualization}{,}WSTID\cite{veena2016least}{, }LSWR\cite{dixon2016aerial}\\FDCNN\cite{janssens2016convolutional} {, }DWT-CNN Fusion\cite{avci2024mfif}{, }DC\\-NN\cite{wang2017virtualization}             \cite{rani2023atcnn,liu2024wavefusionnet,xu2024semantic},modelnode-task1, text width=33em]
                %     ]
                %     [
                %        Transformer-Based, leaf-task10, text width=8.5em
                %         [TPR\cite{li2023efficient}{, }TransMEF\cite{qu2022transmef}{, }
                %         \cite{jiang2024frft}{, }\\\cite{cao2023cfmb}{, }WT-OCT \cite{chen2024waveformer}\\
                %         Laplacian-Former\cite{azad2023laplacian}{, }\cite{korkmaz2024training}\\
                %         \cite{chibani2003redundant,mao2018multi}{, }FFT\cite{zhu2023attention}
                %         \\\cite{kong2023efficient}{, }WECT\cite{wang2024versatile}{, }
                %         DWT\cite{zhang2024waveletformernet}{, }\\
                %         WaveletFormerNet\cite{chen2021pre,liang2021swinir,zhang2022swinfir}{, }\\
                %         Fourier Conv-Transformer Cross-Scale\cite{zhang2022low,zhang2023cross}
                %         , modelnode-task1, text width=33em]  
                %     ]
                % ]
                % [
                %     High-level Vision \\(\textbf{Sec.~\ref{sec:high-level-vision}}), leaf-paradigms,text width=9em
                %     [
                %         Convolution-Based, leaf-task10, text width=8.5em
                %         [DSNet\cite{shang2020dense,liu2022partial}{, }FDCNN\cite{goh2021frequency}{, }HWNN\\\cite{nong2021hypergraph}{, }HWD\cite{xu2023haar}{, }SAR-CNN\cite{heiselberg2022sar}\\FFPF~\cite{lingyun2022fast}{, }DCT\cite{luo2024frequency}{, }FFANet\cite{zhou2024frequency}\\ FT\&CNN~\cite{labbihi2024combining} {,}WTConv\cite{finder2024wavelet}{, }Riesz–Laplace\\-Transform\cite{unser2009multiresolution}{, }HPF-MMCA\cite{magid2021dynamic}, modelnode-task1, text width=33em]
                %     ]
                %     [
                %        Transformer-Based, leaf-task10, text width=8.5em
                %         [ TMG-AFM \cite{chen2020generative,alexey2021image,yang2023discrete}{, }DETR\\\cite{carion2020end,zhu2020deformable}{, }FSA\cite{zhang2023decomformer}{, }MetaISP\\ \cite{chen2021pre}{, }APT\cite{huang2022atrous}{, }GWT\cite{bastos2023learnable}{, }\\DWP\cite{yang2023discrete}{, }AWT\cite{huang2021adaptive}{, }SAN\\\cite{kreuzer2021rethinking}{, }Spectformer\cite{patro2023spectformer}{, }SPT-SEG\cite{xu2024spectral}\\SVT\cite{patro2024scattering}{, }FreqDiMFT\cite{zhang2024frequency}, modelnode-task1, text width=33em]
                %     ]
                % ]  
                [
                    Time Series (\textbf{Sec.~\ref{sec:time-series}}), leaf-others,text width=9em
                    [
                        Convolution-Based, leaf-task10, text width=8.5em
                        [\cite{yu2024method,park2021fast,zhou2024fourier}{, }Tslanet\\\cite{eldele2024tslanet}\cite{kim2024neural,lange2021fourier,zhang2024frnet}\\\cite{cai2024msgnet}{, }RFF\cite{tompkins2018fourier}, modelnode-task1, text width=33em]
                    ]
                    [
                       Transformer-Based, leaf-task10, text width=8.5em
                        [LSTF\cite{ma2023long,zhou2024fourier,ni2024time}{, }TST\\\cite{jin2022time}\cite{chen2023lightweight,tran2023fourier}{, }ETSFormer\\\cite{woo2022etsformer}\cite{kang2023electric,yang2024fedaf}{, }FEDformer~\\\cite{zhou2022fedformer}{, } CRT\cite{zhang2023self}{,}W-Transformer\cite{sasal2022w}\\GRAU\cite{yang2024graformer,zhang2023self,wang2024card}{, }, modelnode-task1, text width=33em]  
                    ]
                ]
                [
                    Spatio-Temporal \\Dynamics (\textbf{Sec.~\ref{sec:spatio-temporal-dynamics}}), leaf-others,text width=9em
                    [
                        Convolution-Based, leaf-task10, text width=8.5em   [FNO~\cite{li2020fourier}{, }\cite{kovachki2021universal,tran2021factorized}\\\cite{song2022high,liu2024neural,zhao2024recfno}\\~\cite{zhao2023local}{,} S-FNOs\cite{bonev2023spherical}{, }U-FNO\cite{wen2022u}{, }\\FNON\cite{guan2023fourier}{, } FNOF\cite{kabri2023resolution}, modelnode-task1, text width=33em]
                    ]
                    [
                       Transformer-Based, leaf-task10, text width=8.5em
                        [\cite{guibas2021efficient,li2023fourier,cao2021choose}{,}Gnot~\cite{hao2023gnot}{, }\\IUFNO\cite{li2024transformer,yang2024enhancing}{, }\\AFNO\cite{guibas2021adaptive}{, }GNOT\cite{hao2023gnot}, modelnode-task1, text width=33em]
                    ]
                ]
                % [
                %     Graph (\textbf{Sec.~\ref{sec:others}}), leaf-others,text width=9em
                %     [
                %         Convolution-Based, leaf-task10, text width=8.5em
                %         [\cite{bruna2014spectral,defferrard2016convolutional,kipf2017semi}\\TIGraNet\cite{khasanova2017graph,lee2021fnet}{, }\\GWNN\cite{xu2023haar}{, }\cite{yadav2021graph}{,}TPAMI\cite{shivakumara2011laplacian}\\ML-GW\cite{rustamov2013wavelets}{, }SGWT\cite{pilavci2019spectral}, modelnode-task1, text width=33em]
                %     ]
                %     [
                %        Transformer-Based, leaf-task10, text width=8.5em
                %         [\cite{wang2022powerful}{, }MGT\cite{ngo2023multiresolution}{, }FSC\cite{jin2024transformer}{, }\\BWN\cite{jeong2024brainwavenet}{, } LSGC\cite{chen2024multi}{, }, modelnode-task1, text width=33em]
                %     ]
                %     % [
                %     %    Diffusion-Based, leaf-task10, text width=8.5em
                %     %     [Wavelet\cite{guth2022wavelet}{,} Wavelet~\cite{phung2023wavelet}{, }~\cite{jiang2023low}{, }\\CWR\cite{hu2024neural}\cite{hu2024neural,huang2024wavedm}{,} DiffLL\\~\cite{jiang2023low}{, }WaveDM\cite{huang2024wavedm}{, } CFWD\cite{xue2024low}, modelnode-task1, text width=33em]
                %     % ]
                % ]
            ]
        \end{forest}
    }
    \caption{Taxonomy of learning in frequency domain (overall version)}
    \label{fig_taxonomy1}
\end{figure*}
% \section{Domain Transform Methods}

% This section provides a concise overview of three fundamental transform methods: Fourier Transform, Laplace Transform, and Wavelet Transform.

% \noindent \textbf{Fourier Transform}. Fourier transform~\cite{nussbaumer1982fast} is a fundamental tool for analyzing the frequency content of signals by decomposing a time-domain signal into sinusoidal components, revealing the frequencies and their magnitudes. 
% And the process of Fourier Transform is shown as follows:
% \begin{equation}
% \begin{aligned}
% X(f) = \int_{-\infty}^{\infty} x(t)e^{-j2\pi ft} dt, \quad j = \sqrt{-1}
% \end{aligned}
% \end{equation}
% where $X(f)$ is the continuous frequency spectrum, $x(t)$ is the continuous-time signal, and $f$ is the continuous frequency variable.


% The Continuous-Time Fourier Transform (CTFT) provides the continuous frequency spectrum for continuous-time signals, while the Discrete Fourier Transform (DFT) gives a discrete frequency representation for discrete-time signals. For computational efficiency, the Fast Fourier Transform (FFT) reduces the direct DFT's $\mathcal{O}(N^2)$ complexity to $\mathcal{O}(N \log N)$, making it essential in fields like signal processing and telecommunications. The FFT achieves this speedup through recursive decomposition, leveraging symmetries in trigonometric calculations to facilitate rapid frequency-domain analysis, which is critical in applications requiring real-time data processing.



% \noindent \textbf{Laplace Transform}. Laplace transform~\cite{schiff2013laplace} is a crucial tool for analyzing linear time-invariant (LTI) systems, converting time-domain functions into functions of a complex frequency variable $s$. The unilateral (one-sided) Laplace transform of a function $f(t)$, defined for $t \ge 0$, is shown:

% \begin{equation} % \label{eq:laplace}
%     F(s) = \mathcal{L}\{f(t)\} = \int_0^\infty f(t)e^{-st} dt
% \end{equation}
% where $F(s)$ represents the Laplace transform of $f(t)$. $s$ is a complex frequency variable, often expressed as $s = \sigma + j\omega$, with $\sigma$ representing the real part (related to exponential decay) and $\omega$ the imaginary part (related to frequency). The integral's lower limit of 0 reflects the typical application to causal signals (signals that are zero for $t < 0$).




% \noindent \textbf{Wavelet Transform}. The wavelet transform~\cite{meyer1989wavelets} is a mathematical tool used in signal processing that allows for the decomposition of a signal in the time-frequency domain. The following equation shows the CWT, which decomposes a continuous signal $f(t)$ into frequency components localized in time:
% \begin{equation}
% \begin{aligned}
%     F(\tau, s) = \frac{1}{\sqrt{\vert s \vert}} \int_{-\infty}^{\infty} f(t) \psi^* \left( \frac{t - \tau}{s} \right) dt
% \end{aligned}
% \end{equation}
% where $\tau$ is the translation parameter, determining the position of the wavelet along the time axis, and $s$ is the scale parameter, which controls the width of the wavelet and thus affects the frequency resolution.

% Unlike the Fourier transform, the wavelet transform can reveal both the frequency characteristics of a signal and the time distribution of these frequency components, making it particularly effective for analyzing non-stationary signals. According to the computation methods, wavelet transforms can be divided into the continuous wavelet transform (CWT) and discrete wavelet transform (DWT). The CWT computes wavelet coefficients by analyzing a signal across different frequencies and time positions, providing a detailed energy distribution at the expense of high computational cost. The DWT performs multi-scale decomposition of a signal, separating it into different frequency bands while maintaining both frequency and time localization. Unlike the CWT, the DWT is computationally efficient and suitable for digital signal processing. 


\section{Problem Definition}
\label{sec:prob}
The input consists of a long-term time series $\mathcal{X} = (x_1, \dots, x_L) \in \mathbb{R}^{L \times V}$, where $L$ is the historical window length and $V$ is the number of variables. The corresponding ground truth for the prediction is $\mathcal{Y} = (x_{L+1}, \dots, x_{L+H}) \in \mathbb{R}^{H \times V}$, with $H$ representing the prediction horizon.

\noindent \textbf{Frequency Transform.} To more effectively capture periodic patterns inherent in time series data, numerous studies have employed transformations that convert the data into the frequency domain. Formally, we denote the frequency domain transformation by a generic operator $\text{FT}(\cdot)$, defined as follows:
\begin{equation}
\begin{aligned}
    \mathbf{X}' = \text{FT}(\mathcal{X})
\end{aligned}
\end{equation}
The primary objective of learning in the frequency domain is to capture periodic information in time series while preserving temporal dependencies. We provide a pipeline for time series analysis through frequency transformation in Figure~\ref{fig:intro}. 
% \tikzset{
        my node/.style={
            draw,
            align=left,
            thin,
            text width=2.5cm, 
            % minimum height=1cm,
            rounded corners=3,
        },
        my leaf/.style={
            draw,
            align=left,
            thin,
            % minimum width=1cm,
            text width=4.5cm, 
            % text height=1cm, 
            % minimum height=0.5cm,
            rounded corners=3,
        }
}
\forestset{
  every leaf node/.style={
    if n children=0{#1}{}
  },
  every tree node/.style={
    if n children=0{minimum width=1em}{#1}
  },
}
\begin{forest}
    for tree={%
        % my node,
        every leaf node={my leaf, font=},
        every tree node={my node, font=\small, l sep-=4.5pt, l-=1.pt},
        anchor=west,
        inner sep=2pt,
        % l = 10pt,
        l sep=10pt, % control leaf to parent nodes gaps (horizontal)
        s sep=5pt, % control node gaps (vertical)
        fit=tight,
        grow'=east,
        edge={thick},
        parent anchor=east,
        child anchor=west,
        if n children=0{tier=last}{},
        edge path={
            \noexpand\path [draw, \forestoption{edge}] (!u.parent anchor) -- +(5pt,0) |- (.child anchor)\forestoption{edge label};
        },
        if={isodd(n_children())}{
            for children={
                if={equal(n,(n_children("!u")+1)/2)}{calign with current}{}
            }
        }{}
    }
    % text width=3.5cm
    % {Organizational Structure}, draw=gray, color=gray!100, fill=gray!15, very thick, text=black, text width=2.1cm,
    % [draw=none, fill=none, text width=0, minimum height=0, inner sep=0pt, outer sep=0pt,
    [{Organizational Structure}, draw=gray, color=gray!100, fill=gray!15, very thick, text=black, text width=2.1cm,
        [\cref{early_srs} Foundations of Reasoning LLMs, color=cyan!100, fill=cyan!15, very thick, text=black, text width=5.1cm
            [\cref{f_llm} Foundational LLMs, color=cyan!100, fill=cyan!15, very thick, text=black, text width=4.5cm
            ]
            [\cref{symb_exp} Symbolic Logic Systems, color=cyan!100, fill=cyan!15, very thick, text=black, text width=4.5cm
            ]
            [\cref{mcts} Monte Carlo Tree Search, color=cyan!100, fill=cyan!15, very thick, text=black, text width=4.5cm
            ]
            [\cref{rl} Reinforcement Learning, color=cyan!100, fill=cyan!15, very thick, text=black, text width=4.5cm
            ]
        ]
        [\cref{replication} Blueprinting Reasoning LLMs, color=lightcoral!100, fill=lightcoral!15, very thick, text=black, text width=5.1cm
            [\cref{o1_features} Feature Analysis, color=lightcoral!100, fill=lightcoral!15, very thick, text=black, text width=4.9cm
                [\cref{output_behaviour} Output Behaviour, color=lightcoral!100, fill=lightcoral!15, very thick, text=black, tier=Task, text width=5.1cm
                ]
                [\cref{dynamic_perspective} Training Dynamics, color=lightcoral!100, fill=lightcoral!15, very thick, text=black, tier=Task, text width=5.1cm
                ]
            ]
            [\cref{foundations} Core Method, color=lightcoral!100, fill=lightcoral!15, very thick, text=black, text width=4.9cm
                [\cref{structure_search} Structure Search, color=lightcoral!100, fill=lightcoral!15, very thick, text=black, tier=Task, text width=5.1cm
                ]
                [\cref{prm} Reward Modeling, color=lightcoral!100, fill=lightcoral!15, very thick, text=black, tier=Task, text width=5.1cm
                ]
                [\cref{self-improve} Self Improvement, color=lightcoral!100, fill=lightcoral!15, very thick, text=black, tier=Task, text width=5.1cm
                ]
                [\cref{macro_action} Macro Action, color=lightcoral!100, fill=lightcoral!15, very thick, text=black, tier=Task, text width=5.1cm
                ]
                [\cref{rl_supervise} Reinforcement Fine-Tuning, color=lightcoral!100, fill=lightcoral!15, very thick, text=black, tier=Task, text width=5.1cm
                ]
            ]
            [\cref{evolutionary} Reasoning LLMs Evolution, color=lightcoral!100, fill=lightcoral!15, very thick, text=black, text width=4.9cm
            ]
        ]
        [\cref{benchmark} Benchmarking Reasoning LLMs, color=darkpastelgreen!100, fill=darkpastelgreen!15, very thick, text=black, text width=5.1cm
            [\cref{benchmark_category} Benchmark Categories, color=darkpastelgreen!100, fill=darkpastelgreen!15, very thick, text=black, text width=4.55cm
            ]
            [\cref{metrics} Evaluation Metrics, color=darkpastelgreen!100, fill=darkpastelgreen!15, very thick, text=black, text width=4.55cm
            ]
            [\cref{performance_compare} Performance Comparison, color=darkpastelgreen!100, fill=darkpastelgreen!15, very thick, text=black, text width=4.55cm
            ]
        ]
    ]
\end{forest}



\section{Approaches for Frequency Transform}\label{sec:time-series}
\label{sec:pre}
% need to compress  (simpler intro and insight of each trasnform)

Frequency transforms are categorized into Fourier, wavelet, and Laplace transforms based on their formulations and applications. To provide a comprehensive overview, Table~\ref{tab:frequency_transform_methods} summarizes representative frequency transform methods. 


\begin{table*}[!ht] % 使用 !ht 替代 htbp,!ht 会强制表格尽可能出现在当前位置,而不强制它浮动到顶部或底部。
    \centering
    \caption{Summary of representative frequency transform methods in our framework.}
    \label{tab:frequency_transform_methods}
    \vspace{-0.3cm}
    \scriptsize
    \setlength{\tabcolsep}{6pt}
    \resizebox{\textwidth}{!}{
        \fontsize{12}{1.25\baselineskip}\selectfont % 设置字体大小为12pt, 1.25倍行距
        \begin{tabular}{c|c|c|l} 
        \toprule[1.2pt]
        \textbf{\makecell[c]{Frequency \\ Transform}} & \textbf{\makecell[c]{Categories and \\ Representative Methods}} & \textbf{Expression} & \textbf{Notes (Advantages \& Disadvantages)} \\ 
        \midrule[1.2pt]
        \multirow{17}{*}{\makecell[c]{\\ \textbf{Fourier} \\ \textbf{Transform}}} 
            & \makecell[c]{Discrete Fourier Transform (DFT)} & \makecell[c]{$X[k] = \sum_{n=0}^{L-1} x[n] e^{-j\frac{2\pi}{L}kn}$} & 
            \makecell[l]{
                \textcolor{green}{\checkmark} Well-suited for stationary signals \\
                \textcolor{green}{\checkmark} Captures global frequency components efficiently \\
                \textcolor{red}{$\times$} Cannot handle non-stationary signals \\
                \textcolor{red}{$\times$} Loses time-domain information (no localization) \\
            } \\
            \cmidrule(r){2-4}
            & \makecell[c]{Continuous Fourier Transform (CFT)} & 
            \makecell[c]{$X(f) = \int_{-\infty}^{\infty} x(t)e^{-j2\pi ft} dt$} & 
            \makecell[l]{
                \textcolor{green}{\checkmark} Used for theoretical frequency analysis \\
                \textcolor{green}{\checkmark} Provides continuous spectrum analysis \\
                \textcolor{red}{$\times$} Not practical for discrete signals \\
            } \\ 
            \cmidrule(r){2-4}
            & \makecell[c]{Fast Fourier Transform (FFT)} & - &  
            \makecell[l]{
                \textcolor{green}{\checkmark} Fast computation ($\mathcal{O}(Nlog N)$ complexity) \\
                \textcolor{green}{\checkmark} Used in real-time applications \\
                \textcolor{red}{$\times$} Shares the same limitations as DFT \\
            } \\
            \cmidrule(r){2-4}
            & \makecell[c]{Short-Time Fourier Transform (STFT)} &  
            \makecell[c]{$X(t, f) = \int_{-\infty}^{\infty} x(\tau) w(\tau - t) e^{-j2\pi f\tau} d\tau$} &  
            \makecell[l]{
                \textcolor{green}{\checkmark} Allows frequency analysis with time localization \\
                \textcolor{green}{\checkmark} Common in speech and signal processing \\
                \textcolor{red}{$\times$} Limited resolution due to fixed window size \\
                \textcolor{red}{$\times$} Trade-off between time and frequency resolution \\
            } \\
            \cmidrule(r){2-4}
            & \makecell[c]{Fractional Fourier Transform (FrFT)} & - &  
            \makecell[l]{
                \textcolor{green}{\checkmark} Generalizes FT for non-stationary signals \\
                \textcolor{green}{\checkmark} Bridges time-frequency representation \\
                \textcolor{red}{$\times$} More complex and computationally intensive \\
            } \\
        \midrule
        \multirow{5}{*}{\makecell[c]{\textbf{Wavelet} \\ \textbf{Transform}}} 
            & \makecell[c]{Discrete Wavelet Transform (DWT)} & $D(a, b) = \frac{1}{\sqrt{b}} \sum_{m=0}^{p-1} f[t_m] \psi \left( \frac{t_m - a}{b} \right)$ &  
            \makecell[l]{
                \textcolor{green}{\checkmark} Captures both time and frequency information \\
                \textcolor{green}{\checkmark} Handles non-stationary signals well \\
                \textcolor{red}{$\times$} Requires careful wavelet selection \\
                \textcolor{red}{$\times$} High computational cost \\
            } \\
            \cmidrule(r){2-4}
            & Continuous Wavelet Transform (CWT) & \makecell[c]{$F(\tau, s) = \frac{1}{\sqrt{|s|}} \int_{-\infty}^{\infty} f(t) \psi^* \left( \frac{t - \tau}{s} \right) dt$} &  
            \makecell[l]{
                \textcolor{green}{\checkmark} Provides continuous time-frequency representation \\
                \textcolor{green}{\checkmark} Better suited for complex signals \\
                \textcolor{red}{$\times$} Computationally expensive \\
                \textcolor{red}{$\times$} Redundant representation due to continuous scaling \\
            } \\ 
            
        \midrule
        \multirow{5}{*}{\makecell[c]{\textbf{Laplace} \\ \textbf{Transform}}}  
            & \makecell[c]{Unilateral Laplace Transform} & \makecell[c]{$F(s) = \mathcal{L}\{f(t)\} = \int_0^\infty f(t)e^{-st} dt$} &  
            \makecell[l]{
                \textcolor{green}{\checkmark} Useful for control systems and differential equations \\
                \textcolor{green}{\checkmark} Helps analyze system stability \\
                \textcolor{red}{$\times$} Less common in traditional time series analysis \\
            } \\
            \cmidrule(r){2-4}
            & Bilateral Laplace Transform &  
            \makecell[c]{$F(s) = \int_{-\infty}^{\infty} f(t)e^{-st} dt$} &  
            \makecell[l]{
                \textcolor{green}{\checkmark} Generalizes the Fourier transform \\
                \textcolor{green}{\checkmark} Used in engineering and systems analysis \\
                \textcolor{red}{$\times$} Computationally intensive \\
            } \\
        \midrule
        \multirow{5}{*}{\makecell[c]{\textbf{Graph} \\ \textbf{Fourier} \\ \textbf{Transform}}}  
            & \makecell[c]{Spectral Graph Fourier Transform (GFT) } &  
            \makecell[c]{$X(\lambda) = U^T x$} &  
            \makecell[l]{
                \textcolor{green}{\checkmark} Extends Fourier Transform to graph data \\
                \textcolor{green}{\checkmark} Useful for irregularly structured time series \\
                \textcolor{red}{$\times$} Requires graph construction and eigen decomposition \\
            } \\
            \cmidrule(r){2-4}
            & \makecell[c]{Wavelet Graph Transform} & - &  
            \makecell[l]{
                \textcolor{green}{\checkmark} Provides localized frequency analysis on graphs \\
                \textcolor{green}{\checkmark} Used in social networks and bioinformatics \\
                \textcolor{red}{$\times$} More complex than traditional wavelets \\
            } \\
        \bottomrule[1.2pt]
        \end{tabular}}
    \vspace{-0.5cm}
\end{table*}


\subsection{Fourier Transform}

The Fourier transform converts a time-domain signal into its frequency-domain representation. Widely used variants include the Discrete Fourier Transform (DFT), Continuous Fourier Transform (CFT), and Fast Fourier Transform (FFT). DFT captures global frequency components efficiently but loses time-domain information and cannot handle non-stationary signals. CFT provides a continuous spectrum but is impractical for discrete signals. FFT offers fast computation with $\mathcal{O}(N \log N)$ complexity, making it suitable for real-time applications, though it shares DFT's limitations. Each method has unique advantages and trade-offs depending on the signal characteristics and application requirements. Other methods, such as the Short-Time Fourier Transform (STFT)~\cite{yao2019stfnets} and the Fractional Fourier Transform (FrFT)~\cite{kocc2022fractional}, address specific needs. STFT enables frequency analysis with time localization, commonly used in speech and signal processing, but involves a trade-off between time and frequency resolution. FrFT generalizes the Fourier transform for non-stationary signals but is more complex and computationally intensive.

To support graph data, extensions like the spectral graph Fourier transform~\cite{defferrard2016convolutional} and wavelet graph transform~\cite{xu2019graph} provide localized frequency analysis for irregularly structured time series. However, they require graph construction and eigen decomposition, and are more complex than traditional methods.

% The Fourier transform is a mathematical technique that converts a time-domain signal into its frequency-domain representation. For an input time series $(x_1, x_2, \cdots, x_L)$, its frequency spectrum $F(\omega_k)$ can be obtained through the discrete Fourier transform as follows~\cite{wen2020time}:
% \begin{equation}
% \begin{aligned}
%     F(\omega_k) = \frac{1}{L} \sum_{t=0}^{L-1} (x_t \cdot e^{-j \omega_k t}) = A(\omega_k) \cdot \mathrm{exp}[j \theta(\omega_k)],
% \end{aligned}
% \end{equation}
% where $\omega = \frac{2 \pi k}{L}$ is the angular frequency, $A(\omega_k)$ is the amplitude spectrum, $\theta(\omega_k)$ is the phase spectrum, and $j = \sqrt{-1}$.

% Recent studies highlight the potential of the Fourier transform as a data augmentation and feature engineering technique in time-series analysis. The most widely used Fourier transforms include the Discrete Fourier Transform (DFT) and Fast Fourier Transform (FFT). For instance, Alaa \textit{et al.}~\cite{alaa2021generative} and Zhang \textit{et al.}~\cite{zhang2022tfad} proposed DFT-based data augmentation techniques to increase labeled data and exploit the unique properties of time-series data. Woo \textit{et al.}~\cite{woo2022cost} learned trend representations in the time domain, while seasonal representations were modeled by a Fourier layer in the frequency domain. Zhou \textit{et al.}~\cite{zhou2022fedformer} and Galán-Sales \textit{et al.}~\cite{galan2023approach} demonstrated FFT's effectiveness as a feature engineering tool for improving time-series forecasting. In addition, advanced Fourier-based architectures, such as FourierGNN~\cite{yi2023fouriergnn} and neural Fourier transform~\cite{koren2024interpretable}, have been proposed, integrating multi-dimensional Fourier transforms, temporal convolutional networks, and graph-based models to enhance the accuracy and interpretability of time-series forecasting. 

% Fourier Neural Operators (FNOs) have proven effective for solving parametric partial differential equations and inspired extensions to diverse applications~\cite{guibas2021adaptive}. Building on this foundation, Li and Yang~\cite{li2023gafno} introduced GAFNO, a gated adaptive FNO specifically designed for time series analysis, leveraging adaptive mechanisms to capture multiscale dependencies. To further enhance time-series modeling, Cho \textit{et al.}~\cite{cho2024operator} proposed Branched Fourier Neural Operators (BFNOs), addressing the limitations of traditional FNOs with a more expressive architecture.



% Yi \textit{et al.}~\cite{yi2023fouriergnn} introduced FourierGNN, a novel graph-based architecture for multivariate time series forecasting, utilizing hypervariate graphs and Fourier Graph Operators to unify spatiotemporal dynamics, achieving efficient and expressive forecasting performance.

% Gao \textit{et al.}~\cite{gao2020robusttad} presents RobustTAD, a framework leveraging time-series decomposition and convolutional neural networks, including frequency-domain data augmentation, to enhance anomaly detection performance.

% Alaa et al.~\cite{alaa2021generative} introduced Fourier Flow, a likelihood model leveraging frequency-domain representations of time series via DFT and spectral filtering for flexible and efficient analysis.

% Zhang \textit{et al.}~\cite{zhang2022tfad} further proposed TFAD, a time-frequency analysis-based model that exploits both time and frequency domains, incorporating time-series decomposition and data augmentation to improve both performance and interpretability.

% Galán-Sales et al.~\cite{galan2023approach} investigated the potential of FFT as a feature engineering method to enhance the accuracy and efficiency of time-series forecasting models.

% Zhou \textit{et al.}~\cite{zhou2022fedformer} presented FEDformer, a Frequency Enhanced Decomposed Transformer that combines seasonal-trend decomposition and Fourier transform to efficiently capture global and detailed structures in time series, achieving superior forecasting performance with reduced complexity.

% Zhou \textit{et al.}~\cite{zhou2022film} introduced FiLM, a Frequency Improved Legendre Memory model that combines Legendre polynomial projections for historical approximation, Fourier projection for noise reduction, and low-rank approximation to enhance long-term time series forecasting.

% Koren \textit{et al.}~\cite{koren2024interpretable} presented the Neural Fourier Transform (NFT), which integrates multi-dimensional Fourier transforms with Temporal Convolutional Networks to enhance the accuracy and interpretability of multivariate time-series forecasting.

% Cao \textit{et al.}~\cite{cao2020spectral} introduced StemGNN, which leverages graph Fourier transform and DFT to jointly model inter-series correlations and temporal dependencies in the spectral domain for improved multivariate time-series forecasting. 


\subsection{Wavelet Transform}

% Wavelet transform transforms a time series using wavelets as basis functions to reduce data size or noise. For a continuous time series $f(t)$, it decomposes the signal into time-localized frequency components:
% \begin{equation}
% \begin{aligned}
%     F(\tau, s) = \frac{1}{\sqrt{\vert s \vert}} \int_{-\infty}^{\infty} f(t) \psi^* \left( \frac{t - \tau}{s} \right) dt,
% \end{aligned}
% \end{equation}
% where $\tau$ is the translation parameter, determining the position of the wavelet along the time axis, and $s$ is the scale parameter, which controls the width of the wavelet and thus affects the frequency resolution.

Wavelet transform uses wavelets as basis functions to transform a time series, reducing data size or noise. The most common wavelet transforms are the Discrete Wavelet Transform (DWT) and Continuous Wavelet Transform (CWT), both widely used in time-series analysis. DWT captures both time and frequency information, making it particularly effective for handling non-stationary signals. However, it requires careful wavelet selection and incurs a high computational cost. In contrast, CWT provides a continuous time-frequency representation, which is better suited for analyzing complex signals. However, CWT is computationally expensive and results in redundant representations due to continuous scaling.

Recent works have leveraged wavelet transforms for diverse tasks, such as optimizing time-frequency representations through non-linear filter-bank transformations~\cite{cosentino2020learnable}, isolating periodic components using the maximal overlap DWT~\cite{wen2021robustperiod}, and integrating wavelet methods into deep learning frameworks to capture both frequency and time-domain features~\cite{yang2023waveform}. Moreover,~\citep{liang2024waverora} introduced wavelet-based frameworks that leverage time-frequency features to enhance forecasting efficiency and accuracy.

% Furthermore, some methods combine the strengths of Fourier and wavelet transforms to improve time-series analysis. Zhou \textit{et al.}~\cite{zhou2022fedformer} proposed FEDformer, which integrates Fourier transform for capturing global patterns and wavelet transform for modeling local structures, achieving a balance between accuracy and computational efficiency. Liu \textit{et al.}~\cite{liu2024wftnet} introduced WFTNet, which incorporates both transforms with a periodicity-weighted coefficient to adaptively balance their contributions. These hybrid approaches effectively exploit the complementary strengths of Fourier and wavelet transforms, setting new benchmarks in long-term time-series forecasting.

\subsection{Laplace Transform}

% Laplace transform is a crucial tool for analyzing linear time-invariant systems, converting time-domain functions into functions of a complex frequency variable $s$. The unilateral Laplace transform of a function $f(t), t \ge 0$, is defined as:
% \begin{equation}
% \begin{aligned}
%     F(s) = \mathcal{L}\{f(t)\} = \int_0^\infty f(t)e^{-st} dt
% \end{aligned}
% \end{equation}
% where $F(s)$ represents the Laplace transform of $f(t)$. $s$ is a complex frequency variable, often expressed as $s = \sigma + j\omega$, with $\sigma$ representing the real part (related to exponential decay) and $\omega$ the imaginary part (related to frequency). % The integral's lower limit of 0 reflects the typical application to causal signals (signals that are zero for $t < 0$).

The Laplace transform is a key tool for analyzing linear time-invariant systems, converting time-domain functions into functions in the complex frequency domain. The two primary types of Laplace transforms are unilateral and bilateral. The unilateral Laplace transform is particularly useful in control systems and the analysis of differential equations, as it aids in system stability analysis. However, it is less commonly used in traditional time-series analysis. The bilateral Laplace transform generalizes the Fourier transform and is used primarily in engineering and systems analysis, but it is computationally intensive due to its broader scope and complexity.

While the Laplace transform has a wide range of applications in machine learning, its direct application to time-series data remains limited, likely due to challenges in integrating it with complex temporal structures. For instance,~\citep{ambhika2024time} proposed a hybrid model combining Laplace transform-based deep recurrent neural networks with long short-term memory networks for time-series prediction. Similarly,~\citep{chen2024laplacian} and~\citep{shu2024low} leveraged Laplacian transforms for traffic time-series imputation, using methods like low-rank completion, Laplacian kernel regularization, and FFT. These studies highlight the Laplace transform’s potential in improving time-series modeling by addressing challenges in data representation and computational efficiency.









% \begin{table*}[htbp]
%     \setlength{\tabcolsep}{10pt}
%     \centering
%     \scriptsize
%     \caption{Summary of representative frequency transform methods in our framework}
%     \label{tab:frequency_transform_methods} % 添加表格标签
%     \setlength{\tabcolsep}{2mm}{
%     \begin{threeparttable}
%         \begin{tabular}{c|c|c|c} % 修正表格列对齐格式
%         \toprule
%         \textbf{\makecell[c]{Frequency \\ Transform}} & \textbf{\makecell[c]{Categories and \\ Representative Methods}} & \textbf{Expression} & \textbf{Notes} \\ % 添加标题行的对齐符号
%         \midrule
%         \multirow{4}{*}{\makecell[c]{Fourier \\ Transform}} 
%             & \makecell[c]{Discrete Fourier Transform~\upcite{eldele2024tslanet}} & \makecell[c]{$X[k] = \sum_{n=0}^{L-1} (x[n] e^{-j\frac{2\pi}{L}kn})$} & \makecell[c]{$X[k]$ denotes the dis-\\crete frequency spectrum} \\
%             \cmidrule(r){2-4}
%             & Continuous Fourier Transform & 
%             \makecell[c]{$X(f) = \int_{-\infty}^{\infty} x(t)e^{-j2\pi ft} dt$} & \makecell[c]{$X(f)$ is the continu-\\ous frequency spectrum} \\
%             % \cmidrule(r){2-4}
%             % & Fast Fourier Transform & - & - \\
%             % \cmidrule(r){2-4}
%             % & \makecell[c]{Fractional Fourier Transform~\upcite{kocc2022fractional}} & - & - \\
%             % \cmidrule(r){2-4}
%             % & \makecell[c]{Graph Fourier Transform~\upcite{defferrard2016convolutional}} & - & - \\ 
%         \midrule
%         \multirow{3}{*}{\makecell[c]{Wavelet \\ Transform}} 
%             & Discrete Wavelet Transform & $D(a, b) = \frac{1}{\sqrt{b}} \sum_{m=0}^{p-1} f[t_m] \psi \left( \frac{t_m - a}{b} \right)$ & - \\
%             \cmidrule(r){2-4}
%             & Continuous Wavelet Transform & \makecell[c]{$F(\tau, s) = \frac{1}{\sqrt{\vert s \vert}} \int_{-\infty}^{\infty} f(t) \psi^* \left( \frac{t - \tau}{s} \right) dt$} & - \\ 
%         \midrule
%         \makecell[c]{Laplace \\ Transform}  & \makecell[c]{Unilateral Laplace Transform~\upcite{ambhika2024time}} & \makecell[c]{$F(s) = \mathcal{L}\{f(t)\} = \int_0^\infty f(t)e^{-st} dt$ } & - \\ 
%         \bottomrule
%         \end{tabular}
%         % \begin{tablenotes}
%         % \scriptsize
%             % \item * where $X[k]$ denotes the discrete frequency spectrum, $k$ represents the discrete frequency index, and $j = \sqrt{-1}$.
%             % \item ** where $X(f)$ is the continuous frequency spectrum, $f$ is the continuous frequency variable. $x(t)$ is the continuous-time signal and $j = \sqrt{-1}$ is the imaginary unit.
%         % \end{tablenotes}
%     \end{threeparttable}}
% \end{table*}



% Predicting long-term trends in time series data (e.g., energy consumption, weather patterns, traffic flow) continues to be a difficult problem. Because the frequency domain is helpful for capturing long-term patterns in time series, frequency domain transformations have been widely used in time series prediction recently, as shown in Figure~\ref{figure-time-series}. One category of methods combines frequency domain with Convolutional models~\cite{krizhevsky2012imagenet}, while another uses Transformer-based models~\cite{vaswani2017attention} as a basis for combination.

% \begin{figure}[ht]
%     \centering
%     \setlength{\belowcaptionskip}{-0.5cm}  % Adjust this to reduce spacing
%     \includegraphics[width=0.475\textwidth]{figures/time-series.pdf}
%     \caption{Time Series in the Frequency Domain}
%     \label{figure-time-series}
%     % \vspace{-0.3cm}  % Adjust this to further reduce spacing
% \end{figure}

% \subsection{Convolution-based}
% Convolution-based methods~\cite{yu2024method,eldele2024tslanet,park2021fast,zhou2024fourier,kim2024neural,lange2021fourier,zhang2024frnet,cai2024msgnet} for time series prediction have been a hot topic recent years. A former study~\cite{park2021fast} proposes a novel method called Partial Fourier Transformation (PFT), which offers a precise and efficient approach for calculating a subset of Fourier coefficients. PFT achieves this by employing polynomials to approximate a portion of the twiddle factors (trigonometric constants), effectively lowering computational complexity resulting from the multitude of these factors. PFT analyzes the asymptotic time complexity of PFT concerning input and output dimensions, as well as tolerance levels. Furthermore, PFT demonstrates that PFT allows users to define a specific approximation error threshold, offering flexibility crucial for scenarios where rapid evaluation is a top priority. Another former study~\cite{lange2021fourier} presents an algorithm that shares similarities with the Fourier transform but operates without relying on assumptions of periodicity. This feature enables forecasting even with irregular sampling intervals. Extending this approach to nonlinear signals involves incorporating Koopman theory. The resultant algorithm conducts a spectral decomposition within a nonlinear, data-driven framework. Despite the highly non-convex nature of the optimization objective in both cases, transforming the objective into the frequency domain facilitates the computation of global optima for the error surface efficiently and at scale, leveraging the computational efficiency of the Fast Fourier Transform. These methods, closely linked to Bayesian Spectral Analysis, naturally yield metrics for quantifying uncertainty in spectral forecasting processes. By contrast, recent studies tend to adopt adaptive frequency methods to solve higher and lower frequencies of time series.
% A recent work~\cite{eldele2024tslanet} proposes a method Tslanet, incorporating an adaptive spectral block via \emph{Fourier transform}, employing Fourier analysis for improved feature representation and the detection of both short-term and long-term dependencies. Noise reduction is achieved through adaptive thresholding. 

% % Furthermore, an interactive convolutional block, trained with self-supervised learning, enhances the model's ability to interpret complex temporal patterns and improves its generalizability across diverse datasets.

% \subsection{Transformer-based}
% Transformer-based methods~\cite{ma2023long,zhou2024fourier,ni2024time,chen2023lightweight,tran2023fourier,kang2023electric,yang2024fedaf}
% for time series prediction aim to provide better performance in the long-term prediction. To capture global-view dependencies of time series, Zhou \textit{et al.}~\cite{zhou2022fedformer} propose a method, namely FEDformer, to decompose Transformer with \emph{Fourier Transform} to compact representations of long-term time series patterns into frequency domain. To be specific, FEDformer integrates the Transformer model with the seasonal-trend decomposition technique, where this method grasps the overall pattern of time series data while Transformers delve into finer intricacies. To boost the Transformer's efficacy in long-range forecasting, FEDformer leverages the observation that many time series can be efficiently represented in common bases like the Fourier transform, leading to the creation of a frequency-enriched transformer. Meanwhile, via \emph{Fourier Transform}, FEDformer aims to capture pieces of information lost in the temporal domain. Another work~\cite{sasal2022w} introduces an innovative approach to learning representations of univariate time series, named W-Transformer, which is built upon a transformer encoder structure utilizing wavelets. The W-Transformers apply a maximal overlap discrete wavelet transformation to the time series information. Meanwhile, local transformers are adopted to effectively capture the nonstationary nature and intricate long-term nonlinear relationships within the time series data. Diferent from other studies, Jin \textit{et al.}~\cite{jin2022time} developed a novel approach for generating token sequences tailored for 1D data, namely TST, a fusion of the time series tokenizer and Transformer architecture. More specifically, TST introduces a way to generate token sequences from one-dimensional data, including time series data. This time series tokenizer is then integrated into a Transformer architecture. In this way, good performance is achieved.


% To capture temporal-spectral correlations effectively~\cite{yang2024graformer,zhang2023self,wang2024card}, to be specific, Zhang \textit{at al.}~\cite{zhang2023self} propose Cross Reconstruction Transformer (CRT). CRT facilitates time series representation learning by employing a cross-domain dropping-reconstruction task via extracting the frequency domain of the time series using the fast \emph{Fourier Transform} and randomly eliminating specific patches in both the time and frequency domains. Woo \textit{et al.}~\cite{woo2022etsformer} proposed ETSFormer, a fresh Transformer architecture tailored for time-series data. This model leverages the concept of exponential smoothing to enhance Transformers for time-series forecasting. Drawing inspiration from classical exponential smoothing techniques in time-series prediction, ETSFormer introduces the innovative concepts of Exponential Smoothing Attention and Frequency Attention. 
% % These mechanisms replace the conventional self-attention module in standard Transformers, enhancing the accuracy and efficiency of the model.


\section{Benchmarking}\label{sec:benchmark}

In this section, we show the performance of the KID approximation for the GER vector \( \b r \) and its computational complexity. In particular, through our experimental evaluation, we aim to do the following:
\begin{enumerate}[label=\bfseries(\roman*),leftmargin=*]
      \item support the asymptotic estimate for the approximation error \eqref{eq:err1} in terms of the number of moments \( M \) and the number of MC-vectors \( N_z \);
      \item support the computational complexity of the KID approximation using the (scaled) oracle choice for the parameters, Theorem~\ref{thm:error};
      \item compare the actual execution time of the approximation to the direct computation for complexes of different sizes and densities.
\end{enumerate}

\paragraph{Vietoris--Rips filtration.} Theorem~\ref{thm:error} and Equation~\eqref{eq:err1} describe the performance of the developed method in terms of the number of simplices \( m_k \). In order to appropriately numerically illustrate these behaviours, one should consider a family of arbitrarily large and dense simplicial complexes. For this reason, we opt to use simplicial complexes induced by the filtration procedure on point clouds. Formally, we proceed as follows:
\begin{enumerate}[leftmargin=*]
      \item  we consider \( m_0 \) points embedded in $\mathbb R^2$, sampled randomly in two clusters, i.e., \( \frac{m_0}{2}\) points are sampled from \( \mc N( \b 0, I )\) and \( \frac{m_0}{2}\) points are sampled from \( \mc N( c \b 1, I )\), for some \( c > 0 \); 
      \item then, for a fixed filtration threshold \( \epsilon > 0 \), a simplex \( \sigma = [v_{i_1}, ... v_{i_p} ] \) on these nodes enters the generated complex \( \mc K \) if and only if $d_{\mc M }(v_{i_j}, v_{i_k}) \le \epsilon$ for all paris $j$ and $k$. 
\end{enumerate}
This straightforward filtration is known as  Vietoris--Rips filtration, and the corresponding complex \( \mc K \) as a VR-complex. An illustrative example is provided in Figure~\ref{fig:example}. In the chosen setup, the value of the filtration parameter \( \epsilon \) naturally governs the density of the generated simplicial complexes of every order, as shown by the right panel in Figure~\ref{fig:example}: larger values of \( \epsilon \) define complexes with a higher number of edges, triangles, tetrahedrons, etc., until every possible simplex is included in \( \mc K \).

\begin{figure}[t]
      \centering
      \includegraphics[width = 1.0\columnwidth]{figures/example.pdf}
      \caption{ Example of VR-filtration. Left pane: point cloud with \( m_0 = 40 \) and filtration \( \epsilon = 1.5 \), inter-cluster distance \( c = 3 \). Right pane: dynamics of the number of simplices of different orders for varying filtration parameter \( \epsilon\). \label{fig:example}}
\end{figure}

\paragraph{Parameter choice and computational complexity.} 
The error estimate from Equation~\ref{eq:err1} suggests that the approximation error for the sparsifying norm \( \b p \) scales as \( M^{-1}\) in terms of the number of moments and as \( N_z^{-1/2}\) in terms of the number of Monte-Carlo vectors (MC-vectors). To illustrate this behaviour, we fix one of the parameters (\( M \) or \( N_z \)) to their theoretical estimates provided by Theorem~\ref{thm:error} and demonstrate the dynamic of the error \( \| \b p - \wh{\b p }\|_\infty \) as the function of the other parameter. As shown by Figure~\ref{fig:M_Nz}, the overall scaling law coincides with the estimates of Equation~\ref{eq:err1} in the case of \(1 \)-sparsification for \( \Lu 1\) operator. Note that all experiments are conducted in the at least minimally-dense setting, i.e. \( m_2 \ge m_1 \ln m_1 \).
\begin{figure}[t]
      \centering
      \includegraphics[width = 1.0\columnwidth]{figures/err_filt.pdf}
      \caption{
            Dependence of the approximation error \( \| \b p - \wh{ \b p } \|_\infty \) on the number of moments \( M \) and number of MC vectors \( N_z \). Values are tested up to (scaled) theoretical bounds from Thm~\ref{thm:error} (in red); line colors correspond to varying \( m_0 \) in the point cloud. Left pane: errors vs the number of moments \( M \) with fixed theoretical \( N_z \); right pane: errors vs the number of MC vectors \( N_z \) with fixed theoretical \( M \).  Errors are averaged over several generated VR-complexes; colored areas correspond to the spread of values. \label{fig:M_Nz}
      }
\end{figure}

Here, we explicitly highlight two observations from Figure~\ref{fig:M_Nz}: (1) larger and denser simplicial complexes tend to exhibit faster convergence in both parameters (especially in the number of moments \( M \)), and (2) Theorem~\ref{thm:error} provides theoretical (greedy) estimates for \( M \) and \( N_z \) that are sufficient for achieving the target approximation quality \(\delta\) and can be interpreted asymptotically. Consequently, one may choose scaled (and empirically sufficient) values for these parameters:
\[
M = \left\lceil \frac{m_{k+1}}{\delta\,m_k} \right\rceil
\quad\text{and}\quad
N_z = \left\lceil \frac{1}{10}\,\frac{8}{\pi^2}\,\frac{m_{k+1}^2}{\delta^2\,m_k^2} \right\rceil.
\]

\begin{figure}[t]
      \centering
      \includegraphics[width = 1.0\columnwidth]{figures/timings.pdf}
      \caption{
            Execution time of KID approximation for effective resistance of triangles, \( \V 2\) (left), and tethrahedrons, \( \V 3 \) (right). Line colors correspond to varying \( m_0 \)  in the point cloud; theoretical estimation of the computational complexity is given in dash.
            Execution times are averaged over several generated VR-complexes; colored areas correspond to the spread of values.  \label{fig:times}
       }
\end{figure}

Given this choice of parameters, in Figure~\ref{fig:times} we demonstrate that the complexity estimate
\(\mathcal{O}\!\bigl(\tfrac{m_{k+1}^4}{m_k^3}\bigr)\)
from Theorem~\ref{thm:error} aligns with the actual execution time of the KID approximation for varying filtration parameters \(\epsilon\) in the cases of \(1\)- and \(2\)-sparsification of VR-complexes.

\paragraph{Comparison with the direct computation.}

Finally, we compare the execution time of the KID approximation with that of the direct computation of the sparsifying measure \( \b{p} \) for \( 1 \)-sparsification, using the approximation parameters mentioned above (see Figure~\ref{fig:comparison}). Note that although the densest case complexity estimate 
\(\mathcal{O}\!\bigl(\delta^{-3}\,m_k^{\,1+\frac{4}{k+1}}\bigr)\)
suggests that the KID method's execution time might be comparable to direct computation, in practice the developed algorithm is significantly faster while still maintaining the target approximation error 
\(\|\b{p} - \widehat{\b{p}}\|_\infty \le \frac{\delta}{m_2}\).


\begin{figure}[t]
      \centering
      \includegraphics[width = 1.0\columnwidth]{figures/filtration100.pdf}
      \caption{
            Computation time comparison between KID-approximation \( \wh{ \b p} \), solid line, and directly computed sparsifying measure \( \b p \), dashed line (left), and corresponding approximation error \( \| \wh{ \b p} - \b p \|_\infty \) (right). Target approximation error is given in dotted line (right pane);  line colors correspond to varying \( m_0 \)  in the point cloud.
            Execution times are averaged over several generated VR-complexes. \label{fig:comparison}      
      }
\end{figure}

Additionally, we note that the performance of the direct computation of \(\b{p}\) for the largest considered point cloud with \(m_0 = 125\) highlights another important advantage of the KID approximation: reduced memory consumption. Indeed, whether one uses the definition of the GER vector 
\[
\b{r} 
= \diag \Bigl( B_{k+1}^\top (\Lu{k})^\dagger B_{k+1} \Bigr) 
\]
or the reformulation in terms of the right singular vector from Theorem~\ref{thm:GER_DOS}, a full SVD of \(\Lu{k}\) is required. In the case of point clouds with \(m_0 = 125\), denser VR-complexes lead to real-valued matrices of size \(10^4 \times 10^4\), resulting in substantial memory demands for the SVD. By contrast, the KID approximation avoids this decomposition and restricts the additional memory usage to storing Monte-Carlo matrices \(Z\) and their \texttt{matvecs} of dimension \(m_{k+1} \times N_z\), which is comparatively smaller.
\section{Application: Harnessing the Linearity}
\label{sec:application}
Leveraging the \emph{linearity} of DMD operator, as well as the intuition of bases exposed by the spectral decomposition, we have developed several novel applications that extend the capabilities of our Koopman-based reduced-order simulation pipeline. In this section, we explore these applications, demonstrating that our method's unique strengths translate into practical tools for graphics and simulation.

\subsection{Direct Editing Temporal Dynamics}
\label{sec:editing}
\begin{figure}[!ht]
    \centering
    \includegraphics[width=1\columnwidth]{figure/karman_vortex_street_editing.pdf}
    \caption{\textbf{Editing temporal dynamics of K\'arm\'an Vortex Street with the Koopman Operator Approximation}. The modifications are applied to the DMD basis coefficients: (a) Scaling the modulus of the DMD basis by factors of 0.5, 1.0, and 1.5, affecting overall amplitude; (b) Adjusting the real part of $\bm{\Omega}$, influencing growth and decay rates of modal contributions; (c) Modifying the imaginary part, altering phase dynamics and wave propagation characteristics. }
    \label{fig:karman_editing}
    \Description{}
\end{figure}


\begin{figure*}[!ht]
    \centering
    \includegraphics[width=1\linewidth]{figure/reversibility.pdf}
    \caption{\textbf{Reversibility of Flows with Inversed DMD Operator}. We compare the reconstruction of two distinct fluid flows using Dynamic Mode Decomposition (DMD). The top row in each panel shows the velocity L2-norm of the field used to train the DMD, while the second and third rows depict the temporal evolution of the reconstructed flow fields as applied to an initial density field. The forward-time training phase is followed by a backward-time testing phase to assess predictive accuracy when advecting backward in time. The bottom plots show the evolution of kinetic energy over time. From the buoyant case, we observe the inverted DMD operator $\bm{A^{-1}}$ can still reasonably trace backward in time without compromising much visual quality. The vortical case exhibits a more challenging example where the symmetry should be reconstructed backwards in time. We see that the inverse operator indeed recovers this symmetry, with some acceptable levels of incurred noise. Bottom plots show the evolution of the total kinetic energy over time, demonstrating that our inverse operator actually correctly reverses the arrow of time, reversing the dissipation-related entropy increase over time. Decreasing kinetic energy also validates the \emph{physical plausibility} of our result.}
    \label{fig:reverse_simulation}
    \Description{}
\end{figure*}


Since our method approximates \refeq{eqn:euler_equations} with a linear operator in the full space, this allows us to transform the operator acting on the velocity field into the evolution of different modes under a linear operator. Therefore, we can directly edit the temporal dynamics of the fluid system by modifying the modes of the reduced \koopman{} $\bm{\hat{K}}$:
we set $t_0$ to be the initial time, $\bm{\Omega} = \nicefrac{\log(\bm\Lambda)}{\Delta t}$, where $\Delta t$ is the time step of the dataset. With this, we can rewrite \refeq{eqn:reduced_koopman_simulation} in the following form:
\begin{equation}
    \begin{aligned}
    \bm{u}(t_0 + k\Delta t) &= \bm{\Phi}\exp(\bm{\Omega} t) \bm{z}(t_0) \\
    &= \bm{\Phi}\exp{\left(k(\log(r) + i\theta)\right)} \bm{z}(t_0) \\
    &= \sum_{i = 1}^{n} {w_i} \bm{\Phi_i} r_i^k \left(\cos(k\theta_i) + \sin(k\theta_i)\right) \bm{z_i}(t_0)\\
    \end{aligned}
    \label{eqn:edit_temporal}
\end{equation}
% explanation for the formula
where ${w_i}$ is a user-defined scalar weight, $r_i = \sqrt{\Re(\lambda_i)^2 + \Im(\lambda_i)^2}$ is the \emph{modulus} and $\theta_i = \arctan\left(\Im(\lambda_i), \Re(\lambda_i)\right)$ is the \emph{phase} of the $i$-th eigenvalue $\lambda_i$ in the diagonal \emph{complex} eigenvalue matrix $\bm{\Lambda}$. Notice that this implies that the modes of the spectral decomposition represent different scales of vorticity, completing the physical intuition of the reduced space modes.
% show the benefits of our method for artist to edit

As shown in \refeq{eqn:edit_temporal}, our method decomposes a simulation sequence into modes with different growth/decay rates and frequencies.
The growth/decay rate of a mode is reflected in $r_i$, where a larger $r_i$ indicates a higher growth rate (or a lower decay rate), and vice versa.
The frequency of a mode is represented by the absolute value of $\theta_i$, with a larger absolute value corresponding to a higher frequency mode, and vice versa.
Furthermore, the different modes are decoupled, allowing for the adjustment of the relative proportions between modes.
As a result, these properties provide the artist with powerful tools to edit the simulation playback. The artist can modify the overall velocity field by adjusting the proportion ($w_i$), growth/decay rate ($r_i$), and frequency ($\theta_i$) of specific modes.
% explanation for what we actually do in code
In the experiments, we directly adjust the real part of $\bm{\Omega_i}$ to control $r_i$, modify the imaginary part of $\bm{\Omega_i}$ to control $\theta_i$, and vary the modulus of $\bm{\Phi_i}$ to control $w_i$.
% explanation for what we did to edit in karman vortex street scene
\paragraph{Editing the K\'arm\'an Vortex Street}
The first example is editing on the classic K\'arm\'an vortex street. We filter the imaginary part of $\bm{\Omega}$ and cluster modes with an absolute value smaller than $0.01$ as \emph{low-frequency cluster}, and the rest as \emph{high-frequency cluster}.
The low-frequency mode manifests as a laminar flow, with its phase changing very slowly over time. The high-frequency mode is represented by vortical structures distributed on both sides of the cylinder, where the phase of this mode changes relatively quickly over time.
As seen in \reffig{fig:karman_editing}, when we adjust the modulus of the high-frequency cluster from $0.5$ to $1.5$, the intensity of the vortices increases, which is as we expected. When we set the real part of $\bm{\Omega}$ to $0.5$, it can be observed that the high-frequency motion decays faster than user input. When we set the real part of $\bm{\Omega}$ to $1.5$, it can be observed that the high-frequency motion decays slower than user input. Similarly, when we tune the imaginary part of $\bm{\Omega}$ from $0.5$ to $1.5$, we could observe the oscillation frequency of the fluid trail transitions from slow to fast compared to user input.
% explanation for what we did to edit in 3D plume scene
\paragraph{Editing the Plume with Bunny}
To evaluate the editing capability of our method, we scale our editing scenario to 3D. With the same filtering procedure as in the K\'arm\'an vortex street example, we set the low-frequency cluster to high-frequency cluster ratio to $4:1$, $2:1$, $1:2$, and $1:4$, and compared the results with the user input. From the results, we observe that when the proportion of low-frequency cluster is increased, with a ratio of $4:1$, the top of the plume lacks "wrinkles" and appears more "fluffy". This is because the velocity field is dominated by smoother, lower-frequency modes than the original user input. Conversely, when the proportion of high-frequency cluster is increased, with ratios of $1:4$, the plume developes more detailed plume structure around the top, as the velocity field now emphasizes more high-frequency details compared to the user input.

\subsection{Reversibility of the Reduced Simulation}
Although physically-based fluid simulations have the capability to generate stunning visuals, when artists aim to direct the fluid's evolution toward a predefined target shape, challenges arise. It is a long standing problem in the community that people aim to enable users with \emph{spatial control}. In this example, we aim to enable users to do \emph{temporal control}, motived by a prior work \citet{oborn2018time}. Compared to previous work \shortcite{oborn2018time} where the authors employ a self-attraction force to replace the arbitrary external forces, providing a stable, physics-motivated, but time-consuming approach, we propose a data-driven, fast, and easy to implement method to address the same problem.

\label{sec:reversibility}

We observe that that given $\bm{\tilde{K}} = \bm{\Phi} \bm{\Lambda} \bm{\Phi}^+$, we could easily compute the \emph{inverse} of the truncated \koopman{} $\bm{\tilde{K}}^{-1} = (\bm{\Phi} \bm{\Lambda} \bm{\Phi}^+)^{-1} = \bm{\Phi} \bm{\Lambda}^{-1} \bm{\Phi}^+$, which is essentially the approximate inverse time evolution $\bm{f}^{-1}(\bm u)$ of the fluid system. This allows us to reverse the simulation by applying the inverse truncated \koopman{} to the current state of the fluid system:
\begin{equation}
    \label{eqn:reverse_simulation}
    \begin{aligned}
        \bm{u}(t) &= \bm{A}^{-1} \bm{u}(t + \Delta t), \\
        \bm{u}(t) &= \bm{\Phi} \bm{\Lambda}^{-1}\bm{\Phi}^+ \bm{u}(t + \Delta t), \\
        \bm{u}(t) &= \bm{\Phi} \bm{\Lambda}^{-1} \bm{z}(t + \Delta t).
    \end{aligned}
\end{equation}

Similar to \refeq{eqn:reduced_koopman_projection}, we could train the reduced \koopman{} on the forward simulation data and then apply the inverse reduced \koopman{} to reverse the simulation, given a state of the fluid system.


\begin{figure*}[!ht]
    \centering
    \includegraphics[width=1\linewidth]{figure/upsample.pdf}
    \caption{\textbf{Upsampling and Generalization to Unseen Sequences with Trained DMD Operator}. Two different input low-resolution fluid simulations (bunny and strawberry) are upscaled using the same DMD operator trained on a different velocity field. Initial velocity fields are seeded as moving down based on the input density field.    
    Naive application of DMD shown in each middle column, and our \emph{augmented DMD upresolution} method shown on the right columns. 
    Schematic of our method presented on the far right. At each frame, we project the low-resolution artist-directed input into the low-order bases of our reduced representation, using these to replace the low-order terms of the DMD field. Notice that naive application of DMD simply moves towards the known input training data, while our augmented field matches the low-resolution input more closely, with extra high-order detail gained from the DMD operator.}
    \label{fig:upsample}
    \Description{}
\end{figure*}


% first explanation for buoyant reversibility
\paragraph{Reversibility of Buoyant Flow}
We experiment our approach on a simple buoyant flow setup (\reffig{fig:reverse_simulation}, left). Our dataset was initialized with a \textit{qian}, a density field shaped like a round coin with a square hole, with the density value set to $1$. A density value of $1$ density field was driven by a velocity field where an upwards velocity of $0.3$ is set within the qian and downwards elsewhere. We run the simulation for $300$ frames to construct the dataset, and trained the DMD operator on this dataset. The inverse operator $\bm{\tilde{K}}^{-1}$ was then applied to the initial velocity field of the dataset at $t=0$ (frame $0$). By iteratively applying the inverse operator, we obtained the velocity fields for the preceding frames, starting from frame $-1$, frame $-2$, and all the way back to frame $-300$.
% stability
When examining the evolution of the density field from frame -300 to frame 300, it is evident that the velocity field remains consistently upward and smooth, indicating that our method is both reasonable and effective.
% energy
Further analysis of the energy of the velocity field obtained through the inverse process and the velocity field from the dataset reveals a downward trend in energy, with a smooth and reasonable curve, consistent with fluids with dissipative properties. This demonstrates that our inverse operator has the ability to predict a \emph{physically-plausible} velocity field prior to the dataset.

% second explanation for vortical reversibility
\paragraph{Reversibility of Vortical Flow}
To challenge the method with a scene of nontrivial vortical structure, we initialized a vortex sheet by placing four vortices at the corners of the domain (\reffig{fig:reverse_simulation}, right). We generated the dataset using the same procedure as in the previous experiment, resulting in a collection of $500$ frames. Subsequently, we constructed the inverse operator to recover the velocity fields preceding the dataset.
% stability
The results show that the density field (counterclockwise) and the dataset (clockwise) rotate in the opposite direction, which indicates that the velocity field predicted by the inverse operator is correct. This is because the vortex sheet velocity field continuously rotates in a clockwise direction, and by examining the density field from frame -500 to frame 500, we observe that the field indeed undergoes continuous clockwise rotation.
% energy
From the energy field analysis, the results show that, except for the significant energy fluctuation between frames -500 and -450, the energy consistently decreases in the remaining frames, with a consistent slope. This further demonstrates the robustness of our method.

\subsection{Upsampling with Reduced Koopman Operator}

The scale of the imaginary part of eigenvalues in $\bm{\Lambda}$ encode different scales of turbulent modes, enabling us to use a trained DMD operator to add in secondary motion to an existing fluid simulation. This is particularly useful for \emph{upscaling} a low-resolution fluid, simulated using stable fluid for example, leveraging the DMD basis to add in turbulent modes that were too small for the low-res sim to capture. This upscaling problem has been explored in prior work \cite{kim2008wavelet, nielsen2009guiding}, but we show that due to the linearity of the Koopman operator, and the physical intuition on each of its reduced bases, this upscaling is essentially attained for \emph{free}, amounting to nothing more than a linear combination of two matrix multiplications. 

\subsubsection{Evolution} \label{sec:upres_direct}

Suppose we have frames of a low-res input velocity field $\{\bm{L}_0, \bm{L}_1, \bm{L}_2, \dots, \bm{L}_T\}$, a high-res initial condition $H_0$. Additionally, we have some DMD basis $\bm{\Phi}$ trained on some high-res simulation distinct from the low-res simulation, with corresponding eigenvalues $\bm{\Lambda}$, sorted by the length of their imaginary parts in increasing order. At the first frame, we can generate the reduced-space initial condition by simply using our basis mapping $R_0 = \bm{\Phi}^TH_0$.

Now, for every subsequent frame $t$, we generate $R_t$ by first applying the DMD evolution on the previous reduced space frame to produce an intermediate state $R^*_t=\bm{\Lambda}R_{t-1}$. We also produce a representation of the current frame of the low-res input in reduced space $P_t = \bm{\Phi}^TL_t$. Now, we have a representation of the \emph{current} frame of the low-res input, and the DMD \emph{time evolution} of the \emph{previous} reduced space frame. We want to keep the low-order bulk flow of the low-res input, and augment it with the high-order turbulent flow learned by the DMD basis. To that end, we split each reduced-space vector into a low-order and high-order part: $R^*_t = \left[R_t^{*L}\ R_t^{*H}\right]$, $P_t=\left[P_t^L\ P_t^H\right]$. Now, we take only the low-order modes of the input flow, and the high-order modes of the DMD-evolved flow, to produce our new reduced space velocity field $R_t=\left[P_t^L\ R_t^{*H}\right]$. From here, we can just apply the basis to return to high-resolution full-space $H_t=\bm{\Phi}R_t$.

We note that the composition operators here are linear. We can simply represent them with selection matrices $S^H$, $S_L$, for the high- and low-order bases respectively, such that $R_t=S^LP_t + S^HR_t^*$. Since the DMD operator is also linear, we note that this entire upscaling method is linear by construction.

Results are shown on Figure \ref{fig:upsample}. We see that even if the initial velocity field is significantly different from the input field, the low-order basis is able to capture the bulk flow of the low-resolution input, and modify the DMD-produced field accordingly. In particular, we note that naively applying the DMD operator, without passing the low-resolution input field into the low-order bases, ends up reconstructing the original training set, rather than a velocity field directed by our input. This is demonstrated by the results for the two initial conditions being very similar, whereas our augmented field matches the input much closer.

\subsubsection{Projection}

The above governs the time evolution of the velocity field. In some cases, where the input velocity field differs significantly from the training data used for the DMD basis, the above as written will still produce velocity fields that are unacceptably different from the input velocity field. This is largely representation error, fields that are far away from the training data are less representable by the reduced space. In these cases, we can again leverage our input low-res field, this time as a constraint. 

Essentially, we would like to project our velocity field $\bm{H}_t$ onto the space of velocity fields that are identical to the input low-res field when downsampled to that resolution. This can be represented as an equality-constrained quadratic problem,
\begin{align}
    &\argmin_x \frac{1}{2}(\bm{x}-\bm{H}_t)^T(\bm{x}-\bm{H}_t) \\
    &\text{subject to } \bm{Ax} = \bm{L}_t,
\end{align}
where $\bm{A}$ is a downsampling operator that converts from high-res to low-res. 
Notice that because the downsampling operator does not change for the duration of the simulation. Thus, the KKT (Karush-Kuhn-Tucker) matrix can be precomputed making the projection a single matrix multiply during runtime.

\begin{wrapfigure}{r}{0.5\columnwidth}
    \vspace{-2pt}
    \includegraphics[width=0.5\columnwidth]{figure/qr.pdf}
    \hspace{5pt}
    \label{fig:qp_project}
\end{wrapfigure}

As a sanity check, we show the effect of this projection here: it is apparent with the projection step,we can recover fields that are much closer to the input, yet retaining extra high-order detail. And of course, because these are all linear, linear combinations of the direct and projected fields can be taken. In particular, because the basis functions of reduced space are orthogonal, a diagonal matrix of linear weights can be taken, preferring projected for low-order modes and direct for high-order modes for example.

\begin{comment}
Given a high-resolution DMD matrix $A \in \mathbb{R}^{N_{hi} \times N_{hi}}$ trained on high-dimensional data, we reconstruct a high-resolution sequence $\bm{x}_{hi}(t) \in \mathbb{R}^{N_{hi}}$ using an initial high-resolution frame $\bm{x}_{hi}(0)$ and subsequent low-resolution frames $\bm{x}_{lo}(t) \in \mathbb{R}^{N_{lo}}$, where $N_{lo} < N_{hi}$. The matrix $A$ is structured as
\begin{equation}
    \label{eqn:slice_A}
    A = \begin{bmatrix} A_{ll} & A_{lh} \\ A_{hl} & A_{hh} \end{bmatrix}
\end{equation}, with $A_{ll} \in \mathbb{R}^{N_{lo} \times N_{lo}}$ map low frequency component to, $A_{lh} \in \mathbb{R}^{N_{lo} \times (N_{hi} - N_{lo})}$, $A_{hl} \in \mathbb{R}^{(N_{hi} - N_{lo}) \times N_{lo}}$, and $A_{hh} \in \mathbb{R}^{(N_{hi} - N_{lo}) \times (N_{hi} - N_{lo})}$.

Starting from the initial condition $\bm{x}_{hi}(0)$, the high-resolution state at time $t + \Delta t$ is updated using:
\begin{equation}
    \label{eqn:upsampling_advect}
    \bm{x}_{hi}(t + \Delta t) = A \bm{x}_{hi}(t) + \begin{bmatrix} \bm{x}_{lo}(t + \Delta t) - A_{ll} \bm{x}_{lo}(t) \\ A_{hl} \left( \bm{x}_{lo}(t + \Delta t) - A_{ll} \bm{x}_{lo}(t) \right) \end{bmatrix}.
\end{equation}

In this equation, $A \bm{x}_{hi}(t)$ evolves the high-resolution dynamics. The term $\bm{x}_{lo}(t + \Delta t) - A_{ll} \bm{x}_{lo}(t)$ represents the correction to the low-frequency component, and $A_{hl} \left( \bm{x}_{lo}(t + \Delta t) - A_{ll} \bm{x}_{lo}(t) \right)$ in-paints the missing high-frequency details. This process ensures the reconstructed high-resolution sequence remains consistent with the initial frame and the low-resolution input while leveraging the full dynamics encoded in $A$.

\end{comment}
\section{Discussion}
\label{sec:discuss}
\textbf{Poisoning attacker.}
Assumption 4.2 is the core assumption of this paper, which implicitly assumes that the attacker's dataset $D_i$ is not a poisoning dataset. This assumption is based on the premise that client $i$ aims to maximize their reward and thus will not poison the grand model $\mathcal{A}(\granddataset)$, as doing so would reduce the reward derived from monetizing/utilizing $\mathcal{A}(\granddataset)$.
However, in certain scenarios, client $i$ may pursue dual objectives: both attacking the grand model and conducting data overvaluation. Addressing this dual-objective scenario requires further exploration.


\textbf{Computational efficiency.}
Similar to computing the SV, computing Truth-Shapley is time-consuming, as it requires $O(2^{N+\max_i M_i})$ times of model retraining. 
Since Truth-Shapley utilizes the SV-style approach to define both its client-level data value and block-level data value, existing techniques for accelerating SV computation can be applied to computing these two levels of data value.
Also, designing more efficient acceleration methods specifically for Truth-Shapley is a promising direction for future research.


\textbf{Extension of data overvaluation attack.}
The data overvaluation attack proposed in Definition \ref{def:overvaluation} allows client $i$ to manipulate the utility $v(\datasubset)$ of a data subset $\datasubset \subset \granddataset$ by misreporting client $i$'s data blocks $\reportedstoi$. 
Similarly, client $i$ can achieve the same objective by violating the training algorithm $\mathcal{A}$.
For example, client $i$ can decrease $v(\datasubset)$ by performing a gradient ascent attack during model training.
Truth-Shapley remains resistant to this extension of data overvaluation attack with a slight modification to Assumption 4.2: we assume that, in client $i$’s belief, following algorithm $\mathcal{A}$ maximizes the expected utility for any $\datasubset \subset \granddataset$.
\section{Theoretical Analysis}\label{sec:theoretical}

\textbf{Different correct answers are competitor.}\quad For any LLM trained with cross-entropy loss, different correct answers are competitors in terms of probability \footnote{The ``same question'' refers to questions that are semantically equivalent but do not need to be identical.}. Continuing with the example of proposing a president, suppose $\tau^{a}$ (``\texttt{Barack}'') is the label of a sample whose $\bm{q}$ is ``\texttt{[INST]Could you give me one name of president?[\textbackslash INST]}'' and a generated token vector $\bm{a}_{t-1}$  can be decoded into ``\texttt{Sure, here is a historical American president:**}'', the loss of the next token at this position during supervised fine-tuning can be written as:
\begin{equation}
\begin{aligned}
 &L^{\tau^a} = - \log \frac{\exp(\mathcal{M}({\tau^a}|\bm{q},\bm{a}_{t-1}))}{\sum_{m=1}^{|\bm{Y}|} \exp(\mathcal{M}(\tau^{m}|\bm{q},\bm{a}_{t-1}))} ,
 % \\   &L^{\tau^b} = - \log \frac{\exp(\mathcal{M}(\tau^b|\bm{q},\bm{a}_{t-1}))}{\sum_{m=1}^{|\bm{Y}|} \exp(\mathcal{M}(\tau^{m}|\bm{q},\bm{a}_{t-1}))} ,
\end{aligned}
\end{equation}
where $L^{\tau^a}$ is the loss on the sample with the next token label $\tau^{a}$.
Consider cases where multiple distinct answers to the same question appear in the training set, the situation becomes different. For example, $\tau^{b}$ (``\texttt{George}'') is the label in another sample with the same question. When the model is simultaneously fine-tuned on both samples, the gradient update for the model will be:
\begin{equation}
\begin{aligned}
 & \nabla_{\mathcal{M}} (L^{\tau^a} + L^{\tau^b}) = \nabla_{\mathcal{M}} L^{\tau^a} + \nabla_{\mathcal{M}} L^{\tau^b} \\
% &= -y_a^{\tau^a}\frac{1}{\Omega_a^{\tau^a}}\nabla_{\mathcal{M}}\Omega_a^{\tau^a}-\sum_{m \neq a}^{|\bm{Y}|} y_a^{\tau^m}\frac{1}{\Omega_a^{\tau^m}}\nabla_{\mathcal{M}}\Omega_a^{\tau^m}
% \\
% &\quad -y_b^{\tau^b}\frac{1}{\Omega_b^{\tau^b}}\nabla_{\mathcal{M}}\Omega_b^{\tau^b}-\sum_{m \neq b}^{|\bm{Y}|} y_b^{\tau^m}\frac{1}{\Omega_b^{\tau^m}}\nabla_{\mathcal{M}}\Omega_b^{\tau^m}
% \\
&\quad= \underbrace{-y_a^{\tau^a}\frac{1}{\Omega_a^{\tau^a}}\nabla_{\mathcal{M}}\Omega_a^{\tau^a}-y_b^{\tau^b}\frac{1}{\Omega_b^{\tau^b}}\nabla_{\mathcal{M}}\Omega_b^{\tau^b}}_{\text{(1) maximizing the probability of annotated answer}}\\& \quad \underbrace{-y_a^{\tau^b}\frac{1}{\Omega_a^{\tau^b}}\nabla_{\mathcal{M}}\Omega_a^{\tau^b}-y_b^{\tau^a}\frac{1}{\Omega_b^{\tau^a}}\nabla_{\mathcal{M}}\Omega_b^{\tau^a}}_{{\text{\textbf{(2)} minimizing the probability of the other annotated answer}}}\\& \quad \underbrace{-\sum_{m \neq a,b}^{|\bm{Y}|}y_{a,b}^{\tau^m} \left[ \frac{1}{\Omega_a^{\tau^m}}\nabla_{\mathcal{M}}\Omega_a^{\tau^m} + \frac{1}{\Omega_b^{\tau^m}}\nabla_{\mathcal{M}}\Omega_b^{\tau^m} \right]}_{\text{(3) minimizing the probability of incorrect answers}},
\end{aligned}\label{eq:competitor}
\end{equation}
where $\Omega_a^{\tau^a}=\frac{\exp(\mathcal{M}(\tau^a|\bm{q},\bm{a}_{t-1}))}{\sum_{m=1}^{|\bm{Y}|} \exp(\mathcal{M}(\tau^{m}|\bm{q},\bm{a}_{t-1}))}$, and $y_a^{\tau^m}$ indicates the next token label of a training sample with ground-truth label ${\tau^a}$, that is, we have $y_a^{\tau^a}=1$ and $y_a^{\tau^b}=0$. In particular, when $\mathcal{M}$ is in a certain state during training, we have $\Omega_a^{\tau^a}=\Omega_b^{\tau^a}$, and we make distinctions to facilitate the reader's understanding here. As we can see, for scenarios with multiple answers, the training objective can be divided into three parts:
(1) For each sample, increase the probability of its own annotation in the output distribution.
(2) For each sample, decrease the probability of another sample's annotation in the output distribution. \textit{\textbf{Note:}} This part leads to the issue where probability cannot anymore capture the reliability of LLM responses, as different correct answers tend to reduce the probability of other correct answers, making low probabilities cannot indicates low reliability.
(3) For both samples, decrease the probability of other outputs not present in the annotations in the output distribution.




\section*{Acknowledgment}
The authors would like to thank Clement Svendsen for valuable measure theoretic insight. 

Kasper Green Larsen is co-funded by a DFF Sapere Aude Research Leader Grant No. 9064-00068B by the Independent Research Fund Denmark and co-funded by the European Union (ERC, TUCLA, 101125203). Natascha Schalburg is funded by the European Union (ERC, TUCLA, 101125203). Views and opinions expressed are however those of the author(s) only and do not necessarily reflect those of the European Union or the European Research Council. Neither the European Union nor the granting authority can be held responsible for them.

\section{Conclusion}
\label{sec:conclu}
This survey underscores the transformative role of frequency domain techniques in advancing time series analysis. By systematically reviewing Fourier, Laplace, and Wavelet Transforms, we provide a comprehensive understanding of their applications, strengths, and limitations. Our up-to-date pipeline highlights recent advancements, offering valuable insights for researchers and practitioners. This work not only fills a critical gap in the literature but also inspires innovative applications and fosters deeper exploration of frequency domain methodologies. The accompanying GitHub repository further enhances accessibility and reproducibility, paving the way for future advancements in the field.

\clearpage
\bibliographystyle{named}
\bibliography{ref}


\end{document}

