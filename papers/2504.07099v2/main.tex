%%%% ijcai25.tex

\typeout{IJCAI--25 Instructions for Authors}

% These are the instructions for authors for IJCAI-25.

\documentclass{article}
\pdfpagewidth=8.5in
\pdfpageheight=11in

% The file ijcai25.sty is a copy from ijcai22.sty
% The file ijcai22.sty is NOT the same as previous years'

\usepackage{ijcai25}
\newcommand{\citep}[1]{\citeauthor{#1} [\citeyear{#1}]}


% Use the postscript times font!
\usepackage{times}
\usepackage{soul}
\usepackage{url}
% \usepackage[hidelinks]{hyperref}
\usepackage[utf8]{inputenc}
\usepackage[small]{caption}
\usepackage{graphicx}
\usepackage{amsthm}
\usepackage{booktabs}
\usepackage{algorithm}
\usepackage{algorithmic}
\usepackage[switch]{lineno}
\usepackage{tabularx}

\usepackage{float}
\usepackage{amsmath,amsfonts}
\usepackage{algorithmic}
\usepackage{times}
\usepackage{latexsym}
\usepackage{xtab,booktabs}
\usepackage{tabularx}
\usepackage{multirow}

\usepackage[T1]{fontenc}
\usepackage{tikz}
\usepackage{xcolor} 
\usepackage[colorlinks=true,     
            linkcolor=black,     
            citecolor=black,      
            urlcolor=blue      
           ]{hyperref}

\usepackage{forest}
\usetikzlibrary{graphs}
\usepackage[utf8]{inputenc}
\usepackage{booktabs}
\usepackage{ragged2e}
% \usetikzlibrary{trees}
\usepackage{microtype}
\usepackage{multirow}
\usepackage{makecell}
\usepackage{url}
\usepackage{colortbl}       % the color of table
\usepackage{threeparttable}
\usepackage{enumitem}

\definecolor{hidden-draw}{RGB}{251,239,214}
\definecolor{hidden-orange}{RGB}{218, 97, 91}
\definecolor{lightred}{RGB}{220,92,96}
\definecolor{deepblue}{RGB}{125,174,224}
\definecolor{lightpurp}{RGB}{179,149,189}
\definecolor{lightpurple}{RGB}{130, 132, 131}
\definecolor{lightgray}{gray}{0.9}

\definecolor{hiddenc1}{RGB}{59, 118, 122}
\definecolor{hiddenc2}{RGB}{69,105,144}
\definecolor{hiddenc3}{RGB}{130,130,170}

\definecolor{hid-vae}{RGB}{251,239,214}
\definecolor{hid-gnn}{RGB}{179,149,189}
\definecolor{hid-trans}{RGB}{122, 199,226}
\definecolor{hid-dm}{RGB}{225, 225, 255}
\definecolor{hid-llm}{RGB}{84,190,170}
\definecolor{hid-ssl}{RGB}{176,217,146}
\definecolor{hid-dms}{RGB}{238, 144, 59}
\newcommand{\hx}[1]{{\bf\color{cyan}[{\sc WHX:} #1]}}
\newcommand{\zz}[1]{{\bf\color{red}[{\sc SZZ:} #1]}}


% Comment out this line in the camera-ready submission
% \linenumbers

\urlstyle{same}

% the following package is optional:
%\usepackage{latexsym}

% See https://www.overleaf.com/learn/latex/theorems_and_proofs
% for a nice explanation of how to define new theorems, but keep
% in mind that the amsthm package is already included in this
% template and that you must *not* alter the styling.
\newtheorem{example}{Example}
\newtheorem{theorem}{Theorem}
\newcommand{\upcite}[1]{\textsuperscript{\textsuperscript{\cite{#1}}}}

\pdfinfo{
/TemplateVersion (IJCAI.2025.0)
}


\title{Beyond the Time Domain: Recent Advances on Frequency\\ Transforms in Time Series Analysis}


\author{
Qianru Zhang$^1$\thanks{Equal contribution. $^\dagger$Corresponding author.}\and
Peng Yang$^{1,*}$\and
Honggang Wen$^{1,*}$\and
Xinzhu Li$^{1,*}$
\and
Haixin Wang$^{2,*}$\and\\
Fang Sun$^{2,*}$
\and
Zezheng Song$^3$
\and
Zhichen Lai$^4$
\and
Rui Ma$^6$
\and
Ruihua Han$^{1}$
\and\\
Tailin Wu$^5$
\and
Siu-Ming Yiu$^{1,\dagger}$
\and
Yizhou Sun$^2$
\and
Hongzhi Yin$^{7,\dagger}$\\
\affiliations
$^1$The University of Hong Kong (HKU),
$^2$University of California, Los Angeles (UCLA),
$^3$University of Maryland, College Park (UMCP),
$^4$Aalborg University,
$^5$Westlake University,
$^7$The University of Queensland (UQ),
$^6$Microsoft
}





\begin{document}

\maketitle

\begin{abstract}
    The field of time series analysis has seen significant progress, yet traditional methods predominantly operate in temporal or spatial domains, overlooking the potential of frequency-based representations. This survey addresses this gap by providing the first comprehensive review of frequency transform techniques-Fourier, Laplace, and Wavelet Transforms-in time series. We systematically explore their applications, strengths, and limitations, offering a comprehensive review and an up-to-date pipeline of recent advancements. By highlighting their transformative potential in time series applications including finance, molecular, weather, etc. This survey serves as a foundational resource for researchers, bridging theoretical insights with practical implementations. A curated GitHub repository further supports reproducibility and future research.
\end{abstract}

\section{Introduction}
\label{sec:intro}

\begin{figure*}[tb]
    \centering
    \includegraphics[width=0.848\linewidth]{figs/circuitnn.pdf} 
    \caption{Illustration of differentiable CircuitNN. CircuitNN is designed based on differentiable NAND gates. After DAS is guided by PI and PO pairs of the truth table, CircuitNN can get the precise circuit architecture logic equivalent to the truth table.}
    \label{fig:circuitnn}
\end{figure*}

% 1. Describe the importance of logic synthesis
% 2. Existing Problems
% (a) Neural Architecture Search: Unstable, Predefined Setting, etc.
% (b) Circuit Generation: Probabilistic Model, Logic Equivalence

With the rapid advancement of technology, the scale of integrated circuits (ICs) has expanded exponentially. 
This expansion has introduced significant challenges in chip manufacturing, particularly concerning power and area metrics.
A primary objective in IC design is achieving the same circuit function with fewer transistors, thereby reducing power usage and area occupancy.

Logic synthesis~\cite{hachtel2005logicsynth}, a critical step in electronic design automation (EDA), transforms behavioral-level circuit designs into optimized gate-level circuits, ultimately yielding the final IC layout. 
The primary goal of logic synthesis is to identify the physical implementation with the fewest gates for a given circuit function. 
This task constitutes a challenging NP-hard combinatorial optimization problem. 
Current logic synthesis tools~\cite{brayton2010abc, wolf2013yosys} rely on human-designed heuristics, often leading to sub-optimal outcomes.

Differentiable architecture search (DAS) techniques~\cite{liu2018darts, chu2020darts} offer novel perspectives on addressing challenges in this problem.
Circuit functions can be represented through truth tables, which map binary inputs to their corresponding outputs. 
Truth tables provide a precise representation of input-output relationships, ensuring the design of functionally equivalent circuits.
Inspired by this, researchers~\cite{deepmind2024ai4sys, wang2024tnet} have begun exploring the application of DAS to synthesize circuits directly from truth tables.
Specifically, \citet{deepmind2024ai4sys} proposed CircuitNN, a framework that learns differentiable connection structures with logic gates, enabling the automatic generation of logic circuits from truth tables.
This approach significantly reduces the complexity of traditional circuit generation. 
Building on this, \citet{wang2024tnet} introduced T-Net, a triangle-shaped variant of CircuitNN, incorporating regularization techniques to enhance the efficiency of DAS.

Despite these advancements, several challenges remain. 
The computational complexity of DAS grows quadratically with the number of gates, posing scalability issues.
Although triangle-shaped architecture~\cite{wang2024tnet} partially mitigates this problem, redundancy persists. 
%Additionally, DAS is susceptible to converging to local optima, limiting the ability to search architectures that satisfy the given truth tables~\cite{liu2018darts}. 
%Furthermore, hyperparameters (network depth and layer width) require extensive searches, introducing complexity and prolonging the synthesis process. 
Additionally, DAS is susceptible to converging to local optima~\cite{liu2018darts} and hyperparameters (network depth and layer width) require extensive searches. 
The challenges arise from the vast search space in DAS. 
% Even with predefined settings for CircuitNN, finding a configuration that meets the truth table requires extensive trial and error during the DAS process. 
Intuitively, limiting the search space through predefined parameters (network depth, gates per layer, and connection probabilities) can significantly reduce the complexity.

Recent advances~\cite{openai2023gpt4, abramson2024alphafold3, esser2024sd3, li2024mar} in conditional generative models have demonstrated remarkable performance across language, vision, and graph generation tasks. 
Motivated by these developments, we propose a novel approach to circuit generation that generates preliminary circuit structures to guide DAS in generating refined circuits matching specified truth tables. 
Firstly, we introduce CircuitVQ, a tokenizer with a discrete codebook for circuit tokenization. 
Built upon our Circuit AutoEncoder framework~\cite{hou2022graphmae,li2023maskgae,wu2025mgvga}, CircuitVQ is trained through a circuit reconstruction task. 
Specifically, the CircuitVQ encoder encodes input circuits into discrete tokens using a learnable codebook, while the decoder reconstructs the circuit adjacency matrix based on these tokens.
Subsequently, the CircuitVQ encoder serves as a circuit tokenizer for CircuitAR pretraining, which employs a masked autoregressive modeling paradigm~\cite{chang2022maskgit, li2023mage}. 
In this process, the discrete codes function as supervision signals. 
After training, CircuitAR can generate discrete tokens progressively, which can be decoded into initial circuit structures by the decoder of the CircuitVQ. 
These prior insights can guide DAS in producing refined circuits that match the target truth tables precisely.

Our key contributions can be summarized as follows:
\begin{itemize}
\item We introduce CircuitVQ, a circuit tokenizer that facilitates graph autoregressive modeling for circuit generation, based on our Circuit AutoEncoder framework;
\item Develop CircuitAR, a model trained using masked autoregressive modeling, which generates initial circuit structures conditioned on given truth tables;
\item Propose a refinement framework that integrates differentiable architecture search to produce functionally equivalent circuits guided by target truth tables;
\item Comprehensive experiments demonstrating the scalability and capability emergence of our CircuitAR and the superior performance of the proposed circuit generation approach.
\end{itemize}

% Motivation
% (a) Diffusion (Vision, Graph), Autoregressive (Language, Vision)
% (b) Circuit Generation for Predefined Setting
% (c) Neural Architecture Search for Strict Logic Equivalence

% Contribution
% (a) Circuit Tokenizer (new transformer arch, training strategy)
% (b) CircuitAR (train and gen strategies, post-ar strategy)
% (c) Extensive Evaluation including BitD (Bit Distance) for Scalability

% 
\tikzstyle{my-box}=[
    rectangle,
    draw=hidden-draw,
    rounded corners,
    align=left,
    text opacity=1,
    minimum height=1.5em,
    minimum width=5em,
    inner sep=2pt,
    fill opacity=.8,
    line width=0.8pt,
]
\tikzstyle{leaf-head}=[my-box, minimum height=1.5em,
    draw=hidden-orange, % 调颜色
    % fill=hidden-draw,  % 调颜色
    text=black, font=\normalsize,
    inner xsep=2pt,
    inner ysep=4pt,
    line width=0.8pt,
]
\tikzstyle{leaf-task}=[my-box, minimum height=2.5em,
    draw=hidden-orange, % 调颜色
    % fill=hidden-draw,  % 调颜色
    text=black, font=\normalsize,
    inner xsep=2pt,
    inner ysep=4pt,
    line width=0.8pt,
]
\tikzstyle{leaf-taska}=[my-box, minimum height=2.5em,
    draw=hidden-orange, % 调颜色
    % fill=hidden-draw,  % 调颜色
    text=black, font=\normalsize,
    inner xsep=2pt,
    inner ysep=4pt,
    line width=0.8pt,
]
\tikzstyle{modelnode-task1}=[my-box, minimum height=1.5em,
    draw=hidden-orange, % 调颜色
    fill=hidden-draw,  % 调颜色
    text=black, font=\normalsize,
    inner xsep=2pt,
    inner ysep=4pt,
    line width=0.8pt,
]
\tikzstyle{leaf-task10}=[my-box, minimum height=1.0em,
    draw=hidden-orange, % 调颜色
    % fill=gray!30,  % 调颜色
    text=black, font=\normalsize,
    inner xsep=2pt,
    inner ysep=4pt,
    line width=0.6pt,
]
\tikzstyle{modelnode-task6}=[my-box, minimum height=1.5em,
    draw=hidden-orange, % 调颜色
    % fill=red!25,  % 调颜色
    text=black, font=\normalsize,
    inner xsep=2pt,
    inner ysep=4pt,
    line width=0.8pt,
]
\tikzstyle{modelnode-task7}=[my-box, minimum height=1.5em,
    draw=hidden-orange, % 调颜色
    % fill=red!25,  % 调颜色
    text=black, font=\normalsize,
    inner xsep=2pt,
    inner ysep=4pt,
    line width=0.8pt,
]
\tikzstyle{modelnode-task8}=[my-box, minimum height=1.5em,
    draw=hidden-orange, % 调颜色
    % fill=red!25,  % 调颜色
    text=black, font=\normalsize,
    inner xsep=2pt,
    inner ysep=4pt,
    line width=0.8pt,
]
\tikzstyle{modelnode-task9}=[my-box, minimum height=1.5em,
    draw=hidden-orange, % 调颜色
    % fill=red!25,  % 调颜色
    text=black, font=\normalsize,
    inner xsep=2pt,
    inner ysep=4pt,
    line width=0.8pt,
]
\tikzstyle{leaf-paradigms}=[my-box, minimum height=2.5em,
    draw=hidden-orange, % 调颜色
    % fill=hiddenc2,  % 调颜色
    text=black, font=\normalsize,
    inner xsep=2pt,
    inner ysep=4pt,
    line width=0.8pt,
]
\tikzstyle{leaf-others}=[my-box, minimum height=2.5em,
    %fill=hidden-pink!80,
    draw=hidden-orange, % 调颜色
    % fill=hiddenc3,  % 调颜色
    text=black, font=\normalsize,
    inner xsep=2pt,
    inner ysep=4pt,
    line width=0.8pt,
]
\tikzstyle{leaf-other}=[my-box, minimum height=2.5em,
    %fill=hidden-pink!80,
    draw=orange!80, % 调颜色
    fill=orange!15,  % 调颜色
    text=black, font=\normalsize,
    inner xsep=2pt,
    inner ysep=4pt,
    line width=0.8pt,
]
\tikzstyle{modelnode-task}=[my-box, minimum height=1.5em,
    draw=black, % 调颜色
    % fill=red!25,  % 调颜色
    text=black, font=\normalsize,
    inner xsep=2pt,
    inner ysep=4pt,
    line width=0.8pt,
]
\tikzstyle{modelnode-paradigms}=[my-box, minimum height=1.5em,
    draw=black, % 调颜色
    % fill=blue!15,  % 调颜色
    text=black, font=\normalsize,
    inner xsep=2pt,
    inner ysep=4pt,
    line width=0.8pt,
]
\tikzstyle{modelnode-others}=[my-box, minimum height=1.5em,
    draw=black, % 调颜色
    % fill=green!15,  % 调颜色
    text=black, font=\normalsize,
    inner xsep=2pt,
    inner ysep=4pt,
    line width=0.8pt,
]
\tikzstyle{modelnode-other}=[my-box, minimum height=1.5em,
    draw=black, % 调颜色
    % fill=orange!15,  % 调颜色
    text=black, font=\normalsize,
    inner xsep=2pt,
    inner ysep=4pt,
    line width=0.8pt,
]
\begin{figure*}[!th]
    \centering
    \resizebox{1\textwidth}{!}{
        \begin{forest}
            for tree={
                grow=east,
                reversed=true,
                anchor=base west,
                parent anchor=east,
                child anchor=west,
                base=left,
                font=\normalsize,
                rectangle,
                draw=hidden-draw,
                rounded corners,
                align=center,
                minimum width=1em,
                edge+={darkgray, line width=1pt},
                s sep=3pt,
                inner xsep=0pt,
                inner ysep=3pt,
                line width=0.8pt,
                ver/.style={rotate=90, child anchor=north, parent anchor=south, anchor=center},
                edge path={
                    \noexpand\path [draw, \forestoption{edge}]
                    (!u.parent anchor) -- ++(5pt,0) |- (.child anchor)\forestoption{edge label};
                },
            },
            [% 中括号之间不要换行!!!
                Learning in Frequency Domain,leaf-head,ver
                % [
                %      Low-level Vision \\(\textbf{Sec.~\ref{sec:low-level-vision}}),leaf-task,text width=9em
                %     [
                %         Convolution-Based, leaf-task10, text width=8.5em
                %         [\cite{chang2000adaptive,chappelier2006oriented,veena2016least}{, }\\WT\cite{chappelier2006oriented}{, }FPA\cite{chen2024large}\\GFD\cite{wang2017virtualization}{,}WSTID\cite{veena2016least}{, }LSWR\cite{dixon2016aerial}\\FDCNN\cite{janssens2016convolutional} {, }DWT-CNN Fusion\cite{avci2024mfif}{, }DC\\-NN\cite{wang2017virtualization}             \cite{rani2023atcnn,liu2024wavefusionnet,xu2024semantic},modelnode-task1, text width=33em]
                %     ]
                %     [
                %        Transformer-Based, leaf-task10, text width=8.5em
                %         [TPR\cite{li2023efficient}{, }TransMEF\cite{qu2022transmef}{, }
                %         \cite{jiang2024frft}{, }\\\cite{cao2023cfmb}{, }WT-OCT \cite{chen2024waveformer}\\
                %         Laplacian-Former\cite{azad2023laplacian}{, }\cite{korkmaz2024training}\\
                %         \cite{chibani2003redundant,mao2018multi}{, }FFT\cite{zhu2023attention}
                %         \\\cite{kong2023efficient}{, }WECT\cite{wang2024versatile}{, }
                %         DWT\cite{zhang2024waveletformernet}{, }\\
                %         WaveletFormerNet\cite{chen2021pre,liang2021swinir,zhang2022swinfir}{, }\\
                %         Fourier Conv-Transformer Cross-Scale\cite{zhang2022low,zhang2023cross}
                %         , modelnode-task1, text width=33em]  
                %     ]
                % ]
                % [
                %     High-level Vision \\(\textbf{Sec.~\ref{sec:high-level-vision}}), leaf-paradigms,text width=9em
                %     [
                %         Convolution-Based, leaf-task10, text width=8.5em
                %         [DSNet\cite{shang2020dense,liu2022partial}{, }FDCNN\cite{goh2021frequency}{, }HWNN\\\cite{nong2021hypergraph}{, }HWD\cite{xu2023haar}{, }SAR-CNN\cite{heiselberg2022sar}\\FFPF~\cite{lingyun2022fast}{, }DCT\cite{luo2024frequency}{, }FFANet\cite{zhou2024frequency}\\ FT\&CNN~\cite{labbihi2024combining} {,}WTConv\cite{finder2024wavelet}{, }Riesz–Laplace\\-Transform\cite{unser2009multiresolution}{, }HPF-MMCA\cite{magid2021dynamic}, modelnode-task1, text width=33em]
                %     ]
                %     [
                %        Transformer-Based, leaf-task10, text width=8.5em
                %         [ TMG-AFM \cite{chen2020generative,alexey2021image,yang2023discrete}{, }DETR\\\cite{carion2020end,zhu2020deformable}{, }FSA\cite{zhang2023decomformer}{, }MetaISP\\ \cite{chen2021pre}{, }APT\cite{huang2022atrous}{, }GWT\cite{bastos2023learnable}{, }\\DWP\cite{yang2023discrete}{, }AWT\cite{huang2021adaptive}{, }SAN\\\cite{kreuzer2021rethinking}{, }Spectformer\cite{patro2023spectformer}{, }SPT-SEG\cite{xu2024spectral}\\SVT\cite{patro2024scattering}{, }FreqDiMFT\cite{zhang2024frequency}, modelnode-task1, text width=33em]
                %     ]
                % ]  
                [
                    Time Series (\textbf{Sec.~\ref{sec:time-series}}), leaf-others,text width=9em
                    [
                        Convolution-Based, leaf-task10, text width=8.5em
                        [\cite{yu2024method,park2021fast,zhou2024fourier}{, }Tslanet\\\cite{eldele2024tslanet}\cite{kim2024neural,lange2021fourier,zhang2024frnet}\\\cite{cai2024msgnet}{, }RFF\cite{tompkins2018fourier}, modelnode-task1, text width=33em]
                    ]
                    [
                       Transformer-Based, leaf-task10, text width=8.5em
                        [LSTF\cite{ma2023long,zhou2024fourier,ni2024time}{, }TST\\\cite{jin2022time}\cite{chen2023lightweight,tran2023fourier}{, }ETSFormer\\\cite{woo2022etsformer}\cite{kang2023electric,yang2024fedaf}{, }FEDformer~\\\cite{zhou2022fedformer}{, } CRT\cite{zhang2023self}{,}W-Transformer\cite{sasal2022w}\\GRAU\cite{yang2024graformer,zhang2023self,wang2024card}{, }, modelnode-task1, text width=33em]  
                    ]
                ]
                [
                    Spatio-Temporal \\Dynamics (\textbf{Sec.~\ref{sec:spatio-temporal-dynamics}}), leaf-others,text width=9em
                    [
                        Convolution-Based, leaf-task10, text width=8.5em   [FNO~\cite{li2020fourier}{, }\cite{kovachki2021universal,tran2021factorized}\\\cite{song2022high,liu2024neural,zhao2024recfno}\\~\cite{zhao2023local}{,} S-FNOs\cite{bonev2023spherical}{, }U-FNO\cite{wen2022u}{, }\\FNON\cite{guan2023fourier}{, } FNOF\cite{kabri2023resolution}, modelnode-task1, text width=33em]
                    ]
                    [
                       Transformer-Based, leaf-task10, text width=8.5em
                        [\cite{guibas2021efficient,li2023fourier,cao2021choose}{,}Gnot~\cite{hao2023gnot}{, }\\IUFNO\cite{li2024transformer,yang2024enhancing}{, }\\AFNO\cite{guibas2021adaptive}{, }GNOT\cite{hao2023gnot}, modelnode-task1, text width=33em]
                    ]
                ]
                % [
                %     Graph (\textbf{Sec.~\ref{sec:others}}), leaf-others,text width=9em
                %     [
                %         Convolution-Based, leaf-task10, text width=8.5em
                %         [\cite{bruna2014spectral,defferrard2016convolutional,kipf2017semi}\\TIGraNet\cite{khasanova2017graph,lee2021fnet}{, }\\GWNN\cite{xu2023haar}{, }\cite{yadav2021graph}{,}TPAMI\cite{shivakumara2011laplacian}\\ML-GW\cite{rustamov2013wavelets}{, }SGWT\cite{pilavci2019spectral}, modelnode-task1, text width=33em]
                %     ]
                %     [
                %        Transformer-Based, leaf-task10, text width=8.5em
                %         [\cite{wang2022powerful}{, }MGT\cite{ngo2023multiresolution}{, }FSC\cite{jin2024transformer}{, }\\BWN\cite{jeong2024brainwavenet}{, } LSGC\cite{chen2024multi}{, }, modelnode-task1, text width=33em]
                %     ]
                %     % [
                %     %    Diffusion-Based, leaf-task10, text width=8.5em
                %     %     [Wavelet\cite{guth2022wavelet}{,} Wavelet~\cite{phung2023wavelet}{, }~\cite{jiang2023low}{, }\\CWR\cite{hu2024neural}\cite{hu2024neural,huang2024wavedm}{,} DiffLL\\~\cite{jiang2023low}{, }WaveDM\cite{huang2024wavedm}{, } CFWD\cite{xue2024low}, modelnode-task1, text width=33em]
                %     % ]
                % ]
            ]
        \end{forest}
    }
    \caption{Taxonomy of learning in frequency domain (overall version)}
    \label{fig_taxonomy1}
\end{figure*}
\section{Preliminaries and Motivation}
\label{sec:prelim}



\subsection{LLM Unlearning}

LLM unlearning refers to techniques that selectively remove specific behaviors or knowledge from a pre-trained language model while maintaining its overall functionality~\cite{yao2023large}. 
With the proliferation of LLMs, unlearning has gained significant attention due to its broad applications in safety alignment, privacy protection, and copyright compliance~\cite{eldan2023s,liu2024rethinking,jia-etal-2024-soul}. The evaluation and auditing of LLM unlearning spans from basic verbatim memorization to deeper knowledge memorization~\cite{shi2024muse}, with this work focusing on the latter.
As depicted in \autoref{fig:overalltask}, LLM unlearning operates as a targeted intervention within the model's knowledge representation framework. 
Its core objective is the selective removal of specific information while preserving the model's broader knowledge base (e.g, on retain set). 
This study focuses on the knowledge unlearning auditing that assesses unlearned models' behaviors through comprehensive audit cases. Given access to both forget and retain corpora, we generate a holistic set of test questions with reference answers to thoroughly evaluate whether an unlearned model exhibits any residual knowledge memorization.
% The effectiveness of unlearning is characterized by a distinctive performance pattern: the model should exhibit significantly reduced performance on tasks involving the targeted knowledge while maintaining competence across all other domains. This selective degradation in performance serves as a key indicator of successful unlearning.

% This paper focuses on auditing LLM knowledge une, specifically developing fine-grained test cases based on training data to identify instances where the unlearning process has failed to achieve its intended objectives.

% Formally, given a pre-trained language model $\mathcal{M}$ with parameters $\theta$, the unlearning process aims to derive a new model $\mathcal{M}'$ with parameters $\theta'$ that demonstrates forgetting of targeted information while preserving other capabilities.
% In the context of exact knowledge unlearning, we specifically focus on removing precise factual information or relationships from the model's knowledge base. Let $\mathcal{D}_{fgt}$ denote the forget dataset containing knowledge to be eliminated, and $\mathcal{D}_{ret}$ represent the retain dataset containing knowledge that should be preserved. The objective of exact knowledge unlearning can be formalized as:

% \begin{equation}
% \theta' = \argmin_{\theta'} \mathcal{L}_{fgt}(\mathcal{M}'(\theta'), \mathcal{D}_{fgt}) + \lambda \mathcal{L}_{ret}(\mathcal{M}'(\theta'), \mathcal{D}_{ret})
% \end{equation}

% where $\mathcal{L}_{fgt}$ measures the model's tendency to generate or recall information from $\mathcal{D}_{fgt}$, $\mathcal{L}_{ret}$ evaluates the preservation of knowledge in $\mathcal{D}_{ret}$, and $\lambda$ balances these competing objectives. A successfully unlearned model should satisfy:

% \begin{equation}
% P(\mathcal{M}'(\theta') \mid \mathcal{D}_{fgt}) \ll P(\mathcal{M}(\theta) \mid \mathcal{D}_{fgt})
% \end{equation}

% \begin{equation}
% P(\mathcal{M}'(\theta') \mid \mathcal{D}_{ret}) \approx P(\mathcal{M}(\theta) \mid \mathcal{D}_{ret})
% \end{equation}

% where $P(\mathcal{M} \mid \mathcal{D})$ represents the model's probability of generating or correctly responding to information in dataset $\mathcal{D}$. These conditions ensure that the unlearned model exhibits significantly reduced ability to recall forgotten knowledge while maintaining its performance on retained knowledge.



\subsection{Knowledge Graph}
\label{sec:pre_kg}

A knowledge graph (KG) is a structured multi-relational graph~\cite{bordes2013translating}, usually representing a collection of facts as a network of entities and the relationships between entities.
Formally, a KG \(\mathcal{G} = \langle \mathcal{E}, \mathcal{R}, \mathcal{F} \rangle\) could be considered a directed edge-labeled graph~\cite{ji2021survey}, which comprises a set \(\mathcal{E}\) of entities (e.g., \textit{Harry Potter}, \textit{Hogwarts School}), a set \(\mathcal{R}\) of relations (e.g., \textit{attends}), and a set $\mathcal{F}$ of facts. A fact is a triple containing the head entity \(e_1 \in \mathcal{E}\), the relation $r \in \mathcal{R}$, and the tail entity \(e_2 \in \mathcal{E}\) to show that there exists the relation from the tail entity to the head entity, denoted as \((e_1, r, e_2) \in \mathcal{F}\)~\cite{hogan2021knowledge}. To illustrate, the fact (\textit{Harry Potter}, \textit{attends}, \textit{Hogwarts School}) shows that there exists the \textit{attends} relation between \textit{Harry Potter} and \textit{Hogwarts School}, which indicates``Harry Potter attends Hogwarts School''.

\begin{figure*}[t]
  \centering
  \includegraphics[width=0.9\linewidth]{fig/fig_overview_fig.pdf}
  \caption{The diagram illustrates the MHA, MLA, and our MHA2MLA. It can be seen that the ``cached'' part is fully aligned with MLA after MHA2MLA. The input to the attention module is also completely aligned with MLA (the \colorbox{lightgray!60}{aligned region below}). Meanwhile, the parameters in MHA2MLA maximize the use of pre-trained parameters from MHA (the \colorbox{lightgray!60}{aligned region above}).}
  \vspace{-0.4cm}
  % https://1drv.ms/p/c/e248d5c415e14c2d/ETjEatb4CaVLpZfGpurOYKsB41mBrk6_He7OiQfC5mh_Vg?e=DJMN0j
  \label{fig:overview}
\end{figure*}

% MHA、MLA以及我们的MHA2MLA的示意图。可以看到,Cached部分在MHA2MLA后完全对齐MLA。注意力模块的输入也完全对齐了MLA(下方的对齐区域)。而参数部分最大化利用了MHA中预训练的参数(上方的对齐区域)。

\subsection{Motivation}
This section aims to illustrate why and how we consider employing KG to facilitate the holistic LLM unlearning audit. 
Two critical factors underpin this task.
\ding{182}\textbf{Audit Adequacy}: The Forget Dataset is an extensive, unstructured corpus. Existing approaches typically rely on the LLM's prior knowledge to directly generate QA pairs or segment the corpus and feed these segments to ChatGPT for automated QA pair generation. Such methods often fail to intuitively reflect or guarantee the sufficiency of the generate dataset.
\ding{183}\textbf{Knowledge Redundancy}: A more subtle and easily overlooked issue is that the Retain Dataset and Forget Dataset may contain overlapping knowledge. As illustrated in \autoref{fig:overalltask}, this overlapping knowledge should be retained by the unlearned model and, therefore not be treated as candidates for the unlearning efficacy audit. Existing evaluation benchmarks like MUSE often neglect this aspect, as evidenced by \autoref{fig:musecase}.



A KG can offer an effective solution to address these two challenges. 
First, the KG inherently captures the knowledge facts within the Forget Dataset at a fine-grained level, with each edge representing a minimal testable unit. 
By ensuring coverage of every edge in the KG, one can achieve a more intuitive and relatively comprehensive audit. 
Moreover, the structured data provided by the KG can facilitate the identification of identical knowledge facts present in both the Retain and Forget Datasets.
This capability allows for refinement of the initial forget knowledge graph by removing potentially retained information.
Finally, owing to recent advances in KG extraction technology, numerous automated extraction models and pipelines are available to support the automated construction of an audit dataset.


% \tikzset{
        my node/.style={
            draw,
            align=left,
            thin,
            text width=2.5cm, 
            % minimum height=1cm,
            rounded corners=3,
        },
        my leaf/.style={
            draw,
            align=left,
            thin,
            % minimum width=1cm,
            text width=4.5cm, 
            % text height=1cm, 
            % minimum height=0.5cm,
            rounded corners=3,
        }
}
\forestset{
  every leaf node/.style={
    if n children=0{#1}{}
  },
  every tree node/.style={
    if n children=0{minimum width=1em}{#1}
  },
}
\begin{forest}
    for tree={%
        % my node,
        every leaf node={my leaf, font=},
        every tree node={my node, font=\small, l sep-=4.5pt, l-=1.pt},
        anchor=west,
        inner sep=2pt,
        % l = 10pt,
        l sep=10pt, % control leaf to parent nodes gaps (horizontal)
        s sep=5pt, % control node gaps (vertical)
        fit=tight,
        grow'=east,
        edge={thick},
        parent anchor=east,
        child anchor=west,
        if n children=0{tier=last}{},
        edge path={
            \noexpand\path [draw, \forestoption{edge}] (!u.parent anchor) -- +(5pt,0) |- (.child anchor)\forestoption{edge label};
        },
        if={isodd(n_children())}{
            for children={
                if={equal(n,(n_children("!u")+1)/2)}{calign with current}{}
            }
        }{}
    }
    % text width=3.5cm
    % {Organizational Structure}, draw=gray, color=gray!100, fill=gray!15, very thick, text=black, text width=2.1cm,
    % [draw=none, fill=none, text width=0, minimum height=0, inner sep=0pt, outer sep=0pt,
    [{Organizational Structure}, draw=gray, color=gray!100, fill=gray!15, very thick, text=black, text width=2.1cm,
        [\cref{early_srs} Foundations of Reasoning LLMs, color=cyan!100, fill=cyan!15, very thick, text=black, text width=5.1cm
            [\cref{f_llm} Foundational LLMs, color=cyan!100, fill=cyan!15, very thick, text=black, text width=4.5cm
            ]
            [\cref{symb_exp} Symbolic Logic Systems, color=cyan!100, fill=cyan!15, very thick, text=black, text width=4.5cm
            ]
            [\cref{mcts} Monte Carlo Tree Search, color=cyan!100, fill=cyan!15, very thick, text=black, text width=4.5cm
            ]
            [\cref{rl} Reinforcement Learning, color=cyan!100, fill=cyan!15, very thick, text=black, text width=4.5cm
            ]
        ]
        [\cref{replication} Blueprinting Reasoning LLMs, color=lightcoral!100, fill=lightcoral!15, very thick, text=black, text width=5.1cm
            [\cref{o1_features} Feature Analysis, color=lightcoral!100, fill=lightcoral!15, very thick, text=black, text width=4.9cm
                [\cref{output_behaviour} Output Behaviour, color=lightcoral!100, fill=lightcoral!15, very thick, text=black, tier=Task, text width=5.1cm
                ]
                [\cref{dynamic_perspective} Training Dynamics, color=lightcoral!100, fill=lightcoral!15, very thick, text=black, tier=Task, text width=5.1cm
                ]
            ]
            [\cref{foundations} Core Method, color=lightcoral!100, fill=lightcoral!15, very thick, text=black, text width=4.9cm
                [\cref{structure_search} Structure Search, color=lightcoral!100, fill=lightcoral!15, very thick, text=black, tier=Task, text width=5.1cm
                ]
                [\cref{prm} Reward Modeling, color=lightcoral!100, fill=lightcoral!15, very thick, text=black, tier=Task, text width=5.1cm
                ]
                [\cref{self-improve} Self Improvement, color=lightcoral!100, fill=lightcoral!15, very thick, text=black, tier=Task, text width=5.1cm
                ]
                [\cref{macro_action} Macro Action, color=lightcoral!100, fill=lightcoral!15, very thick, text=black, tier=Task, text width=5.1cm
                ]
                [\cref{rl_supervise} Reinforcement Fine-Tuning, color=lightcoral!100, fill=lightcoral!15, very thick, text=black, tier=Task, text width=5.1cm
                ]
            ]
            [\cref{evolutionary} Reasoning LLMs Evolution, color=lightcoral!100, fill=lightcoral!15, very thick, text=black, text width=4.9cm
            ]
        ]
        [\cref{benchmark} Benchmarking Reasoning LLMs, color=darkpastelgreen!100, fill=darkpastelgreen!15, very thick, text=black, text width=5.1cm
            [\cref{benchmark_category} Benchmark Categories, color=darkpastelgreen!100, fill=darkpastelgreen!15, very thick, text=black, text width=4.55cm
            ]
            [\cref{metrics} Evaluation Metrics, color=darkpastelgreen!100, fill=darkpastelgreen!15, very thick, text=black, text width=4.55cm
            ]
            [\cref{performance_compare} Performance Comparison, color=darkpastelgreen!100, fill=darkpastelgreen!15, very thick, text=black, text width=4.55cm
            ]
        ]
    ]
\end{forest}



\section{Approaches for Frequency Transform}\label{sec:time-series}
\label{sec:pre}
% need to compress  (simpler intro and insight of each trasnform)

Frequency transforms are categorized into Fourier, wavelet, and Laplace transforms based on their formulations and applications. To provide a comprehensive overview, Table~\ref{tab:frequency_transform_methods} summarizes representative frequency transform methods. 


\begin{table*}[!ht] % 使用 !ht 替代 htbp,!ht 会强制表格尽可能出现在当前位置,而不强制它浮动到顶部或底部。
    \centering
    \caption{Summary of representative frequency transform methods in our framework.}
    \label{tab:frequency_transform_methods}
    \vspace{-0.3cm}
    \scriptsize
    \setlength{\tabcolsep}{6pt}
    \resizebox{\textwidth}{!}{
        \fontsize{12}{1.25\baselineskip}\selectfont % 设置字体大小为12pt, 1.25倍行距
        \begin{tabular}{c|c|c|l} 
        \toprule[1.2pt]
        \textbf{\makecell[c]{Frequency \\ Transform}} & \textbf{\makecell[c]{Categories and \\ Representative Methods}} & \textbf{Expression} & \textbf{Notes (Advantages \& Disadvantages)} \\ 
        \midrule[1.2pt]
        \multirow{17}{*}{\makecell[c]{\\ \textbf{Fourier} \\ \textbf{Transform}}} 
            & \makecell[c]{Discrete Fourier Transform (DFT)} & \makecell[c]{$X[k] = \sum_{n=0}^{L-1} x[n] e^{-j\frac{2\pi}{L}kn}$} & 
            \makecell[l]{
                \textcolor{green}{\checkmark} Well-suited for stationary signals \\
                \textcolor{green}{\checkmark} Captures global frequency components efficiently \\
                \textcolor{red}{$\times$} Cannot handle non-stationary signals \\
                \textcolor{red}{$\times$} Loses time-domain information (no localization) \\
            } \\
            \cmidrule(r){2-4}
            & \makecell[c]{Continuous Fourier Transform (CFT)} & 
            \makecell[c]{$X(f) = \int_{-\infty}^{\infty} x(t)e^{-j2\pi ft} dt$} & 
            \makecell[l]{
                \textcolor{green}{\checkmark} Used for theoretical frequency analysis \\
                \textcolor{green}{\checkmark} Provides continuous spectrum analysis \\
                \textcolor{red}{$\times$} Not practical for discrete signals \\
            } \\ 
            \cmidrule(r){2-4}
            & \makecell[c]{Fast Fourier Transform (FFT)} & - &  
            \makecell[l]{
                \textcolor{green}{\checkmark} Fast computation ($\mathcal{O}(Nlog N)$ complexity) \\
                \textcolor{green}{\checkmark} Used in real-time applications \\
                \textcolor{red}{$\times$} Shares the same limitations as DFT \\
            } \\
            \cmidrule(r){2-4}
            & \makecell[c]{Short-Time Fourier Transform (STFT)} &  
            \makecell[c]{$X(t, f) = \int_{-\infty}^{\infty} x(\tau) w(\tau - t) e^{-j2\pi f\tau} d\tau$} &  
            \makecell[l]{
                \textcolor{green}{\checkmark} Allows frequency analysis with time localization \\
                \textcolor{green}{\checkmark} Common in speech and signal processing \\
                \textcolor{red}{$\times$} Limited resolution due to fixed window size \\
                \textcolor{red}{$\times$} Trade-off between time and frequency resolution \\
            } \\
            \cmidrule(r){2-4}
            & \makecell[c]{Fractional Fourier Transform (FrFT)} & - &  
            \makecell[l]{
                \textcolor{green}{\checkmark} Generalizes FT for non-stationary signals \\
                \textcolor{green}{\checkmark} Bridges time-frequency representation \\
                \textcolor{red}{$\times$} More complex and computationally intensive \\
            } \\
        \midrule
        \multirow{5}{*}{\makecell[c]{\textbf{Wavelet} \\ \textbf{Transform}}} 
            & \makecell[c]{Discrete Wavelet Transform (DWT)} & $D(a, b) = \frac{1}{\sqrt{b}} \sum_{m=0}^{p-1} f[t_m] \psi \left( \frac{t_m - a}{b} \right)$ &  
            \makecell[l]{
                \textcolor{green}{\checkmark} Captures both time and frequency information \\
                \textcolor{green}{\checkmark} Handles non-stationary signals well \\
                \textcolor{red}{$\times$} Requires careful wavelet selection \\
                \textcolor{red}{$\times$} High computational cost \\
            } \\
            \cmidrule(r){2-4}
            & Continuous Wavelet Transform (CWT) & \makecell[c]{$F(\tau, s) = \frac{1}{\sqrt{|s|}} \int_{-\infty}^{\infty} f(t) \psi^* \left( \frac{t - \tau}{s} \right) dt$} &  
            \makecell[l]{
                \textcolor{green}{\checkmark} Provides continuous time-frequency representation \\
                \textcolor{green}{\checkmark} Better suited for complex signals \\
                \textcolor{red}{$\times$} Computationally expensive \\
                \textcolor{red}{$\times$} Redundant representation due to continuous scaling \\
            } \\ 
            
        \midrule
        \multirow{5}{*}{\makecell[c]{\textbf{Laplace} \\ \textbf{Transform}}}  
            & \makecell[c]{Unilateral Laplace Transform} & \makecell[c]{$F(s) = \mathcal{L}\{f(t)\} = \int_0^\infty f(t)e^{-st} dt$} &  
            \makecell[l]{
                \textcolor{green}{\checkmark} Useful for control systems and differential equations \\
                \textcolor{green}{\checkmark} Helps analyze system stability \\
                \textcolor{red}{$\times$} Less common in traditional time series analysis \\
            } \\
            \cmidrule(r){2-4}
            & Bilateral Laplace Transform &  
            \makecell[c]{$F(s) = \int_{-\infty}^{\infty} f(t)e^{-st} dt$} &  
            \makecell[l]{
                \textcolor{green}{\checkmark} Generalizes the Fourier transform \\
                \textcolor{green}{\checkmark} Used in engineering and systems analysis \\
                \textcolor{red}{$\times$} Computationally intensive \\
            } \\
        \midrule
        \multirow{5}{*}{\makecell[c]{\textbf{Graph} \\ \textbf{Fourier} \\ \textbf{Transform}}}  
            & \makecell[c]{Spectral Graph Fourier Transform (GFT) } &  
            \makecell[c]{$X(\lambda) = U^T x$} &  
            \makecell[l]{
                \textcolor{green}{\checkmark} Extends Fourier Transform to graph data \\
                \textcolor{green}{\checkmark} Useful for irregularly structured time series \\
                \textcolor{red}{$\times$} Requires graph construction and eigen decomposition \\
            } \\
            \cmidrule(r){2-4}
            & \makecell[c]{Wavelet Graph Transform} & - &  
            \makecell[l]{
                \textcolor{green}{\checkmark} Provides localized frequency analysis on graphs \\
                \textcolor{green}{\checkmark} Used in social networks and bioinformatics \\
                \textcolor{red}{$\times$} More complex than traditional wavelets \\
            } \\
        \bottomrule[1.2pt]
        \end{tabular}}
    \vspace{-0.5cm}
\end{table*}


\subsection{Fourier Transform}

The Fourier transform converts a time-domain signal into its frequency-domain representation. Widely used variants include the Discrete Fourier Transform (DFT), Continuous Fourier Transform (CFT), and Fast Fourier Transform (FFT). DFT captures global frequency components efficiently but loses time-domain information and cannot handle non-stationary signals. CFT provides a continuous spectrum but is impractical for discrete signals. FFT offers fast computation with $\mathcal{O}(N \log N)$ complexity, making it suitable for real-time applications, though it shares DFT's limitations. Each method has unique advantages and trade-offs depending on the signal characteristics and application requirements. Other methods, such as the Short-Time Fourier Transform (STFT)~\cite{yao2019stfnets} and the Fractional Fourier Transform (FrFT)~\cite{kocc2022fractional}, address specific needs. STFT enables frequency analysis with time localization, commonly used in speech and signal processing, but involves a trade-off between time and frequency resolution. FrFT generalizes the Fourier transform for non-stationary signals but is more complex and computationally intensive.

To support graph data, extensions like the spectral graph Fourier transform~\cite{defferrard2016convolutional} and wavelet graph transform~\cite{xu2019graph} provide localized frequency analysis for irregularly structured time series. However, they require graph construction and eigen decomposition, and are more complex than traditional methods.

% The Fourier transform is a mathematical technique that converts a time-domain signal into its frequency-domain representation. For an input time series $(x_1, x_2, \cdots, x_L)$, its frequency spectrum $F(\omega_k)$ can be obtained through the discrete Fourier transform as follows~\cite{wen2020time}:
% \begin{equation}
% \begin{aligned}
%     F(\omega_k) = \frac{1}{L} \sum_{t=0}^{L-1} (x_t \cdot e^{-j \omega_k t}) = A(\omega_k) \cdot \mathrm{exp}[j \theta(\omega_k)],
% \end{aligned}
% \end{equation}
% where $\omega = \frac{2 \pi k}{L}$ is the angular frequency, $A(\omega_k)$ is the amplitude spectrum, $\theta(\omega_k)$ is the phase spectrum, and $j = \sqrt{-1}$.

% Recent studies highlight the potential of the Fourier transform as a data augmentation and feature engineering technique in time-series analysis. The most widely used Fourier transforms include the Discrete Fourier Transform (DFT) and Fast Fourier Transform (FFT). For instance, Alaa \textit{et al.}~\cite{alaa2021generative} and Zhang \textit{et al.}~\cite{zhang2022tfad} proposed DFT-based data augmentation techniques to increase labeled data and exploit the unique properties of time-series data. Woo \textit{et al.}~\cite{woo2022cost} learned trend representations in the time domain, while seasonal representations were modeled by a Fourier layer in the frequency domain. Zhou \textit{et al.}~\cite{zhou2022fedformer} and Galán-Sales \textit{et al.}~\cite{galan2023approach} demonstrated FFT's effectiveness as a feature engineering tool for improving time-series forecasting. In addition, advanced Fourier-based architectures, such as FourierGNN~\cite{yi2023fouriergnn} and neural Fourier transform~\cite{koren2024interpretable}, have been proposed, integrating multi-dimensional Fourier transforms, temporal convolutional networks, and graph-based models to enhance the accuracy and interpretability of time-series forecasting. 

% Fourier Neural Operators (FNOs) have proven effective for solving parametric partial differential equations and inspired extensions to diverse applications~\cite{guibas2021adaptive}. Building on this foundation, Li and Yang~\cite{li2023gafno} introduced GAFNO, a gated adaptive FNO specifically designed for time series analysis, leveraging adaptive mechanisms to capture multiscale dependencies. To further enhance time-series modeling, Cho \textit{et al.}~\cite{cho2024operator} proposed Branched Fourier Neural Operators (BFNOs), addressing the limitations of traditional FNOs with a more expressive architecture.



% Yi \textit{et al.}~\cite{yi2023fouriergnn} introduced FourierGNN, a novel graph-based architecture for multivariate time series forecasting, utilizing hypervariate graphs and Fourier Graph Operators to unify spatiotemporal dynamics, achieving efficient and expressive forecasting performance.

% Gao \textit{et al.}~\cite{gao2020robusttad} presents RobustTAD, a framework leveraging time-series decomposition and convolutional neural networks, including frequency-domain data augmentation, to enhance anomaly detection performance.

% Alaa et al.~\cite{alaa2021generative} introduced Fourier Flow, a likelihood model leveraging frequency-domain representations of time series via DFT and spectral filtering for flexible and efficient analysis.

% Zhang \textit{et al.}~\cite{zhang2022tfad} further proposed TFAD, a time-frequency analysis-based model that exploits both time and frequency domains, incorporating time-series decomposition and data augmentation to improve both performance and interpretability.

% Galán-Sales et al.~\cite{galan2023approach} investigated the potential of FFT as a feature engineering method to enhance the accuracy and efficiency of time-series forecasting models.

% Zhou \textit{et al.}~\cite{zhou2022fedformer} presented FEDformer, a Frequency Enhanced Decomposed Transformer that combines seasonal-trend decomposition and Fourier transform to efficiently capture global and detailed structures in time series, achieving superior forecasting performance with reduced complexity.

% Zhou \textit{et al.}~\cite{zhou2022film} introduced FiLM, a Frequency Improved Legendre Memory model that combines Legendre polynomial projections for historical approximation, Fourier projection for noise reduction, and low-rank approximation to enhance long-term time series forecasting.

% Koren \textit{et al.}~\cite{koren2024interpretable} presented the Neural Fourier Transform (NFT), which integrates multi-dimensional Fourier transforms with Temporal Convolutional Networks to enhance the accuracy and interpretability of multivariate time-series forecasting.

% Cao \textit{et al.}~\cite{cao2020spectral} introduced StemGNN, which leverages graph Fourier transform and DFT to jointly model inter-series correlations and temporal dependencies in the spectral domain for improved multivariate time-series forecasting. 


\subsection{Wavelet Transform}

% Wavelet transform transforms a time series using wavelets as basis functions to reduce data size or noise. For a continuous time series $f(t)$, it decomposes the signal into time-localized frequency components:
% \begin{equation}
% \begin{aligned}
%     F(\tau, s) = \frac{1}{\sqrt{\vert s \vert}} \int_{-\infty}^{\infty} f(t) \psi^* \left( \frac{t - \tau}{s} \right) dt,
% \end{aligned}
% \end{equation}
% where $\tau$ is the translation parameter, determining the position of the wavelet along the time axis, and $s$ is the scale parameter, which controls the width of the wavelet and thus affects the frequency resolution.

Wavelet transform uses wavelets as basis functions to transform a time series, reducing data size or noise. The most common wavelet transforms are the Discrete Wavelet Transform (DWT) and Continuous Wavelet Transform (CWT), both widely used in time-series analysis. DWT captures both time and frequency information, making it particularly effective for handling non-stationary signals. However, it requires careful wavelet selection and incurs a high computational cost. In contrast, CWT provides a continuous time-frequency representation, which is better suited for analyzing complex signals. However, CWT is computationally expensive and results in redundant representations due to continuous scaling.

Recent works have leveraged wavelet transforms for diverse tasks, such as optimizing time-frequency representations through non-linear filter-bank transformations~\cite{cosentino2020learnable}, isolating periodic components using the maximal overlap DWT~\cite{wen2021robustperiod}, and integrating wavelet methods into deep learning frameworks to capture both frequency and time-domain features~\cite{yang2023waveform}. Moreover,~\citep{liang2024waverora} introduced wavelet-based frameworks that leverage time-frequency features to enhance forecasting efficiency and accuracy.

% Furthermore, some methods combine the strengths of Fourier and wavelet transforms to improve time-series analysis. Zhou \textit{et al.}~\cite{zhou2022fedformer} proposed FEDformer, which integrates Fourier transform for capturing global patterns and wavelet transform for modeling local structures, achieving a balance between accuracy and computational efficiency. Liu \textit{et al.}~\cite{liu2024wftnet} introduced WFTNet, which incorporates both transforms with a periodicity-weighted coefficient to adaptively balance their contributions. These hybrid approaches effectively exploit the complementary strengths of Fourier and wavelet transforms, setting new benchmarks in long-term time-series forecasting.

\subsection{Laplace Transform}

% Laplace transform is a crucial tool for analyzing linear time-invariant systems, converting time-domain functions into functions of a complex frequency variable $s$. The unilateral Laplace transform of a function $f(t), t \ge 0$, is defined as:
% \begin{equation}
% \begin{aligned}
%     F(s) = \mathcal{L}\{f(t)\} = \int_0^\infty f(t)e^{-st} dt
% \end{aligned}
% \end{equation}
% where $F(s)$ represents the Laplace transform of $f(t)$. $s$ is a complex frequency variable, often expressed as $s = \sigma + j\omega$, with $\sigma$ representing the real part (related to exponential decay) and $\omega$ the imaginary part (related to frequency). % The integral's lower limit of 0 reflects the typical application to causal signals (signals that are zero for $t < 0$).

The Laplace transform is a key tool for analyzing linear time-invariant systems, converting time-domain functions into functions in the complex frequency domain. The two primary types of Laplace transforms are unilateral and bilateral. The unilateral Laplace transform is particularly useful in control systems and the analysis of differential equations, as it aids in system stability analysis. However, it is less commonly used in traditional time-series analysis. The bilateral Laplace transform generalizes the Fourier transform and is used primarily in engineering and systems analysis, but it is computationally intensive due to its broader scope and complexity.

While the Laplace transform has a wide range of applications in machine learning, its direct application to time-series data remains limited, likely due to challenges in integrating it with complex temporal structures. For instance,~\citep{ambhika2024time} proposed a hybrid model combining Laplace transform-based deep recurrent neural networks with long short-term memory networks for time-series prediction. Similarly,~\citep{chen2024laplacian} and~\citep{shu2024low} leveraged Laplacian transforms for traffic time-series imputation, using methods like low-rank completion, Laplacian kernel regularization, and FFT. These studies highlight the Laplace transform’s potential in improving time-series modeling by addressing challenges in data representation and computational efficiency.









% \begin{table*}[htbp]
%     \setlength{\tabcolsep}{10pt}
%     \centering
%     \scriptsize
%     \caption{Summary of representative frequency transform methods in our framework}
%     \label{tab:frequency_transform_methods} % 添加表格标签
%     \setlength{\tabcolsep}{2mm}{
%     \begin{threeparttable}
%         \begin{tabular}{c|c|c|c} % 修正表格列对齐格式
%         \toprule
%         \textbf{\makecell[c]{Frequency \\ Transform}} & \textbf{\makecell[c]{Categories and \\ Representative Methods}} & \textbf{Expression} & \textbf{Notes} \\ % 添加标题行的对齐符号
%         \midrule
%         \multirow{4}{*}{\makecell[c]{Fourier \\ Transform}} 
%             & \makecell[c]{Discrete Fourier Transform~\upcite{eldele2024tslanet}} & \makecell[c]{$X[k] = \sum_{n=0}^{L-1} (x[n] e^{-j\frac{2\pi}{L}kn})$} & \makecell[c]{$X[k]$ denotes the dis-\\crete frequency spectrum} \\
%             \cmidrule(r){2-4}
%             & Continuous Fourier Transform & 
%             \makecell[c]{$X(f) = \int_{-\infty}^{\infty} x(t)e^{-j2\pi ft} dt$} & \makecell[c]{$X(f)$ is the continu-\\ous frequency spectrum} \\
%             % \cmidrule(r){2-4}
%             % & Fast Fourier Transform & - & - \\
%             % \cmidrule(r){2-4}
%             % & \makecell[c]{Fractional Fourier Transform~\upcite{kocc2022fractional}} & - & - \\
%             % \cmidrule(r){2-4}
%             % & \makecell[c]{Graph Fourier Transform~\upcite{defferrard2016convolutional}} & - & - \\ 
%         \midrule
%         \multirow{3}{*}{\makecell[c]{Wavelet \\ Transform}} 
%             & Discrete Wavelet Transform & $D(a, b) = \frac{1}{\sqrt{b}} \sum_{m=0}^{p-1} f[t_m] \psi \left( \frac{t_m - a}{b} \right)$ & - \\
%             \cmidrule(r){2-4}
%             & Continuous Wavelet Transform & \makecell[c]{$F(\tau, s) = \frac{1}{\sqrt{\vert s \vert}} \int_{-\infty}^{\infty} f(t) \psi^* \left( \frac{t - \tau}{s} \right) dt$} & - \\ 
%         \midrule
%         \makecell[c]{Laplace \\ Transform}  & \makecell[c]{Unilateral Laplace Transform~\upcite{ambhika2024time}} & \makecell[c]{$F(s) = \mathcal{L}\{f(t)\} = \int_0^\infty f(t)e^{-st} dt$ } & - \\ 
%         \bottomrule
%         \end{tabular}
%         % \begin{tablenotes}
%         % \scriptsize
%             % \item * where $X[k]$ denotes the discrete frequency spectrum, $k$ represents the discrete frequency index, and $j = \sqrt{-1}$.
%             % \item ** where $X(f)$ is the continuous frequency spectrum, $f$ is the continuous frequency variable. $x(t)$ is the continuous-time signal and $j = \sqrt{-1}$ is the imaginary unit.
%         % \end{tablenotes}
%     \end{threeparttable}}
% \end{table*}



% Predicting long-term trends in time series data (e.g., energy consumption, weather patterns, traffic flow) continues to be a difficult problem. Because the frequency domain is helpful for capturing long-term patterns in time series, frequency domain transformations have been widely used in time series prediction recently, as shown in Figure~\ref{figure-time-series}. One category of methods combines frequency domain with Convolutional models~\cite{krizhevsky2012imagenet}, while another uses Transformer-based models~\cite{vaswani2017attention} as a basis for combination.

% \begin{figure}[ht]
%     \centering
%     \setlength{\belowcaptionskip}{-0.5cm}  % Adjust this to reduce spacing
%     \includegraphics[width=0.475\textwidth]{figures/time-series.pdf}
%     \caption{Time Series in the Frequency Domain}
%     \label{figure-time-series}
%     % \vspace{-0.3cm}  % Adjust this to further reduce spacing
% \end{figure}

% \subsection{Convolution-based}
% Convolution-based methods~\cite{yu2024method,eldele2024tslanet,park2021fast,zhou2024fourier,kim2024neural,lange2021fourier,zhang2024frnet,cai2024msgnet} for time series prediction have been a hot topic recent years. A former study~\cite{park2021fast} proposes a novel method called Partial Fourier Transformation (PFT), which offers a precise and efficient approach for calculating a subset of Fourier coefficients. PFT achieves this by employing polynomials to approximate a portion of the twiddle factors (trigonometric constants), effectively lowering computational complexity resulting from the multitude of these factors. PFT analyzes the asymptotic time complexity of PFT concerning input and output dimensions, as well as tolerance levels. Furthermore, PFT demonstrates that PFT allows users to define a specific approximation error threshold, offering flexibility crucial for scenarios where rapid evaluation is a top priority. Another former study~\cite{lange2021fourier} presents an algorithm that shares similarities with the Fourier transform but operates without relying on assumptions of periodicity. This feature enables forecasting even with irregular sampling intervals. Extending this approach to nonlinear signals involves incorporating Koopman theory. The resultant algorithm conducts a spectral decomposition within a nonlinear, data-driven framework. Despite the highly non-convex nature of the optimization objective in both cases, transforming the objective into the frequency domain facilitates the computation of global optima for the error surface efficiently and at scale, leveraging the computational efficiency of the Fast Fourier Transform. These methods, closely linked to Bayesian Spectral Analysis, naturally yield metrics for quantifying uncertainty in spectral forecasting processes. By contrast, recent studies tend to adopt adaptive frequency methods to solve higher and lower frequencies of time series.
% A recent work~\cite{eldele2024tslanet} proposes a method Tslanet, incorporating an adaptive spectral block via \emph{Fourier transform}, employing Fourier analysis for improved feature representation and the detection of both short-term and long-term dependencies. Noise reduction is achieved through adaptive thresholding. 

% % Furthermore, an interactive convolutional block, trained with self-supervised learning, enhances the model's ability to interpret complex temporal patterns and improves its generalizability across diverse datasets.

% \subsection{Transformer-based}
% Transformer-based methods~\cite{ma2023long,zhou2024fourier,ni2024time,chen2023lightweight,tran2023fourier,kang2023electric,yang2024fedaf}
% for time series prediction aim to provide better performance in the long-term prediction. To capture global-view dependencies of time series, Zhou \textit{et al.}~\cite{zhou2022fedformer} propose a method, namely FEDformer, to decompose Transformer with \emph{Fourier Transform} to compact representations of long-term time series patterns into frequency domain. To be specific, FEDformer integrates the Transformer model with the seasonal-trend decomposition technique, where this method grasps the overall pattern of time series data while Transformers delve into finer intricacies. To boost the Transformer's efficacy in long-range forecasting, FEDformer leverages the observation that many time series can be efficiently represented in common bases like the Fourier transform, leading to the creation of a frequency-enriched transformer. Meanwhile, via \emph{Fourier Transform}, FEDformer aims to capture pieces of information lost in the temporal domain. Another work~\cite{sasal2022w} introduces an innovative approach to learning representations of univariate time series, named W-Transformer, which is built upon a transformer encoder structure utilizing wavelets. The W-Transformers apply a maximal overlap discrete wavelet transformation to the time series information. Meanwhile, local transformers are adopted to effectively capture the nonstationary nature and intricate long-term nonlinear relationships within the time series data. Diferent from other studies, Jin \textit{et al.}~\cite{jin2022time} developed a novel approach for generating token sequences tailored for 1D data, namely TST, a fusion of the time series tokenizer and Transformer architecture. More specifically, TST introduces a way to generate token sequences from one-dimensional data, including time series data. This time series tokenizer is then integrated into a Transformer architecture. In this way, good performance is achieved.


% To capture temporal-spectral correlations effectively~\cite{yang2024graformer,zhang2023self,wang2024card}, to be specific, Zhang \textit{at al.}~\cite{zhang2023self} propose Cross Reconstruction Transformer (CRT). CRT facilitates time series representation learning by employing a cross-domain dropping-reconstruction task via extracting the frequency domain of the time series using the fast \emph{Fourier Transform} and randomly eliminating specific patches in both the time and frequency domains. Woo \textit{et al.}~\cite{woo2022etsformer} proposed ETSFormer, a fresh Transformer architecture tailored for time-series data. This model leverages the concept of exponential smoothing to enhance Transformers for time-series forecasting. Drawing inspiration from classical exponential smoothing techniques in time-series prediction, ETSFormer introduces the innovative concepts of Exponential Smoothing Attention and Frequency Attention. 
% % These mechanisms replace the conventional self-attention module in standard Transformers, enhancing the accuracy and efficiency of the model.


\section{Benchmarking}\label{sec:benchmark}

In this section, we show the performance of the KID approximation for the GER vector \( \b r \) and its computational complexity. In particular, through our experimental evaluation, we aim to do the following:
\begin{enumerate}[label=\bfseries(\roman*),leftmargin=*]
      \item support the asymptotic estimate for the approximation error \eqref{eq:err1} in terms of the number of moments \( M \) and the number of MC-vectors \( N_z \);
      \item support the computational complexity of the KID approximation using the (scaled) oracle choice for the parameters, Theorem~\ref{thm:error};
      \item compare the actual execution time of the approximation to the direct computation for complexes of different sizes and densities.
\end{enumerate}

\paragraph{Vietoris--Rips filtration.} Theorem~\ref{thm:error} and Equation~\eqref{eq:err1} describe the performance of the developed method in terms of the number of simplices \( m_k \). In order to appropriately numerically illustrate these behaviours, one should consider a family of arbitrarily large and dense simplicial complexes. For this reason, we opt to use simplicial complexes induced by the filtration procedure on point clouds. Formally, we proceed as follows:
\begin{enumerate}[leftmargin=*]
      \item  we consider \( m_0 \) points embedded in $\mathbb R^2$, sampled randomly in two clusters, i.e., \( \frac{m_0}{2}\) points are sampled from \( \mc N( \b 0, I )\) and \( \frac{m_0}{2}\) points are sampled from \( \mc N( c \b 1, I )\), for some \( c > 0 \); 
      \item then, for a fixed filtration threshold \( \epsilon > 0 \), a simplex \( \sigma = [v_{i_1}, ... v_{i_p} ] \) on these nodes enters the generated complex \( \mc K \) if and only if $d_{\mc M }(v_{i_j}, v_{i_k}) \le \epsilon$ for all paris $j$ and $k$. 
\end{enumerate}
This straightforward filtration is known as  Vietoris--Rips filtration, and the corresponding complex \( \mc K \) as a VR-complex. An illustrative example is provided in Figure~\ref{fig:example}. In the chosen setup, the value of the filtration parameter \( \epsilon \) naturally governs the density of the generated simplicial complexes of every order, as shown by the right panel in Figure~\ref{fig:example}: larger values of \( \epsilon \) define complexes with a higher number of edges, triangles, tetrahedrons, etc., until every possible simplex is included in \( \mc K \).

\begin{figure}[t]
      \centering
      \includegraphics[width = 1.0\columnwidth]{figures/example.pdf}
      \caption{ Example of VR-filtration. Left pane: point cloud with \( m_0 = 40 \) and filtration \( \epsilon = 1.5 \), inter-cluster distance \( c = 3 \). Right pane: dynamics of the number of simplices of different orders for varying filtration parameter \( \epsilon\). \label{fig:example}}
\end{figure}

\paragraph{Parameter choice and computational complexity.} 
The error estimate from Equation~\ref{eq:err1} suggests that the approximation error for the sparsifying norm \( \b p \) scales as \( M^{-1}\) in terms of the number of moments and as \( N_z^{-1/2}\) in terms of the number of Monte-Carlo vectors (MC-vectors). To illustrate this behaviour, we fix one of the parameters (\( M \) or \( N_z \)) to their theoretical estimates provided by Theorem~\ref{thm:error} and demonstrate the dynamic of the error \( \| \b p - \wh{\b p }\|_\infty \) as the function of the other parameter. As shown by Figure~\ref{fig:M_Nz}, the overall scaling law coincides with the estimates of Equation~\ref{eq:err1} in the case of \(1 \)-sparsification for \( \Lu 1\) operator. Note that all experiments are conducted in the at least minimally-dense setting, i.e. \( m_2 \ge m_1 \ln m_1 \).
\begin{figure}[t]
      \centering
      \includegraphics[width = 1.0\columnwidth]{figures/err_filt.pdf}
      \caption{
            Dependence of the approximation error \( \| \b p - \wh{ \b p } \|_\infty \) on the number of moments \( M \) and number of MC vectors \( N_z \). Values are tested up to (scaled) theoretical bounds from Thm~\ref{thm:error} (in red); line colors correspond to varying \( m_0 \) in the point cloud. Left pane: errors vs the number of moments \( M \) with fixed theoretical \( N_z \); right pane: errors vs the number of MC vectors \( N_z \) with fixed theoretical \( M \).  Errors are averaged over several generated VR-complexes; colored areas correspond to the spread of values. \label{fig:M_Nz}
      }
\end{figure}

Here, we explicitly highlight two observations from Figure~\ref{fig:M_Nz}: (1) larger and denser simplicial complexes tend to exhibit faster convergence in both parameters (especially in the number of moments \( M \)), and (2) Theorem~\ref{thm:error} provides theoretical (greedy) estimates for \( M \) and \( N_z \) that are sufficient for achieving the target approximation quality \(\delta\) and can be interpreted asymptotically. Consequently, one may choose scaled (and empirically sufficient) values for these parameters:
\[
M = \left\lceil \frac{m_{k+1}}{\delta\,m_k} \right\rceil
\quad\text{and}\quad
N_z = \left\lceil \frac{1}{10}\,\frac{8}{\pi^2}\,\frac{m_{k+1}^2}{\delta^2\,m_k^2} \right\rceil.
\]

\begin{figure}[t]
      \centering
      \includegraphics[width = 1.0\columnwidth]{figures/timings.pdf}
      \caption{
            Execution time of KID approximation for effective resistance of triangles, \( \V 2\) (left), and tethrahedrons, \( \V 3 \) (right). Line colors correspond to varying \( m_0 \)  in the point cloud; theoretical estimation of the computational complexity is given in dash.
            Execution times are averaged over several generated VR-complexes; colored areas correspond to the spread of values.  \label{fig:times}
       }
\end{figure}

Given this choice of parameters, in Figure~\ref{fig:times} we demonstrate that the complexity estimate
\(\mathcal{O}\!\bigl(\tfrac{m_{k+1}^4}{m_k^3}\bigr)\)
from Theorem~\ref{thm:error} aligns with the actual execution time of the KID approximation for varying filtration parameters \(\epsilon\) in the cases of \(1\)- and \(2\)-sparsification of VR-complexes.

\paragraph{Comparison with the direct computation.}

Finally, we compare the execution time of the KID approximation with that of the direct computation of the sparsifying measure \( \b{p} \) for \( 1 \)-sparsification, using the approximation parameters mentioned above (see Figure~\ref{fig:comparison}). Note that although the densest case complexity estimate 
\(\mathcal{O}\!\bigl(\delta^{-3}\,m_k^{\,1+\frac{4}{k+1}}\bigr)\)
suggests that the KID method's execution time might be comparable to direct computation, in practice the developed algorithm is significantly faster while still maintaining the target approximation error 
\(\|\b{p} - \widehat{\b{p}}\|_\infty \le \frac{\delta}{m_2}\).


\begin{figure}[t]
      \centering
      \includegraphics[width = 1.0\columnwidth]{figures/filtration100.pdf}
      \caption{
            Computation time comparison between KID-approximation \( \wh{ \b p} \), solid line, and directly computed sparsifying measure \( \b p \), dashed line (left), and corresponding approximation error \( \| \wh{ \b p} - \b p \|_\infty \) (right). Target approximation error is given in dotted line (right pane);  line colors correspond to varying \( m_0 \)  in the point cloud.
            Execution times are averaged over several generated VR-complexes. \label{fig:comparison}      
      }
\end{figure}

Additionally, we note that the performance of the direct computation of \(\b{p}\) for the largest considered point cloud with \(m_0 = 125\) highlights another important advantage of the KID approximation: reduced memory consumption. Indeed, whether one uses the definition of the GER vector 
\[
\b{r} 
= \diag \Bigl( B_{k+1}^\top (\Lu{k})^\dagger B_{k+1} \Bigr) 
\]
or the reformulation in terms of the right singular vector from Theorem~\ref{thm:GER_DOS}, a full SVD of \(\Lu{k}\) is required. In the case of point clouds with \(m_0 = 125\), denser VR-complexes lead to real-valued matrices of size \(10^4 \times 10^4\), resulting in substantial memory demands for the SVD. By contrast, the KID approximation avoids this decomposition and restricts the additional memory usage to storing Monte-Carlo matrices \(Z\) and their \texttt{matvecs} of dimension \(m_{k+1} \times N_z\), which is comparatively smaller.
\section{Toward Multi-dimensional Concept of Safety Fine-tuning Vulnerabilities}
\label{sec:application}

Previous analysis presents a multi-dimensional framework for understanding learned safety behaviors, where distinct features and dynamics emerge along different directions in residual space. In this section, we demonstrate how this framework provides practical insights into safety fine-tuning vulnerabilities by showing manipulating non-dominant directions can bypass learned safety capabilities. We explore two methods to circumvent the learned safety capabilities while preserving the model's refusal ability: (1) suppressing non-dominant components and (2) removing or rephrasing trigger tokens from jailbreak prompts. Here, we define "trigger tokens" as specific token sequences that induce changes in feature directions, as demonstrated in \autoref{tab:plrp_logitlens}.

\paragraph{Suppressing Non-Dominant Directions}
As shown in \autoref{iterpret_tokens}, removing \texttt{L14-C6} explains the model's learned ability to refuse PAIR-like jailbreaks. Building on this insight, we investigate the effect of suppressing most non-dominant components while leaving dominant components untouched. Formally:

\[
    \mathbf{x} := \mathbf{x} - \sum_{v_i \in V^{t:}} \alpha_i \mathbf{v}_i
    \label{eq:intervene_all}
\]

This approach allows us to examine whether safety alignment can be reversed by blocking only indirect features. To preserve the model's ability to refuse plainly harmful prompts, we exclude component directions with harmfulness correlations above 0.7. 


\paragraph{Trigger Removal Attack}
We next introduce a procedure to remove trigger tokens from jailbreaks. First, we apply token-wise PLRP to dominant directions of the final layers to identify a list of top trigger tokens that explain the refusal output. Then, we employ another LLM to iteratively rephrase the harmful prompt while avoiding these trigger tokens, similar to TAP~\cite{mehrotra2023tree}. These modified jailbreak prompts are incorporated into the safety fine-tuning dataset, and we evaluate the detection accuracy on a validation split. The detailed algorithm is provided in the Appendix~\ref{appd:trigger_removal}.

\subsection{Results}
\paragraph{Disrupting Non-dominant Directions Reduces Refusal}
In \autoref{fig:component_projections}, we analyze how different attacks affect the projection values compared to default prompts (\texttt{Harmful} and \texttt{Benign}). Both non-dominant suppression and trigger removal attacks cause the dominant component projection to deviate from harmful samples. This deviation leads to a lower refusal rate as projection values on the dominant component increase. Our analysis reveals that indirect features from non-dominant directions greatly influence the dominant directions. Interestingly, while trigger removal attacks shift projections closer to benign samples, non-dominant suppression pushes them in the opposite direction.

\paragraph{Trigger Removal is Resilient to Safety Fine-tuning}

\autoref{tab:exposure_acc} shows that removing triggers effectively prevents safety fine-tuning from generalizing to these attacks. The initial attack success rate is comparable to other methods for a pre-fine-tuned model. However, after fine-tuning on 80 samples per jailbreak, while the success rate of other jailbreaks drops to near zero, the Trigger Removal Attack maintains approximately 40\% effectiveness.


Overall, these findings confirm that non-dominant directions causally impact both the dominant component and safety behavior. Since these non-dominant directions capture features beyond query harmfulness like specific jail-break patterns, this suggests that safety training may model \emph{spurious correlations}~\cite{geirhos2020shortcut} in certain jailbreak patterns, allowing out-of-domain jailbreaks like the Trigger Removal Attack to weaken or bypass the learned alignment.

\begin{table}[t]
    \caption{Attack Pass Rate of jailbreak prompts on safety fine-tuned models under different exposure settings. \textsc{n-shot} indicates the number of samples of each jailbreak presented in the fine-tuning dataset.}
    \label{tab:exposure_acc}
    \vskip 0.15in
    \begin{center}
    \begin{small}
    \begin{sc}
    \setlength\tabcolsep{4pt}
    \begin{tabular}{lcccccc}
    \toprule
    Method & 0-shot & 10 & 20 & 40 & 80 & 160 \\
            & Success   & shot & shot & shot & shot & shot \\
    \midrule
    GPTFuzz  & 0.02 & 0.02 & 0.02 & 0.03 & 0.03 & 0.03 \\
    Flip     & 0.78 & 0.12 & 0.22 & 0.03 & 0.03 & 0.03 \\
    Pair     & 0.82 & 0.75 & 0.45 & 0.17 & 0.12 & 0.05 \\
    ReNellm  & 0.61 & 0.00 & 0.00 & 0.00 & 0.00 & 0.00 \\
    \midrule
    \begin{tabular}[c]{@{}l@{}} Trigger \\ Removal \end{tabular}     & 0.77 & 0.78 & 0.62 & 0.52 & 0.42 & 0.30 \\
    \bottomrule
    \end{tabular}
    \end{sc}
    \end{small}
    \end{center}
    \vskip -0.2in
\end{table}

We introduced the Bidirectional Diffusion Bridge Model (BDBM), a novel
framework for bidirectional image-to-image (I2I) translation using
a single network. By leveraging the Chapman-Kolmogorov Equation, BDBM
models the shared components of forward and backward transitions,
enabling efficient bidirectional generation with minimal computational
overhead. Empirical results demonstrated that BDBM consistently outperforms
existing I2I translation methods across diverse datasets.

Despite these strengths, BDBM has so far been applied exclusively
to the image domain. Extending it to other domains, such as text,
presents an exciting direction for future research. In particular,
exploring BDBM for multimodal tasks like image$\leftrightarrow$text
generation would be a promising avenue.



% !TEX root = main.tex

\section{Future research}\label{sec:future}
Below we list a few research questions, which we find interesting and
particularly promising directions after our contribution.

\para{Exact complexity for $3$-VASS}
We have shown that shortest paths in binary $3$-VASS are of at most triply-exponential length.
It is tempting to conjecture that actually the upper bound for the length of the paths is shorter,
at most doubly-exponential. We conjecture so and leave this conjecture to the future research.

\para{Example of a $3$-VASS with doubly-exponential path}
We have shown that shortest paths in binary $3$-VASS are of at most triple-exponential length.
However, currently we still do not know any example in which even a path of doubly-exponential length is needed,
it might be that paths of exponential length are sufficient leading to \pspace-completeness for binary $3$-VASS.
It would be very interesting to find an example of a binary $3$-VASS with shortest path between two configurations
being doubly exponential. An example of binary $4$-VASS of doubly-exponential shortest path is known (see Section 5 in~\cite{DBLP:conf/concur/Czerwinski0LLM20}). Maybe some modification of this $4$-VASS would allow to design a $3$-VASS with similar properties.

\para{Reachability for $d$-VASS with $d \geq 4$}
It is a natural question whether our techniques extend to higher dimensions.
The answer is: possibly yes, but we would need a few other structural results for $3$-VASS
to make a similar approach to $4$-VASS possible. In the proof of Lemma~\ref{lem:main} we do not only
use $2$-VASS reachability as a black box, but we use a deep understanding of the reachability relation in $2$-VASS
from~\cite{DBLP:conf/focs/0001CMOSW24}. Probably a similar understanding of the reachability relation for $3$-VASS would be needed
to advance understanding of $4$-VASS along our lines. 

In general it is very interesting to determine the complexity of the reachability problem for $d$-VASS.
We have excluded that for each $d \geq 3$ the problem is $\F_d$-completely, but it is still possible that
the problem is $\F_{d-C}$-complete for some constant $C \in \N$ and $d$ big enough.
Recall that in~\cite{DBLP:conf/fsttcs/CzerwinskiJ0LO23}
it was shown that the reachability problem for $(2d+4)$-VASS is $\F_d$-hard for any $d \geq 3$ and this
is the best currently known lower bound for arbitrary dimension.
Therefore the other natural possibility is that the reachability problem for $(2d+C)$-VASS is $\F_d$-complete for some
constant $C \in \N$. 

\para{Applications of the approximation technique}
Another natural research direction is to search for other applications of the technique of approximating the reachability sets,
which allows to lower the complexity down, below the size of the reachability set.
One particular case, which seems to be prone to such techniques is the $2$-VASS with some number of $\Z$-counters, namely counters, which can take values below zero.
The best complexity lower bound for the reachability problem in this model is \pspace-hardness inherited from~\cite{BlondinFGHM15},
while the best upper bound is Ackermann membership inherited from VASS reachability~\cite{LS19}.
The reachability sets for that systems are not necessarily semilinear.
This disqualifies most of the techniques relying on the semilinearity of reachability sets, but our techniques
seem to be promising for that model.



\section*{Acknowledgment}
The authors would like to thank Clement Svendsen for valuable measure theoretic insight. 

Kasper Green Larsen is co-funded by a DFF Sapere Aude Research Leader Grant No. 9064-00068B by the Independent Research Fund Denmark and co-funded by the European Union (ERC, TUCLA, 101125203). Natascha Schalburg is funded by the European Union (ERC, TUCLA, 101125203). Views and opinions expressed are however those of the author(s) only and do not necessarily reflect those of the European Union or the European Research Council. Neither the European Union nor the granting authority can be held responsible for them.

\section{Conclusion}
\label{sec:conclu}
This survey underscores the transformative role of frequency domain techniques in advancing time series analysis. By systematically reviewing Fourier, Laplace, and Wavelet Transforms, we provide a comprehensive understanding of their applications, strengths, and limitations. Our up-to-date pipeline highlights recent advancements, offering valuable insights for researchers and practitioners. This work not only fills a critical gap in the literature but also inspires innovative applications and fosters deeper exploration of frequency domain methodologies. The accompanying GitHub repository further enhances accessibility and reproducibility, paving the way for future advancements in the field.

\clearpage
\bibliographystyle{named}
\bibliography{ref}


\end{document}

