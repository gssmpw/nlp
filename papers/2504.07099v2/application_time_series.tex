\section{Approaches for Frequency Transform}\label{sec:time-series}
\label{sec:pre}
% need to compress  (simpler intro and insight of each trasnform)

Frequency transforms are categorized into Fourier, wavelet, and Laplace transforms based on their formulations and applications. To provide a comprehensive overview, Table~\ref{tab:frequency_transform_methods} summarizes representative frequency transform methods. 


\begin{table*}[!ht] % 使用 !ht 替代 htbp,!ht 会强制表格尽可能出现在当前位置,而不强制它浮动到顶部或底部。
    \centering
    \caption{Summary of representative frequency transform methods in our framework.}
    \label{tab:frequency_transform_methods}
    \vspace{-0.3cm}
    \scriptsize
    \setlength{\tabcolsep}{6pt}
    \resizebox{\textwidth}{!}{
        \fontsize{12}{1.25\baselineskip}\selectfont % 设置字体大小为12pt, 1.25倍行距
        \begin{tabular}{c|c|c|l} 
        \toprule[1.2pt]
        \textbf{\makecell[c]{Frequency \\ Transform}} & \textbf{\makecell[c]{Categories and \\ Representative Methods}} & \textbf{Expression} & \textbf{Notes (Advantages \& Disadvantages)} \\ 
        \midrule[1.2pt]
        \multirow{17}{*}{\makecell[c]{\\ \textbf{Fourier} \\ \textbf{Transform}}} 
            & \makecell[c]{Discrete Fourier Transform (DFT)} & \makecell[c]{$X[k] = \sum_{n=0}^{L-1} x[n] e^{-j\frac{2\pi}{L}kn}$} & 
            \makecell[l]{
                \textcolor{green}{\checkmark} Well-suited for stationary signals \\
                \textcolor{green}{\checkmark} Captures global frequency components efficiently \\
                \textcolor{red}{$\times$} Cannot handle non-stationary signals \\
                \textcolor{red}{$\times$} Loses time-domain information (no localization) \\
            } \\
            \cmidrule(r){2-4}
            & \makecell[c]{Continuous Fourier Transform (CFT)} & 
            \makecell[c]{$X(f) = \int_{-\infty}^{\infty} x(t)e^{-j2\pi ft} dt$} & 
            \makecell[l]{
                \textcolor{green}{\checkmark} Used for theoretical frequency analysis \\
                \textcolor{green}{\checkmark} Provides continuous spectrum analysis \\
                \textcolor{red}{$\times$} Not practical for discrete signals \\
            } \\ 
            \cmidrule(r){2-4}
            & \makecell[c]{Fast Fourier Transform (FFT)} & - &  
            \makecell[l]{
                \textcolor{green}{\checkmark} Fast computation ($\mathcal{O}(Nlog N)$ complexity) \\
                \textcolor{green}{\checkmark} Used in real-time applications \\
                \textcolor{red}{$\times$} Shares the same limitations as DFT \\
            } \\
            \cmidrule(r){2-4}
            & \makecell[c]{Short-Time Fourier Transform (STFT)} &  
            \makecell[c]{$X(t, f) = \int_{-\infty}^{\infty} x(\tau) w(\tau - t) e^{-j2\pi f\tau} d\tau$} &  
            \makecell[l]{
                \textcolor{green}{\checkmark} Allows frequency analysis with time localization \\
                \textcolor{green}{\checkmark} Common in speech and signal processing \\
                \textcolor{red}{$\times$} Limited resolution due to fixed window size \\
                \textcolor{red}{$\times$} Trade-off between time and frequency resolution \\
            } \\
            \cmidrule(r){2-4}
            & \makecell[c]{Fractional Fourier Transform (FrFT)} & - &  
            \makecell[l]{
                \textcolor{green}{\checkmark} Generalizes FT for non-stationary signals \\
                \textcolor{green}{\checkmark} Bridges time-frequency representation \\
                \textcolor{red}{$\times$} More complex and computationally intensive \\
            } \\
        \midrule
        \multirow{5}{*}{\makecell[c]{\textbf{Wavelet} \\ \textbf{Transform}}} 
            & \makecell[c]{Discrete Wavelet Transform (DWT)} & $D(a, b) = \frac{1}{\sqrt{b}} \sum_{m=0}^{p-1} f[t_m] \psi \left( \frac{t_m - a}{b} \right)$ &  
            \makecell[l]{
                \textcolor{green}{\checkmark} Captures both time and frequency information \\
                \textcolor{green}{\checkmark} Handles non-stationary signals well \\
                \textcolor{red}{$\times$} Requires careful wavelet selection \\
                \textcolor{red}{$\times$} High computational cost \\
            } \\
            \cmidrule(r){2-4}
            & Continuous Wavelet Transform (CWT) & \makecell[c]{$F(\tau, s) = \frac{1}{\sqrt{|s|}} \int_{-\infty}^{\infty} f(t) \psi^* \left( \frac{t - \tau}{s} \right) dt$} &  
            \makecell[l]{
                \textcolor{green}{\checkmark} Provides continuous time-frequency representation \\
                \textcolor{green}{\checkmark} Better suited for complex signals \\
                \textcolor{red}{$\times$} Computationally expensive \\
                \textcolor{red}{$\times$} Redundant representation due to continuous scaling \\
            } \\ 
            
        \midrule
        \multirow{5}{*}{\makecell[c]{\textbf{Laplace} \\ \textbf{Transform}}}  
            & \makecell[c]{Unilateral Laplace Transform} & \makecell[c]{$F(s) = \mathcal{L}\{f(t)\} = \int_0^\infty f(t)e^{-st} dt$} &  
            \makecell[l]{
                \textcolor{green}{\checkmark} Useful for control systems and differential equations \\
                \textcolor{green}{\checkmark} Helps analyze system stability \\
                \textcolor{red}{$\times$} Less common in traditional time series analysis \\
            } \\
            \cmidrule(r){2-4}
            & Bilateral Laplace Transform &  
            \makecell[c]{$F(s) = \int_{-\infty}^{\infty} f(t)e^{-st} dt$} &  
            \makecell[l]{
                \textcolor{green}{\checkmark} Generalizes the Fourier transform \\
                \textcolor{green}{\checkmark} Used in engineering and systems analysis \\
                \textcolor{red}{$\times$} Computationally intensive \\
            } \\
        \midrule
        \multirow{5}{*}{\makecell[c]{\textbf{Graph} \\ \textbf{Fourier} \\ \textbf{Transform}}}  
            & \makecell[c]{Spectral Graph Fourier Transform (GFT) } &  
            \makecell[c]{$X(\lambda) = U^T x$} &  
            \makecell[l]{
                \textcolor{green}{\checkmark} Extends Fourier Transform to graph data \\
                \textcolor{green}{\checkmark} Useful for irregularly structured time series \\
                \textcolor{red}{$\times$} Requires graph construction and eigen decomposition \\
            } \\
            \cmidrule(r){2-4}
            & \makecell[c]{Wavelet Graph Transform} & - &  
            \makecell[l]{
                \textcolor{green}{\checkmark} Provides localized frequency analysis on graphs \\
                \textcolor{green}{\checkmark} Used in social networks and bioinformatics \\
                \textcolor{red}{$\times$} More complex than traditional wavelets \\
            } \\
        \bottomrule[1.2pt]
        \end{tabular}}
    \vspace{-0.5cm}
\end{table*}


\subsection{Fourier Transform}

The Fourier transform converts a time-domain signal into its frequency-domain representation. Widely used variants include the Discrete Fourier Transform (DFT), Continuous Fourier Transform (CFT), and Fast Fourier Transform (FFT). DFT captures global frequency components efficiently but loses time-domain information and cannot handle non-stationary signals. CFT provides a continuous spectrum but is impractical for discrete signals. FFT offers fast computation with $\mathcal{O}(N \log N)$ complexity, making it suitable for real-time applications, though it shares DFT's limitations. Each method has unique advantages and trade-offs depending on the signal characteristics and application requirements. Other methods, such as the Short-Time Fourier Transform (STFT)~\cite{yao2019stfnets} and the Fractional Fourier Transform (FrFT)~\cite{kocc2022fractional}, address specific needs. STFT enables frequency analysis with time localization, commonly used in speech and signal processing, but involves a trade-off between time and frequency resolution. FrFT generalizes the Fourier transform for non-stationary signals but is more complex and computationally intensive.

To support graph data, extensions like the spectral graph Fourier transform~\cite{defferrard2016convolutional} and wavelet graph transform~\cite{xu2019graph} provide localized frequency analysis for irregularly structured time series. However, they require graph construction and eigen decomposition, and are more complex than traditional methods.

% The Fourier transform is a mathematical technique that converts a time-domain signal into its frequency-domain representation. For an input time series $(x_1, x_2, \cdots, x_L)$, its frequency spectrum $F(\omega_k)$ can be obtained through the discrete Fourier transform as follows~\cite{wen2020time}:
% \begin{equation}
% \begin{aligned}
%     F(\omega_k) = \frac{1}{L} \sum_{t=0}^{L-1} (x_t \cdot e^{-j \omega_k t}) = A(\omega_k) \cdot \mathrm{exp}[j \theta(\omega_k)],
% \end{aligned}
% \end{equation}
% where $\omega = \frac{2 \pi k}{L}$ is the angular frequency, $A(\omega_k)$ is the amplitude spectrum, $\theta(\omega_k)$ is the phase spectrum, and $j = \sqrt{-1}$.

% Recent studies highlight the potential of the Fourier transform as a data augmentation and feature engineering technique in time-series analysis. The most widely used Fourier transforms include the Discrete Fourier Transform (DFT) and Fast Fourier Transform (FFT). For instance, Alaa \textit{et al.}~\cite{alaa2021generative} and Zhang \textit{et al.}~\cite{zhang2022tfad} proposed DFT-based data augmentation techniques to increase labeled data and exploit the unique properties of time-series data. Woo \textit{et al.}~\cite{woo2022cost} learned trend representations in the time domain, while seasonal representations were modeled by a Fourier layer in the frequency domain. Zhou \textit{et al.}~\cite{zhou2022fedformer} and Galán-Sales \textit{et al.}~\cite{galan2023approach} demonstrated FFT's effectiveness as a feature engineering tool for improving time-series forecasting. In addition, advanced Fourier-based architectures, such as FourierGNN~\cite{yi2023fouriergnn} and neural Fourier transform~\cite{koren2024interpretable}, have been proposed, integrating multi-dimensional Fourier transforms, temporal convolutional networks, and graph-based models to enhance the accuracy and interpretability of time-series forecasting. 

% Fourier Neural Operators (FNOs) have proven effective for solving parametric partial differential equations and inspired extensions to diverse applications~\cite{guibas2021adaptive}. Building on this foundation, Li and Yang~\cite{li2023gafno} introduced GAFNO, a gated adaptive FNO specifically designed for time series analysis, leveraging adaptive mechanisms to capture multiscale dependencies. To further enhance time-series modeling, Cho \textit{et al.}~\cite{cho2024operator} proposed Branched Fourier Neural Operators (BFNOs), addressing the limitations of traditional FNOs with a more expressive architecture.



% Yi \textit{et al.}~\cite{yi2023fouriergnn} introduced FourierGNN, a novel graph-based architecture for multivariate time series forecasting, utilizing hypervariate graphs and Fourier Graph Operators to unify spatiotemporal dynamics, achieving efficient and expressive forecasting performance.

% Gao \textit{et al.}~\cite{gao2020robusttad} presents RobustTAD, a framework leveraging time-series decomposition and convolutional neural networks, including frequency-domain data augmentation, to enhance anomaly detection performance.

% Alaa et al.~\cite{alaa2021generative} introduced Fourier Flow, a likelihood model leveraging frequency-domain representations of time series via DFT and spectral filtering for flexible and efficient analysis.

% Zhang \textit{et al.}~\cite{zhang2022tfad} further proposed TFAD, a time-frequency analysis-based model that exploits both time and frequency domains, incorporating time-series decomposition and data augmentation to improve both performance and interpretability.

% Galán-Sales et al.~\cite{galan2023approach} investigated the potential of FFT as a feature engineering method to enhance the accuracy and efficiency of time-series forecasting models.

% Zhou \textit{et al.}~\cite{zhou2022fedformer} presented FEDformer, a Frequency Enhanced Decomposed Transformer that combines seasonal-trend decomposition and Fourier transform to efficiently capture global and detailed structures in time series, achieving superior forecasting performance with reduced complexity.

% Zhou \textit{et al.}~\cite{zhou2022film} introduced FiLM, a Frequency Improved Legendre Memory model that combines Legendre polynomial projections for historical approximation, Fourier projection for noise reduction, and low-rank approximation to enhance long-term time series forecasting.

% Koren \textit{et al.}~\cite{koren2024interpretable} presented the Neural Fourier Transform (NFT), which integrates multi-dimensional Fourier transforms with Temporal Convolutional Networks to enhance the accuracy and interpretability of multivariate time-series forecasting.

% Cao \textit{et al.}~\cite{cao2020spectral} introduced StemGNN, which leverages graph Fourier transform and DFT to jointly model inter-series correlations and temporal dependencies in the spectral domain for improved multivariate time-series forecasting. 


\subsection{Wavelet Transform}

% Wavelet transform transforms a time series using wavelets as basis functions to reduce data size or noise. For a continuous time series $f(t)$, it decomposes the signal into time-localized frequency components:
% \begin{equation}
% \begin{aligned}
%     F(\tau, s) = \frac{1}{\sqrt{\vert s \vert}} \int_{-\infty}^{\infty} f(t) \psi^* \left( \frac{t - \tau}{s} \right) dt,
% \end{aligned}
% \end{equation}
% where $\tau$ is the translation parameter, determining the position of the wavelet along the time axis, and $s$ is the scale parameter, which controls the width of the wavelet and thus affects the frequency resolution.

Wavelet transform uses wavelets as basis functions to transform a time series, reducing data size or noise. The most common wavelet transforms are the Discrete Wavelet Transform (DWT) and Continuous Wavelet Transform (CWT), both widely used in time-series analysis. DWT captures both time and frequency information, making it particularly effective for handling non-stationary signals. However, it requires careful wavelet selection and incurs a high computational cost. In contrast, CWT provides a continuous time-frequency representation, which is better suited for analyzing complex signals. However, CWT is computationally expensive and results in redundant representations due to continuous scaling.

Recent works have leveraged wavelet transforms for diverse tasks, such as optimizing time-frequency representations through non-linear filter-bank transformations~\cite{cosentino2020learnable}, isolating periodic components using the maximal overlap DWT~\cite{wen2021robustperiod}, and integrating wavelet methods into deep learning frameworks to capture both frequency and time-domain features~\cite{yang2023waveform}. Moreover,~\citep{liang2024waverora} introduced wavelet-based frameworks that leverage time-frequency features to enhance forecasting efficiency and accuracy.

% Furthermore, some methods combine the strengths of Fourier and wavelet transforms to improve time-series analysis. Zhou \textit{et al.}~\cite{zhou2022fedformer} proposed FEDformer, which integrates Fourier transform for capturing global patterns and wavelet transform for modeling local structures, achieving a balance between accuracy and computational efficiency. Liu \textit{et al.}~\cite{liu2024wftnet} introduced WFTNet, which incorporates both transforms with a periodicity-weighted coefficient to adaptively balance their contributions. These hybrid approaches effectively exploit the complementary strengths of Fourier and wavelet transforms, setting new benchmarks in long-term time-series forecasting.

\subsection{Laplace Transform}

% Laplace transform is a crucial tool for analyzing linear time-invariant systems, converting time-domain functions into functions of a complex frequency variable $s$. The unilateral Laplace transform of a function $f(t), t \ge 0$, is defined as:
% \begin{equation}
% \begin{aligned}
%     F(s) = \mathcal{L}\{f(t)\} = \int_0^\infty f(t)e^{-st} dt
% \end{aligned}
% \end{equation}
% where $F(s)$ represents the Laplace transform of $f(t)$. $s$ is a complex frequency variable, often expressed as $s = \sigma + j\omega$, with $\sigma$ representing the real part (related to exponential decay) and $\omega$ the imaginary part (related to frequency). % The integral's lower limit of 0 reflects the typical application to causal signals (signals that are zero for $t < 0$).

The Laplace transform is a key tool for analyzing linear time-invariant systems, converting time-domain functions into functions in the complex frequency domain. The two primary types of Laplace transforms are unilateral and bilateral. The unilateral Laplace transform is particularly useful in control systems and the analysis of differential equations, as it aids in system stability analysis. However, it is less commonly used in traditional time-series analysis. The bilateral Laplace transform generalizes the Fourier transform and is used primarily in engineering and systems analysis, but it is computationally intensive due to its broader scope and complexity.

While the Laplace transform has a wide range of applications in machine learning, its direct application to time-series data remains limited, likely due to challenges in integrating it with complex temporal structures. For instance,~\citep{ambhika2024time} proposed a hybrid model combining Laplace transform-based deep recurrent neural networks with long short-term memory networks for time-series prediction. Similarly,~\citep{chen2024laplacian} and~\citep{shu2024low} leveraged Laplacian transforms for traffic time-series imputation, using methods like low-rank completion, Laplacian kernel regularization, and FFT. These studies highlight the Laplace transform’s potential in improving time-series modeling by addressing challenges in data representation and computational efficiency.









% \begin{table*}[htbp]
%     \setlength{\tabcolsep}{10pt}
%     \centering
%     \scriptsize
%     \caption{Summary of representative frequency transform methods in our framework}
%     \label{tab:frequency_transform_methods} % 添加表格标签
%     \setlength{\tabcolsep}{2mm}{
%     \begin{threeparttable}
%         \begin{tabular}{c|c|c|c} % 修正表格列对齐格式
%         \toprule
%         \textbf{\makecell[c]{Frequency \\ Transform}} & \textbf{\makecell[c]{Categories and \\ Representative Methods}} & \textbf{Expression} & \textbf{Notes} \\ % 添加标题行的对齐符号
%         \midrule
%         \multirow{4}{*}{\makecell[c]{Fourier \\ Transform}} 
%             & \makecell[c]{Discrete Fourier Transform~\upcite{eldele2024tslanet}} & \makecell[c]{$X[k] = \sum_{n=0}^{L-1} (x[n] e^{-j\frac{2\pi}{L}kn})$} & \makecell[c]{$X[k]$ denotes the dis-\\crete frequency spectrum} \\
%             \cmidrule(r){2-4}
%             & Continuous Fourier Transform & 
%             \makecell[c]{$X(f) = \int_{-\infty}^{\infty} x(t)e^{-j2\pi ft} dt$} & \makecell[c]{$X(f)$ is the continu-\\ous frequency spectrum} \\
%             % \cmidrule(r){2-4}
%             % & Fast Fourier Transform & - & - \\
%             % \cmidrule(r){2-4}
%             % & \makecell[c]{Fractional Fourier Transform~\upcite{kocc2022fractional}} & - & - \\
%             % \cmidrule(r){2-4}
%             % & \makecell[c]{Graph Fourier Transform~\upcite{defferrard2016convolutional}} & - & - \\ 
%         \midrule
%         \multirow{3}{*}{\makecell[c]{Wavelet \\ Transform}} 
%             & Discrete Wavelet Transform & $D(a, b) = \frac{1}{\sqrt{b}} \sum_{m=0}^{p-1} f[t_m] \psi \left( \frac{t_m - a}{b} \right)$ & - \\
%             \cmidrule(r){2-4}
%             & Continuous Wavelet Transform & \makecell[c]{$F(\tau, s) = \frac{1}{\sqrt{\vert s \vert}} \int_{-\infty}^{\infty} f(t) \psi^* \left( \frac{t - \tau}{s} \right) dt$} & - \\ 
%         \midrule
%         \makecell[c]{Laplace \\ Transform}  & \makecell[c]{Unilateral Laplace Transform~\upcite{ambhika2024time}} & \makecell[c]{$F(s) = \mathcal{L}\{f(t)\} = \int_0^\infty f(t)e^{-st} dt$ } & - \\ 
%         \bottomrule
%         \end{tabular}
%         % \begin{tablenotes}
%         % \scriptsize
%             % \item * where $X[k]$ denotes the discrete frequency spectrum, $k$ represents the discrete frequency index, and $j = \sqrt{-1}$.
%             % \item ** where $X(f)$ is the continuous frequency spectrum, $f$ is the continuous frequency variable. $x(t)$ is the continuous-time signal and $j = \sqrt{-1}$ is the imaginary unit.
%         % \end{tablenotes}
%     \end{threeparttable}}
% \end{table*}



% Predicting long-term trends in time series data (e.g., energy consumption, weather patterns, traffic flow) continues to be a difficult problem. Because the frequency domain is helpful for capturing long-term patterns in time series, frequency domain transformations have been widely used in time series prediction recently, as shown in Figure~\ref{figure-time-series}. One category of methods combines frequency domain with Convolutional models~\cite{krizhevsky2012imagenet}, while another uses Transformer-based models~\cite{vaswani2017attention} as a basis for combination.

% \begin{figure}[ht]
%     \centering
%     \setlength{\belowcaptionskip}{-0.5cm}  % Adjust this to reduce spacing
%     \includegraphics[width=0.475\textwidth]{figures/time-series.pdf}
%     \caption{Time Series in the Frequency Domain}
%     \label{figure-time-series}
%     % \vspace{-0.3cm}  % Adjust this to further reduce spacing
% \end{figure}

% \subsection{Convolution-based}
% Convolution-based methods~\cite{yu2024method,eldele2024tslanet,park2021fast,zhou2024fourier,kim2024neural,lange2021fourier,zhang2024frnet,cai2024msgnet} for time series prediction have been a hot topic recent years. A former study~\cite{park2021fast} proposes a novel method called Partial Fourier Transformation (PFT), which offers a precise and efficient approach for calculating a subset of Fourier coefficients. PFT achieves this by employing polynomials to approximate a portion of the twiddle factors (trigonometric constants), effectively lowering computational complexity resulting from the multitude of these factors. PFT analyzes the asymptotic time complexity of PFT concerning input and output dimensions, as well as tolerance levels. Furthermore, PFT demonstrates that PFT allows users to define a specific approximation error threshold, offering flexibility crucial for scenarios where rapid evaluation is a top priority. Another former study~\cite{lange2021fourier} presents an algorithm that shares similarities with the Fourier transform but operates without relying on assumptions of periodicity. This feature enables forecasting even with irregular sampling intervals. Extending this approach to nonlinear signals involves incorporating Koopman theory. The resultant algorithm conducts a spectral decomposition within a nonlinear, data-driven framework. Despite the highly non-convex nature of the optimization objective in both cases, transforming the objective into the frequency domain facilitates the computation of global optima for the error surface efficiently and at scale, leveraging the computational efficiency of the Fast Fourier Transform. These methods, closely linked to Bayesian Spectral Analysis, naturally yield metrics for quantifying uncertainty in spectral forecasting processes. By contrast, recent studies tend to adopt adaptive frequency methods to solve higher and lower frequencies of time series.
% A recent work~\cite{eldele2024tslanet} proposes a method Tslanet, incorporating an adaptive spectral block via \emph{Fourier transform}, employing Fourier analysis for improved feature representation and the detection of both short-term and long-term dependencies. Noise reduction is achieved through adaptive thresholding. 

% % Furthermore, an interactive convolutional block, trained with self-supervised learning, enhances the model's ability to interpret complex temporal patterns and improves its generalizability across diverse datasets.

% \subsection{Transformer-based}
% Transformer-based methods~\cite{ma2023long,zhou2024fourier,ni2024time,chen2023lightweight,tran2023fourier,kang2023electric,yang2024fedaf}
% for time series prediction aim to provide better performance in the long-term prediction. To capture global-view dependencies of time series, Zhou \textit{et al.}~\cite{zhou2022fedformer} propose a method, namely FEDformer, to decompose Transformer with \emph{Fourier Transform} to compact representations of long-term time series patterns into frequency domain. To be specific, FEDformer integrates the Transformer model with the seasonal-trend decomposition technique, where this method grasps the overall pattern of time series data while Transformers delve into finer intricacies. To boost the Transformer's efficacy in long-range forecasting, FEDformer leverages the observation that many time series can be efficiently represented in common bases like the Fourier transform, leading to the creation of a frequency-enriched transformer. Meanwhile, via \emph{Fourier Transform}, FEDformer aims to capture pieces of information lost in the temporal domain. Another work~\cite{sasal2022w} introduces an innovative approach to learning representations of univariate time series, named W-Transformer, which is built upon a transformer encoder structure utilizing wavelets. The W-Transformers apply a maximal overlap discrete wavelet transformation to the time series information. Meanwhile, local transformers are adopted to effectively capture the nonstationary nature and intricate long-term nonlinear relationships within the time series data. Diferent from other studies, Jin \textit{et al.}~\cite{jin2022time} developed a novel approach for generating token sequences tailored for 1D data, namely TST, a fusion of the time series tokenizer and Transformer architecture. More specifically, TST introduces a way to generate token sequences from one-dimensional data, including time series data. This time series tokenizer is then integrated into a Transformer architecture. In this way, good performance is achieved.


% To capture temporal-spectral correlations effectively~\cite{yang2024graformer,zhang2023self,wang2024card}, to be specific, Zhang \textit{at al.}~\cite{zhang2023self} propose Cross Reconstruction Transformer (CRT). CRT facilitates time series representation learning by employing a cross-domain dropping-reconstruction task via extracting the frequency domain of the time series using the fast \emph{Fourier Transform} and randomly eliminating specific patches in both the time and frequency domains. Woo \textit{et al.}~\cite{woo2022etsformer} proposed ETSFormer, a fresh Transformer architecture tailored for time-series data. This model leverages the concept of exponential smoothing to enhance Transformers for time-series forecasting. Drawing inspiration from classical exponential smoothing techniques in time-series prediction, ETSFormer introduces the innovative concepts of Exponential Smoothing Attention and Frequency Attention. 
% % These mechanisms replace the conventional self-attention module in standard Transformers, enhancing the accuracy and efficiency of the model.

