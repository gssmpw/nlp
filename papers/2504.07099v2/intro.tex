\section{Introduction}
% The field of time series has witnessed remarkable advancements, driven by the development of sophisticated algorithms and the availability of massive datasets. 
% However, most existing methods represent the data in the vanilla temporal domain. 
% In fact, in traditional signal processing, signals are analyzed in the time domain, and their frequency components are obtained through transformations such as the Fourier, wavelet, and Laplace transform~\cite{wu2023timesnet}.
% % Frequency domain analysis includes Fourier, wavelet, and Laplace signals~\cite{wu2023timesnet}, which is widely used on sequential data representations~\cite{wilson2017data}.
% To be specific, frequency transform has been applied in various time-series tasks like forecasting, anomaly detection, and classification to effectively capture periodic patterns and uncover hidden frequency-domain features.


Traditional approaches for time series analysis have predominantly focused on representing data in the temporal or spatial domains, leveraging techniques such as auto-regressive models~\cite{kaur2023autoregressive}, moving averages~\cite{hansun2013new}, and recurrent neural networks~\cite{che2018recurrent} to capture temporal dependencies. 
These methods have proven effective in tasks like forecasting, anomaly detection, and pattern recognition. 
However, they often struggle with complex temporal structures, noise, and high-dimensional data, limiting their ability to fully exploit the underlying information. 

In traditional signal processing, signals are analyzed in the time domain, meanwhile, their frequency components are obtained through transformations such as the Fourier, wavelet, and Laplace transform~\cite{wu2023timesnet}.
As the field progressed, it becomes evident that relying solely on the temporal domain could hinder the extraction of deeper insights, particularly in scenarios requiring robust feature separation and noise reduction. This realization paves the way for exploring alternative representations, particularly in the frequency domain, which promises to address these limitations and unlock new potential for time series analysis.


Frequency domain methodologies have emerged as a cornerstone in modern time series analysis, offering transformative advantages that address critical challenges inherent in traditional temporal and spatial representations. By leveraging frequency-based representations, these techniques significantly enhance feature separability, enabling models to disentangle complex patterns by isolating low-frequency components (e.g., contours) from high-frequency details (e.g., edges), thereby capturing intricate structural nuances with remarkable precision~\cite{he2023idsn}. This capability is particularly vital in scenarios where subtle yet meaningful features are embedded within noisy or high-dimensional data. Furthermore, frequency domain techniques excels in noise reduction, as demonstrated by frequency-domain filtering approaches~\cite{souden2009optimal}, which effectively suppress interference while preserving essential characteristics, thereby enhancing model robustness in noisy environments. Another pivotal advantage lies in their ability to facilitate dimensionality reduction. For instance, wavelet transforms condense information into a compact set of coefficients~\cite{maji2018reduction}, drastically reducing computational overhead during both training and inference phases. This efficiency is invaluable for real-time applications or resource-constrained settings, where scalability and speed are paramount. Collectively, these transformative benefits underscore the indispensability of frequency domain techniques in advancing time series analysis, driving innovations in model performance, and unlocking novel insights for data representation and interpretation.


\begin{figure*}[bht]
    \centering
    \setlength{\abovecaptionskip}{0.5mm}  % Adjust this to reduce spacing
    \includegraphics[width=0.99\textwidth]{figures/intro-v3.0.pdf}
    \caption{Overview of how frequency transform acts in the time series analysis framework.}
    \label{fig:intro}
    \vspace{-0.3cm}
\end{figure*}



In recent years, the field of time series analysis has witnessed unprecedented growth, driven by advancements in algorithms, computational power, and the availability of large-scale datasets. However, amidst this rapid evolution, there remains a critical gap: the absence of a comprehensive and up-to-date survey that systematically reviews the applications, advancements, and challenges of frequency transforms in time series research. Such a survey is not merely a convenience but a necessity, as it would provide researchers with a consolidated understanding of the progress made in leveraging frequency domain techniques, such as Fourier, wavelet, and Laplace transforms, across diverse domains. These transforms have proven indispensable in addressing fundamental challenges in time series analysis, including feature extraction, dimensionality reduction, signal denoising, and model building. For instance, Fourier transforms have been instrumental in analyzing periodic patterns, while wavelet transforms excel in capturing localized temporal and frequency variations~\cite{retter2016uncovering}. Similarly, Laplace transforms have found applications in modeling spatio-temporal dynamics~\cite{keil2022recommendations}. Despite their widespread use, the strengths and limitations of these techniques remain underexplored in a unified framework. This survey seeks to fill this void by examining recent developments over the past three years, offering insights into how these transforms have been applied to time series. By highlighting their transformative potential and addressing their constraints, this survey aims to guide researchers in selecting appropriate methodologies, inspire innovative applications, and foster a deeper understanding of the role of frequency domain techniques in advancing time series analysis. In doing so, it will serve as a foundational resource for both newcomers and seasoned practitioners, bridging the gap between theoretical advancements and practical implementations.



To summarize, the contributions of this survey are multifaceted and address critical gaps in the current literature. 

\noindent\textbf{(1)} We are the first to provide a dedicated survey that systematically reviews and synthesizes studies leveraging frequency domain techniques, filling a long-standing void in the field.

\noindent\textbf{(2)} Our work offers a comprehensive exploration of methodologies rooted in \textit{Fourier, wavelet, and Laplace transforms}, highlighting their applications, strengths, and limitations in machine learning for time series analysis. 

\noindent\textbf{(3)} We present an up-to-date pipeline that captures the latest advancements in time series dynamics research over the past three years, ensuring relevance to contemporary challenges and innovations. 

Additionally, to foster reproducibility and further research, we provide a curated GitHub repository accessible via \url{https://github.com/lizzyhku/Awesome_Frequency_Transform}, which serves as a valuable resource for researchers and practitioners alike. Collectively, these contributions aim to guide future research, inspire novel applications, and establish a foundational reference for different kinds of frequency transforms in time series analysis.


The paper is structured as follows: Section~\ref{sec:prob} defines the frequency transform problem. Section~\ref{sec:pre} illustrates various frequency transforms and their specifics. Section~\ref{sec:libra} introduces the frequency transform library, including datasets, models, and code. Section~\ref{sec:appli} explores applications in time series. Section~\ref{sec:chall} discusses challenges in frequency-domain learning. Section~\ref{sec:dis} outlines key discussions. Section~\ref{sec:future} highlights future directions. Finally, Section~\ref{sec:conclu} concludes with key takeaways.
% The remainder of this paper unfolds as follows: Section~\ref{sec:prob} delineates the problem definition of the frequency transform. Section~\ref{sec:pre} presents illustrations of different kinds of frequency transform and their specifics. Section~\ref{sec:libra} shows the library for frequency transform, including benchmark datasets, model structures, and code. Section~\ref{sec:appli} shows frequency transform applications in time series. Section~\ref{sec:chall} shows challenges and open problems in learning in the frequency domain. Section~\ref{sec:dis} outlines discussions on frequency transform. Section~\ref{sec:future} shows potential future directions. Lastly, Section~\ref{sec:conclu} encapsulates the conclusion, highlighting key takeaways.








































