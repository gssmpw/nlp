\section{Libraries for Frequency Transform}
\label{sec:libra}

\paragraph{Benchmark Datasets.} Table~\ref{tab:benchmark_dataset} provides the statistics and feature details of commonly used datasets for time series analysis. These datasets cover applications ranging from energy consumption and meteorological indicators to healthcare and anomaly detection, offering a comprehensive foundation for research and model benchmarking.

% Table~\ref{tab:benchmark_dataset} provides the statistics and feature details of commonly used datasets for time series analysis:
% \begin{itemize}[leftmargin=0pt]
%     \item Electricity Transformer Temperature (ETT)~\citep{zhou2021informer} contains four sub-dataset: ETTm1\&m2 and ETTh1\&h2, collected from two electricity transformers at two stations in different resolutions (15min or 1h). ETT dataset contains multiple series of loads and one series of oil temperatures;
%     \item Weather dataset contains 21 meteorological indicators for a range of 1 year in Germany;
%     \item Electricity (ECL) dataset contains the electricity consumption of clients with each column corresponding to one client;
%     \item Traffic dataset contains the occupation rate of freeway system across the State of California;
%     \item Exchange dataset~\citep{lai2018modeling} contains the current exchange of 8 countries;
%     \item Illness (ILI) dataset contains the influenza-like illness patients in the United States.
%     \item UEA time-series dataset is curated by the University of East Anglia (UEA) for various research tasks, including five collections for series-level anomaly detection.
%     \item Server Machine Dataset (SMD) is a 5-week dataset from 28 machines in three groups, designed for anomaly detection with labeled points and contributing dimensions, trained/tested separately.
% \end{itemize}



\begin{table}[!ht]
    \setlength{\tabcolsep}{10pt}
    \centering
    \caption{A list of commonly used and publicly accessible datasets.}
    \label{tab:benchmark_dataset}
    \vspace{-0.3cm}
    \setlength{\tabcolsep}{6pt}
    \resizebox{0.49\textwidth}{!}{      % 限制表格宽度为页面宽度
        \fontsize{12}{1.25\baselineskip}\selectfont % 设置字体大小为12pt, 1.25倍行距
        \begin{threeparttable}
        \begin{tabular}{lcccc}
        \toprule
        \textbf{Datasets} & \textbf{Length}  & \textbf{Dimension}  & \textbf{Frenqucy} & \textbf{Task} \\
        \midrule
        ETTm1\&m2$^{[1]}$ & 69680 & 8 & 15 mins  & \makecell[c]{Forecasting \\and Imputation} \\
    
        \midrule
        ETTh1\&h2$^{[1]}$ & 17420 & 8 & 1h  & \makecell[c]{Forecasting \\and Imputation} \\
    
        \midrule
        Weather$^{[2]}$ & 52696 & 22 & 10 mins & \makecell[c]{Forecasting \\and Imputation} \\
    
        \midrule
        Electricity$^{[3]}$ & 26304 & 322 & 1h & \makecell[c]{Forecasting \\and Imputation} \\
    
        \midrule
        Traffic$^{[4]}$ & 17544 & 863 & 1 h & Forecasting \\
    
        \midrule
        Exchange$^{[5]}$ & 7588 & 9 & 1 day & Forecasting \\
    
        \midrule
        Illness$^{[6]}$ & 966 & 8 & 7 day & Forecasting \\
    
        \midrule
        UEA$^{[7]}$ & 8 \textasciitilde 17984 & 2 \textasciitilde 1345 & - & Classification \\
    
        \midrule
        SMD$^{[8]}$ & 1416825 & 38 & 1 mins & \makecell[c]{Anomaly Detection} \\
        \bottomrule
        \end{tabular}
        
        \begin{tablenotes}
            \footnotesize
            \item[] [1] ETT dataset: \url{https://shorturl.at/3rJse}. [2] Weather dataset: \url{https://shorturl.at/dUSD9}.
            \item[] [3] Electricity dataset: \url{https://shorturl.at/asn1o}. [4] Traffic dataset: \url{https://shorturl.at/L83ZN}.
            \item[] [5] Exchange dataset: \url{https://shorturl.at/5Dia3}. [6] Illness dataset: \url{https://shorturl.at/vafkb}.
            \item[] [7] UED dataset: \url{https://shorturl.at/wJUSs}. [8] SMD dataset: \url{https://shorturl.at/QcNcW}.
        \end{tablenotes}
    \end{threeparttable}}
\end{table}

\paragraph{Model Structures.} Frequency transform methods are essential for processing time-series data, enabling efficient transformations and feature extraction. Table~\ref{tab:frequency_transform_methods} summarizes the representative frequency transform methods used in our framework, highlighting their applications in time-series analysis. These methods offer distinct benefits but also present challenges. Understanding their strengths and limitations provides insights for selecting the most suitable approach for specific tasks and guiding future research.

\paragraph{Code.} To facilitate access to empirical analysis, we summarize the open-source codes of representative frequency transform methods for time series in Table~\ref{tab:representative_mdthod}. Additionally, we list the applied tasks and corresponding benchmark datasets for each method. Due to space limitations, a more comprehensive summary is available in our GitHub repository at \url{https://github.com/lizzyhku/Awesome_Frequency_Transform}. Furthermore, we will update the repository in real-time as new methods and implementations become available.







































