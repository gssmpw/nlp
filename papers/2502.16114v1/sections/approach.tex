\section{\tool}

\autoref{sec:relationship_space} introduces relationship spaces, outlining the types of relationships based on content--text, \yanna{code}, and outputs--and granularity, from whole cells to segments.
\autoref{sec:relationship_formulation} then  describes how to formulate the relationships. 
Finally, \autoref{sec:system_design} details the design of \tool, starting with an overview of the system, followed by the layout computation, relationship visualization, and interaction.


\begin{figure}[t!]
    \centering
    \includegraphics[width=\linewidth,alt={Diagram illustrating the relationship space of a tool, with circles representing segments and rectangles representing entire cells. The top part shows inter-category relationships among code, text, and output, connected by solid and dashed lines with 8 key relationships highlighted in red to represent the primary focus of the tool. The bottom part demonstrates intra-category relationships within the same content type, depicted with dashed black lines. A legend on the right details the symbols and line styles used to indicate cell-cell, cell-segment, and segment-segment relationships.}]{figures/relationship_space_new.png}
    \caption{The relationship space in computational notebooks, categorized by content types (\ie~text, code, and output) and granularity (entire cell and segments).
    It identifies 27 distinct relationship types: 12 stemming from inter-category and 15 from intra-category relationships.
    Specifically, 8 inter-category relationship types highlighted in \textcolor[rgb]{0.796,0.255,0.329}{red} are the primary focus of \tool.}
    \Description{Diagram illustrating the relationship space of a tool, with circles representing segments and rectangles representing entire cells. The top part shows inter-category relationships among code, text, and output, connected by solid and dashed lines with 8 key relationships highlighted in red to represent the primary focus of the tool. The bottom part demonstrates intra-category relationships within the same content type, depicted with dashed black lines. A legend on the right details the symbols and line styles used to indicate cell-cell, cell-segment, and segment-segment relationships.}
    \label{fig:relationship}
    \vspace{-1em}
\end{figure}



\subsection{Relationship Space}
\label{sec:relationship_space}

Considering both the content categories (including text, code, and output) and the granularity of the cells (including the whole cell and \final{segments} of cells), we build a complete relationship space between all of them, as shown in~\autoref{fig:relationship}.
Regarding content, relationships within notebooks can be categorized into both intra-category and inter-category relationships: 1) Intra-category relationships occur within the same category of notebook content (\ie~text-text, code-code, and output-output relationships);
and 2) Inter-category relationships span different categories of content (\ie~text-code, text-output, and code-output relationships).
Additionally, we defined the granularity of these relationships, identifying cell-cell relationships that exist between entire cells, and further nuanced relationships including cell-segment and segment-segment relationships. 
Through this classification, we have identified a total of 27 unique relationship types—12 stemming from inter-category and 15 from intra-category relationships.
There are more intra-category relationships because there are more segment-segment relationships within the same cell category,
\yanna{\eg~multiple code segments within a single code cell contributing to a shared function.}


Feedback from participants in the formative study mainly emphasized challenges associated with text-code and text-output relationship\final{s}.
Thus, in this paper, our focus narrows to these two kinds of relationships, which include eight relationship types (marked in red in \autoref{fig:relationship}). 




\subsection{Relationship Formulation}
\label{sec:relationship_formulation}

\yanna{ 
We propose a formal specification for a relationship $r$ between text, code, and outputs, as illustrated in~\autoref{fig:rm_json} (A).
Each relationship consists of two components, each defined by several attributes. 
The \textit{CellId} and \textit{CellType} identify the ID and type of the cell involved, while \textit{GranularityType} specifies whether the relationship applies to the entire cell or specific segments. 
For text or code content, \textit{SpanPos} precisely locates the relevant segment by specifying its starting index and length.
For output annotations, \textit{Sketch} accommodates the multi-modal nature of outputs (\eg~images, text, and tables) by supporting rectangular bounding boxes or freehand sketches recorded as paths. 
Additionally, \textit{ViewSize} captures the SVG dimensions of the annotations to ensure precise positioning across varying screen sizes.
To provide an overview of relationships between notebook cells without specific details, we further introduce an aggregated relationship $r'$ derived from $r$, represented by the IDs of the two involved cells.
The relationships in a notebook are collectively represented as $R = [r_1, r_2, \dots]$ and the corresponding aggregated relationships as  $R' = [r'_1, r'_2, \dots]$.

\autoref{fig:rm_json}~(B) illustrates an example of how relationships are represented. (a) shows four examples of how the components in relationships are represented according to the defined formulation.
These components form a relationship set $R$ with two distinct relationships, linking two text segments to one code cell and one output segment, respectively (b). 
The corresponding aggregated relationship set $R'$ is shown in (c).




}




\begin{figure}[t!]
    \centering
    \includegraphics[width=\linewidth, alt={The figure is a screenshot of the InterLink interface. On the left side, there is a series of text cells, with one labeled ``Observations'' providing a list of insights about Spotify's song data. On the right side, the code and output cells are displayed. The code cells include Python code, utilizing the seaborn library to create visualizations of the data. The output cell shows a line plot titled ``Count of Tracks Added'', displaying the number of tracks added over time. In the middle, connecting lines visually link the corresponding text cells with their related code or output cells, highlighting the relationships between explanations and their data-driven visualizations.}]{figures/interlink_interface.png}
    \caption{\revise{Screenshot of the \tool interface, with text cells on the left, code and output cells on the right, and lines in the middle denoting aggregated relationships between them.}}
    \Description{The figure is a screenshot of the InterLink interface. On the left side, there is a series of text cells, with one labeled ``Observations'' providing a list of insights about Spotify's song data. On the right side, the code and output cells are displayed. The code cells include Python code, utilizing the seaborn library to create visualizations of the data. The output cell shows a line plot titled ``Count of Tracks Added'', displaying the number of tracks added over time. In the middle, connecting lines visually link the corresponding text cells with their related code or output cells, highlighting the relationships between explanations and their data-driven visualizations.}
\label{fig:interlink_interface}
\end{figure}



\begin{figure*}[t!]
    \centering    \includegraphics[width=\textwidth,alt={xx}]{figures/rm_json.png}
    \caption{\yanna{This figure shows the relationship formulation and its example.
    (A) describes the formulation of a relationship and the corresponding aggregated relationship. 
    (B) provides an example of how relationships are represented.  (a) shows four examples of how the components in a relationship are represented according to the defined formulation. 
    (b) shows a relationship set $R$, consisting of two relationships formed from these four components. (c) shows the corresponding aggregated relationship set $R'$. }}
    \Description{This figure illustrates the formulation and representation of relationships in computational notebooks, divided into two panels: (A) Relationship Formulation and (B) Relationship Example. Panel (A) defines the structure of a relationship, which consists of a source and target, both represented as ContentAttributes. These attributes include the CellId, a unique identifier for the cell; the CellType, which specifies whether the cell is a "text," "code," or "output" cell; the GranularityType, which indicates whether the relationship applies to the entire cell or a specific segment; and the SpanPos, which identifies the start index and length for locating text or code content, or is marked as "none" if not applicable. For output annotations, the Sketch attribute accommodates either bounding boxes defined as [X, Y, Width, Height, Angle] or freehand sketches recorded as paths, with "none" used if no sketch is provided. The ViewSize attribute records the dimensions of the SVG associated with the annotation as [ViewWidth, ViewHeight], or is marked as "none" if not applicable. Below the main formulation, the aggregated relationship, AggRelationship, is defined as a higher-level summary that uses only the CellId of the two connected cells. Panel (B) provides an example of how these relationships are instantiated and aggregated in a notebook.  Subfigure (a) of (B) displays four examples of how the components in a relationship are represented according to the defined formulation.  Subfigure (b) shows how these four components form two distinct relationships, collectively constituting a relationship set $R$. The first relationship links two segments of a markdown cell, identified as m6, to a code cell, identified as c19. The second relationship links the same markdown cell to an output cell, identified as o19. For each component, attributes such as CellId, CellType, GranularityType, and SpanPos are specified. Subfigure (c) illustrates the corresponding aggregated relationship set $R'$, which simplifies the representation by focusing on the CellId of the two connected cells. In this example, $R'$ consists of two pairs:{"m6", "c19"} and {"m6", "o19"}, summarizing the high-level connections between the notebook cell.}
    \label{fig:rm_json}
    \vspace{-1em}
\end{figure*}



\subsection{System Design}
\label{sec:system_design}
In this section, we first present an overview of \tool.
Then we introduce the details of the layout computation, relationship visualization, and supported user interactions.

\subsubsection{System Overview}
\label{sec:overview}
We developed \tool, a plugin for JupyterLab. \tool equips readers with visualizations and interactive features that facilitate \final{identifying and navigating} relationships between text with \yanna{code} and outputs (\textbf{DR5}), thereby making it easier to \final{understand} notebooks \revise{(\autoref{fig:interlink_interface}).}

When readers launch \tool in a notebook by clicking a button in the toolbar (\autoref{fig: teaser}~(b1)), the notebook is shown in two side-by-side columns: the left column dedicated to the text cells, and the right column for code and output cells (\textbf{DR3, DR4}),
% as shown in ~\autoref{fig: teaser}~(A) and~\autoref{fig:interlink_interface}.
\revise{as illustrated in~\autoref{fig: teaser}~(B).
\autoref{fig:interlink_interface} provides an actual screenshot of the interface.}
Between the two distinct columns, there are lines connecting text cells with their related code or output cells (\textbf{DR1}), delineating the presence of relationships at a cell level without specifying the details within~(\autoref{fig: teaser}~(b3)).
This design choice emphasizes connections between whole cells without the granularity of segment-specific relationships, aiming to reduce visual clutter (\textbf{DR1}).
To detail the nuanced, multi-granular information underpinning these relationships, individual cells have visual cues (\textbf{DR1}).
Specifically, color-coded underlines within text and code cells draw attention to specific text segments involved in a relationship (\autoref{fig: teaser}~(b5)), while dashed sketches over the output cell emphasize particular segments of interest (\autoref{fig: teaser}~(b7)). 
Additionally, a dashed border surrounding an entire cell signals that the relationship encompasses the whole cell (\autoref{fig: teaser}~(b6)). 


Exploration of the relationships is facilitated through interaction (\textbf{DR2}). 
Key interactions include hovering over elements to immediately highlight related content, which simultaneously invokes tooltips for rapid insights (\autoref{fig:focus_mode} (a1-a5)). 
A ``Shift'' key-triggered focus mode allows for an in-depth examination of relationships, effectively minimizing distractions from unrelated information (\autoref{fig:focus_mode}). 
Additionally, readers can click to fix the position of selected cells, anchoring their focus for detailed analysis (\autoref{fig: teaser}~(b2)). 



% Hovering over any visual cues or cells triggers immediate highlighting of related content for straightforward identification.
% This interaction further prompts a tooltip that presents relevant content for a quick grasp.
% For user aiming for a more structured overview of related content, \tool enables to transform the tooltip into a ``sticky note'' for a focused mode to streamline the focus by exclusively displaying cells related to the hovered cell, diminishing the efforts of scroll to related information. 
% This mode aids users in concentrating on relevant relationships without the distraction of extraneous information, offering a cleaner, more targeted exploration within the notebook.
% Additionally, \tool allows users to click to lock the position of a selected cell and maintain the highlight effect, further facilitating focused analysis.


% All these links and visual cues support interaction to explore them (\textbf{DR2}).
% Links allows users to grasp the broad relationships at a glance, offering a high-level overview that simplifies the initial comprehension. 
% Users are empowered to delve deeper into these relationships using interactions, uncovering the specifics as needed (\textbf{DR3}).
% For example, hovering over a visual cue of related information highlighted connected information for easy identification in the other column, and also triggers a tooltip that shows only the related content for a quick grasp.
% Users seeking a consolidated view of related information in toolttip can trigger a "sticky note" view in the other column, effectively gathering and presenting related snippets in an accessible format. 
% Moreover, \tool offers a split-view mode that filters and displays only the content related to selected elements, enabling users to delve into specific relationships without the clutter of unrelated information.







% It allows users to grasp the broad relationships at a glance, offering a high-level overview that simplifies the initial comprehension. 
% Subsequently, through interactive elements, users are empowered to delve deeper into these relationships, uncovering the specifics as needed (\textbf{DR3}).




%%%%% ============ New 4.3.2 ============
\subsubsection{Layout Computation} \label{sec:layout_computation}
% \textcolor{red}{The writing was generally clear though several sections of the text felt long relative to their contribution. For example, the Layout Computation in section 4.3.2. is described in far more detail than needed as this aspect of the system is not a primary contribution of the paper.
% Sections 4.1 and 4.2 similarly might be condensed or removed as these are not key contributions of the manuscript (R2);
% (iii) the layout algorithm could be described both more concisely and more clearly.(2AC)}

% \textcolor{red}{I don’t understand how O3 is different from O1, in that aren’t both about preserving the order of cells? (edit: on second look, is O3 specifically about staggering the presentation of text and code/outputs when there is no relationship to be found to a text cell?) }

% \textcolor{red}{in general, I”m not sure what “zigzag” means with regards to staggering text and code/output blocks ;
% }
% \textcolor{red}{if a text cell is cut off and some of the text content that is cut off contains a link to a code/output cell, how can a user get access to that content? }


We use \final{a} \yanna{heuristic-based approach guided by} a set of objectives to strategically arrange the cells in a side-by-side layout. In this layout, text cells are on the left, and code and output cells are on the right, as shown in~\autoref{fig:interlink_interface}.
% This design was informed by feedback from our formative study and previous research findings.
% Participants reported that the traditional single-column layout often disrupted their focus and necessitated frequent scrolling to track related information across different notebook elements (text, code, and outputs). 
% This observation aligns with findings from~\cite{zhi2019linking}, which demonstrated that readers engaged in more text-visualization transition actions in vertical layouts compared to side-by-side slideshow layouts.
% Zhi~\etal also suggested that arranging descriptive texts and visualizations side by side significantly improves comprehension over traditional vertical arrangements~\cite{zhi2019linking}. 
Next, we explain how \tool handles the positions and sizes of cells in the layout.
In terms of positions, \tool retains the primary order of code and output cells. 
For text cells, \tool repositions them to \yanna{follow the original order and place them as close as possible to the first related code or output cell.}
This is informed by~\cite{wang2022documentation}, which suggests that the positions of text cells relative to related code cells are more flexible (\eg~above, below, or adjacent).
In terms of sizes, \tool limits the maximum height of text cells while maintaining the original size of code and output cells.
This decision is informed by the observation that the content length of text cells can be approximately ten times that of code cells~\cite{wang2022documentation}, causing significant height discrepancies. 
In a side-by-side layout with reduced cell width\final{s}, these differences are magnified, leading to misalignment and excessive scrolling similar to single-column layouts.
\yanna{If the content of a text cell exceeds this maximum height, users can access the overflow through manual scrolling or auto-scroll triggered by hovering over its related information.}


    % \item \yanna{O2: Position text cells according to their relationships with code and outputs.
    % If a text cell is related to code or output cells, it should be placed close to the first related cell for easy reference. If there is no direct relationship, the text cell should follow the original sequence in the notebook to preserve the implicit relationships and logical flow.
    % }

Specifically, the following objectives are applied:
\begin{itemize}
    \item[O1] Maintain the original textual narrative order and computational order.
    \item[O2] Be close to the first related code or output cells. This close arrangement aims to enable users to quickly associate textual explanations with relevant computational content and facilitate easy reference. 
    \yanna{\item[O3] Follow the original sequence in the notebook if the text cell is not related to any code or output cells.}
    \item[O4] Adjust the height of text cells according to the cumulative height of \final{the} related cells.  \yanna{\tool assumes that the importance of a text cell is positively correlated with the amount of its related computational content~\cite{lin2023dashboard}. \tool sets the maximum height of a text cell to the cumulative height of its associated code or output cells.}
    \item[O5] Adjust the height of text cells to improve space utilization. 
    \yanna{If a text cell is unrelated to any code or output cells and is followed by a code cell without accompanying text (leaving space to the left of the code cell), its height matches that of the code cell. 
    Otherwise, \tool adds space within the code sequence to fit the text cell with a default height, sacrificing some space but ensuring alignment between text and associated code and output.}
    % In this case, $m_i$ mirrors the height of the subsequent code cell to maintain a zigzag layout and preserve the original order. 
    \item[O6] Ensure that the height of text cells does not exceed the height required to display their content. 
\end{itemize}

The highest priority is given to objective O1, with the others being of equal priority. A detailed pseudocode can be found in the supplementary material.


\autoref{fig:layout_computation} demonstrates the implementation results.
In terms of positions, the textual narrative order on the left and the computational order on the right follow\final{s} the original order strictly (meeting O1).
Notably,  $m1$ is placed near $c1$ to align with O2 for easy referencing.
Meanwhile, $m2$ also tries to approach $c1$ but is placed below $m1$ to maintain the original order, prioritizing order over proximity (O1 over O2).
Moreover, the position\final{s} of $c2, m3, c3$ and $m4$ maintain the original sequence (satisfying O3).
In terms of sizes, the height of $m_1$ equals the sum of the heights of $c_1$ and $o_1$ (meeting O4), while the height of $m_3$ mirrors that of $c_3$ to maintain the original sequence without introducing additional spaces (meeting O4 and O6). Moreover, $m_4$ is set to a default height, and additional spaces are introduced in the computational sequence since the next code cell $c_4$ has relationships with $m_5$ (meeting O5 and O6). 


%%%%% ============ Original 4.3.2 ============
% \subsubsection{Layout Computation}
% \textcolor{red}{The writing was generally clear though several sections of the text felt long relative to their contribution. For example, the Layout Computation in section 4.3.2. is described in far more detail than needed as this aspect of the system is not a primary contribution of the paper.
% Sections 4.1 and 4.2 similarly might be condensed or removed as these are not key contributions of the manuscript (R2);
% (iii) the layout algorithm could be described both more concisely and more clearly.(2AC)}
                
% As shown in~\autoref{fig:interlink_interface}, \tool adopts a side-by-side design to separate notebooks into two columns.
% This design was informed by feedback from our formative study and previous research findings.
% Participants reported that the traditional single-column layout often disrupted their focus and necessitated frequent scrolling to track related information across different notebook elements (text, code, and outputs). 
% This observation aligns with findings from~\cite{zhi2019linking}, which demonstrated that readers engaged in more text-visualization transition actions in vertical layouts compared to side-by-side slideshow layouts.
% Zhi~\etal also suggested that arranging descriptive texts and visualizations side by side significantly improves comprehension over traditional vertical arrangements~\cite{zhi2019linking}. 


% \revise{
% In this side-by-side layout, cell width is fixed and depends on screen space.
% Thus, the arrangement for each cell $e$ in the notebook can be described by:
% \begin{compactitem}
%     \item $y_s(e)$, the vertical starting position of $e$ on the display;
%     \item $h_a(e)$, the assigned height of $e$ on the display;
%     \item $h_c(e)$, the content height of $e$; and
%     \item $y_e(e) = y_s(e) + h_a(e)$, the vertical ending position of $e$ on the display.
% \end{compactitem}
% }


% We use heuristics to strategically arrange the cells in the side-by-side layout.
% Considering the more flexible position of text cells relative to associated code and output cells (\eg~above, below, or adjacent)~\cite{wang2022documentation}, \tool retains the primary order of code and output cells while adjusting the position and size of the text cells accordingly.
% Specifically, we mainly reposition and resize the text cells while making small adjustments to code and output cells as necessary.

% To reposition the text cells in the \tool, we apply objectives that respect the original order and the relationships between text cells and associated code and output cells:
% \textcolor{red}{I don’t understand how O3 is different from O1, in that aren’t both about preserving the order of cells? (edit: on second look, is O3 specifically about staggering the presentation of text and code/outputs when there is no relationship to be found to a text cell?) }
% \begin{compactitem}
%     \item O1: Maintain the original textual narrative order and computational order.
%     \item O2: Be close to the first related code or output cells. This close arrangement aims to enable users to quickly associate textual explanations with relevant computational content and facilitate easy reference.
%     \item O3: Keep original order for unrelated cells. If a text cell $m_i$ does not relate to any code or output cells (\ie~$\nexists e : (m_i, e) \in R'$), it maintains its original sequence, adopting a zigzag-like layout that follows preceding computational elements $c/o_j$. This arrangement ensures preservation of order and logical progression.
% \end{compactitem}

% Priority is given to these objectives in the following order: O1 is first, followed by O2 and  O3.
% According to these objectives, the position of a text cell $m_i$ is defined as:
% \begin{equation}
% \begin{aligned}
% y_s(m_i) &= \begin{cases} 
% \max \left( \min \left( \{ y_s(e) \mid (m_i, e) \in R' \} \right), y_e(m_{i-1}) \right), & \text{if } \exists e : (m_i, e) \in R' \\
% \max \left( y_e(c/o_j), y_e(m_{i-1}) \right), & \text{if } \nexists e : (m_i, e) \in R'
% \end{cases}
% \end{aligned}
% \end{equation}
% where $c/o_j$ represents preceding computational elements relative to $m_i$ in the computational notebook.
% As shown in~\autoref{fig:layout_computation}, both the textual narrative order on the left and the computational order on the right follow the original order strictly (meeting O1).
% Notably, the position of $m1$ is close to $c1$, aligning with O2 to facilitate easy referencing.
% Meanwhile, $m2$ also tries to approach $c1$ but is placed below $m1$ to maintain the original order, prioritizing order over proximity (O1 over O2).
% Moreover, the position of $c2, m3, c3$ and $m4$ follow a zigzag layout to maintain the original order (satisfying O3).


% Regarding the size arrangement, we maintain the original size of code and output cells while specifically limiting the maximum height of text cells.
% This decision is informed by the observation that the content length of text cells can be approximately ten times that of code cells~\cite{wang2022documentation}.
% This discrepancy can lead to significant height differences between text cells and associated code or output cells.
% In a side-by-side layout, where the width of cells is reduced compared to a single-column format, such height difference are further magnified.
% The significant difference in cell height can lead to misalignments between text and computational cells, causing drawbacks similar to single-column layouts, such as excessive scrolling required to navigate between related information.
% To address these issues, we propose the following objectives:
% \begin{compactitem}
%     \item O4: Adjust the height of text cells according to the cumulative height of related cells. \tool assumes that the importance of a text cell is positively correlated with the amount of its related computational content. Moreover, following a common dashboard design pattern~\cite{lin2023dashboard}, where the size of a view implies its importance, \tool assigns more space (\ie~a larger height) to  more significant text cell. Specifically, \tool sets the maximum height of a text cell to the cumulative height of its associated code or output cells.
    
%     \item O5: Adjust the height of text cells to improve space utilization while maintaining the original order. For a text cell $m_i$ that does not describe specific \yanna{code} or outputs, if its subsequent cell is a code cell without associated descriptive text (\ie~$\nexists e :  (e, c_j) \text{ in } R'$), then no text cell is positioned on the left of this code cell.
%     In this case, $m_i$ mirrors the height of the subsequent code cell to maintain a zigzag layout and preserve the original order. 
%     Otherwise, the space on the left of subsequent cell is occupied.
%     Then, \tool introduces additional space within the code sequences to accommodate the text cell with a default height.
%     This adjustment sacrifices some layout space within the code and output sequence but ensures that descriptive text remains aligned with its related \yanna{code} or outputs.
    
%     \item O6: Ensure that the height of text cells does not exceed the height required to display their content. 
% \end{compactitem}

% Following O4-O6, the size of the text cell $m_i$ is defined as:
% \begin{equation}
% h_a(m_i) = \begin{cases} 
% \min \left( \sum \left( \{ h_a(e) \mid (m_i, e) \in R' \} \right), h_c(m_i) \right), & \text{if } \exists e : (m_i, e) \in R' \\
% \min \left( h_a(c_j), h_c(m_i) \right) , & \text{if } \nexists e : (m_i, e) \in R' \text{ and }  \nexists e :  (e, c_j) \text{ in } R' \\
% \min \left(  \text{default height}, h_c(m_i) \right), & \text{otherwise}
% \end{cases}
% \end{equation}
% where $c_j$ is the next code cells of $m_i$ in the computational notebook.
% As shown in~\autoref{fig:layout_computation}, the height of $m_1$ equals the sum of the heights of $c_1$ and $o_1$, while the height of $m_3$ mirrors that of $c_3$ to keep zigzag order without introducing additional spaces.
% \textcolor{red}{in general, I”m not sure what “zigzag” means with regards to staggering text and code/output blocks ;
% }
% \textcolor{red}{if a text cell is cut off and some of the text content that is cut off contains a link to a code/output cell, how can a user get access to that content? }
% Moreover, $m_4$ is set to a default height and additional spaces is introduced in the computational sequence since the next code cell $c_4$ has relationships with $m_5$. 
% For detailed pseudocode of the layout calculation, please refer to the supplementary material.


% Considering the order in computational notebook is meaningful, we retain original textual narrative and computational order, respectively (Rule1).
% While for arranging  the text cells, we based on their relationships with \yanna{code} and outputs, i.e., 1) if the text cell has relationships with code or outputs, it should be arranged as close as possible to the first related code or output cells for easy reference (Rule2), and  2) if not, it should follow the original order in computational notebook from left to right, like a zig-zag layout (Rule3).
% Specifically, when there are conflicts, Rule1 is the most important, followed by Rule2, and then Rule3.
% \revise{\autoref{fig:mi_pos} demonstrated four cases about how a specific text cell will be arranged.
% Specifically, as shown in (A), if it has corresponding relationships in $R'$, which indicates that $m_i$ describes specific code and outputs, its position is optimized to be as close as possible to the first related code or output cell, yet below the last text cell $m_{i-1}$ to maintain the order of the narrative.}
% For a specific text cell $m_i$, if it has corresponding relationships in $R'$, which indicates that $m_i$ describes specific code and outputs, its position is optimized to be as close as possible to the first related code or output cell, yet below the last text cell $m_{i-1}$ to maintain the order of the narrative.
% The maximum height allocated for $m_{i}$ equals the cumulative height of its related \yanna{code} and outputs.
% \revise{As shown in (B) nad (C), if the text cell does not describe specific code or outputs, }
% % In cases where text cells do not describe specific code or outputs,
% their placement seeks to maintain the narrative flow.
% They are logically situated between the preceding and following cells.
% As shown in (B), if the subsequent cell is code cell without an associated relationship in $R'$—implying it lacks related descriptive text—the text cell $m_i$ is positioned to the left of this code cell $c_j$, mirroring its height.
% When both preceding and subsequent cells are text cells or code cells accompanying their descriptive texts, \revise{as shown in (C),} indicating that the space on the left is already used, we introduce additional space within the code sequences to accommodate the text cell $m_i$. 
% We sacrifice some layout space in code sequence to ensure that descriptive text remains aligned with its related \yanna{code} or outputs.
% For a detailed pseudocode of the layout calculation, please refer to the supplementary material.





% But also they follows a specific narrative order, thus, the next documentation cell must be below the last documentation cell to keep the narrtive order.
% While for those documentation cells not related to specific code and outputs, the position should be follow the original order, that is below the last cells while above the next cells.
% We tried to follow the zig-zag reading order, which are a common reading style.

% Considering these constraints, we formulate we have the 

% $P_{m_i} = f(P_{m_{i-1}}, P_{code_{j}} P_{related \yanna{code} or outputs})$


% In this case, we can calculate the position (top) of the $m_i$.




\begin{figure}[t!]
    \centering
    \includegraphics[width=\linewidth, alt={The figure is divided into two parts: the top part illustrates the original notebook and its re-layout after calculating the position and size of each text cell, with corresponding objectives, while the bottom part explains these calculations and objectives in detail. In the top part, the left side shows the original arrangement of the text (m1-m5) and code/output cells (c1-c4 and o1). The re-layout section shows how the text cells are repositioned, with additional space added to align text cell m5 with code cell c4, ensuring computational order and visual clarity. In the bottom section, the left side outlines the formulas for calculating the position and height of text cells based on relationships with code/output cells. On the right, two categories of objectives are listed: "Objectives - Position" (O1-O3) focuses on maintaining narrative and computational order, proximity to related cells, and respecting original order for unrelated cells; "Objectives - Size" (O4-O6) aims to adjust the height of text cells for better space utilization while ensuring content is fully displayed.}]{figures/layout_computation.png}
    \caption{\revise{An example of repositioning and resizing text cells while respecting their original order and relationships with associated code and output cells, as described by the six objectives in \autoref{sec:layout_computation}.}}
    \Description{The figure is divided into two parts: the top part illustrates the original notebook and its re-layout after calculating the position and size of each text cell, with corresponding objectives, while the bottom part explains these calculations and objectives in detail. In the top part, the left side shows the original arrangement of the text (m1-m5) and code/output cells (c1-c4 and o1). The re-layout section shows how the text cells are repositioned, with additional space added to align text cell m5 with code cell c4, ensuring computational order and visual clarity. In the bottom section, the left side outlines the formulas for calculating the position and height of text cells based on relationships with code/output cells. On the right, two categories of objectives are listed: "Objectives - Position" (O1-O3) focuses on maintaining narrative and computational order, proximity to related cells, and respecting original order for unrelated cells; "Objectives - Size" (O4-O6) aims to adjust the height of text cells for better space utilization while ensuring content is fully displayed.}
\label{fig:layout_computation}
\end{figure}



\subsubsection{Relationship Visualization}
\label{sec:relationship_visualization}

Besides structuring notebooks side-by-side, \tool shows relationships in $R$ in the UI.

To help readers infer potential relationships within notebooks, \tool adopts visual cues to signal the existence of \yanna{relationships}. \revise{The designs are inspired by~\cite{Srinivasan2019voder}, which summarizes embellishment options for highlighting visualizations to aid interpretation}.
\autoref{fig:visual_cues} shows visual cues, tailored to content type, granularity, and interaction status.
% The visual cues are tailored to the content's type and granularity.
% \revise{Inspired  we chose }
For entire cells, \tool uses dashed borders to signify \final{that} they share a relationship with other information within the notebook.
% \revise{\tool does not use background color since it represents the type of cell in original design, e.g., grey for code.}
For segments, the visual cues become more nuanced.
Within text cells, color-coded underlines distinguish the text types:  blue for code-related text, green for output-related text, and purple for text relating to both \yanna{code} and outputs. 
Code segments within code cells use the same green underlines as the code-related text for emphasis.
Specific areas within outputs are outlined with sketches in dashed borders to signify relevant segments, \revise{catering to \final{their} multi-modal nature (\eg~images, tables, and texts).}
% \revise{To help users understand what they are focusing on, \tool  further adopts blue borders to signify the focused cell and also the related cells in relationships.
% For detailed contents, we adopts pink background for text format and red dashed borders for multi-modal outputs to avoid content masking. }

% we introduced visual cues to indicate the relationships among \yanna{code}, outputs, and texts, as depicted in \autoref{fig:deign}.
% To communicate each relationship defined in $R$, visual cues were tailored to the content's type and granularity. 


% Hegarty and Just argued that linking t
% learning can be seen as having three kinds of cognitive demands: essential processing, incidental processing, and representational holding [HJ89]. 
% The associated representational holding can be potentially reduced and essential  processing can be increased if readers can easily link the text and pictures in multimedia content.

% linking descrease users' confusion about the mapping by providing a clear reference between text and vis.


To effectively show the detailed relationships identified in $R$, \tool uses a two-tiered approach for visualizing connections \revise{following the ``overview first, then details on demand''~\cite{shneiderman2003eyes}.
Specifically, \tool adopts \textbf{explicit links} to present the high-level aggregated relationships $R'$, providing readers with an overview of the connected information. Additionally, it offers \textbf{implicit interactions} for the granular correspondences within $R$, enabling readers to explore the details as needed. }
% }: \textbf{explicit links} for the high-level relationship $R'$ and \textbf{implicit highlights} for the granular correspondences within $R$. 
We initially considered using explicit lines for all relationships in $R$.
% to directly link related elements.
However, this approach causes visual clutter due to the large number of relationships, and content being hidden by links related to detailed segments. 
To mitigate these issues, \tool opts for a more nuanced visual strategy. 
\tool adopts explicit links to visualize the aggregated relationships $R'$, thus offering an overview of the interconnections between text, code, and output cells. 
For instance, \autoref{fig: teaser}~(B) depicts how text cells are interconnected with code and output cells via lines, signifying their relational bonds. 
Specifically, (b3) is the line that depicts the ``observation'' cell related to the code cell (b4).
\revise{Although visualizing the aggregated relationship in $R'$ instead of $R$ can decrease the number of lines displayed, thus reducing visual clutter,  this method may obscure the specifics of the relationships.}
\revise{To address this problem, \tool introduces interactions to facilitate exploring and understanding detailed relationships as needed, which is detailed in~\autoref{interaction_design}.}

% Specifically, interacting with lines or cells—or the specific visual cues within a cell—triggers a highlighting effect, with both the cell’s border and its interconnecting lines turning blue. 
% This feature helps readers identify the relationships.
% For those detailed relationships captured in $R$, \tool uses implicit highlights. 
% This allows readers to delve deeper into specific relationships as needed, without overwhelming the interface with excessive visual information. 
% When readers interact with visual cues (like hovering over them) that signify the presence of relationships, \tool distinctively highlights the relevant information. 
% These visual cues of active status are illustrated in~\autoref{fig:visual_cues}. 
% Specifically, it applies a pink background to highlight the associated content, ensuring clarity and focus. 
% For output content, to prevent any obstruction of the underlying data visualizations, a pink border is utilized instead of a background color. 
% \autoref{fig:focus_mode} exemplifies this approach: hovering over an element not only changes its border to pink (a1) but also triggers a background highlight in related content (as seen in (a2)) and simultaneously displays a tooltip (a3), providing immediate access to relevant descriptive text. 



% When users encounter areas of interest or potential connections indicated by these visual hints, hovering over the cues triggers highlighting of the detailed relationships. 

% As users go through the high-level overview, and  then find some specific area of interests with visual cues that indicating potential related information,  then hover over corresponding visual cues to highlighting the detailed relationships, enabling users to discover and understand the nuanced interplays between text, code, and outputs with minimal visual distraction and cognitive load.

% one potential solution is that using explicit lines to link all corresponding relationships.
% However, this will be visual clutter due to 1) there are too many relationships will confuse readers;
% 2) for those relationship with segments, the direct link to the segments may overlap other segments, covering other contents.



% Explict link and implicit links:

% - 2) Literature review:  Hegarty and Just argued
% that learning can be seen as having three kinds of cognitive demands:
% essential processing, incidental processing, and representational
% holding [HJ89]. The associated representational holding
% can be potentially reduced and essential processing can be increased if readers can easily link the text and pictures in multimedia
% content.

% linking descrease users' confusion about the mapping by providing a clear reference between text and vis.
% where the explanatory visual element is highlighted when users mouse over narrative text ==> Hover and highlight to link is a common way to link two contents ==> linking and layout

% \subsubsection{Relationship Display}




\begin{figure}[t!]
    \centering    \includegraphics[width=\linewidth,alt={This figure presents a table by \tool that demonstrates visual cues for varying granularity levels, content types, and interaction statuses within a computational notebook. The 'Entire Cell' row indicates inactive state with a dashed border, while the active state is shown with a pink background and a solid blue border. In the 'Segment' row, the content type differentiation is depicted: text segments have underlines in green for code-related, blue for output-related, and purple for both. Their active states are shown with a pink background. The 'Code' segment is underlined in green, which changes to a pink background when active. For 'Output', a dashed black border highlights the area of interest, which changes to a solid red border when active.}]{figures/visual_cues_new.png}
    
    \caption{Demonstration of how \tool uses visual cues to indicate relationships within the notebook, tailored specifically to the content type, granularity, and interaction status.}
\Description{This figure presents a table by \tool that demonstrates visual cues for varying granularity levels, content types, and interaction statuses within a computational notebook. The 'Entire Cell' row indicates inactive state with a dashed border, while the active state is shown with a pink background and a solid blue border. In the 'Segment' row, the content type differentiation is depicted: text segments have underlines in green for code-related, blue for output-related, and purple for both. Their active states are shown with a pink background. The 'Code' segment is underlined in green, which changes to a pink background when active. For 'Output', a dashed black border highlights the area of interest, which changes to a solid red border when active.}

    \label{fig:visual_cues}
    \vspace{-1em}
\end{figure}


\begin{figure*}[t!]
    \centering
    \includegraphics[width=1.0\textwidth, alt={The figure contains four sub-images, illustrating different interaction modes in a re-layout computational notebook. Top-left image: The original re-layout of the computational notebook. The notebook contains multiple cells, including text cells and code/output cells. Lines visually connect related content between text cells (left) and code/output cells (right), showing how content and code are linked. Top-right image (A): Depicts the "hover-to-highlight" interaction. When a user hovers over a specific part of a text cell, a blue outline highlights the hovered text cell itself and the related code or output cell on the right, along with the detailed contents in pink background. Bottom-left image (B): Shows the "key-activated focus" interaction. When activated, unrelated cells are filtered out, leaving only the related cells in screen. This mode helps the user concentrate on relevant content by minimizing distractions from other cells. Bottom-right image (C): Represents the "click-to-fix" interaction. This allows the user to click on a specific cell to pin it in place. A red pin icon appears in the corner of the cell, visually indicating that the cell is now fixed, while the rest of the content can be scrolled independently. This disrupts the default narrative flow but provides easy access for referencing fixed cells.}]{figures/interaction_space.png}
    \caption{Illustration of three interactions supported by \tool: (A) hover-to-highlight, which aids in identifying detailed corresponding content; (B) key-activated focus, which enables users to concentrate on relevant information while minimizing distractions from unrelated content; and (C) click-to-fix, which allows users to pin a cell, thereby disrupting the original narrative flow to facilitate easy referencing.}
    \Description{The figure contains four sub-images, illustrating different interaction modes in a re-layout computational notebook. Top-left image: The original re-layout of the computational notebook. The notebook contains multiple cells, including text cells and code/output cells. Lines visually connect related content between text cells (left) and code/output cells (right), showing how content and code are linked. Top-right image (A): Depicts the "hover-to-highlight" interaction. When a user hovers over a specific part of a text cell, a blue outline highlights the hovered text cell itself and the related code or output cell on the right, along with the detailed contents in pink background. Bottom-left image (B): Shows the "key-activated focus" interaction. When activated, unrelated cells are filtered out, leaving only the related cells in screen. This mode helps the user concentrate on relevant content by minimizing distractions from other cells. Bottom-right image (C): Represents the "click-to-fix" interaction. This allows the user to click on a specific cell to pin it in place. A red pin icon appears in the corner of the cell, visually indicating that the cell is now fixed, while the rest of the content can be scrolled independently. This disrupts the default narrative flow but provides easy access for referencing fixed cells.}
\label{fig:interaction_space}
\end{figure*}



\subsubsection{Interaction}
\label{interaction_design}

% To facilitate \final{identifying and navigating} relationships within the notebook, 
\tool is designed to make it easier to \final{identify} and navigate these relationships, as illustrated in ~\autoref{fig:interaction_space}. 
\revise{These interactions include hover-to-highlight (A), key-activated focus (B), and click-to-fix interactions (C).}
 

\textbf{Hover-to-highlight.}
Interacting with lines, cells, or specific visual cues within the cells triggers a highlighting effect.
The borders of related cells and their interconnecting lines turn blue. 
Visual cues related to the hovered content are also marked as ``active'' status with pink backgrounds or red borders, as detailed in~\autoref{fig:visual_cues}. 
This interaction is further complemented by a tooltip that presents concise, relevant content for a quick overview. 
For instance, as shown in \autoref{fig:focus_mode}, hovering over an output segment (a1) changes the background color of the associated content (a2) to pink and brings up a tooltip (a3) next to it for immediate insight. 
Similarly, hovering over a text segment (a4) displays relevant output details in a tooltip (a5).
This feature helps readers identify the relationships.

\revise{\textbf{``Shift'' key-activated focus.} Considering that related cells might be scattered or distant from each other, interleaved with many unrelated ones across the UI, \tool introduces a ``Shift'' key-activated focus mode to address these challenges.}
The selection of the ``Shift'' key as the activation mechanism was deliberate, chosen to avoid conflicts with JupyterLab's existing keyboard shortcuts, such as double-clicking for edit mode, right-clicking for contextual menus, and pressing \final{the} ``Space'' key for scrolling.
\revise{Focus mode filters out all unrelated cells, allowing users to concentrate solely on the information related to the cell or visual cue under examination.
This mode aims to streamline the exploration process within the notebook by minimizing distractions from irrelevant content.}
\autoref{fig:focus_mode} displays the focus mode for the ``observation'' text cell that contrasts with the standard right column interface seen in \autoref{fig: teaser} (B). 
In this mode, only information related to the ``observation'' cell is shown and presented concisely. 

\begin{figure}[t!]
    \centering    \includegraphics[width=\linewidth,alt={This figure demonstrates the `focus mode' of an `observation' cell in a computational notebook interface, with the observation cell on the left and the related information composed on the right, devoid of irrelevant details. The visualization includes labels a1 through a5, which signify a bidirectional hovering mechanism. a3 and a5 are two tooltips appearing to offer succinct explanations.}]{figures/focus_mode.png}
    \caption{
    The focus mode of ``observation'' cell,  displaying only relevant information and omitting unrelated content. \yanna{
    % The alphanumeric system such as (a1) is used to identify portions of the figures. 
    Hovering over an output segment (a1) highlights the related text in pink (a2) and triggers a tooltip for quick information access (a3). Similarly, hovering over a text segment (a4) triggers a corresponding output snapshot in a tooltip (a5).
    } 
    % (a1)-(a5) demonstrate the bidirectional hovering mechanism that highlights relevant content with a pink background or border and provides tooltips (a3 and a5) for quick information grasp.
    % \yanna{This figure showcases the bidirectional hovering mechanism in the  focus mode of ``observation'' cell, which displays only relevant information and omitting unrelated content. The alphanumeric system such as (a1) is used to identify portions of the figures. (A) Hovering over an output segment (a1) highlights the related text in pink (a2) and triggers a tooltip for quick information access (a3). 
    % (B) Hovering over a text segment (b1) reveals corresponding output details in a tooltip (b2).} 
    }
    \label{fig:focus_mode}
    \Description{This figure demonstrates the `focus mode' of an `observation' cell in a computational notebook interface, with the observation cell on the left and the related information composed on the right, devoid of irrelevant details. The visualization includes labels a1 through a5, which signify a bidirectional hovering mechanism. a3 and a5 are two tooltips appearing to offer succinct explanations.}
    \vspace{-1em}
\end{figure}


% Moreover, to address the challenge of navigating through notebook cells that may be scattered or distant from one another, 
\textbf{Click-to-fix. }\revise{To further address the challenge of related cells not being on the same screen—thus eliminating the need for users to frequently scroll back and forth—\tool supports click-to-fix interactions that allow readers to pin a cell. }
This action locks the cell's position on the screen and maintains its highlight, facilitating a focused examination of its related content without the necessity to adhere strictly to the notebook's linear narrative.
To aid in locating related cells more easily, \tool scrolls automatically upon fixing a cell. 
Hovering over any visual cue automatically brings the first related cell into view.
An illustration of this mode is in \autoref{fig: teaser} (B), showcasing the outcome of fixing the ``observation'' cell. 
Despite being positioned at the end of the notebook, its reference to preceding code and output cells becomes easier. 
The click-to-fix feature ensures the ``observation'' cell remains visible alongside related \yanna{code} and outputs, enhancing content correlation and {understanding}. 
The fixed icon, as seen in \autoref{fig: teaser} (b2), serves as an indicator of the cell's fixed status.
Readers can easily \final{exit} this fixed mode with another click.


In summary, \tool uses these interactions to support exploring the detailed relationships in $R$, allowing readers to delve deeper into specific relationships as needed, without overwhelming the interface with excessive visual information. 
% When readers interact with visual cues that signify the presence of relationships, \tool highlights the relevant information.

% Though \tool choose to visualize the summarized relationships $MR$ to reduce visual clutter.
% The relationships of $MR$ still be confused since its complex many-to-many relationships.
% Previous research about clutter reduction have summarized the strategies, such as change the appearance through sampling, filtering and change opacity.
%  % Taxonomy of Clutter Reduction for Information Visualisation,"
%  Inspired this, \tool integrates a suit of interactions to further reduce visual clutter and facilitate the relationship understanding.

%  \tool enables users to hover a specific cell, to highlight the corresponding related cells by decreasing the opacity of other unrelated components.
%  \tool also support users to click a specific cell to fixed the highlight and opacity effect of the users.
%  Considering in some case that some related cells are really distant, users need to scroll down to find the corresponding cells.
%  In this case,tool support users to filter the infomration by hover the element and press the "option" key. 
%  Then those related information will be collected in a tooltip let users filter the information and focus on those related information.
%  Also, the tools support user to hover and press the control key to filter those relevant information, while all others are disappear.
 
 








