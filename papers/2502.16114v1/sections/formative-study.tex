


\section{Formative Study} \label{sec: formative study}

We conducted a formative study to capture the challenges of understanding computational notebooks and to derive design requirements for facilitating the understanding of notebooks. 
This study involved semi-structured interviews with six participants who have experience working with shared computational notebooks.




\subsection{Participants}
We recruited six participants (3 females, 3 males; aged $26.8\pm2.2$ years; identified as FP1-6) through online advertisements on social media and word-of-mouth referrals.
All participants are postgraduate students, with five pursuing PhD degrees. 
They have extensive experience using Python and Jupyter Notebooks, \final{with an average of} $6.8\pm1.9$ years and $5.3\pm0.8$ years, respectively. 
On a 5-point Likert scale (where 1 indicates ``not at all familiar'' and 5 indicates ``extremely familiar'')~\cite{likert}, all participants rated themselves as either ``moderately familiar'' (4) or ``extremely familiar'' (5) with Python and Jupyter Notebooks. 
Their backgrounds span diverse fields, including data visualization, computer vision, natural language processing, reinforcement learning, and human-computer interaction.




\subsection{Procedure}

The formative study was executed through one-on-one online meetings. 
Participants began by signing a consent form that authorized use to record videos of the session and to collect their demographic information and feedback for research purposes.

\concise{We first collected participants' familiarity with Python and Jupyter Notebooks and their demographic information.
After that, participants were asked to describe their experience with understanding computational notebooks.
% , including the challenges they faced and their general understanding process.
They were asked to share recent notebooks they had attempted to understand to provide a concrete context and details of the challenges encountered.}
%%%After being collected demographic information, participants were asked to recall and describe their experiences with reading and understanding computational notebooks, including detailing scenarios.
%%%To delve deeper into their understanding process and further understand the challenges they faced, participants were encouraged to share recent notebooks they had attempted to understand for recalling their experiences and providing context for the challenges encountered.
\concise{When} sharing notebooks was impractical due to data confidentiality, participants were asked to select a notebook of interest from Kaggle competitions using their domain expertise as relevant keywords (\eg~``computer vision'').
% prioritizing well-voted notebook in highly engaged competitions.
Participants were then asked to thoroughly comprehend the selected or shared notebooks until they felt confident enough to reuse the notebooks or for up to 20 minutes (limited to prevent fatigue). 
Then we asked the participants to describe the challenges they encountered in understanding the notebook.
We \concise{asked} open-ended questions such as, ``Did you encounter any challenges while understanding this notebook?'', ``What aspects, if any, hindered your understanding of this notebook?'', ``Have you faced similar challenges in previous experiences with notebooks?'', and ``Beyond this notebook, have you encountered other obstacles that made understanding notebooks difficult?''.
\concise{We further asked participants to provide detailed explanations and concrete examples with follow-up questions like, ``Can you elaborate on why this aspect hinders your understanding?'' and ``Could you provide specific examples?''}
%  
% Questions posed included, ``Do you encounter ant challenges when understanding this notebook'', 
% ``What prevents you from understanding this notebook'', ``Have these challenges occurred in your previous experiences with notebooks'' and ``Besides this notebook, have you encountered other challenges that prevent you from understanding notebooks based on your previous experience''.
% To elucidate these proposed challenges further to avoid misunderstandings, participants were asked to offer detailed explanations and concrete examples from the notebooks or describe the related scenarios they had previously encountered.
% This included probing questions like, ``Can you elaborate on why this aspect hinders your understanding?'' and ``Could you provide specific examples?''.
Finally, we invited the participants to propose what tool features they expected to enhance the understanding process.
Each session lasted around an hour, with participants compensated with US \$15 for their time and insights.

% \textcolor{red}{I think it needs to be shortened substantially (by ~40\%) and would be more appropriate as a~6,000–7,000 word submission. I admit that I'm struggling to identify too many sections or subsections that should be removed (I personally think 3.4 and references to it could be removed) but I do think (i) the writing can be tightened up throughout (ii) the "core" idea of linking text, code, and output could be introduced faster given prior work(2AC)}
\subsection{{Findings}}
\label{sec:findings}

% \reviewerCom{Formative study: 1) there is no presentation of what data were collected or how they were analyzed. There is mention of a consent form that requests authorization for recording (audio? video?). Were these recordings transcribed? Were they analyzed in a particular way? Or are the findings more of a gut feeling or informal characterization of the interviewers' findings? While all of these are perfectly fine (although more rigor would generally be better), it is still necessary to provide the reader with enough detail to be able to understand how much weight to put in the findings.
% }

% \revise{
% This section summarizes the formative study findings about obstacles in understanding notebooks including the individual elements of text, \yanna{code}, and outputs, as well as the relationships among these elements.}
\yanna{Participants in \final{the} formative study reported both obstacles in understanding individual elements of notebooks (i.e., text, code, and outputs), as well as the structure and relationships of these elements.
While previous studies (\autoref{sec:rw_understanding}) have predominantly discussed and addressed problems related to individual elements and structured outlines, this section reports obstacles caused by the complex relationships and interleaving structure of these elements.}


% \yanna{\textbf{Difficult navigation of unclear, multi-granular relationships among code, outputs, and texts.}}
\textbf{\yanna{C1: Difficult relationship inference due to} unclear, multi-granular relationships between code, outputs, and text.}
% \yanna{Participants reported two challenges when trying to understand the relationships among text, code, and outputs in computational notebooks.}
% \yanna{\textbf{First},} 
Participants identified challenges in inferring the relationships among text, code, and outputs (FP1-FP4).
They mentioned spending extra time determining the source or reference point of the text within the notebook.
These challenges were notably intensified when attempting to associate specific segments across these elements.
% \yanna{For example, FP1 and FP3 encountered unfamiliar functions in code cells and struggled to find accompanying textual explanations.}
For instance,
FP1, who often reviews notebooks for research purposes, noticed a trend where notebooks use markdown cells with summaries to describe findings from previous analyses. 
Yet, he struggled to trace back each finding to its corresponding output and encountered even greater difficulty when attempting to pinpoint specific segments within those outputs.
Similarly, when delving into code and outputs, participants needed additional efforts to identify related textual descriptions for understanding (FP1, FP3). 
% For example, FP3 encountered difficulties in comprehending a set of complex visualizations and sought textual descriptions that might elucidate the key takeaways from these visuals.
% He found it more difficult to determine which segments within a series of visualizations were responsible for deriving the findings.
% struggled to figure out which segments in multiple visualizations derive the findings.

\yanna{\textbf{C2: Demanding cross-reference due to single-column layout.}
Participants reported that the traditional single-column layout made it difficult to cross-reference related information, requiring frequent scrolling to check information across different notebook cells (FP1, FP2, FP4). 
FP4 mentioned, ``\textit{When the code or text extends beyond my screen, I have to scroll back and forth to check details, which is mentally demanding as I have to remember and recall them.}''
FP1 echoed this concern, adding that the challenge intensifies when related cells are far apart.}
% \yanna{\textbf{Third,} participants reported that the traditional single-column layout exacerbated these issues, as it required frequent scrolling to track and reference related information across different notebook elements (FP1, FP2, FP4). 
% This observation aligns with findings from~\cite{zhi2019linking}, which demonstrated that readers engaged in more text-visualization transition actions in vertical layouts compared to side-by-side slideshow layouts.
% }

\textbf{C3: Disrupted focus due to interleaving structure.}
Participants in our study (FP1, FP2, FP5) reported that the interleaving structure of notebook elements often disrupted their focus, \yanna{making it difficult to maintain sustained attention on specific type\final{s} of elements}.
% hindering their ability to engage deeply with the content for understanding
% (FP1, FP2, FP5).
FP1 noted a feeling of interference when attempting to concentrate on a single aspect of the notebook. 
\yanna{Similarly, FP2 described difficulty focusing on text during the understanding phase and code during the applying phase~\cite{forehand2010bloom}, due to the interleaving text and code}.
% Similarly, FP2 articulated a shift of needs in different stages of interaction with the notebook—requiring text during the understanding phase and focusing on \yanna{code} during the applying phase~\cite{forehand2010bloom}. 
% \yanna{This process was disjointed and lacked smoothness due to the interleaving structure.}
FP5 mentioned that displaying various types of information (including text, \yanna{code}, and outputs) in limited vertical screen space is \yanna{dizzying} and overwhelming, \yanna{making it hard to focus on one type of element at a time.}





\subsection{Design \yanna{Requirements}}

\yanna{Findings in~\autoref{sec:findings} reveal that readers encounter numerous obstacles in \final{identifying and navigating} the relationships among text, code, and outputs.}
\yanna{Based on these findings, we derive five key design requirements for our system design.}

\yanna{\textbf{DR1: Provide clear visualization with clutter-reducing mechanisms for multi-granular relationship inference.}}
To address the difficulties in inferring the unclear and multi-granular relationships between \yanna{code}, outputs, and text, the tool should visually show these relationships \textbf{(C1)}.
These visual aids should support the full spectrum of relationship granularities, ranging from entire cells to specific segments within them.
\yanna{Moreover, the tool should incorporate mechanisms to minimize visual clutter caused by numerous relationships.}
With the tool, readers should be able to effortlessly infer both the existence and positions of the \yanna{relationships}.


\yanna{\textbf{DR2: Offer flexible interactions to streamline relationship inference and navigation.}}
% Given the necessity to understand and synthesize the specifics of relationships across different levels of detail, t
The tool should offer flexible interactions to allow users to \yanna{easily infer and \final{navigate} relationships and cross-reference the connected information \textbf{(C1 and C2)}. 
With the tool, readers should be able to easily locate, navigate, and cross-reference interconnected text, code, and outputs} at various granularities dispersed across the notebook, thereby facilitating the synthesis and interpretation of these interconnected elements.
% This includes enabling readers to easily pinpoint relevant text, \yanna{code}, and outputs, thereby facilitating the synthesis and interpretation of these interconnected elements.
% With the tool,  readers should be able to efficiently locate, cross-reference, synthesize, and interpret interconnected text, \yanna{code}, and outputs dispersed across the notebook.

\yanna{\textbf{DR3: Provide alternative layouts for efficient cross-referencing.}
The tool should explore alternatives to the single-column layout, organizing related information in a more accessible manner for quick cross-referencing \textbf{(C2)}.
With the tool, users should be able to easily infer multi-granular relationships and cross-reference related information for better understanding.
}


\textbf{DR4: Enable separate focus on text or code and output \yanna{to reduce focus disruption}.}
The narrative text and computational \yanna{code} or outputs are often intertwined, which can disrupt user focus when attempting to concentrate on one type of element.
To address this, the tool should provide mechanisms to isolate the content of the text with \yanna{code} and outputs \textbf{(C3)}.
With the tools, readers should be able to selectively concentrate on either the text or the \yanna{code} and outputs independently. 
% Providing mechanisms to isolate or emphasize one type of content at a time can help mitigate distractions and enhance concentration.

\textbf{DR5: Integrate with existing platforms seamlessly.} \yanna{Supported by literature showing that integrating with existing platforms reduces users' learning curve~\cite{li2023notable}, we propose this design requirement.}
The tool should integrate seamlessly with common computational notebook environments, eliminating the need for readers to learn a new interface.

\revise{In this research, we aim to understand whether a reading-optimized design can improve the experience of reading computational notebooks. We leave authoring tools for future work.} 
