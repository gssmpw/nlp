\section{Discussion}
This section discusses the lessons learned during our study and the limitations and future work of the research.

\revise{\textbf{Clear relationships \final{help users focus on understanding by streamlining retrieval of related content}.}
\tool~\final{makes computational notebooks easier to understand} by designing a new side-by-side layout with clear visualization and interactions to \final{identify and navigate} relationships between text, \yanna{code}, and outputs. 
% This design allows participants to quickly and accurately identify connections and related information to synthesize a coherent understanding, as evidenced by higher task accuracy.
This design allows participants to quickly and accurately locate and \final{synthesize} related information, as evidenced by higher task accuracy.
\yanna{
Some participants reported reduced efforts in retrieving relevant content, enabling more focus on \final{integrating and understanding relevant information}.}
Future research should further streamline the burdensome basic tasks of \final{identifying and navigating} relationships of relevant content, enabling users to concentrate more on understanding and higher-level tasks, such as applying and creating~\cite{forehand2010bloom}.}

% In other words, 
% \tool redefines the user experience by prioritizing understanding over simple information retrieval, thus elevating user focus from basic retrieval tasks to higher-level understanding~\cite{forehand2010bloom} tasks.

% \bigskip

% \noindent
\textbf{Gaps between current computational notebook practices and the literate programming principle.}
Computational notebooks, designed on the literate programming principle, aim to integrate \yanna{code}, outputs, and explanatory text into a cohesive narrative~\cite{rule2018exploration,wang2022documentation, chattopadhyay2023make}.
% that makes the rationale behind computations accessible
% This integration bridges text, \yanna{code}, and outputs into a cohesive narrative. 
However, current practices often adopt a linear layout with unclear relationships between \yanna{code}, outputs, and text~\cite{chattopadhyay2023make}.
% often diverge from this ideal.
% They adopt a linear layout that, while supportive of exploratory tasks, tends to overlook the importance of clear explanations due to the unclear relationship between \yanna{code}, outputs, and text~\cite{chattopadhyay2023make}.
Such designs weaken the explanatory role of the text, turning notebooks into loosely connected scripts that diverge from literate programming principles~\cite{rule2018aiding, wagemann2022five}.
To address this, \tool~introduces a side-by-side layout to present relationships more clearly to help readers integrate information. It therefore realigns notebooks with literate programming principles.
\yanna{User ratings suggest that \tool{} \final{helps} participants infer, locate, navigate, and integrate relevant information, and enhance their overall understanding of the notebook.}
% Further exploration is needed to better align computational notebooks with literate programming. 
We \final{suggest} future research and tool development should further align computational notebooks with literate programming to maximize their potential for clear and effective communication.
% prioritize literate programming principles to maximize the potential of notebooks for clear, effective, and engaging communication.

% \bigskip

% \noindent
\textbf{Limitations and future work.}
Our efforts to make computational notebooks more readable and understandable unveil several future research opportunities.
% Our efforts in facilitating computational notebook \final{readability} with \tool unveil several future research opportunities.
% avenues for future improvement.
First, enhancing the system design would be beneficial. 
Participants expressed a desire for more intuitive methods to navigate and manipulate relationships within the notebook environment. 
Future iterations could focus on simplifying non-intuitive key press interactions (P1, P6-P10) and enabling dynamic layout adjustments, such as adjustable proportions of descriptive text, \yanna{code}, and outputs to accommodate diverse user needs more effectively (P3).
Moreover, integrating advanced language models such as GPT-4~\cite{openai2023gpt4} for content summarization was highlighted, especially valuable when dealing with extensive textual information (P11, P12). 
\revise{Considering all types of relationships could also be promising to \final{make notebooks easier to understand}}.
% Second, our approach relies on predefined relationships that are manually created.
\yanna{Second, our approach relies on predefined relationships that are manually created and require textual description.
Future work could explore description generation techniques~\cite{lin2023inksight} to enhance notebook quality and thus extend the applicability of \tool.
Additionally, advancing technologies for automatic relationship mining~\cite{lin2023dashboard} and developing authoring tools that facilitate easier relationship establishment between different notebook components~\cite{latif2022kori, sultanum2021leveraging} could further enhance the system.
These advancements should address challenges in relationship maintenance, such as cell edits, rearrangements, and re-runs.}
Third, our evaluation contains limitations that warrant future consideration.
One limitation is that participants may not be proficient enough in \tool given the short training time.
Moreover, due to time management of the study, participants were asked to answer the questions rather than \final{carefully read} the entire notebook, which may not fully align with their practice of  \final{reading and understanding notebooks}.
In the future, we hope to conduct long-term and real-world user studies.