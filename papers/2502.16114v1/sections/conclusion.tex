\section{Conclusion}


In this work, we introduce \tool, a computational notebook plugin designed to \final{present relationships between text, code, and outputs, thereby making notebooks easier to understand}. 
\tool aims to address the challenges readers face when trying to \final{read and} understand computational notebooks shared by others, which often requires synthesizing text, code, and outputs \final{by identifying and navigating their relationships}. 
Specifically, it features a side-by-side layout that reorganizes notebook elements into two columns and incorporates visual cues and interactions to aid cross-referencing and \final{information integration}.
This reading-optimized design has been shown to improve \final{finding and integrating relevant information for understanding notebooks}, as evidenced by increased accuracy and reduced IES scores in our user study.
We hope our work emphasizes the importance of design considerations in computational notebooks that prioritize not only analysis but also communication. 
