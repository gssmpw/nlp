

\section{Related Work}
Our research is related to prior studies on how people understand computational notebooks, computational notebook layouts, and \yanna{links between code, outputs, and text.}

\subsection{Computational Notebook Understanding}
\label{sec:rw_understanding}

\yanna{Existing tools communicate analysis results in computational notebook\final{s} by transforming them} into other formats such as slides~\cite{zheng2022nb2slides, li2023notable, wang2024outlinespark, wang2023slide4n} and data comics~\cite{kang2021toonnote}. 
\yanna{While these tools focus on transforming notebook\final{s into other formats}, our work emphasizes \final{reading and} understanding computational notebooks in their original form to support continuous analysis and reuse.}

\yanna{
The iterative nature of exploratory data analysis often leads to long, messy notebooks, compounded by insufficient or absent textual explanations, which hinder reading, understanding, and collaboration~\cite{rule2018aiding}.
To tackle these issues, 
some research has explored textual explanation generation methods to improve readability and understanding~\cite{lin2023inksight, wang2022documentation, wang2020callisto}. 
Other efforts focus on enhancing code organization to mitigate messiness, including code cleaning and organization~\cite{head2019managing, wang2020assessing, shankar2022bolt}, adapting notebooks for non-linear analysis workflows~\cite{weinman2021fork, wang2022stickyland, harden2022exploring}, and providing analytical workflow overviews~\cite{ramasamy2023visualising, wenskovitch2019albireo}.
}
Despite these advancements, these approaches typically focus on a single aspect—either \yanna{code} or text—without fully addressing the complex interplay between them, a critical component for reading and understanding notebooks~\cite{kery2018story}. 

To fill this gap, recent research has introduced structured outlines to better organize the loose \yanna{code} and text. 
For instance, Rule \etal~\cite{rule2018aiding} developed a plugin that allows users to group code cells with annotated text, incorporating features like text cell folding for detailed exploration. 
\yanna{Similarly, Chattopadhyay \etal~\cite{chattopadhyay2023make} organized notebooks into labeled, expandable chapters for improved navigation.}
Though useful, they predominantly emphasize headers, overlooking a significant portion of descriptive content—approximately 68\%—that exists in forms other than headers \cite{wang2022documentation}. 
Users have indicated a need for understanding content \final{at a finer level of granularity}, such as individual code statements over broader, header-level summaries~\cite{chattopadhyay2023make}. 
Aligning with these observations, our work aims to address this gap by providing \revise{an interface} that facilitates a clearer understanding of multi-granular relationships between text, code and output.








\subsection{Computational Notebook Layout}
\label{sec:layout}
% \textcolor{red}{Dom: Related work should describe B2 which proposes a very different layout from the traditional notebook. 
% Related work could cite the Observable framework article which basically says that notebooks are not optimized for reading and that observable released framework to address that gap. We are arguing that having a different representation of notebooks optimized for reading can be valuable. Video: https://vis.csail.mit.edu/pubs/b2/}
% \textcolor{red}{Also https://openreview.net/pdf?id=Gkogn48LeI}

Computational notebooks have become essential in data science and computational research, improving the way computational narratives are constructed and presented~\cite{rule2018exploration}.
Yet, the prevailing linear structure of these notebooks often poses challenges for non-linear analyses~\cite{wang2022stickyland, weinman2021fork}, effective navigation through extensive content~\cite{harden2022exploring}, and the orderly management of complex information~\cite{head2019managing, liu2019understanding, shankar2022bolt}, ultimately affecting usability and comprehension.

Researchers have explored various strategies to address these limitations by redesigning the notebook layout to leverage large displays and support complex analytical tasks.
For example, Wang \etal proposed StickyLand~\cite{wang2022stickyland}, a plugin that treats each cell as a movable sticky note, allowing for a flexible,
non-linear organization of \yanna{code} that aligns with the exploratory nature of data analysis. 
Weinman~\etal~\cite{weinman2021fork} introduced forking mechanisms within notebooks facilitating the exploration of multiple analytical pathways.
Additionally, empirical studies further showed the benefits of multi-column or workboard layouts of notebook\final{s} to branching and comparative analyses~\cite{harden2022exploring, harden2023there}.
Besides non-linear analysis, B2~\cite{wu2020b2} offers interactive visualizations alongside notebook content, aiming to bridge the gap between the iterative, two-dimensional nature of interactive visualizations and the linear layout of computational notebooks. 
Janus~\cite{rule2018aiding} further enhances notebook functionality by introducing collapsible content sections, which can be selectively displayed to minimize clutter and focus on relevant analyses. 



% by minizing the scope and need for virtual nativation (scrolling). 
% Further exploration by Harden\etal~ into how users would prefer to arrange cells in a 2D space revealed preferences for linear, multi-column, and workboard patterns~\cite{harden2022exploring}. 
% This study underscored the potential advantages of 2D layouts for branching and comparative analyses but also highlighted the need for additional navigational and organizational aids.
% Similarity, Jesse \etal~further conduted emperical study to learn the advantanges of ``2D computational notebooks'' by organizing notebook into multi-columns, which give the emprival evidence that multi-column 2d computation notebooks provide enhance efficiency and usability.

Observable reflected on the traditional notebook single-column, narrow layout for presentation purposes and noted its limited information density.
\revise{They pointed out that while this format is suitable for ad-hoc exploration, it is inadequate for presentations and displays~\cite{observable}.}
% In response, Observable proposed new framework to offer a more flexible layout design that caters to creating aesthetically pleasing pages and interactive applications 
Thus, a different representation of notebooks, optimized for reading and understanding, would be valuable.
In light of these developments, our work proposes a side-by-side layout with interactions and visual cues for computational notebook\final{s}, aiming to enhance information integration and understanding by presenting the relationships between text with code and output.


% \subsection{Creating navigable links between Code-text and Code-output}
% \subsection{\final{Linking Code with Text and Outputs}}
\subsection{\final{Code-text and Code-output Links}}


\yanna{

Computational notebooks are widely used in data science and research because they integrate executable code, outputs, and descriptive text into a cohesive document, making the rationale behind computations more transparent~\cite{knuth1984literate, wang2022documentation}.
% This aligns with Mayer's multimedia principle, which suggests that combining information from multiple media enhances understanding compared to using a single medium~\cite{mayer2002multimedia}.
However, synthesizing different information often challenging especially when they are spatially separated~\cite{kong2014extracting}.  

To address this issue, research in programming environments has explored linking code with relevant texts or outputs to streamline tasks like coding and debugging. 
Some tools enhance coding workflows by linking code to related texts, such as interactive documentation~\cite{oney2012codelets, ko2006barista}, chat messages~\cite{oney2018creating}, or code explanations~\cite{canvas, yan2024ivie}. 
These tools often present information inline, adjacent, or in floating panels for efficient information seeking and understanding.
While effective, these approaches typically focus on simple, one-to-one relationships between code and text.
Others connect source code of GUI application\final{s} with corresponding UI outputs to support code review and debugging~\cite{huang2024unfold, chi2018doppio}.
Extending beyond one-to-one relationships, UNFOLD~\cite{huang2024unfold} labels code pieces with multiple UI output identifiers, yet it still requires users to piece together the relationships manually.

% In computational notebook, recent tools have explored linking notebook content with user-generated artifacts through interactions.  
Recent tools, which extend computational notebook interface for additional content creation, provide interactions for users to link back to the original content in computational notebooks.
For instance, when using OutlineSpark~\cite{wang2024outlinespark} to create slides next to a computational notebook interface, users can click the slide outline to highlight related notebook cells.
Similarly, Callisto~\cite{shi2020calliope} uses a two-column layout to link chat messages with specific code or outputs.
% For instance, OutlineSpark~\cite{wang2024outlinespark} connects notebook cells with user-generated outlines via LLMs to aid slide creation, while Callisto~\cite{shi2020calliope} employs a two-column layout to link chat messages with specific code or outputs.
% These tools employ highlights and interactions to help users identify and explore relationships.
% However, such interactive linking requires users to actively trigger connections and lacks mechanisms to provide an overview of relationships across text, code, and outputs.
% This limits users' \final{flexibility} to fully understand the relationships within a notebook.
However, these tools require users to actively trigger connections and lack mechanisms to provide an overview of relationships across text, code, and outputs within a notebook.

% mind, the way that InterLink computes layouts depending on relationships seems different from prior work but isn't really emphasized in subsection 2.3
% \final{To address these gaps, \tool ororganizing the code, outputs, and text by computing the layouts  depending on their relationships.
% This layout aims to enables users have a quick  an overview of the relationships.
% Moreover, \tool provides other information 
% }
To address these gaps, our work introduces \tool, a notebook plugin designed to present complex many-to-many relationships between code, outputs, and text.
\final{\tool reorganizes code, outputs, and text by computing layouts based on their relationships, making it easier for users to cross-reference information.
The layout is further enhanced by line connections and visual cues, allowing users to see the relationships, thereby facilitating locating and integrating relevant information.}}
% It enables users to overview the relationships with line connections~\cite{Steinberger2011context} and explore them in detail interactively, facilitating understanding and synthesis of information.}


% \bigskip

% \noindent
% \vspace{-4em}
% \subsection{\final{Linking Text with Visual Elements}}
\subsection{\final{Text-visual Links}}
\yanna{
Visual elements, including charts and images, are often closely connected with descriptive text, and linking them is critical for effective data communication and comprehension~\cite{badam2019elastic, lalle2021gaze, zhi2019linking}. 
Prior research has extensively explored the value of explicit connections between text and visuals to improve readability.
For example, 
Zhi~\etal validated that connecting text with visualizations significantly reduces user reading time while improving comprehension task performance~\cite{zhi2019linking}. 
Thus, several authoring tools
have been developed to help authors link text with visual elements~\cite{WonderFlow,dataplaywright, sultanum2021leveraging}.
These tools use video or scrollytelling~\cite{seyser2018scrollytelling} to present text-visual links. 
However, these methods rely on a fixed content exploration order determined by the author, which limits the reader's ability to explore the content flexibly according to their individual needs.
% Vizflow~\cite{sultanum2021leveraging} further incorporates scrollytelling to present the text-visual linkage.
% Specifically, it integrates textual narratives with various media (\eg~charts and videos) through spatial and scrolling interactions\cite{sultanum2021leveraging}, allowing sequential exploration of content. 

Active reading systems have attempted to address this limitation by enabling readers to organize or interact with linked information. 
For example, LiquidText allows readers to freely select and organize content on a workspace through multitouch interactions~\cite{tashman2011liquidtext}.
While effective, such an approach sacrifices the original document structure, which is essential in computational notebooks where computation order conveys important contextual meaning.
Other tools have supported in-situ information presentation to enhance reading through context-aware linking~\cite{badam2019elastic, lalle2021gaze, zhi2019linking}.
For instance, Badam~\etal developed a system that allows users to select text sentences and link them to corresponding visual elements in tables or visualizations using synchronized visual highlighting~\cite{badam2019elastic}. 
However, these methods primarily support on-demand, localized comprehension while failing to provide a global overview of how different content components are interconnected.

While previous work shows that linking text with visuals eases reading, existing methods cannot be directly applied to computational notebooks.
To address this gap, we propose \tool, a notebook plugin that links text with code and outputs \final{to make notebooks easier to understand}. 
\tool visualizes relationships by using connecting lines that provide a global cell-cell relationship overview and subtle highlights for fine-grained relationship details, while preserving the computation flow of the notebook.
It also offers interactions for exploring these relationships flexibly.
% while preserving computation flow. 
% It also offers both a global overview and in-situ detailed information for deeper understanding.
}

% Linking descriptive text to visual elements is a fundamental technique in data visualization, essential for effective storytelling and enhancing data communication~\cite{badam2019elastic, lalle2021gaze, Steinberger2011context, zhi2019linking, sultanum2021leveraging} .
% Authoring tools like Kori~\cite{latif2022kori} and Vizflow~\cite{sultanum2021leveraging} utilize opacity, visual annotations (\eg~arrows and rectangles), and cross-highlight interactions to link text with associated visual elements.
% These techniques use visual embellishments to signify relationships and enable users to retrieve them through interaction.
% Scrollytelling extends these techniques by integrating textual narratives with other media (\eg~charts and videos) through spatial and scroll interactions~\cite{sultanum2021leveraging}, allowing sequential exploration of text-visual links.
% However, this method requires a predetermined order for content exploration, making it less suitable for non-linear contexts like computational notebooks.

% Building on these advancements, \tool introduces a design for computational notebooks to present and explore relationships between text, code, and outputs, as well as to investigate the benefits of such linkages.




% \textcolor{red}{Fail to insert and compare to papers suggested by R1: Active Reading}
% } 

% \revise{Contrasting with existing methods that primarily contributes on text-visual linkage techniques or the corresponding authoring tools, \tool aims to investigate the benefits of multimedia integration within computational notebooks. 
% \tool introduces a novel side-by-side layout that integrates multimodal content by linking text with codes and outputs, and explores how such a layout can  enhance the readability and comprehension of computational notebooks.}

% \textcolor{red}{how to tell the difference between ours with text-visual linking. Kpri + VizFlow: Semantically linking of text \& visual ⇒ help emphaize the role of linking.}

% - What is the relationship between ours vs text-visual links

% - Scenarios difference ==> what the unique challenges & unique features?







% \subsection{Multimedia Integration}

% \textcolor{red}{\textbf{Version 2:}}
 
% Mayer's multimedia principle highlights that combining information from multiple media significantly boosts understanding beyond a single media~\cite{mayer2002multimedia}.
% However, synthesizing information across distinct media is often challenging especially when they are spatially separated~\cite{kong2014extracting}. 
% Readers must frequently shift their attention between disparate media, such as explanatory text and exploratory code, to build connections~\cite{latif2022kori}. 

% Numerous research efforts have explored linking different types of information (\eg~code, output and text) to facilitate downstream tasks.
% In coding environments, 
% tools link code to contextually relevant information, such as interactive code documentation~\cite{oney2012codelets, ko2006barista}, chat messages~\cite{oney2018creating}, or code explanations~\cite{canvas, yan2024ivie}, to support coding or collaboration.
% Other tools connect source code with corresponding UI outputs, enabling developers to interactively track and review changes with highlights~\cite{huang2024unfold, chi2018doppio}.
% In visualization domain, tools integrates text with visualizations to augment  visualizations for effective stroytelling~\cite{latif2022kori, sultanum2021leveraging, zhi2019linking}.
% While these tools effectively link two modalities (\eg~text-code, code-output, and text-output), they cannot be directly applied to computational notebooks, which involve diverse modalities (\ie~explanatory text, code, and visual outputs) and varying granularities (\ie~cells vs. segments). 


% Recent tools have made progress in computational notebooks by enabling users to link notebook content with user-generated artifacts. 
% For example, Callisto\cite{shi2020calliope} allows users to create chat messages linked to specific code or output regions through a two-column layout and navigate bidirectionally. 
% OutlineSpark\cite{wang2024outlinespark} leverages LLMs to link user-created outlines with notebook cells for slide creation. 
% While these tools primarily focus on the linking process itself, \tool explores how to present these links.


% Building on prior work in aiding comprehension, such as scrollytelling~\cite{seyser2018scrollytelling, epperson2022leveraging} and active reading~\cite{tashman2011liquidtext}, we introduce \tool, a reading-optimized design specifically tailored for computational notebooks. 
% \tool adopts a side-by-side layout that aligns descriptive text alongside with corresponding computational elements for immediate reference.
% It uses explicit links for cell-cell relationships and subtle highlights for fine-grained relationships, such as cell-segment or segment-segment relationships, minimizing visual clutter and avoid overwhelming users. 
% Additionally, \tool~provides interactions to explore these complex relationships.

% % Other research has explored aiding comprehension through scrollytelling, which organizes information into linear stories~\cite{seyser2018scrollytelling, epperson2022leveraging}, and active reading techniques~\cite{tashman2011liquidtext}. 
% % Other research has explored aiding comprehension, such as  scrollytelling to organize different information in a linear story~\cite{seyser2018scrollytelling, epperson2022leveraging} and active reading~\cite{tashman2011liquidtext}.
% % Considering that the explanatory text in computational notebook is not a complete story and also non-linear, \tool proposes a side-by-side layout with explicit links and subtle highlights to present the relationships between text with code and outputs in computational notebook.
% % \tool follows this line of research, trying to find a better way to 

% % Notebooks encompass diverse modalities (\ie~explanatory text, code, and visual outputs), varying granularities (\ie~cells vs. segments), and intricate relationships (such as many-to-many, one-to-many, and many-to-one).
% \textcolor{red}{Concerns: Not related work to support our design}



% \textcolor{red}{\textbf{Version 1}}

% The integration of multimedia content is critical for enhancing learning and comprehension. 
% Mayer's multimedia principle highlights that combining information from multiple media significantly boosts understanding beyond a single media~\cite{mayer2002multimedia}.
% However, synthesizing information across distinct media is often challenging especially when they are spatially separated~\cite{kong2014extracting}. 
% Readers must frequently shift their attention between disparate media, such as explanatory text and exploratory code, to build connections~\cite{latif2022kori}. 
% This challenge, commonly discussed as the split-attention effect~\cite{ayres2005split} and the contiguity principle~\cite{mayer201412} in multimedia learning theory, increases readers' mental effort and working memory and further hinders comprehension and learning outcomes.

% In response to these challenges, numerous research efforts have explored linking different types of information, such as text and code~\cite{ko2006barista, oney2012codelets, omar2012active, yan2024ivie}, code and visual output~\cite{chi2018doppio, huang2024unfold}, or text and output~\cite{zhi2019linking, sultanum2021leveraging, latif2022kori, badam2019elastic, lalle2021gaze, Steinberger2011context, kong2014extracting}.
% For instance, tools like Ivie~\cite{yan2024ivie} and Codelets~\cite{oney2012codelets} provide in-situ annotations that link code to contextually relevant information, such as interactive code documentation or code explanations for AI-generated code (\eg~Copilot-generated code)~\cite{canvas}, to faciliate coding tasks.
% chat.codes~\cite{oney2018creating} enables users to create chat messages referencing specific code regions in a separated chat interface  adjacent to the code environment, enhancing code understanding and collaboration.
% Tools like Doppio~\cite{chi2018doppio} and UNFOLD~\cite{huang2024unfold} connect source code with corresponding UI outputs, enabling developers to interactively track and review code alongside UI revisions using highlights.
% Other tools focus on integrating text with visualizations. 
% For example, Kori~\cite{latif2022kori} and Vizflow~\cite{sultanum2021leveraging}, authoring tools for linking text and visuals, utilize opacity, visual annotations (such as arrow and rectangle), and cross-highlight interactions to identify the links, claiming that interactive links between text and visuals facilitate data communication.
% % Zhi~\etal further validated that linking text with its visualization can significantly reduce user time spent on reading story, and help users performed better in comprehension tasks~\cite{zhi2019linking}.
% While these tools demonstrate the value of linking two modalities (\eg~text-code, text-output, or code-output), they fall short in addressing the complexity inherent in computational notebooks. 
% Notebooks encompass diverse modalities (\ie~explanatory text, code, and visual outputs), varying granularities (\ie~cells vs. segments), and intricate relationships (such as many-to-many, one-to-many, and many-to-one).



% More relevantly, recent tools enable users to create chat messages~\cite{shi2020calliope} or outlines~\cite{wang2024outlinespark} based on notebook content and link the created text to corresponding notebook components. 
% For instance, OutlineSpark~\cite{wang2024outlinespark} links user-created outline text with Jupyter notebook cells for slide creation, but lacks support for finer granularity, such as linking to specific segments of code or output.
% Callisto~\cite{shi2020calliope} adopts a two-column layout with a chat inferface ajancent to the notebook,  enabling users to reference specific code or output regions and navigate bidirectionally. 
% However, it does not address complexities like segment-level navigation or one-to-many text-code/output relationships.

% In this paper, we present \tool, designed to address these limitations by supporting text-code and text-output relationships across different granularities and complexities in computational notebooks
% Specifically, \tool adopts a side-by-side layout, aligning related descriptive text alongside with its corresponding computational elements for immediate reference.
% To present relationships across varying granularities and complexities, \tool uses explicit links for cell-cell relationships and subtle highlights for fine-grained relationships, such as cell-segment or segment-segment relationships, minimizing visual clutter and avoid overwhelming users. 
% Additionally, \tool~provides interactions to facilitation relationship exploration.
