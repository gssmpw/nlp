
\section{Introduction}

Computational notebooks, such as Jupyter~\cite{jupyter} and RStudio~\cite{rstudio}, have become essential tools in data science and research.
Adhering to the literate programming principle, they combine executable \yanna{code}, code outputs, and descriptive text in a single document, making the rationale behind computations accessible~\cite{knuth1984literate}.
These notebooks are structured into discrete units of computation and descriptive text, known as ``cells'', which can be executed and edited independently. 
This flexible design not only streamlines the iterative process of code testing and refinement but also simplifies the integration of comprehensive descriptive text alongside computational content.
As a result, computational notebooks are frequently created and shared among various stakeholders, so there is a wide audience that needs to read and understand the notebooks~\cite{ramasamy2023visualising}.

% However, reading \final{and therefore understanding} shared computational notebooks is challenging.
However, understanding shared computational notebooks is challenging.
Though a notebook author may make the best effort to clean the code and output cells and write descriptive and explanatory text, readers still face three major challenges when \final{trying to identify and navigate the relationships between the text, code, and outputs}\footnote{\final{For brevity, the term ``relationship'' in the remainder of this paper refers to the relationships between the text, code, and outputs.}}.
% synthesizing the text, code, and outputs.
% for \final{reading and} understanding.
\yanna{\textbf{The first is to be aware of related text for code or outputs.}}
% \textbf{The first is to be aware of and locate related text for \yanna{code} or outputs.} 
The location of text relative to its corresponding code or outputs can vary significantly. 
Text describing some code or output can appear above, below, or far away from the relevant code or outputs, depending on the author's preferences or the type of descriptive text~\cite{wang2022documentation}.
Consequently, \yanna{readers may not realize the existence of related text for extra information when reading specific code and output cells, and vice versa}.
% or locate it promptly.
% Consequently, readers frequently encounter uncertainty about whether explanatory text exists for specific \yanna{code} or outputs and, if present, its location. 
% \textbf{The second is sorting out the many-to-many relationships between these elements of varied granularity.}
\textbf{The second is sorting out the complex relationships between code, outputs, and text, which may exhibit many-to-many relationships at different granularities.}
\yanna{Even when readers are aware that relevant information exists, having a clear view of all the relationships and locating them is challenging.
The relationships between code, outputs, and text often involve many-to-many mappings at different levels of granularity.}
In many-to-many relationships, a single text cell can be related to one or more code or output cells, and several text cells can relate to a single code or output cell. 
Furthermore, these elements can be related at different granularities.
For example, a specific output cell may be related to a segment of a text cell that summarizes findings from multiple output cells.
In such cases, reader\final{s} might find it challenging to \yanna{locate the related content and identify the specific connections} because they need to process and filter extensive information.
\textbf{The third is retaining \yanna{and synthesizing} the relationship information.}
Even when relationships of \yanna{code}, outputs, and text are clarified, readers may still struggle to retain and synthesize them for understanding.
When the notebook is long or \final{contains} unrelated cells interspersed among relevant information, readers have to scroll back and forth to collect, remember, and \final{process} relevant information. 

Existing research on making computational notebooks more readable has predominantly focused on isolated aspects, such as code organization~\cite{head2019managing, weinman2021fork, wang2022stickyland} and descriptive text generation~\cite{lin2023inksight,wang2022documentation}.
These approaches, while valuable, often overlook the interplay between \yanna{code}, outputs, and text.
Recent efforts have aimed to address this gap by implementing structured outlines within computational notebooks~\cite{chattopadhyay2023make, rule2018aiding}. 
Nonetheless, they mainly applied headlines to structure computational notebooks and neglected a substantial portion of the narrative content in other types, which constitutes approximately 68\% of the descriptive text~\cite{wang2022documentation}.
Furthermore, previous studies have reported readers' desire for more granular explanations that extend beyond headers to single code statement explanations~\cite{chattopadhyay2023make}.

\revise{To fill this gap, our work proposes \yanna{a tool}, \tool, that clearly presents relationships \final{between the text, code, and outputs to help readers identify and navigate related information in notebooks, thereby making them easier to understand.}
\final{Rather than specifying these relationships, which we leave for future work, this work investigates whether explicitly presenting these relationships in a reading-optimized design can improve the readability of computational notebooks.}
\yanna{To inform \tool's design, we first conducted a formative study involving semi-structured interviews with six participants.
The study revealed challenges in \final{identifying and navigating} the relationships between text, \yanna{code}, and outputs.
Based on these findings, we derived five key design requirements to guide the design of ~\tool.
}
% a formative study involving semi-structured interviews with six participants,
% revealing the challenges of understanding notebooks regarding the individual elements of text, \yanna{code}, and outputs, as well as their relationships.
% , including dirupted focus due to interweaving structure 
% Based on these insights and the literature review, our work focused on relationships among elements and derived five design requirements.
% These requirements guided the design of ~\tool.
\yanna{Before designing \tool to present relationships, we first designed a relationship space to identify the types of relationships to present and proposed their corresponding formulation.}
This space defines 27 distinct relationship types, considering both content type (\yanna{code}, outputs, and text) and their granularity (ranging from a whole cell to specific segments). 
It includes relationships such as a text segment describing ``the song with \final{the} longest duration'' and a corresponding visual output segment indicating ``the tallest bar in a bar chart''.
Based on user feedback from the formative study that emphasized the importance of the text-code and text-output relationships,
% we developed \tool, a plugin to enhance notebook comprehension by clarifying these relationships.
we developed \tool, a plugin to \final{make notebooks easier to understand} by clarifying these relationships. 
}
Specifically, \tool separates text from \yanna{code} and outputs by presenting them in two columns, aligning related descriptive text side by side with its corresponding computational elements for immediate reference, as illustrated in~\autoref{fig: teaser} (B). 
It uses explicit connection lines for cell-cell connections and subtle highlights for intricate, fine-grained relationships. 
\revise{\tool~also provides interactions for \final{navigating} relationships.}
To demonstrate the usefulness and effectiveness of \tool, we conducted a user study with 12 participants.
\yanna{
Our study shows that \tool improves task accuracy \final{by 13.6\% over the traditional notebook interface} \final{in identifying, navigating, and integrating} relevant information within computational notebooks.
}
Finally, we discussed future directions for facilitating the understanding of notebooks.
\revise{The main contributions of this paper are as follows:
\begin{itemize}
\item 
% A formative study to identify the readers' pain points and derive design requirements of understanding notebooks;
A formative study to identify the readers' pain points and derive design requirements of \final{identifying and navigating the relationships between text, code, and outputs in computational notebooks;}

\item
\yanna{A novel computational notebook plugin}, \tool, to present relationships between descriptive text and associated \yanna{code} and outputs for facilitating the understanding of notebooks;
\item
A user study to demonstrate the usefulness and effectiveness of \tool.
\end{itemize}
}
