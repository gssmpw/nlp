\section{Related Work}
\label{sub:literature}
The papers that are closest to us are the ones on information design in auctions. 
For selling a single item, \citet{bergemann2007information} show that the revenue optimal information structure is partitional. 
However, \citet{cai2024algorithmic} show that computing the optimal partition is NP-hard, and they provide a PTAS for computing the optimal partition. 
The design of optimal information structures in mechanisms has also been considered for second-price auctions \citep{Bergemann2022Optimal}, bilateral trade \citep{schottmuller2023optimal}
and non-linear pricing \citep{bergemann2022screening}.
We consider the design of optimal information structures in multi-dimensional environments and show that the ability to design information structures makes simple mechanisms competitive. 

There is an extensive study for providing approximation guarantee of simple mechanisms in auctions environments for selling a single-item \citep[e.g.,][]{bulow1989simple,hartline2009simple,yan2011mechanism,alaei2019optimal,jin2019tight,feng2019optimal,beyhaghi2021improved,feng2023simple}
or multiple items with combinatorial values \citep[e.g.,][]{chawla2010multi,chawla2010power,babaioff2020simple,hart2017approximate,cai2016duality,cai_simple_2017,cai_computing_2022, cai_simultaneous_2023,daskalakis_multi-item_2022, babaioff2017menu}. 
For combinatorial auctions, these papers focus exclusively on settings with independent item values. In contrast, \citet{hart2019selling} show that when item values are correlated, even in the case of selling just two items to a single buyer, the revenue gap between the optimal mechanism and simple mechanisms, such as item pricing, can be unbounded.
Recently, \citet{chawla2019buy} consider the buy-many model for selling multiple items with correlated valuations to a unit-demand buyer, where the buyer can make multiple purchases from the menu offered from the seller. The authors show that item pricing achieves an $O(\log m)$-approximation to the optimal revenue. 

The simplicity notion we adopted in our paper is the simplicity in the auction format, such as item pricing for selling multiple items \citep{carroll2017robustness} or linear contracts in contracting environments \citep{carroll2015robustness}.
This is different from the requirement for strategic simplicity such as dominant strategy-proof \citep{chung2007foundations}, strategy-proof based on first order reasoning \citep{borgers2019strategically} or obviously strategy-proof \citep{li2017obviously}. 

The idea that endogenous information makes simple mechanism competitive has also been observed in other auction environments where the endogeneity arises since the buyers are acquiring information optimally. 
For example, with optimal buyer learning, \citet{deb2021multi} show that bundling is optimal for selling multiple items with additive valuations under the exchangeable prior assumption, 
and \citet{li2022selling} shows that selling full information using posted pricing is a 2-approximation to the optimal.