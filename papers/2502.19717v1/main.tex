
\documentclass{article} % For LaTeX2e
\usepackage{iclr2025_conference,times}

% Optional math commands from https://github.com/goodfeli/dlbook_notation.
%%%%% NEW MATH DEFINITIONS %%%%%

\usepackage{amsmath,amsfonts,bm}
\usepackage{derivative}
% Mark sections of captions for referring to divisions of figures
\newcommand{\figleft}{{\em (Left)}}
\newcommand{\figcenter}{{\em (Center)}}
\newcommand{\figright}{{\em (Right)}}
\newcommand{\figtop}{{\em (Top)}}
\newcommand{\figbottom}{{\em (Bottom)}}
\newcommand{\captiona}{{\em (a)}}
\newcommand{\captionb}{{\em (b)}}
\newcommand{\captionc}{{\em (c)}}
\newcommand{\captiond}{{\em (d)}}

% Highlight a newly defined term
\newcommand{\newterm}[1]{{\bf #1}}

% Derivative d 
\newcommand{\deriv}{{\mathrm{d}}}

% Figure reference, lower-case.
\def\figref#1{figure~\ref{#1}}
% Figure reference, capital. For start of sentence
\def\Figref#1{Figure~\ref{#1}}
\def\twofigref#1#2{figures \ref{#1} and \ref{#2}}
\def\quadfigref#1#2#3#4{figures \ref{#1}, \ref{#2}, \ref{#3} and \ref{#4}}
% Section reference, lower-case.
\def\secref#1{section~\ref{#1}}
% Section reference, capital.
\def\Secref#1{Section~\ref{#1}}
% Reference to two sections.
\def\twosecrefs#1#2{sections \ref{#1} and \ref{#2}}
% Reference to three sections.
\def\secrefs#1#2#3{sections \ref{#1}, \ref{#2} and \ref{#3}}
% Reference to an equation, lower-case.
\def\eqref#1{equation~\ref{#1}}
% Reference to an equation, upper case
\def\Eqref#1{Equation~\ref{#1}}
% A raw reference to an equation---avoid using if possible
\def\plaineqref#1{\ref{#1}}
% Reference to a chapter, lower-case.
\def\chapref#1{chapter~\ref{#1}}
% Reference to an equation, upper case.
\def\Chapref#1{Chapter~\ref{#1}}
% Reference to a range of chapters
\def\rangechapref#1#2{chapters\ref{#1}--\ref{#2}}
% Reference to an algorithm, lower-case.
\def\algref#1{algorithm~\ref{#1}}
% Reference to an algorithm, upper case.
\def\Algref#1{Algorithm~\ref{#1}}
\def\twoalgref#1#2{algorithms \ref{#1} and \ref{#2}}
\def\Twoalgref#1#2{Algorithms \ref{#1} and \ref{#2}}
% Reference to a part, lower case
\def\partref#1{part~\ref{#1}}
% Reference to a part, upper case
\def\Partref#1{Part~\ref{#1}}
\def\twopartref#1#2{parts \ref{#1} and \ref{#2}}

\def\ceil#1{\lceil #1 \rceil}
\def\floor#1{\lfloor #1 \rfloor}
\def\1{\bm{1}}
\newcommand{\train}{\mathcal{D}}
\newcommand{\valid}{\mathcal{D_{\mathrm{valid}}}}
\newcommand{\test}{\mathcal{D_{\mathrm{test}}}}

\def\eps{{\epsilon}}


% Random variables
\def\reta{{\textnormal{$\eta$}}}
\def\ra{{\textnormal{a}}}
\def\rb{{\textnormal{b}}}
\def\rc{{\textnormal{c}}}
\def\rd{{\textnormal{d}}}
\def\re{{\textnormal{e}}}
\def\rf{{\textnormal{f}}}
\def\rg{{\textnormal{g}}}
\def\rh{{\textnormal{h}}}
\def\ri{{\textnormal{i}}}
\def\rj{{\textnormal{j}}}
\def\rk{{\textnormal{k}}}
\def\rl{{\textnormal{l}}}
% rm is already a command, just don't name any random variables m
\def\rn{{\textnormal{n}}}
\def\ro{{\textnormal{o}}}
\def\rp{{\textnormal{p}}}
\def\rq{{\textnormal{q}}}
\def\rr{{\textnormal{r}}}
\def\rs{{\textnormal{s}}}
\def\rt{{\textnormal{t}}}
\def\ru{{\textnormal{u}}}
\def\rv{{\textnormal{v}}}
\def\rw{{\textnormal{w}}}
\def\rx{{\textnormal{x}}}
\def\ry{{\textnormal{y}}}
\def\rz{{\textnormal{z}}}

% Random vectors
\def\rvepsilon{{\mathbf{\epsilon}}}
\def\rvphi{{\mathbf{\phi}}}
\def\rvtheta{{\mathbf{\theta}}}
\def\rva{{\mathbf{a}}}
\def\rvb{{\mathbf{b}}}
\def\rvc{{\mathbf{c}}}
\def\rvd{{\mathbf{d}}}
\def\rve{{\mathbf{e}}}
\def\rvf{{\mathbf{f}}}
\def\rvg{{\mathbf{g}}}
\def\rvh{{\mathbf{h}}}
\def\rvu{{\mathbf{i}}}
\def\rvj{{\mathbf{j}}}
\def\rvk{{\mathbf{k}}}
\def\rvl{{\mathbf{l}}}
\def\rvm{{\mathbf{m}}}
\def\rvn{{\mathbf{n}}}
\def\rvo{{\mathbf{o}}}
\def\rvp{{\mathbf{p}}}
\def\rvq{{\mathbf{q}}}
\def\rvr{{\mathbf{r}}}
\def\rvs{{\mathbf{s}}}
\def\rvt{{\mathbf{t}}}
\def\rvu{{\mathbf{u}}}
\def\rvv{{\mathbf{v}}}
\def\rvw{{\mathbf{w}}}
\def\rvx{{\mathbf{x}}}
\def\rvy{{\mathbf{y}}}
\def\rvz{{\mathbf{z}}}

% Elements of random vectors
\def\erva{{\textnormal{a}}}
\def\ervb{{\textnormal{b}}}
\def\ervc{{\textnormal{c}}}
\def\ervd{{\textnormal{d}}}
\def\erve{{\textnormal{e}}}
\def\ervf{{\textnormal{f}}}
\def\ervg{{\textnormal{g}}}
\def\ervh{{\textnormal{h}}}
\def\ervi{{\textnormal{i}}}
\def\ervj{{\textnormal{j}}}
\def\ervk{{\textnormal{k}}}
\def\ervl{{\textnormal{l}}}
\def\ervm{{\textnormal{m}}}
\def\ervn{{\textnormal{n}}}
\def\ervo{{\textnormal{o}}}
\def\ervp{{\textnormal{p}}}
\def\ervq{{\textnormal{q}}}
\def\ervr{{\textnormal{r}}}
\def\ervs{{\textnormal{s}}}
\def\ervt{{\textnormal{t}}}
\def\ervu{{\textnormal{u}}}
\def\ervv{{\textnormal{v}}}
\def\ervw{{\textnormal{w}}}
\def\ervx{{\textnormal{x}}}
\def\ervy{{\textnormal{y}}}
\def\ervz{{\textnormal{z}}}

% Random matrices
\def\rmA{{\mathbf{A}}}
\def\rmB{{\mathbf{B}}}
\def\rmC{{\mathbf{C}}}
\def\rmD{{\mathbf{D}}}
\def\rmE{{\mathbf{E}}}
\def\rmF{{\mathbf{F}}}
\def\rmG{{\mathbf{G}}}
\def\rmH{{\mathbf{H}}}
\def\rmI{{\mathbf{I}}}
\def\rmJ{{\mathbf{J}}}
\def\rmK{{\mathbf{K}}}
\def\rmL{{\mathbf{L}}}
\def\rmM{{\mathbf{M}}}
\def\rmN{{\mathbf{N}}}
\def\rmO{{\mathbf{O}}}
\def\rmP{{\mathbf{P}}}
\def\rmQ{{\mathbf{Q}}}
\def\rmR{{\mathbf{R}}}
\def\rmS{{\mathbf{S}}}
\def\rmT{{\mathbf{T}}}
\def\rmU{{\mathbf{U}}}
\def\rmV{{\mathbf{V}}}
\def\rmW{{\mathbf{W}}}
\def\rmX{{\mathbf{X}}}
\def\rmY{{\mathbf{Y}}}
\def\rmZ{{\mathbf{Z}}}

% Elements of random matrices
\def\ermA{{\textnormal{A}}}
\def\ermB{{\textnormal{B}}}
\def\ermC{{\textnormal{C}}}
\def\ermD{{\textnormal{D}}}
\def\ermE{{\textnormal{E}}}
\def\ermF{{\textnormal{F}}}
\def\ermG{{\textnormal{G}}}
\def\ermH{{\textnormal{H}}}
\def\ermI{{\textnormal{I}}}
\def\ermJ{{\textnormal{J}}}
\def\ermK{{\textnormal{K}}}
\def\ermL{{\textnormal{L}}}
\def\ermM{{\textnormal{M}}}
\def\ermN{{\textnormal{N}}}
\def\ermO{{\textnormal{O}}}
\def\ermP{{\textnormal{P}}}
\def\ermQ{{\textnormal{Q}}}
\def\ermR{{\textnormal{R}}}
\def\ermS{{\textnormal{S}}}
\def\ermT{{\textnormal{T}}}
\def\ermU{{\textnormal{U}}}
\def\ermV{{\textnormal{V}}}
\def\ermW{{\textnormal{W}}}
\def\ermX{{\textnormal{X}}}
\def\ermY{{\textnormal{Y}}}
\def\ermZ{{\textnormal{Z}}}

% Vectors
\def\vzero{{\bm{0}}}
\def\vone{{\bm{1}}}
\def\vmu{{\bm{\mu}}}
\def\vtheta{{\bm{\theta}}}
\def\vphi{{\bm{\phi}}}
\def\va{{\bm{a}}}
\def\vb{{\bm{b}}}
\def\vc{{\bm{c}}}
\def\vd{{\bm{d}}}
\def\ve{{\bm{e}}}
\def\vf{{\bm{f}}}
\def\vg{{\bm{g}}}
\def\vh{{\bm{h}}}
\def\vi{{\bm{i}}}
\def\vj{{\bm{j}}}
\def\vk{{\bm{k}}}
\def\vl{{\bm{l}}}
\def\vm{{\bm{m}}}
\def\vn{{\bm{n}}}
\def\vo{{\bm{o}}}
\def\vp{{\bm{p}}}
\def\vq{{\bm{q}}}
\def\vr{{\bm{r}}}
\def\vs{{\bm{s}}}
\def\vt{{\bm{t}}}
\def\vu{{\bm{u}}}
\def\vv{{\bm{v}}}
\def\vw{{\bm{w}}}
\def\vx{{\bm{x}}}
\def\vy{{\bm{y}}}
\def\vz{{\bm{z}}}

% Elements of vectors
\def\evalpha{{\alpha}}
\def\evbeta{{\beta}}
\def\evepsilon{{\epsilon}}
\def\evlambda{{\lambda}}
\def\evomega{{\omega}}
\def\evmu{{\mu}}
\def\evpsi{{\psi}}
\def\evsigma{{\sigma}}
\def\evtheta{{\theta}}
\def\eva{{a}}
\def\evb{{b}}
\def\evc{{c}}
\def\evd{{d}}
\def\eve{{e}}
\def\evf{{f}}
\def\evg{{g}}
\def\evh{{h}}
\def\evi{{i}}
\def\evj{{j}}
\def\evk{{k}}
\def\evl{{l}}
\def\evm{{m}}
\def\evn{{n}}
\def\evo{{o}}
\def\evp{{p}}
\def\evq{{q}}
\def\evr{{r}}
\def\evs{{s}}
\def\evt{{t}}
\def\evu{{u}}
\def\evv{{v}}
\def\evw{{w}}
\def\evx{{x}}
\def\evy{{y}}
\def\evz{{z}}

% Matrix
\def\mA{{\bm{A}}}
\def\mB{{\bm{B}}}
\def\mC{{\bm{C}}}
\def\mD{{\bm{D}}}
\def\mE{{\bm{E}}}
\def\mF{{\bm{F}}}
\def\mG{{\bm{G}}}
\def\mH{{\bm{H}}}
\def\mI{{\bm{I}}}
\def\mJ{{\bm{J}}}
\def\mK{{\bm{K}}}
\def\mL{{\bm{L}}}
\def\mM{{\bm{M}}}
\def\mN{{\bm{N}}}
\def\mO{{\bm{O}}}
\def\mP{{\bm{P}}}
\def\mQ{{\bm{Q}}}
\def\mR{{\bm{R}}}
\def\mS{{\bm{S}}}
\def\mT{{\bm{T}}}
\def\mU{{\bm{U}}}
\def\mV{{\bm{V}}}
\def\mW{{\bm{W}}}
\def\mX{{\bm{X}}}
\def\mY{{\bm{Y}}}
\def\mZ{{\bm{Z}}}
\def\mBeta{{\bm{\beta}}}
\def\mPhi{{\bm{\Phi}}}
\def\mLambda{{\bm{\Lambda}}}
\def\mSigma{{\bm{\Sigma}}}

% Tensor
\DeclareMathAlphabet{\mathsfit}{\encodingdefault}{\sfdefault}{m}{sl}
\SetMathAlphabet{\mathsfit}{bold}{\encodingdefault}{\sfdefault}{bx}{n}
\newcommand{\tens}[1]{\bm{\mathsfit{#1}}}
\def\tA{{\tens{A}}}
\def\tB{{\tens{B}}}
\def\tC{{\tens{C}}}
\def\tD{{\tens{D}}}
\def\tE{{\tens{E}}}
\def\tF{{\tens{F}}}
\def\tG{{\tens{G}}}
\def\tH{{\tens{H}}}
\def\tI{{\tens{I}}}
\def\tJ{{\tens{J}}}
\def\tK{{\tens{K}}}
\def\tL{{\tens{L}}}
\def\tM{{\tens{M}}}
\def\tN{{\tens{N}}}
\def\tO{{\tens{O}}}
\def\tP{{\tens{P}}}
\def\tQ{{\tens{Q}}}
\def\tR{{\tens{R}}}
\def\tS{{\tens{S}}}
\def\tT{{\tens{T}}}
\def\tU{{\tens{U}}}
\def\tV{{\tens{V}}}
\def\tW{{\tens{W}}}
\def\tX{{\tens{X}}}
\def\tY{{\tens{Y}}}
\def\tZ{{\tens{Z}}}


% Graph
\def\gA{{\mathcal{A}}}
\def\gB{{\mathcal{B}}}
\def\gC{{\mathcal{C}}}
\def\gD{{\mathcal{D}}}
\def\gE{{\mathcal{E}}}
\def\gF{{\mathcal{F}}}
\def\gG{{\mathcal{G}}}
\def\gH{{\mathcal{H}}}
\def\gI{{\mathcal{I}}}
\def\gJ{{\mathcal{J}}}
\def\gK{{\mathcal{K}}}
\def\gL{{\mathcal{L}}}
\def\gM{{\mathcal{M}}}
\def\gN{{\mathcal{N}}}
\def\gO{{\mathcal{O}}}
\def\gP{{\mathcal{P}}}
\def\gQ{{\mathcal{Q}}}
\def\gR{{\mathcal{R}}}
\def\gS{{\mathcal{S}}}
\def\gT{{\mathcal{T}}}
\def\gU{{\mathcal{U}}}
\def\gV{{\mathcal{V}}}
\def\gW{{\mathcal{W}}}
\def\gX{{\mathcal{X}}}
\def\gY{{\mathcal{Y}}}
\def\gZ{{\mathcal{Z}}}

% Sets
\def\sA{{\mathbb{A}}}
\def\sB{{\mathbb{B}}}
\def\sC{{\mathbb{C}}}
\def\sD{{\mathbb{D}}}
% Don't use a set called E, because this would be the same as our symbol
% for expectation.
\def\sF{{\mathbb{F}}}
\def\sG{{\mathbb{G}}}
\def\sH{{\mathbb{H}}}
\def\sI{{\mathbb{I}}}
\def\sJ{{\mathbb{J}}}
\def\sK{{\mathbb{K}}}
\def\sL{{\mathbb{L}}}
\def\sM{{\mathbb{M}}}
\def\sN{{\mathbb{N}}}
\def\sO{{\mathbb{O}}}
\def\sP{{\mathbb{P}}}
\def\sQ{{\mathbb{Q}}}
\def\sR{{\mathbb{R}}}
\def\sS{{\mathbb{S}}}
\def\sT{{\mathbb{T}}}
\def\sU{{\mathbb{U}}}
\def\sV{{\mathbb{V}}}
\def\sW{{\mathbb{W}}}
\def\sX{{\mathbb{X}}}
\def\sY{{\mathbb{Y}}}
\def\sZ{{\mathbb{Z}}}

% Entries of a matrix
\def\emLambda{{\Lambda}}
\def\emA{{A}}
\def\emB{{B}}
\def\emC{{C}}
\def\emD{{D}}
\def\emE{{E}}
\def\emF{{F}}
\def\emG{{G}}
\def\emH{{H}}
\def\emI{{I}}
\def\emJ{{J}}
\def\emK{{K}}
\def\emL{{L}}
\def\emM{{M}}
\def\emN{{N}}
\def\emO{{O}}
\def\emP{{P}}
\def\emQ{{Q}}
\def\emR{{R}}
\def\emS{{S}}
\def\emT{{T}}
\def\emU{{U}}
\def\emV{{V}}
\def\emW{{W}}
\def\emX{{X}}
\def\emY{{Y}}
\def\emZ{{Z}}
\def\emSigma{{\Sigma}}

% entries of a tensor
% Same font as tensor, without \bm wrapper
\newcommand{\etens}[1]{\mathsfit{#1}}
\def\etLambda{{\etens{\Lambda}}}
\def\etA{{\etens{A}}}
\def\etB{{\etens{B}}}
\def\etC{{\etens{C}}}
\def\etD{{\etens{D}}}
\def\etE{{\etens{E}}}
\def\etF{{\etens{F}}}
\def\etG{{\etens{G}}}
\def\etH{{\etens{H}}}
\def\etI{{\etens{I}}}
\def\etJ{{\etens{J}}}
\def\etK{{\etens{K}}}
\def\etL{{\etens{L}}}
\def\etM{{\etens{M}}}
\def\etN{{\etens{N}}}
\def\etO{{\etens{O}}}
\def\etP{{\etens{P}}}
\def\etQ{{\etens{Q}}}
\def\etR{{\etens{R}}}
\def\etS{{\etens{S}}}
\def\etT{{\etens{T}}}
\def\etU{{\etens{U}}}
\def\etV{{\etens{V}}}
\def\etW{{\etens{W}}}
\def\etX{{\etens{X}}}
\def\etY{{\etens{Y}}}
\def\etZ{{\etens{Z}}}

% The true underlying data generating distribution
\newcommand{\pdata}{p_{\rm{data}}}
\newcommand{\ptarget}{p_{\rm{target}}}
\newcommand{\pprior}{p_{\rm{prior}}}
\newcommand{\pbase}{p_{\rm{base}}}
\newcommand{\pref}{p_{\rm{ref}}}

% The empirical distribution defined by the training set
\newcommand{\ptrain}{\hat{p}_{\rm{data}}}
\newcommand{\Ptrain}{\hat{P}_{\rm{data}}}
% The model distribution
\newcommand{\pmodel}{p_{\rm{model}}}
\newcommand{\Pmodel}{P_{\rm{model}}}
\newcommand{\ptildemodel}{\tilde{p}_{\rm{model}}}
% Stochastic autoencoder distributions
\newcommand{\pencode}{p_{\rm{encoder}}}
\newcommand{\pdecode}{p_{\rm{decoder}}}
\newcommand{\precons}{p_{\rm{reconstruct}}}

\newcommand{\laplace}{\mathrm{Laplace}} % Laplace distribution

\newcommand{\E}{\mathbb{E}}
\newcommand{\Ls}{\mathcal{L}}
\newcommand{\R}{\mathbb{R}}
\newcommand{\emp}{\tilde{p}}
\newcommand{\lr}{\alpha}
\newcommand{\reg}{\lambda}
\newcommand{\rect}{\mathrm{rectifier}}
\newcommand{\softmax}{\mathrm{softmax}}
\newcommand{\sigmoid}{\sigma}
\newcommand{\softplus}{\zeta}
\newcommand{\KL}{D_{\mathrm{KL}}}
\newcommand{\Var}{\mathrm{Var}}
\newcommand{\standarderror}{\mathrm{SE}}
\newcommand{\Cov}{\mathrm{Cov}}
% Wolfram Mathworld says $L^2$ is for function spaces and $\ell^2$ is for vectors
% But then they seem to use $L^2$ for vectors throughout the site, and so does
% wikipedia.
\newcommand{\normlzero}{L^0}
\newcommand{\normlone}{L^1}
\newcommand{\normltwo}{L^2}
\newcommand{\normlp}{L^p}
\newcommand{\normmax}{L^\infty}

\newcommand{\parents}{Pa} % See usage in notation.tex. Chosen to match Daphne's book.

\DeclareMathOperator*{\argmax}{arg\,max}
\DeclareMathOperator*{\argmin}{arg\,min}

\DeclareMathOperator{\sign}{sign}
\DeclareMathOperator{\Tr}{Tr}
\let\ab\allowbreak


\usepackage{hyperref}
\usepackage{url}



% ====================== customized ======================
\usepackage{comment}
\usepackage{xcolor}
\usepackage{graphicx}
\usepackage{duckuments}
\usepackage{caption}
\usepackage{subcaption}
\usepackage[capitalize,noabbrev]{cleveref}
\usepackage{algorithm}
\usepackage{algorithmic}
\usepackage{booktabs}
\usepackage{multirow}
\usepackage{enumitem}
\usepackage{soul}
\usepackage[normalem]{ulem}               % to striketrhourhg text
\usepackage{threeparttable}
\usepackage{bbm}
\usepackage{amsthm}
\setlist[itemize]{leftmargin=0.2in}
\newcommand\redout{\bgroup\markoverwith
{\textcolor{red}{\rule[0.5ex]{2pt}{0.8pt}}}\ULon}
\newtheorem{theorem}{Theorem}
\newtheorem{remark}{Remark}
\newtheorem{lemma}{Lemma}

\title{Exponential Topology-enabled Scalable Communication in Multi-agent Reinforcement Learning}

% Authors must not appear in the submitted version. They should be hidden
% as long as the \iclrfinalcopy macro remains commented out below.
% Non-anonymous submissions will be rejected without review.

\author{
% Antiquus S.~Hippocampus, Natalia Cerebro \& Amelie P. Amygdale \thanks{ Use footnote for providing further information
% about author (webpage, alternative address)---\emph{not} for acknowledging
% funding agencies.  Funding acknowledgements go at the end of the paper.} \\
% Department of Computer Science\\
% Cranberry-Lemon University\\
% Pittsburgh, PA 15213, USA \\
% \texttt{\{hippo,brain,jen\}@cs.cranberry-lemon.edu} \\
% \And
% Ji Q. Ren \& Yevgeny LeNet \\
% Department of Computational Neuroscience \\
% University of the Witwatersrand \\
% Joburg, South Africa \\
% \texttt{\{robot,net\}@wits.ac.za} \\
% \AND
Xinran Li\textsuperscript{1,2}
\quad Xiaolu Wang\textsuperscript{3}\thanks{Corresponding author.}
\quad Chenjia Bai\textsuperscript{2}
\quad Jun Zhang\textsuperscript{1} \\
\textsuperscript{1}The Hong Kong University of Science and Technology \\
\textsuperscript{2}Institute of Artificial Intelligence (TeleAI), China Telecom \\
\textsuperscript{3}Software Engineering Institute, East China Normal University \\
% Address \\
\texttt{xinran.li@connect.ust.hk, xiaoluwang@sei.ecnu.edu.cn} \\
\texttt{baicj@chinatelecom.cn, eejzhang@ust.hk}
% \AND
% Coauthor \\
% Affiliation \\
% Address \\
% \texttt{email}
}

% The \author macro works with any number of authors. There are two commands
% used to separate the names and addresses of multiple authors: \And and \AND.
%
% Using \And between authors leaves it to \LaTeX{} to determine where to break
% the lines. Using \AND forces a linebreak at that point. So, if \LaTeX{}
% puts 3 of 4 authors names on the first line, and the last on the second
% line, try using \AND instead of \And before the third author name.

\newcommand{\fix}{\marginpar{FIX}}
\newcommand{\new}{\marginpar{NEW}}

\iclrfinalcopy % Uncomment for camera-ready version, but NOT for submission.
\begin{document}


\maketitle
% We introduce ExpoComm, a scalable communication protocol that leverages exponential topologies for efficient information dissemination among many agents in large-scale multi-agent reinforcement learning.
\vskip -0.2in
\begin{abstract}
% The abstract paragraph should be indented 1/2~inch (3~picas) on both left and
% right-hand margins. Use 10~point type, with a vertical spacing of 11~points.
% The word \textsc{Abstract} must be centered, in small caps, and in point size 12. Two
% line spaces precede the abstract. The abstract must be limited to one
% paragraph.
% In cooperative multi-agent reinforcement learning, well-designed communication protocols can effectively enhance cooperation and improve task performance. Despite the extensive efforts to optimize the communication protocol , how to 
% how to extend such ideas to large-scale multi-agent systems has not yet been investigated. To address this research gap, we aim to design a scalable communication protocol for MARL, as effective information exchange is vital in these systems due to severe partial observability.
% The challenges posed by a large number of agents can be summarized 
% which is imperative given the prevalence of large-scale systems in real-world applications. 
% In cooperative multi-agent reinforcement learning (MARL), well-designed communication protocols can effectively facilitate consensus among agents and therefore enhance task performance. Despite significant efforts to optimize these protocols for improved efficacy and reduced cost, few studies address their scalability, which is critical for large-scale systems commonly found in real-world applications.
% In such systems, effective communication is even more crucial due to the heightened challenges of partial observability compared to smaller-scale setups. To address this research gap, we aim to design a scalable communication protocol for MARL. 
% Unlike previous methods that focus on selecting the most useful pairwise communication links—a task that becomes increasingly challenging as the number of agents grows—we adopt an alternative approach and look into the communication topology from a global perspective. In particular, we explore utilizing exponential topologies to enable rapid information dissemination among all agents. By leveraging the small diameter and size properties of exponential graphs, we introduce ExpoComm as a scalable solution for MARL communication. To fully unlock the potential of exponential graphs as communication topologies, we employ memory-based message processors and auxiliary tasks to ground messages, allowing them to reflect global information and assist in decision-making. Extensive experiments on large-scale cooperative benchmarks, including MAgent and large-scale Infrastructure Management Planning, demonstrate the superior performance and strong zero-shot transferability of ExpoComm compared to existing communication strategies. 

% In cooperative multi-agent reinforcement learning (MARL), well-designed communication protocols can effectively facilitate consensus among agents and therefore enhance task performance. Despite significant efforts to optimize these protocols for improved efficacy and reduced cost, few studies address their scalability, which is critical for large-scale systems commonly found in real-world applications. In such systems, effective communication is even more vital due to the heightened challenges of partial observability compared to smaller-scale setups. To address this gap, we propose a scalable communication protocol for MARL. Unlike previous methods that focus on selecting optimal pairwise communication links—an increasingly complex task as the number of agents grows—we adopt a global perspective on communication topology. Specifically, we utilize exponential topologies to enable rapid information dissemination among agents. By leveraging the small diameter and size properties of exponential graphs, we introduce ExpoComm as a scalable solution for MARL communication. To fully unlock the potential of exponential graphs as communication topologies, we employ memory-based message processors and auxiliary tasks to ground messages, ensuring they reflect global information and aid in decision-making. Extensive experiments on large-scale cooperative benchmarks, including MAgent and Infrastructure Management Planning, demonstrate the superior performance and robust zero-shot transferability of ExpoComm compared to existing strategies. 

In cooperative multi-agent reinforcement learning (MARL), well-designed communication protocols can effectively facilitate consensus among agents, thereby enhancing task performance. Moreover, in large-scale multi-agent systems commonly found in real-world applications, effective communication plays an even more critical role due to the escalated challenge of partial observability compared to smaller-scale setups. In this work, we endeavor to develop a scalable communication protocol for MARL. Unlike previous methods that focus on selecting optimal pairwise communication links—a task that becomes increasingly complex as the number of agents grows—we adopt a global perspective on communication topology design. Specifically, we propose utilizing the exponential topology to enable rapid information dissemination among agents by leveraging its small-diameter and small-size properties. This approach leads to a scalable communication protocol, named ExpoComm. To fully unlock the potential of exponential graphs as communication topologies, we employ memory-based message processors and auxiliary tasks to ground messages, ensuring that they reflect global information and benefit decision-making. Extensive experiments on large-scale cooperative benchmarks, including MAgent and Infrastructure Management Planning, demonstrate the superior performance and robust zero-shot transferability of ExpoComm compared to existing communication strategies. The
code is publicly available at \url{https://github.com/LXXXXR/ExpoComm}.\looseness=-1

\end{abstract}


\section{Introduction}
% \textcolor{blue}{1.5 page}
% MARL -> communication in MARL -> scalability to many agents -> 

% few focus on this scenario 
% fully-connected and learn the connectivity 

% -> challenges: too expensive, not effective 

% focus too much on local and communication cost scale 

% quadratically with the number of agents 


% one-step 
% multi-step


Cooperative multi-agent reinforcement learning (MARL) has recently emerged as a promising approach for complex decision-making tasks across diverse real-world applications, such as resource allocation~\citep{marl_dis}, package delivery~\citep{marl_delivery}, autonomous driving~\citep{MARL_autonomous_driving}, robot control~\citep{marl_robot_swamy2020scaled}, and 
% infrastructure management planning
infrastructure management planning~\citep{benchmark_IMP}. Under the widely adopted centralized training and decentralized execution (CTDE)
paradigm~\citep{ctde_kraemer2016multi,ctde_lyu2021contrasting}, algorithms like MADDPG~\citep{MADDPG}, COMA~\citep{COMA}, MATD3~\citep{MATD3}, QMIX~\citep{QMIX}, and MAPPO~\citep{MAPPO} have achieved notable success.

To enhance agent collaboration in partially observable scenarios, communication mechanisms have been incorporated into multi-agent systems (MASs) to assist in decentralized decision-making~\citep{CommNet}.
% facilitate better cooperation. 
Enabling information exchange during execution helps MARL algorithms to address non-stationarity and partial observability prevalent in these environments. Building upon this foundation, researchers have devoted efforts to designing effective communication protocols, focusing on three core considerations:
1) \textit{whom} the agents should communicate with~\citep{i2c,Commformer};
2) \textit{when} communication should occur~\citep{ETC,schedNet};
and 3) \textit{how} the agents should design and utilize the communication messages effectively~\citep{TarMAC,MASIA}.
By leveraging tools such as attention and graph neural networks (GNNs), learnable and adaptive communication mechanisms have significantly advanced MARL performance. \looseness=-1


% Nevertheless, most of the previous works primarily focus on developing sophisticated communication protocols , paying less attention to the actual applicability 
% tested small-scale MASs with the number of agents less than ten \textcolor{blue}{cite papers here}, 
% leaving scalability of the communication strategies are largely overlooked. 
% Motivated by this research gap, this work focuses on the scalability of communication protocol in MARL. The challenges 
% Motivated by this research gap, this work aims to develop a scalable communication protocol in MARL that is both highly effective and low-cost.
% To address this research gap of scalability in MARL communication protocols, the challenges are two-fold: First, with many agent in the system, how . Second, how to manage the cost of the communication during execution?
% Therefore, to enable feasible applications of MARL communication protocols to real-world MASs, the research gap in scalability must be addressed. 
% To scale the current method, which involves learn the connectivity among agents, to many-agent systems, two challenges lie in the way: First, it becomes 
% In such many-agent systems, 
% Second, \underline{with a fixed sparsity of the connectivity graph typical of such methods} {\color{blue}(not very clear)}, the communication cost overall scales quadratically with \redout{respect to} the number of agents, which can be \redout{infeasibly} {\color{red}prohibitively} large in \redout{large-scale} {\color{red}many-agent} systems.
% with the increasing number in the systems 
Despite considerable success, most existing communication strategies are designed for small-scale MASs~\citep{MADDPG,smac,facmac_mamujoco} and may struggle as systems scale to dozens or even hundreds of agents, which are ubiquitous in real-world applications~\citep{population_survey,traffic_scalibility,inventory_scalibility,large_scale_networked_MARL}. 
In these \textit{many-agent} systems, existing methods that learn pairwise connectivity among agents falter for two reasons: First, these methods often require agents to receive messages only from ``useful'' peers. However, identifying these peers becomes increasingly challenging as the number of agents grows, potentially compromising the effectiveness of communication protocols~\citep{MASIA}.
Second, the overhead of these methods scales poorly. Specifically, training memory consumption quickly becomes prohibitively large, as shown in our empirical evaluation, and the communication overhead during execution scales quadratically with the number of agents, which is infeasible for many-agent systems. \looseness=-1

This motivates a fundamental rethinking of scalable MARL communication: Can we adopt a global perspective and design an overall topology that propagates information among all agents effectively and at low cost, rather than relying on finding task-specific pairwise connectivity? In this vein, we propose an exponential topology-enabled communication protocol, termed \textit{ExpoComm}, as a scalable solution for MARL communication. Unlike previous works that seek to identify useful communication links at each timestep, ExpoComm draws inspiration from graph theory and leverages the small-diameter property of exponential topologies to ensure effective communication by facilitating message flow across all agents within a limited number of timesteps. The inherent sparsity (small size) of exponential topologies allows ExpoComm's communication cost to scale (near-)linearly with the number of agents. Moreover, to fully leverage the small-size and small-diameter properties of exponential graphs for efficient information dissemination, we employ memory-based blocks for message processing and auxiliary tasks to ground messages, ensuring that they effectively reflect global information. Extensive experiments across twelve scenarios on large-scale benchmarks, including MAgent~\citep{benchmark_Magent} and Infrastructure Management Planning (IMP)~\citep{benchmark_IMP}, demonstrate the superior performance of ExpoComm compared to baseline algorithms when handling large numbers of agents up to a hundred. Additionally, owing to its global perspective without pairwise reliance, ExpoComm exhibits remarkable zero-shot transferability to larger numbers of agents during test time. \looseness=-1

% To overcome the above limitations, we propose an exponential graph-based communication protocol, termed \textit{ExpoComm}, as a scalable solution in MARL communication. Unlike the previous work that tries to identify the useful communication links in each timestep, ExpoComm takes inspiration from classic graph literature and ensures the effectiveness of communication by finding the message flow to traverse all agents within limited timesteps. In the meanwhile, with the \redout{sparse} {\color{red}sparsity} nature of exponential graphs, the communication cost of ExpoComm only scales \redout{sub-quadratically or linearly} {\color{red}(nearly) linearly} with \redout{respect to} the number of agents in the system\redout{, depending on the particular variant of exponential graphs utilized}. Moreover, to fully leverage the properties of exponential graphs to traverse all agents within a few timesteps, we propose to use memory-based blocks to process the messages and an auxiliary loss to restore the global information to ground messages. With extensive experiments across twelve scenarios on large-scale benchmarks MAgent~\citep{benchmark_Magent} and large-scale \redout{Infrastructure Management Planning (IMP-MARL)} {\color{red}IMP-based MARL}~\citep{benchmark_IMP}, we demonstrate the superior performance of ExpoComm compared to baseline algorithms when dealing with dozens and hundreds of agents. Furthermore, we also showcase the remarkable zero-shot \redout{transfer ability} {\color{red}transferability} of ExpoComm to more agents during test time, crediting to the global perspective taken by ExpoComm without relying on pairwise relationships among agents.




% Subsequent works took a ste vast different communication protocols mainly focusing on three aspects: 
% To facilitate cooperation among agents and 







\section{Related Work} \label{sec: related_work}
% \textcolor{blue}{1 page}
\paragraph{Communication in MASs}
% many communication work -> many progress for better performance -> some consider realistic concerns -> scalability -> primary issue is the communication topology -> fixed / dynamic (distance-based) -> 
% individually controlled (sender/receiver)
% GNN-learnt  

% First introduced by \citet{CommNet,DIAL_RIAL}, communication among agents in MARL has been an active research area due to its great potential to enhance cooperation and improve task performance. Owing to the high flexibility of communication protocols, finding the most effective protocols tailored to the MARL paradigm is a challenging task~\citep{MARL_comm_survey}.
% % To this end, extensive later works adopt the straightforward approach to learn the communication protocol in an end-to-end fashion, which is termed ``learning to communicate''. Notably, . M
% To this end, extensive works take the learning perspective and optimize the communication components~\citep{BiCNet}, such as message generators, message aggregators, and connectivity among agents, through end-to-end training. From the sender side, ToM2C~\citep{ToM2C} and MAIC~\citep{MAIC} enhanced the message generation process by utilizing teammate modeling. CACL~\citep{CACL} focuses on the decentralized training paradigm and proposes to learn communication encoding with contrastive learning. From the receiver side, TarMAC~\citep{TarMAC}, G2ANet~\citep{G2ANet} and MASIA~\citep{MASIA} improve message aggregation by employing attention-based aggregation strategies.

Communication among agents in MARL was first introduced by \citet{CommNet,DIAL_RIAL} and has since become an active research area due to its potential to enhance cooperation and improve task performance. The flexibility of communication protocols makes finding effective solutions for the MARL paradigm challenging~\citep{MARL_comm_survey}.
To address this difficulty, many studies have focused on optimizing communication components, such as message generators, message aggregators, and connectivity among agents, through end-to-end training~\citep{BiCNet}. From the sender side, ToM2C~\citep{ToM2C} and MAIC~\citep{MAIC} enhance message generation through teammate modeling, while CACL~\citep{CACL} uses contrastive learning techniques to learn communication encoding in a decentralized training paradigm. From the receiver side, TarMAC~\citep{TarMAC}, G2ANet~\citep{G2ANet}, and MASIA~\citep{MASIA} improve message aggregation using attention-based strategies. \looseness=-1

Recently, researchers have addressed challenges posed by real-world communication systems. Notably, NDQ~\citep{NDQ} and TMC~\citep{TMC} reduce communication costs by crafting succinct messages, while ATOC~\citep{atoc}, IC3~\citep{ic3}, I2C~\citep{i2c}, and CommFormer~\citep{Commformer} manage overhead by pruning unnecessary communication links. Additionally, \citet{noisy_channel} propose a stochastic encoding/decoding scheme to handle noisy channels, and DACOM \citep{DACOM} introduces delay-aware communication to account for the high latency of wireless channels.

Despite these advancements, scalability in communication mechanisms has been largely overlooked, often due to the quadratically increasing communication cost associated with fully-connected graphs as the number of agents grows. Although few works explicitly address the scalability issue, efforts to design communication topologies among agents offer potential solutions. These can be categorized into fully-connected, rule-based, and learned topologies. 
Early works~\citep{CommNet,DIAL_RIAL,BiCNet} typically adopt fully-connected topologies to demonstrate communication benefits, but at the cost of high bandwidth requirements. Later on, to reduce the overall communication overhead, \citet{DGN} and \citet{neighbor_graph} restrict communication to nearby neighbors based on distance, while NeurComm~\citep{NeurComm} limits communication to neighboring agents in networked MASs. In spite of achieving significant performance gains, their further applicability may be limited since they require extra information beyond local observation to determine the communication topology. In contrast, learned topology methods assume no such requirements and offer high flexibility. In particular, ATOC~\citep{atoc}, IC3~\citep{ic3}, I2C~\citep{i2c} locally deploy gates for agents to decide if they should engage in communication. However, these methods may result in uncontrollable overall communication costs due to individual control schemes. Alternatively, MAGIC~\citep{MAGIC} utilizes graph attention mechanisms to learn the communication topology, while CommFormer~\citep{Commformer} extends the idea and enables control over the overall communication sparsity. Although effective in small-scale MASs, peer-wise connectivity becomes increasingly difficult to learn in large-scale MASs, and high sparsity may impair performance, as discussed by \citet{Commformer}.
% In particular, VBC~\citep{VBC}, ETC~\citep{ETC} and MBC~\citep{MBC_sparse_comm} 

% Our proposed ExpoComm incorporates a rule-based topology, supplementing the aforementioned progress on MARL communication by explicitly addressing the scalability of communication.
Our proposed ExpoComm, which incorporates rule-based topologies for rapid information dissemination among all agents, complements existing efforts in MAS communication by explicitly addressing scalability challenges. \looseness=-1

% Subsequent works then made concrete progress 

% {\color{red}
% \textbf{Our Contributions.}
% \begin{enumerate}[label=\arabic*),leftmargin=14pt]
%     \item 
    
%     \item  
    
%     \item 
% \end{enumerate}}
\paragraph{Exponential Graphs}
% \textcolor{blue}{Should I put this paragraph above the last one?}
Exponential graphs are a class of graph topologies that exhibit strong scalability properties with respect to the number of nodes. They have been primarily used in distributed learning to periodically synchronize model updates across devices. \citet{exp_graph_decentralized_learning_first} investigate exponential graphs with gossip algorithms and achieve high consensus rates for decentralized learning. Follow-up works~\citep{SlowMo,exp_graph_decentralized_exact_avg,expG_consensus,ExpG_largebs} build upon this topology, optimizing model weight update algorithms and providing empirical evidence and theoretical guarantees for the effectiveness of exponential graphs. Beyond distributed learning, exponential graphs also have applications in chip design~\citep{wang2014rpnoc,ExpG_chip_design}. Overall, exponential graphs demonstrate efficient information dissemination across many nodes, making them a promising candidate topology for achieving scalable communication in MARL. \looseness=-1



\section{Scalable Communication with Exponential Graph in MARL} \label{sec: method}

In this section, we propose ExpoComm, which leverages exponential graphs as communication topologies among agents in MARL to enable scalable communication. We structure the following subsections to address three key questions: 1) Why and how should exponential graphs be adapted for agent communication? 2) How can the corresponding neural network architecture be designed to effectively utilize the messages transmitted through these topologies? 3) How can messages propagated among agents be grounded to ensure their usefulness?

In \cref{sec: expG_as_comm_topologies}, we outline the requirements for scalable communication: effective information dissemination among agents and low communication overhead. We translate these requirements into the challenge of identifying topologies with small diameters and sizes, key properties of exponential topologies. In \cref{sec: network_design}, we discuss how memory-based message processors can enable meaningful message encoding, leveraging the small-diameter property over multiple timesteps within exponential topologies. In \cref{sec: training_details}, we adopt a global perspective to ground messages using a global state reconstruction auxiliary task and contrastive learning, as ExpoComm aims to facilitate message flow across the entire graph rather than focusing on local features.

\subsection{Exponential Graph as the Communication Topology} \label{sec: expG_as_comm_topologies}


\subsubsection{Problem Setting} 
In this work, we consider a fully cooperative partially observable multi-agent task, which can be modeled as a decentralized partially observable Markov decision process (Dec-POMDP)~\citep{pomdp_oliehoek2016concise}. The Dec-POMDP is defined by a tuple $\mathcal{M} = \langle \mathcal{S}, A, P, R, \Omega, O, N, \gamma \rangle$ with $N$ being the number of agents and $\gamma \in (0, 1]$ being the discount factor. 
At each timestep $t$, with the global observation $s^t \in \mathcal{S}$, agent $i$ receives a local observation $ o_i^t \in \Omega$ and then communicates with other agents. Upon receiving the messages from other agents, agent $i$ then selects an action $a_i^t \in A$ based on its local policy $\pi_i$. These individual actions collectively form a joint action $\boldsymbol{a}^t \in A^N$, leading to a transition to the next global observation $s^{t+1} \sim P(s^{t+1}| s^t, \boldsymbol{a}^t)$ and inducing a global reward $r^t = R(s^t, \boldsymbol{a}^t)$. The team objective is to learn the policies that maximize the expected discounted cumulative return $G_t = \sum_t \gamma^t r^t$. 

\subsubsection{Communication Topologies} \label{sec: comm_graph_desiderata}

To design an effective and scalable communication protocol in many-agent systems, it is essential to determine whom to communicate with, i.e., to construct the communication topology so that communication is both beneficial for decision-making and cost-effective. While previous work~\citep{Commformer} assumes a static communication topology, we adopt a more flexible, time-varying directed graph $\textstyle \gG^t = \langle \gV, \gE^t \rangle$, where node $\textstyle v_i \in \gV$ denotes agent $i$ and edge $e_{i \rightarrow j}^t \in \gE^t$ indicates a communication link from agent $i$ to agent $j$ at timestep $t$. 

From a graph perspective, we consider the following desiderata for the communication topology:

\begin{enumerate}[label=$\bullet$,leftmargin=14pt]
    \item \textbf{Small graph diameter for fast information dissemination}: Formally defined as $\textstyle \text{diameter}(\gG^t) =  \max_{v_i, v_j \in \gV} d(v_i, v_j)$ with $d(v_i, v_j)$ representing the shortest path distance from $v_i$ to $v_j$, the graph diameter indicates how quickly messages travel through the graph. Since communication aids multi-agent decision-making by providing the locally observant agents with global information and alleviating the non-stationarity, a graph with a small diameter can expedite message exchange and is therefore desirable.
    % \item \textbf{Small size for low communication overhead}: Formally defined as $\textstyle \lvert \gE^t \rvert$, the size of a graph denotes the total number of edges, corresponding to communication overhead in an MAS. Given the high hardware requirement for communication modules and the potential delays induced by densely connected communication topologies, we prefer graphs with a small size in many-agent settings. 
    \item \textbf{Small size for low communication overhead}: Formally defined as $\textstyle \lvert \gE^t \rvert$, the size of a graph denotes the total number of edges, corresponding to the number of communication links in an MAS. We assume that any message transmission incurs the same overhead, therefore the total overhead scales with the number of links. Given the high hardware requirement for communication modules and the potential delays induced by densely connected communication topologies, we prefer graphs with a small size in many-agent settings. 
\end{enumerate}



\subsubsection{Exponential Graphs} \label{sec: exp_graph}

Based on the desiderata above for the communication topologies, we draw inspiration from graph literature and choose exponential graphs~\citep{exp_graph_decentralized_learning_first,exp_graph_decentralized_exact_avg} as a promising candidate for communication topology in many-agent systems. Below, we introduce two variants of exponential graphs and demonstrate their small-diameter and small-size properties through an illustrative example.

\begin{figure}[t]
  \centering
  \begin{subfigure}[t]{.21\textwidth}
    \centering
    \includegraphics[width=\textwidth]{figs/stat_exp_v2.pdf}
    \caption{Static exponential graph.}
    \label{fig: demo_static_exp_graph}
   \end{subfigure}
   \hfill
  \begin{subfigure}[t]{.67\textwidth}
    \centering
    \includegraphics[width=\textwidth]{figs/one_peer_exp_v2.pdf}
    \caption{One-peer exponential graph. $k$ is an integer.}
    \label{fig: demo_one_peer_exp_graph}
   \end{subfigure}
  \vskip -0.1in
  \caption{Illustration of exponential graphs with $N=8$.}
  \label{fig: demo_exp_graph}
  \vskip -0.2in
\end{figure}

\paragraph{Static Exponential Graph}
Assuming a randomly sequential ordering of agents $\textstyle 0, 1, \ldots,  N-1$ and the corresponding adjacency matrix $\textstyle E \in \{0,1 \}^{N \times N}$, in the static exponential graph, each agent communicates with peers that are $\textstyle 2^0, 2^1, \ldots, 2^{\lfloor \log_2{(N-1)} \rfloor}$ hops away, which is illustrated by \cref{fig: demo_static_exp_graph}. Formally, we have
\begin{align}
    \displaystyle
    E^{t(\text{stat})}_{ij}  =
    \begin{cases}
    1 & \text{if} \log_2\left( (j-i) \bmod N \right) \text{is an integer or } i = j\\
    0 & \text{otherwise}
    \end{cases}
    .
    \label{eq: static_exp_adj}
\end{align}


\paragraph{One-peer Exponential Graph}
In the one-peer exponential graph, each agent iterates through different peers that are $\textstyle 2^0, 2^1, \ldots, 2^{\lfloor \log_2{(N-1)} \rfloor}$ hops away, which is illustrated by \cref{fig: demo_one_peer_exp_graph}. Formally, we have
\begin{align}
    \displaystyle
    E^{t(\text{one-peer})}_{ij}  =
    \begin{cases}
    1 & \text{if} \log_2\left( (j-i) \bmod N \right) = t \bmod \lfloor \log_2{(N-1)} \rfloor \text{or } i = j \\
    0 & \text{otherwise}
    \end{cases}
    .
    \label{eq: one_peer_exp_adj}
\end{align}


\begin{figure}[t]
  \centering
  \begin{subfigure}[t]{.48\textwidth}
    \centering
    \includegraphics[width=\textwidth]{figs/demo/dense_dist_based.pdf}
    \caption{Distance-based graph with $\textstyle \lvert \gE^t \rvert = N \cdot \log_2N$.}
    \label{fig: demo_dense_dist}
   \end{subfigure}
   \begin{subfigure}[t]{.48\textwidth}
    \centering
    \includegraphics[width=\textwidth]{figs/demo/sparse_dist_based.pdf}
    \caption{Distance-based graph with $\textstyle \lvert \gE^t \rvert = N$.}
    \label{fig: demo_sparse_dist}
   \end{subfigure}
   \begin{subfigure}[t]{.48\textwidth}
    \centering
    \includegraphics[width=\textwidth]{figs/demo/dense_ER.pdf}
    \caption{Erdős–Rényi graph with $\textstyle \lvert \gE^t \rvert = N \cdot \log_2N$.}
    \label{fig: demo_dense_ER}
   \end{subfigure}
   \begin{subfigure}[t]{.48\textwidth}
    \centering
    \includegraphics[width=\textwidth]{figs/demo/sparse_ER.pdf}
    \caption{Erdős–Rényi graph with $\textstyle \lvert \gE^t \rvert = N$.}
    \label{fig: demo_sparse_ER}
   \end{subfigure}
   % \hfill
  \begin{subfigure}[t]{.48\textwidth}
    \centering
    \includegraphics[width=\textwidth]{figs/demo/static_exp.pdf}
    \caption{Static exponential graph with $\textstyle \lvert \gE^t \rvert = N \cdot \log_2N$.}
    \label{fig: demo_static_exp}
   \end{subfigure}
   \begin{subfigure}[t]{.48\textwidth}
    \centering
    \includegraphics[width=\textwidth]{figs/demo/one_peer_exp.pdf}
    \caption{One-peer exponential graph with $\textstyle \lvert \gE^t \rvert = N$.}
    \label{fig: demo_one_peer_exp}
   \end{subfigure}
  \caption{A toy example to illustrate the message dissemination with different graph topologies. We demonstrate how the messages, represented by red dots, travel from a random agent to other agents over time, following different graph structures. In distance-based graphs~\citep{DGN}, agents are connected to top-$K$ nearest neighbors. In Erdős–Rényi random graphs~\citep{ER_graph}, the adjacency matrices are sampled uniformly from all the graphs satisfying the diameter and size conditions. In exponential graphs, the adjacency matrices follow \cref{eq: static_exp_adj,eq: one_peer_exp_adj}. \looseness=-1}
  \label{fig: toy_graph_diameter}
  \vskip -0.2in
\end{figure}


\paragraph{Properties}

Using the adjacency matrices defined above, we verify that the graph diameter for both static and one-peer exponential graphs is $\lceil \log_2{(N-1)} \rceil$ (see \cref{supp: theory} for details). As discussed in \cref{sec: comm_graph_desiderata}, a small diameter facilitates efficient information dissemination, especially when the number of agents $N$ is large. 

Regarding communication costs, static exponential graphs have a size of $N \cdot \lfloor \log_2{(N-1)} \rfloor$, while one-peer exponential graphs have a size of $N$. Notably, the size of one-peer exponential graphs scales linearly with the number of agents, meaning the overall communication overhead also scales linearly.

To illustrate these properties, we provide a toy example in \cref{fig: toy_graph_diameter}. We visualize the message dissemination abilities of different communication topologies under varying communication budgets. In this example, with $N=256$ agents, graph sizes (communication budgets) $\textstyle \lvert \gE^t \rvert$ are set to $N \cdot \log_2N$ and $N$, respectively.
In \cref{fig: toy_graph_diameter}, we observe that for each communication topology, reducing the graph sizes (as shown in \cref{fig: demo_sparse_dist,fig: demo_sparse_ER,fig: demo_one_peer_exp}) typically slows down dissemination speed due to increased graph diameters. This illustrates a trade-off between graph diameter and size, reflecting the trade-off between communication performance and overhead in many-agent systems. Sparser graphs with smaller sizes result in slower message dissemination but lighter communication overhead. However, exponential topologies strike a balance in this trade-off, demonstrating strong information diffusion even with a minimal communication budget of $N$.
% From \cref{fig: toy_graph_diameter}, we observe that under communication budget $N \cdot \log_2N$, both Erdős–Rényi random graphs and static exponential graphs demonstrate fast message dissemination speed. Under low communication budgets $N$, only one-peer exponential graphs show strong abilities to diffuse information among all agents. 
% Nevertheless, exponential topologies find a sweet spot in such trade-off and show strong abilities to diffuse information among all agents, even under an extremely low communication budget of $N$.

Based on these observations, we conclude that exponential topologies are well-suited for many-agent communication because: 1) In exponential topologies, any two agents can exchange messages in at most $\lceil \log_2{(N-1)} \rceil$ timesteps, ensuring timely information exchange in decentralized decision-making problems. 2) The communication overhead scales nearly linearly with the number of agents, which is crucial for many-agent systems. 3) With a rule-based topology, exponential graphs are easy to deploy and adapt to systems with varying numbers of agents, as empirically verified in \cref{sec: results}.


\subsection{Neural Network Architecture Design} \label{sec: network_design}
\begin{figure}[t]
  \centering
    \centering
    \includegraphics[width=0.8\textwidth]{figs/arch_v6.pdf}
    \caption{Neural network architecture for ExpoComm. For the static exponential topologies, attention blocks are used for message aggregation. For the one-peer exponential topologies, RNN blocks are used for message aggregation. }
    \label{fig: network_arch}
    \vskip -0.2in
\end{figure}

With exponential graphs serving as the communication topology in ExpoComm, we we elaborate on the neural network architecture to help agents utilize received messages for better decision-making. The overall architecture is illustrated in \cref{fig: network_arch}. 
ExpoComm is based on the concept of facilitating message flow across all agents within a certain timeframe, where the graph diameter indicates the length of such timeframe. To capitalize on the small graph diameter of exponential graphs, the message-processing module at each agent should ideally preserve all information received within $\textstyle \text{diameter}(\gG^t)$ timesteps. However, preserving all messages across multiple timesteps is not memory-efficient, so we employ sequential neural networks, such as attention blocks and recurrent neural networks (RNNs), for message processing.


\subsection{Training and Execution Details} \label{sec: training_details}
% \textcolor{blue}{need to revise this paragraph because the update on my method}
Following the QMIX~\citep{QMIX} algorithm, we update the network parameters $\theta$ with the objective of minimizing the temporal difference (TD) error loss:
\begin{equation}
    \displaystyle
    \mathcal{L}^{\text{TD}}(\theta) = \mathbb{E}_{(s^t, \bm{o}^t,  \bm{a}^t, r^t, s^{t+1}, \bm{o}^{t+1}) \sim \mathcal{D}} \left[\left(y^{tot} - Q_{tot}(s^t, \boldsymbol{o}^t, \boldsymbol{a}^t; \theta) \right)^2\right],
\end{equation}
where $\textstyle y^{tot} = r + \gamma \max_{\boldsymbol{a}} Q_{tot}(s^{t+1}, \boldsymbol{o}^{t+1}, \boldsymbol{a}; \theta^-)$ and $\theta^-$ represents the parameters of the target network as in DQN.

% Nevertheless, solely with the MARL training objective, 
% Moreover, since the communication 
However, communication inevitably enlarges the policy space, making it more challenging to find the optimal policy relying solely on the MARL training objective~\citep{comm_aux_loss}. To facilitate learning meaningful messages, we introduce auxiliary tasks to restore global information from local messages. From a message perspective, we aim for it to traverse among agents over multiple timesteps, accumulating new information along the way, and ultimately reflecting global information useful for decision-making.

\paragraph{Message grounding with the global state}
In scenarios where the global state is available during training, the auxiliary loss is given by the prediction error of the current global state: 
\begin{align}
    \displaystyle
    \mathcal{L}^{\text{Aux}}_{\text{pred}}(\theta, \phi) = \mathbb{E}_{(s^t, \bm{o}^t) \sim \mathcal{D}} \left[ s^t - f(m_i^t; \phi))^2 \right],
    \label{eq: aux_pred}
\end{align}
where the learnable auxiliary network for prediction $\textstyle f(\cdot; \phi)$ is used to ground the messages and can be discarded after training.

\paragraph{Message grounding without the global state} Alternatively, when the global state is unavailable during training, we use contrastive learning for meaningful message encoding, similar to \citet{CACL}. Specifically, we treat messages from different agents at the same timestep as positive pairs and messages with intervals larger than $\textstyle \text{diameter}(\gG^t)$ as negative pairs, encouraging local messages $m_i^t$ to reflect the current global latent state. The corresponding auxiliary loss is given as the InfoNCE loss~\citep{InfoNCE}: 
\begin{align}
    \displaystyle
    \mathcal{L}^{\text{Aux}}_{\text{cont}}(\theta) = - \mathbb{E}_{i, j, t, t'} \left[ \log{\frac{\text{exp} \left( g(m_i^t) \cdot g(m_j^t) / \tau \right) }{\sum_{m \in \mathcal{M}} \text{exp} \left( g(m_i^t) \cdot g(m) / \tau \right)}} \right],
    \label{eq: aux_cont}
\end{align}
where $i$ is uniformly sampled from $\{0, \ldots, N\}$, $j$ is uniformly sampled from $\{0, \ldots, N : j \neq i\}$, $\mathcal{M} = \{m_k^{t'}: k \in \{0, \ldots, N\} ,t' \notin [t-\text{diameter}(\gG^t), t+\text{diameter}(\gG^t) ] \} \cup \{m_j^t\}$ with $|\mathcal{M}| = M+1$ and $m$ is uniformly sampled from $\mathcal{M}$. $g(\cdot)$ is the normalization function, $M$ is the hyperparameter indicating the number of negative pairs and $\tau$ is the temperature hyperparameter.
The overall training loss is:
\begin{align}
    \displaystyle
    \mathcal{L}^{\text{TD}}(\theta) = \mathcal{L}^{\text{TD}}(\theta) + \alpha \cdot \mathcal{L}^{\text{Aux}}(\theta; \phi),
    \label{eq: training_loss}
\end{align}
where $\alpha$ is the hyperparameter and $\mathcal{L}^{\text{Aux}}(\cdot)$ is the auxiliary loss defined by \cref{eq: aux_pred} or \cref{eq: aux_cont}, depending on whether global information is available during training. The training and execution procedures are summarized in \cref{algo: ExpoComm_training}.

\begin{algorithm}[t]
\caption{Training and Execution Procedure of ExpoComm}
\begin{algorithmic}[1]

\STATE {\bfseries Init:} Network parameters $\theta$, $\phi$, $\mathcal{D} = \emptyset$, $\text{step} = 0$, $\theta^- =\theta$
\WHILE{$\text{step} < \text{step}_\text{max}$}
    \STATE $t=0$. Reset the environment. 

    % \STATE {\bfseries Init:} Local history  
    \FOR{$t = 1, 2, ..., \text{episode\_limit}$} 
        \STATE \textit{// Decentralized execution at agent $i$}
        \STATE Update local history $h^t_i$ based on current observation $o_i^t$ and previous history $h^{t-1}_i$
        \STATE Update agent $i$'s message $m_i^t$ based on previous local message $m^{t-1}_i$ and previously received messages $\left[ m_j^{t-1} \right]_{E_{ij}^{t-1}=1}$
        \STATE \textit{// Communication}
        \STATE Send message $m_i^t$ to peers $\{j \mid E_{ij}^t=1\}$ \hfill\COMMENT{$\triangleright$ \cref{eq: static_exp_adj,eq: one_peer_exp_adj}}
        \STATE \textit{// Action, which can happen concurrently with communication}
        \STATE Sample action $a_i^t$ based on current history $h^t_i$ and current message $m_i^t$ 
        \STATE Interact with the environment $(s^{t+1}, \boldsymbol{o}^{t+1}, r^t) = \text{env}.\text{step}(\boldsymbol{a}^t)$ 

        \STATE Save the experience $\mathcal{D} = \mathcal{D} \cup (s^t, \boldsymbol{o}^t, \boldsymbol{a}^t, r^t, s^{t+1}, \boldsymbol{o}^{t+1})$
    \ENDFOR
    \STATE At some interval, update network parameters $\theta$, $\phi$ and $\theta^-$ \hfill\COMMENT{$\triangleright$ \cref{eq: training_loss}}
\ENDWHILE
\STATE {\bfseries Output:} Policy networks parameters $\theta$
% \RETURN{} Parameters for exploitation networks $\zeta$
\end{algorithmic}
\label{algo: ExpoComm_training}
\end{algorithm}


\section{Experimental Results}
% \textcolor{blue}{3.5 pages}

In this section, we evaluate ExpoComm on two large-scale multi-agent benchmarks: MAgent~\citep{benchmark_Magent} and Infrastructure Management Planning (IMP)~\citep{benchmark_IMP}. All experiments are averaged over five random seeds and
the shaded areas represent the $95\%$ confidence interval. Details on network architecture and the training hyperparameters are available in \cref{supp: net_arch_hyperparams}.

\subsection{Experimental Setups} \label{sec: experiment_setups}
\paragraph{Environment descriptions}
In this section, We test ExpoComm and baselines across twelve scenarios in two large-scale benchmarks, with the number of agents ranging from 20 to 100. Specifically, MAgent is a particle-based gridworld environment representative of the typical MARL gaming benchmarks. To expand the variety of tasks, we also include the IMP benchmark, with tasks oriented from real-world applications. More details regarding the environment settings are provided in \cref{supp: env_details}. \looseness=-1


\paragraph{Communication Budgets}
Denoting the number of agents each agent communicates to by $K$, for each baseline in each scenario, we test two communication budgets: $K = \lceil\log_2N\rceil$ and $K = 1$, where $N$ is the number of agents in the systems. 

\paragraph{Baselines}
In the following, we compare our proposed ExpoComm with four baselines: (i) \textit{IDQN/QMIX}~\citep{QMIX}: Base algorithms without communication; (ii) \textit{DGN+TarMAC}~\citep{DGN,TarMAC}: Position-based communication topologies in which agents communicate with their nearest neighbors and use TarMAC structure to aggregate messages; (iii) \textit{ER}: ExpoComm with the exponential graph topologies replaced by random graph communication topologies following the Erdős–Rényi model; (iv) \textit{CommFormer}~\citep{Commformer}: Learned communication topologies using GNN. 
For ExpoComm, we use the static exponential graph variant for $K = \lceil\log_2N\rceil$ and the one-peer exponential graph variant for $K = 1$. For DGN+TarMAC, agents communicate to top-$K$ nearest neighbors. For ER, communication graphs are sampled uniformally from all the $K$-in-regular directed graphs. CommFormer uses constraints with varying sparsity levels for different communication budgets. Official implementations of these baselines are utilized wherever available; otherwise, we closely follow the descriptions from their respective papers, integrating them into the base algorithms. More implementation details can be found in \cref{supp: implement_details}. \looseness=-1
% For each baseline in each scenario, we test the cases with budgets for communication links of $N \cdot \log_2N$ and $N$, where $N$ is the number of agents in the systems. More details on implementation can be found in \cref{supp: implement_details}.

\subsection{Results} \label{sec: results}

% ,often out performing its counterpart method ExpoComm with static 

\begin{figure}[t]
\centering
% \subfigure[]{
% \includegraphics[width=0.7\textwidth]{Styles/figs/results/0515_legend_cropped.pdf}
%     }8
\raisebox{-\height}{\includegraphics[width=0.82\textwidth]{figs/results/0928_magent_legend.pdf}}
\par
\begin{subfigure}[t]{.32\textwidth}
    \centering
    \includegraphics[width=\textwidth]{figs/results/1120_magent_adv_pursuit25.pdf}
    \caption{AdversarialPursuit w/ 25 agents}
\end{subfigure}
\begin{subfigure}[t]{.32\textwidth}
    \centering
    \includegraphics[width=\textwidth]{figs/results/1120_magent_adv_pursuit45.pdf}
    \caption{AdversarialPursuit w/ 45 agents}
\end{subfigure}
\begin{subfigure}[t]{.32\textwidth}
    \centering
    \includegraphics[width=\textwidth]{figs/results/1120_magent_adv_pursuit61.pdf}
    \caption{AdversarialPursuit w/ 61 agents}
\end{subfigure}
\begin{subfigure}[t]{.32\textwidth}
    \centering
    \includegraphics[width=\textwidth]{figs/results/1120_magent_battle20.pdf}
    \caption{Battle w/ 20 agents}
\end{subfigure}
\begin{subfigure}[t]{.32\textwidth}
    \centering
    \includegraphics[width=\textwidth]{figs/results/1120_magent_battle42.pdf}
    \caption{Battle w/ 42 agents}
\end{subfigure}
\begin{subfigure}[t]{.32\textwidth}
    \centering
    \includegraphics[width=\textwidth]{figs/results/1120_magent_battle64.pdf}
    \caption{Battle w/ 64 agents}
\end{subfigure}
\caption{Performance comparison with baselines on MAgent tasks. Solid lines represent communication budgets of $K = 1$, while dashed lines represent budgets of $K = \lceil\log_2N\rceil$. Runs requiring more than 40 GB of GPU memory are excluded due to extreme training costs compared to other methods. \looseness=-1}
\label{fig:results_magent}
\vskip -0.25in
\end{figure}


\begin{table}[t]
\centering
\begin{threeparttable}
\caption{Performance comparison with baselines on IMP tasks. Results are reported as the mean and standard deviation of the percentage of normalized discounted rewards relative to expert-based heuristic policies, following \citet{benchmark_IMP}, with details in \cref{supp: env_details}. The best-performing method is indicated in \textbf{bold}, and the second best is \underline{underlined}.}
\label{tab:results_imp}
\begin{tabular}{lccccc}
\toprule
\multirow{2}{*}{Scenario} & \multicolumn{1}{c}{QMIX} & \multicolumn{2}{c}{ER} & \multicolumn{2}{c}{ExpoComm} \\
\cmidrule(lr){2-2} \cmidrule(lr){3-4} \cmidrule(lr){5-6}
& $K=0$ & $K=1$ & $K=\lceil\log_2N\rceil$ & $K=1$ & $K=\lceil\log_2N\rceil$ \\
\midrule
\multicolumn{6}{c}{$N=50$} \\
\midrule
Uncorrelated   & $26.42\,\scalebox{0.8}{($3.43$)}$     & $24.91\,\scalebox{0.8}{($3.77$)}$  & $26.62\,\scalebox{0.8}{($2.03$)}$  & $\underline{27.31\,\scalebox{0.8}{($2.26$)}}$  & $\mathbf{28.26\,\scalebox{0.8}{($2.51$)}}$ \\
Correlated     & $24.81\,\scalebox{0.8}{($4.16$)}$     & $34.63\,\scalebox{0.8}{($9.72$)}$  & $34.76\,\scalebox{0.8}{($5.07$)}$  & $\mathbf{43.82\,\scalebox{0.8}{($6.33$)}}$  & $\underline{40.01\,\scalebox{0.8}{($3.19$)}}$ \\
OWF            & $62.45\,\scalebox{0.8}{($3.46$)}$     & $62.99\,\scalebox{0.8}{($3.02$)}$  & $61.70\,\scalebox{0.8}{($4.62$)}$  & $\underline{64.66\,\scalebox{0.8}{($0.26$)}}$  & $\mathbf{65.19\,\scalebox{0.8}{($0.51$)}}$ \\
\midrule
\multicolumn{6}{c}{$N=100$} \\
\midrule
Uncorrelated  & $12.86\,\scalebox{0.8}{($6.88$)}$     & $21.94\,\scalebox{0.8}{($5.97$)}$  & $18.36\,\scalebox{0.8}{($12.92$)}$ & $\underline{27.34\,\scalebox{0.8}{($13.32$)}}$ & $\mathbf{27.81\,\scalebox{0.8}{($5.71$)}}$ \\
Correlated    & $-40.20\,\scalebox{0.8}{($96.35$)}$ & $-65.14\,\scalebox{0.8}{($65.08$)}$ & $9.84\,\scalebox{0.8}{($32.27$)}$ & $\mathbf{19.17\,\scalebox{0.8}{($23.94$)}}$ & $\underline{17.25\,\scalebox{0.8}{($22.70$)}}$ \\
OWF           & $65.55\,\scalebox{0.8}{($0.53$)}$     & $\mathbf{66.70\,\scalebox{0.8}{($0.50$)}}$    & $65.92\,\scalebox{0.8}{($0.87$)}$    & $65.26\,\scalebox{0.8}{($1.34$)}$    & $\underline{66.23\,\scalebox{0.8}{($0.38$)}}$ \\
\bottomrule
\end{tabular}
\begin{tablenotes}\footnotesize
\item[1] DGN+TarMAC is not suitable for this benchmark because it requires the physical positions of agents, which are not available in this environment.
\item[2] Methods that require more than 40 GB GPU memory are excluded from comparison due to the extreme training cost compared to other methods.
\end{tablenotes}
\end{threeparttable}
\end{table}

\paragraph{Benchmark results} We present the comparative performance of ExpoComm and baselines in MAgent and IMP environments with \cref{fig:results_magent} and \cref{tab:results_imp}, respectively. Overall, ExpoComm demonstrates superior performance in these large-scale benchmarks under both communication budgets, underscoring the scalability and robustness of ExpoComm strategies. 
Notably, the one-peer version of ExpoComm achieves the best performance in most scenarios, despite communication costs that only grow linearly with the number of agents. This makes it the most suitable method for handling large-scale MARL communication problems under very low communication budgets. Additional visualization results to illustrate the learned policies are provided in \cref{supp: vis_results}.


\begin{figure}
    \centering
    \includegraphics[width=\linewidth]{figs/results/0929_transfer_s.pdf}
    \caption{Zero-shot transfer results on \texttt{Battle} scenario. The subtitle ``X to Y'' indicates that methods are trained with X agents and tested with Y agents. Filled bars represent communication budgets of $K = \lceil\log_2N\rceil$, while hatched bars represent budgets of $K = 1$. Baseline CommFormer is not excluded in this experiment because it learns a fixed peer-to-peer communication topology among agents in a specific scenario and it is non-trivial to transfer such topology to scenarios with different numbers of agents. \looseness-1}
    \label{fig:results_transfer}
    \vskip -0.2in
\end{figure}

\paragraph{Zero-shot transfer}
Similar to the experimental settings suggested by \citet{ToM2C}, we test the zero-shot transfer ability of our proposed ExpoComm and the baseline methods, reporting the results in \cref{fig:results_transfer}. Specifically, we train the agent policies and their corresponding communication policies in scenarios with smaller numbers of agents and directly test these policies against each other in larger agent scenarios in the competitive task \texttt{Battle}. We test each pair of methods over 200 games, record the method with more wins as the winner, and summarize the results in \cref{fig:results_transfer}. We observe that both ER and ExpoComm demonstrate good transfer ability compared to other baselines, with ExpoComm performing better under smaller communication budgets. The superior transfer ability of ER and ExpoComm may be attributed to the grounding of messages, which reflects global information.


\begin{figure}[t]
\centering
% \subfigure[]{
% \includegraphics[width=0.7\textwidth]{Styles/figs/results/0515_legend_cropped.pdf}
%     }
\raisebox{-\height}{\includegraphics[width=0.7\textwidth]{figs/results/0930_abla_legend.pdf}}
\par
\begin{subfigure}[t]{.32\textwidth}
    \centering
    \includegraphics[width=\textwidth]{figs/results/1120_abla_magent_adv_pursuit25.pdf}
    \caption{AdversarialPursuit w/ 25 agents}
\end{subfigure}
\begin{subfigure}[t]{.32\textwidth}
    \centering
    \includegraphics[width=\textwidth]{figs/results/1120_abla_imp_owf50.pdf}
    \caption{OWF w/ 50 agents}
\end{subfigure}
\begin{subfigure}[t]{.32\textwidth}
    \centering
    \includegraphics[width=\textwidth]{figs/results/1120_abla_imp_uc50.pdf}
    \caption{Uncorrelated w/ 50 agents}
\end{subfigure}

\caption{Ablation studies on MAgent and IMP benchmarks.}
\label{fig:results_ablation}
\vskip -0.2in
\end{figure}

\paragraph{Ablation studies}
We conduct ablation studies to assess the impact of various design elements in ExpoComm, with results presented in \cref{fig:results_ablation}. In particular, we compare ExpoComm with two ablations: (i) \textit{ExpoComm w/o mem}, in which the message generators are not memory-based as described in \cref{sec: network_design}. (ii) \textit{ExpoComm w/o aux}, which lacks the auxiliary loss term described in \cref{sec: training_details}. From the results, we see that removing the memory blocks from the message generators hinders effective message generation, especially in scenarios with strong time correlation such as \texttt{AdversarialPursuit}. Auxiliary tasks primarily aid in grounding the messages, without which the messages could lack guidance and even be detrimental to decision-making.


\paragraph{Discussion}
As discussed in \cref{sec: related_work}, existing methods suitable for large-scale multi-agent communication fall into two categories: position-based methods (e.g., DGN+TarMAC) and GNN-based methods (e.g., CommFormer). We propose a third category based on specific graph structures. In this category, ExpoComm utilizes exponential topologies, and we construct the baseline ER using Erdős–Rényi random graphs. Based on our analysis and experiments, the advantages of ExpoComm are as follows:
\begin{itemize}
    \item \textbf{Superior task performance}: Due to the small diameter of exponential graphs and memory-based message generators, ExpoComm facilitates fast message dissemination among all agents. It efficiently collects and carries local information from all agents, aiding decentralized decision-making. This advantage is supported by experiments shown in \cref{fig:results_magent} and \cref{tab:results_imp}.
    \item \textbf{Low communication costs}: The compact size of exponential graphs ensures that ExpoComm's communication costs scale (near-)linearly with the number of agents $N$, crucial for managing communication costs in multi-agent systems. Unlike position-based methods or ER, which can only control the number of in-edges or out-edges without global scheduling, ExpoComm naturally balances communication overhead across agents.
    \item \textbf{Versatile adaptability}: ExpoComm shows strong transferability across different numbers of agents, as seen in \cref{fig:results_transfer}. This is due to its global message dissemination strategies, which focus on overall communication rather than pairwise relationships, allowing it to adapt to more agents. Additionally, ExpoComm handles a wide range of tasks, regardless of the task nature or agent count. Unlike position-based methods, which may struggle with non-gridworld tasks like IMP due to assumptions about knowledge of agent locations, ExpoComm makes no such assumptions. Moreover, while learning effective pairwise communication topologies using GNNs can lead to significant GPU memory consumption, ExpoComm bypasses these challenges. It does not rely on the expensive task of learning a scenario-specific communication topology guided by the MARL task itself but instead uses a well-designed rule-based topology based on the communication desiderata analyzed in \cref{sec: comm_graph_desiderata}.

\end{itemize}

% why others fail
% the advantages of ExpoComm

\section{Conclusions}
% \textcolor{blue}{0.5 page}

In this work, we explored scalable communication strategies in MARL and introduced ExpoComm, an exponential topology-enabled communication protocol. We proposed a framework with communication topologies featuring small diameters for fast information dissemination and small graph sizes for low communication overhead. This framework is complemented by memory-based message processors and message grounding through auxiliary objectives to achieve effective global information representation. Despite requiring only (near-)linear communication costs relative to the number of agents, ExpoComm demonstrated superior performance and strong transferability on large-scale benchmarks like MAgent and IMP. This study highlights the potential for enhancing the scalability of MARL communication strategies through the explicit design of communication topologies. \looseness=-1

% \textcolor{blue}{maybe I should explicitly summarize our method again?}
% , underlying  the potential 
% Empowered by exponential topologies, our proposed ExpoComm enjoys superior MARL task performance and requires only (near) linear communication costs with respect to the number of agents. With extensive empirical evaluation under MAgent and IMP benchmarks, we have demonstrated 

% \textcolor{blue}{maybe add future works}


\begin{comment}
\section{Submission of conference papers to ICLR 2025}

ICLR requires electronic submissions, processed by
\url{https://openreview.net/}. See ICLR's website for more instructions.

If your paper is ultimately accepted, the statement {\tt
  {\textbackslash}iclrfinalcopy} should be inserted to adjust the
format to the camera ready requirements.

The format for the submissions is a variant of the NeurIPS format.
Please read carefully the instructions below, and follow them
faithfully.

\subsection{Style}

Papers to be submitted to ICLR 2025 must be prepared according to the
instructions presented here.

%% Please note that we have introduced automatic line number generation
%% into the style file for \LaTeXe. This is to help reviewers
%% refer to specific lines of the paper when they make their comments. Please do
%% NOT refer to these line numbers in your paper as they will be removed from the
%% style file for the final version of accepted papers.

Authors are required to use the ICLR \LaTeX{} style files obtainable at the
ICLR website. Please make sure you use the current files and
not previous versions. Tweaking the style files may be grounds for rejection.

\subsection{Retrieval of style files}

The style files for ICLR and other conference information are available online at:
\begin{center}
   \url{http://www.iclr.cc/}
\end{center}
The file \verb+iclr2025_conference.pdf+ contains these
instructions and illustrates the
various formatting requirements your ICLR paper must satisfy.
Submissions must be made using \LaTeX{} and the style files
\verb+iclr2025_conference.sty+ and \verb+iclr2025_conference.bst+ (to be used with \LaTeX{}2e). The file
\verb+iclr2025_conference.tex+ may be used as a ``shell'' for writing your paper. All you
have to do is replace the author, title, abstract, and text of the paper with
your own.

The formatting instructions contained in these style files are summarized in
sections \ref{gen_inst}, \ref{headings}, and \ref{others} below.

\section{General formatting instructions}
\label{gen_inst}

The text must be confined within a rectangle 5.5~inches (33~picas) wide and
9~inches (54~picas) long. The left margin is 1.5~inch (9~picas).
Use 10~point type with a vertical spacing of 11~points. Times New Roman is the
preferred typeface throughout. Paragraphs are separated by 1/2~line space,
with no indentation.

Paper title is 17~point, in small caps and left-aligned.
All pages should start at 1~inch (6~picas) from the top of the page.

Authors' names are
set in boldface, and each name is placed above its corresponding
address. The lead author's name is to be listed first, and
the co-authors' names are set to follow. Authors sharing the
same address can be on the same line.

Please pay special attention to the instructions in section \ref{others}
regarding figures, tables, acknowledgments, and references.


There will be a strict upper limit of 10 pages for the main text of the initial submission, with unlimited additional pages for citations. 

\section{Headings: first level}
\label{headings}

First level headings are in small caps,
flush left and in point size 12. One line space before the first level
heading and 1/2~line space after the first level heading.

\subsection{Headings: second level}

Second level headings are in small caps,
flush left and in point size 10. One line space before the second level
heading and 1/2~line space after the second level heading.

\subsubsection{Headings: third level}

Third level headings are in small caps,
flush left and in point size 10. One line space before the third level
heading and 1/2~line space after the third level heading.

\section{Citations, figures, tables, references}
\label{others}

These instructions apply to everyone, regardless of the formatter being used.

\subsection{Citations within the text}

Citations within the text should be based on the \texttt{natbib} package
and include the authors' last names and year (with the ``et~al.'' construct
for more than two authors). When the authors or the publication are
included in the sentence, the citation should not be in parenthesis using \verb|\citet{}| (as
in ``See \citet{Hinton06} for more information.''). Otherwise, the citation
should be in parenthesis using \verb|\citep{}| (as in ``Deep learning shows promise to make progress
towards AI~\citep{Bengio+chapter2007}.'').

The corresponding references are to be listed in alphabetical order of
authors, in the \textsc{References} section. As to the format of the
references themselves, any style is acceptable as long as it is used
consistently.

\subsection{Footnotes}

Indicate footnotes with a number\footnote{Sample of the first footnote} in the
text. Place the footnotes at the bottom of the page on which they appear.
Precede the footnote with a horizontal rule of 2~inches
(12~picas).\footnote{Sample of the second footnote}

\subsection{Figures}

All artwork must be neat, clean, and legible. Lines should be dark
enough for purposes of reproduction; art work should not be
hand-drawn. The figure number and caption always appear after the
figure. Place one line space before the figure caption, and one line
space after the figure. The figure caption is lower case (except for
first word and proper nouns); figures are numbered consecutively.

Make sure the figure caption does not get separated from the figure.
Leave sufficient space to avoid splitting the figure and figure caption.

You may use color figures.
However, it is best for the
figure captions and the paper body to make sense if the paper is printed
either in black/white or in color.
\begin{figure}[h]
\begin{center}
%\framebox[4.0in]{$\;$}
\fbox{\rule[-.5cm]{0cm}{4cm} \rule[-.5cm]{4cm}{0cm}}
\end{center}
\caption{Sample figure caption.}
\end{figure}

\subsection{Tables}

All tables must be centered, neat, clean and legible. Do not use hand-drawn
tables. The table number and title always appear before the table. See
Table~\ref{sample-table}.

Place one line space before the table title, one line space after the table
title, and one line space after the table. The table title must be lower case
(except for first word and proper nouns); tables are numbered consecutively.

\begin{table}[t]
\caption{Sample table title}
\label{sample-table}
\begin{center}
\begin{tabular}{ll}
\multicolumn{1}{c}{\bf PART}  &\multicolumn{1}{c}{\bf DESCRIPTION}
\\ \hline \\
Dendrite         &Input terminal \\
Axon             &Output terminal \\
Soma             &Cell body (contains cell nucleus) \\
\end{tabular}
\end{center}
\end{table}

\section{Default Notation}

In an attempt to encourage standardized notation, we have included the
notation file from the textbook, \textit{Deep Learning}
\cite{goodfellow2016deep} available at
\url{https://github.com/goodfeli/dlbook_notation/}.  Use of this style
is not required and can be disabled by commenting out
\texttt{math\_commands.tex}.


\centerline{\bf Numbers and Arrays}
\bgroup
\def\arraystretch{1.5}
\begin{tabular}{p{1in}p{3.25in}}
$\displaystyle a$ & A scalar (integer or real)\\
$\displaystyle \va$ & A vector\\
$\displaystyle \mA$ & A matrix\\
$\displaystyle \tA$ & A tensor\\
$\displaystyle \mI_n$ & Identity matrix with $n$ rows and $n$ columns\\
$\displaystyle \mI$ & Identity matrix with dimensionality implied by context\\
$\displaystyle \ve^{(i)}$ & Standard basis vector $[0,\dots,0,1,0,\dots,0]$ with a 1 at position $i$\\
$\displaystyle \text{diag}(\va)$ & A square, diagonal matrix with diagonal entries given by $\va$\\
$\displaystyle \ra$ & A scalar random variable\\
$\displaystyle \rva$ & A vector-valued random variable\\
$\displaystyle \rmA$ & A matrix-valued random variable\\
\end{tabular}
\egroup
\vspace{0.25cm}

\centerline{\bf Sets and Graphs}
\bgroup
\def\arraystretch{1.5}

\begin{tabular}{p{1.25in}p{3.25in}}
$\displaystyle \sA$ & A set\\
$\displaystyle \R$ & The set of real numbers \\
$\displaystyle \{0, 1\}$ & The set containing 0 and 1 \\
$\displaystyle \{0, 1, \dots, n \}$ & The set of all integers between $0$ and $n$\\
$\displaystyle [a, b]$ & The real interval including $a$ and $b$\\
$\displaystyle (a, b]$ & The real interval excluding $a$ but including $b$\\
$\displaystyle \sA \backslash \sB$ & Set subtraction, i.e., the set containing the elements of $\sA$ that are not in $\sB$\\
$\displaystyle \gG$ & A graph\\
$\displaystyle \parents_\gG(\ervx_i)$ & The parents of $\ervx_i$ in $\gG$
\end{tabular}
\vspace{0.25cm}


\centerline{\bf Indexing}
\bgroup
\def\arraystretch{1.5}

\begin{tabular}{p{1.25in}p{3.25in}}
$\displaystyle \eva_i$ & Element $i$ of vector $\va$, with indexing starting at 1 \\
$\displaystyle \eva_{-i}$ & All elements of vector $\va$ except for element $i$ \\
$\displaystyle \emA_{i,j}$ & Element $i, j$ of matrix $\mA$ \\
$\displaystyle \mA_{i, :}$ & Row $i$ of matrix $\mA$ \\
$\displaystyle \mA_{:, i}$ & Column $i$ of matrix $\mA$ \\
$\displaystyle \etA_{i, j, k}$ & Element $(i, j, k)$ of a 3-D tensor $\tA$\\
$\displaystyle \tA_{:, :, i}$ & 2-D slice of a 3-D tensor\\
$\displaystyle \erva_i$ & Element $i$ of the random vector $\rva$ \\
\end{tabular}
\egroup
\vspace{0.25cm}


\centerline{\bf Calculus}
\bgroup
\def\arraystretch{1.5}
\begin{tabular}{p{1.25in}p{3.25in}}
% NOTE: the [2ex] on the next line adds extra height to that row of the table.
% Without that command, the fraction on the first line is too tall and collides
% with the fraction on the second line.
$\displaystyle\frac{d y} {d x}$ & Derivative of $y$ with respect to $x$\\ [2ex]
$\displaystyle \frac{\partial y} {\partial x} $ & Partial derivative of $y$ with respect to $x$ \\
$\displaystyle \nabla_\vx y $ & Gradient of $y$ with respect to $\vx$ \\
$\displaystyle \nabla_\mX y $ & Matrix derivatives of $y$ with respect to $\mX$ \\
$\displaystyle \nabla_\tX y $ & Tensor containing derivatives of $y$ with respect to $\tX$ \\
$\displaystyle \frac{\partial f}{\partial \vx} $ & Jacobian matrix $\mJ \in \R^{m\times n}$ of $f: \R^n \rightarrow \R^m$\\
$\displaystyle \nabla_\vx^2 f(\vx)\text{ or }\mH( f)(\vx)$ & The Hessian matrix of $f$ at input point $\vx$\\
$\displaystyle \int f(\vx) d\vx $ & Definite integral over the entire domain of $\vx$ \\
$\displaystyle \int_\sS f(\vx) d\vx$ & Definite integral with respect to $\vx$ over the set $\sS$ \\
\end{tabular}
\egroup
\vspace{0.25cm}

\centerline{\bf Probability and Information Theory}
\bgroup
\def\arraystretch{1.5}
\begin{tabular}{p{1.25in}p{3.25in}}
$\displaystyle P(\ra)$ & A probability distribution over a discrete variable\\
$\displaystyle p(\ra)$ & A probability distribution over a continuous variable, or over
a variable whose type has not been specified\\
$\displaystyle \ra \sim P$ & Random variable $\ra$ has distribution $P$\\% so thing on left of \sim should always be a random variable, with name beginning with \r
$\displaystyle  \E_{\rx\sim P} [ f(x) ]\text{ or } \E f(x)$ & Expectation of $f(x)$ with respect to $P(\rx)$ \\
$\displaystyle \Var(f(x)) $ &  Variance of $f(x)$ under $P(\rx)$ \\
$\displaystyle \Cov(f(x),g(x)) $ & Covariance of $f(x)$ and $g(x)$ under $P(\rx)$\\
$\displaystyle H(\rx) $ & Shannon entropy of the random variable $\rx$\\
$\displaystyle \KL ( P \Vert Q ) $ & Kullback-Leibler divergence of P and Q \\
$\displaystyle \mathcal{N} ( \vx ; \vmu , \mSigma)$ & Gaussian distribution %
over $\vx$ with mean $\vmu$ and covariance $\mSigma$ \\
\end{tabular}
\egroup
\vspace{0.25cm}

\centerline{\bf Functions}
\bgroup
\def\arraystretch{1.5}
\begin{tabular}{p{1.25in}p{3.25in}}
$\displaystyle f: \sA \rightarrow \sB$ & The function $f$ with domain $\sA$ and range $\sB$\\
$\displaystyle f \circ g $ & Composition of the functions $f$ and $g$ \\
  $\displaystyle f(\vx ; \vtheta) $ & A function of $\vx$ parametrized by $\vtheta$.
  (Sometimes we write $f(\vx)$ and omit the argument $\vtheta$ to lighten notation) \\
$\displaystyle \log x$ & Natural logarithm of $x$ \\
$\displaystyle \sigma(x)$ & Logistic sigmoid, $\displaystyle \frac{1} {1 + \exp(-x)}$ \\
$\displaystyle \zeta(x)$ & Softplus, $\log(1 + \exp(x))$ \\
$\displaystyle || \vx ||_p $ & $\normlp$ norm of $\vx$ \\
$\displaystyle || \vx || $ & $\normltwo$ norm of $\vx$ \\
$\displaystyle x^+$ & Positive part of $x$, i.e., $\max(0,x)$\\
$\displaystyle \1_\mathrm{condition}$ & is 1 if the condition is true, 0 otherwise\\
\end{tabular}
\egroup
\vspace{0.25cm}



\section{Final instructions}
Do not change any aspects of the formatting parameters in the style files.
In particular, do not modify the width or length of the rectangle the text
should fit into, and do not change font sizes (except perhaps in the
\textsc{References} section; see below). Please note that pages should be
numbered.

\section{Preparing PostScript or PDF files}

Please prepare PostScript or PDF files with paper size ``US Letter'', and
not, for example, ``A4''. The -t
letter option on dvips will produce US Letter files.

Consider directly generating PDF files using \verb+pdflatex+
(especially if you are a MiKTeX user).
PDF figures must be substituted for EPS figures, however.

Otherwise, please generate your PostScript and PDF files with the following commands:
\begin{verbatim}
dvips mypaper.dvi -t letter -Ppdf -G0 -o mypaper.ps
ps2pdf mypaper.ps mypaper.pdf
\end{verbatim}

\subsection{Margins in LaTeX}

Most of the margin problems come from figures positioned by hand using
\verb+\special+ or other commands. We suggest using the command
\verb+\includegraphics+
from the graphicx package. Always specify the figure width as a multiple of
the line width as in the example below using .eps graphics
\begin{verbatim}
   \usepackage[dvips]{graphicx} ...
   \includegraphics[width=0.8\linewidth]{myfile.eps}
\end{verbatim}
or % Apr 2009 addition
\begin{verbatim}
   \usepackage[pdftex]{graphicx} ...
   \includegraphics[width=0.8\linewidth]{myfile.pdf}
\end{verbatim}
for .pdf graphics.
See section~4.4 in the graphics bundle documentation (\url{http://www.ctan.org/tex-archive/macros/latex/required/graphics/grfguide.ps})

A number of width problems arise when LaTeX cannot properly hyphenate a
line. Please give LaTeX hyphenation hints using the \verb+\-+ command.

\subsubsection*{Author Contributions}
If you'd like to, you may include  a section for author contributions as is done
in many journals. This is optional and at the discretion of the authors.

\subsubsection*{Acknowledgments}
Use unnumbered third level headings for the acknowledgments. All
acknowledgments, including those to funding agencies, go at the end of the paper.
\end{comment}

\newpage

\section*{Reproducibility Statement}
Method and implementation details are provided in \cref{sec: method}, \cref{supp: net_arch_hyperparams}, and \cref{supp: implement_details}. Experiment settings and details are described in \cref{sec: experiment_setups} and \cref{supp: env_details}. Information about the experimental infrastructure is available in \cref{supp: infrast}. The
code is publicly available at \url{https://github.com/LXXXXR/ExpoComm}.
% The code will be released publicly after the review process.

\section*{Acknowledgment}
This work was supported by the Hong Kong Research Grants Council under the NSFC/RGC Collaborative Research Scheme grant CRS\_HKUST603/22 and the Shanghai Sailing Program 24YF2710200. We thank the anonymous reviewers for their valuable feedback and suggestions.


\bibliography{ref}
\bibliographystyle{iclr2025_conference}

\appendix
% \section{Appendix}


% You may include other additional sections here.

\newpage

\section{Theoretical Analysis} \label{supp: theory}
In this section, we analyze the communication effect of exponential topologies.

\begin{theorem}
Suppose that $E^{t}_{ij}$ is defined by \cref{eq: one_peer_exp_adj}. Let $\tau = \lceil \log_2{(N-1)} \rceil$. Then, the following holds:
\begin{equation}
    E^{t}_{ij} \times_b E^{t+1}_{ij} \times_b \ldots E^{t+\tau-1}_{ij} = \mathbbm{1}\mathbbm{1}^T,
\end{equation}
where $\times_b$ denotes logical (Boolean) matrix multiplication.
\label{theorem: comm_effect}
\end{theorem}

\begin{remark}
    If the information at each agent remains valid within $\tau$ timesteps and there is no information loss during aggregation, the one-peer exponential topology ensures information exchange between any two agents in the system with $\tau$ timesteps.
\end{remark}

\begin{remark}
    Static exponential topologies exhibit a similar communication effect as described in \cref{theorem: comm_effect}. Specifically, $\forall i, j$ that $E^{t(\text{one-peer})}_{ij} = 1$, it holds that $E^{t(\text{stat})}_{ij} = 1$.
\end{remark}

\begin{proof}
Define function $Z:\mathbb{R}_+ \rightarrow \{0,1\}$ such that
    \begin{equation}
        Z(x) = \begin{cases}
    1,& x > 0,\\
    0,& x = 0.
    \end{cases}
    \end{equation}
    Then, for all $x,y,u,v\geq 0$, the following equivalence holds:
    \begin{equation}
        xy + wv = 0 \iff (Z(x) \times_b Z(y)) +_b (Z(u) \times_b Z(v)) = 0,
    \end{equation}
    where $\times_b$ denotes logical (Boolean) \texttt{And}, and $+_b$ denotes logical (Boolean) \texttt{Or}.
    \label{lemma: add_mul}
Now, consider the connection between $Z(x)$ and the structure of an all-one matrix. For a non-negative matrix $X$, it holds that $X_{ij} \in \mathbb{R}^+, \forall i, j \iff Z(X) = \mathbbm{1}\mathbbm{1}^T$. 

Therefore, by applying \cref{lemma: add_mul} to $E^{t}_{ij}  E^{t+1}_{ij} \ldots E^{t+\tau-1}_{ij} = \frac{2^\tau}{N} \mathbbm{1}\mathbbm{1}^T$~\citep{exp_graph_decentralized_exact_avg}, we have $E^{t}_{ij} \times_b E^{t+1}_{ij} \times_b \ldots E^{t+\tau-1}_{ij} = \mathbbm{1}\mathbbm{1}^T$.
\end{proof}


\color{black}


\section{Experiment Details} 
\subsection{Network architecture and hyperparameters} \label{supp: net_arch_hyperparams}

\paragraph{Codebase} 
Our implementation of ExpoComm and baseline algorithms is based on the following codebase:
\begin{itemize}
    \item EPyMARL~\citep{epymarl}: \url{https://github.com/uoe-agents/epymarl}
    \item CommFormer~\cite{Commformer}: \url{https://github.com/charleshsc/CommFormer}
    \item CommNet~\citep{CommNet}: \url{https://github.com/isp1tze/MAProj}
\end{itemize}

The code for ExpoComm is publicly available at \url{https://github.com/LXXXXR/ExpoComm}.

\paragraph{Neural network architecture}
Following previous work \cite{epymarl}, we employ deep neural networks consisting of multilayer perceptrons (MLPs) with rectified linear unit (ReLU) activation functions and gated recurrent units (GRUs) to parameterize the agent networks. 
In ExpoComm, the message memory blocks described in \cref{sec: network_design} are implemented using a single GRU or an attention block. The prediction network $f(\cdot; \phi)$ described in \cref{sec: training_details} is implemented using a two-layer MLP. 

\paragraph{Hyperparameters} 
To ensure a fair comparison, we implement our method and self-constructed baselines using the same codebase with the same set of hyperparameters, with the exception of method-specific ones and the learning rate. In general, we follow the common settings provided by \citet{epymarl} for MAgent benchmark and adopt the settings in IMP paper~\citep{benchmark_IMP} for the IMP benchmark. The common hyperparameters are listed in \cref{tab: comm_hyper}. The ExpoComm-specific hyperparameters are provided in \cref{tab: hyper_ExpoComm}. For learning rate, we search among $(0.0001, 0.0005)$ for ExpoComm and baselines. We use the value of $0.0005$ for base algorithms without communication; $0.0005$ for DGN+TarMAC in MAgent and $0.0001$ for DGN+TarMAC in IMP; $0.0001$ for ExpoComm in IMP with 50 agents and $0.0005$ for other scenarios. For CommFormer, we adopt the optimal value in its official implementation.

\begin{table}[th]
    \centering
    \caption{Common hyperparameters.}
    \label{tab: comm_hyper}
    \begin{threeparttable}
    \begin{tabular}{ccc}
    \toprule
    Hyperparameter & Benchmark & Value \\
    \midrule
    Hidden sizes & - & $64$ \\
    \multirow{2}{*}{Discount factor $\gamma$} & MAgent &  $0.99$ \\
    & IMP & $0.95$ \\
    \multirow{2}{*}{Batch size} & MAgent & $32$ \\
    & IMP & $64$ \\
    Replay buffer size & - &  $2000$ \\
    \multirow{2}{*}{Number of environment steps} & MAgent & $5 \times 10^{6}$ \\
    & IMP & $2 \times 10^{6}$ \\
    \multirow{2}{*}{Epsilon anneal steps} & MAgent & $5 \times 10^{5}$ \\
    & IMP & $5 \times 10^{3}$ \\
    \multirow{2}{*}{Test interval steps} & MAgent & $5 \times 10^{4}$ \\
    & IMP & $2.5 \times 10^{4}$ \\
    Number of test episode & - & $100$ \\
    \bottomrule
    \end{tabular}
    % \begin{tablenotes}
    %     \item[1] Here we adopt the per-scenario finetuned value for this hyperparameter as provided by HARL.
    % \end{tablenotes}
    \end{threeparttable}
\end{table}

% \begin{table}[th]
%     \centering
%     \caption{Common hyperparameters.}
%     \label{tab:comm_hyper}
%     \begin{threeparttable}
%     \begin{tabular}{lcc}
%     \toprule
%     Hyperparameter & MAgent & IMP \\
%     \midrule
%     Hidden sizes & \multicolumn{2}{c}{$64$} \\
%     Discount factor $\gamma$ & $0.99$ & $0.95$ \\
%     Batch size & $32$ & $64$ \\
%     Replay buffer size & \multicolumn{2}{c}{$2000$} \\
%     Number of environment steps & $5 \times 10^{6}$ & $2 \times 10^{6}$ \\
%     Epsilon anneal steps & $5 \times 10^{5}$ & $5 \times 10^{3}$ \\
%     Test interval steps & $5 \times 10^{4}$ & $2.5 \times 10^{4}$ \\
%     Number of test episodes & \multicolumn{2}{c}{$100$} \\
%     \bottomrule
%     \end{tabular}
%     \end{threeparttable}
% \end{table}

\begin{table}[th]
    \centering
    \caption{Hyperparameters used for ExpoComm.}
    \label{tab: hyper_ExpoComm}
    \begin{tabular}{cc}
    \toprule
    Hyperparameter & Value \\
    \midrule
    Auxillary loss coefficient $\alpha$  & $0.1$\\
    Temperature $\tau$  & $0.07$\\
    Number of negative pairs $M$  & $20$ \\
    \bottomrule
    \end{tabular}
\end{table}

\subsection{Environmental details} \label{supp: env_details}

\paragraph{Codebase} The environments used in this work are listed below with descriptions in \cref{tab: env}.
\begin{itemize}
    \item MAgent~\citep{benchmark_Magent,magent2}: \\\url{https://github.com/Farama-Foundation/MAgent2}
    \item IMP~\citep{benchmark_IMP}: \url{https://github.com/moratodpg/imp_marl}
\end{itemize}

\begin{table}[th]
    \centering
    \caption{Environments details.}
    \label{tab: env}
    \begin{threeparttable}
    \begin{tabular}{ccc}
    \toprule
    Environment &  Scenarios & Number of agents \\
    \midrule
    \multirow{2}{*}{MAgent} &  Adversarial Pursuit & $(25, 45, 61)$ \tnote{1} \\
    &  Battle & $(20, 42, 64)$ \tnote{2}\\
    % \hline \noalign{\vskip 2pt}
    \midrule
    \multirow{3}{*}{IMP} &  Uncorrelated: uncorrelated k-out-of-n; campaign cost & $(50, 100)$ \\
    &  Correlated: correlated k-out-of-n; campaign cost & $(50, 100)$ \\
    &  OWF: offshore wind farm; campaign cost & $(50, 100)$ \\
    \bottomrule
    \end{tabular}
    \begin{tablenotes}
        \item[1] The number of agents in this scenario is determined by setting the map size to $25, 35, 40$, respectively.
        \item[2] The number of agents in this scenario is determined by setting the map size to $45, 60, 70$, respectively.
    \end{tablenotes}
    \end{threeparttable}
\end{table}


\begin{figure}[t]
\centering
\begin{subfigure}[t]{.26\textwidth}
    \centering
    \includegraphics[width=\textwidth]{figs/adversarial_pursuit-24_env.png}
    \caption{AdversarialPursuit.}
\end{subfigure}
\begin{subfigure}[t]{.26\textwidth}
    \centering
    \includegraphics[width=\textwidth]{figs/battle-113_env.png}
    \caption{Battle.}
\end{subfigure}
\caption{Environments from the MAgent benchmark suite~\citep{magent2}. In each scenario, the MARL algorithms control the red agents, while the blue adversary agents are controlled by pretrained policies.}
\label{fig: Magent_env}
\end{figure}


\begin{figure}[t]
  \centering
    \centering
    \includegraphics[width=\textwidth]{figs/imp_env_v2.pdf}
    \caption{IMP environment~\citep{benchmark_IMP}. This environment simulates an engineering system with multiple components controlled by agents. The objective is to minimize overall system failure risk at low costs. The system risk depends on component damage probabilities, which evolve over time and can be influenced by agent inspection or repair actions.}
    \label{fig: IMP_env}
\end{figure}



\paragraph{MAgent} MAgent is a highly scalable gridworld gaming benchmark shown in \cref{fig: Magent_env}. In \texttt{AdversarialPursuit}, agents aim to tag adversaries while adversaries try to escape. Agents can choose actions from \texttt{move}, \texttt{tag} or \texttt{do nothing}. Agents are rewarded for successfully tagging an adversary and penalized for unsuccessful tagging attempts. In \texttt{Battle}, agents attempt to attack and eliminate adversaries, with the same goal for adversaries. Agents can choose actions from \texttt{move}, \texttt{attack} or \texttt{do nothing}. A team wins by eliminating all opponents or having more surviving agents when the episode ends.

Following the official implementation~\citep{magent2}, we use an individual reward setting with IDQN as the base algorithm. 
Due to the high-dimensional observations in MAgent, storing experience in a replay buffer can be challenging because of hardware constraints. We adopt a preprocessing procedure following \citet{DGN}, compressing observations by concatenating [my\_team\_hp - obstacle/off the map, other\_team\_hp - obstacle/off the map]. To facilitate the use of communication, we use small view ranges ($8$ for \texttt{AdversarialPursuit} and $7$ for \texttt{Battle}). In both scenarios, we pretrain the adversary policies using the IDQN algorithm with a self-play scheme and use these pretrained policies to test the performance of different algorithms.

\paragraph{IMP} IMP is a platform for benchmarking the scalability of cooperative MARL methods in real-world engineering applications, as illustrated in \cref{fig: IMP_env}. This environment simulates an infrastructure management planning problem with agents controlling different components. Agents can choose actions from \texttt{inspection}, \texttt{repair} or \texttt{do nothing}. In different scenarios, the correlation between agent deterioration processes and the system failure function are defined differently, posing unique challenges.

Following the official implementation of IMP, we use a global reward setting and choose QMIX as the base algorithm due to its stable performance across scenarios. We adopt the campaign cost setting, which requires higher cooperation among agents. As recommended by \citet{benchmark_IMP}, results are normalized with respect to expert-based heuristic policies using $(x-H)/|H|$, where $x$ is the discounted rewards of the tested algorithm, and $H$ is the discounted rewards achieved by heuristic policies listed in \cref{tab: heuristic}.

\begin{table}[th]
    \centering
    \caption{Heuristic policies performance on the IMP benchmark.}
    \label{tab: heuristic}
    \begin{tabular}{ccc}
    \toprule
    Scenario & Number of agents $N$ & Discounted reward $H$ \\
    \midrule
    \multirow{2}{*}{Uncorrelated} & $50$ &  $-232.7$ \\
    & $100$ & $-231.5$ \\
    \midrule
    \multirow{2}{*}{Correlated} & $50$ &  $-211.0$ \\
    & $100$ & $-194.0$ \\
    \midrule
    \multirow{2}{*}{OWF} & $50$ &  $-1248.2$ \\
    & $100$ & $-2436.3$ \\
    \bottomrule
    \end{tabular}

\end{table}


\subsection{Implementation details} \label{supp: implement_details}
In MAgent, we implement our proposed ExpoComm along with baselines DGN+TarMAC and ER on top of the IDQN base algorithm. In IMP, these are implemented on top of QMIX. For ExpoComm, we use \cref{eq: aux_pred} for MAgent benchmark because the global state is provided in this environment,  and \cref{eq: aux_cont} for IMP, as the global state is a concatenation of all observations and is not compact or suitable for message grounding.

\subsection{Experimental Infrastructure} \label{supp: infrast}
The experiments were conducted using NVIDIA GeForce RTX 3080 GPUs and NVIDIA A100GPUs. Each experimental run required less than 2 days to complete.


\section{More results and discussion}

\subsection{Visualization results} \label{supp: vis_results}


\begin{figure}[t]
\centering
\begin{subfigure}[t]{0.8\textwidth}
    \centering
    \includegraphics[width=\textwidth]{figs/results/1117_adv_pursuit_61_ExpoComm_one_peer.pdf}
    \caption{ExpoComm policies with $K=1$. Red agents push blue adversary agents to the edge of the frame and trap them there to obtain high rewards by repeatedly tagging them.}
    \label{fig: vis_adv_pursuit_ExpoComm}
\end{subfigure}
\begin{subfigure}[t]{0.8\textwidth}
    \centering
    \includegraphics[width=\textwidth]{figs/results/1117_adv_pursuit_61_IQL.pdf}
    \caption{IDQN policies.}
    \label{fig: vis_adv_pursuit_IDQN}
\end{subfigure}
\caption{Visualization in AdversarialPursuit w/ 61 agents.}
\label{fig: vis_adv_pursuit}
\end{figure}

\begin{figure}[t]
\centering
\begin{subfigure}[t]{0.8\textwidth}
    \centering
    \includegraphics[width=\textwidth]{figs/results/1117_battle_64_ExpoComm_one_peer.pdf}
    \caption{ExpoComm policies with $K=1$. Agents coordinate to ensure red agents outnumber blue adversaries on the front line ($t=11$, $t=21$), securing an advantage. Once red agents substantially outnumber blue adversaries, they surround the remaining adversaries ($t=61$, $t=71$) to eliminate them.}
    \label{fig: vis_battle_ExpoComm}
\end{subfigure}
\begin{subfigure}[t]{0.8\textwidth}
    \centering
    \includegraphics[width=\textwidth]{figs/results/1117_battle_64_IQL.pdf}
    \caption{IDQN policies.}
    \label{fig: vis_battle_IDQN}
\end{subfigure}
\caption{Visualization in Battle w/ 64 agents.}
\label{fig: vis_battle}
\end{figure}



We visualize the final trained policies of ExpoComm and IDQN in \texttt{dversarialPursuit} and \texttt{Battle} with \cref{fig: vis_adv_pursuit} and \cref{fig: vis_battle} respectively to demonstrate how ExpoComm enhances cooperation among agents. As shown in \cref{fig: vis_adv_pursuit_ExpoComm} and \cref{fig: vis_battle_ExpoComm}, agents adopt a global perspective and act cooperatively with ExpoComm policies, demonstrating effectiveness even under extreme communication budgets($K=1$). In comparison, IDQN agents focus only on local observations and often become trapped in suboptimal solutions due to lack of coordination.



\subsection{Comparison with proxy-based communication}

Although we primarily focus on decentralized communication-based MASs without centralized proxies, we also compare ExpoComm against the proxy-based CommNet~\citep{CommNet}. As seen in \cref{fig:results_magent_CommNet} and \cref{tab:results_imp_CommNet}, ExpoComm outperforms CommNet in most scenarios, especially in IMP benchmarks. However, CommNet achieves comparable performance on the \texttt{AdversarialPursuit} tasks. This implies that a global perspective is more crucial for success in these scenarios, possibly explaining ExpoComm's larger advantage over other baselines in this scenario. 


\begin{figure}[t]
\centering
% \subfigure[]{
% \includegraphics[width=0.7\textwidth]{Styles/figs/results/0515_legend_cropped.pdf}
%     }8
\raisebox{-\height}{\includegraphics[width=0.75\textwidth]{figs/results/1118_magent_legend_CommNet.pdf}}
\par
\begin{subfigure}[t]{.32\textwidth}
    \centering
    \includegraphics[width=\textwidth]{figs/results/1120_magent_adv_pursuit25_CommNet.pdf}
    \caption{AdversarialPursuit w/ 25 agents}
\end{subfigure}
\begin{subfigure}[t]{.32\textwidth}
    \centering
    \includegraphics[width=\textwidth]{figs/results/1120_magent_adv_pursuit45_CommNet.pdf}
    \caption{AdversarialPursuit w/ 45 agents}
\end{subfigure}
\begin{subfigure}[t]{.32\textwidth}
    \centering
    \includegraphics[width=\textwidth]{figs/results/1120_magent_adv_pursuit61_CommNet.pdf}
    \caption{AdversarialPursuit w/ 61 agents}
\end{subfigure}
\begin{subfigure}[t]{.32\textwidth}
    \centering
    \includegraphics[width=\textwidth]{figs/results/1120_magent_battle20_CommNet.pdf}
    \caption{Battle w/ 20 agents}
\end{subfigure}
\begin{subfigure}[t]{.32\textwidth}
    \centering
    \includegraphics[width=\textwidth]{figs/results/1120_magent_battle42_CommNet.pdf}
    \caption{Battle w/ 42 agents}
\end{subfigure}
\begin{subfigure}[t]{.32\textwidth}
    \centering
    \includegraphics[width=\textwidth]{figs/results/1120_magent_battle64_CommNet.pdf}
    \caption{Battle w/ 64 agents}
\end{subfigure}
\caption{Performance comparison with proxy-based baselines on MAgent tasks.}
\label{fig:results_magent_CommNet}
\end{figure}


\begin{table}[t]
\centering
\begin{threeparttable}
\caption{Performance comparison with proxy-based baselines on IMP tasks. Results are reported as the mean and standard deviation of the percentage of normalized discounted rewards relative to expert-based heuristic policies, following \citet{benchmark_IMP}, with details in \cref{supp: env_details}. The best-performing method is indicated in \textbf{bold}, and the second best is \underline{underlined}.}
\label{tab:results_imp_CommNet}
\begin{tabular}{lccc}
\toprule
\multirow{2}{*}{Scenario} & \multicolumn{1}{c}{CommNet} & \multicolumn{2}{c}{ExpoComm} \\
\cmidrule(lr){2-2} \cmidrule(lr){3-4}
& with communication proxy & $K=1$ & $K=\lceil\log_2N\rceil$ \\
\midrule
\multicolumn{4}{c}{$N=50$} \\
\midrule
Uncorrelated & $26.07\,\scalebox{0.8}{($6.82$)}$ & $\underline{27.31\,\scalebox{0.8}{($2.26$)}}$ & $\mathbf{28.26\,\scalebox{0.8}{($2.51$)}}$ \\
Correlated & $26.14\,\scalebox{0.8}{($16.87$)}$ & $\mathbf{43.82\,\scalebox{0.8}{($6.33$)}}$ & $\underline{40.01\,\scalebox{0.8}{($3.19$)}}$ \\
OWF & $53.71\,\scalebox{0.8}{($1.27$)}$ & $\underline{64.66\,\scalebox{0.8}{($0.26$)}}$ & $\mathbf{65.19\,\scalebox{0.8}{($0.51$)}}$ \\
\midrule
\multicolumn{4}{c}{$N=100$} \\
\midrule
Uncorrelated & $-65.92\,\scalebox{0.8}{($125.03$)}$ & $\underline{27.34\,\scalebox{0.8}{($13.32$)}}$ & $\mathbf{27.81\,\scalebox{0.8}{($5.71$)}}$ \\
Correlated & $-82.76\,\scalebox{0.8}{($48.62$)}$ & $\mathbf{19.17\,\scalebox{0.8}{($23.94$)}}$ & $\underline{17.25\,\scalebox{0.8}{($22.70$)}}$ \\
OWF & $34.71\,\scalebox{0.8}{($5.34$)}$ & $\underline{65.26\,\scalebox{0.8}{($1.34$)}}$ & $\mathbf{66.23\,\scalebox{0.8}{($0.38$)}}$ \\
\bottomrule
\end{tabular}
\end{threeparttable}
\end{table}


\subsection{Limitations and future work}
While ExpoComm demonstrates strong performance and scalability in cooperative multi-agent tasks, some limitations remain.

First, ExpoComm does not explicitly incorporate agent heterogeneity or properties of the underlying environmental MDP when constructing the communication topology. This could result in suboptimal performance in scenarios requiring targeted messaging between specific agents~\citep{MAIC} or in networked MDPs~\citep{networked_MA, large_scale_networked_MARL}. Therefore, Incorporating factors like agent identities or relationships presents a promising direction for further improvements in such settings.

Second, we evaluated ExpoComm primarily in fully cooperative tasks. Partially competitive settings requiring agents to learn to communicate only when necessary remain challenging. Examining ExpoComm's capabilities and limitations in such partially competitive tasks presents an important avenue for future work.

Finally, communication scalability in multi-agent systems remains an under-explored area despite the attempt of this work. For instance, incorporating finer graph topologies beyond exponential graphs may enhance performance, and exploiting temporal communication sparsity could further reduce costs. There are still many open questions in scaling communication efficiently.





\end{document}
