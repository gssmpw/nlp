\section{Related Work}
\label{sec: related_work}
% \textcolor{blue}{1 page}
\paragraph{Communication in MASs}
% many communication work -> many progress for better performance -> some consider realistic concerns -> scalability -> primary issue is the communication topology -> fixed / dynamic (distance-based) -> 
% individually controlled (sender/receiver)
% GNN-learnt  

% First introduced by \citet{CommNet,DIAL_RIAL}, communication among agents in MARL has been an active research area due to its great potential to enhance cooperation and improve task performance. Owing to the high flexibility of communication protocols, finding the most effective protocols tailored to the MARL paradigm is a challenging task~\citep{MARL_comm_survey}.
% % To this end, extensive later works adopt the straightforward approach to learn the communication protocol in an end-to-end fashion, which is termed ``learning to communicate''. Notably, . M
% To this end, extensive works take the learning perspective and optimize the communication components~\citep{BiCNet}, such as message generators, message aggregators, and connectivity among agents, through end-to-end training. From the sender side, ToM2C~\citep{ToM2C} and MAIC~\citep{MAIC} enhanced the message generation process by utilizing teammate modeling. CACL~\citep{CACL} focuses on the decentralized training paradigm and proposes to learn communication encoding with contrastive learning. From the receiver side, TarMAC~\citep{TarMAC}, G2ANet~\citep{G2ANet} and MASIA~\citep{MASIA} improve message aggregation by employing attention-based aggregation strategies.

Communication among agents in MARL was first introduced by \citet{CommNet,DIAL_RIAL} and has since become an active research area due to its potential to enhance cooperation and improve task performance. The flexibility of communication protocols makes finding effective solutions for the MARL paradigm challenging~\citep{MARL_comm_survey}.
To address this difficulty, many studies have focused on optimizing communication components, such as message generators, message aggregators, and connectivity among agents, through end-to-end training~\citep{BiCNet}. From the sender side, ToM2C~\citep{ToM2C} and MAIC~\citep{MAIC} enhance message generation through teammate modeling, while CACL~\citep{CACL} uses contrastive learning techniques to learn communication encoding in a decentralized training paradigm. From the receiver side, TarMAC~\citep{TarMAC}, G2ANet~\citep{G2ANet}, and MASIA~\citep{MASIA} improve message aggregation using attention-based strategies. \looseness=-1

Recently, researchers have addressed challenges posed by real-world communication systems. Notably, NDQ~\citep{NDQ} and TMC~\citep{TMC} reduce communication costs by crafting succinct messages, while ATOC~\citep{atoc}, IC3~\citep{ic3}, I2C~\citep{i2c}, and CommFormer~\citep{Commformer} manage overhead by pruning unnecessary communication links. Additionally, \citet{noisy_channel} propose a stochastic encoding/decoding scheme to handle noisy channels, and DACOM \citep{DACOM} introduces delay-aware communication to account for the high latency of wireless channels.

Despite these advancements, scalability in communication mechanisms has been largely overlooked, often due to the quadratically increasing communication cost associated with fully-connected graphs as the number of agents grows. Although few works explicitly address the scalability issue, efforts to design communication topologies among agents offer potential solutions. These can be categorized into fully-connected, rule-based, and learned topologies. 
Early works~\citep{CommNet,DIAL_RIAL,BiCNet} typically adopt fully-connected topologies to demonstrate communication benefits, but at the cost of high bandwidth requirements. Later on, to reduce the overall communication overhead, \citet{DGN} and \citet{neighbor_graph} restrict communication to nearby neighbors based on distance, while NeurComm~\citep{NeurComm} limits communication to neighboring agents in networked MASs. In spite of achieving significant performance gains, their further applicability may be limited since they require extra information beyond local observation to determine the communication topology. In contrast, learned topology methods assume no such requirements and offer high flexibility. In particular, ATOC~\citep{atoc}, IC3~\citep{ic3}, I2C~\citep{i2c} locally deploy gates for agents to decide if they should engage in communication. However, these methods may result in uncontrollable overall communication costs due to individual control schemes. Alternatively, MAGIC~\citep{MAGIC} utilizes graph attention mechanisms to learn the communication topology, while CommFormer~\citep{Commformer} extends the idea and enables control over the overall communication sparsity. Although effective in small-scale MASs, peer-wise connectivity becomes increasingly difficult to learn in large-scale MASs, and high sparsity may impair performance, as discussed by \citet{Commformer}.
% In particular, VBC~\citep{VBC}, ETC~\citep{ETC} and MBC~\citep{MBC_sparse_comm} 

% Our proposed ExpoComm incorporates a rule-based topology, supplementing the aforementioned progress on MARL communication by explicitly addressing the scalability of communication.
Our proposed ExpoComm, which incorporates rule-based topologies for rapid information dissemination among all agents, complements existing efforts in MAS communication by explicitly addressing scalability challenges. \looseness=-1

% Subsequent works then made concrete progress 

% {\color{red}
% \textbf{Our Contributions.}
% \begin{enumerate}[label=\arabic*),leftmargin=14pt]
%     \item 
    
%     \item  
    
%     \item 
% \end{enumerate}}
\paragraph{Exponential Graphs}
% \textcolor{blue}{Should I put this paragraph above the last one?}
Exponential graphs are a class of graph topologies that exhibit strong scalability properties with respect to the number of nodes. They have been primarily used in distributed learning to periodically synchronize model updates across devices. \citet{exp_graph_decentralized_learning_first} investigate exponential graphs with gossip algorithms and achieve high consensus rates for decentralized learning. Follow-up works~\citep{SlowMo,exp_graph_decentralized_exact_avg,expG_consensus,ExpG_largebs} build upon this topology, optimizing model weight update algorithms and providing empirical evidence and theoretical guarantees for the effectiveness of exponential graphs. Beyond distributed learning, exponential graphs also have applications in chip design~\citep{wang2014rpnoc,ExpG_chip_design}. Overall, exponential graphs demonstrate efficient information dissemination across many nodes, making them a promising candidate topology for achieving scalable communication in MARL. \looseness=-1