\section{Related Work}
\label{section:related_works}
It is important to note that regret-based UED approaches provide a minimax regret guarantee at Nash Equilibrium; however, they provide no explicit guarantee of convergence to such equilibrium. \citet{beukman2024Refining} demonstrated that the minimax regret objective does not necessarily align with learnability: an agent may encounter UPOMDPs with high regret on certain levels where it already performs optimally (given the partial observability constraints), while there exist other levels with lower regret where it could still improve. Consequently, selecting levels solely based on regret can lead to {\em regret stagnation}, where learning halts prematurely. This suggests that focusing exclusively on minimax regret may inhibit the exploration of levels where overall regret is non-maximal, but opportunities for acquiring transferable skills for generalization are significant. Thus, there is a compelling need for a complementary objective, such as novelty, to explicitly guide level selection towards enhancing zero-shot generalization performance and mitigating regret stagnation.

The {\em Paired Open-Ended Trailblazer} (POET;~\cite{wang2019poet}) algorithm computes novelty based on environment encodings---a vector of parameters that define level configurations. POET maintains a record of the encodings from previously generated levels and computes the novelty of a new level by measuring the average distance between the k-nearest neighbors of its encoding. However, this method for computing novelty is domain-specific and relies on human expertise in designing environment encodings, posing challenges for scalability to complex domains. Moreover, due to UED's underspecified nature, where free parameters may yield a one-to-many mapping between parameters and environments instances, each inducing distinct agent behaviors, quantifying novelty based on parameters alone is futile. 

{\em Enhanced POET} (EPOET;~\cite{wang2020enhanced}) improves upon its predecessor by introducing a domain-agnostic approach to quantify a level's novelty. EPOET is grounded in the insight that novel levels offer new insights into how the behaviors of agents within them differ. EPOET evaluates both active and archived agents' performance in each environment, converting their performance rankings into rank-normalized vectors. The level's novelty is then computed by measuring the Euclidean distance between these vectors. Despite addressing POET's domain-specific limitations, EPOET encounters its own challenges. The computation of rank-normalized vectors only works for population-based approaches as it requires evaluating multiple trained student agents and incurs substantial computational costs. Furthermore, EPOET remains curriculum-agnostic, as its novelty metric relies on the ordering of raw returns within the agent population, failing to directly assess whether the environment elicits rarely observed states and actions in the existing curriculum.

{\em Diversity Induced Prioritized Level Replay} (DIPLR;~\cite{li2023effective}), calculates novelty using the Wasserstein distance between occupancy distributions of agent trajectories from different levels. DIPLR maintains a level buffer and determines a level's novelty as the minimum distance between the agent's trajectory on the candidate level and those in the buffer. While DIPLR incorporates the agent’s experiences into its novelty calculation, it faces significant scalability and robustness issues. First, relying on the Wasserstein distance is notoriously costly. Additionally, DIPLR requires pairwise distance computations between all levels in the buffer, causing computational costs to grow exponentially with more levels. Finally, although DIPLR promotes diversity within the active buffer, it fails to evaluate whether state-action pairs in the current trajectory have already been adequately explored through past curriculum experiences, making it arguably still curriculum-agnostic. Further discussions on relevant literature can be found in Appendix~\ref{section:extended_related_work}.