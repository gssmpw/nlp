\documentclass{article}


% if you need to pass options to natbib, use, e.g.:
    \PassOptionsToPackage{numbers, compress}{natbib}
% before loading neurips_2024


% ready for submission
% \usepackage{neurips_2024}


% to compile a preprint version, e.g., for submission to arXiv, add the
% [preprint] option:
%     \usepackage[preprint]{neurips_2024}


% to compile a camera-ready version, add the [final] option, e.g.:
    \usepackage[final]{neurips_2024}


% to avoid loading the natbib package, add option nonatbib:
%    \usepackage[nonatbib]{neurips_2024}


\usepackage[utf8]{inputenc} % allow utf-8 input
\usepackage[T1]{fontenc}    % use 8-bit T1 fonts
\usepackage{hyperref}       % hyperlinks
\usepackage{url}            % simple URL typesetting
\usepackage{booktabs}       % professional-quality tables
\usepackage{amsfonts}       % blackboard math symbols
\usepackage{nicefrac}       % compact symbols for 1/2, etc.
\usepackage{microtype}      % microtypography
\usepackage{xcolor}         % colors

% added packages
\usepackage{amsmath}
\usepackage{amsfonts}
\usepackage{graphicx}
\usepackage{subfigure}
\usepackage{algorithm}
\usepackage{algorithmic}
\usepackage{multirow}
\usepackage{color}
\usepackage{svg}
\usepackage{wrapfig}

\DeclareMathOperator*{\argmax}{arg\,max}
\DeclareMathOperator*{\argmin}{arg\,min}



\title{Improving Environment Novelty Quantification for Effective Unsupervised Environment Design}



% The \author macro works with any number of authors. There are two commands
% used to separate the names and addresses of multiple authors: \And and \AND.
%
% Using \And between authors leaves it to LaTeX to determine where to break the
% lines. Using \AND forces a line break at that point. So, if LaTeX puts 3 of 4
% authors names on the first line, and the last on the second line, try using
% \AND instead of \And before the third author name.


% compact foramt
\author{
Jayden Teoh$^*$, 
Wenjun Li$^*$$^\dag$, 
Pradeep Varakantham\\
Singapore Management University \\
\texttt{\{jxteoh.2023, wjli.2020, pradeepv\}@smu.edu.sg} 
}

\begin{document}


\maketitle
\def\thefootnote{*}\footnotetext{Equal contribution.}
% \def\thefootnote{\dag}\footnotetext{Corresponding author: \texttt{wjli.2020@phdcs.smu.edu.sg}}
\def\thefootnote{\dag}\footnotetext{Corresponding author.}


\begin{abstract}
Unsupervised Environment Design (UED) formalizes the problem of autocurricula through interactive training between a teacher agent and a student agent. The teacher generates new training environments with high learning potential, curating an adaptive curriculum that strengthens the student's ability to handle unseen scenarios. Existing UED methods mainly rely on {\em regret}, a metric that measures the difference between the agent's optimal and actual performance, to guide curriculum design. Regret-driven methods generate curricula that progressively increase environment complexity for the student but overlook environment {\em novelty}--a critical element for enhancing an agent's generalizability. Measuring environment novelty is especially challenging due to the underspecified nature of environment parameters in UED, and existing approaches face significant limitations. To address this, this paper introduces the {\em Coverage-based Evaluation of Novelty In Environment} (CENIE) framework. CENIE proposes a scalable, domain-agnostic, and curriculum-aware approach to quantifying environment novelty by leveraging the student's state-action space coverage from previous curriculum experiences. We then propose an implementation of CENIE that models this coverage and measures environment novelty using Gaussian Mixture Models. By integrating both regret and novelty as complementary objectives for curriculum design, CENIE facilitates effective exploration across the state-action space while progressively increasing curriculum complexity. Empirical evaluations demonstrate that augmenting existing regret-based UED algorithms with CENIE achieves state-of-the-art performance across multiple benchmarks, underscoring the effectiveness of novelty-driven autocurricula for robust generalization.
\end{abstract}



\section{Introduction}
Although recent advancements in Deep Reinforcement Learning (DRL) have led to many successes, e.g., super-human performance in games~\cite{hu2019simplified,berner2019dota} and reliable control in robotics~\cite{akkaya2019solving,andrychowicz2020learning}, training generally-capable agents remains a significant challenge. DRL agents often fail to generalize well to environments only slightly different from those encountered during training~\cite{cobbe2019quantifying,zhou2022domain}. To address this problem, there has been a surge of interest in {\em Unsupervised Environment Design} (UED;~\cite{wang2019poet,dennis2020emergent,wang2020enhanced,jiang2021prioritized,jiang2021replay,parker2022evolving,li2023effective,azad2023clutr}), which formulates the autocurricula~\cite{leibo2019autocurricula} generation problem as a two-player zero-sum game between a {\em teacher} and a {\em student} agent. In UED, the teacher constantly adapts training environments (e.g., mazes with varying obstacles and car-racing games with different track designs) in the curriculum to improve the student's ability to generalize across all possible levels. 

To design effective autocurricula, researchers have proposed various metrics to capture learning potential, enabling teacher agents to create training levels that adapt to the student's capabilities. The most popular metric, {\em regret}, measures the student's maximum improvement possible in a level. While regret-based UED algorithms~\cite{dennis2020emergent,jiang2021replay,jiang2021prioritized} are effective in producing levels at the frontier of the student's capability, they do not guarantee diversity in the student's experiences, limiting the training of generally-capable agents especially in large environment design spaces. Another line of work in UED recognizes this limitation, leading to methods exploring the prioritization of novel levels during curriculum generation~\cite{wang2019poet,wang2020enhanced,li2023effective}. This strategic shift empowers the teacher to introduce novel levels into the curriculum such that the student agent can actively explore the environment space and enhance its generalization capabilities. 

To more effectively evaluate environment novelty, we introduce the {\em Coverage-based Evaluation of Novelty In Environment} (CENIE) framework. CENIE operates on the intuition that a novel environment should induce unfamiliar experiences, pushing the student agent into unexplored regions of the state space and introducing variability in its actions. Therefore, signals about an environment's novelty can be derived by modeling and comparing its state-action space coverage with those of environments already encountered in the curriculum. We refer to this method of estimating novelty based on the agent’s past experiences as {\em curriculum-aware}. By evaluating novelty in relation to the experiences induced by other environments within the curriculum, CENIE prevents redundant environments—those that elicit similar experiences as existing ones—from being classified as novel. Curriculum-aware approaches ensure that levels in the student's curriculum collectively drive the agent toward novel experiences in a sample-efficient manner. 
% CENIE leverages {\em Gaussian Mixture Models} (GMMs) to represent the distribution of past aggregated experiences and assess the likelihood of experiences in new environments. This likelihood indicates the dissimilarity in state-action space coverage, enabling direct comparison of novelty between levels. 

Our contributions are threefold. First, we introduce CENIE, a scalable, domain-agnostic, and curriculum-aware framework for quantifying environment novelty via the agent’s state-action space coverage. CENIE addresses shortcomings in existing methods for environment novelty quantification, as discussed further in Sections \ref{section:related_works} and \ref{section:cenie_approach}. Second, we present implementations for CENIE using {\em Gaussian Mixture Models} (GMM) and integrated its novelty objective with PLR$^\perp$\cite{jiang2021prioritized} and ACCEL\cite{parker2022evolving}, the leading UED algorithms in zero-shot transfer performance. Finally, we conduct a comprehensive evaluation of the CENIE-augmented algorithms across three distinct benchmark domains. By incorporating CENIE into these leading UED algorithms, we empirically validate that CENIE's novelty-based objective not only exposes the student agent to a broader range of scenarios in the state-action space, but also contributes to achieving state-of-the-art zero-shot generalization performance. This paper underscores the importance of novelty and the effectiveness of the CENIE framework in enhancing UED.


\section{Background}
We briefly review the background of Unsupervised Environment Design (UED) in this section. UED problems are modeled as an Underspecified Partially Observable Markov Decision Process (UPOMDP) defined by the tuple:
\begin{align}
\langle S, A, O, \mathcal{I}, \mathcal{T}, \mathcal{R}, \gamma, \Theta \rangle \nonumber
\end{align}
where $S$, $A$ and $O$ are the sets of states, actions, and observations, respectively. $\Theta$ represents a set of free parameters where each $\theta \in \Theta$ defines a specific instantiation of an environment (also known as a {\em level}). We use the terms ``environments'' and ``levels'' interchangeably throughout this paper. The level-conditional observation and transition functions are defined as $\mathcal{I}: S \times \Theta \rightarrow O$ and $\mathcal{T}:S \times A \times \Theta \rightarrow \Delta(S)$, respectively. The student agent, with policy $\pi$, receives a reward based on the reward function $\mathcal{R}:S \times A \rightarrow \mathbb{R}$ with a discount factor $\gamma \in [0, 1]$. The student seeks to maximize its expected value for each $\theta$ denoted by $V^{\theta}(\pi) = \mathbb{E}_\pi [\sum_{t=0}^T R(s_t, a_t) \gamma^t$]. The teacher's goal is to select levels forming the curriculum by maximizing a utility function $U(\pi, \theta)$, which depends on $\pi$. 

Different UED approaches vary primarily in the teacher's utility function. {\em Domain Randomization} (DR;~\cite{tobin2017domain}) uniformly randomizes environment parameters, with a constant utility $U^\mathcal{U}(\pi, \theta) = C$, making it agnostic to the student's policy. {\em Minimax training}~\cite{pinto2017robust} adversarially generates challenging levels, with utility $U^\mathcal{M}(\pi, \theta) = -V^\theta (\pi)$, to minimize the student's return. However, this approach incentivizes the teacher to make the levels completely unsolvable, limiting room for learning. Recent UED methods address this by using a teacher that maximizes {\em regret}, defined as the difference between the return of the optimal policy and the current policy. Regret-based utility is defined as $U^\mathcal{R}(\pi, \theta) = \normalfont\textsc{Regret}^\theta(\pi, \pi^*) =  V^\theta(\pi^*) - V^\theta(\pi)$ where $\pi^*$ is the optimal policy on $\theta$. Regret-based objectives have been shown to promote the simplest levels that the student cannot solve optimally, and benefit from the theoretical guarantee of a minimax regret robust policy upon convergence in the two-player zero-sum game. However, since $\pi^*$ is generally unknown, regret must be approximated. ~\citet{dennis2020emergent}, the pioneer UED work, introduced a principled level generation based on the regret objective and proposed the {\em PAIRED} algorithm, where regret is estimated by the difference between the returns attained by an antagonist agent and the protagonist (student) agent. Later on, ~\citet{jiang2021replay} introduced {\em PLR$^\perp$} which combines DR with regret using {\em Positive Value Loss} (PVL), an approximation of regret based on Generalized Advantage Estimation (GAE;~\cite{schulman2015high}):
\begin{align}
\normalfont\textsc{PVL}^\theta(\pi) &= \frac{1}{T} \sum_{t=0}^{T} \max \left( \sum_{k=t}^{T} (\gamma \lambda)^{k-t} \delta^{\theta}_k, 0 \right), \label{eq:gae}     
\end{align}
where $\lambda$ and $T$ are the GAE discount factor and MDP horizon, respectively. $\delta^{\theta}_k$ is the TD-error at time step $k$ for $\theta$. The state-of-the-art UED algorithm, {\em ACCEL}~\cite{parker2022evolving}, improves PLR$^\perp$~\cite{jiang2021replay} by replacing its random level generation with an editor that mutates previously curated levels to gradually introduce complexity into the curriculum.

% The goal of the teacher's policy $\Lambda$ is to generate a distribution over the set of environment parameters, $\Delta(\Theta)$, to train the student to be robust to any $\theta \in \Theta$:
% \begin{align}
% & \Lambda: \Pi \rightarrow \Delta(\Theta) \quad \text{s.t.} \quad \mathop{\max}\limits_{\pi} V^\theta(\pi)= \mathop{\max}\limits_{\pi} \mathbb{E}_{\tau \sim \pi} V^\theta(\tau) = \mathop{\max}\limits_{\pi} \mathbb{E}_{\pi} \Big[\sum_{t=0}^T r_t^\theta \cdot \gamma^t \Big] \nonumber 
% \end{align}
% where $\Pi$ is the set of possible policies of the teacher, $r^{\theta}_t$ is the reward received by student policy $\pi$ in an environment conditioned by $\theta$ at time step $t$, and $T$ is the MDP horizon. 


\section{Related Work} \label{section:related_works}
It is important to note that regret-based UED approaches provide a minimax regret guarantee at Nash Equilibrium; however, they provide no explicit guarantee of convergence to such equilibrium. \citet{beukman2024Refining} demonstrated that the minimax regret objective does not necessarily align with learnability: an agent may encounter UPOMDPs with high regret on certain levels where it already performs optimally (given the partial observability constraints), while there exist other levels with lower regret where it could still improve. Consequently, selecting levels solely based on regret can lead to {\em regret stagnation}, where learning halts prematurely. This suggests that focusing exclusively on minimax regret may inhibit the exploration of levels where overall regret is non-maximal, but opportunities for acquiring transferable skills for generalization are significant. Thus, there is a compelling need for a complementary objective, such as novelty, to explicitly guide level selection towards enhancing zero-shot generalization performance and mitigating regret stagnation.

The {\em Paired Open-Ended Trailblazer} (POET;~\cite{wang2019poet}) algorithm computes novelty based on environment encodings---a vector of parameters that define level configurations. POET maintains a record of the encodings from previously generated levels and computes the novelty of a new level by measuring the average distance between the k-nearest neighbors of its encoding. However, this method for computing novelty is domain-specific and relies on human expertise in designing environment encodings, posing challenges for scalability to complex domains. Moreover, due to UED's underspecified nature, where free parameters may yield a one-to-many mapping between parameters and environments instances, each inducing distinct agent behaviors, quantifying novelty based on parameters alone is futile. 

{\em Enhanced POET} (EPOET;~\cite{wang2020enhanced}) improves upon its predecessor by introducing a domain-agnostic approach to quantify a level's novelty. EPOET is grounded in the insight that novel levels offer new insights into how the behaviors of agents within them differ. EPOET evaluates both active and archived agents' performance in each environment, converting their performance rankings into rank-normalized vectors. The level's novelty is then computed by measuring the Euclidean distance between these vectors. Despite addressing POET's domain-specific limitations, EPOET encounters its own challenges. The computation of rank-normalized vectors only works for population-based approaches as it requires evaluating multiple trained student agents and incurs substantial computational costs. Furthermore, EPOET remains curriculum-agnostic, as its novelty metric relies on the ordering of raw returns within the agent population, failing to directly assess whether the environment elicits rarely observed states and actions in the existing curriculum.

{\em Diversity Induced Prioritized Level Replay} (DIPLR;~\cite{li2023effective}), calculates novelty using the Wasserstein distance between occupancy distributions of agent trajectories from different levels. DIPLR maintains a level buffer and determines a level's novelty as the minimum distance between the agent's trajectory on the candidate level and those in the buffer. While DIPLR incorporates the agent’s experiences into its novelty calculation, it faces significant scalability and robustness issues. First, relying on the Wasserstein distance is notoriously costly. Additionally, DIPLR requires pairwise distance computations between all levels in the buffer, causing computational costs to grow exponentially with more levels. Finally, although DIPLR promotes diversity within the active buffer, it fails to evaluate whether state-action pairs in the current trajectory have already been adequately explored through past curriculum experiences, making it arguably still curriculum-agnostic. Further discussions on relevant literature can be found in Appendix~\ref{section:extended_related_work}. 

\section{Approach: CENIE}
\label{section:cenie_approach}
The limitations of prior approaches to quantifying environment novelty underscore the need for a more robust framework, motivating the development of CENIE. CENIE quantifies environment novelty through state-action space coverage derived from the agent’s accumulated experiences across previous environments in its curriculum. In single-environment online reinforcement learning, coverage within the training distribution is often linked to sample efficiency~\cite{xie2022role}, providing inspiration for the CENIE framework. Given UED’s objective to enhance a student’s generalizability across a vast and often unseen (during training) environment space, quantifying novelty in terms of state-action space coverage has several benefits. By framing novelty in this way, CENIE enables a sample-efficient exploration of the environment search space by prioritizing levels that drive the agent towards unfamiliar state-action combinations. This provides a principled basis for directing the environment design towards enhancing the generalizability of the student agent. Additionally, a distinctive benefit of this approach is that it is not confined to any particular UED or DRL algorithms since it solely involves modeling the agent's state-action space coverage. This flexibility allows us to implement CENIE atop any UED algorithm.

CENIE’s approach to novelty quantification through state-action coverage introduces three key attributes, effectively addressing the limitations of previous methods: (1) \textbf{domain-agnostic}, (2) \textbf{curriculum-aware}, and (3) \textbf{scalable}. CENIE is domain-agnostic, as it quantifies novelty solely based on the state-action pairs of the student, thus eliminating any dependency on the encoding of the environment. CENIE achieves curriculum-awareness by quantifying novelty using a model of the student's past aggregated experiences, i.e., state-action space coverage, ensuring that the selection of environments throughout the curriculum is sample-efficient with regards to expanding the student's state-action coverage. Lastly, CENIE demonstrates scalability by avoiding the computational burden associated with exhaustive pairwise comparisons or costly distance metrics.

% To model the state-action space coverage, we propose the use of {\em Gaussian Mixture Models} (GMMs) to represent the distribution of aggregated experiences from the agent's previous interactions within environments in the curriculum. Note that CENIE describes a general framework for quantifying novelty through state-action space coverage, and GMMs are simply one method among many that could be applied to model the state-action coverage. Future research may explore alternatives to model state-action space coverage within the CENIE framework (see Section ~\ref{section:future_work_limitations} in the appendix for more discussions). Not only do GMMs allow the construction of a parametric approximation of the state-action space coverage GMMs also allow us to assess the likelihood of experiences in new environments. This likelihood indicates the dissimilarity in state-action space coverage, enabling direct comparison of novelty between levels. 
% CENIE's complexity, governed by the Expectation-Maximization (EM) algorithm, is $O(tnkd^3)$, where $t$ is the number of iterations for performing the EM algorithm, $n$ is the number of data points, $k$ is the number of Gaussian kernels and $d$ is the dimensionality of the data. The algorithm's linear dependency on the number of data points enables CENIE to seamlessly model probability density distributions across an extensive horizon of past experiences. While the algorithm does exhibit cubic complexity concerning the number of dimensions due to the matrix inversion step, this constraint can be effortlessly mitigated through dimensionality reduction techniques like Principal Component Analysis (PCA). 
% By addressing the limitations of prior work on quantifying environment novelty, CENIE emerges as a landmark framework within the UED paradigm, allowing for a robust and efficient method for incorporating environment novelty into curricula generation.

\begin{figure}[h]
  \centering
  \includegraphics[width=0.5\linewidth]{figures/cenie/cenie_overview.png}
  \caption{An overview of the CENIE framework. The teacher will utilise environment regret and novelty for curating student's curriculum. $\Gamma$ contains past experiences and $\tau$ is the recent trajectory.}
  \label{fig:algo_overview}
\end{figure}

\subsection{Measuring the Novelty of a Level}
To evaluate the novelty of new environments using the agent's state-action pairs, the teacher needs to first model the student's past state-action space coverage distribution. We propose to use GMMs as they are particularly effective due to their robustness in representing high-dimensional continuous distributions~\cite{bouveyron2007high,assent2012clustering}. A GMM is a probabilistic clustering model that represents the underlying distribution of data points using a weighted combination of multivariate Gaussian components. Once the state-action distribution is modeled using a GMM, we can leverage it for density estimation. Specifically, the GMM allows us to evaluate the likelihood of state-action pairs induced by new environments, where lower likelihoods indicate experiences that are less represented in the student's current state-action space. This likelihood provides a quantitative measure of dissimilarity in state-action space coverage, enabling a direct comparison of novelty between levels. It is important to note that CENIE defines a general framework for quantifying novelty through state-action space coverage; GMMs represent just one possible method for modeling this coverage. Future research may explore alternatives to model state-action space coverage within the CENIE framework (see Section ~\ref{section:future_work_limitations} in the appendix for more discussions). 

Before detailing our approach, we first define the notations used in this section. Let $l_{\theta}$ be a particular level conditioned by an environment parameter $\theta$. We refer to $l_{\theta}$ as the candidate level, for which we aim to determine its novelty. The agent's trajectory on $l_{\theta}$ is denoted as $\tau_{\theta}$, and can be decomposed into a set of sample points, represented as $X_{\theta}=\left\{x=(s, a) \sim \tau_{\theta}\right\}$. The set of past training levels is represented by $L$ and $\Gamma=\left\{x=(s, a) \sim \tau_{L} \right\}$ is a buffer containing the state-action pairs collected from levels across $L$. We treat $\Gamma$ as the ground truth of the agent's state-action space coverage, against which we evaluate the novelty of state-action pairs from the candidate level $X_{\theta}$.

To fit a GMM on $\Gamma$, we must find a set of Gaussian mixture parameters, denoted as $\lambda_\Gamma=\left\{(\alpha_1, \mu_1, \Sigma_1), ..., (\alpha_K, \mu_K, \Sigma_K)\right\}$, that best represents the underlying distribution. Here, $K$ denotes the predefined number of Gaussians in the mixture, where each Gaussian component is characterized by its weight ($\alpha_k$), mean vector ($\mu_k$), and covariance matrix ($\Sigma_k$), with $k \in \left\{1, ..., K\right\}$. We employ the {\em kmeans++} algorithm~\cite{blomer2013simple, arthur2007k} for a fast and efficient initialization of $\lambda_\Gamma$. The likelihood of observing $\Gamma$ given the initial GMM parameters $\lambda_{\Gamma}$ is expressed as:
\begin{align}
P(\Gamma \mid \lambda_{\Gamma}) = \prod_{j=1}^J\sum_{k=1}^{K} \alpha_k \mathcal{N}(x_j \mid \mu_k, \Sigma_k) \label{eq:gmm_ground_truth}
\end{align}
where $x_j$ is a state-action pair sample from $\Gamma$. $\mathcal{N}(x_j \mid \mu_k, \Sigma_k)$ represents the multi-dimensional Gaussian density function for the $k$-th component with mean vector $\mu_k$ and covariance matrix $\Sigma_k$. To optimise $\lambda_{\Gamma}$, we use the Expectation Maximization (EM) algorithm~\cite{dempster1977maximum,redner1984mixture} because Eq. \ref{eq:gmm_ground_truth} is a non-linear function of $\lambda_{\Gamma}$, making direct maximization infeasible. The EM algorithm iteratively refines the initial $\lambda_{\Gamma}$ to estimate a new $\lambda_{\Gamma}'$ such that $P(X \mid \lambda_{\Gamma}') > P(X \mid \lambda_{\Gamma})$. This process is repeated iteratively until some convergence, i.e., $\|\ \lambda_{\Gamma}'-\lambda_{\Gamma} \| < \epsilon$, where $\epsilon$ is a small threshold.

Once the GMM is fitted, we can use $\lambda_{\Gamma}$ to perform density estimation and calculate the novelty of the candidate level $l_\theta$. Specifically, we consider the set of state-action pairs from the agent's trajectory, $X_{\theta}$, and compute their posterior likelihood under the GMM. This likelihood indicates how similar the new state-action pairs are to the learned distribution of past state-action coverage. Consequently, the novelty score of $l_\theta$ is represented as follows:
\begin{align}
\normalfont\textsc{Novelty}_{l_\theta} = -\frac{1}{\lvert X_{\theta}\rvert} \log \mathcal{L}(X_{\theta} \mid \lambda_{\Gamma}) = -\frac{1}{\lvert X_{\theta} \rvert} \sum_{t=1}^T \log p(x_t \mid \lambda_{\Gamma}) \label{eq:log_novelty}
\end{align}
where $x_t$ is a sample state-action pair from $X_{\theta}$. As shown in Eq. \ref{eq:log_novelty}, we take the negative mean log-likelihood across all samples in $X_{\theta}$ to attribute higher novelty scores to levels with state-action pairs that are less likely to originate from the aggregated past experiences, $\Gamma$. This novelty metric promotes candidate levels that induce more novel experiences for the agent during training. More details on fitting GMMs are explained in Appendix \ref{section:fit_gaussian_mixtures}.

\paragraph{Design considerations for the GMM} First, we specifically designate the state-action coverage buffer, i.e., $\Gamma$, as a First-In-First-Out (FIFO) buffer with a fixed window length. By focusing on a fixed window rather than the entire history of state-action pairs, our novelty metric avoids bias toward experiences that are outdated and have not appeared in recent trajectories. This design choice helps reduce the effects of catastrophic forgetting prevalent in DRL. Next, it is known that by allowing the adaptation of the number of Gaussians in the mixture, i.e., $K$ in Eq. \ref{eq:gmm_ground_truth}, any smooth density distribution can be approximated arbitrarily close~\cite{figueiredo2000gaussian}. Therefore, to optimize the GMM's representation of the agent's state-action coverage distribution, we fit multiple GMMs with varying numbers of Gaussians within a predefined range at each time step and select the best one based on the silhouette score~\cite{rousseeuw1987silhouettes}, an approach inspired by~\citet{portelas2019alpgmm}. The silhouette score evaluates clustering quality by measuring both intra-cluster cohesion and inter-cluster separation. This approach enables CENIE to construct a pseudo-online GMM model that dynamically adjusts its complexity to accommodate the agent's changing state-action coverage distribution.

\subsection{State-Action Space Coverage Directed Training Agent}
\label{section:state-action-coverage-training}

%%%%%%%%%%%%%%%%%%%%%%%%% pseudocodes %%%%%%%%%%%%%%%%%%%%%%%%%
\begin{algorithm}[h]
    \caption{ACCEL-CENIE}
    \label{alg:accel_cenie}
    \textbf{Input}: Level buffer size $N$, \textcolor{blue}{Component range $[K_{\text{min}}$}, \textcolor{blue}{$K_{\text{max}}]$, FIFO window size $\mathcal{W}$}, level generator $\mathcal{G}$ \\
    \textbf{Initialize}: Student policy $\pi_\eta$, level buffer $\mathcal{B}$, \textcolor{blue}{state-action buffer $\Gamma$, GMM parameters $\lambda_{\Gamma}$}
    
    \begin{algorithmic}[1]
    \STATE Generate $N$ initial levels by $\mathcal{G}$ to populate $\mathcal{B}$ 
    \STATE Collect $\pi_\eta$'s trajectories on each level in $\mathcal{B}$ and fill up $\Gamma$ 
    
    \WHILE{not converged}
    \STATE Sample replay decision, $\epsilon \sim U[0, 1]$
    \IF {$\epsilon \geq 0.5$}
    \STATE Generate a new level $l_{\theta}$ by $\mathcal{G}$
    \STATE Collect trajectories $\tau$ on $l_{\theta}$, with stop-gradient $\eta_{\perp}$ 
    \begingroup
    \color{blue}
    \STATE {Compute novelty score for $l_{\theta}$ using $\lambda_{\Gamma}$} (Eq.\ref{eq:log_novelty} and Eq.\ref{eq:replay_prob})
    \endgroup
    \STATE Compute regret score for $l_{\theta}'$ (Eq.\ref{eq:gae} and Eq.\ref{eq:replay_prob})
    \STATE Update $\mathcal{B}$ with $l_{\theta}$ if $P_{replay}(l_{\theta})$ is greater than that of any levels in $\mathcal{B}$ (Eq.\ref{eq:level_replay_weightage})
    \ELSE
    \STATE Sample a replay level $l_{\theta} \sim \mathcal{B}$ according to $P_{replay}$
    \STATE Collect trajectories $\tau$ on $l_{\theta}$
    \STATE Update $\pi_\eta$ with rewards $R(\tau)$
    \begingroup
    \color{blue}
    \STATE {Update $\Gamma$ with $\tau$ and resize to $\mathcal{W}$}
    \FOR {$k$ in range $K_{\text{min}}$ to $K_{\text{max}}$}
    \STATE Fit a GMM$_k$ with $k$ components on $\Gamma$ and compute its silhouette score
    \ENDFOR
    \STATE Select GMM parameters with the highest silhouette score to replace $\lambda_{\Gamma}$
    \endgroup
    \STATE Perform edits on $l_{\theta}$ to produce $l_{\theta}'$
    \STATE Collect trajectories $\tau$ on $l_{\theta}'$, with stop-gradient $\eta_{\perp}$ 
    \begingroup
    \color{blue}
    \STATE {Compute novelty score for $l_{\theta}'$ using $\lambda_{\Gamma}$} (Eq.\ref{eq:log_novelty} and Eq.\ref{eq:replay_prob})
    \endgroup
    \STATE Compute regret score for $l_{\theta}'$ (Eq.\ref{eq:gae} and Eq.\ref{eq:replay_prob})
    \STATE Update $\mathcal{B}$ with $l_{\theta}'$ if $P_{replay}(l_{\theta}')$ is greater than that of any levels in $\mathcal{B}$ (Eq.\ref{eq:level_replay_weightage})
    \ENDIF
    \ENDWHILE
    \end{algorithmic}
\end{algorithm}
%%%%%%%%%%%%%%%%%%%%%%%%% pseudocodes %%%%%%%%%%%%%%%%%%%%%%%%%

With a scalable method to quantify the novelty of levels, we demonstrate its versatility and effectiveness by deploying it on top of the leading UED algorithms, PLR$^\perp$ and ACCEL. For convenience, in subsequent sections, we will refer to this CENIE-augmented methodology of PLR$^\perp$ and ACCEL using GMMs as PLR-CENIE and ACCEL-CENIE, respectively. Both PLR$^\perp$ and ACCEL utilize a replay mechanism to train their students on the highest-regret levels curated within the level buffer. To integrate CENIE within these algorithms, we use normalized outputs of a prioritization function to convert the level scores (novelty and regret) into level replay probabilities ($P_{S}$):
\begin{align}
P_{S} = \frac{h(S_i)^{\beta}}{\sum_j h(S_j)^{\beta}} \label{eq:replay_prob}
\end{align}
where $h$ is a prioritization function (e.g. rank) with a tunable temperature $\beta$ that defines the prioritization of levels with regards to any arbitrary score $S$. Following the implementations in PLR$^\perp$ and ACCEL, we employ $h$ as the rank prioritization function, i.e., $h(S_i) = 1/{\text{rank}(S_i)}$, where $\text{rank}(S_i)$ is the rank of level score $S_i$ among all scores sorted in descending order. In ACCEL-CENIE and PLR-CENIE, we use both the novelty and regret scores to determine the level replay probability:
\begin{align}
P_{replay} = \alpha \cdot P_N + (1-\alpha) \cdot P_R
\label{eq:level_replay_weightage}
\end{align}
where $P_N$ and $P_R$ are the novelty-prioritized probability and regret-prioritized probability respectively, and $\alpha$ allows us to adjust the weightage of each probability. The complete procedures for ACCEL-CENIE are provided in Algorithm \ref{alg:accel_cenie}, and for PLR-CENIE in the appendix (see Algorithm \ref{alg:plr_cenie}). Key steps specific to CENIE using GMMs are highlighted in \textcolor{blue}{blue}.


\section{Experiments} \label{section:experiment_section}
In this section, we benchmark PLR-CENIE and ACCEL-CENIE against their predecessors and a set of baseline algorithms: Domain Randomization (DR), Minimax, PAIRED, and DIPLR. The technical details of each algorithm are presented in Appendix \ref{app:baseline_algos}. We empirically demonstrated the effectiveness of CENIE on three distinct domains: Minigrid, BipedalWalker, and CarRacing. Minigrid is a partially observable navigation task under discrete control with sparse rewards, while BipedalWalker and CarRacing are partially observable continuous control tasks with dense rewards. Due to the complexity of mutating racing tracks, CarRacing is the only domain where ACCEL and ACCEL-CENIE are excluded. The experiment details are provided in Appendix \ref{section:implementation_details}. Following standard UED practices, all agents were trained using Proximal Policy Optimization (PPO; \cite{schulman2017proximal}) across the domains, and we present their zero-shot performance on held-out tasks. We also conducted ablation studies to assess the isolated effectiveness of CENIE's novelty metric (see Appendix \ref{section:ablation_studies}).

For reliable comparison, we employ the standardized DRL evaluation metrics \cite{agarwal2021deep}, presenting the aggregate inter-quartile mean (IQM) and optimality gap plots after normalizing the performance with a min-max range of solved-rate/returns. Specifically, IQM focuses on the middle 50\% of combined runs, discarding the bottom and top 25\%, thereby providing a robust measure of overall performance. Optimality gap captures the amount by which the algorithm fails to meet a desirable target (e.g., 95\% solved rate), beyond which further improvements are considered unimportant. Higher IQM and lower optimality gap scores are better. The hyperparameters for the algorithms in each experiment are specified in the appendix.

% Note that the state-action spaces in these domains are highly dimensional, particularly in Minigrid and CarRacing, where the student processes visual observations (i.e., RGB images) resulting in state-action space dimensions of 257 and 259, respectively. In BipedalWalker, the state-action space has 28 dimensions. Evaluating the performance of CENIE-augmented algorithms in these domains enables us to empirically validate the robustness of GMMs in handling high-dimensional spaces.


\subsection{Minigrid Domain}
\label{subsection:minigrid}
First, we validated the CENIE-augmented methods in Minigrid~ \cite{dennis2020emergent,MinigridMiniworld23}, a popular benchmark due to its ease of interpretability and customizability. Given its sparse reward signals and partial observability, navigating Minigrid requires the agent to explore multiple possible paths before successfully solving the maze and receiving rewards for policy updates. Therefore, Minigrid is an ideal domain to validate the exploration capabilities of the CENIE-augmented algorithms.

\begin{figure}[h]
  \centering
  \includegraphics[width=1.0\linewidth]{figures/cenie/mg_results.png}
  \caption{Zero-shot transfer performance in eight human-designed test environments. The plots are based on the median and interquartile range of solved rates across 5 independent runs.}
  \label{fig:mg_results}
  % \vspace{-1em}
\end{figure}


Following prior UED works, we train all student agents for 30k PPO updates (approximately 250 million steps) and evaluate their generalization on eight held-out environments (see Figure \ref{fig:mg_results}). Figure \ref{fig:mg_results} demonstrates that ACCEL-CENIE outperforms ACCEL in all testing environments. Moreover, PLR-CENIE shows significantly better performance in seven test environments compared to PLR$^\perp$. This underscores the ability of CENIE's novelty metric to complement the UED framework, particularly in improving generalization performance beyond the conventional learning potential metric, regret. Further empirical validation in Figure \ref{figure:mg_iqm} confirms ACCEL-CENIE's superiority over ACCEL in both IQM and optimality gap. PLR-CENIE also outperforms its predecessor, PLR$^\perp$, by a significant margin. Notably, PLR-CENIE's performance is able to match ACCEL's, which is significant considering PLR-CENIE uses a random generator while ACCEL uses an editing mechanism to introduce gradual complexity to environments. 

\begin{figure}[htbp]
\centering
\subfigure[]{
    \begin{minipage}[t]{0.5\linewidth}
    \centering
    \includegraphics[width=2.6in]{figures/cenie/mg_iqm.png}
    \end{minipage}%
    \label{figure:mg_iqm}
}%
\subfigure[]{
    \begin{minipage}[t]{0.5\linewidth}
    \centering
    \includegraphics[width=2.6in]{figures/cenie/mg_largemaze.png}
    \end{minipage}%
    \label{figure:mg_largemaze}
}%
\centering
\caption{(a) Aggregate zero-shot transfer performance in Minigrid domain across 5 independent runs. (b) Zero-shot test performance of PLR$^\perp$, PLR-CENIE, ACCEL, and ACCEL-CENIE on PerfectMazeLarge across 5 independent runs.}
\label{fig:mg_iqm_largemaze_results}
% \vspace{-1em}
\end{figure}

Beyond the normal-size testing mazes, we consider a more challenging evaluation setting. We evaluate the fully-trained student policy of PLR$^\perp$, PLR-CENIE, ACCEL, and ACCEL-CENIE on \texttt{PerfectMazeLarge} (shown in Figure \ref{figure:mg_largemaze}), an out-of-distribution environment which has $51 \times 51$ tiles and a episode length of 5000 timesteps -- a much larger scale than training levels. We evaluate the agents for 100 episodes (per seed), using the same checkpoints in Figure \ref{fig:mg_results}. ACCEL-CENIE and ACCEL achieved comparable zero-shot transfer performance. Notably, PLR-CENIE achieved close to 50\% solved rate, matching ACCEL's performance. This is a significant improvement from PLR$^\perp$, which had less than a 10\% solved rate. 


\subsection{BipedalWalker Domain}
\begin{figure}[ht]
  \centering
  \includegraphics[width=1.0\linewidth]{figures/cenie/bw_results.png}
  \caption{Student's generalization performance on 6 BipedalWalker testing environments during training. Each curve is measured across 5 independent runs (mean and standard error).}
  \label{fig:bw_results}
  % \vspace{-1em}
\end{figure}

We also evaluated the CENIE-augmented algorithms in the BipedalWalker domain~\cite{wang2019poet,parker2022evolving}, which is a partially observable continuous domain with dense rewards. We train all the algorithms for 30k PPO updates ($\sim$1B timesteps) and then evaluate their generalization performance on six distinct test environments: BipedalWalker (default), Hardcore, Stair, PitGap, Stump, and Roughness (visualized in Figure~\ref{figure:bw_domain}). To monitor the student's generalization performance evolution, we assess the student policy every 100 PPO updates across six testing environments during the training period.


In Figure \ref{fig:bw_results}, ACCEL-CENIE outperforms ACCEL in five testing environments, with both achieving parity in the Roughness challenge, establishing ACCEL-CENIE as the leading UED algorithm in BipedalWalker. Similarly, PLR-CENIE consistently outperforms PLR$^\perp$ across all testing instances, except for the Stump challenge, where both algorithms exhibit similar performance. We present the aggregate results after min-max normalization (with range=[0, 300] on all test environments) in Figure \ref{figure:bw_iqm}. Both ACCEL-CENIE and PLR-CENIE exhibit better performance compared to their predecessors in the IQM and optimality gap metrics. Notably, ACCEL-CENIE outperforms all benchmarks by a substantial margin, achieving close to 55\% of optimal performance. 

\begin{table}[h]
\caption{Coverage of state-action space across 30k PPO updates in the BipedalWalker domain.}
\label{tab:state_action_coverage_percentage}
\centering
\begin{tabular}{lcccc}
\toprule
\textbf{} & PLR$^\perp$ & PLR-CENIE & ACCEL & ACCEL-CENIE \\ 
\midrule
\begin{tabular}[c]{@{}c@{}}State-action \\ Space Coverage\end{tabular} &43.4\% &55.3\% &42.5\% &47.6\% \\ 
\bottomrule
\end{tabular}
% \vspace{-1em}
\end{table}

Next, we tracked the evolution of state-action space coverage throughout training to evaluate the impact of CENIE’s novelty objective on the curriculum’s exploration of the state-action space. During training, state-action pairs encountered by the agent were collected for both PLR$^\perp$ and ACCEL, along with their CENIE-augmented versions. To visualize the distribution of these high-dimensional state-action pairs, we applied t-distributed Stochastic Neighbor Embedding (t-SNE;~\cite{van2008visualizing}) to project them into a 2-D space. The resulting evolution plot and detailed implementation steps are provided in Appendix \ref{section:bipedal-walker-extended}. Afterwards, we quantified state-action space coverage by discretizing the 2-D scatterplot into cells and calculating the percentage of total cells occupied by each algorithm. As shown in Table \ref{tab:state_action_coverage_percentage}, CENIE drives ACCEL-CENIE and PLR-CENIE to achieve significantly broader state-action coverage by the end of 30k PPO updates compared to their predecessors. This evidence supports that the inclusion of CENIE's novelty objective for level replay prioritization contributes to broader state-action space coverage. 

\begin{figure}[ht]
    \centering
    \includegraphics[width=0.65\linewidth]{figures/cenie/level_composition.png}
    \caption{Difficulty composition of levels replayed by ACCEL and ACCEL-CENIE during training.}
    \label{fig:level_composition}
    % \vspace{-1em}
\end{figure}

To understand ACCEL-CENIE's improvement over ACCEL, we analyzed the difficulty composition of replayed levels at various training intervals across five seeds, as shown in Figure \ref{fig:level_composition}. Level difficulty is assessed based on environment parameters such as stump height and pit gap width, using metrics adapted from \citet{wang2019poet} (details in Appendix~\ref{section:bipedal-walker-extended}). It is evident that ACCEL predominantly favors ``Easy'' to ``Moderate'' difficulty levels, whereas ACCEL-CENIE progressively incorporates ``Challenging'' levels into its replay selection throughout training. 

The disparity in level difficulty distribution between ACCEL and ACCEL-CENIE is a critical factor in understanding their observed performance differences. ACCEL's training curriculum tends to remain within a comfort zone, consistently selecting a limited subset of simpler levels where the agent experiences high regret. However, this can be problematic when considering the regret stagnation problem. Specifically, in the event where the easier levels exhibit {\em irreducible regret}, it can restrict the agent's exposure to more complex scenarios, thereby constraining its generalization potential. In contrast, ACCEL-CENIE’s integration of a novelty objective actively selects challenging levels, pushing the agent beyond its comfort zone into unfamiliar, complex environments. This novelty-based regularization fosters the exploration of under-explored regions in the state-action space, even if regret levels are low, thereby enhancing the agent’s generalization capabilities. Furthermore, with a mutation-based approach like ACCEL, this environment selection strategy may generate or mutate new levels with high learning potential, further enriching the training curriculum.


% Note that our figure differs from Figure 12 in \citet{parker2022evolving} which shows the difficulty distribution of the levels \textbf{generated and added into the buffer}, but not the actual levels selected by the teacher for the student to \textbf{replay/train on}. On that note, this also demonstrates that CENIE remedies an inefficiency in the original ACCEL algorithm, where mutation-based generation constantly produces high complexity levels but are not selected for student training due to solely depending on regret for level prioritization. 


\begin{figure}[htbp]
\centering
\subfigure[]{
    \begin{minipage}[t]{0.5\linewidth}
    \centering
    \includegraphics[width=2.6in]{figures/bw_domain.png}
    \end{minipage}%
    \label{figure:bw_domain}
}%
\subfigure[]{
    \begin{minipage}[t]{0.5\linewidth}
    \centering
    \includegraphics[width=2.6in]{figures/cenie/bw_iqm_results.png}
    \end{minipage}%
    \label{figure:bw_iqm}
}%
\centering
\caption{(a) BipedalWalker domain and (b) Aggregate zero-shot transfer performance in BipedalWalker.}
\label{fig:bw_cr_domain}
% \vspace{-1em}
\end{figure}

\subsection{CarRacing Domain}
Finally, we evaluated the effectiveness of CENIE by implementing it on PLR$^\perp$ within the \texttt{CarRacing} domain~\cite{brockman2016openai,jiang2021replay}. In this domain, the teacher manipulates the curvature of racing tracks using Bézier curves defined by a sequence of 12 control points, while the student drives on the track under continuous control with dense rewards. We train the students in each algorithm for 2.75k PPO updates ($\sim$5.5M steps), after which we test the zero-shot transfer performance of the different algorithms on 20 levels replicating real-world Formula One (F1) tracks introduced by \citet{jiang2021replay}. These tracks are guaranteed to be OOD as their configuration cannot be defined by Bézier curves with only 12 control points. The middle image in Figure \ref{fig:bw_cr_domain}b shows a track generated by domain randomization and the rightmost image shows a bird's-eye view of the F1-USA benchmark track.

\begin{figure}[htbp]
\centering
\subfigure[]{
    \begin{minipage}[t]{0.5\linewidth}
    \centering
    \includegraphics[width=2.6in]{figures/cr_domain.png}
    \end{minipage}%
    \label{figure:cr_domain}
}%
\subfigure[]{
    \begin{minipage}[t]{0.5\linewidth}
    \centering
    \includegraphics[width=2.6in]{figures/cenie/cr_iqm_results.png}
    \end{minipage}%
    \label{figure:cr_iqm}
}%
\centering
\caption{(a) CarRacing domain and (b) Aggregate zero-shot transfer performance in CarRacing.}
\label{fig:bw_cr_iqm_results}
% \vspace{-1em}
\end{figure}


The aggregate performance after min-max normalization of all algorithms is summarized in Figure \ref{fig:bw_cr_iqm_results}b. Note that the min-max range varies across F1 tracks due to different specifications on the maximum episode steps (see Table \ref{tab:car_racing_min_max} in the appendix for more details). Once again, the CENIE-augmented algorithm, PLR-CENIE, achieves the best generalization performance in both IQM and optimality gap scores. Table \ref{tab:carracing_all_results} in the appendix shows the zero-shot transfer returns on all 20 F1 tracks. PLR-CENIE consistently outperforms or matches the best-performing baseline on all tracks.


\begin{wrapfigure}{r}{0.35\textwidth}
    \centering
    \vspace{-1.5em}
    \includegraphics[width=1.0\linewidth]{figures/cenie/regret_comparison_cr.png}
    \vspace{-1.5em}
    \caption{Total regret in level replay buffer for PLR$^\perp$ and PLR-CENIE over training in CarRacing.}
    \label{fig:regret_compairson}
\end{wrapfigure}

Figure \ref{fig:regret_compairson} presents the total regret in the level replay buffer for both PLR$^\perp$ and PLR-CENIE throughout the training process. Interestingly, PLR-CENIE maintains comparable, or even slightly higher, levels of regret across the training distribution, despite not directly optimizing for it. This outcome suggests that CENIE's novelty objective synergizes with the discovery of high-regret levels, providing counterintuitive evidence that optimizing solely for regret is not the only, nor necessarily the most effective, strategy for identifying levels with high learning potential (as approximated by regret). Intuitively, value predictions are inherently less reliable in regions of lower coverage density--areas characterized by higher entropy or high uncertainty regarding optimal actions--since these regions are less frequently sampled for agent's learning. These high-entropy regions are prime candidates for high-regret outcomes, especially when using a bootstrapped regret estimate, as in Eq.~\ref{eq:gae}, due to the value estimation error in such states. By pursuing novel environments based on coverage, CENIE indirectly enhances the discovery of high-regret states, highlighting that novelty-driven autocurricula can effectively complement regret-based methods in uncovering diverse and challenging training scenarios.



\section{Conclusion}
In this paper, we introduced Coverage-based Evaluation of Novelty In Environment (CENIE), a scalable, domain-agnostic, and curriculum-aware framework for quantifying environment novelty in UED. We then proposed an implementation of CENIE that models this coverage and measures environment novelty using Gaussian Mixture Models. By incorporating CENIE with existing UED algorithms, we validated the framework's effectiveness in enhancing agent exploration capabilities and zero-shot transfer performance across three distinct benchmark domains. This promising approach marks a significant step towards unifying novelty-driven exploration and regret-driven exploitation for training generally capable RL agents. We encourage motivated readers to refer to the appendix for further studies and discussions on CENIE.



\section*{Acknowledgments}
This research/project is supported by the National Research Foundation Singapore and DSO National Laboratories under the AI Singapore Programme (AISG Award No: AISG2-RP-2020-017) and Lee Kuan Yew Fellowship awarded to Pradeep Varakantham.



\documentclass{article}


% if you need to pass options to natbib, use, e.g.:
%     \PassOptionsToPackage{numbers, compress}{natbib}
\PassOptionsToPackage{sort,numbers}{natbib}
% before loading neurips_2024




% to compile a preprint version, e.g., for submission to arXiv, add add the
% [preprint] option:
\usepackage[preprint]{neurips_2024}


% to compile a camera-ready version, add the [final] option, e.g.:
%     \usepackage[final]{neurips_2024}


% to avoid loading the natbib package, add option nonatbib:
%    \usepackage[nonatbib]{neurips_2024}


\usepackage[utf8]{inputenc} % allow utf-8 input
\usepackage[T1]{fontenc}    % use 8-bit T1 fonts
\usepackage{hyperref}       % hyperlinks
\usepackage{url}            % simple URL typesetting
\usepackage{booktabs}       % professional-quality tables
\usepackage{amsfonts}       % blackboard math symbols
\usepackage{nicefrac}       % compact symbols for 1/2, etc.
\usepackage{microtype}      % microtypography
\usepackage{xcolor}         % colors
\usepackage{amsmath}
\usepackage{amssymb}
\usepackage{graphicx}
\usepackage{systeme}   
\usepackage{amsthm}
\usepackage{enumitem}
\usepackage[thinc]{esdiff}
\usepackage{amsmath}
\usepackage{amssymb}
\usepackage{mathtools}
\usepackage{amsthm}
\usepackage{bbm}
\usepackage{mathtools}
\usepackage{bm}
\usepackage{enumitem}
\usepackage{tcolorbox}
\usepackage{subfigure}



\usepackage[capitalise]{cleveref}





\renewenvironment{quote}
  {\list{}{\leftmargin=1em \rightmargin=1em}\item\relax}
  {\endlist}


\DeclarePairedDelimiter{\ceil}{\lceil}{\rceil}

% if you use cleveref..


\newcommand{\cA}{\mathcal{A}}
\newcommand{\cG}{\mathcal{G}}
\newcommand{\cV}{\mathcal{V}}
\newcommand{\cD}{\mathcal{D}}
\newcommand{\cM}{\mathcal{M}}
\newcommand{\cN}{\mathcal{N}}
\newcommand{\cH}{\mathcal{H}}
\newcommand{\cC}{\mathcal{C}}
\newcommand{\bN}{\mathbb{N}}
\newcommand{\bR}{\mathbb{R}}
\newcommand{\bS}{\mathbb{S}}
\newcommand{\bP}{\mathbb{P}}

\newcommand{\1}{\mathbf{1}}
\newcommand{\0}{\mathbf{0}}






\newcommand{\norm}[1]{ \left\| #1 \right\| }
\newcommand{\softmax}{\mathrm{softmax}}
\DeclareMathOperator{\Conv}{Conv}
\DeclareMathOperator{\LN}{LN}
\DeclareMathOperator{\Rank}{Rank}
\DeclareMathOperator{\SRank}{SRank}

\DeclareMathOperator{\dist}{dist}
\DeclareMathOperator{\diag}{diag}
\DeclareMathOperator{\ew}{\leq_{ew}}
\DeclareMathOperator{\res}{res}
\DeclareMathOperator{\Tr}{Tr}
\DeclareMathOperator*{\argmin}{arg\,min}

\DeclareMathOperator{\decay}{decay}
\DeclareMathOperator{\RoPE}{RoPE}

%%%%%%%%%%%%%%%%%%%%%%%%%%%%%%%%
% THEOREMS
%%%%%%%%%%%%%%%%%%%%%%%%%%%%%%%%
\theoremstyle{plain}
\newtheorem{theorem}{Theorem}[section]
\newtheorem{proposition}[theorem]{Proposition}
\newtheorem{lemma}[theorem]{Lemma}
\newtheorem{corollary}[theorem]{Corollary}
\theoremstyle{definition}
\newtheorem{definition}[theorem]{Definition}
\newtheorem{assumption}[theorem]{Assumption}
\newtheorem{remark}[theorem]{Remark}



\title{On the Emergence of Position Bias in Transformers}

% The \author macro works with any number of authors. There are two commands
% used to separate the names and addresses of multiple authors: \And and \AND.
%
% Using \And between authors leaves it to LaTeX to determine where to break the
% lines. Using \AND forces a line break at that point. So, if LaTeX puts 3 of 4
% authors names on the first line, and the last on the second line, try using
% \AND instead of \And before the third author name.


\author{Xinyi Wu$^{1}$\qquad Yifei Wang$^2$ \qquad Stefanie Jegelka$^{3,2}$\qquad Ali Jadbabaie$^{1}$\\  
$^1$MIT IDSS \& LIDS \qquad $^2$MIT CSAIL \qquad $^3$TU Munich\\
\texttt{\{xinyiwu,yifei\_w,stefje,jadbabai\}@mit.edu}}



\begin{document}


\maketitle


\begin{abstract}
Recent studies have revealed various manifestations of position bias in transformer architectures, from the ``lost-in-the-middle" phenomenon to attention sinks, yet a comprehensive theoretical understanding of how attention masks and positional encodings shape these biases remains elusive. This paper introduces a novel graph-theoretic framework to analyze position bias in multi-layer attention.  Modeling attention masks as directed graphs, we quantify how tokens interact with contextual information based on their sequential positions. We uncover two key insights: First, causal masking inherently biases attention toward earlier positions, as tokens in deeper layers attend to increasingly more contextualized representations of earlier tokens. Second, we characterize the competing effects of the causal mask and relative positional encodings, such as the decay mask and rotary positional encoding (RoPE): while both mechanisms introduce distance-based decay within individual attention maps, their aggregate effect across multiple attention layers -- coupled with the causal mask -- leads to a trade-off between the long-term decay effects and the cumulative importance of early sequence positions. 
Through controlled numerical experiments, we not only validate our theoretical findings but also reproduce position biases observed in real-world LLMs. Our framework offers a principled foundation for understanding positional biases in transformers, shedding light on the complex interplay of attention mechanism components and guiding more informed architectural design.
\end{abstract}

\section{Introduction}

The attention mechanism is central to transformer architectures~\cite{Vaswani2017AttentionIA}, which form the backbone of state-of-the-art foundation models, including large language models (LLMs). Its success lies in its ability to dynamically weigh input elements based on their relevance, enabling efficient handling of complex dependencies~\cite{Kim2017StructuredAN, Bahdanau2014NeuralMT}. However, despite this widespread success, many questions remain unanswered regarding how these mechanisms process information and the artifacts they may introduce. Developing a deeper theoretical understanding of their inner workings is essential -- not only to better interpret existing models but also to guide the design of more robust and powerful architectures.


One particularly intriguing aspect that demands such a theoretical investigation is \emph{position bias}, i.e., the bias of the model to focus on certain regions of the input, which significantly impacts the performance and reliability of transformers and LLMs~\cite{Zheng2023JudgingLW, Wang2024EliminatingPB,Hou2023LargeLM}. For instance, these models often suffer from the ``lost-in-the-middle" problem, where retrieval accuracy significantly degrades for information positioned in the middle of the input sequence compared to information at the beginning or end~\cite{liu2024lost, Zhang2024FoundIT,Guo2024SerialPE}. Similarly, in-context learning is highly sensitive to the order of illustrative examples: simply shuffling independently and identically distributed (i.i.d.) examples can lead to significant performance degradation~\cite{Min2022RethinkingTR, Lu2021FantasticallyOP, zhao2021calibrateuseimprovingfewshot}. Moreover, recent research has also revealed that attention sinks~\cite{Xiao2023EfficientSL, Gu2024WhenAS, guo2024activedormantattentionheadsmechanistically} -- positions that attract disproportionately high attention weights -- arise at certain positions regardless of semantic relevance, suggesting an inherent positional bias.




These empirical findings suggest that while transformers effectively encode and process positional information through the combined use of attention masks and positional encodings (PEs)~\cite{Wang2024EliminatingPB,Fan2025InvICL}, these design elements also appear to introduce systematic positional biases, often independent of semantic content. This raises a fundamental and intriguing question about the role of positional information in attention mechanisms:
\begin{quote}
\hspace{-3ex}\textit{How do attention masks and positional encodings shape position bias in transformers?}
\end{quote}



% Understanding of these position-dependent effects is essential for improving model reliability, interpretability, and performance.
% Although attention masks and positional encodings (PEs) were initially designed to enable transformers to process sequential information effectively, growing evidence reveals that these mechanisms can also lead to unintended artifacts in model behavior. As transformer architectures continue to set the standard in state-of-the-art language modeling, a deeper understanding of these position-dependent effects is essential for improving model reliability, interpretability, and performance.

\begin{table*}[t] 
    \centering
       \caption{Summary of our results and their connections to empirical observations on position bias reported in the literature. }   
       \vspace{1ex}
     \resizebox{\textwidth}{!}{
    \begin{tabular}{cc}
    \toprule
        Empirical Observations on Position Bias & Our Results \\
         \midrule
         Positional information induced by the causal mask~{\cite{Kazemnejad2023TheIO, Wang2024EliminatingPB, Barbero2024TransformersNG}} & Theorem~\ref{thm: causal_mask}, Section~\ref{exp:causal}\\
         Decay effects induced by relative PEs~\cite{su2023roformerenhancedtransformerrotary} & Lemma~\ref{lem:decay_mask_attn}-\ref{lem: attn_rope}, Section~\ref{sec:depth}\\
         Interplay between the causal mask and relative PEs~\cite{Wang2024EliminatingPB} & Theorem~\ref{thm:aggregate_decay_effect}-\ref{thm: rope}, Section~\ref{sec:depth}\\
         Attention sinks~\cite{Xiao2023EfficientSL,Gu2024WhenAS} & Theorem~\ref{thm: causal_mask}-\ref{thm:prefix_mask}, Appendix~\ref{app:attn_sinks}\\
         The ``lost-in-the-middle" phenomenon~\cite{liu2024lost} & Section~\ref{exp:causal}\\
    \bottomrule
    \end{tabular}
    }
    \label{tab:summary}
\end{table*}





To address the question, we propose a novel graph-theoretic framework for analyzing attention score distributions in multi-layer attention settings. Building upon~\citet{Wu2024OnTR}, we model attention masks as directed graphs, enabling rigorous mathematical analysis of attention patterns. This approach proves particularly powerful for studying multi-layer attention mechanisms, as it allows us to precisely quantify how each token's contextual representation is composed from information at different positions in the sequence. By tracking the information flow through the attention layers, we can systematically examine how positional biases emerge and propagate across layers, providing insights into the complex interplay between attention masks, PEs, and the network's depth.

\textbf{Our contributions are summarized as follows:}
\begin{itemize}[leftmargin = 3ex]
    \item We develop a graph-theoretic framework that unifies and advances understanding of position bias in transformers, offering deeper insights into diverse empirical observations documented in the literature (\Cref{tab:summary}). 
    \vspace{-1ex}
    \item We show that causal masking in transformers inherently biases attention toward earlier positions in deep networks. This happens as tokens in deeper layers attend to increasingly more contextualized representations of earlier tokens, thereby amplifying the influence of initial positions. We derive analogous results for the sliding-window mask and the prefix mask, highlighting the generalizability of our framework.
    \vspace{-1ex}
    \item We uncover a nuanced interaction between causal masking and relative PEs, such as decay masks and rotary positional encoding (RoPE). Our findings highlight a trade-off in multi-layer attention networks, where local decay effects within individual layers are counterbalanced by the cumulative importance of early sequence positions. These results provide a deeper understanding of how PE and masking interact in deep attention-based architectures, with design implications about how to balance local and global context.
    \vspace{-1ex}
    \item We support our theoretical findings with experiments, empirically validating that deeper attention layers amplify the bias toward earlier parts of the sequence, while relative PEs partially mitigate this effect. Through carefully controlled numerical experiments, we further investigate the role of data in shaping position bias and how causal masking implicitly leverages positional information.
\end{itemize}

\section{Related Work}



\paragraph{Position bias in transformers}
Position bias in transformer models has emerged as a critical challenge across diverse applications. In information retrieval and ranking, \citet{liu2024lost, Guo2024SerialPE, Hou2023LargeLM, Zheng2023JudgingLW} demonstrated systematic degradation of performance due to positional dependencies. Similarly, in in-context learning, model performance can vary dramatically based solely on the order of examples \cite{Lu2021FantasticallyOP, Min2022RethinkingTR, zhao2021calibrateuseimprovingfewshot, Fan2025InvICL}. While mitigation strategies such as novel PEs  \cite{Kazemnejad2023TheIO, Zhang2024FoundIT}, alternative masking techniques~\cite{Wang2024EliminatingPB,Fan2025InvICL} and bootstrapping~\cite{Hou2023LargeLM} have been proposed, they remain task-specific and empirically driven. This gap between empirical observations and theoretical understanding highlights the need for a rigorous analysis of how transformers process and integrate positional information through attention.


\paragraph{The effect of attention masks and PEs in transformers} 
The role of attention masks and PEs in transformers has been explored from various perspectives. \citet{Yun2020OnCA} analyzed the function approximation power of transformers under different masking schemes, while \citet{Wu2024OnTR, Wu2023Demystify} investigated the role of attention masks in mitigating rank collapse. Moreover, \citet{Gu2024WhenAS} empirically examined how attention masks affect the emergence of attention sinks. As for PEs, \citet{Kazemnejad2023TheIO} studied their role in length generalization, and \citet{Barbero2024RoundAR} analyzed RoPE’s use of feature dimensions. Additionally, \citet{Wang2024EliminatingPB} empirically observed that both causal masking and RoPE introduce position dependencies in LLMs. Despite these advances, fundamental questions remain about the mechanisms through which attention masks and PEs enable transformers to process and integrate positional information, as well as the nature of the systematic positional biases that emerge as a result.

\section{Problem Setup}

\paragraph{Notation}

 We use the shorthand $[n]:=\{1,\ldots,n\}$. For a matrix \( M \), we denote its \( i \)-th row by \( M_{i,:} \) and its \( j \)-th column by \( M_{:,j} \). Throughout the analysis in the paper, we formalize the attention mask to be a directed graph $\cG$. Formally, we represent a directed graph with $N$ nodes by $\mathcal{G}$ and let $E(\cG)$ be the set of directed edges of $\cG$. A directed edge $(j,i)\in E(\cG)$  from node $j$ to $i$ in $\mathcal{G}$ means that in the attention mechanism, token $j$ serves as a direct context for token $i$ or token $i$ attends to token $j$. The set $\cN_i$ of all neighbors of node $i$ is then $\{k: (k,i)\in E(\cG)\}$.  We say a node $v$ is \emph{reachable} from node $u$ in a directed graph $\cG$ if there is a directed path $(u, n_1), (n_1, n_2), ..., (n_k,v )$ from $u$ to $v$. In the attention mechanism, this means that token $u$ serves as a direct or indirect context for token $v$.

Furthermore, we will be using the following graph-theoretic terminology (see~\Cref{fig:masks} for a schematic illustration):
\begin{definition} [Center Node]
    A node $v$ from which every node in the directed graph $\cG$ is reachable is called a \emph{center node}.
    \label{def:center_nodes}
\end{definition}

\subsection{(Masked) Attention Mechanism}
Given the representation $X\in\bR^{N\times d}$ of $N$ tokens, the raw attention score matrix is computed as 
\[Z = XW_Q (XW_K)^\top/\sqrt{d_{QK}}\,,\]
where $W_Q, W_K\in\bR^{d\times d'}$ are the query and the key matrix, respectively, and $\sqrt{d_{QK}}$ is a temperature term to control the scale of raw attention scores. Without loss of generality, we assume $d_{QK}=1$ in our analysis. To enforce a masked attention, we create a sparse attention matrix $A \in \bR^{N \times N}$ based on $Z$ whose sparsity pattern is specified by a directed graph $\cG$: we normalize $Z_{ij}$ among all allowed token attention interactions $(k,i) \in E(\cG)$ such that if $(j,i) \in E(\cG)$,
\[A_{ij} = {\softmax}_\cG (Z_{ij}) = \frac{\exp(Z_{ij})}{\sum_{k\in\cN_i}\exp(Z_{ik})}\, \;\,,\] and $A_{ij} = 0$ otherwise. 
% $A_{ij}$ relates to edge $(j,i)$ or $(i,j)$? It makes sense, just checking, all good as long as it's consistent. Check with Figure 1. 
% \sj{2nd: below $A$ is the full attention matrix. We could also have a binary masking matrix and then Hadamard product with the full attention $A$}

\subsection{Attention Update}
For our analysis, we consider %the following definition of 
single-head (masked) self-attention networks (SANs). The layerwise update rule can be written as
\vspace{-1ex}
\begin{equation*}
    A^{(t)} = \softmax_{\cG^{(t)}}\left( X^{(t)}W^{(t)}_Q (X^{(t)}W^{(t)}_K)^\top/\sqrt{d_{QK}} \right)  
\end{equation*}
\vspace{-2ex}
\begin{equation}
     X^{(t+1)} = A^{(t)}X^{(t)}W_V^{(t)}\,,\label{eq: update_no_LN}
\end{equation}
where $W_V^{(t)}\in\bR^{d\times d'}$ is the value matrix. For simplicity, throughout the paper, we assume that $d=d'$
and $\cG^{(t)}=\cG$. Yet the results can be
easily generalized to the case where masks are time-varying and satisfy regularity conditions.

\subsection{Relative Positional Encoding}
\paragraph{Decay Mask}
The decay mask represents the relative distance between two tokens by introducing an explicit bias favoring more recent tokens. Formally, it can be written as:
\begin{equation*}
     D_{ij} = 
    \begin{cases}
        {-(i-j)m} & \text{if } j \leq i \\
        0 & \text{otherwise}\,.
    \end{cases}
\end{equation*}
Then applying the decay mask is essentially 
\begin{equation}
   A_{\decay}^{(t)} = \softmax_{\cG}(X^{(t)}W_Q^{(t)}(X^{(t)}W_K^{(t)})^\top + D) \,.
   \label{eq:decay_mask_update}
\end{equation}
Note that while the decay mask formulation follows ALiBi~\cite{Press2021TrainST}, it can be generalized to more complex variants such as KERPLE~\cite{Chi2022KERPLEKR}.

\paragraph{Rotary Positional Encoding (RoPE)}

Another way to encode the relative positional information is through RoPE~\cite{su2023roformerenhancedtransformerrotary}, which applies a rotation to  query and key embeddings by an angle proportional to
the token’s position index within the sequence. Formally, the rotation operation applied to each query or key $X_{i,:}W_{\{Q,K\}}$ can be written as 
\begin{equation}
    (\hat{X}_{\{Q,K\}})_{i,:} = X_{i,:}W_{\{Q,K\}}R^d_{\Theta, i}
    \label{eq:rope_to_qk}
\end{equation}
where
\begin{equation*}
\begin{aligned}
    R^d_{\Theta, i} = &
    \resizebox{0.75\columnwidth}{!}{$
    \begin{bmatrix}
        \cos i\theta_1  & \sin i \theta_1  & 0 & 0 & \cdots & 0 & 0 \\
        -\sin i  \theta_1 & \cos i \theta_1  & 0 & 0 & \cdots & 0 & 0 \\
        0 & 0 & \cos i \theta_2 & \sin i \theta_2 & \cdots & 0 & 0 \\
        0 & 0 & -\sin i \theta_2 & \cos i \theta_2 & \cdots & 0 & 0 \\
        \vdots & \vdots & \vdots & \vdots & \ddots & \vdots & \vdots \\
        0 & 0 & & 0 & 0 & \cos i \theta_{d/2} & \sin i \theta_{d/2} \\
        0 & 0 & & 0 & 0 & -\sin i \theta_{d/2} & \cos i \theta_{d/2}
   \end{bmatrix}
    $}
\end{aligned}
\end{equation*}
is the rotation matrix with a set of pre-defined base rotational angles $\Theta = \{0\leq\theta_1\leq \cdots \leq \theta_{d/2}\}$. Then the raw attention
scores under RoPE $Z_{\RoPE}$ become
\begin{align*}
    (Z_{\RoPE})_{ij} & = (X_{i,:}W_{Q}R^d_{\theta, i})(X_{j,:}W_{K}R^d_{\theta, j})^\top\\
    & = X_{i,:}W_{Q}R^d_{\theta, i-j}W_{K}^\top X_{j,:}^\top,
\end{align*}
which distorts the original raw attention scores based on the relative token distances. The final attention scores under RoPE are calculated as $
    A^{(t)}_{\RoPE} = \softmax_{\cG}\left(Z_{\RoPE}^{(t)} \right)$.


\section{Main Results}

% \subsection{The Contextualization Effect of Attention }

In the transformer model, the attention mechanism is the sole module that allows tokens to interact with one another and incorporate contextual information from the sequence. It iteratively refines the contextual representation of each token across layers, allowing information to flow and accumulate based on relevance. This concept of contextualization through attention has its origins in the development of attention mechanisms, which predate transformers~\cite{Kim2017StructuredAN, Bahdanau2014NeuralMT}. From the perspective of contextualization, the attention mechanism can be expressed in the following form~\cite{Kim2017StructuredAN}:
\begin{align}
    X_{i,:}^{(t+1)} 
    & = \sum_{j=1}^N 
        \underbrace{(A^{(t)} \cdots A^{(0)})_{ij}}_{\mathclap{ \bP^{(t)}(z_i = j \mid X^{(0)})}} 
        \,\cdot\, 
        \underbrace{X^{(0)}_{j,:}W_V^{(0)} \cdots W_V^{(t)}}_{\mathclap{f^{(t)}(X^{(0)}_{z_i,:})}},
        \vspace{-3ex}
\end{align}
where \(z_i\) is a categorical latent variable with a sample space \(\{1, \ldots, N\}\) that selects the input \(X_{j,:}\) to provide context for token \(i\). In this formulation, \(A^{(t)}\) represents the attention matrix at layer \(t\), \(\bP^{(t)}(z_i = j \mid X^{(0)})\) denotes the cumulative probability of selecting input token \(j\) as the context for token \(i\) at depth $t$ , and \(f^{(t)}(\cdot)\) is a learned transformation function.

This probabilistic formulation reveals two key aspects of the attention mechanism: it acts as both a context selector and a feature aggregator. As a selector, it assigns probabilities $\bP^{(t)}$ that quantify the relevance of each token $j$ to target token $i$ at depth $t$. As an aggregator, it combines these selected contexts weighted by their respective probabilities $\bP^{(t)}$ to form the contextualized representation $X^{(t)}$.  Since position bias fundamentally manifests as systematic preferences in how tokens select and incorporate context from different positions, analyzing the attention mechanism's behavior is crucial for understanding these biases. By examining how attention masks and PEs affect the probability distribution $\bP^{(t)}$, we can investigate how position-dependent patterns emerge and propagate through multi-layer attention in transformers.

Finally, we adopt the following assumptions in our analysis:
\begin{enumerate}
    \item [\textbf{A1}] There exists  $C\in\bR$ such that  
    $\underset{t\in\bN}{\max}\big\{\|W_Q^{(t)}\|_2, \|W_K^{(t)}\|_2\big\} \leq C$.
    \item [\textbf{A2}] The sequence $\big\{\|\prod_{t=0}^k W_V^{(t)}\|_2\big\}_{k=0}^\infty$ is bounded.
\end{enumerate}
\textbf{A1} assumes that the key and query weight matrices are bounded, which is crucial for efficient attention computation in practice~\cite{Alman2023FastAR}, whereas \textbf{A2} is to ensure boundedness of the node representations' trajectories $X^{(t)}$ for all $t\geq 0$~\cite{Wu2024OnTR}. 

\begin{figure}
    \centering   \includegraphics[width=0.6\linewidth]{figs/icml-25-mask.pdf}
    \caption{\small{Three types of attention masks and their corresponding directed graphs $\cG$ used in the analysis (self-loops are omitted for clarity). A directed edge from token $j$ to $i$ indicates that $i$ attends to $j$. The center node(s) (\Cref{def:center_nodes}), highlighted in yellow, represent tokens that can be directly or indirectly attended to by all other tokens in the sequence. As depicted in the top row, the graph-theoretic formulation captures both direct and indirect contributions of tokens to the overall context, providing a comprehensive view of the token interactions under multi-layer attention.}
 }
 \vspace{-3ex}
    \label{fig:masks}
\end{figure}

\subsection{Attention Masks: A Graph-Theoretic View}


We first analyze the case without PEs, focusing on the effect of attention masks. A graph-theoretic perspective offers a powerful framework for analyze multi-layer attention: the flow of attention across tokens can be represented as paths in a directed graph defined by the mask, where each path captures how information is transmitted between tokens (see~\Cref{fig:masks} for an illustration). The number of steps in a path corresponds to the number of layers. By accounting for all such paths, we can quantify the cumulative influence of each token in the context computation of other tokens. 

Through the graph-theoretic view, our first result states that for a causal mask $\cG$, as tokens in deeper layers attend to increasingly more contextualized representations of earlier tokens, the context of each token converges exponentially toward the first token in the sequence.


\begin{theorem}
Let $\cG$ be the causal mask. Under $\textup{\textbf{A1}}$-$\textup{\textbf{A2}}$, 
    given $X^{(0)}\in\bR^{N\times d}$, for every token $i\in[N]$,
    \begin{equation*}
        \lim_{t\to\infty} \bP^{(t)}(z_i = 1|X^{(0)}) = 1\,.
    \end{equation*}
Moreover, there exist $0< C,\epsilon < 1$ where $N\epsilon < 1$ such that 
\begin{equation*}
        \bP^{(t)}(z_i = j|X^{(0)}) \leq  C(1-(j-1)\epsilon)^{t}\,.
    \end{equation*}
for all $1< j \leq i$ and $t\geq 0$.
\label{thm: causal_mask}
\end{theorem}

\Cref{thm: causal_mask} reveals that in multi-layer causal attention, positional bias toward earlier sequence positions intensifies with depth -- regardless of semantic content. This phenomenon arises from the nature of multi-layer attention: starting from the second layer, tokens no longer attend to raw inputs but instead to contextualized tokens, i.e., representations transformed by prior attention layers. Combined with the sequential structure of the causal mask, this iterative process amplifies the role of earlier tokens, as they influence later ones not only as direct context but also indirectly through intermediate tokens along the path. We discuss a few intriguing implications below. 

\paragraph{The role of softmax} 
      The key property that leads to the above result is that the softmax operation in the attention mechanism cannot fundamentally disconnect any directed edge in the graph $\cG$. As a result, center nodes (\cref{def:center_nodes}), which appear in the context directly or indirectly for all tokens in the sequence, will eventually gain a dominant role in the context as their direct and indirect contributions propagate through the graph. Empirically, \citet{Xiao2023EfficientSL, Gu2024WhenAS} found that changing softmax to ReLU, which can disconnect edges in the graph, indeed mitigates the emergence of attention sinks. 
\vspace{-1ex}
\paragraph{How No PE induces positional information} Previous works have observed that the causal mask alone amplifies the position bias~\cite{Yu2024MitigatePB,Wang2024EliminatingPB}. Despite these observations, it remains insufficiently understood how the causal mask captures positional information and what information is being captured. One hypothesis in~\citet{Kazemnejad2023TheIO} suggests that the causal mask may be simulating either an absolute PE or a relative PE with specific weight matrices.

\Cref{thm: causal_mask} offers a different perspective. The causal mask results in earlier tokens being utilized more frequently during computation, inducing a sequential order.
%
%inherently imposes a sequential order on tokens~\cite{Wu2024OnTR}, resulting in earlier tokens being utilized more frequently during computation. 
This %design choice introduces a 
bias aligns with the token order in the sequence. To validate this perspective, we present additional experimental results in \cref{exp:causal}, providing empirical evidence that the causal mask is not simulating any PE but instead just exhibits a bias toward the earlier parts of a sequence.


\paragraph{Trade-off between representation power and position bias} \Cref{thm: causal_mask} also highlights a trade-off between representational power and positional bias as the depth of attention layers increases. While numerous studies have demonstrated that deeper attention models are crucial for improving representation power~\cite{Yun2019AreTU, Merrill2022ThePT, Li2024ChainOT,Sanford2024OnelayerTF}, our findings reveal that these benefits come at a cost. As the model depth increases, the initial tokens in a sequence are utilized more frequently, amplifying the positional bias toward the beginning of the sequence. This trade-off underscores the importance of carefully balancing depth and positional bias in the design of attention-based architectures. 



% \subsection{Attention Masks: Sliding-Window and Prefix}

\Cref{thm: causal_mask} on the causal mask can be generalized to encompass other types of attention masks, notably the sliding-window mask~\cite{Jiang2023Mistral7, Beltagy2020LongformerTL} and the prefix mask~\cite{2020t5, Lewis2019BARTDS}. In the sliding-window mask, each token is allowed to attend to a fixed number of preceding tokens. Let $w$
denote the width of the sliding-window, representing the maximal number of tokens each token can access. The following result shows how limiting the context window size affects the propagation of contextual information in attention mechanism.
\begin{theorem}
Let $\cG$ be the sliding-window mask with width $w\geq 2$. Under $\textup{\textbf{A1}}$-$\textup{\textbf{A2}}$,  
    given $X^{(0)}\in\bR^{N\times d}$, for every token $i\in[N]$,
    \begin{equation*}\lim_{t\to\infty} \bP^{(t)}(z_i = 1|X^{(0)}) = 1\,.
    \end{equation*}
Moreover, there exist $0<C, \epsilon<1$ where $N\epsilon^{\left\lceil \frac{N-1}{w-1} \right\rceil} < 1$ such that 
\begin{equation*}
        \bP^{(t)}(z_i = j|X^{(0)}) \leq  C(1-(j-1)\epsilon^{\left\lceil \frac{N-1}{w-1} \right\rceil} )^{t/\left(2\left\lceil \frac{N-1}{w-1} \right\rceil\right)}\,.
    \end{equation*}
for all $1< j \leq i$ and $t\geq 0$.  
\label{thm:sliding-window}
\end{theorem}
The above result suggests that a smaller window size $w$ helps mitigate the model's bias toward early tokens in the sequence. However, such a moderating effect has its limit -- the contextual information will still exponentially converge toward the first token over successive layers, though at a rate determined by the ratio between the sequence length $N$ and the window size $w$.


Finally, for the case of a prefix mask, where the first $K$ tokens in the sequence serve as a prefix and all subsequent tokens attend to them, contextual information exponentially converges toward these $K$ tokens rather than being dominated by just the first one, with each of these $K$ tokens having a non-trivial influence.

\begin{theorem}
Let $\cG$ be the prefix mask with the first $K$ tokens being the prefix tokens. Under $\textup{\textbf{A1}}$-$\textup{\textbf{A2}}$, 
    given $X^{(0)}\in\bR^{N\times d}$, for every token $i\in[N]$, 
    \begin{equation*}\lim_{t\to\infty} \bP^{(t)}(z_i \in [K] |X^{(0)}) = 1\,,
        \label{eq: first_K_tokens_total}
    \end{equation*}
and there exists $\kappa > 0$ such that
 \begin{equation*} 
        \liminf_{t\to\infty} \bP^{(t)}(z_i = k|X^{(0)}) \geq  \kappa. \qquad\forall k\in [K]\,.
        \label{eq: first_k_distribution}
    \end{equation*}
Moreover, there exist $0<C,\epsilon < 1$ where $N\epsilon < 1$ such that
\begin{equation*} 
        \bP^{(t)}(z_i = j|X^{(0)}) \leq  C(1-(j-K)\epsilon)^t\,.
         \label{eq: prefix_exp}
    \end{equation*}
for all $K< j \leq i$ and $t\geq 0$.
\label{thm:prefix_mask}
\end{theorem}

\paragraph{Attention sink and center node} The above result connects the emergence of attention sinks to the structural role of center nodes in the graph $\cG$ defined by the mask. Specifically, in~\citet{Gu2024WhenAS}, the authors observed two interesting phenomena: 1) when using the sliding-window mask, attention sinks still appear on the absolute first token in the sequence, but not on the first token within each context window; 2) when using the prefix mask, attention sinks emerge on all prefix tokens, rather than just on the first token.

These empirical results align well with Theorems~\ref{thm:sliding-window} and \ref{thm:prefix_mask}. Our results suggest that the absolute first token and the prefix tokens act as center nodes for the sliding-window mask and prefix mask, respectively. The context for each token, after multi-layer attention, exponentially converges to these center nodes. This connection between attention sinks and center nodes suggests that attention sinks are not arbitrary artifacts but arise naturally from the underlying graph structure induced by the attention mask.

\subsection{Relative PEs: A Competing Decay Effect}

Having analyzed how attention masks bias the model toward the beginning of the sequence, we now shift our focus to studying PEs, the other key mechanism for representing positional information in transformers. 

Relative PE, as the name suggests, incorporates positional information by modifying the original attention scores in a way that reflects the relative positions of tokens. Among these, the decay mask~\cite{Press2021TrainST} explicitly introduces a distance-based decay effect into the attention mechanism. We begin by examining the effect of the decay mask on individual attention layers.

\begin{lemma}
    Consider the decay mask in~(\ref{eq:decay_mask_update}) where $\cG$ is causal. Under $\textup{\textbf{A1}}$-$\textup{\textbf{A2}}$, given $X^{(0)}\in\bR^{N\times d}$, there exists $C_{\max}, C_{\min} > 0$ such that for all $t\geq 0$,
    \[C_{\min} e^{-(i-j)m} \leq (A^{(t)}_{\decay})_{ij} \leq C_{\max} e^{-(i-j)m}.\]
    \vspace{-3ex}
\label{lem:decay_mask_attn}
\end{lemma}
\Cref{lem:decay_mask_attn} demonstrates that the decay mask introduces an exponential decay effect into each attention map, with the strength of the effect determined by the token distances. However, while this result characterizes the behavior of individual attention layers, the interaction between layers in a multi-layer setting leads to more intricate behaviors. Building on~\Cref{lem:decay_mask_attn} , \Cref{thm:aggregate_decay_effect} examines the cumulative effect of the decay mask across multiple layers when combined with the causal mask. 

\begin{theorem}
Consider the decay mask in~(\ref{eq:decay_mask_update}) where $\cG$ is causal. Fix $T \geq 0$. Under $\textup{\textbf{A1}}$-$\textup{\textbf{A2}}$, given $X^{(0)}\in\bR^{N\times d}$, it holds for all $t\leq T$, 
\[\bP_{_{\decay}}^{(t)}(z_i = j|X^{(0)})= \Theta\left({t+i-j \choose i-j} e^{-(i-j)m} \right)\,.\]
\label{thm:aggregate_decay_effect}
\vspace{-3ex}
\end{theorem}
Notably, if we denote\[L(x) = \log \left({t+x  \choose x} e^{-xm} \right)\,,\] then $L(x)$ is not a monotone function of the distance $x$ between two tokens. More precisely, under Stirling's approximation, the critical point, where the highest attention score occurs, is at $x^* = t/(e^m-1)\,.$ This means that increasing the decay strength $m$ decreases $x^*$, making the model more biased towards recent tokens,   whereas increasing the number of attention layers increases $x^*$, making the model more biased towards initial tokens.  

Compared to~\cref{lem:decay_mask_attn}, while the decay mask imposes a stronger decay effect on earlier tokens within individual attention layers, these tokens gain more cumulative importance across multiple layers. This trade-off between layer-wise decay and cross-layer accumulation transforms the initially monotonic decay pattern within each attention map into a more intricate, non-monotonic behavior when aggregated throughout the network.


\subsection{A Closer Look at RoPE}
Having analyzed the effect of the decay mask, which directly incorporates a distance-based decay into the attention score calculation, we now turn our attention to another popular form of relative positional encoding: RoPE~\cite{su2023roformerenhancedtransformerrotary}.

RoPE’s inherent complexity has made a clear theoretical understanding challenging. However, recent empirical observations in~\citet{Barbero2024RoundAR} suggest that in practice, LLMs tend to predominantly utilize feature dimensions that rotate slowly. This phenomenon introduces additional structure, enabling a more refined analysis of RoPE’s effects by focusing on these slowly rotating feature dimensions. For simplicity and without loss of generality, we consider the case where only the slowest-rotating feature dimensions with base rotational angle $\theta_1$ are used by the model, i.e. effectively reducing the embedding dimension to $d=2$. See~\cref{app:rope_d_geq_2} for more results on the general case $d\geq 2$.


Recall from \eqref{eq:rope_to_qk} that RoPE operates by rotating the original query and key embeddings by an angle proportional to the token's position index within the sequence. Similar to the decay mask, which incorporates distance-based decay into attention scores, RoPE adjusts raw attention scores via these rotations. To formalize this relationship mathematically, we define the original angle between query $q_i^{(t)} \vcentcolon= X^{(t)}_{i,:}W^{(t)}_{Q}$ and key $k_j^{(t)} \vcentcolon=X^{(t)}_{j,:}W^{(t)}_{K}$ as $\phi^{(t)}_{i,j}$. Then the following result analyzes how RoPE's position-dependent rotations systematically modify the computation of attention scores.

\begin{lemma}
    Let $\cG$ be the causal mask and $d=2$. Suppose for $t\geq 0$, $\|q_i^{(t)}\|_2, \|k_j^{(t)}\|_2 > 0$, and $|\phi^{(t)}_{i,j}| \leq \delta\theta_1$, where $\delta > 0$ and $(\delta+N-1)\theta_1 \leq \pi$. Then under $\textup{\textbf{A1}}$-$\textup{\textbf{A2}}$, given $X^{(0)}\in\bR^{N\times d}$, there exists $C_{\max}, C_{\min,} c, c' > 0$ such that
    \[C_{\min} e^{-c(i-j)^2\theta_1^2} \leq (A^{(t)}_{\RoPE})_{ij} \leq C_{\max} e^{-c'(i-j)^2\theta_1^2}\,.\]
    \label{lem: attn_rope}
    \vspace{-3ex}
\end{lemma}
The result shows that by solely leveraging feature dimensions that rotate slowly, RoPE effectively induces a distance-based decay effect, which aligns with the intuition in~\citet{su2023roformerenhancedtransformerrotary}. However, it is worth noting that the decay effect induced by RoPE is significantly smaller compared to that of the decay mask (\cref{lem:decay_mask_attn}). This is because the base rotational angle $\theta_1$ is typically chosen to be small, i.e. $\approx1/10000$ per token~\cite{su2023roformerenhancedtransformerrotary, Dubey2024TheL3}, resulting in a more gradual decay.

However, similar to the case of the decay mask, when considering the effect of RoPE across multiple layers of attention, the long-term decay effects within individual layers are counteracted by the increasing influence of earlier tokens given by the causal mask.

\begin{theorem}
   Fix $T> 0$. Under the same conditions as in~\cref{lem: attn_rope} for $t \leq T$, given $X^{(0)}\in\bR^{N\times d}$, there exists $c>0$ such that for all $t\leq T$, 
   \[\bP_{_{\RoPE}}^{(t)}(z_i = j|X^{(0)}) = \Theta\left({t+i-j  \choose i-j} e^{-c(i-j)^2\theta_1^2} \right)\,.\]
   \label{thm: rope}
   \vspace{-2ex}
\end{theorem}

Again, if we write\[L(x) = \log \left({t+i-j  \choose i-j} e^{-x^2\theta_1^2} \right)\,,\]
then, by implicit differentiation, the critical point $x^*$ is an increasing function of the depth $t$ and a decreasing function of the base rotational angle $\theta_1$ (see~\cref{app:implicit differentiation}). This implies that increasing the base rotational angle $\theta_1$ reduces the optimal distance $x^*$, amplifying the long-term decay effect and causing tokens to focus more on nearby tokens. In contrast, increasing the number of attention layers $t$ increases $x^*$ and hence deeper models become more biased toward initial tokens.


\section{Experiments}\label{sec:exp}
In this section, we validate our theoretical findings via carefully designed numerical experiments. To ensure a controlled setup that enables precise manipulation of positional biases in the data, we adopt the synthetic data-generating process and simplified self-attention network framework proposed in~\citet{Reddy2023TheMB}. This setup allows us to systematically isolate and examine the effects of different components on the emergence of position bias.
\paragraph{Task structure} Following~\citet{Reddy2023TheMB}, we adopt the following information retrieval task: The model is trained to predict the label $y_{\text{query}}$
of a target $x_{\text{query}}$ using the cross-entropy loss, given an alternating sequence of $n$ items and $n$ labels:  \( x_1, y_1, \dots, x_n, y_n, x_{\text{query}} \). The sequence is embedded in $d$ dimensions.  Each \( x_i \) is sampled from a Gaussian mixture model with $K$ classes, and $y_i$ is the corresponding class label assigned prior to training from the total $L$ labels ($L\leq K$). The burstiness $B$ is the number of occurrences of \( x_i \) from a particular class in an input sequence.  Importantly, at least one item in the context belongs to the same class as the query. To control position bias in the training data,  $x_\text{query}$ can either be explicitly assigned to the class of a specific $x_i$, introducing position-dependent bias in the data, or randomly assigned to the class of any $x_i$, simulating a scenario without position bias in the data. 
\paragraph{Tracking position bias} To quantify position bias, we evaluate model performance using sequences containing novel classes not seen during training. Specifically, by generating new class centers for the Gaussian mixture and randomly assigning one of the $L$ existing labels to these novel classes, we ensure that the model relies on contextual information rather than memorized class features. Crucially, we can systematically vary the position of the correct answer within test sequences to measure retrieval accuracy changes, thereby isolating and quantifying position-dependent biases in the model's behavior.
\paragraph{Network architecture}
The input sequences are passed through an attention-only network followed by a classifier. Each
attention layer has one attention head. The classifier is then a three-layer MLP with ReLU activations and a softmax layer which predicts the probabilities of the $L$
labels.

Following~\citet{Reddy2023TheMB}, we set $n=8$ and $d=64$. Additional experimental details are provided in~\cref{app:exps}. Despite our use of a simplified experimental setup, we observe the emergence of key phenomena documented in real-world LLMs, such as the ``lost-in-the-middle" phenomenon (\cref{exp:causal}) and the formation of attention sinks (\cref{app:attn_sinks}). This convergence between our controlled environment and real-world observations validates our choice of abstraction, suggesting that we have preserved the essential mechanisms driving position bias while enabling systematic investigation.




\subsection{The Effects of Depth and Relative PEs}\label{sec:depth}


To investigate the position bias arising solely from the architectural design of the attention mechanism, we use training sequences without positional bias, where the position of $x_i$ sharing the same class as $x_{\text{query}}$ is uniformly random in $\{1, 2, \ldots, n\}$. To evaluate the position bias in the trained model, we construct test sequences of the form $[\bm{a}, b]$. Here, the bolded term $\bm{a}$ explicitly marks the correct position, ensuring $y_a$ matches $y_{\text{query}}$, while position $b$ serves as a baseline. In these sequences, $x_a$ and $x_b$ are identical vectors, allowing us to control for the influence of semantic information on the model's retrieval accuracy. We then measure the retrieval accuracy gap between pairs of sequences where the content at positions $a$ and $b$ is identical, but the correct position varies. This gap, defined as $[\bm{a},b]-[\bm{b},a]$, quantifies the model’s positional preference independent of semantic information. To perform this evaluation, we construct three pairs of test sets, each containing $10,000$ sequences: [\textbf{first}, middle] vs. [\textbf{middle}, first], [\textbf{first}, last] vs. [\textbf{last}, first], and [\textbf{middle}, last] vs. [\textbf{last}, middle]. Here ``first'' (position~1), ``middle'' (position~\(n/2\)), and ``last'' (position~\(n\))  
denote fixed positions within a sequence. 

 \begin{figure*}[t]
    \centering
    \includegraphics[width=\linewidth]{figs/depth.pdf}\caption{Position bias arising solely from the architectural design of the attention mechanism, with \textbf{no positional bias in the training data}. $a$ vs.\ $b$ denotes the gap for the case $[\bm{a},b]-[\bm{b},a]$, where bar magnitude indicates gap size, positive indicates bias toward earlier position, and negative indicates bias toward later position. Deeper attention amplifies the bias toward earlier tokens, regardless of the PE used. Furthermore, decay mask introduce stronger distance-based decay effects that increase focus on recent tokens than RoPE. }
    \label{fig:depth-PE}
\end{figure*}

 \begin{figure*}[t]
    \centering
    \includegraphics[width=\linewidth]{figs/how_causal_learn.pdf}\caption{Position bias when \textbf{trained on data biased toward the first and last positions}. Compared with no mask, a causal mask without PE indeed introduces positional
dependencies. However, pure causal mask captures positional bias only at the first position but not at the last, whereas both sin PE and RoPE successfully capture biases at both ends. Moreover, the performance under PEs also displays a ``lost-in-the-middle" pattern, which is absent under other types of positional bias in the training data (see~\cref{app:additional_positional_bias} for more details). }
    \label{fig:how_causal_learn}
\end{figure*}

\Cref{fig:depth-PE} shows the average results over five runs, where $a$ vs.\ $b$ denotes the gap $[\bm{a},b]-[\bm{b},a]$.  The magnitude of each bar represents the size of the performance gap, and the sign of each bar reflects the direction of the bias: a positive sign indicates a bias toward earlier positions, while a negative sign indicates a bias toward later positions. We highlight several key observations. First, increasing model depth consistently amplifies the bias toward earlier parts of the sequence, regardless of the PE used. Also note that the performance gap between the middle and last positions is notably smaller than that involving the first position. This aligns with our theory, which suggests that as the model focuses more on the initial part of the sequence, information near the sequence's end becomes less distinguishable, consistent with the patterns observed in~\citet{Barbero2024TransformersNG}. Furthermore, both the decay mask and RoPE introduce distance-based decay effects that reduce the bias toward the beginning induced by the causal mask and increase the focus on recent tokens. However, the decay effect induced by the decay mask is substantially more pronounced than that by RoPE, as predicted by our theory.  
\subsection{Can Causal Mask Induce Usable Positional Information?}\label{exp:causal}
Next, we empirically examine how the causal mask leverages positional information. \citet{Kazemnejad2023TheIO} hypothesized that without PE (No PE), the causal mask can implicitly simulate absolute or relative PE through specific weight matrix configurations. To test this hypothesis, we train models on sequences with positional bias at either the beginning or the end. Specifically, in the training data, $x_\text{query}$ is assigned to the class of $x_1$ or $x_n$ with equal probability. We then evaluate two types of attention masks: no mask ($\cG$ is complete) and causal, and three types of PEs: No PE~\cite{Kazemnejad2023TheIO}, absolute sinusoidal PE (sin PE)~\cite{Vaswani2017AttentionIA}, and relative PE using RoPE. For evaluation, we construct six types of test sets as described in~\cref{sec:depth}, each with $10,000$ sequences. 

\Cref{fig:how_causal_learn} shows the average results using a $2$-layer network over five runs. Notably, in the left subplot, the causal mask without PE demonstrates a clear position bias toward the first position compared to the no mask without PE. This indicates that the causal mask indeed introduces a notion of position. However, when strong positional biases are present in the training data, both sin PE and RoPE allow the model to effectively capture these biases at both ends, regardless of the mask used. In contrast, a causal mask without PE only enables the model to learn a position bias at the beginning of the sequence. If the hypothesis by~\citet{Kazemnejad2023TheIO} were correct, that the causal mask uses positional information by simulating PEs, then the model should be able to capture positional bias at any location. This discrepancy suggests that the causal mask does not inherently implement PE but instead introduces a bias toward earlier positions via iterative attention, capturing positional bias only when it aligns with this predisposition.

\paragraph{The role of data in creating positional bias}
It is worth noting that in~\Cref{fig:how_causal_learn}, we observe the ``lost-in-the-middle" phenomenon~\cite{liu2024lost}, where information retrieval accuracy follows a U-shape relative to the position of the answer, with performance at the beginning of the sequence slightly better than at the end. More experimental results under different types of positional bias in the training data can be found in~\Cref{app:additional_positional_bias}. Notably, this phenomenon does not occur when the training data lacks positional bias (\Cref{fig:masks}) or contains other types of positional bias considered (\Cref{app:additional_positional_bias}).  This suggests that specific types of positional bias in the training data also play a role in how the model learns to prioritize positions within a sequence.

\section{Conclusion}
In this paper, we study position bias in transformers through a probabilistic and graph-theoretic lens, developing a theoretical framework that quantifies how positional information influences context construction across multi-layer attention. Our analysis reveals two key findings about position bias in transformers: the causal mask's inherent bias toward earlier tokens, as deeper layers increasingly attend to these positions through iterative attention, and the interplay between causal masking and relative positional encodings, which results in a nuanced, non-monotonic balance between distance-based decay effects and the cumulative influence of earlier positions. These findings open several promising directions for future work. One potential direction is leveraging these insights to design bias-free transformers, mitigating positional biases to improve model robustness and generalization capabilities. Alternatively, our framework can also inform the strategic exploitation of positional bias in specific applications, such as emphasizing early positions for text summarization or prioritizing recent interactions in recommendation systems. By deepening our understanding of how architectural choices in transformers shape positional dependencies, our work provides a foundation for designing attention mechanisms with predictable and task-aligned positional properties.


\bibliography{references}
\bibliographystyle{plainnat}


\appendix

\section{Proof of~\Cref{thm: causal_mask}}
\subsection{Auxiliary results}

\begin{lemma}
    Under~\textup{\textbf{A1}}-\textup{\textbf{A2}}, there exists $\epsilon > 0$ such that $A^{(t)}_{ij} \geq \epsilon$ for all $t\geq 0$, $(j,i) \in E$.
    \label{lem: matrix_A}
\end{lemma}

\begin{proof}
    Writing~\eqref{eq: update_no_LN} recursively, we get that the token trajectories 
    \begin{equation}
        X^{(t+1)} = A^{(t)}...A^{(0)}X^{(0)}W_V^{(0)}...W_V^{(t)}\,,
        \label{eq: token_traj_no_LN}
    \end{equation}
    stay uniformly bounded for all $t\geq 0$ by~\textup{\textbf{A2}}.  Then it follows from~\textup{\textbf{A1}} that there exists $C\in \bR$ such that for all $t\geq 0$,
\begin{equation}
\begin{aligned}\norm{\left(X^{(t)}W_Q^{(t)}\right)_{i,:}}_2 = \norm{X^{(t)}_{i,:}W_Q^{(t)}}_2 \leq C\,,\\
    \norm{\left(X^{(t)}W_K^{(t)}\right)_{i,:}}_2 = \norm{X^{(t)}_{i,:}W_K^{(t)}}_2 \leq C\,. 
    \end{aligned}
     \label{eq: qk_norm_bound}
    \end{equation}
   
    Hence for all $i,j\in[N]$,
    \[ - C^2 \leq (X^{(t)}W^{(t)}_Q (X^{(t)}W^{(t)}_K)^\top)_{ij} \leq C^2\,.\]
    This implies that there exists $\epsilon > 0$ such that 
    $A^{(t)}_{ij} \geq \epsilon$ for all $(j,i) \in E$.
\end{proof}

\subsection{Proof of~\cref{thm: causal_mask}}

We denote 
$P^{(t)} \vcentcolon = A^{(t)}\cdots A^{(0)}$. It suffices to show that there exists $ 0 < C < 1$ and $0 < \epsilon <1$ such that 
\begin{equation}
    P^{(t)}_{ij} \leq C(1-(j-1)\epsilon)^t
    \label{eq: causal_exponetial_decay}
\end{equation}
for all $1<j\leq i$ and $t\geq 0$. 

The proof will go by induction:
\paragraph{Base case}
By~\cref{lem: matrix_A}, it follows that 
\begin{equation*}
    P^{(0)}_{ij} \leq (1-\epsilon)
\end{equation*}
for all $1<j\leq i$. Then let $C\vcentcolon=1-\epsilon$.
\paragraph{Induction step} Assume that~\eqref{eq: causal_exponetial_decay} holds, it follows that for all $1<j\leq i$. 
\begin{equation*}
    P_{ij}^{(t+1)} = \sum_{k=j}^i A^{(t)}_{ik}P^{(t)}_{kj} \leq (1-(j-1)\epsilon)C(1-(j-1)\epsilon)^{t} = C(1-(j-1)\epsilon)^{t+1}\,.
\end{equation*}
From above, we conclude the theorem.


\section{Proof of~\cref{thm:sliding-window}}
For $t_0 \leq t_1$, we denote 
$$A^{(t_1:t_0)} = A^{(t_1)}\ldots A^{(t_0)}\,.$$
Without loss of generality, we assume in the following proof that $N-1$ can be divided by $w-1$.
\subsection{Auxiliary results}

\begin{lemma}
    Let $\cG$ be the sliding-window mask with the window size $w \geq 2$. Then there exists $c > 0$ such that for all $t_0\geq 0$, 
    \[c \leq A^{\left(t_0+\frac{N-1}{w-1}-1:t_0\right)}_{ij} \leq 1 \,,\quad\forall j\leq i\in[N]\,.\]
    \label{lem:batch_P_positive}
\end{lemma}

\begin{proof}
    Given the connectivity pattern of the sliding-window mask $\cG$ and~\cref{lem: matrix_A}, it follows that for all $t_0\geq 0$, $A^{\left(t_0+\frac{N}{w}-1:t_0\right)}$ is a lower triangular matrix. Moreover, since $A_{ij}^{\left(t_0+\frac{N}{w}-1:t_0\right)}$ counts the aggregate probability of the walks of length $\frac{N-1}{w-1}$
 between token $i$ and token $j$ where by~\cref{lem: matrix_A}, each walk has probability at least $\epsilon^{\frac{N-1}{w-1}}$.
 
 Thus we conclude that there exists $c>0$ such that for all $t_0\geq 0$,
    \begin{equation*}A_{ij}^{\left(t_0+\frac{N}{w}-1:t_0\right)} \geq c, \quad \forall j\leq i\in[ N]\,. 
    \end{equation*}
\end{proof}

\subsection{Proof of~\cref{thm:sliding-window}}

For $k\geq 0$, denote 
\[\tilde{A}^{(k)} = A^{\left((k+1)\left(\frac{N-1}{w-1}\right)-1:k\left(\frac{N-1}{w-1}\right)\right)}\,\]
and 
\[\tilde{P}^{(k)} = \tilde{A}^{(k)}\cdots \tilde{A}^{(0)}\,.\]

Then by~\cref{lem:batch_P_positive} and~\cref{thm: causal_mask}, we get that there exists $0<C<1$ and $0<c<1$ such that for all $k\geq 0$

\begin{equation*}
\tilde{P}_{ij}^{(k)} \leq C(1-(j-1)c)^k\,, \quad \forall j\leq i\in[N]\,.
\end{equation*}

Denote $Q_j^{(t)} = \underset{1\leq i\leq N}{\max} P_{ij}^{(t)}$ and $\tilde{Q}_j^{(k)} = \underset{1\leq i\leq N}{\max} \tilde{P}_{ij}^{(k)}$. 
Then it follows that for all $k\geq 0$,
\begin{equation*}
    \tilde{Q}_j^{(k)} \leq C(1-(j-1)c)^k.
\end{equation*}

Observe that 

\begin{equation}
    \forall i\in [N], P^{(t)}_{ij} \leq Q^{(t)}_j\,,
    \label{eq: obs_1}
\end{equation}
and
\begin{equation}
    \forall j\in [N], Q_j^{(t+1)}\leq Q_j^{(t)}\,.
    \label{eq: obs_2}
\end{equation}


Let $q_j = (1-(j-1)c)^{\frac{1}{2\frac{N-1}{w-1}}}$.
Then for all $k\geq 1$ and $0\leq r < \frac{N-1}{w-1}$,
\[q_j^{k\left(\frac{N-1}{w-1}\right) +r } \geq (1-(j-1)c)^k.\]
This implies that for all $t\geq \frac{N-1}{w-1}$,
\[P_{ij}^{(t)} \leq q_j^{t} = C(1-(j-1)c)^{t/\left(2\frac{N-1}{w-1} \right)}\,.\]

As for $t<\frac{N-1}{w-1}$, notice that $P_{ij}^{(0)}\leq 1-\epsilon$ for all $j\leq i\in[N]$ by~\cref{lem: matrix_A}. Then by~\eqref{eq: obs_1} and \eqref{eq: obs_2}, we deduce that 
\[P_{ij}^{(t)}\leq (1-\epsilon)^\frac{t+1}{\frac{N-1}{w-1}}, \quad \forall j\leq i \in [N]. \]

We thus conclude the statement.


\section{Proof of~\cref{thm:prefix_mask}}

First note that the first and third statements:
\begin{equation}\lim_{t\to\infty} \bP^{(t)}(z_i \in [K] |X^{(0)}) = 1\,,
        \label{eq: first_K_tokens_total}
    \end{equation}

for all $\in [N]$ and 

\begin{equation*}
        \bP^{(t)}(z_i = j|X^{(0)}) \leq  C(1-(j-K)\epsilon)^t\,.
         \label{eq: prefix_exp}
    \end{equation*}
for all $K< j \leq i$ and $t\geq 0$, follow immediately from~\Cref{thm: causal_mask} by regarding the first K tokens as a super node in the 
 causal graph $\cG$ and aggregate the edges accordingly. Thus it suffices to show that there exists $\kappa > 0$ such that
 \begin{equation}
        \liminf_{t\to\infty} \bP^{(t)}(z_i = k|X^{(0)}) \geq  \kappa. \qquad\forall k\in [K]\,.
        \label{eq: first_k_distribution}
    \end{equation}
 For $t > 0$, consider $$P^{(t)}_{ik} = \sum_{l=1}^N P_{il}^{(t:1)}A_{lk}^{(0)}\,.$$

 Then for $k_1, k_2 \in [K]$, 
$$\frac{P^{(t)}_{ik_1}}{P^{(t)}_{ik_2}} = \frac{\sum_{l=1}^{\max\{i,K\}} P_{il}^{(t:1)}A_{lk_1}^{(0)}}{\sum_{l=1}^{\max\{i,K\}} P_{il}^{(t:1)}A_{lk_2}^{(0)}}\,,$$

  which then follows 
$$\frac{P^{(t)}_{ik_1}}{P^{(t)}_{ik_2}} \geq \underset{1\leq l \leq \min\{i,K\}}{\min} \frac{A^{(0)}_{lk_1}}{A^{(0)}_{lk_2}}\,.$$


 Then by~\Cref{lem: matrix_A}, there exists $C > 0$ such that for all $k_1, k_2 \in [K]$, 
$$C \leq \liminf_{t\to \infty} \frac{P^{(t)}_{ik_1}}{P^{(t)}_{ik_2}}\,.$$
Since $\lim_{t\to\infty}\sum_{k=1}^K P_{ik}^{(t)} = 1$ by~\eqref{eq: first_K_tokens_total}, we deduce~\eqref{eq: first_k_distribution} as desired.

\section{Proof of~\cref{lem:decay_mask_attn}}
Fix $t\geq 0$. Let \[Z^{(t)}_{ij} = (X^{(t)}W^{(t)}_Q)_{i,:} (X^{(t)}W^{(t)}_K)_{:,j}.\]
Following from~\cref{lem: matrix_A}, there exists $I_{\min}, I_{\max} \in \bR$ such that for all $j\leq i\in [N]$, \[Z^{(t)}_{ij}\in[I_{\min}, I_{\max} ].\]
Consider the denominator in the $\softmax(\cdot)$ operation in the calculation of $(A^{(t)}_{\decay})_{ij}$:
\begin{align*}
\sum_{k=1}^i e^{Z^{(t)}_{ik}-(i-k)m} & \geq e^{I_{\min}}  \sum_{k=0}^i e^{-(i-k)m}\\
& = e^{I_{\min}}\frac{1-e^{-(i+1)m}}{1-e^{-m}} \\
& \geq e^{I_{\min}} \frac{1-e^{-2m}}{1-e^{-m}} \\
& = e^{I_{\min}} (1+e^{-m})
\end{align*}
and 
\begin{align*}
\sum_{k=1}^i e^{Z^{(t)}_{ik}-(i-k)m} & \leq e^{I_{\max}}  \sum_{k=0}^\infty e^{-km}\\
& = \frac{e^{I_{\max}}}{1-e^{-m}}
\end{align*}
It follows that 
\begin{align*}
    (A^{(t)}_{\decay})_{ij} \leq 
\frac{e^{I_{\max}-(i-j)m}}{e^{I_{\min}}(1+e^{-m})} = C_{\max} e^{-(i-j)m}
\end{align*}
and 
\begin{align*}
    (A^{(t)}_{\decay})_{ij} \geq 
\frac{e^{I_{\min}-(i-j)m}}{e^{I_{\max}}/(1-e^{-m})} = C_{\min} e^{-(i-j)m}
\end{align*}
where $C_{\max} \vcentcolon=e^{(I_{\max}-I_{\min})}/({1+e^{-m}})$ and $C_{\min} \vcentcolon=(1-e^{-m})e^{(I_{\min}-I_{\max})}$.

\section{Proof of~\cref{thm:aggregate_decay_effect}}
Note that in the causal graph $\cG$, there are $t+i-j \choose i-j$ paths of length $t+1$ from token $j$ to token $i$.

Since going from token $j$ to token $i$ in the causal graph, the connectivity patterns ensure that the token indices along the path are non-decreasing, i.e. if we denote the directed path as $(j,l_1), (l_1,l_2),...,(l_t,i)$, it holds that $j\leq l_1\leq l_2 \leq ...\leq l_{t}\leq i$. Together with~\cref{lem:decay_mask_attn}, we conclude the theorem statement.

\section{Proof of~\cref{lem: attn_rope}}
    Fix $t\ge0$. Denote the angle after rotation to be $\psi^{(t)}_{i,j}$. Then it follows from the definition of RoPE that 
    \begin{equation*}
        \psi^{(t)}_{i,j} = \phi^{(t)}_{i,j} - (i-j)\theta_1\,.
    \end{equation*}
    Thus 
    \begin{equation*}
        |\psi^{(t)}_{i,j}| = |\phi^{(t)}_{i,j} - (i-j)\theta_1| \geq| |(i-j)\theta_1| - |\phi^{(t)}_{i,j}|| \geq |(i-j)-\delta|\theta_1\,.
    \end{equation*}
 
    
    \begin{equation*}
        |\psi^{(t)}_{i,j}| = |\phi^{(t)}_{i,j} - (i-j)\theta_1| \leq |(i-j)\theta_1| +  |\phi^{(t)}_{i,j}| \leq (i-j+\delta)\theta_1\,.
    \end{equation*}

Let the original query $i$ and key $j$ embeddings be $q^{(t)}_i \vcentcolon= X^{(t)}_{i,:}W^{(t)}_Q$ and $k^{(t)}_j\vcentcolon=  X^{(t)}_{j,:}W^{(t)}_K$, respectively, and the corresponding query $i$ and key $j$ embeddings after rotation be $q_i'^{(t)}$ and $k_j'^{(t)}$, respectively.

Since $\langle q'^{(t)}_i, k'^{(t)}_j\rangle = \|q^{(t)}_i\|_2 \|k^{(t)}_j\|_2 \cos \psi^{(t)}_{i,j}$, it follows from  that there exists $C_{\min}, C_{\max} \geq 0$ such that for all $i,j\in[N]$,
\begin{equation*}
    C_{\min} \cos ((i-j + \delta)\theta_1) \leq \langle q'^{(t)}_i, k'^{(t)}_j\rangle \leq C_{\max} \cos (|(i-j)-\delta|\theta_1)\,.
\end{equation*}
Since for $|x| \leq \pi$ there exists $c>0$ such that $1-x^2/2 \leq \cos x  \leq 1-x^2/c$, we get that
\begin{equation*}
    C_{\min} \left(1-\frac{((i-j) + \delta)^2\theta_1^2}{2}\right) \leq \langle q'^{(t)}_i, k'^{(t)}_j\rangle \leq C_{\max} \left(1-\frac{((i-j)-\delta)^2 \theta_1^2}{c}\right) \,.
\end{equation*}
and hence 
\begin{equation*}
    C_{\min} \left(1 - \delta^2\theta_1^2-(i-j)^2 \theta_1^2\right) \leq \langle q'^{(t)}_i, k'^{(t)}_j\rangle \leq C_{\max} \left(1-\frac{((i-j)^2/2-\delta^2) \theta_1^2}{c}\right) \,.
\end{equation*}

Consider $Y^{(t)}_i = \sum_{k=1}^{i}e^{\langle q'^{(t)}_i, k'^{(t)}_j \rangle}$. Then by~\eqref{eq: qk_norm_bound}, we get that there exists $Y_{\max}, Y_{\min} > 0$ such that 
\begin{equation*}
    Y_{\max} \leq Y^{(t)}_i \leq Y_{\min}\,.
\end{equation*}
We thus conclude the statement.


\section{Proof of~\cref{thm: rope}}

    Notice that
    \begin{equation}
        (P^{(t)}_{\RoPE})_{ij} = \sum_{l_1\leq\cdots\leq l_{t-1}\in[N]^{t-1}} A^{(t-1)}_{il_{t-1}}A^{(t-2)}_{l_{t-1}l_{t-2}}\cdots A^{(0)}_{l_1j}
    \end{equation}
    Given that when $\cG$ is the causal graph, due to the connectivity the directed path of length $t$ from token $j$ to token $i$ must be non-decreasing, i.e. $j\leq l_1 \leq l_2 \leq \cdots\leq l_{t-1} \leq i$, and there would be in total ${t+i-j  \choose i-j}$ such paths. For each such path $j\leq l_1 \leq l_2 \leq \cdots\leq l_{t-1} \leq i$, notice that by~\Cref{lem: attn_rope}, we get that fix $T\geq0$, there exists $C_{\min}$, $C_{\max} > 0$ such that 
    \begin{equation}
        A^{(t-1)}_{i,l_{t-1}}A^{(t-2)}_{l_{t-1},l_{t-2}}\cdots A^{(0)}_{l_1,j} \geq C_{\min} e^{-c((i-l_{t-1})^2 + (l_{t-1}-l_{t-2})^2 + \cdots + 
        (l_1-j)^2)\theta_1^2}
        \label{eq: rope_P_lower_raw}
    \end{equation}
    and 
    \begin{equation}
        A^{(t-1)}_{i,l_{t-1}}A^{(t-2)}_{l_{t-1},l_{t-2}}\cdots A^{(0)}_{l_1,j} \leq C_{\max} e^{-c'((i-l_{t-1})^2 + (l_{t-1}-l_{t-2})^2 + \cdots + 
        (l_1-j)^2)\theta_1^2}
        \label{eq: rope_P_upper_raw}
    \end{equation}

From~\eqref{eq: rope_P_lower_raw}, since $j\leq l_1 \leq l_2 \leq \cdots\leq l_{t-1} \leq i$, we further get that
  \begin{equation}
        A^{(t-1)}_{i,l_{t-1}}A^{(t-2)}_{l_{t-1},l_{t-2}}\cdots A^{(0)}_{l_1,j} \geq C_{\min} e^{-c(i-j)^2\theta_1^2}\,,
        \label{eq: rope_P_lower_final}
    \end{equation}
and similarly
 \begin{equation}
        A^{(t-1)}_{i,l_{t-1}}A^{(t-2)}_{l_{t-1},l_{t-2}}\cdots A^{(0)}_{l_1,j} \leq C_{\max} e^{-\frac{c'}{2}(i-j)^2\theta_1^2}\,.
        \label{eq: rope_P_lower_final}
    \end{equation}

\section{Implicit differentiation of $x$ with respect to $\theta_1$ and $t$}\label{app:implicit differentiation}

Recall that under RoPE, $$L(x) = \log \left({t+i-j  \choose i-j} e^{-(i-j)^2\theta_1^2} \right)\,.$$
Then by Stirling's approximation,
$$L(x) \approx \left((t+x)\log(t+x) - (t+x) \right) - (x\log x -x) - \theta_1^2x^2\,,$$
and thus 
$$L'(x) = \log\left(\frac{t+x}{x}\right) -2\theta_1 x\,.$$

Taking implicit differentiation of $t$:

\[\frac{\partial}{\partial t}\log \left(\frac{t+x}{x}\right)= \frac{-t}{x(x+t)}\frac{\partial x}{\partial t} + \frac{1}{x+t}\]

and 
\[\frac{\partial}{\partial t} 2\theta_1 x = 2\theta_1 \frac{\partial x}{\partial t}.\]

So let
\[\frac{1}{x+t} = (2\theta_1 +\frac{t}{(x+t)x})\frac{\partial x}{\partial t}\]
and thus
\[\frac{\partial x}{\partial t} = \frac{1}{2\theta_1(x+t)+\frac{t}{x}} >0\,.\]
Taking implicit differentiation of $\theta_1$:

\[\frac{\partial}{\partial \theta_1}\log \left(\frac{t+x}{x}\right)= \frac{-t}{x(x+t)}\frac{\partial x}{\partial \theta_1}\]
and 
\[\frac{\partial}{\partial \theta_1} 2\theta_1 x = 2x +2\frac{\partial x}{\partial \theta_1} \theta_1.\]
So let
\[\left(2\theta_1 +\frac{t}{x(x+t)}\right)\frac{\partial x}{\partial \theta_1} = -2x\,,\]
and thus 
\[\frac{\partial x}{\partial \theta_1} = \frac{-2x}{2\theta_1+\frac{t}{x(x+t)}} < 0\,.\]
Hence we observe that $x^*$ is an increasing function of $t$ and a decreasing function of $\theta_1$. This implies that increasing the base rotational angle $\theta_1$ reduces the optimal distance $x^*$, amplifying the long-term decay effect and causing tokens to focus more on nearby tokens. In contrast, increasing the number of attention layers $t$ increases $x^*$ and hence deeper models become more biased toward initial tokens.


\section{The effect of RoPE: case for $d\geq 2$}\label{app:rope_d_geq_2}
In this section, we present a generalized version of~\cref{thm: rope} for the case $d\geq2$. 

Let the query $q$ and key $k$ be vectors in $\bR^d$, where $d$ is even, and let $\phi$ be the angle between $q$ and $k$, which we assume to be well-defined, with:
\[\cos\phi = \frac{\langle q, k\rangle}{\|q\|_2\|k\|_2}\,.\] 

Define the length-2 segments of query $q$ and 
and key $k$ as 
\[q_l = (q_{2l-1}, q_{2l}),  \quad k_l = (q_{2l-1}, q_{2l}),\]
for $l\in [d/2]$, and let $\phi_l$ be the angle between $q_l$ and $k_l$, with:
\[\cos\phi_l = \frac{\langle q_l, k_l\rangle}{\|q_l\|_2\|k_l\|_2}\,.\]


Without loss of generality, we make the following assumption:

\begin{enumerate}[leftmargin=*, labelindent=2em]
   \item [\textbf{A3}] There exists $\beta_q, \beta_k>0$ such that $\|q^{(t)}_{l}\|_2 \geq  \beta_q \|q^{(t)}\|_2$ and $\|k^{(t)}_{l}\|_2 \geq  \beta_k \|k^{(t)}\|_2$ for all $l\in[d/2]$ for all $t\geq 0$.
\end{enumerate}

The condition means that all segments makes a nontrivial contribution to the norm. In practice, since LLMs tend to tend to predominantly utilize feature dimensions that rotate slowly~\cite{Barbero2024RoundAR}, the effective $d/2$ tends to be a small number.


Given the pre-defined set of base rotational angles $\Theta = \{0\leq\theta_1\leq \cdots \leq\theta_{d/2}\}$, we reparametrize as $\theta_i = \alpha_i\theta_1$. 
\subsection{Results}
We present the general version of~\cref{lem: attn_rope} and~\cref{thm: rope} as follows:
\begin{lemma}
    Let $\cG$ be the causal mask and $\textup{\textbf{A1}}$-$\textup{\textbf{A3}}$ hold. Suppose for $t\geq 0$, $\|q_i\|_2, \|k_j\|_2 > 0$, and $|\phi^{(t)}_{i,j}| \leq \delta\theta_1$, where $\delta > 0$ and $$\left(\sqrt{\frac{1}{\beta_q\beta_k}}\delta\pi+2(N-1)\alpha_{d/2}\right)\theta \leq 2\pi\,.$$ Then there exists $C_{\max}, C_{\min}, c,c' > 0$ such that 
    $$C_{\min} e^{-c\sum_{l=1}^{d/2}(i-j)^2\alpha_l^2\theta_1^2} \leq (A^{(t)}_{\RoPE})_{i,j} \leq C_{\max} e^{-c'\sum_{l=1}^{d/2}(i-j)^2\alpha_l^2\theta_1^2}\,.$$
    \label{lem: attn_rope_general}
\end{lemma}

\begin{theorem}
   Fix $T> 0$. Under the same conditions as in~\cref{lem: attn_rope_general} for $t\leq T$, there exists $c>0$ such that for all $t\leq T$, 
   $$(P^{(t)}_{\RoPE})_{i,j} = \Theta\left({t+i-j  \choose i-j} e^{-c\sum_{l=1}^{d/2}(i-j)^2\alpha_l^2\theta_1^2} \right)\,.$$
   \label{thm: rope_general}
   \vspace{-3ex}
\end{theorem}




\subsection{Proofs of~\cref{lem: attn_rope_general}}
We first show the following auxiliary result:
\begin{lemma}
    Under \textup{\textbf{A3}},  it holds that
    \[|\phi_l| \leq \frac{\pi}{2}\sqrt{\frac{1}{\beta_q\beta_k}} |\phi|\,,\]
    for all $l\in[d/2]$.
\end{lemma}
\begin{proof}
    By definition, since
    \[\sum_{l=1}^{d/2}\|q_l\|_2\|k_l\|_2 \cos \phi_l = \|q\|_2\|k\|_2\cos\phi\,,\]
    then the Cauchy–Schwarz inequality implies that 
    \begin{equation}
        \sum_{l=1}^{d/2}\|q_l\|_2\|k_l\|_2 (1-\cos \phi_l) \leq \|q\|_2\|k\|_2(1-\cos\phi)\,.
        \label{eq:inner_product_identity_cs}
    \end{equation}
    By~\textbf{A3}, \eqref{eq:inner_product_identity_cs} becomes 
     \begin{equation*}
        \sum_{l=1}^{d/2} (1-\cos \phi_l) \leq \frac{1}{\beta_q\beta_k}(1-\cos\phi)\,.
    \label{eq:inner_product_identity_cs_2}
    \end{equation*}
    Given the trigonometric identity $1-\cos 2x = 2\sin^2 x$, we get that 
     \begin{equation}
        \sum_{l=1}^{d/2} \sin^2\left(\frac{\phi_l}{2}\right) \leq \frac{1}{\beta_q\beta_k}\sin^2\left(\frac{\phi}{2}\right) \,.
    \label{eq:inner_product_identity_cs_2}
    \end{equation}

    Notice for all $x\in\bR$,
    \begin{equation}
        \sin^2 x\leq x^2\,,
        \label{eq:ineq_1}
    \end{equation}
    and  for all $|x|\leq \pi/2$, 
     \begin{equation}
        \frac{4}{\pi^2}x^2 \leq \sin^2x\,.
         \label{eq:ineq_2}
    \end{equation}
    Apply~\eqref{eq:ineq_1} and~\eqref{eq:ineq_2} to~\eqref{eq:inner_product_identity_cs_2}, we get that 
    \begin{equation*}
        \sum_{i=1}^{d/2} \phi_l^2 \leq \frac{\pi^2}{4\beta_q\beta_k} \phi^2\,.
    \end{equation*}
   Hence for all $l\in[d/2]$, it follows that
   \begin{equation*}
       |\phi_l| \leq \frac{\pi}{2}\sqrt{\frac{1}{\beta_q\beta_k}} |\phi|\,.
   \end{equation*}
\end{proof}


 Denote the angle after rotation to be $\psi_{i,j,l}$. Then it follows that 
    \begin{equation*}
        \psi_{i,j,l} = \phi_{i,j,l} - (i-j)\alpha_l\theta_1\,.
    \end{equation*}
It follows similarly as in the proof of~\cref{lem: attn_rope} that 
\begin{equation*}
    C_{\min} \left(1 - \delta'^2\theta^2-(i-j)^2 \alpha_l^2 \theta_1^2\right) \leq \langle (q'_i)_l, (k'_j)_l\rangle \leq C_{\max} \left(1-\frac{((i-j)^2\alpha_l^2/2-\delta'^2) \theta_1^2}{c}\right) \,,
\end{equation*}
where $\delta' = \frac{\pi}{2}\sqrt{\frac{1}{\beta_q\beta_k}}  \delta  $, for all $l\in[d/2]$.

Since \[\langle q_i',k_j'\rangle = \sum_{l=1}^{d/2} \langle (q'_i)_l, (k'_j)_l \rangle,\] we conclude the statement.

\subsection{Proof of~\cref{thm: rope_general}}
The result is a direct corollary of~\cref{lem: attn_rope} and~\cref{thm: rope}.


\section{Experiments}\label{app:exps}

Here we provide more details on the numerical experiments presented in~\cref{sec:exp}. All models were
implemented with PyTorch~\cite{Paszke2019PyTorchAI}.

\paragraph{Parameterizing the data distribution} As defined in~\cref{sec:exp}, the input data distribution is modulated by tuning various parameters. In addition to the parameters described in the main text, for the Gaussian mixture with $K$ classes, 
each class $k$ is defined by a $d$-dimensional
vector $\mu_k$ whose components are sampled $i.i.d.$ from a normal distribution with mean zero and variance $1/d$. Then the value of $x_i$ is given by $\frac{\mu_k+\gamma\eta}{\sqrt{1+\gamma^2}}$, where $\eta$ is drawn from the same distribution as the $\mu_k$’s and $\gamma$ sets the within-class variability. Each class is assigned to one of $L$ labels ($L\leq K$). The contents of the labels are drawn prior to training from the same distribution as the $\mu_k$'s. 







In~\citet{Reddy2023TheMB}, the author found that different configurations of the data generating process give rise to different learning regimes. To enable better information retrieval ability of the model, we choose the configuration suggested by~\citet{Reddy2023TheMB} that corresponds to the difficult in-weight learning and easy in-context-learning regime to ensure the information retrieval ability of the model. Specifically, we set $\gamma=0.75$, $K=2048$, $L=32$, and $B = 4$.

\paragraph{Relative PE hyperparameters} For the decay mask, we set $m=0.8$. For RoPE, we set $\theta_i = 10000^{-2(i-1)/d}$, as in~\citet{su2023roformerenhancedtransformerrotary}.


\paragraph{Compute} We trained all of our models on a Tesla V100 GPU.

\paragraph{Training details} In all experiments, we used the AdamW optimizer~\cite{Loshchilov2017DecoupledWD} with a learning rate of $10^{-3}$, a weight decay of $10^{-6}$, a batch size of $128$, and trained for $100,000$ iterations.

\section{Additional Experimental Results}\label{app:additional_exps}
\subsection{The role of training data on positional bias}\label{app:additional_positional_bias}


In this section, we present additional experimental results building on the experiment described in~\cref{exp:causal}, but focusing on cases where positional bias in the training sequences is introduced at other positions. Specifically, we consider three types of training sequences where $x_\text{query}$ is assigned the class of 1) $x_1$ (the first position), 2) $x_{n/2}$ (the middle position), or 3) $x_n$ (the last position). The corresponding results are shown in~\Cref{fig:bias_0}, \Cref{fig:bias_4}, and \Cref{fig:bias_-1}, respectively.



Observe that, compared with no mask, the causal mask without PEs indeed introduces a sense of position across all cases. Specifically, it enables the model to learn a positional bias favoring the beginning of the sequence, as earlier tokens tend to receive more attention through the mechanism of iterative attention. In contrast, both sin PE and RoPE allow the model to effectively capture different positional biases regardless of their location in the training sequences.

\begin{figure}[h]
        \centering
        \includegraphics[width=\linewidth]{figs/bias_0.pdf}
        \caption{Position bias when trained on data biased toward the \textbf{first} position.}
        \label{fig:bias_0}
        \vspace{1cm}
        \centering
        \includegraphics[width=\linewidth]{figs/bias_4.pdf}
        \caption{Position bias when trained on data biased toward the \textbf{middle} positions.}
        \label{fig:bias_4}
        \vspace{1cm}
        \centering
        \includegraphics[width=\linewidth]{figs/bias_-1.pdf}
        \caption{Position bias when trained on data biased toward the \textbf{last} positions.}
        \label{fig:bias_-1}
    \end{figure}

Interestingly, when comparing this behavior to the case shown in~\Cref{fig:how_causal_learn}, we note that the ``lost-in-the-middle'' phenomenon only emerges when the training sequences are biased toward both the beginning and the end. This suggests that specific types of positional bias in the training data play a crucial role in shaping how the model learns to process and prioritize positions within a sequence.




As the structure of positional bias in natural language remains unclear, this observation raises the following question:
\begin{quote}
    \textit{Does positional bias in natural language sequences shape the ``lost-in-the-middle'' phenomenon in a similar way to what we observe in this simplified case?}
\end{quote}

This question connects to broader inquiries about the parallels between artificial and human attention. In neuroscience, the primacy-recency effect highlights that human attention often gravitates toward the beginning and end of sequences~\cite{Glanzer1966TwoSM,Li2024EEGDT}, a phenomenon that may have influenced the structure of human languages, where critical information is frequently placed in these positions~\cite{Halliday2004AnIT}. As demonstrated in~\cref{exp:causal}, when such patterns are present in training data, attention-based architectures seem to develop analogous biases~\cite{Hollenstein2021MultilingualLM}, aligning with natural language characteristics for improved performance. This raises deeper, perhaps philosophical questions: To what extent are these biases intrinsic to effective sequential processing? How closely should neural networks emulate human cognitive patterns? Investigating these connections can deepen our understanding of both human and artificial intelligence while guiding the design of more effective machine learning models. 


\subsection{Attention sinks}\label{app:attn_sinks}


Despite our use of a simplified experimental setup in this work, we observe the emergence of key phenomena documented in more complex settings. In addition to the ``lost-in-the-middle" phenomenon discussed in~\cref{exp:causal} and~\cref{app:additional_positional_bias}, in this section, we report the formation of attention sinks in our setting.

\begin{figure}[h]
    \centering
    \includegraphics[width=0.75\linewidth]{figs/causal_layers.pdf}
    \caption{Example of the emergence of attention sinks in our experimental setting. In particular, the sequences used for training and inference are all \textbf{free of position bias}.}
    \label{fig:attn_sinks_causal}
\end{figure}

\Cref{fig:attn_sinks_causal} shows an example of the attention maps of a two-layer self-attention networks under the causal mask without PEs, where the sequences used for training and inference are all free of position bias. We observe the similar phenomenon of attention sinks as reported in~\citet{Xiao2023EfficientSL}.

More quantitatively, following~\citet{Gu2024WhenAS}, we calculate their metric for measuring the emergence of attention sinks, over $10,000$ sequences free of position bias. 
Specifically, denote the adjacency matrix of the mask $\cG$ to be $M$. Then the metric for attention sink at token $j$ is calculated as 
\[\text{Attention Sink}_{j} = \frac{1}{T}\sum_{t=0}^{T-1} \frac{1}{\sum_{i=1}^NM_{ij}}\sum_{i=1}^{N} \1\{A^{(t)}_{ij} > \tau\}\,.\]
The threshold $\tau$ we choose is $0.2$. The results for the causal mask, the sliding-window masks (with $w=5,9,13$), and the prefix masks (with $K=2,4,6$) are shown in Figures ~\ref{fig:attnsink_causal}, \ref{fig:attnsink_window}, and \ref{fig:attnsink_prefix}, respectively.
In particular, we make the following observations: 
\begin{enumerate}
     \item Attention sinks emerge on the absolute first token under the causal mask. 
    \item Attention sinks tend to emerge on the absolute first token when the window size $w$ is larger, under the sliding-window mask.
    \item Attention sinks emerge on the $K$ prefix tokens, not just on the first token alone, under the prefix mask.
\end{enumerate}
   All of these phenomena have been observed in real-world LLMs in~\citet{Gu2024WhenAS}. This alignment between our controlled setup and real-world observations affirms the validity of our abstraction, indicating that we have captured the key mechanisms underlying position bias while facilitating a systematic analysis.

    \begin{figure}[h]
        \centering\includegraphics[width=0.34\linewidth]{figs/attnsink_causal.pdf}
        \caption{Attention sinks emerge on the first token under the causal attention mask.}
        \label{fig:attnsink_causal}
     \vspace{1cm} % Adjust vertical spacing if needed
        \centering
        \includegraphics[width=\linewidth]{figs/attnsink_window.pdf}
        \caption{Attention sinks tend to emerge on the absolute first token when the context window size $w$ is larger, under the sliding-window
mask.}
        \label{fig:attnsink_window}
    \vspace{1cm}
        \centering
        \includegraphics[width=\linewidth]{figs/attnsink_prefix.pdf}
        \caption{Attention sinks emerge on the $K$ prefix tokens, not just on the first token alone, under the prefix mask.}
        \label{fig:attnsink_prefix}
    \end{figure}









\end{document}
\bibliographystyle{abbrvnat}

\clearpage
\appendix
\subsection{Lloyd-Max Algorithm}
\label{subsec:Lloyd-Max}
For a given quantization bitwidth $B$ and an operand $\bm{X}$, the Lloyd-Max algorithm finds $2^B$ quantization levels $\{\hat{x}_i\}_{i=1}^{2^B}$ such that quantizing $\bm{X}$ by rounding each scalar in $\bm{X}$ to the nearest quantization level minimizes the quantization MSE. 

The algorithm starts with an initial guess of quantization levels and then iteratively computes quantization thresholds $\{\tau_i\}_{i=1}^{2^B-1}$ and updates quantization levels $\{\hat{x}_i\}_{i=1}^{2^B}$. Specifically, at iteration $n$, thresholds are set to the midpoints of the previous iteration's levels:
\begin{align*}
    \tau_i^{(n)}=\frac{\hat{x}_i^{(n-1)}+\hat{x}_{i+1}^{(n-1)}}2 \text{ for } i=1\ldots 2^B-1
\end{align*}
Subsequently, the quantization levels are re-computed as conditional means of the data regions defined by the new thresholds:
\begin{align*}
    \hat{x}_i^{(n)}=\mathbb{E}\left[ \bm{X} \big| \bm{X}\in [\tau_{i-1}^{(n)},\tau_i^{(n)}] \right] \text{ for } i=1\ldots 2^B
\end{align*}
where to satisfy boundary conditions we have $\tau_0=-\infty$ and $\tau_{2^B}=\infty$. The algorithm iterates the above steps until convergence.

Figure \ref{fig:lm_quant} compares the quantization levels of a $7$-bit floating point (E3M3) quantizer (left) to a $7$-bit Lloyd-Max quantizer (right) when quantizing a layer of weights from the GPT3-126M model at a per-tensor granularity. As shown, the Lloyd-Max quantizer achieves substantially lower quantization MSE. Further, Table \ref{tab:FP7_vs_LM7} shows the superior perplexity achieved by Lloyd-Max quantizers for bitwidths of $7$, $6$ and $5$. The difference between the quantizers is clear at 5 bits, where per-tensor FP quantization incurs a drastic and unacceptable increase in perplexity, while Lloyd-Max quantization incurs a much smaller increase. Nevertheless, we note that even the optimal Lloyd-Max quantizer incurs a notable ($\sim 1.5$) increase in perplexity due to the coarse granularity of quantization. 

\begin{figure}[h]
  \centering
  \includegraphics[width=0.7\linewidth]{sections/figures/LM7_FP7.pdf}
  \caption{\small Quantization levels and the corresponding quantization MSE of Floating Point (left) vs Lloyd-Max (right) Quantizers for a layer of weights in the GPT3-126M model.}
  \label{fig:lm_quant}
\end{figure}

\begin{table}[h]\scriptsize
\begin{center}
\caption{\label{tab:FP7_vs_LM7} \small Comparing perplexity (lower is better) achieved by floating point quantizers and Lloyd-Max quantizers on a GPT3-126M model for the Wikitext-103 dataset.}
\begin{tabular}{c|cc|c}
\hline
 \multirow{2}{*}{\textbf{Bitwidth}} & \multicolumn{2}{|c|}{\textbf{Floating-Point Quantizer}} & \textbf{Lloyd-Max Quantizer} \\
 & Best Format & Wikitext-103 Perplexity & Wikitext-103 Perplexity \\
\hline
7 & E3M3 & 18.32 & 18.27 \\
6 & E3M2 & 19.07 & 18.51 \\
5 & E4M0 & 43.89 & 19.71 \\
\hline
\end{tabular}
\end{center}
\end{table}

\subsection{Proof of Local Optimality of LO-BCQ}
\label{subsec:lobcq_opt_proof}
For a given block $\bm{b}_j$, the quantization MSE during LO-BCQ can be empirically evaluated as $\frac{1}{L_b}\lVert \bm{b}_j- \bm{\hat{b}}_j\rVert^2_2$ where $\bm{\hat{b}}_j$ is computed from equation (\ref{eq:clustered_quantization_definition}) as $C_{f(\bm{b}_j)}(\bm{b}_j)$. Further, for a given block cluster $\mathcal{B}_i$, we compute the quantization MSE as $\frac{1}{|\mathcal{B}_{i}|}\sum_{\bm{b} \in \mathcal{B}_{i}} \frac{1}{L_b}\lVert \bm{b}- C_i^{(n)}(\bm{b})\rVert^2_2$. Therefore, at the end of iteration $n$, we evaluate the overall quantization MSE $J^{(n)}$ for a given operand $\bm{X}$ composed of $N_c$ block clusters as:
\begin{align*}
    \label{eq:mse_iter_n}
    J^{(n)} = \frac{1}{N_c} \sum_{i=1}^{N_c} \frac{1}{|\mathcal{B}_{i}^{(n)}|}\sum_{\bm{v} \in \mathcal{B}_{i}^{(n)}} \frac{1}{L_b}\lVert \bm{b}- B_i^{(n)}(\bm{b})\rVert^2_2
\end{align*}

At the end of iteration $n$, the codebooks are updated from $\mathcal{C}^{(n-1)}$ to $\mathcal{C}^{(n)}$. However, the mapping of a given vector $\bm{b}_j$ to quantizers $\mathcal{C}^{(n)}$ remains as  $f^{(n)}(\bm{b}_j)$. At the next iteration, during the vector clustering step, $f^{(n+1)}(\bm{b}_j)$ finds new mapping of $\bm{b}_j$ to updated codebooks $\mathcal{C}^{(n)}$ such that the quantization MSE over the candidate codebooks is minimized. Therefore, we obtain the following result for $\bm{b}_j$:
\begin{align*}
\frac{1}{L_b}\lVert \bm{b}_j - C_{f^{(n+1)}(\bm{b}_j)}^{(n)}(\bm{b}_j)\rVert^2_2 \le \frac{1}{L_b}\lVert \bm{b}_j - C_{f^{(n)}(\bm{b}_j)}^{(n)}(\bm{b}_j)\rVert^2_2
\end{align*}

That is, quantizing $\bm{b}_j$ at the end of the block clustering step of iteration $n+1$ results in lower quantization MSE compared to quantizing at the end of iteration $n$. Since this is true for all $\bm{b} \in \bm{X}$, we assert the following:
\begin{equation}
\begin{split}
\label{eq:mse_ineq_1}
    \tilde{J}^{(n+1)} &= \frac{1}{N_c} \sum_{i=1}^{N_c} \frac{1}{|\mathcal{B}_{i}^{(n+1)}|}\sum_{\bm{b} \in \mathcal{B}_{i}^{(n+1)}} \frac{1}{L_b}\lVert \bm{b} - C_i^{(n)}(b)\rVert^2_2 \le J^{(n)}
\end{split}
\end{equation}
where $\tilde{J}^{(n+1)}$ is the the quantization MSE after the vector clustering step at iteration $n+1$.

Next, during the codebook update step (\ref{eq:quantizers_update}) at iteration $n+1$, the per-cluster codebooks $\mathcal{C}^{(n)}$ are updated to $\mathcal{C}^{(n+1)}$ by invoking the Lloyd-Max algorithm \citep{Lloyd}. We know that for any given value distribution, the Lloyd-Max algorithm minimizes the quantization MSE. Therefore, for a given vector cluster $\mathcal{B}_i$ we obtain the following result:

\begin{equation}
    \frac{1}{|\mathcal{B}_{i}^{(n+1)}|}\sum_{\bm{b} \in \mathcal{B}_{i}^{(n+1)}} \frac{1}{L_b}\lVert \bm{b}- C_i^{(n+1)}(\bm{b})\rVert^2_2 \le \frac{1}{|\mathcal{B}_{i}^{(n+1)}|}\sum_{\bm{b} \in \mathcal{B}_{i}^{(n+1)}} \frac{1}{L_b}\lVert \bm{b}- C_i^{(n)}(\bm{b})\rVert^2_2
\end{equation}

The above equation states that quantizing the given block cluster $\mathcal{B}_i$ after updating the associated codebook from $C_i^{(n)}$ to $C_i^{(n+1)}$ results in lower quantization MSE. Since this is true for all the block clusters, we derive the following result: 
\begin{equation}
\begin{split}
\label{eq:mse_ineq_2}
     J^{(n+1)} &= \frac{1}{N_c} \sum_{i=1}^{N_c} \frac{1}{|\mathcal{B}_{i}^{(n+1)}|}\sum_{\bm{b} \in \mathcal{B}_{i}^{(n+1)}} \frac{1}{L_b}\lVert \bm{b}- C_i^{(n+1)}(\bm{b})\rVert^2_2  \le \tilde{J}^{(n+1)}   
\end{split}
\end{equation}

Following (\ref{eq:mse_ineq_1}) and (\ref{eq:mse_ineq_2}), we find that the quantization MSE is non-increasing for each iteration, that is, $J^{(1)} \ge J^{(2)} \ge J^{(3)} \ge \ldots \ge J^{(M)}$ where $M$ is the maximum number of iterations. 
%Therefore, we can say that if the algorithm converges, then it must be that it has converged to a local minimum. 
\hfill $\blacksquare$


\begin{figure}
    \begin{center}
    \includegraphics[width=0.5\textwidth]{sections//figures/mse_vs_iter.pdf}
    \end{center}
    \caption{\small NMSE vs iterations during LO-BCQ compared to other block quantization proposals}
    \label{fig:nmse_vs_iter}
\end{figure}

Figure \ref{fig:nmse_vs_iter} shows the empirical convergence of LO-BCQ across several block lengths and number of codebooks. Also, the MSE achieved by LO-BCQ is compared to baselines such as MXFP and VSQ. As shown, LO-BCQ converges to a lower MSE than the baselines. Further, we achieve better convergence for larger number of codebooks ($N_c$) and for a smaller block length ($L_b$), both of which increase the bitwidth of BCQ (see Eq \ref{eq:bitwidth_bcq}).


\subsection{Additional Accuracy Results}
%Table \ref{tab:lobcq_config} lists the various LOBCQ configurations and their corresponding bitwidths.
\begin{table}
\setlength{\tabcolsep}{4.75pt}
\begin{center}
\caption{\label{tab:lobcq_config} Various LO-BCQ configurations and their bitwidths.}
\begin{tabular}{|c||c|c|c|c||c|c||c|} 
\hline
 & \multicolumn{4}{|c||}{$L_b=8$} & \multicolumn{2}{|c||}{$L_b=4$} & $L_b=2$ \\
 \hline
 \backslashbox{$L_A$\kern-1em}{\kern-1em$N_c$} & 2 & 4 & 8 & 16 & 2 & 4 & 2 \\
 \hline
 64 & 4.25 & 4.375 & 4.5 & 4.625 & 4.375 & 4.625 & 4.625\\
 \hline
 32 & 4.375 & 4.5 & 4.625& 4.75 & 4.5 & 4.75 & 4.75 \\
 \hline
 16 & 4.625 & 4.75& 4.875 & 5 & 4.75 & 5 & 5 \\
 \hline
\end{tabular}
\end{center}
\end{table}

%\subsection{Perplexity achieved by various LO-BCQ configurations on Wikitext-103 dataset}

\begin{table} \centering
\begin{tabular}{|c||c|c|c|c||c|c||c|} 
\hline
 $L_b \rightarrow$& \multicolumn{4}{c||}{8} & \multicolumn{2}{c||}{4} & 2\\
 \hline
 \backslashbox{$L_A$\kern-1em}{\kern-1em$N_c$} & 2 & 4 & 8 & 16 & 2 & 4 & 2  \\
 %$N_c \rightarrow$ & 2 & 4 & 8 & 16 & 2 & 4 & 2 \\
 \hline
 \hline
 \multicolumn{8}{c}{GPT3-1.3B (FP32 PPL = 9.98)} \\ 
 \hline
 \hline
 64 & 10.40 & 10.23 & 10.17 & 10.15 &  10.28 & 10.18 & 10.19 \\
 \hline
 32 & 10.25 & 10.20 & 10.15 & 10.12 &  10.23 & 10.17 & 10.17 \\
 \hline
 16 & 10.22 & 10.16 & 10.10 & 10.09 &  10.21 & 10.14 & 10.16 \\
 \hline
  \hline
 \multicolumn{8}{c}{GPT3-8B (FP32 PPL = 7.38)} \\ 
 \hline
 \hline
 64 & 7.61 & 7.52 & 7.48 &  7.47 &  7.55 &  7.49 & 7.50 \\
 \hline
 32 & 7.52 & 7.50 & 7.46 &  7.45 &  7.52 &  7.48 & 7.48  \\
 \hline
 16 & 7.51 & 7.48 & 7.44 &  7.44 &  7.51 &  7.49 & 7.47  \\
 \hline
\end{tabular}
\caption{\label{tab:ppl_gpt3_abalation} Wikitext-103 perplexity across GPT3-1.3B and 8B models.}
\end{table}

\begin{table} \centering
\begin{tabular}{|c||c|c|c|c||} 
\hline
 $L_b \rightarrow$& \multicolumn{4}{c||}{8}\\
 \hline
 \backslashbox{$L_A$\kern-1em}{\kern-1em$N_c$} & 2 & 4 & 8 & 16 \\
 %$N_c \rightarrow$ & 2 & 4 & 8 & 16 & 2 & 4 & 2 \\
 \hline
 \hline
 \multicolumn{5}{|c|}{Llama2-7B (FP32 PPL = 5.06)} \\ 
 \hline
 \hline
 64 & 5.31 & 5.26 & 5.19 & 5.18  \\
 \hline
 32 & 5.23 & 5.25 & 5.18 & 5.15  \\
 \hline
 16 & 5.23 & 5.19 & 5.16 & 5.14  \\
 \hline
 \multicolumn{5}{|c|}{Nemotron4-15B (FP32 PPL = 5.87)} \\ 
 \hline
 \hline
 64  & 6.3 & 6.20 & 6.13 & 6.08  \\
 \hline
 32  & 6.24 & 6.12 & 6.07 & 6.03  \\
 \hline
 16  & 6.12 & 6.14 & 6.04 & 6.02  \\
 \hline
 \multicolumn{5}{|c|}{Nemotron4-340B (FP32 PPL = 3.48)} \\ 
 \hline
 \hline
 64 & 3.67 & 3.62 & 3.60 & 3.59 \\
 \hline
 32 & 3.63 & 3.61 & 3.59 & 3.56 \\
 \hline
 16 & 3.61 & 3.58 & 3.57 & 3.55 \\
 \hline
\end{tabular}
\caption{\label{tab:ppl_llama7B_nemo15B} Wikitext-103 perplexity compared to FP32 baseline in Llama2-7B and Nemotron4-15B, 340B models}
\end{table}

%\subsection{Perplexity achieved by various LO-BCQ configurations on MMLU dataset}


\begin{table} \centering
\begin{tabular}{|c||c|c|c|c||c|c|c|c|} 
\hline
 $L_b \rightarrow$& \multicolumn{4}{c||}{8} & \multicolumn{4}{c||}{8}\\
 \hline
 \backslashbox{$L_A$\kern-1em}{\kern-1em$N_c$} & 2 & 4 & 8 & 16 & 2 & 4 & 8 & 16  \\
 %$N_c \rightarrow$ & 2 & 4 & 8 & 16 & 2 & 4 & 2 \\
 \hline
 \hline
 \multicolumn{5}{|c|}{Llama2-7B (FP32 Accuracy = 45.8\%)} & \multicolumn{4}{|c|}{Llama2-70B (FP32 Accuracy = 69.12\%)} \\ 
 \hline
 \hline
 64 & 43.9 & 43.4 & 43.9 & 44.9 & 68.07 & 68.27 & 68.17 & 68.75 \\
 \hline
 32 & 44.5 & 43.8 & 44.9 & 44.5 & 68.37 & 68.51 & 68.35 & 68.27  \\
 \hline
 16 & 43.9 & 42.7 & 44.9 & 45 & 68.12 & 68.77 & 68.31 & 68.59  \\
 \hline
 \hline
 \multicolumn{5}{|c|}{GPT3-22B (FP32 Accuracy = 38.75\%)} & \multicolumn{4}{|c|}{Nemotron4-15B (FP32 Accuracy = 64.3\%)} \\ 
 \hline
 \hline
 64 & 36.71 & 38.85 & 38.13 & 38.92 & 63.17 & 62.36 & 63.72 & 64.09 \\
 \hline
 32 & 37.95 & 38.69 & 39.45 & 38.34 & 64.05 & 62.30 & 63.8 & 64.33  \\
 \hline
 16 & 38.88 & 38.80 & 38.31 & 38.92 & 63.22 & 63.51 & 63.93 & 64.43  \\
 \hline
\end{tabular}
\caption{\label{tab:mmlu_abalation} Accuracy on MMLU dataset across GPT3-22B, Llama2-7B, 70B and Nemotron4-15B models.}
\end{table}


%\subsection{Perplexity achieved by various LO-BCQ configurations on LM evaluation harness}

\begin{table} \centering
\begin{tabular}{|c||c|c|c|c||c|c|c|c|} 
\hline
 $L_b \rightarrow$& \multicolumn{4}{c||}{8} & \multicolumn{4}{c||}{8}\\
 \hline
 \backslashbox{$L_A$\kern-1em}{\kern-1em$N_c$} & 2 & 4 & 8 & 16 & 2 & 4 & 8 & 16  \\
 %$N_c \rightarrow$ & 2 & 4 & 8 & 16 & 2 & 4 & 2 \\
 \hline
 \hline
 \multicolumn{5}{|c|}{Race (FP32 Accuracy = 37.51\%)} & \multicolumn{4}{|c|}{Boolq (FP32 Accuracy = 64.62\%)} \\ 
 \hline
 \hline
 64 & 36.94 & 37.13 & 36.27 & 37.13 & 63.73 & 62.26 & 63.49 & 63.36 \\
 \hline
 32 & 37.03 & 36.36 & 36.08 & 37.03 & 62.54 & 63.51 & 63.49 & 63.55  \\
 \hline
 16 & 37.03 & 37.03 & 36.46 & 37.03 & 61.1 & 63.79 & 63.58 & 63.33  \\
 \hline
 \hline
 \multicolumn{5}{|c|}{Winogrande (FP32 Accuracy = 58.01\%)} & \multicolumn{4}{|c|}{Piqa (FP32 Accuracy = 74.21\%)} \\ 
 \hline
 \hline
 64 & 58.17 & 57.22 & 57.85 & 58.33 & 73.01 & 73.07 & 73.07 & 72.80 \\
 \hline
 32 & 59.12 & 58.09 & 57.85 & 58.41 & 73.01 & 73.94 & 72.74 & 73.18  \\
 \hline
 16 & 57.93 & 58.88 & 57.93 & 58.56 & 73.94 & 72.80 & 73.01 & 73.94  \\
 \hline
\end{tabular}
\caption{\label{tab:mmlu_abalation} Accuracy on LM evaluation harness tasks on GPT3-1.3B model.}
\end{table}

\begin{table} \centering
\begin{tabular}{|c||c|c|c|c||c|c|c|c|} 
\hline
 $L_b \rightarrow$& \multicolumn{4}{c||}{8} & \multicolumn{4}{c||}{8}\\
 \hline
 \backslashbox{$L_A$\kern-1em}{\kern-1em$N_c$} & 2 & 4 & 8 & 16 & 2 & 4 & 8 & 16  \\
 %$N_c \rightarrow$ & 2 & 4 & 8 & 16 & 2 & 4 & 2 \\
 \hline
 \hline
 \multicolumn{5}{|c|}{Race (FP32 Accuracy = 41.34\%)} & \multicolumn{4}{|c|}{Boolq (FP32 Accuracy = 68.32\%)} \\ 
 \hline
 \hline
 64 & 40.48 & 40.10 & 39.43 & 39.90 & 69.20 & 68.41 & 69.45 & 68.56 \\
 \hline
 32 & 39.52 & 39.52 & 40.77 & 39.62 & 68.32 & 67.43 & 68.17 & 69.30  \\
 \hline
 16 & 39.81 & 39.71 & 39.90 & 40.38 & 68.10 & 66.33 & 69.51 & 69.42  \\
 \hline
 \hline
 \multicolumn{5}{|c|}{Winogrande (FP32 Accuracy = 67.88\%)} & \multicolumn{4}{|c|}{Piqa (FP32 Accuracy = 78.78\%)} \\ 
 \hline
 \hline
 64 & 66.85 & 66.61 & 67.72 & 67.88 & 77.31 & 77.42 & 77.75 & 77.64 \\
 \hline
 32 & 67.25 & 67.72 & 67.72 & 67.00 & 77.31 & 77.04 & 77.80 & 77.37  \\
 \hline
 16 & 68.11 & 68.90 & 67.88 & 67.48 & 77.37 & 78.13 & 78.13 & 77.69  \\
 \hline
\end{tabular}
\caption{\label{tab:mmlu_abalation} Accuracy on LM evaluation harness tasks on GPT3-8B model.}
\end{table}

\begin{table} \centering
\begin{tabular}{|c||c|c|c|c||c|c|c|c|} 
\hline
 $L_b \rightarrow$& \multicolumn{4}{c||}{8} & \multicolumn{4}{c||}{8}\\
 \hline
 \backslashbox{$L_A$\kern-1em}{\kern-1em$N_c$} & 2 & 4 & 8 & 16 & 2 & 4 & 8 & 16  \\
 %$N_c \rightarrow$ & 2 & 4 & 8 & 16 & 2 & 4 & 2 \\
 \hline
 \hline
 \multicolumn{5}{|c|}{Race (FP32 Accuracy = 40.67\%)} & \multicolumn{4}{|c|}{Boolq (FP32 Accuracy = 76.54\%)} \\ 
 \hline
 \hline
 64 & 40.48 & 40.10 & 39.43 & 39.90 & 75.41 & 75.11 & 77.09 & 75.66 \\
 \hline
 32 & 39.52 & 39.52 & 40.77 & 39.62 & 76.02 & 76.02 & 75.96 & 75.35  \\
 \hline
 16 & 39.81 & 39.71 & 39.90 & 40.38 & 75.05 & 73.82 & 75.72 & 76.09  \\
 \hline
 \hline
 \multicolumn{5}{|c|}{Winogrande (FP32 Accuracy = 70.64\%)} & \multicolumn{4}{|c|}{Piqa (FP32 Accuracy = 79.16\%)} \\ 
 \hline
 \hline
 64 & 69.14 & 70.17 & 70.17 & 70.56 & 78.24 & 79.00 & 78.62 & 78.73 \\
 \hline
 32 & 70.96 & 69.69 & 71.27 & 69.30 & 78.56 & 79.49 & 79.16 & 78.89  \\
 \hline
 16 & 71.03 & 69.53 & 69.69 & 70.40 & 78.13 & 79.16 & 79.00 & 79.00  \\
 \hline
\end{tabular}
\caption{\label{tab:mmlu_abalation} Accuracy on LM evaluation harness tasks on GPT3-22B model.}
\end{table}

\begin{table} \centering
\begin{tabular}{|c||c|c|c|c||c|c|c|c|} 
\hline
 $L_b \rightarrow$& \multicolumn{4}{c||}{8} & \multicolumn{4}{c||}{8}\\
 \hline
 \backslashbox{$L_A$\kern-1em}{\kern-1em$N_c$} & 2 & 4 & 8 & 16 & 2 & 4 & 8 & 16  \\
 %$N_c \rightarrow$ & 2 & 4 & 8 & 16 & 2 & 4 & 2 \\
 \hline
 \hline
 \multicolumn{5}{|c|}{Race (FP32 Accuracy = 44.4\%)} & \multicolumn{4}{|c|}{Boolq (FP32 Accuracy = 79.29\%)} \\ 
 \hline
 \hline
 64 & 42.49 & 42.51 & 42.58 & 43.45 & 77.58 & 77.37 & 77.43 & 78.1 \\
 \hline
 32 & 43.35 & 42.49 & 43.64 & 43.73 & 77.86 & 75.32 & 77.28 & 77.86  \\
 \hline
 16 & 44.21 & 44.21 & 43.64 & 42.97 & 78.65 & 77 & 76.94 & 77.98  \\
 \hline
 \hline
 \multicolumn{5}{|c|}{Winogrande (FP32 Accuracy = 69.38\%)} & \multicolumn{4}{|c|}{Piqa (FP32 Accuracy = 78.07\%)} \\ 
 \hline
 \hline
 64 & 68.9 & 68.43 & 69.77 & 68.19 & 77.09 & 76.82 & 77.09 & 77.86 \\
 \hline
 32 & 69.38 & 68.51 & 68.82 & 68.90 & 78.07 & 76.71 & 78.07 & 77.86  \\
 \hline
 16 & 69.53 & 67.09 & 69.38 & 68.90 & 77.37 & 77.8 & 77.91 & 77.69  \\
 \hline
\end{tabular}
\caption{\label{tab:mmlu_abalation} Accuracy on LM evaluation harness tasks on Llama2-7B model.}
\end{table}

\begin{table} \centering
\begin{tabular}{|c||c|c|c|c||c|c|c|c|} 
\hline
 $L_b \rightarrow$& \multicolumn{4}{c||}{8} & \multicolumn{4}{c||}{8}\\
 \hline
 \backslashbox{$L_A$\kern-1em}{\kern-1em$N_c$} & 2 & 4 & 8 & 16 & 2 & 4 & 8 & 16  \\
 %$N_c \rightarrow$ & 2 & 4 & 8 & 16 & 2 & 4 & 2 \\
 \hline
 \hline
 \multicolumn{5}{|c|}{Race (FP32 Accuracy = 48.8\%)} & \multicolumn{4}{|c|}{Boolq (FP32 Accuracy = 85.23\%)} \\ 
 \hline
 \hline
 64 & 49.00 & 49.00 & 49.28 & 48.71 & 82.82 & 84.28 & 84.03 & 84.25 \\
 \hline
 32 & 49.57 & 48.52 & 48.33 & 49.28 & 83.85 & 84.46 & 84.31 & 84.93  \\
 \hline
 16 & 49.85 & 49.09 & 49.28 & 48.99 & 85.11 & 84.46 & 84.61 & 83.94  \\
 \hline
 \hline
 \multicolumn{5}{|c|}{Winogrande (FP32 Accuracy = 79.95\%)} & \multicolumn{4}{|c|}{Piqa (FP32 Accuracy = 81.56\%)} \\ 
 \hline
 \hline
 64 & 78.77 & 78.45 & 78.37 & 79.16 & 81.45 & 80.69 & 81.45 & 81.5 \\
 \hline
 32 & 78.45 & 79.01 & 78.69 & 80.66 & 81.56 & 80.58 & 81.18 & 81.34  \\
 \hline
 16 & 79.95 & 79.56 & 79.79 & 79.72 & 81.28 & 81.66 & 81.28 & 80.96  \\
 \hline
\end{tabular}
\caption{\label{tab:mmlu_abalation} Accuracy on LM evaluation harness tasks on Llama2-70B model.}
\end{table}

%\section{MSE Studies}
%\textcolor{red}{TODO}


\subsection{Number Formats and Quantization Method}
\label{subsec:numFormats_quantMethod}
\subsubsection{Integer Format}
An $n$-bit signed integer (INT) is typically represented with a 2s-complement format \citep{yao2022zeroquant,xiao2023smoothquant,dai2021vsq}, where the most significant bit denotes the sign.

\subsubsection{Floating Point Format}
An $n$-bit signed floating point (FP) number $x$ comprises of a 1-bit sign ($x_{\mathrm{sign}}$), $B_m$-bit mantissa ($x_{\mathrm{mant}}$) and $B_e$-bit exponent ($x_{\mathrm{exp}}$) such that $B_m+B_e=n-1$. The associated constant exponent bias ($E_{\mathrm{bias}}$) is computed as $(2^{{B_e}-1}-1)$. We denote this format as $E_{B_e}M_{B_m}$.  

\subsubsection{Quantization Scheme}
\label{subsec:quant_method}
A quantization scheme dictates how a given unquantized tensor is converted to its quantized representation. We consider FP formats for the purpose of illustration. Given an unquantized tensor $\bm{X}$ and an FP format $E_{B_e}M_{B_m}$, we first, we compute the quantization scale factor $s_X$ that maps the maximum absolute value of $\bm{X}$ to the maximum quantization level of the $E_{B_e}M_{B_m}$ format as follows:
\begin{align}
\label{eq:sf}
    s_X = \frac{\mathrm{max}(|\bm{X}|)}{\mathrm{max}(E_{B_e}M_{B_m})}
\end{align}
In the above equation, $|\cdot|$ denotes the absolute value function.

Next, we scale $\bm{X}$ by $s_X$ and quantize it to $\hat{\bm{X}}$ by rounding it to the nearest quantization level of $E_{B_e}M_{B_m}$ as:

\begin{align}
\label{eq:tensor_quant}
    \hat{\bm{X}} = \text{round-to-nearest}\left(\frac{\bm{X}}{s_X}, E_{B_e}M_{B_m}\right)
\end{align}

We perform dynamic max-scaled quantization \citep{wu2020integer}, where the scale factor $s$ for activations is dynamically computed during runtime.

\subsection{Vector Scaled Quantization}
\begin{wrapfigure}{r}{0.35\linewidth}
  \centering
  \includegraphics[width=\linewidth]{sections/figures/vsquant.jpg}
  \caption{\small Vectorwise decomposition for per-vector scaled quantization (VSQ \citep{dai2021vsq}).}
  \label{fig:vsquant}
\end{wrapfigure}
During VSQ \citep{dai2021vsq}, the operand tensors are decomposed into 1D vectors in a hardware friendly manner as shown in Figure \ref{fig:vsquant}. Since the decomposed tensors are used as operands in matrix multiplications during inference, it is beneficial to perform this decomposition along the reduction dimension of the multiplication. The vectorwise quantization is performed similar to tensorwise quantization described in Equations \ref{eq:sf} and \ref{eq:tensor_quant}, where a scale factor $s_v$ is required for each vector $\bm{v}$ that maps the maximum absolute value of that vector to the maximum quantization level. While smaller vector lengths can lead to larger accuracy gains, the associated memory and computational overheads due to the per-vector scale factors increases. To alleviate these overheads, VSQ \citep{dai2021vsq} proposed a second level quantization of the per-vector scale factors to unsigned integers, while MX \citep{rouhani2023shared} quantizes them to integer powers of 2 (denoted as $2^{INT}$).

\subsubsection{MX Format}
The MX format proposed in \citep{rouhani2023microscaling} introduces the concept of sub-block shifting. For every two scalar elements of $b$-bits each, there is a shared exponent bit. The value of this exponent bit is determined through an empirical analysis that targets minimizing quantization MSE. We note that the FP format $E_{1}M_{b}$ is strictly better than MX from an accuracy perspective since it allocates a dedicated exponent bit to each scalar as opposed to sharing it across two scalars. Therefore, we conservatively bound the accuracy of a $b+2$-bit signed MX format with that of a $E_{1}M_{b}$ format in our comparisons. For instance, we use E1M2 format as a proxy for MX4.

\begin{figure}
    \centering
    \includegraphics[width=1\linewidth]{sections//figures/BlockFormats.pdf}
    \caption{\small Comparing LO-BCQ to MX format.}
    \label{fig:block_formats}
\end{figure}

Figure \ref{fig:block_formats} compares our $4$-bit LO-BCQ block format to MX \citep{rouhani2023microscaling}. As shown, both LO-BCQ and MX decompose a given operand tensor into block arrays and each block array into blocks. Similar to MX, we find that per-block quantization ($L_b < L_A$) leads to better accuracy due to increased flexibility. While MX achieves this through per-block $1$-bit micro-scales, we associate a dedicated codebook to each block through a per-block codebook selector. Further, MX quantizes the per-block array scale-factor to E8M0 format without per-tensor scaling. In contrast during LO-BCQ, we find that per-tensor scaling combined with quantization of per-block array scale-factor to E4M3 format results in superior inference accuracy across models. 


% \clearpage
% \fbox{\begin{minipage}{38em}

\subsubsection*{Scaling Law Reproducilibility Checklist}\label{sec:checklist}


\small

\begin{minipage}[t]{0.48\textwidth}
\raggedright
\paragraph{Scaling Law Hypothesis (\S\ref{sec:power-law-form})}

\begin{itemize}[leftmargin=*]
    \item What is the form of the power law?
    \item What are the variables related by (included in) the power law?
    \item What are the parameters to fit?
    \item On what principles is this form derived?
    \item Does this form make assumptions about how the variables are related?
    % \item How are each of these variables counted? (For example, how is compute cost/FLOPs counted, if applicable? How are parameters of the model counted?)
    % \item Are code/code snippets provided for calculating these variables if applicable? 
\end{itemize}


\paragraph{Training Setup (\S\ref{sec:model_training})}
\begin{itemize}[leftmargin=*]
    \item How many models are trained?
    \item At which sizes?
    \item On how much data each? On what data? Is any data repeated within the training for a model?
    \item How are model size, dataset size, and compute budget size counted? For example, how are parameters of the model counted? Are any parameters excluded (e.g., embedding layers)?
    \item Are code/code snippets provided for calculating these variables if applicable?
    % embedding  For example, how is compute cost counted, if applicable? 
    \item How are hyperparameters chosen (e.g., optimizer, learning rate schedule, batch size)? Do they change with scale?
    \item What other settings must be decided (e.g., model width vs. depth)? Do they change with scale?
    \item Is the training code open source?
    % \item How is the correctness of the scaling law considered SHOULD WE?
\end{itemize}

\end{minipage}
\begin{minipage}[t]{0.48\textwidth}
\raggedright


\paragraph{Data Collection(\S\ref{sec:data})}
\begin{itemize}[leftmargin=*]
    \item Are the model checkpoints provided openly?
    % \item Are these checkpoints modified in any way before evaluation? (say, checkpoint averaging)
    % \item If the above is done, is code for modifying the checkpoints provided?
    \item How many checkpoints per model are evaluated to fit each scaling law?
    \item What evaluation metric is used? On what dataset?
    \item Are the raw evaluation metrics modified, e.g., through loss interpolation, centering around a mean, scaling logarithmically, etc?
    \item If the above is done, is code for modifying the metric provided? 
\end{itemize}

\paragraph{Fitting Algorithm (\S\ref{sec:opt})}
\begin{itemize}[leftmargin=*]
    \item What objective (loss) is used?
    \item What algorithm is used to fit the equation?
    \item What hyperparameters are used for this algorithm?
    \item How is this algorithm initialized?
    \item Are all datapoints collected used to fit the equations? For example, are any outliers dropped? Are portions of the datapoints used to fit different equations?
    \item How is the correctness of the scaling law considered? Extrapolation, Confidence Intervals, Goodness of Fit?
\end{itemize}

\end{minipage}

% \paragraph{Other}
% \begin{itemize}
%     \item Is code for 
% \end{itemize}

\end{minipage}}


\end{document}