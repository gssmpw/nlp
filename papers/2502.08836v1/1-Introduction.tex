\section{Introduction}
\label{sec:intro}
Single-image reflection removal (SIRR) is a critical task in image processing, focusing on recovering the true scene behind reflections from reflective surfaces (e.g., transparent glasses). 

Over the years, various techniques have been proposed to solve the SIRR problem. Traditional methods typically relied on a non-learning paradigm. Since the goal of SIRR is to recover the transmission image (i.e., the true scene) from a blended image containing both the scene and reflections, this is an ill-posed problem. In the absence of additional information about the scene, there are an infinite number of possible decompositions of the image~\cite{levin2004separating}. Therefore, these traditional methods often exploit prior knowledge (or priors) to constrain the solution space and guide the recovery process. One widely used prior is sparsity priors~\cite{levin2002learning, levin2007user}. These approaches impose gradient sparsity constraints to find the minimum edges and corners of the layer decomposition. The fundamental idea is to require the image gradient histogram to have a long-tail distribution~\cite{li2014single}. Another common assumption is the use of smoothness priors~\cite{li2014single, yang2019fast, wan2016depth}. This prior is based on the observation that reflection layers are more likely to be blurred compared to the background scene, primarily due to differences in the distance from the camera. Additionally, Shih et al.~\cite{shih2015reflection} introduced the concept of examining ghosting effects caused by double-pane windows. These ghosting effects lead to multiple reflections in the captured image, which can be challenging to separate. To address this, they proposed modeling these effects using Gaussian Mixture Models (GMM) as a patch-based prior.

However, these priors often struggle to generalize well across different types of reflections and scenes, particularly when dealing with complex real-world environments. Most of these priors are based on the assumption that the captured image $I$ is a linear combination of the transmitted scene $T$ and a reflection $R$, i.e., $I = T + R$. While this assumption is simple, it often deviates significantly from reality, as real-world reflections are more complex and influenced by factors such as lighting conditions, surface characteristics, and so on~\cite{song2023robust}. Moreover, both $T$ and $R$ may contain content from real-world scenes, leading to overlapping appearance distributions that make the separation of the two components more challenging~\cite{wei2019single}. To address these issues, researchers have recently shifted focus toward learning-based methods, particularly data-driven deep learning approaches, driven by the success of deep neural networks in tackling versatile computer vision problems. Instead of relying on handcrafted priors, deep learning models can be trained on large, labeled datasets, enabling them to effectively handle a broad spectrum of scenarios. 

This article presents a comprehensive survey of SIRR research utilizing deep learning methods. We analyze current research trends and identify future opportunities. Compared to existing surveys on similar topics, this article offers a more thorough review of the literature, focusing on key journals and conferences, with the aim to of presenting concise, critical, and recent research advances. For example, although Wan et al.\cite{wan2017benchmarking, wan2022benchmarking} provided a brief survey, their focus was primarily on introducing their new datasets (SIR$^2$ and SIR$^{2+}$) and establishing benchmarks for different algorithms. Amanlou et al.\cite{amanlou2022single} conducted a survey on SIRR using deep learning, but their coverage is limited to work published between 2015 and 2021. In contrast, our review covers the most recent advancements and organizes the research within a more detailed framework.

The article is organized as follows: Section 2 describes the methodology used to conduct the bibliographic search. Section 3 introduces the mathematical hypotheses for modeling SIRR. Section 4 surveys the SIRR research from three perspectives: single-stage, two-stage, and multi-stage approaches. Section 5 discusses currently available public datasets and commonly used evaluation metrics. Finally, Section 6 explores future research opportunities in SIRR field, followed by brief concluding remarks in Section 7.