% Template for ICIP-2022 paper; to be used with:
%          spconf.sty  - ICASSP/ICIP LaTeX style file, and
%          IEEEbib.bst - IEEE bibliography style file.
% --------------------------------------------------------------------------
\documentclass{article}
\usepackage{spconf,amsmath,graphicx}
\usepackage{multirow}
\usepackage{tabularray}
\usepackage{subcaption}
\usepackage[shortlabels]{enumitem}
\usepackage[hidelinks]{hyperref}
\usepackage{indentfirst}
\usepackage{booktabs}
\usepackage{arydshln}
\usepackage{dcolumn}
\usepackage[table]{xcolor}
\usepackage{amssymb}
\usepackage{cite}

% Example definitions.
% --------------------
\def\x{{\mathbf x}}
\def\L{{\cal L}}

% Title.
% ------
\title{Survey on Single-Image Reflection Removal using Deep Learning Techniques}
%
% Single address.
% ---------------
% \name{Author(s) Name(s)\thanks{Thanks to XYZ agency for funding.}}
% \address{Author Affiliation(s)}
\name{Kangning Yang, Huiming Sun, Jie Cai, Lan Fu, Jiaming Ding, Jinlong Li, Chiu Man Ho, Zibo Meng}
\address{OPPO AI Center}

%
% For example:
% ------------
%\address{School\\
%	Department\\
%	Address}
%
% Two addresses (uncomment and modify for two-address case).
% ----------------------------------------------------------
%\twoauthors
%  {A. Author-one, B. Author-two\sthanks{Thanks to XYZ agency for funding.}}
%	{School A-B\\
%	Department A-B\\
%	Address A-B}
%  {C. Author-three, D. Author-four\sthanks{The fourth author performed the work
%	while at ...}}
%	{School C-D\\
%	Department C-D\\
%	Address C-D}
%
\begin{document}
% \topmargin=0mm
%\ninept
%
\maketitle
%
\begin{abstract}
The phenomenon of reflection is quite common in digital images, posing significant challenges for various applications such as computer vision, photography, and image processing. Traditional methods for reflection removal often struggle to achieve clean results while maintaining high fidelity and robustness, particularly in real-world scenarios. Over the past few decades, numerous deep learning-based approaches for reflection removal have emerged, yielding impressive results. In this survey, we conduct a comprehensive review of the current literature by focusing on key venues such as ICCV, ECCV, CVPR, NeurIPS, etc., as these conferences and journals have been central to advances in the field. Our review follows a structured paper selection process, and we critically assess both single-stage and two-stage deep learning methods for reflection removal. The contribution of this survey is three-fold: first, we provide a comprehensive summary of the most recent work on single-image reflection removal; second, we outline task hypotheses, current deep learning techniques, publicly available datasets, and relevant evaluation metrics; and third, we identify key challenges and opportunities in deep learning-based reflection removal, highlighting the potential of this rapidly evolving research area.
\end{abstract}
%
\begin{keywords}
single-image reflection removal, deep learning
\end{keywords}
%

\section{Introduction}

Deep Reinforcement Learning (DRL) has emerged as a transformative paradigm for solving complex sequential decision-making problems. By enabling autonomous agents to interact with an environment, receive feedback in the form of rewards, and iteratively refine their policies, DRL has demonstrated remarkable success across a diverse range of domains including games (\eg Atari~\citep{mnih2013playing,kaiser2020model}, Go~\citep{silver2018general,silver2017mastering}, and StarCraft II~\citep{vinyals2019grandmaster,vinyals2017starcraft}), robotics~\citep{kalashnikov2018scalable}, communication networks~\citep{feriani2021single}, and finance~\citep{liu2024dynamic}. These successes underscore DRL's capability to surpass traditional rule-based systems, particularly in high-dimensional and dynamically evolving environments.

Despite these advances, a fundamental challenge remains: DRL agents typically rely on deep neural networks, which operate as black-box models, obscuring the rationale behind their decision-making processes. This opacity poses significant barriers to adoption in safety-critical and high-stakes applications, where interpretability is crucial for trust, compliance, and debugging. The lack of transparency in DRL can lead to unreliable decision-making, rendering it unsuitable for domains where explainability is a prerequisite, such as healthcare, autonomous driving, and financial risk assessment.

To address these concerns, the field of Explainable Deep Reinforcement Learning (XRL) has emerged, aiming to develop techniques that enhance the interpretability of DRL policies. XRL seeks to provide insights into an agent’s decision-making process, enabling researchers, practitioners, and end-users to understand, validate, and refine learned policies. By facilitating greater transparency, XRL contributes to the development of safer, more robust, and ethically aligned AI systems.

Furthermore, the increasing integration of Reinforcement Learning (RL) with Large Language Models (LLMs) has placed RL at the forefront of natural language processing (NLP) advancements. Methods such as Reinforcement Learning from Human Feedback (RLHF)~\citep{bai2022training,ouyang2022training} have become essential for aligning LLM outputs with human preferences and ethical guidelines. By treating language generation as a sequential decision-making process, RL-based fine-tuning enables LLMs to optimize for attributes such as factual accuracy, coherence, and user satisfaction, surpassing conventional supervised learning techniques. However, the application of RL in LLM alignment further amplifies the explainability challenge, as the complex interactions between RL updates and neural representations remain poorly understood.

This survey provides a systematic review of explainability methods in DRL, with a particular focus on their integration with LLMs and human-in-the-loop systems. We first introduce fundamental RL concepts and highlight key advances in DRL. We then categorize and analyze existing explanation techniques, encompassing feature-level, state-level, dataset-level, and model-level approaches. Additionally, we discuss methods for evaluating XRL techniques, considering both qualitative and quantitative assessment criteria. Finally, we explore real-world applications of XRL, including policy refinement, adversarial attack mitigation, and emerging challenges in ensuring interpretability in modern AI systems. Through this survey, we aim to provide a comprehensive perspective on the current state of XRL and outline future research directions to advance the development of interpretable and trustworthy DRL models.
\section{Methodology}
\label{sec:methodology}
We performed a two-step methodology—filtering and investigation—to construct the Rust flaky test dataset.

\subsection{Filtering}
To accurately investigate the root causes of flakiness, we focus on \textit{already fixed} flaky tests. Our approach involves leveraging the GitHub REST API to search for issues in repositories using Rust, specifically querying for the term \textit{'flaky'}. The search was narrowed to only closed issues that have linked pull requests. Finally, we create a dataset containing all issues captured on October 29, 2024. 
This process resulted in a dataset of 1,146 issues potentially related to Rust flaky tests. 
% \yang{from my understanding, \hl{1,146} is the number of issues, then what's the number of tests?}\tom{Unknown. Only by reviewing an issue can we know the actual number of flaky tests.}

% \tom{This might be a good spot to mention that the results of the pull from GH were shuffled before serializing to the initial CSV file}\yang{mentioned below}

\subsection{Investigation}
Manually investigating over 1,000 issues is time-consuming. To ensure the diversity of tests in our study within a limited time, we shuffled the dataset and performed our analysis from start to finish. At the time of submission, we have manually investigated a subset of 53 tests from 49 projects of their root causes and fix strategies to conduct a deep analysis through the following process: 
We first reviewed the description provided in each issue, which may include the test failure scenario, the assertion failure, and the hypothesized source of the issue. We then reviewed the linked pull request to verify it as the fix for flakiness. Finally, we categorized the cause of flakiness and its fixing strategy. Note that we had to exclude certain cases because (1) some issues identified in our GitHub search using 'flaky' or its variations were actually not relevant to flaky tests, and (2) the issues pertained to parts of the repository not utilizing Rust, such as Python bindings or CI infrastructure.

% \yang{@Tom, how do you filter those N/A cases? Please elaborate on details of filtering out those flaky tests} 
% \tom{Captured general process and filtering of N/A below}
% We first reviewed the description provided in each issue, which may include the test failure scenario, the assertion failure, and the hypothesized source of the issue. We then reviewed its linked pull request to mark it as fixing the issue. Finally, we determined the category of the cause of the flakiness and the strategy for fixing it.

% In some cases, issues identified in the GitHub search query that we analyzed used the term 'flaky' or some variation but were not flaky tests. In other cases, the issue related to a portion of the repository that was not using Rust and instead was, for example, Python bindings or CI infrastructure. These issues were noted as 'not applicable' and filtered out of the final results for these cases.
\section{Mathematical Hypothesis}
\label{sec:hypothesis}
As previously mentioned, SIRR is inherently an ill-posed problem. To this end, researchers have proposed various hypotheses.

\subsection{Linear Hypothesis}
The linear hypothesis posits that a captured image $I$ is perceived as the superimposition of a transmission layer and a reflection layer, a concept inspired by the human visual system~\cite{levin2002learning}.

Early studies, mainly non-learning approaches~\cite{li2014single, levin2007user} and early deep learning works~\cite{zhang2018single, fan2017generic, hu2021trash}, adopted this hypothesis, assuming that an image containing reflections can be mathematically modeled as the sum of the transmission and reflection layers, i.e., $I = T + R$. However, this assumption is heuristic; the reflection and transmission layers are likely to degrade due to diffusion during the superposition process~\cite{hu2023single, wan2020reflection}, and it may not hold true in cases involving intense and bright reflections~\cite{li2023two}.

Furthermore, researchers~\cite{wan2018crrn, yang2018seeing, li2020single} introduced blending scalars to build a more nuanced model, $I = \alpha T + \beta R$, where $\alpha$ and $\beta$ represent scaling factors for the transmission and reflection layers, respectively. For instance, Li et al.~\cite{li2020single} assumes $I = \alpha T + R$, while Yang et al.~\cite{yang2018seeing} assumes that $I = \alpha T + (1-\alpha)R$, and Wan et al.~\cite{wan2018crrn} treats $\alpha$ and $\beta$ as mixing coefficients balancing the transmission and reflection layers. 

Despite these improvements, blending two images using constant values does not accurately simulate the complex real-world reflection process. The formation of the reflected image depends on factors such as the relative position of the camera to the image plane and the lighting conditions~\cite{wen2019single}.

\subsection{Non-linear Hypothesis}
To address the complex process of image reflection, some studies have leveraged the full potential of deep learning to efficiently incorporate prior information mined from labeled training data into network structures. These studies also introduced non-linearity to develop more sophisticated models that better approximate the physical mechanisms involved in image formation~\cite{dong2021location, li2020single, wei2019single}. One such approach utilizes alpha matting~\cite{dong2021location} to model the blending process. In this formulation, an alpha blending mask $W$ is introduced to represent the relative contribution of the transmission layer at each pixel. The synthesis process is then represented as: $I = W \circ T + R$, where $\circ$ denotes the element-wise multiplication. Similarly, Wen et al.~\cite{wen2019single} approximates the reflection mechanisms as: $I = W \circ T + (1 - W) \circ R$. This approach enables a more flexible and accurate separation of the scene and reflection layers, particularly in cases involving complex light interactions or semi-transparent surfaces. 

Additionally, Zheng et al.~\cite{zheng2021single} proposed the model $I = \Omega T + \Phi R$, where $\Omega$ and $\Phi$ represent refractive and reflective amplitude coefficient maps, respectively. Likewise, Wan et al.~\cite{wan2020reflection} considered degradation in both $T$ and $R$, expressing the formation of a mixed image as $I = g(T) + f(R)$, where $g(\cdot)$ and $f(\cdot)$ represent the various degradation processes learned by the network structure. More recently, Hu and Guo~\cite{hu2023single} deliver a more general formulation as: $I = T + R + \Phi(T, R)$, where $\Phi(T, R)$ represents the residue in the reconstruction process, which may arise due to factors such as attenuation, overexposure, etc.

\begin{table*}
\centering
\renewcommand{\dashlinedash}{0.5pt} 
\renewcommand{\dashlinegap}{2pt}    
\begin{tabular}{lllll}
\hline\hline
 & \shortstack[l]{\space \\ \space \\ \textbf{Methods} \\ \space } & 
   \shortstack[l]{\space \\ \space \\ \textbf{Venue} \\ \space } & 
   \shortstack[l]{\space \\ \space \\ \textbf{Scheme} \\ \space } & 
   \shortstack[l]{\space \\ \space \\ \textbf{Cross-stage fusion} \\ \space} \\
\bottomrule   \\
Single-stage
& Zhang et al.~\cite{zhang2018single} & CVPR 2018 & $I \rightarrow [T, R]$ & - \\ 
& ERRNet~\cite{wei2019single} & CVPR 2019 & $I \rightarrow T$ & - \\ 
& RobustSIRR~\cite{song2023robust} & CVPR 2023 & $I_{multiscale} \rightarrow T$ & - \\
& YTMT~\cite{hu2021trash} & NeurIPS 2021 & $I \rightarrow [T, R]$ & - \\   \bottomrule   \\
Two-stage
& CoRRN~\cite{wan2019corrn} & TPAMI 2019 & \shortstack[l]{\space \\$I \rightarrow E_{T}$ \\ $[I, E_{T}] \rightarrow T$} & Convolutional Fusion \\ \cline{2-5}
& DMGN~\cite{feng2021deep} & TIP 2021 & \shortstack[l]{\space \\ $I \rightarrow [T_{1}, R]$ \\ $[I, T_{1}, R] \rightarrow T$} & Convolutional Fusion \\ \cline{2-5}
& RAGNet~\cite{li2023two} & Appl. Intell. 2023 & \shortstack[l]{\space \\ $I \rightarrow R$ \\ $ [I, R] \rightarrow T$} & Convolutional Fusion \\ \cline{2-5}
& CEILNet~\cite{fan2017generic} & ICCV 2017 & \shortstack[l]{\space \\ $[I, E_{I}] \rightarrow E_{T}$ \\ $[I, E_{T}] \rightarrow T$ } & Concat \\ \cline{2-5}
& DSRNet~\cite{hu2023single} & ICCV 2023 & \shortstack[l]{\space \\ $I \rightarrow (T_{1}, R_{1})$ \\ $(R_{1}, T_{1}) \rightarrow (R, T, residue)$} & N/A \\ \cline{2-5}
& SP-net BT-net~\cite{kim2020single} & CVPR 2020 & \shortstack[l]{\space \\ $I \rightarrow [T_{1}, R_{1}]$ \\ $R_{1} \rightarrow R$} & N/A \\ \cline{2-5}
& Wan et al.~\cite{wan2020reflection} & CVPR 2020  & \shortstack[l]{\space \\ $[I, E_{I}] \rightarrow R_{1}$ \\ $R_{1} \rightarrow R$} & N/A \\ \cline{2-5}
& Zheng et al.~\cite{zheng2021single} & CVPR 2021 & \shortstack[l]{\space \\ $I \rightarrow e$ \\ $[I, e] \rightarrow T$} & Concat \\ \cline{2-5}
& Zhu et al.~\cite{zhu2024revisiting} & CVPR 2024 & \shortstack[l]{\space \\ $I \rightarrow E_{R}$ \\ $ [I, E_{R}] \rightarrow T$} & Concat \\ \cline{2-5}
& Language-Guided~\cite{zhong2024language} & CVPR 2024 & \shortstack[l]{\space \\ $[I, Texts] \rightarrow R\ or\ T$ \\ $[I, R\ or\ T] \rightarrow T\ or\ R$} & Feature-Level Concat
\\   \bottomrule   \\
Multi-stage
& BDN~\cite{yang2018seeing} & ECCV 2018 & \shortstack[l]{\space \\ $I \rightarrow T_{1}$ \\ $[I, T_{1}] \rightarrow R$ \\ $[I, R] \rightarrow T$ } & Concat \\ \cline{2-5}
& IBCLN~\cite{li2020single} & CVPR 2020 & \shortstack[l]{\space \\ $[I, R_{0}, T_{0}] \rightarrow [R_{1}, T_{1}]$ \\ $[I, R_{1}, T_{1}] \rightarrow [R_{2}, T_{2}]$ \\ \ldots} & \shortstack[l]{Concat \\ Recurrent} \\ \cline{2-5}
& Chang et al.~\cite{chang2021single} & WACV 2021 & \shortstack[l]{\space \\ $I \rightarrow E_{T}$ \\ $[I, E_{T}] \rightarrow T_{1} \rightarrow R_{1} \rightarrow T_{2}$ \\ $[I, E_{T}, T_{2}] \rightarrow R \rightarrow T$} & \shortstack[l]{Concat \\ Recurrent} \\ \cline{2-5}
& LANet~\cite{dong2021location} & ICCV 2021 & \shortstack[l]{\space \\ $[I, T_{0}] \rightarrow R_{1} \rightarrow T_{1}$ \\ $[I, T_{1}] \rightarrow R_{2} \rightarrow T_{2}$ \\ \ldots} & \shortstack[l]{Concat \\ Recurrent} \\ \cline{2-5}
& V-DESIRR~\cite{prasad2021v} & ICCV 2021 & \shortstack[l]{\space \\ $I_{1} \rightarrow T_{1}$ \\ $[I_{1}, T_{1}, I_{2}] \rightarrow T_{2}$ \\ \ldots \\ $[I_{n-1}, T_{n-1}, I_{n}] \rightarrow T$ } & \shortstack[l]{Convolutional Fusion \\ Recurrent} \\
\hline\hline
\end{tabular}
\caption{\textbf{I}, \textbf{R}, \textbf{T}, and \textbf{E} represent the \textbf{I}nput, \textbf{R}eflection, \textbf{T}ransmission, and \textbf{E}dge map, respectively. The subscripts of \textbf{T} and \textbf{R} represent intermediate process outputs. The Absorption Effect $e$ is introduced in~\cite{zheng2021single} to describe light attenuation as it passes through the glass. The output $residue$ term, proposed in~\cite{hu2023single}, is used to correct errors in the additive reconstruction of the reflection and transmission layers. Language descriptions in~\cite{zhong2024language} provide contextual information about the image layers, assisting in addressing the ill-posed nature of the reflection separation problem.}
\label{tab:1}
\end{table*}


\section{Reflection Removal Approaches}
We first present one-stage SIRR approaches in Section~\ref{sec:single-stage}, followed by two-stage and multi-stage approaches in Sections~\ref{sec:two-stage} and~\ref{sec:multi-stage}. The learning objective is introduced in Section~\ref{sec:learning-objective}, with a comparative analysis of models in Table~\ref{tab:1}.

\subsection{Single-Stage Approaches}
\label{sec:single-stage}

In the field of reflection removal, most academic approaches adopt a multi-stage architecture. However, some studies have also proposed one-stage architectures. Given an image \( I \in [0, 1]^{m \times n \times 3} \) with reflections, these approaches typically decompose \( I \) into a transmission layer \( f_T(I; \theta) \) and/or a reflection layer \( f_R(I; \theta) \), using a single network such that $I \approx f_T(I; \theta) + f_R(I; \theta)$, where \( \theta \) represents the network parameters.
ERRNet~\cite{wei2019single} and RobustSIRR~\cite{song2023robust} take \( I \) as input and output only \( T \), while Zhang et al.~\cite{zhang2018single} and YTMT~\cite{hu2021trash} take \( I \) as input and output both \( R \) and \( T \). Additionally, RobustSIRR~\cite{song2023robust} utilizes multi-resolution inputs alongside to enhance feature extraction and improve reflection removal performance.

Zhang et al.~\cite{zhang2018single} propose utilizing a deep neural network with perceptual losses to address the problem of SIRR. ERRNet~\cite{wei2019single} enhances a fundamental image reconstruction neural network by simplifying residual blocks through the elimination of batch normalization, expanding capacity by widening the network to 256 feature maps, and enriching the input image \( I \)  with hypercolumn features extracted from a pretrained VGG-19 network to incorporate semantic information for better performance. RobustSIRR~\cite{song2023robust} presents a robust transformer-based model for SIRR, integrating cross-scale attention modules, multi-scale fusion modules, and an adversarial image discriminator to improve performance. YTMT~\cite{hu2021trash} introduces a simple yet effective interactive strategy called "Your Trash is My Treasure". This approach constructs dual-stream decomposition networks by facilitating block-wise communication between the streams and transferring deactivated ReLU information from one stream to the other, leveraging the additive property of the components.


\subsection{Two-Stage Approaches}
\label{sec:two-stage}

Due to the inherent ambiguity and complexity of SIRR, solving this problem is very challenging. To address this, researchers typically use deep learning models in a cascade or sequence manner, which helps manage the uncertainty in estimating transmission layer while also simplifying the training of SIRR systems.
Some academic approaches adopt a two-stage architecture, where an intermediate output, such as a reflection layer~\cite{feng2021deep,li2023two,hu2023single,kim2020single,wan2020reflection,zhong2024language}, a coarse transmission layer~\cite{feng2021deep,hu2023single,kim2020single,zhong2024language}, an edge map~\cite{wan2019corrn,fan2017generic,zhu2024revisiting}, and so on~\cite{zheng2021single}, is first estimated, followed by the reconstruction of the final transmission layer and/or reflection layer. Besides, the techniques for fusing features across different stages vary among studies. Some methods use convolutional fusion techniques~\cite{wan2019corrn, feng2021deep, li2023two}, while others apply image-level~\cite{fan2017generic, zheng2021single, zhu2024revisiting} or feature-level concatenation~\cite{zhong2024language}. Additionally, certain approaches~\cite{hu2023single, kim2020single, wan2020reflection} do not employ any fusion strategies.

CoRRN\cite{wan2019corrn} proposes a network that uses feature sharing to tackle the problem within a cooperative framework, combining image context and multi-scale gradient information. 
DMGN~\cite{feng2021deep} presents a unified framework for background restoration, employing the Residual Deep-Masking Cell to progressively refine and control information flow. 
The RAG~\cite{li2023two} module is designed to improve the use of the estimated reflection for more accurate transmission layer prediction. 
CEILNet~\cite{fan2017generic} introduces a cascaded pipeline for edge prediction followed by image reconstruction.
DSRNet~\cite{hu2023single} architecture features two cascaded stages and a learnable residue module (LRM). Stage 1 gathers hierarchical semantic information, while Stage 2 refines the decomposition using the LRM to separate components that break the linear assumption.
The structure in~\cite{kim2020single} includes SP-net, which decomposes an input image into the predicted transmission layer and background reflection layer. The BT-net then eliminates glass and lens effects from the predicted reflection, enhancing image clarity and enabling more accurate error matching.
In this paper~\cite{wan2020reflection}, instead of eliminating reflection components from the mixed image, the goal is to recover the reflection scenes from the mixture.
This paper~\cite{zheng2021single} addresses SIRR by incorporating the absorption effect $(e)$, which is approximated using the average refractive amplitude coefficient map. It proposes a two-step solution: the first step estimates the absorption effect from the reflection-contaminated image, and the second step recovers the transmission image using both the reflection-contaminated image and the estimated absorption effect.
The framework~\cite{zhu2024revisiting} consists of RDNet and RRNet, where RDNet utilizes a pretrained backbone with residual blocks and interpolation to estimate the reflection mask, and RRNet uses this estimate to assist in the reflection removal process.
This paper~\cite{zhong2024language} addresses language-guided reflection separation by using language descriptions to provide layer content. It proposes a unified framework that uses cross-attention and contrastive learning to align language descriptions with image layers, while a gated network and randomized training strategy help resolve layer ambiguity.



\subsection{Multi-Stage Approaches}
\label{sec:multi-stage}

Some studies extend beyond a two-stage architecture by using a multi-stage cascaded structure. Similar to the two-stage design, the multi-stage approach generates intermediate outputs in a recurrent fashion, eventually reconstructing the final transmission and/or reflection layer.
Some methods use convolutional fusion techniques~\cite{prasad2021v}, while others utilize concatenation~\cite{yang2018seeing, li2020single, chang2021single, dong2021location}.

DBN~\cite{yang2018seeing} introduces a cascaded deep neural network that simultaneously estimates both background and reflection components. The network follows a bidirectional approach: first using the estimated background to predict the reflection, and then refining the background prediction using the estimated reflection. This dual-estimation strategy improves reflection removal performance.
IBCLN~\cite{li2020single} is designed for reflection removal by progressively refining the estimates of the transmission and reflection layers, with each iteration improving the prediction of the other. By utilizing LSTM to transfer information between steps and incorporating residual reconstruction loss, IBCLN tackles the vanishing gradient issue and improves training across multiple cascade steps.
The model in~\cite{chang2021single} takes a reflection-contaminated image and separates it into the reflection and transmission layers. To ensure high-quality transmission, three auxiliary techniques are employed: edge guidance, a reflection classifier, and recurrent decomposition.
This paper~\cite{dong2021location} presents a LANet for SIRR. It employs a reflection detection module that generates a probabilistic confidence map using multi-scale Laplacian features. The network, designed as a recurrent model, progressively refines reflection removal, with Laplacian kernel parameters highlighting strong reflection boundaries to improve detection and enhance the quality of the results.
V-DESIRR~\cite{prasad2021v} introduces a lightweight model for reflection removal using an innovative scale-space architecture, which processes the corrupted image in two stages: a Low Scale Sub-network (LSSNet) for the lowest scale and a Progressive Inference (PI) stage for higher scales. To minimize computational complexity, the PI stage sub-networks are significantly shallower than LSSNet, and weight sharing across scales enables the model to generalize to high resolutions without the need for retraining.



\subsection{Learning Objective}
\label{sec:learning-objective}

To train SIRR models, several commonly used loss functions are combined to ensure high-quality reflection removal. These include Reconstruction Loss, Perceptual Loss\cite{johnson2016perceptual}, and Adversarial Loss\cite{goodfellow2014generative}. Each of these loss functions contributes to different aspects of the model’s learning process:

\textbf{Reconstruction loss} is typically defined using the $L1$ or $L2$ loss, which directly measures the pixel-wise difference between the predicted reflection-free image and the ground truth image. This loss ensures that the output image is as close as possible to the desired reflection-free image in a pixel-wise sense. However, relying solely on this loss can lead to overly smooth results, as it does not consider high-level perceptual differences. The $L1$ loss formulation is as follows:

\begin{equation}
    \mathcal{L}_{\text{rec}} = \| \hat{T} - T \|_1
\end{equation}

where \( \hat{T} \) is the predicted reflection-free image and \( T \) is the ground truth image.

To mitigate the oversmoothing effect of reconstruction loss and preserve important structural details, \textbf{Gradient Consistency Loss} is utilized in~\cite{wei2019single}. This loss ensures that the predicted transmission layer \( \hat{T} \) retains the edge structures of the ground truth \( T \) by minimizing the difference between their gradients along both the \( x \)- and \( y \)-directions:

\begin{equation}
    \mathcal{L}_{\text{grad}} = \|\nabla_x \hat{T} - \nabla_x T\|_1 + \|\nabla_y \hat{T} - \nabla_y T\|_1
\end{equation}

where \( \nabla_x \) and \( \nabla_y \) are the gradient operators along the horizontal and vertical directions, respectively.

By combining these two losses, SIRR achieves a balance between accurate pixel-wise reconstruction and the preservation of structural details.



\textbf{Perceptual loss} utilizes a pre-trained deep neural network (e.g., VGG) to extract high-level feature representations of both the predicted and ground truth images. Instead of measuring pixel-wise differences, this loss compares the differences in feature space, making the generated images more visually realistic and closer to human perception. The perceptual loss can be expressed as:

\begin{equation}
    \mathcal{L}_{\text{per}} = \sum_{i} \| \phi_i(\hat{T}) - \phi_i(T) \|_1
\end{equation}

where \( \phi_i(\cdot) \) represents the feature map extracted from the \( i \)-th layer of the pre-trained network.


\textbf{Adversarial loss} is inspired by Generative Adversarial Networks (GANs) and is used to improve the realism of the generated images. A discriminator \( D \) is introduced to distinguish between real reflection-free images and the generated images. The adversarial loss is formulated as:

\begin{equation}
    \mathcal{L}_{\text{adv}} = \mathbb{E}[\log D(T)] + \mathbb{E}[\log (1 - D(\hat{T}))]
\end{equation}

where \( D(\cdot) \) represents the discriminator network. The generator aims to minimize this loss, making the generated images indistinguishable from real ones.



In addition to reconstruction loss, perceptual loss, and adversarial loss, other loss functions are also utilized to enhance reflection removal performance. One such example is the \textbf{exclusion loss}~\cite{li2023two,zhang2018single}, which encourages the separation of the transmission (\(T\)) and reflection (\(R\)) layers by minimizing their structural correlation. This loss enforces gradient decorrelation at multiple scales. It is formulated as:


\begin{equation}
    \mathcal{L}_{\text{excl}} = \frac{1}{N+1} \sum_{n=0}^{N} \sqrt{\|\Psi(T\downarrow^n, R\downarrow^n)\|_F}
\end{equation}

where \(T\downarrow^n\) and \(R\downarrow^n\) are the downsampled versions at different scales, $\|\cdot\|_F$ is the Frobenius norm, and $\Psi(T, R)$ measures the correlation between their gradients.

% \begin{equation}
%     \Psi(T, R) = \tanh(\lambda_T |\nabla T|) \circ \tanh(\lambda_R |\nabla R|)
% \end{equation}

% measures the correlation between their gradients. The Frobenius norm \(\|\cdot\|_F\) quantifies this correlation. The normalization factors are set as:

% \begin{equation}
%     \lambda_T = \frac{1}{2}, \quad \lambda_R = \frac{\|\nabla T\|_1}{\|\nabla R\|_1}.
% \end{equation}


\textbf{Total Variation Loss} is a regularization technique commonly used in image processing tasks to promote smoothness and reduce noise or artifacts. It encourages spatial continuity by minimizing the differences between neighboring pixels, preventing excessive sharp variations. TVLoss is particularly useful in reflection removal~\cite{zhu2024revisiting}, denoising, and super-resolution tasks, where it helps generate cleaner and more visually appealing results by reducing undesired texture artifacts while preserving important image structures.


\textbf{Contextual Loss} is used in~\cite{prasad2021v} to preserve fine-grained details in image generation tasks by focusing on feature similarity rather than direct pixel-wise differences. The formula for Contextual Loss is typically expressed as:

\[
\mathcal{L}_{\text{CX}}(F, F^*) = - \log \left( \max_j \ \text{CX}_{ij} \right)
\]

where \( F \) and \( F^* \) represent feature activations from a pre-trained network for the generated and target images, respectively. The contextual similarity \( \text{CX}_{ij} \) measures the correlation between feature vectors, ensuring that the generated image retains important structural patterns from the reference. CX Loss is particularly useful in reflection removal, style transfer, and image synthesis, as it helps maintain perceptual consistency while allowing flexibility in pixel arrangements.


In addition to traditional loss functions, some evaluation metrics are also directly used as loss terms. For instance, Zheng et al.~\cite{zheng2021single,wan2019corrn} directly incorporates PSNR, SSIM, and SI. PSNR ensures high-fidelity reconstruction, SSIM preserves structural similarity, and SI enhances overall structural consistency. By combining these metrics, the model improves reflection removal performance.

\section{Datasets \& Evaluation Metrics}
\label{sec:datasets}

\subsection{Data Acquisition}

The datasets for SIRR are a critical aspect of developing effective deep learning models. These datasets vary in size, image source, and the type of annotations provided. In general, they can be categorized into two main types: synthetic datasets and real-world datasets.

\subsubsection{Synthetic Datasets}
Synthetic datasets are created by simulating the reflection phenomenon using computer graphics techniques. This allows for precise control over various factors such as the intensity and blurriness of the reflection, as well as the presence of ghosting effects. Common methods for creating synthetic datasets include:

\textbf{Image Mixing:} Combining two images with different coefficients to represent the background and reflection layers.

\textbf{Reflection Blur:} Applying Gaussian blur to the reflection layer to mimic the out-of-focus effect.

\textbf{Brightness Adjustment:} Adaptively adjusting brightness and contrast to create realistic reflections.


\textbf{Physics-based Rendering:} Using physics-based methods to render reflections.

\subsubsection{Real-world Datasets}

Real-world datasets are captured using cameras in real-world environments. This provides more realistic and diverse data, but it also makes it more challenging to obtain accurate ground truth images. Common methods for creating real-world datasets include:

\textbf{Manual Glass Removal:} Capturing images with and without the glass to obtain ground truth.

\textbf{Raw Data Subtraction:} Subtracting the reflection from the mixed image in the raw data space.

\textbf{Flash/no-flash Pairs:} Capturing images with and without flash to exploit flash-only cues.

\textbf{Polarization:} Using polarization cameras to capture images with different polarization angles.

\textbf{Controlled Environments:} Capturing images in controlled environments with varying lighting, glass thickness, and camera settings.



\subsection{Current Public Datasets}

Several public datasets have been created to facilitate the development and evaluation of deep learning models for SIRR. These datasets vary in size, image source, and the type of annotations provided. Table~\ref{tab:dataset} summarizes the most important datasets for SIRR, including their usage (training, testing, or both), the number of image pairs, average resolution, and whether they collect from real or synthetic images.


\begin{table*}[t]
  \centering
  
\begin{subtable}[h]{0.62\textwidth}
\vspace{-3mm}

\resizebox{1\columnwidth}{!}{
\begin{tabular}{llllll}
\toprule
\textbf{Methods} & Photo & Art & Cartoon & Sketch & \textit{Mean} \\
 \midrule
ERM & 41.2  &   40.9 & 53.7 &46.2  &45.5  \\
Wang \etal \cite{wang2021tent} & 42.0  &  41.6 & 56.5 & 53.8 & 48.5 \\
Zhou \etal \cite{zhou2021domain} & 56.3 & 44.5 & 55.8 & 46.7 & 50.8 \\
Li \etal \cite{li2022uncertainty} & \textbf{60.2} & 45.0 & 54.4 & 49.2 & 52.2\\
Zhang \etal \cite{zhang2022exact} & 57.9 & \textbf{46.0} & 55.3 & 50.0 & 52.3 \\
 \rowcolor{lightorange}
{\textit{\textbf{This paper}}} & 50.0 \scriptsize{$\pm$0.2} & 45.7 \scriptsize{$\pm$0.3} & \textbf{61.2} \scriptsize{$\pm$0.28} & \textbf{55.5} \scriptsize{$\pm$0.5} & \textbf{53.1} \scriptsize{$\pm$0.31} \\
\bottomrule
\end{tabular}}
\caption{Output-level shift.}
\label{table:dist_shift}
\end{subtable}
\begin{subtable}[h]{0.293\textwidth}
\vspace{-3mm}

\resizebox{1\columnwidth}{!}{
\begin{tabular}{ll}
\toprule
\textbf{Methods} & \textbf{Accuracy} \\ 
\midrule
ERM & 69.3\\
Wang \etal \cite{wang2021tent} & 68.8 \\
Long \etal \cite{long2018conditional} & 70.7 \\
Ganin \etal \cite{ganin2016domain} & 72.1 \\
\rowcolor{lightorange}
\textit{\textbf{{\textit{\textbf{This paper}}}}} & \textbf{73.7}  \scriptsize{$\pm$0.7} \\ 
\bottomrule
\end{tabular}}
\caption{Feature-level shift}
\label{tab:feautre_level}
\end{subtable}
\vspace{-1em}
\caption{\textbf{Comparisons on output-level and feature-level shifts} for ResNet-18 on PACS. Our method achieves the best overall performance.}
\vspace{-4mm}
\end{table*}


Some of these datasets, such as Nature and Real, are relatively small and contain only real-world images with ground truth transmission layers. Others, such as SIR$^2$~\cite{wan2017benchmarking} and CEIL~\cite{fan2017generic}, are larger and include both synthetic and real-world images.  The SIR$^2$~\cite{wan2017benchmarking} dataset is particularly notable for its diversity, as it includes images with varying blur levels and glass thicknesses. The CEIL dataset, on the other hand, is designed to be more challenging, as it includes images with strong reflections. 

More recent datasets, such as CDR~\cite{lei2022categorized}, has been created to address the limitations of earlier datasets.  The CDR dataset is categorized according to reflection types and contains images with perfect alignment between the mixed and transmission images includes misaligned raw flash/ambient images. 

The largest and most recent dataset is RRW~\cite{zhu2024revisiting}, which contains over 14,950 high-resolution real-world reflection pairs. This dataset is particularly valuable for training deep learning models, as it provides a large number of images.

The choice of dataset depends on the specific application and the desired properties of the reflection removal algorithm. In general, larger and more diverse datasets are preferred, as they enable the training of more robust and generalizable models.

\subsection{Evaluation Metrics}

In this survey, we provide a comprehensive summary of evaluation metrics commonly employed in deep learning-based reflection removal. These metrics can be broadly classified into two categories: quantitative metrics and qualitative metrics.

\subsubsection{Quantitative Metrics}

Quantitative metrics are used to objectively measure the difference between the predicted transmission layer and the ground-truth transmission layer. Commonly used quantitative metrics include:

\begin{itemize}
\item \textbf{PSNR} (Peak Signal-to-Noise Ratio): Measures the difference between the predicted transmission layer and the ground-truth transmission layer. 
\item \textbf{SSIM} (Structural Similarity): Evaluates the similarity between the predicted and ground-truth transmission layers from three aspects: luminance, contrast, and structure. 
\item \textbf{MSE} (Mean Squared Error): Calculates the average squared difference between the predicted and ground-truth transmission layers. 
\item \textbf{Local MSE} (LMSE): Evaluates the local structure similarity by calculating the similarity of each local patch. 
\item \textbf{Normalized Cross Correlation} (NCC): Measures the correlation between the predicted and ground-truth transmission layers after normalizing their overall intensity. 
\item \textbf{Structure Index} (SI): Evaluates the structural similarity between the predicted and ground-truth transmission layers based on their covariance and variance. 
\end{itemize}

\subsubsection{Qualitative Metrics}

Qualitative metrics, on the other hand, are used to subjectively evaluate the visual quality of the predicted transmission images. One common qualitative metric is the \textbf{perceptual user study} \cite{zhang2018single}, where human users compare the predicted transmission images with the ground-truth images and rate their quality.

The choice of evaluation metrics depends on the specific application and the desired properties of the reflection removal algorithm. In general, a combination of quantitative and qualitative metrics is used to provide a comprehensive evaluation of the algorithm's performance.

\section{Discussion}
\label{sec:discussion}
\subsection{Challenges in current SIRR research}
One of the biggest challenges in SIRR research is the lack of large, high-quality training datasets that represent a variety of reflection types across different surfaces and lighting conditions. Reflection removal relies on supervised learning, which requires a well-labeled dataset with clear ground-truth images for training. However, creating or collecting such datasets is both time- and labor-consuming. Moreover, the absence of suitable test sets with real-world reflection scenarios presents another challenge. These test sets should include not only high-quality images but also a wide range of reflective surfaces, lighting conditions, and material properties (e.g., building glass, car window, smooth metal surface, rough metal surface, etc.). Without such comprehensive datasets, model evaluation remains limited and often unreliable when deploying into the real world.

Furthermore, the lack of datasets in SIRR research has also made exploring complex network architectures less valuable. With limited data, simpler models like UNet is already achieving state-of-the-art results, making the development of more complex models unnecessary and slowing progress in the evolution of SIRR network designs. These intricate architectures are prone to overfitting on small datasets, preventing them from reaching their full potential. Consequently, academic research in this field is stagnating, resulting fewer innovations in network design for SIRR.

In addition, the inherent complexity of the SIRR task itself compounds these challenges. Reflections vary in type, intensity, and interaction with the scene—some completely obscure the background, turning the task into a form of image inpainting. Such a wide variety makes it difficult to clearly define what a SIRR task should involve: is it purely about removing reflections, or does it also need to reconstruct missing background details? The lack of a comprehensive task definition hinders the development of consistent methodologies, reliable evaluation metrics, and meaningful comparisons. To make real progress, the academic community must come up with a clearer and more inclusive task definition that better captures the complexity of reflection removal and develop specific guidelines for handling different reflection scenarios.

\subsection{Future Directions}
One of the most pressing future directions in SIRR research is the creation of large-scale, high-quality datasets that cover a wide variety of reflection types, surfaces, and lighting conditions. Efforts should also focus on curating test sets that contain not only clean ground-truth images without reflections, but also cover real-world reflective materials (e.g., different types of glass and metals) and various environmental factors (e.g., lighting variations and reflections in challenging environments). Such test sets will enable more accurate training and better evaluation metrics, ultimately improving the generalization capabilities of SIRR models. To address the data scarcity issues, we appeal for collaborations between research institutions, industry, and the development of more powerful synthetic data generation methods.

On the other hand, future SIRR development could greatly benefit from the integration of advanced AIGC models. By utilizing large vision foundation models, researchers can enhance scene understanding and improve semantic reasoning, while large language models can provide deeper, descriptive insights into scene content and the relationships between reflections and transmissions.


Additionally, combining multimodal information (e.g., text descriptions, depth information, and semantic segmentation) will strengthen the reflection-transmission separation by leveraging complementary insights from multimodal resources. We believe that this fusion could result in more context-aware and precise reflection removal systems.

As mentioned earlier, clarifying the definition of SIRR task is another critical direction. The field needs to precisely define whether the goal is limited to reflection removal or extends to background reconstruction in cases of severe reflections. Establishing these clear boundaries will enable standardized evaluations and fair comparisons, ultimately leading to the development of more effective solutions.

\subsection{Limitations}
Our work has several limitations. First, due to the keywords and databases chosen for the search query, some related research may have been missed. In addition, research that is not published in English, exceeds the prescribed time frame, or does not provide enough technical information, were not included. Nevertheless, our paper provides a comprehensive analysis of the existing literature, based on a sample of 28 papers sourced from key venues. The aim of our review is to provide readers with a clear and rapid understanding of the key developments in SIRR field, including its current state, challenges, and future directions. Second, we did not include benchmark results in this paper, as different methods and datasets often have their own unique training strategies and evaluation process. To address this, we will develop a unified evaluation framework in the future, which will provide a fair platform to compare all publicly available datasets.


% \section{Discussion and Future Work}

% %Onscad is insample data cause these LLMS have seen openscad

% The decision space of language design is enormous, so we had to make some decisions about what to explore in the language design of AIDL. In particular, we did not build a new constraint system from scratch and instead developed ours based on an open-source constraint solver. This limited the types of primitives we allow, e.g. ellipses are not currently supported. \jz{Additionally, rectangles in AIDL are constrained to be axis-aligned by default because we found that in most use cases, a rectangle being rotated by the solver was unintuitive, and we included a parameter in the language allows rectangles to be marked as rotatable. While this feature was included in the prompts to the LLM, it was never used by the model. We hope to explore prompt-engineering techniques to rectify this issue in the future. Similarly, we hope to reduce the frequency of solver errors by providing better prompts for explaining the available constraints.} \adriana{Add two other limitations to this paragraph: that we typically noticed that things are axis alignment, say why we use this as default and in the future could try to get the gpt to not use default more often. Mention that we still have Solver failures that could be addressed by better engineering in future. }

% In testing our front-end, we observed that repeated instances of feedback tends to reduce the complexity of models as the LLM would frequently address the errors by removing the offending entity. This leads to unnecessarily removed details. More extensive prompt-engineering could be employed in future work to encourage the LLM to more frequently modify, rather than remove, to fix these errors. \adriana{no idea what this paragraph is trying to say}


% \adriana{This seems  like a future work paragraph so maybe start by saying that in the future you could do other front end or fine tune a model with aidl, we just tested the few shot.  } \jz{In the future, we hope to improve our front-end generation pipeline by finetuning a pretrained LLM on example AIDL programs.} In addition, multi-modal vision-langauge model development has exploded in recent months. Visual modalities are an obvious fit for CAD modeling -- in fact, most procedural CAD models are produced in visual editors -- but we decided not to explore visual inputs yet based on reports ([PH] cite OPENAIs own GPT4V paper) that current vision-language models suffter from the same spatial reasoning issues as purely textual models do (identifying relative positions like above, left of, etc.). This also informed our decision to omit spline curves which are difficult to describe in natural language. This deficit is being addressed by the development of new spatial reasoning datasets ([PH] cite visual math reasoning paper), so allowing visual user input as well as visual feedback in future work with the next generation of models seems promising.

% The decision space of language design is enormous, so it was impossible to explore it all here. We had to make some decisions about what to explore, guided by experience, conjecture, technical limitations, and anecdotal experience. Since we primarily explore the interaction between language design and language models in order to overcome the shortcomings in the latter, we did not wish to focus effort on building new constraint systems. This led us to use an open-source constraint solver to build our solver off of. This limited the types of primitives we allow; in particular, most commercial geometric solvers also support ellipses.

% In testing our generation frontend, we observed that repeated instances of feedback tended to reduce the complexity of models as the LLM would frequently address the errors by removing the offending entity. This is a fine strategy for over-constrained systems, but can unnecessarily remove detail when done in response to a syntax or validation error. More extensive prompt-engineering could be employed to encourage the LLM to more frequently modify, rather than remove, to fix these errors


% In recent months, multi-modal vision-language model development has exploded. Visual modalities are an obvious fit for CAD modeling -- in fact, most procedural CAD models are produced in visual editors -- but we decided not to explore visual input yet based on reports (cite OpenAIs own GPT4V paper) that current vision-language models suffer from the same spatial reasoning issues as purely textual models do (identifying relative positions like above, left of, etc.). This also informed our decision to omit spline curves; they are not easily described in natural language. This deficit is being addressed by the development of new spatial reasoning datasets (cite visual math reasoning paper), so allowing visual user input as well as visual feedback in future work with the next generation of models seems promising.



\section{Conclusion}

AIDL is an experiment in a new way of building graphics systems for language models; what if, instead of tuning a model for a graphics system, we build a graphics system tailored for language models? By taking this approach, we are able to draw on the rich literature of programming languages, crafting a language that supports language-based dependency reasoning through semantically meaningful references, separation of concerns with a modular, hierarchical structure, and that compliments the shortcomings of LLMs with a solver assistance. Taking this neurosymbolic, procedural approach allows our system to tap into the general knowledge of LLMs as well as being more applicable to CAD by promoting precision, accuracy, and editability. Framing AI CAD generation as a language design problem is a complementary approach to model training and prompt engineering, and we are excited to see how advance in these fields will synergize with AIDL and its successors, especially as the capabilities of multi-modal vision-language models improve. AI-driven, procedural design coming to CAD, and AIDL provides a template for that future.

% Using procedural generation instead of direct geometric generation enables greater editability, accuracy, and precision
% Using a general language model allows for generalizability beyond existing CAD datasets and control via common language.
% Approaches code gen in LLMs through language design rather than training the model or constructing complexing querying algorithms. This could be a complimentary approach
% Embedding as a DSL in a popular language allows us to leverage the LLMs syntactic knowledge while exploiting our domain knowledge in the language design
% LLM-CAD languages should hierarchical, semantic, support constraints and dependencies




%In this paper, we proposed AIDL, a language designed specifically for LLM-driven CAD design. The AIDL language simultaneously supports 1) references to constructed geometry (\dgone{}), 2) geometric constraints between components (\dgtwo{}), 3) naturally named operators (\dgthree{}), and 4) first-class hierarchical design (\dgfour{}), while none of the existing languages supports all the above. These novel designs in AIDL allow users to tap into LLMs' knowledge about objects and their compositionalities and generate complex geometry in a hierarchical and constrained fashion. Specifically, the solver for AIDL supports iterative editing by the LLM by providing intermediate feedback, and remedies the LLM's weakness of providing explicit positions for geometries.

%\adriana{This seems  like a future work paragraph so maybe start by saying that in the future you could do other front end or fine tune a model with aidl, we just tested the few shot.  }
%\paragraph{Future work} In recent months, multi-modal vision-language model development has exploded. Visual modalities are an obvious fit for CAD modeling -- in fact, most procedural CAD models are produced in visual editors -- but we decided not to explore visual input yet based on reports (cite OpenAIs own GPT4V paper) that current vision-language models suffer from the same spatial reasoning issues as purely textual models do (identifying relative positions like above, left of, etc.). This also informed our decision to omit spline curves; they are not easily described in natural language. This deficit is being addressed by the development of new spatial reasoning datasets (cite visual math reasoning paper), so allowing visual user input as well as visual feedback in future work with the next generation of models seems promising. 

% \section{Introduction}
% \label{sec:intro}

% These guidelines include complete descriptions of the fonts, spacing, and
% related information for producing your proceedings manuscripts. Please follow
% them and if you have any questions, direct them to Conference Management
% Services, Inc.: Phone +1-979-846-6800 or email
% to \\\texttt{icip2022@cmsworkshops.com}.

% \section{Formatting your paper}
% \label{sec:format}

% All printed material, including text, illustrations, and charts, must be kept
% within a print area of 7 inches (178 mm) wide by 9 inches (229 mm) high. Do
% not write or print anything outside the print area. The top margin must be 1
% inch (25 mm), except for the title page, and the left margin must be 0.75 inch
% (19 mm).  All {\it text} must be in a two-column format. Columns are to be 3.39
% inches (86 mm) wide, with a 0.24 inch (6 mm) space between them. Text must be
% fully justified.

% \section{PAGE TITLE SECTION}
% \label{sec:pagestyle}

% The paper title (on the first page) should begin 1.38 inches (35 mm) from the
% top edge of the page, centered, completely capitalized, and in Times 14-point,
% boldface type.  The authors' name(s) and affiliation(s) appear below the title
% in capital and lower case letters.  Papers with multiple authors and
% affiliations may require two or more lines for this information. Please note
% that papers should not be submitted blind; include the authors' names on the
% PDF.

% \section{TYPE-STYLE AND FONTS}
% \label{sec:typestyle}

% To achieve the best rendering both in printed proceedings and electronic proceedings, we
% strongly encourage you to use Times-Roman font.  In addition, this will give
% the proceedings a more uniform look.  Use a font that is no smaller than nine
% point type throughout the paper, including figure captions.

% In nine point type font, capital letters are 2 mm high.  {\bf If you use the
% smallest point size, there should be no more than 3.2 lines/cm (8 lines/inch)
% vertically.}  This is a minimum spacing; 2.75 lines/cm (7 lines/inch) will make
% the paper much more readable.  Larger type sizes require correspondingly larger
% vertical spacing.  Please do not double-space your paper.  TrueType or
% Postscript Type 1 fonts are preferred.

% The first paragraph in each section should not be indented, but all the
% following paragraphs within the section should be indented as these paragraphs
% demonstrate.

% \section{MAJOR HEADINGS}
% \label{sec:majhead}

% Major headings, for example, "1. Introduction", should appear in all capital
% letters, bold face if possible, centered in the column, with one blank line
% before, and one blank line after. Use a period (".") after the heading number,
% not a colon.

% \subsection{Subheadings}
% \label{ssec:subhead}

% Subheadings should appear in lower case (initial word capitalized) in
% boldface.  They should start at the left margin on a separate line.
 
% \subsubsection{Sub-subheadings}
% \label{sssec:subsubhead}

% Sub-subheadings, as in this paragraph, are discouraged. However, if you
% must use them, they should appear in lower case (initial word
% capitalized) and start at the left margin on a separate line, with paragraph
% text beginning on the following line.  They should be in italics.

% \section{PRINTING YOUR PAPER}
% \label{sec:print}

% Print your properly formatted text on high-quality, 8.5 x 11-inch white printer
% paper. A4 paper is also acceptable, but please leave the extra 0.5 inch (12 mm)
% empty at the BOTTOM of the page and follow the top and left margins as
% specified.  If the last page of your paper is only partially filled, arrange
% the columns so that they are evenly balanced if possible, rather than having
% one long column.

% In LaTeX, to start a new column (but not a new page) and help balance the
% last-page column lengths, you can use the command ``$\backslash$pagebreak'' as
% demonstrated on this page (see the LaTeX source below).

% \section{PAGE NUMBERING}
% \label{sec:page}

% Please do {\bf not} paginate your paper.  Page numbers, session numbers, and
% conference identification will be inserted when the paper is included in the
% proceedings.

% \section{ILLUSTRATIONS, GRAPHS, AND PHOTOGRAPHS}
% \label{sec:illust}

% Illustrations must appear within the designated margins.  They may span the two
% columns.  If possible, position illustrations at the top of columns, rather
% than in the middle or at the bottom.  Caption and number every illustration.
% All halftone illustrations must be clear black and white prints.  Colors may be
% used, but they should be selected so as to be readable when printed on a
% black-only printer.

% Since there are many ways, often incompatible, of including images (e.g., with
% experimental results) in a LaTeX document, below is an example of how to do
% this \cite{Lamp86}.

% \section{FOOTNOTES}
% \label{sec:foot}

% Use footnotes sparingly (or not at all!) and place them at the bottom of the
% column on the page on which they are referenced. Use Times 9-point type,
% single-spaced. To help your readers, avoid using footnotes altogether and
% include necessary peripheral observations in the text (within parentheses, if
% you prefer, as in this sentence).

% % Below is an example of how to insert images. Delete the ``\vspace'' line,
% % uncomment the preceding line ``\centerline...'' and replace ``imageX.ps''
% % with a suitable PostScript file name.
% % -------------------------------------------------------------------------
% \begin{figure}[htb]

% \begin{minipage}[b]{1.0\linewidth}
%   \centering
%   \centerline{\includegraphics[width=8.5cm]{image1}}
% %  \vspace{2.0cm}
%   \centerline{(a) Result 1}\medskip
% \end{minipage}
% %
% \begin{minipage}[b]{.48\linewidth}
%   \centering
%   \centerline{\includegraphics[width=4.0cm]{image3}}
% %  \vspace{1.5cm}
%   \centerline{(b) Results 3}\medskip
% \end{minipage}
% \hfill
% \begin{minipage}[b]{0.48\linewidth}
%   \centering
%   \centerline{\includegraphics[width=4.0cm]{image4}}
% %  \vspace{1.5cm}
%   \centerline{(c) Result 4}\medskip
% \end{minipage}
% %
% \caption{Example of placing a figure with experimental results.}
% \label{fig:res}
% %
% \end{figure}


% % To start a new column (but not a new page) and help balance the last-page
% % column length use \vfill\pagebreak.
% % -------------------------------------------------------------------------
% %\vfill
% %\pagebreak
% \section{COPYRIGHT FORMS}
% \label{sec:copyright}

% You must submit your fully completed, signed IEEE electronic copyright release
% form when you submit your paper. We {\bf must} have this form before your paper
% can be published in the proceedings.

% \section{RELATION TO PRIOR WORK}
% \label{sec:prior}

% The text of the paper should contain discussions on how the paper's
% contributions are related to prior work in the field. It is important
% to put new work in  context, to give credit to foundational work, and
% to provide details associated with the previous work that have appeared
% in the literature. This discussion may be a separate, numbered section
% or it may appear elsewhere in the body of the manuscript, but it must
% be present.

% You should differentiate what is new and how your work expands on
% or takes a different path from the prior studies. An example might
% read something to the effect: "The work presented here has focused
% on the formulation of the ABC algorithm, which takes advantage of
% non-uniform time-frequency domain analysis of data. The work by
% Smith and Cohen \cite{Lamp86} considers only fixed time-domain analysis and
% the work by Jones et al \cite{C2} takes a different approach based on
% fixed frequency partitioning. While the present study is related
% to recent approaches in time-frequency analysis [3-5], it capitalizes
% on a new feature space, which was not considered in these earlier
% studies."

\vfill\pagebreak


% \section{REFERENCES}
% \label{sec:refs}

% List and number all bibliographical references at the end of the
% paper. The references can be numbered in alphabetic order or in
% order of appearance in the document. When referring to them in
% the text, type the corresponding reference number in square
% brackets as shown at the end of this sentence \cite{C2}. An
% additional final page (the fifth page, in most cases) is
% allowed, but must contain only references to the prior
% literature.

% References should be produced using the bibtex program from suitable
% BiBTeX files (here: strings, refs, manuals). The IEEEbib.bst bibliography
% style file from IEEE produces unsorted bibliography list.
% -------------------------------------------------------------------------
\let\oldthebibliography=\thebibliography
\let\endoldthebibliography=\endthebibliography
\renewenvironment{thebibliography}[1]{%
   \begin{oldthebibliography}{#1}%
     \setlength{\itemsep}{-.3ex}%
}%
{%
   \end{oldthebibliography}%
}

{
\footnotesize
\bibliographystyle{IEEEbib}
% \bibliographystyle{IEEEtran}
% \bibliographystyle{jabbrv_unsrt}
% \bibliography{strings,refs}
\bibliography{refs}
}
\end{document}
