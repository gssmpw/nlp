\section{Discussion}
\label{sec:discussion}
\subsection{Challenges in current SIRR research}
One of the biggest challenges in SIRR research is the lack of large, high-quality training datasets that represent a variety of reflection types across different surfaces and lighting conditions. Reflection removal relies on supervised learning, which requires a well-labeled dataset with clear ground-truth images for training. However, creating or collecting such datasets is both time- and labor-consuming. Moreover, the absence of suitable test sets with real-world reflection scenarios presents another challenge. These test sets should include not only high-quality images but also a wide range of reflective surfaces, lighting conditions, and material properties (e.g., building glass, car window, smooth metal surface, rough metal surface, etc.). Without such comprehensive datasets, model evaluation remains limited and often unreliable when deploying into the real world.

Furthermore, the lack of datasets in SIRR research has also made exploring complex network architectures less valuable. With limited data, simpler models like UNet is already achieving state-of-the-art results, making the development of more complex models unnecessary and slowing progress in the evolution of SIRR network designs. These intricate architectures are prone to overfitting on small datasets, preventing them from reaching their full potential. Consequently, academic research in this field is stagnating, resulting fewer innovations in network design for SIRR.

In addition, the inherent complexity of the SIRR task itself compounds these challenges. Reflections vary in type, intensity, and interaction with the scene—some completely obscure the background, turning the task into a form of image inpainting. Such a wide variety makes it difficult to clearly define what a SIRR task should involve: is it purely about removing reflections, or does it also need to reconstruct missing background details? The lack of a comprehensive task definition hinders the development of consistent methodologies, reliable evaluation metrics, and meaningful comparisons. To make real progress, the academic community must come up with a clearer and more inclusive task definition that better captures the complexity of reflection removal and develop specific guidelines for handling different reflection scenarios.

\subsection{Future Directions}
One of the most pressing future directions in SIRR research is the creation of large-scale, high-quality datasets that cover a wide variety of reflection types, surfaces, and lighting conditions. Efforts should also focus on curating test sets that contain not only clean ground-truth images without reflections, but also cover real-world reflective materials (e.g., different types of glass and metals) and various environmental factors (e.g., lighting variations and reflections in challenging environments). Such test sets will enable more accurate training and better evaluation metrics, ultimately improving the generalization capabilities of SIRR models. To address the data scarcity issues, we appeal for collaborations between research institutions, industry, and the development of more powerful synthetic data generation methods.

On the other hand, future SIRR development could greatly benefit from the integration of advanced AIGC models. By utilizing large vision foundation models, researchers can enhance scene understanding and improve semantic reasoning, while large language models can provide deeper, descriptive insights into scene content and the relationships between reflections and transmissions.


Additionally, combining multimodal information (e.g., text descriptions, depth information, and semantic segmentation) will strengthen the reflection-transmission separation by leveraging complementary insights from multimodal resources. We believe that this fusion could result in more context-aware and precise reflection removal systems.

As mentioned earlier, clarifying the definition of SIRR task is another critical direction. The field needs to precisely define whether the goal is limited to reflection removal or extends to background reconstruction in cases of severe reflections. Establishing these clear boundaries will enable standardized evaluations and fair comparisons, ultimately leading to the development of more effective solutions.

\subsection{Limitations}
Our work has several limitations. First, due to the keywords and databases chosen for the search query, some related research may have been missed. In addition, research that is not published in English, exceeds the prescribed time frame, or does not provide enough technical information, were not included. Nevertheless, our paper provides a comprehensive analysis of the existing literature, based on a sample of 28 papers sourced from key venues. The aim of our review is to provide readers with a clear and rapid understanding of the key developments in SIRR field, including its current state, challenges, and future directions. Second, we did not include benchmark results in this paper, as different methods and datasets often have their own unique training strategies and evaluation process. To address this, we will develop a unified evaluation framework in the future, which will provide a fair platform to compare all publicly available datasets.

