%\documentclass[letterpaper,10pt]{article} 
%\documentclass[letter,10pt]{article} 
\documentclass[article,12pt]{article} 
%\documentclass{article}


%The most common document-classes in LaTex are:
%%%book
%%%report
%%%article
%%%letter

\usepackage{amscd,amsmath,amssymb,amsfonts,latexsym,mathrsfs,amsthm,mathtools}
\usepackage{graphicx} 
\usepackage{pifont}
\usepackage{setspace} 
\usepackage{graphics,epsfig}
\graphicspath{{figures/}} %Setting the graphicspath
\usepackage{sectsty}
\usepackage{natbib}
\usepackage[english]{babel}
\usepackage{bm}
\usepackage{mathrsfs}
\usepackage{subfigure}
%\usepackage{subcaption}
\usepackage[usenames]{color}
%\usepackage{showlabels}
\usepackage{rotating}
\usepackage{url}
\usepackage{bbm}
\usepackage{float}
\usepackage{flushend}
\usepackage{verbatim}
\usepackage{tabularx}
\usepackage{multirow} 
%\usepackage{tabu}
\usepackage{arydshln}
\usepackage{hyperref}
\usepackage[toc,page]{appendix}
\usepackage{bigints}
\usepackage{diagbox}
%\usepackage{varwidth}
%\usepackage{fbox}
\usepackage{cancel}
\usepackage[normalem]{ulem}


\newtheorem{theo}{Theorem}
\newtheorem{defi}{Definition}
\newtheorem{coro}{Corollary}
\newtheorem{prop}{Proposition}[section]
%\theoremstyle{remark}
%\newtheorem*{rem}{{\bf Remark}}
\newtheorem{Ex}{{\bf Example}}
\newtheorem{rem}{{\bf Remark}}
\newtheorem{lemma}{Lemma}
\newtheorem{proper}{Property}[section]
\newtheorem{algo}{Algorithm}




\newcommand{\R}{\mathbb{R}}
\newcommand{\E}{\mathbb{E}}
\newcommand{\Var}{\mathrm{Var}}
\newcommand{\Cov}{\mathrm{Cov}}

%\newcommand{\x}{{\bf x}}
\newcommand{\x}{{\bm \theta}}

\newcommand{\z}{{\bf z}}
\newcommand{\X}{{\bf X}}
\newcommand{\y}{{\bf y}}
\newcommand{\Y}{{\bf Y}}
\newcommand{\post}{\bar{\pi}}

\newcommand{\s}{{\bf s}}

\newcommand{\norm}[1]{\left\lVert#1\right\rVert}

\UseRawInputEncoding


 %Margins
\textwidth 17.5cm                 % Text width without margins 
\textheight 22cm                % Text height without margins 
\evensidemargin 0cm             % Even margin 4 cm - x cm 
\oddsidemargin -0.8cm       % Odd margin 4cm - x cm
\topmargin -1cm


\tolerance=10000
\pretolerance=10000


% Title.
% ------
\title{Optimality in importance sampling: a gentle survey}

\author{F. Llorente$^\dagger$, L. Martino$^{\star}$, \\
{\small$^\dagger$  Stony Brook University, New York, USA.}\\
{\small$^\star$  Universit{\'a} degli Studi di Catania,  Italia.} \\
}

%\author{Luca Martino\thanks{Universidad Carlos III (Spain). E-mail: {\tt luca@tsc.uc3m.es}} }
%%%%%%\title{Effective Sample Size functions for Importance Sampling based on the Discrepancy} 

\date{}
 
\begin{document}



%
\maketitle
%


%\thispagestyle{empty}


%



\begin{abstract}
The performance of the Monte Carlo sampling methods relies on the crucial choice of a proposal density.
The notion of optimality is fundamental to design suitable adaptive procedures of the proposal density within Monte Carlo schemes. This work is an exhaustive review around the concept of optimality in importance sampling. Several frameworks are described and analyzed, such as the marginal likelihood approximation for model selection, the use of multiple proposal densities, a sequence of tempered posteriors, and noisy scenarios including the applications to approximate Bayesian computation (ABC) and reinforcement learning, to name a few. Some theoretical and empirical comparisons are also provided.
\newline
\newline
%\citep{Djuric03,Gordon93,Martino15PF}. It has been widely accepted as a valid measure of effective sample size. 
{ \bf Keywords:} 
Importance sampling; adaptive Monte Carlo methods; Bayesian inference; optimal proposal density.
\end{abstract} 



\section{Introduction}

Monte Carlo (MC) methods are  powerful tools for numerical inference and optimization widely employed in statistics, signal processing and machine learning \cite{Liu04b,Robert04}. They are mainly used for computing approximately   the solution of definite integrals, and by extension, of differential equations (for this reason, MC schemes can be considered stochastic quadrature rules). Although exact analytical solutions to integrals are always desirable, such �unicorns� are rarely available, specially in real-world systems. Many applications inevitably require the approximation of intractable integrals. Specifically, Bayesian methods need the computation of expectations with respect to posterior  probability density function (pdf) which, generally, are analytically intractable \cite{gelman2013bayesian}. The MC methods can be divided in four main families: direct methods (based on transformations or random variables), accept-reject techniques,  Markov chain Monte Carlo (MCMC) algorithms, and importance sampling (IS)  schemes \cite{LuengoMartino2020,martino2018independent}. The last two families are the most popular for the facility and universality of their possible application \cite{Liang10,Liu04b,Robert04}.
\newline
\newline
All the MC methods require the choice of a suitable proposal density that is crucial for their performance  \cite{LuengoMartino2020,Robert04}. For this reason, adaptive strategies that update the proposal density are often employed \cite{Bugallo15,bugallo2017adaptive,Cappe04,Liang10}. In order to design a suitable adaptation procedure, the notion of optimal proposal density (at  least associated to a specific task)  is required. For instance, let us consider a parametric proposal family of densities $q_{{\bm \xi}}(\x)$ (where ${\bm \xi}$ is a parameter vector), that is used in a MC scheme for approximating a specific integral. 
 One idea could be minimizing a divergence $\mbox{D}(q_\text{opt},q_{{\bm \xi}})$ between the optimal proposal for the specific MC scheme and integral to compute, $q_\text{opt}$, and the parametric proposal $q_{{\bm \xi}}$ \cite{akyildiz2024global,Akyildiz2021,Dieng_NIPS2017,perello2023adaptively}. Hence, the optimal parameter vector would be
$$
\widehat{{\bm \xi}}=\arg\min_{{\bm \xi}} J({\bm \xi})= \arg\min_{{\bm \xi}}\mbox{D}(q_\text{opt}(\x), q_{{\bm \xi}}(\x)).
$$
However, in order to minimize $J({\bm \xi})$, it is essential the knowledge of $q_\text{opt}(\x)$ for the specific that task we desire to solve.  Note that the parametric family of $q_{{\bm \xi}}(\x)$ must be chosen such that: {\bf (a)} we are  able to draw samples  from $q_{{\bm \xi}}(\x)$ and {\bf (b)} we are evaluate point-wise  $q_{{\bm \xi}}(\x)$, for each possible value of $\x$ and the parameter vector ${\bm \xi}$ \cite{akyildiz2024global,perello2023adaptively}.  A particular suitable divergence for importance sampling is the chi-squared divergence, a.k.a., Pearson divergence \cite{Agapiou17,Akyildiz2021,CHEN2005,Dieng_NIPS2017}.
% Importance sampling is
% For this reason, several adaptive importance sampling (AIS) schemes have been proposed in the literature.}
\newline
\newline
In this work, we focus on the IS class of methods. It is important to remark that an IS scheme employing a proposal density close to the optimal one (with respect to the specific framework of application) is able to outperform the ideal Monte Carlo technique. This is the reason why the IS approaches are also known as variance reduction methods \cite{Arouna2004,Lapeyre2011,owen2013monte}.
 We address several frameworks of practical interest and provide the corresponding optimal proposal density $q_\text{opt}$ \cite{rainforth2020target,gelman1998simulating,meng2002warp,Owen00}.  For this purpose, we have the opportunity to review numerous IS schemes proposed in the literature during the last years, describing also several related properties and results. We consider the use of a unique or multiple proposal densities for approximating an integral \cite{CORNUET12,ElviraMIS15,Owen00,Veach95}. Moreover, we consider the joint approximation 
of several integrals in different contexts \cite{Llorente2023_TBI,rainforth2020target}, including a sequence of tempered posteriors \cite{Locatelli00,neal1996sampling,Neal01}. The specific scenario of the approximation of the marginal likelihood for model selection purpose is also discussed \cite{llorenteREV_ML,gelman1998simulating,meng2002warp}. 
The noisy framework (when the evaluation of the posterior is a random variable itself), which includes the reinforcement learning and approximate Bayesian computation (ABC) as special cases, is also addressed \cite{DenizNoisy,LLORENTEnoisyIS,LlorenteABC_RF,newton1994approximate}. In this sense, this work can be considered an exhaustive survey around the concept of optimality in importance sampling. The range of applications is wider than only the Monte Carlo world: recently, related notions of optimality have also acquired a relevant place in other fields such as {\it contrastive learning}, that has been proved to have a close theoretical development to importance sampling    
 \cite{chehab2023,gutmann12a,Gutman_2019}.
 
 













\setcounter{tocdepth}{3}
\tableofcontents







%%%%%%%%%%%%%%%%%%%%%%%%%%%%%%%%%%%%%
\section{Problem statement and main notation}\label{ProbStatSect}
%%%%%%%%%%%%%%%%%%%%%%%%%%%%%%%%%%%%%

%%%%%%%%%%%%%%%%%%
\subsection{Bayesian inference}
%%%%%%%%%%%%%%%%%%

In Bayesian inference, the goal is to extract information from the posterior density $\post(\x)=p(\x|{\bf y})$ of a parameter vector $\x = [\theta_1,\dots,\theta_{D}]^\top\subset  \Theta$ given the data ${\bf y}\in \mathbb{R}^{D_y}$, i.e.,   
\begin{equation}
\post(\x)=p(\x|{\bf y})=\frac{\ell({\bf y}|\x) g(\x)}{p({\bf y})},
\end{equation}
where $\ell({\bf y}|\x)$ is the likelihood function, $g(\x)$ is the prior density, and 
\begin{equation}\label{MargLike}
Z=p({\bf y})= \int_\Theta \ell({\bf y}|\x) g(\x) d\x= \int_\Theta \pi(\x) d\x,
\end{equation}
represents the marginal likelihood (a.k.a., Bayesian evidence) \cite{Liu04b,owen2013monte,Robert04}. Moreover, above we have defined the unnormalized posterior 
$$
\pi(\x)= \ell({\bf y}|\x) g(\x),
$$
 i.e., $\pi(\x)\propto \post(\x)$ and $ \post(\x)=\frac{1}{Z}\pi(\x)$. 
The marginal likelihood $Z=p({\bf y})$, which plays the role of a normalizing constant, is particularly important for model selection purposes \cite{llorenteREV_ML}. 


%%%%%%%%%%%%%%%%%%%%%%%%%%%%%%%%%%
\subsection{Integrals of interest}
%%%%%%%%%%%%%%%%%%%%%%%%%%%%%%%%%%%

In order to extract  information about the posterior $\post(\x)$, often we are interested in computing integrals which generally involve the product of a generic function $f$ and the posterior $\post$. We can distinguish four cases depending on the possible vectorial nature of $f$ and/or $\post$. We also highlight the corresponding application frameworks.
\newline
\newline
{\bf Case 1: both scalar functions.} In this scenario we have an integral of form
\begin{align}\label{eq_I_of_interest0}
I = \int_\Theta  f(\x)\post(\x)d\x=\frac{1}{Z}\int_\Theta f(\x)\pi(\x)d\x,
\end{align}
where $\post(\x)=\frac{1}{Z} \pi(\x)$ is a pdf with support $\Theta$, and $f(\x): \Theta \to \mathbb{R}$  is a generic integrable function. The  function $f(\x)$ defines the specific expectation with respect to the posterior, that we are interested in computing. We also desire to calculate the normalizing constant of $\pi(\x)$, i.e., 
\begin{align}\label{eq_Z_of_int}
	Z = \int_\Theta\pi(\x)d\x.
\end{align}
{\bf Case 2: vectorial function ${\bf f}(\x)$.} Moreover, since $\x$ is a multidimensional vector, in many other cases we need to consider a vectorial function ${\bf f}(\x)=[f_1(\x),...,f_P(\x)]^{\top}: \Theta \to \mathbb{R}^{P}$ with $P\geq 1$. For instance,  in order to express a moment of order $\alpha$ of a random variable with density $\post(\x)$, for instance, we could ${\bf f}(\x)=\x^\alpha$ (where $P=D$). In these scenarios, we have a multidimensional integral of interest,
\begin{align}\label{eq_I_of_interest}
{\bf I} = \int_\Theta {\bf f}(\x)\post(\x)d\x=\frac{1}{Z}\int_\Theta {\bf f}(\x)\pi(\x)d\x,
\end{align}
 %moreover ${\bf f}(\x)=[f_1(\x),...,f_P(\x)]^{\top}: \Theta \to \mathbb{R}^{P}$ is a generic integrable function on $\Theta$ with $P\geq 1$ (scalar or vector-valued), and 
 with
$$
{\bf I}=[I_1,...,I_p,...,I_P]^{\top}, \quad \mbox{ where } \quad  I_p=\int_\Theta f_p(\x)\post(\x)d\x.
$$
 For instance, setting ${\bf f}(\x)=\x$, the integral ${\bf I}$  represents the expected value of the r.v. $\x \sim \post(\x)$, that is also known as  minimum mean square error (MMSE) estimator. An interesting special case is addressed in Section \ref{FantSectionLuca},
 where $P=2$ and the estimation of the vector ${\bf I} = [I_1,I_2=Z]^\top=[I,Z]^\top$, i.e.,  $f_1(\x)=f(\x)$ and $f_2(\x)=Z$.\footnote{Note that if $f(\x)=Z$ then $I=\frac{1}{Z}\int_\Theta Z\pi(\x)d\x=\int_\Theta \pi(\x)d\x=Z$.} \footnote{This special case is also related to the approach in Section \ref{Twoor3prop}.} Whereas the general case for a generic ${\bf f}(\x)$ (and $P$) is addressed in Section \ref{SectVariasF}.
 %that also includes the evidence $Z$ as a component. 
 \newline
\newline
{\bf Case 3: $f$ scalar but several posteriors.}   Above, we have to compute a set of $P$ different integrals. We can have other scenarios of interest with several integrals due to the use of $M$ different target pdfs, i.e.,
\begin{align}\label{eq_I_of_interest2}
{\bf I} = \int_\Theta  f(\x){\bm \post}(\x)d\x
\end{align}
 %moreover ${\bf f}(\x)=[f_1(\x),...,f_P(\x)]^{\top}: \Theta \to \mathbb{R}^{P}$ is a generic integrable function on $\Theta$ with $P\geq 1$ (scalar or vector-valued), and 
 with
\begin{align}\label{eq_I_of_interest3}
{\bf I}=[I_1,...,I_m,...,I_M], \quad \mbox{ where } \quad  I_m=\int_\Theta f(\x)\post_m(\x)d\x,
\end{align}
and ${\bm \post}(\x)=[\post_1(\x),...,\post_M(\x)]$. Note that the different target pdfs can be produced by a sequence of tempered posterior distributions \cite{Locatelli00,neal1996sampling,Neal01}.  This case is discussed in Section \ref{VariasTargets}.
\newline
\newline
{\bf Case 4: both vectorial functions.}  The most general scenario is when ${\bf I}$ and ${\bm \pi}$ are both vectorial functions, i.e.,  
\begin{align}\label{eq_I_of_interest4_0}
{\bf I}=[I_1,...,I_p,...,I_P]^{\top} \quad \mbox{ where } \quad  I_{p}=\int_\Theta f_p(\x)\post_p(\x)d\x
\end{align} 
that is addressed in Section \ref{MoreGenCaseSect}.
Generally, the integrals ${\bf I}$ and/or $Z$ cannot be computed analytically, and their computations require approximations by quadrature rules, variational algorithms and/or Monte Carlo methods. Here, we focus on the importance sampling (IS) family of techniques. 

\subsection{The baseline ideal  Monte Carlo (MC) estimator}
For simplicity, let us consider the integral
\begin{align}\label{eq_I_of_interest0_otra}
I =\mathbb{E}_{\bar{\pi}}\left[f(\x)\right]= \int_\Theta  f(\x)\post(\x)d\x,
\end{align}
in Eq. \eqref{eq_I_of_interest0}. The MC estimator of $I$ is given by
\begin{align}\label{MCest}
\widehat{I}_{\texttt{MC}} = \frac{1}{N}\sum_{n=1}^N  f(\x_n), \qquad \x_n \sim \post(\x). 
\end{align}
The estimator above is an unbiased estimation of $I$, i.e., 
\begin{align}\label{MCest2}
\mathbb{E}_{\bar{\pi}}\left[\widehat{I}_{\texttt{MC}}\right] = \frac{1}{N}\sum_{n=1}^N  \mathbb{E}_{\bar{\pi}}\left[f(\x_n)\right]= \frac{1}{N} \left(N  I\right)=I, 
\end{align}
 since $\x_n \sim \post(\x)$ and $\mathbb{E}_{\bar{\pi}}\left[f(\x_n)\right]=I$. Hence, $\mbox{Bias}_{\bar{\pi}}\big[\widehat{I}_{\texttt{MC}}\big]=0$, so that the mean squared error (MSE) is 
 $\mbox{MSE}_{\bar{\pi}}\big[\widehat{I}_{\texttt{MC}}\big]= \mbox{Var}_{\bar{\pi}}\big[\widehat{I}_{\texttt{MC}}\big] +\mbox{Bias}_{\bar{\pi}}\big[\widehat{I}_{\texttt{MC}}\big]^2=\mbox{Var}_{\bar{\pi}}\big[\widehat{I}_{\texttt{MC}}\big]$,
where
\begin{align}
\mbox{Var}_{\bar{\pi}}\big[\widehat{I}_{\texttt{MC}}\big]=\frac{1}{N} \mbox{Var}_{\bar{\pi}}\left[f(\boldsymbol{\theta})\right]&=\frac{1}{N} \int_{\Theta}(f(\boldsymbol{\theta})-I)^2 \bar{\pi}(\boldsymbol{\theta}) d \boldsymbol{\theta}=\frac{1}{N} \left(\mathbb{E}_{\bar{\pi}}\left[f(\boldsymbol{\theta})^2\right]-I^2 \right).
\end{align}
This variance and, as a consequence, the MSE converge to zero as $N\rightarrow 0$. It also depends on the variance of the random variable $F=f(\boldsymbol{\theta})$ with $\x \sim \post(\x)$. Unfortunately,  the estimator $\widehat{I}_{\texttt{MC}} $  cannot be applied in many practical problems, since we cannot draw samples directly from $\bar{\pi}(\boldsymbol{\theta})$. However, other types of MC sampling algorithms, such as rejection sampling schemes, Markov chain Monte Carlo (MCMC) techniques, importance sampling (IS) methods can be applied.
Generally, these alternative estimators have bigger variances than the baseline MC estimator. However, we will show the IS schemes using the optimal proposal densities can beat the baseline MC estimator, in terms of efficiency.
%Note that (5) is equivalent to stating that $\widehat{I}_M \xrightarrow{d} \mathcal{N}\left(I, V_M\right)$ as $M \rightarrow \infty$.
%Unfortunately, Algorithm 1 cannot be applied in many practical problems, because we cannot draw samples directly from $\bar{\pi}(\boldsymbol{\theta} \mid \mathbf{y})$. In these cases, if we can perform point-wise evaluations of the target function, $\pi(\boldsymbol{\theta} \mid \mathbf{y})=\ell(\mathbf{y} \mid \boldsymbol{\theta}) p_0(\boldsymbol{\theta})$, we can apply other types of Monte Carlo algorithms: Rejection Sampling (RS) schemes, Markov chain Monte Carlo (MCMC) techniques and importance sampling (IS) methods. These two large classes of algorithms, MCMC and IS, are the core of this paper and will be described in detail in the rest of this work. Before, we briefly recall the basis of the RS approach, which is one of the key ingredients of MCMC methods, in the following section.

%%%%%%%%%%%%%%%%%%%%%%%%%%%%%%%%%%%%%%%%
%\section{Part 1: minimizing the MSE with an unique proposal density}\label{FirstPart}
%%%%%%%%%%%%%%%%%%%%%%%%%%%%%%%%%%%%%%%%%

%%%%%%%%%%%%%%%%%%%%%%%%%%%%%%%%%%
%%%%%%%%%%%%%%%%%%%%%%%%%%%%%%%%%%
\section{The two fundamental families of  IS estimators }
%%%%%%%%%%%%%%%%%%%%%%%%%%%%%%%%%%
%%%%%%%%%%%%%%%%%%%%%%%%%%%%%%%%%%

In this section, we describe the two  basic IS schemes in the simplest scenario: considering one proposal density $q$ and one integral $I$. Note that, trough all the manuscript, the proposal density $q(\x)$ is always  normalized, i.e., $\int_{\Theta} q(\x) d\x=1$.

%%%%%%%%%%%%%%%%%%%%%%%%%%%%%%%%%%
\subsection{$Z$ known: standard importance sampling estimator $\widehat{I}_\text{IS}$}
%%%%%%%%%%%%%%%%%%%%%%%%%%%%%%%%%
 Let $q(\x)$ be a pdf with support $\Theta$, and denote as $\x_i$ a sample drawn from it, i.e., $\x_i\sim q(\x)$. The IS estimator of $I$ in Eq. \eqref{eq_I_of_interest} based on ${q}$ is given by
\begin{align}\label{eq_std_IS_est}
	\widehat{I}_\text{IS} &= \frac{1}{N}\sum_{i=1}^N \frac{f(\x_i)\post(\x_i)}{q(\x_i)}, \nonumber \\
	&=\frac{1}{NZ}\sum_{i=1}^N \frac{\pi(\x_i)}{q(\x_i)} f(\x_i),  \nonumber\\
	&=\frac{1}{NZ}\sum_{i=1}^N w_i f(\x_i),   \qquad \x_i\sim q(\x),
\end{align}
where we have set $w_i=\frac{\pi(\x_i)}{q(\x_i)}\geq 0$ for the non-negative weight assigned to sample $\x_i$.
The estimator $\widehat{I}_\text{IS}$ is an unbiased estimator of $I$  in Eq. \eqref{eq_I_of_interest0}, obtained with the sufficient condition that $q(\x) >0$ whenever $\post(\x)>0$. 

%%%%%%%%%%%%%%
\subsection{ $Z$ unknown: the self-normalized IS estimator $\widehat{I}_\text{SNIS} $}
%%%%%%%%%%%%%%%%%%
 When $\post(\x) = \frac{1}{Z}\pi(\x)$ and $Z$ is unknown, we need to resort to the so-called {\it self-normalized} IS (SNIS) estimator. 
 By using a standard IS estimator, we can estimate $Z$ ({\it reusing} the set of samples drawn from $q(\x)$) as 
 \begin{align}\label{eq_Z_est}
	\widehat{Z}_{\text{IS}}=\widehat{Z} =\frac{1}{N}\sum_{k=1}^{N}w_k= \frac{1}{N}\sum_{k=1}^N\frac{\pi(\x_k)}{q(\x_k)}, \qquad \x_k\sim q(\x).
\end{align}
 Then,  we can replace $Z$ with $\widehat{Z}$ into the  standard IS estimator $\widehat{I}_\text{IS}$ in Eq. \eqref{eq_std_IS_est}, obtaining 
\begin{align}\label{eq_SNIS_est}
	\widehat{I}_\text{SNIS} &=\frac{1}{N\widehat{Z} }\sum_{i=1}^N w_i f(\x_i), \nonumber \\
	%= \frac{1}{\sum_{k=1}^{N}\frac{\pi(\x_k)}{q(\x_k)}}\sum_{i= 1}^{N}\frac{f(\x_i)\pi(\x_i)}{q(\x_i)}, \nonumber\\
	  &= \frac{1}{\sum_{k=1}^{N}w_k}\sum_{i= 1}^{N}w_i f(\x_i), \\ 
	  &=\sum_{i= 1}^{N}\bar{w}_i f(\x_i),  \qquad \x_i\sim q(\x), \nonumber 
\end{align}
where $w_i=\frac{\pi(\x_i)}{q(\x_i)}$ and $\bar{w}_i=\frac{w_i}{\sum_{k=1}^{N}w_k}$, so that $\sum_{i=1}^{N}\bar{w}_i=1$. Unlike  $\widehat{I}_\text{IS}$,  the SNIS estimator is biased, but is still a consistent estimator of Eq. \eqref{eq_I_of_interest}. However, $\widehat{I}_\text{SNIS}$ can be generally more efficient than $\widehat{I}_\text{IS}$ and, in many real-world applications, it is the {\it only} applicable IS estimator \cite{Robert04,Liu04b}. Since $\widehat{I}_\text{SNIS}$ is a {\it convex}  combination of $ f(\x_i)$, $\widehat{I}_\text{SNIS}$ is always bounded (unlike $\widehat{I}_\text{IS}$). Indeed,  we can write 
\begin{align}
	\min_i f(\x_i) \leq \widehat{I}_\text{SNIS} \leq \max_i f(\x_i).
\end{align} 
This is a good property that allows, in some cases,  $\widehat{I}_\text{SNIS}$ to have better performance than $\widehat{I}_\text{IS}$. For instance, see Remark \ref{Rem4}. Note also that $\widehat{I}_\text{SNIS}$ can be seen as a quotient of two standard IS estimators, using the same set of samples drawn from $q(\x)$, i.e., 
\begin{align}
	\widehat{I}_\text{SNIS}= \frac{\widehat{E}}{\widehat{Z}} =\frac{\frac{1}{N} \sum_{i= 1}^{N}w_i f(\x_i)}{\frac{1}{N}\sum_{k=1}^{N}w_k},  \qquad \x_i\sim q(\x),
	% = \frac{\frac{1}{N}\sum_{i=1}^N\frac{f(\x_i)\pi(\x_i)}{q(\x_i)}}{\frac{1}{N}\sum_{i=1}^N\frac{\pi(\x_i)}{q(\x_i)}},
\end{align}
where the numerator $\widehat{E}$ is the estimator of $E=\int_{\Theta} f(\x)\pi(\x)d\x$, whereas the denominator $\widehat{Z}$ in Eq. \eqref{eq_Z_est} is the estimator of $Z=\int_{\Theta} \pi(\x)d\x$. Observe that both estimators are using the same proposal $q(\x)$ and also the same set of samples $\x_i$'s, so that they are correlated.   

%%%%%%%%%%%%%%%%%%%%%%%%%%%%%%%%%%%%%%%%
\section{Optimal IS schemes with an unique proposal density}\label{FirstPart}
%%%%%%%%%%%%%%%%%%%%%%%%%%%%%%%%%%%%%%%%%

%%%%%%%%%%%%%%%%%%%%%%%%%%%%%%%%%%%%%
\subsection{One proposal pdf, one function $f$ and one target $\post$}\label{SectAquiIStand}
%%%%%%%%%%%%%%%%%%%%%%%%%%%%%%%%%%%%%
In this section, we recall the optimal proposals of classical IS estimators: standard IS where $Z$ is assumed known, and the self-normalized IS where  $Z$ is unknown. Hence, here we consider the simplest case, i.e., ${\bf f}(\x)=f(\x)$ and ${\bf I}=I$ as in Eq. \eqref{eq_I_of_interest0}.



\subsubsection{Optimal proposal density in standard IS}

 The variance of the estimator $\widehat{I}_\text{IS}$ above is 
\begin{align}
	\mbox{Var}_q[\widehat{I}_\text{IS}]&=\frac{1}{N} \mbox{Var}_{q}\left[\frac{f(\x)\post(\x)}{q(\x)}\right] 
	= \frac{1}{N}\E_{q}\left[\left(\frac{f(\x)\post(\x)}{q(\x)}\right)^2\right] - \frac{I^2}{N}, \nonumber \\
	&= \frac{1}{N}\sigma_\text{IS}^2, \label{eq_var_std_IS}
%	&= \int_{\Theta_{q}}\frac{(f(\x)\post(\x) - Iq(\x))^2}{q(\x)}d\x.
\end{align}
where we have set 
$$
\sigma_\text{IS}^2=\E_{q}\left[\left(\frac{f(\x)\post(\x)}{q(\x)}\right)^2\right] - I^2.
$$
Thus, by applying Jensen's inequality in Eq. \eqref{eq_var_std_IS},  we have that
\begin{align}\label{eq_Jensen1}
\mathbb{E}_{q}\left[\left(\frac{f(\x)\post(\x)}{q(\x)}\right)^2\right] \geq \left(\mathbb{E}_{q}\left[\frac{f(\x)\post(\x)}{q(\x)}\right] \right)^2,
\end{align}
and the equality holds if and only if $\frac{f(\x)\post(\x)}{q(\x)}$ is constant, i.e., we should have $q(\x) \propto f(\x)\post(\x)$. However, for $f(\x)$ taking both negative and positive values, the product $f(\x)\post(\x)$ does not define a pdf. In this case, the only possibility is to take
%\framebox{dsds}
\begin{align}\label{OptimalProposalStandIS}
\fbox{$q_\text{opt}(\x)= \frac{|f(\x)|\post(\x)}{\int_{\Theta}   |f(\x)|\post(\x) d\x} = \frac{|f(\x)|\post(\x)}{I} \propto |f(\x)|\post(\x).$}
\end{align}
{\rem Note that the normalization $\int_{\Theta}   |f(\x)|\post(\x) d\x$ of $q_\text{opt}(\x)$ is unknown, hence we can evaluate $q_\text{opt}(\x)$ only approximately or up to the normalizing constant.  Moreover, it is difficult to sample from it.
}
\newline
\newline
The minimum possible value of the variance is
%However, contrary to the previous case, this choice does not yield a zero variance estimator. Indeed we have that
\begin{align}\label{aquifer}
	\mbox{Var}_{q_\text{opt}}[\widehat{I}_\text{IS}] = \frac{1}{N} \left[\left(\int |f(\x)|\post(\x)d\x\right)^2 - I^2\right].
\end{align}
Hence, when $f(\x)$ assumes both negative and positive values, the  lowest possible variance the standard IS can achieve is given by the expression above. If  $f(\x)$ is non-negative or non-positive, for all $\x$, the minimum  variance is zero  since the terms within parenthesis in Eq. \eqref{aquifer} cancel out.

{\rem $\mbox{Var}_{q_\text{opt}}[\widehat{I}_\text{IS}]=0$ when  $f(\x)$ is non-negative or non-positive for all $\x\in \Theta$. }

{\rem Note that $q_\text{opt}(\x)$ in Eq. \eqref{OptimalProposalStandIS} is the optimal proposal for a specific integral, i.e., considering a specific function $f(\x)$ \cite{Liu04b,Robert04}. %For every different expectation with respect to $\post(\x)$, we have a different optimal proposal $q_\text{opt}(\x)$. See Section \ref{AquiGenCase} for a further details.
}


%We consider two scenarios: (i) $f(\x)$ is non-negative, (ii) $f(\x)$ is arbitrary.
%%\newline
%For $f(\x)\geq 0$, and using the result from Eq. \eqref{eq_Jensen1}, the variance of $\widehat{I}_\text{IS}$ is minimized when
%\begin{align}\label{eq_q_opt_f_pos}
%	q_\text{opt}(\x) \propto f(\x)\post(\x).
%\end{align} 
%Indeed, this choice yields $\mbox{Var}[\widehat{I}_\text{IS}] = 0$. This implies that $\widehat{I}_\text{IS}$ is always exact (i.e. $\widehat{I}_\text{IS}=I$) regardless the specific values of $\x_i$'s or the sample size $N$, since
%\begin{align}
%	\widehat{I}_\text{IS}^{\texttt{opt}} = \frac{1}{N}\sum_{i=1}^N\frac{f(\x_i)\post(\x_i)}{f(\x_i)\post(\x_i)/I} = \frac{1}{N}\sum_{i=1}^N I = I.
%\end{align}
%Note that this choice is possible since $f(\x)\geq 0$ for all $\x$. 


\subsubsection{Optimal proposal in SNIS}\label{Sect_optSNIS} 
%Let use denote $\widehat{E} = \frac{1}{N}\sum_{i=1}^N\frac{f(\x_i)\pi(\x_i)}{q(\x_i)}$ and $\widehat{Z} = \frac{1}{N}\sum_{i=1}^N\frac{\pi(\x_i)}{q(\x_i)}$. Hence, $\widehat{I}_\text{SNIS}= \frac{\widehat{E}}{\widehat{Z}}$.
We have seen that $\widehat{I}_\text{SNIS}= \frac{\widehat{E}}{\widehat{Z}}$. If $N$ is large enough, the variance of the ratio $\widehat{I}_\text{SNIS}= \frac{\widehat{E}}{\widehat{Z}}$ can be approximated  as \cite{Robert04},
\begin{align*}
\text{Var}_{q}[\widehat{I}_{\text{SNIS}}] = \text{Var}_{q}\left[\frac{\widehat{E}}{\widehat{Z}}\right] 
\approx \frac{1}{Z^2}\text{Var}_{q}\big[\widehat{E}\big] - 2\frac{E}{Z}\text{Cov}_{q}\big[\widehat{E},\widehat{Z}\big]+ \frac{E^2}{Z^4}\text{Var}_{q}\big[\widehat{Z}\big].
\end{align*}
After some algebra over $\text{Cov}_{q}\big[\widehat{E},\widehat{Z}\big]$,  it is possible to show that 
\begin{align}\label{eq_var_SNIS}
	\mbox{Var}_{q}[\widehat{I}_\text{SNIS}] \approx
%	\widetilde{\mbox{var}}_{q}(\widehat{I}_\text{SNIS}) = 
	\frac{\sigma^2_{\text{SNIS}}}{N} =\frac{1}{N} \mathbb{E}_{q}\left[\left(\frac{\post(\x)}{q(\x)}\left(f(\x) - I\right)\right)^2\right].
\end{align}
 The optimal choice of $q(\x)$, for a specific $f(\x)$ (i.e., for a specific integral), is thus
\begin{align}\label{eq_q_opt_SNIS}
\fbox{$q_\text{opt}(\x)=\frac{|f(\x) - I|\post(\x)}{C_q}   \propto |f(\x) - I|\post(\x)$}.
\end{align} 
The normalizing constant 
$$
C_q=\int_{\Theta} |f(\x) - I|\post(\x) d\x=\mathbb{E}_{\post}[|f(\x)-I|],
$$
 is again unknown. The minimum reachable variance is
\begin{align}
 \mbox{Var}_{q_\text{opt}} [\widehat{I}_\text{SNIS}] & \approx
 \frac{1}{N} \int_{\Theta}\left(\frac{\post(\x)(f(\x) - I)}{q_\text{opt}(\x)} \right)^2 q_\text{opt}(\x) d\x, \nonumber\\
 & \approx
 \frac{1}{N} \int_{\Theta}C_q^2 \frac{1}{C_q} |f(\x) - I|\post(\x)d\x, \nonumber\\
  & \approx
 \frac{1}{N} C_q \int_{\Theta} |f(\x) - I|\post(\x)d\x, \nonumber\\
 & \approx
 \frac{1}{N} C_q^2=
  \frac{1}{N} \Big[\mathbb{E}_{\post}[|f(\x)-I|]\Big]^2.
\end{align}
%{\color{red}(deberiamos distinguir la expresion general $\sigma_\text{SNIS}^2$ de cuando se ha cogido la $q_\text{opt}$...)}
The above expression defines a fundamental lower bound for any SNIS estimator. 

{\rem\label{Rem4} In this case, unlike for $\widehat{I}_\text{IS}$, there does not exist a proposal density $q(\x)$ such that $\sigma^2_{\text{SNIS}}=0$   (even if $f(\x)$ is non-negative or non-positive). However, an interesting special case is when $f(\x)= c$, i.e., $f(\x)$ is a constant value. Indeed, if $f(\x)= c$, we will always have $\widehat{I}_\text{SNIS}=c$, i.e., we have zero bias and zero variance, while generally $\widehat{I}_\text{IS} \neq c$.}

{\rem Note that $q_\text{opt}(\x)$  in Eq. \eqref{eq_q_opt_SNIS} depends on the unknown integral  $I$. However, this expression has a theoretical value. Furthermore, the value of the integral $I$ could be also replaced  with an estimator $\widehat{I}$ (using also iterative procedures that we discuss in Section \ref{BridgeSect}).}
%\newline
%\newline
%{\color{red}
\subsubsection{Optimal proposal for estimating $Z$}\label{Sect_optZ}

The estimator $\widehat{Z}$ in Eq. \eqref{eq_Z_est} is a standard IS estimator of the integral in Eq. \eqref{eq_Z_of_int} and its variance is given by
\begin{align}\label{eq_var_Z}
\text{Var}[\widehat{Z}] = \frac{1}{N}\E_q\left[\frac{\pi(\x)^2}{q(\x)^2}\right] - \frac{1}{N}Z^2.
\end{align}
The optimal proposal is thus
\begin{align}
\fbox{$q_\text{opt}(\x) = \post(\x)$.}
\end{align}
 In this scenario, the optimal (minimum) variance is zero, i.e., $\text{Var}_{q_\text{opt}}[\widehat{Z}]=0$. Table  \ref{TablaIS_menos_1} summarizes all the considerations so far.

%}
\begin{table}[!h]	
	\caption{Summary of optimal proposal pdfs in Section \ref{SectAquiIStand}.}\label{TablaIS_menos_1}
	\vspace{-0.3cm}
	\begin{center}
		\begin{tabular}{|c|c|c| } 
			\hline
		 \diagbox{{\bf Scheme}}{{\bf Target}}
%		 {\bf Integral of interest}	
		 &  $I=\int_\Theta f(\x)\post(\x)d\x$  &  $Z=\int_\Theta \post(\x)d\x$      \\
%			&&& \\
			\hline 
			\hline		 
			&& \\
	Stand IS	&	$q_\text{opt}(\x)\propto |f(\x)|\post(\x)$ &
			 ---------------  \\
			  && \\
		  SNIS &  $q_\text{opt}(\x)\propto |f(\x) - I|\post(\x)$ 
			  & $q_\text{opt}(\x)=\post(\x)$ \\
			  && \\
%		     \hline
%		      \hline
%		     {\bf Section}  & \ref{SectAquiIStand} & \ref{SectVariasF}& \ref{AquiGenCase} \\
			\hline
		\end{tabular}
	\end{center}
\end{table}


%%%%%%%%%%%%%%%%%%%%%%%%%%%%%%%%%%%%%%%%%%%%%%%%%
%\subsubsection{Optimal proposal density for a generic particle approximation} \label{AquiGenCase}
%%%%%%%%%%%%%%%%%%%%%%%%%%%%%%%%%%%%%%%%%%%%%%%%%
%
%
%We have seen that the IS techniques provide estimators for specific expectations with respect to $\post(\x)$. 
% More generally, the IS scheme provides the following particle approximation of the measure of $\post(\x)$, i.e.,
%\begin{align}\label{eq_part_approx}
%	\post(\x)\approx \ \widehat{\pi}(\x) = \sum_{i=1}^N \bar{w}_i \delta(\x-\x_i), \qquad \x_i \sim q(\x),
%\end{align}
%where $\bar{w}_i =\frac{1}{\widehat{Z}} w(\x_i)$ and $w(\x_i) = \pi(\x_i)/q(\x_i)$ .
%%{\bf Optimal proposal for SNIS ``in general''. {\color{red}(lo de Deniz y Joaquin...)}} In the general setting when one is not interested in a specific $f(\x)$, it is usually stated that the ``optimal'' $q(\x)$ in SNIS corresponds to the target density $\post(\x)$. 

\subsubsection{Related theoretical results} \label{AquiGenCase}

In the literature, MSE bounds for the SNIS estimator can be found \cite{Akyildiz2020,Akyildiz2021}, for instance,
\begin{align}
	\mathbb{E}\left[\left(I - \widehat{I}_\text{SNIS}\right)^2\right] \leq \frac{c_f\rho}{N},
\end{align}
where $c_f = 4\norm{f}_\infty$ and $\rho = \mathbb{E}_{q}\left[\frac{\post(\x)^2}{q(\x)^2}\right]= \mathbb{E}_{q}\left[\frac{\pi(\x)^2}{Z^2q(\x)^2}\right]=\mathbb{E}_{q}\left[\frac{1}{Z^2}w(\x)^2\right]$ is the second moment of  $\post(\x)/q(\x)$ \cite{Akyildiz2020}.
The variance of the unnormalized weight $\mbox{Var}_{q}[w(\x)]$  can be related to a measure of divergence between the posterior and proposal \cite{Agapiou17,CHEN2005}, \cite[App. A.2]{llorente2021deep},
\begin{align}
\mbox{Var}_{q}[w(\x)]&= 
\int (w(\x) - Z)^2q(\x)d\x, \nonumber \\ 
&= Z^2 \int \left(\frac{\pi(\x)}{Zq(\x)} - 1\right)^2q(\x)d\x,  \nonumber  \\
		&= Z^2\int\frac{(\post(\x)- q(\x))^2}{q(\x)}d\x = Z^2 D_{\chi^2}(\post,q), \label{Eqchi}
\end{align} 
where we have used $\post(\x)=\frac{1}{Z}\pi(\x)$ and $D_{\chi^2}(\post,q)$ denotes the Pearson divergence between the posterior $\post$ and proposal $q$. Since $\E_q[w(\x)]=Z$, the relative MSE is 
\begin{align}
\mbox{rel-MSE}=\frac{\E_q[(w(\x)-Z)^2]}{Z^2}=\frac{\mbox{Var}_{q}[w(\x)]}{Z^2}&\propto D_{\chi^2}(\post,q). \label{Eqchi0}
\end{align} 
 Note that $D_{\chi^2}(\post,q) = \norm{(\post-q)\left(\frac{\post-q}{q}\right)}_{L_1}$ and, by Holder's inequality, we have 
\begin{align}\label{Eqchi2}
	D_{\chi^2}(\post,q) = \norm{(\post-q)\left(\frac{\post-q}{q}\right)}_{L_1}
	\leq \norm{\post- q}_{L_2}\norm{\frac{\post-q}{q}}_{L_2}.
\end{align}
Hence, by reducing the $L_2$ distance between $\post$ and $q$, we are diminishing the chi-squared divergence and equivalently the variance of the weight function $w(\x)=\pi(\x)/q(\x)$ \cite[App. A.3]{llorente2021deep}. Moreover, the MSE of $\widehat{I}_\text{SNIS}$ is shown to be bounded also in terms of  $D_{\chi^2}(\post,q)$ \cite{Agapiou17,Akyildiz2021,CHEN2005}.
In this sense, we can assert that the optimal particle approximation can be obtained with $q(\x)=\post(\x)$, as we have seen for $Z$ \cite{Dieng_NIPS2017}.  





%%%%%%%%%%%%%%%%%%%%%%%%%%%%%%%%%%%
\subsection{Unique optimal proposal pdf for multiple related integrals}
%%%%%%%%%%%%%%%%%%%%%%%%%%%%%%%%%%%



In this section, we address the problem of estimating multiple related integrals using a single proposal pdf, and derive the optimal choice for the different cases. Namely, here we search for a {\it unique} optimal proposal density for the simultaneous estimation of several quantities.



% ---------------------------------------------------- %



\subsubsection{Optimal proposal for the simultaneous estimation of $I$ and $Z$}\label{FantSectionLuca} 
%esto me lo enviaste en el whatapp

The choice $q(\x)=\post(\x)$ is very common when one is considering the SNIS estimator.  %as we have discussed in Section \ref{AquiGenCase}. 
With this choice, the variance of the estimator $\widehat{Z}$ is zero, but it is not optimal for estimating $I$ using SNIS.  Conversely, if one uses the optimal proposal for SNIS in Eq. \eqref{eq_q_opt_SNIS}, this choice is not optimal for estimating $Z$. 
\newline
Let us consider the case, where we seek an optimal proposal density for {\it simultaneously} estimating both $Z$ and $I$, using respectively standard IS and SNIS.
We think of the problem of estimating the vector of multiple integrals, where $P=2$ and the estimation of the vector
$$
{\bf I} =[I,Z]^\top,
$$ 
(i.e.,  $f_1(\x)=f(\x)$ and $f_2(\x)=Z$),  using a single proposal $q(\x)$, resulting in the vector estimator $\widehat{{\bf I}} = [\widehat{I}_\text{SNIS},\widehat{Z}]^\top$, where $\widehat{Z}$ from Eq. \eqref{eq_Z_est} and $\widehat{I}_\text{SNS}$ from Eq. \eqref{eq_SNIS_est}.
\newline
\newline
Since we are considering a vector-valued estimator, the variance $\text{Var}[\widehat{{\bf I}}]$ corresponds to a $2 \times 2$ covariance matrix.
We aim to find the proposal which minimizes the sum of variances in the diagonal of the covariance matrix.
For a scalar $f(\x)$, we have in \eqref{eq_var_SNIS} that
$$
\text{Var}[\widehat{I}_\text{SNIS}] \approx \frac{1}{NZ^2}\E\left[\frac{\pi(\x)^2(f(\x)-I)^2}{q(\x)^2}\right].
$$
Recall  also that $\text{Var}[\widehat{Z}] = \frac{1}{N}\left\{\E\left[\frac{\pi(\x)^2}{q(\x)^2}\right]-Z^2\right\}$. 
Thus, 
considering  the following definition of optimal density
\begin{align}
q_\text{opt}(\x) &= \arg\min_{q} \left( \text{Var}[\widehat{I}_\text{SNIS}] + \text{Var}[\widehat{Z}] \right) \\
&=  \arg\min_{q} \E \left[
\frac{Z^2\pi(\x)^2 + \pi(\x)^2(f(\x) - I)^2}{q(\x)^2}.
\right]
\end{align}
%{\color{red} using the  Jensen's inequality as we did above (a lo mejor necesitamos un paso intermedio o hacer un appendix...)} {\color{blue} - bien appendix,} 
Using the  Jensen's inequality as in the previous sections, we have that
\begin{align}
\fbox{$q_\text{opt}(\x) \propto \pi(\x)\sqrt{(f(\x)- I)^2 + Z^2}$.}
\end{align}
Note again that  the density $q_\text{opt}(\x)$  depends on two unknowns $I$ and $Z$. However, this expression has a theoretical value and  iterative procedures could be employed, as shown in Section \ref{BridgeSect}.

%%{\color{red} add proof in appendix}












% ---------------------------------------------------- %


%%%%%%%%%%%%%%%%%%%%%%%%%%%%%%%%%%%%%
\subsubsection{Optimal proposal for vector-valued functions}\label{SectVariasF}
%%%%%%%%%%%%%%%%%%%%%%%%%%%%%%%%%%%%%

Let us consider a vector-valued function ${\bf f}(\x) = [f_1(\x),\dots,f_P(\x)]^\top$
%$${\bf f}(\x) = \begin{bmatrix}
%f_1(\x) \\
%f_2(\x) \\
%\vdots \\
%f_P(\x)
%\end{bmatrix}$$
with $P$ components (and $M=1$).
%where the $p$-th component
% {\color{red}$f_p(\x) \geq 0$ (creo que podemos quitar esta condicion...)}. 
We are interested in a vector of integrals
$${\bf I} = \begin{bmatrix}
I_1 \\
I_2 \\
\vdots \\
I_P
\end{bmatrix}
= \begin{bmatrix}
\int_\Theta  f_1(\x)\post(\x)d\x \\
\int_\Theta  f_2(\x)\post(\x)d\x \\
\vdots \\
\int_\Theta  f_P(\x)\post(\x)d\x
\end{bmatrix}.$$
Note that all the integrals share the presence of the posterior $\post(\x)$ (hence, they are in some sense connected). If one function $f_p(\x) =Z$ for some $p$, then we have $I_p=Z$. 
We aim to obtain the optimal proposal for estimating the whole vector ${\bf I}$ in standard IS and in SNIS. Since the posterior $\post(\x)=\frac{1}{Z} \pi(\x)$ is the same in each component of ${\bf I}$, so that we have also a unique normalizing constant $Z$.
\newline
\newline
{\bf Knowing $Z$.} Using the standard IS scheme,
we already know that the $p$-th integral can be estimated through IS with optimal proposal  $q_{p,\text{opt}}(\x) \propto |f_p(\x)|\post(\x)$  for generic $f_p$, for which the variance is minimum.  On the contrary, if we consider a unique proposal for estimating all $I_p$'s, we need to study the variance of the vector-valued estimator ${\bf \widehat{I}}$ whose $p$-th component is
\begin{align}
	\left({\bf \widehat{I}}\right)_p =  \widehat{I}_p = \frac{1}{N}\sum_{i=1}^N\frac{\post(\x_i)f_p(\x_i)}{q(\x_i)},
\end{align}
for $p=1,\dots, P$. Note that all $\widehat{I}_p$'s use the same set $\{\x_i\}_{i=1}^N \sim q$.
%The variance of ${\bf \widehat{I}}$ is a covariance matrix
%$
%\mbox{Var}({\bf \widehat{I}}) = \left(\mbox{Cov}(\widehat{I}_{n}, \widehat{I}_{m})\right)_{n,m}$,  $1\leq n,m \leq P,$
%where
%\begin{align}
%	\mbox{Cov}(\widehat{I}_n,\widehat{I}_m)
%	 =
%	 \frac{1}{N}\E_{q}\left[\frac{\post(\x)^2}{q(\x)^2}f_n(\x)f_m(\x)\right] - \frac{1}{N}I_nI_m.
%\end{align}
%If we use $q(\x) = q_{p,\text{opt}}(\x) \propto f_p(\x)\post(\x)$, we have $\mbox{cov}(\widehat{I}_p, \widehat{I}_j)=0$ for $j= 1, \dots, P$.
%{\color{red}toda la discusion anterior sobre los terminos de covarianza puede confundir... en la sect anterior simplemente dijimos de minimizar la suma de las varianzas y aqui parece que tenemos que justificarlo...} 
It is natural to look for the proposal that minimizes the sum of the variance of each component, i.e., 
\begin{align}
	q_\text{opt}(\x) = \arg \min_{q} \sum_{p=1}^{P}\mbox{Var}_{q}[\widehat{I}_p].
\end{align}
This is justified from the MSE of ${\bf \widehat{I}}$
\begin{align}\label{eq_MSE_vec}
	\mbox{MSE}({\bf \widehat{I}}) =	\mathbb{E}[({\bf \widehat{I}} - {\bf I})^\top ({\bf \widehat{I}} - {\bf I})]
		=\sum_{p=1}^{P}\E\left[(\widehat{I}_p - I_p)^2\right]
	=\sum_{p=1}^{P}\mbox{Var}_{q}[\widehat{I}_p],
\end{align}
where in the last equality we use $\mathbb{E}[\widehat{I}_p] = I_p$ for $p=1,\dots,P$, i.e., we have unbiased estimators. Hence the $q_\text{opt}$ above is the choice for which ${\bf \widehat{I}}$ is the MMSE estimator of ${\bf I}$. 
Let us rewrite the sum of variances as follows
\begin{align*}
	\sum_{p=1}^{P}\mbox{Var}_{q}(\widehat{I}_p) &= \frac{1}{N}\sum_{p=1}^{P}\mbox{Var}\left[\frac{\post(\x) f_p(\x)}{q(\x)}\right] \\
	&=\frac{1}{N}\sum_{p=1}^{P}\left(
	\mathbb{E}_{q}\left[\frac{\post(\x)^2 f_p(\x)^2}{q(\x)^2}\right] - I_p^2 
	\right) \\
	&=\frac{1}{N}\sum_{p=1}^{P}
	\E_{q}\left[\frac{\post(\x)^2 f_p(\x)^2}{q(\x)^2}\right] - \frac{1}{N}\sum_{p=1}^{P}I_p^2 \\
	&=\frac{1}{N}	\E_{q}\left[\frac{\post(\x)^2 \sum_{p=1}^{P}
	f_p(\x)^2}{q(\x)^2}\right] - \frac{1}{N}\sum_{p=1}^{P}I_p^2. 
\end{align*}
Thus, by Jensen's inequality, we have that
\begin{align}
	\E_{q}\left[\frac{\post(\x)^2 \sum_{p=1}^{P}
		f_p(\x)^2}{q(\x)^2}\right] 
	\geq \left(\E_{q}\left[\frac{\post(\x) \sqrt{\sum_{p=1}^{P}
			f_p(\x)^2
		}}{q(\x)}\right]\right)^2 = \left(\E_{q}\left[\frac{\post(\x) \norm{{\bf 
	f}(\x)}_2}{q(\x)}\right]\right)^2,
\end{align}
where the equality holds if and only if 
$\frac{\post(\x) \norm{{\bf f}(\x)}_2}{q(\x)}$ is constant.
Thus, we have that 
\begin{align}
\fbox{$q_\text{opt}(\x) \propto \post(\x)\norm{{\bf f}(\x)}_2.$}
\end{align}
%and
%\begin{align}
% \sum_{p=1}^{P}\mbox{var}_{q_\text{opt}}(\widehat{I}_p) = \frac{1}{N}\left( \int_\Theta  \post(\x)\norm{{\bf f}(\x)}_2d\x\right)^2 - \frac{1}{N}\sum_{p=1}^{P}I_p^2. 
%\end{align}
%\newline
\newline
{\bf With an unknown $Z$.} Let us consider now estimating the vector ${\bf I}$ using the SNIS approach, i.e., the $p$-th estimator is
\begin{align}
\left({\bf \widehat{I}}\right)_p =  \widehat{I}_p = \frac{1}{\sum_{j=1}^N\frac{\pi(\x_j)}{q(\x_j)}}\sum_{i=1}^N\frac{\pi(\x_i)f_p(\x_i)}{q(\x_i)}.
\end{align}
In this case, the MSE in Eq. \eqref{eq_MSE_vec} does not correspond exactly to the sum of variances, since the SNIS estimators are biased, i.e.,
\begin{align}
	\text{MSE}\left(\widehat{{\bf I}}_\text{SNIS}\right) = \sum_{p=1}^{P}\left\{\text{Var}\left(\widehat{I}^\text{SNIS}_p\right) + \text{Bias}\left(\widehat{I}^\text{SNIS}_p\right)^2\right\} \approx \sum_{p=1}^{P}\text{Var}\left(\widehat{I}^\text{SNIS}_p\right),
\end{align}
where the last approximation fulfills if the sample size $N$ is big enough (since the bias terms are dominated by the variances in the limit $N \to \infty$).
%{\color{blue}la ecuacion de arriba me gusta pero, de nuevo, cuando derivamos la opt en SNIS por primera vez no escribimos lo del sesgo y aqui si....}
Its variance is given (approximately) by
	$$ 
	\mbox{Var}\left[\widehat{I}_p\right] \approx \frac{1}{N}\E_{q}\left[\left(\frac{\post(\x)}{q(\x)}(f_p(\x) - I_p)\right)^2\right].
	$$
The sum of diagonal variances is thus
\begin{align}
\sum_{p=1}^P\mbox{Var}\left[\widehat{I}_p\right] &\approx	\sum_{p= 1}^P\E_{q}\left[\left(\frac{\post(\x)}{q(\x)}(f_p(\x) - I_p)\right)^2\right] \\
&= \E_{q}\left[\frac{\post(\x)^2}{q(\x)^2}\sum_{p= 1}^P(f_p(\x) - I_p)^2\right].
\end{align}
Hence, the optimal proposal is given by
\begin{align}
\fbox{	$q_\text{opt}(\x) 
%	&= \arg \min_{q}	\sum_{p= 1}^P\E_{q}\left[\left(\frac{\post(\x)}{q(\x)}(f_p(\x) - I_p)\right)^2\right] \nonumber \\
	\propto \post(\x)\norm{{\bf f}(\x) - {\bf I}}_2.$}
\end{align}










% ------------------------------------------------------------- %





%%%%%%%%%%%%%%%%%%%%%%%%%%%%%%%%%%%%%%%%%%%%
\subsubsection{Optimal proposal for integrals involving several target pdfs}\label{VariasTargets}
%%%%%%%%%%%%%%%%%%%%%%%%%%%%%%%%%%%%%%%%%%%%

 Now, instead of a set of functions as in the previous section, we are interested in a vector of integrals induced by having a set of target pdfs and a fixed scalar function. This setting corresponds to, e.g., robust Bayesian analysis, where one is interested in computing a lower bound on expectations of a specific function with respect to a family of posterior distributions \cite{cruz2022iterative}.
Let us denote with $\post_m(\x)$ for $m=1,\dots,M$ a set of target pdfs.
We are interested in the following vector of integrals
$${\bf I} = \begin{bmatrix}
	I_1 \\
	I_2 \\
	\vdots \\
	I_M
\end{bmatrix}
= \begin{bmatrix}
	\int_\Theta  f(\x)\post_1(\x)d\x \\
	\int_\Theta  f(\x)\post_2(\x)d\x \\
	\vdots \\
	\int_\Theta  f(\x)\post_M(\x)d\x
\end{bmatrix}.$$
Note that we consider the same $f(\x)$ for all the integrals, each w.r.t. $\post_m(\x)=\frac{1}{Z_m}\pi_m(\x)$. Note that, in this scenario, we also have $P$ different normalizing constants $Z_m$.
\newline
\newline
{\bf All the $Z_m$ are known.} In the case we can evaluate $\post_m(\x)$ for all $m$ (i.e. we have available $Z_m=\int_\Theta  \pi_m(\x)d\x$ for all $p$), the variance of each
$$
\widehat{I}_m = \frac{1}{N}\sum_{i=1}^{N}\frac{\post_m(\x_i)f(\x_i)}{q(\x_i)}, \quad  \x_i\sim q(\x_i),
$$
is given by
$$
\mbox{Var}(\widehat{I}_m) = \frac{1}{N}\mbox{Var}\left[\frac{\post_m(\x)f(\x)}{q(\x)}\right].
$$
Applying the Jensen's inequality, we can see that the sum of these variances $\sum_{m=1}^M \mbox{Var}(\widehat{I}_m)$ is minimized when we take the proposal as
\begin{align}
&\fbox{$q_\text{opt}(\x) \propto |f(\x)|\sqrt{\post_1(\x)^2+\dots+\post_M(\x)^2}$,}  \nonumber\\
&\fbox{$ q_\text{opt}(\x)\propto f(\x)\|\bm{\bar{\pi}}(\x)\|_2$,}
\end{align}
where we have defined 
$$
\bm{\bar{\pi}}(\x)=[\post_1(\x),...,\post_M(\x)].
$$
%\newline
%\newline
{\bf The $Z_m$ are unknown.} In the case we can only evaluate $\pi_m(\x)$ for all $m$ (i.e. $Z_m=\int_\Theta  \pi_m(\x)d\x$  are not availanle for all $m$), we need to consider the self-normalized estimators
$$
\widehat{I}_m = \frac{1}{\sum_{j=1}^N\frac{\pi_m(\x_j)}{q(\x_j)}}\sum_{i=1}^{N}\frac{\pi_m(\x_i)f(\x_i)}{q(\x_i)}, \quad  \x_i\sim q(\x_i)
$$
whose asymptotic variance is
$$
\mbox{var}(\widehat{I}_m) \approx
\frac{1}{N}\E_{q}\left[\left(\frac{\post_m(\x)}{q(\x)}(f(\x) - I_m)\right)^2\right], \qquad \mbox{ as } \qquad N\rightarrow \infty.
$$
 Again using the Jensen's inequality, we can see that the sum of asymptotic variances is minimized when we take the following proposal density:
\begin{align}
&\fbox{$q_\text{opt}(\x) \propto \sqrt{\post_1(\x)^2(f(\x)-I_1)^2+\dots+\post_M(\x)^2(f(\x)-I_M)^2}$,} \nonumber \\
&\fbox{$q_\text{opt}(\x)\propto \|\bm{\bar{\pi}}(\x)\odot(f(\x)\bm{1}_M - {\bf I})\|_2$,}
\end{align}
where $\bm{1}_M=[1,....1]$ is a $1\times M$ unit vector and $\odot$ denotes the element-wise product. Table \ref{TablaIS_0} summarizes these results.


\begin{table}[!h]	
	\caption{Summary for scalar and vector integrals. The related sections are also provided.}\label{TablaIS_0}
	\vspace{-0.7cm}
	\begin{center}
		\begin{tabular}{|c|c|c|c| } 
			\hline
%			\multicolumn{6}{|c|}{ } \\
%			\multicolumn{6}{|c|}{ $\widehat{Z}_{IS1} = \frac{1}{N} \sum_{i=1}^N\frac{g(\x_i)}{q(\x_i)} \ell(\y|\x_i)=\frac{1}{N}\sum_{i=1}^N \rho_i \ell(\y|\x_i)$, \quad  $\rho_i=\frac{g(\x_i)}{q(\x_i)}$}\\
%			\multicolumn{6}{|c|}{ } \\ 
%%			\hline
			 &&& \\
			{\bf Integrals} &  $I=\int_\Theta f(\x)\post(\x)d\x$  &  ${\bf I}=\int_\Theta{\bf f}(\x)\post(\x)d\x$   &  ${\bf I}=\int_\Theta f(\x)\bm{\bar{\pi}}(\x)d\x$   \\
			 {\bf of interest} &&& \\
			\hline 
			\hline		 
			&&& \\
	Stand IS	&	$q_\text{opt}(\x)\propto |f(\x)|\post(\x)$ &
			  $q_\text{opt}(\x)\propto \|{\bf f}(\x)\|_2\post(\x)$ & 
			  $q_\text{opt}(\x)\propto f(\x)\|\bm{\bar{\pi}}(\x)\|_2$  \\
			  &&& \\
		  SNIS &  $q_\text{opt}(\x)\propto |f(\x) - I|\post(\x)$ 
			  & $q_\text{opt}(\x)\propto \|{\bf f}(\x) - {\bf I}\|_2\post(\x)$ &
			  $q_\text{opt}(\x)\propto \|\bm{\bar{\pi}}(\x)\odot(f(\x)\bm{1}_P - {\bf I})\|_2$ \\
			  &&& \\
		     \hline
		      \hline
		     			 {\bf Section}  & \ref{SectAquiIStand} & \ref{SectVariasF}& \ref{VariasTargets} \\
			\hline
		\end{tabular}
	\end{center}
	\begin{center}
		\begin{tabular}{|c|c|c|} 
		\hline
		    &  &  \\ 
		{\bf Integrals}  & ${\bf I} = \int_\Theta {\bf f}(\x) \odot {\bm \post}(\x)d\x$   &    $ \left[I,Z\right]$\\
		   {\bf of interest}      & &  \\   
	       \hline	
	         \hline	
	       & & \\ 
	  %      & & \\
	Stand IS	&  $q_\text{opt}(\x)\propto  \|{\bf f}(\x)\odot\post(\x)\|_2$  &  --------------- \\
			 & & \\
		  SNIS & $q_\text{opt}(\x)\propto \|\bm{\bar{\pi}}(\x)\odot({\bf f}(\x) - {\bf I})\|_2$  &  $q_\text{opt}(\x) \propto \pi(\x)\sqrt{(f(\x)- I)^2 + Z^2}$  \\
			&  & \\
	         \hline
	          \hline
	           {\bf Section}  & \ref{MoreGenCaseSect} & \ref{FantSectionLuca} \\
	           \hline	
       	\end{tabular}
	\end{center}
\end{table}


















%%%%%%%%%%%%%%%%%%%%%%%%%%%
\subsubsection{Optimal proposal for vector-valued functions and several target densities}\label{MoreGenCaseSect}
%%%%%%%%%%%%%%%%%%%%%%%%%%%

%{\color{red} {\bf esto es solo el caso `diagonal' ...no con la matriz}}

Let us consider now the following vector of integrals
$${\bf I} = \begin{bmatrix}
	I_1 \\
	I_2 \\
	\vdots \\
	I_P
\end{bmatrix}
= \begin{bmatrix}
	\int_\Theta  f_1(\x)\post_1(\x)d\x \\
	\int_\Theta  f_2(\x)\post_2(\x)d\x \\
	\vdots \\
	\int_\Theta  f_P(\x)\post_P(\x)d\x
\end{bmatrix},
$$
which can be summarized in the following vectorial form,
\begin{align}\label{eq_I_of_interest_gen}
{\bf I} = \int_\Theta {\bf f}(\x) \odot {\bm \post}(\x)d\x,
\end{align}
where both ${\bf f}(\x)$ and ${\bm \post}(\x)$ are vector-valued functions with $P$ components. This scenario can appear when using tempered posteriors or posteriors considering different mini-batches of data, for instance. 
\newline
\newline
%{\color{red}Luca says: se puede hacer este caso? a�adir a tabla resumen si fuera posible}
%yo creo que va ser algo as�...
{\bf With known $Z_p$'s.}  We look for the proposal that optimizes the variance of the vector-valued estimator whose $p$-th component is $\widehat{I}_p = \frac{1}{N}\sum_{i=1}^{N}\frac{f_p(\x_i)\post_p(\x_i)}{q(\x_i)}$, since $\widehat{I}_p$ are all unbiased. Hence, we have
\begin{align}
	\text{Var}_q[\widehat{{\bf I}}_\text{IS}] &= \frac{1}{N} \sum_{p=1}^P \text{Var}_q\left[\frac{f_p(\x)\post_p(\x)}{q(\x)}\right],\nonumber \\
	&= \frac{1}{N} \sum_{p=1}^{P} \E\left[\frac{f_p(\x)^2\post_p(\x)^2}{q(\x)^2}\right] + \text{constant terms}, \nonumber \\
	&=  \frac{1}{N} \E\left[\frac{\sum_{p=1}^{P}f_p(\x)^2\post_p(\x)^2}{q(\x)^2}\right] + \text{constant terms}.
\end{align}
Then, applying the Jensen's inequality as in the previous sections, we obtain  
\begin{align}
\fbox{$ q_\text{opt}(\x)\propto \|{\bf f}(\x) \odot \bm{\bar{\pi}}(\x)\|_2$.}
\end{align}
\newline
{\bf With unknown $Z_p$'s.} The sum of the variances of the SNIS estimators is
\begin{align}
	\text{Var}_q[\widehat{{\bf I}}_\text{SNIS}] &\approx \sum_{p=1}^P \frac{1}{NZ_p^2} \E_q\left[\frac{\pi_p(\x)^2(f_p(\x)-I_p)^2}{q(\x)}\right] \\
	&=  \frac{1}{N} \E_q\left[\frac{\sum_{p=1}^P\post_p(\x)^2(f_p(\x)-I_p)^2}{q(\x)}\right].
\end{align}
Hence, following the same procedure, we finally get
\begin{align}
&\fbox{$q_\text{opt}(\x) \propto \| 
{\bf \post}(\x) \odot ({\bf f}(\x) - {\bf I})		
 \|_2$.} 
\end{align}



\noindent
{\rem All the optimal proposal densities $q_\text{opt}(\x)$ in this section are just known up to a normalizing constant, and their point-wise evaluation is also intractable in unnormalized form, since they depend on the unknown quantities we want to estimate, such as $I$ or $Z$. 
% Therefore, their use for computing the marginal likelihood in Eqs. \eqref{MargLike} would require the  additional approximation of the normalizing constant of the proposal pdf as well.
}

%%%%%%%%%%%%%%%%%%%%%%%%%%%%%%%%%%%%%%%%%%%%%
%\section{Part 2: Optimal IS schemes for posterior expectations with multiple proposal pdfs}
\section{Optimal IS schemes with multiple proposal pdfs}
\label{sec_ISconvariasprop}
%%%%%%%%%%%%%%%%%%%%%%%%%%%%%%%%%%%%%%%%%%%%%%


%{\color{red}- con el self-normalized los ingles y franceses dice que es mejor dos.... $\Longleftarrow$ esta en Sect. \ref{sec_gen_results_IS} 
%}
 It is interesting to note that we can beat optimal {\it one-proposal} IS estimators by introducing (new) additional proposals, each one tuned and/or optimized for a specific task. These optimal {\it two/three-proposal} IS estimators can overcome the performance limits of previous analyzed estimators \cite{rainforth2020target}.
\newline
In this section, we present some results regarding the optimality in IS schemes where the use of more than one proposal pdfs is jointly considered. In Sections \ref{Twoor3prop}- \ref{Twoor3prop2}, we describe an optimal use of two and three proposal pdfs in the standard IS and SNIS schemes, respectively.
In Sect. \ref{sec_MIS}, we present the general setting of multiple IS (MIS), and discuss the optimal MIS scheme.
Here we focus in the approximation of the posterior expectation $I$ in Eq. \eqref{eq_I_of_interest0}.
%In the next Section \ref{sec_marglike}, we consider the case of the marginal likelihood computation with  multiple proposals.


%{\color{red} {\bf faltan mas refs/citas, no olvidar  - tambien la nuestra Computational statistics}}



\subsection{Two proposals for the standard IS estimator $\widehat{I}_\text{IS}$}\label{Twoor3prop}


%The variance of the classical estimators (standard IS and SNIS), even with the optimal proposal choice, is greater than a lower bound. 
%In other words, there is a limit in the performance these estimators can have for a given problem and a given sample size.
%The use of more than one proposals allows to bypass this limitation and hence produce estimators with arbitrary small variance, for a given problem and sample size. 
%\newline
%{\bf Standard IS with two proposals.}
In a standard IS scheme (i.e., when $Z$ is known), with a generic function $f(\x)$, is possible to obtain an estimator with zero variance with the so-called `positivisation trick' of $f$ and the use of two proposals $q_1(\x)$ and $q_2(\x)$ \cite{Owen00,rainforth2020target}. 
The positivisation trick consists in dividing the integral of interest \cite{Llorente2023_TBI,rainforth2020target},
$$
I=\int_\Theta f(\x)\post(\x)d\x,
$$
 in two different integrals,
\begin{align}
I = I_+ - I_- = \int_\Theta f_+(\x)\post(\x)d\x - \int_\Theta f_-(\x)\post(\x)d\x,
\end{align}
where $f_+(\x) = \max\{0, f(\x)\}$ and $f_-(\x) = \max\{0, -f(\x)\}$ are non-negative functions. 
Thus, we can address the approximation of the two integrals, 
$$
I_+ = \E_{q_1}\left[\frac{f_+(\x)\post(\x)}{q_1(\x)}\right] \quad  \text{and} \quad  I_-= \E_{q_2}\left[\frac{f_-(\x)\post(\x)}{q_2(\x)}\right].
$$
Hence, we can design the following two-proposal IS estimator using {\it two different sets of samples}, $\x_i$ and $\widetilde{\x}_j$,
\begin{align}
	\widehat{I}_\text{IS-2q} = \widehat{I}_+ - \widehat{I}_- 
	=\frac{1}{N_1}\sum_{i=1}^{N_1}\frac{f_+(\x_i)\post(\x_i)}{q_{1}(\x_i)} - 
	\frac{1}{N_2}\sum_{j=1}^{N_2}\frac{f_-(\widetilde{\x}_j)\post(\widetilde{\x}_j)}{q_{2}(\widetilde{\x}_j)}, \quad \x_i\sim q_{1}(\x), \ \widetilde{\x}_j \sim q_2(\x), 
\end{align}
with $i=1,\dots,N_1$ and $j=1,\dots,N_2$.
A zero-variance estimator could be obtained by choosing, respectively, 
\begin{align}\label{eq_2_q_opt}
\fbox{$q_{1,\text{opt}}(\x) \propto f_+(\x)\post(\x),$} \quad \text{and} \quad 	\fbox{$q_{2,\text{opt}}(\x) \propto f_-(\x)\post(\x).$}
\end{align}
%Finally, the resulting estimator is the difference of two IS estimators
%\begin{align}
%\widehat{I}_\text{IS-2q} = \widehat{I}_+ - \widehat{I}_- 
%=\frac{1}{N_1}\sum_{i=1}^{N_!}\frac{f_+(\x_i)\post(\x_i)}{q_{1,\text{opt}}(\x_i)} - 
%\frac{1}{N_2}\sum_{i=1}^{N_2}\frac{f_-(\widetilde{\x}_i)\post(\widetilde{\x}_i)}{q_{2,\text{opt}}(\widetilde{\x}_i)},
%\end{align}
%{\color{red}no se si poner la expression con la q optimas merece la pena (es trivial $\widehat{I}_+-\widehat{I}_- = I_+ - I_-$).... seria mejor ponerla con q genericas y luego decir las optimas...{\color{blue} es decir, lo que he puesto en rojo arriba, y luego simplemente decir que se puede hacer Var$[\widehat{I}_{2q}]=0$}...}
%where $\x_i \sim q_{1,\text{opt}}(\x)$ ($i=1,\dots,N_1$) and $\widetilde{\x}_i \sim q_{2,\text{opt}}(\x)$ ($i=1,\dots,N_2$).
%\newline

%%%%%%%%%%%%%%%%%%%%%%%%%%%%%%%
\subsection{Two proposals for SNIS estimator $\widehat{I}_\text{SNIS}$}\label{Twoor2prop2}
%%%%%%%%%%%%%%%%%%%%%%%%%%%%%%%
If  $Z$ is unknown, we should use a SNIS approach.
In this case, if we want to have zero variance, we need to consider more than two proposal pdfs \cite{rainforth2020target}. Indeed, the SNIS estimator is the ratio of two standard IS estimators, 
$$  I = \frac{E}{Z} = \frac{\E_q\left[\frac{f(\x)\pi(\x)}{q(\x)}\right]}{\E_q\left[\frac{\pi(\x)}{q(\x)}\right]}\approx \widehat{I}_\text{SNIS}=\frac{\widehat{E}}{\widehat{Z}}
=\dfrac{
	\dfrac{1}{N}\sum_{i=1}^{N}\frac{f(\x_i)\pi(\x_i)}{q(\x_i)} 
}{
\dfrac{1}{N}\sum_{i=1}^{N}\frac{\pi(\x_i)}{q(\x_i)}
}, \qquad \x_i \sim q(\x).
$$ 
So far, for both estimators $\widehat{E}$ and $\widehat{Z}$ we employ the same samples from the same unique proposal pdf $q(\x)$. Note that it is impossible to design a proposal density that works arbitrary well for both numerator and denominator \cite{Llorente2023_TBI,rainforth2020target}.
Hence, we could employ a different proposal density for each estimator,
$$ 
 I = \frac{E}{Z} = \frac{\E_{q_1}\left[\frac{f(\x)\pi(\x)}{q_1(\x)}\right]}{\E_{q_2}\left[\frac{\pi(\x)}{q_2(\x)}\right]},
$$ 
so that the final estimator is the ratio of two estimators using different samples from different proposal pdfs,
\begin{align}
 \widehat{I}_\text{SNIS-2q}=\frac{\widehat{E}_{q_1}}{\widehat{Z}_{q2}}=
 \dfrac{
	\dfrac{1}{N}\sum_{i=1}^{N_1}\frac{f(\x_i)\pi(\x_i)}{q_1(\x_i)} 
}{
\dfrac{1}{N_2}\sum_{k=1}^{N_2}\frac{\pi(\z_k)}{q_2(\z_k)}
}, \qquad  \x_i \sim q_1(\x), \quad \z_k \sim q_2(\z).
\end{align}
We can use the two proposals $q_{1,\text{opt}}(\x)$ as in Section \ref{SectAquiIStand} for estimating the numerator $E$, and take a second optimal proposal as $q_{2,\text{opt}}(\x) \propto \pi(\x)$ for estimating $Z$, i.e.,
\begin{align}
\fbox{$q_{1,\text{opt}}(\x) \propto |f(\x)|\post(\x),$}   \quad \text{and} \quad \fbox{$q_{2,\text{opt}}(\x) \propto \pi(\x)$.}
\end{align}
Note that $\widehat{I}_\text{SNS-2q}$ can provide better performance than $\widehat{I}_\text{SNS}$ (considering the same number of total samples and evaluation of $\pi(\x)$, i.e.,  $N=N_1+N_2$), since each optimal proposal is tailored to each specific estimator (instead of a unique proposal addressing the whole ratio of estimators).

%%%%%%%%%%%%%%%%%%%%%%%%%%%%%%%
\subsection{Three proposals for SNIS estimator $\widehat{I}_\text{SNIS}$}\label{Twoor3prop2}
%%%%%%%%%%%%%%%%%%%%%%%%%%%%%%%

The previous estimator $ \widehat{I}_\text{SNS-2q}$ can be improved using an additional proposal density. Indeed, we can split  $f(\x)$ as we have done in Section \ref{Twoor3prop}. %.... can generally positive and/or negative, for different values of $\x$.
%Instead of using the same $q(\x)$, we can devise two optimal proposals, one for $E$, and another one for $Z$. However, if $f(\x)$ is not non-negative, the optimal standard IS estimator $\widehat{E}$ does not have zero variance (see positivisation trick above).
The idea is to divide the integral $I$ in three different parts, i.e.,
$$ 
 I = \frac{E^+ - E^-}{Z} = \frac{\E_{q_1}\left[\frac{f_+(\x)\pi(\x)}{q_1(\x)}\right] - \E_{q_2}\left[\frac{f_-(\x)\pi(\x)}{q_2(\x)}\right]}{\E_{q_3}\left[\frac{\pi(\x)}{q_3(\x)}\right]},
 $$ 
 where we have applied the positivisation trick of $f$ in the numerator, i.e., we denote as $f_+(\x) = \max\{0, f(\x)\}$ and $f_-(\x) = \max\{0, -f(\x)\}$ two non-negative functions. The resulting estimator is, in this case,
\begin{align}
\widehat{I}_\text{SNIS-3q}=  \dfrac{
	\dfrac{1}{N_1}\sum_{i=1}^{N_1}\frac{f_+(\x_i)\pi(\x_i)}{q_{1}(\x_i)} - 
	\dfrac{1}{N_2}\sum_{j=1}^{N_2}\frac{f_-(\widetilde{\x}_j)\pi(\widetilde{\x}_j)}{q_{2}(\widetilde{\x}_j)}
}{
	\dfrac{1}{N_3}\sum_{k=1}^{N_3}\frac{\pi(\z_k)}{q_{3}(\z_k)}},
\end{align}
where $\x_i \sim q_{1}(\x)$ ($i=1,\dots,N_1$), $\widetilde{\x}_j \sim q_{2}(\x)$ ($j=1,\dots,N_2$) and $\z_k \sim q_{3}(\x)$ ($k=1,\dots,N_3$).
The estimator above with three generic proposal pdfs $q_i$, with $i=1,2,3$, is generally more efficient than SNIS when there is significant mismatch between $\pi(\x)$ and $f(\x)\pi(\x)$, since in that scenario, it is difficult to find a proposal that produces low variance estimates of both numerator and denominator \cite{rainforth2020target}. 
Regarding the optimal choices of the three proposal pdfs, we can use the two proposals $q_{1,\text{opt}}(\x)$ and $q_{2,\text{opt}}(\x)$ as in Eq. \eqref{eq_2_q_opt} for building a zero variance estimators for $E^+$ and $E^-$, and take a third optimal proposal as $q_{3,\text{opt}}(\x) \propto \pi(\x)$ for estimating $Z$, i.e.,
\begin{align}
\fbox{$q_{1,\text{opt}}(\x) \propto f_+(\x)\post(\x),$} \quad	\fbox{$q_{2,\text{opt}}(\x) \propto f_-(\x)\post(\x).$}  \quad \text{and} \quad \fbox{$q_{3,\text{opt}}(\x) \propto \pi(\x).$}
\end{align}
Note that $\widehat{I}_{\text{IS}-2q}$ and $\widehat{I}_{\text{SNIS}-3q}$ can achieve zero variance with suitable choices of proposal pdfs, contrary to standard IS, where we could only have zero variance when $Z$ is known and $f(\x)$ is either non-positive or non-negative. Table \ref{TablaIS_superSUM} summarizes the different optimal IS schemes for the scalar integral $I=\int_\Theta f(\x)\post(\x)d\x$, considering the possible use of a different numbers of proposal densities.
%{\color{red}
%\subsection{Summary: several strategies for estimating one integral $I$ by IS}\label{sec_obs}
%
%In this section, we can do a summary of the results in the previous sections.
%Recall the (scalar) integral of interest
%\begin{align}\label{eq_I_of_interest0_2}
%I = \int_\Theta  f(\x)\post(\x)d\x=\frac{1}{Z}\int_\Theta f(\x)\pi(\x)d\x,
%\end{align}
% Looking $I$ as unique/sole integral, if $Z$ is known,
%the optimal proposal densities are
% \begin{align}\label{OptimalProposalStandIS_2}
%\fbox{$q_\text{opt}(\x) \propto |f(\x)|\post(\x).$}
%\end{align}
%or, if $Z$ is unknown, 
%\begin{align}\label{eq_q_opt_SNIS_2}
%\fbox{$q_\text{opt}(\x) \propto |f(\x) - I|\post(\x)$,}
%\end{align} 
%
% Looking $I$ as formed by two integrals,  i.e.,
% $$
% I =
% $$
%
%Important observation:
%I can use one proposal, two proposals, or three proposals....

\begin{table}[!h]	
	\caption{Different optimal IS approximations for the scalar integral $I=\int_\Theta f(\x)\post(\x)d\x$. The column regarding the possible reachable zero variance takes into account a generic non-constant function $f(\x)$ (that takes both positive and negative values), and the use of the optimal proposal pdfs. }\label{TablaIS_superSUM}
	\vspace{-0.5cm}
	\begin{center}
		\begin{tabular}{|c|c|c|c|} 
			\hline
%			\multicolumn{6}{|c|}{ } \\
%			\multicolumn{6}{|c|}{ $\widehat{Z}_{IS1} = \frac{1}{N} \sum_{i=1}^N\frac{g(\x_i)}{q(\x_i)} \ell(\y|\x_i)=\frac{1}{N}\sum_{i=1}^N \rho_i \ell(\y|\x_i)$, \quad  $\rho_i=\frac{g(\x_i)}{q(\x_i)}$}\\
%			\multicolumn{6}{|c|}{ } \\ 
%%			\hline
{\bf $Z$ known}		& {\bf Identity}  & {\bf Optimal proposals} & {\bf Min. zero variance} \\
			\hline
		        \hline
		              && & only if $f$ is positive \\ 
\checkmark		& $I= \E_{q}\left[\frac{f(\x)\post(\x)}{q(\x)}\right]$ &  $q_\text{opt}(\x) \propto |f(\x)|\post(\x)$  & or negative \\
		       && & \\ 
\multirow{2}{*}{\checkmark}		& 
		\multirow{2}{*}{$I= \E_{q_1}\left[\frac{f_+(\x)\post(\x)}{q_1(\x)}\right] - \E_{q_2}\left[\frac{f_-(\x)\post(\x)}{q_2(\x)}\right]$}
		 & $q_{1,\text{opt}}(\x) \propto f_+(\x)\post(\x)$ & \multirow{2}{*}{\checkmark} \\
		 && $q_{2,\text{opt}}(\x) \propto f_-(\x)\post(\x)$  & \\ 
		 	 &&  &   \\ 
		 \hline
		 \hline
				 && & \\ 
\ding{55}	&	 $ 
 I =  \frac{\E_{q}\left[\frac{f(\x)\pi(\x)}{q(\x)}\right]}{\E_{q}\left[\frac{\pi(\x)}{q(\x)}\right]}
 $ &  $q_\text{opt}(\x) \propto |f(\x) - I|\post(\x)$ & \ding{55}\\	
				 && & \\ 
\multirow{2}{*}{\ding{55}}				 &
				\multirow{2}{*}{ $ I =  \frac{\E_{q_1}\left[\frac{f(\x)\pi(\x)}{q_1(\x)}\right]}{\E_{q_2}\left[\frac{\pi(\x)}{q_2(\x)}\right]}
 $		}		 
				 & $q_\text{1,opt}(\x) \propto |f(\x)|\post(\x)$ & only if $f$ is positive \\ 
	&	  &  $q_\text{2,opt}(\x) \propto \post(\x)$ & or negative \\		 
		 &&  &\\ 
\multirow{3}{*}{\ding{55}}		  &
		\multirow{3}{*}{    $ I =  \frac{\E_{q_1}\left[\frac{f_+(\x)\pi(\x)}{q_1(\x)}\right] - \E_{q_2}\left[\frac{f_-(\x)\pi(\x)}{q_2(\x)}\right]}{\E_{q_3}\left[\frac{\pi(\x)}{q_3(\x)}\right]}$ }		  
		  & $q_{1,\text{opt}}(\x) \propto f_+(\x)\post(\x)$ &\\ 
	&& $q_{2,\text{opt}}(\x) \propto f_-(\x)\post(\x)$  & \checkmark	  \\
  && $q_\text{3,opt}(\x) \propto \post(\x)$ &\\ 
   &&  & \\ 
			\hline
		\end{tabular}
	\end{center}
	\end{table}










\subsection{Multiple importance sampling (MIS): optimal weights and sampling scheme}\label{sec_MIS}

%{\color{red}no se si esta sect pega mucho en este paper... aqui solo se habla de como muestrear y pesar.... quiza seria mejor hacer una sect pequen\~a donde hablemos de ``other optimality criteria in IS''....}

So far, we have found optimal proposal densities in order to minimize the MSE in approximation of an integral (or several integrals). Here, we show the optimal form of the importance weights and optimal sampling scheme to reduce the variance of the final IS estimators when multiple proposal pdfs are employed. Below, we describe {\it the optimal sampling and weighting} in this scenario. 
%We consider to use of $R$ pro
\newline
\newline
{\bf Optimal sampling.} Regarding the sampling, the best strategy is to employ the {\it deterministic mixture approach} \cite{CORNUET12,EfficientMIS,HereticalMIS,ElviraMIS15}.
This scheme can be used each time the number of samples $N$ is a multiple of the number $R$ of proposal densities, $N=KR$ where $K$ is an integer. 
Indeed, let us consider the joint use of $R$ different proposal pdfs $q_r(\x)$,  and  we can draw one sample from each one, i.e.,
$$
\x_{r,k} \sim q_r(\x), \qquad \mbox{for  $r=1,...,R$ and $k=1,...,K$}.
$$
Collecting all these samples $\{\x_{r,k}\}$ in the same ``urn''  and using  the samples $\{\x_{r,k}\}$  all together indiscriminately, they are distributed according to the mixture of $q_r$'s with equal weights. Clearly, we have avoided the random selection of the components so that this strategy has less variance with respect to the standard one. The deterministic mixture approach can be also applied when the weights  of the mixture are not equal {\it but} are rational numbers (i.e., they can still expressed as fractions). In that case, the number of samples from each proposal density should be different (according to the weigths).
\newline
\newline
{\bf Proper weighting schemes.}  In a multiple proposal scenario, different {\it proper} importance weights can be employed \cite{ElviraMIS15,HereticalMIS,EfficientMIS}, i.e.,
\begin{eqnarray}
w_{r,k}=\frac{\pi(\x_{r,k})}{\psi(\x_{r,k})},
\end{eqnarray}
which differ for the possible denominator $\psi(\x_{r,k})$. The easiest and cheaper possibility (but the worst in terms of performance) is the classical choice $\psi(\x)=q_r(\x)$. For other possible choices see \cite{ElviraMIS15,HereticalMIS,EfficientMIS}.
\newline
\newline
{\bf Optimal weighting.}  Considering the sampling scheme above (with the same number of samples $K$ per proposal), it is possible to show that the best choice for the denominator \cite{ElviraMIS15,he2014optimal,Veach95}, in terms of minimum variance of the resulting estimator, is 
\begin{eqnarray}
\psi_{\texttt{opt}}(\x_{m,r})=\frac{1}{RK} \sum_{K=1}^{K} \sum_{r=1}^{R} q_i(\x_{r,k}),
\end{eqnarray}
which is usually called {\it full-deterministic mixture} denominator ({\it f}-DM).
Hence, the optimal MIS estimators employ the weights 
\begin{eqnarray}
w_{r,k}^{\texttt{(opt)}}=\frac{\pi(\x_{r,k})}{\frac{1}{MK} \sum_{K=1}^{K} \sum_{r=1}^{R} q_i(\x_{r,k})}.
\end{eqnarray}
It can be shown that $\text{Var}[\widehat{I}_\text{ful-DM}] \leq \text{Var}[\widehat{I}_\text{MIS}]$ (where $\widehat{I}_\text{ful-DM}$ uses the optimal weights $w_{r,k}^{\texttt{(opt)}}$ above) for any $\widehat{I}_\text{MIS}$ that is built using other valid sampling and weighting strategy \cite{ElviraMIS15}. 






















%%%%%%%%%%%%%%%%%%%%%%%%%%%%%%%%%%%%
\section{Optimal proposal with noisy evaluations of the target density}\label{NoisyIS_schemes}
%%%%%%%%%%%%%%%%%%%%%%%%%%%%%%%%%%%%

In many applications, the direct pointwise evaluation of  $\pi(\x)$ is not possible \cite{LLORENTEnoisyIS,LlorenteABC_RF}. In this section, we deal with noisy evaluations $\widetilde{\pi}(\x)$ that is related to $\pi(\x)$, instead of direct evaluations of $\pi(\x)$. More specifically,  $\widetilde{\pi}(\x)$ is a random variable and we have access to realizations of this random variable. Furthermore,  let denote with
$$
m(\x) = \E[\widetilde{\pi}(\x)|\x], \quad \text{and} \quad s(\x)^2 = \text{Var}[\widetilde{\pi}(\x)|\x],
$$
 the expectation and variance of $\widetilde{\pi}(\x)$ respectively, given a fixed value of $\x$.  Then, we can consider the following noisy IS estimators
\begin{align}
	\widetilde{Z}=\frac{1}{N}\sum_{n=1}^N \frac{\widetilde{\pi}(\x)}{q(\x)}, 
\end{align}
and
\begin{align}
	%\label{eq_std_IS}
	\widetilde{{\bf I}}_\text{IS}=\frac{1}{N \bar{Z}} \sum_{n=1}^N \frac{\widetilde{\pi}(\x)}{q(\x)} {\bf f}(\x_n),  \qquad 
	\widetilde{{\bf I}}_\text{SNIS}=\frac{1}{N \widetilde{Z}} \sum_{n=1}^N \frac{\widetilde{\pi}(\x)}{q(\x)} {\bf f}(\x_n). \label{AquiSelfNormISnoisy}
\end{align}
where $\bar{Z} =\int_{\Theta}  m(\x) d\x$. The above estimators converge, respectively, to \cite{LLORENTEnoisyIS,tran2013importance,fearnhead2010random}
\begin{equation}\label{Goal_in_Integral}
	\bar{Z}=\int_{\Theta}  m(\x) d\x, \quad \bar{{\bf I}}=\frac{1}{\bar{Z}}\int_{\Theta} {\bf f}(\x) m(\x) d\x.
\end{equation}
{\bf Unbiased scenario.} In the unbiased case, we would have $\E[\widetilde{\pi}(\x)|\x]=m(\x)=\pi(\x)$, we have $\bar{Z}=Z$ in Eq. \eqref{eq_Z_of_int} and $\bar{{\bf I}} = {\bf I}$ in Eq. \eqref{eq_I_of_interest}.
\newline
As in the non-noisy framework, the estimator $\widetilde{{\bf I}}_\text{IS}$ requires the knowledge of $\bar{Z}$, that is not needed in the so-called self-normalized estimator, $\widetilde{{\bf I}}_\text{SNIS}$.
In the following, we show the optimal proposals for $\widetilde{Z}$, $\widetilde{{\bf I}}_\text{IS}$ and $\widetilde{{\bf I}}_\text{SNIS}$.
%Below, we describe the optimality of the proposal for the estimators above.

%%%%%%%%%%%%%%%%%%%%%%%%%%%%%
\subsection{Optimal proposal pdf for estimating $\bar{Z}$}
%%%%%%%%%%%%%%%%%%%%%%%%%%%%%
The variance of $\widetilde{Z}$ (w.r.t. the samples and the noisy realizations) is given by \cite{LLORENTEnoisyIS},
\begin{align}
	\mbox{Var}[\widetilde{Z}] = \frac{1}{N}\mathbb{E}\left[\frac{m(\x)^2+s(\x)^2}{q(\x)^2}\right] - \frac{1}{N}\bar{Z}^2.
\end{align}
The minimum variance, denoted as $\mbox{V}_\text{opt}$, is attained at
\begin{align}
	\fbox{$q_\text{opt}(\x)= \frac{1}{\widetilde{C}_q}\sqrt{m(\x)^2+s(\x)^2}\propto \sqrt{m(\x)^2+s(\x)^2}$},
\end{align}
where $\widetilde{C}_q=\int_\Theta  \sqrt{m(\x)^2+s(\x)^2}d\x$.
Note that $\mbox{V}_\text{opt}=\min_q\mbox{Var}[\widetilde{Z}]$ is always greater than 0, specifically,
\begin{align}
\mbox{V}_\text{opt} &= \frac{1}{N}\mathbb{E}\left[\widetilde{C}_q^2\right] - \frac{1}{N}\bar{Z}^2, \nonumber\\
&=\frac{1}{N}\widetilde{C}_q^2 - \frac{1}{N}\bar{Z}^2,\nonumber \\
&=\frac{1}{N}\left[\int_\Theta  \sqrt{m(\x)^2+s(\x)^2}d\x\right]^2 - \frac{1}{N}\bar{Z}^2.
\end{align}
Hence, differently from the non-noisy setting in Section \ref{Sect_optZ}, in the noisy IS scenario the optimal estimator of $Z$ does not reach a null variance, i.e., is not equal to 0, as long as $s(\x)$ is not null everywhere.  Recall that with $s(\x)=0$ we recover the non-noisy scenario. 
%\newline
%\noindent{\bf Illustration of $q_\text{opt}$.} Assume a multiplicative noisy $\widetilde{\pi}(\x) = \epsilon \pi(\x)$ with Var$[\epsilon]=\sigma^2$, hence $s(\x)^2 = \pi(\x)^2\mbox{Var}[\epsilon] = \sigma^2 \pi(\x)^2$. In this case, the optimal proposal coincides with the optimal one in the non-noisy setting, since 
%\begin{align}
%	q_\text{opt} &\propto \sqrt{\sigma^2\pi^2(\x)+\pi^2(\x)} = \pi(\x)\sqrt{1+\sigma^2}\\
%	&\propto \post(\x).
%\end{align}
%As a second example, Let us consider a Bernoulli-type noise where $\widetilde{\pi}(\x) = \epsilon\pi_\text{max}$, and $\epsilon \sim \mbox{Bernoulli}\left(\frac{\pi(\x)}{\pi_\text{max}}\right)$, where $\pi_\text{max} = \max_{\x} \pi(\x)$. Then, $s^2(\x) = \pi(\x)(\pi_\text{max}-\pi(\x))$, and the optimal proposal is
%\begin{align}
%		q_\text{opt}(\x) \propto \pi(\x)\sqrt{1 + (\pi_\text{max}-\pi(\x))^2}.
%\end{align}
%\newline
%%%%%%%%%%%%%%%%%%%%%%%%%%%%%
\subsection{Optimal proposal for standard noisy IS}
%%%%%%%%%%%%%%%%%%%%%%%%%%%%%

%We have already seen that the optimal proposal that minimizes the variance of $\widetilde{Z}$ is $q_\text{opt}(\x) \propto \sqrt{m(\x)^2 + s(\x)^2}$. 
Let us consider now the estimator $\widetilde{{\bf I}}_\text{IS}$. Note that this estimator assumes we can evaluate $\bar{Z}=\int_{\Theta} m(\x)d\x$. 
Since we are considering a vector-valued function, the estimator has $P$ components  $\widetilde{{\bf I}}_\text{IS}=[\widetilde{I}_{\text{IS},1} \dots \widetilde{I}_{\text{IS},P}]^\top$, and $\text{Var}[\widetilde{{\bf I}}_\text{IS}]$ corresponds to a $P \times P$ covariance matrix.
We aim to find the proposal that minimizes the sum of diagonal variances. From the results of the previous section, it is straightforward to show that the variance of the $p$-th component is
\begin{align*}
	\text{Var}[\widetilde{I}_{\text{IS},p}] = \frac{1}{N\bar{Z}^2}\E\left[\frac{f_p(\x)^2(m(\x)^2+s(\x)^2)}{q(\x)^2}\right] - \frac{1}{N\bar{Z}^2}\bar{I}_p^2,
\end{align*}
where $f_p(\x)$ and $\bar{I}_p$ are respectively the $p$-th components of ${\bf f}(\x)$ and $\bar{{\bf I}}$.
Thus,
{\footnotesize$$
	\sum_{p=1}^{P} \text{Var}[\widetilde{I}_{\text{IS},p}] = \frac{1}{N\bar{Z}^2}\E\left[\frac{\sum_{p=1}^{P} f_p(\x)^2(m(\x)^2+s(\x)^2)}{q(\x)^2}\right] - \frac{1}{N\bar{Z}^2}\sum_{p=1}^{P} \bar{I}_p^2.
	$$}
%By Jensen's inequality, we have
%{\footnotesize
%	\begin{align*}
%		\mathbb{E}\left[\frac{\sum_{p=1}^{P} f_p(\x)^2(m(\x)^2+s(\x)^2)}{q(\x)^2}\right] 
%		%	&\geq \left(\mathbb{E}\left[\frac{\sqrt{\sum_{p=1}^\chi f_p(\x)^2(m(\x)^2+s(\x)^2)}}{q(\x)}\right]\right)^2 \\
%		&\geq\left(\mathbb{E}\left[\frac{\sqrt{m(\x)^2+s(\x)^2}\norm{{\bf f}(\x)}_2}{q(\x)}\right]\right)^2,
%	\end{align*}
%}where $\norm{{\bf f}(\x)}_2$ denotes the euclidean norm.
%The equality holds if and only if $\frac{\sqrt{m(\x)^2+s(\x)^2}\norm{{\bf f}(\x)}_2}{q(\x)}$ is constant.
Hence, the optimal proposal is
\begin{align}
	\fbox{$q_\text{opt}(\x)\propto \norm{{\bf f}(\x)}_2\sqrt{m(\x)^2+s(\x)^2}.$}
\end{align}
%\newline
%{\bf Optimal proposal for $\widetilde{{\bf I}}_{SNIS}$.}
%%%%%%%%%%%%%%%%%%%%%%%%%%%%%
\subsection{Optimal proposal for self-normalized noisy IS}
%%%%%%%%%%%%%%%%%%%%%%%%%%%%%
Let us consider the case of the self-normalized estimator $\widetilde{{\bf I}}_\text{SNIS}$.
Recall that $\widetilde{{\bf I}}_\text{SNIS} = \frac{\widetilde{{\bf E}}}{\widetilde{Z}}$,
where $\widetilde{{\bf E}}$ denotes the noisy estimator of ${\bf E} = \int_{\Theta} {\bf f}(\x)m(\x)d\x$, so that we are considering ratios of estimators.
Again, we aim to find the proposal that minimizes the variance of the vector-valued estimator $\widetilde{{\bf I}}_\text{SNIS}$.
When $N$ is large enough, the variance of $p$-th ratio is approximated  as \cite{LLORENTEnoisyIS}
\begin{align*}
	\text{Var}_q[\widetilde{I}_{\text{self},p}] = \text{Var}_q\left[\frac{\widetilde{E}_p}{\widetilde{Z}}\right] 
	\approx \frac{1}{\bar{Z}^2}\text{Var}_q[\widetilde{E}_p] - 2\frac{E_p}{\bar{Z}}\text{Cov}_q[\widetilde{E}_p,\widetilde{Z}] + \frac{E_p^2}{\bar{Z}^4}\text{Var}_q[\widetilde{Z}],
\end{align*}
where $E_p$ is the $p$-th component of ${\bf E}$, and it is possible to show that 
\begin{align*}
	\text{Var}_q[\widetilde{E}_p] &= \frac{1}{N}\E\left[\frac{f_p(\x)^2(m(\x)^2+s(\x)^2)}{q(\x)^2}\right] - \frac{1}{N}E_p^2, \\
	\text{Var}_q[\widetilde{Z}] &= \frac{1}{N}\E_q\left[\frac{m(\x)^2+s(\x)^2}{q(\x)^2}\right] - \frac{1}{N}\bar{Z}^2, \\
	\text{Cov}_q[\widetilde{E}_p,\widetilde{Z}] &= \frac{1}{N}\E_q\left[\frac{f_p(\x)(m(\x)^2+s(\x)^2)}{q(\x)^2}\right] - \frac{1}{N}E_p\bar{Z}.
\end{align*}
%The first two results have been already obtained in the previous sections. The third result is given in Appendix \ref{App1}.
The sum of the variances is thus
\begin{align*}
	\sum_{p=1}^{P}\text{Var}_q[\widetilde{I}_{\text{self},p}] \approx \frac{1}{N\bar{Z}^2}\E_q\left[\frac{(m(\x)^2+s(\x)^2)\sum_{p=1}^{P}(f_p(\x) - \bar{I}_p)^2}{q(\x)^2}\right],
\end{align*}
and, by Jensen's inequality, we obtain that the optimal proposal density is 
\begin{align}
	\fbox{$q_\text{opt}(\x) \propto \norm{{\bf f}(\x) - \bar{{\bf I}}}_2\sqrt{m(\x)^2 + s(\x)^2}.$}	
\end{align}


















%%%%%%%%%%%%%%%%%%%%%%%%%%%%%%%%%%%%%%%%%%%%%%
\section{Optimality in IS schemes for computing the evidence $Z$}\label{sec_marglike}
%%%%%%%%%%%%%%%%%%%%%%%%%%%%%%%%%%%%%%%%%%%%%
In this section, we focus on the computation of normalizing constants or ratios of normalizing constants. From a practical point of view, these problems appear in the computation of marginal likelihoods,  $Z= \int_\Theta \pi(\x)d\x$, and/or Bayes factors, $Z_1/Z_2$  \cite{gelman1998simulating,llorenteREV_ML,meng2002warp}.
The methods in this section rely on different identities, some of them using multiple proposal pdfs. In some cases, $\post(\x) = \frac{\pi(\x)}{Z}$ is itself employed as a proposal density, from which samples are drawn. 
Clearly, in this scenario, we imply the use of MCMC algorithms or other Monte Carlo schemes from drawing from $\post(\x)$.
\newline
\newline
We recall that  $\widehat{Z}_{\text{IS}}$ in Eq. \eqref{eq_Z_est} is the simplest estimator of $Z= \int_\Theta \pi(\x)d\x$, and its variance is given by
\begin{align}\label{eq_var_Z2}
\text{Var}_q[\widehat{Z}_{\text{IS}}] = \frac{1}{N}\E_q\left[\frac{\pi(\x)^2}{q(\x)^2}\right] - \frac{1}{N}Z^2.
\end{align}
We also recall that optimal proposal pdf in this case is  \fbox{$q_\text{opt}(\x)\propto \pi(\x)$}. Here, we discuss different concepts of optimality of specific IS schemes specifically devoted to the approximation of $Z$, and can improve in some way the performance of $\widehat{Z}$ in Eq. \eqref{eq_Z_est}. 


%%%%%%%%%%%%%%%%%%%%%%%%%%%%
\subsection{Reverse Importance Sampling (RIS)}\label{RISsect} 
%%%%%%%%%%%%%%%%%%%%%%%%%%%%%
It is also possible to estimate $Z$ using the so-called reverse importance sampling, also known as {\it reciprocal} IS \cite{gelfand1994bayesian,llorenteREV_ML}.
The RIS scheme can be derived from the identity   
\begin{align}\label{ReverseISidentity}
\frac{1}{Z} =\E_{\post}\left[ \frac{\varphi(\x)}{\pi(\x)} \right] =\int_{\Theta}\frac{\varphi(\x)}{\pi(\x)}\post(\x)d\x= \frac{1}{Z}\int_{\Theta} \varphi(\x) d\x, 
\end{align}
where  we consider an auxiliary normalized density $\varphi(\x)$, i.e., $\int_{\Theta} \varphi(\x) d\x=1$. Then, one could consider the estimator
\begin{align}\label{ReverseIS}
\widehat{Z}_\text{RIS} = \left(\frac{1}{N}\sum_{i=1}^N\frac{\varphi(\x_i)}{\pi(\x_i)}\right)^{-1},
%=\left(\frac{1}{N}\sum_{i=1}^N\frac{\varphi(\x_i)}{\ell(\y|\x_i)g(\x_i)}\right)^{-1}
\quad \x_i \sim \post(\x).
\end{align}
The samples $ \x_i \sim \post(\x)$ can be obtained approximately by an MCMC algorithm, for instance. See also Figure \ref{USfig}.
\newline
\newline
 The expression $\frac{1}{N}\sum_{i=1}^N\frac{\varphi(\x_i)}{\pi(\x_i)}$ is an unbiased estimator of $1/Z$.   The estimator $\widehat{Z}_\text{RIS} $ above is consistent but, however, is a biased estimator of $Z$.
Here,  $\post(\x)$ plays the role of the proposal density from which we need to draw from. Indeed, in this case,  we do not need samples from $\varphi(\x)$, although its choice affects the precision of the approximation \cite{llorenteREV_ML}. 
 Unlike in the standard IS approach, $\varphi(\x)$ must have lighter tails than $\pi(\x)$\cite{llorenteREV_ML}.   See the experiment in Section \ref{IS_vs_RIScomp} for more details.  
Taking into account the inverse estimator $\frac{1}{\widehat{Z}_\text{RIS}}$ that is unbiased with respect to $1/Z$, we can write that the variance of $\frac{1}{\widehat{Z}_\text{RIS}}$ is
\begin{align}\label{eq_var_RIS}
\mbox{Var}\left[\frac{1}{\widehat{Z}_\text{RIS}}\right] = \frac{1}{N}\E_{\post}\left[\frac{\varphi(\x)^2}{\pi(\x)^2}\right] - \frac{1}{NZ^2}.
\end{align}
Then, the optimal choice of the auxiliary density $\varphi(\x)$ is
\begin{align}
\fbox{$\varphi_\text{opt}(\x) = \post(\x).$}
\end{align}
 However,  although $\x_i \sim \post(\x)$, recall that  $\varphi_\text{opt}(\x)$ is not the proposal density but, in this scenario, plays the role of an auxiliary/reference pdf. See also Figure \ref{USfig2}.
 %\newline
%\newline



 

%{\color{red}the paper also considers two extensions: (i) unbounded domains, and (ii) general basis functions (not constant)}


%%%%%%%%%%%%%%%%%%%%%%%%%%%%%
%\subsection{ Expressing $Z$ as the ratio of constants} 
%%%%%%%%%%%%%%%%%%%%%%%%%%%%%%
%
%We have seen the standard IS estimator of $Z$ and the RIS estimator based on the estimation of $\frac{1}{Z}$. Here, we consider a more general perspective. 
%Let us assume that we are interested in estimating the ratio of two normalizing constants (as in the computation of Bayes factors),
%\begin{align}
%r = \frac{c_1}{c_2} = \frac{\int_\Theta\widetilde{q}_1(\x)d\x}{\int_\Theta\widetilde{q}_2(\x)d\x},
%\end{align}
%where $q_i(\x)=\frac{\widetilde{q}_i(\x)}{c_i}$, $i=1,2$, denote the normalized pdfs, $\widetilde{q}_i(\x)$ denote the unnormalized pdfs, and $c_i$ represents the normalizing constant of $q_i(\x)$.  
%The following identity, and corresponding estimator,
%\begin{align}\label{OnePropIS}
%r = \E_{q_2}\left[\frac{\widetilde{q}_1(\x)}{\widetilde{q}_2(\x)}\right] \approx \widehat{r} = \frac{1}{N}\sum_{i=1}^{N}\frac{\widetilde{q}_1(\x_i)}{\widetilde{q}_2(\x_i)}, \qquad \x_i\sim q_2(\x),
%\end{align}
%encompasses both standard IS and RIS. Taking $\widetilde{q}_1(\x)=\pi(\x)$ and $c_2=1$, hence $r=Z$ and the estimator $\widehat{r}$ corresponds to standard IS for $Z$ in Eq. \eqref{eq_Z_est}.
%The RIS estimator is obtained by setting $\widetilde{q}_1(\x) = \varphi(\x)$, $c_1=1$, $q_2(\x)=\post(\x)$, $\widetilde{q}_2(\x)=\pi(\x)$ and hence $r=\frac{1}{Z}$. 
%Table \ref{TablaISRIS} summarizes these cases.
% 
% \begin{table}[!h]	
%	\caption{Summary of techniques considering the expression \eqref{OnePropIS}.}\label{TablaISRIS}
%	\vspace{-0.2cm}
%	\begin{center}
%		\begin{tabular}{|c|c| c|c|c|c|c|c|} 
%			\hline 
%		{\bf Scheme} & $\widetilde{q}_1(\x)$  &$\widetilde{q}_2(\x)$ & $c_1$ & $c_2$  &  {\bf Proposal pdf} $q_2(\x)$ &  $c_1/c_2$ & {\bf Optimal function}  \\ 
%			\hline 
%			\hline 
%			Standard IS &  $\pi(\x)$   & $q(\x)$ & $Z$ &  $1$  & $q(\x)$ &   $Z$ & $q_\text{opt}(\x) = \post(\x)$\\
%			RIS   & $\varphi(\x)$  & $\pi(\x)$ & $1$ &  $Z$ & $\post(\x)$ &  $1/Z$ & $\varphi_\text{opt}(\x) = \post(\x)$\\ 
%			\hline
%		\end{tabular}
%	\end{center}
%\end{table}
%
%
%
%\noindent
%For both estimators, we already know that the variance of the estimators is zero when $q_1(\x)$ and $q_2(\x)$ coincide, corresponding to the choice $q_2(\x)=\widetilde{q}_2(\x)=\post(\x)$ in standard IS, and $q_1(\x) = \widetilde{q}_1(\x)= \post(\x)$ in RIS. 
%More generally, the relative mean-squared error of $\widehat{r}$ scales with Pearson divergence between those pdfs \cite{llorente2021deep}
%$$\frac{\E[(r-\widehat{r})^2]}{r}=\frac{1}{N}D_{\chi^2}(q_1,q_2).$$
%%Note that we assume $\widetilde{q}_1(\x)$ and $\widetilde{q}_2(\x)$ are defined over the same dominion. Sometimes this is true in some applications. In other cases, we are only interested in one of the constant, and introduce a second density that is easy to sample and whose normalizing constant is known.
%%There exist two general identities for expressing $r$ as the ratio of expectations,  generalizing Eq. \eqref{OnePropIS}, that give rise to well-known methods for computing normalizing constants/marginal likelihoods, namely, the {\it ratio importance sampling} identity in Eq. \eqref{eq_umbrella_para_Z} and the {\it brige sampling} identity in Eq. \eqref{BridgeSamplingIdentity}.
%%%First, we present a basic identity for building an estimator of $r$, and connect it to standard and reverse IS. 
%%%\newline
%%The identities and estimators that we present below were proposed to improve upon the efficiency of the estimator in Eq. \eqref{OnePropIS}, by using an ``intermediate'' density (between $q_1(\x)$ and $q_2(\x)$). 
%

%%%%%%%%%%%%%%%%%%%%%%%%%%%%%%%%%%
\subsection{Ratio Importance Sampling for $Z$ (a.k.a, umbrella sampling)} 
%%%%%%%%%%%%%%%%%%%%%%%%%%%%%%%%%%
%\noindent{\color{red}{\bf Self-normalized/ratio IS for Z}.} 
%Setting $\widetilde{q}_1(\x)=\pi(\x)$, $q_2(\x)=\widetilde{q}_2(\x) = \varphi(\x)$, $c_2=1$, hence $r=Z$ in Eq. \eqref{eq_US_identity}. Also setting $q_3(\x)=q(\x)$ and $\widetilde{q}_3(\x)=\widetilde{q}(\x)$, the resulting estimator in Eq. \eqref{eq_US_estimator},
Let $\varphi(\x)$ and $q(\x)$ denote two normalized densities, where $\varphi(\x)$ is some normalized auxiliary pdf and $q(\x)$ is the proposal pdf (from which we draw samples from). The following identity expresses $Z$ as the ratio of two expectations, producing the following estimator called ``ratio importance sampling" (and ``umbrella sampling'' in the physics literature) \cite{chen1997monte},
\begin{align}\label{eq_umbrella_para_Z}
Z = \frac{\E_q\left[\frac{\pi(\x)}{q(\x)}\right]}{\E_q\left[\frac{\varphi(\x)}{q(\x)}\right]}\approx\widehat{Z}_\text{ratio} = \frac{\frac{1}{N}\sum_{i=1}^{N}\frac{\pi(\z_i)}{q(\z_i)}}{\frac{1}{N}\sum_{i=1}^{N}\frac{\varphi(\z_i)}{q(\z_i)}},  \qquad {\bf z}_i \sim q(\x).
\end{align}
%where ${\bf z}_i \sim q(\x)$ for $i=1,\dots,N$.
{\rem Since $\varphi(\x)$ and $q(\x)$ are normalized, the denominator above is an estimator of the value $1$. Surprisingly, $\widehat{Z}_\text{ratio}$ can be more efficient than $\widehat{Z}_\text{IS}$ in Eq. \eqref{eq_Z_est} \cite{chen1997monte,llorenteREV_ML} (see below).
}
\newline
\newline
This general estimator encompasses many known estimators of normalizing constants/marginal likelihoods that use samples from one proposal $q(\x)$ \cite{llorenteREV_ML}. For instance, standard IS and RIS are obtained as special cases by setting $\varphi(\x)=q(\x)$ or $q(\x)=\post(\x)$, respectively. Table \ref{TablaUmbrella} shows different techniques as special case of the estimator $\widehat{Z}_\text{ratio}$.  



\begin{table}[!h]	
	 \caption{Famous special cases of the estimator $\widehat{Z}_\text{ratio}$ \cite{llorenteREV_ML}. Recall $g(\x)$ represents a normalized prior density.  }\label{TablaUmbrella}
	 \vspace{-0.2cm}
	\begin{center}
		\begin{tabular}{|c||c|c|} 
		\hline 
			{\bf Methods} & $\varphi(\x)$  & $q(\x)$ \\ 
			 \hline
			 \hline
			 Naive Monte Carlo & $g(\x)$ & $g(\x)$  \\
			 Harmonic Mean  &     $g(\x)$ & $\post(\x)$   \\ 
		% stand IS   & $q(\x)$ & $q(\x)$ (quitar)   \\ 
		 RIS & $\varphi(\x)$ & $\post(\x)$   \\
			\hline
		\end{tabular}
	\end{center}

\end{table}



{\rem The motivation for using the identity \eqref{eq_umbrella_para_Z} is the idea of taking advantage of an intermediate proposal pdf $q(\x)$, that is ``in the middle'' of $\post(\x)$ and $\varphi(\x)$ \cite{chen1997monte,meng1996simulating,gelman1998simulating}. Figure \ref{USfig3} represents the umbrella sampling idea compared to other approaches.
}
\newline
\newline
The optimal choice of the auxiliary density $\varphi(\x)$ is always
\begin{align}\label{optVARphi}
	\fbox{$\varphi_\text{opt}(\x)=\post(\x)$},
\end{align}  
which gives the exact solution $\widehat{Z}_\text{ratio}=Z$, for any choice of $q(\x)$.
Fixing a generic $\varphi(\x)$, the optimal choice of $q(\x)$, that minimizes the asymptotic relative mean-squared error (rel-MSE) of $\widehat{Z}_\text{ratio}$, is \cite{chen1997monte,llorenteREV_ML}
\begin{align}\label{eq:OptProUmbrella}
\fbox{$q_\text{opt}(\x) = \dfrac{|\post(\x) - \varphi(\x)|}{\int_\Theta  |\post(\x') - \varphi(\x')|d\x '} \propto \left|\dfrac{1}{Z}\pi(\x) - \varphi(\x)\right|.$}
\end{align}
With this optimal choice of the (intermediate) proposal pdf $q(\x)$, the relative MSE (rel-MSE) in estimation of $\widehat{Z}_\text{ratio}$  is given by 
\begin{align}
	\mbox{rel-MSE} =\frac{\E\left[(Z - \widehat{Z}_\text{ratio})^2\right]}{Z^2}  &\approx \frac{1}{N} \left[\int_\Theta |\post(\x)-\varphi(\x)|d\x\right]^2, \nonumber \\
	 &\approx \frac{1}{N}L_1^2(\post,\varphi), \quad \mbox{(for $N$  great enough),}
\end{align}
where $L_1(\post,\varphi)$ denotes the $L_1$-distance  between $\post$ and $\varphi$ \cite[Theorem 3.2]{chen1997monte}.  %Note that, in ratio IS, using jointly both $\varphi_\text{opt}(\x)$ in Eq. \eqref{optVARphi} and $q_\text{opt}(\x)$ in Eq. \eqref{eq:OptProUmbrella}, we get
%\begin{equation}
%\fbox{$q_\text{opt}(\x)=\varphi_\text{opt}(\x)= \post(\x),$}
%\end{equation}
%and we obtain a zero variance estimator in this case.


{\rem Since $L_1^2(\cdot,\cdot)\leq D_{\chi^2}(\cdot,\cdot)$ \cite{chen1997monte} and Eq. \eqref{Eqchi0}, the optimal estimator $\widehat{Z}_\text{ratio}$, using $q_\text{opt}(\x)$,  is asymptotically more efficient than a standard IS estimator using $\varphi(\x)$ as proposal, and than a RIS estimator using $\varphi(\x)$ as auxiliary pdf. }
%{\color{red} creo que lo he entendido.., se necesitaria una figura...o un ejemplo....}{\color{blue}MIRAR FIG 1 (b)-(c)}  }
\newline
\newline
However, the pdf $q_{\text{opt}}(\x)$ depends on $Z$ (hence we cannot evaluate it), and $q_{\text{opt}}(\x)$  is not easy to draw from.   In order to implement the optimal umbrella estimator in practice, one can employ the following iterative procedure \cite{chen1997monte}:
%First, use a initial approximation $\widehat{r}_0$ to build and sample $q_{3,\text{opt}}(\x)$. Then, use the samples to obtain the final estimator $\widehat{r}_1$.
%The following two-stage procedure is often used in practice: 
\newline
\newline
- Start with an arbitrary density $q^{(1)}(\x) \propto \widetilde{q}^{(1)}(\x)$.
\newline
- For $t=2,...,T:$
\begin{enumerate}
	 \item Draw $N$ samples from $q^{(t-1)}(\x)$ (using an MCMC or other Monte Carlo method), and use them to obtain 
	\begin{align}\label{eq_TwoStageRatioIS_1}
		\widehat{Z}_\text{ratio}^{(t)} = \frac{\sum_{i=1}^{N}\frac{\pi_1(\x_i)}{\widetilde{q}^{(t-1)}(\x_i)}}{\sum_{i=1}^{N}\frac{\varphi(\x_i)}{\widetilde{q}^{(t-1)}(\x_i)}}, \quad \{\x_i \}_{i=1}^{N} \sim q^{(t-1)}(\x),
	\end{align}
	\item Set
	\begin{align}\label{eq_TwoStageRatioIS_2}
		q^{(t)}(\x) \propto \widetilde{q}^{(t)}(\x) = |\pi(\x) - \widehat{Z}_\text{ratio}^{(t)}\varphi(\x)|.
	\end{align}
%	\item  	
%	{\it Stage 2}: Draw $N_2$ samples from $q^{(t)}(\x)$ via, e.g., MCMC and compute
%	\begin{align}\label{eq_TwoStageRatioIS_3}
%		\widehat{Z}_\text{ratio}^{(t+1)} = \frac{\sum_{i=1}^{N_2}\frac{\pi(\x_i)}{\widetilde{q}^{(2)}(\x_i)}}{\sum_{i=1}^{N_2}\frac{\varphi(\x_i)}{\widetilde{q}^{(2)}(\x_i)}}, \quad \{\x_i \}_{i=1}^{N_2} \sim q^{(2)}(\x).
	%\end{align}
\end{enumerate}
%\newline
%\newline
A graphical representation of umbrella sampling is given in Figure \ref{USfig3}. Fixing the reference pdf $\varphi$, the proposal pdf $q$ is a pdf ''in between'' $\varphi$ and $\post$.



%%%%%%%%%%%%%%%%%%%%%%%%%%%%
\subsection{Bridge sampling} \label{BridgeSect}
%%%%%%%%%%%%%%%%%%%%%%%%%%%%%
 In the previous section, devoted to umbrella sampling, we have employed two densities which play the role of the proposal, $q$, and of an auxiliary reference pdf, $\varphi$. We only draw samples from the proposal pdf $q$. From Eq. \eqref{eq:OptProUmbrella}, we can interpreted that the reference pdf and the posterior act as two ``extremes'' (using the analogy of an interval),  and the proposal pdf represents a function  ``in between'' of both extremes.  This is graphically shown in Figure \ref{USfig3}.
\newline
In this section, we describe the bridge sampling technique. Using the same analogy, in this case  the ``extremes'' are the proposal, $q$, and the posterior $\post$. The density  ``in between'' , used as a ``bridge'', is the auxiliary pdf  $\varphi$. Another difference with umbrella sampling is that here we generated from both  ``extremes'' i.e., from $q$, and $\post$. This is depicted in Figure \ref{USfig4}.
% which considers the opposite case: we sample from the ``extremes'' (being one of the proposal pdf and the other one the posterior pdf), and use a ``bridge'' (i.e., an ``intermediate'' density) to improve performance. 
\newline
\newline
Bridge sampling is a technique for computing ratios of constants by using samples drawn from their corresponding densities.  It is based on other identity that can be adapted for computing a single constant, $Z$, as follows \cite{meng1996simulating,llorenteREV_ML}
\begin{align}\label{BridgeSamplingIdentity}
Z = \frac{\E_{q}\left[\frac{\varphi(\x)}{q(\x)}\right]}{\E_{\post}\left[\frac{\varphi(\x)}{\pi(\x)}\right]},
\end{align}
%where $q(\x)$ is the proposal pdf, and $\varphi(\x)$ is an arbitrary pdf  defined on the intersection of the supports of $q(\x)$ and $\post(\x)$. 
and the corresponding estimator uses samples from  $q(\x)$ and $\post(\x)$,
\begin{align}\label{eq_BS_para_Z}
\widehat{Z}_\text{bridge} = \frac{\frac{1}{N_2}\sum_{i=1}^{N_2}\frac{\varphi(\z_i)}{q(\z_i)}}{\frac{1}{N_1}\sum_{j=1}^{N_1}\frac{\varphi(\x_j)}{\pi(\x_j)}}, \quad \x_j\sim\post(\x), \quad \z_i \sim q(\x),
\end{align}
where $j=1,\dots,N_1$ and $i=1,\dots,N_2$. 
The function $\varphi(\x)$ is an arbitrary density, defined on the intersection of the supports of $q(\x)$ and $\post(\x)$.  Note that $\varphi(\x)$ can be evaluated up to a normalizing constant, i.e., it can be an unnormalized pdf.
\newline
\newline
For any $\varphi(\x)$, the optimal proposal is 
\begin{align}
	\fbox{$q_\text{opt}(\x) \propto \pi(\x),$}
\end{align} 
which produces the exact solution $\widehat{Z}_\text{bridge}=Z$ (i.e., a zero variance solution).
Regarding $\varphi(\x)$, keeping fixed a generic $q(\x)$,  the asymptotic relative mean-squared error (rel-MSE) is minimized by the choice 
\begin{align}\label{eq_optBridgeBS}
	\fbox{$\varphi_\text{opt}(\x) = \dfrac{1}{\frac{N_2}{N_1+N_2}\post(\x)^{-1} + \frac{N_1}{N_1+N_2}q(\x)^{-1}} \propto \dfrac{q(\x)\pi(\x)}{N_1\pi(\x) + N_2Zq(\x)},$}
\end{align}
which is a weighted harmonic mean of the ``extreme'' densities $q$ and $\post$. 

{\rem Also in bridge sampling, employing jointly both $q_\text{opt}(\x)$ and $\varphi_\text{opt}(\x)$, we have
\begin{equation}
\fbox{$q_\text{opt}(\x)=\varphi_\text{opt}(\x)\propto \pi(\x),$}
\end{equation}
and we obtain a zero variance estimator.
}
\newline
\newline
Since $q_\text{opt}(\x)$ and $\varphi_\text{opt}(\x)$ depends on the unknown quantity $Z$, in order to use the optimal bridge sampling estimator we need again to use  an iterative procedure \cite{meng1996simulating}.  Starting with an initial estimate $\widehat{Z}^{(0)}$, we iteratively update it as  
\begin{align}
	\label{IterativeBS}
	\widehat{Z}^{(t)} = \frac{\frac{1}{N_2}\sum_{i=1}^{N_2}\dfrac{\pi(\z_i)}{N_1\pi(\z_i) + N_2\widehat{Z}^{(t-1)}q(\z_i)}}{\frac{1}{N_1}\sum_{i=1}^{N_1}\dfrac{q(\x_i)}{N_1\pi(\x_i) + N_2\widehat{Z}^{(t-1)}q(\x_i)}},\quad  \text{for} \quad t=1,...,T,
\end{align}
where $\{ \z_i \}_ {i=1}^{N_2} \sim q(\x)$ and $ \{\x_i\}_{i=1}^{N_1} \sim \post(\x)$. 

{\rem In the iterative procedure above, note that the sampling part and the evaluations of $\pi(\x)$ and $q(\x)$ are performed only once.}
\newline
\newline
The authors in \cite{meng1996simulating} demonstrate that this iterative scheme has a unique limit, and that achieves the same optimal variance of the optimal bridge sampling estimator.


%%%%%%%%%%%%%%%%%%%%%%%%%%%%%%%
\section{Some numerical and theoretical comparisons}
%%%%%%%%%%%%%%%%%%%%%%%%%%%%%%%

This section is devoted to provide some numerical and theoretical comparisons, in order to highlight the importance of the notion of optimality in IS. We can avoid catastrophic situations (when the variance of the estimators explodes to infinity), and improve the baseline, ideal Monte Carlo scenario. Theoretical and numerical results are provided and checked. In section \ref{FirstSectNum}, we consider different functions $f(\x)$ and different proposal densities $q(\x)$, including the optimal ones. In section \ref{IS_vs_RIScomp}, we focus on the comparison between IS and RIS for the estimation of the marginal likelihood $Z$. 

%%%%%%%%%%%%%%%%%%%%%%%%%%%%%%%%%%%%%
\subsection{The reason why IS is a variance reduction method}\label{FirstSectNum}
%%%%%%%%%%%%%%%%%%%%%%%%%%%%%%%%%%%%%
This section is divided in two parts. In the first part,  we show the shape of optimal densities considering different functions $f$  (different integrals), and having the same target density $\bar{\pi}$. In the second part, we show that the IS estimators can have better performance (in terms of smaller MSE) than the ideal Monte Carlo, using the optimal proposal density (or a proposal close to the optimal one).  %In the last part, we show that the IS estimators can beat the ideal Monte Carlo even with a proposal different from the optimal one, but close to it (in some sense). 
\newline
\newline
{\bf First part.} For the sake of simplicity, let us consider a one-dimensional Gaussian target  distribution, i.e.,
$$
\bar{\pi}(\theta)=\frac{1}{\sqrt{2\pi}} \exp\left( -\frac{(\theta+1)^2}{2}\right),
$$
i.e., with mean $\mu=-1$ and variance $\sigma^2=1$. We assume the following integrals of interest,
\begin{align*}
I_1 = \int_\Theta  \theta\post(\theta)d\theta, \quad  I_2 = \int_\Theta  \sqrt{|\theta|}\post(\theta)d\theta, \quad I_3 = \int_\Theta  \theta^2\post(\theta)d\theta,
\end{align*}
i.e., $f_1(\theta)=\theta$, $f_2(\theta)=\sqrt{|\theta|}$ and $f_3(\theta)=\theta^2$, respectively. The optimal proposals for the standard IS and the SNIS schemes are, respectively,
\begin{align*}
q_\text{opt}(\theta) \propto |f_k(\theta)|\post(\theta), \quad \mbox{ and } \quad q_\text{opt}(\theta) \propto |f_k(\theta)-I_k|\post(\theta), \quad k=1,2,3.
\end{align*}
The corresponding optimal proposal densities are depicted in Figures \ref{LucaEx1}, \ref{LucaEx2} and \ref{LucaEx3}. If compared with the ideal MC  (where the proposal coincides with the target $\post(\theta)$), their shapes are quite surprising, since some of them present regions of low probabilities around the mode of $\post(\theta)$. More generally, they differ substantially to the shape of the target density $\post(\theta)$: for instance, all of them are at least bimodal (in Figures \ref{Tremodes1}-\ref{Tremodes2}, there are three modes), instead of just unimodal as $\post(\theta)$.
\newline
\newline
{\bf Second part.} Assuming now $f(\theta)=\theta$, we compute the theoretical effective sample size (ESS) \cite{ESSmartino,ESSelvira} defined as 
$$
\mbox{ESS}=N \cdot \frac{\mbox{MSE of ideal MC}}{\mbox{MSE of $\widehat{I}_{\text{SNIS}}$}}.
$$
We have always that $\mbox{ESS}>0$ and, with a bad or regular choice of the proposal $q$, we generally have $\mbox{ESS} < N$, i.e., the  ideal MC performs better than an IS scheme. However, it is possible to obtain $\mbox{ESS} \geq N$. Indeed, we show that with a good choice of the proposal density $q$, the IS estimators can have better performance that the baseline MC estimators, so that we obtain $\mbox{ESS}\geq N$. This is the reason why IS is often included within the class of variance reduction techniques \cite{Arouna2004,Lapeyre2011,owen2013monte}. Firstly, we employ the optimal proposal $q_{\text{opt}}(\theta)$ for SNIS, assuming $f(\theta)=\theta$. Then, we also consider
 \begin{align*}
q(\theta) = \mathcal{N}(\theta|-1,h^2) = \frac{1}{\sqrt{2\pi h^2}}\exp\left(- \frac{1}{2h^2}(\theta+1)^2\right),
\end{align*}
 as proposal density in SNIS. Note that $q(\theta)$ has the same mean of 
  $\post(\theta)$ and variance $h^2$. We test the values $h=1.5$ and $h=5$. Finally, recall that in the baseline MC we employ $q(\theta)=\post(\theta)$. We set different values of $N \in \{10,50,100, 500,1000, 5000\}$. 
  The results averaged  over $1000$ independent simulations, are given in Table \ref{TablaResultsESS}. The MSE of the SNIS estimators with $q_{\text{opt}}(\theta)$ and with $q(\theta), h=1.5$,  is always lower than the MSE of the ideal MC scheme and, as a consequence, the ESS is always bigger than 1, in these cases. Whereas the MSE of the SNIS estimator with $q(\theta)$ and $h=5$ is bigger  than the MSE of the ideal MC scheme. The reason of this change in the performance is that $q(\theta)$ with $h=1.5$ covers the two modes of the optimal proposal in Figure \ref{CitabLuca}. Hence, with $h=1.5$, the proposal $q(\theta)$ is more similar to $q_{\text{opt}}(\theta)$, than $q(\theta)$ with $h=5$ and also than $q(\theta)$ with $h=1$ (that coincides with $\post(\theta)$).
  

%308.65   62.20   32.30    6.33    3.21    0.66    0.32
%276.73   46.79   23.22    4.39    2.11    0.44    0.21
   
 
 \begin{table}[!h]	
	 \caption{\footnotesize MSE and ESS comparing the ideal MC and the SNIS estimators, as function of the number of samples $N$ and for  different proposal densities. We can see that the MSE of the SNIS estimator with the optimal proposal (and with a $q$ close to the the optimal proposal, i.e., with $h=1.5$)  is always lower and, as a consequence, the ratio $\frac{\mbox{ESS}}{N}$ is always bigger than 1.}\label{TablaResultsESS}
	 \vspace{-0.2cm}
	\begin{center}
	\footnotesize
		\begin{tabular}{|c||c|c|c|c|c|c|} 
		\hline 
		Number of samples, $N$ 	 & $10$  & $50$ & $100$ &  $500$ & $1000$ & $5000$   \\ 
			 \hline
			 \hline
			 Ideal Monte Carlo, i.e., $q(\x)=\post(\x)$ & 
			 0.0986  &   0.0201 &   0.0101 &    0.0020  &  0.0010    & 0.0002		       \\
\hline 
\hline 			 
			 SNIS with $q_{\text{opt}}(\x)$  &    0.0834    & 0.0146 &   0.0067  &  0.0013   & 0.0006 &    0.0001    			    \\ 
              %      \hline 
                  %    \hline 
		 ESS$/N$ with $q_{\text{opt}}(\x)$ & 1.1823
		   &  1.3809  &  1.5103 &   1.5912  &  1.5475  &  1.5262   \\
			\hline 
 \hline 			 
			 SNIS, $q(\x)$ with $h=1.5$  & 0.0910 & 0.0158 &    0.0078  &  0.0016   & 0.0008  &   0.0002 \\    			      
		 ESS$/N$, $q(\x)$ with $h=1.5$ & 1.0834  &   1.2722 &   1.2949    &1.2500  &  1.2500 &    1.0000         \\
			\hline	
			\hline 			 
			 SNIS, $q(\x)$ with $h=5$  & 0.3916   & 0.0400&    0.0189    &0.0037   & 0.0019  &   0.0004        			    \\  
		 ESS$/N$, $q(\x)$ with $h=5$ &  0.2518  &  0.5031  &  0.5342    &0.5456   & 0.5404  &  0.5479         \\
			\hline		
		\end{tabular}
	\end{center}

\end{table}

 
	\begin{figure}[!h]
		\centering
		\subfigure[Standard IS with $f(\theta)=\theta$.]{\includegraphics[width=8cm]{SuperFigExLuca}}
		\subfigure[\label{CitabLuca}Self-Normalized IS with $f(\theta)=\theta$.]{\includegraphics[width=8cm]{SuperFig2ExLuca}}	
		%\vspace{-0.4cm}
		\caption{\footnotesize Target density $\post(\theta)$ (blue line) and optimal proposal densities $q_{\text{opt}}(\theta)$ for the standard IS (red line) and SNIS (magenta line) schemes, when $f(\theta)=\theta$.  }
		\label{LucaEx1}
	\end{figure}

	\begin{figure}[!h]
		\centering
		\subfigure[Standard IS with $f(\theta)=\sqrt{|\theta|}$.]{\includegraphics[width=8cm]{SuperFig3ExLuca}}
		\subfigure[\label{Tremodes1}Self-Normalized IS with $f(\theta)=\sqrt{|\theta|}$]{\includegraphics[width=8cm]{SuperFig4ExLuca}}	
		%\vspace{-0.4cm}
		\caption{\footnotesize Target density $\post(\theta)$ (blue line) and optimal proposal densities $q_{\text{opt}}(\theta)$ for the standard IS (red line) and SNIS (magenta line) schemes, when $f(\theta)=\sqrt{|\theta|}$. }
		\label{LucaEx2}
	\end{figure}
	
	\begin{figure}[!h]
		\centering
		\subfigure[Standard IS with $f(\theta)=\theta^2$.]{\includegraphics[width=8cm]{SuperFig5ExLuca}}
		\subfigure[\label{Tremodes2}Self-Normalized IS with $f(\theta)=\theta^2$]{\includegraphics[width=8cm]{SuperFig6ExLuca}}
		%\vspace{-0.4cm}
		\caption{\footnotesize Target density $\post(\theta)$ (blue line) and optimal proposal densities $q_{\text{opt}}(\theta)$ for the standard IS (red line) and SNIS (magenta line) schemes, when $f(\theta)=\theta^2$. }
		\label{LucaEx3}
	\end{figure}


 

%\subsection{Second experiment}\label{Comparison IS vs RIS}

%%%%%%%%%%%%%%%%%%%%%%%%%%%%%%%%%%%%
\subsection{Theoretical and numerical comparisons between IS and RIS} \label{IS_vs_RIScomp} 
%%%%%%%%%%%%%%%%%%%%%%%%%%%%%%%%%%%%
%%%%%%%%%%%%%%%%%%%%%%%%%%%%%%%%%%%%%%%%%%%%
\subsubsection{Theoretical comparison}
%%%%%%%%%%%%%%%%%%%%%%%%%%%%%%%%%%%%%%%%%%%%

%We consider the scenario in the example in Section \ref{Comparison IS vs RIS}.
%{decimos que $\pi(\theta)$ no tiene dependencia en $\y$ en este toy example?}
In this section, the goal is to compare theoretically the standard IS and RIS schemes for estimating the normalizing constant of a target density $Z$. For simplicity, we consider again Gaussian target $\pi(\theta) = \exp( -\frac{1}{2}\theta^2)$, since we know the ground-truth  $Z =\int_{-\infty}^{\infty}\pi(\theta)d\theta = \sqrt{2\pi}$, so that $\post(\theta) = \frac{\pi(\theta)}{Z} = \mathcal{N}(\theta|0,1)$. 
%{Since this is a data-independent example, $\pi(\theta)$ and $\post(\theta)$ have no dependence on $\y$.}
The standard IS estimator of $Z$ with proposal $q(\theta)$ and the RIS estimator with auxiliary density $\varphi(\theta)$ are the following:
\begin{align*}
\widehat{Z}_\text{IS} = \frac{1}{N}\sum_{i=1}^{N} \frac{\pi(\theta_i)}{q(\theta_i)}, \quad  \theta_i \sim q(\theta), \quad \widehat{Z}_\text{RIS} = \frac{1}{\frac{1}{N}\sum_{k=1}^{N}\frac{\varphi(\theta_k)}{\pi(\theta_k)}}, \quad \theta_k \sim \post(\theta).
\end{align*}
For a fair theoretical and empirical comparison, we consider
\begin{align*}
\varphi(\theta)=q(\theta) = \mathcal{N}(\theta|0,h^2) = \frac{1}{\sqrt{2\pi h^2}}\exp\left(- \frac{1}{2h^2}\theta^2\right),
\end{align*}
where $h>0$ is the standard deviation. Thus,  both estimators depend on $q(\theta)$, although the density $q(\theta)$ plays a different role inside each estimator. We desire to study the performance of the two estimators as $h$ varies.
\newline
Now, we study the variances of the estimators $\widehat{Z}_\text{IS}$ and $\widehat{Z}_\text{RIS}$ as function of $h$, starting from $\widehat{Z}_\text{IS}$. Note that by the i.i.d. assumption, we can write
\begin{align}\label{VarIS_ex}
\text{Var}_{q}[\widehat{Z}_\text{IS}]&= \frac{1}{N}\text{Var}_{q}\left[\frac{\pi(\theta)}{q(\theta)}\right] = \frac{1}{N}\left\{ \mathbb{E}_{q}\left[\frac{\pi(\theta)}{q(\theta)}\right]^2 - Z^2 \right\},
\end{align}
Substituting $\pi(\theta) = \exp(-\frac{1}{2}\theta^2)$ and $q(\theta)=\frac{1}{\sqrt{2\pi h ^2}}\exp(-\frac{1}{2h^2}\theta^2)$, then we obtain
\begin{align*}
\mathbb{E}_{q}\left[\frac{\pi(\theta)}{q(\theta)}\right]^2 &= \int_{-\infty}^{\infty}\left(\frac{\pi(\theta)}{q(\theta)} \right)^2q(\theta)d\theta,\\
&= \int_{-\infty}^{\infty}\frac{\pi(\theta)^2}{q(\theta)}d\theta, \\
&= \sqrt{2\pi h^2}\int_{-\infty}^{\infty}\exp\left\{-\left(1-\frac{1}{2h^2}\right)\theta^2 \right\}d\theta, \\
&=  2\pi \frac{h}{\sqrt{2-\frac{1}{h^2}}}. 
\end{align*}
Replacing the last expression above in Eq. \eqref{VarIS_ex}, we obtain that the variance of $\widehat{Z}_\text{IS}$ is given by 
\begin{align}\label{VAR_IS_teo}
\text{Var}_{q}\left[\widehat{Z}_\text{IS}\right] = \frac{2\pi}{N}\left\{ \frac{h}{ \sqrt{2 - \frac{1}{h^2}}} - 1,
\right\}.
\end{align}
that is depicted in Figure \ref{Fig4a}.
This variance reaches its minimum, $\text{Var}_q[\widehat{Z}_\text{IS}]=0$, at $h=1$, i.e., when the proposal is optimal, coinciding exactly the posterior $q(\theta)=\mathcal{N}(\theta|0,1)=\post(\theta)$ as expected (recall that we are estimating $Z$). For $h < 1$, $\text{Var}_q[\widehat{Z_\text{IS}}]$ grows exponentially until reaching $h = \frac{1}{\sqrt{2}}$ where is infinite. For $0 < h < \frac{1}{\sqrt{2}}$, $\text{Var}_q[\widehat{Z_\text{IS}}]$ is not defined. Finally,  $\text{Var}_q[\widehat{Z_\text{IS}}]$ grows linearly from $h=1$ onwards, i.e., diverges to infinity as $h\rightarrow \infty$. Figure \ref{Fig4a} shows that behavior when $N=500$.  Clearly, this is  perfectly in line the well-known theoretical requirement that the proposal pdf must have fatter tails than the posterior density in a IS  scheme. Moreover, this confirms that the use of proposals with variance bigger than that of the target is generally not catastrophic. The opposite could yield  catastrophic results.  Recall also that $\mathbb{E}_{q}[\widehat{Z}_\text{IS}]=Z$, i.e., the bias of $\widehat{Z}_\text{IS}$ is zero. 
\newline
\newline
Regarding RIS, it is easier to compute analytically the variance of $\widehat{r} =\frac{1}{\widehat{Z}_\text{RIS}}$, rather than $\widehat{Z}_\text{RIS}$ itself. Namely,  we consider the estimator $\widehat{r} =\frac{1}{N} \sum_{i=1}^N \frac{q(\theta_k)}{\pi(\theta_k)}$, with $\theta_k \sim \post(\theta)$, which is an unbiased estimator of $\frac{1}{Z}$. 
%Note that
%$$
%\widehat{Z}_\text{RIS} = \frac{1}{\widehat{r}} \ .
%$$
Since $\theta_k$'s are i.i.d. from $\post(\theta)$, then we have 
%{Mediante el metodo delta, podemos decir que el error de RIS es $\mathbb{E}[(\widehat{Z}_\text{RIS}-Z)^2]=\mbox{var}\left[\widehat{r}\right] + \mathcal{O}(\frac{1}{N^2})$ y asi nos ahorramos lo de la inversa....}
\begin{align*}
\text{Var}_{\post}\left[\widehat{r}\right] = \text{Var}_{\post}\left[\frac{1}{\widehat{Z}_\text{RIS}}\right] &= \frac{1}{N}\text{Var}_{\post}\left[ \frac{f(\theta)}{\pi(\theta)} \right], \\
&= \frac{1}{N}\left\{ \mathbb{E}_{\post}\left[ \frac{f(\theta)}{\pi(\theta)} \right]^2 - \frac{1}{Z^2} \right\},
\end{align*}
Substituting $\pi(\theta) = \exp\left(-\frac{1}{2}\theta^2\right)$ and $f(\theta)=\frac{1}{\sqrt{2\pi h ^2}}\exp\left(-\frac{1}{2h^2}\theta^2\right)$, we obtain
\begin{align*}
\mathbb{E}_{\post}\left[ \frac{f(\theta)}{\pi(\theta)} \right]^2 &= \int_{-\infty}^{\infty} \left(\frac{f(\theta)}{\pi(\theta)} \right)^2 \frac{\pi(\theta)}{Z}d\theta, \\
&= \frac{1}{Z}\int_{-\infty}^{\infty} \frac{f(\theta)^2}{\pi(\theta)} d\theta, \\
&= \frac{1}{2\pi h^2 \sqrt{2\pi}}\int_{-\infty}^{\infty} \exp \left\{ -\left( \frac{1}{h^2} - \frac{1}{2}\right)\theta^2 \right\}d\theta, \\
&= \frac{1}{2\pi}\frac{1}{h^2\sqrt{\frac{2}{h^2}-1}}.
\end{align*}
Hence the variance of $\widehat{r}$ is given by 
\begin{align}\label{VAR_RIS_inv}
\text{Var}_{\post}[\widehat{r}] = \text{Var}_{\post}\left[\frac{1}{\widehat{Z}_\text{RIS}}\right] = \frac{1}{2\pi N}\left\{ \frac{1}{h^2\sqrt{\frac{2}{h^2}-1}} - 1 \right\},
\end{align}
which reaches its minimum, $\text{Var}_{\post}[\widehat{r}]=0$, again at $h=1$ as expected. Recall that in RIS, $q(\theta)$ is playing the role of an auxiliary density, and it is not a proposal pdf. 
 Note that $\text{Var}[\widehat{r}]$ is defined when $0<h<\sqrt{2}$ (there are two vertical asymptotes). Moreover, $\text{Var}_{\post}[\widehat{r}]$ grows more quickly in $1<h<\sqrt{2}$ than in $0<h<1$. In Figure \ref{Fig4b}, we show $\text{Var}_{\post}[1/\widehat{Z}_\text{RIS}]$ for $N=500$. Observe that  $\widehat{r} =\frac{1}{\widehat{Z}_\text{RIS}}$ has the same behavior as the IS estimator when the variance of the denominator (in this case $\post(\theta)$) is smaller than the numerator (in this case $q(\theta)$), but the asymptote is reached only at $h=0$ (not before). Therefore, the case $h<1$ is less catastrophic than in the standard IS scheme. However, RIS presents an additional catastrophic scenario for $h>1$, at $h=1.4$, where there is another vertical asymptote. However, studying numerically $\widehat{Z}_\text{RIS}$ instead of $1/\widehat{Z}_\text{RIS}$, we can see that second  vertical asymptote disappears (see below).   The variance around the optimal value $h=1$ is flatter than in the standard IS. 
\newline
Therefore, choosing properly $h$, RIS can provide better performance than standard IS. However, it seems that the only safe region for avoiding catastrophic scenarios of infinite variance (for estimation of $Z$) is given by the use of a standard IS scheme with a variance of the proposal density greater than the  variance of the target density.
  
%Then, we see that $q(\theta)$ should have non-zero variance and less variance than $\post(\theta)$, in order to avoid infinite variance of the resulting estimator $\widehat{r} =\frac{1}{\widehat{Z}_\text{RIS}}$.  We can observe two vertical asymptotes when we analyze $\text{Var}_{\post}[1/\widehat{Z}_\text{RIS}]$.  However, note that  $\text{Var}_{\post}[\widehat{Z}_\text{RIS}]$ has just one vertical asymptote at $h=0$ as shown below.


%{\color{red} 
%In Figure \ref{varZteo_ISandRIS}, we show var$[\widehat{Z}_\text{IS}]$ for $M=500$. In fact, the effect of the sample size in this case is simply scaling the whole curve.} 




\begin{figure}[!h]
	\centering
	\subfigure[\label{Fig4a}Variance of $\widehat{Z}_\text{IS}$.]{\includegraphics[width=8cm]{var_Z_IS.pdf}}
	%\hspace{0.5cm}
	\subfigure[\label{Fig4b}Variance of $\frac{1}{\widehat{Z}_\text{RIS}}$.]{\includegraphics[width=8cm]{var_Z_RIS.pdf}}
	%\vspace{-0.4cm}
	\caption{The variances  $\text{Var}_{\post}[\widehat{Z}_\text{IS}]$ and $\text{Var}_{\post}[1/\widehat{Z}_\text{RIS}]$ in Eqs. \eqref{VAR_IS_teo} and \eqref{VAR_RIS_inv}, respectively ($N=500$).
	}
	\label{varZteo_ISandRIS}
\end{figure}
%%%%%%%%%%%%%%%%%%%%%%%%%%%%%%%
\subsubsection{Numerical comparison} 
%%%%%%%%%%%%%%%%%%%%%%%%%%%%%%%

In the previous part of this section, we have compared $\text{Var}_{q}\left[\widehat{Z}_\text{IS}\right]$ with $\text{Var}_{\post}\left[\frac{1}{\widehat{Z}_\text{RIS}}\right]$. Recall that 
 $\mathbb{E}_{q}[\widehat{Z}_\text{IS}]=Z$ and $\mathbb{E}_{\post}\left[\frac{1}{\widehat{Z}_\text{RIS}}\right]=Z$ but $\mathbb{E}_{\post}\left[\widehat{Z}_\text{RIS}\right]\neq Z$,  i.e., the bias is non-zero in this last case. 
 \newline
Thus, setting $N=500$, we compute numerically the mean square error (MSE) of both $\widehat{Z}_\text{IS}$ and $\widehat{Z}_\text{RIS}$, and the variance and bias of both $\widehat{Z}_\text{RIS}$, averaging the results over 5000 independent runs. We show the results in Fig. \ref{cacalabel}.  In Figure \ref{FigRIS1}, we provide bias and variance of the estimator $\widehat{Z}_\text{RIS}$. Note that its variance the bias have only one asymptote at $0$ instead of two asymptotes, unlike the variance of $\widehat{r}=1/\widehat{Z}_\text{RIS}$. Indeed, the variance and bias  of  $\widehat{Z}_\text{RIS}$ diverge  also as $h\rightarrow \infty$ (bur without an additional vertical asymptote). Observe also that the bias is negligible for $0.1<h<1.6$,  with respect to the value of the variance.  
\newline
 In Fig. \ref{FigRIS2}, we can see that the MSE of $\widehat{Z}_\text{IS}$ corresponds to its theoretical variance shown in Fig. \ref{varZteo_ISandRIS}, as we expect since $\widehat{Z}_\text{IS}$ has zero bias, hence $\text{MSE}_q(\widehat{Z}_\text{IS}) = \text{Var}_q(\widehat{Z}_\text{IS})$. Although $\widehat{Z}_\text{RIS}$ is not unbiased, we see that its MSE, also shown in Fig. \ref{FigRIS2}, is virtually identical to its variance shown in  Fig. \ref{FigRIS1}, where the bias seems to be negligible for the majority of values of $h$. {In fact, applying the Delta method to $\widehat{Z}_\text{RIS}$ shows that $\mathbb{E}[(\widehat{Z}_\text{RIS}-Z)^2]=\mbox{var}\left[\widehat{r}\right] + \mathcal{O}(\frac{1}{N^2})$, i.e., the MSE of $\widehat{Z}_\text{RIS}$ coincides with $\mbox{var}\left[\widehat{r}\right]$ up to $\mathcal{O}(\frac{1}{N^2})$ terms.}




% See also Figure for a numerical comparison of the MSE of $\widehat{Z}_\text{IS}$ and $\widehat{Z}_\text{RIS}$ using $M=N=500$ for a range of $h$ from $h=0.1$ to $h=5$. The results were averaged over 1000 independent simulations. 
\begin{figure}[h!] 
	\centering
	\subfigure[\label{FigRIS1}Bias and variance of $\widehat{Z}_\text{RIS}$.]{\includegraphics[width=8cm]{FIG_SIMU1_EX1.pdf}}
	\subfigure[MSE of $\widehat{Z}_\text{IS}$ and $\widehat{Z}_\text{RIS}$.]{\label{FigRIS2}\includegraphics[width=8cm]{FIG_SIMU2_EX1.pdf}}
	\caption{(a) Bias (dashed line) and variance (solid line) of $\widehat{Z}_\text{RIS}$ as a function of $h$ ($N=500$). (b) MSE of $\widehat{Z}_\text{IS}$ (solid line) and $\widehat{Z}_\text{RIS}$ (dashed line) as a function of $h$ ($N=500$).}
	%\label{MSE, var and bias numerically of IS and RIS}
	\label{cacalabel}
\end{figure}


	
	
	
	
	
	
	
	
	









\section{Conclusions}\label{FinalDiscSect}

The choice of the proposal density is crucial  for the performance of Monte Carlo sampling methods and, specifically, in IS schemes. Hence, knowing the optimal proposal density in the specific scenario of interest is essential in order to design suitable adaptive procedures within modern IS schemes. In this review, we have provided an exhaustive and accessible introduction to different results about the optimality in IS schemes, that were spread in the literature during the last decades. We have also included novel variants and several settings, including the noisy target scenario and the marginal likelihood estimation. The relationships among the different frameworks and schemes have been widely described in the text, by means of several summary tables and figures. Theoretical and empirical comparisons have been also provided. 
\newline
This work also should be of particular interest for practitioners and researchers involved in the development of new methods that seek to address the growing list of challenges modern day statistical science is being called upon to address.  As an example of future work and research challenge, we suggest the analysis of the relevant connection between importance sampling and {\it contrastive learning} \cite{gutmann12a}, where the concept of {\it optimal reference density} has been recently started to study \cite{chehab2023}.  

%This review has aimed to provide an accessible introduction to the necessary differential geometry, with a specific focus on the elements required to formally describe Hamiltonian Monte Carlo in particular. This formal understanding is necessary to gain insights into more advanced methods, including Shadow Hamiltonian and Riemann Manifold Hamiltonian methods. This should also be of interest to readers interested in the development of new methods that seek to address the growing list of challenges modern day statistical science is being called upon to address. More generally, we believe the use of geometry is essential to even attempt to tackle sampling issues related to the curse of dimensionality and concentration of measure in for example Deep Learning.


%\section{Practical optimization....Deniz....gradiente estocasctico ...y otros descenso...}\label{IteraSect}
%
%\subsection{Gradient descent}
%\subsection{Adaptive schemes}
%
%just literatures y general schemes...SHORT
%
%}
%%{\color{red}
%%\section{Discussion about implementation of optimal proposals}
%%}
%All the optimal proposals discussed in this paper are not of direct use since they cannot be sampled or evaluated.
%In some cases, we can evaluate them up to a normalizing constant, hence we are able to sample them by using, e.g., MCMC algorithms. 
%In other cases, the optimal proposal is {\it doubly intractable }, as it cannot be evaluated up to a normalizing constant either, making it impossible to run MCMC algorithms (or any Monte Carlo method) on it.
%However, even if we can sample via MCMC, we cannot evaluate them in normalized form since their normalizing constant is also intractable. This is necessary in standard IS, although in SNIS it is enough to be able to evaluate them in unnormalized form. 
%
%One solution is to build tractable (e.g., parametric or piecewise constant) approximations $\widehat{q}_\text{opt}(\x)$ of the optimal proposal $q_\text{opt}$, such that $\widehat{q}_\text{opt}(\x)$ are easy to sample (using direct methods) and can be evaluated in closed-form. For instance, this is the case of the Vegas algorithm \cite{lepage1978new}.
%Adaptive importance sampling (AIS) algorithms can be used for this task, since many of these algorithms consider adapting a population of parametric proposal in order to reduce the mismatch with respect to a given target function. In our case, this target function corresponds to the unnormalized proposal density (when it can be evaluated).
%
%{\color{red}
%For doubly intractable proposals: Starting with an initial estimate, an iterative procedure can be used to build and refine approximations of the optimal proposal. For instance, considering the optimal proposal in SNIS, we can take $\widetilde{q}^{(t)}_\text{opt}(\x) = \pi(\x)|f(\x) - \widehat{I}^{(t-1)}|$, where $\widehat{I}^{(t-1)}$ is running estimate of the intractable quantity. This procedure has its drawbacks though \cite{rainforth2020target}.
%}

%%In this work, we focus on the IS class of methods. We describe several frameworks of practical interest and provide the corresponding optimal proposal density $q_\text{opt}$.  For this purpose, we have the opportunity to review numerous IS schemes proposed in the literature during the last years, describing also several related properties and results. We consider the use of a unique or multiple proposal densities for approximating an integral. Moreover, we consider the joint approximation 
%of several integrals. The noisy framework (when the evaluation of the posterior is a random variable itself), which includes the reinforcement learning and approximate Bayesian computation (ABC) as special cases, is also addressed. In this sense, this work can be considered an exhaustive survey around the concept of optimality in importance sampling.







%\bibliographystyle{unsrt}
\bibliographystyle{IEEEtranN}
\bibliography{bibliografia} 






%\begin{appendices}
%	\section{Covariance between $\widehat{E}$ and $\widehat{Z}$}\label{App1}
%Let $\widehat{E} = \frac{1}{N}\sum_{i=1}^{N}\frac{\pi(\x_i)f(\x_i)}{q(\x_i)}$ and $\widehat{Z} = \frac{1}{N}\sum_{i=1}^N\frac{\pi(\x_i)}{q(\x_i)}$ be the standard IS estimators of $E = \int_\Theta f(\x)\pi(\x)d\x$ and $Z=\int_{\Theta}\pi(\x)d\x$, respectively. 
%We show that
%$$
%\text{Cov}[\widehat{E},\widehat{Z}] = \frac{1}{N}\E_q\left[\frac{f(\x)\pi(\x)^2}{q(\x)^2}\right] - \frac{1}{N}EZ.
%$$
%First, recall that $\text{Cov}[\widehat{E},\widehat{Z}] = \E_q[\widehat{E}\widehat{Z}] - EZ$.
%Denote $w_i=\frac{\pi(\x_i)}{q(\x_i)}$ and let us rewrite the product $\widehat{E}\widehat{Z}$ as
%{\footnotesize\begin{align*}
%	\widehat{E}\widehat{Z} 
%	&=  \frac{1}{N^2}\sum_{i=1}^{N}w_i^2f(\x_i) + \frac{2}{N^2}\sum_{i=1}^N\sum_{j>i}^{N}w_iw_jf(\x_i).
%	\end{align*}}
%Hence, we obtain
%{\footnotesize\begin{align*}
%	\E_q\left[\widehat{E}\widehat{Z}\right] &= \frac{1}{N}\E_q\left[\frac{f(\x)\pi(\x)^2}{q(\x)^2}\right] 
%	+ \frac{2}{N^2}\sum_{i=1}^N\sum_{j>i}^{N}\E\left[\frac{\pi(\x_i)f(\x_i)}{q(\x_i)}\right]\E_q\left[\frac{\pi(\x_j)}{q(\x_j)}\right] \\
%	&=\frac{1}{N}\E_q\left[\frac{f(\x)\pi(\x)^2}{q(\x)^2}\right] +
%	\frac{2}{N^2}\sum_{i=1}^N\sum_{j>i}^{N}EZ \\
%	&=\frac{1}{N}\E_q\left[\frac{f(\x)\pi(\x)^2}{q(\x)^2}\right] +
%	EZ\left(1-\frac{1}{N}\right).
%	\end{align*}}
%Combining the results, we obtain the desired expression.
%	
%\end{appendices}

	\begin{figure}[!h]
		\centering
		%\subfigure[]{\includegraphics[width=8cm]{vsBS_MIS_MselfIS.png}}	
		\subfigure[\label{USfig1}]{\includegraphics[width=12cm]{StandardIS}}
		\subfigure[\label{USfig2}]{\includegraphics[width=12cm]{RISfig}}
	\subfigure[\label{USfig3}]{\includegraphics[width=12cm]{UmbSampl}}
	\subfigure[\label{USfig4}]{\includegraphics[width=12cm]{BrSampl}}		
		\caption{Graphical representation and comparison of {\bf (a)} standard IS, {\bf (b)} RIS, {\bf (c)} the umbrella sampling and  {\bf (d)} bridge sampling, for estimating $Z$.}	
		\label{USfig}
	\end{figure}




\end{document}


