\begin{figure}[ht]
    %\vspace{-1em}
    \begin{center}
        \adjustbox{valign=t}{
            \begin{minipage}{0.30\textwidth}
                \includegraphics[width=\textwidth]{figures/figs/math_flash_v_pro_v_sonnet_3hints.pdf}
            \end{minipage}
        }
        \adjustbox{valign=t}{
            \begin{minipage}{0.30\textwidth}
                \includegraphics[width=\textwidth]{figures/figs/gpqa_gemma_v_flash_sonnet_hints.pdf}
            \end{minipage}
        }
        \hfill
        \adjustbox{valign=t}{
            \begin{minipage}{0.75\textwidth}
            \vspace{0.5em}
            \caption{
                \small
                \textbf{Hints improve separation on \NS problems.}
                On \MATHAdv and \GPQA, giving no hint results in too difficult problems while giving all hints makes the problems too easy.
                In both cases we need 1 or 2 hints to reliably separate candidate models.
                Thus hints synthesized from PI effectively interpolate the difficulty of \NS problems, which helps separate weaker models from stronger ones.
                \eat{
                These figures show:
                1. We can interpolate the difficulty of GPQA problems with hints. (slopes monotonically increase)
                2. We can clearly separate Gemma and Flash with 1 or 2 hints. (unclear w/out hints or w/ too many hints)
                }
            }
            \label{fig:better_separation}
            \end{minipage}
        }
    \end{center}
    \vspace{-1em}
\end{figure}