\begin{figure}[ht]
    \vspace{1em}
    \begin{center}
        \adjustbox{valign=t}{
            \begin{minipage}{0.4199\textwidth}
                \includegraphics[width=\textwidth]{figures/figs/reka_vibe_eval.pdf}
            \end{minipage}
        }
        \hfill
        \adjustbox{valign=t}{
            \begin{minipage}{0.2549\textwidth}
                \includegraphics[width=\textwidth]{figures/figs/reka_pro_v_human.pdf}
                \vfill
            \end{minipage}
        }
        \adjustbox{valign=t}{
            \begin{minipage}{0.2549\textwidth}
                \includegraphics[width=\textwidth]{figures/figs/reka_normal_vs_hard.pdf}
                \vfill
            \end{minipage}
        }
        \begin{minipage}[t]{0.9\textwidth}
            \vspace{1em}
            \caption{
                \small
                \textbf{On \VibeEval, graders with \PI outperform individual human graders.}
                Spearman correlation is measured against the average vote of 5 human graders.
                \textbf{Left}: Both Gemini 1.5 Flash and Pro can outperform individual human graders, and they both perform best when given different sources of \PI.
                \textbf{Middle}: Individual humans also benefit from \PI, albeit not as much as automatic graders.
                \textbf{Right}: Gemini 1.5 Pro benefits from \PI especially on the \emph{Hard} split of \VibeEval, indicating \PI is especially useful for \NS benchmarks.
                \eat{
                    The left figure shows:
                    1. Combining privileged information improves the rater's performance.
                    2. Both Flash and Pro can outperform individual humans (vs avg human).
                    The middle figure shows:
                    1. Humans also benefit from PI.
                    2. LLMs benefit a lot more from PI than humans.
                    The right figure shows: Pro benefits from PI a lot more on hard prompts than normal ones.
                    Here PI = gold-ref + rating guidelines.
                }
            }
            \label{fig:reka_vibe_eval}
        \end{minipage}
    \end{center}
    %\vspace{-1em}
\end{figure}