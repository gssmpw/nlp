\documentclass[preprint,12pt]{elsarticle}

%% Use the option review to obtain double line spacing
%% \documentclass[authoryear,preprint,review,12pt]{elsarticle}
\usepackage[english]{babel}
\usepackage[strings]{underscore}
\usepackage{microtype}
\usepackage{graphicx}
\usepackage{subcaption}
 \graphicspath{{./img/}}
 \DeclareGraphicsExtensions{.pdf}
\usepackage{booktabs}
\usepackage{multirow}
\usepackage{amsmath}
\usepackage[export]{adjustbox}
\usepackage{algcompatible}
\usepackage{lscape}
\usepackage[noend]{algpseudocode}
\usepackage{algorithm}
\usepackage[table]{xcolor}
\usepackage{xcolor}

\usepackage{xr}
\makeatletter
\newcommand*{\addFileDependency}[1]{
  \typeout{(#1)}
  \@addtofilelist{#1}
  \IfFileExists{#1}{}{\typeout{No file #1.}}
}
\makeatother

\newcommand*{\myexternaldocumenta}[1]{
    \externaldocument{#1}
    \addFileDependency{#1.tex}
    \addFileDependency{#1.aux}
}
%%% END HELPER CODE

% put all the external documents here!
\myexternaldocumenta{./JPDC2018}

% just to see what's happening
\listfiles


\begin{document}

\appendix
%%%%%%%%%%%%%%%%O
\section{Extended Simulation Results}
\label{results_appendix}

Figure \ref{fig:RLFT_HS10} and \ref{fig:RLFT_HS25} show the simulation results when 10\% and 25\% of the end-nodes generate congested traffic addressed to a single end-node (i.e., traffic pattern HS10-1 and HS25-1). 

As we can see in these figures, the 1-VC configuration cannot deal with congestion just applying restricted adaptive routing, due to the HoL blocking (see Figures \ref{fig_RLFT_HS10_1q},  \ref{fig_RLFT_HS10_1q-voq}, \ref{fig_RLFT_HS_1q} and \ref{fig_RLFT_HS_1q-voq}).


Note that when no VOQs are used, the adaptive routing using \emph{triggering thresholds} is able to raise a bit the performance, since this restriction delays the adaptivity decisions. The results show that better performance is achieved in a more congested scenario. 
This strange effect is due to the congestion tree. In fact, a higher incast congestion scenario creates a larger congestion tree that reaches the level 1 switches in the routing upward phase. Note that the performance raise in Figure \ref{fig_RLFT_HS10_1q} and \ref{fig_RLFT_HS_1q} when they reach 70\% and 50\% of the generated traffic. 
As a consequence, the performance of adaptive algorithms, restricted only to the second stage,  drops to 0.
By contrast, when VOQs are used, 1VC performance is close to 0\% as congestion trees grow into the VOQs, regardless the used routing algorithm.

DBBM (see Figures \ref{fig_RLFT_HS10_dbbm3}, \ref{fig_RLFT_HS10_dbbm3-voq}, \ref{fig_RLFT_HS_dbbm3} and \ref{fig_RLFT_HS_dbbm3-voq}) outperforms 1VC regardless the routing configurations.
When no VOQs are used,the results are homogeneous until the percentage of injected traffic exceeds 60\% and they show that K/$\Delta$ and 2S restrictions do not work as good as the other routing configurations.
The 2S restriction does not work well due to routing configuration uses $D$-mod-$K$ in the first stage of the Fat Tree, so that DBBM maps all the destinations to the same VC (see \figurename~\ref{fig_RLFT_3_2_destro}). Something similar happens with K/$\Delta$ restriction. This restriction uses a module to decide which upward port is chosen in the routing function \ref{algorithmAdaptive}. Since the module for choosing the port and mapping in the queue matches, then all adapted packets are mapped in the same queue. That's why the worst results are obtained when packets are adapted only in the second stage without any restrictions.
In other words, DBBM improves the deterministic routing when it adapts the routes in the first stage and manages to balance the use of queues.
However, when VOQs are used, low order HoL-blocking disappears and the results converge regardless the used routing algorithm.

When we use vFtree (see Figures \ref{fig_RLFT_HS10_vftree3}, \ref{fig_RLFT_HS10_vftree3-voq}, \ref{fig_RLFT_HS_vftree3-voq} and \ref{fig_RLFT_HS_vftree3}) the network efficiency drops significantly when adaptive routing is applied in the second stage (see the top-right switch \figurename~\ref{fig_RLFT_3_2_adapt} and downward stages in red), since all the destinations are mapped to all the VCs so that the HoL blocking probability increases.
In this case, the K/$\Delta$ restrictions avoid this defect and achieve productivity even when packets are adapted just in the second stage.
Note that *S-K configurations are able to raise the performance when the injection traffic is over 70\% and 60\%.  
This effect is more noticeable in the VOQ architecture because of its higher productivity.

When we use Flow2SL without VOQs (see Figures \ref{fig_RLFT_HS10_flow2sl3}, \ref{fig_RLFT_HS_flow2sl3}), we can see that better results are obtained when adaptive routing is applied in the first stage (FS and S*)  with triggering restrictions.
On the other hand, when switches implement VOQs (see Figures \ref{fig_RLFT_HS10_flow2sl3-voq}, \ref{fig_RLFT_HS_flow2sl3-voq}), the obtained results are similar.
The reason is that VOQs spread the congestion throughout all the VCs in the same buffer, since flow-control is performed at VC level, but not at VOQ level.

Figures \ref{fig:RLFT_HS104} and \ref{fig:RLFT_HS254} show experiment results for the same scenarios described before when we generate traffic creating four congestion trees  (i.e., traffic HS10-4 and HS25-4 depicted in \figurename~\ref{fig_traffic_HS4}).


This is a very strong congestion situation, since we generate four congestion trees whose branches affect to different VCs at the same time and thus reduces the effectiveness of the static queuing scheme.
Therefore, this situation makes complex, even for restricted routing combined with queuing schemes, to reduce the HoL blocking.
When no VOQs are used, the SQS show similar behaviour as described above but they yield less performance.
By contrast, when VOQs are used, the only queuing scheme that works is vFtree (see Figures \ref{fig_RLFT_HS_vftree3} and \ref{fig_RLFT_HS_vftree3}) due to its mapping properties. It works well with deterministic routing but a little more productivity can be achieved when we adapt packets using K/$\Delta$ restriction even when there are four strong congestion trees.

Figure \ref{fig:RLFT_IHS} shows simulation results for traffic generating hot-spots in intermediate stages of the topology (see \figurename~\ref{fig_traffic_HSW}).
In particular, hot-spots (i.e., congestion tree roots) are generated in the output ports of some switches placed at the second stage of the RLFT.
As we can see in the figures, this traffic pattern generates a congestion situation where $D$-mod-$K$, oblivious and fully-adaptive routing algorithms (both with and without restrictions) can deal with the HoL blocking appearing in the network.
For switches without VOQs, vFtree queuing scheme achieves the best results combined with adaptive routing using combined restrictions, such as ADAP-NoTH-SS-K, ADAP-TH-AS-K or ADAP-2TH-AS-$K/3$ outperform 1VC, DBBM and Flow2SL. 
This behavior is the same for vFtree when using switches with VOQs.
Note that switches with VOQs using the configuration ADAP-NoTH-SS-K achieve excellent results regardless the use of queuing schemes (see Figures  \ref{fig_RLFT_IHS_1q-voq}, \ref{fig_RLFT_IHS_dbbm3-voq},  \ref{fig_RLFT_IHS_vftree3-voq}, \ref{fig_RLFT_IHS_flow2sl3-voq}).
Although the use of restricted routing and queuing schemes are not necessary in this traffic scenario when using VOQs, note that their use preserves the performance gains shown in the previous traffic scenarios when using restricted adaptive routing and queuing schemes.
Therefore, the use of restricted adaptive routing combined with queuing schemes significantly increases the network performance under congested scenarios, compared to when we use deterministic and oblivious routing.

Figures \ref{fig:RLFT_histo_1q}, \ref{fig:RLFT_histo_dbbm}, \ref{fig:RLFT_histo_vftree} and \ref{fig:RLFT_histo_flow2sl} show in histograms the same results as the Tables \ref{tab:NOVOQ} and \ref{tab:VOQ}. This allows the comparison of different switch architectures for a given queuing scheme


% JESÜS: a partir de aquí está el texto anterior


%We have renamed the non-restricted adaptive routing as ``fully-adaptive'' routing, since it is the way it is called in the literature.
%More precisely, Figure \ref{fig:RLFT_HS10_vTime} shows the situation when 10\% of the end-nodes generate congesting packets addressed to a single destination (i.e., traffic pattern HS10-1), and \ref{fig:RLFT_HS25_vTime} shows the simulation results when 25\% of the end-nodes generate congested traffic addressed to a single end-node (i.e., traffic pattern HS25-1).


%As we can see in these figures, the 1-VC configuration (with and without VOQs) cannot deal with congestion just applying restricted and non-restricting adaptive routing, due to the HoL blocking (see Figures \ref{fig_RLFT_HS10_1q},  \ref{fig_RLFT_HS10_1q-voq}, \ref{fig_RLFT_HS_1q} and \ref{fig_RLFT_HS_1q-voq}).
%Note that when no VOQs are used, the adaptive routing using \emph{triggering thresholds} (i.e., ADAP-TH-*S-K and ADAP-2TH-*S-K) is able to raise a bit the performance (up to 30\% approximately), since this restriction delays the adaptivity decisions.
%By contrast, when VOQs are used, 1VC performance is close to 0\% as congestion trees grow into the VOQs, regardless the used routing algorithm.
%Figures \ref{fig_time_RLFT_HS10_1q} and  \ref{fig_time_RLFT_HS10_1q-voq} show similar results when simulations are run during $3$ms and we generate 100\% of traffic load.
%The hot-spot is generated by 10\% of the nodes.
%Similar results are shown in Figures \ref{fig_time_RLFT_HS_1q} and \ref{fig_time_RLFT_HS_1q-voq}, when 25\% of end-nodes generate congested traffic.

%When queuing schemes are used without VOQs, the network efficiency results raise from 30\% achieved by 1VC using the ADAP-2TH-AS-K configuration up to 55\% for DBBM, 64\% for vFtree and 73\%.
%DBBM (see Figures \ref{fig_RLFT_HS10_dbbm3}, \ref{fig_time_RLFT_HS10_dbbm3}, \ref{fig_RLFT_HS_dbbm3} and \ref{fig_time_RLFT_HS_dbbm3}) outperforms 1VC regardless the routing configurations.
%The best routing configuration for DBBM is 2TH-*S-H in these scenarios of moderate congestion.
%This routing configuration uses $D$-mod-$K$ in the first stage of the Fat Tree, so that DBBM maps all the destinations to the same VC (see \figurename~\ref{fig_RLFT_3_2_destro}).
%In the second stage, adaptive routing is used, but the path diversity is reduced to the destinations reaching the second stage of the Fat Tree (as $D$-mod-$K$ balances traffic flows in the first stage).
%Then, DBBM shares more destinations in the same VC and the queuing scheme effectiveness is reduced.
%For this reason, DBBM combined with the adaptive routing configurations using the restriction in the first stage (FS) obtains good performance results, regardless the traffic pattern (i.e., HS10-1 and HS25-1).
%Note also that $D$-mod-$K$ routing achieves 40\% of network efficiency (less than 55\% achieved by routing configurations with restrictions).
%Moreover, oblivious and fully-adaptive routing take advantage of the DBBM mapping policy, since they achieve almost 50\% of the network efficiency.

%By contrast, when we use vFtree (see Figures \ref{fig_RLFT_HS10_vftree3}, \ref{fig_time_RLFT_HS10_vftree3}, \ref{fig_RLFT_HS_vftree3} and \ref{fig_time_RLFT_HS_vftree3}) the network efficiency drops significantly when adaptive routing is applied in the second stage (SS) even through vFtree is using 3VCs (see the top-right switch \figurename~\ref{fig_RLFT_3_2_adapt} and downward stages in red), since all the destinations are mapped to all the VCs so that the HoL blocking probability increases.
%In the case of HS10-1 traffic, when the triggering restrictions are applied (TH or 2TH), the network efficiency raises after generation rate reaches 70\%, while, in the case of HS25-1 traffic it raisers when the generated traffic load is close to 50\%.
%Note that the congestion tree generated by the traffic HS25-1 is more intense than that generated by HS10-1.
%In the case of HS10-1, the congestion appears first in switches near the hot-spot destination end-node, and then it propagates backwards so that it reaches slower the switches in the first stage of the Fat Tree (upward-path), compared to the traffic pattern HS25-1.
%In the case of HS25-1, the congestion appears first in the first stages as more flows gather, then it propagates downwards.
%As we can see, vFtree requires more restrictions to the adaptive routing than DBBM, in order to perform properly.
%Note that vFtree with routing restrictions reaches 60\% of network efficiency, while $D$-mod-$K$, which can be considered as the most restricted routing, achieves 51\% of network effficiency.
%Indeed, the routing restrictions to vFtree also involve the number of used port counts ($K$).
%When this value is $K/3$ then vFtree performs better, although it requires to restrict the adaptivity per stages (FS or SS) and using the triggering thresholds (TH or 2TH).
%Note that the oblivious and fully-adaptive routing configurations results drop near 0\% due to the specific mapping of vFtree, which needs to reduce adaptivity mainly in the second stage, as we have described before.

%When we use Flow2SL (see Figures \ref{fig_RLFT_HS10_flow2sl3}, \ref{fig_time_RLFT_HS10_flow2sl3},  \ref{fig_RLFT_HS_flow2sl3} and \ref{fig_time_RLFT_HS_flow2sl3}), we can see that better results are obtained when adaptive routing is applied in the first stage (FS) and in all the stages (AS), compared to deterministic ($D$-mod-$K$) or oblivious routing configurations.
%Note that Flow2SL achieves the maximum throughput (around 70\% in Figures \ref{fig_RLFT_HS10_flow2sl3} and \ref{fig_RLFT_HS_flow2sl3}) with certain configurations of restrictions in the routing, such as ADAPT-2TH-AS-$K/3$.

%On the other hand, when switches implement VOQs, the obtained results are better, in general, than those obtained for switches without VOQs for DBBM, vFtree and Flow2SL.
%The reason is that VOQs spread the congestion throughout all the VCs in the same buffer, since flow-control is performed at VC level, but not at VOQ level.
%Hence, applying restrictions to the adaptive routing is unnecessary, since an efficient queuing scheme achieves good performance (as it happens for DBBM and Flow2SL). By contrast, vFtree needs to apply some of these restrictions when using VOQ-based switches, since the certain adaptivity degrees (i.e., mostly in the second stage) cause that the  vFtree mapping assigns many destinations to many VCs, thus increasing HoL blocking probability.
%As a consequence, another interesting observation is the homogeneous throughput obtained by VOQ-based switches.
%VOQ-based buffer organization switches show homogeneous results when generation rate achieves 60\% for different routing configurations while non-based show an unstable throughput, as in \figurename~\ref{fig_RLFT_HS10_1q-voq}, \ref{fig_RLFT_HS10_dbbm3-voq} ,\ref{fig_RLFT_HS10_vftree3-voq}, and \ref{fig_RLFT_HS10_flow2sl3-voq}.
%
%It is worth mentioning the additional throughput achieved in VOQ-based buffer organization configurations when the generation rates are lower than 40\% when adaptive routing is applied in the second stage (SS), as in \figurename~\ref{fig_RLFT_HS10_1q-voq}, \ref{fig_RLFT_HS10_vftree3-voq}, and \ref{fig_RLFT_HS10_flow2sl3-voq}. 
%Actually, the best results are usually obtained using path restrictions (K/3).

%Figures \ref{fig:RLFT_HS104}, \ref{fig:RLFT_HS104_vTime} \ref{fig:RLFT_HS254} and \ref{fig:RLFT_HS254_vTime} show experiment results for the same scenarios described before when we generate traffic creating four congestion trees  (i.e., traffic HS10-4 and HS25-4 depicted in \figurename~\ref{fig_traffic_HS4}).
%Note that, the four hot-spots in these scenarios are generated, respectively, by 10\% and 25\% of the source end-nodes generating traffic addressed to end-nodes $600$, $3400$, $5200$ and $9500$.
%This is a very strong congestion situation, since we generate four congestion trees whose branches  affect to different VCs at the same time.
%It is even possible that four different branches belonging to four different congestion trees grow within the same VC, producing HoL blocking in all the VCs. 
%Therefore, this situation makes complex, even for restricted routing combined with queuing schemes, to reduce the HoL blocking.


%As in the previous scenario, 1-VC with and without VOQs obtains the worst results, since the HoL blocking dramatically degrades the network performance.
%On the other hand, as the mapping of Flow2SL and DBBM is unfortunate, they achieve worse results than vFtree, since its mapping suits better the traffic situation.
%Note that $D$-mod-$K$ and oblivious routing achieve a performance near to 0\% for DBBM and Flow2SL queuing schemes, regardless the generated traffic load and the use of VOQs in the switches.
%Another reason for this bad performance is that this traffic pattern (see \figurename~\ref{fig_traffic_HS4}) produces very strong congestion situations in the first stage of the upward path and the adaptive and oblivious routing algorithms spread the congestion through the VCs and VOQs in the first stage of the RLFT.

%In the case of DBBM, it maps all the hot-spot destinations to all the VCs, since destination end-node $600$ is mapped to VC0, destinations $3400$ and $5200$ are mapped to VC1, and destination $9500$ is mapped to VC2.
%Note that the best results obtained for DBBM are those achieved by adaptive routing configurations using the triggering restrictions (i.e., TH and 2TH) and not restricting adaptivity to a particular stage (i.e., AS).
%By contrast, Flow2SL maps hot-spot destinations to all the groups in the topology, since destination $600$ is mapped to group $0$ (so to VC0), destinations $3400$ and $5200$ are mapped to group $1$ (so to VC1) and destination $9500$ is mapped to group $2$ (so to VC2).
%Hence, it is highly possible that all VCs suffer from HoL blocking, then degrading the Flow2SL efficiency.
%By contrast, vFtree maps destinations $600$ and $5200$ to VC0, while destinations $3400$ and $9500$ are mapped to VC2.
%Hence, the flows mapped to VC1 do not interact with congested flows in VC0 and VC1.
%We can see this effect in the results achieved by  $D$-mod-$K$.
%As for DBBM and Flow2SL, vFtree also takes advantage of the configurations using combined restrictions, such as ADAPT-2TH-AS-$K/3$.
%When VOQs are used, we can observe the unfortunate mapping effects for DBBM and Flow2SL, described in Section \ref{sec:problem}.

%Figures \ref{fig:RLFT_IHS} and \ref{fig:RLFT_IHS_vTime} show simulation results for traffic generating hot-spots in intermediate stages of the topology (see \figurename~\ref{fig_traffic_HSW}).
%In particular, hot-spots (i.e., congestion tree roots) are generated in the output ports of some switches placed at the second stage of the RLFT.
%As we can see in the figures, this traffic pattern generates a congestion situation where $D$-mod-$K$, oblivious and fully-adaptive routing algorithms (both with and without restrictions) can deal with the HoL blocking appearing in the network.
%For switches without VOQs, vFtree queuing scheme achieves the best results combined with adaptive routing using combined restrictions, such as ADAP-NoTH-SS-K, ADAP-TH-AS-K or ADAP-2TH-AS-$K/3$ outperform 1VC, DBBM and Flow2SL. 
%This behavior is the same for vFtree when using switches with VOQs.
%Note that switches with VOQs using the configuration ADAP-NoTH-SS-K achieve excellent results regardless the use of queuing schemes (see Figures  \ref{fig_RLFT_IHS_1q-voq}, \ref{fig_RLFT_IHS_dbbm3-voq},  \ref{fig_RLFT_IHS_vftree3-voq}, \ref{fig_RLFT_IHS_flow2sl3-voq}).
%Although the use of restricted routing and queuing schemes are not necessary in this traffic scenario when using VOQs, note that their use preserves the performance gains shown in the previous traffic scenarios when using restricted adaptive routing and queuing schemes.
%Therefore, the use of restricted adaptive routing combined with queuing schemes significantly increases the network performance under congested scenarios, compared to when we use deterministic and oblivious routing.

%Figures \ref{fig:RLFT_HS10} and \ref{fig:RLFT_HS25} show simulation results for the queuing schemes described before combined with deterministic ($D$-mod-$K$), oblivious and adaptive routing (restricted and non-restricted).
%We have re-named the non-restricted adaptive routing as ``fully-adaptive'' routing, since it is the way it is called in the literature.
%Also, we have added zooms to some of the figures in order to better reflect the differences among the data series.

\begin{figure*}[!htb]

\begin{subfigure}[!th]{\textwidth}
 \centering 
\includegraphics[width=1.0\textwidth]{leyenda3.pdf}
\end{subfigure}

 \begin{subfigure}[!th]{0.47\textwidth}
 \centering 
\includegraphics[width=0.82\textwidth]{./1q/Graphics/1q_synthetic_hotspot10_throughput_load.pdf}
\caption{1VC.}
\label{fig_RLFT_HS10_1q}
\end{subfigure}
 \begin{subfigure}[!th]{0.47\textwidth}
 \centering 
\includegraphics[width=0.82\textwidth]
{./1q-voq/Graphics/1q-voq_synthetic_hotspot10_throughput_load.pdf}
\caption{1VC and VOQs.}
\label{fig_RLFT_HS10_1q-voq}
\end{subfigure}

 \begin{subfigure}[!th]{0.47\textwidth}
 \centering 
\includegraphics[width=0.82\textwidth]
{./dbbm3/Graphics/dbbm3_synthetic_hotspot10_throughput_load.pdf}
\caption{DBBM.}
\label{fig_RLFT_HS10_dbbm3}
\end{subfigure}
 \begin{subfigure}[!th]{0.47\textwidth}
 \centering 
\includegraphics[width=0.82\textwidth]
{./dbbm3-voq/Graphics/dbbm3-voq_synthetic_hotspot10_throughput_load.pdf}
\caption{DBBM and VOQs.}
\label{fig_RLFT_HS10_dbbm3-voq}
\end{subfigure}

 \begin{subfigure}[!th]{0.47\textwidth}
 \centering 
\includegraphics[width=0.82\textwidth]
{./vftree3/Graphics/vftree3_synthetic_hotspot10_throughput_load.pdf}
\caption{vFtree.}
\label{fig_RLFT_HS10_vftree3}
\end{subfigure}
 \begin{subfigure}[!th]{0.47\textwidth}
 \centering 
\includegraphics[width=0.82\textwidth]
{./vftree3-voq/Graphics/vftree3-voq_synthetic_hotspot10_throughput_load.pdf}
\caption{vFtree and VOQs.}
\label{fig_RLFT_HS10_vftree3-voq}
\end{subfigure}

 \begin{subfigure}[!th]{0.47\textwidth}
 \centering 
\includegraphics[width=0.82\textwidth]
{./flow2sl3/Graphics/flow2sl3_synthetic_hotspot10_throughput_load.pdf}
\caption{Flow2SL.}
\label{fig_RLFT_HS10_flow2sl3}
\end{subfigure}
 \begin{subfigure}[!th]{0.47\textwidth}
 \centering 
\includegraphics[width=0.82\textwidth]
{./flow2sl3-voq/Graphics/flow2sl3-voq_synthetic_hotspot10_throughput_load.pdf}
\caption{Flow2SL and VOQs.}
\label{fig_RLFT_HS10_flow2sl3-voq}
\end{subfigure}

\caption{Normalized Throughput versus Generated Traffic Load in a $11664$-node RLFT under HS10-1 synthetic traffic pattern.}
\label{fig:RLFT_HS10}
\end{figure*}


%%%%%%%%%%%%%%%%
\begin{figure*}[!htb]
\vspace{-.5cm}
\begin{subfigure}[!th]{1\textwidth}
 \centering 
\includegraphics[width=1.0\textwidth]
{leyenda3.pdf}
\end{subfigure}

 \begin{subfigure}[!th]{0.47\textwidth}
 \centering 
\includegraphics[width=0.82\textwidth]
{./1q/Graphics/1q_synthetic_hotspot_throughput_load.pdf}
\caption{1VC.}
\label{fig_RLFT_HS_1q}
\end{subfigure}
 \begin{subfigure}[!th]{0.47\textwidth}
 \centering 
\includegraphics[width=0.82\textwidth]
{./1q-voq/Graphics/1q-voq_synthetic_hotspot_throughput_load.pdf}
\caption{1VC and VOQs.}
\label{fig_RLFT_HS_1q-voq}
\end{subfigure}

 \begin{subfigure}[!th]{0.47\textwidth}
 \centering 
\includegraphics[width=0.82\textwidth]
{./dbbm3/Graphics/dbbm3_synthetic_hotspot_throughput_load.pdf}
\caption{DBBM.}
\label{fig_RLFT_HS_dbbm3}
\end{subfigure}
 \begin{subfigure}[!th]{0.47\textwidth}
 \centering 
\includegraphics[width=0.82\textwidth]
{./dbbm3-voq/Graphics/dbbm3-voq_synthetic_hotspot_throughput_load.pdf}
\caption{DBBM and VOQs.}
\label{fig_RLFT_HS_dbbm3-voq}
\end{subfigure}

 \begin{subfigure}[!th]{0.47\textwidth}
 \centering 
\includegraphics[width=0.82\textwidth]
{./vftree3/Graphics/vftree3_synthetic_hotspot_throughput_load.pdf}
\caption{vFtree.}
\label{fig_RLFT_HS_vftree3}
\end{subfigure}
 \begin{subfigure}[!th]{0.47\textwidth}
 \centering 
\includegraphics[width=0.82\textwidth]
{./vftree3-voq/Graphics/vftree3-voq_synthetic_hotspot_throughput_load.pdf}
\caption{vFtree and VOQs.}
\label{fig_RLFT_HS_vftree3-voq}
\end{subfigure}

 \begin{subfigure}[!th]{0.47\textwidth}
 \centering 
\includegraphics[width=0.82\textwidth]
{./flow2sl3/Graphics/flow2sl3_synthetic_hotspot_throughput_load.pdf}
\caption{Flow2SL.}
\label{fig_RLFT_HS_flow2sl3}
\end{subfigure}
 \begin{subfigure}[!th]{0.47\textwidth}
 \centering 
\includegraphics[width=0.82\textwidth]
{./flow2sl3-voq/Graphics/flow2sl3-voq_synthetic_hotspot_throughput_load.pdf}
\caption{Flow2SL and VOQs.}
\label{fig_RLFT_HS_flow2sl3-voq}
\end{subfigure}

\caption{Normalized Throughput versus Generated Traffic Load in a $11664$-node RLFT under HS25-1 synthetic traffic pattern.}
\label{fig:RLFT_HS25}
\end{figure*}


%%%%%%%%%%%%%%%%
\begin{figure*}[!htb]
\vspace{-.5cm}
\begin{subfigure}[!th]{1\textwidth}
 \centering 
\includegraphics[width=1.0\textwidth]
{leyenda3.pdf}
\end{subfigure}

 \begin{subfigure}[!th]{0.47\textwidth}
 \centering 
\includegraphics[width=0.82\textwidth]
{./1q/Graphics/1q_synthetic_hotspot104_throughput_load.pdf}
\caption{1VC.}
\label{fig_RLFT_HS104_1q}
\end{subfigure}
 \begin{subfigure}[!th]{0.47\textwidth}
 \centering 
\includegraphics[width=0.82\textwidth]
{./1q-voq/Graphics/1q-voq_synthetic_hotspot104_throughput_load.pdf}
\caption{1VC and VOQs.}
\label{fig_RLFT_HS104_1q-voq}
\end{subfigure}

 \begin{subfigure}[!th]{0.47\textwidth}
 \centering 
\includegraphics[width=0.82\textwidth]
{./dbbm3/Graphics/dbbm3_synthetic_hotspot104_throughput_load.pdf}
\caption{DBBM.}
\label{fig_RLFT_HS104_dbbm3}
\end{subfigure}
 \begin{subfigure}[!th]{0.47\textwidth}
 \centering 
\includegraphics[width=0.82\textwidth]
{./dbbm3-voq/Graphics/dbbm3-voq_synthetic_hotspot104_throughput_load.pdf}
\caption{DBBM and VOQs.}
\label{fig_RLFT_HS104_dbbm3-voq}
\end{subfigure}

 \begin{subfigure}[!th]{0.47\textwidth}
 \centering 
\includegraphics[width=0.82\textwidth]
{./vftree3/Graphics/vftree3_synthetic_hotspot104_throughput_load.pdf}
\caption{vFtree.}
\label{fig_RLFT_HS104_vftree3}
\end{subfigure}
 \begin{subfigure}[!th]{0.47\textwidth}
 \centering 
\includegraphics[width=0.82\textwidth]
{./vftree3-voq/Graphics/vftree3-voq_synthetic_hotspot104_throughput_load.pdf}
\caption{vFtree and VOQs.}
\label{fig_RLFT_HS104_vftree3-voq}
\end{subfigure}

 \begin{subfigure}[!th]{0.47\textwidth}
 \centering 
\includegraphics[width=0.82\textwidth]
{./flow2sl3/Graphics/flow2sl3_synthetic_hotspot104_throughput_load.pdf}
\caption{Flow2SL.}
\label{fig_RLFT_HS104_flow2sl3}
\end{subfigure}
 \begin{subfigure}[!th]{0.47\textwidth}
 \centering 
\includegraphics[width=0.82\textwidth]
{./flow2sl3-voq/Graphics/flow2sl3-voq_synthetic_hotspot104_throughput_load.pdf}
\caption{Flow2SL and VOQs.}
\label{fig_RLFT_HS104_flow2sl3-voq}
\end{subfigure}

\caption{Normalized Throughput versus Generated Traffic Load in a $11664$-node RLFT under HS10-4 synthetic traffic pattern.}
\label{fig:RLFT_HS104}
\end{figure*}

%%%%%%%%%%%%%%%%
\begin{figure*}[!htb]
\vspace{-.5cm}
\begin{subfigure}[!th]{1\textwidth}
 \centering 
\includegraphics[width=1.0\textwidth]
{leyenda3.pdf}
\end{subfigure}

 \begin{subfigure}[!th]{0.47\textwidth}
 \centering 
\includegraphics[width=0.82\textwidth]
{./1q/Graphics/1q_synthetic_hotspot254_throughput_load.pdf}
\caption{1VC.}
\label{fig_RLFT_HS254_1q}
\end{subfigure}
 \begin{subfigure}[!th]{0.47\textwidth}
 \centering 
\includegraphics[width=0.82\textwidth]
{./1q-voq/Graphics/1q-voq_synthetic_hotspot254_throughput_load.pdf}
\caption{1VC and VOQs.}
\label{fig_RLFT_HS254_1q-voq}
\end{subfigure}

 \begin{subfigure}[!th]{0.47\textwidth}
 \centering 
\includegraphics[width=0.82\textwidth]
{./dbbm3/Graphics/dbbm3_synthetic_hotspot254_throughput_load.pdf}
\caption{DBBM.}
\label{fig_RLFT_HS254_dbbm3}
\end{subfigure}
 \begin{subfigure}[!th]{0.47\textwidth}
 \centering 
\includegraphics[width=0.82\textwidth]
{./dbbm3-voq/Graphics/dbbm3-voq_synthetic_hotspot254_throughput_load.pdf}
\caption{DBBM and VOQs.}
\label{fig_RLFT_HS254_dbbm3-voq}
\end{subfigure}

 \begin{subfigure}[!th]{0.47\textwidth}
 \centering 
\includegraphics[width=0.82\textwidth]
{./vftree3/Graphics/vftree3_synthetic_hotspot254_throughput_load.pdf}
\caption{vFtree.}
\label{fig_RLFT_HS254_vftree3}
\end{subfigure}
 \begin{subfigure}[!th]{0.47\textwidth}
 \centering 
\includegraphics[width=0.82\textwidth]
{./vftree3-voq/Graphics/vftree3-voq_synthetic_hotspot254_throughput_load.pdf}
\caption{vFtree and VOQs.}
\label{fig_RLFT_HS254_vftree3-voq}
\end{subfigure}

 \begin{subfigure}[!th]{0.47\textwidth}
 \centering 
\includegraphics[width=0.82\textwidth]
{./flow2sl3/Graphics/flow2sl3_synthetic_hotspot254_throughput_load.pdf}
\caption{Flow2SL.}
\label{fig_RLFT_HS254_flow2sl3}
\end{subfigure}
 \begin{subfigure}[!th]{0.47\textwidth}
 \centering 
\includegraphics[width=0.82\textwidth]
{./flow2sl3-voq/Graphics/flow2sl3-voq_synthetic_hotspot254_throughput_load.pdf}
\caption{Flow2SL and VOQs.}
\label{fig_RLFT_HS254_flow2sl3-voq}
\end{subfigure}

\caption{Normalized Throughput versus Generated Traffic Load in a $11664$-node RLFT under HS25-4 synthetic traffic pattern.}
\label{fig:RLFT_HS254}
\end{figure*}



%%%%%%%%%%%%%%%%
\begin{figure*}[!htb]
\vspace{-.5cm}
\begin{subfigure}[!th]{1\textwidth}
 \centering 
\includegraphics[width=1.0\textwidth]
{leyenda3.pdf}
\end{subfigure}

 \begin{subfigure}[!th]{0.47\textwidth}
 \centering 
\includegraphics[width=0.82\textwidth]
{./1q/Graphics/1q_synthetic_interhotspot_throughput_load.pdf}
\caption{1VC.}
\label{fig_RLFT_IHS_1q}
\end{subfigure}
 \begin{subfigure}[!th]{0.47\textwidth}
 \centering 
\includegraphics[width=0.82\textwidth]
{./1q-voq/Graphics/1q-voq_synthetic_interhotspot_throughput_load.pdf}
\caption{1VC and VOQs.}
\label{fig_RLFT_IHS_1q-voq}
\end{subfigure}

 \begin{subfigure}[!th]{0.47\textwidth}
 \centering 
\includegraphics[width=0.82\textwidth]
{./dbbm3/Graphics/dbbm3_synthetic_interhotspot_throughput_load.pdf}
\caption{DBBM.}
\label{fig_RLFT_IHS_dbbm3}
\end{subfigure}
 \begin{subfigure}[!th]{0.47\textwidth}
 \centering 
\includegraphics[width=0.82\textwidth]
{./dbbm3-voq/Graphics/dbbm3-voq_synthetic_interhotspot_throughput_load.pdf}
\caption{DBBM and VOQs.}
\label{fig_RLFT_IHS_dbbm3-voq}
\end{subfigure}

 \begin{subfigure}[!th]{0.47\textwidth}
 \centering 
\includegraphics[width=0.82\textwidth]
{./vftree3/Graphics/vftree3_synthetic_interhotspot_throughput_load.pdf}
\caption{vFtree.}
\label{fig_RLFT_IHS_vftree3}
\end{subfigure}
 \begin{subfigure}[!th]{0.47\textwidth}
 \centering 
\includegraphics[width=0.82\textwidth]
{./vftree3-voq/Graphics/vftree3-voq_synthetic_interhotspot_throughput_load.pdf}
\caption{vFtree and VOQs.}
\label{fig_RLFT_IHS_vftree3-voq}
\end{subfigure}

 \begin{subfigure}[!th]{0.47\textwidth}
 \centering 
\includegraphics[width=0.82\textwidth]
{./flow2sl3/Graphics/flow2sl3_synthetic_interhotspot_throughput_load.pdf}
\caption{Flow2SL.}
\label{fig_RLFT_IHS_flow2sl3}
\end{subfigure}
 \begin{subfigure}[!th]{0.47\textwidth}
 \centering 
\includegraphics[width=0.82\textwidth]
{./flow2sl3-voq/Graphics/flow2sl3-voq_synthetic_interhotspot_throughput_load.pdf}
\caption{Flow2SL and VOQs.}
\label{fig_RLFT_IHS_flow2sl3-voq}
\end{subfigure}

\caption{Normalized Throughput versus Generated Traffic Load in a $11664$-node RLFT under IHS synthetic traffic pattern.}
\label{fig:RLFT_IHS}
\end{figure*}



\begin{figure*}[!htb]
\vspace{-.5cm}
\begin{subfigure}[!th]{1\textwidth}
 \centering 
\includegraphics[width=1.0\textwidth]
{leyenda3.pdf}
\end{subfigure}

 \begin{subfigure}[!th]{1\textwidth}
 \centering 
\includegraphics[width=0.8\textwidth]
{./img/1q.pdf}
\caption{1VC.}
\label{fig_histo_1q}
\end{subfigure}
 \begin{subfigure}[!th]{1\textwidth}
 \centering 
\includegraphics[width=0.8\textwidth]
{./img/1q-voq.pdf}
\caption{1VC and VOQs.}
\label{fig_histo_1q-voq}
\end{subfigure}
\caption{Normalized Throughput after warmup period in a $11664$-node RLFT using 1Q and 1Q-VOQ.}
\label{fig:RLFT_histo_1q}
\end{figure*}

\begin{figure*}[!htb]
\vspace{-.5cm}
\begin{subfigure}[!th]{1\textwidth}
 \centering 
\includegraphics[width=1.0\textwidth]
{leyenda3.pdf}
\end{subfigure}

 \begin{subfigure}[!th]{1\textwidth}
 \centering 
\includegraphics[width=0.8\textwidth]
{./img/dbbm3.pdf}
\caption{DBBM.}
\label{fig_histo_dbbm}
\end{subfigure}
 \begin{subfigure}[!th]{1\textwidth}
 \centering 
\includegraphics[width=0.8\textwidth]
{./img/dbbm3-voq.pdf}
\caption{DBBM and VOQs.}
\label{fig_histo_dbbm-voq}
\end{subfigure}
\caption{Normalized Throughput after warmup period in a $11664$-node RLFT using DBBM and DBBM-VOQ.}
\label{fig:RLFT_histo_dbbm}
\end{figure*}

\begin{figure*}[!ht]
\vspace{-.5cm}
\begin{subfigure}[!th]{1\textwidth}
 \centering 
\includegraphics[width=1.0\textwidth]
{leyenda3.pdf}
\end{subfigure}

 \begin{subfigure}[!th]{1\textwidth}
 \centering 
\includegraphics[width=0.8\textwidth]
{./img/vftree3.pdf}
\caption{vFtree.}
\label{fig_histo_vftree}
\end{subfigure}
 \begin{subfigure}[!th]{1\textwidth}
 \centering 
\includegraphics[width=0.8\textwidth]
{./img/vftree3-voq.pdf}
\caption{vFtree and VOQs.}
\label{fig_histo_vftree-voq}
\end{subfigure}
\caption{Normalized Throughput after warmup period in a $11664$-node RLFT using vFtree and vFtree-VOQ.}
\label{fig:RLFT_histo_vftree}
\end{figure*}


\begin{figure*}[!ht]
\vspace{-.5cm}
\begin{subfigure}[!th]{1\textwidth}
 \centering 
\includegraphics[width=1.0\textwidth]
{leyenda3.pdf}
\end{subfigure}

 \begin{subfigure}[!th]{1\textwidth}
 \centering 
\includegraphics[width=0.8\textwidth]
{./img/flow2sl3.pdf}
\caption{Flow2SL.}
\label{fig_histo_flow2sl}
\end{subfigure}
 \begin{subfigure}[!th]{1\textwidth}
 \centering 
\includegraphics[width=0.8\textwidth]
{./img/flow2sl3-voq.pdf}
\caption{Flow2SL and VOQs.}
\label{fig_histo_flow2sl-voq}
\end{subfigure}
\caption{Normalized Throughput after warmup period in a $11664$-node RLFT using Flow2SL and Flow2SL-VOQ.}
\label{fig:RLFT_histo_flow2sl}
\end{figure*}

\end{document}

\endinput