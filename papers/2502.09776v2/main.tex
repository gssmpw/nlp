\pdfoutput=1
%%
%% This is file `sample-manuscript.tex',
%% generated with the docstrip utility.
%%
%% The original source files were:
%%
%% samples.dtx  (with options: `manuscript')
%% 
%% IMPORTANT NOTICE:
%% 
%% For the copyright see the source file.
%% 
%% Any modified versions of this file must be renamed
%% with new filenames distinct from sample-manuscript.tex.
%% 
%% For distribution of the original source see the terms
%% for copying and modification in the file samples.dtx.
%% 
%% This generated file may be distributed as long as the
%% original source files, as listed above, are part of the
%% same distribution. (The sources need not necessarily be
%% in the same archive or directory.)
%%
%% Commands for TeXCount
%TC:macro \cite [option:text,text]
%TC:macro \citep [option:text,text]
%TC:macro \citet [option:text,text]
%TC:envir table 0 1
%TC:envir table* 0 1
%TC:envir tabular [ignore] word
%TC:envir displaymath 0 word
%TC:envir math 0 word
%TC:envir comment 0 0
%%
%%
%% The first command in your LaTeX source must be the \documentclass command.
%%%% Small single column format, used for CIE, CSUR, DTRAP, JACM, JDIQ, JEA, JERIC, JETC, PACMCGIT, TAAS, TACCESS, TACO, TALG, TALLIP (formerly TALIP), TCPS, TDSCI, TEAC, TECS, TELO, THRI, TIIS, TIOT, TISSEC, TIST, TKDD, TMIS, TOCE, TOCHI, TOCL, TOCS, TOCT, TODAES, TODS, TOIS, TOIT, TOMACS, TOMM (formerly TOMCCAP), TOMPECS, TOMS, TOPC, TOPLAS, TOPS, TOS, TOSEM, TOSN, TQC, TRETS, TSAS, TSC, TSLP, TWEB.
% \documentclass[acmsmall]{acmart}

%%%% Large single column format, used for IMWUT, JOCCH, PACMPL, POMACS, TAP, PACMHCI
% \documentclass[acmlarge,screen]{acmart}

%%%% Large double column format, used for TOG
% \documentclass[acmtog, authorversion]{acmart}

%%%% Generic manuscript mode, required for submission
%%%% and peer review
%\documentclass[manuscript,screen, review, anonymous]{acmart}
\documentclass[acmsmall]{acmart}
%% Fonts used in the template cannot be substituted; margin 
%% adjustments are not allowed.
%%
%% \BibTeX command to typeset BibTeX logo in the docs
\AtBeginDocument{%
  \providecommand\BibTeX{{%
    \normalfont B\kern-0.5em{\scshape i\kern-0.25em b}\kern-0.8em\TeX}}}
\usepackage{multirow}
\usepackage{xcolor} 
\usepackage{graphicx}
\usepackage{subcaption}
\newenvironment{renv}
  {\color{black}} % Code to execute at the beginning
  {} % Code to execute at the end

\newcommand{\todo}[1]{\textcolor{purple}{\{TODO: #1\}}}
\newcommand{\italquote}[1]{\begin{quote}``\textit{#1}''\end{quote}}
%\newcommand{\redacted}[1]{\textbf{REDACTED}}
\newcommand{\redacted}[1]{#1}
\newcommand{\topic}[1]{\textcolor{black}{#1}}
\newcommand{\revision}[1]{\textcolor{black}{#1}}
%\newcommand{\revisionCSCW}[1]{\textcolor{red}{#1}}

%% Rights management information.  This information is sent to you
%% when you complete the rights form.  These commands have SAMPLE
%% values in them; it is your responsibility as an author to replace
%% the commands and values with those provided to you when you
%% complete the rights form.
\setcopyright{acmcopyright}
\copyrightyear{2018}
\acmYear{2018}
\acmDOI{XXXXXXX.XXXXXXX}

%% These commands are for a PROCEEDINGS abstract or paper.
\acmConference[Conference acronym 'XX]{Make sure to enter the correct
  conference title from your rights confirmation emai}{June 03--05,
  2018}{Woodstock, NY}
%
%  Uncomment \acmBooktitle if the title of the proceedings is different
%  from ``Proceedings of ...''!
%
\acmBooktitle{Woodstock '18: ACM Symposium on Neural Gaze Detection,
 June 03--05, 2018, Woodstock, NY} 
\acmPrice{15.00}
\acmISBN{978-1-4503-XXXX-X/18/06}


%%
%% Submission ID.
%% Use this when submitting an article to a sponsored event. You'll
%% receive a unique submission ID from the organizers
%% of the event, and this ID should be used as the parameter to this command.
%%\acmSubmissionID{123-A56-BU3}

%%
%% For managing citations, it is recommended to use bibliography
%% files in BibTeX format.
%%
%% You can then either use BibTeX with the ACM-Reference-Format style,
%% or BibLaTeX with the acmnumeric or acmauthoryear sytles, that include
%% support for advanced citation of software artefact from the
%% biblatex-software package, also separately available on CTAN.
%%
%% Look at the sample-*-biblatex.tex files for templates showcasing
%% the biblatex styles.
%%

%%
%% The majority of ACM publications use numbered citations and
%% references.  The command \citestyle{authoryear} switches to the
%% "author year" style.
%%
%% If you are preparing content for an event
%% sponsored by ACM SIGGRAPH, you must use the "author year" style of
%% citations and references.
%% Uncommenting
%% the next command will enable that style.
%%\citestyle{acmauthoryear}

%%
%% end of the preamble, start of the body of the document source.
\begin{document}

%%
%% The "title" command has an optional parameter,
%% allowing the author to define a "short title" to be used in page headers.
\title[Disruptors of Engagement with DMH Tools]{Investigating the Role of Situational Disruptors in Engagement with Digital Mental Health Tools}

%%
%% The "author" command and its associated commands are used to define
%% the authors and their affiliations.
%% Of note is the shared affiliation of the first two authors, and the
%% "authornote" and "authornotemark" commands
%% used to denote shared contribution to the research.

\author{Ananya Bhattacharjee}
\affiliation{%
  \institution{Computer Science, University of Toronto}
  \city{Toronto}
  \state{Ontario}
  \country{Canada}
}

\author{Joseph Jay Williams}
\affiliation{%
  \institution{Computer Science, University of Toronto}
  \city{Toronto}
  \state{Ontario}
  \country{Canada}
}

\author{Miranda L. Beltzer}
\affiliation{%
  \institution{Preventive Medicine, Northwestern University}
  \city{Chicago}
  \state{Illinois}
  \country{USA}
}

\author{Jonah Meyerhoff}
\affiliation{%
  \institution{Preventive Medicine, Northwestern University}
  \city{Chicago}
  \state{Illinois}
  \country{USA}
}

\author{Harsh Kumar}
\affiliation{%
  \institution{Computer Science, University of Toronto}
  \city{Toronto}
  \state{Ontario}
  \country{Canada}
}
%\email{harsh@cs.toronto.edu}

\author{Haochen Song}
\affiliation{%
  \institution{Statistical Sciences, University of Toronto}
  \city{Toronto}
  \state{Ontario}
  \country{Canada}
}
%\email{fred.song@mail.utoronto.ca}

\author{David C. Mohr}
\affiliation{%
  \institution{Preventive Medicine, Northwestern University}
  \city{Chicago}
  \state{Illinois}
  \country{USA}
}


\author{Alex Mariakakis}
\affiliation{%
  \institution{Computer Science, University of Toronto}
  \city{Toronto}
  \state{Ontario}
  \country{Canada}
}

\author{Rachel Kornfield}
\affiliation{%
  \institution{Preventive Medicine, Northwestern University}
  \city{Chicago}
  \state{Illinois}
  \country{USA}
}

%%
%% By default, the full list of authors will be used in the page
%% headers. Often, this list is too long, and will overlap
%% other information printed in the page headers. This command allows
%% the author to define a more concise list
%% of authors' names for this purpose.
\renewcommand{\shortauthors}{Bhattacharjee et al.}

%%
%% The abstract is a short summary of the work to be presented in the
%% article.
\begin{abstract}
\begin{abstract}  
Test time scaling is currently one of the most active research areas that shows promise after training time scaling has reached its limits.
Deep-thinking (DT) models are a class of recurrent models that can perform easy-to-hard generalization by assigning more compute to harder test samples.
However, due to their inability to determine the complexity of a test sample, DT models have to use a large amount of computation for both easy and hard test samples.
Excessive test time computation is wasteful and can cause the ``overthinking'' problem where more test time computation leads to worse results.
In this paper, we introduce a test time training method for determining the optimal amount of computation needed for each sample during test time.
We also propose Conv-LiGRU, a novel recurrent architecture for efficient and robust visual reasoning. 
Extensive experiments demonstrate that Conv-LiGRU is more stable than DT, effectively mitigates the ``overthinking'' phenomenon, and achieves superior accuracy.
\end{abstract}  

\end{abstract}

%%
%% The code below is generated by the tool at http://dl.acm.org/ccs.cfm.
%% Please copy and paste the code instead of the example below.
%%
\begin{CCSXML}
<ccs2012>
<concept>
<concept_id>10003120.10003121.10011748</concept_id>
<concept_desc>Human-centered computing~Empirical studies in HCI</concept_desc>
<concept_significance>500</concept_significance>
</concept>
</ccs2012>
\end{CCSXML}

\ccsdesc[500]{Human-centered computing~Empirical studies in HCI}

%%
%% Keywords. The author(s) should pick words that accurately describe
%% the work being presented. Separate the keywords with commas.
\keywords{Digital mental health, engagement, text message, design workshop, goal setting, prioritization, context}

%% A "teaser" image appears between the author and affiliation
%% information and the body of the document, and typically spans the
%% page.


%%
%% This command processes the author and affiliation and title
%% information and builds the first part of the formatted document.
\maketitle
\section{Introduction}


\begin{figure}[t]
\centering
\includegraphics[width=0.6\columnwidth]{figures/evaluation_desiderata_V5.pdf}
\vspace{-0.5cm}
\caption{\systemName is a platform for conducting realistic evaluations of code LLMs, collecting human preferences of coding models with real users, real tasks, and in realistic environments, aimed at addressing the limitations of existing evaluations.
}
\label{fig:motivation}
\end{figure}

\begin{figure*}[t]
\centering
\includegraphics[width=\textwidth]{figures/system_design_v2.png}
\caption{We introduce \systemName, a VSCode extension to collect human preferences of code directly in a developer's IDE. \systemName enables developers to use code completions from various models. The system comprises a) the interface in the user's IDE which presents paired completions to users (left), b) a sampling strategy that picks model pairs to reduce latency (right, top), and c) a prompting scheme that allows diverse LLMs to perform code completions with high fidelity.
Users can select between the top completion (green box) using \texttt{tab} or the bottom completion (blue box) using \texttt{shift+tab}.}
\label{fig:overview}
\end{figure*}

As model capabilities improve, large language models (LLMs) are increasingly integrated into user environments and workflows.
For example, software developers code with AI in integrated developer environments (IDEs)~\citep{peng2023impact}, doctors rely on notes generated through ambient listening~\citep{oberst2024science}, and lawyers consider case evidence identified by electronic discovery systems~\citep{yang2024beyond}.
Increasing deployment of models in productivity tools demands evaluation that more closely reflects real-world circumstances~\citep{hutchinson2022evaluation, saxon2024benchmarks, kapoor2024ai}.
While newer benchmarks and live platforms incorporate human feedback to capture real-world usage, they almost exclusively focus on evaluating LLMs in chat conversations~\citep{zheng2023judging,dubois2023alpacafarm,chiang2024chatbot, kirk2024the}.
Model evaluation must move beyond chat-based interactions and into specialized user environments.



 

In this work, we focus on evaluating LLM-based coding assistants. 
Despite the popularity of these tools---millions of developers use Github Copilot~\citep{Copilot}---existing
evaluations of the coding capabilities of new models exhibit multiple limitations (Figure~\ref{fig:motivation}, bottom).
Traditional ML benchmarks evaluate LLM capabilities by measuring how well a model can complete static, interview-style coding tasks~\citep{chen2021evaluating,austin2021program,jain2024livecodebench, white2024livebench} and lack \emph{real users}. 
User studies recruit real users to evaluate the effectiveness of LLMs as coding assistants, but are often limited to simple programming tasks as opposed to \emph{real tasks}~\citep{vaithilingam2022expectation,ross2023programmer, mozannar2024realhumaneval}.
Recent efforts to collect human feedback such as Chatbot Arena~\citep{chiang2024chatbot} are still removed from a \emph{realistic environment}, resulting in users and data that deviate from typical software development processes.
We introduce \systemName to address these limitations (Figure~\ref{fig:motivation}, top), and we describe our three main contributions below.


\textbf{We deploy \systemName in-the-wild to collect human preferences on code.} 
\systemName is a Visual Studio Code extension, collecting preferences directly in a developer's IDE within their actual workflow (Figure~\ref{fig:overview}).
\systemName provides developers with code completions, akin to the type of support provided by Github Copilot~\citep{Copilot}. 
Over the past 3 months, \systemName has served over~\completions suggestions from 10 state-of-the-art LLMs, 
gathering \sampleCount~votes from \userCount~users.
To collect user preferences,
\systemName presents a novel interface that shows users paired code completions from two different LLMs, which are determined based on a sampling strategy that aims to 
mitigate latency while preserving coverage across model comparisons.
Additionally, we devise a prompting scheme that allows a diverse set of models to perform code completions with high fidelity.
See Section~\ref{sec:system} and Section~\ref{sec:deployment} for details about system design and deployment respectively.



\textbf{We construct a leaderboard of user preferences and find notable differences from existing static benchmarks and human preference leaderboards.}
In general, we observe that smaller models seem to overperform in static benchmarks compared to our leaderboard, while performance among larger models is mixed (Section~\ref{sec:leaderboard_calculation}).
We attribute these differences to the fact that \systemName is exposed to users and tasks that differ drastically from code evaluations in the past. 
Our data spans 103 programming languages and 24 natural languages as well as a variety of real-world applications and code structures, while static benchmarks tend to focus on a specific programming and natural language and task (e.g. coding competition problems).
Additionally, while all of \systemName interactions contain code contexts and the majority involve infilling tasks, a much smaller fraction of Chatbot Arena's coding tasks contain code context, with infilling tasks appearing even more rarely. 
We analyze our data in depth in Section~\ref{subsec:comparison}.



\textbf{We derive new insights into user preferences of code by analyzing \systemName's diverse and distinct data distribution.}
We compare user preferences across different stratifications of input data (e.g., common versus rare languages) and observe which affect observed preferences most (Section~\ref{sec:analysis}).
For example, while user preferences stay relatively consistent across various programming languages, they differ drastically between different task categories (e.g. frontend/backend versus algorithm design).
We also observe variations in user preference due to different features related to code structure 
(e.g., context length and completion patterns).
We open-source \systemName and release a curated subset of code contexts.
Altogether, our results highlight the necessity of model evaluation in realistic and domain-specific settings.





\putsec{related}{Related Work}

\noindent \textbf{Efficient Radiance Field Rendering.}
%
The introduction of Neural Radiance Fields (NeRF)~\cite{mil:sri20} has
generated significant interest in efficient 3D scene representation and
rendering for radiance fields.
%
Over the past years, there has been a large amount of research aimed at
accelerating NeRFs through algorithmic or software
optimizations~\cite{mul:eva22,fri:yu22,che:fun23,sun:sun22}, and the
development of hardware
accelerators~\cite{lee:cho23,li:li23,son:wen23,mub:kan23,fen:liu24}.
%
The state-of-the-art method, 3D Gaussian splatting~\cite{ker:kop23}, has
further fueled interest in accelerating radiance field
rendering~\cite{rad:ste24,lee:lee24,nie:stu24,lee:rho24,ham:mel24} as it
employs rasterization primitives that can be rendered much faster than NeRFs.
%
However, previous research focused on software graphics rendering on
programmable cores or building dedicated hardware accelerators. In contrast,
\name{} investigates the potential of efficient radiance field rendering while
utilizing fixed-function units in graphics hardware.
%
To our knowledge, this is the first work that assesses the performance
implications of rendering Gaussian-based radiance fields on the hardware
graphics pipeline with software and hardware optimizations.

%%%%%%%%%%%%%%%%%%%%%%%%%%%%%%%%%%%%%%%%%%%%%%%%%%%%%%%%%%%%%%%%%%%%%%%%%%
\myparagraph{Enhancing Graphics Rendering Hardware.}
%
The performance advantage of executing graphics rendering on either
programmable shader cores or fixed-function units varies depending on the
rendering methods and hardware designs.
%
Previous studies have explored the performance implication of graphics hardware
design by developing simulation infrastructures for graphics
workloads~\cite{bar:gon06,gub:aam19,tin:sax23,arn:par13}.
%
Additionally, several studies have aimed to improve the performance of
special-purpose hardware such as ray tracing units in graphics
hardware~\cite{cho:now23,liu:cha21} and proposed hardware accelerators for
graphics applications~\cite{lu:hua17,ram:gri09}.
%
In contrast to these works, which primarily evaluate traditional graphics
workloads, our work focuses on improving the performance of volume rendering
workloads, such as Gaussian splatting, which require blending a huge number of
fragments per pixel.

%%%%%%%%%%%%%%%%%%%%%%%%%%%%%%%%%%%%%%%%%%%%%%%%%%%%%%%%%%%%%%%%%%%%%%%%%%
%
In the context of multi-sample anti-aliasing, prior work proposed reducing the
amount of redundant shading by merging fragments from adjacent triangles in a
mesh at the quad granularity~\cite{fat:bou10}.
%
While both our work and quad-fragment merging (QFM)~\cite{fat:bou10} aim to
reduce operations by merging quads, our proposed technique differs from QFM in
many aspects.
%
Our method aims to blend \emph{overlapping primitives} along the depth
direction and applies to quads from any primitive. In contrast, QFM merges quad
fragments from small (e.g., pixel-sized) triangles that \emph{share} an edge
(i.e., \emph{connected}, \emph{non-overlapping} triangles).
%
As such, QFM is not applicable to the scenes consisting of a number of
unconnected transparent triangles, such as those in 3D Gaussian splatting.
%
In addition, our method computes the \emph{exact} color for each pixel by
offloading blending operations from ROPs to shader units, whereas QFM
\emph{approximates} pixel colors by using the color from one triangle when
multiple triangles are merged into a single quad.


\section{Procedure of Study 1}
We conducted our first study with 20 participants in an eight-week text messaging program designed to improve psychological wellbeing. The duration of the program enabled us to examine broader social contexts that hindered and shaped their engagement with a DMH tool. The study received approval from the Research Ethics Board at \redacted{Northwestern University}, and its logistics are described below.

\subsection{Participants}

Participant recruitment for this study was facilitated by \redacted{our research center's} Online Research Registry as well as a web-based screening platform of a nonprofit mental health advocacy organization called \redacted{Mental Health America (MHA)}. %Participants were recruited between August and September 2022. 
The study was presented as a field trial to test a text-messaging tool constructed to aid individuals in managing their symptoms of depression and anxiety. Across both recruitment sources, participants were required to be at least 18 years old, own a mobile phone, be a resident of the United States, and not currently receiving psychotherapy or planning to start psychotherapy in the study timeframe. 

Participants also completed self-screening surveys for depression and anxiety to confirm that they reflect potential users who stand to benefit from engaging with a DMH tool. Eligibility was determined based on self-reported scores from the Patient Health Questionnaire-9 (PHQ-9) \cite{kroenke2001phq} and the General Anxiety Disorder-7 (GAD-7) \cite{williams2014gad}. Those who scored 10 or higher on either questionnaire were considered eligible for participation.


%Participants recruited through the advocacy organization took the Patient Health Questionnaire-9 (PHQ-9) \cite{kroenke2001phq} and the General Anxiety Disorder-7 (GAD-7) \cite{williams2014gad}, while those recruited through the registry were only given the PHQ-9. Those who indicated moderate to severe symptoms according to PHQ-9 or GAD-7 — determined by a score of 10 or more for both questionnaires — were eligible to participate. 
% Subsequently, prospective participants were directed to an additional screening survey. Those recruited through the registry were emailed an invitation to learn about the study and complete screening procedures if interested. Those screening procedures included the PHQ-9, with an individual being eligible if they had a score of 10 or higher.

The final cohort comprised 20 participants with an average age of $31.0\pm3.0$ years. The participants identified with multiple genders (17 women, 2 men, 1 non-binary) and several racial groups (10 White, 1 Black/African American, 1 Asian, 1 American Indian/Alaskan Native, 3 mixed race, and 4 undisclosed). 
\topic{Participants were asked whether a specialist or healthcare provider had ever diagnosed them with major depression or anxiety disorder; 11 participants (55.0\%) reported being diagnosed with depression, and 8 participants (40.0\%) reported being diagnosed with an anxiety disorder. However, a formal diagnosis of depression or anxiety was not required for participation.} We refer to these participants as FP1--FP20. 



%. For depression, 11 participants (55.0\%) responded "Yes", 8 participants (40.0\%) responded "No", and 1 participant (5.0\%) responded "I don't know". For anxiety, 8 participants (40.0\%) responded "Yes", while 12 participants (60.0\%) responded "No".}
%\topic{Participants were asked whether they had ever been diagnosed with a mental health disorder by a specialist or healthcare provider;
%11 participants (55.0\%) reported being diagnosed with depression, and 8 participants (40.0\%) reported being diagnosed with an anxiety disorder.}

% \topic{Additional details about participant characteristics are presented in Table \ref{tab:participant_characteristics}.}

% \begin{table}[h!]
% \caption{Participant Characteristics}
% \label{tab:participant_characteristics}
% \centering
% \begin{tabular}{|c|c|}\hline
% \textbf{Characteristic} & \textbf{Total Sample ($n = 20$)} \\ \hline
% History of Depression or Anxiety Diagnosis\# (\%) & X (X\%) \\\hline
% Previous history of psychotherapy \# (\%) & 13 (65\%) \\\hline
% History of psychiatric medication use \# (\%) & 10 (50\%) \\\hline
% Current psychiatric medication use \# (\%) & 3 (15\%) \\ \hline
% \end{tabular}


% \end{table}

%The study was advertised as a research trial of a text messaging prototype designed to support self-management of depression and anxiety symptoms. Individuals who showed at least moderate levels of depression or anxiety symptoms according to the Patient Health Questionnaire-9 (PHQ-9) \cite{kroenke2001phq} and the General Anxiety Disorder-7 (GAD-7) \cite{williams2014gad} (i.e., scores of 10 or higher) were invited to learn more about study activities by following a link presented alongside their results. Potential participants completed an additional screening survey and were eligible if they were located in the United States, were above 18 years old, and owned a mobile phone.

\subsection{Overview of the Text Messaging Program}

After undergoing the eligibility and consent processes, participants were enrolled in the Small Steps SMS program \cite{meyerhoff2024small} -- an automated, interactive program created by \citet{meyerhoff2024small} to build users' capability to self-manage depression and anxiety symptoms. \revision{The program sought to sustain engagement by offering a variety of ways for users to interact \cite{meyerhoff2024small}.} It offered daily interactive dialogues that taught users how to apply 11 evidence-based psychological strategies. The presentation of these strategies largely drew from works by \citet{kornfield2022involving} and \citet{meyerhoff2022system} 
and were grounded in theories including acceptance and commitment therapy \cite{hayes2006acceptance}, cognitive behavioral therapy \cite{willson2019cognitive}, and social rhythm therapy \cite{frank2022interpersonal}.
%The eight-week structure drew inspiration from the previous research by \citet{meyerhoff2022system}. 

The program provided active personalization, enabling users to select which psychological strategies they wanted to use over time. Within a subset of dialogues, users had the option to choose between different topics or exercises.
Figure \ref{fig:conversations} illustrates a subset of the message dialogues sent to users. Within each dialogue, users responded to sequential messages that used branching logic to tailor replies from the program. The program's daily dialogues varied over time in their approach.
Some of the messages included:
\begin{itemize}
    \item Introductions to specific psychological strategies and prompts to participate in skill-building exercises \cite{meyerhoff2022system},
    \item Stories of individuals who had successfully harnessed these strategies to navigate challenges \cite{bhattacharjee2022kind},
    \item Invitations to compose and receive supportive texts for peers \cite{meyerhoff2022system},
    \item Succinct tips or prompts for self-reflection, including messages that directed users to more extensive psychoeducational resources \cite{bhattacharjee2023investigating}, and 
    \item Prompts for users to think about or respond to open-ended questions \cite{bhattacharjee2023investigating} 
\end{itemize}
A more thorough description of the Small Steps SMS program can be found in \citet{meyerhoff2024small}, \topic{and additional diagrams illustrating the sequence of certain text messaging dialogues are provided in Appendix~\ref{sec: flow}.} \topic{The program spanned eight weeks, which aligns with typical durations for clinical DMH interventions and is comparable to the expected engagement period required for people to benefit from in-person therapy~\cite{lipschitz2022digital, mohr2019randomized}. 
% Additionally, there is strong evidence to suggest that an eight-week duration is acceptable for text messaging-based behavioral interventions. 
% This duration supports the notion that building new routines and incorporating new skills into daily life requires time, practice, and reinforcement. 
%Prior literature has demonstrated that this duration is suitable for people to build new routines and incorporate new skills into their lives, reaping similar benefits to what can be gained from in-person therapy
%\cite{agyapong2012supportive, agyapong2017randomized, ranney2018emergency}. 
% The Small Steps SMS program was specifically designed to provide opportunities not only for learning but also for implementing and reinforcing new habits. 
This duration reflects that building new routines and incorporating new skills into daily life requires time, practice, and reinforcement. The Small Steps SMS program was specifically designed to provide opportunities not only for learning but also for implementing and reinforcing new habits in order to achieve downstream effects on symptoms. However, it is important to recognize that not all DMH tools require such extended durations; for example, some are designed as single-session interventions lasting around 10 to 20 minutes, focusing on teaching a specific micro-skill or providing immediate support for stress management  \cite{bhattacharjee2024exploring, paredes2014poptherapy, schleider2020future}}.
%Nevertheless, it is important to recognize that not all DMH tools require such extended durations; many are designed as single-session interventions lasting around 10 to 20 minutes, focusing on teaching a specific micro-skill or providing immediate support for stress management \cite{bhattacharjee2024exploring, paredes2014poptherapy}.}


\begin{figure}
    \centering
    \includegraphics[width=1\linewidth]{Figures/conversations8}
    \caption{Examples of message dialogues seen by participants in the Small Steps SMS Program: (a) A dialogue showcasing specific psychological strategies and skill-building exercises, (b) illustrative accounts of individuals successfully employing strategies to overcome challenges, and (c) messages designed to prompt self-reflection.}
    \label{fig:conversations}
\end{figure}

%The program included both active and passive personalization. As far as active personalization, users could select which psychological strategies to start with, and which to exclude. They could also decide when they were ready to move on from a strategy. Within a subset of dialogues, users could also choose the topics of stories to receive or could select between activities. 

%As far as passive personalization, the system utilized reinforcement learning algorithms \cite{kaelbling1996reinforcement} to guide multiple decisions, such as whether to include or exclude components like weblinks and reminders, and when to send messages to align with users' availability and favored time slots. In particular, it employed contextual bandit algorithms \cite{figueroa2023ratings, figueroa2022daily}, basing decisions about what to send and when on users' past engagement with the system and ratings of daily dialogues, while also considering contextual factors such as day of the week, time on study, or symptom levels at sign up. 

%During the eight-week program, participants demonstrated a relatively high level of engagement. On average, they responded at least once on 70\% of the study days, with individual engagement rates spanning from 34\% to 100\%.
%While we acknowledge that variations in message content and algorithmic design could enhance participants' engagement, the substantial involvement from participants in the program allowed us to delve deeper into the broader social contextual factors influencing user engagement. This paper primarily concentrates on illuminating these influential social contexts.

\subsection{Data Collection and Procedures}

Participants were asked to complete two semi-structured interviews, one midway and another at the end of the eight-week program, conducted via Zoom by two research team members. These interviews were intended to gather insights into how participants' social contexts, personal responsibilities, and other circumstances evolved over the course of the study and how these contexts influenced participants' interaction with the text messaging program. Examples of interview questions included:


\begin{itemize}
\item Have there been any changes in your interaction with or perceptions of the program over the last month? Can you describe these changes?
\item Over the past month, have there been any life changes that have influenced your usage of the program? Can you describe how they affected your use of the texting program?
\item Engagement levels can fluctuate in programs like this. What are your thoughts on why this might occur?
\item Have you observed any changes in your daily activities, personal life, or emotional state since starting the program? What were they?
\end{itemize}

All 20 program participants were invited to interviews, with all but one completing at least one. Eighteen participated in the mid-program interviews at four weeks, and 17 in the final interviews at eight weeks. Interview durations ranged from 20 to 45 minutes, with participants compensated \$20 USD per hour.


\subsection{Data Analysis}
After transcribing the interviews, our team undertook a thematic analysis \cite{cooper2012apa} of the qualitative data. Two team members, termed ``coders,'' began by thoroughly reviewing all transcripts to acquaint themselves with the content. Following an open-coding process \cite{khandkar2009open}, each coder independently created a preliminary codebook. Subsequent meetings enabled the team to consolidate the codes into a unified codebook. \topic{During these discussions, the team refined code definitions, identified recurring codes, assessed each code’s relevance to the research questions, and eliminated those that did not directly pertain to the research questions at hand.} The coders then tested the shared codebook on a subset of the data consisting of eight interview transcripts. This iterative process allowed for further refinements to the codebook. Once a consensus was reached, each coder applied the finalized codebook to separate halves of the remaining data.

\subsection{Ethical Considerations}
\label{study1_ethics}
The research team was comprised of faculty members and graduate students from various fields including computer science, human-computer interaction, cognitive science, and clinical psychology. Given that  mental health research presents multiple ethical challenges, we took measures to address these at all stages of the research.

Participants were explicitly informed at the commencement of the study that the text messaging system was not a crisis intervention service. They were also given a list of emergency contacts, including crisis text lines and suicide helplines, in the possible event that such information would be useful. We did not solicit suicide-related information at any point in the study; however, given the open-ended nature of text messaging, we recognized the unlikely possibility that participants may disclose unprompted suicidal thoughts or behaviors. Hence, we took measures to ensure the safety of participants. We reviewed text messages from participants on a daily basis. We were prepared to reach out to any participants indicating a risk of suicidal ideation or self-harm and conduct the Columbia Suicide Risk Assessment protocol \cite{posner2008columbia}. However, no such risks emerged during the study, and no follow-up assessment was necessary. 

Similar considerations were applied to the interviews, where interviewees were informed they could skip questions or exit at any time. Interviewers were trained to conduct the Columbia Suicide Risk Assessment protocol as well, but again, no such measures were necessary. 
\section{Findings from Study 1}
%It is worth noting that the actual engagement might be even greater, considering the possibility that participants might have chosen to read the messages without necessarily responding.
%However, participants cited several reasons for occasional non-engagement with the messages. While some of these reasons touched upon personal preferences, the majority were predominantly influenced by broader social factors, which we elaborate on in the subsequent sections.

\topic{Participants offered varied interpretations of what constitutes a DMH tool. They reported using tools like Woebot (FP11) and mood-tracking apps (FP20) as well as leveraging social media for peer support (FP17). Others (FP4, FP5, and FP9) mentioned exploring digital reflection techniques online, while FP10 highlighted using apps for weight management. Several participants (FP5, FP17, FP19) had also mentioned that they had familiarity with some of the psychological concepts and techniques introduced by the Small Steps SMS program.} %although few (e.g., ) had used messaging tools for mental health self-management before this study.}

Our interviews illuminated various factors that disrupted user engagement with the Small Steps SMS program. While some experiences of disengagement were linked to program-specific issues like habituation and individual preferences, the majority stemmed from the broader social contexts affecting participants. We first briefly present disengagement factors linked to habituation and individual preferences before delving into those influenced by social context.



\subsection{Disengagement Due to Habituation and Personal Preferences}

During the eight-week program, participants demonstrated a relatively high level of responsiveness to the messages sent by the system. On average, they responded at least once on 70\% of the study days, with individual daily responsiveness spanning from 34\% to 100\%. The level of participation also declined over the study, with average daily responsiveness falling from 93\% in the first week, to 47\% in the 8th week. Participants relayed that these trends partly reflected diminishing novelty, as well as occasional dissatisfaction regarding the content and frequency of the messages.

\topic{A prominent reason for gradually decreasing responsiveness to the messages was habituation. Over the course of the program, participants became accustomed to the repetitive nature of the daily text messages, rendering them increasingly predictable. This predictability sometimes reduced the novelty of the messages and engendered a sense of monotony.} As the routine of reading and responding to messages became more ingrained, participants found the task increasingly tedious, eroding both their motivation to engage with the program and their perception of its utility. Over time, the diminishing returns of the repetitive interactions led participants to question the program's usefulness, thereby lowering their overall responsiveness. FP10 expressed this sentiment by saying:


%The repetitiveness of the daily interactions could lead to a sense of predictability. This familiarity could make the task of reading and responding to messages tedious over time, sometimes diminishing individual motivation and the perceived utility of the program. 


\italquote{It’s like a daily thing that you constantly have to check up every day. Then, yeah, it’s like a chore. It’s something that I’m not really looking forward to eventually. It’s like `Oh, right. I have to do that.' I’ve kinda lost interest in it.}

FP2 emphasized that the frequency of the messages also contributed to their disengagement. Specifically, receiving messages at sporadic intervals throughout the day could become overwhelming rather than helpful. Elaborating on the experience of interacting with several messages, FP11 stated:

\italquote{I just feel like if I don’t accomplish at least reading it and responding back, I’ve failed at something. \ldots So, it can get a little bit overwhelming.}

To enhance user engagement, participants provided suggestions for additional features. For example, they recommended implementing features for journaling or documenting daily events. They also suggested a more organized messaging interface to separate and retrieve distinct dialogues for future reference.

%These findings indicate  factors such as habituation and message frequency could contribute to a decline in responsiveness


%Despite employing contextual bandit algorithms and collecting preliminary preferences from participants regarding content and timing, 
%occasional dissatisfaction with the content or timing of messages led to sporadic disengagement. For instance, FP6 considered gratitude messages to be too other-focused since they felt that they were already dedicating most of their time to supporting their child as a young parent. On the other hand, P336 occasionally found content on supporting others rudimentary based on their familiarity with social support literature and prior experience in assisting others. Other preferences regarding tone (formal vs. informal) and the style of guidance (action-oriented vs. indirect) of the messages arose as well. In cases where the content, tone, or guidance style of the messages did not align with an individual's preferences, they were more likely to disengage in the future.

%Despite specifying their preferred timings, some participants found the timing of messages occasionally inopportune. %Unforeseen work commitments or unexpected life events meant they could not always engage with the messages as intended. 
%FP2 highlighted that the frequency of messages could lead to disengagement too, pointing out that receiving messages sporadically throughout the day could be distracting. Additionally, for participants like FP7, receiving multiple messages in a single batch could be overwhelming and deter engagement.

%FP2 even suggested receiving all messages collectively, to avoid multiple interruptions throughout their day. However, individuals like FP7 preferred sporadic interruptions throughout the day.

%In summary, these findings underscore how misalignment with people's personal preferences could lead to disengagement. However, these issues may be attributable to broader social context of the participants, a topic we discuss further in the subsequent subsection.

%\todo{Does this need to be longer? I intentionally kept this short.}

\subsection{Broader Social Contexts Underlying Disengagement} 
%In contrast to participants whose engagement waned gradually due to perceived limitations of the tool, a number of participants described disengagement that was more abrupt, reflecting a range of life circumstances that could overwhelm them and demand their attention and mental energy. In these instances, disengagement often happened despite satisfaction with the tool.

Unlike instances where engagement decreased gradually due to perceived shortcomings of the tool, there were often times when participants abruptly disengaged due to life circumstances that demanded their full attention and energy. In these instances, disengagement often happened despite satisfaction with the tool. We refer to these circumstances as situational engagement disruptors (SEDs), \topic{which are summarized in Table \ref{tab:sed_summary}}. We describe them in detail below.

\begin{table}[h!]
\centering
\caption{\topic{Summary of SEDs identified from Study 1}}
\label{tab:sed_summary}
\begin{renv}
    
\begin{tabular}{|p{4cm}|p{9cm}|}
\hline
\textbf{Category}                     & \textbf{Examples of Disruptors}                                                                                     \\ \hline
Academic and Workplace Responsibilities & Academic coursework,  phone restrictions, deadlines, and unpredictable schedule                         \\ \hline
Family Responsibilities                & Childcare, household tasks, and financial concerns                                   \\ \hline
Other Disruptors                      & Illness, relocation, sleep disturbances, and grief                                       \\ \hline
\end{tabular}
\end{renv}

\end{table}


\subsubsection{Academic and Workplace Responsibilities}

Several participants shared their experiences dealing with academic stress and its impact on their ability to engage with text messages. For instance, FP16 expressed feeling overwhelmed by their coursework, stating that the initial enthusiasm they had when signing up for the program was overshadowed by mounting academic pressures. As the semester progressed, FP7 reflected on the increasing number of assignments and exams that demanded their full attention. Both FP7 and FP16 noted that they derived value from the program by gaining insights into new psychological theories and considering how to integrate these learnings into their daily lives. However, the demands of their course load and the stress associated with final exams subsequently made it challenging for them to continue dedicating time for such reflection.


Participants also extended their list of challenges preventing them from engaging with text messages to workplace rules and responsibilities. Reflecting on their work environment, FP8 shared the following:

\italquote{Unfortunately, I work in an environment where you're not allowed your phone. \ldots It's like a secure room, so any text I get during the workday tends to all come together. I read them, but it's hard to engage with all of them, like, all at once.}

\noindent
Several participants highlighted categories of occupations that may inherently conflict with the ability to respond to text messages promptly. FP14 pointed out truck drivers as an example, explaining that the nature of their job may often prevent them from safely and conveniently engaging with text messages while driving. The irregular and unpredictable timing of their breaks compounds this issue as it complicates anticipating when they will have the chance to interact with such messages.

Even individuals who do not face strict rules or difficulties regarding phone usage found it challenging to prioritize self-care during working hours, stemming from the perception that taking a moment for personal wellbeing might detract from their focus on work. For example, FP2 noted that when their phone buzzes repeatedly with notifications, they feel compelled to redirect their attention from work to the messages. Although they found the messages useful for learning how to cope with stress, they were concerned about continually switching back and forth between work and text messages. They explained: 
\italquote{That experience of, you’re focused at work and you hear your phone go off -- it sounds like you have a reaction like ``Okay, what is it?'' You have to completely redirect your attention.}

%In a similar vein, %FP4 expressed that one may not prioritize spending a moment on the messages because they have \textit{`nothing to do with my work.}' 

%Academic and workplace responsibilities not only limit the time available for self-care but also generate a prioritization conflict. Participants perceived a trade-off between engaging in self-care activities and focusing on the tasks at hand. 

%Participants relayed their experiences of facing challenges where they had to strike a balance between engaging in self-care activities and fulfilling their immediate responsibilities or tasks. 
The tensions between work tasks and personal wellbeing activities were highlighted further in FP4's experience. They indicated that they only felt compelled to promptly respond to messages that concerned urgent job responsibilities or impending work-related deadlines. Messages suggesting strategies to handle negative emotions did not hold the same sense of urgency or offer benefits that could surpass the importance of completing time-bound professional tasks.
%Text messages about learning to manage negative emotions seemed to lack the same sense of urgency and did not provide tangible benefits that outweighed completing academic or professional tasks with deadlines. 
Consequently, these messages were often relegated to a lower priority and could be easily forgotten or overlooked. 
% To summarize, balancing work commitments seemed to overshadow the importance of self-care activities, leading to personal health and wellbeing sometimes being underserved.


%To summarize, the pressing need to devote time and energy to meet work commitments could eclipse the perceived importance of self-care activities, thereby fostering a cycle of stress and neglect of personal health and wellbeing.
%The need to allocate time and effort towards income generation may overshadow the value and necessity of self-care practices, contributing to a cycle of overwhelm and neglect of personal wellbeing.

\subsubsection{Family Responsibilities}
A few participants expressed feelings of overwhelm regarding their family responsibilities, most notably women who felt responsible for managing household duties. FP2 and FP6 shared their experiences as young parents and the impact it has on their ability to engage with text messages. FP2 explained:

\italquote{I think that [the text messaging program] was really helpful for me. \ldots But I'm in school, I'm working, my oldest is in kindergarten, my youngest is in daycare, so I'm all over the place. And part of what I struggle with is being – I'm also 25. \ldots When I have that free time, I never use it as `me time'. It's clean up the kids' room, get my laundry done, make sure everything's good so that when they get back home, we've got a new space to tear apart again. \ldots And it's never me just taking a nap. \ldots If I'm home alone, I'm getting homework done. I'm getting chores around the house done. \ldots It goes along with being a parent. All of your time is dedicated to other people's needs and not so much your own.}

\noindent
Caring for children and other family members had a significant impact on people's ability to plan and set personal goals. The demands of managing household duties, childcare, and other familial obligations can create a sense of overwhelm and restrict people's capacity to allocate time and energy for reflection or self-care. Hence, FP6 emphasized that young parents would benefit from adding more structure to their lives, whether it be through creating routines or setting aside dedicated time for self-care activities. 
% Other participants echoed a desire for support in planning and incorporating self-care activities into their daily lives, recognizing the positive impact it can have on their overall well-being.

%These participants shared that with the responsibilities of taking care of their children as well as other family members, they tend to struggle with planning and setting goals for their life. They feel that they do not have much bandwidth to reflect on their life or take care of themselves. FP6, as a result, expressed that young parents like hers would most benefit from forming some sort of structure in their life and some support in planning activities around self-care. 

For some, anxiety and overwhelm centered on their families' financial stability. FP17 described ongoing worries and efforts around maintaining a balanced budget while meeting their family's basic needs, noting that such worries could easily lead people to deprioritize engaging with text messages. They commented:

\italquote{Even when I was busy, I would take the time \ldots People are busy, and a lot of people are just not gonna take the time to do it depending on how depressed the person is. They've got bills. They've got other things. The way things are right now, people are struggling just to make it. So, it’s gonna be a challenge. People are thinking, 'I’ve got a family to feed and if I don’t go do this, this, and this, I’m not gonna \ldots surmount any of the challenges'.}

\noindent
This preoccupation with financial challenges consumed their thoughts to such an extent that it became difficult for them to redirect their focus to other areas of life, namely self-care. Despite recognizing the potential benefits of the program's messages, participants sometimes found themselves unable to fully engage with them.


%expressed the belief that engaging in self-care activities would require time that they perceived would be better spent on finding ways to increase their income. %They reflected that some of the self-care activities might require some time, that they would think would be better spent in finding ways to increase income. 


\subsubsection{Other Disruptors}

Outside of academics, work, and family, participants highlighted other circumstances that disrupted their usual routines and hindered their engagement with text messages. These disruptions encompassed a range of unexpected events and situations, including brief illnesses, changes in residence, and the death of loved ones. Throughout the eight-week period, some participants reported experiencing health issues like COVID-19 and bronchitis that impacted their mood and energy levels, making it difficult to engage with the messages.
FP8 revealed that missing messages for a few days during their COVID-19 illness led them to disengage from the program altogether as the perceived difficulty in catching up with the missed messages seemed daunting. They stated:

\italquote{Getting a text is just actually really nice. \ldots When I got COVID and I was getting messages \ldots that kind of thing skews my engagement with it a little bit because I was too tired to even read my phone let alone actually engage with anything. And sometimes, it is annoying when you get several messages all at once.}

Meanwhile, participants like FP5 and FP8 shared how insomnia and changes in their sleep schedule disrupted their ability to interact with the messaging program. They relayed that sleep deprivation led them to a state of heightened busyness yet diminished presence in their daily activities. This lack of energy and focus negatively impacted their engagement with text messages, making comprehension and response more challenging. FP5 echoed a similar sentiment by underscoring the importance of establishing a regular sleep routine to better manage their interaction with the program.


%FP54 commented that even though they knew that the messages were well-intentioned, they felt that the messages interrupted their attempts to adjust their sleep schedule.

Additionally, participants highlighted the difficulties they faced when adapting to uncertain and unfamiliar housing environments due to new job opportunities. FP12 mentioned that they had to constantly move between two cities for work. The uncertainty surrounding their living arrangements consumed their thoughts and hindered their ability to engage with text messages. Participants also mentioned personal, unanticipated reasons for disengagement. FP4, for example, shared the experience of losing a close friend on the same day they enrolled in the program. The grieving process consumed their emotional energy, leading them to ignore the messages for the first few days.


\section{Procedure of Study 2}
%Following the identification of SEDs in our first study, we aimed to discover possible solutions that could address and account for these SEDs. 

The first study identified several critical SEDs within the context of engagement with a DMH tool. These SEDs emerged as broader factors originating from the social contexts surrounding an individual. Spurred by these insights, our subsequent efforts were directed towards identifying potential solutions to address and alleviate these SEDs. To this end, our second study involved conducting design workshops with \topic{25 participants} aimed at generating ideas for further exploration. \topic{While we drew inspiration from the deployment of a text messaging tool for these workshops, we aimed to explore how any sort of DMH tool can be designed to address SEDs.}
This phase of the study was approved by \redacted{the University of Toronto's} Research Ethics Board, and its logistics are described below. 

\subsection{Participants}

%To ensure a more diverse pool of participants, 
We utilized promotional calls across various social media platforms to recruit diverse participants who felt overwhelmed by their life circumstances. %Participants were recruited between June and July 2023. 
The study was framed as a collaborative design activity to understand challenges in everyday life that could impede engagement with DMH tools and to explore strategies for addressing these challenges. 

\topic{In the first study, participants with elevated symptoms of depression and anxiety frequently highlighted feeling overwhelmed by competing demands \cite{kabigting2019conceptual, hopps1995power} as a primary barrier to sustained interaction with DMH tools. Recognizing that this challenge is also prevalent in populations without a formal diagnosis or clinically elevated symptoms, we expanded the focus in the second study to include other populations that would benefit from DMH tools. This second study included individuals experiencing elevated stress irrespective of their current levels of depression or anxiety symptoms, allowing us to incorporate their diverse perspectives in our workshops.}
% The shift in participant groups reflects the distinct goals of the two studies.} %Recognizing that this experience of overwhelm is a common, cross-cutting issue not exclusive to depression and anxiety, we expanded our focus in the second study to include individuals experiencing elevated stress. This approach enabled us to examine SEDs that contribute to overwhelm across a broader range of individuals. The shift in participant groups reflects the distinct goals of the two studies.}

%To align with the overarching theme of 'overwhelm' from our first study, 
Our eligibility criteria required participants to self-report moderate to severe levels of stress according to the Perceived Stress Scale (PSS) (determined by a score of 14 or more) \cite{cohen1994perceived}.  Eligibility criteria also required participants to be at least 18 years old and residents of North America. 
Interested individuals could follow a link to learn more about the study online. After completing a brief eligibility survey, eligible participants were asked to provide informed consent.


\topic{The final cohort for this study consisted of 25 participants} with an average age of $25.0\pm4.2$ years old. Participants identified with multiple genders (15 men, 9 women, 1 non-binary) and multiple racial groups (9 Asian, 7 White, 6 African American, 1 Native Hawaiian/Other Pacific Islander, and 2 mixed race). 
\topic{Similar to the first study, participants were asked whether they had ever been diagnosed with major depression or anxiety disorder by a specialist or healthcare provider, although a formal diagnosis of depression or anxiety was not required for participation.
Nine participants (36.0\%) reported being diagnosed with depression, and 12 participants (48.0\%) reported being diagnosed with an anxiety disorder.} We refer to these participants as SP1--SP25. 

\subsection{Data Collection and Procedures}

We hosted five online design workshops through the Google Meet videoconferencing platform, with each accommodating 4–6 participants and two research team members. Each participant attended only one session, and every workshop lasted approximately 90 minutes. Participants were compensated \$30 USD for their participation.

The workshops began with a brief introduction to the workshop's goals, followed by a presentation from a research team member detailing the findings of our first study. After this overview, participants shared their personal experiences with DMH tools similar to the Small Steps SMS program and their general thoughts about using technology (e.g., apps, text messaging) to support mental health self-management. \topic{They were also asked to reflect on SEDs they had experienced or considered relevant to their interactions with DMH tools, both within and beyond the findings of the first study.} We then presented them with a series of hypothetical scenarios distilled from common issues that emerged from the formative study. Each scenario presented a set of circumstances in which an individual might become overwhelmed, spurring participants to reflect on the challenges one might face while trying to engage with a DMH tool as intended. \topic{These scenarios are summarized in Table~\ref{tab:scenario} in Appendix \ref{app: scenarios}.}




Following the hypothetical scenarios, participants engaged in three rounds of activities during which they documented their thoughts in online documents and shared their insights verbally. Each of these rounds lasted 15--20 minutes and revolved around one of the following questions:

\begin{enumerate}
\item Reflecting on times when you faced similar challenges, which self-care practices did you find most beneficial? What specific strategies would you recommend for managing such challenges in the future?
\item In what ways do you believe programs like the Small Steps SMS Program can be enhanced to help people adopt these self-care practices, especially during challenging times?
\item Are there strategies that you feel are broadly applicable regardless of specific life challenges? %Conversely, which strategies do you deem less broadly applicable?
\end{enumerate}

\topic{We clarified that participants did not need to limit their recommendations to modifications for the Small Steps SMS program. They were encouraged to share insights that could be broadly applicable to the design of various DMH tools.}

\subsection{Data Analysis}
After consolidating the workshop transcriptions and participant notes in the online documents, we employed the same thematic analysis procedures that we utilized in the first study, although a distinct codebook was developed for this phase.

\subsection{Ethical Considerations}

We used similar measures to the ones detailed in Section \ref{study1_ethics} in order to ensure that participants were treated in an ethical manner. All researchers who spoke with participants were prepared to conduct the Columbia-Suicide Risk Assessment protocol, yet no risks emerged and no follow-ups were necessary.
\section{Findings from Study 2}

\topic{Similar to the first study, participants shared varied interpretations of DMH tools.
Participants mentioned enrolling in text messaging services (DP1, DP7, DP13, DP14), participating in digital journaling (DP1), engaging in social media channels (DP4, DP8, DP20), connecting with therapists over digital platforms (DP12), and using apps like Headspace (DP21), Calm (DP13), and Forest (DP5).}
They recommended several ways to account for the SEDs in the design of DMH tools. We describe these potential solutions below.
%We delve into these recommended solutions below.


\subsection{Prioritization of Responsibilities}
During the workshop, participants identified that feelings of overwhelm often arise from juggling numerous personal and professional responsibilities. For instance, SP11 recalled how they had to balance multiple exams, complete their thesis, and work at a part-time job to support their household during their final year as an undergraduate student. 
% On the other hand, individuals like SP23 highlighted the demands of caring for sick family members. 
Reflecting on these diverse experiences, participants acknowledged the crucial need to prioritize responsibilities. SP6 suggested that DMH tools could help individuals with multiple demands on their time as follows:

\italquote{Try to weigh all of these responsibilities and see which of them take precedence over the rest, and try to fulfill those responsibilities in that order. Prompt the user to deeply think about these responsibilities so they can get a better handle on things.}

Tied into these discussions was a recurring theme of procrastination. Participants often cited an accumulation of tasks that were deferred out of anxiety or a lack of motivation.
% propelled by a combination of the sheer volume of tasks and a tight timeframe for completion. 
To mitigate this issue, participants like SP7 proposed that DMH tools could help users deconstruct larger responsibilities into smaller, more manageable tasks. They believed that this approach would make tasks more actionable, foster a sense of productivity, and subsequently disrupt the cycle of procrastination.
However, participants in the workshop recognized that prioritizing responsibilities often necessitates carefully considering and endorsing a rationale for why certain tasks take priority. They thought this could be facilitated by soliciting users' short-term and long-term objectives, and aiding them in formulating action plans to meet these goals. %SP14 personally attested to the benefits of prioritizing tasks during moments of crisis, as it only required momentarily breaking from extracurricular activities to focus on work and academic obligations. 
% SP14 attested to the benefits of prioritizing tasks during moments of crisis, recalling from a personal experience when they could identify that they could take a break from extracurricular activities to focus on work and academic obligations.
SP1 noted that seeing their responsibilities in written form \textit{`just makes it feel like grounded and like more in control'}.

Participants were also aware of the mental effort and cognitive resources required for these practices, making them more suitable for periods of relative downtime like weekends and holidays. SP7 suggested that DMH tools could help users articulate goals and plan actions during these periods. To keep these goals at the forefront, participants recommended reminders throughout the week, with SP7 emphasizing morning reminders to align daily activities with long-term objectives.

Still, SP21 cautioned that writing down detailed, elaborate plans on mobile devices could contribute to additional cognitive overload, even during free time. Instead, they suggested that users could be offered the option to either formulate their plans mentally or using traditional pen-and-paper methods.

\subsection{Incorporating Self-Care into Daily Routines}

Participants emphasized the importance of maintaining a balanced lifestyle with respect to sleep, diet, and exercise. This was regarded as an important factor in managing one's overall wellbeing, especially during periods of stress or overwhelm. Based on challenges identified in our first study related to maintaining a consistent sleep schedule in the face of SEDs, workshop participants proposed that DMH tools should include features to encourage sleep hygiene. SP10 suggested that users' preferred sleeping hours should be gathered during the registration process, allowing DMH tools to send prompts when individuals' activities on their phones continue beyond the specified bedtime.
During periods of illness or heightened stress, participants acknowledged that neglecting meals had an adverse effect on their mental health. In such scenarios, the recommendation was to focus on ensuring users remember to eat and hydrate. SP7 commented:

\italquote{Instead of like constant messages, add more general reminders like eating, drinking water, etc. Depending on the time of the day, you can send messages like `have a snack,' `drink some water,' `do breathing exercises,' or food and snack ideas.}

Reflecting on the toll of substantial familial responsibilities, participants like SP23 and SP25 expressed a tendency to neglect self-care in favor of attending to the needs of others. They often attributed this inclination to the absence of a systematic approach to time management. To address this issue, they suggested that DMH tools could assist users in carving out dedicated time for self-care. SP23 elaborated that these reserved times could be spent meeting friends or engaging in restful activities like reading or sleeping. However, we observed varied preferences for how this personal time should be distributed; while some preferred an hour each day, others leaned towards larger blocks of time or even full days every few weeks.


To support users in navigating their daily activities and appropriately allocating time for self-care, participants like SP5 and SP14 suggested that DMH tools should consider integrating with users' digital calendars or other platforms commonly used for academic or professional communication (e.g., Microsoft Teams, Slack). They proposed that this synchronization would allow such tools to gain a nuanced understanding of users' schedules and obligations, consequently enabling more personalized and timely interventions. For example, reminders or prompts for self-care activities could be strategically scheduled during free periods identified in users' calendars, thus assisting them in maintaining a healthy work-life balance amidst their professional or academic responsibilities. Participants also anticipated that seeing a time slot reserved for self-care in their calendar would reinforce their commitment to these activities.

\subsection{Alternate Framing of Disengagement}

Participants acknowledged that irrespective of the quality and effectiveness of a DMH tool, users are likely to experience occasional disengagement due to either voluntary or involuntary reasons. They highlighted the mental hurdle of re-engaging with these tools after a period of non-use. Reflecting on their past experiences with other DMH tools, SP1 expressed a negative sentiment regarding the tools' emphasis on continuous use. They noted that when they took a break from the tools, the feeling of losing their streak demotivated them. This demotivation stemmed from the perception that they would have to put in significant effort to regain the level of consistent use they had previously achieved. They commented:

\italquote{If I consistently used it every day and then I forgot one day, or like I forgot a couple of times, I'd be like, `Oh, but I already lost the streak.` Then I might as well just like give up the whole thing, which is not great.}

\noindent
To support users during periods of heightened stress or activity, SP2 suggested enabling them to specify days when they might be occupied with exams or work responsibilities. Marking such days in a DMH tool would prompt the system to lower engagement expectations, helping prevent users from completely withdrawing. For example, users could be asked to reflect on their mood or energy level with a number or emoji rather than a full journal entry. 
% SP5 also echoed this sentiment, suggesting that on these busy days, users should not be asked to write long texts.

Despite these recommendations, participants acknowledged that occasional disengagement is inevitable.  
During prolonged periods of user inactivity, SP22 proposed that DMH tools should respond by issuing straightforward and actionable suggestions to re-engage the users. However, SP5 and SP9 warned that the frequency of such suggestions should not be overwhelming, as too many notifications could potentially irritate users and cause them to develop a strong aversion to the tool. 
%SP5 and SP9 cautioned that the frequency of such suggestions should not be overwhelming. They reminded us that an overload of notifications during periods of disengagement could have the opposite effect, provoking irritation or even triggering a strong aversion towards the tool. 
SP9 shed light on this perspective from their previous experiences:

\italquote{From the meditation app, I actually got a little bit annoyed because it would sort of bloat my notification list. I kind of get mad if I don't see the value in the amount of times my phone is being blown up. \ldots If it's sending me a lot of messages and long messages, I might go into rebellious mode and just not even answer out of pure hate or maybe just possibly hit skip.}

%Hence, participants went on to suggest that DMH tools should accept and normalize such behaviors.  SP6 suggested that the text messaging application should remind people that struggles in life are temporary and with the things they can control, they can improve their mental state, usage of the app being one of them.
%They emphasized that the success of using a tool should not be solely defined by continuous usage but rather appreciated for the fact that users are utilizing the tool to improve their wellbeing. 
%However, 

In light of these reflections, workshop participants suggested a shift in perspective. Instead of perceiving periods of disengagement as user failures, DMH tools should understand and accept these moments as part of the user's journey. SP6 highlighted the importance of acknowledging struggles and promoting the idea that focusing on more controllable aspects of life, such as engagement with a DMH tool, can positively influence one's mental health. 
The goal here would be not to pressure users into continuous usage, but to foster an environment where they feel supported and understood, even during periods of disengagement. Participants emphasized that the success of using DMH tools should not be solely defined by continuous usage but rather the fact that users are choosing to utilize the tool to improve their wellbeing, despite barriers and setbacks. 


\subsection{Leveraging External Support and Community Resources}
Participants underlined the significant role external support and community resources can play when it comes to managing overwhelming situations. 
Drawing on their academic experiences, participants like SP2 and SP8 expressed the ease with which people can overlook potential help from their peer network. They described times when they grappled with academic assignments on their own, forgetting they could lean on friends and teaching assistants for guidance. They suggested that a gentle reminder to consider support resources could be beneficial in moments of high stress. Participants also spoke about how they appreciated reading narratives that echoed their specific struggles. They expressed that seeing examples of how others have managed to navigate similar issues could offer them hope, strengthen their belief in their ability to improve their situation, and potentially provide practical directions to explore.

On the whole, participants described that addressing individuals' psychological wellbeing sometimes must go beyond merely providing supportive messages or teaching psychological strategies. In addition to encouragement to draw on their existing social networks for support, some participants suggested that people dealing with financial burdens may benefit from DMH tools that direct users to unconventional financial resources. Drawing upon a hypothetical scenario of a single mother juggling multiple jobs and having little time for self-care, SP13 posited one such example:

\italquote{For single mothers, there are many government and non-government organizations that give grants. \dots And if possible, she could collect grants, or maybe someone wants to start up a small business.}

\noindent
Concerning individuals with unstable housing conditions, participants pointed to websites and social media groups dedicated to helping people find suitable housing. They relayed that many users are often unaware of these resources, so DMH tools could provide a convenient gateway to them. 

Participants recognized the difficulty in tailoring DMH tools to users' specific ongoing life challenges, as these issues are often open-ended and unique to each individual. Therefore, %SP10 proposed that DMH tools might draw inspiration from platforms like Snapchat to gather information about the primary challenges users face in their lives. 
SP10 proposed that DMH tools could learn from the approach employed by the social media platform Snapchat, which collects data on users' preferences and interests as part of its sign-up process. They went on to say: %that DMH tools could feature an onboarding process that allows users to indicate the specific challenges they are currently facing. They explained:
\revision{\italquote{When you first log in as a user, you get the option to choose what kind of problem you're struggling with, kind of the scenario you're in. And you know, you can get responses based on that scenario.}}


In a complementary suggestion, SP1 recommended that DMH tools could then compile a repository of resources tailored to address those challenges. When users express their concerns, a matchmaking process could be initiated to direct people to resources related to their issues. Given recent excitement for large language models (LLMs), a few participants theorized that LLMs could be used to match people's open-ended concerns to resources. Nevertheless, participants were wary of the reliability, suitability, and scalability of such a process.
% also noted the limitations of these approaches, with scalability being a significant concern.

% In light of this, a few participants suggested that DMH tools might consider utilizing large language models (LLMs) due to their ability to respond to open-ended queries. They felt that LLMs could potentially assist them in identifying solutions tailored to their specific circumstances. However, they also voiced concerns about the reliability and suitability of advice generated by LLMs.


% \subsection{Alternate Modes of Interaction}

% \todo{Ananya Note: I am thinking of cutting this section because this provides a counter-argument to the need for text message and also not really a novel finding. Reviewers had issues about this section too.}

% Echoing Study 1, a few participants shared their reluctance to read lengthy text messages, particularly when they are distracted or engrossed in other tasks. In these situations, audio-based messages offer the distinct advantage of enabling engagement with minimal diversion from ongoing activities. Underlining the practicality of voice messages as an alternative, SP16 shared: 


% %Workshop participants, similar to participants from the first study, indicated that there could be people who do not prefer to type or even those who do may not be able to do so in different circumstances. People like SP1, SP16, and SP21 suggested that it might be easier for some people to talk about their experiences regarding mental health or stressful situations, rather than texting. Explaining the conveniences of voice messages, SP16 commented:

% \italquote{If possible, messages can be sent in voice form. I feel it's much, much easier to listen than to write. You can listen while on the go rather than sitting to read a text message and trying to reply.}

% Even outside the work context, some participants posited that occasional messages enriched with multimedia elements would be more likely to capture their attention, providing more engaging and visually stimulating content. They suggested that infographics, short video clips, or animations would provide a more vivid visualization of best practices in self-care. For example, participants suggested that a short video demonstrating mindfulness techniques or an infographic showing the steps to manage stress could be more impactful and easier to follow than plain text.


% In line with findings from our initial study, several participants including SP1, SP16, and SP21 remarked on the challenges that users may encounter when attempting to articulate their thoughts through written text. They suggested that speaking might pose less of a cognitive burden than writing, particularly on mobile devices where manual typing can be cumbersome. Verbal communication, they argued, could offer a more intuitive and mentally less demanding alternative, especially when individuals are by themselves.
% %Aligning with findings from the first study, participants also commented on factors that dissuade people from writing or typing out their thoughts. SP1, SP16, and SP21 thought that users might find it easier to verbalize their experiences rather than formulating them in written text. They argued that replying with long responses could pose a cognitive burden, especially when it would involve manual typing on a mobile device. On the other hand, verbal communication might come across as more intuitive and less mentally challenging, particularly in scenarios where individuals are alone. 

% To further diversify user interaction, participants recommended the use of images and videos as input to DMH tools in order to demonstrate their progress. For instance, a user could share a photo of a quiet spot in nature where they practiced mindfulness, or a screenshot of a completed to-do list as a record of their productivity. SP1 commented that this image-based approach could help users generate a personalized depiction of their mental health journey, thereby making the task of documenting progress less strenuous and more immersive.

% %When it comes to interacting with messages, workshop participants like SP1, SP16, and SP21 highlighted that some individuals may find it easier to articulate their mental health experiences or stressful situations verbally rather than in written form. They mentioned that writing could be cognitively demanding, particularly when one is dealing with mental health issues. In contrast, speaking about their problems might feel more intuitive and less taxing. Furthermore, participants suggested when users are asked what they did regarding practicing a new psychological strategy or adopting a new behavior, users might be allowed to reply with images, instead of writing down their progress. SP1 thought that it could also offer a more concrete and personal representation of their experiences and allow them to document their progress without much effort.


% %This could involve the use of visuals, such as pictures taken by participants themselves, which could provide a more engaging and less cognitively demanding alternative. 
\section{Discussion of Assumptions}\label{sec:discussion}
In this paper, we have made several assumptions for the sake of clarity and simplicity. In this section, we discuss the rationale behind these assumptions, the extent to which these assumptions hold in practice, and the consequences for our protocol when these assumptions hold.

\subsection{Assumptions on the Demand}

There are two simplifying assumptions we make about the demand. First, we assume the demand at any time is relatively small compared to the channel capacities. Second, we take the demand to be constant over time. We elaborate upon both these points below.

\paragraph{Small demands} The assumption that demands are small relative to channel capacities is made precise in \eqref{eq:large_capacity_assumption}. This assumption simplifies two major aspects of our protocol. First, it largely removes congestion from consideration. In \eqref{eq:primal_problem}, there is no constraint ensuring that total flow in both directions stays below capacity--this is always met. Consequently, there is no Lagrange multiplier for congestion and no congestion pricing; only imbalance penalties apply. In contrast, protocols in \cite{sivaraman2020high, varma2021throughput, wang2024fence} include congestion fees due to explicit congestion constraints. Second, the bound \eqref{eq:large_capacity_assumption} ensures that as long as channels remain balanced, the network can always meet demand, no matter how the demand is routed. Since channels can rebalance when necessary, they never drop transactions. This allows prices and flows to adjust as per the equations in \eqref{eq:algorithm}, which makes it easier to prove the protocol's convergence guarantees. This also preserves the key property that a channel's price remains proportional to net money flow through it.

In practice, payment channel networks are used most often for micro-payments, for which on-chain transactions are prohibitively expensive; large transactions typically take place directly on the blockchain. For example, according to \cite{river2023lightning}, the average channel capacity is roughly $0.1$ BTC ($5,000$ BTC distributed over $50,000$ channels), while the average transaction amount is less than $0.0004$ BTC ($44.7k$ satoshis). Thus, the small demand assumption is not too unrealistic. Additionally, the occasional large transaction can be treated as a sequence of smaller transactions by breaking it into packets and executing each packet serially (as done by \cite{sivaraman2020high}).
Lastly, a good path discovery process that favors large capacity channels over small capacity ones can help ensure that the bound in \eqref{eq:large_capacity_assumption} holds.

\paragraph{Constant demands} 
In this work, we assume that any transacting pair of nodes have a steady transaction demand between them (see Section \ref{sec:transaction_requests}). Making this assumption is necessary to obtain the kind of guarantees that we have presented in this paper. Unless the demand is steady, it is unreasonable to expect that the flows converge to a steady value. Weaker assumptions on the demand lead to weaker guarantees. For example, with the more general setting of stochastic, but i.i.d. demand between any two nodes, \cite{varma2021throughput} shows that the channel queue lengths are bounded in expectation. If the demand can be arbitrary, then it is very hard to get any meaningful performance guarantees; \cite{wang2024fence} shows that even for a single bidirectional channel, the competitive ratio is infinite. Indeed, because a PCN is a decentralized system and decisions must be made based on local information alone, it is difficult for the network to find the optimal detailed balance flow at every time step with a time-varying demand.  With a steady demand, the network can discover the optimal flows in a reasonably short time, as our work shows.

We view the constant demand assumption as an approximation for a more general demand process that could be piece-wise constant, stochastic, or both (see simulations in Figure \ref{fig:five_nodes_variable_demand}).
We believe it should be possible to merge ideas from our work and \cite{varma2021throughput} to provide guarantees in a setting with random demands with arbitrary means. We leave this for future work. In addition, our work suggests that a reasonable method of handling stochastic demands is to queue the transaction requests \textit{at the source node} itself. This queuing action should be viewed in conjunction with flow-control. Indeed, a temporarily high unidirectional demand would raise prices for the sender, incentivizing the sender to stop sending the transactions. If the sender queues the transactions, they can send them later when prices drop. This form of queuing does not require any overhaul of the basic PCN infrastructure and is therefore simpler to implement than per-channel queues as suggested by \cite{sivaraman2020high} and \cite{varma2021throughput}.

\subsection{The Incentive of Channels}
The actions of the channels as prescribed by the DEBT control protocol can be summarized as follows. Channels adjust their prices in proportion to the net flow through them. They rebalance themselves whenever necessary and execute any transaction request that has been made of them. We discuss both these aspects below.

\paragraph{On Prices}
In this work, the exclusive role of channel prices is to ensure that the flows through each channel remains balanced. In practice, it would be important to include other components in a channel's price/fee as well: a congestion price  and an incentive price. The congestion price, as suggested by \cite{varma2021throughput}, would depend on the total flow of transactions through the channel, and would incentivize nodes to balance the load over different paths. The incentive price, which is commonly used in practice \cite{river2023lightning}, is necessary to provide channels with an incentive to serve as an intermediary for different channels. In practice, we expect both these components to be smaller than the imbalance price. Consequently, we expect the behavior of our protocol to be similar to our theoretical results even with these additional prices.

A key aspect of our protocol is that channel fees are allowed to be negative. Although the original Lightning network whitepaper \cite{poon2016bitcoin} suggests that negative channel prices may be a good solution to promote rebalancing, the idea of negative prices in not very popular in the literature. To our knowledge, the only prior work with this feature is \cite{varma2021throughput}. Indeed, in papers such as \cite{van2021merchant} and \cite{wang2024fence}, the price function is explicitly modified such that the channel price is never negative. The results of our paper show the benefits of negative prices. For one, in steady state, equal flows in both directions ensure that a channel doesn't loose any money (the other price components mentioned above ensure that the channel will only gain money). More importantly, negative prices are important to ensure that the protocol selectively stifles acyclic flows while allowing circulations to flow. Indeed, in the example of Section \ref{sec:flow_control_example}, the flows between nodes $A$ and $C$ are left on only because the large positive price over one channel is canceled by the corresponding negative price over the other channel, leading to a net zero price.

Lastly, observe that in the DEBT control protocol, the price charged by a channel does not depend on its capacity. This is a natural consequence of the price being the Lagrange multiplier for the net-zero flow constraint, which also does not depend on the channel capacity. In contrast, in many other works, the imbalance price is normalized by the channel capacity \cite{ren2018optimal, lin2020funds, wang2024fence}; this is shown to work well in practice. The rationale for such a price structure is explained well in \cite{wang2024fence}, where this fee is derived with the aim of always maintaining some balance (liquidity) at each end of every channel. This is a reasonable aim if a channel is to never rebalance itself; the experiments of the aforementioned papers are conducted in such a regime. In this work, however, we allow the channels to rebalance themselves a few times in order to settle on a detailed balance flow. This is because our focus is on the long-term steady state performance of the protocol. This difference in perspective also shows up in how the price depends on the channel imbalance. \cite{lin2020funds} and \cite{wang2024fence} advocate for strictly convex prices whereas this work and \cite{varma2021throughput} propose linear prices.

\paragraph{On Rebalancing} 
Recall that the DEBT control protocol ensures that the flows in the network converge to a detailed balance flow, which can be sustained perpetually without any rebalancing. However, during the transient phase (before convergence), channels may have to perform on-chain rebalancing a few times. Since rebalancing is an expensive operation, it is worthwhile discussing methods by which channels can reduce the extent of rebalancing. One option for the channels to reduce the extent of rebalancing is to increase their capacity; however, this comes at the cost of locking in more capital. Each channel can decide for itself the optimum amount of capital to lock in. Another option, which we discuss in Section \ref{sec:five_node}, is for channels to increase the rate $\gamma$ at which they adjust prices. 

Ultimately, whether or not it is beneficial for a channel to rebalance depends on the time-horizon under consideration. Our protocol is based on the assumption that the demand remains steady for a long period of time. If this is indeed the case, it would be worthwhile for a channel to rebalance itself as it can make up this cost through the incentive fees gained from the flow of transactions through it in steady state. If a channel chooses not to rebalance itself, however, there is a risk of being trapped in a deadlock, which is suboptimal for not only the nodes but also the channel.

\section{Conclusion}
This work presents DEBT control: a protocol for payment channel networks that uses source routing and flow control based on channel prices. The protocol is derived by posing a network utility maximization problem and analyzing its dual minimization. It is shown that under steady demands, the protocol guides the network to an optimal, sustainable point. Simulations show its robustness to demand variations. The work demonstrates that simple protocols with strong theoretical guarantees are possible for PCNs and we hope it inspires further theoretical research in this direction.
\section{Conclusion}
In this work, we propose a simple yet effective approach, called SMILE, for graph few-shot learning with fewer tasks. Specifically, we introduce a novel dual-level mixup strategy, including within-task and across-task mixup, for enriching the diversity of nodes within each task and the diversity of tasks. Also, we incorporate the degree-based prior information to learn expressive node embeddings. Theoretically, we prove that SMILE effectively enhances the model's generalization performance. Empirically, we conduct extensive experiments on multiple benchmarks and the results suggest that SMILE significantly outperforms other baselines, including both in-domain and cross-domain few-shot settings.


\bibliographystyle{ACM-Reference-Format}
\bibliography{sample-base}
\subsection{Lloyd-Max Algorithm}
\label{subsec:Lloyd-Max}
For a given quantization bitwidth $B$ and an operand $\bm{X}$, the Lloyd-Max algorithm finds $2^B$ quantization levels $\{\hat{x}_i\}_{i=1}^{2^B}$ such that quantizing $\bm{X}$ by rounding each scalar in $\bm{X}$ to the nearest quantization level minimizes the quantization MSE. 

The algorithm starts with an initial guess of quantization levels and then iteratively computes quantization thresholds $\{\tau_i\}_{i=1}^{2^B-1}$ and updates quantization levels $\{\hat{x}_i\}_{i=1}^{2^B}$. Specifically, at iteration $n$, thresholds are set to the midpoints of the previous iteration's levels:
\begin{align*}
    \tau_i^{(n)}=\frac{\hat{x}_i^{(n-1)}+\hat{x}_{i+1}^{(n-1)}}2 \text{ for } i=1\ldots 2^B-1
\end{align*}
Subsequently, the quantization levels are re-computed as conditional means of the data regions defined by the new thresholds:
\begin{align*}
    \hat{x}_i^{(n)}=\mathbb{E}\left[ \bm{X} \big| \bm{X}\in [\tau_{i-1}^{(n)},\tau_i^{(n)}] \right] \text{ for } i=1\ldots 2^B
\end{align*}
where to satisfy boundary conditions we have $\tau_0=-\infty$ and $\tau_{2^B}=\infty$. The algorithm iterates the above steps until convergence.

Figure \ref{fig:lm_quant} compares the quantization levels of a $7$-bit floating point (E3M3) quantizer (left) to a $7$-bit Lloyd-Max quantizer (right) when quantizing a layer of weights from the GPT3-126M model at a per-tensor granularity. As shown, the Lloyd-Max quantizer achieves substantially lower quantization MSE. Further, Table \ref{tab:FP7_vs_LM7} shows the superior perplexity achieved by Lloyd-Max quantizers for bitwidths of $7$, $6$ and $5$. The difference between the quantizers is clear at 5 bits, where per-tensor FP quantization incurs a drastic and unacceptable increase in perplexity, while Lloyd-Max quantization incurs a much smaller increase. Nevertheless, we note that even the optimal Lloyd-Max quantizer incurs a notable ($\sim 1.5$) increase in perplexity due to the coarse granularity of quantization. 

\begin{figure}[h]
  \centering
  \includegraphics[width=0.7\linewidth]{sections/figures/LM7_FP7.pdf}
  \caption{\small Quantization levels and the corresponding quantization MSE of Floating Point (left) vs Lloyd-Max (right) Quantizers for a layer of weights in the GPT3-126M model.}
  \label{fig:lm_quant}
\end{figure}

\begin{table}[h]\scriptsize
\begin{center}
\caption{\label{tab:FP7_vs_LM7} \small Comparing perplexity (lower is better) achieved by floating point quantizers and Lloyd-Max quantizers on a GPT3-126M model for the Wikitext-103 dataset.}
\begin{tabular}{c|cc|c}
\hline
 \multirow{2}{*}{\textbf{Bitwidth}} & \multicolumn{2}{|c|}{\textbf{Floating-Point Quantizer}} & \textbf{Lloyd-Max Quantizer} \\
 & Best Format & Wikitext-103 Perplexity & Wikitext-103 Perplexity \\
\hline
7 & E3M3 & 18.32 & 18.27 \\
6 & E3M2 & 19.07 & 18.51 \\
5 & E4M0 & 43.89 & 19.71 \\
\hline
\end{tabular}
\end{center}
\end{table}

\subsection{Proof of Local Optimality of LO-BCQ}
\label{subsec:lobcq_opt_proof}
For a given block $\bm{b}_j$, the quantization MSE during LO-BCQ can be empirically evaluated as $\frac{1}{L_b}\lVert \bm{b}_j- \bm{\hat{b}}_j\rVert^2_2$ where $\bm{\hat{b}}_j$ is computed from equation (\ref{eq:clustered_quantization_definition}) as $C_{f(\bm{b}_j)}(\bm{b}_j)$. Further, for a given block cluster $\mathcal{B}_i$, we compute the quantization MSE as $\frac{1}{|\mathcal{B}_{i}|}\sum_{\bm{b} \in \mathcal{B}_{i}} \frac{1}{L_b}\lVert \bm{b}- C_i^{(n)}(\bm{b})\rVert^2_2$. Therefore, at the end of iteration $n$, we evaluate the overall quantization MSE $J^{(n)}$ for a given operand $\bm{X}$ composed of $N_c$ block clusters as:
\begin{align*}
    \label{eq:mse_iter_n}
    J^{(n)} = \frac{1}{N_c} \sum_{i=1}^{N_c} \frac{1}{|\mathcal{B}_{i}^{(n)}|}\sum_{\bm{v} \in \mathcal{B}_{i}^{(n)}} \frac{1}{L_b}\lVert \bm{b}- B_i^{(n)}(\bm{b})\rVert^2_2
\end{align*}

At the end of iteration $n$, the codebooks are updated from $\mathcal{C}^{(n-1)}$ to $\mathcal{C}^{(n)}$. However, the mapping of a given vector $\bm{b}_j$ to quantizers $\mathcal{C}^{(n)}$ remains as  $f^{(n)}(\bm{b}_j)$. At the next iteration, during the vector clustering step, $f^{(n+1)}(\bm{b}_j)$ finds new mapping of $\bm{b}_j$ to updated codebooks $\mathcal{C}^{(n)}$ such that the quantization MSE over the candidate codebooks is minimized. Therefore, we obtain the following result for $\bm{b}_j$:
\begin{align*}
\frac{1}{L_b}\lVert \bm{b}_j - C_{f^{(n+1)}(\bm{b}_j)}^{(n)}(\bm{b}_j)\rVert^2_2 \le \frac{1}{L_b}\lVert \bm{b}_j - C_{f^{(n)}(\bm{b}_j)}^{(n)}(\bm{b}_j)\rVert^2_2
\end{align*}

That is, quantizing $\bm{b}_j$ at the end of the block clustering step of iteration $n+1$ results in lower quantization MSE compared to quantizing at the end of iteration $n$. Since this is true for all $\bm{b} \in \bm{X}$, we assert the following:
\begin{equation}
\begin{split}
\label{eq:mse_ineq_1}
    \tilde{J}^{(n+1)} &= \frac{1}{N_c} \sum_{i=1}^{N_c} \frac{1}{|\mathcal{B}_{i}^{(n+1)}|}\sum_{\bm{b} \in \mathcal{B}_{i}^{(n+1)}} \frac{1}{L_b}\lVert \bm{b} - C_i^{(n)}(b)\rVert^2_2 \le J^{(n)}
\end{split}
\end{equation}
where $\tilde{J}^{(n+1)}$ is the the quantization MSE after the vector clustering step at iteration $n+1$.

Next, during the codebook update step (\ref{eq:quantizers_update}) at iteration $n+1$, the per-cluster codebooks $\mathcal{C}^{(n)}$ are updated to $\mathcal{C}^{(n+1)}$ by invoking the Lloyd-Max algorithm \citep{Lloyd}. We know that for any given value distribution, the Lloyd-Max algorithm minimizes the quantization MSE. Therefore, for a given vector cluster $\mathcal{B}_i$ we obtain the following result:

\begin{equation}
    \frac{1}{|\mathcal{B}_{i}^{(n+1)}|}\sum_{\bm{b} \in \mathcal{B}_{i}^{(n+1)}} \frac{1}{L_b}\lVert \bm{b}- C_i^{(n+1)}(\bm{b})\rVert^2_2 \le \frac{1}{|\mathcal{B}_{i}^{(n+1)}|}\sum_{\bm{b} \in \mathcal{B}_{i}^{(n+1)}} \frac{1}{L_b}\lVert \bm{b}- C_i^{(n)}(\bm{b})\rVert^2_2
\end{equation}

The above equation states that quantizing the given block cluster $\mathcal{B}_i$ after updating the associated codebook from $C_i^{(n)}$ to $C_i^{(n+1)}$ results in lower quantization MSE. Since this is true for all the block clusters, we derive the following result: 
\begin{equation}
\begin{split}
\label{eq:mse_ineq_2}
     J^{(n+1)} &= \frac{1}{N_c} \sum_{i=1}^{N_c} \frac{1}{|\mathcal{B}_{i}^{(n+1)}|}\sum_{\bm{b} \in \mathcal{B}_{i}^{(n+1)}} \frac{1}{L_b}\lVert \bm{b}- C_i^{(n+1)}(\bm{b})\rVert^2_2  \le \tilde{J}^{(n+1)}   
\end{split}
\end{equation}

Following (\ref{eq:mse_ineq_1}) and (\ref{eq:mse_ineq_2}), we find that the quantization MSE is non-increasing for each iteration, that is, $J^{(1)} \ge J^{(2)} \ge J^{(3)} \ge \ldots \ge J^{(M)}$ where $M$ is the maximum number of iterations. 
%Therefore, we can say that if the algorithm converges, then it must be that it has converged to a local minimum. 
\hfill $\blacksquare$


\begin{figure}
    \begin{center}
    \includegraphics[width=0.5\textwidth]{sections//figures/mse_vs_iter.pdf}
    \end{center}
    \caption{\small NMSE vs iterations during LO-BCQ compared to other block quantization proposals}
    \label{fig:nmse_vs_iter}
\end{figure}

Figure \ref{fig:nmse_vs_iter} shows the empirical convergence of LO-BCQ across several block lengths and number of codebooks. Also, the MSE achieved by LO-BCQ is compared to baselines such as MXFP and VSQ. As shown, LO-BCQ converges to a lower MSE than the baselines. Further, we achieve better convergence for larger number of codebooks ($N_c$) and for a smaller block length ($L_b$), both of which increase the bitwidth of BCQ (see Eq \ref{eq:bitwidth_bcq}).


\subsection{Additional Accuracy Results}
%Table \ref{tab:lobcq_config} lists the various LOBCQ configurations and their corresponding bitwidths.
\begin{table}
\setlength{\tabcolsep}{4.75pt}
\begin{center}
\caption{\label{tab:lobcq_config} Various LO-BCQ configurations and their bitwidths.}
\begin{tabular}{|c||c|c|c|c||c|c||c|} 
\hline
 & \multicolumn{4}{|c||}{$L_b=8$} & \multicolumn{2}{|c||}{$L_b=4$} & $L_b=2$ \\
 \hline
 \backslashbox{$L_A$\kern-1em}{\kern-1em$N_c$} & 2 & 4 & 8 & 16 & 2 & 4 & 2 \\
 \hline
 64 & 4.25 & 4.375 & 4.5 & 4.625 & 4.375 & 4.625 & 4.625\\
 \hline
 32 & 4.375 & 4.5 & 4.625& 4.75 & 4.5 & 4.75 & 4.75 \\
 \hline
 16 & 4.625 & 4.75& 4.875 & 5 & 4.75 & 5 & 5 \\
 \hline
\end{tabular}
\end{center}
\end{table}

%\subsection{Perplexity achieved by various LO-BCQ configurations on Wikitext-103 dataset}

\begin{table} \centering
\begin{tabular}{|c||c|c|c|c||c|c||c|} 
\hline
 $L_b \rightarrow$& \multicolumn{4}{c||}{8} & \multicolumn{2}{c||}{4} & 2\\
 \hline
 \backslashbox{$L_A$\kern-1em}{\kern-1em$N_c$} & 2 & 4 & 8 & 16 & 2 & 4 & 2  \\
 %$N_c \rightarrow$ & 2 & 4 & 8 & 16 & 2 & 4 & 2 \\
 \hline
 \hline
 \multicolumn{8}{c}{GPT3-1.3B (FP32 PPL = 9.98)} \\ 
 \hline
 \hline
 64 & 10.40 & 10.23 & 10.17 & 10.15 &  10.28 & 10.18 & 10.19 \\
 \hline
 32 & 10.25 & 10.20 & 10.15 & 10.12 &  10.23 & 10.17 & 10.17 \\
 \hline
 16 & 10.22 & 10.16 & 10.10 & 10.09 &  10.21 & 10.14 & 10.16 \\
 \hline
  \hline
 \multicolumn{8}{c}{GPT3-8B (FP32 PPL = 7.38)} \\ 
 \hline
 \hline
 64 & 7.61 & 7.52 & 7.48 &  7.47 &  7.55 &  7.49 & 7.50 \\
 \hline
 32 & 7.52 & 7.50 & 7.46 &  7.45 &  7.52 &  7.48 & 7.48  \\
 \hline
 16 & 7.51 & 7.48 & 7.44 &  7.44 &  7.51 &  7.49 & 7.47  \\
 \hline
\end{tabular}
\caption{\label{tab:ppl_gpt3_abalation} Wikitext-103 perplexity across GPT3-1.3B and 8B models.}
\end{table}

\begin{table} \centering
\begin{tabular}{|c||c|c|c|c||} 
\hline
 $L_b \rightarrow$& \multicolumn{4}{c||}{8}\\
 \hline
 \backslashbox{$L_A$\kern-1em}{\kern-1em$N_c$} & 2 & 4 & 8 & 16 \\
 %$N_c \rightarrow$ & 2 & 4 & 8 & 16 & 2 & 4 & 2 \\
 \hline
 \hline
 \multicolumn{5}{|c|}{Llama2-7B (FP32 PPL = 5.06)} \\ 
 \hline
 \hline
 64 & 5.31 & 5.26 & 5.19 & 5.18  \\
 \hline
 32 & 5.23 & 5.25 & 5.18 & 5.15  \\
 \hline
 16 & 5.23 & 5.19 & 5.16 & 5.14  \\
 \hline
 \multicolumn{5}{|c|}{Nemotron4-15B (FP32 PPL = 5.87)} \\ 
 \hline
 \hline
 64  & 6.3 & 6.20 & 6.13 & 6.08  \\
 \hline
 32  & 6.24 & 6.12 & 6.07 & 6.03  \\
 \hline
 16  & 6.12 & 6.14 & 6.04 & 6.02  \\
 \hline
 \multicolumn{5}{|c|}{Nemotron4-340B (FP32 PPL = 3.48)} \\ 
 \hline
 \hline
 64 & 3.67 & 3.62 & 3.60 & 3.59 \\
 \hline
 32 & 3.63 & 3.61 & 3.59 & 3.56 \\
 \hline
 16 & 3.61 & 3.58 & 3.57 & 3.55 \\
 \hline
\end{tabular}
\caption{\label{tab:ppl_llama7B_nemo15B} Wikitext-103 perplexity compared to FP32 baseline in Llama2-7B and Nemotron4-15B, 340B models}
\end{table}

%\subsection{Perplexity achieved by various LO-BCQ configurations on MMLU dataset}


\begin{table} \centering
\begin{tabular}{|c||c|c|c|c||c|c|c|c|} 
\hline
 $L_b \rightarrow$& \multicolumn{4}{c||}{8} & \multicolumn{4}{c||}{8}\\
 \hline
 \backslashbox{$L_A$\kern-1em}{\kern-1em$N_c$} & 2 & 4 & 8 & 16 & 2 & 4 & 8 & 16  \\
 %$N_c \rightarrow$ & 2 & 4 & 8 & 16 & 2 & 4 & 2 \\
 \hline
 \hline
 \multicolumn{5}{|c|}{Llama2-7B (FP32 Accuracy = 45.8\%)} & \multicolumn{4}{|c|}{Llama2-70B (FP32 Accuracy = 69.12\%)} \\ 
 \hline
 \hline
 64 & 43.9 & 43.4 & 43.9 & 44.9 & 68.07 & 68.27 & 68.17 & 68.75 \\
 \hline
 32 & 44.5 & 43.8 & 44.9 & 44.5 & 68.37 & 68.51 & 68.35 & 68.27  \\
 \hline
 16 & 43.9 & 42.7 & 44.9 & 45 & 68.12 & 68.77 & 68.31 & 68.59  \\
 \hline
 \hline
 \multicolumn{5}{|c|}{GPT3-22B (FP32 Accuracy = 38.75\%)} & \multicolumn{4}{|c|}{Nemotron4-15B (FP32 Accuracy = 64.3\%)} \\ 
 \hline
 \hline
 64 & 36.71 & 38.85 & 38.13 & 38.92 & 63.17 & 62.36 & 63.72 & 64.09 \\
 \hline
 32 & 37.95 & 38.69 & 39.45 & 38.34 & 64.05 & 62.30 & 63.8 & 64.33  \\
 \hline
 16 & 38.88 & 38.80 & 38.31 & 38.92 & 63.22 & 63.51 & 63.93 & 64.43  \\
 \hline
\end{tabular}
\caption{\label{tab:mmlu_abalation} Accuracy on MMLU dataset across GPT3-22B, Llama2-7B, 70B and Nemotron4-15B models.}
\end{table}


%\subsection{Perplexity achieved by various LO-BCQ configurations on LM evaluation harness}

\begin{table} \centering
\begin{tabular}{|c||c|c|c|c||c|c|c|c|} 
\hline
 $L_b \rightarrow$& \multicolumn{4}{c||}{8} & \multicolumn{4}{c||}{8}\\
 \hline
 \backslashbox{$L_A$\kern-1em}{\kern-1em$N_c$} & 2 & 4 & 8 & 16 & 2 & 4 & 8 & 16  \\
 %$N_c \rightarrow$ & 2 & 4 & 8 & 16 & 2 & 4 & 2 \\
 \hline
 \hline
 \multicolumn{5}{|c|}{Race (FP32 Accuracy = 37.51\%)} & \multicolumn{4}{|c|}{Boolq (FP32 Accuracy = 64.62\%)} \\ 
 \hline
 \hline
 64 & 36.94 & 37.13 & 36.27 & 37.13 & 63.73 & 62.26 & 63.49 & 63.36 \\
 \hline
 32 & 37.03 & 36.36 & 36.08 & 37.03 & 62.54 & 63.51 & 63.49 & 63.55  \\
 \hline
 16 & 37.03 & 37.03 & 36.46 & 37.03 & 61.1 & 63.79 & 63.58 & 63.33  \\
 \hline
 \hline
 \multicolumn{5}{|c|}{Winogrande (FP32 Accuracy = 58.01\%)} & \multicolumn{4}{|c|}{Piqa (FP32 Accuracy = 74.21\%)} \\ 
 \hline
 \hline
 64 & 58.17 & 57.22 & 57.85 & 58.33 & 73.01 & 73.07 & 73.07 & 72.80 \\
 \hline
 32 & 59.12 & 58.09 & 57.85 & 58.41 & 73.01 & 73.94 & 72.74 & 73.18  \\
 \hline
 16 & 57.93 & 58.88 & 57.93 & 58.56 & 73.94 & 72.80 & 73.01 & 73.94  \\
 \hline
\end{tabular}
\caption{\label{tab:mmlu_abalation} Accuracy on LM evaluation harness tasks on GPT3-1.3B model.}
\end{table}

\begin{table} \centering
\begin{tabular}{|c||c|c|c|c||c|c|c|c|} 
\hline
 $L_b \rightarrow$& \multicolumn{4}{c||}{8} & \multicolumn{4}{c||}{8}\\
 \hline
 \backslashbox{$L_A$\kern-1em}{\kern-1em$N_c$} & 2 & 4 & 8 & 16 & 2 & 4 & 8 & 16  \\
 %$N_c \rightarrow$ & 2 & 4 & 8 & 16 & 2 & 4 & 2 \\
 \hline
 \hline
 \multicolumn{5}{|c|}{Race (FP32 Accuracy = 41.34\%)} & \multicolumn{4}{|c|}{Boolq (FP32 Accuracy = 68.32\%)} \\ 
 \hline
 \hline
 64 & 40.48 & 40.10 & 39.43 & 39.90 & 69.20 & 68.41 & 69.45 & 68.56 \\
 \hline
 32 & 39.52 & 39.52 & 40.77 & 39.62 & 68.32 & 67.43 & 68.17 & 69.30  \\
 \hline
 16 & 39.81 & 39.71 & 39.90 & 40.38 & 68.10 & 66.33 & 69.51 & 69.42  \\
 \hline
 \hline
 \multicolumn{5}{|c|}{Winogrande (FP32 Accuracy = 67.88\%)} & \multicolumn{4}{|c|}{Piqa (FP32 Accuracy = 78.78\%)} \\ 
 \hline
 \hline
 64 & 66.85 & 66.61 & 67.72 & 67.88 & 77.31 & 77.42 & 77.75 & 77.64 \\
 \hline
 32 & 67.25 & 67.72 & 67.72 & 67.00 & 77.31 & 77.04 & 77.80 & 77.37  \\
 \hline
 16 & 68.11 & 68.90 & 67.88 & 67.48 & 77.37 & 78.13 & 78.13 & 77.69  \\
 \hline
\end{tabular}
\caption{\label{tab:mmlu_abalation} Accuracy on LM evaluation harness tasks on GPT3-8B model.}
\end{table}

\begin{table} \centering
\begin{tabular}{|c||c|c|c|c||c|c|c|c|} 
\hline
 $L_b \rightarrow$& \multicolumn{4}{c||}{8} & \multicolumn{4}{c||}{8}\\
 \hline
 \backslashbox{$L_A$\kern-1em}{\kern-1em$N_c$} & 2 & 4 & 8 & 16 & 2 & 4 & 8 & 16  \\
 %$N_c \rightarrow$ & 2 & 4 & 8 & 16 & 2 & 4 & 2 \\
 \hline
 \hline
 \multicolumn{5}{|c|}{Race (FP32 Accuracy = 40.67\%)} & \multicolumn{4}{|c|}{Boolq (FP32 Accuracy = 76.54\%)} \\ 
 \hline
 \hline
 64 & 40.48 & 40.10 & 39.43 & 39.90 & 75.41 & 75.11 & 77.09 & 75.66 \\
 \hline
 32 & 39.52 & 39.52 & 40.77 & 39.62 & 76.02 & 76.02 & 75.96 & 75.35  \\
 \hline
 16 & 39.81 & 39.71 & 39.90 & 40.38 & 75.05 & 73.82 & 75.72 & 76.09  \\
 \hline
 \hline
 \multicolumn{5}{|c|}{Winogrande (FP32 Accuracy = 70.64\%)} & \multicolumn{4}{|c|}{Piqa (FP32 Accuracy = 79.16\%)} \\ 
 \hline
 \hline
 64 & 69.14 & 70.17 & 70.17 & 70.56 & 78.24 & 79.00 & 78.62 & 78.73 \\
 \hline
 32 & 70.96 & 69.69 & 71.27 & 69.30 & 78.56 & 79.49 & 79.16 & 78.89  \\
 \hline
 16 & 71.03 & 69.53 & 69.69 & 70.40 & 78.13 & 79.16 & 79.00 & 79.00  \\
 \hline
\end{tabular}
\caption{\label{tab:mmlu_abalation} Accuracy on LM evaluation harness tasks on GPT3-22B model.}
\end{table}

\begin{table} \centering
\begin{tabular}{|c||c|c|c|c||c|c|c|c|} 
\hline
 $L_b \rightarrow$& \multicolumn{4}{c||}{8} & \multicolumn{4}{c||}{8}\\
 \hline
 \backslashbox{$L_A$\kern-1em}{\kern-1em$N_c$} & 2 & 4 & 8 & 16 & 2 & 4 & 8 & 16  \\
 %$N_c \rightarrow$ & 2 & 4 & 8 & 16 & 2 & 4 & 2 \\
 \hline
 \hline
 \multicolumn{5}{|c|}{Race (FP32 Accuracy = 44.4\%)} & \multicolumn{4}{|c|}{Boolq (FP32 Accuracy = 79.29\%)} \\ 
 \hline
 \hline
 64 & 42.49 & 42.51 & 42.58 & 43.45 & 77.58 & 77.37 & 77.43 & 78.1 \\
 \hline
 32 & 43.35 & 42.49 & 43.64 & 43.73 & 77.86 & 75.32 & 77.28 & 77.86  \\
 \hline
 16 & 44.21 & 44.21 & 43.64 & 42.97 & 78.65 & 77 & 76.94 & 77.98  \\
 \hline
 \hline
 \multicolumn{5}{|c|}{Winogrande (FP32 Accuracy = 69.38\%)} & \multicolumn{4}{|c|}{Piqa (FP32 Accuracy = 78.07\%)} \\ 
 \hline
 \hline
 64 & 68.9 & 68.43 & 69.77 & 68.19 & 77.09 & 76.82 & 77.09 & 77.86 \\
 \hline
 32 & 69.38 & 68.51 & 68.82 & 68.90 & 78.07 & 76.71 & 78.07 & 77.86  \\
 \hline
 16 & 69.53 & 67.09 & 69.38 & 68.90 & 77.37 & 77.8 & 77.91 & 77.69  \\
 \hline
\end{tabular}
\caption{\label{tab:mmlu_abalation} Accuracy on LM evaluation harness tasks on Llama2-7B model.}
\end{table}

\begin{table} \centering
\begin{tabular}{|c||c|c|c|c||c|c|c|c|} 
\hline
 $L_b \rightarrow$& \multicolumn{4}{c||}{8} & \multicolumn{4}{c||}{8}\\
 \hline
 \backslashbox{$L_A$\kern-1em}{\kern-1em$N_c$} & 2 & 4 & 8 & 16 & 2 & 4 & 8 & 16  \\
 %$N_c \rightarrow$ & 2 & 4 & 8 & 16 & 2 & 4 & 2 \\
 \hline
 \hline
 \multicolumn{5}{|c|}{Race (FP32 Accuracy = 48.8\%)} & \multicolumn{4}{|c|}{Boolq (FP32 Accuracy = 85.23\%)} \\ 
 \hline
 \hline
 64 & 49.00 & 49.00 & 49.28 & 48.71 & 82.82 & 84.28 & 84.03 & 84.25 \\
 \hline
 32 & 49.57 & 48.52 & 48.33 & 49.28 & 83.85 & 84.46 & 84.31 & 84.93  \\
 \hline
 16 & 49.85 & 49.09 & 49.28 & 48.99 & 85.11 & 84.46 & 84.61 & 83.94  \\
 \hline
 \hline
 \multicolumn{5}{|c|}{Winogrande (FP32 Accuracy = 79.95\%)} & \multicolumn{4}{|c|}{Piqa (FP32 Accuracy = 81.56\%)} \\ 
 \hline
 \hline
 64 & 78.77 & 78.45 & 78.37 & 79.16 & 81.45 & 80.69 & 81.45 & 81.5 \\
 \hline
 32 & 78.45 & 79.01 & 78.69 & 80.66 & 81.56 & 80.58 & 81.18 & 81.34  \\
 \hline
 16 & 79.95 & 79.56 & 79.79 & 79.72 & 81.28 & 81.66 & 81.28 & 80.96  \\
 \hline
\end{tabular}
\caption{\label{tab:mmlu_abalation} Accuracy on LM evaluation harness tasks on Llama2-70B model.}
\end{table}

%\section{MSE Studies}
%\textcolor{red}{TODO}


\subsection{Number Formats and Quantization Method}
\label{subsec:numFormats_quantMethod}
\subsubsection{Integer Format}
An $n$-bit signed integer (INT) is typically represented with a 2s-complement format \citep{yao2022zeroquant,xiao2023smoothquant,dai2021vsq}, where the most significant bit denotes the sign.

\subsubsection{Floating Point Format}
An $n$-bit signed floating point (FP) number $x$ comprises of a 1-bit sign ($x_{\mathrm{sign}}$), $B_m$-bit mantissa ($x_{\mathrm{mant}}$) and $B_e$-bit exponent ($x_{\mathrm{exp}}$) such that $B_m+B_e=n-1$. The associated constant exponent bias ($E_{\mathrm{bias}}$) is computed as $(2^{{B_e}-1}-1)$. We denote this format as $E_{B_e}M_{B_m}$.  

\subsubsection{Quantization Scheme}
\label{subsec:quant_method}
A quantization scheme dictates how a given unquantized tensor is converted to its quantized representation. We consider FP formats for the purpose of illustration. Given an unquantized tensor $\bm{X}$ and an FP format $E_{B_e}M_{B_m}$, we first, we compute the quantization scale factor $s_X$ that maps the maximum absolute value of $\bm{X}$ to the maximum quantization level of the $E_{B_e}M_{B_m}$ format as follows:
\begin{align}
\label{eq:sf}
    s_X = \frac{\mathrm{max}(|\bm{X}|)}{\mathrm{max}(E_{B_e}M_{B_m})}
\end{align}
In the above equation, $|\cdot|$ denotes the absolute value function.

Next, we scale $\bm{X}$ by $s_X$ and quantize it to $\hat{\bm{X}}$ by rounding it to the nearest quantization level of $E_{B_e}M_{B_m}$ as:

\begin{align}
\label{eq:tensor_quant}
    \hat{\bm{X}} = \text{round-to-nearest}\left(\frac{\bm{X}}{s_X}, E_{B_e}M_{B_m}\right)
\end{align}

We perform dynamic max-scaled quantization \citep{wu2020integer}, where the scale factor $s$ for activations is dynamically computed during runtime.

\subsection{Vector Scaled Quantization}
\begin{wrapfigure}{r}{0.35\linewidth}
  \centering
  \includegraphics[width=\linewidth]{sections/figures/vsquant.jpg}
  \caption{\small Vectorwise decomposition for per-vector scaled quantization (VSQ \citep{dai2021vsq}).}
  \label{fig:vsquant}
\end{wrapfigure}
During VSQ \citep{dai2021vsq}, the operand tensors are decomposed into 1D vectors in a hardware friendly manner as shown in Figure \ref{fig:vsquant}. Since the decomposed tensors are used as operands in matrix multiplications during inference, it is beneficial to perform this decomposition along the reduction dimension of the multiplication. The vectorwise quantization is performed similar to tensorwise quantization described in Equations \ref{eq:sf} and \ref{eq:tensor_quant}, where a scale factor $s_v$ is required for each vector $\bm{v}$ that maps the maximum absolute value of that vector to the maximum quantization level. While smaller vector lengths can lead to larger accuracy gains, the associated memory and computational overheads due to the per-vector scale factors increases. To alleviate these overheads, VSQ \citep{dai2021vsq} proposed a second level quantization of the per-vector scale factors to unsigned integers, while MX \citep{rouhani2023shared} quantizes them to integer powers of 2 (denoted as $2^{INT}$).

\subsubsection{MX Format}
The MX format proposed in \citep{rouhani2023microscaling} introduces the concept of sub-block shifting. For every two scalar elements of $b$-bits each, there is a shared exponent bit. The value of this exponent bit is determined through an empirical analysis that targets minimizing quantization MSE. We note that the FP format $E_{1}M_{b}$ is strictly better than MX from an accuracy perspective since it allocates a dedicated exponent bit to each scalar as opposed to sharing it across two scalars. Therefore, we conservatively bound the accuracy of a $b+2$-bit signed MX format with that of a $E_{1}M_{b}$ format in our comparisons. For instance, we use E1M2 format as a proxy for MX4.

\begin{figure}
    \centering
    \includegraphics[width=1\linewidth]{sections//figures/BlockFormats.pdf}
    \caption{\small Comparing LO-BCQ to MX format.}
    \label{fig:block_formats}
\end{figure}

Figure \ref{fig:block_formats} compares our $4$-bit LO-BCQ block format to MX \citep{rouhani2023microscaling}. As shown, both LO-BCQ and MX decompose a given operand tensor into block arrays and each block array into blocks. Similar to MX, we find that per-block quantization ($L_b < L_A$) leads to better accuracy due to increased flexibility. While MX achieves this through per-block $1$-bit micro-scales, we associate a dedicated codebook to each block through a per-block codebook selector. Further, MX quantizes the per-block array scale-factor to E8M0 format without per-tensor scaling. In contrast during LO-BCQ, we find that per-tensor scaling combined with quantization of per-block array scale-factor to E4M3 format results in superior inference accuracy across models. 



\end{document}
\endinput
%%
%% End of file `sample-authordraft.tex'.
