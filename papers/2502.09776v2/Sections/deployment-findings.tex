\section{Findings from Study 2}

\topic{Similar to the first study, participants shared varied interpretations of DMH tools.
Participants mentioned enrolling in text messaging services (DP1, DP7, DP13, DP14), participating in digital journaling (DP1), engaging in social media channels (DP4, DP8, DP20), connecting with therapists over digital platforms (DP12), and using apps like Headspace (DP21), Calm (DP13), and Forest (DP5).}
They recommended several ways to account for the SEDs in the design of DMH tools. We describe these potential solutions below.
%We delve into these recommended solutions below.


\subsection{Prioritization of Responsibilities}
During the workshop, participants identified that feelings of overwhelm often arise from juggling numerous personal and professional responsibilities. For instance, SP11 recalled how they had to balance multiple exams, complete their thesis, and work at a part-time job to support their household during their final year as an undergraduate student. 
% On the other hand, individuals like SP23 highlighted the demands of caring for sick family members. 
Reflecting on these diverse experiences, participants acknowledged the crucial need to prioritize responsibilities. SP6 suggested that DMH tools could help individuals with multiple demands on their time as follows:

\italquote{Try to weigh all of these responsibilities and see which of them take precedence over the rest, and try to fulfill those responsibilities in that order. Prompt the user to deeply think about these responsibilities so they can get a better handle on things.}

Tied into these discussions was a recurring theme of procrastination. Participants often cited an accumulation of tasks that were deferred out of anxiety or a lack of motivation.
% propelled by a combination of the sheer volume of tasks and a tight timeframe for completion. 
To mitigate this issue, participants like SP7 proposed that DMH tools could help users deconstruct larger responsibilities into smaller, more manageable tasks. They believed that this approach would make tasks more actionable, foster a sense of productivity, and subsequently disrupt the cycle of procrastination.
However, participants in the workshop recognized that prioritizing responsibilities often necessitates carefully considering and endorsing a rationale for why certain tasks take priority. They thought this could be facilitated by soliciting users' short-term and long-term objectives, and aiding them in formulating action plans to meet these goals. %SP14 personally attested to the benefits of prioritizing tasks during moments of crisis, as it only required momentarily breaking from extracurricular activities to focus on work and academic obligations. 
% SP14 attested to the benefits of prioritizing tasks during moments of crisis, recalling from a personal experience when they could identify that they could take a break from extracurricular activities to focus on work and academic obligations.
SP1 noted that seeing their responsibilities in written form \textit{`just makes it feel like grounded and like more in control'}.

Participants were also aware of the mental effort and cognitive resources required for these practices, making them more suitable for periods of relative downtime like weekends and holidays. SP7 suggested that DMH tools could help users articulate goals and plan actions during these periods. To keep these goals at the forefront, participants recommended reminders throughout the week, with SP7 emphasizing morning reminders to align daily activities with long-term objectives.

Still, SP21 cautioned that writing down detailed, elaborate plans on mobile devices could contribute to additional cognitive overload, even during free time. Instead, they suggested that users could be offered the option to either formulate their plans mentally or using traditional pen-and-paper methods.

\subsection{Incorporating Self-Care into Daily Routines}

Participants emphasized the importance of maintaining a balanced lifestyle with respect to sleep, diet, and exercise. This was regarded as an important factor in managing one's overall wellbeing, especially during periods of stress or overwhelm. Based on challenges identified in our first study related to maintaining a consistent sleep schedule in the face of SEDs, workshop participants proposed that DMH tools should include features to encourage sleep hygiene. SP10 suggested that users' preferred sleeping hours should be gathered during the registration process, allowing DMH tools to send prompts when individuals' activities on their phones continue beyond the specified bedtime.
During periods of illness or heightened stress, participants acknowledged that neglecting meals had an adverse effect on their mental health. In such scenarios, the recommendation was to focus on ensuring users remember to eat and hydrate. SP7 commented:

\italquote{Instead of like constant messages, add more general reminders like eating, drinking water, etc. Depending on the time of the day, you can send messages like `have a snack,' `drink some water,' `do breathing exercises,' or food and snack ideas.}

Reflecting on the toll of substantial familial responsibilities, participants like SP23 and SP25 expressed a tendency to neglect self-care in favor of attending to the needs of others. They often attributed this inclination to the absence of a systematic approach to time management. To address this issue, they suggested that DMH tools could assist users in carving out dedicated time for self-care. SP23 elaborated that these reserved times could be spent meeting friends or engaging in restful activities like reading or sleeping. However, we observed varied preferences for how this personal time should be distributed; while some preferred an hour each day, others leaned towards larger blocks of time or even full days every few weeks.


To support users in navigating their daily activities and appropriately allocating time for self-care, participants like SP5 and SP14 suggested that DMH tools should consider integrating with users' digital calendars or other platforms commonly used for academic or professional communication (e.g., Microsoft Teams, Slack). They proposed that this synchronization would allow such tools to gain a nuanced understanding of users' schedules and obligations, consequently enabling more personalized and timely interventions. For example, reminders or prompts for self-care activities could be strategically scheduled during free periods identified in users' calendars, thus assisting them in maintaining a healthy work-life balance amidst their professional or academic responsibilities. Participants also anticipated that seeing a time slot reserved for self-care in their calendar would reinforce their commitment to these activities.

\subsection{Alternate Framing of Disengagement}

Participants acknowledged that irrespective of the quality and effectiveness of a DMH tool, users are likely to experience occasional disengagement due to either voluntary or involuntary reasons. They highlighted the mental hurdle of re-engaging with these tools after a period of non-use. Reflecting on their past experiences with other DMH tools, SP1 expressed a negative sentiment regarding the tools' emphasis on continuous use. They noted that when they took a break from the tools, the feeling of losing their streak demotivated them. This demotivation stemmed from the perception that they would have to put in significant effort to regain the level of consistent use they had previously achieved. They commented:

\italquote{If I consistently used it every day and then I forgot one day, or like I forgot a couple of times, I'd be like, `Oh, but I already lost the streak.` Then I might as well just like give up the whole thing, which is not great.}

\noindent
To support users during periods of heightened stress or activity, SP2 suggested enabling them to specify days when they might be occupied with exams or work responsibilities. Marking such days in a DMH tool would prompt the system to lower engagement expectations, helping prevent users from completely withdrawing. For example, users could be asked to reflect on their mood or energy level with a number or emoji rather than a full journal entry. 
% SP5 also echoed this sentiment, suggesting that on these busy days, users should not be asked to write long texts.

Despite these recommendations, participants acknowledged that occasional disengagement is inevitable.  
During prolonged periods of user inactivity, SP22 proposed that DMH tools should respond by issuing straightforward and actionable suggestions to re-engage the users. However, SP5 and SP9 warned that the frequency of such suggestions should not be overwhelming, as too many notifications could potentially irritate users and cause them to develop a strong aversion to the tool. 
%SP5 and SP9 cautioned that the frequency of such suggestions should not be overwhelming. They reminded us that an overload of notifications during periods of disengagement could have the opposite effect, provoking irritation or even triggering a strong aversion towards the tool. 
SP9 shed light on this perspective from their previous experiences:

\italquote{From the meditation app, I actually got a little bit annoyed because it would sort of bloat my notification list. I kind of get mad if I don't see the value in the amount of times my phone is being blown up. \ldots If it's sending me a lot of messages and long messages, I might go into rebellious mode and just not even answer out of pure hate or maybe just possibly hit skip.}

%Hence, participants went on to suggest that DMH tools should accept and normalize such behaviors.  SP6 suggested that the text messaging application should remind people that struggles in life are temporary and with the things they can control, they can improve their mental state, usage of the app being one of them.
%They emphasized that the success of using a tool should not be solely defined by continuous usage but rather appreciated for the fact that users are utilizing the tool to improve their wellbeing. 
%However, 

In light of these reflections, workshop participants suggested a shift in perspective. Instead of perceiving periods of disengagement as user failures, DMH tools should understand and accept these moments as part of the user's journey. SP6 highlighted the importance of acknowledging struggles and promoting the idea that focusing on more controllable aspects of life, such as engagement with a DMH tool, can positively influence one's mental health. 
The goal here would be not to pressure users into continuous usage, but to foster an environment where they feel supported and understood, even during periods of disengagement. Participants emphasized that the success of using DMH tools should not be solely defined by continuous usage but rather the fact that users are choosing to utilize the tool to improve their wellbeing, despite barriers and setbacks. 


\subsection{Leveraging External Support and Community Resources}
Participants underlined the significant role external support and community resources can play when it comes to managing overwhelming situations. 
Drawing on their academic experiences, participants like SP2 and SP8 expressed the ease with which people can overlook potential help from their peer network. They described times when they grappled with academic assignments on their own, forgetting they could lean on friends and teaching assistants for guidance. They suggested that a gentle reminder to consider support resources could be beneficial in moments of high stress. Participants also spoke about how they appreciated reading narratives that echoed their specific struggles. They expressed that seeing examples of how others have managed to navigate similar issues could offer them hope, strengthen their belief in their ability to improve their situation, and potentially provide practical directions to explore.

On the whole, participants described that addressing individuals' psychological wellbeing sometimes must go beyond merely providing supportive messages or teaching psychological strategies. In addition to encouragement to draw on their existing social networks for support, some participants suggested that people dealing with financial burdens may benefit from DMH tools that direct users to unconventional financial resources. Drawing upon a hypothetical scenario of a single mother juggling multiple jobs and having little time for self-care, SP13 posited one such example:

\italquote{For single mothers, there are many government and non-government organizations that give grants. \dots And if possible, she could collect grants, or maybe someone wants to start up a small business.}

\noindent
Concerning individuals with unstable housing conditions, participants pointed to websites and social media groups dedicated to helping people find suitable housing. They relayed that many users are often unaware of these resources, so DMH tools could provide a convenient gateway to them. 

Participants recognized the difficulty in tailoring DMH tools to users' specific ongoing life challenges, as these issues are often open-ended and unique to each individual. Therefore, %SP10 proposed that DMH tools might draw inspiration from platforms like Snapchat to gather information about the primary challenges users face in their lives. 
SP10 proposed that DMH tools could learn from the approach employed by the social media platform Snapchat, which collects data on users' preferences and interests as part of its sign-up process. They went on to say: %that DMH tools could feature an onboarding process that allows users to indicate the specific challenges they are currently facing. They explained:
\revision{\italquote{When you first log in as a user, you get the option to choose what kind of problem you're struggling with, kind of the scenario you're in. And you know, you can get responses based on that scenario.}}


In a complementary suggestion, SP1 recommended that DMH tools could then compile a repository of resources tailored to address those challenges. When users express their concerns, a matchmaking process could be initiated to direct people to resources related to their issues. Given recent excitement for large language models (LLMs), a few participants theorized that LLMs could be used to match people's open-ended concerns to resources. Nevertheless, participants were wary of the reliability, suitability, and scalability of such a process.
% also noted the limitations of these approaches, with scalability being a significant concern.

% In light of this, a few participants suggested that DMH tools might consider utilizing large language models (LLMs) due to their ability to respond to open-ended queries. They felt that LLMs could potentially assist them in identifying solutions tailored to their specific circumstances. However, they also voiced concerns about the reliability and suitability of advice generated by LLMs.


% \subsection{Alternate Modes of Interaction}

% \todo{Ananya Note: I am thinking of cutting this section because this provides a counter-argument to the need for text message and also not really a novel finding. Reviewers had issues about this section too.}

% Echoing Study 1, a few participants shared their reluctance to read lengthy text messages, particularly when they are distracted or engrossed in other tasks. In these situations, audio-based messages offer the distinct advantage of enabling engagement with minimal diversion from ongoing activities. Underlining the practicality of voice messages as an alternative, SP16 shared: 


% %Workshop participants, similar to participants from the first study, indicated that there could be people who do not prefer to type or even those who do may not be able to do so in different circumstances. People like SP1, SP16, and SP21 suggested that it might be easier for some people to talk about their experiences regarding mental health or stressful situations, rather than texting. Explaining the conveniences of voice messages, SP16 commented:

% \italquote{If possible, messages can be sent in voice form. I feel it's much, much easier to listen than to write. You can listen while on the go rather than sitting to read a text message and trying to reply.}

% Even outside the work context, some participants posited that occasional messages enriched with multimedia elements would be more likely to capture their attention, providing more engaging and visually stimulating content. They suggested that infographics, short video clips, or animations would provide a more vivid visualization of best practices in self-care. For example, participants suggested that a short video demonstrating mindfulness techniques or an infographic showing the steps to manage stress could be more impactful and easier to follow than plain text.


% In line with findings from our initial study, several participants including SP1, SP16, and SP21 remarked on the challenges that users may encounter when attempting to articulate their thoughts through written text. They suggested that speaking might pose less of a cognitive burden than writing, particularly on mobile devices where manual typing can be cumbersome. Verbal communication, they argued, could offer a more intuitive and mentally less demanding alternative, especially when individuals are by themselves.
% %Aligning with findings from the first study, participants also commented on factors that dissuade people from writing or typing out their thoughts. SP1, SP16, and SP21 thought that users might find it easier to verbalize their experiences rather than formulating them in written text. They argued that replying with long responses could pose a cognitive burden, especially when it would involve manual typing on a mobile device. On the other hand, verbal communication might come across as more intuitive and less mentally challenging, particularly in scenarios where individuals are alone. 

% To further diversify user interaction, participants recommended the use of images and videos as input to DMH tools in order to demonstrate their progress. For instance, a user could share a photo of a quiet spot in nature where they practiced mindfulness, or a screenshot of a completed to-do list as a record of their productivity. SP1 commented that this image-based approach could help users generate a personalized depiction of their mental health journey, thereby making the task of documenting progress less strenuous and more immersive.

% %When it comes to interacting with messages, workshop participants like SP1, SP16, and SP21 highlighted that some individuals may find it easier to articulate their mental health experiences or stressful situations verbally rather than in written form. They mentioned that writing could be cognitively demanding, particularly when one is dealing with mental health issues. In contrast, speaking about their problems might feel more intuitive and less taxing. Furthermore, participants suggested when users are asked what they did regarding practicing a new psychological strategy or adopting a new behavior, users might be allowed to reply with images, instead of writing down their progress. SP1 thought that it could also offer a more concrete and personal representation of their experiences and allow them to document their progress without much effort.


% %This could involve the use of visuals, such as pictures taken by participants themselves, which could provide a more engaging and less cognitively demanding alternative. 