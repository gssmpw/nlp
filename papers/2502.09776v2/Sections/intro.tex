\section{Introduction}
%DMH tools are helpful, but they face a lot of challenges
%Digital mental health (DMH) tools enable users to access resources and strategies for managing their psychological wellbeing at their own convenience \todo{cite}. While a range of DMH tools have demonstrated success in various settings, a common concern against most of them is that users are often not able to engage with them and as a result, not able to extract the potential benefits DMH tools can offer. A significant amount of literature has looked into the main causes of disengagement, highlighting issues like generic content, interventions not matched to users' state of mind at a particular time, or failure to identify opportune moments to intervene users.

Mental health conditions such as anxiety and depression are highly prevalent among adults \cite{vahratian2021symptoms, thomeer2023racial}. These issues are often associated with substantial distress and disability and can lead to serious disruptions in areas like work, education, and social relationships \cite{goodwin2022trends, kessler2022estimated}. Traditional therapeutic interventions can be effective in alleviating mental health symptoms and improving overall wellbeing, but are often inaccessible due to barriers like cost, societal stigma, or a preference for self-reliance \cite{gulliver2010perceived, robards2018marginalized, oguamanam2023intersectional, lattie2020designing, frik2023model}. 
These barriers have led to the development of digital mental health (DMH) tools accessible through everyday devices like computers and smartphones to complement traditional interventions \cite{fairburn2017impact, mohr2017personal, bhattacharjee2023investigating}. Their versatility has been showcased across a spectrum of settings, offering notable benefits like accessibility and personalization of care \cite{lattie2022overview, kornfield2022meeting, pendse2022treatment, mohr2018solution}.
%Digital mental health (DMH) tools have gained substantial popularity for their potential to conveniently deliver resources that help people manage their psychological wellbeing. 
However, frequent user disengagement presents a significant obstacle to the effective utilization of DMH tools \cite{borghouts2021barriers}. 
% While quantified in numerous ways (e.g., days of sustained use, user responsiveness, daily sessions of use)

In the context of self-guided DMH tools, engagement broadly refers to the process through which a user creates, sustains, and eventually terminates a relationship with a computerized system over time \cite{sidner2004look, xu2013designing, doherty2018engagement}. Although it is not necessary for every DMH tool to be continuously used over weeks or months, the acquisition of knowledge and self-management skills through many such tools is predicated on their usage over an extended period \cite{scholten2019use, pendse2022treatment}. Thus, engagement often focuses on the extent to which an individual makes productive use of a tool in their daily life \cite{schueller2017integrating, zech2023integrative}. While many individuals are eager to try out new DMH tools, they often lack the sustained engagement needed to support learning and applying new skills and perspectives \cite{muench2017more, lipschitz2023engagement}. As such, the benefits of DMH tools that require prolonged use often remain unrealized relative to their potential \cite{figueroa2022daily, bhattacharjee2023integrating}. 

Several studies have explored the underlying causes of disengagement with DMH tools, identifying factors such as the provision of generic content and failures to recognize the optimal moments for user intervention \cite{bhattacharjee2023investigating, rennick2016health, bhattacharjee2022design, kornfield2020energy}.
%Most of these challenges are viewed individualistically, but no literature on how socio-economic factors impact engagement with DMH tools
%Research on enhancing engagement with DMH tools and digital tools, in general, focuses on integrating features that may enhance engagement (e.g., coaching, peer-to-peer communication, gamification), fixing usability problems, and tailoring the content or the timing of the intervention elements to match individual preferences \cite{marangunic2015technology, venkatesh2008technology, davis1989perceived, venkatesh2003user, dwivedi2019re, borghouts2021barriers}. 
To address these issues, HCI and CSCW researchers have largely concentrated on integrating new DMH tool features that enhance user interest (e.g., coaching, peer-to-peer communication, gamification), improve usability, and tailor content or timing to users' needs \cite{saleem2021understanding, dwivedi2019re, borghouts2021barriers, kornfield2022involving, ng2019provider, yoo2024missed}. However, recent studies have also noted the limits of such %technology-centric 
approaches to improving engagement when disengagement can also stem from fundamental disruptions in daily life that dramatically shift user's priorities~\cite{lipschitz2023engagement, pendse2022treatment}. For instance, an unexpected change in one's work schedule or a sudden financial crisis might derail their ability or motivation to engage with a DMH tool, regardless of how well the tool is designed or tailored to their preferences. These circumstantial issues are tied to social context, which encompasses the specific settings in which human relationships and personal circumstances influence the way one thinks and behaves~\cite{dourish2004we, bronfenbrenner1977toward, helliwell2004social}. The impact of social context on engagement is well-established in face-to-face therapy, where issues like job constraints and financial hardship are often cited as reasons for early termination, sometimes more so than dissatisfaction with therapy  \cite{renk2002reasons, roe2006clients, hynan1990client}. Since DMH tools are intended to be more convenient and self-directed, the role of circumstantial factors in disrupting user engagement with them is less studied.

%we posit that these influences may extend to interactions with digital platforms. 
Given that key factors related to one's social context can profoundly influence individuals' goals and priorities \cite{gallie2019research, allen2014social, slavich2020social, tachtler2020supporting, sabie2020memory, bhattacharjee2022kind}, \revision{we are interested in exploring how these factors impact their use of DMH tools that require ongoing engagement over long periods during which one's social context is likely to evolve.} Changes in social context can exacerbate mental health symptoms, inducing a sense of overwhelm and undermining the prioritization of self-care \cite{slavich2020social, mills2020prioritising}; these setbacks pose a fundamental challenge to sustained usage of these tools.
%Changes in social context can disrupt engagement while exacerbating mental health symptoms, inducing a sense of overwhelm, and undermining the prioritization of self-care \cite{wilson2020job, allen2014social}, thereby posing a fundamental challenge to sustained usage of these tools.
% Previous research has generally not taken into account these long-term dynamics that may play out as individuals with mental health concerns experience changes in their life circumstances as they try to incorporate a digital tool into their daily lives while balancing various roles and responsibilities.
%The relevance of such disruptors increases in the context of DMH tools given that needed engagement with a tool may extend over weeks or months. Thus, prior works have generally overlooked the long-term dynamics that might arise among users with mental health concerns as they incorporate a tool into their daily lives while juggling diverse roles and responsibilities.  
%These individual challenges and social factors may present fundamental barriers to engaging with DMH tools, necessitating more than just superficial adjustments in design or content. These obstacles could be highly disruptive to engagement, intricately tied to daily life and long-term goals. For instance, an unexpected change in work schedule or a sudden financial crisis might derail an individual's ability or motivation to engage with a DMH tool, regardless of how well the tool is designed or tailored to their preferences. The complexity, variability, and unpredictability of these factors might require not viewing engagement merely as a matter of personal preference and system improvement, but rather as a complex interaction rooted in their unique social contexts and individual circumstances.
%Such factors may constitute more fundamental barriers, requiring a comprehensive consideration that shapes every aspect of tool design and implementation, rather than merely selecting among existing offerings. These barriers could be highly disruptive to engagement, and ongoing monitoring might be unfeasible, given the diverse and variable nature of social factors.
% Motivated by these opportunities, we explore the role of one's social context in inhibiting or shaping their engagement with DMH tools. 
We specifically focus on text-messaging DMH tools designed for long-term use spanning weeks to months. 
Text messaging is quickly becoming one of the well-established DMH platforms for promoting psychological wellbeing \cite{bhattacharjee2023investigating,  kornfield2022meeting}. Compared to alternatives like mobile applications and online programs, text messaging has been embraced for its wider reach and accessibility in supporting psychological wellbeing \cite{kretzschmar2019can, chikersal2020understanding, rathbone2017use, inkster2018empathy, morris2018towards}.
In addition, reflecting their long-term use, this focus allows for insight into engagement patterns as they emerge over time and across changes in life circumstances.
% to tools that are likely to be susceptible to situations when an individuals, as DMH tools that require shorter adherence periods (e.g., a few days or a week) may not need to deeply consider how fluctuations in social context affect engagement, which falls outside our study's scope.

These considerations motivate the following two research questions:

\begin{itemize}
\item \textbf{RQ1:} How does an individual's social context influence their engagement with DMH tools \revision{that require extended use spanning weeks or months}?
\item \textbf{RQ2:} How can \revision{those} DMH tools be designed to better account for the challenges and opportunities presented by an individual's unique social context?
\end{itemize}

\noindent
By investigating these questions and developing a deeper understanding of how one's broad societal context shapes engagement, designers may be able to create tools that enable users to navigate challenges without disrupting their interaction with the system. %Users may also be able to preserve their connection with the tools during challenging circumstances and continue to utilize it when they have the capacity to do so in the future. 
Furthermore, DMH tools may be designed in such a way that they are not just resilient in the face of these disruptors but are actively helpful in navigating them. This shift in perspective may allow the DMH tools to be supportive of users' unique situations, rather than merely functioning in spite of them. DMH tools that require shorter adherence periods (e.g., a few minutes or hours) may not need to deeply consider how fluctuations in social context affect engagement and, hence, fall outside our study's scope.

%Motivated by these opportunities, we aim to explore the role of what we term Situational Engagement Disruptors (SEDs) — specific obstacles originating from an individual's particular social context and personal circumstances that might inhibit or shape their engagement with DMH tools. By identifying these specific SEDs, we aspire to generate design recommendations that are more closely aligned with users' needs, acknowledging and addressing the multifaceted complexities of their social context. 


%The relevance of these factors becomes even more accentuated when considering DMH tools. A plethora of literature underscores the profound influence of such factors on shaping and determining one's psychological wellbeing and needs \cite{alegria2018social, allen2014social, bhattacharjee2021understanding, lewis2019cultural, slavich2023social, tachtler2021unaccompanied, lattie2020designing, bhattacharjee2023integrating, slavich2020social}. These elements can impact not only an individual's specific preferences but also dictate their perceptions regarding the utility and accessibility of a digital tool. 


%For instance, a professional navigating a demanding work schedule might gravitate towards a DMH tool that provides concise, actionable insights during office breaks rather than one that demands lengthy interactive sessions. Conversely, an individual struggling with financial challenges might prioritize immediate day-to-day necessities, relegating extended mental health interventions to the background. The nuances in these priorities can significantly influence how individuals engage with DMH tools, an aspect that frequently goes unnoticed.
%Such distinctions in priorities can play important roles in shaping their engagement patterns with DMH tools, which are often overlooked.



%Research on engagement with DMH tools, and even digital tools at large, predominantly concentrates on addressing users' individual preferences, be it in terms of content, timing, perceived benefits, or ease of use, all aimed at delivering personalized, context-aware support \todo{cite}. Although effective to a degree, this strategy may neglect the encompassing social environments and circumstances individuals navigate — from their socio-cultural upbringing, familial and workplace obligations, to their socio-economic status that, in turn, shapes their aspirations and life priorities \todo{cite}. In the context of DMH tools, these factors might be even more important since there is substantial literature that supports that these factors play significant roles in shaping and determining one's mental health and the needs \todo{cite}. These factors are fundamental facets of a person's life, influencing not only their specific needs and preferences but also their perceived usefulness and comfort in engaging with a digital tool \todo{cite}. For example, a professional grappling with an intense work schedule might lean towards a DMH tool that dispenses brief, actionable advice during office hours, rather than one demanding extended interactive sessions. Again, someone from a less privileged socio-economic bracket might place immediate necessities and survival over protracted mental health interventions, thereby influencing their engagement patterns with DMH tools.


%The current literature on engagement with DMH tools, and to some extent, with general tools, largely focus on addressing one's individual preferences in terms of content and timing as well as their perceived benefits and ease of use to provide personalized and contextualized support \todo{cite}. While this approach can provide considerable success, they are less likely to capture the broad social environments and contexts they live in -- the conditions they were born, the responsibilities they have for families as well as occupation, and their socio-economic status which determines their goals and priorities in life \todo{cite}. These contextual factors are core components of an individual's life, playing important roles in shaping individual needs and preferences, and how easy or beneficial they would find engaging with a tool \todo{cite}. For example, a working professional with demanding job hours might prioritize a tool that offers brief, actionable strategies during working hours rather than lengthy interactive sessions, given their limited time. Again, an individual from a lower socio-economic background might prioritize immediate survival needs over long-term mental health interventions.





%Therefore, it is crucial to understand how one's social context impacts their engagement with DMH tools \todo{cite}.

%ignore the role one's social contexts play in shaping engagement. Social context could encompass a broad range of elements in one life -- they could range from the conditions they were born in, the responsibilities they have for families as well as occupation, and their socio-economic status which determines their goals and priorities in life \todo{cite}. These factors are core components of an individual's life, and there is substantial literature that supports that a combination of these factors plays significant roles in shaping and determining one's mental health.


%In our research, we aim to delve into the relationship between an individual's social contexts and their engagement with DMH tools. Specifically, we seek to understand the broader social factors that might hinder users from engaging with these tools. 

%\todo{define social context}

%\todo{preview of the findings}

We commenced our research by deploying an eight-week text messaging program for managing psychological wellbeing to 20 individuals. Utilizing two rounds of interviews, we examined the influence of participants' social contexts on their engagement with the program. Through this exploration, we identified various situational engagement disruptors (SEDs) that include academic challenges, workplace demands, family obligations, and unexpected life events such as health issues and relocation. \topic{These SEDs were often characterized by a sense of overwhelm -- the feeling of being unable to keep up with the various demands on oneself, marked by high levels of stress and a perceived loss of control over one’s life \cite{kabigting2019conceptual, hopps1995power}. Consequently, SEDs hindered participants' ability to prioritize self-care and dedicate time to interact with DMH tools.}
To address these disruptors, we conducted a subsequent study featuring five design workshops with 25 participants to identify strategies for incorporating these considerations into DMH tool design. Our research yielded several key design considerations: prioritizing self-care through structured goal-setting, designs that accommodate flexible engagement, and addressing SEDs within social contexts.
Our contributions include:

\begin{itemize}
    \item The inclusion of social context as a critical variable in the study of engagement with DMH tools,
    \item The identification of specific SEDs that impact user engagement, and
    \item The provision of targeted design strategies to mitigate the identified SEDs and enhance user engagement.
\end{itemize}

