\section{Procedure of Study 1}
We conducted our first study with 20 participants in an eight-week text messaging program designed to improve psychological wellbeing. The duration of the program enabled us to examine broader social contexts that hindered and shaped their engagement with a DMH tool. The study received approval from the Research Ethics Board at \redacted{Northwestern University}, and its logistics are described below.

\subsection{Participants}

Participant recruitment for this study was facilitated by \redacted{our research center's} Online Research Registry as well as a web-based screening platform of a nonprofit mental health advocacy organization called \redacted{Mental Health America (MHA)}. %Participants were recruited between August and September 2022. 
The study was presented as a field trial to test a text-messaging tool constructed to aid individuals in managing their symptoms of depression and anxiety. Across both recruitment sources, participants were required to be at least 18 years old, own a mobile phone, be a resident of the United States, and not currently receiving psychotherapy or planning to start psychotherapy in the study timeframe. 

Participants also completed self-screening surveys for depression and anxiety to confirm that they reflect potential users who stand to benefit from engaging with a DMH tool. Eligibility was determined based on self-reported scores from the Patient Health Questionnaire-9 (PHQ-9) \cite{kroenke2001phq} and the General Anxiety Disorder-7 (GAD-7) \cite{williams2014gad}. Those who scored 10 or higher on either questionnaire were considered eligible for participation.


%Participants recruited through the advocacy organization took the Patient Health Questionnaire-9 (PHQ-9) \cite{kroenke2001phq} and the General Anxiety Disorder-7 (GAD-7) \cite{williams2014gad}, while those recruited through the registry were only given the PHQ-9. Those who indicated moderate to severe symptoms according to PHQ-9 or GAD-7 — determined by a score of 10 or more for both questionnaires — were eligible to participate. 
% Subsequently, prospective participants were directed to an additional screening survey. Those recruited through the registry were emailed an invitation to learn about the study and complete screening procedures if interested. Those screening procedures included the PHQ-9, with an individual being eligible if they had a score of 10 or higher.

The final cohort comprised 20 participants with an average age of $31.0\pm3.0$ years. The participants identified with multiple genders (17 women, 2 men, 1 non-binary) and several racial groups (10 White, 1 Black/African American, 1 Asian, 1 American Indian/Alaskan Native, 3 mixed race, and 4 undisclosed). 
\topic{Participants were asked whether a specialist or healthcare provider had ever diagnosed them with major depression or anxiety disorder; 11 participants (55.0\%) reported being diagnosed with depression, and 8 participants (40.0\%) reported being diagnosed with an anxiety disorder. However, a formal diagnosis of depression or anxiety was not required for participation.} We refer to these participants as FP1--FP20. 



%. For depression, 11 participants (55.0\%) responded "Yes", 8 participants (40.0\%) responded "No", and 1 participant (5.0\%) responded "I don't know". For anxiety, 8 participants (40.0\%) responded "Yes", while 12 participants (60.0\%) responded "No".}
%\topic{Participants were asked whether they had ever been diagnosed with a mental health disorder by a specialist or healthcare provider;
%11 participants (55.0\%) reported being diagnosed with depression, and 8 participants (40.0\%) reported being diagnosed with an anxiety disorder.}

% \topic{Additional details about participant characteristics are presented in Table \ref{tab:participant_characteristics}.}

% \begin{table}[h!]
% \caption{Participant Characteristics}
% \label{tab:participant_characteristics}
% \centering
% \begin{tabular}{|c|c|}\hline
% \textbf{Characteristic} & \textbf{Total Sample ($n = 20$)} \\ \hline
% History of Depression or Anxiety Diagnosis\# (\%) & X (X\%) \\\hline
% Previous history of psychotherapy \# (\%) & 13 (65\%) \\\hline
% History of psychiatric medication use \# (\%) & 10 (50\%) \\\hline
% Current psychiatric medication use \# (\%) & 3 (15\%) \\ \hline
% \end{tabular}


% \end{table}

%The study was advertised as a research trial of a text messaging prototype designed to support self-management of depression and anxiety symptoms. Individuals who showed at least moderate levels of depression or anxiety symptoms according to the Patient Health Questionnaire-9 (PHQ-9) \cite{kroenke2001phq} and the General Anxiety Disorder-7 (GAD-7) \cite{williams2014gad} (i.e., scores of 10 or higher) were invited to learn more about study activities by following a link presented alongside their results. Potential participants completed an additional screening survey and were eligible if they were located in the United States, were above 18 years old, and owned a mobile phone.

\subsection{Overview of the Text Messaging Program}

After undergoing the eligibility and consent processes, participants were enrolled in the Small Steps SMS program \cite{meyerhoff2024small} -- an automated, interactive program created by \citet{meyerhoff2024small} to build users' capability to self-manage depression and anxiety symptoms. \revision{The program sought to sustain engagement by offering a variety of ways for users to interact \cite{meyerhoff2024small}.} It offered daily interactive dialogues that taught users how to apply 11 evidence-based psychological strategies. The presentation of these strategies largely drew from works by \citet{kornfield2022involving} and \citet{meyerhoff2022system} 
and were grounded in theories including acceptance and commitment therapy \cite{hayes2006acceptance}, cognitive behavioral therapy \cite{willson2019cognitive}, and social rhythm therapy \cite{frank2022interpersonal}.
%The eight-week structure drew inspiration from the previous research by \citet{meyerhoff2022system}. 

The program provided active personalization, enabling users to select which psychological strategies they wanted to use over time. Within a subset of dialogues, users had the option to choose between different topics or exercises.
Figure \ref{fig:conversations} illustrates a subset of the message dialogues sent to users. Within each dialogue, users responded to sequential messages that used branching logic to tailor replies from the program. The program's daily dialogues varied over time in their approach.
Some of the messages included:
\begin{itemize}
    \item Introductions to specific psychological strategies and prompts to participate in skill-building exercises \cite{meyerhoff2022system},
    \item Stories of individuals who had successfully harnessed these strategies to navigate challenges \cite{bhattacharjee2022kind},
    \item Invitations to compose and receive supportive texts for peers \cite{meyerhoff2022system},
    \item Succinct tips or prompts for self-reflection, including messages that directed users to more extensive psychoeducational resources \cite{bhattacharjee2023investigating}, and 
    \item Prompts for users to think about or respond to open-ended questions \cite{bhattacharjee2023investigating} 
\end{itemize}
A more thorough description of the Small Steps SMS program can be found in \citet{meyerhoff2024small}, \topic{and additional diagrams illustrating the sequence of certain text messaging dialogues are provided in Appendix~\ref{sec: flow}.} \topic{The program spanned eight weeks, which aligns with typical durations for clinical DMH interventions and is comparable to the expected engagement period required for people to benefit from in-person therapy~\cite{lipschitz2022digital, mohr2019randomized}. 
% Additionally, there is strong evidence to suggest that an eight-week duration is acceptable for text messaging-based behavioral interventions. 
% This duration supports the notion that building new routines and incorporating new skills into daily life requires time, practice, and reinforcement. 
%Prior literature has demonstrated that this duration is suitable for people to build new routines and incorporate new skills into their lives, reaping similar benefits to what can be gained from in-person therapy
%\cite{agyapong2012supportive, agyapong2017randomized, ranney2018emergency}. 
% The Small Steps SMS program was specifically designed to provide opportunities not only for learning but also for implementing and reinforcing new habits. 
This duration reflects that building new routines and incorporating new skills into daily life requires time, practice, and reinforcement. The Small Steps SMS program was specifically designed to provide opportunities not only for learning but also for implementing and reinforcing new habits in order to achieve downstream effects on symptoms. However, it is important to recognize that not all DMH tools require such extended durations; for example, some are designed as single-session interventions lasting around 10 to 20 minutes, focusing on teaching a specific micro-skill or providing immediate support for stress management  \cite{bhattacharjee2024exploring, paredes2014poptherapy, schleider2020future}}.
%Nevertheless, it is important to recognize that not all DMH tools require such extended durations; many are designed as single-session interventions lasting around 10 to 20 minutes, focusing on teaching a specific micro-skill or providing immediate support for stress management \cite{bhattacharjee2024exploring, paredes2014poptherapy}.}


\begin{figure}
    \centering
    \includegraphics[width=1\linewidth]{Figures/conversations8}
    \caption{Examples of message dialogues seen by participants in the Small Steps SMS Program: (a) A dialogue showcasing specific psychological strategies and skill-building exercises, (b) illustrative accounts of individuals successfully employing strategies to overcome challenges, and (c) messages designed to prompt self-reflection.}
    \label{fig:conversations}
\end{figure}

%The program included both active and passive personalization. As far as active personalization, users could select which psychological strategies to start with, and which to exclude. They could also decide when they were ready to move on from a strategy. Within a subset of dialogues, users could also choose the topics of stories to receive or could select between activities. 

%As far as passive personalization, the system utilized reinforcement learning algorithms \cite{kaelbling1996reinforcement} to guide multiple decisions, such as whether to include or exclude components like weblinks and reminders, and when to send messages to align with users' availability and favored time slots. In particular, it employed contextual bandit algorithms \cite{figueroa2023ratings, figueroa2022daily}, basing decisions about what to send and when on users' past engagement with the system and ratings of daily dialogues, while also considering contextual factors such as day of the week, time on study, or symptom levels at sign up. 

%During the eight-week program, participants demonstrated a relatively high level of engagement. On average, they responded at least once on 70\% of the study days, with individual engagement rates spanning from 34\% to 100\%.
%While we acknowledge that variations in message content and algorithmic design could enhance participants' engagement, the substantial involvement from participants in the program allowed us to delve deeper into the broader social contextual factors influencing user engagement. This paper primarily concentrates on illuminating these influential social contexts.

\subsection{Data Collection and Procedures}

Participants were asked to complete two semi-structured interviews, one midway and another at the end of the eight-week program, conducted via Zoom by two research team members. These interviews were intended to gather insights into how participants' social contexts, personal responsibilities, and other circumstances evolved over the course of the study and how these contexts influenced participants' interaction with the text messaging program. Examples of interview questions included:


\begin{itemize}
\item Have there been any changes in your interaction with or perceptions of the program over the last month? Can you describe these changes?
\item Over the past month, have there been any life changes that have influenced your usage of the program? Can you describe how they affected your use of the texting program?
\item Engagement levels can fluctuate in programs like this. What are your thoughts on why this might occur?
\item Have you observed any changes in your daily activities, personal life, or emotional state since starting the program? What were they?
\end{itemize}

All 20 program participants were invited to interviews, with all but one completing at least one. Eighteen participated in the mid-program interviews at four weeks, and 17 in the final interviews at eight weeks. Interview durations ranged from 20 to 45 minutes, with participants compensated \$20 USD per hour.


\subsection{Data Analysis}
After transcribing the interviews, our team undertook a thematic analysis \cite{cooper2012apa} of the qualitative data. Two team members, termed ``coders,'' began by thoroughly reviewing all transcripts to acquaint themselves with the content. Following an open-coding process \cite{khandkar2009open}, each coder independently created a preliminary codebook. Subsequent meetings enabled the team to consolidate the codes into a unified codebook. \topic{During these discussions, the team refined code definitions, identified recurring codes, assessed each code’s relevance to the research questions, and eliminated those that did not directly pertain to the research questions at hand.} The coders then tested the shared codebook on a subset of the data consisting of eight interview transcripts. This iterative process allowed for further refinements to the codebook. Once a consensus was reached, each coder applied the finalized codebook to separate halves of the remaining data.

\subsection{Ethical Considerations}
\label{study1_ethics}
The research team was comprised of faculty members and graduate students from various fields including computer science, human-computer interaction, cognitive science, and clinical psychology. Given that  mental health research presents multiple ethical challenges, we took measures to address these at all stages of the research.

Participants were explicitly informed at the commencement of the study that the text messaging system was not a crisis intervention service. They were also given a list of emergency contacts, including crisis text lines and suicide helplines, in the possible event that such information would be useful. We did not solicit suicide-related information at any point in the study; however, given the open-ended nature of text messaging, we recognized the unlikely possibility that participants may disclose unprompted suicidal thoughts or behaviors. Hence, we took measures to ensure the safety of participants. We reviewed text messages from participants on a daily basis. We were prepared to reach out to any participants indicating a risk of suicidal ideation or self-harm and conduct the Columbia Suicide Risk Assessment protocol \cite{posner2008columbia}. However, no such risks emerged during the study, and no follow-up assessment was necessary. 

Similar considerations were applied to the interviews, where interviewees were informed they could skip questions or exit at any time. Interviewers were trained to conduct the Columbia Suicide Risk Assessment protocol as well, but again, no such measures were necessary. 