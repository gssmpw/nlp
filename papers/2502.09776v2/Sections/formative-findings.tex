\section{Findings from Study 1}
%It is worth noting that the actual engagement might be even greater, considering the possibility that participants might have chosen to read the messages without necessarily responding.
%However, participants cited several reasons for occasional non-engagement with the messages. While some of these reasons touched upon personal preferences, the majority were predominantly influenced by broader social factors, which we elaborate on in the subsequent sections.

\topic{Participants offered varied interpretations of what constitutes a DMH tool. They reported using tools like Woebot (FP11) and mood-tracking apps (FP20) as well as leveraging social media for peer support (FP17). Others (FP4, FP5, and FP9) mentioned exploring digital reflection techniques online, while FP10 highlighted using apps for weight management. Several participants (FP5, FP17, FP19) had also mentioned that they had familiarity with some of the psychological concepts and techniques introduced by the Small Steps SMS program.} %although few (e.g., ) had used messaging tools for mental health self-management before this study.}

Our interviews illuminated various factors that disrupted user engagement with the Small Steps SMS program. While some experiences of disengagement were linked to program-specific issues like habituation and individual preferences, the majority stemmed from the broader social contexts affecting participants. We first briefly present disengagement factors linked to habituation and individual preferences before delving into those influenced by social context.



\subsection{Disengagement Due to Habituation and Personal Preferences}

During the eight-week program, participants demonstrated a relatively high level of responsiveness to the messages sent by the system. On average, they responded at least once on 70\% of the study days, with individual daily responsiveness spanning from 34\% to 100\%. The level of participation also declined over the study, with average daily responsiveness falling from 93\% in the first week, to 47\% in the 8th week. Participants relayed that these trends partly reflected diminishing novelty, as well as occasional dissatisfaction regarding the content and frequency of the messages.

\topic{A prominent reason for gradually decreasing responsiveness to the messages was habituation. Over the course of the program, participants became accustomed to the repetitive nature of the daily text messages, rendering them increasingly predictable. This predictability sometimes reduced the novelty of the messages and engendered a sense of monotony.} As the routine of reading and responding to messages became more ingrained, participants found the task increasingly tedious, eroding both their motivation to engage with the program and their perception of its utility. Over time, the diminishing returns of the repetitive interactions led participants to question the program's usefulness, thereby lowering their overall responsiveness. FP10 expressed this sentiment by saying:


%The repetitiveness of the daily interactions could lead to a sense of predictability. This familiarity could make the task of reading and responding to messages tedious over time, sometimes diminishing individual motivation and the perceived utility of the program. 


\italquote{It’s like a daily thing that you constantly have to check up every day. Then, yeah, it’s like a chore. It’s something that I’m not really looking forward to eventually. It’s like `Oh, right. I have to do that.' I’ve kinda lost interest in it.}

FP2 emphasized that the frequency of the messages also contributed to their disengagement. Specifically, receiving messages at sporadic intervals throughout the day could become overwhelming rather than helpful. Elaborating on the experience of interacting with several messages, FP11 stated:

\italquote{I just feel like if I don’t accomplish at least reading it and responding back, I’ve failed at something. \ldots So, it can get a little bit overwhelming.}

To enhance user engagement, participants provided suggestions for additional features. For example, they recommended implementing features for journaling or documenting daily events. They also suggested a more organized messaging interface to separate and retrieve distinct dialogues for future reference.

%These findings indicate  factors such as habituation and message frequency could contribute to a decline in responsiveness


%Despite employing contextual bandit algorithms and collecting preliminary preferences from participants regarding content and timing, 
%occasional dissatisfaction with the content or timing of messages led to sporadic disengagement. For instance, FP6 considered gratitude messages to be too other-focused since they felt that they were already dedicating most of their time to supporting their child as a young parent. On the other hand, P336 occasionally found content on supporting others rudimentary based on their familiarity with social support literature and prior experience in assisting others. Other preferences regarding tone (formal vs. informal) and the style of guidance (action-oriented vs. indirect) of the messages arose as well. In cases where the content, tone, or guidance style of the messages did not align with an individual's preferences, they were more likely to disengage in the future.

%Despite specifying their preferred timings, some participants found the timing of messages occasionally inopportune. %Unforeseen work commitments or unexpected life events meant they could not always engage with the messages as intended. 
%FP2 highlighted that the frequency of messages could lead to disengagement too, pointing out that receiving messages sporadically throughout the day could be distracting. Additionally, for participants like FP7, receiving multiple messages in a single batch could be overwhelming and deter engagement.

%FP2 even suggested receiving all messages collectively, to avoid multiple interruptions throughout their day. However, individuals like FP7 preferred sporadic interruptions throughout the day.

%In summary, these findings underscore how misalignment with people's personal preferences could lead to disengagement. However, these issues may be attributable to broader social context of the participants, a topic we discuss further in the subsequent subsection.

%\todo{Does this need to be longer? I intentionally kept this short.}

\subsection{Broader Social Contexts Underlying Disengagement} 
%In contrast to participants whose engagement waned gradually due to perceived limitations of the tool, a number of participants described disengagement that was more abrupt, reflecting a range of life circumstances that could overwhelm them and demand their attention and mental energy. In these instances, disengagement often happened despite satisfaction with the tool.

Unlike instances where engagement decreased gradually due to perceived shortcomings of the tool, there were often times when participants abruptly disengaged due to life circumstances that demanded their full attention and energy. In these instances, disengagement often happened despite satisfaction with the tool. We refer to these circumstances as situational engagement disruptors (SEDs), \topic{which are summarized in Table \ref{tab:sed_summary}}. We describe them in detail below.

\begin{table}[h!]
\centering
\caption{\topic{Summary of SEDs identified from Study 1}}
\label{tab:sed_summary}
\begin{renv}
    
\begin{tabular}{|p{4cm}|p{9cm}|}
\hline
\textbf{Category}                     & \textbf{Examples of Disruptors}                                                                                     \\ \hline
Academic and Workplace Responsibilities & Academic coursework,  phone restrictions, deadlines, and unpredictable schedule                         \\ \hline
Family Responsibilities                & Childcare, household tasks, and financial concerns                                   \\ \hline
Other Disruptors                      & Illness, relocation, sleep disturbances, and grief                                       \\ \hline
\end{tabular}
\end{renv}

\end{table}


\subsubsection{Academic and Workplace Responsibilities}

Several participants shared their experiences dealing with academic stress and its impact on their ability to engage with text messages. For instance, FP16 expressed feeling overwhelmed by their coursework, stating that the initial enthusiasm they had when signing up for the program was overshadowed by mounting academic pressures. As the semester progressed, FP7 reflected on the increasing number of assignments and exams that demanded their full attention. Both FP7 and FP16 noted that they derived value from the program by gaining insights into new psychological theories and considering how to integrate these learnings into their daily lives. However, the demands of their course load and the stress associated with final exams subsequently made it challenging for them to continue dedicating time for such reflection.


Participants also extended their list of challenges preventing them from engaging with text messages to workplace rules and responsibilities. Reflecting on their work environment, FP8 shared the following:

\italquote{Unfortunately, I work in an environment where you're not allowed your phone. \ldots It's like a secure room, so any text I get during the workday tends to all come together. I read them, but it's hard to engage with all of them, like, all at once.}

\noindent
Several participants highlighted categories of occupations that may inherently conflict with the ability to respond to text messages promptly. FP14 pointed out truck drivers as an example, explaining that the nature of their job may often prevent them from safely and conveniently engaging with text messages while driving. The irregular and unpredictable timing of their breaks compounds this issue as it complicates anticipating when they will have the chance to interact with such messages.

Even individuals who do not face strict rules or difficulties regarding phone usage found it challenging to prioritize self-care during working hours, stemming from the perception that taking a moment for personal wellbeing might detract from their focus on work. For example, FP2 noted that when their phone buzzes repeatedly with notifications, they feel compelled to redirect their attention from work to the messages. Although they found the messages useful for learning how to cope with stress, they were concerned about continually switching back and forth between work and text messages. They explained: 
\italquote{That experience of, you’re focused at work and you hear your phone go off -- it sounds like you have a reaction like ``Okay, what is it?'' You have to completely redirect your attention.}

%In a similar vein, %FP4 expressed that one may not prioritize spending a moment on the messages because they have \textit{`nothing to do with my work.}' 

%Academic and workplace responsibilities not only limit the time available for self-care but also generate a prioritization conflict. Participants perceived a trade-off between engaging in self-care activities and focusing on the tasks at hand. 

%Participants relayed their experiences of facing challenges where they had to strike a balance between engaging in self-care activities and fulfilling their immediate responsibilities or tasks. 
The tensions between work tasks and personal wellbeing activities were highlighted further in FP4's experience. They indicated that they only felt compelled to promptly respond to messages that concerned urgent job responsibilities or impending work-related deadlines. Messages suggesting strategies to handle negative emotions did not hold the same sense of urgency or offer benefits that could surpass the importance of completing time-bound professional tasks.
%Text messages about learning to manage negative emotions seemed to lack the same sense of urgency and did not provide tangible benefits that outweighed completing academic or professional tasks with deadlines. 
Consequently, these messages were often relegated to a lower priority and could be easily forgotten or overlooked. 
% To summarize, balancing work commitments seemed to overshadow the importance of self-care activities, leading to personal health and wellbeing sometimes being underserved.


%To summarize, the pressing need to devote time and energy to meet work commitments could eclipse the perceived importance of self-care activities, thereby fostering a cycle of stress and neglect of personal health and wellbeing.
%The need to allocate time and effort towards income generation may overshadow the value and necessity of self-care practices, contributing to a cycle of overwhelm and neglect of personal wellbeing.

\subsubsection{Family Responsibilities}
A few participants expressed feelings of overwhelm regarding their family responsibilities, most notably women who felt responsible for managing household duties. FP2 and FP6 shared their experiences as young parents and the impact it has on their ability to engage with text messages. FP2 explained:

\italquote{I think that [the text messaging program] was really helpful for me. \ldots But I'm in school, I'm working, my oldest is in kindergarten, my youngest is in daycare, so I'm all over the place. And part of what I struggle with is being – I'm also 25. \ldots When I have that free time, I never use it as `me time'. It's clean up the kids' room, get my laundry done, make sure everything's good so that when they get back home, we've got a new space to tear apart again. \ldots And it's never me just taking a nap. \ldots If I'm home alone, I'm getting homework done. I'm getting chores around the house done. \ldots It goes along with being a parent. All of your time is dedicated to other people's needs and not so much your own.}

\noindent
Caring for children and other family members had a significant impact on people's ability to plan and set personal goals. The demands of managing household duties, childcare, and other familial obligations can create a sense of overwhelm and restrict people's capacity to allocate time and energy for reflection or self-care. Hence, FP6 emphasized that young parents would benefit from adding more structure to their lives, whether it be through creating routines or setting aside dedicated time for self-care activities. 
% Other participants echoed a desire for support in planning and incorporating self-care activities into their daily lives, recognizing the positive impact it can have on their overall well-being.

%These participants shared that with the responsibilities of taking care of their children as well as other family members, they tend to struggle with planning and setting goals for their life. They feel that they do not have much bandwidth to reflect on their life or take care of themselves. FP6, as a result, expressed that young parents like hers would most benefit from forming some sort of structure in their life and some support in planning activities around self-care. 

For some, anxiety and overwhelm centered on their families' financial stability. FP17 described ongoing worries and efforts around maintaining a balanced budget while meeting their family's basic needs, noting that such worries could easily lead people to deprioritize engaging with text messages. They commented:

\italquote{Even when I was busy, I would take the time \ldots People are busy, and a lot of people are just not gonna take the time to do it depending on how depressed the person is. They've got bills. They've got other things. The way things are right now, people are struggling just to make it. So, it’s gonna be a challenge. People are thinking, 'I’ve got a family to feed and if I don’t go do this, this, and this, I’m not gonna \ldots surmount any of the challenges'.}

\noindent
This preoccupation with financial challenges consumed their thoughts to such an extent that it became difficult for them to redirect their focus to other areas of life, namely self-care. Despite recognizing the potential benefits of the program's messages, participants sometimes found themselves unable to fully engage with them.


%expressed the belief that engaging in self-care activities would require time that they perceived would be better spent on finding ways to increase their income. %They reflected that some of the self-care activities might require some time, that they would think would be better spent in finding ways to increase income. 


\subsubsection{Other Disruptors}

Outside of academics, work, and family, participants highlighted other circumstances that disrupted their usual routines and hindered their engagement with text messages. These disruptions encompassed a range of unexpected events and situations, including brief illnesses, changes in residence, and the death of loved ones. Throughout the eight-week period, some participants reported experiencing health issues like COVID-19 and bronchitis that impacted their mood and energy levels, making it difficult to engage with the messages.
FP8 revealed that missing messages for a few days during their COVID-19 illness led them to disengage from the program altogether as the perceived difficulty in catching up with the missed messages seemed daunting. They stated:

\italquote{Getting a text is just actually really nice. \ldots When I got COVID and I was getting messages \ldots that kind of thing skews my engagement with it a little bit because I was too tired to even read my phone let alone actually engage with anything. And sometimes, it is annoying when you get several messages all at once.}

Meanwhile, participants like FP5 and FP8 shared how insomnia and changes in their sleep schedule disrupted their ability to interact with the messaging program. They relayed that sleep deprivation led them to a state of heightened busyness yet diminished presence in their daily activities. This lack of energy and focus negatively impacted their engagement with text messages, making comprehension and response more challenging. FP5 echoed a similar sentiment by underscoring the importance of establishing a regular sleep routine to better manage their interaction with the program.


%FP54 commented that even though they knew that the messages were well-intentioned, they felt that the messages interrupted their attempts to adjust their sleep schedule.

Additionally, participants highlighted the difficulties they faced when adapting to uncertain and unfamiliar housing environments due to new job opportunities. FP12 mentioned that they had to constantly move between two cities for work. The uncertainty surrounding their living arrangements consumed their thoughts and hindered their ability to engage with text messages. Participants also mentioned personal, unanticipated reasons for disengagement. FP4, for example, shared the experience of losing a close friend on the same day they enrolled in the program. The grieving process consumed their emotional energy, leading them to ignore the messages for the first few days.

