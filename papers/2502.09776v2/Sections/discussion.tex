\section{Discussion}

Our work offers evidence and actionable insights for the HCI and CSCW communities, emphasizing the essential need to consider the broader social contexts in which DMH tools operate. \topic{Although our observations originate from a text messaging tool used for self-managing depression and anxiety symptoms, the SEDs we identified likely have similar impact on the usage of other DMH tools. For instance, several prior studies have reported that social circumstances can pose barriers to people's engagement with DMH mobile apps and online programs \cite{borghouts2021barriers, bhattacharjee2024exploring, harjumaa2015user, bhattacharjee2023investigating}.} We enrich the existing research by exploring how social contexts affect interactions with DMH tools, highlighting the inherently sporadic nature of engagement. Our proposed solutions—community support, socio-technical interventions, and individual goal-setting—aim to redefine successful engagement by adapting to varying user involvement levels and addressing specific SEDs.

%We enrich the existing body of work by providing a nuanced understanding of the complex interplay between an individual's social contexts and their interaction with DMH tools. Recognizing that engagement can be inherently sporadic and influenced by external life events, we advocate for redefining what constitutes successful engagement in diverse social environments. This redefinition may involve accommodating varied interaction patterns and acknowledging different levels of user involvement, as well as helping individuals manage specific SEDs contributing to their disengagement. The solutions we propose, including community support, socio-technical interventions, and individual goal-setting, are designed to integrate deeply within users' social contexts, addressing the complex dynamics that affect user engagement.

In the following discussion, we first outline how our findings respond to our research questions and contribute to existing literature on engagement with DMH tools. We then discuss the limitations of our study.

\subsection{RQ1: Influence of SEDs in Dictating Engagement with DMH Tools}
% Our study underscores the intricate manner in which an individual's social context impacts their engagement with DMH tools. 
Prior literature on disengagement has emphasized individual correlates like motivation or symptom severity \cite{lipschitz2023engagement, doherty2018engagement, venkatesh2003user, marangunic2015technology, miller2016game, yoo2024missed} and the ways in which disengagement may reflect misalignment between a tool and one's preferences \cite{bhattacharjee2023investigating, jardine2023between, perski2017conceptualising}. 
\topic{Previous researchers have also commented on how circumstantial constraints like job-related issues and financial hardships frequently lead to termination of traditional in-person therapy \cite{renk2002reasons, roe2006clients, hynan1990client, carolan2018employees}.
Our work extends this literature by identifying contextual constraints from SEDs that instigate disengagement with DMH tools. We observed that these constraints significantly influence both subjective (e.g., perceived commitment and burden) and behavioral (e.g., interaction frequency, long-term adherence) measures of engagement. While the SEDs identified in our study emerged from one specific intervention, they reflect broader engagement challenges that transcend any given technology. These issues can arise in interactions with different types of DMH tools including mindfulness apps \cite{laurie2016making}, digital self-care platforms \cite{borghouts2021barriers}, or tools designed to manage symptoms of depression and anxiety \cite{jardine2023between, bhattacharjee2023investigating}}. We observed that SEDs can make it challenging for users to maintain consistent engagement with DMH tools even when they command as little effort as reading a text message. 
%These observations align with existing research that highlights the considerable influence of diverse socioeconomic factors on psychological wellbeing \cite{allen2014social, slavich2020social, tachtler2020supporting, guan2022financial, li2022suffered, bhattacharjee2023investigating, oguamanam2023intersectional, murnane2018personal, perski2017conceptualising}. 

\subsubsection{The Role of Feeling Overwhelmed in Disengagement}
Our investigation uncovered a handful of diverse SEDs that can influence engagement with DMH tools: academic challenges, workplace expectations, familial duties, and unanticipated life events. Although DMH tools are often theorized to overcome circumstantial constraints because of their convenience and accessibility, we demonstrate that SEDs remain significant impediments to sustained engagement. \topic{Echoing previous research \cite{borghouts2021barriers, carolan2018employees, bhattacharjee2023investigating}, we observed that consistent engagement faltered when individuals found it difficult to weave these digital interventions into their everyday routines.} When demands exceeded coping capacity, individuals often felt overwhelmed \cite{kabigting2019conceptual, hopps1995power}, leading to the deprioritization of self-care activities like DMH tools in favor of immediate or tangible tasks.

%A unifying theme across individuals' commentary was a pervasive sense of feeling overwhelmed. 
%When presented with urgent and competing responsibilities, participants often postponed or abandoned self-care activities in favor of focusing on tasks with immediate deadlines or clear outcomes \cite{wong2023mental}. 

\topic{Further complicating this situation were unexpected life events and disruptions, ranging from health issues and fluctuations in sleep patterns to relocations, all of which have been found to impact engagement with DMH tools \cite{borghouts2021barriers, karlgren2023sleep, lattie2020designing, perski2017conceptualising}. This cascade of challenges hindered people's ability to plan and set priorities, contributing to heightened distress.} Consequently, self-care activities could be perceived as additional burdens, even when participants acknowledged their value and expressed a genuine affinity for them.

%consistent use of the text messaging intervention suffered even when participants acknowledged its value and expressed a genuine affinity for it.  %exacerbating the feeling of being overwhelmed

\subsubsection{Handling Disengagement}
%\todo{fix -- based on David's comments} 
Once participants experienced voluntary or involuntary disengagement with a DMH tool, they frequently harbored the belief that they had irrevocably fallen behind. \topic{As prior work on mental health apps has also acknowledged \cite{six2021examining}, the notion of `falling behind' or `losing a streak' might evolve from a simple logistical concern into a formidable emotional barrier.} This perception weakened their resolve to continue with self-care efforts, initiating a cycle where re-engagement with the tool became increasingly difficult. The pressure to maintain a streak occasionally transformed into a reminder of their shortcomings, leading to self-disappointment \cite{bekk2022all, toda2017dark, hadi2022gamification}. \topic{These observations contribute to technology adoption models \cite{venkatesh2003user, dwivedi2019re} by underscoring how emotional barriers can shape users' motivation and perceived usefulness of DMH tools.} %To escape these negative emotions, some participants decided that discontinuing the use of the DMH tool was the best course of action.


%\topic{Our work enriches technology adoption models by highlighting the impact of overwhelm on user engagement, suggesting that tools that accommodate the natural fluctuations of user interaction without penalizing breaks can enhance long-term adoption.}
%It undermined their commitment to self-care initiatives, leading to a self-perpetuating cycle that made it even harder to re-engage with the program. Emphasis on maintaining a streak might become triggers for reminders that they are not able to complete it or self-recrimination, resulting in negative feelings of disappointment in oneself \cite{mirandanewpaper}. Eventually, discontinuation might become a reasonable solution to stopping those aversive feelings.
%This emotional toll ultimately might lead to total withdrawal for some participants.
%The revelation that these broader, overwhelming challenges often underlie individuals' disengagement suggests that DMH tools should factor in users' responsibilities, daily routines, and social connections \cite{borghouts2021barriers}.

The revelation that these broader challenges often underlie individuals' disengagement suggests that a more nuanced and comprehensive understanding of users' lives is imperative for crafting engagement strategies that genuinely support psychological wellbeing \cite{pendse2022treatment, allen2014social, slavich2020social, murnane2018personal, kaziunas2017caring}. \topic{Since engagement can often be sporadic, DMH tools should accommodate varying levels and types of interaction without penalizing users or creating additional stress. Taking inspiration from slow and temporal design philosophies \cite{grosse2013slow, pschetz2018temporal}, DMH tools should prioritize thoughtful, user-paced interactions to help users achieve long-term goals over time. They could include flexible goal-setting activities to integrate self-care into users’ daily lives or targeted support for managing the underlying causes of overwhelm; we elaborate upon some potential design solutions in Section~\ref{subsec: RQ2}. DMH tools could also offer customizable schedules that adapt to users’ unique routines, especially during periods of overwhelm. At the same time, it is important to avoid introducing unpredictability that could exacerbate overwhelm, as expressed by participants like FP11. DMH tools should also ensure that flexibility and adaptability are transparently communicated and predictable in their operation, so users feel supported rather than disrupted.} 
%By reducing the pressure of meeting fixed expectations, such designs can better accommodate the diverse and dynamic capacities of users, ensuring the tools remain supportive and acceptable.}

%Our findings further advocate that Tools that adapt to natural fluctuations in user interaction and allow for breaks without penalties may foster better long-term adoption. This perspective also aligns with temporal design philosophy, which critiques externally imposed time structures and instead advocates for systems that respect users’ personal rhythms and accommodate their dynamic, context-specific needs. In general, we argue that DMH tools can shift from enforcing system-defined schedules to offering flexible interactions around SEDs that support psychological wellbeing in a sustainable way.

%This reconsideration should account for the varied capacities of individuals to engage, especially when they are overwhelmed. \topic{Our findings advocate revising technology adoption models by suggesting that DMH tools should be adaptable, recognizing that interactions may not always be consistent or frequent. This critique aligns with temporal design philosophy, which challenges system-driven timing and advocates for approaches that respect the relational and contextual nature of time in users’ lives. Temporal design emphasizes that tools should align with users' personal rhythms and accommodate fluctuating demands, offering flexibility rather than enforcing rigid, system-imposed schedules.}}

 %This perspective urges the development of DMH tools that are not only responsive to the fluctuating capabilities of their users but also redefine success in engagement to include diverse patterns of use. This perspective underscores the need to redefine success in DMH tools, moving away from rigid metrics such as streaks and consistent usage to more inclusive patterns of use that reflect users' changing circumstances. Drawing on slow design principles, DMH tools could encourage thoughtful, user-paced interaction rather than pressuring users to meet system-driven timelines. For instance, tools could prioritize reflective prompts or user-customizable interaction schedules, fostering meaningful engagement that fits seamlessly into users’ lives.

%In this context, DMH tools can take motivation from the \textit{slow design} philosophy in many ways. Participants reported feeling overwhelmed by the sporadic frequency of text messages and the associated fear of failure. This experience underscores how DMH tools often enforce rigid or unpredictable temporalities, prioritizing system-driven timing over user-centered rhythms. Temporal Design critiques such externally imposed time structures, advocating for approaches that account for the relational and contextual nature of time in users' lives.  

%While self-management of psychological wellbeing could be perceived as an individual responsibility \cite{lorig2003self}, more recent studies have highlighted the limitations of such individualistic perspectives and advocate instead for approaches that acknowledge one's social context \cite{sallis2015ecological, murnane2018personal, borghouts2021barriers}. In alignment with such an approach, our findings suggest that DMH tools should factor in users' responsibilities, life aspirations, daily routines, and social connections. 

%These disruptors indicate that work that focuses on tweaking content or tailoring algorithms or otherwise making superficial adjustments to a tool is unlikely to succeed in achieving engagement for populations facing mental health challenges, especially when considered over weeks or months. Rather, Our findings highlighted the importance of embracing a comprehensive understanding of users' lives when offering strategies to bolster psychological wellbeing \cite{pendse2022treatment, allen2014social, slavich2020social, murnane2018personal}. While self-management of psychological wellbeing could be perceived as an individual responsibility \cite{lorig2003self}, more recent studies have highlighted the limitations of such individualistic perspectives \cite{sallis2015ecological, murnane2018personal}, advocating for acknowledging the context in which an individual lives and is surrounded by.
%In the context of our findings, this implies that DMH tools should factor in users' myriad responsibilities, life aspirations, daily routines, and social connections. A salient reason for disengagement, as indicated by our data, is users' struggle to align their engagement with DMH tools within the framework of their overarching goals and duties. In instances where such alignment falters, individuals often sideline their psychological wellbeing. This observation echoes earlier research advocating that mental health initiatives should resonate with the sociocultural milieu and life circumstances of individuals \cite{slavich2020social, slavich2023social, allen2014social, glanz2008health, tachtler2021unaccompanied, oguamanam2023intersectional, murnane2018personal}.

%Central to our observations was the pervasive sense of being overwhelmed by various SEDs, including academic challenges, workplace expectations, familial duties, and unanticipated life events like health setbacks or relocations. While the specific circumstances and disruptors varied among individuals, there were commonalities across their experiences related to the feeling of overwhelm. This state often resulted in difficulty planning and prioritizing beyond immediate needs, along with increased distress about their broader situation. Such feelings led to failure to routine usage of the tool, even when participants expressed a genuine affinity for the platform and saw its potential benefits. Therefore, work that focuses on tweaking content, tailoring algorithms, or otherwise making superficial adjustments to a tool is unlikely to succeed in achieving engagement for populations facing mental health challenges, especially when considered over weeks or months. 


%Although individuals' circumstances and specific disruptors varied widely, there were commonalities across their experiences as relating to the feeling of overwhelm, which included a loss of ability to plan and prioritize beyond meeting their most pressing obligations, as well as distress about their situation. Naturally, this typically resulted in failure to routinely use of a mental health tool, about which many participants felt some regret. These experiences of disruption occurred despite the users' best intentions. Indeed, participants reported generally liking the text messaging tool and the content it delivers, and thought it could be helpful to them. Thus, work that focuses on tweaking content or tailoring algorithms or otherwise making superficial adjustments to a tool is unlikely to succeed in achieving engagement for populations facing mental health challenges, especially when considered over weeks or months.

%The intense sense of overwhelm stemming from academic and professional commitments may shed light on why participants in previous studies noted a preference for minimal to no support during work hours, even though there might be intermittent needs for respite and support \cite{bhattacharjee2023investigating, bhattacharjee2022design, howe2022design, wong2023mental}. Such pervasive stress can often relegate mental wellbeing initiatives to the sidelines, notably when the benefits of self-care practices remain intangible in the short term. In the face of a cascade of urgent responsibilities, participants reported frequently deferring self-care initiatives, finding themselves overshadowed by tasks with immediate deadlines and tangible rewards \cite{wong2023mental}. Compounding this challenge is the difficulty in maintaining a consistent healthy lifestyle. Factors such as shifts in dietary habits or disruptions in sleep patterns could further erode an individual's capacity to manage time efficiently \cite{gipson2019effects, karlgren2023sleep, lattie2020designing}. Such disturbances could adversely impact task performance and diminish the attention devoted to mental health practices.

%Family obligations and financial concerns were also found to be important SEDs that can pervade every facet of an individual's life.  Prior research in the mental health domain has demonstrated the profound implications of several broad socio-economic factors, revealing their potential to exert adverse effects on mental wellbeing \cite{allen2014social, slavich2020social, tachtler2020supporting, guan2022financial, li2022suffered}. Our findings resonate with this narrative, emphasizing that these factors not only influence mental health but also play a pivotal role in determining the extent to which individuals engage with DMH tools. For instance, the demands of child-rearing can be a dominant source of stress, often leaving scarce opportunities for individuals to prioritize their self-care. Analogously, the weight of managing financial intricacies can exact a similar toll. These overarching social factors underscore the intertwined nature of life's responsibilities and their resultant influence on psychological wellbeing and engagement with DMH tools.

\subsection{RQ2: Designing and Implementing DMH Tools to Account for the Social Contexts}
\label{subsec: RQ2}
%Our findings underscored the importance of adopting a more holistic view of users' lives in the realm of DMH tool design, especially when it pertains to engagement. %Current tools like ours, often with a narrow focus on specific metrics or behaviors, may neglect the broader, interconnected aspects of users' daily lives, leading to decreased engagement and potential disconnection. Participants frequently voiced the sentiment that such tools, while well-intentioned, sometimes felt misaligned with their intricate realities, spanning personal, social, and professional spheres. This misalignment can result in tools feeling more burdensome than beneficial, prompting sporadic or short-lived engagement. To bolster sustained and meaningful user interaction, there is a pronounced need for tools that genuinely resonate with the multifaceted nature of users' experiences. Embracing a more holistic understanding of users' responsibilities, priorities, life goals, and surroundings, hence, will ensure that DMH tools are not just interacted with, but integrated into, the diverse tapestry of users' lives.

We recognize the complexity involved in proposing design solutions that respond to diverse SEDs. Even though participants across both studies came from varied life circumstances, they shared common experiences of overwhelm and disengagement and largely found similar types of solutions to be valuable. \topic{As summarized in Table \ref{tab:socio_technical_solutions}, these suggestions were often socio-technical rather than purely technological, extending frameworks like Social Safety Theory \cite{slavich2020social} by demonstrating how DMH tools can alleviate perceived social threats. In doing so, they open up new discussions on how DMH tools can foster a sense of social safety within users' daily interactions. We elaborate on these ideas below.} 

    

\begin{table}[ht]

\caption{\topic{Socio-technical solutions based on our findings}}
\label{tab:socio_technical_solutions}
\centering
\begin{renv}

\begin{tabular}{|p{3cm}|p{3cm}|p{7cm}|}
\hline
\textbf{Theme} & \textbf{Action} & \textbf{Description} \\ \hline
\multirow[t]{3}{3cm}{Prioritizing Self-Care Through Structured Goal-Setting} 
& \raggedright Balancing Self-Care and Responsibilities & Encourage users to prioritize self-care alongside personal and professional responsibilities by offering structured prompts during quieter periods, such as weekends or evenings, to help reduce perceived overwhelm. \\ \cline{2-3}
& \raggedright Decomposition of Tasks & Simplify complex goals by breaking them into smaller, actionable steps, which makes achieving tasks more manageable and less daunting for users. \\ \cline{2-3}
& \raggedright Facilitating Long-Term Planning & Encourage the development of plans that align with users' broader life goals, supported by interactive features that aid in strategic planning. \\ \cline{1-3}

\multirow[t]{3}{3cm}{Designing for Flexible Engagement} 
& \raggedright Adaptive Engagement Strategies & Design tools to accommodate varying levels of user engagement, offering simple, low-effort options during high-stress periods and more intensive, high-effort activities during times of greater user availability. \\ \cline{2-3}
& \raggedright Establishing Foundational Support & Foster a consistent relationship between users and DMH tools by providing unintrusive reminders during low-engagement periods and allowing flexible, pressure-free re-engagement opportunities. \\ \cline{2-3}
& \raggedright Reframing Disengagement & Normalize occasional disengagement by emphasizing the role of DMH tools as a natural part of the user journey, thereby reducing feelings of guilt or failure. \\ \cline{1-3}

\multirow[t]{4}{3cm}{Addressing SEDs Within Social Contexts} 
& \raggedright Targeted Socio-Economic Support & Integrate pathways to practical resources, including financial counseling, housing assistance, and grants for vulnerable populations. \\ \cline{2-3}
& \raggedright Peer and Community Integration & Enable tools to foster a sense of belonging by sharing relatable narratives and connecting users with peer or community-based support systems. \\ \cline{2-3}
& \raggedright Multi-Stakeholder Collaboration & Collaborate with financial technology companies, local banks, and other institutions to provide accessible workshops or tools addressing life complexities. \\ \hline
\end{tabular}
\end{renv}

\end{table}


% These technologies will only effectively support well-being if they assist users in managing their various social roles and responsibilities as they go about their daily lives.
%As described below, they involve helping users manage their various social roles and responsibilities as they go about their daily lives.
% helping users set and carry out structured goals, changing how they think about engagement and diversifying the ways they can use a tool during periods of overwhelm, and connecting users to additional networks of support.

\subsubsection{Prioritizing Self-Care Through Structured Goal-Setting}

In line with previous studies \cite{bhattacharjee2022kind, brown2014health, chaudhry2022formative, jardine2023between, epstein2015lived}, our findings underscore the difficulty users encounter when trying to weave self-care activities seamlessly into their daily lives. Conflicts arise when people's presumed responsibilities interfere with the timing and perceived importance of these activities, calling for tools that assist people in prioritizing their myriad responsibilities. Our findings indicate that structured goal setting, often a standard objective of many therapeutic interventions \cite{chand2023cognitive, agapie2022longitudinal, epstein2015lived}, holds significant importance in general wellbeing management when engaging with DMH tools. It becomes imperative for these tools to consider the broader goals of the users, thereby ensuring that the activities proposed are in harmony with, and not contradictory to, those goals. 

\topic{These findings highlight the critical role of DMH tools could play in counteracting the tendency people have to sideline self-care during busy periods.}
%Participants in our studies proposed multiple strategies to counteract the habit of consistently sidelining self-care during busy periods. 
To prevent academic or professional obligations from continually overshadowing self-care activities, DMH tools could assist users in setting aside designated times for self-care while emphasizing how self-care activities directly contribute to personal objectives. These personalized goals could correspond with their broader life objectives, highlighting the real-world implications of self-care practices \cite{jennings2018personalized, lindhiem2016meta, agapie2022longitudinal, epstein2015lived}. Incorporating visual analytics tools as suggested in health and wellbeing visualization studies \cite{baumer2014reviewing, kocielnik2018reflection} could provide users with graphical representations of improvements to their wellbeing, linking them to enhanced performance in other life areas. Further reinforcement can come from integrating educational modules that elucidate the scientific connections between wellbeing practices, work performance, and overall life improvement \cite{howe2022design, bhattacharjee2023understanding}. %\topic{By providing structured external prompts, DMH tools can help users integrate self-care activities into their daily routines, which might otherwise be deprioritized under stress. This form of technological mediation would allow self-care practices to coexist with other responsibilities, ensuring that these essential activities receive the attention they deserve.}

In line with literature on collaborative goal-setting \cite{agapie2022longitudinal, lee2021sticky, zakaria2019stressmon, jung2023enjoy, xu2023technology}, DMH tools can also serve as a collaborative partner in sorting through overwhelming obligations. For example, DMH tools could lead users through guided conversations that help them decompose broad objectives into distinct, actionable steps \cite{o2018suddenly, bowman2022pervasive, borghouts2021barriers}. Tools could also help users identify which responsibilities are essential, which could be considered deferred, and which could be ignored altogether. Such dialogues would help users allocate their time wisely and concentrate on high-priority activities \cite{pereira2021struggling, grover2020design}.
Recognizing that these guided interactions might require dedicated time and thoughtful reflection, participants in our study suggested that users should be encouraged to complete these brief exercises during their idle time \cite{poole2013hci, bhattacharjee2022design}. Idle time could include moments when individuals are engaged in low-attention activities, such as checking messages or browsing social networking platforms. Once a plan has been crafted, DMH tools could then offer occasional reminders and reflection prompts, helping users stay on track with the incremental steps needed to achieve their overall goals. \topic{This form of technological mediation would allow self-care practices to coexist with other responsibilities, ensuring that these essential activities receive the attention they deserve.}

\topic{However, the potential for collaboration between users and DMH tools can be disrupted by a multitude of factors \cite{flathmann2024empirically, caldwell2022agile}. Power imbalances may emerge when the tool imposes too much control over the user’s decision-making process, potentially diminishing their sense of personal agency \cite{flathmann2024empirically, bhattacharjee2023informing}. Furthermore, trust issues can severely disrupt collaboration, whether they stem from a sudden recommendation that undermines the tool's reliability or from concerns about the security of personal data \cite{caldwell2022agile}. Future studies should explore the landscape of human-DMH tool collaboration to identify these disruptors as well as strategies to mitigate them.}

%Additionally, participants suggested solutions to counteract patterns of repeatedly deferring and deprioritzing self-care in busy periods. So that immediate academic or professional tasks do not always take precedence over self-care, they thought DMH tools should help users block off dedicated time for self-care. In addition, systems could highlight how self-care strategies align with individuals' personalized goals. 


%However, merely providing guided interactions may not be adequate, particularly when confronting multifaceted challenges and users are uncertain where to begin. Participants of our study indicated that there could be times they were not aware of existing resources or may not remember them. In such cases, a supplementary approach could include sharing narratives of individuals navigating similar challenges or offering access to relevant community resources that can offer additional insight and guidance \cite{bhattacharjee2022kind, thaker2022exploring, baumel2018digital}.

%\todo{user may not know the importance, may need to be reminded}

%We also observed that participants indicated a need for scheduling such activities that would not clash with those responsibilities. 

\subsubsection{Designing for Flexible Engagement}

%Our participants expressed a desire for greater flexibility when interacting with DMH tools. They noted encountering many DMH tools that have employed a somewhat rigid framing of engagement as a key metric of success (e.g., in gamified systems that collect and reward ``streaks''). Participants relayed wanting a system to allow for new conceptualizations of engagement, recognizing it as naturally disrupted at times, often by forces that feel outside the user's control. Such a re-framing is essential in part because a sense of demoralization over disengagement can feed a vicious cycle where users become less and less likely to engage over time. To re-frame engagement, participants thought systems should allow for setting breaks (without compromising streaks), normalize disruptions, and find ways to pull disengaged users back in. Given how frequent disruptions are, identifying appropriate interaction types and schedules to re-engage users represents a high priority area. Yet, this design space is also rife with challenges, with our findings suggesting that participants are divided when considering how often a system should reach out to a disengaged user, since efforts to re-engage users could risk further demoralizing or annoying them. One strategy that could be explored is for systems to explicitly recognize the action of coming back after a hiatus or a major SED as a significant accomplishment and demonstration of commitment.

%\todo{cite and improve}
%Many current DMH tools prioritize sustained interaction, often through gamified metrics like ``streaks''. Although the core intent of DMH tools is to foster wellbeing, the overwhelming emphasis on sustained engagement can inadvertently defeat this purpose. Our research underscores that a relentless push for streaks and frequent notifications can introduce unnecessary stress, echoing commentaries against over-gamification that prioritize these targets \cite{etkin2016hidden, bekk2022all, liu2017toward, jia2016personality, hamari2014does}. 
Participants in our study expressed a strong desire for a reframing of what constitutes successful engagement with DMH tools. Although the core intent of DMH tools is to foster wellbeing, an emphasis on sustained engagement can inadvertently defeat this purpose \revision{even when such engagement is necessary.} Our research underscores that a relentless push for streaks and frequent notifications can introduce unnecessary stress, echoing commentaries against over-gamification \cite{etkin2016hidden, bekk2022all, liu2017toward, jia2016personality, hamari2014does, epstein2015lived}. %The daunting challenge of maintaining long streaks can discourage users from taking even incremental steps toward wellbeing, especially when disengagement occurs due to circumstances beyond their control. 
To address this, DMH tools can consider offering options for taking short breaks without jeopardizing accumulated progress. Additionally, these tools should normalize brief periods of disengagement and formulate strategies to encourage users to return and continue their wellbeing journey.

Nevertheless, disruptions are inevitable, so it is equally important to identify appropriate cadences for re-engaging users. Our findings suggested that participants are divided when considering how often a system should reach out to a disengaged user, as efforts to re-engage users could risk further demoralizing or annoying them \cite{bhattacharjee2022kind, epstein2015lived}. A promising strategy could be to frame acts of re-engagement after a hiatus as a meaningful achievement. Rather than dwelling on periods of inactivity, DMH tools should thus focus on fostering a foundational relationship with users, utilizing unintrusive reminders during periods of low engagement as a way to encourage continued interaction~\cite{bhattacharjee2023integrating, yardley2015person}. \topic{This foundational relationship can play a critical role in helping users maintain a consistent practice of self-care. When this relationship is absent, individuals may be forced to entirely rely on their own discipline and time management abilities;  if overwhelmed, the lack of external support can lead to sporadic or neglected self-care. Therefore, it is vital for DMH tools to regularly foster this relationship, allowing users to flexibly engage without undue pressure.}

\topic{We acknowledge the challenges in maintaining this relationship as integrating personal, academic, or work schedules with supportive resources in DMH tools may shift them from lightweight, low-burden systems to potentially more complex, high-burden ones.} %Lightweight systems offer quick, accessible support, potentially helpful for high-stress or time-limited situations \cite{bhattacharjee2023investigating}, though their simplicity may limit deeper, sustained benefits. Conversely, high-burden systems, with features like proactive notifications, reminders, and detailed tracking, may enable holistic, long-term support \cite{agapie2022longitudinal} but can overwhelm users by demanding intensive engagement.}
%\topic{We acknowledge that establishing this conection might be difficult as integrating personal academic or work schedules and actively pushing supportive resources into DMH tools may shift them from lightweight, low-burden systems to potentially more complex, high-burden ones that may demand greater cognitive effort. Lightweight systems offer quick, accessible support with minimal interaction, which could be helpful for users in high-stress or time-limited situations \cite{bhattacharjee2023investigating}, though their simplicity may limit deeper, sustained benefits. In contrast, high-burden systems—with features like proactive notifications, reminders, and detailed tracking—may enable holistic, long-term support \cite{agapie2022longitudinal} but can overwhelm users by demanding intensive engagement.} 
\topic{Approaches like causal pathway diagramming \cite{klasnja2024getting} can help identify mechanisms to achieve a balance between these extremes. DMH tools could dynamically adjust their functionality based on user context, factoring in preconditions such as user availability and moderators like their energy levels. During high-stress periods or busy workweeks, the system could offer lightweight support that does not require a response or involves only quick and simple tasks \cite{bhattacharjee2023investigating}. These low-effort interactions can aim to provide immediate support, potentially facilitating proximal outcomes like stress relief or a temporary mood boost. Conversely, when users have more energy or availability (e.g., idle time \cite{poole2013hci, bhattacharjee2022design}), the tool should encourage intensive activities like goal-setting or reflection to achieve distal outcomes such as overcoming life stressors or reaching professional goals \cite{kornfield2020energy}. Such an adaptive model should ensure all interactions, regardless of effort level, align with users' broader life objectives.}

%To re-frame engagement, participants thought systems should allow for setting breaks (without compromising streaks), normalize disruptions, and find ways to pull disengaged users back in. Given how frequent disruptions are, identifying appropriate interaction types and schedules to re-engage users also represents a high-priority area. Yet, this design space is also rife with challenges, with our findings suggesting that participants are divided when considering how often a system should reach out to a disengaged user, since efforts to re-engage users could risk further demoralizing or annoying them. One strategy that could be explored is for systems to explicitly recognize the action of coming back after a hiatus or a major SED as a significant accomplishment and demonstration of commitment.  A balanced approach that tolerates brief periods of disengagement and values both active and passive interactions should be considered. Instead of emphasizing lapses in use, DMH tools could focus on building a foundational connection with the user, employing gentle nudges during periods of low engagement \cite{bhattacharjee2023integrating}.

%This desire for flexibility was evident not only in participants' suggestions for alternative framing of engagement but also in their interest in a variety of interaction modes.
%This need for flexibility was visible in the suggestions of alternative framings of disengagement as well as looking for diverse modes of interaction. 
%The core intent of DMH tools is to foster wellbeing, yet the overwhelming emphasis on sustained engagement can inadvertently defeat this purpose. Our research underscores that a relentless push for streaks and frequent notifications can introduce unnecessary stress, echoing commentaries against over-gamification that prioritize these targets \cite{etkin2016hidden, bekk2022all, liu2017toward, jia2016personality, hamari2014does}. The daunting challenge of maintaining long streaks can lead to perceived setbacks that discourage users from taking even incremental steps to improve their wellbeing. A balanced approach that tolerates brief periods of disengagement and values both active and passive interactions should be considered. Instead of emphasizing lapses in use, DMH tools could focus on building a foundational connection with the user, employing gentle nudges during periods of low engagement \cite{bhattacharjee2023integrating}.

%Flexible engagement strategies can extend beyond the timing of interventions~\cite{burke2010social, verduyn2021impact, haidet2004controlled}. Relying on a single mode like text messages may not be suitable for all circumstances, as people may be more willing to express themselves through alternative means of interaction. For example, during periods of heightened activity or stress, DMH tools can require minimal effort by letting users passively read messages or acknowledge a message with a single tap. %\todo{Will cut this if we remove 6.5 -- Some individuals may be more comfortable communicating verbally when reflecting on stressful situations since it would allow them to convey their thoughts at a quicker rate than in text \cite{murray1994emotional, stigall2022towards}. Allowing users to respond with voice recordings might also provide an accessible option when their hands are occupied, whether they are driving or carrying bags. People may also lack the concentration required to read lengthy texts when focused on other tasks \cite{bhattacharjee2022kind}, but images or animations could convey essential information in a more compelling manner \cite{bryant2003imaginal, howe2022design}.} 
%By incorporating flexibility to engage through diverse modes of communication, DMH tools can enhance accessibility and responsiveness to individual needs.

%Engagement is often key to aiding users in managing their wellbeing. However, many DMH tools appear overly preoccupied with sustaining user engagement. They frequently inundate users with notifications and emphasize maintaining streaks. While such features can be helpful in keeping users aligned with DMH tools, our study highlighted potential drawbacks. For some participants, the pursuit of maintaining streaks could inadvertently morph into a stressor. These streaks, perceived as lofty or unattainable, sometimes deter users from even initiating small, progressive steps. Conversely, an occasional lapse in engagement could leave users with the disheartening feeling of starting from scratch, undermining the very purpose of the tool. These findings extend the prior literature on quantifying and gamifying engagement \cite{etkin2016hidden, bekk2022all, liu2017toward, jia2016personality, hamari2014does}, by highlighting the potential negative impacts of streaks and rewards. As our participants reminded us, the goal of the DMH tools should instead be allowing users to manage their wellbeing, which might sometimes be possible through passive engagement by briefly skimming messages, or even allowing for occasional disengagement. In periods of low or no engagement, users should not be reminded about their failure to engage, rather they can be requested to have a baseline connection with DMH tools \cite{bhattacharjee2023integrating}.
%Thus, an overemphasis on maintaining unbroken streaks might sometimes be counterproductive.

%\subsubsection{Multi-Level Interventions for Psychological Wellbeing Support}
\subsubsection{Addressing SEDs Within Social Contexts}

Participants also pointed to the need for DMH tools that help them directly manage the specific SEDs underlying their overwhelm and disengagement. \topic{They described that tools should assess the specific SEDs individuals were facing and, where possible, match them to solutions \cite{ma2023contextbot, jardine2023between, zakaria2019stressmon, fang2022matching}.}  For example, targeted digital interventions could address challenges in carrying out family obligations by providing access to virtual family counseling and online parental education to help individuals navigate the complexities of familial responsibilities \cite{chi2015systematic, smout2023enabling}. To better manage finances, digital financial literacy workshops and access to personal finance management tools might empower individuals to take control of their financial situations, mitigating anxiety and potential mental health deterioration \cite{boyd2016earned, mehra2018prayana}. Collaborative endeavors with financial technology companies or local banks could facilitate these online workshops, providing a practical approach to managing financial complexities \cite{dillahunt2022village, moulder2014hci}. Moreover, online community support groups could provide digital resources for childcare, eldercare, or other family-related needs in order to alleviate some of the burden on those balancing both family and work responsibilities \cite{gisore2012community, chaudhry2022formative, butler2012relationship, compton2015social, smout2023enabling}. Thus, past works emphasize the necessity for multi-level interventions that extend across and address various life dimensions and societal factors \cite{xu2023technology, sallis2015ecological, scholmerich2016translating, bauer2022community, murnane2018personal}. 

%However, as others have noted \cite{murnane2018personal, allen2014social}, targeting individual behavior alone is unlikely to address the root causes of many of the SEDs participants face. Thus, past work emphasizes the necessity for multi-level interventions that extend across and address various life dimensions and societal factors \cite{sallis2015ecological, scholmerich2016translating, bauer2022community, murnane2018personal}. 
%Addressing SEDs may also require policy-level changes that recognize the interconnectedness between mental wellbeing and conditions within work and educational settings \cite{murnane2018personal}.
% which recommends that effective interventions should be integrated with the actual settings in which people encounter them. 

Previous research in implementation science underscores the importance of the processes through which technologies are disseminated and delivered to users for effective psychological wellbeing support \cite{graham2020implementation, hermes2019measuring, powell2012compilation, powell2015refined}. Whether in educational environments or workplaces, numerous opportunities exist for enhancing the integration of psychological wellbeing support within existing technological frameworks \cite{graham2020implementation, haldar2022collaboration, xu2023technology}. Interventions can be introduced at opportune moments using digital platforms such as email and calendars \cite{howe2022design}. %The involvement of peer relationships in mental wellbeing initiatives can also offer an essential layer of support \cite{bhattacharjee2022kind, burgess2019think}. 
Furthermore, \citet{graham2020implementation} recommend strategies like providing access to educational materials to help users learn to use a tool and designated contact persons for assistance. It should be noted that support does not necessarily have to emanate solely from mental health professionals; trained peers and paraprofessionals can periodically extend their assistance, thereby fostering a sense of accountability and support \cite{sultana2019parar, anvari2022behavioral, okolo2021cannot, burgess2019think, jardine2023between, haldar2022collaboration}.


%\todo{improve}
%\topic{However, designing interventions that are technically feasible, socioculturally acceptable, and supportive of diverse populations involves understanding and managing the expectations and capabilities of a varied user base. The complexity increases as these tools are tailored to fit not only individual needs but also the specific social and professional environments in which they operate. This dual approach—balancing the deployment of lightweight, immediate support tools with more comprehensive systems designed for long-term outcomes—creates a spectrum of use that can cater to both urgent and ongoing support needs. By adopting a modular approach, where users can engage with the tool at different levels of complexity based on their immediate needs and environmental context, DMH tools can evolve from being mere support mechanisms to integral components of educational and professional wellbeing strategies}


 %We illustrate a simplified model in Figure~\ref{fig:cpd}, wherein the user's availability is considered a precondition and their stress level is considered a moderator; however, it is important to note that this proposed model requires further validation in future work.}


% \begin{figure}
%     \centering
%     \includegraphics[width=0.9\linewidth]{Figures/CPD.drawio.png}
%     \caption{An example causal pathway diagram that shows how a DMH tool might provide support for long-term engagement, considering the user's availability as a precondition and their stress level as a moderator.}
%     \label{fig:cpd}
% \end{figure}
%In the educational and professional spheres, there exist significant opportunities for organizations to weave mental health support into their existing technological infrastructure. Such integration should acknowledge that mental wellbeing is not isolated from other facets of life but is intimately linked with work and study environments. Policy-level changes, as suggested by \citet{murnane2018personal}, could profoundly influence psychological well-being, either enhancing or deteriorating mental health support within organizations. Building on the work of \citet{howe2022design}, interventions can be delivered at opportune moments identified through email interactions, digital calendar usage, or physiological data monitored in office settings. The inclusion of close relations and peers in an individual's mental wellbeing initiatives can add another layer of support \cite{bhattacharjee2022kind, burgess2019think}. 

%Participants in our study pointed out that intense periods of stress and overwhelm could be mitigated collectively with colleagues and peers, who support one another. DMH tools can further facilitate these interactions by enabling group support and shared narratives. This communal approach can not only validate emotions but foster a sense of belonging, encouraging individuals to face their struggles with a sense of solidarity and shared experience. 

%Social translucence 
%https://dl.acm.org/doi/fullHtml/10.1145/2775388 

%Sign-up Survey

%audio-video what people are used to do

%personal holistic, personalized, individual personalization, maybe subgroup

%\todo{social contexts become disruptions -- check otter}

\subsection{Limitations and Future Work}
Our work has a few limitations. First, our research focused on the experiences of individuals residing in North America. A more diverse participant pool from various geographical and cultural backgrounds might have revealed additional ones. 
\topic{For instance, norms around workplace and educational practices vary significantly across cultures \cite{southard2011vacation, parekh2022cross}. Some countries have regular short breaks embedded in their work culture while others do not, and such norms could influence people's willingness to interact with DMH tools during work hours. Even within regions of the same country, workplace practices and norms can differ \cite{parekh2022cross}. Future research should encompass diverse populations, including those both within and beyond North America, to explore how varying cultural and environmental contexts impact engagement with DMH interventions.}

Second, the specific context of our study — such as its duration and focus on a text messaging program — may limit the generalizability of our findings. Many of our insights should be applicable to the broader usage of DMH tools, yet different intervention designs might reveal distinct disruptors. \topic{Restricting the second study to participants with elevated depression and anxiety symptoms, as in the first study, might have provided findings more tailored to that population. Future studies could also explore how prior DMH tool usage influences future engagement with other tools.}

Third, we acknowledge that the optimal level of engagement with DMH tools is a complex topic that intertwines with the clinical model underlying the tool itself \cite{knights2023framework, saleem2021understanding, yardley2016understanding}. We also recommend that the proposed solutions from the second study should undergo thorough validation and be implemented in collaboration with clinicians and domain experts. Our research does not suggest that all DMH tools necessitate or should strive for continuous, long-term engagement. Rather, \topic{our insights are particularly relevant to DMH tools designed for extended engagement over weeks or months, as interventions requiring only brief user involvement, such as single-session interventions \cite{schleider2020future}, are less likely to be impacted by the disruptors we identified. Therefore, we limit our claims to tools intended for long-term use and do not extend them to short-term DMH interventions.}