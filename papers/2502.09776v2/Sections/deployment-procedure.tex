\section{Procedure of Study 2}
%Following the identification of SEDs in our first study, we aimed to discover possible solutions that could address and account for these SEDs. 

The first study identified several critical SEDs within the context of engagement with a DMH tool. These SEDs emerged as broader factors originating from the social contexts surrounding an individual. Spurred by these insights, our subsequent efforts were directed towards identifying potential solutions to address and alleviate these SEDs. To this end, our second study involved conducting design workshops with \topic{25 participants} aimed at generating ideas for further exploration. \topic{While we drew inspiration from the deployment of a text messaging tool for these workshops, we aimed to explore how any sort of DMH tool can be designed to address SEDs.}
This phase of the study was approved by \redacted{the University of Toronto's} Research Ethics Board, and its logistics are described below. 

\subsection{Participants}

%To ensure a more diverse pool of participants, 
We utilized promotional calls across various social media platforms to recruit diverse participants who felt overwhelmed by their life circumstances. %Participants were recruited between June and July 2023. 
The study was framed as a collaborative design activity to understand challenges in everyday life that could impede engagement with DMH tools and to explore strategies for addressing these challenges. 

\topic{In the first study, participants with elevated symptoms of depression and anxiety frequently highlighted feeling overwhelmed by competing demands \cite{kabigting2019conceptual, hopps1995power} as a primary barrier to sustained interaction with DMH tools. Recognizing that this challenge is also prevalent in populations without a formal diagnosis or clinically elevated symptoms, we expanded the focus in the second study to include other populations that would benefit from DMH tools. This second study included individuals experiencing elevated stress irrespective of their current levels of depression or anxiety symptoms, allowing us to incorporate their diverse perspectives in our workshops.}
% The shift in participant groups reflects the distinct goals of the two studies.} %Recognizing that this experience of overwhelm is a common, cross-cutting issue not exclusive to depression and anxiety, we expanded our focus in the second study to include individuals experiencing elevated stress. This approach enabled us to examine SEDs that contribute to overwhelm across a broader range of individuals. The shift in participant groups reflects the distinct goals of the two studies.}

%To align with the overarching theme of 'overwhelm' from our first study, 
Our eligibility criteria required participants to self-report moderate to severe levels of stress according to the Perceived Stress Scale (PSS) (determined by a score of 14 or more) \cite{cohen1994perceived}.  Eligibility criteria also required participants to be at least 18 years old and residents of North America. 
Interested individuals could follow a link to learn more about the study online. After completing a brief eligibility survey, eligible participants were asked to provide informed consent.


\topic{The final cohort for this study consisted of 25 participants} with an average age of $25.0\pm4.2$ years old. Participants identified with multiple genders (15 men, 9 women, 1 non-binary) and multiple racial groups (9 Asian, 7 White, 6 African American, 1 Native Hawaiian/Other Pacific Islander, and 2 mixed race). 
\topic{Similar to the first study, participants were asked whether they had ever been diagnosed with major depression or anxiety disorder by a specialist or healthcare provider, although a formal diagnosis of depression or anxiety was not required for participation.
Nine participants (36.0\%) reported being diagnosed with depression, and 12 participants (48.0\%) reported being diagnosed with an anxiety disorder.} We refer to these participants as SP1--SP25. 

\subsection{Data Collection and Procedures}

We hosted five online design workshops through the Google Meet videoconferencing platform, with each accommodating 4–6 participants and two research team members. Each participant attended only one session, and every workshop lasted approximately 90 minutes. Participants were compensated \$30 USD for their participation.

The workshops began with a brief introduction to the workshop's goals, followed by a presentation from a research team member detailing the findings of our first study. After this overview, participants shared their personal experiences with DMH tools similar to the Small Steps SMS program and their general thoughts about using technology (e.g., apps, text messaging) to support mental health self-management. \topic{They were also asked to reflect on SEDs they had experienced or considered relevant to their interactions with DMH tools, both within and beyond the findings of the first study.} We then presented them with a series of hypothetical scenarios distilled from common issues that emerged from the formative study. Each scenario presented a set of circumstances in which an individual might become overwhelmed, spurring participants to reflect on the challenges one might face while trying to engage with a DMH tool as intended. \topic{These scenarios are summarized in Table~\ref{tab:scenario} in Appendix \ref{app: scenarios}.}




Following the hypothetical scenarios, participants engaged in three rounds of activities during which they documented their thoughts in online documents and shared their insights verbally. Each of these rounds lasted 15--20 minutes and revolved around one of the following questions:

\begin{enumerate}
\item Reflecting on times when you faced similar challenges, which self-care practices did you find most beneficial? What specific strategies would you recommend for managing such challenges in the future?
\item In what ways do you believe programs like the Small Steps SMS Program can be enhanced to help people adopt these self-care practices, especially during challenging times?
\item Are there strategies that you feel are broadly applicable regardless of specific life challenges? %Conversely, which strategies do you deem less broadly applicable?
\end{enumerate}

\topic{We clarified that participants did not need to limit their recommendations to modifications for the Small Steps SMS program. They were encouraged to share insights that could be broadly applicable to the design of various DMH tools.}

\subsection{Data Analysis}
After consolidating the workshop transcriptions and participant notes in the online documents, we employed the same thematic analysis procedures that we utilized in the first study, although a distinct codebook was developed for this phase.

\subsection{Ethical Considerations}

We used similar measures to the ones detailed in Section \ref{study1_ethics} in order to ensure that participants were treated in an ethical manner. All researchers who spoke with participants were prepared to conduct the Columbia-Suicide Risk Assessment protocol, yet no risks emerged and no follow-ups were necessary.