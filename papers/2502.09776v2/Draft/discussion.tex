\section{Discussion}

%Holistically viewing users' life

%\subsection{Key Insights}

\subsection{RQ1: Influence of SEDs in Dictating Engagement with DMH Tools}

%\todo{connect with engagement}

Our study offers nuanced insights into the HCI discussion around DMH tools, underscoring the intricate manner in which an individual's social context impacts their engagement with DMH tools. Central to our observations was the pervasive sense of being overwhelmed by various SEDs, including academic challenges, workplace expectations, familial duties, and unanticipated life events like health setbacks or relocations. While the specific circumstances and disruptors varied among individuals, there were commonalities across their experiences related to the feeling of overwhelm. This state often resulted in difficulty planning and prioritizing beyond immediate needs, along with increased distress about their broader situation. Such feelings led to failure to routine usage of the tool, even when participants expressed a genuine affinity for the platform and saw its potential benefits. Therefore, work that focuses on tweaking content, tailoring algorithms, or otherwise making superficial adjustments to a tool is unlikely to succeed in achieving engagement for populations facing mental health challenges, especially when considered over weeks or months.


%Although individuals' circumstances and specific disruptors varied widely, there were commonalities across their experiences as relating to the feeling of overwhelm, which included a loss of ability to plan and prioritize beyond meeting their most pressing obligations, as well as distress about their situation. Naturally, this typically resulted in failure to routinely use of a mental health tool, about which many participants felt some regret. These experiences of disruption occurred despite the users' best intentions. Indeed, participants reported generally liking the text messaging tool and the content it delivers, and thought it could be helpful to them. Thus, work that focuses on tweaking content or tailoring algorithms or otherwise making superficial adjustments to a tool is unlikely to succeed in achieving engagement for populations facing mental health challenges, especially when considered over weeks or months.

The intense sense of overwhelm stemming from academic and professional commitments may shed light on why participants in previous studies noted a preference for minimal to no support during work hours, even though there might be intermittent needs for respite and support \cite{bhattacharjee2023investigating, bhattacharjee2022design, howe2022design, wong2023mental}. Such pervasive stress can often relegate mental wellbeing initiatives to the sidelines, notably when the benefits of self-care practices remain intangible in the short term. In the face of a cascade of urgent responsibilities, participants reported frequently deferring self-care initiatives, finding themselves overshadowed by tasks with immediate deadlines and tangible rewards \cite{wong2023mental}. Compounding this challenge is the difficulty in maintaining a consistent healthy lifestyle. Factors such as shifts in dietary habits or disruptions in sleep patterns could further erode an individual's capacity to manage time efficiently \cite{gipson2019effects, karlgren2023sleep, lattie2020designing}. Such disturbances could adversely impact task performance and diminish the attention devoted to mental health practices.

Family obligations and financial concerns were also found to be important SEDs that can pervade every facet of an individual's life.  Prior research in the mental health domain has demonstrated the profound implications of several broad socio-economic factors, revealing their potential to exert adverse effects on mental wellbeing \cite{allen2014social, slavich2020social, tachtler2020supporting, guan2022financial, li2022suffered}. Our findings resonate with this narrative, emphasizing that these factors not only influence mental health but also play a pivotal role in determining the extent to which individuals engage with DMH tools. For instance, the demands of child-rearing can be a dominant source of stress, often leaving scarce opportunities for individuals to prioritize their self-care. Analogously, the weight of managing financial intricacies can exact a similar toll. These overarching social factors underscore the intertwined nature of life's responsibilities and their resultant influence on psychological wellbeing and engagement with DMH tools.


Our findings highlighted the importance of embracing a comprehensive understanding of users' lives when offering strategies to bolster psychological wellbeing \cite{pendse2022treatment, allen2014social, slavich2020social, murnane2018personal}. While self-management of psychological wellbeing could be perceived as an individual responsibility \cite{lorig2003self}, more recent studies have highlighted the limitations of such individualistic perspectives \cite{sallis2015ecological, murnane2018personal}, advocating for acknowledging the context in which an individual lives and is surrounded by.
In the context of our findings, this implies that DMH tools should factor in users' myriad responsibilities, life aspirations, daily routines, and social connections. A salient reason for disengagement, as indicated by our data, is users' struggle to align their engagement with DMH tools within the framework of their overarching goals and duties. In instances where such alignment falters, individuals often sideline their psychological wellbeing. This observation echoes earlier research advocating that mental health initiatives should resonate with the sociocultural milieu and life circumstances of individuals \cite{slavich2020social, slavich2023social, allen2014social, glanz2008health, tachtler2021unaccompanied, oguamanam2023intersectional, murnane2018personal}.


\subsection{RQ2: Designing DMH Tools to Account for the Social Contexts}

%Our findings underscored the importance of adopting a more holistic view of users' lives in the realm of DMH tool design, especially when it pertains to engagement. %Current tools like ours, often with a narrow focus on specific metrics or behaviors, may neglect the broader, interconnected aspects of users' daily lives, leading to decreased engagement and potential disconnection. Participants frequently voiced the sentiment that such tools, while well-intentioned, sometimes felt misaligned with their intricate realities, spanning personal, social, and professional spheres. This misalignment can result in tools feeling more burdensome than beneficial, prompting sporadic or short-lived engagement. To bolster sustained and meaningful user interaction, there is a pronounced need for tools that genuinely resonate with the multifaceted nature of users' experiences. Embracing a more holistic understanding of users' responsibilities, priorities, life goals, and surroundings, hence, will ensure that DMH tools are not just interacted with, but integrated into, the diverse tapestry of users' lives.

We recognize the complexity embedded in translating the understanding of broad SEDs into actionable design. Nevertheless, as elucidated below, our insights pave the way for design suggestions that can facilitate promising avenues for imminent DMH tools.


%This could necessitate users to input exhaustive details about their lives, or demand considerable computational capabilities, coupled with intensive tracking — all of which usher in privacy and ethical dilemmas \todo{cite}. 



%\subsection{Design Considerations}
\subsubsection{Streamlined Self-Care Through Structured Goal-Setting}


In line with previous studies \cite{bhattacharjee2022kind, brown2014health}, our findings underscore the difficulty users encounter when trying to weave self-care activities seamlessly into their daily lives. Conflicts arise when participants' perceived responsibilities, both familial and professional, interfere with the timing and perceived significance of these activities. Feedback from participants illuminates a pressing need: for tools that offer a clearer, more structured overview of their lives, assisting them in prioritizing their myriad responsibilities.

To effectively structure and compartmentalize these responsibilities, insights can be drawn from literature on collaborative goal-setting \cite{agapie2022longitudinal, lee2021sticky, hsueh2020exploring}. A plausible approach to simplifying complex objectives may be the use of guided conversations \cite{o2018suddenly, bowman2022pervasive}, designed to instruct users on breaking down broad objectives into distinct, actionable steps with rationales. Recognizing that these guided interactions might require thoughtful engagement, participants in our study suggested that such interventions be reserved for users' \textit{idle times} \cite{poole2013hci}. These idle times could include moments when individuals are engaged in low-attention activities, such as checking messages or browsing social networking platforms. Once a plan has been crafted, DMH tools could then offer occasional reminders, helping users to stay on track with the incremental steps needed to achieve their overall goals, thus facilitating a more manageable approach to balancing their myriad responsibilities.


Additionally, the challenge of recognizing the benefits of wellbeing activities over immediate academic or professional tasks can be addressed by implementing strategies that align with individuals' personalized goals and visually representing their progress. DMH tools could enable users to set personalized psychological wellbeing goals that correspond with their broader life objectives, highlighting the real-world implications of their self-care practices \cite{jennings2018personalized, lindhiem2016meta, agapie2022longitudinal}. Additionally, incorporating visual analytics tools as suggested in health and well-being visualization studies \cite{baumer2014reviewing, kocielnik2018reflection} could provide users with graphical representations of their wellbeing improvements, linking them to enhanced performance in other life areas. Further reinforcement can come from integrating educational modules that elucidate the scientific connections between wellbeing practices and work performance and overall life improvement \cite{howe2022design}. Through these combined approaches, users may be able to prioritize self-care amidst their manifold responsibilities.


%However, merely providing guided interactions may not be adequate, particularly when confronting multifaceted challenges and users are uncertain where to begin. Participants of our study indicated that there could be times they were not aware of existing resources or may not remember them. In such case,
%a supplementary approach could include sharing narratives of individuals navigating similar challenges or offering access to relevant community resources that can offer additional insight and guidance \cite{bhattacharjee2022kind, thaker2022exploring, baumel2018digital}.

%\todo{Should I add LLMs?}

%\todo{user may not know the importance, may need to be reminded}











%We also observed that participants indicated a need for scheduling such activities that would not clash with those responsibilities . 


\subsubsection{Designing for Flexible Engagement}

Our participants expressed a desire for greater flexibility when interacting with DMH tools. This need for adaptability spanned from a reimagined understanding of disengagement to diverse modes of interaction.

The core intent of DMH tools is to foster wellbeing, yet the overwhelming emphasis on sustained engagement can inadvertently defeat this purpose. Our research underscores that a relentless push for streaks and frequent notifications can introduce unnecessary stress. The daunting challenge of maintaining long streaks can discourage users from taking even small, incremental steps. A missed day of engagement might amplify feelings of setback. These observations resonate with prior studies that caution against over-gamifying engagement (e.g., focus on maintaining a streak, reaching a certain number of step counts) \cite{etkin2016hidden, bekk2022all, liu2017toward, jia2016personality, hamari2014does}. A more balanced approach that may allow users to derive value even from passive interactions or occasional breaks could be explored in this context. Rather than highlighting missed engagements, DMH tools should try to cultivate a foundational connection, nudging users gently during low engagement periods \cite{bhattacharjee2023integrating}.

This flexibility extends to the mode of interaction with DMH tools, as relying solely on a single mode, such as text messages, may not be suitable for all circumstances and may miss opportunities that might be captured through alternative means of communication. For instance, some individuals may find comfort in verbal communication when reflecting on stressful situations \cite{murray1994emotional, stigall2022towards}. Additionally, allowing users to respond via voice might provide an accessible option when their hands are occupied, such as when driving. People may also lack the concentration required to read lengthy texts when focused on other tasks \cite{bhattacharjee2022kind}, but images or animations could convey essential information in a more engaging manner \cite{bryant2003imaginal, howe2022design}. By incorporating flexibility to engage through diverse modes of communication, DMH tools can enhance accessibility and responsiveness to individual needs.

%Engagement is often key to aiding users in managing their wellbeing. However, many DMH tools appear overly preoccupied with sustaining user engagement. They frequently inundate users with notifications and emphasize maintaining streaks. While such features can be helpful in keeping users aligned with DMH tools, our study highlighted potential drawbacks. For some participants, the pursuit of maintaining streaks could inadvertently morph into a stressor. These streaks, perceived as lofty or unattainable, sometimes deter users from even initiating small, progressive steps. Conversely, an occasional lapse in engagement could leave users with the disheartening feeling of starting from scratch, undermining the very purpose of the tool. These findings extend the prior literature on quantifying and gamifying engagement \cite{etkin2016hidden, bekk2022all, liu2017toward, jia2016personality, hamari2014does}, by highlighting the potential negative impacts of streaks and rewards. As our participants reminded us, the goal of the DMH tools should instead be allowing users to manage their wellbeing, which might sometimes be possible through passive engagement by briefly skimming messages, or even allowing for occasional disengagement. In periods of low or no engagement, users should not be reminded about their failure to engage, rather they can be requested to have a baseline connection with DMH tools \cite{bhattacharjee2023integrating}.
%Thus, an overemphasis on maintaining unbroken streaks might sometimes be counterproductive.

\subsubsection{Multi-Level Interventions for Psychological Wellbeing Support}


Our findings emphasize the necessity for multi-level interventions that extend across various dimensions of life, including educational and workplace technologies, along with broader societal factors \cite{sallis2015ecological, scholmerich2016translating, bauer2022community, murnane2018personal}. In the educational and professional spheres, there exist  significant opportunities for organizations to weave mental health support into their existing technological infrastructure. Such integration acknowledges that mental wellbeing is not isolated from other facets of life but is intimately linked with work and study environments. Policy-level changes, as suggested by \citet{murnane2018personal}, could profoundly influence psychological well-being, either enhancing or deteriorating mental health support within organizations.

Building on the work of \citet{howe2022design}, interventions can be delivered at opportune moments identified through email interactions, digital calendar usage, or physiological data monitored in office settings. The inclusion of close relations and peers in an individual's mental wellbeing initiatives can add another layer of support \cite{bhattacharjee2022kind, burgess2019think}. Participants in our study pointed out that intense periods of stress and overwhelm could be mitigated collectively with colleagues and peers, who support one another. DMH tools can further facilitate these interactions by enabling group support and shared narratives. This communal approach can not only validate emotions but foster a sense of belonging, encouraging individuals to face their struggles with a sense of solidarity and shared experience. 

Regarding other SEDs, such as challenges in carrying out family obligations and managing financial support, targeted interventions could be strategically implemented. For example, access to programs providing family counseling and parental education can help individuals navigate the complexities of familial responsibilities, thereby reducing associated stress \cite{chi2015systematic, smout2023enabling}. Financial literacy workshops and access to personal finance management tools can empower individuals to take control of their financial situations, mitigating anxiety and potential mental health deterioration \cite{boyd2016earned, mehra2018prayana}. Community-based partnerships with local banks and financial institutions might facilitate these workshops, providing a practical approach to managing financial intricacies \cite{dillahunt2022village, moulder2014hci}. Moreover, support for childcare, eldercare, or other family-related needs could be organized and distributed through community support groups, relieving some of the burdens on those juggling family and work \cite{gisore2012community, butler2012relationship, compton2015social, smout2023enabling}.

\todo{Should I add LLMs}


%tangibility
%quantifiability
%structure

%addressing overwhelm -- address life problems and goal setting

%community support -- other models and cscw papers



%\subsection{Designing Multi-Level Interventions}




%\subsection{Allowing People to See tangible Benefits}


%\subsection{Family, community}

%Social translucence 
%https://dl.acm.org/doi/fullHtml/10.1145/2775388 



Sign-up Survey



%audio-video what people are used to do


%personal holistic
%personalized
%individual personalization
%maybe subgroup

\todo{social contexts become disruptions -- check otter}