\section{Related Work}

In our overview of related work, we first outline the diverse social factors that negatively impact psychological wellbeing.
We then describe how DMH tools, more specifically text messaging services, have been used to promote behavior change and psychological wellbeing.
We follow this overview with a discussion on the challenges that designers and users face in sustaining engagement with DMH tools.


\subsection{Frameworks on the Social Determinants of Mental Health}
The impact of social contexts on mental health has garnered significant attention in the fields of HCI and psychology \cite{alegria2018social, allen2014social, bhattacharjee2021understanding, lewis2019cultural, slavich2023social, tachtler2021unaccompanied, lattie2020designing, bhattacharjee2023integrating, slavich2020social}. 
\todo{People have developed frameworks, we ground our findings in the following ones:}
%Another related theory with some overlap
%Social factors are important
    %social determinant by Allen
    %Other works have looked into social experiences -- social safety theory

% while they impact mental health, not sure how much they impact engagement with mental health tools

%Plenty of research works have suggested that one's social and economic backgrounds play key roles in shaping one's mental health and state of mind. 
%Social determinants refer to the conditions in which people ``are born, grow, live, work, and age'' \cite{marmot2011global}. The concept of social determinants has its inspiration from social gradients \cite{honorio2021social}, which suggests that people with higher social status have a higher life expectancy and fewer health risks compared to those with lower status. Differences in social determinants have been found to stem from an unequal distribution of resources, and hence, ways to reduce the differences should incorporate targeted social and economic measures \cite{marmot2008closing}
%\todo{difficulty in using one word}

\subsubsection{Social Safety Theory}
One notable framework that researchers have employed to methodically structure conversations around this topic is the Social Safety Theory by \citet{slavich2020social}. This framework highlights two contrasting forces on one's life circumstances: (1) socially safe environments characterized by aspects like acceptance, stability, predictability, and harmony; and (2) socially threatening situations marked by aspects like conflict, turbulence, and unpredictability. Social Safety Theory suggests that these two components impose an observable impact on one's mental wellbeing. 
As an example, the repercussions of performing poorly on an examination can permeate beyond the academic sphere. It can lead to negative social evaluations and critiques from peers or family members, subsequently impacting self-esteem and social status \cite{diamond2022rethinking, noronha2018study, bhattacharjee2021understanding}. Similarly, losing a job is often regarded as a negative economic event, but it can also trigger the loss of close social connections within the workplace and a decline in social standing amongst friends and family members \cite{diamond2022rethinking, katz2017self}.

%\subsubsection{Another Name}
A pivotal study by \citet{allen2014social} examined how individuals' living conditions, working conditions, socio-economic standing, and surrounding environment influence their mental health outcomes. Allen et al. employed a multi-level framework to investigate these social determinants, suggesting preventive measures for mental health disorders in the general populace. Their framework integrated a life-course perspective that encompassed various life stages (e.g., childhood, adulthood), community-level contexts (e.g., the natural and built environment, primary healthcare), and country-level contexts (e.g., social and political factors, cultural norms). The study's findings highlighted the detrimental effects on mental health outcomes stemming from adverse situations like unemployment, inadequate education, and instances of gender discrimination.

%Various 
\subsection{Social Contexts as Determinants of Mental Health}
Several studies have delved into the connection between social factors and mental health across different settings  \cite{brydsten2018health, lecerof2015does, han2015social, reibling2017depressed, li2022suffered, groot2022impact, de2017gender, frik2023model, oguamanam2023intersectional}. 
For instance, a study by \citet{katz2017self} found a correlation between lower income and symptoms of depression and suicidal thoughts among transgender individuals in the USA. Similarly, a study by \citet{guan2022financial} found that economic factors like the ownership of tangible possessions and subjective experiences of financial strain were strong predictors of depressive symptoms. Elsewhere, Danish researchers demonstrated that youths living independently in subpar housing conditions were particularly vulnerable to mental health challenges \cite{groot2022impact}. Meanwhile, a study conducted during the COVID-19 pandemic observed that female parents of young children were experiencing higher stress levels than other demographic groups~\cite{li2022suffered}. 
% Collectively, these investigations reveal that mental health conditions can be notably susceptible to various socioeconomic challenges.
% , such as financial strain, housing conditions, and familial responsibilities.


%Research conducted by \citet{guan2022financial} revealed that  tangible possessions, non-collateralized debt, and personal assessments of financial strain, are relatively strong and persistent predictors of depressive symptoms. A group of Danish researchers concluded that youths living alone in poor housing conditions were particularly prone to mental health concerns~\cite{groot2022impact}. Another study reported that female parents of young children endured higher stress levels during the COVID-19 pandemic compared to other demographic groups~\cite{li2022suffered}. 
%These studies, along with others,  indicate that mental health conditions can be particularly susceptible to challenges such as financial difficulties, housing conditions, and familial responsibilities.


The diverse ways in which social factors intersect with mental health are exemplified in the experiences of various demographic groups. Academic performance is often a primary stressor for university students, although other factors like high expectations from parents, economic pressures, and relationship problems can also exacerbate their mental health \cite{de2016relationship, bruffaerts2018mental, bhattacharjee2021understanding, lattie2020designing}. Meanwhile, the mental health of immigrant youths can be detrimentally impacted by concerns such as fears of deportation and anxiety over families left in their home countries \cite{tachtler2021unaccompanied, tachtler2020supporting, jakobsen2017impact}. Older adults often face unique challenges that can impact their mental health: social isolation, physical health issues, caregiving stress, and cognitive decline \cite{carod2017social, mccrone2008paying, huisman2013socioeconomic}.
These studies underscore the complex interplay between socioeconomic hardships and mental health, reinforcing the necessity for comprehensive approaches that consider these socioeconomic contexts when addressing mental health challenges.

%\citet{starkey2013financial} also observed similar results for African women and suggested that treatment strategies should be adapted based on one's level of financial distress. Indicators of poor financial conditions like food insecurity, 
%A British study \cite{pevalin2017impact} found that consistently living in poor housing conditions can worsen mental health of youth and adults. 

%\todo{some study on stress during end of semester, start of new jobs, etc}

%\todo{social safety schemas -- maybe}

%\todo{while they impact mental health, not sure how much they impact engagement with mental health tools}

%\todo{look into literature doc, look into Shirley's summary}

\subsection{Text Messaging as a DMH Tool for Promoting
Behavior Change and Psychological Wellbeing}
\todo{sentence that emphasizes that text messages are a type of DMH tool}
In recent years, text messaging services have shown considerable success in supporting behavior change across various physical and mental health challenges \cite{haug2013efficacy, haug2013pre, yun2013text, suffoletto2023effectiveness, shalaby2022text, figueroa2022daily, bhandari2022effectiveness, villanti2022tailored, gipson2019effects}. For instance, \citet{villanti2022tailored} conducted a randomized controlled trial and found that a tailored text message intervention for socioeconomically disadvantaged smokers led to greater smoking abstinence rates, reduced smoking frequency, and increased desire to cease smoking compared to online resources. A similar study by \citet{liao2018effectiveness} demonstrated that frequent text messages (3-5 messages per day) can contribute to lower cigarette consumption rates among adult smokers. 
%Moreover, \citet{suffoletto2023effectiveness} highlighted the benefits of text messaging interventions for non-treatment-seeking young adults with hazardous drinking habits. Their study showed that messages providing feedback on drinking plans, alcohol consumption, and drinking limit goals resulted in sustained reductions in binge drinking days. 
Regarding other forms of behavior change, \cite{gipson2019effects} developed an interaction aimed at improving sleep hygiene among college students. 
Additionally, \citet{glasner2022promising} developed an intervention aimed at improving medication adherence and reducing heavy drinking in adults with HIV and substance use disorders. 
Other areas where text messaging has shown promise include weight management \cite{siopis2015systematic, donaldson2014text}, promoting physical activity \cite{kim2013text, smith2020text, murnane2020designing}, and enhancing vaccine uptake \cite{buttenheim2022effects, mehta2022effect}, among many others.

There has also been a surge of text-messaging services aimed to promote psychological wellbeing~\cite{kretzschmar2019can, chikersal2020understanding, rathbone2017use, inkster2018empathy, morris2018towards, stowell2018designing, kornfield2022meeting, agyapong2020changes}. The breadth of the content they cover and the goals they seek to support are vast. Most text messaging services employ therapeutic techniques from clinical psychology — cognitive behavioral therapy (CBT) \cite{willson2019cognitive}, dialectical behavior therapy (DBT) \cite{linehan2014dbt}, acceptance and commitment therapy (ACT) \cite{hayes2004acceptance}, and motivational interviewing \cite{hettema2005motivational} — to deliver personalized content. Text messages can function as reminders, assisting individuals in setting aside time for leisure and physical exercise \cite{bhattacharjee2023investigating, figueroa2022daily}. They can also facilitate expressions of gratitude, nudging individuals to reflect on life's positive aspects \cite{bhattacharjee2022design}. Furthermore, text messages can convey narratives, offering insights into how peers have navigated and surmounted challenging life situations, thereby aiding individuals in applying psychological strategies in their own lives \cite{bhattacharjee2022kind}. These advancements have spurred researchers to explore the potential of AI-powered chatbots to simulate human conversations and foster self-reflection \cite{inkster2018empathy, kocielnik2018reflection, tielman2017should}. The momentum in this domain has further accelerated with the recent rise of large language models (LLMs) \cite{van2023global, kumar2022exploring}.

There have been numerous initiatives that harness text messaging services to assist individuals in navigating their wellbeing during particularly stressful events \cite{agyapong2020changes, levin2019outcomes, arps2018promoting, nobles2018identification, bhattacharjee2023investigating, agyapong2022text4hope, agyapong2022text4hope}. \citet{agyapong2022text4hope} implemented such a system during the COVID-19 pandemic, dispatching daily support messages that led to decreased self-reported anxiety and stress. \citet{levin2019outcomes} combined psychoeducational content and reminders to assist those with bipolar disorder and hypertension. Meanwhile, \citet{arps2018promoting} tackled adolescent depression by sending daily gratitude-focused texts.

Yet, there are several challenges pertaining to one's individual and social contexts that could prevent text messaging systems and other DMH tools from providing more benefits to users. We discuss these issues in the next subsection.


% Various text messaging services have been used to help people manage wellbeing during critical periods. For example, Agyapong et al.
% [1] deployed a text messaging service to help people manage their
% mental wellness during the COVID-19 pandemic. The service sent
% daily supportive messages to people along with requests to reflect
% on their stress, anxiety, and depression level. The authors found
% that their messages were able to significantly reduce self-reported
% anxiety scores among users. Levin et al . [68] delivered messages
% ranging from psychoeducational texts, reminders, and EMA queries
% to help people with bipolar disorder and hypertension manage
% their symptoms. Arps et al . [2] were able to reduce depressive
% symptoms among adolescents by sending them daily gratitude mes-
% sages. Lastly, researchers have explored the design of artificially
% intelligent chatbots that try to promote self-refection by engaging
% users in human-like conversations [50, 59, 123]




%In the past few years, text messaging services have  been quite successful in supporting behavior change for a range of physical and mental health challenges \cite{haug2013efficacy, haug2013pre, yun2013text, suffoletto2023effectiveness, shalaby2022text, figueroa2022daily, bhandari2022effectiveness, villanti2022tailored}. For example, 
%\citet{villanti2022tailored} reported through a randomized controlled trial that a tailored text message intervention for socio-economically disadvantaged young adult smokers could result in greater smoking abstinence rates,  reductions in smoking frequency, and increased desire to quit compared to online quit resources. Deployment of a similar service \cite{liao2018effectiveness} suggested that frequent delivery of messages (3--5 messages every day) can promote lower cigarette consumption rates among adult smokers. \citet{suffoletto2023effectiveness} noted the benefits of text messaging interventions in helping non-treatment-seeking young adults with hazardous drinking habits, showing that messages that provided feedback on drinking plans, alcohol consumption, and drinking limit goals resulted in long-lasting reductions in binge drinking days. \citet{glasner2022promising} developed an intervention aimed at enhancing medication adherence and reducing heavy drinking in adults with HIV and substance use disorders. Other promising avenues for text messaging applications include weight management \cite{siopis2015systematic, donaldson2014text}, physical activity promotion \cite{kim2013text, smith2020text, murnane2020designing}, vaccine uptake \cite{buttenheim2022effects, mehta2022effect}, among many others.




\subsection{Challenges of Sustaining Engagement with DMH Tools}

%Despite the well-documented benefits provided by digital mental health tools, they face lots of challenges in engaging users. The commonly cited reasons involve the delivery of generic content that people fail to see the benefits of or how they might apply in their life. To reach a broader population, interventions often try to address issues that might apply to a broad group of people, but in doing so they risk being perceived as insincere or surface-level. Additionally, the intervention contents may also have limited variety; ultimately, users end up engaging with repeated content in a short timeframe, making them possess a negative attitude toward the novelty aspect of the intervention.

While DMH tools offer an array of benefits, their practical utility is often limited by several substantial challenges that hinder user engagement. One of the most frequently cited issues lies in the delivery of generic content. In an effort to achieve widespread accessibility and help the maximum number of people, digital interventions often take a generalist approach \cite{rennick2016health} that addresses issues broadly applicable to a diverse population. However, such interventions run the risk of being perceived as insincere or superficial, making users feel that the information is too generic to be truly valuable or that it fails to consider the nuanced complexities of their individual experiences \cite{bhattacharjee2023investigating, slovak2023designing}. 
In turn, users are led to question the applicability of generic content to their unique circumstances or how to implement the suggestions from such content in their daily lives \cite{lederman2014moderated, bhattacharjee2022design}.
A further complication arises from the limited content variety within these digital tools \cite{bhattacharjee2022design}. Users often encounter repetitive content in DMH tools. This can reduce the intervention's perceived novelty and create a sense of monotony, negatively impacting engagement and leading to boredom \cite{brown2014mobile}. %The lack of variety can disengage users, potentially diminishing the tool's effectiveness and user satisfaction.
Furthermore, DMH tools face the challenge of competing with various ongoing and urgent demands for user attention \cite{muench2017more}. Even with personalized content, these tools run the risk of being ignored if they do not intervene at opportune moments \cite{bhattacharjee2022design}. 


%Several research studies have focused on identifying the ideal time window to deliver interventions~\cite{figueroa2022daily, hardeman2019systematic, gonul2019expandable, lin2006fish, rabbi2015automated, goldstein2017return, konrad2015finding, ghandeharioun2016kind}. Some studies have relied on users' digital calendars to determine when to send massages~\cite{howe2022design}. Meanwhile, other studies have relied on information about users' physical or mental state, including self-reported stress scores~\cite{paredes2014poptherapy}, physiological data (e.g., heart rate \cite{clarke2017mstress}) or activity level (e.g., walking, running, or being in a vehicle \cite{ismail2022design}). These works often utilize personalization algorithms, such as contextual multi-armed bandit algorithms \cite{figueroa2023ratings, li2010contextual}, to determine the timing and content of interventions that will have the highest likelihood of benefiting and engaging each user. By leveraging personalized algorithms and considering various contextual factors, DMH tools aim to optimize their interventions to be delivered when they are most likely to have an impact and capture the user's attention.

%\todo{review lit from https://www.jmir.org/2021/3/e24387/

%Literature has explored the impact of technology design on user engagement, but the profound role of social context has largely been unrecognized. Predominant models like the Technology Acceptance Model (TAM) \cite{marangunic2015technology, venkatesh2008technology, davis1989perceived} and the Unified Theory of Acceptance and Use of Technology (UTAUT) \cite{venkatesh2003user, dwivedi2019re} anchor their analyses on factors such as perceived usefulness and ease of use. Social influence is included to a lesser extent, but only basic elements like peer influence or subjective norms. These models frequently overlook the nuanced interplay between users and their diverse social environments: familial ties, workplace rules and responsibilities, and broader societal interactions. Given the known significance of these dynamics in influencing mental health \cite{allen2014social, lattie2020designing, slavich2023social}, our work seeks to explore their role in determining the effectiveness and receptiveness of digital interventions.




Researchers have endeavored to address these barriers and enhance engagement with DMH tools through adjustments in tool design or by tailoring algorithms to ensure personalization in terms of content, timing, or delivery method \cite{doherty2012engagement, harding2015hci, poole2013hci, xu2013designing, howe2022design}. This view is also prominent in the design of influential technology adoption models. Models such as the Technology Acceptance Model (TAM) \cite{marangunic2015technology, venkatesh2008technology, davis1989perceived} and the Unified Theory of Acceptance and Use of Technology (UTAUT) \cite{venkatesh2003user, dwivedi2019re} have centered on factors like perceived usefulness and ease of use. Social influence has been included but to a lesser extent, with an emphasis on rudimentary elements such as peer influence and subjective norms.

However, there is a lack of comprehension about the dynamics of engagement in the context of an individual's life circumstances, which makes it challenging to determine whether the right approach to promote engagement requires only minor adjustments or calls for a more fundamental reassessment. At least in the context of DMH tools, prior research on various social contexts and their associations with psychological wellbeing  \cite{bhattacharjee2023investigating, alegria2018social, allen2014social, bhattacharjee2021understanding, lewis2019cultural, slavich2023social, tachtler2021unaccompanied, lattie2020designing, bhattacharjee2023integrating, slavich2020social} suggests that unexplored facets, particularly those tied to diverse social environments like familial relationships, workplace obligations, and broader societal interactions, may demand more profound alterations in how engagement with DMH tools is perceived. These experiences might be multifaceted and unpredictable, impacting a substantial subset of users suddenly and leading to disruptions or even halting tool use entirely. Therefore, it might be crucial to conceptualize tools that can assist users in overcoming challenges without disturbing tool functionality. This consideration may help maintain the user's relationship with the system for future use. %Moreover, tools may be crafted to navigate specific disruptors, signifying that they can be advantageous not merely in spite of these challenges but actually for addressing them.














%HCI and implementation science have tried to promote engagement of DMH tools and in general technology by making adjustments in the design of a tool or tailoring algorithms to ensure personaliztion in terms of content, timing, or delivery method. In the context of technology adoption, predominant models like the Technology Acceptance Model (TAM) \cite{marangunic2015technology, venkatesh2008technology, davis1989perceived} and the Unified Theory of Acceptance and Use of Technology (UTAUT) \cite{venkatesh2003user, dwivedi2019re} have been built upon factors such as perceived usefulness and ease of use, with social influence included to a lesser extent, focusing on basic elements like peer influence or subjective norms. 


%There is a prevailing lack of understanding regarding the mechanics of engagement hampers our ability to ascertain whether the right approach involves only superficial modifications or calls for a more fundamental reassessment.
%At least in the context of DMH tools, prior works on various social contexts and their connections with psychological wellbeing indicate that there could be other aspects, particularly those related to diverse social environments: familial ties, workplace rules and responsibilities, and broader societal interactions, that have not been thoroughly explored and might require more profound changes to how we view engagement in the context of DMH tools. These experiences could be diverse and hard to predict. they affect a significant subset of users at a given moment, and stall or disrupt use of a tool in a dramatic way. It is important to think about designing tools that can help users "weather" challenges without disrupting tool use. This can keep the relationship to the system in place so they have ability to use it when they have ability in the future. Tools may also be designed so that can be helpful in navigating specific disruptors, i.e. they can be helpful *for* and not just *in spite of* disruptors.




%Research on engagement with DMH tools, as well as various other forms of technology, has largely concentrated on making minor adjustments or aligning the tool with the specific circumstances of users. Predominant models like the Technology Acceptance Model (TAM) \cite{marangunic2015technology, venkatesh2008technology, davis1989perceived} and the Unified Theory of Acceptance and Use of Technology (UTAUT) \cite{venkatesh2003user, dwivedi2019re} anchor their analyses on factors such as perceived usefulness and ease of use, with social influence included to a lesser extent, focusing on basic elements like peer influence or subjective norms. Nevertheless, there could be other aspects, particularly those related to disruptions or challenging life situations, that have not been thoroughly explored and might require more profound changes to the content, delivery timing, and method. The prevailing lack of understanding regarding the mechanics of engagement hampers our ability to ascertain whether the right approach involves only superficial modifications or calls for a more fundamental reassessment.  This perspective emphasizes the need for a thorough examination of the complex interplay between tool use and life circumstances, to ensure a more resilient and adaptive design.




%Such dynamics, integral to a user's life, can profoundly shape priorities, receptiveness to digital interventions, and even the interpretation of content.



%Comparison With Other Models of Technology and Digital Health Intervention Engagement
%}

%However,  DMH tools have to compete with many ongoing and often urgent issues for getting user attention \cite{muench2017more}, and even with a variety of personalized contents, they face the risk of being ignored, if they do not intervene users at opportune moments \cite{bhattacharjee2022design}. Many research works have attempted to identify the ideal time window to deliver interventions \cite{figueroa2022daily, hardeman2019systematic, gonul2019expandable, lin2006fish, rabbi2015automated, goldstein2017return, konrad2015finding, ghandeharioun2016kind}; some have used sensors to anticipate their physical state (e.g., heart rate) \citet{clarke2017mstress} and activity level (e.g., walking, on vehicle) \cite{ismail2022design} and deliver interventions accordingly, some have relied on user-reported scores \cite{paredes2014poptherapy}, some others have prompted users in between meetings from digital calendars \cite{howe2022design}. This line of work relies on personalization algorithms (e.g.,  contextual multi-armed bandit algorithm \cite{figueroa2023ratings, li2010contextual}) to identify the timing and content of interventions that have the highest probability to benefit and receive engagement from a particular user.


%\todo{Social and Broader issues}
%However, a major limitation in much of the HCI literature is that disengagement with DMH tools is often viewed individualistically; technologies try to identify suitable times for intervening users or deliver contents that can address their state of mind or activity. But there are numerous external barriers that can cause users to not engage with DMH tools, many of which may not be within the control of the users. These barriers can contribute to user engagement in nuanced ways that cannot be captured by the collection of individual behavioral markers. 

%\todo{adaptive algos are yet to capture external factors}
%However, a significant limitation in much of the HCI literature is that disengagement with Digital Mental Health (DMH) tools is often viewed from an individualistic perspective, focusing primarily on individual factors without considering the broader social context in which individuals live. 
%Prior research suggests that an individual's mental health and state of mind are influenced by a range of social and environmental factors, including their financial status, career prospects, and familial responsibilities \todo{cite}. 
%These external factors may significantly impact an individual's engagement with DMH tools, but their influence is often underexplored in the literature. \todo{algorithms cannot capture them}
%Recognizing the importance of understanding how these external factors impact engagement with DMH tools, we aimed to investigate the interplay between these factors and users' engagement with DMH tools.