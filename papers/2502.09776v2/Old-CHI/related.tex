\section{Related Work}

In our overview of related work, we first outline the diverse social factors that negatively impact psychological wellbeing.
We then describe how DMH tools, more specifically text messaging services, have been used to promote behavior change and psychological wellbeing.
We follow this overview with a discussion on the challenges that designers and users face in sustaining engagement with DMH tools.

\subsection{Social Context and Individual Challenges as Determinants of Psychological Wellbeing}

Several frameworks have been developed in recent years to explain the role of social context and individual challenges as determinants of psychological wellbeing \cite{slavich2020social, slavich2023social, allen2014social, glanz2008health, tachtler2021unaccompanied, oguamanam2023intersectional, sallis2015ecological}. Among these, the Social Safety Theory by \citet{slavich2020social} provides insights into the dichotomy of (1) socially safe environments characterized by elements such as acceptance, stability, and harmony and (2) socially threatening situations marked by elements such as conflict, turbulence, and unpredictability. According to the theory, socially threatening situations can exert a considerable impact on one's psychological wellbeing. Examples of this can be found in everyday situations like performing poorly on an exam or losing a job. These events can extend beyond their immediate academic or economic effects, leading to negative social evaluations, a decline in self-esteem, and a shift in social status \cite{diamond2022rethinking, noronha2018study, bhattacharjee2021understanding, katz2017self}. Adding to this understanding, a study by \citet{allen2014social} delved into the complex interplay between living and working conditions, socio-economic standing, and the surrounding environment. Through a multi-level framework that encompassed various life stages and contextual factors, they illuminated the detrimental mental health consequences stemming from adverse situations such as unemployment, inadequate education, and instances of gender discrimination.


Many studies have delved into the connections between social factors and psychological wellbeing across diverse settings and demographic groups \cite{brydsten2018health, lecerof2015does, han2015social, reibling2017depressed, li2022suffered, groot2022impact, de2017gender, frik2023model, oguamanam2023intersectional, murnane2018personal}. For example, economic challenges have been found to significantly impact psychological wellbeing. A study by \citet{guan2022financial} discovered that tangible possessions and financial strain were strong predictors of depressive symptoms, while a study by \citet{katz2017self} found a correlation between lower income and symptoms of depression among transgender individuals in the USA. Among university students, academic performance, parental expectations, and economic stress have been identified as primary stressors \cite{de2016relationship, bruffaerts2018mental, lattie2020designing}. The psychological wellbeing of older adults has also been explored, with unique challenges including social isolation and physical health issues \cite{carod2017social, mccrone2008paying}. Additional studies have illuminated the diverse ways social factors can intersect with psychological wellbeing, such as immigrant youths fearing deportation \cite{tachtler2021unaccompanied, tachtler2020supporting} and parents experiencing stress during the COVID-19 pandemic \cite{li2022suffered}.

%\todo{multi-level interventions}
%https://dl.acm.org/doi/pdf/10.1145/3274396?casa_token=vo9GehKfd44AAAAA:6_PcS6e8RsNOVzEazag2mnfcPHoERYmW73sgsJBKide_EhIIVcpeGr63xzIhDlua4xGz1sshew4ER44

%Recognizing the influence of various social and economic challenges on mental health outcomes, researchers have advocated for multi-level interventions that holistically address mental health by considering an array of social and economic factors affecting individual lives \cite{aubry2016multiple, kerman2018effects, leung2015household, murnane2018personal, sallis2015ecological, li2010stage}. 

Recognizing the influence of various social and economic challenges on mental health outcomes, some studies developed targeted interventions that address the unique challenges faced by specific groups. These groups include but are not limited to low-income populations, minoritized populations, and those at risk of homelessness \cite{aubry2016multiple, kerman2018effects, leung2015household, murnane2018personal, sallis2015ecological, li2010stage, schueller2019use}. Interventions in this space are structured to account for the technological resources available to these targeted populations and the underlying issues contributing to their mental health conditions. 
For instance, targeted efforts to improve housing conditions have been shown to lessen the demand for inpatient psychiatric services among the homeless who suffer from mental disorders \cite{kerman2018effects}. %Additionally, initiatives promoting community involvement to build neighborhood trust and safety have yielded positive mental health outcomes \cite{butler2012relationship, compton2015social}. 
Other studies advocate for the consideration of temporal aspects like daily routines and life transitions when crafting interventions for psychological wellbeing \cite{doherty2010fieldwork, murnane2018personal, reddy2006temporality}. Implementation science literature \cite{graham2020implementation, urquhart2020defining, powell2015refined} complements these works by emphasizing the importance of social context and structures that can either facilitate or impede the effective use of mental health resources. Rather than concentrating solely on tool design, this field explores strategies for the successful deployment of existing tools in specific settings. Such strategies often include user and staff training as well as monitoring tool uptake \cite{powell2015refined, powell2012compilation}. Collectively, these works underscore the importance of a holistic approach that integrates mental wellbeing considerations into broader societal contexts. 



%Relatedly, the implementation science literature focuses on the real-world contexts and structures around tool use that can either support or undermine it. This work is less focused on how tools are designed, but rather the strategies for successfully deploying established tools in particular settings (e.g., providing training to users and organizational staff, monitoring uptake). These works collectively underscore the importance of a holistic approach, integrating mental well-being considerations into broader societal contexts.  
%Efforts in this avenue include developing initiatives to improve housing stability, which has been linked to decreased use of inpatient psychiatric hospitals among homeless people with mental illnesses \cite{kerman2018effects}, and enhancing neighborhood trust and safety through community participation, with beneficial impacts on mental health \cite{butler2012relationship, compton2015social}. Past work \cite{doherty2010fieldwork, murnane2018personal, reddy2006temporality} has also advocated for considering temporal layer that represents patterns and transitions throughout the course of the individual’s life and integrating wellbeing support with organizational entities.   


%Community-based interventions have sought to enhance neighborhood trust and safety and mitigate violence and crime, with beneficial impacts on mental health \cite{butler2012relationship, compton2015social}. Interventions  have also been identified as having a positive impact on mental wellbeing \cite{webber2017review, murnane2018personal}. \citet{murnane2018personal} 

%Urban planning programs focusing on access to green spaces have demonstrated potential in reducing depressive symptoms \cite{larson2016public, mceachan2016association}, and targeted programs to engage youth through social media and community organizations have positively influenced various behavioral health outcomes \cite{dunne2017review}. 


%\todo{interventions from this paper https://link.springer.com/article/10.1007/s11920-018-0969-9#Sec4}
%\todo{community level interventions from this paper https://link.springer.com/article/10.1007/s11920-018-0969-9#Sec4}


\subsection{Text Messaging as a DMH Platform for Promoting
Behavior Change and Psychological Wellbeing}
%\todo{sentence that emphasizes that text messages are a type of DMH tool}
Text messaging has emerged as a powerful DMH platform, providing an accessible and personalized means of supporting individuals in their health journeys. The broad reach of this medium extends across various demographic groups, making it particularly useful for populations that encounter obstacles in accessing more conventional digital platforms like apps and online programs \cite{muench2017more, bhattacharjee2022kind}.
In recent years, text messaging interventions have shown considerable success in supporting behavior change across various physical and mental health challenges \cite{haug2013efficacy, haug2013pre, yun2013text, suffoletto2023effectiveness, shalaby2022text, figueroa2022daily, bhandari2022effectiveness, villanti2022tailored, gipson2019effects}. For instance, \citet{villanti2022tailored} conducted a randomized controlled trial and found that a tailored text message intervention for socioeconomically disadvantaged smokers led to greater smoking abstinence rates, reduced smoking frequency, and increased desire to cease smoking compared to online resources. A similar study by \citet{liao2018effectiveness} demonstrated that frequent text messages (3-5 messages per day) could contribute to lower cigarette consumption rates among adult smokers. 
%Moreover, \citet{suffoletto2023effectiveness} highlighted the benefits of text messaging interventions for non-treatment-seeking young adults with hazardous drinking habits. Their study showed that messages providing feedback on drinking plans, alcohol consumption, and drinking limit goals resulted in sustained reductions in binge drinking days. 
Regarding other forms of behavior change, \citet{gipson2019effects} developed an intervention to improve sleep hygiene among college students, and \citet{glasner2022promising} developed an intervention to improve medication adherence and reduce heavy drinking in adults with HIV and substance use disorders. 
Other areas where text messaging has shown promise include weight management \cite{siopis2015systematic, donaldson2014text}, promoting physical activity \cite{kim2013text, smith2020text, murnane2020designing}, and enhancing vaccine uptake \cite{buttenheim2022effects, mehta2022effect}, among many others.

There has also been a surge of text-messaging services aimed at promoting psychological wellbeing~\cite{kretzschmar2019can, chikersal2020understanding, rathbone2017use, inkster2018empathy, morris2018towards, stowell2018designing, kornfield2022meeting, agyapong2020changes}. The breadth of the content they cover and the goals they seek to support are vast. Most text messaging services employ therapeutic techniques from clinical psychology — cognitive behavioral therapy \cite{willson2019cognitive}, dialectical behavior therapy \cite{linehan2014dbt}, acceptance and commitment therapy \cite{hayes2004acceptance}, and motivational interviewing \cite{hettema2005motivational} — to deliver support. Text messages can function as reminders, assisting individuals in setting aside time for leisure and physical exercise \cite{bhattacharjee2023investigating, figueroa2022daily}. They can also facilitate expressions of gratitude, nudging individuals to reflect on life's positive aspects \cite{bhattacharjee2022design}. Furthermore, text messages can share narratives that offer insights into how peers have navigated and surmounted challenging life situations, thereby aiding individuals in applying psychological strategies in their own lives \cite{bhattacharjee2022kind}. %These advancements have spurred researchers to explore the potential of AI-powered chatbots to simulate human conversations and foster self-reflection \cite{inkster2018empathy, kocielnik2018reflection, tielman2017should}. The momentum in this domain has further accelerated with the recent rise of large language models (LLMs) \cite{van2023global, kumar2022exploring}.

Several studies have leveraged the utility of text messaging as a medium to offer support for individuals grappling with various mental health challenges \cite{agyapong2020changes, levin2019outcomes, arps2018promoting, nobles2018identification, bhattacharjee2023investigating, agyapong2022text4hope}. \citet{agyapong2022text4hope} implemented such a system during the COVID-19 pandemic, dispatching daily support messages that led to decreased self-reported anxiety and stress. \citet{levin2019outcomes} combined psychoeducational content and reminders to assist those with bipolar disorder and hypertension. Meanwhile, \citet{arps2018promoting} tackled adolescent depression by sending daily texts centered around gratitude. Yet, there are several challenges pertaining to one's individual and social context that could prevent text messaging systems and other DMH tools from providing more benefits to users. We discuss these issues in the next subsection.


% Various text messaging services have been used to help people manage wellbeing during critical periods. For example, Agyapong et al.
% [1] deployed a text messaging service to help people manage their
% mental wellness during the COVID-19 pandemic. The service sent
% daily supportive messages to people along with requests to reflect
% on their stress, anxiety, and depression level. The authors found
% that their messages were able to significantly reduce self-reported
% anxiety scores among users. Levin et al . [68] delivered messages
% ranging from psychoeducational texts, reminders, and EMA queries
% to help people with bipolar disorder and hypertension manage
% their symptoms. Arps et al . [2] were able to reduce depressive
% symptoms among adolescents by sending them daily gratitude mes-
% sages. Lastly, researchers have explored the design of artificially
% intelligent chatbots that try to promote self-refection by engaging
% users in human-like conversations [50, 59, 123]




%In the past few years, text messaging services have  been quite successful in supporting behavior change for a range of physical and mental health challenges \cite{haug2013efficacy, haug2013pre, yun2013text, suffoletto2023effectiveness, shalaby2022text, figueroa2022daily, bhandari2022effectiveness, villanti2022tailored}. For example, 
%\citet{villanti2022tailored} reported through a randomized controlled trial that a tailored text message intervention for socio-economically disadvantaged young adult smokers could result in greater smoking abstinence rates,  reductions in smoking frequency, and increased desire to quit compared to online quit resources. Deployment of a similar service \cite{liao2018effectiveness} suggested that frequent delivery of messages (3--5 messages every day) can promote lower cigarette consumption rates among adult smokers. \citet{suffoletto2023effectiveness} noted the benefits of text messaging interventions in helping non-treatment-seeking young adults with hazardous drinking habits, showing that messages that provided feedback on drinking plans, alcohol consumption, and drinking limit goals resulted in long-lasting reductions in binge drinking days. \citet{glasner2022promising} developed an intervention aimed at enhancing medication adherence and reducing heavy drinking in adults with HIV and substance use disorders. Other promising avenues for text messaging applications include weight management \cite{siopis2015systematic, donaldson2014text}, physical activity promotion \cite{kim2013text, smith2020text, murnane2020designing}, vaccine uptake \cite{buttenheim2022effects, mehta2022effect}, among many others.




\subsection{Challenges of Sustaining Engagement with DMH Tools}

%Despite the well-documented benefits provided by digital mental health tools, they face lots of challenges in engaging users. The commonly cited reasons involve the delivery of generic content that people fail to see the benefits of or how they might apply in their life. To reach a broader population, interventions often try to address issues that might apply to a broad group of people, but in doing so they risk being perceived as insincere or surface-level. Additionally, the intervention contents may also have limited variety; ultimately, users end up engaging with repeated content in a short timeframe, making them possess a negative attitude toward the novelty aspect of the intervention.

%While DMH tools offer an array of benefits, their practical utility is often limited by several substantial challenges that hinder user engagement. Researchers have quantified engagement in numerous ways, such as days of sustained use, user responsiveness, and daily sessions of use \todo{cite}. To study engagement, studies have examined who is likely to initiate and sustain engagement with DMH tools, finding that individuals who are more highly motivated for behavior change tend to sustain use. Studies have examined who is likely to initiate and sustain engagement with DMH tools, finding that individuals who are more highly motivated for behavior change tend to sustain use...Studies also show that engagement tends to be low. For example, users engage with downloaded mental health apps an average of... This suggests that many individuals may not have exposure to tools sufficient to learn and apply new skills. Indeed, research shows that greater engagement also tends to predict greater levels of benefit from using a tool. 


Although there is no universal conceptualization of engagement for DMH tools \cite{zech2023integrative}, researchers have quantified engagement in several ways: days of sustained use, user responsiveness, frequency of daily sessions, etc. \cite{lipschitz2023engagement, wang2023metrics}. Regardless of the metric, studies often report low engagement rates with DMH tools \cite{baumel2019objective, lipschitz2023engagement}. \citet{baumel2019objective} reported that individuals with a mental health app installed exhibit a median daily open rate of 4.0\%, and the 15-day median retention rate stands at  3.9\%. These statistics suggest that users might access mental health apps sporadically, and engagement beyond the first 15 days is uncommon. This poses a challenge for DMH tools, as sustained engagement has been significantly related to improved mental health outcomes \cite{gan2021effect}.


%While DMH tools offer a range of benefits, their effectiveness is often undermined by challenges that limit user engagement. Researchers have defined engagement in multiple ways, such as days of sustained use, user responsiveness, and frequency of daily sessions \cite{lipschitz2023engagement, wang2023metrics}. %Previous studies have indicated that individuals with higher motivation for behavioral change may be more likely to sustain engagement. Despite this, general engagement levels are often low, implying that many users do not interact with these tools enough to derive significant benefits \cite{gan2021effect}. Some works have focused on early identification of users who may face engagement challenges, aiming to provide them with additional support, such as coaching. However, our study aims to show a more nuanced perspective, showing whether motivated individuals can disengage due to disruptions stemming from social context. In that case, the measurement of initial engagement in the first days or weeks may not be an accurate predictor of long-term usage, as life circumstances can disrupt even initially committed users.

Prior work has identified some elements of DMH tools that can cause disengagement. A common issue is the use of generic and repetitive content \cite{bhattacharjee2022design, rennick2016health, brown2014mobile}. While aiming for a wide reach, DMH interventions risk seeming insincere as generalized material may not resonate with users' unique situations \cite{bhattacharjee2023investigating, slovak2023designing}. This lack of personal relevance, compounded by limited content variety, often leads to decreased engagement due to monotony and boredom \cite{brown2014mobile}.
%A common criticism lies in the delivery of generic content \cite{bhattacharjee2022design}. Aiming at broader reach, digital interventions often take a generalist approach \cite{rennick2016health} that addresses issues broadly applicable to a diverse population. However, such interventions risk appearing insincere, as users may find the content too generic and question its relevance to their unique situations or how to apply its suggestions in daily life. \cite{bhattacharjee2023investigating, slovak2023designing}. 
%A further complication arises from the limited content variety within these digital tools \cite{bhattacharjee2022design}. Users often encounter repetitive content in DMH tools, reducing perceived novelty and creating a sense of monotony that can negatively impact engagement and lead to boredom \cite{brown2014mobile}. %The lack of variety can disengage users, potentially diminishing the tool's effectiveness and user satisfaction.
Furthermore, DMH tools face the challenge of competing with various ongoing and urgent demands for user attention \cite{muench2017more}. Even with personalized content, these tools run the risk of being ignored if they do not intervene at opportune moments \cite{bhattacharjee2022design}. 

%Several research studies have focused on identifying the ideal time window to deliver interventions~\cite{figueroa2022daily, hardeman2019systematic, gonul2019expandable, lin2006fish, rabbi2015automated, goldstein2017return, konrad2015finding, ghandeharioun2016kind}. Some studies have relied on users' digital calendars to determine when to send massages~\cite{howe2022design}. Meanwhile, other studies have relied on information about users' physical or mental state, including self-reported stress scores~\cite{paredes2014poptherapy}, physiological data (e.g., heart rate \cite{clarke2017mstress}) or activity level (e.g., walking, running, or being in a vehicle \cite{ismail2022design}). These works often utilize personalization algorithms, such as contextual multi-armed bandit algorithms \cite{figueroa2023ratings, li2010contextual}, to determine the timing and content of interventions that will have the highest likelihood of benefiting and engaging each user. By leveraging personalized algorithms and considering various contextual factors, DMH tools aim to optimize their interventions to be delivered when they are most likely to have an impact and capture the user's attention.

%\todo{review lit from https://www.jmir.org/2021/3/e24387/

%Literature has explored the impact of technology design on user engagement, but the profound role of social context has largely been unrecognized. Predominant models like the Technology Acceptance Model (TAM) \cite{marangunic2015technology, venkatesh2008technology, davis1989perceived} and the Unified Theory of Acceptance and Use of Technology (UTAUT) \cite{venkatesh2003user, dwivedi2019re} anchor their analyses on factors such as perceived usefulness and ease of use. Social influence is included to a lesser extent, but only basic elements like peer influence or subjective norms. These models frequently overlook the nuanced interplay between users and their diverse social environments: familial ties, workplace rules and responsibilities, and broader societal interactions. Given the known significance of these dynamics in influencing mental health \cite{allen2014social, lattie2020designing, slavich2023social}, our work seeks to explore their role in determining the effectiveness and receptiveness of digital interventions.

Researchers have endeavored to address these barriers and enhance engagement with DMH tools by refining features that inhibit engagement, making improvements in UI/UX, and tailoring algorithms to ensure content and delivery personalization \cite{doherty2012engagement, harding2015hci, poole2013hci, xu2013designing, howe2022design, lipschitz2023engagement}. This view toward engagement is also prominent in the design of influential technology adoption models, such as the Technology Acceptance Model (TAM) \cite{venkatesh2008technology, davis1989perceived} and the Unified Theory of Acceptance and Use of Technology (UTAUT) \cite{venkatesh2003user, dwivedi2019re}. %, or the Lived Informatics Model (LIM) \cite{epstein2015lived}. 
These models have centered on factors like perceived usefulness and ease of use. However, these factors have often been considered in isolation, neglecting the broader influences of one's surroundings. Social influence has been included but to a lesser extent, with an emphasis on elements such as peer influence and subjective norms \cite{marangunic2015technology}. Some initiatives have proactively identified individuals at risk of low engagement to implement adaptive interventions, such as providing specialized coaching or peer support to enhance their engagement \cite{arnold2019predicting, lipschitz2023engagement}. While we acknowledge that these efforts and models are important steps in improving engagement with DMH tools, there exists a lack of knowledge about the dynamics of engagement in the context of an individual's life circumstances. This gap makes it challenging to determine whether the right approach to promote engagement requires only minor adjustments in tool design or calls for a more fundamental reassessment.


%Prior research on various social and economic factors and their associations with psychological wellbeing suggests that unexplored disruptors, particularly those tied to diverse social environments like familial relationships, workplace obligations, and broader societal interactions, may demand more profound alterations in how engagement with DMH tools is perceived \cite{bhattacharjee2023investigating, alegria2018social, allen2014social, bhattacharjee2021understanding, lewis2019cultural, slavich2023social, tachtler2021unaccompanied, lattie2020designing, bhattacharjee2023integrating, slavich2020social}. 

%check the rewrite--rachel
Building on previous research that has thoroughly investigated the impact of diverse social and economic factors on mental health \cite{slavich2020social, mills2020prioritising, allen2014social}, we posit that fluctuations in social context could significantly impede the consistent and sustained engagement with DMH tools.
These disruptors of engagement are likely to be multifaceted, unpredictable, and possibly challenging to monitor \cite{bhattacharjee2022design, allen2014social, slavich2023social}. They can manifest broadly and suddenly, leading to unexpected disruptions or even causing individuals to halt engagement altogether. Traditional user studies spanning only a few days or weeks may fail to capture these disruptors \cite{bhattacharjee2022design}. Instead, the influence of these factors may only emerge over time as users with mental health concerns integrate DMH tools into their daily lives. 

In our first study, we aim to bridge this gap by extending our understanding of engagement with DMH tools. Through an eight-week-long study, we seek to identify these disruptors and provide a more comprehensive view of how these complex factors influence engagement. In our second study, we build on this knowledge by eliciting potential strategies for engaging end-users of DMH tools who are facing disruptors. 

%Therefore, it might be crucial to conceptualize tools that can assist users in overcoming challenges without disturbing tool functionality. This consideration may help maintain the user's relationship with the system for future use. Moreover, tools may be crafted to navigate specific disruptors, signifying that they can be advantageous not merely in spite of these challenges but actually for addressing them.


%These disruptors might be multifaceted,  unpredictable, and possibly difficult to monitor \cite{bhattacharjee2022design, allen2014social, slavich2023social}, impacting a substantial subset of users suddenly and leading to disruptions or even halting tool use entirely. Shorter user studies of a few days or a couple of weeks may not capture these SEDs \cite{bhattacharjee2022design}, rather they may emerge over time among users with mental health concerns as they engage a tool in daily life.
%\todo{review 2.3}










%HCI and implementation science have tried to promote engagement of DMH tools and in general technology by making adjustments in the design of a tool or tailoring algorithms to ensure personaliztion in terms of content, timing, or delivery method. In the context of technology adoption, predominant models like the Technology Acceptance Model (TAM) \cite{marangunic2015technology, venkatesh2008technology, davis1989perceived} and the Unified Theory of Acceptance and Use of Technology (UTAUT) \cite{venkatesh2003user, dwivedi2019re} have been built upon factors such as perceived usefulness and ease of use, with social influence included to a lesser extent, focusing on basic elements like peer influence or subjective norms. 


%There is a prevailing lack of understanding regarding the mechanics of engagement hampers our ability to ascertain whether the right approach involves only superficial modifications or calls for a more fundamental reassessment.
%At least in the context of DMH tools, prior works on various social contexts and their connections with psychological wellbeing indicate that there could be other aspects, particularly those related to diverse social environments: familial ties, workplace rules and responsibilities, and broader societal interactions, that have not been thoroughly explored and might require more profound changes to how we view engagement in the context of DMH tools. These experiences could be diverse and hard to predict. they affect a significant subset of users at a given moment, and stall or disrupt use of a tool in a dramatic way. It is important to think about designing tools that can help users "weather" challenges without disrupting tool use. This can keep the relationship to the system in place so they have ability to use it when they have ability in the future. Tools may also be designed so that can be helpful in navigating specific disruptors, i.e. they can be helpful *for* and not just *in spite of* disruptors.




%Research on engagement with DMH tools, as well as various other forms of technology, has largely concentrated on making minor adjustments or aligning the tool with the specific circumstances of users. Predominant models like the Technology Acceptance Model (TAM) \cite{marangunic2015technology, venkatesh2008technology, davis1989perceived} and the Unified Theory of Acceptance and Use of Technology (UTAUT) \cite{venkatesh2003user, dwivedi2019re} anchor their analyses on factors such as perceived usefulness and ease of use, with social influence included to a lesser extent, focusing on basic elements like peer influence or subjective norms. Nevertheless, there could be other aspects, particularly those related to disruptions or challenging life situations, that have not been thoroughly explored and might require more profound changes to the content, delivery timing, and method. The prevailing lack of understanding regarding the mechanics of engagement hampers our ability to ascertain whether the right approach involves only superficial modifications or calls for a more fundamental reassessment.  This perspective emphasizes the need for a thorough examination of the complex interplay between tool use and life circumstances, to ensure a more resilient and adaptive design.




%Such dynamics, integral to a user's life, can profoundly shape priorities, receptiveness to digital interventions, and even the interpretation of content.



%Comparison With Other Models of Technology and Digital Health Intervention Engagement
%}

%However,  DMH tools have to compete with many ongoing and often urgent issues for getting user attention \cite{muench2017more}, and even with a variety of personalized contents, they face the risk of being ignored, if they do not intervene users at opportune moments \cite{bhattacharjee2022design}. Many research works have attempted to identify the ideal time window to deliver interventions \cite{figueroa2022daily, hardeman2019systematic, gonul2019expandable, lin2006fish, rabbi2015automated, goldstein2017return, konrad2015finding, ghandeharioun2016kind}; some have used sensors to anticipate their physical state (e.g., heart rate) \citet{clarke2017mstress} and activity level (e.g., walking, on vehicle) \cite{ismail2022design} and deliver interventions accordingly, some have relied on user-reported scores \cite{paredes2014poptherapy}, some others have prompted users in between meetings from digital calendars \cite{howe2022design}. This line of work relies on personalization algorithms (e.g.,  contextual multi-armed bandit algorithm \cite{figueroa2023ratings, li2010contextual}) to identify the timing and content of interventions that have the highest probability to benefit and receive engagement from a particular user.


%\todo{Social and Broader issues}
%However, a major limitation in much of the HCI literature is that disengagement with DMH tools is often viewed individualistically; technologies try to identify suitable times for intervening users or deliver contents that can address their state of mind or activity. But there are numerous external barriers that can cause users to not engage with DMH tools, many of which may not be within the control of the users. These barriers can contribute to user engagement in nuanced ways that cannot be captured by the collection of individual behavioral markers. 

%\todo{adaptive algos are yet to capture external factors}
%However, a significant limitation in much of the HCI literature is that disengagement with Digital Mental Health (DMH) tools is often viewed from an individualistic perspective, focusing primarily on individual factors without considering the broader social context in which individuals live. 
%Prior research suggests that an individual's mental health and state of mind are influenced by a range of social and environmental factors, including their financial status, career prospects, and familial responsibilities \todo{cite}. 
%These external factors may significantly impact an individual's engagement with DMH tools, but their influence is often underexplored in the literature. \todo{algorithms cannot capture them}
%Recognizing the importance of understanding how these external factors impact engagement with DMH tools, we aimed to investigate the interplay between these factors and users' engagement with DMH tools.