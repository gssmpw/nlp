\section{Related Work}
\label{sec:relatedWork}

ENF variations due to load fluctuations and grid frequency control help to localize audio recordings. Grigoras's research demonstrated this by correlating ENF from audio recordings with reference ENF signals from different power grids to estimate the location of the recording \cite{grigoras2005digital}. Extensive research was conducted in grid localization using ENF by employing diverse datasets~\cite{yao2017source}. Additionally, location estimation at various scales was addressed in \cite{sarkar2019application} and \cite{garg2021}. In~\cite{hajj2015}, a machine learning system was developed to ascertain where an ENF-containing media file was recorded, even when no simultaneous ENF reference was available. Five machine learning algorithms were explored to identify the recording location of power and audio recordings obtained from ten distinct power grids in~\cite{vsaric2016improving}. The hypothesis that variations in load conditions could generate unique location-specific patterns within the ENF signal was assessed in~\cite{garg2013geo}. In \cite{li2024advanced}, an ENF region classification model, UniTS-SinSpec, was introduced within the UniTS framework, integrating a sinusoidal activation function and a spectral attention mechanism and trained on a public dataset. Addressing the complexities of inter-grid classification, field specialists have formulated methodologies to distinguish audio recordings across global power grids, exemplified by the 2016 SP Cup. This work substantially improved the forensic analysis based on ENF, fortifying the verification of the authenticity of multimedia recordings. These distinctive patterns could pinpoint the precise location within a grid where the recording was made.