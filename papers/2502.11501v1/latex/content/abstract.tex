\begin{abstract}
Multimodal large language models (MLLMs) have shown remarkable performance for cross-modal understanding and generation, yet still suffer from severe inference costs. 
Recently, abundant works have been proposed to solve this problem with token pruning, which identifies the redundant tokens in MLLMs and then prunes them to reduce the computation and KV storage costs, leading to significant acceleration without training. While these methods claim efficiency gains, critical questions about their fundamental design and evaluation remain unanswered: 
\emph{Why do many existing approaches underperform even compared to naive random token selection?
Are attention-based scoring sufficient for reliably identifying redundant tokens?
Is language information really helpful during token pruning?
What makes a good trade-off between token importance and duplication?
Are current evaluation protocols comprehensive and unbiased?} 
The ignorance of previous research on these problems hinders the long-term development of token pruning. 
In this paper, we answer these questions one by one, providing insights into the design of future token pruning methods. 
% Codes are available in the supplementary materials.


%Token pruning has gained huge popularity due to its ability to accelerate MLLMs without any training costs. However, in this paper, we study the question \emph{does token pruning really work in MLLMs?} Surprisingly, we find that X of the X published token pruning methods are weaker than very simple baselines such as random token pruning or directly reducing the resolution of images, and some default and ``well-motivated'' techniques do not bring any benefits, and even harm the efficiency of MLLMs. \emph{Codes are in the supplementary materials and will be released in Github.}
\end{abstract}