\section{Introduction}
Time-Sensitive Networking (TSN) and 5G are widely recognized as key enabling technologies for dependable (wireless) industrial networking~\cite{9299391}.
Envisioned use cases \ldash like networked control systems to govern a fleet of automated guided vehicles~(AGV)~\cite{3gpp.22.104, det6g_usecases} \rdash require ultra-reliable low latency communication, making end-to-end reliability and end-to-end latency the key performance indicators for dependability.
3GPP defines the reliability of a stream as the percentage of frames that arrive within their time constraints~\cite{3gpp.22.261}. 
However, it is crucial to note that 5G reliability solely pertains to the inner mobile networks and does not provide end-to-end reliability in TSN.
Neglecting this difference can nullify any quality-of-service~(QoS) guarantee made by the 5G system.
For instance, our simulation results show a $\qty{99.99}{\percent}$ 5G reliability plummeting to an end-to-end reliability of below $\qty{10}{\percent}$.

Time-Sensitive Networking captures a set of standards under IEEE 802.1, ranging from a general TSN bridge specification~\cite{802.1Q} to specific profiles for industrial automation~\cite{60802}.
In this work, we focus on scheduling time-triggered streams in networks with both wired and wireless elements.
Time-triggered traffic is periodic and specifies its QoS requirements through bounded end-to-end reliability, end-to-end latency, and arrival time jitter.
Compared to analytical techniques that inspect the guarantees of a given schedule (e.g., network calculus~\cite{le2001network}) or frameworks that provide network diagnostics (e.g., fault localization~\cite{10682902}), we consider the problem of synthesizing a schedule that satisfies given QoS requirements.
For this purpose, we employ the IEEE 802.1Qbv Time-Aware Shaper~(TAS)~\cite{802.1Qbv} to forward frames according to precise timetables, called gate control lists (GCLs), along each hop.
Computing GCLs that satisfy the QoS requirements of all streams is, in general, an NP-hard optimization problem.

An extensive body of research exists on synthesizing GCLs in wired TSN, e.g., with ILP solvers~\cite{10.1145/3139258.3139289}, SMT solvers~\cite{8894249}, (meta-)heuristics~\cite{nwps}, or deep learning~\cite{10228875}.
However, TAS is highly susceptible to runtime uncertainties, with timing inaccuracies in the range of mere microseconds or sporadic frame loss causing total schedule breakdowns~\cite{Craciunas2016RTNS}.
In comparison, 5G introduces stochastic packet delays with variations that are often in the range of milliseconds~\cite{downlink_example_histogram}.
This makes conventional scheduling techniques with deterministic system models unsuitable (e.g., the above~\cite{nwps,10228875,10.1145/3139258.3139289,8894249}), as they would have to rely on (non-robust) scalar or worst-case 5G channel assumptions.
In alignment with state-of-the-art solutions of related scheduling domains~\cite{infocom24_best_paper,diaz2023robust}, we therefore advocate for moving towards stochastic and robust scheduling approaches that can provide formal end-to-end QoS guarantees.

Within this context, we introduce Full Interleaving Packet Scheduling~(FIPS) as a wireless-friendly IEEE 802.1Qbv scheduler that does not rely on non-robust 5G channel assumptions.
Instead, FIPS computes 5G packet delay budgets (PDBs) based on the streams QoS requirements and the 5G link information, e.g., through packet delay histograms.
PDBs specify the target delay for the 5G system and are of the form:
``With a 5G reliability of $\qty{99.99}{\percent}$, the 5G packet delays for stream $F$ are lower- and upper-bounded by the budget interval $[\qty{3}{\ms}, \qty{15}{\ms}]$.''
Under the condition that the 5G system satisfies this requirement, FIPS extends the QoS guarantees end-to-end, for example, to:
``With an end-to-end reliability of $\qty{99.99}{\percent}$, each frame of $F$ arrives at the TSN listener with a latency below $\qty{20}{\ms}$ and jitter below $\qty{100}{\us}$.''

We summarize our main contributions as follows:

\textbf{Formal End-to-End QoS:}
FIPS computes TSN configurations that are robust against bounded runtime uncertainties, yielding (provable) end-to-end QoS guarantees as above.

\textbf{Fault Isolation:}
Errors do not cascade through the entire network but stay isolated to the affected streams.
In particular, FIPS continuously upholds the QoS of wired streams even if the 5G system can no longer sustain the 5G PDBs.

\textbf{Improved Schedulability:} 
FIPS relaxes conventional temporal isolation constraints from IEEE 802.1Qbv scheduling techniques in wired networks~\cite{nwps,Craciunas2016RTNS}.
Our evaluations show that this relaxation improves schedulability in terms of the number of wireless streams by a factor of up to $\times 45$.

The remainder of this paper is structured as follows: 
Section~\ref{sec:related_work} discusses related work.
Section~\ref{sec:background} provides background on 5G/TSN and defines our system model.
Section~\ref{sec:problem_description} captures the problems of current IEEE 802.1Qbv scheduling techniques in wireless TSN and summarizes our problem statement.
Sections~\ref{sec:robustness_and_feasibility} and Section~\ref{sec:fips} introduces the concept of robust scheduling and defines FIPS.
Finally, Section~\ref{sec:eval} contains our evaluation results and Section~\ref{sec:conclusion} concludes this work.
