\section{Computing Feasible TSN Configurations} \label{sec:fips}
We introduce Full Interleaving Packet Scheduling~(FIPS) to compute feasible TSN configurations $\Conf$, without residing to strict temporal isolation.
In the following, we assume that the CNC already allocated the 5G PDBs, as presented in Section~\ref{sec:robustness:pdb}.
We therefore focus on computing robust TSN configurations that satisfy the latency and jitter constraints of all accepted TSN streams.
We start with an overview.

\subsection{Overview of FIPS}
Given a stream set $\TT$, we follow an incremental scheduling approach that adds one stream $F_j \in \TT$ to an existing TSN configuration (initially empty) at a time.
At the core of the IEEE 802.1Qbv scheduling problem, the scheduler has to decide the transmission order of frames $f_{i_1} \prec f_{i_2} \prec \ldots \prec f_{i_n}$ at each egress port~$[u,v]$.
To reduce the scalability bottleneck of strict temporal isolation, we allow batched frame transmissions over $[u,v]$ that result in a transmission ordering of the form
\begin{equation*}
  B_1 = \{f_{i_1}, f_{i_2}, \ldots, f_{i_{k_1}}\} \prec B_2 \prec \ldots \prec B_n.
\end{equation*}
The operation of FIPS can therefore be split in two parts:
Section~\ref{sec:fips:incremental} presents how to add a new stream $F_j$ to an existing transmission ordering, based on the QoS requirements of $F_j$ and the (feasible) TSN configuration $\Conf_{j-1}$ of the previous iteration.
Section~\ref{sec:fips:configuration} then shows how to derive a robust TSN configuration $\Conf_j$ from the new transmission ordering.
FIPS accepts the stream $F_j$ if and only if $\Conf_j$ satisfies the latency and jitter constraints of $F_j$ and all previously accepted streams.

\subsection{An Incremental Heuristic for Transmission Orderings} \label{sec:fips:incremental}
Let $\TT_{j-1} = \{F_1, \ldots, F_{j-1}\}$ be the set of streams that are already feasibly scheduled under the FIPS configuration $\Conf_{j-1}$.
To add a new stream $F$, we adjust the transmission orderings
\begin{equation*}
  B_1^k \prec B_2^k \prec \ldots \prec B_n^k
\end{equation*}
at each port $[\v{k}{F}, \v{k+1}{F}] \in \path{F}$ by inserting all frames $f \in F$ that are sent within one hypercycle.
We start by \textit{inserting} the frames in the transmission ordering as a new batch, and then check if the batch should be \textit{merged} with its immediate transmission predecessor or successor. 
Thus, the transmission ordering encodes the scheduling decisions, which is later used in Section~\ref{sec:fips:configuration} to derive a robust TSN configuration.

\textbf{Inserting Frames:}
For each frame $f \in F$, we compute a lower bound for the earliest possible transmission at $\v{k}{F}$ with
\begin{equation*}
  \phi([\v{k}{F}, \v{k+1}{F}], f) = \release{f} + \sum_{i = 1}^{k-1} \dpdbmax{[\v{i}{F}, \v{i+1}{F}], f}.
\end{equation*}
Similarly, let $\S{[\v{k}{F}, \v{k+1}{F}], B_i^k}$ denote the transmission start of the batch $B_i^k$ under the previous TSN configuration $\Conf_{j-1}$.
FIPS selects the maximum $i \in \{1, \ldots, n\}$ with
\begin{equation} \label{eq:insertion:1}
  \S{[\v{k}{F}, \v{k+1}{F}], B_i^k} \leq \phi([\v{k}{F}, \v{k+1}{F}], f)
\end{equation}
and insert $f$ in the transmission ordering as $B_i^k \prec f \prec B_{i+1}^k$.
Moreover, to avoid inconsistent transmission orderings, we fix the ordering $B_i^k \prec f$ for all consecutive transmission hops where $B_i^k$ and $f$ share the same FIFO queue.

\begin{table}
  \centering
  \caption{Policy to avoid FIFO inconsistencies.} \label{tab:inconsistency}
  \begin{tabular}{rccc}
    \toprule
     & $[T_1^{DS}, B^{NW}]$ & $[B^{NW}, B_1]$ & $[B_1, L_1]$ \\
    \midrule
    $\S{[u,v], f_1}$ & $\qty{0}{\ms}$ & $\qty{14}{\ms}$ & $\qty{14.01}{\ms}$ \\
    $\phi([u,v], f)$ & $\qty{1}{\ms}$ & $\qty{11}{\ms}$ & $\qty{11.01}{\ms}$ \\[1.5mm]
    Ordering of (\ref{eq:insertion:1}) & $f_1 \prec f$ & $\;\;f\; \prec \, f_1$ & $\;\;f\; \prec \, f_1$ \\
    FIFO Policy & $f_1 \prec f$ & $f_1 \prec f$ & $f_1 \prec f$ \\
    \bottomrule
  \end{tabular}
\end{table}

Table~\ref{tab:inconsistency} shows the calculations for inserting a new stream $F$ into a transmission ordering that already contains $F_1$ from our previous examples (e.g., see Fig.~\ref{fig:problem}).
$F$ has the same specification as $F_1$, but has a phase offset of $\qty{1}{\ms}$ and a smaller 5G PDB with $\dpdbmax{} = \qty{10}{\ms}$.
Solely applying~(\ref{eq:insertion:1}) would imply that the frame $f \in F$ has to overtake $f_1 \in F_1$, which is impossible if both frames share the same FIFO queues along $[T_1^{DS}, B^{NW}, B_1, L_1]$.
We therefore prioritize the already accepted $F_1$ by fixing the transmission ordering $f_1 \prec f$ for all consecutive transmission hops.

\textbf{Merging Batches.}
For wired streams $F$, there is no need for batching and the ordering of the previous step is returned. 
Otherwise, let $\v{k}{F}$ be the TSN translator that follows the 5G link.
With a 5G PDB of $\dpdbfull{}$, FIPS opts to de-jitter frames $f \in F$ at $\v{k}{F}$ for a minimum duration of 
\begin{equation*}
  \sub{\dpdbmax}{\dpdbmin}{[\v{k-1}{F}, \v{k}{F}], f}.
\end{equation*}
Recall that this duration can be in the range of milliseconds, where a strict transmission isolation leads to a major scalability bottleneck.
Hence, when $B_i^k \prec f \prec B_{i+1}^k$ is the transmission ordering of the previous step, FIPS considers three options:
merge $f$ with $B_i^k$, merge $f$ with $B_{i+1}^k$, or return the ordering as it is.
FIPS derives a robust TSN configuration, as described next, for all three options and verifies the QoS guarantees for all streams in $\TT_{j-1} \cup \{F\}$.

\subsection{Robust TSN Configurations from Transmission Orderings} \label{sec:fips:configuration}
Next, we present how to efficiently derive a FIPS configuration $\Conf$ from a given transmission ordering and prove that every FIPS configuration is robust (Theorem~\ref{theorem:fips_robust}).
By checking the latency (\ref{eq:latency}) and jitter (\ref{eq:jitter}) constraints for each frame, the procedure returns successfully if and only if $\Conf$ is feasible.

We start by defining the accumulated PDBs of a frame batch $B_i$.
For Ethernet links, the transmission delay is the summed delays of all frames $f \in B_i$, i.e.,
\begin{equation*}
  \dpdbtransmin{[u,v], B_i} = \dpdbtransmax{[u,v], B_i} = \sum_{f \in B_i} \frac{\size{f}}{\drate{[u,v]}},
\end{equation*}
whereas the total delay $\dpdb{[u,v], B_i}$ includes the propagation delay $\dprop{[u,v]}$ and the processing delay $\dproc{v}$ once.
For 5G links, frequency-division multiplexing allows the simultaneous transmission of all frames in $B_i$, under the delay conditions captured by the respective histograms;
hence, $\dpdb{[u,v], B_i}$ is set to the minimum and maximum bounds of the individual 5G PDBs $\dpdb{[u,v], f}$, for $f \in B_i$.

FIPS continues by determining the earliest possible transmission start $\S{[u,v], B_i}$ for each link $[u,v]$ and each batch $B_i$ by enforcing the following constraints C1--C3: 

\begin{figure}
  \centering
  \tikzset{
  diagonal fill/.style 2 args={fill=#2, path picture={
  \fill[#1, sharp corners] (path picture bounding box.south west) -|
			   (path picture bounding box.north east) -- cycle;}},
  reversed diagonal fill/.style 2 args={fill=#2, path picture={
  \fill[#1, sharp corners] (path picture bounding box.north west) |- 
			   (path picture bounding box.south east) -- cycle;}}
  }
  \begin{tikzpicture}
      \def\offset{0.1}
      \def\scale{0.35}
      \def\steps{2}
      \def\y{-0.75}

      \draw (0,\y*0) edge[thick,->] node[pos=0,left,font=\scriptsize,anchor=east,xshift=-2pt,yshift=8pt] {$[T^{DS}_1, B^{NW}]$} (7,\y*0);
      \draw (0,\y*1) edge[thick,->] node[pos=0,left,font=\scriptsize,anchor=east,xshift=-2pt,yshift=8pt] {$[T^{DS}_2, B^{NW}]$} (7,\y*1);
      \draw (0,\y*2) edge[thick,->] node[pos=0,below,font=\scriptsize,anchor=east,xshift=-2pt,yshift=8pt] {$B^{NW}$ (PSFP)} (7,\y*2);
      \draw (0,\y*3) edge[thick,->] node[pos=0,below,font=\scriptsize,anchor=east,xshift=-2pt,yshift=8pt] {$[B^{NW}, B_1]$} (7,\y*3);
      \draw (0,\y*4) edge[thick,->] node[pos=0,below,font=\scriptsize,anchor=east,xshift=-2pt,yshift=8pt] {$[T_3, B_1]$} (7,\y*4);
      \draw (0,\y*5) edge[thick,->] node[pos=0,below,font=\scriptsize,anchor=east,xshift=-2pt,yshift=8pt] {$[B_1, L_1]$} (7,\y*5);
      \draw (0,\y*6) edge[thick,->] node[pos=0,below,font=\scriptsize,anchor=east,xshift=-2pt,yshift=8pt] {$[B_1, L_2]$} (7,\y*6);

      % T_1 -> B^NW
      \draw[rounded corners=1mm, fill=f1] (\scale*0+\offset,\y*0) rectangle ++(\scale*1.5,0.4) node[pos=0.5,font=\scriptsize] {$f_1$};
      \draw (\scale*1.5+\offset,\y*0+0.5) edge[thick,-{Latex[length=1.5mm]}] node[above=1.5mm,font=\scriptsize] {$tx$} (\scale*1.5+\offset,\y*0);

      % T_2 -> B^NW
      \draw[rounded corners=1mm, fill=f2] (\scale*0+\offset,\y*1) rectangle ++(\scale*1.5,0.4) node[pos=0.5,font=\scriptsize] {$f_2$};
      \draw (\scale*1.5+\offset,\y*1+0.5) edge[thick,-{Latex[length=1.5mm]}] node[above=1.5mm,font=\scriptsize] {$tx$} (\scale*1.5+\offset,\y*1);

      % B^NW PSFP
      \draw[pattern=north east lines] (0,\y*2) rectangle ++(\scale*4+\offset,0.5);
      \draw[diagonal fill={f2}{f1}] (\scale*4+\offset,\y*2) rectangle ++(\scale*4,0.5);  
      \draw[pattern=north east lines] (\scale*8+\offset,\y*2) rectangle ++(\scale*10.5+\offset,0.5);

      % B^NW -> B_1
      \draw[rounded corners=1mm, diagonal fill={f2}{f1}] (\scale*4+\offset,\y*3) rectangle ++(\scale*4.5,0.4) node[pos=0.5,font=\scriptsize] {$\{f_1,f_2\}$};
      \draw (\scale*8.5+\offset,\y*3+0.5) edge[thick,-{Latex[length=1.5mm]}] node[above=1.5mm,font=\scriptsize] {$tx$} (\scale*8.5+\offset,\y*3);
      % T_3 -> B_1
      \draw[rounded corners=1mm, fill=f3] (\scale*8.5+\offset,\y*4) rectangle ++(\scale*1.5,0.4) node[pos=0.5,font=\scriptsize] {$f_3$};
      \draw (\scale*10+\offset,\y*4+0.5) edge[thick,-{Latex[length=1.5mm]}] node[above=1.5mm,font=\scriptsize] {$tx$} (\scale*10+\offset,\y*4);

      % B_1 -> L_1
      \draw[rounded corners=1mm, fill=f1] (\scale*8.5+\offset,\y*5) rectangle ++(\scale*1.5,0.4) node[pos=0.5,font=\scriptsize] {$f_1$};
      \draw (\scale*10+\offset,\y*5+0.5) edge[thick,-{Latex[length=1.5mm]}] node[above=1.5mm,font=\scriptsize] {$tx$} (\scale*10+\offset,\y*5);

      % B_1 -> L_2
      \draw[rounded corners=1mm, fill=f2] (\scale*8.5+\offset,\y*6) rectangle ++(\scale*1.5,0.4) node[pos=0.5,font=\scriptsize] {$f_2$};
      \draw (\scale*10+\offset,\y*6+0.5) edge[thick,-{Latex[length=1.5mm]}] node[above=1.5mm,font=\scriptsize] {$tx$} (\scale*10+\offset,\y*6);
      \draw[rounded corners=1mm, fill=f3] (\scale*10.1+\offset,\y*6) rectangle ++(\scale*1.5,0.4) node[pos=0.5,font=\scriptsize] {$f_3$};
      \draw (\scale*11.6+\offset,\y*6+0.5) edge[thick,-{Latex[length=1.5mm]}] node[above=1.5mm,font=\scriptsize] {$tx$} (\scale*11.6+\offset,\y*6);
  \end{tikzpicture}
  \caption{FIPS configuration when batching $\{f_1, f_2\}$ at $[B^{NW}, B_1]$ and choosing $f_2 \prec f_3$ at $[B_1, L_2]$.} \label{fig:fips}
\end{figure}


\textbf{C1) Sequential Transmissions.}
For each frame $f \in B_i$, the transmission start $\S{[u,v], B_i}$ is deferred until the latest arrival of $f$ at the bridge $u$, i.e.,
\begin{equation*}
  \S{[u,v], B_i} \geq \Rmax{u, f}.
\end{equation*}
In Fig.~\ref{fig:fips}, the transmission of $\{f_1, f_2\}$ via $[B^{NW}, B_1]$ has to wait until after the latest possible arrival of $f_1$ and $f_2$.
In contrast, for the first transmission hop of a frame, $\Rmax{T_1^{DS}, f_1}$ is set to the earliest release time of $f_1$ by the talker.

\textbf{C2) Transmission Ordering.}
If $B_i$ is not the first batch (within the hypercycle) to be transmitted over $[u,v]$, its transmission is deferred until $B_{i-1}$ is fully transmitted, i.e.,
\begin{equation*}
  \S{[u,v], B_i} \geq \add{\S}{\dpdbmax}{[u,v], B_{i-1}}.
\end{equation*}
For example, the transmission ordering $f_2 \prec f_3$ implies in Fig.~\ref{fig:fips} that the transmission of $f_3$ via $[B_1, L_2]$ can only start after the transmission of $f_2$ has finished.
 
\textbf{C3) Batch Fault Isolation.}
For each frame $f \in B_i$, it must be ensured that $f$ takes its intended transmission slot over the subsequent hop $[v,w]$.
Formally, let $B_j'$ denote the frame batch of $f$ at $[v,w]$.
To ensure $f$ never takes the slot of $B_{j-1}'$, the transmission start of $\S{[u,v], B_i}$ is delayed so that $f$ never arrives at $v$ before the transmission of $B_{j-1}'$ has finished, i.e.,
\begin{equation*}
  \S{[u,v], B_i} \geq \add{\S}{\dpdbmax}{[v,w], B_{j-1}'} - \dpdbmin{[u,v], f}.
\end{equation*}
For example, Fig.~\ref{fig:fips} delays $\S{[T_3, B_1], f_3}$ to ensure $f_3$ never takes $f_2$'s transmission slot over the subsequent link, even if $f_2$ is discarded by PSFP for violating its 5G PDB.
Also note that, in a multi-queue setting, C3 is only necessary if $f_2$ and $f_3$ share the same queue at $[B_1, L_2]$.
Moreover, C3 directly implies that frames are transmitted in a FIFO manner.
 
\textbf{Robust TSN Configuration.}
C1--C3 define a natural recursion order where the transmission start $\S{[u,v], B_i}$ depends on, at most, three earlier transmissions.
In case a recursion cycle is detected, the procedure is aborted;
otherwise, the GCL at link $[u,v]$ is derived as $\GCL([u,v]) = \{[o_i, c_i]\}_{i=1}^n$ with
\begin{equation*}
  o_i = \S{[u,v], B_i} \quad \text{and} \quad c_i = \add{\S}{\dpdbmax}{[u,v], B_i}.
\end{equation*}
Similarly, the PSFP configuration $\R{}$ is set for each frame $f$ and each bridge $v \in \path{f}$ with
\begin{align*}
  \Rmin{v, f} &= \S{[u,v], B_i} + \dpdbmin{[u,v], f}, \quad \text{and} \\
  \Rmax{v, f} &= \add{\S}{\dpdbmax}{[u,v], B_i},
\end{align*}
which is the earliest and latest arrival of $f$ at $v$ under FIPS.

With $\Conf = (\GCL, \R{})$, it is straightforward to check if the latency~(\ref{eq:latency}) and jitter~(\ref{eq:jitter}) requirements are satisfied for all streams. 
As detailed before, FIPS therefore accepts a new stream $F$ if and only if $\Conf$ satisfies the latency and jitter constraints of $F$ and all previously accepted streams.
Finally, we show that FIPS is provably robust (Theorem~\ref{theorem:fips_robust}) and thereby provides formal end-to-end QoS guarantees (Corollary~\ref{corollary:fips_feasibility}).

\begin{theorem} \label{theorem:fips_robust}
  Every TSN configuration $\Conf = (\GCL, \R{})$ derived by FIPS robustly schedules all TSN streams $F \in \TT$.
\end{theorem}
\begin{proof}
  The proof is given in Appendix~\ref{appendix:theorem_proof}.
\end{proof}

\begin{corollary} \label{corollary:fips_feasibility}
  Every TSN configuration $\Conf = (\GCL, \R{})$ derived by FIPS feasibly schedules all accepted TSN streams.
\end{corollary}
\begin{proof}
  By Theorem~\ref{theorem:feasibility}, feasibility is induced by \textit{(i)}--\textit{(iii)}.
  FIPS satisfies
  \begin{enumerate*}[label=(\roman*)]
    \item with Theorem~\ref{theorem:fips_robust}, 
    \item by allocating 5G PDBs as detailed in Section~\ref{sec:robustness:pdb}, and
    \item by accepting only streams for which $\Conf$ satisfies the latency and jitter requirements.
  \end{enumerate*}
\end{proof}
