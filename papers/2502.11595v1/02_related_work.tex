\section{Related Work} \label{sec:related_work}
Scheduling in IEEE 802.1Qbv is closely related to other NP-hard optimization problems.
We therefore review literature both within and outside of the TSN scheduling domain.

\subsection{Scheduling in Time-Sensitive Networks}
System components in wired Time-Sensitive Networks are often modelled to be strictly deterministic to configure the IEEE 802.1Qbv GCLs~\cite{nwps,10228875,10.1145/3139258.3139289,8894249}.
Approaches that consider uncertain runtime behavior most commonly focus on time synchronization errors~\cite{10228980,10144237,Craciunas2016RTNS}, sporadic frame loss~\cite{Craciunas2016RTNS,10.1145/3458768,FENG2022102381,9488750}, or permanent link failures~\cite{9488750,8607244,8474201,reusch2022dependability}.
While these solutions remain applicable in wired network partitions, none of them are designed to cope with non-negligible 5G packet delay variations in the range of milliseconds.

Out of the above solutions, \cite{nwps} and~\cite{Craciunas2016RTNS} are most closely related to our approach and considered as state-of-the-art in wired TSN.
The authors strictly separate frame transmissions to isolate potential transmission faults and can account for bounded clock skew.
While their temporal isolation constraints can be extended to cover 5G packet delay variations, our measurements show that these extensions do not scale in terms of the number of accepted wireless streams.
FIPS improves schedulability by allowing a controlled batching of streams.

The approaches in \cite{9212049} and \cite{9940254} aim to advance the integration of 5G and TSN by allowing the IEEE 802.1Qbv scheduler to allocate 5G resource blocks for wireless streams.
However, they rely on scalar (e.g., average, median, or maximum) 5G packet delays, for which we demonstrate that they cannot provide any end-to-end QoS guarantees. 
Moreover, we argue that 5G-internal resource management (e.g., resource allocation~\cite{8736403}, link adaptation~\cite{9488790}, and QoS provisioning in 5G vRAN~\cite{infocom24_5gvran}) remains a complementary task by itself.
Compared to requiring the 5G system to expose its internal state, we therefore design FIPS to work with 5G PDBs.

More recently, the authors of~\cite{10682834} recognized the problem of cascading faults under non-robust TSN configurations.
They propose a shielding mechanism that discards frames with unexpected delays if they endanger the latency guarantees of other streams.
Complementary to their approach, FIPS computes TSN configurations with an increased tolerance for delay variations (i.e., as captured by the 5G PDBs), preventing excessive frame loss caused by such shielding mechanisms.

\subsection{Scheduling with Uncertainty in Related Domains}
Outside of the TSN literature, state-of-the-art scheduling solutions in multi-hop networks provide remarkable near-optimal latency and reliability guarantees~\cite{6155625,infocom24_best_paper}.
However, they often assume high-cost forwarding buffers, e.g., one dedicated buffer per stream per bridge~\cite{6155625}.
This makes them unsuitable for scheduling in TSN, where each egress port is equipped with a maximum of eight FIFO queues.

Job-shop scheduling covers a vast research domain that encounters similar obstacles when transitioning from deterministic to uncertain task durations;
a survey is provided in~\cite{XIONG2022105731}.
For instance, D{\'\i}az et al.~\cite{diaz2023robust} showed that unattuned (e.g., scalar) approaches are not robust against runtime uncertainty.
In alignment with their observations, FIPS is designed to compute TSN configurations that are robust against bounded uncertainty intervals, as captured by the 5G PDBs.

