\section{Problem Description} \label{sec:problem_description}
The integration of the 5G system in TSN as a logical 5G-TSN bridge enables IEEE 802.1Qbv scheduling in networks with wired and wireless network elements.
However, logical 5G-TSN bridges and wired TSN bridges differ significantly in their packet delay characteristics (see Fig.~\ref{fig:delays}).
Next, we show how this difference can effectively nullify the end-to-end QoS of streams and how it can induce a central bottleneck in scalability.
In particular, we argue that current IEEE 802.1Qbv scheduling approaches fall into the following two categories.

\textbf{Non-Robust Approaches}
are unable to capture 5G packet delay variations.
This can lead to unintended frame reordering and cascading QoS violations that spread throughout the entire network.
Most prominently, this category includes scheduling algorithms that consider scalar values for the 5G packet delays \ldash e.g., the average, median, or maximum delay of a 5G packet delay histogram (e.g., \cite{9212049, 9940254}).

This problem is illustrated in Fig.~\ref{fig:problem}, where three streams $F_1, F_2$, and $F_3$ (of the same priority) traverse the network as shown on the left.
The talkers $T_1^{DS}$ and $T_2^{DS}$ start the transmission of the frames $f_1 \in F_1$ and $f_2 \in F_2$ at times $\qty{0}{\ms}$ and $\qty{3}{\ms}$, respectively. 
With the 5G packet delay histograms of Fig.~\ref{fig:delays}(b), $f_1$ and $f_2$ can arrive at the NW-TT $B^{NW}$ within the intervals $[\qty{4}{\ms}, \qty{14}{\ms}]$ and $[\qty{7}{\ms}, \qty{17}{\ms}]$.

Even considering the latest possible arrival time ($\qty{14}{\ms}$ and $\qty{17}{\ms}$) will result in a schedule that breaks down irreparably after a short period of operation.
This is caused by the overlapping arrival intervals at $B^{NW}$, where a seemingly small probability of $f_2$ arriving before $f_1$ (denoted by $f_2 \prec f_1$ in Fig.~\ref{fig:problem}) has the following consequences:
\begin{enumerate}[label=(\roman*)]
  \item $f_1$ is forwarded later than intended and also misses its subsequent transmission slot at $[B_1, L_1]$,
  \item $f_2$ is forwarded earlier than intended and can steal the transmission slot of $f_3$ at $[B_1, L_2]$.
\end{enumerate}
Thus, even in such simplified settings, a single frame reordering has cascading effects that spread across multiple streams.
Our evaluations in Section~\ref{sec:eval} show that these effects cause a $\qty{99.99}{\percent}$ reliability in the inner 5G network to plummet below $\qty{10}{\percent}$ when looking at the end-to-end communication in TSN.

\textbf{Strict Transmission Isolation (STI).}
An apparent fix to the problems of non-robust approaches would be to allow no overlap in the arrival intervals of $f_1$ and $f_2$ at $B^{NW}$.
STI is often employed implicitly for IEEE 802.1Qbv scheduling in wired networks~\cite{nwps, Craciunas2016RTNS}.
However, compared to the small delay variations in wired networks, 5G packet delay variations are three orders of magnitude larger (cf. Fig.~\ref{fig:delays}).
Thus, STI would defer the initial transmission of $f_2$ by an additional $\qty{7}{\ms}$ to avoid an overlapping arrival interval with $f_1$ at $B^{NW}$.
This already prohibits $f_2$ from being accepted if its end-to-end latency must be below $\qty{20}{\ms}$.
Employing STI would therefore introduce a major scalability bottleneck in the number of schedulable wireless streams.

In this work, we aim to devise an IEEE 802.1Qbv scheduling technique that provides formal end-to-end QoS guarantees for each TSN stream (according to Definition~\ref{def:reliability}) without suffering the same scalability bottleneck as STI.
To quantify the scalability issue, we consider the number of schedulable TSN streams as our optimization objective.
