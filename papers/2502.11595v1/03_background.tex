\begin{figure}[b]
  \centering
  \resizebox{0.74\columnwidth}{!}{%
  \begin{tikzpicture}
    \node[draw, trapezium, align=center, minimum width=3.5cm, minimum height=0.55cm, trapezium stretches body] (psfp) at (0,0) {PSFP $\R{}$};
    \draw (0,0.6) edge[->,thick] node[pos=0,yshift=2mm] {ingress port} (psfp);

    \node[align=center,black!70] (cnc) at (3.5,1) {configured by the CNC};
    \draw[black!50] (cnc) edge[->,dashed] (0.8,0);
    \draw[black!50] (cnc) edge[->,dashed] (4,0.2);

    \node[draw,pattern=queue,minimum width=0.6cm,minimum height=1.5cm,rounded corners=2pt] (q0) at (-1.45,-1.5) {};
    \node[draw,pattern=queue,minimum width=0.6cm,minimum height=1.5cm,rounded corners=2pt] (q1) at (-0.55,-1.5) {};
    \node (dots) at (0.45,-1.5) {\Large $\cdots$};
    \node[below=0mm of dots] {\scriptsize $\leq 8$ queues};
    \node[draw,pattern=queue,minimum width=0.6cm,minimum height=1.5cm,rounded corners=2pt] (q2) at (1.45,-1.5) {};

    \draw (q0) edge[<-,thick] node[right,pos=0.6] {0} (\tikztostart |- psfp.south);
    \draw (q1) edge[<-,thick] node[right,pos=0.6] {1} (\tikztostart |- psfp.south);
    \draw (q2) edge[<-,thick] node[right,pos=0.6] {7} (\tikztostart |- psfp.south);

    \node[draw,below=2mm of q0,rounded corners,minimum width=0.6cm,fill=black!60] (g0) {c};
    \node[draw,below=2mm of q1,rounded corners,minimum width=0.6cm,fill=black!60] (g1) {c};
    \node[draw,below=2mm of q2,rounded corners,minimum width=0.6cm,fill=white] (g2) {o};
    \node (egress) at (0,-3.7) {egress port};

    \draw (q0) edge[thick] (g0);
    \draw (g0) edge[->,thick,out=-90,in=90] (egress);
    \draw (q1) edge[thick] (g1);
    \draw (g1) edge[->,thick,out=-90,in=90] (egress);
    \draw (q2) edge[thick] (g2);
    \draw (g2) edge[->,thick,out=-90,in=90] (egress);

    \node (gcl0) at (4,0) {TAS $\GCL$};
    \node (gcl1) at (4,-1.6) {
      \begin{tabular}{ll}
	\toprule
	time & gates\\
	\midrule
	$[0,t_0)$ & oo$\ldots$o\\
	$[t_0,t_1)$ & cc$\ldots$o\\
	$[t_1,t_2)$ & cc$\ldots$o\\
	$[t_2,H)$ & co$\ldots$c\\
      \end{tabular}
    };
    \node[fill=black!20,inner ysep=1pt,inner xsep=2pt] (gcl2) at (4,-2.18) {
      \begin{tabular}{cc}
	$[t_1,t_2)$ & cc$\ldots$o\\
      \end{tabular}
    };
    \node[draw,black!60,rounded corners,fit={(gcl0) (gcl1)}, inner sep=1pt] {};

    \draw (gcl2) edge[dashed,thick,black!50,out=180,in=0] (g2);
    \draw (g2) edge[dashed,thick,black!50] (g1);
    \draw (g1) edge[dashed,thick,black!50] (g0);

    \node[below=0.6cm of gcl2,xshift=7mm,align=left] {o = open\\c = closed};
  \end{tikzpicture}
  }
  \caption{Port-to-port model of TSN bridges with Per-Stream Filtering and Policing~(PSFP) and the Time-Aware Shaper~(TAS).} \label{fig:tsn_model}
\end{figure}


\section{Background and System Model} \label{sec:background}
Next, we provide an overview of relevant standards for Time-Sensitive Networking~(TSN) and its proposed integration with 5G.
We then introduce our system model and define the end-to-end QoS requirements for streams in wireless TSN.

\subsection{Time-Sensitive Networking} \label{sec:tsn}
TSN is a set of IEEE 802.1 standards to provide real-time communication in Ethernet networks with bounded latency and low packet delay variations for each stream.
Streams originate at exactly one talker, traverse multiple bridges, and terminate at one or more listeners.
To schedule time-triggered streams, we consider a centralized network controller~(CNC)~\cite{8514112} with a global view of the network and streams to configure the TSN mechanisms (as described next) at each bridge.
Moreover, we assume a bounded clock skew between network devices~\cite{802.1AS} and that talkers are synchronized to the network schedule.

For time-triggered streams, i.e., periodic traffic with fixed frame sizes, two TSN mechanisms are of central interest: the Time-Aware Shaper~(TAS)~\cite{802.1Qbv} and Per-Stream Filtering and Policing~(PSFP)~\cite{802.1Qci}.
Fig.~\ref{fig:tsn_model} illustrates the operation of TAS and PSFP for a bridge.
When a frame $f$ arrives at the bridge $u$, each frame is first mapped to its corresponding TSN stream via its source/destination MAC address, VLAN identifier, and priority code point~(PCP) value.
PSFP then verifies that the frame adheres to the stream specification, i.e., by arriving within an expected interval $\R{u, f} = [\text{rx}^{min}, \text{rx}^{max}]$. 
If a violation is detected, the frame is discarded; otherwise the frame is enqueued.
To determine the queue where $f$ is appended, the bridge consults its forwarding table (specifying the egress port) and the 3-bit PCP value in the frame's header (specifying one out of eight FIFO queues per egress port).

When the egress port is free, TAS uses gates (one per queue) to select the next frame for transmission.
Whether the gates are currently open or closed is determined by the active entry in the gate control list~(GCL).
While finding suitable GCLs is a computationally hard problem that is done by the CNC in advance, they simplify the transmission selection at runtime:
TAS selects the frame at the head of the highest-priority queue that is non-empty and has an open gate.
In the case of Fig.~\ref{fig:tsn_model}, the third GCL entry is currently active and only opens the gate of the highest-priority queue; 
this allows the first frame of the queue to be forwarded without any delays from lower-priority traffic.
Finally, note that the GCL is also periodic and repeats after the least common multiple of the streams' periods; we call this the hypercycle $H$ of the TSN configuration.

\begin{figure}[b]
  \centering
  \resizebox{\columnwidth}{!}{%
  \begin{tikzpicture}
      \node[draw,minimum width=5.75cm,minimum height=3.5cm,rounded corners] at (2.5,0.25) {};
      \node[draw,minimum width=5cm,fill=white] (5G) at (2.5,2) {Logical 5G-TSN Bridge};

      \node[draw,fill=black!20,minimum height=0.8cm,] (dstt1) at (0,1) {DS-TT};
      \node[draw,minimum height=0.8cm] (ue1) at (0.97,1) {UE};
      \draw (-1,1) edge[thick] (dstt1);

      \node[draw,fill=black!20,minimum height=0.8cm] (dstt2) at (0,-0.9) {DS-TT};
      \node[draw,minimum height=0.8cm] (ue2) at (0.97,-0.9) {UE};
      \draw (-1,-0.9) edge[thick] (dstt2);

      \node[draw,minimum height=0.8cm] (gnb) at (2.5,0.05) {RAN};
      \node[draw,minimum height=0.8cm] (upf) at (4,0.05) {CN};
      \node[draw,fill=black!20,minimum height=0.8cm] (nwtt) at (5,0.05) {NW-TT};

      \draw (ue1) edge[thick,dashed,out=0,in=140] (gnb);
      \draw (ue2) edge[thick,dashed,out=0,in=-140] (gnb);
      \draw (gnb) edge[thick] (upf);
      \draw (nwtt) edge[thick] (6,0.05);

      \node[draw, fill=black!15, shape=cloud, aspect=1.7, inner sep=1pt, align=center] (TSN1) at (7.4,0.05) {TSN-capable\\Edge Platform};
      \node[draw, fill=black!15, shape=cloud, aspect=1.7, align=center] (TSN2) at (-2,1) {Robot\\\scriptsize(TSN System)};
      \node[draw, fill=black!15, shape=cloud, aspect=1.7, align=center] (TSN3) at (-2,-0.9) {Robot\\\scriptsize(TSN System)};

      \node[draw, align=center, fill=black!30] (CNC) at (2.5,3.25) {TSN Centralized\\Network Controller};

      \draw (CNC) edge[<->,black!50,very thick] (5G);
      \draw (CNC) edge[<->,black!50,very thick,out=0,in=110] (TSN1);
      \draw (CNC) edge[<->,black!50,very thick,out=180,in=70] (TSN2);
      \draw (CNC) edge[<->,black!50,very thick,out=180,in=70] (TSN3);
  \end{tikzpicture}
  }
  \caption{Integration of 5G and TSN, as standardized by~\cite{3gpp.23.501}.}\label{fig:5g_tsn_bridge}
\end{figure}


\subsection{The 5G System as a Logical TSN Bridge} \label{sec:5g_tsn}
To provide the same TSN mechanisms in networks that require both wired and wireless elements, e.g., free-moving robots in an industrial setting, 3GPP recently standardized an architecture to expose the 5G system as a logical TSN bridge~\cite{3gpp.23.501}. 
An overview of this integration is provided in Fig.~\ref{fig:5g_tsn_bridge}:
The TSN systems on the left show two moving robots, containing a single TSN end-station or a more complex internal network.
Each robot has a user equipment~(UE) to connect wirelessly to the 5G radio access network~(RAN).
When data is sent to the edge computing platform on the right, the RAN forwards the frames via the 5G core network (CN).

Instead of exposing the entire internal state (e.g., 5G resource allocation or session management), the 5G system provides device-side~(DS-TT) and network-side TSN translators~(NW-TT).
From the perspective of the TSN controller, the 5G system thus appears as a logical TSN bridge that supports the TSN mechanisms, as in Section~\ref{sec:tsn}.
Still, Fig.~\ref{fig:delays} shows an evident non-functional difference in the port-to-port delays of wired bridges and logical 5G-TSN bridges:
For wired bridges, the port-to-port delay corresponds to the processing delay and exhibits only small delay variations below $\qty{1}{\us}$.
For logical 5G-TSN bridges, however, the uplink port-to-port delay covers the entirety from the DS-TT transmission start until the NW-TT reception time, yielding millisecond delays and millisecond delay variations (a difference by three orders of magnitude).

The scheduler must account for this difference when synthesizing TSN configurations.
We therefore assume that the 5G system reports port-to-port delay histograms to the CNC, as in Fig.~\ref{fig:delays}(b).
Different streams can be served with different histograms.
For this work, we assume that the histograms are stationary and stay valid for any TSN scheduling decision.
However, even in this idealized setting, we show that existing approaches cannot provide any QoS guarantees or do not scale.

\section{Support for Selection Predicates}
\label{sec:histograms}

\system can support arbitrary selection predicates on a relation. As long as we can provide $\ell_p$-norms on the degree sequences of the join columns for those tuples that satisfy the selection predicate, \system can use these norms in the statistics constraints. In the following, we discuss the case of equality and range predicates, and their conjunction and disjunction; \texttt{IN} and \texttt{LIKE} predicates can be accommodated using data structures like for \safebound~\cite{SafeBound:SIGMOD23}.

As data structures to support predicates, \system uses simple and effective adaptations of existing data structures in databases: Most Common Values (MCVs) and histograms. Yet instead of a count for each MCV or histogram bucket, \system keeps a set of $\ell_p$-norms on the degree sequences of the tuples for that MCV or histogram bucket. The simplicity and ubiquity of these data structures make \system easy to incorporate in database systems.

In the following, let a relation $R({\bf X},{\bf Y},A)$ with join attributes ${\bf X} = \{X_1,\ldots,X_n\}$, a predicate attribute $A$, and other attributes ${\bf Y}$.

\paragraph{Equality Predicate.} For each MCV $a$ of $A$, we compute $\ell_p$-norms for the full and simple degree sequences $\deg_R(*|X_i, A=a)$ for $i=1,n$. The number of MCVs can significantly affect the accuracy of \system (Fig.~\ref{fig:lpbound-MCVs}), as it does for \safebound and \psql. 

We also construct one degree sequence ${\bf d}_i$ for all non-MCVs of $A$ and each $i=1,n$. Let $r_i$ be the maximum number of $X_i$-values per non-MCV of $A$ and ${\bf d}_i$ be the degree sequence of the $r_i$ largest degrees of $X_i$-values. \nop{We can construct the degrees in the sequence by inspecting individually each non-MCV or aggregate over all non-MCVs.} We compute a set of $\ell_p$-norms of each degree sequence ${\bf d}_i$. An alternative, more expensive approach is to compute $\ell_p$-norms for each non-MCV and take their max for each $p$. 

To estimate for the equality predicate $A=v$, we use the $\ell_p$-norms for the degree sequences $\deg_R(*|X_i,$ $A=v)$ if $v$ is an MCV. Otherwise, we use the $\ell_p$-norms for the degree sequences ${\bf d}_i$. 

\paragraph{Range Predicate.} Range predicates are supported in \system using a hierarchy of histograms: Each layer is a histogram whose number of buckets is half the number of buckets of the histogram at the layer below. We ensure that the histogram at each layer covers the entire domain range of the attribute $A$. For each histogram bucket with boundaries $[s_i,e_i]$, we create $\ell_p$-norms on the full and simple degree sequences $\deg_R(*|X_i,$ $A\in [s_i,e_i])$.

To estimate for the range predicate $A\in [s,e]$, we find the smallest histogram bucket that contains the range $[s,e]$ from the predicate and then use the $\ell_p$-norms from that bucket.

\paragraph{Multiple Predicates.} In case of a conjunction of  predicates, we take as $\ell_p$-norm the minimum of the $\ell_p$-norms for the predicates, for each $p$. This is correct as the records must satisfy all predicates and in particular the most selective one. In case of a disjunction, we take as the $\ell_p$-norm the sum of the $\ell_p$-norms for the predicates, for each $p$. This {\em computed} quantity upper bounds the {\em desired} $\ell_p$-norm of the degree sequence for those tuples that satisfy the disjunction of the predicates, yet we cannot compute the latter norm unless we evaluate the predicates. To see this, observe that the desired $\ell_p$-norm is less than or equal to the $\ell_p$-norm of the degree sequence, which is obtained by the entry-wise sum of the degree sequences for the predicates. By Minkowski inequality, the latter norm is less than or equal to the computed norm.

\nop{\color{green}
In case of a disjunction, we take as $\ell_p$-norm the sum of the $\ell_p$-norms for the predicates, for each $p$. This is correct: the $\ell_p$-norm of the degree sequence for the disjunction of two predicates is less than or equal to the $\ell_p$-norm of the sum of two degree sequences, where the largest degrees of the two sequences are summed in order, which leads to the largest possible $\ell_p$-norm. Then, this $\ell_p$-norm of the sum of two degree sequences is less than or equal to the sum of the $\ell_p$-norms of the two degree sequences due to the Minkowski inequality, which proves that we take the upper bound.
}

\paragraph{Optimizations.} A challenge for \system is to estimate the cardinality of a join, where one operand is orders of magnitude larger than the other operands and has many dangling key values. This happens when a join operand has a selective predicate. By using norms that incorporate degrees  of dangling key values, \system returns a large overestimate. To address this challenge, it combines two orthogonal optimizations:  {\em predicate propagation} and {\em prefix degree sequences}. 
Predicate propagation is used in case of a predicate on a primary-key (PK) relation that is joined with a foreign-key (FK) relation. We propagate the predicate and its attribute through the join to the FK relation without increasing its size. The new predicate on the FK relation is then supported using MCVs and histograms to yield smaller and more accurate $\ell_p$-norms. \revtwo{For instance, assume we have a table $R(K,A)$ with primary key $K$ and attribute $A$ on which we have a predicate $\phi(A)$. We also have a table $S(K,B)$ with foreign key $K$ and some attribute $B$. By propagating $\phi(A)$ from $R$ to $S$, we mean that we join the two relations to obtain a new relation $S'(K,B,A)$. This relation $S'$ has the same cardinality as $S$, yet every $K$-value in $S'$ is now accompanied by the $A$-value from $R$. We can now construct MCVs and histograms on the data column $A$ in $S'$. A variant of this optimization is also used by \safebound~\cite{SafeBound:SIGMOD23}.
}


In case of a large degree sequence, \system also keeps its length ($\ell_0$-norm) and the $\ell_p$-norms on its prefixes with the $2^i$ largest degrees, for $i\geq 0$. Then, for a join, \system first fetches the $\ell_0$-norm of each of the operands. The minimum $m$ of these $\ell_0$-norms tells us the maximum number of key values that join at each operand. \system uses $m$ to pick the $\ell_p$-norms for the $i$-th prefix\footnote{The degrees typically decrease exponentially and sequence prefixes for $i>4$ have norms close to those for the entire degree sequence. For each large degree sequence, we therefore only keep the norms for the first 4 prefixes and for the entire sequence.} of the degree sequences of each of the join operands, for $2^{i-1}\leq m \leq 2^i$.



\nop{

TODO.  Things to say here:

\begin{itemize}
\item For a given attribute $Z$, we compute and store the maximum of
  the $\ell_p$ statistics of all relations of the form
  $\sigma_{Z=z}(R)$, for $z \in R.Z$.  We use these whenever the query
  contains a selection $R.Z=??$.  TO GIVE A NAME TO THIS STATISTICS,
  e.g. the \emph{generic conditionals}.
\item Better: construct a histogram on $R.Z$, by partitioning the
  domain into $b$ buckets, and storing separate generic conditionals
  per bucket.
\item Better: for the most common values $z_1, \ldots, z_k$ we store
  the values the $\ell_p$-statistics separately.  TO SAY HOW LARGE WE
  TAKE $k$.
\item For range queries, we need a different kind of histogram.
  Describe. 
\item For LIKE predicates we use this data structure XXX.
\item I really like Haoze's idea of the $\ell_p$ norms of a prefix.
  Can we include that?  Do we have experiments for that?
\item what else?
\end{itemize}



Relation $R(X,A_1,A_2,\ldots)$, conjunction of predicates $\bigwedge_i$ \textcolor{mMediumBrown}{$(A_i\text{ op } v_i)$}
\vspace*{1em}

Use the following $\ell_p$-norms for the conjunction of predicates:     \vspace*{1em}
\begin{itemize}
    \item Fetch the $\ell_p$-norms for each predicate \textcolor{mMediumBrown}{$A_i\text{ op } v_i$}
    \vspace*{1em}
    \item For each $p$, take the minimum of the $\ell_p$-norms for the predicates
\end{itemize}

    





}

\section{Problem Statement}
\label{sec:problem}

This work studies two popular models of opinion exchange on networks. The overarching goal is for a network of truthful, rational agents to learn a binary piece of information, which we call the state of the world or \emph{ground truth}. We can also think of this state as an optimal binary action (buying or selling a stock, voting for a political party's candidate, etc.). The agents are arranged on a directed graph $G=(V,E)$ and broadcast which of the two states they believe is more probable to their neighbors. 

More explicitly, we encode the ground truth in $ \theta \in \{0,1\} $, distributed according to $ \berd{q} $, Bernoulli distribution with probability of $q$ of taking $1$ and $1-q$ of taking $0$. Every agent $v \in V$ initially receives an independent \emph{private signal} $ s_v \in \{0,1\}$ correlated with the ground truth. They then announce a prediction $a_v \in \{ 0,1 \} $ of the ground truth along outgoing edges, so out-neighbors of $v$ may use $a_v$ to improve their own predictions. Importantly, the probabilities $p$ and $q$, as well as the graph $G$ are all common knowledge. 
This is captured in the following formal definition of a network.

\begin{definition}[Social Network]
    A \emph{social network} is $ \network \deq ( G,q,p ) $,
    where \begin{enumerate}
        \item $ G = ( V,E ) $ is a directed graph with agents as vertices,
        \item $ q \in ( 0,1 ) $ is the prior probability of $\theta = 1$,
        \item $ p \in ( \frac 12, 1 ) $ is the accuracy of agents' private signals $s_v \in \{0,1\}$, such that \[\pr{s_v = 1 \suchthat \theta = 1} = \pr{s_v = 0 \suchthat \theta = 0} = p, \quad \forall v \in V.
            \]
    \end{enumerate}
    We further denote $ n \deq \absolute{V} $.
\end{definition}

We consider a classic asynchronous \emph{sequential} model ~\cite{Golub2017-qo}, in which agents announce their predictions in a \emph{decision ordering}, given by a one-to-one mapping $\sigma: V \to [n]$.
We denote the set of all possible orderings by $\Sigma_n$.
At every time step $i$, agent $v = \sigma^{-1}(i)$ makes an announcement $a_v \in \{0,1\}$.
The announcement depends on the agent's private measurement $ s_v $, along with the \emph{previous announcements of in-neighbors}, which we denote as a tuple $ N_v $, defined as \[
    N_v \deq ( a_u \suchthat u \in V \land uv \in E \land \sigma(u)<\sigma(v) ).
\] 
We call the tuple $X_v = (s_v) \cup N_v$ the \emph{inputs} of node $v$. 
This setup of limiting visibility to an agent's neighborhood has been studied in a number of recent papers~\cite{Bahar2020-am,arieli2020social,lu24enabling}.

When making announcements, agents follow an \emph{aggregation rule}, which is a function $ \mu:  ( X_v, G, \sigma ) \mapsto a_v $.
Broadly speaking, aggregation rules can either be Bayesian or non-Bayesian. 
In the \emph{Bayesian} model,  agents are fully rational
and make predictions according to their posterior probability for $\theta$, given their inputs and knowledge of the network topology $G$. In particular, agents take into account the correlation between their inputs resulting from the network topology and from the current ordering $ \sigma $. \[
     \mu^B(X_v, G, \sigma) \deq \begin{cases}
         1 & \text{if $\Pr_{G,\sigma}[\theta = 1 \suchthat X_v] > \frac 12$,} \\
         0 & \text{if $\Pr_{G,\sigma}[\theta = 0 \suchthat X_v] > \frac 12$,} \\
         \berd{\frac 12} & \text{otherwise.}
     \end{cases}
 \]

We also consider a non-Bayesian model, in which agents have bounded rationality and instead use simpler heuristic rules.
This is perhaps a more practical model, as computing posterior probabilities in arbitrary networks can become computationally expensive. 
In particular, we examine the \emph{majority dynamics} model, in which agents simply follow the majority among their inputs~\cite{Bahar2020-am,Shoham1992-ir,Laland2004-ej}. Since this model does not require agents to take into account correlations between their inputs derived from the network topology or the ordering, we omit $G$ and $ \sigma $ as inputs to $\mu^M$: \[
    \mu^M(X_v) \deq \begin{cases}
        1 & \text{if $  \frac 1{\absolute{X_v}}\sum_{x \in X_v} x > \frac 12 $,} \\
        0 & \text{if $  \frac 1{\absolute{X_v}}\sum_{x \in X_v} x < \frac 12 $,} \\
        s_v & \text{otherwise.}
    \end{cases}
\]

Finally, we quantify how successful the network is in predicting the ground truth by defining the following notion of a learning rate.
\begin{definition}
    The \emph{cumulative learning rate} (CLR) of a network $ \network $ under the ordering $ \sigma $ and an aggregation rule $ \mu $ is \[
		\clr (\network, \sigma, \mu) \deq \E_{\theta, s} \left[ \sum_{v \in V}^{} \mathds{1}_{\{a_v = \theta\}} \right] = \sum_{v \in V} \Pr_{\theta, s}\left[a_v = \theta\right],
	\]
    where the equality follows from linearity of expectation.
    Further, the \emph{learning rate} (LR) of a network $ \network $ under the ordering $ \sigma $ is simply \[
		\lr (\network, \sigma, \mu) \deq \tfrac 1 n \clr (\network, \sigma, \mu).
	\]
\end{definition}

We are mainly interested in the \emph{optimal} learning rate of a network, defined as follows.

\begin{definition}[Optimal LRs]
    The \emph{optimal cumulative learning rate} of a network $ \network $ is \[
        \oclr (\network, \mu) \deq \max_{\sigma \in \Sigma_n} \clr (\network, \sigma, \mu),
    \]
    and the \emph{optimal learning rate} of a network $ \network $ is \[
        \olr (\network, \mu) \deq \max_{\sigma \in \Sigma_n} \lr (\network, \sigma, \mu).
    \]
\end{definition}

Note that when $ p $ and $ q $ are clear from the context, we use the learning rate notation with only the graph, for example $ \olr (G, \mu) = \olr(( G,p,q ), \mu) $.
We can now present a formal definition of our main focus, the \netlearnopt{} optimization problem, and \netlearn{}, its decision version.

\begin{definition}[\netlearnopt{}]
    Suppose $\mu$ is a fixed aggregation rule.
    Given a network $ \network $, the \netlearnopt{} problem is to maximize  $ \lr (\network, \sigma, \mu) $, over $ \sigma \in \Sigma_n $.
\end{definition}


\begin{definition}[\netlearn{}]\label{def:decnetlearn}
    Suppose $\mu$ is a fixed aggregation rule.
    Given a network $ \network $ and a constant threshold $ \varepsilon \in (0,1)$,
    the \netlearn{} decision problem asks whether \[
            ( \exists \sigma \in \Sigma_n )\quad \lr (\network, \sigma, \mu) \geq 1-\varepsilon.
        \]
\end{definition}

Note that \Cref{def:decnetlearn} can be formulated equivalently by asking whether an optimal ordering $\sigma^*$ which maximizes the network learning rate achieves LR at least $1-\varepsilon$.

In \Cref{sec:bayes,sec:maj}, we focus on the decision problem, offering a proof that it is \np-hard for $ \mu = \mu^B $ and $ \mu = \mu^M $.
Finally, in \Cref{sec:approx}, we use insights from the \np-hardness proofs to show \netlearnopt{} is hard to even approximate.
Surprisingly, this gives us a stronger \np-hardness statement, showing that \netlearn{} is \np-hard even if we arbitrarily fix the agents' accuracy $ p \in ( \frac 12, 1 ) $.


\subsection{System Model and Notation}
We model the network as a directed graph $G = (V, E)$.
A vertex $u \in V$ represents a network entity visible from the TSN controller (i.e., end-devices, wired TSN bridges, and TSN translators), whereas an edge $[u,v] \in E$ (i.e., $u,v \in V$) specifies an Ethernet or 5G link between two adjacent network entities.
Note that, in the case of Fig.~\ref{fig:5g_tsn_bridge}, we do not use a single vertex to model the logical 5G-TSN bridge;
instead, we use three vertices for the two DS-TTs and the NW-TT.
We argue that this way of modelling makes our results also applicable for other wireless technologies like IEEE 802.11.

We denote the set of (time-triggered) streams by $\TT$.
Each stream $F \in \TT$ defines its path $\path{F}$, period $\period{F}$, phase $\phase{F}$, and frame size $\size{F}$.
For this work, we assume unicast paths of the form $(\v{1}{F}, \ldots, \vl{F})$, where $\v{1}{F}$ is the talker and $\vl{F}$ is the listener. 
The hypercycle $H$ is defined by $\text{lcm}_{F \in \TT}(\period{F})$.
To find a feasible TSN configuration, the scheduler has to incorporate $H / \period{F}$ many frames of the stream $F$.
We denote frames of $F$ by $f \in F$, and the $i$th frame in a hypercycle is released at time
\begin{equation*}
  \release{f} = \phase{F} + i \times \period{F}.
\end{equation*}

\begin{definition}[End-to-End QoS] \label{def:reliability}
  Each stream $F \in \TT$ specifies its minimum end-to-end reliability $\rel{F}$ w.r.t. latency $\ete{F}$ and jitter $\jitter{F}$.
The TSN configuration must ensure, with a probability of at least $\rel{F}$, that each frame $f \in F$ arrives at its listener $\vl{F}$ within a predetermined arrival interval $\R{\vl{F}, f}$, as constrained by
\begin{align}
  \Rmax{\vl{F}, f} - \release{f} &\leq \ete{F} \qquad \text{and} \label{eq:latency} \\
  \sub{\Rmax}{\Rmin}{\vl{F}, f} &\leq \jitter{F}. \label{eq:jitter}
\end{align}
\end{definition}

This work aims to compute feasible TSN configurations $\Conf = (\GCL, \R{})$ that satisfy Definition~\ref{def:reliability}.
$\Conf$ defines the gating operations $\GCL([u,v])$ of the Time-Aware Shaper at each egress port $[u,v]$ and the allowed arrival interval $\R{\v{k}{F}, f}$ of PSFP for each frame $f \in F$ at each hop $\v{k}{F} \in \path{F}$.

\subsection{Modelling TSN Runtime Behavior} \label{sec:execution_sequences}
To analyze the QoS guarantees of a TSN configuration $\Conf$, we introduce a model that captures possible runtime behavior under $\Conf$.
To this end, we define execution sequences $\E = (\T{}, \D{})$ to capture the possible transmission offsets $\T{}$ and transmission delays $\D{}$ for each frame $f \in \TT$ and each hop $[u,v] \in \path{f}$.
In detail, $\E$ encodes the following semantic:
\begin{itemize}
  \item $\T{[u,v], f}$ denotes the time when the bridge $u$ starts the transmission of $f$ at the egress port $[u,v]$, and
  \item $\D{[u,v], f}$ denotes the delay until $f$ is enqueued at $v$.
\end{itemize}
For Ethernet links, $\D{[u,v], f}$ equals the sum of the serialization delay $\size{f} / \drate{[u,v]}$, the propagation delay $\dprop{[u,v]}$, and the processing delay $\dproc{v}$ of $v$.
For 5G links, $\D{[u,v], f}$ is the outcome of the random variable capturing the 5G port-to-port delay, as described in Section~\ref{sec:5g_tsn}.
In both cases, the arrival time of frame $f$ at bridge $v$ equals
\begin{equation*}
  \add{\T}{\D}{[u,v], f} = \T{[u,v], f} + \D{[u,v], f}.
\end{equation*}
Moreover, $\E$ is constrained to satisfy the following:

\textbf{FIFO Queueing:}
The transmission of a frame $f$ via $[u,v]$ can only start after $f$ arrived at $u$. 
If another frame $f'$ is enqueued in the same FIFO queue at $[u,v]$ before $f$, then $u$ can only start the transmission of $f$ once that of $f'$ is completed.

\textbf{Multiplexing:}
Only a single frame can be transmitted over an Ethernet link at once.
In contrast, frequency-division multiplexing allows multiple frames to be sent over 5G links in parallel.
As noted in Section~\ref{sec:5g_tsn}, we assume that the induced traffic load is captured by the 5G packet delay histograms.

\textbf{GCL Consistency:}
A frame can be transmitted via $[u,v]$ if the gate configured by $\GCL([u,v])$ is open.
At the same time, the first frame of the egress queue must immediately start its transmission once the gate opens.

\textbf{Stream Policing:}
A frame $f$ is discarded by PSFP at bridge $u$ if and only if $f$ arrives at $u$ outside the interval $\R{u, f}$.
