\section{Robust and Feasible TSN Configurations} \label{sec:robustness_and_feasibility}
The previous section showed that highly stochastic 5G packet delays can nullify any QoS guarantee of non-robust approaches.
The underlying problem is rooted in the disconnect between the intended transmission ordering of the GCL and the actual transmission behavior at runtime. 
To overcome this obstacle, this section proposes a notion of robust scheduling that account for bounded packet delay uncertainty. 

\subsection{5G Packet Delay Budgets} \label{sec:robustness:pdb}
Each TSN stream $F$ specifies QoS requirements in terms of latency $\ete{F}$, jitter $\jitter{F}$, and reliability $\rel{F}$.
Due to significant 5G packet delays and packet delay variations, each of these metrics is strongly affected for wireless streams.
We therefore start by defining the allowed 5G packet delay budget~(PDB) that the TSN configuration has to account for.
For simplicity, we assume that each wireless stream traverses at most one logical 5G-TSN bridge and that there are no transmission faults in wired network partitions.\footnote{
  An extension to multiple wireless hops is straightforward by using (\ref{eq:pdb2}) multiplicatively for each 5G link.
  To address wired link/bridge failures, however, complementary techniques like frame replication are more suitable.
}

\subsubsection{Design Rationale}
A 5G PDB $\dpdb{} = [\dpdbmin{}, \dpdbmax{}]$ lower- and upper-bounds the allowed packet delay for traversing the logical 5G-TSN bridge. 
When regarding $\ete{F}$ as the end-to-end delay budget of $F$, the key question is: how much of that budget should be allocated to its 5G PDB $\dpdb{F}$?
We capture the trade-off between choosing $\dpdb{F}$ too narrow or too large by formulating the following optimization problem:
\begin{subequations} \label{eq:pdb}
\begin{align}
  \min_{\dpdb{}} \quad &\dpdbmax{F}, \label{eq:pdb1}\\
  \text{s.t.} \quad &\Prob\Big(\D{F} \in \dpdbfull{F}\Big) \geq \rel{F}. \label{eq:pdb2}
\end{align}
\end{subequations}
Here, $\D{F}$ denotes the random variable of 5G delays experienced by frames of stream $F$, and $\Prob(\cdot)$ denotes the corresponding probability function.
In the following, we describe the rationale behind the choice of (\ref{eq:pdb1}) and (\ref{eq:pdb2}).

\begin{figure}[t]
  \begin{subfigure}{0.48\columnwidth}
    \centering
    \resizebox{\textwidth}{!}{%
    \begin{tikzpicture}
	\begin{axis}[
	    ymin=0, ymax=0.1,
	    xmax=15,
	    minor y tick num = 3,
	    scaled y ticks=real:0.01,
	    area style,
	    xlabel={uplink packet delay [\si{\ms}]},
	    ylabel={probability density},
	    font=\Large,
	    ]
	    \path[name path=lower9999] (axis cs:3.803,0) -- (axis cs:3.803,1);
	    \path[name path=upper9999] (axis cs:13.176,0) -- (axis cs:13.176,1);
	    \addplot [black!10] fill between[of = lower9999 and upper9999];
	    \node[rotate=90,anchor=south] at (axis cs:13.176,0.08) {$\qty{99.99}{\percent}$};

	    \path[name path=lower99] (axis cs:3.803,0) -- (axis cs:3.803,1);
	    \path[name path=upper99] (axis cs:9.983,0) -- (axis cs:9.983,1);
	    \addplot [black!20] fill between[of = lower99 and upper99];
	    \node[rotate=90,anchor=south] at (axis cs:9.981,0.08) {$\qty{99}{\percent}$};

	    \path[name path=lower90] (axis cs:3.803,0) -- (axis cs:3.803,1);
	    \path[name path=upper90] (axis cs:7.717,0) -- (axis cs:7.717,1);
	    \addplot [black!30] fill between[of = lower90 and upper90];
	    \node[rotate=90,anchor=south] at (axis cs:7.717,0.08) {$\qty{90}{\percent}$};

	    \addplot+[ybar interval,mark=no,black,fill=black,semitransparent] plot coordinates { 
		(3.803,1e-05) (3.906,0.00011) (4.009,0.00011) (4.112,0.00029) (4.215,0.00041) (4.318,0.00055) (4.421,0.00088) (4.524,0.00136) (4.627,0.00161) (4.73,0.00224) (4.833,0.00372) (4.936,0.00452) (5.039,0.00866) (5.142,0.01582) (5.245,0.02163) (5.348,0.03399) (5.451,0.03653) (5.554,0.0576) (5.657,0.06254) (5.76,0.08382) (5.863,0.06751) (5.966,0.03065) (6.069,0.01778) (6.172,0.0207) (6.275,0.01947) (6.378,0.02322) (6.481,0.02457) (6.584,0.02623) (6.687,0.03085) (6.79,0.03164) (6.893,0.0405) (6.996,0.03587) (7.099,0.04674) (7.202,0.03896) (7.305,0.0416) (7.408,0.03652) (7.511,0.03066) (7.614,0.03047) (7.717,0.02029) (7.82,0.01577) (7.923,0.00345) (8.026,0.00041) (8.129,0.00026) (8.232,0.00044) (8.335,0.00048) (8.438,0.00033) (8.541,0.00041) (8.644,0.00093) (8.747,0.00057) (8.85,0.00075) (8.953,0.00116) (9.056,0.00262) (9.159,0.0018) (9.262,0.00104) (9.365,0.00103) (9.468,0.00219) (9.571,0.00187) (9.674,0.00109) (9.777,0.00183) (9.88,0.00148) (9.983,0.00039) (10.086,0.00017) (10.189,0.0002) (10.292,0.0002) (10.395,0.00028) (10.498,0.00024) (10.601,0.0003) (10.704,0.00171) (10.807,0.00045) (10.91,0.00027) (11.013,0.00037) (11.116,0.0005) (11.219,0.00033) (11.322,0.00046) (11.425,0.0007) (11.528,0.00158) (11.631,0.00057) (11.734,0.00023) (11.837,0.0002) (11.94,5e-05) (12.043,1e-05) (12.146,0.0) (12.249,1e-05) (12.352,1e-05) (12.455,0.0) (12.558,3e-05) (12.661,6e-05) (12.764,1e-05) (12.867,1e-05) (12.97,1e-05) (13.073,1e-05) (13.176,1e-05) (13.279,0.0) (13.382,0.0) (13.485,4e-05) (13.588,0.0) (13.691,2e-05) (13.794,0.0) (13.897,2e-05) (14.0,0.0) 
	    };

	    \draw (axis cs: 3.803,0) edge[thick] node[fill=white, inner sep=1pt] {min} (axis cs: 3.803,0.1);
	    \draw (axis cs: 14,0) edge[thick] node[fill=white, inner sep=1pt] {max} (axis cs: 14,0.1);
	\end{axis}
    \end{tikzpicture}
    }
  \end{subfigure}
  \begin{subfigure}{0.48\columnwidth}
    \centering
    %\resizebox{\textwidth}{!}{%
    \begin{tikzpicture}
      \def\offset{0.1}
      \def\scale{0.35}
      \def\steps{2}

      \draw (0,0) edge[thick,->] node[pos=0,below,font=\scriptsize,anchor=north west,xshift=-2pt] {DS-TT} (3.5,0);
      \draw (0,-1) edge[thick,->] node[pos=0,below,font=\scriptsize,anchor=north west,xshift=-2pt] {NW-TT (PSFP)} (3.5,-1);
      \draw (0,-2) edge[thick,->] node[pos=0,below,font=\scriptsize,anchor=north west,xshift=-2pt] {NW-TT (egress queue)} (3.5,-2);

      \draw[rounded corners=1mm] (\scale*0+\offset,0) rectangle ++(\scale*1,0.4) node[pos=0.5,font=\scriptsize] {$f$};
      \draw (\scale*1+\offset,0.5) edge[thick,-{Latex[length=1.5mm]}] node[above=2mm,font=\scriptsize] {$tx$} (\scale*1+\offset,0);

      \draw [pattern=north east lines] (0,-1) rectangle ++(\scale*4+\offset,0.5);
      \draw (\scale*4+\offset,0.5) edge[dashed] node[above=9mm,xshift=7pt,font=\scriptsize,rotate=30] {$tx + \dpdbmin{}$} (\scale*4+\offset,-1);
      \draw (\scale*4+\offset,-0.5) edge[thick] (\scale*4+\offset,-1);
      \draw (\scale*8+\offset,0.5) edge[dashed] node[above=9mm,xshift=7pt,font=\scriptsize,rotate=30] {$tx + \dpdbmax{}$} (\scale*8+\offset,-1);
      \draw (\scale*8+\offset,-0.5) edge[thick] (\scale*8+\offset,-1);
      \draw [pattern=north east lines] (\scale*8+\offset,-1) rectangle ++(\scale*0.35+\offset,0.5) node[right,pos=0.5,xshift=2pt,font=\scriptsize] {$\cdots$};

      \draw[rounded corners=1mm] (\scale*5+\offset,-2) rectangle ++(\scale*3.5,0.4) node[pos=0.5,font=\scriptsize] {$f$};
      \draw[black!40] (\scale*1+\offset,0) -- (\scale*5+\offset,-0.5) -- (\scale*5+\offset,-2);
      \draw (\scale*5+\offset,-0.5) edge[thick,-{Latex[length=1.5mm]}] node[above=2mm,font=\scriptsize] {$rx$} (\scale*5+\offset,-1);
      \draw (\scale*8.5+\offset,-1.5) edge[thick,-{Latex[length=1.5mm]}] node[above=2mm,font=\scriptsize] {$tx$} (\scale*8.5+\offset,-2);
      
      \node at (0,-2.5) {};
    \end{tikzpicture}
    %}
  \end{subfigure}
  \caption{5G Packet Delay Budgets $[\mathbf{d}^{min}, \mathbf{d}^{max}]$ from a 5G perspective (left) and from a TSN perspective (right).} \label{fig:pdb}
\end{figure}



First, allocating a too narrow 5G PDB, either in its upper bound $\dpdbmax{}$ or its size $\dpdbmax{} - \dpdbmin{}$, results in a degradation of the stream's end-to-end reliability and increases the difficulty for the 5G system to allocate sufficient resources.
%Moreover, to shield other (high-priority) traffic, TSN bridges and TSN translators can be instructed with PSFP to discard frames that arrive outside their expected arrival interval.
Moreover, for the uplink stream $F$ in Fig.~\ref{fig:pdb}, the expected arrival of a frame $f \in F$ at the NW-TT is given by
\begin{equation*}
  [tx + \dpdbmin{F}, tx + \dpdbmax{F}],
\end{equation*}
where $tx$ is the transmission offset of $f$ at the DS-TT.
As any arrival of $f$ outside this interval will likely result in uncontrolled frame reordering, PSFP at the NW-TT must discard $f$ in such cases.
Therefore, constraint~(\ref{eq:pdb2}) ensures that the 5G PDB of $F$ is large enough so that the probability of frames $f \in F$ passing PSFP is greater than $\rel{F}$.

Second, allocating a too large 5G PDB increases the difficulty for the IEEE 802.1Qbv scheduler to deliver frames within the allowed end-to-end latency.
For instance, Fig.~\ref{fig:pdb} shows how the PDB for a stream $F$ with $\rel{F} = \qty{90}{\percent}$ can cut off the long tail of the packet delay histogram, reducing the end-to-end latency by over $\qty{6}{\ms}$.
With the objective function~(\ref{eq:pdb1}), we therefore opt to minimize $\dpdbmax{F}$.

\subsubsection{5G PDB Contracts}
With the port-to-port delay histograms reported by the 5G system, the CNC can compute the 5G PDB of $F$ by accumulating the bin counts until $\rel{F}$ is exceeded. 
In detail, we can solve (\ref{eq:pdb}) optimally with
\begin{align*}
  \dpdbmin{F} &= \text{hist}[0].\text{low}, \\
  \dpdbmax{F} &= \min\Big\{\text{hist}[i].\text{up} \mid \sum\nolimits_{j=0}^{i} \text{hist}[j].\text{count} \geq \rel{F}\Big\},
\end{align*}
where $\text{hist}[i]$ denotes the $i$th bin of the histogram, with its respective lower/upper bound and (normalized) bin count.

The 5G PDB of a stream $F$ yields the following contract between the 5G system and the TSN system:
``The delay of a frame $f \in F$ traversing the 5G system is bounded by $\dpdb{F}$, with a probability greater than $\rel{F}$.''
The IEEE 802.1Qbv scheduler solely requires this contract to compute the GCLs.
That is, the CNC does not require additional state information of the 5G system (e.g., the current channel quality indicator) and it does not have to reschedule the entire TSN domain with every change in the 5G network state (e.g., which can occur every few milliseconds).
Instead, the 5G system can internally adapt its resource allocation to uphold the PDB contract.

\subsection{Robust and Feasible TSN Configurations} \label{sec:robustness}
With 5G PDBs capturing the guarantees of the 5G system, TSN has to extend these guarantees end-to-end from talker to listener.
This section introduces the notion of robustness that guarantees the frame's punctual delivery to its TSN listener as long as the 5G system complies with the 5G PDB contracts.

\input{strict_isolation.tex}

Before formalizing robustness in Definition~\ref{def:robustness}, we describe the underlying idea with Fig.~\ref{fig:strict_isolation}.
To protect the QoS of $F_2$, the correct delivery of $f_2 \in F_2$ must be independent of potential transmission faults of $f_1$ and $f_3$.
Robustness must therefore ensure, for every execution sequence $\E$, that $f_2$ can only take its intended transmission slots, i.e., the second gate opening at $[B^{NW}, B_1]$ and at $[B_1,L_2]$.
This assurance is twofold: 
First, there must not be a queueing backlog that defers $f_2$.
Second, $f_2$ may never take an earlier transmission slot, even if the previous slot is empty (e.g., when $f_1$ violates its 5G PDB).

At the same time, robustness should be considered as a conditional guarantee that builds on the 5G PDB contracts of Section~\ref{sec:robustness:pdb}.
For instance, $f_2$ can only take its intended transmission slots if the 5G packet delay of $f_2$ stays within its PDB $\dpdb{[T_2^{DS}, B^{NW}], F_2}$;
otherwise, $B^{NW}$ may have to discard $f_2$ to protect other streams.
We extend this idea inductively in the number of transmission hops:
$f_2$ arrives at its $k$th hop within the expected arrival interval if all individual packet delays up to that point lie within their respective PDBs.\footnote{To unify notation, we extend the notion of 5G PDBs to Ethernet links, e.g., to $\dpdb{[B_1, L_2], F_2}$, which may capture small delay variations as in Fig.~\ref{fig:delays}(a).}
This leads to the following definition of robustness:

\begin{definition}[Robustness] \label{def:robustness}
  The TSN configuration $\Conf = (\GCL, \R{})$ robustly schedules a stream $F \in \TT$ if for every execution sequence $\E = (\T{}, \D{})$, every frame $f \in F$, and every hop $\v{k}{F} \in \path{F}$ the following holds true:
  If, up to bridge~$\v{k}{F}$, the packet delays lie within their PDBs, i.e.,
  \begin{equation*} \label{eq:robustness:precondition}
    \D{[\v{l}{F}, \v{l+1}{F}], f} \in \dpdb{[\v{l}{F}, \v{l+1}{F}], F}, \quad \forall 1 \leq l < k,
  \end{equation*}
  then $\v{k}{F}$ receives $f$ within its expected PSFP interval, i.e.,
  \begin{equation*} \label{eq:robustness:postcondition}
    \add{\T}{\D}{[\v{k-1}{F}, \v{k}{F}], f} \in \R{\v{k}{F}, f}.
  \end{equation*}
\end{definition}

Finally, we summarize the required properties of a TSN configuration $\Conf$ to achieve formal end-to-end QoS guarantees:
\begin{theorem}[Feasibility] \label{theorem:feasibility}
  A TSN configuration $\Conf$ feasibly schedules a stream $F \in \TT$ (according to Definition~\ref{def:reliability}) if
  \begin{enumerate}[label=(\roman*)]
    \item $\Conf$ robustly schedules $F$,
    \item $\Conf$ allocates sufficiently large PDBs, according to (\ref{eq:pdb}), and
    \item $\Conf$ satisfies the latency~(\ref{eq:latency}) and jitter~(\ref{eq:jitter}) bounds of $F$.
  \end{enumerate}
\end{theorem}
\begin{proof}
  The proof is given in Appendix~\ref{appendix:feasibility}.
\end{proof}

\noindent For example, combining \textit{(i)} and \textit{(ii)} ensures \ldash with a probability of at least $\rel{F_2}$ \rdash that $f_2$ arrives at its listener $L_2$ within the expected arrival interval $\R{L_2, f_2} = [rx_1, rx_2]$.
Moreover, \textit{(iii)} constrains $rx_2 \leq \ete{F_2}$ and $rx_2 - rx_1 \leq \jitter{F_2}$.
