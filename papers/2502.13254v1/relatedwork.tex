\section{Related Work}
Our research builds on prior work in Sustainable Human-Computer Interaction (SHCI), the unmaking framework, and qualitative research on making and the maker movement. Each of these areas provides critical insights into how sustainability intersects with the practices and spaces of making.

\subsection{Sustainable HCI}

The concept of Sustainable Interaction Design (SID), introduced by Blevis~\cite{SID}, laid a foundational framework for what has now developed into the broader field of SHCI. 
Over the past decade, this field has grown to explore sustainability from multiple angles~\cite{10.1145/3411764.3445069, 10.1145/1753326.1753625}. Blevis’s original work emphasizes the need to integrate sustainability into every design phase, urging designers to reflect critically on how their values, methods, and decisions impact sustainability. 
This approach extends to material concerns such as the use, reuse, and disposal of products, with the ultimate goal of promoting longer product lifecycles through practices like repair and sharing.

Following this foundation, the theme of repair has emerged as a significant thread in SHCI, offering both theoretical and practical insights. Jackson and Kang, for example, conducted ethnographic studies of artists working with discarded technologies to challenge common assumptions about creativity and functionality~\cite{jackson2014breakdown}. Similarly, Lu and Lopes investigated e-waste practitioners, advocating for a reevaluation of users’ roles within capitalist systems~\cite{lu2024unmaking}. 
Kim and Paulos developed a framework for creative reuse, encouraging the transformation of e-waste through practices such as remaking and remanufacturing~\cite{kim2011practices}. 
Beyond creative reuse, repair itself has been studied in various contexts, such as Orr's ethnography of Xerox technicians, which highlights the collaborative nature of repair work~\cite{orr2016talking}, and research by Houston et al., which examined repair practices across diverse global settings~\cite{houston2016values}.


In parallel, SHCI research has explored persuasive technologies designed to influence end-user behavior, especially in relation to energy consumption. 
These technologies often use sensing and data visualization to encourage sustainable actions~\cite{10.1007/978-3-540-77006-0_7, 10.1145/3375182, 10.1145/1056808.1056932, 10.1145/3294109.3295634}. 
Mobile technologies have been particularly effective in promoting self-reflection and providing eco-feedback~\cite{10.1145/2786567.2795398, 10.1145/1753326.1753629, 10.1145/1518701.1518861}. 
However, much of this research positions end-users primarily as consumers, focusing mainly on goals such as reducing energy and water use, minimizing carbon footprints, and lowering electricity consumption.

In contrast to these consumer-oriented perspectives, our research explores sustainable practices among makers, who are not only consumers but also active creators of physical designs using tangible materials. This shift in focus allows us to examine sustainability from a different perspective, investigating how makers engage with materials and navigate sustainability in their daily activities, often without explicitly prioritizing material reduction.

\subsection{Unmaking and Sustainable Making}

The concept of unmaking, an extension of SID, offers a critical lens for examining the rapid production cycles driven by capitalism. 
The prefix \textit{un} suggests a deliberate rethinking of the assumption that products are inherently designed for longevity, exposing the realities of planned obsolescence, disposal, and replacement under the guise of technical advancement. 
This theme was recently highlighted as a special topic in TOCHI~\cite{song2025unmaking} and has been explored in projects such as Unmaking~\cite{Unmaking}, Un-crafting~\cite{murer2015crafting}, and Unfabricate~\cite{10.1145/3313831.3376227}, which use speculative and participatory design methods to investigate the afterlives of objects and materials.


For example, Song and Paulos experimented with 3D printing to explore how objects could have dynamic afterlives through unmaking~\cite{Unmaking}, while Sabie et al. used participatory design to foster critical conversations around unmaking as a way to provoke dialogue and embrace diverse perspectives~\cite{sabie2022unmaking}. 
Khan et al. discussed the pragmatics of sustainable unmaking through folk strategies involved in the e-waste recycling industry~\cite{khan2023pragmatics}. Meanwhile, Wu and Devendorf, through their Unfabricate project, reflected on the time and labor involved in the design-disassembly of smart yarn, framing this process as part of a broader critique of capitalist material production~\cite{10.1145/3313831.3376227}.

The connection between unmaking and sustainability is clear, particularly in initiatives such as the Sustainable Making workshop at UIST~\cite{yan2023future} and the Sustainable Unmaking workshop at CHI~\cite{song2024sustainable}. 
Our work aligns with these efforts but shifts the focus away from theoretical critique. 
Instead, we provide an empirical account of how sustainable practices in making are voluntarily implemented—or why they are not. 
This first-hand perspective offers valuable insights for future research aimed at promoting sustainable making practices at large.

\subsection{Make, Maker, and Makerspace}

The democratization of personal fabrication tools has transformed the landscape of making, leading to the rapid expansion of makerspaces and revitalizing maker and DIY cultures. 
Initially, these spaces served as grassroots hubs for technologists and hobbyists to engage in hands-on fabrication and hacking~\cite{lindtner2014emerging}.
Over time, the concept of a makerspace has broadened to include a wide range of physical spaces dedicated to making, from libraries to classrooms~\cite{hira2014classroom}, and even spaces supported by governments, businesses, or universities~\cite{freeman2018bottom, matthiesen2015replacing}.

In the HCI literature, the rise of makerspaces and maker culture has been extensively documented~\cite{lindtner2014emerging, bardzell2014now, tanenbaum2013democratizing}. 
For example, Shewbridge et al. explored the early uses of 3D printers in home makerspaces~\cite{shewbridge2014everyday}, while Hudson et al.~\cite{hudson2016understanding} examined how casual makers engage with 3D printing services in public makerspaces. 
Additionally, HCI research has highlighted the supportive and collaborative nature of these spaces. 
Kolko et al. documented Hackademia, a semi-formal learning environment that fosters creative engineering experiences through making in a shared, welcoming setting~\cite{kolko2012hackademia}. 
Similar findings emerged from a 19-month-long ethnographic study conducted by Toombs et al.~\cite{toombs2015proper}.

The growth of maker culture has positioned makerspaces as ideal environments for researchers to investigate the material and environmental aspects of physical creation. 
For example, Dew and Rosner~\cite{10.1145/3322276.3322320} presented a series of design studies that aim to understand how designers conceptualize, manage, and repurpose waste materials within on-campus makerspaces. 
Vyas et al.~\cite{vyas2023democratizing} conducted maker workshops with under-resourced communities to explore how e-waste can be leveraged to support engagement in technology design. 
Research outside of HCI has also looked into the intertwining of making, sustainability, and materiality. 
For example, research in environmental science has analyzed data from two separate makerspace printing locations to assess the uncertainties and variations in energy and material balances associated with democratized 3D printing~\cite{song2019uncertainty}.

Perhaps the most pertinent study related to our work is by Unterfrauner et al., as discussed in~\cite{unterfrauner2017environmental}. Their research, based on interviews with 39 makers across Europe, documented numerous maker projects and initiatives that either address environmental issues directly or adopt environmentally friendly making practices. 
For example, the study found that it is common for makers to use reclaimed pallets, repurpose scrap from different industries for prototyping, or utilize available household items to develop eco-friendly solutions for home appliances.

Our research also adopts a qualitative approach, involving semi-structured interviews with U.S.-based makers in various roles. Unlike previous studies, however, our work highlights many of the challenges associated with implementing sustainable making practices within today's maker infrastructure and provides design insights and considerations to address them as future research directions.