
\documentclass{article} % For LaTeX2e
\usepackage{arxiv,times}

% Optional math commands from https://github.com/goodfeli/dlbook_notation.
% %%%%% NEW MATH DEFINITIONS %%%%%

\usepackage{amsmath,amsfonts,bm}
\usepackage{derivative}
% Mark sections of captions for referring to divisions of figures
\newcommand{\figleft}{{\em (Left)}}
\newcommand{\figcenter}{{\em (Center)}}
\newcommand{\figright}{{\em (Right)}}
\newcommand{\figtop}{{\em (Top)}}
\newcommand{\figbottom}{{\em (Bottom)}}
\newcommand{\captiona}{{\em (a)}}
\newcommand{\captionb}{{\em (b)}}
\newcommand{\captionc}{{\em (c)}}
\newcommand{\captiond}{{\em (d)}}

% Highlight a newly defined term
\newcommand{\newterm}[1]{{\bf #1}}

% Derivative d 
\newcommand{\deriv}{{\mathrm{d}}}

% Figure reference, lower-case.
\def\figref#1{figure~\ref{#1}}
% Figure reference, capital. For start of sentence
\def\Figref#1{Figure~\ref{#1}}
\def\twofigref#1#2{figures \ref{#1} and \ref{#2}}
\def\quadfigref#1#2#3#4{figures \ref{#1}, \ref{#2}, \ref{#3} and \ref{#4}}
% Section reference, lower-case.
\def\secref#1{section~\ref{#1}}
% Section reference, capital.
\def\Secref#1{Section~\ref{#1}}
% Reference to two sections.
\def\twosecrefs#1#2{sections \ref{#1} and \ref{#2}}
% Reference to three sections.
\def\secrefs#1#2#3{sections \ref{#1}, \ref{#2} and \ref{#3}}
% Reference to an equation, lower-case.
\def\eqref#1{equation~\ref{#1}}
% Reference to an equation, upper case
\def\Eqref#1{Equation~\ref{#1}}
% A raw reference to an equation---avoid using if possible
\def\plaineqref#1{\ref{#1}}
% Reference to a chapter, lower-case.
\def\chapref#1{chapter~\ref{#1}}
% Reference to an equation, upper case.
\def\Chapref#1{Chapter~\ref{#1}}
% Reference to a range of chapters
\def\rangechapref#1#2{chapters\ref{#1}--\ref{#2}}
% Reference to an algorithm, lower-case.
\def\algref#1{algorithm~\ref{#1}}
% Reference to an algorithm, upper case.
\def\Algref#1{Algorithm~\ref{#1}}
\def\twoalgref#1#2{algorithms \ref{#1} and \ref{#2}}
\def\Twoalgref#1#2{Algorithms \ref{#1} and \ref{#2}}
% Reference to a part, lower case
\def\partref#1{part~\ref{#1}}
% Reference to a part, upper case
\def\Partref#1{Part~\ref{#1}}
\def\twopartref#1#2{parts \ref{#1} and \ref{#2}}

\def\ceil#1{\lceil #1 \rceil}
\def\floor#1{\lfloor #1 \rfloor}
\def\1{\bm{1}}
\newcommand{\train}{\mathcal{D}}
\newcommand{\valid}{\mathcal{D_{\mathrm{valid}}}}
\newcommand{\test}{\mathcal{D_{\mathrm{test}}}}

\def\eps{{\epsilon}}


% Random variables
\def\reta{{\textnormal{$\eta$}}}
\def\ra{{\textnormal{a}}}
\def\rb{{\textnormal{b}}}
\def\rc{{\textnormal{c}}}
\def\rd{{\textnormal{d}}}
\def\re{{\textnormal{e}}}
\def\rf{{\textnormal{f}}}
\def\rg{{\textnormal{g}}}
\def\rh{{\textnormal{h}}}
\def\ri{{\textnormal{i}}}
\def\rj{{\textnormal{j}}}
\def\rk{{\textnormal{k}}}
\def\rl{{\textnormal{l}}}
% rm is already a command, just don't name any random variables m
\def\rn{{\textnormal{n}}}
\def\ro{{\textnormal{o}}}
\def\rp{{\textnormal{p}}}
\def\rq{{\textnormal{q}}}
\def\rr{{\textnormal{r}}}
\def\rs{{\textnormal{s}}}
\def\rt{{\textnormal{t}}}
\def\ru{{\textnormal{u}}}
\def\rv{{\textnormal{v}}}
\def\rw{{\textnormal{w}}}
\def\rx{{\textnormal{x}}}
\def\ry{{\textnormal{y}}}
\def\rz{{\textnormal{z}}}

% Random vectors
\def\rvepsilon{{\mathbf{\epsilon}}}
\def\rvphi{{\mathbf{\phi}}}
\def\rvtheta{{\mathbf{\theta}}}
\def\rva{{\mathbf{a}}}
\def\rvb{{\mathbf{b}}}
\def\rvc{{\mathbf{c}}}
\def\rvd{{\mathbf{d}}}
\def\rve{{\mathbf{e}}}
\def\rvf{{\mathbf{f}}}
\def\rvg{{\mathbf{g}}}
\def\rvh{{\mathbf{h}}}
\def\rvu{{\mathbf{i}}}
\def\rvj{{\mathbf{j}}}
\def\rvk{{\mathbf{k}}}
\def\rvl{{\mathbf{l}}}
\def\rvm{{\mathbf{m}}}
\def\rvn{{\mathbf{n}}}
\def\rvo{{\mathbf{o}}}
\def\rvp{{\mathbf{p}}}
\def\rvq{{\mathbf{q}}}
\def\rvr{{\mathbf{r}}}
\def\rvs{{\mathbf{s}}}
\def\rvt{{\mathbf{t}}}
\def\rvu{{\mathbf{u}}}
\def\rvv{{\mathbf{v}}}
\def\rvw{{\mathbf{w}}}
\def\rvx{{\mathbf{x}}}
\def\rvy{{\mathbf{y}}}
\def\rvz{{\mathbf{z}}}

% Elements of random vectors
\def\erva{{\textnormal{a}}}
\def\ervb{{\textnormal{b}}}
\def\ervc{{\textnormal{c}}}
\def\ervd{{\textnormal{d}}}
\def\erve{{\textnormal{e}}}
\def\ervf{{\textnormal{f}}}
\def\ervg{{\textnormal{g}}}
\def\ervh{{\textnormal{h}}}
\def\ervi{{\textnormal{i}}}
\def\ervj{{\textnormal{j}}}
\def\ervk{{\textnormal{k}}}
\def\ervl{{\textnormal{l}}}
\def\ervm{{\textnormal{m}}}
\def\ervn{{\textnormal{n}}}
\def\ervo{{\textnormal{o}}}
\def\ervp{{\textnormal{p}}}
\def\ervq{{\textnormal{q}}}
\def\ervr{{\textnormal{r}}}
\def\ervs{{\textnormal{s}}}
\def\ervt{{\textnormal{t}}}
\def\ervu{{\textnormal{u}}}
\def\ervv{{\textnormal{v}}}
\def\ervw{{\textnormal{w}}}
\def\ervx{{\textnormal{x}}}
\def\ervy{{\textnormal{y}}}
\def\ervz{{\textnormal{z}}}

% Random matrices
\def\rmA{{\mathbf{A}}}
\def\rmB{{\mathbf{B}}}
\def\rmC{{\mathbf{C}}}
\def\rmD{{\mathbf{D}}}
\def\rmE{{\mathbf{E}}}
\def\rmF{{\mathbf{F}}}
\def\rmG{{\mathbf{G}}}
\def\rmH{{\mathbf{H}}}
\def\rmI{{\mathbf{I}}}
\def\rmJ{{\mathbf{J}}}
\def\rmK{{\mathbf{K}}}
\def\rmL{{\mathbf{L}}}
\def\rmM{{\mathbf{M}}}
\def\rmN{{\mathbf{N}}}
\def\rmO{{\mathbf{O}}}
\def\rmP{{\mathbf{P}}}
\def\rmQ{{\mathbf{Q}}}
\def\rmR{{\mathbf{R}}}
\def\rmS{{\mathbf{S}}}
\def\rmT{{\mathbf{T}}}
\def\rmU{{\mathbf{U}}}
\def\rmV{{\mathbf{V}}}
\def\rmW{{\mathbf{W}}}
\def\rmX{{\mathbf{X}}}
\def\rmY{{\mathbf{Y}}}
\def\rmZ{{\mathbf{Z}}}

% Elements of random matrices
\def\ermA{{\textnormal{A}}}
\def\ermB{{\textnormal{B}}}
\def\ermC{{\textnormal{C}}}
\def\ermD{{\textnormal{D}}}
\def\ermE{{\textnormal{E}}}
\def\ermF{{\textnormal{F}}}
\def\ermG{{\textnormal{G}}}
\def\ermH{{\textnormal{H}}}
\def\ermI{{\textnormal{I}}}
\def\ermJ{{\textnormal{J}}}
\def\ermK{{\textnormal{K}}}
\def\ermL{{\textnormal{L}}}
\def\ermM{{\textnormal{M}}}
\def\ermN{{\textnormal{N}}}
\def\ermO{{\textnormal{O}}}
\def\ermP{{\textnormal{P}}}
\def\ermQ{{\textnormal{Q}}}
\def\ermR{{\textnormal{R}}}
\def\ermS{{\textnormal{S}}}
\def\ermT{{\textnormal{T}}}
\def\ermU{{\textnormal{U}}}
\def\ermV{{\textnormal{V}}}
\def\ermW{{\textnormal{W}}}
\def\ermX{{\textnormal{X}}}
\def\ermY{{\textnormal{Y}}}
\def\ermZ{{\textnormal{Z}}}

% Vectors
\def\vzero{{\bm{0}}}
\def\vone{{\bm{1}}}
\def\vmu{{\bm{\mu}}}
\def\vtheta{{\bm{\theta}}}
\def\vphi{{\bm{\phi}}}
\def\va{{\bm{a}}}
\def\vb{{\bm{b}}}
\def\vc{{\bm{c}}}
\def\vd{{\bm{d}}}
\def\ve{{\bm{e}}}
\def\vf{{\bm{f}}}
\def\vg{{\bm{g}}}
\def\vh{{\bm{h}}}
\def\vi{{\bm{i}}}
\def\vj{{\bm{j}}}
\def\vk{{\bm{k}}}
\def\vl{{\bm{l}}}
\def\vm{{\bm{m}}}
\def\vn{{\bm{n}}}
\def\vo{{\bm{o}}}
\def\vp{{\bm{p}}}
\def\vq{{\bm{q}}}
\def\vr{{\bm{r}}}
\def\vs{{\bm{s}}}
\def\vt{{\bm{t}}}
\def\vu{{\bm{u}}}
\def\vv{{\bm{v}}}
\def\vw{{\bm{w}}}
\def\vx{{\bm{x}}}
\def\vy{{\bm{y}}}
\def\vz{{\bm{z}}}

% Elements of vectors
\def\evalpha{{\alpha}}
\def\evbeta{{\beta}}
\def\evepsilon{{\epsilon}}
\def\evlambda{{\lambda}}
\def\evomega{{\omega}}
\def\evmu{{\mu}}
\def\evpsi{{\psi}}
\def\evsigma{{\sigma}}
\def\evtheta{{\theta}}
\def\eva{{a}}
\def\evb{{b}}
\def\evc{{c}}
\def\evd{{d}}
\def\eve{{e}}
\def\evf{{f}}
\def\evg{{g}}
\def\evh{{h}}
\def\evi{{i}}
\def\evj{{j}}
\def\evk{{k}}
\def\evl{{l}}
\def\evm{{m}}
\def\evn{{n}}
\def\evo{{o}}
\def\evp{{p}}
\def\evq{{q}}
\def\evr{{r}}
\def\evs{{s}}
\def\evt{{t}}
\def\evu{{u}}
\def\evv{{v}}
\def\evw{{w}}
\def\evx{{x}}
\def\evy{{y}}
\def\evz{{z}}

% Matrix
\def\mA{{\bm{A}}}
\def\mB{{\bm{B}}}
\def\mC{{\bm{C}}}
\def\mD{{\bm{D}}}
\def\mE{{\bm{E}}}
\def\mF{{\bm{F}}}
\def\mG{{\bm{G}}}
\def\mH{{\bm{H}}}
\def\mI{{\bm{I}}}
\def\mJ{{\bm{J}}}
\def\mK{{\bm{K}}}
\def\mL{{\bm{L}}}
\def\mM{{\bm{M}}}
\def\mN{{\bm{N}}}
\def\mO{{\bm{O}}}
\def\mP{{\bm{P}}}
\def\mQ{{\bm{Q}}}
\def\mR{{\bm{R}}}
\def\mS{{\bm{S}}}
\def\mT{{\bm{T}}}
\def\mU{{\bm{U}}}
\def\mV{{\bm{V}}}
\def\mW{{\bm{W}}}
\def\mX{{\bm{X}}}
\def\mY{{\bm{Y}}}
\def\mZ{{\bm{Z}}}
\def\mBeta{{\bm{\beta}}}
\def\mPhi{{\bm{\Phi}}}
\def\mLambda{{\bm{\Lambda}}}
\def\mSigma{{\bm{\Sigma}}}

% Tensor
\DeclareMathAlphabet{\mathsfit}{\encodingdefault}{\sfdefault}{m}{sl}
\SetMathAlphabet{\mathsfit}{bold}{\encodingdefault}{\sfdefault}{bx}{n}
\newcommand{\tens}[1]{\bm{\mathsfit{#1}}}
\def\tA{{\tens{A}}}
\def\tB{{\tens{B}}}
\def\tC{{\tens{C}}}
\def\tD{{\tens{D}}}
\def\tE{{\tens{E}}}
\def\tF{{\tens{F}}}
\def\tG{{\tens{G}}}
\def\tH{{\tens{H}}}
\def\tI{{\tens{I}}}
\def\tJ{{\tens{J}}}
\def\tK{{\tens{K}}}
\def\tL{{\tens{L}}}
\def\tM{{\tens{M}}}
\def\tN{{\tens{N}}}
\def\tO{{\tens{O}}}
\def\tP{{\tens{P}}}
\def\tQ{{\tens{Q}}}
\def\tR{{\tens{R}}}
\def\tS{{\tens{S}}}
\def\tT{{\tens{T}}}
\def\tU{{\tens{U}}}
\def\tV{{\tens{V}}}
\def\tW{{\tens{W}}}
\def\tX{{\tens{X}}}
\def\tY{{\tens{Y}}}
\def\tZ{{\tens{Z}}}


% Graph
\def\gA{{\mathcal{A}}}
\def\gB{{\mathcal{B}}}
\def\gC{{\mathcal{C}}}
\def\gD{{\mathcal{D}}}
\def\gE{{\mathcal{E}}}
\def\gF{{\mathcal{F}}}
\def\gG{{\mathcal{G}}}
\def\gH{{\mathcal{H}}}
\def\gI{{\mathcal{I}}}
\def\gJ{{\mathcal{J}}}
\def\gK{{\mathcal{K}}}
\def\gL{{\mathcal{L}}}
\def\gM{{\mathcal{M}}}
\def\gN{{\mathcal{N}}}
\def\gO{{\mathcal{O}}}
\def\gP{{\mathcal{P}}}
\def\gQ{{\mathcal{Q}}}
\def\gR{{\mathcal{R}}}
\def\gS{{\mathcal{S}}}
\def\gT{{\mathcal{T}}}
\def\gU{{\mathcal{U}}}
\def\gV{{\mathcal{V}}}
\def\gW{{\mathcal{W}}}
\def\gX{{\mathcal{X}}}
\def\gY{{\mathcal{Y}}}
\def\gZ{{\mathcal{Z}}}

% Sets
\def\sA{{\mathbb{A}}}
\def\sB{{\mathbb{B}}}
\def\sC{{\mathbb{C}}}
\def\sD{{\mathbb{D}}}
% Don't use a set called E, because this would be the same as our symbol
% for expectation.
\def\sF{{\mathbb{F}}}
\def\sG{{\mathbb{G}}}
\def\sH{{\mathbb{H}}}
\def\sI{{\mathbb{I}}}
\def\sJ{{\mathbb{J}}}
\def\sK{{\mathbb{K}}}
\def\sL{{\mathbb{L}}}
\def\sM{{\mathbb{M}}}
\def\sN{{\mathbb{N}}}
\def\sO{{\mathbb{O}}}
\def\sP{{\mathbb{P}}}
\def\sQ{{\mathbb{Q}}}
\def\sR{{\mathbb{R}}}
\def\sS{{\mathbb{S}}}
\def\sT{{\mathbb{T}}}
\def\sU{{\mathbb{U}}}
\def\sV{{\mathbb{V}}}
\def\sW{{\mathbb{W}}}
\def\sX{{\mathbb{X}}}
\def\sY{{\mathbb{Y}}}
\def\sZ{{\mathbb{Z}}}

% Entries of a matrix
\def\emLambda{{\Lambda}}
\def\emA{{A}}
\def\emB{{B}}
\def\emC{{C}}
\def\emD{{D}}
\def\emE{{E}}
\def\emF{{F}}
\def\emG{{G}}
\def\emH{{H}}
\def\emI{{I}}
\def\emJ{{J}}
\def\emK{{K}}
\def\emL{{L}}
\def\emM{{M}}
\def\emN{{N}}
\def\emO{{O}}
\def\emP{{P}}
\def\emQ{{Q}}
\def\emR{{R}}
\def\emS{{S}}
\def\emT{{T}}
\def\emU{{U}}
\def\emV{{V}}
\def\emW{{W}}
\def\emX{{X}}
\def\emY{{Y}}
\def\emZ{{Z}}
\def\emSigma{{\Sigma}}

% entries of a tensor
% Same font as tensor, without \bm wrapper
\newcommand{\etens}[1]{\mathsfit{#1}}
\def\etLambda{{\etens{\Lambda}}}
\def\etA{{\etens{A}}}
\def\etB{{\etens{B}}}
\def\etC{{\etens{C}}}
\def\etD{{\etens{D}}}
\def\etE{{\etens{E}}}
\def\etF{{\etens{F}}}
\def\etG{{\etens{G}}}
\def\etH{{\etens{H}}}
\def\etI{{\etens{I}}}
\def\etJ{{\etens{J}}}
\def\etK{{\etens{K}}}
\def\etL{{\etens{L}}}
\def\etM{{\etens{M}}}
\def\etN{{\etens{N}}}
\def\etO{{\etens{O}}}
\def\etP{{\etens{P}}}
\def\etQ{{\etens{Q}}}
\def\etR{{\etens{R}}}
\def\etS{{\etens{S}}}
\def\etT{{\etens{T}}}
\def\etU{{\etens{U}}}
\def\etV{{\etens{V}}}
\def\etW{{\etens{W}}}
\def\etX{{\etens{X}}}
\def\etY{{\etens{Y}}}
\def\etZ{{\etens{Z}}}

% The true underlying data generating distribution
\newcommand{\pdata}{p_{\rm{data}}}
\newcommand{\ptarget}{p_{\rm{target}}}
\newcommand{\pprior}{p_{\rm{prior}}}
\newcommand{\pbase}{p_{\rm{base}}}
\newcommand{\pref}{p_{\rm{ref}}}

% The empirical distribution defined by the training set
\newcommand{\ptrain}{\hat{p}_{\rm{data}}}
\newcommand{\Ptrain}{\hat{P}_{\rm{data}}}
% The model distribution
\newcommand{\pmodel}{p_{\rm{model}}}
\newcommand{\Pmodel}{P_{\rm{model}}}
\newcommand{\ptildemodel}{\tilde{p}_{\rm{model}}}
% Stochastic autoencoder distributions
\newcommand{\pencode}{p_{\rm{encoder}}}
\newcommand{\pdecode}{p_{\rm{decoder}}}
\newcommand{\precons}{p_{\rm{reconstruct}}}

\newcommand{\laplace}{\mathrm{Laplace}} % Laplace distribution

\newcommand{\E}{\mathbb{E}}
\newcommand{\Ls}{\mathcal{L}}
\newcommand{\R}{\mathbb{R}}
\newcommand{\emp}{\tilde{p}}
\newcommand{\lr}{\alpha}
\newcommand{\reg}{\lambda}
\newcommand{\rect}{\mathrm{rectifier}}
\newcommand{\softmax}{\mathrm{softmax}}
\newcommand{\sigmoid}{\sigma}
\newcommand{\softplus}{\zeta}
\newcommand{\KL}{D_{\mathrm{KL}}}
\newcommand{\Var}{\mathrm{Var}}
\newcommand{\standarderror}{\mathrm{SE}}
\newcommand{\Cov}{\mathrm{Cov}}
% Wolfram Mathworld says $L^2$ is for function spaces and $\ell^2$ is for vectors
% But then they seem to use $L^2$ for vectors throughout the site, and so does
% wikipedia.
\newcommand{\normlzero}{L^0}
\newcommand{\normlone}{L^1}
\newcommand{\normltwo}{L^2}
\newcommand{\normlp}{L^p}
\newcommand{\normmax}{L^\infty}

\newcommand{\parents}{Pa} % See usage in notation.tex. Chosen to match Daphne's book.

\DeclareMathOperator*{\argmax}{arg\,max}
\DeclareMathOperator*{\argmin}{arg\,min}

\DeclareMathOperator{\sign}{sign}
\DeclareMathOperator{\Tr}{Tr}
\let\ab\allowbreak

%%%%% NEW MATH DEFINITIONS %%%%%

\usepackage{amsmath,amsfonts,bm}

% Mark sections of captions for referring to divisions of figures
\newcommand{\figleft}{{\em (Left)}}
\newcommand{\figcenter}{{\em (Center)}}
\newcommand{\figright}{{\em (Right)}}
\newcommand{\figtop}{{\em (Top)}}
\newcommand{\figbottom}{{\em (Bottom)}}
\newcommand{\captiona}{{\em (a)}}
\newcommand{\captionb}{{\em (b)}}
\newcommand{\captionc}{{\em (c)}}
\newcommand{\captiond}{{\em (d)}}

% Highlight a newly defined term
\newcommand{\newterm}[1]{{\bf #1}}


% Figure reference, lower-case.
\def\figref#1{figure~\ref{#1}}
% Figure reference, capital. For start of sentence
\def\Figref#1{Figure~\ref{#1}}
\def\twofigref#1#2{figures \ref{#1} and \ref{#2}}
\def\quadfigref#1#2#3#4{figures \ref{#1}, \ref{#2}, \ref{#3} and \ref{#4}}
% Section reference, lower-case.
\def\secref#1{section~\ref{#1}}
% Section reference, capital.
\def\Secref#1{Section~\ref{#1}}
% Reference to two sections.
\def\twosecrefs#1#2{sections \ref{#1} and \ref{#2}}
% Reference to three sections.
\def\secrefs#1#2#3{sections \ref{#1}, \ref{#2} and \ref{#3}}
% Reference to an equation, lower-case.
\def\eqref#1{equation~\ref{#1}}
% Reference to an equation, upper case
\def\Eqref#1{Equation~\ref{#1}}
% A raw reference to an equation---avoid using if possible
\def\plaineqref#1{\ref{#1}}
% Reference to a chapter, lower-case.
\def\chapref#1{chapter~\ref{#1}}
% Reference to an equation, upper case.
\def\Chapref#1{Chapter~\ref{#1}}
% Reference to a range of chapters
\def\rangechapref#1#2{chapters\ref{#1}--\ref{#2}}
% Reference to an algorithm, lower-case.
\def\algref#1{algorithm~\ref{#1}}
% Reference to an algorithm, upper case.
\def\Algref#1{Algorithm~\ref{#1}}
\def\twoalgref#1#2{algorithms \ref{#1} and \ref{#2}}
\def\Twoalgref#1#2{Algorithms \ref{#1} and \ref{#2}}
% Reference to a part, lower case
\def\partref#1{part~\ref{#1}}
% Reference to a part, upper case
\def\Partref#1{Part~\ref{#1}}
\def\twopartref#1#2{parts \ref{#1} and \ref{#2}}

\def\ceil#1{\lceil #1 \rceil}
\def\floor#1{\lfloor #1 \rfloor}
\def\1{\bm{1}}
\newcommand{\train}{\mathcal{D}}
\newcommand{\valid}{\mathcal{D_{\mathrm{valid}}}}
\newcommand{\test}{\mathcal{D_{\mathrm{test}}}}

\def\eps{{\epsilon}}


% Random variables
\def\reta{{\textnormal{$\eta$}}}
\def\ra{{\textnormal{a}}}
\def\rb{{\textnormal{b}}}
\def\rc{{\textnormal{c}}}
\def\rd{{\textnormal{d}}}
\def\re{{\textnormal{e}}}
\def\rf{{\textnormal{f}}}
\def\rg{{\textnormal{g}}}
\def\rh{{\textnormal{h}}}
\def\ri{{\textnormal{i}}}
\def\rj{{\textnormal{j}}}
\def\rk{{\textnormal{k}}}
\def\rl{{\textnormal{l}}}
% rm is already a command, just don't name any random variables m
\def\rn{{\textnormal{n}}}
\def\ro{{\textnormal{o}}}
\def\rp{{\textnormal{p}}}
\def\rq{{\textnormal{q}}}
\def\rr{{\textnormal{r}}}
\def\rs{{\textnormal{s}}}
\def\rt{{\textnormal{t}}}
\def\ru{{\textnormal{u}}}
\def\rv{{\textnormal{v}}}
\def\rw{{\textnormal{w}}}
\def\rx{{\textnormal{x}}}
\def\ry{{\textnormal{y}}}
\def\rz{{\textnormal{z}}}

% Random vectors
\def\rvepsilon{{\mathbf{\epsilon}}}
\def\rvtheta{{\mathbf{\theta}}}
\def\rva{{\mathbf{a}}}
\def\rvb{{\mathbf{b}}}
\def\rvc{{\mathbf{c}}}
\def\rvd{{\mathbf{d}}}
\def\rve{{\mathbf{e}}}
\def\rvf{{\mathbf{f}}}
\def\rvg{{\mathbf{g}}}
\def\rvh{{\mathbf{h}}}
\def\rvu{{\mathbf{i}}}
\def\rvj{{\mathbf{j}}}
\def\rvk{{\mathbf{k}}}
\def\rvl{{\mathbf{l}}}
\def\rvm{{\mathbf{m}}}
\def\rvn{{\mathbf{n}}}
\def\rvo{{\mathbf{o}}}
\def\rvp{{\mathbf{p}}}
\def\rvq{{\mathbf{q}}}
\def\rvr{{\mathbf{r}}}
\def\rvs{{\mathbf{s}}}
\def\rvt{{\mathbf{t}}}
\def\rvu{{\mathbf{u}}}
\def\rvv{{\mathbf{v}}}
\def\rvw{{\mathbf{w}}}
\def\rvx{{\mathbf{x}}}
\def\rvy{{\mathbf{y}}}
\def\rvz{{\mathbf{z}}}

% Elements of random vectors
\def\erva{{\textnormal{a}}}
\def\ervb{{\textnormal{b}}}
\def\ervc{{\textnormal{c}}}
\def\ervd{{\textnormal{d}}}
\def\erve{{\textnormal{e}}}
\def\ervf{{\textnormal{f}}}
\def\ervg{{\textnormal{g}}}
\def\ervh{{\textnormal{h}}}
\def\ervi{{\textnormal{i}}}
\def\ervj{{\textnormal{j}}}
\def\ervk{{\textnormal{k}}}
\def\ervl{{\textnormal{l}}}
\def\ervm{{\textnormal{m}}}
\def\ervn{{\textnormal{n}}}
\def\ervo{{\textnormal{o}}}
\def\ervp{{\textnormal{p}}}
\def\ervq{{\textnormal{q}}}
\def\ervr{{\textnormal{r}}}
\def\ervs{{\textnormal{s}}}
\def\ervt{{\textnormal{t}}}
\def\ervu{{\textnormal{u}}}
\def\ervv{{\textnormal{v}}}
\def\ervw{{\textnormal{w}}}
\def\ervx{{\textnormal{x}}}
\def\ervy{{\textnormal{y}}}
\def\ervz{{\textnormal{z}}}

% Random matrices
\def\rmA{{\mathbf{A}}}
\def\rmB{{\mathbf{B}}}
\def\rmC{{\mathbf{C}}}
\def\rmD{{\mathbf{D}}}
\def\rmE{{\mathbf{E}}}
\def\rmF{{\mathbf{F}}}
\def\rmG{{\mathbf{G}}}
\def\rmH{{\mathbf{H}}}
\def\rmI{{\mathbf{I}}}
\def\rmJ{{\mathbf{J}}}
\def\rmK{{\mathbf{K}}}
\def\rmL{{\mathbf{L}}}
\def\rmM{{\mathbf{M}}}
\def\rmN{{\mathbf{N}}}
\def\rmO{{\mathbf{O}}}
\def\rmP{{\mathbf{P}}}
\def\rmQ{{\mathbf{Q}}}
\def\rmR{{\mathbf{R}}}
\def\rmS{{\mathbf{S}}}
\def\rmT{{\mathbf{T}}}
\def\rmU{{\mathbf{U}}}
\def\rmV{{\mathbf{V}}}
\def\rmW{{\mathbf{W}}}
\def\rmX{{\mathbf{X}}}
\def\rmY{{\mathbf{Y}}}
\def\rmZ{{\mathbf{Z}}}

% Elements of random matrices
\def\ermA{{\textnormal{A}}}
\def\ermB{{\textnormal{B}}}
\def\ermC{{\textnormal{C}}}
\def\ermD{{\textnormal{D}}}
\def\ermE{{\textnormal{E}}}
\def\ermF{{\textnormal{F}}}
\def\ermG{{\textnormal{G}}}
\def\ermH{{\textnormal{H}}}
\def\ermI{{\textnormal{I}}}
\def\ermJ{{\textnormal{J}}}
\def\ermK{{\textnormal{K}}}
\def\ermL{{\textnormal{L}}}
\def\ermM{{\textnormal{M}}}
\def\ermN{{\textnormal{N}}}
\def\ermO{{\textnormal{O}}}
\def\ermP{{\textnormal{P}}}
\def\ermQ{{\textnormal{Q}}}
\def\ermR{{\textnormal{R}}}
\def\ermS{{\textnormal{S}}}
\def\ermT{{\textnormal{T}}}
\def\ermU{{\textnormal{U}}}
\def\ermV{{\textnormal{V}}}
\def\ermW{{\textnormal{W}}}
\def\ermX{{\textnormal{X}}}
\def\ermY{{\textnormal{Y}}}
\def\ermZ{{\textnormal{Z}}}

% Vectors
\def\vzero{{\bm{0}}}
\def\vone{{\bm{1}}}
\def\vmu{{\bm{\mu}}}
\def\vtheta{{\bm{\theta}}}
\def\va{{\bm{a}}}
\def\vb{{\bm{b}}}
\def\vc{{\bm{c}}}
\def\vd{{\bm{d}}}
\def\ve{{\bm{e}}}
\def\vf{{\bm{f}}}
\def\vg{{\bm{g}}}
\def\vh{{\bm{h}}}
\def\vi{{\bm{i}}}
\def\vj{{\bm{j}}}
\def\vk{{\bm{k}}}
\def\vl{{\bm{l}}}
\def\vm{{\bm{m}}}
\def\vn{{\bm{n}}}
\def\vo{{\bm{o}}}
\def\vp{{\bm{p}}}
\def\vq{{\bm{q}}}
\def\vr{{\bm{r}}}
\def\vs{{\bm{s}}}
\def\vt{{\bm{t}}}
\def\vu{{\bm{u}}}
\def\vv{{\bm{v}}}
\def\vw{{\bm{w}}}
\def\vx{{\bm{x}}}
\def\vy{{\bm{y}}}
\def\vz{{\bm{z}}}

% Elements of vectors
\def\evalpha{{\alpha}}
\def\evbeta{{\beta}}
\def\evepsilon{{\epsilon}}
\def\evlambda{{\lambda}}
\def\evomega{{\omega}}
\def\evmu{{\mu}}
\def\evpsi{{\psi}}
\def\evsigma{{\sigma}}
\def\evtheta{{\theta}}
\def\eva{{a}}
\def\evb{{b}}
\def\evc{{c}}
\def\evd{{d}}
\def\eve{{e}}
\def\evf{{f}}
\def\evg{{g}}
\def\evh{{h}}
\def\evi{{i}}
\def\evj{{j}}
\def\evk{{k}}
\def\evl{{l}}
\def\evm{{m}}
\def\evn{{n}}
\def\evo{{o}}
\def\evp{{p}}
\def\evq{{q}}
\def\evr{{r}}
\def\evs{{s}}
\def\evt{{t}}
\def\evu{{u}}
\def\evv{{v}}
\def\evw{{w}}
\def\evx{{x}}
\def\evy{{y}}
\def\evz{{z}}

% Matrix
\def\mA{{\bm{A}}}
\def\mB{{\bm{B}}}
\def\mC{{\bm{C}}}
\def\mD{{\bm{D}}}
\def\mE{{\bm{E}}}
\def\mF{{\bm{F}}}
\def\mG{{\bm{G}}}
\def\mH{{\bm{H}}}
\def\mI{{\bm{I}}}
\def\mJ{{\bm{J}}}
\def\mK{{\bm{K}}}
\def\mL{{\bm{L}}}
\def\mM{{\bm{M}}}
\def\mN{{\bm{N}}}
\def\mO{{\bm{O}}}
\def\mP{{\bm{P}}}
\def\mQ{{\bm{Q}}}
\def\mR{{\bm{R}}}
\def\mS{{\bm{S}}}
\def\mT{{\bm{T}}}
\def\mU{{\bm{U}}}
\def\mV{{\bm{V}}}
\def\mW{{\bm{W}}}
\def\mX{{\bm{X}}}
\def\mY{{\bm{Y}}}
\def\mZ{{\bm{Z}}}
\def\mBeta{{\bm{\beta}}}
\def\mPhi{{\bm{\Phi}}}
\def\mLambda{{\bm{\Lambda}}}
\def\mSigma{{\bm{\Sigma}}}

% Tensor
\DeclareMathAlphabet{\mathsfit}{\encodingdefault}{\sfdefault}{m}{sl}
\SetMathAlphabet{\mathsfit}{bold}{\encodingdefault}{\sfdefault}{bx}{n}
\newcommand{\tens}[1]{\bm{\mathsfit{#1}}}
\def\tA{{\tens{A}}}
\def\tB{{\tens{B}}}
\def\tC{{\tens{C}}}
\def\tD{{\tens{D}}}
\def\tE{{\tens{E}}}
\def\tF{{\tens{F}}}
\def\tG{{\tens{G}}}
\def\tH{{\tens{H}}}
\def\tI{{\tens{I}}}
\def\tJ{{\tens{J}}}
\def\tK{{\tens{K}}}
\def\tL{{\tens{L}}}
\def\tM{{\tens{M}}}
\def\tN{{\tens{N}}}
\def\tO{{\tens{O}}}
\def\tP{{\tens{P}}}
\def\tQ{{\tens{Q}}}
\def\tR{{\tens{R}}}
\def\tS{{\tens{S}}}
\def\tT{{\tens{T}}}
\def\tU{{\tens{U}}}
\def\tV{{\tens{V}}}
\def\tW{{\tens{W}}}
\def\tX{{\tens{X}}}
\def\tY{{\tens{Y}}}
\def\tZ{{\tens{Z}}}


% Graph
\def\gA{{\mathcal{A}}}
\def\gB{{\mathcal{B}}}
\def\gC{{\mathcal{C}}}
\def\gD{{\mathcal{D}}}
\def\gE{{\mathcal{E}}}
\def\gF{{\mathcal{F}}}
\def\gG{{\mathcal{G}}}
\def\gH{{\mathcal{H}}}
\def\gI{{\mathcal{I}}}
\def\gJ{{\mathcal{J}}}
\def\gK{{\mathcal{K}}}
\def\gL{{\mathcal{L}}}
\def\gM{{\mathcal{M}}}
\def\gN{{\mathcal{N}}}
\def\gO{{\mathcal{O}}}
\def\gP{{\mathcal{P}}}
\def\gQ{{\mathcal{Q}}}
\def\gR{{\mathcal{R}}}
\def\gS{{\mathcal{S}}}
\def\gT{{\mathcal{T}}}
\def\gU{{\mathcal{U}}}
\def\gV{{\mathcal{V}}}
\def\gW{{\mathcal{W}}}
\def\gX{{\mathcal{X}}}
\def\gY{{\mathcal{Y}}}
\def\gZ{{\mathcal{Z}}}

% Sets
\def\sA{{\mathbb{A}}}
\def\sB{{\mathbb{B}}}
\def\sC{{\mathbb{C}}}
\def\sD{{\mathbb{D}}}
% Don't use a set called E, because this would be the same as our symbol
% for expectation.
\def\sF{{\mathbb{F}}}
\def\sG{{\mathbb{G}}}
\def\sH{{\mathbb{H}}}
\def\sI{{\mathbb{I}}}
\def\sJ{{\mathbb{J}}}
\def\sK{{\mathbb{K}}}
\def\sL{{\mathbb{L}}}
\def\sM{{\mathbb{M}}}
\def\sN{{\mathbb{N}}}
\def\sO{{\mathbb{O}}}
\def\sP{{\mathbb{P}}}
\def\sQ{{\mathbb{Q}}}
\def\sR{{\mathbb{R}}}
\def\sS{{\mathbb{S}}}
\def\sT{{\mathbb{T}}}
\def\sU{{\mathbb{U}}}
\def\sV{{\mathbb{V}}}
\def\sW{{\mathbb{W}}}
\def\sX{{\mathbb{X}}}
\def\sY{{\mathbb{Y}}}
\def\sZ{{\mathbb{Z}}}

% Entries of a matrix
\def\emLambda{{\Lambda}}
\def\emA{{A}}
\def\emB{{B}}
\def\emC{{C}}
\def\emD{{D}}
\def\emE{{E}}
\def\emF{{F}}
\def\emG{{G}}
\def\emH{{H}}
\def\emI{{I}}
\def\emJ{{J}}
\def\emK{{K}}
\def\emL{{L}}
\def\emM{{M}}
\def\emN{{N}}
\def\emO{{O}}
\def\emP{{P}}
\def\emQ{{Q}}
\def\emR{{R}}
\def\emS{{S}}
\def\emT{{T}}
\def\emU{{U}}
\def\emV{{V}}
\def\emW{{W}}
\def\emX{{X}}
\def\emY{{Y}}
\def\emZ{{Z}}
\def\emSigma{{\Sigma}}

% entries of a tensor
% Same font as tensor, without \bm wrapper
\newcommand{\etens}[1]{\mathsfit{#1}}
\def\etLambda{{\etens{\Lambda}}}
\def\etA{{\etens{A}}}
\def\etB{{\etens{B}}}
\def\etC{{\etens{C}}}
\def\etD{{\etens{D}}}
\def\etE{{\etens{E}}}
\def\etF{{\etens{F}}}
\def\etG{{\etens{G}}}
\def\etH{{\etens{H}}}
\def\etI{{\etens{I}}}
\def\etJ{{\etens{J}}}
\def\etK{{\etens{K}}}
\def\etL{{\etens{L}}}
\def\etM{{\etens{M}}}
\def\etN{{\etens{N}}}
\def\etO{{\etens{O}}}
\def\etP{{\etens{P}}}
\def\etQ{{\etens{Q}}}
\def\etR{{\etens{R}}}
\def\etS{{\etens{S}}}
\def\etT{{\etens{T}}}
\def\etU{{\etens{U}}}
\def\etV{{\etens{V}}}
\def\etW{{\etens{W}}}
\def\etX{{\etens{X}}}
\def\etY{{\etens{Y}}}
\def\etZ{{\etens{Z}}}

% The true underlying data generating distribution
\newcommand{\pdata}{p_{\rm{data}}}
% The empirical distribution defined by the training set
\newcommand{\ptrain}{\hat{p}_{\rm{data}}}
\newcommand{\Ptrain}{\hat{P}_{\rm{data}}}
% The model distribution
\newcommand{\pmodel}{p_{\rm{model}}}
\newcommand{\Pmodel}{P_{\rm{model}}}
\newcommand{\ptildemodel}{\tilde{p}_{\rm{model}}}
% Stochastic autoencoder distributions
\newcommand{\pencode}{p_{\rm{encoder}}}
\newcommand{\pdecode}{p_{\rm{decoder}}}
\newcommand{\precons}{p_{\rm{reconstruct}}}

\newcommand{\laplace}{\mathrm{Laplace}} % Laplace distribution

\newcommand{\E}{\mathbb{E}}
\newcommand{\Ls}{\mathcal{L}}
\newcommand{\R}{\mathbb{R}}
\newcommand{\emp}{\tilde{p}}
\newcommand{\lr}{\alpha}
\newcommand{\reg}{\lambda}
\newcommand{\rect}{\mathrm{rectifier}}
\newcommand{\softmax}{\mathrm{softmax}}
\newcommand{\sigmoid}{\sigma}
\newcommand{\softplus}{\zeta}
\newcommand{\KL}{D_{\mathrm{KL}}}
\newcommand{\Var}{\mathrm{Var}}
\newcommand{\standarderror}{\mathrm{SE}}
\newcommand{\Cov}{\mathrm{Cov}}
% Wolfram Mathworld says $L^2$ is for function spaces and $\ell^2$ is for vectors
% But then they seem to use $L^2$ for vectors throughout the site, and so does
% wikipedia.
\newcommand{\normlzero}{L^0}
\newcommand{\normlone}{L^1}
\newcommand{\normltwo}{L^2}
\newcommand{\normlp}{L^p}
\newcommand{\normmax}{L^\infty}

\newcommand{\parents}{Pa} % See usage in notation.tex. Chosen to match Daphne's book.

\DeclareMathOperator*{\argmax}{arg\,max}
\DeclareMathOperator*{\argmin}{arg\,min}

\DeclareMathOperator{\sign}{sign}
\DeclareMathOperator{\Tr}{Tr}
\let\ab\allowbreak


\usepackage{hyperref}
\usepackage{url}
\usepackage{graphicx}
\usepackage{multirow}
\usepackage{amssymb}
\usepackage{amsmath}
\usepackage{colortbl}
\usepackage{adjustbox}
\usepackage{makecell}
\usepackage{multicol}
\usepackage{booktabs}
\usepackage{bbding}
\usepackage{pifont}
\usepackage{wasysym}
\usepackage{utfsym}
\usepackage{fontawesome}
\usepackage{algorithm}
\usepackage{algorithmic}
% \usepackage[table]{xcolor} % 加载xcolor包并启用其表格功能
\usepackage{colortbl} 
\usepackage{bm}
\usepackage{subcaption}
\usepackage{mathrsfs}
\usepackage{caption}

\definecolor{mygray}{gray}{0.9}
% \hypersetup{
%     colorlinks=true, 
%     % urlcolor=magenta,   
% }

\title{CLIPure: Purification in Latent Space \\ via CLIP for Adversarially Robust \\ Zero-Shot Classification}
% Directional Adversarial Purification in Latent Space via Likelihood Maximization

% Latent Space Purification via CLIP \\ with Polar Coordinates

% Authors must not appear in the submitted version. They should be hidden
% as long as the \iclrfinalcopy macro remains commented out below.
% Non-anonymous submissions will be rejected without review.

\author{ Mingkun Zhang$^{1,3}$, Keping Bi$^{2,3,}$\thanks{corresponding authors}\ \ , Wei Chen$^{1,3,*}$, Jiafeng Guo$^{2,3}$, Xueqi Cheng$^{1,3}$ \\
  % \dag*, Yun Yang*, Richard Y. Zhang* \\%\thanks{Use footnote for providing further information
    %about author (webpage, alternative address)---\emph{not} for acknowledging
    %funding agencies.} \\
  %Department of Computer Science\\
  % $^{1}$CAS Key Laboratory of AI Safety \\
  % $^{2}$CAS Key Laboratory of Network Data Science and Technology \\
  % $^{3}$Institute of Computing Technology, Chinese Academy of Sciences, Beijing, China \\
  % $^{4}$University of Chinese Academy of Sciences, Beijing, China \\
  $^{1}$State Key Lab of AI Safety, CAS Key Lab of AI Safety, \\
  \ \ Institute of Computing Technology, Chinese Academy of Sciences, Beijing, China \\
  % CAS Key Laboratory of AI Safety, \\
  % \ \ Institute of Computing Technology, Chinese Academy of Sciences, Beijing, China \\
  $^{2}$CAS Key Laboratory of Network Data Science and Technology, \\ 
  \ \ Institute of Computing Technology, Chinese Academy of Sciences, Beijing, China \\
  $^{3}$University of Chinese Academy of Sciences, Beijing, China \\
  % \; \; 
  % University of Southern California\dag \\
  % Pittsburgh, PA 15213 \\
  \texttt{\{zhangmingkun20z, bikeping, chenwei2022, guojiafeng, cxq\}@ict.ac.cn} \\
}


% \author{Mingkun~Zhang \\
% CAS Key Laboratory of AI Safety,\\
% Institute of Computing Technology, CAS \\
% \texttt{zhangmingkun20z@ict.ac.cn} \\
% \And
% Keping~Bi \thanks{Corresponding Authors.} \\
% Key Laboratory of Network Data Science and\\Technology, Institute of Computing Technology, CAS \\
% \texttt{bikeping@ict.ac.cn}\\
% \And
% Wei~Chen~$^*$ \\
% CAS Key Laboratory of AI Safety, \\
% Institute of Computing Technology, CAS \\
% \texttt{chenwei2022@ict.ac.cn} \\
% \And
% Jiafeng~Guo \\
% Key Laboratory of Network Data Science and\\Technology, Institute of Computing Technology, CAS \\
% \texttt{guojiafeng@ict.ac.cn} \\
% \And
% Xueqi~Cheng \\
% CAS Key Laboratory of AI Safety, \\
% Institute of Computing Technology, CAS \\
% \texttt{cxq@ict.ac.cn} \\
% }


% \author{Natalia Cerebro \& Amelie P. Amygdale \thanks{ Use footnote for providing further information
% about author (webpage, alternative address)---\emph{not} for acknowledging
% funding agencies.  Funding acknowledgements go at the end of the paper.} \\
% Department of Computer Science\\
% Cranberry-Lemon University\\
% Pittsburgh, PA 15213, USA \\
% \texttt{\{hippo,brain,jen\}@cs.cranberry-lemon.edu} \\
% \And
% Ji Q. Ren \& Yevgeny LeNet \\
% Department of Computational Neuroscience \\
% University of the Witwatersrand \\
% Joburg, South Africa \\
% \texttt{\{robot,net\}@wits.ac.za} \\
% \AND
% Coauthor \\
% Affiliation \\
% Address \\
% \texttt{email}
% }

% The \author macro works with any number of authors. There are two commands
% used to separate the names and addresses of multiple authors: \And and \AND.
%
% Using \And between authors leaves it to \LaTeX{} to determine where to break
% the lines. Using \AND forces a linebreak at that point. So, if \LaTeX{}
% puts 3 of 4 authors names on the first line, and the last on the second
% line, try using \AND instead of \And before the third author name.

\newcommand{\fix}{\marginpar{FIX}}
\newcommand{\new}{\marginpar{NEW}}

\iclrfinalcopy % Uncomment for camera-ready version, but NOT for submission.
\begin{document}
 
% \definecolor{mycolor_blue}{RGB}{96,150,230}

\maketitle
\begin{abstract}
% Adversarial examples, recognized as out-of-distribution samples, challenge models to produce accurate classification results. Adversarial purification seeks to revert these samples to the clean data region where models demonstrate remarkable performance. Traditional purification methods, which directly modify image pixels, inevitably lead to information loss. To address this, we employ Stochastic Differential Equations (SDE) to model the purification process and define a purification risk. Our analysis underscores the information loss tied to the sparsity of pixel space, advocating for purification in latent space due to its uniform distribution of sample densities. Given the challenges in learning high-quality latent semantics solely from uni-modal images, we leverage the latent spaces modeled by the multimodal CLIP model, which are semantically aligned with text captions. Consequently, we introduce \textbf{CLIPure}, a method for purifying adversarial examples within CLIP’s latent space by optimizing the direction properties of image embeddings while disregarding magnitude. CLIPure includes two versions—one leveraging a diffusion model and the other utilizing an off-the-shelf CLIP model. Validated across multiple datasets, CLIPure consistently surpasses top-ranked models on RobustBench for CIFAR-10 and ImageNet. Furthermore, each version significantly outperforms adversarially trained CLIP models, achieving an average absolute increment of 30.9\% and 35.0\% in robustness against an $l_{\infty} = 4/255$ threat model across 13 datasets for zero-shot classification, also demonstrating exceptional inference efficiency compared with other purification strategies.
In this paper, we aim to build an adversarially robust zero-shot image classifier. We ground our work on CLIP, a vision-language pre-trained encoder model that can perform zero-shot classification by matching an image with text prompts ``a photo of a $<$class-name$>$.''. Purification is the path we choose since it does not require adversarial training on specific attack types and thus can cope with any foreseen attacks. We then formulate purification risk as the KL divergence between the joint distributions of the purification process of denoising the adversarial samples and the attack process of adding perturbations to benign samples, through bidirectional Stochastic Differential Equations (SDEs). The final derived results inspire us to explore purification in the multi-modal latent space of CLIP. We propose two variants for our CLIPure approach: \textit{CLIPure-Diff} which models the likelihood of images' latent vectors with the DiffusionPrior module in DaLLE-2 (modeling the generation process of CLIP's latent vectors), and \textit{CLIPure-Cos} which models the likelihood with the cosine similarity between the embeddings of an image and ``a photo of a.''. As far as we know, CLIPure is the first purification method in multi-modal latent space and CLIPure-Cos is the first purification method that is not based on generative models, which substantially improves defense efficiency. We conducted extensive experiments on CIFAR-10, ImageNet, and 13 datasets that previous CLIP-based defense methods used for evaluating zero-shot classification robustness. Results show that CLIPure boosts the SOTA robustness by a large margin, e.g., from 71.7\% to \textbf{91.1\%} on CIFAR10, from 59.6\% to \textbf{72.6\%} on ImageNet, and \textbf{108\%} relative improvements of average robustness on the 13 datasets over previous SOTA. The code is available at \href{https://github.com/TMLResearchGroup-CAS/CLIPure}{https://github.com/TMLResearchGroup-CAS/CLIPure}.
\end{abstract}

\section{Introduction}


% Neural networks have achieved impressive performance across various tasks but remain vulnerable to imperceptible perturbations, showing vulnerability even in the straightforward image classification task under adversarial attacks \citep{szegedy2013intriguing}.
% Previous efforts to enhance model robustness against adversarial examples mainly focus on two approaches: adversarial training \citep{mkadry2017towards, wang2023better} and purification \citep{song2017pixeldefend, nie2022diffusion}. Since adversarial examples are considered out-of-distribution samples \citep{song2017pixeldefend}, models perform poorly in these regions not encountered during training. Adversarial training aims to mitigate this by incorporating adversarial examples, generated by specific attacks, into the training set. 
% % This method helps improve model performance on these out-of-distribution samples. 
% In contrast, adversarial purification uses generative models to map adversarial examples back to the clean data distribution, where the model performs well.
% added by Keping
% The need for a zero-shot robust classifier. neural classifiers are not robust. zero-shot classifier CLIP is not robust. 

Image classifiers are usually trained in a supervised manner with training data and evaluated on the corresponding test data until recently several vision-language models have emerged as zero-shot classifiers \citep{li2023your,radford2021learning,li2022grounded}. Among them, CLIP \citep{radford2021learning} is an example that is popular, effective, and efficient. CLIP performs zero-shot classification by forming text prompts ``a photo of a $<$class-name$>$.'' of all the candidate categories, and selecting the class with the highest similarity with the image embedding. Despite its efficacy, when facing adversarial attacks, its accuracy can drop to zero, similarly vulnerable to other neural classifiers. 
 
Existing methods to enhance adversarial robustness follow two primary paths: adversarial training and purification. Adversarial Training (AT) \citep{mkadry2017towards, rebuffi2021fixing, wang2023better} incorporates adversarial examples into model training to boost robustness. It often achieves
% Adversarial defense: adversarial training and purification. 
% zero-shot classification or unseen attack. 
% Existing work based on CLIP. FARE and TeCoA
% their pros and cons. 
% Purification: certain types of datasets, worse than AT. 
% CLIP+Purification 
% Purification risk: purification p(x) of benign and adversarial examples. KL()
% smoothiness of log p(x). 
% is pixel space the best choice? 
% latent space (vision-language aligned). 
% We propose: two variants. 
% four versions comparisons. 
\begin{minipage}[c]{0.55\textwidth}
advanced performance in defending against the same attacks while failing to defend against unseen attacks \citep{chen2023robust}. FARE \citep{schlarmann2024robust} and TeCoA \citep{mao2022understanding} are two AT approaches integrated with CLIP, which enhance CLIP's zero-shot classification robustness while harming clean accuracy significantly and do not generalize to other types of attacks. Adversarial purification \citep{song2017pixeldefend} seeks to eliminate adversarial perturbations by optimizing samples to align them with the distribution of benign samples. Instead of adversarial samples, this approach often requires a generative model that can model the probability of benign samples. It can handle unforeseen attacks but often performs worse than AT methods on seen attacks and has lower inference efficiency. 
In this paper, we aim to produce an adversarially robust zero-shot classifier, so we opt for integrating purification with CLIP. To explore a better purification method, we formalize the purification risk and theoretically
\end{minipage}
\hfill % Horizontal fill to ensure the minipages are separated
\begin{minipage}[c]{0.44\textwidth}
    \includegraphics[width=\textwidth]{figs/zeroshot_classification.pdf} % Adjust the image path and dimensions as necessary
    \captionof{figure}{Adversarial robustness of two CLIPure versions versus adversarially trained CLIP models, evaluated against AutoAttack with $\ell_{\infty}=4/255$ across 13 zero-shot classification datasets.\\}
    \label{figure:13-zeroshot}
  \end{minipage}
analyze what may affect purification performance. Concretely, inspired by \citet{song2020score} that models the diffusion process with bidirectional Stochastic Differential Equations (SDEs) (\cite{anderson1982reverse}), we model the process of attack (i.e., adding perturbations to benign examples) with a forward SDE and purification (i.e., denoising adversarial examples) with a reverse SDE. Within this framework, we evaluate the risk of purification methods by measuring the KL divergence between the joint distributions of the purification and attack steps. After some derivation, we find that purification risk is related to 1) the negative KL divergence of the probability distributions of adversarial and benign examples (i.e., $-$KL$(p(\boldsymbol{x}_{\text{adv}}) || p(\boldsymbol{x}_{\text{ben}}))$), and 2) the $\ell_2$ norm of the gradients of adversarial samples' probability distribution 
% regarding $x_{\text{adv}}$
(i.e., $\nabla \log p(\boldsymbol{x}_{\text{adv}})$). This indicates that purification risk can be affected by: 1) the differences between $p(\boldsymbol{x}_{\text{adv}})$ and $p(\boldsymbol{x}_{\text{ben}})$; 2) the smoothness of $p(\boldsymbol{x}_{\text{adv}})$ and possibly its dimension.

To the best of our knowledge, existing adversarial purification methods are all conducted in pixel space. Given the above factors that can affect purification risk, it is natural to ask: \textit{are there purification approaches better than in pixel space?} As we know, pixel space is high-dimensional and sparse while latent embedding space is denser and smoother. Moreover, multi-modal latent representations are theoretically proven to have better quality than uni-modal \citep{huang2021makes}. CLIP, as a vision-language-aligned encoder model, has shown the superiority of its multi-modal embeddings on many tasks \citep{radford2021learning}. Accordingly, we propose to conduct purification in CLIP's latent space for adversarially robust zero-shot classification. 

Our method - CLIPure has two variants that model the likelihood of images' latent vectors with CLIP differently: 1) CLIPure-Diff: a generative version that employs the DiffusionPrior module of DaLLE-2 \citep{ramesh2022hierarchical} to model the likelihood of an image embedding with a diffusion model; 2) CLIPure-Cos: a discriminative version that models the likelihood with the cosine similarity between the image embedding and text embedding of a blank template ``a photo of a .''. Compared to likelihood modeling in pixel space and uni-modal latent space, we find that both our methods have several orders of magnitude larger KL divergence between the distributions of adversarial and benign samples than the former, and significantly higher than the latter (shown in Figure \ref{figure: distribution of clean and adv examples on px and pz}). Remarkably, CLIPure-Cos becomes the first purification method that does not rely on generative models and thus boosts defense efficiency by two or three orders of magnitude (See Table \ref{table: inference time}). 
%KL$(p(\boldsymbol{x}_{\text{adv}}) || p(\boldsymbol{x}_{\text{ben}}))$

% Disregard the effect of vector length. 
Since CLIP aligns image and text embeddings by their cosine similarities \citep{radford2021learning}, vector lengths are not important to reflect the vector relationship in CLIP's latent space. Hence, during likelihood maximization in CLIPure, we normalize the latent vectors to unit vectors to diminish the effect of vector length. This is critical to our approach, as our experiments indicate that the purification process could be obstructed by vector magnitude and ultimately fail.

We compare the robustness of CLIPure and SOTA methods against the strongest adaptive attacks, AutoAttack\citep{croce2020reliable}, with different $\ell_2$ or $\ell_{\infty}$ bounds on various image classification tasks, including the popular CIFAR10, ImageNet, and 13 other datasets (e.g., CIFAR100, ImagetNet-R) that evaluate zero-shot classification robustness in FARE \citep{schlarmann2024robust} and TeCoA \citep{mao2022understanding}. Note that CLIPure always conducts zero-shot classification and defense without the need for any dataset-specific training while the baselines can be any methods that are current SOTA. We are delighted to see that CLIPure boosts the SOTA robustness on all the datasets by a large margin, e.g., from 71.7\% to 91.1\% on CIFAR10 when $\ell_{\infty}=8/255$, from 59.6\% to 72.6\% on ImageNet when $\ell_{\infty}=4/255$. CLIPure achieves 45.9\% and 108\% relative improvements over previous SOTA - FARE\citep{schlarmann2024robust} regarding average robustness across the 13 zero-shot test datasets facing AutoAttack with $\ell_{\infty}=2/255$ and $4/255$, depicted in Figure \ref{figure:13-zeroshot}. Our work shows that purification in multi-modal latent space is promising for zero-shot adversarial robustness, shedding light on future research including but not limited to image classification. 

% Due to the high-dimensional and sparse data space, it is impractical to train on all adversarial examples to defend against every possible attack, making adversarial training ineffective against unseen attacks \citep{laidlaw2020perceptual}. Moreover, adversarial training can also diminish the zero-shot capabilities of pre-trained models. Conversely, adversarial purification does not target specific attacks but instead aims to purify adversarial examples back to regions where the model demonstrates better accuracy, making it universally applicable.

% However, adversarial purification in pixel space directly manipulating image pixels can corrupt the image, resulting in information loss. Consequently, we reconsider the fundamental principle of adversarial purification to explore what affects the quality of purification. We employ a bidirectional Stochastic Differential Equation (SDE) approach to model both the adversarial attack and the defense process. The forward SDE represents the transformation from benign to adversarial distributions, driven by a specific attack method. The corresponding reverse SDE models the purification process, transitioning from adversarial back to purified sample distributions. Within this framework, we evaluate the risk of purification methods by measuring the KL divergence between the purification actions and the attack process.

% Given the derivation of the purification risk related to the $\ell_2$ norm of the gradient of log-likelihood around the sample, we hypothesize that information loss is due to the sparsity of pixel space, which can lead to abrupt changes in log-likelihood, making purification unstable. Therefore, we propose conducting adversarial purification in latent space, which is denser and smoother, corresponding to a reduced purification risk. Typically, the latent space has a denser distribution, and the log-likelihood near samples is smoother, which could lead to lower purification risk and better purification outcomes.

% Considering the challenges of extracting accurate semantic information from uni-modal spaces, multimodal models like CLIP, which leverage textual information to supervise the latent space, result in high-quality latent embeddings. To determine which type of latent space is more effective for adversarial purification, we compared the ability of uni-modal and multimodal latent spaces to distinguish between distributions of clean and adversarial examples. The results demonstrate that the multimodal latent space, as modeled by CLIP \citep{radford2021learning}, offers a distinct advantage in differentiating these distributions (i.e., $\log p(\boldsymbol{z}^i_{\text{ben}})$ is larger than $\log p(\boldsymbol{z}^i_{\text{adv}})$), which may enhance adversarial purification. Therefore, we utilize the latent space modeled by CLIP for purification. We designate this approach as \textbf{CLIPure} that purifying adversarial examples in latent space via CLIP. 

% Following the concept from \citet{song2017pixeldefend}, we focus on maximizing the log-likelihood of samples for our purification objective. However, unlike their approach, we target the log-likelihood of image embeddings. As for estimating log-likelihood in CLIP's latent space, we propose two methods: a generative version based on the Diffusion model (CLIPure-Diff) and a discriminative version based on cosine similarity (CLIPure-Cos), both designed to calculate $\log p(\boldsymbol{z}^i)$.
% \emph{CLIPure-Diff} uses DiffusionPrior learned upon CLIP's latent space via DaLLE-2 \citep{ramesh2022hierarchical}. DiffusionPrior models $\log p(\boldsymbol{z}^i|\boldsymbol{z}^t)$, the distribution of image embedding $\boldsymbol{z}^i$ under the condition of text embedding $\boldsymbol{z}^t$. To utilize the unconditional diffusion model for $\log p(\boldsymbol{z}^i)$ calculations, we input an blank template text embedding $\bar{\boldsymbol{z}}^i$ as a control condition. For classification \citep{radford2021learning}, it matches with 'a photo of a {class}', e.g., 'a photo of a dog', and our blank template uses the blank template without a class name, i.e., 'a photo of a.'. Regarding \emph{CLIPure-Cos}, we try to use the CLIP model to directly compute the cosine similarity between image vector $\boldsymbol{z}^i$ and $\bar{z}^t$ to estimate $\log p(\boldsymbol{z}^i)$, finding it could also differentiate between benign and adversarial sample distributions since clean samples generally have higher cosine similarity with $\bar{\boldsymbol{z}}^t$ than adversarial samples. This is because CLIP, trained on clean samples, matched image-related text vectors via contrastive learning, making natural data directionally close to text space. In contrast, adversarial samples, being out-of-distribution, had not been matched with text during training, therefore showing lower cosine similarity with text space. Thus, we can use such cosine similarity to purify adversarial samples using a discriminative model. It is surprising that adversarial purification does not necessarily rely on a generative model, which brought a high computational burden during inference, thereby limiting use in efficiency-sensitive scenarios.


% \begin{figure}[t]
% \centering
% \includegraphics[width=\linewidth]{figs/pipline.png} % Adjust the path and options as needed
% \caption{Pipline of CLIPure: }
% \label{figure: pipline}
% \end{figure}

% Given the objective to maximize the log-likelihood $p(\boldsymbol{z}^i)$, we consider whether traditional optimization mechanisms from pixel space purification are applicable in latent space. In CLIP's latent space, trained via contrastive learning, the semantic representation of latent vectors is encoded in their direction, not magnitude, due to reliance on cosine similarity for both training and inference. This involves adjusting the direction of $\boldsymbol{z}^i$ while maintaining its length to maximize the log-likelihood $\log p(\boldsymbol{z}^i)$. Our experiments confirm that this approach significantly enhances robustness. Conversely, direct optimization of latent vectors without disentangled the magnitude often fails to find effective purification paths; however, modifying only direction properties, this challenge is mitigated, making purification feasible and effective in the latent space of CLIP.


% We then validate the effectiveness of latent space purification with polar coordinates against AutoAttack \citep{croce2020reliable}. Our method includes two versions, LLM-Diff based on DiffusionPrior's latent diffusion model \citep{ramesh2022hierarchical} for estimating $\log p(\boldsymbol{z}^i)$ and LLM-CLIP based on CLIP's cosine similarity for estimating $\log p(\boldsymbol{z}^i)$. We compare these with adversarial training, pixel space purification, and uni-modal latent space purification methods, confirming that both LLM-Diff and LLM-CLIP achieved significant improvements in robustness and also demonstrated clear advantages in inference efficiency. We also compare our methods with the widely recognized RobustBench \citep{croce2020reliable} baseline methods and surpass the top-ranked model in adversarial robustness on the CIFAR10 and ImageNet datasets, setting a new SOTA. Notably, our method, based directly on off-the-shelf CLIP and DaLLE-2 for purification, requires no additional training, allowing our method to also perform zero-shot classification tasks. We tested on 13 zero-shot datasets, surpassing existing models in robustness, with an average robustness improvement of 30.9\% by LLM-DP and 35.0\% by LLM-CLIP under the AutoAttack with $\ell_\infty=4/255$ threat model.

%% In summary, we introduce a novel perspective on adversarial purification and propose a new framework that utilizes the latent space of multimodal models for purification in polar coordinates. This innovation revolutionizes the paradigm of adversarial purification by shifting the focus to latent space and no longer depending on the traditionally computationally heavy generative models. Empirically, this approach significantly enhances both the adversarial robustness and inference efficiency.

\section{Related Work}
\textbf{Zero-Shot Image Classification.}
Unlike traditional models that are limited to predefined categories, vision-language models (VLMs) are trained on open-vocabulary data and align the embeddings of images and their captions into a common semantic space. This enables them to perform as zero-shot classifiers by matching the semantics of images to textual categories, offering superior generality and flexibility. CLIP \citep{radford2021learning}, trained on extensive internet image-text pairs, achieves advanced results in zero-shot classification tasks. Additionally, other VLMs including Stable Diffusion \citep{rombach2022high}, Imagen \citep{saharia2022palette}, and DaLLE-2 \citep{ramesh2022hierarchical} also possess zero-shot classification capabilities \citep{li2023your, clark2024text}.

\textbf{Adversarial Purification in Pixel Space.}
A prevalent paradigm of adversarial purification aims to maximize the log-likelihood of samples to remove perturbations in pixel space. Since purification has no assumption of the attack type, enabling it to defend against unseen attacks using pre-trained generative models such as PixelCNN \citep{song2017pixeldefend}, GANs \citep{samangouei2018defense}, VAEs \citep{li2020defense}, Energy-based models \citep{hill2020stochastic, yoon2021adversarial}, and Diffusion Models \citep{nie2022diffusion, chen2023robust, zhang2025causaldiff}. Owing to the capability of diffusion models, diffusion-based adversarial purification achieves state-of-the-art robustness among these techniques.



\textbf{CLIP-based Defense.}
While CLIP achieves impressive accuracy in zero-shot classification, it remains vulnerable to imperceptible perturbations \citep{Fort2021CLIPadversarial, mao2022understanding}. Adversarially training the CLIP model on ImageNet / Tiny-ImageNet \citep{schlarmann2024robust, mao2022understanding, wang2024pre} enhances its robustness but undermines its zero-shot capabilities and struggles against unseen attacks. \citet{choi2025adversarial} suggests smoothing techniques for certification. \citet{li2024one} advocates using robust prompts for image classification, but the defensive effectiveness is limited. Additionally, other research focuses on the out-of-distribution (OOD) robustness of the CLIP model \citep{tu2024closer, galindounderstanding}, which is orthogonal to our adversarial defense objectives.

% \section{Preliminary: Training and Inference Strategies of CLIP}
\section{Preliminary: CLIP as a Zero-shot Classifier}
\label{section: preliminary of CLIP}
In this section, 
% we introduce the training and inference methods of CLIP, which are fundamental to our CLIPure approach. 
we introduce how CLIP is trained and how CLIP acts as a zero-shot classifier. 
CLIP (Contrastive Language Image Pre-training) \citep{radford2021learning}, consists of an image encoder $\text{Enc}^i$ and a text encoder $\text{Enc}^t$. It is trained on 400 million image-text pairs from the internet, aiming to align image embeddings with their corresponding text captions through contrastive learning:
\begin{equation}
\begin{aligned}
\mathcal{L}_{\text{CLIP}} = -\frac{1}{2N} \sum_{n=1}^{N} \left[ \log \frac{\exp(\cos(\boldsymbol{z}^i_n, \boldsymbol{z}^t_n) / \tau)}{\sum_{m=1}^{N} \exp(\cos(\boldsymbol{z}^i_n, \boldsymbol{z}^t_m) / \tau)} + \log \frac{\exp(\cos(\boldsymbol{z}^i_n, \boldsymbol{z}^t_n) / \tau)}{\sum_{m=1}^{N} \exp(\cos(\boldsymbol{z}^i_m, \boldsymbol{z}^t_n) / \tau)} \right],
\end{aligned}
\end{equation}
where $N$ represents the number of image-caption pairs, $\boldsymbol{z}_n^i = \text{Enc}^i(\text{image$_n$})$ and $\boldsymbol{z}_n^t = \text{Enc}^t(\text{text$_n$})$ are the embeddings of the $n$-th image and text respectively, $\tau$ is a temperature parameter, and $\cos(\cdot, \cdot)$ denotes the cosine similarity function.

% This alignment enables CLIP to perform zero-shot classification by matching image embeddings to text embeddings corresponding to class descriptions. Class descriptions typically use a templated format such as 'a photo of a {class}', e.g., 'a photo of a dog' for categorizing dog images. The text embedding for each class $j$ is generated by $\boldsymbol{z}^t_j = \text{Enc}^t(\text{'a photo of a {\text{class}$_j$}'})$. The classification of an image $\boldsymbol{x}$ is then determined by: 
This alignment enables CLIP to perform zero-shot classification by matching image embeddings with text embeddings of a template ``a photo of a $<$class-name$>$.'', where $<$class-name$>$ iterates all the possible classes of a dataset. Without loss of generality, given an image, let $\boldsymbol{z}^i$ denote its CLIP-encoded image embedding, and $\boldsymbol{z}^t_c$ be the text embedding of a possible class description, i.e.,  $\boldsymbol{z}^t_c=\text{Enc}^t(\text{``a photo of a {\text{class} $c$}.''})$. The predicted class $\hat{y}$ is determined by:
\begin{equation} 
\label{equation:zero-shot-classification}
\begin{aligned} 
\hat{y} = \arg \max_c \cos(\boldsymbol{z}^i, \boldsymbol{z}^t_c).
\end{aligned} 
\end{equation} 
% where $\boldsymbol{z}^i = \text{Enc}^i(\boldsymbol{x})$ and $\boldsymbol{z}^t_j$ is the text embedding of the class description. 
For enhanced classification stability, as in \citet{radford2021learning}, we use 80 templates of diverse descriptions in combination with class names, such as ``a \textit{good} photo of $<$class-name$>$''. In our experiments, each class $c$'s embedding, $\boldsymbol{z}^t_c$ in Eq. \ref{equation:zero-shot-classification} is the average text embedding of $c$ paired with all the templates. 

\section{CLIPure: Adversarial Purification in Latent Space via CLIP}
In this section, we outline the methodology of our CLIPure, focusing on adversarial purification within CLIP's latent space. We first define purification risk through a Stochastic Differential Equation (SDE) perspective and derive its lower bound in Section~\ref{section: adversarial purification risk}. Section~\ref{section: purification in CLIP latent space} introduces the rationale for CLIPure to potentially achieve a smaller purification risk and two variants of modeling sample likelihood. In Section~\ref{section: purification on direction}, we propose normalizing latent vectors to diminish the effect of vector length during purification to align with CLIP's latent space modeled using cosine similarity. 
% purifying in CLIP's latent space by disregarding magnitude and solely modifying the embedding direction, which aligns with how CLIP models its latent space.

% In this section, 首先定义了purificatin risk through an SDE view and推导了purification risk 的lower bound in Section~\ref{section: adversarial purification risk}. Section~\ref{section: purification in CLIP latent space} introduce the rationale of CLIP's latent space for smaller purification risk for potential for better adversarial purification. Section~\ref{section: purification on direction} we propose 在CLIP隐空间的净化时消除magnitue 的影响只modify embedding direction property, which alighs with the modeling of CLIP’s latent space.


% our CLIPure including:


% \begin{itemize}
%     \item highlights the smoothness of log-likelihood in latent space may reduce purification risk leading to better robustness.
%     \item 
    
%     \textbf{Which} latent space is better for adversarial purification? Section~\ref{} outlines the multimodal model's (i.e., CLIP) latent space better distinguishes clean from adversarial distributions, thus enhancing purification effectiveness.
%     \item \textbf{How} to purify in CLIP latent space? n Section~\ref{section: purification on direction}, we propose to modify the direction of image embeddings in CLIP’s latent space, focusing on semantic representation rather than magnitude.
% \end{itemize}


\subsection{Adversarial Purification Risk}
\label{section: adversarial purification risk}
Considering that adversarial attacks progressively add perturbations to an image while purification gradually removing noise to restore the original image, we formulate both the attack and purification processes through the lens of Stochastic Differential Equations (SDEs). This framework allows us to propose a measure of purification risk based on the divergence between the attack and purification processes, providing insights into what affects purification effectiveness. 
% designing more effective purification strategies.


We formulate the attack process as a transformation from the benign distribution $p_{\text{ben}}(\boldsymbol{x})$ to an adversarial example distribution $p_{\text{adv}}(\boldsymbol{x})$ by an attack algorithm. Note that for simplicity we use $p(\boldsymbol{x})$ to represent $p_{\text{ben}}(\boldsymbol{x})$ in this paper. 
Take untargeted PGD-attack \citep{mkadry2017towards} for instance, the adversarial attack behavior on a benign sample $\boldsymbol{x}_0$ can be described as:
\begin{equation}
\begin{aligned}
\mathrm{d} \boldsymbol{x} = \alpha \mathrm{sign} (\nabla_{\boldsymbol{x}} \mathcal{L}(\theta; \boldsymbol{x}_t, y_{\text{true}})) \mathrm{d} t + \sigma \mathrm{d} \boldsymbol{w}_t, \quad \boldsymbol{x}_0 \sim p(\boldsymbol{x}), \quad \text{s.t.,} \quad \| \boldsymbol{x}_T - \boldsymbol{x}_0 \|_{\rho} \leq \epsilon,
\label{equation: forward sde of untargeted attack}
\end{aligned}
\end{equation}
where $\alpha$ represents the attack step size, $p(\boldsymbol{x})$ denotes the distribution of benign samples, $\mathcal{L}(\theta; \boldsymbol{x}_t, y_{\text{true}})$ denotes the loss of $\boldsymbol{x}_t$ classified by the model with parameters $\theta$ as the ground truth category $y_{\text{true}}$ at attack step $t$ (where $t \in [0, T]$), $\mathrm{d} \boldsymbol{w}_t$ denotes the Wiener process (Brownian motion). The constant $\sigma$ serves as a scaling factor for the noise component and the adversarial example $\boldsymbol{x}_T$ is bounded by $\epsilon$ in $\ell_{\rho}$ norm.

The corresponding reverse-time SDE \citep{anderson1982reverse} of Eq.~\ref{equation: forward sde of untargeted attack} describes the process from the adversarial example distribution $p_{\text{adv}}$ to the purified sample distribution $p_{\text{pure}}$, and is expressed as:
\begin{equation}
\begin{aligned}
d\boldsymbol{x} = [\alpha \mathrm{sign} (\underbrace{\nabla_{\boldsymbol{x}} \mathcal{L}(\theta; \boldsymbol{x}_t, y_{\text{true}}}_{\text{classifier guidance}})) - \sigma^2 \underbrace{\nabla \log p(\boldsymbol{x}_t}_{\text{purification}})] \mathrm{d} t + \sigma \mathrm{d} \tilde{\boldsymbol{w}}_t, \quad \boldsymbol{x}_T \sim p_{\mathrm{adv}}(\boldsymbol{x}),
\label{equation: reverse sde of untargeted attack}
\end{aligned}
\end{equation}
where $\log p(\boldsymbol{x}_t)$ represents the log-likelihood of $\boldsymbol{x}_t$ concerning the distribution of clean samples, analogous to the score function described in Score SDE \citep{song2020score}, $\mathrm{d} \tilde{\boldsymbol{w}}_t$ represents the reverse-time Wiener process. A detailed discussion on the form of the reverse-time SDE can be found in Appendix~\ref{appendix: discussion of the reverse SDE}.

Note that in the reverse-time SDE, $t$ progresses from $T$ to $0$, implying that $\mathrm{d}t$ is negative. According to Eq.~\ref{equation: reverse sde of untargeted attack}, the reverse SDE aims to increase the sample's log-likelihood while simultaneously decreasing the loss of classifying $\boldsymbol{x}$ to $y_{\text{true}}$. In Eq.~\ref{equation: reverse sde of untargeted attack}, the purification term is related to the common objective of adversarial purification $\boldsymbol{x}_{\text{pure}} = \arg \max_{\boldsymbol{x}} \log p(\boldsymbol{x})$. The classifier guidance term has been employed to enhance purification \citep{zhang2024classifier}, and their objective aligns well with Eq. \ref{equation: reverse sde of untargeted attack}. We will incorporate this guidance term with CLIPure in Appendix ~\ref{section: combination} and see its impact.

% Regarding the classifier guidance term, previous work \citep{zhang2024classifier} has discussed enhancing adversarial robustness by increasing the classifier's confidence, and their purification objective aligns well with the reverse-time SDE framework described in Eq.~\ref{equation: reverse sde of untargeted attack}. Since the ground truth category $y_{\text{true}}$ is unknown, estimating $y_{\text{true}}$ constitutes an additional topic that does not impact our exploration of purification strategies. We will discuss this further in Section~\ref{section: combination}.


Then, we define the joint distribution of the attack process described by the forward SDE in Eq.~\ref{equation: forward sde of untargeted attack} as $\mathcal{P}_{0:T} = p(\boldsymbol{x}_0=\boldsymbol{x}_{\text{ben}}, \boldsymbol{x}_1, ..., \boldsymbol{x}_T=\boldsymbol{x}_{\text{adv}}) \in \mathbb{R}^{(T+1)\times d}$, where each $\boldsymbol{x}_t \in \mathbb{R}^d$. For the purification process defined by the reverse-time SDE in Eq.~\ref{equation: reverse sde of untargeted attack}, we denote the joint distribution as $\mathcal{Q}_{0:T} = p(\boldsymbol{x}_{0}=\boldsymbol{x}_{\text{pure}}, \boldsymbol{x}_{1}, ..., \boldsymbol{x}_{T}=\boldsymbol{x}_{\text{adv}})$. Here, $\boldsymbol{x}_{\text{ben}}$, $\boldsymbol{x}_{\text{adv}}$, and $\boldsymbol{x}_{\text{pure}}$ denotes the benign sample, adversarial example, and purified sample respectively. We define the purification risk $\mathcal{R}(\mathcal{Q})$ by the KL divergence between the reverse SDE (corresponding to the purification process) and the joint distribution of forward SDE (representing the attack process):
\begin{equation}
\begin{aligned}
\mathcal{R}(\mathcal{Q}) & := \mathrm{KL}(\mathcal{Q}_{0:T} \| \mathcal{P}_{0:T}) = \mathrm{KL}(\mathcal{Q}_{0, T} \| \mathcal{P}_{0,T}) + \mathbb{E}_{\mathcal{Q}_{0,T}}[\mathrm{KL}(\mathcal{Q}_{1:T-1|0,T} \| \mathcal{P}_{1:T-1|0,T})] \\
& \geq \  \mathrm{KL}(\mathcal{Q}_{0, T} \| \mathcal{P}_{0,T}).
\end{aligned}
\end{equation}

Then, we focus solely on the purified example, i.e., $\mathrm{KL}(\mathcal{Q}_{0, T} \| \mathcal{P}_{0,T})$ rather than the entire purification trajectory. The forward SDE in Eq.~\ref{equation: forward sde of untargeted attack} describes the transformation from $p(\boldsymbol{x}_{\text{ben}})$ to $p(\boldsymbol{x}_{\text{adv}})$, enabling us to obtain the conditional probability $p(\boldsymbol{x}_{\text{adv}}|\boldsymbol{x}_{\text{ben}})$. Simultaneously, the reverse SDE in Eq.~\ref{equation: reverse sde of untargeted attack} supports the transformation from $p(\boldsymbol{x}_{\text{adv}})$ back to $p(\boldsymbol{x}_{\text{pure}})$, helping us to quantify $p(\boldsymbol{x}_{\text{pure}}|\boldsymbol{x}_{\text{adv}})$. Then we can derive that:
\begin{equation}
\begin{aligned}
\mathcal{R}(\mathcal{Q}) \geq & \  \mathrm{KL}(\mathcal{Q}_{0, T} \| \mathcal{P}_{0,T}) = \mathrm{KL}(p(\boldsymbol{x}_{\text{pure}}, \boldsymbol{x}_{\text{adv}}) \| p(\boldsymbol{x}_{\text{ben}}, \boldsymbol{x}_{\text{adv}}))\\
% = & \ \mathbb{E}_{\boldsymbol{x}_{\text{adv}}} \left[ \mathrm{KL}(p(\boldsymbol{x}_{\text{pure}} | \boldsymbol{x}_{\text{adv}}) \| p(\boldsymbol{x}_{\text{adv}} | \boldsymbol{x}_{\text{ben}})) \right] - \mathrm{KL}(p(\boldsymbol{x}_{\text{adv}}) \| p(\boldsymbol{x}_{\text{ben}})) \\
= & \ \frac{1}{2} \mathbb{E}_{\boldsymbol{x}_{\text{adv}}} \left[ \nabla \log p(\boldsymbol{x}_{\text{adv}})^T \nabla \log p(\boldsymbol{x}_{\text{adv}}) \sigma^2 \Delta t \right] - \mathrm{KL}(p(\boldsymbol{x}_{\text{adv}}) \| p(\boldsymbol{x}_{\text{ben}})),
% \geq & - \mathrm{KL}(p(\boldsymbol{x}_{\text{adv}}) \| p(\boldsymbol{x}_{\text{ben}}))
\label{equation: purification risk}
\end{aligned}
\end{equation}
where $\Delta t$ denotes a small time interval for attack and purification, related to the perturbation magnitude. A detailed proof of the result is provided in Appendix~\ref{appendix: proof of purification quality function}.

% This quantifies the adjustments to the probability density brought about by the purification process from $\boldsymbol{x}_{\text{adv}}$ to $\boldsymbol{x}_{\text{purified}}$.

The result in Eq.~\ref{equation: purification risk} highlights that the lower bound of the purification risk is influenced by two factors: 1) the smoothness of the log-likelihood function at adversarial examples and possibly the sample dimension, as indicated by the $\ell_2$ norm of $\nabla \log p(\boldsymbol{x}_{\text{adv}})$, 2) the differences between the likelihood of clean and adversarial samples in the benign example space.  
% to distinguish between the distributions of clean and adversarial samples, 
% which is a necessary condition for effective adversarial purification. 
% In Section~\ref{section: purification in CLIP latent space}, we will further explore how to lower purification risks for better adversarial purification.


% Due to the high-dimensional nature of pixel space, the distribution is sparse, potentially resulting in large gradients of likelihood, which may explain the information loss in pixel-level purification. This is because the purified image can have significant differences from the benign image, indicated by a higher purification risk.


% Since $\mathrm{KL}(p(\boldsymbol{x}_{\text{adv}}) \| p(\boldsymbol{x}_{\text{ben}}))$ is independent of the purification process and $\sigma^2 \Delta t$ is determined by attack behavior, we simplify the purification risk $\mathcal{R}(\mathcal{Q})$ as:
% \begin{equation}
% \begin{aligned}
% \mathcal{R}(\mathcal{Q}) := \mathbb{E}_{\boldsymbol{x}_{\text{adv}}} \left[ \nabla \log p(\boldsymbol{x}_{\text{adv}})^T \nabla \log p(\boldsymbol{x}_{\text{adv}}) \right]
% \label{equation: purification risk}
% \end{aligned}
% \end{equation}
% This formulation highlights that the purification risk is influenced by the smoothness of the log-likelihood function at the adversarial example, as indicated by the $\ell_2$ norm of $\nabla \log p(\boldsymbol{x}_{\text{adv}})$. Due to the high-dimensional nature of pixel space, the distribution is sparse, potentially resulting in large gradients of likelihood, which may explain the information loss in pixel-level purification. This is because the purified image can have significant differences from the benign image, indicated by a higher purification risk.


\begin{figure}[t]
\centering
% \label{figure:px-pz-multi-uni-modal}
\newlength{\commonheight}
\setlength{\commonheight}{2.7cm}
\begin{subfigure}[b]{0.23\linewidth}  % Width set to one third of text width
    \includegraphics[width=\linewidth, height=\commonheight]{figs/distribution_px_EDM.pdf}
    \caption{Pixel Space}
    \label{subfigure: log-likelihood distribution of px}
\end{subfigure}
\hfill  % Add space between subfigures
\begin{subfigure}[b]{0.23\linewidth}
    \includegraphics[width=\linewidth, height=\commonheight]{figs/distribution_pz_SD.pdf}
    \caption{Vision Latent Space}
    \label{subfigure: log-likelihood distribution of pz by Stable Diffusion}
\end{subfigure}
\hfill  % Add space between subfigures
\begin{subfigure}[b]{0.23\linewidth}
    \includegraphics[width=\linewidth, height=\commonheight]{figs/distribution_pz_DP.pdf}
    \caption{Multi-modal(Diff)}
    \label{subfigure: log-likelihood distribution of pz by DiffusionPrior}
\end{subfigure}
\hfill
\begin{subfigure}[b]{0.23\linewidth}
    \includegraphics[width=\linewidth, height=\commonheight]{figs/distribution_pz_CLIP_log.pdf}
    \caption{Multi-modal(Cos)}
    \label{subfigure: log-likelihood distribution of pz by CLIP}
\end{subfigure}
\caption{Negative log-likelihood estimated by diffusion models on (a) pixel space via EDM, (b) uni-modal latent space via VQVAE, (c) multi-modal latent space via DiffusionPrior, and (d) multi-modal latent space via CLIP (using cosine similarity for log-likelihood estimation). KL represents the value of $\mathrm{KL}(p(\boldsymbol{x}_{\text{adv}}) \| p(\boldsymbol{x}_{\text{ben}}))$ discussed in Section~\ref{section: purification in CLIP latent space} indicating the difference between clean and adversarial example distribution.}
\label{figure: distribution of clean and adv examples on px and pz}
\end{figure}



\subsection{Adversarial Purification in CLIP's Latent Space}
\label{section: purification in CLIP latent space}

In this section, we further explore how to achieve a smaller purification risk $\mathcal{R}(\mathcal{Q})$.
Considering $\mathbb{E}_{\boldsymbol{x}_{\text{adv}}} \left[ \nabla \log p(\boldsymbol{x}_{\text{adv}})^T \nabla \log p(\boldsymbol{x}_{\text{adv}}) \sigma^2 \Delta t \right]$ within $\mathcal{R}(\mathcal{Q})$ in Eq. \ref{equation: purification risk}, standard pixel space purification may lead to higher purification risks due to its sparsity and possibly peaked gradient distribution in high dimensionality. Thus, we are curious to investigate purification in latent space where the distribution of sample densities is more uniform and smoother. 
% the sparse and high-dimensional nature of pixel space often results in large gradients of likelihood. This leads to a high purification risk, potentially causing significant differences between purified and benign images. 
% Thus, we propose purifying adversarial examples in latent space, where the distribution of sample densities is more uniform and the characteristics are smoother. 

$\mathrm{KL}(p(\boldsymbol{x}_{\text{adv}}) \| p(\boldsymbol{x}_{\text{ben}}))$ in Eq. \ref{equation: purification risk} implies
% the ability to distinguish between clean and adversarial distributions significantly impacts the purification risk. 
representations that excel at detecting out-of-distribution adversarial examples are likely to carry a lower risk of purification errors. \citet{huang2021makes} suggest that multi-modal latent spaces offer superior quality compared to uni-modal counterparts. Inspired by this observation, CLIP's well-aligned multi-modal latent space, where image embeddings are guided by the semantics of finer-grained words in an open vocabulary, may provide a foundation for purification with a lower risk.  
% Textual data, offering fine-grained guidance through an open vocabulary, enhances the quality of image embeddings in multimodal models, as supported by \citet{huang2021makes}. 
% Given the advanced capabilities and extensive training of the CLIP model on diverse image-text datasets, we propose leveraging CLIP's latent space for lower purification risk, expecting this approach to yield a better adversarial purification. 

% Regarding image classification, embeddings represent image semantics, aiming to extract key information for categorization. While uni-modal models like classifiers or VAEs lack good semantic supervision, making adversarial purification challenging. In contrast, multimodal models like CLIP use image-corresponding captions during training. This approach aligns image content with textual semantics, providing a higher-quality latent space that enhances adversarial purification.

To validate the efficacy of different spaces in modeling sample likelihood for adversarial purification, we focus on the term $\mathrm{KL}(p(\boldsymbol{x}_{\text{adv}}) \| p(\boldsymbol{x}_{\text{ben}}))$ in the lower bound of the purification risk $\mathcal{R}(\mathcal{Q})$.
% This focus is due to the challenge in fairly comparing the first term across pixel and latent spaces, as it tends to be smaller and less variable.
% To validate this idea, we prioritize understanding how well adversarial and benign distributions can be differentiated in each space to evaluate their potential for effective purification. We specifically examine the term $\mathrm{KL}(p(\boldsymbol{x}_{\text{adv}}) \| p(\boldsymbol{x}_{\text{ben}}))$, which represents the lower bound of the purification risk $\mathcal{R}(\mathcal{Q})$. This is due to the difficulty in fairly comparing the first term $\mathbb{E}_{\boldsymbol{x}_{\text{adv}}} \left[ \nabla \log p(\boldsymbol{x}_{\text{adv}})^T \nabla \log p(\boldsymbol{x}_{\text{adv}}) \sigma^2 \Delta t \right]$ between pixel space and latent space, as this term is usually smaller and less variable.
As illustrated in Figure~\ref{figure: distribution of clean and adv examples on px and pz}, we extract 512 samples from ImageNet \citep{deng2009imagenet} and generate adversarial examples by AutoAttack \citep{croce2020reliable} on the CLIP \citep{radford2021learning} classifier. We explore four types of likelihood modeling approaches:

\textbf{In Pixel Space:}
Sample likelihood of the joint distribution of image pixels is estimated by a generative model (we use an advanced diffusion model - EDM\citep{karras2022elucidating}). Figure~\ref{subfigure: log-likelihood distribution of px} indicates that even EDM struggles to distinguish between clean and adversarial sample distributions at the pixel level. We use the Evidence Lower Bound (ELBO) to estimate log-likelihood, expressed as $\log p_{\theta}(\boldsymbol{x}) \geq -\mathbb{E}_{\boldsymbol{\epsilon},t}[\|\epsilon_{\theta}(\boldsymbol{x}_{t}, t) - \boldsymbol{\epsilon} \|_2^2] + C$, where $\boldsymbol{\epsilon} \sim \mathcal{N}(\boldsymbol{0}, \boldsymbol{I})$, and $C$ is typically negligible \citep{ho2020denoising, song2020score}.

\textbf{In Vision Latent Space:}
The likelihood of the joint distribution of an image embedding in a uni-modal space is estimated with the VQVAE component of the Stable Diffusion (SD) model \citep{rombach2022high}. Note that although SD is multi-modal, its VQVAE component is trained solely on image data and keeps frozen while training the other parameters, making it a vision-only generative model of latent vectors. 
Compared to Figure~\ref{subfigure: log-likelihood distribution of px}, Figure~\ref{subfigure: log-likelihood distribution of pz by Stable Diffusion} demonstrates improved capability in distinguishing clean and adversarial sample distributions. 
% , as shown In Figure~\ref{subfigure: log-likelihood distribution of pz by Stable Diffusion}. 
% As for log-likelihood estimation in vision latent space, we leverage the VQVAE component of the Stable Diffusion (SD) model \citep{rombach2022high}. 
% This VQVAE is trained solely on image data while remains parameter-frozen during the training of the latent diffusion model conditioned on the image-corresponding caption. This makes the VQVAE a purely vision-based component, even though SD is a vision-language model. 

\textbf{In Vision-Language Latent Space:}
For CLIP's latent space, we present two approaches to estimate the log-likelihood of image embeddings, i.e., $\log p(\boldsymbol{z}^i)$:
% , detailed as follows: 

(1) \textbf{Diffusion-based} Likelihood Estimation (corresponds to \textit{CLIPure-Diff}): The DiffusionPrior module in DaLLE-2 (based on CLIP) \citep{ramesh2022hierarchical} models the generative process of image embeddings conditioned on text embeddings. Employing DiffusionPrior, we estimate the log-likelihood $\log p_{\theta}(\boldsymbol{z}^i)$ by conditioning on a blank template ``a photo of a .'': \begin{equation} \log p_{\theta}(\boldsymbol{z}^i) \approx \log p_{\theta}(\boldsymbol{z}^i | \bar{\boldsymbol{z}}^t) = -\mathbb{E}_{\boldsymbol{\epsilon},t}[|\epsilon_{\theta}(\boldsymbol{z}^i_{t}, t, \bar{\boldsymbol{z}}^t) - \boldsymbol{\epsilon} |_2^2] + C, \end{equation} where $\epsilon_{\theta} (\boldsymbol{z}^i, t, \boldsymbol{z}^t)$ is the UNet \citep{ronneberger2015u} in DiffusionPrior \citep{ramesh2022hierarchical}, parameterized by $\theta$, with the noised image embedding $\boldsymbol{z}^i_{t}$ at timestep $t$ under the condition of text embedding $\boldsymbol{z}^t$, and $C$ is a constant typically considered negligible \citep{ho2020denoising}. 

(2) \textbf{Cosine Similarity-based} Likelihood Estimation (corresponds to \textit{CLIPure-Cos}): We estimate $\log p_{\theta}(\boldsymbol{z}^i)$ by computing the cosine similarity between image embedding $\boldsymbol{z}^i$ and the blank template’s text embedding:
\vspace{-1mm}
   \begin{equation}
   \log p_{\theta}(\boldsymbol{z}^i) \propto \cos(\boldsymbol{z}^i, \bar{\boldsymbol{z}}^t).
   % \approx \log \frac{\cos(\boldsymbol{z}^i, \bar{\boldsymbol{z}}^t) + 1}{2}.
   \end{equation}
% Consider the value range, we estimate $p_{\theta}(\boldsymbol{z}^i)$ by $(\cos(\boldsymbol{z}^i, \bar{\boldsymbol{z}}^t) + 1)/2$. For simplicity, we directly optimize $\cos(\boldsymbol{z}^i, \bar{\boldsymbol{z}}^t)$ in our experiments.
Note that by modeling likelihood without using generative models, the defense efficiency can be significantly boosted. The inference time of CLIPure-Cos is only 1.14 times of the vanilla CLIP for zero-shot classification, shown in Table \ref{table: inference time}. 

Figure \ref{subfigure: log-likelihood distribution of pz by DiffusionPrior} and Figure~\ref{subfigure: log-likelihood distribution of pz by CLIP} show that CLIPure-Diff and CLIPure-Cos have several orders larger magnitude KL divergence between clean and adversarial samples than modeling likelihood in pixel space. Modeling likelihood in uni-modal latent space also leads to larger KL divergence than pixel space but is smaller than multi-modal space. It indicates that purification in CLIP's latent space is promising to have lower purification risk and enhance adversarial robustness.


% As shown in Figure~\ref{subfigure: log-likelihood distribution of pz by CLIP}, clean samples exhibit higher semantic similarity with a blank template than adversarial samples, reflecting the alignment of text and image embeddings during training compared with adversarial examples. This method thus provides a feasible way to estimate log-likelihood for adversarial purification in CLIP's latent space, underpinning our CLIPure-Cos approach detailed in Algorithm~\ref{algorithm: latent purification by CLIP and DaLLE2.DiffusionPrior}.


% These approaches showcase how different spaces contribute to the effectiveness of adversarial purification, emphasizing the potential of vision-language latent space in distinguishing between benign and adversarial distributions more effectively than traditional pixel or single-modality latent spaces. Consequently, the latent space of CLIP offers a lower purification risk and holds greater potential for enhancing adversarial robustness.

% Notably, our method of adversarial purification operates in a plug-and-play manner, requiring no additional training of the models.


\begin{algorithm*}[t]
  \caption{\textbf{CLIPure}: Adversarial Purification in Latent Space via CLIP}
  \label{algorithm: latent purification by CLIP and DaLLE2.DiffusionPrior}
  \textbf{Required:}{ An off-the-shelf CLIP model including an image encoder $\text{Enc}^{i}$ and a text encoder $\text{Enc}^t$, textual embedding $\bar{\boldsymbol{z}}^t$ of the blank templates (without class names, e.g., 'a photo of a.'),  purification step $N$, step size $\eta$, and a DiffusionPrior model $\epsilon_{\theta}$ from DaLLE-2 (for generative version).} \\
  \textbf{Input: }{Latent embedding of input example $\boldsymbol{z}^i$} \\
  % \textbf{Output: }{Purified latent embedding}
  \textbf{Output: }{label $y$}
  % \vspace{1mm} 
  
  \textbf{for} $i = 1$ \textbf{to} $N$, \textbf{do}
    
    \quad \textbf{step1: Obtain latent embedding in polar coordinates}

    \vspace{0.5mm}
    \quad \quad direction $\boldsymbol{u} = \boldsymbol{z}^i / \|\boldsymbol{z}^i \|_2^2$, \  magnitude $r = \boldsymbol{z}^i / \boldsymbol{u}$

    % \vspace{0.5mm}
    % \quad \quad magnitude $r = \boldsymbol{z}^i / \boldsymbol{u}$

    \vspace{0.6mm} 
    \quad \textbf{step2: Estimate log-likelihood}
    \vspace{0.6mm}
    
  \noindent\begin{minipage}[t]{.49\textwidth}
    \quad \quad \textbf{CLIPure-Diff} via DiffusionPrior $\epsilon_{\theta}$

    % \vspace{0.4mm}
    % \quad \quad \quad $\bar{\boldsymbol{z}}^t = \text{Enc}^{t}(\mathcal{T})$

    \vspace{0.4mm}
    \quad \quad \quad sample $\boldsymbol{\epsilon} \sim \mathcal{N}(\mathbf{0}, \mathbf{1})$

    \vspace{0.4mm}
    \quad \quad \quad $ \log p(\boldsymbol{z}^i) \leftarrow -\|\epsilon_{\theta}(\boldsymbol{z}^i_{t}, t, \bar{\boldsymbol{z}}^t) - \boldsymbol{\epsilon} \|_2^2 $
    
    

  \end{minipage}%
  \hfill\vline\hfill
  \begin{minipage}[t]{.49\textwidth}
    \textbf{CLIPure-Cos} via CLIP
    
    % \quad \quad \textbf{Discriminative version} via CLIP

    % \vspace{1.5mm}
    % \quad $\bar{\boldsymbol{z}}^t = \text{Enc}^{t}(\mathcal{T})$

    \vspace{1.8mm}
    \quad $ \log p(\boldsymbol{z}^i) \leftarrow \cos(\boldsymbol{z}^i, \bar{\boldsymbol{z}}^t)
     % \approx \log \frac{\cos(\boldsymbol{z}^i, \bar{\boldsymbol{z}}^t) + 1}{2}
     $

    

    
  \end{minipage}

    \vspace{1.2 mm} 
    \quad \textbf{step3: Update latent variable}
    \vspace{0.2mm}
    
    % \noindent\begin{minipage}[t]{.49\textwidth}
    
    % \quad \quad $\boldsymbol{u} \leftarrow \boldsymbol{u} + \eta \nabla_{\boldsymbol{z}^i} \log p(\boldsymbol{z}^i) \cdot \nabla_{\boldsymbol{u}} \boldsymbol{z}^i$
    
    \quad \quad $\boldsymbol{u} \leftarrow \boldsymbol{u} + \eta \frac{\partial \log p(\boldsymbol{z}^i)}{\partial \boldsymbol{z}^i} \cdot \frac{\partial \boldsymbol{z}^i}{\partial \boldsymbol{u}}$
    % , \ $\boldsymbol{z}^i \leftarrow r \cdot \boldsymbol{u}$
    % \quad \quad $\boldsymbol{u} \leftarrow \boldsymbol{u} + \eta \frac{\partial \log p(\boldsymbol{z}^i) }{\partial \boldsymbol{z}^i} \cdot \frac{\partial \boldsymbol{z}^i}{\partial \boldsymbol{u}}$
    % \vspace{0.2mm}
    \ , \  $\boldsymbol{z}^i \leftarrow r \cdot \boldsymbol{u}$
  %   \end{minipage}%
  % \hfill\vline\hfill
  % \begin{minipage}[t]{.49\textwidth}
  % \vspace{1.2 mm} 
  %   \quad 
  %   \vspace{0.2mm}
    
  %   \quad $\boldsymbol{u} \leftarrow \boldsymbol{u} + \eta \frac{\partial \log p(\boldsymbol{z}^i)}{\partial \boldsymbol{z}^i} \cdot \frac{\partial \boldsymbol{z}^i}{\partial \boldsymbol{u}}$
  %   \ , \  $\boldsymbol{z}^i \leftarrow r \cdot \boldsymbol{u}$
  % \end{minipage}

    \quad \textbf{step4: Classification based on purified embedding}

    \vspace{0.2mm}
    \quad \quad predict label $y$ across candidate categories according to Eq.~\ref{equation:zero-shot-classification}
    
    \color{black}{}
    \textbf{end for}
    % \quad return purified latent $\boldsymbol{z}^i$
    
    return predicted label $y$

\end{algorithm*}



% \subsection{Adversarial Purification on Direction Proporties}
\subsection{Adversarial Purification Based on Normalized Unit Vectors}
\label{section: purification on direction}
Typically, purification in pixel space is conducted through gradient ascent on the sample $\boldsymbol{x}$ by using the derivative of the log-likelihood $\log p(\boldsymbol{x})$: $\boldsymbol{x} \leftarrow \boldsymbol{x} + \alpha \nabla \log p(\boldsymbol{x})$, where $\alpha$ denotes the step size for updates. However, given that the cosine similarities between vectors in CLIP's latent space are the criterion for alignment where vector lengths do not take effect, it is inappropriate to directly apply the typical purification update manner to this space. Thus, we normalize the image vectors to unit vectors at each purification step to diminish the effect of vector length. 
% given CLIP's method of modeling semantic characteristics of image embeddings through contrastive learning, directly modifying image embeddings in the same manner is inappropriate. Instead, we propose purifying the image embedding by maximizing log-likelihood focusing on its directional properties, disregarding vector magnitude which may disturb purification stability. This approach also aligns well with the CLIP model's reliance on the embedding direction for zero-shot classification, as discussed in Section~\ref{section: preliminary of CLIP}.
Specifically, we first normalize the vector $\boldsymbol{z}^i = \text{Enc}^i(\boldsymbol{x})$, obtained from the CLIP model image encoder for an input sample $\boldsymbol{x}$ (potentially an adversarial sample), to a unit vector $\boldsymbol{u} = \boldsymbol{z}^i / \|\boldsymbol{z}^i \|_2^2$.
Then we calculate the sample's log-likelihood $\log p(\boldsymbol{z}^i)$ and compute the gradient $\boldsymbol{g}_{\boldsymbol{u}}$ by the chain rule:
\begin{equation}
\begin{aligned}
\boldsymbol{g}_{\boldsymbol{u}} = \frac{\partial \log p(\boldsymbol{z}^i)}{\partial \boldsymbol{u}} =  \frac{\partial \log p(\boldsymbol{z}^i)}{\partial \boldsymbol{z}^i} \cdot \frac{\partial \boldsymbol{z}^i}{\partial \boldsymbol{u}}.
\end{aligned}
\end{equation}
This gradient $\boldsymbol{g}_{\boldsymbol{u}}$ is then used to update the direction $\boldsymbol{z}^i$ for adversarial purification, detailed in Algorithm~\ref{algorithm: latent purification by CLIP and DaLLE2.DiffusionPrior}. Note that CLIPure is based on the original CLIP and does not need any extra training.
% For a detailed description of this process, please refer to Algorithm~\ref{algorithm: latent purification by CLIP and DaLLE2.DiffusionPrior}.

We also attempted adversarial purification by directly optimizing vectors instead of the normalized version. We experimented extensively with various steps, parameters, and momentum-based methods, but found it challenging to achieve robustness over 10\% on ImageNet, indicating that it is difficult to find an effective purification path with vector lengths taking effect.  
% Despite extensive experiments with various steps, parameters, and momentum-based methods, achieving robustness over 10\% on ImageNet (same setting with Table~\ref{table: main result on imagenet}) remained challenging, While we do not deny the feasibility of purification, our findings indicate that focusing solely on vector direction properties facilitates more effective purification.


% \section{Algorithm}
% \label{section: algorithm}
% As explored in Section~\ref{}, we utilize two methods to estimate the log-likelihood of image embeddings in CLIP's latent space, laying the groundwork for adversarial purification through $\arg \max_{\boldsymbol{z}^i} \log p(\boldsymbol{z}^i)$. In this section, we delve into the detailed algorithm of CLIPure, which comprises a generative version (CLIPure-Diff) and a discriminative version (CLIPure-Cos), as described in Algorithm~\ref{algorithm: latent purification by CLIP and DaLLE2.DiffusionPrior}.

% The generative approach leverages the Evidence Lower Bound (ELBO) \citep{ho2020denoising} within a latent diffusion model framework to estimate $\log p(\boldsymbol{z}^i)$. Specifically, we use the DiffusionPrior model trained in CLIP's latent space as part of the DaLLE-2 framework \citep{ramesh2022hierarchical}, denoted as $\epsilon_{\theta} (\boldsymbol{z}^i, t, \boldsymbol{z}^t)$. This model aims to estimate $\log p_{\theta}(\boldsymbol{z}^i | \boldsymbol{z}^t)$. To compute $\log p_{\theta}(\boldsymbol{z}^i)$, we condition on a blank template, such as 'a photo of a .' disregarding the class name. We denote the textual embedding of the blank template as $\bar{\boldsymbol{z}}^t$. The estimation is represented as:
% \begin{equation}
% \begin{aligned}
% \log p_{\theta}(\boldsymbol{z}^i) \approx \log p_{\theta}(\boldsymbol{z}^i | \bar{\boldsymbol{z}}^t) = -\mathbb{E}_{\boldsymbol{\epsilon},t}[\|\epsilon_{\theta}(\boldsymbol{z}^i_{t}, t, \bar{\boldsymbol{z}}^t) - \boldsymbol{\epsilon} \|_2^2] + C, \ \boldsymbol{\epsilon} \sim \mathcal{N}(\boldsymbol{0}, \boldsymbol{I}),
% \label{equation: unconditional log-likelihood estimation by diffusionprior}
% \end{aligned}
% \end{equation}
% where $C$ is a constant typically considered negligible \citep{ho2020denoising}.


% The discriminative approach leverages the off-the-shelf CLIP model \citep{radford2021learning} to compute the cosine similarity with a blank template's text embedding. As detailed in Section~\ref{}, this method can effectively distinguish between adversarial and benign samples. We estimate the log-likelihood of image embeddings $\boldsymbol{z}^i$ by calculating their cosine similarity with the blank template text embedding $\bar{\boldsymbol{z}}^t$:
% \begin{equation}
% \begin{aligned}
% \log p_{\theta}(\boldsymbol{z}^i) := \cos(\boldsymbol{z}^i, \bar{\boldsymbol{z}}^t).
% \label{equation: unconditional log-likelihood estimation by CLIP}
% \end{aligned}
% \end{equation}
% It is very exciting that adversarial purification no longer relies solely on generative models, which significantly reduces the burden of computational efficiency during inference. CLIPure-Cos offers inference efficiency comparable to standard CLIP models. CLIPure-Diff achieves at least a 25-fold improvement over other purification methods based on diffusion models, due to the reduced data dimensionality in latent space purification. For a detailed discussion on inference efficiency, please refer to Section~\ref{section: inference efficiency}.






\section{Experiments}
\subsection{Experimental Settings}
\label{section: experimental settings}
\textbf{Datasets.}
Following the RobustBench \citep{croce2020reliable} settings, we assess robustness on CIFAR-10 and ImageNet. To compare against CLIP-based zero-shot classifiers with adversarial training \citep{schlarmann2024robust, mao2022understanding}, we conduct additional tests across 13 image classification datasets (detailed in Appendix~\ref{appendix: zero-shot datasets}). In line with \citet{schlarmann2024robust}, we randomly sampled 1000 examples from the test set for our evaluations. 
% Notably, our tests utilize the off-the-shelf CLIP model without further training, making all dataset evaluations zero-shot for CLIPure.

\textbf{Baselines.}
We evaluate the performance of \textbf{pixel space purification} strategies employing generative models, including Purify-EBM \citep{hill2020stochastic} and ADP \citep{yoon2021adversarial} based on Energy-Based Models; DiffPure based on Score SDE \citep{nie2022diffusion} and DiffPure-DaLLE2.Decoder based on Decoder of DaLLE2 \citep{ramesh2022hierarchical}; GDMP based on DDPM \citep{ho2020denoising}; and likelihood maximization approaches such as LM-EDM \citep{chen2023robust}     based on the EDM model \citep{karras2022elucidating} and LM-DaLLE2.Decoder which adapts LM to the Decoder of DaLLE-2. Furthermore, we perform an ablation study adapting LM to the latent diffusion model to achieve \textbf{ uni-modal latent space purification} using the Stable Diffusion Model \citep{rombach2022high}, denoted as LM-StableDiffusion. Furthermore, we also compare with the state-of-the-art \textbf{adversarial training} methods such as AT-ConvNeXt-L \citep{singh2024revisiting} and AT-Swin-L \citep{liu2024comprehensive}, along with innovative training approaches such as MixedNUTS \citep{bai2024mixednuts} and MeanSquare \citep{amini2024meansparse}, and methods that utilize DDPM and EDM to generate adversarial samples for training dataset expansion: AT-DDPM \citep{rebuffi2021fixing} and AT-EDM \citep{wang2023better}. Moreover, we also compare CLIPure with adversarially trained CLIP model according to TeCoA \citep{mao2022understanding} and FARE \citep{schlarmann2024robust}. Additionally, we also evaluate the performance of \textbf{classifiers without defense} strategies, including CLIP, WideResNet (WRN), and Stable Diffusion.
% Our baselines include purification in \textbf{pixel space} and in \textbf{uni-modal latent space}, as well as \textbf{adversarial training} strategies.
Due to space constraints, we list the details of baselines in Section~\ref{sec: detailed baselines in appendix} in Appendix~\ref{appendix: experimental settings}.


\textbf{Adversarial Attack.}
Following the setup used by FARE \citep{schlarmann2024robust}, we employ AutoAttack's \citep{croce2020reliable} strongest white-box APGD for both targeted and untargeted attacks across 100 iterations, focusing on an $\ell_{\infty}$ threat model (typically $\epsilon=4/255$ or $\epsilon=8/255$) as well as $\ell_2$ threat model (typically $\epsilon=0.5$) for evaluation.
We leverage \textbf{adaptive attack} with full access to the model parameters and inference strategies, including the purification mechanism to expose the model's vulnerabilities thoroughly. It means attackers can compute gradients against the entire CLIPure process according to the chain rule:
% \begin{equation}
% \begin{aligned}
$\frac{\partial \mathcal{L}}{\partial \boldsymbol{x}} = \frac{\partial \mathcal{L}}{\partial \boldsymbol{z}^i_{\text{pure}}} \cdot \frac{\partial \boldsymbol{z}^i_{\text{pure}}}{\partial \boldsymbol{z}^i} \cdot \frac{\partial \boldsymbol{z}^i}{\partial \boldsymbol{x}}.$
We also evaluate CLIPure against the \textbf{BPDA} (short for Backward Pass Differentiable Approximation) with \textbf{EOT} (Expectation Over Transformation) \citep{hill2020stochastic} and latent-based attack \citep{rombach2022high} detailed in Appendix~\ref{sec: detailed setting of attacks}.


% We evaluate our CLIPure and baselines against adaptive attacks with AutoAttack \citep{croce2020reliable}, BPDA with EOT \citep{athalye2018obfuscated}, and latent-based attack \citep{shukla2023generating}, detailed in Section~\ref{sec: detailed setting of attacks} in Appendix~\ref{appendix: experimental settings}.


% \begin{table}[t]
% \centering
% \caption{Comparison with the top-ranked models in their respective settings on RobustBench \citep{croce2020reliable} for CIFAR-10 dataset against $\ell_{\infty}$ threat model ($\epsilon=8/255$) and $\ell_2$ threat model ($\epsilon=0.5$), as well as ImageNet against $\ell_{\infty}$ threat model ($\epsilon=4/255$).}
% \label{table: robustbench results}
% \begin{tabular}{lccccccc}
% \toprule
% \multirow{2}{*}{Method} & \multicolumn{2}{c}{CIFAR10} &  \multicolumn{2}{c}{CIFAR10} &  \multicolumn{2}{c}{ImageNet} \\
% \cmidrule(lr){2-3} \cmidrule(lr){4-5} \cmidrule(lr){6-7}
%  & Clean acc & $\ell_{\infty}=8/255$ & Clean acc & $\ell_{2}=0.5$ &  Clean acc & $\ell_{\infty}=4/255$ \\
% \midrule 
% Rank \#1 & 93.7 & 73.7 & 95.5 & 85.0 & 78.0 & 59.6 \\
% \bf LLM-DP & 95.2 &  88.0 & 95.2 &  91.3 & 73.1 & 65.0 \\
% \bf LLM-CLIP & 95.6 & \bf 91.1 & 95.6 & \bf 91.9 & 76.3 & \bf 72.6 \\
% \bottomrule
% \end{tabular}
% \end{table}


\begin{table}[t]
\caption{Comparison of performance against AutoAttack under $\ell_{\infty}$ ($\epsilon=8/255$) and $\ell_2$ ($\epsilon=0.5$) threat model on CIFAR-10 dataset, showcasing various defense methods including adversarial training and purification mechanisms. We highlight defenses that operate in pixel or latent space (``defense space''), and indicates the modal information used in defense strategies with ``V'' for Vision and ``V-L'' for Vision-Language multimodal representations. We use underlining to highlight the best robustness for baselines, and bold font to denote the state-of-the-art (SOTA) across all methods.}
\label{table: main result on CIFAR10}
\begin{center}
\setlength{\tabcolsep}{2.3pt}
\begin{tabular}{clcc|ccc}
% \hline 
% \multirow{4}{*}{\makecell[c]{\text{Adv.} \\ \text{Train}}} 
\toprule
& \multirow{2}{*}{\textbf{Method}} & \multirow{2}{*}{\makecell[c]{\textbf{Defense} \\ \textbf{Space}}} & \multirow{2}{*}{\textbf{\makecell[c]{\text{Latent} \\ \text{Modality}}}} & \multirow{2}{*}{\textbf{\makecell[c]{\text{Clean} \\ \text{Acc (\%)}}}} & \multicolumn{2}{c}{\textbf{Robust Acc (\%)}} \\
& & & & & $\ell_\infty=8/255$ & $\ell_2=0.5$ \\ 
\midrule
\multirow{3}{*}{\makecell[c]{\text{w/o} \\ \text{Defense}}} & WRN-28-10 \citep{zagoruyko2016wide}& - & V & 94.8 & 0.0&0.0\\
& StableDiffusion \citep{li2023your} & - &V & 87.8 & 0.0 & 38.8 \\
& CLIP \citep{radford2021learning} & - & V-L & 95.2 & 0.0 & 0.0 \\
\hline
\multirow{4}{*}{\makecell[c]{\text{Adv.} \\ \text{Train}}} & AT-DDPM-$\ell_{2}$ \citep{rebuffi2021fixing}& Pixel & V & 93.2 & 49.4 & 81.1 \\
& AT-DDPM-$\ell_{\infty}$ \citep{rebuffi2021fixing} & Pixel & V & 88.8 & 63.3 & 64.7 \\
& AT-EDM-$\ell_{2}$ \citep{wang2023better}& Pixel & V & 95.9 & 53.3 & \underline{84.8} \\
& AT-EDM-$\ell_{\infty}$ \citep{wang2023better} & Pixel & V & 93.4 & 70.9 & 69.7 \\
\hline
\multirow{2}{*}{Other}& TETRA \citep{blau2023classifier} & Pixel & V & 88.2 & 72.0 & 75.9 \\
& RDC \citep{chen2023robust} & Pixel & V & 89.9 & \underline{75.7} &82.0 \\
% ARL \citep{bartoldson2024adversarial} & Pixel & Vision & 93.7 & 73.7 &  \\
% MeanSparse \citep{amini2024meansparse}& Pixel & Vision & 93.2 & 72.1 & \\
% FARE \citep{schlarmann2024robust} & Latent & Vision-Language & 77.7 & 5.1 & 30.6 \\
% TeCoA \citep{mao2022understanding} & Pixel & Vision-Language & 79.6 & 8.1 & 32.4 \\
\hline
\multirow{5}{*}{Purify} & LM - StableDiffusion & Latent & V & 37.9 & 6.9 & 8.6 \\
& DiffPure \citep{nie2022diffusion} & Pixel & V & 90.1 & 71.3 & 80.6 \\
& LM - EDM \citep{chen2023robust} & Pixel & V & 87.9 & 71.7 & 75.0 \\
% LM - Diff & Pixel & Vision & & & \\
& \textbf{Our CLIPure - Diff} & Latent & V-L & 95.2 & \bf 88.0 & \bf 91.3 \\
& \textbf{Our CLIPure - Cos} & Latent & V-L & 95.6 & \bf 91.1 & \bf 91.9 \\
\bottomrule
\end{tabular}
\vspace{-2mm}
\end{center}
\end{table}


\subsection{Main Results}
In this section, we compare CLIPure-Diff and CLIPure-Cos with SOTA methods and examine the model robustness from various perspectives. 
% In this section, we compare our CLIPure approach, which includes both the CLIPure-Diff and CLIPure-Cos versions, with top-ranked methods on RobsutBench and baselines.
% Importantly, our approach achieves efficient inference capabilities without relying on generative models. Moreover, our CLIPure uses the off-the-shelf CLIP model, fully preserving its zero-shot capabilities and simultaneously significantly improving its robustness, thus marking a significant advancement in addressing the long-standing issue of neural network vulnerability to adversarial attacks.

% \textbf{Ranking on RobustBench.} As for overall performance, we benchmark against the top-ranked models on the well-recognized RobustBench \citep{croce2020reliable} for CIFAR-10 and ImageNet datasets in Table~\ref{table: robustbench results}. Both versions of CLIPure demonstrate remarkable performance. Notably, CLIPure-Cos reduces the gap between robustness and clean accuracy to only 4.4\% under the $\ell_{\infty}$ attack setting on CIFAR-10. On the ImageNet dataset, CLIPure-Cos significantly narrows this gap to just 4.3\% from the clean accuracy benchmark of 76.3\%. 

\textbf{Discussion of CLIPure-Diff and CLIPure-Cos.} We compare the performance of CLIPure-Diff and CLIPure-Cos against AutoAttack across CIFAR-10, ImageNet, and 13 datasets in Tables~\ref{table: main result on CIFAR10}, \ref{table: main result on imagenet}, and \ref{table: zero-shot results} respectively, as well as defense against BPDA+EOT and latent-based attack in Table~\ref{table: BPDA+EOT on CIFAR10} and \ref{table: latent attack on CIFAR10} in Appendix~\ref{appendix: more expxperimental results}. 
Results indicate that both models have achieved new SOTA performance on all the datasets and CLIPure-Cos uniformly outperforms CLIPure-Diff in clean accuracy and robustness. It is probably because the DiffusionPrior component used in CLIPure-Diff models the generation process by adding noise to the original image embeddings encoded by CLIP without diminishing the effect of vector magnitude. 
% This underperformance of CLIPure-Diff is likely due to its latent diffusion model (LDM) applying noise directly to image embeddings without eliminating the inference of vector magnitude, unlike CLIPure-Cos, which optimizes semantic direction, aligning with CLIP's cosine similarity approach. 
Specifically, the noise is added as $\boldsymbol{z}^i_{t} = \sqrt{\bar{\alpha}} \boldsymbol{z}^i_{0} + \sqrt{1-\bar{\alpha}} \boldsymbol{\epsilon}, \boldsymbol{\epsilon} \sim \mathcal{N}(\boldsymbol{0}, \boldsymbol{I})$. We expect that the performance will be boosted if the generation process also eliminates the effect of vector length.  
In contrast, CLIPure-Cos has no such issue and it models the likelihood with cosine similarities that are consistent with CLIP. 
% Specifically, LDM uses a pixel space diffusion mechanism, where noise is added as $\boldsymbol{z}^i_{t} = \sqrt{\bar{\alpha}} \boldsymbol{z}^i_{0} + \sqrt{1-\bar{\alpha}} \boldsymbol{\epsilon}, \boldsymbol{\epsilon} \sim \mathcal{N}(\boldsymbol{0}, \boldsymbol{I})$. 
% Addressing these methodological gaps in LDM could further enhance CLIPure-Diff's robustness, representing a promising direction for future research to fully realize its potential as demonstrated in Figure~\ref{subfigure: log-likelihood distribution of pz by DiffusionPrior}.


% \textbf{Comparison with Adversarial Training.} CLIPure showcases superior robustness against unseen attacks, outperforming adversarially trained CLIP models even without additional training on the evaluation dataset.
% In Table~\ref{table: main result on CIFAR10}, CLIPure is compared against state-of-the-art adversarial training methods like AT-$\ell_2$ and AT-$\ell_\infty$ from \citet{rebuffi2021fixing} and \citet{wang2023better}, demonstrating more stable and superior performance across various threat models. Notably, since CLIPure does not train on adversarial examples, each attack scenario is effectively an unseen attack.

% In Table~\ref{table: main result on imagenet} and Table~\ref{table: zero-shot results}, against adversarial training methods FARE \citep{schlarmann2024robust} and TeCoA \citep{mao2022understanding}, CLIPure-Diff and CLIPure-Cos surpass the best recorded robustness by significant margins (65.0\% and 72.6\%, respectively).
% Additionally, Table~\ref{table: zero-shot results} shows our comparison of zero-shot performance across 13 datasets. Unlike FARE and TeCoA, which compromise zero-shot performance due to training on ImageNet, CLIPure retains clean accuracy and enhances adversarial robustness, validating its effectiveness without the need for additional training.




\textbf{Comparisons with Purification in Pixel Space.}
Compared to the SOTA purification methods
% of adversarial samples has traditionally been conducted in pixel space via a generative model, with the current state-of-the-art method 
, DiffPure \citep{nie2022diffusion} and LM-EDM \citep{chen2023robust} (the likelihood maximization approach based on the advanced diffusion model EDM \citep{karras2022elucidating}), our CLIPure achieves better adversarial robustness (shown in Table \ref{table: main result on CIFAR10} and \ref{table: main result on imagenet}) as well as superior inference efficiency (shown in Table \ref{table: inference time}). Table~\ref{table: main result on CIFAR10} shows that on CIFAR10 under the $\ell_\infty$ and $\ell_2$ threat models, our CLIPure-Cos showed improvements of 27.1\% and 22.5\% over LM-EDM and improvements of 27.8\% and 14.0\% over DiffPure. On the ImageNet dataset, shown in Table~\ref{table: main result on imagenet}, CLIPure-Cos achieves a relative increment of 288.2\% over LM-EDM and 63.5\% over DiffPure. Additionally, the inference efficiency of our CLIPure is multiple orders of magnitude higher than DiffPure and LM-EDM (see Table \ref{table: inference time}). 
% , details of which can be seen in Table~\ref{table: inference time} in Section~\ref{section: inference efficiency}.

Moreover, in Table~\ref{table: main result on imagenet}, we also compared the pixel space purification based on the Decoder component of DaLLE-2 \citep{ramesh2022hierarchical} (LM-DaLLE2.Decoder).
% that models that generation of an image based on the latent vector output by DiffusionPrior component.
% where the Decoder modeled the pixel distribution of images under the guidance of image semantics.
% We utilized an unconditional diffusion model to estimate the log-likelihood $\log p(\boldsymbol{x})$ by dropping the guidance signal for purification in pixel space. 
Results show that the LM-DaLLE2.Decoder suffers from a drop in accuracy, possibly because the diffusion architecture ADM \citep{dhariwal2021diffusion} it employs is slightly inferior to EDM in terms of generation quality.
% The drop in clean accuracy is a common issue faced by pixel space methods, but our LLM approach has shown significant mitigation, even achieving clean accuracy on the ImageNet dataset that is comparable to discriminative models (WideResNet-50).




\textbf{Comparisons with Purification in Uni-modal Latent Space.}
We evaluate latent space purification using Stable Diffusion \citep{rombach2022high} (i.e., LM-StableDiffusion) on the CIFAR-10 dataset, with results detailed in Table~\ref{table: main result on CIFAR10}. As discussed in Section~\ref{section: purification in CLIP latent space}, Stable Diffusion (SD) models an uni-modal image latent space through its VQVAE \citep{rombach2022high}. Direct comparisons are challenging due to SD and CLIP utilizing different training data and fundamentally distinct backbones (diffusion model versus discriminative model). As noted in \citet{li2023your}, SD, as an generative model, is designed for generation rather than classification tasks, so its zero-shot classification performance is not good enough, which limits its potential for robustness. 
% Consequently, while LM-StableDiffusion experiences a significant drop in clean accuracy, it indicates potential for improvement.


\textbf{Comparisons with CLIP-based Baselines.}
Figure \ref{figure:13-zeroshot} and Table~\ref{table: zero-shot results} show the zero-shot adversarial robustness on 13 datasets, compared to CLIP-based baselines enhanced with adversarial training, i.e., FARE \citep{schlarmann2024robust} and TeCoA \citep{mao2022understanding}, CLIPure-Diff and CLIPure-Cos surpass their best-reported average robustness by significant margins, 39.4\% and 45.9\% when $\ell_{\infty}=2/255$, 95.4\% and 108\% when the attacks are stronger with $\ell_{\infty}=4/255$. FARE and TeCoA often have much lower clean accuracy than vanilla CLIP due to their adversarial training on ImageNet, which harms zero-shot performance on other datasets.  
Additionally, Table \ref{table: main result on imagenet} shows that even on ImageNet against the attacks they have been trained with, CLIPure still outperforms them by a huge margin (65\% and 72.6\% versus 33.3\% and 44.3\%). 
The robustness of CLIPure against unseen attacks is much better than their performance on seen attacks, showing that we are on the right path of leveraging the power of pre-trained vision-language models.  


\textbf{More Experiments and Analysis.} Due to space constraints, in the Appendix, we include a detailed case study, showcasing the visualization of image embeddings during the purification process using a diffusion model in Figure~\ref{figure: purification process shown by generated images} in Appendix~\ref{section: case study}. Figure~\ref{subfigure: combination} in Appendix~\ref{section: combination} illustrates the effects of combining our approach with adversarial training and pixel space purification methods, while Figure~\ref{subfigure: guidance} displays the outcomes of integrating classifier guidance. Additionally, we employ T-SNE to visualize the distribution of image and text embeddings in CLIP's latent space in Figure~\ref{subfigure: polar plot of embeddings} and analyze the impact of step size on performance in Figure~\ref{subfigure: hyperparam}.



\begin{table}[t]
\vspace{-2mm}
\caption{Performance comparison of defense methods on ImageNet against AutoAttack with $\ell_{\infty}$ threat model ($\epsilon=4/255$). Indicates whether the methods use ImageNet training set for training as zero-shot. ``V'' stands for Vision, and ``V-L'' for Vision-Language multimodal representations.}
\label{table: main result on imagenet}
\begin{center}
\setlength{\tabcolsep}{4pt}
\begin{tabular}{clccc|cc}
% \hline 
% \multirow{4}{*}{\makecell[c]{\text{Adv.} \\ \text{Train}}} 
% \multirow{2}{*}{\textbf{Method}} & \multirow{2}{*}{\makecell[c]{\text{Defense} \\ \text{Space}}} & \multirow{2}{*}{\textbf{\makecell[c]{\text{Latent} \\ \text{Modality}}}} & \multirow{2}{*}{\textbf{\makecell[c]{\text{Clean} \\ \text{Acc (\%)}}}} & \multicolumn{2}{c}{\textbf{Robust Acc (\%)}} \\
% & & & & \textbf{AA }\bm{$\ell_\infty$} & \textbf{AA }\bm{$\ell_2$} \\ 
\toprule
&\bf Method & \bf \makecell[c]{\text{Defense} \\ \text{Space}} & \bf \makecell[c]{\text{Latent} \\ \text{Modality}}  & \bf \makecell[c]{\text{Zero} \\ \text{-Shot}} & \bf \textbf{\makecell[c]{\text{Clean} \\ \text{Acc (\%)}}} & \textbf{\makecell[c]{\text{Robust} \\ \text{Acc (\%)}}} \\
\midrule
\multirow{2}{*}{\makecell[c]{\text{w/o} \\ \text{Defense}}} & WideResNet-50 \citep{zagoruyko2016wide} & - & V& \ding{55} & 76.5 & 0.0 \\
& CLIP \citep{radford2021learning} & - & V-L &\checkmark & 74.9 & 0.0 \\
\hline
% ARL \citep{bartoldson2024adversarial} & Pixel & Vision & &  \\
\multirow{4}{*}{\makecell[c]{Adv.\\ Train}}& FARE \citep{schlarmann2024robust} & Latent & V-L &\ding{55} & 70.4 & 33.3 \\
& TeCoA \citep{mao2022understanding} & Latent & V-L &\ding{55} & 75.2 & 44.3 \\
& AT-ConvNeXt-L \citep{singh2024revisiting} & Pixel & V &\ding{55} & 77.0 & 57.7\\
& AT-Swin-L \citep{liu2024comprehensive} & Pixel & V &\ding{55} & 78.9 & \underline{59.6} \\
\hline
\multirow{2}{*}{Others} & MixedNUTS \citep{bai2024mixednuts}& Pixel & V &\ding{55} & 81.5 & 58.6 \\
& MeanSparse \citep{amini2024meansparse}& Pixel & V &\ding{55} & 78.0 & \underline{59.6} \\
\hline
\multirow{5}{*}{Purify} & LM - DaLLE2.Decoder & Pixel & V-L & \checkmark &36.9 & 9.2 \\
& DiffPure - DaLLE2.Decoder & Pixel & V-L & \checkmark & 31.2 & 9.0 \\
& LM - EDM \citep{chen2023robust} & Pixel & V &\ding{55} & 69.7 & 18.7\\
& DiffPure \citep{nie2022diffusion} & Pixel & V &\ding{55} & 71.2 & 44.4\\
% LM - Diff & Pixel & Vision & & & \\
& \textbf{Our CLIPure - Diff} & Latent & V-L &\checkmark & 73.1 & \bf 65.0 \\
& \textbf{Our CLIPure - Cos} & Latent & V-L &\checkmark & 76.3 & \bf 72.6 \\
\bottomrule
\end{tabular}
\vspace{-4mm}
\end{center}
\end{table}


% \subsection{Analysis}
% \label{section: analysis}




\vspace{-2mm}
\subsection{Inference Efficiency}
\label{section: inference efficiency}
We evaluate the inference efficiency of our CLIPure and baseline models by measuring the inference time of an averaged over 100 examples from the CIFAR-10 dataset on a single 4090 GPU. The results are displayed in Table~\ref{table: inference time}. Our CLIPure-Cos innovatively uses a discriminative approach for adversarial sample purification, achieving inference times comparable to those of discriminative models.
Traditional purification methods typically leverage a generative model like diffusion and significantly involve complex inference procedures. Such complexity limits their applicability in efficiency-sensitive tasks, such as autonomous driving. 
% It also demonstrates robust performance under actual adversarial attacks and does not require additional training, thus offering superior advantages in terms of efficient and robust inference.



\subsection{CLIP Variants}
To investigate the influence of backbone models and their scale, we evaluate the performance of CLIPure-Cos across various versions of CLIP \citep{radford2021learning} and its variants, including EVA2-CLIP \citep{sun2023eva}, CLIPA \citep{li2024inverse}, and CoCa \citep{yu2205coca}. The results in Table~\ref{table: performance across backbones} and Figure~\ref{figure: backbones for CLIPure} indicate: 
1) Larger models generally exhibit stronger capabilities (i.e., clean accuracy), which in turn enhances the robustness of CLIPure. Notably, the CLIPure-Cos with CLIPA with ViT-H-14 achieves the best performance, with robustness reaching 79.3\% on ImageNet—a truly remarkable achievement.
2) CLIPA-based CLIPure-Cos outperforms other CLIP models of comparable size in both accuracy and robustness.
3) ViT-based CLIPure-Cos models show better results than those based on ResNet.


\begin{table}[t]
\centering
% \caption{Comparison of different methods in terms of inference time. Dis. denotes the purification with a discriminative model, and Gen. denotes that with a generative model.}
\caption{Comparison of different methods in terms of average inference time over 100 samples on CIFAR10. ``Relative Time'' expresses each method's inference time as a multiple of the CLIP model's time for classification. Dis. denotes purification with a discriminative model, and Gen. denotes that with a generative model.}
\label{table: inference time}
\begin{tabular}{lccccc}
\toprule
Method& CLIPure-Cos & CLIPure-Diff & DiffPure & LM-EDM & CLIP  \\ \midrule
Gen. or Dis. & Dis. & Gen. & Gen. & Gen. & Dis.  \\ 
Inference Time (s)& 4.1 x $10^{-4}$ & 0.01 & 2.22 & 0.25 & 3.6 x $10^{-4}$ \\
Relative Time  & 1.14x & 27.78x & 6166.67x & 694.44x & 1x\\
\bottomrule
\end{tabular}
\end{table}

\begin{figure}[t]
\centering
% \captionsetup{labelfont={color=blue}}
% \newlength{\commonheight}
% \setlength{\commonheight}{3.5cm}
\begin{subfigure}[b]{0.49\linewidth}
    \includegraphics[width=\linewidth]{figs/backbone_CLIP.pdf}
    % \caption{\color{blue}{Different Version of CLIP}}
    \label{subfigure: backbone_CLIP}
\end{subfigure}
% \hfill  % Add space between subfigures
\begin{subfigure}[b]{0.49\linewidth}  % Width set to one third of text width
    \includegraphics[width=\linewidth]{figs/backbones.pdf}
    % \caption{\color{blue}{Different backbone of CLIPure}}
    \label{subfigure: backbones}
\end{subfigure}
\vspace{-3mm}
\caption{Accuracy and robustness (detailed in Table~\ref{table: performance across backbones}) of CLIPure-Cos for (Left) various versions of CLIP and (Right) different backbone models. The bubble size represents the number of parameters, which is also indicated alongside each bubble. The left figure presents CLIPure-Cos based on ResNet-based models (including RN50, RN101, RN50x64, marked in red) and ViT-based models (ViT-B-16, ViT-B-32, ViT-L-14, ViT-H-14, and ViT-bigG-14, marked in blue). The right figure depicts CLIPure-Cos based on CLIP (including ViT-B-16, ViT-L-14, ViT-H-14, ViT-bigG-14, marked in blue), EVA2-CLIP (including ViT-B-16 and ViT-L-14, marked in green), CLIPA (including ViT-L-14 and ViT-H-14, marked in red), and CoCa (ViT-B-32, marked in yellow). 
The blue dashed line represents the point where robust accuracy equals clean accuracy, serving as the upper bound of robustness, since successful defense against adversarial attacks hinges on accurate classification.
}
\label{figure: backbones for CLIPure}
\vspace{-2mm}
\end{figure}
% \textbf{Comparison with Euclidean Space Purification.}
% We compare purification in Cartesian coordinate latent space for both the LLM-Diff and LLM-CLIP models. Extensive experimental searches covered various purification steps and step size parameters, alongside trials of momentum-based optimization methods. We found it challenging to devise an effective strategy for achieving robustness on the ImageNet dataset to exceed 10\%, echoing the difficulties encountered with stable diffusion. We do not deny the feasibility of purification in Euclidean space but we found that purification in polar coordinates easily facilitates effective purification.


% \section{Discussion}

\section{Conclusion}
We develop CLIPure, a novel adversarial purification method that operates within the CLIP model's latent space to enhance adversarial robustness on zero-shot classification without additional training. CLIPure consists of two variants: CLIPure-Diff and CLIPure-Cos, both achieving state-of-the-art performance across diverse datasets including CIFAR-10 and ImageNet. CLIPure-Cos, notably, does not rely on generative models, significantly enhancing defense efficiency. Our findings reveal that purification in a multi-modal latent space holds substantial promise for adversarially robust zero-shot classification, pointing the way for future research that extends beyond image classification.


\subsection*{Acknowledgments}
This work is supported by the Strategic Priority Research Program of the Chinese Academy of Sciences, Grant No. XDB0680101, CAS Project for Young Scientists in Basic Research under Grant No. YSBR-034, the National Key Research and Development Program of China under Grants No. 2023YFA1011602, the National Natural Science Foundation of China (NSFC) under Grants No. 62302486, the Innovation Project of ICT CAS under Grants No. E361140, the CAS Special Research Assistant Funding Project, the project under Grants No. JCKY2022130C039, the Strategic Priority Research Program of the CAS under Grants No. XDB0680102, and the NSFC Grant No. 62441229.

% \clearpage

\bibliography{arxiv}
\bibliographystyle{arxiv}

\clearpage
\appendix
\section{Discussion of the Reverse-Time SDE}
\label{appendix: discussion of the reverse SDE}
% discussion of the form of reverse SDE

\begin{figure}[t]
\centering
\includegraphics[width=\linewidth]{figs/schrodinger_bridge.png} % Adjust the path and options as needed
\caption{Comparison of the Schrödinger Bridge \citep{schrodinger1932theorie} framework with our adversarial attack and purification process}
\label{figure: schrodinger bridge compared with attack and purification process}
\end{figure}

Motivated by the Schrödinger Bridge theory \citep{schrodinger1932theorie, tang2024simplified} which models the transition between two arbitrary distributions, we utilize this theoretical framework to model the transformation between benign and adversarial sample distributions. 

Within the Schrödinger Bridge framework, the forward Stochastic Differential Equation (SDE) describes a pre-defined transition from a data distribution $p_{\text{data}}$ to a prior distribution $p_{\text{prior}}$, noted as a reference density $p_{\text{ref}} \in \mathscr{P}_{T+1}$, the space of joint distributions on $\mathbb{R}^{(T+1)\times d} $, where $T$ represents the timesteps and $d$ denotes the data dimension. Consequently, we have $p_{\text{ref};0}=p_{\text{data}}$ and $p_{\text{ref};T}=p_{\text{prior}}$. Both  $p_{\text{data}}$ and $p_{\text{prior}}$ are defined over 
the data space $\mathbb{R}^d$. The optimal solution $\pi^*$, starting from $p_{\text{prior}}$ and transferring to $p_{\text{data}}$, is termed the Schrödinger Bridge, described by the reverse SDE.

Specifically, the forward SDE is expressed as:
\begin{equation}
\begin{aligned}
\mathrm{d} \boldsymbol{x} = [f(\boldsymbol{x}_t, t) + g^2(t) \nabla \log \Psi(\boldsymbol{x}_t, t)] \mathrm{d} t + g(t) \mathrm{d} \boldsymbol{w}_t, \quad x_0 \sim p_{\text{data}},
\end{aligned}
\end{equation}
where $f(\boldsymbol{x}_t, t)$ is the drift function and $g(t)$ is the diffusion term. $\boldsymbol{w}_t$ is the standard Wiener process, and $\Psi$ and $\tilde{\Psi}$ are time-varying energy potentials constrained by the following Partial Differential Equations (PDEs):
\begin{equation}
\begin{aligned}
& \frac{\partial \Psi}{\partial t} = -\nabla_{\boldsymbol{x}} \Psi^T f - \frac{1}{2} \operatorname{Tr}(g^2 \nabla_{\boldsymbol{x}}^2 \Psi) \\
& \frac{\partial \tilde{\Psi}}{\partial t} = -\nabla_{\boldsymbol{x}} \tilde{\Psi}^T f - \frac{1}{2} \operatorname{Tr}(g^2 \nabla_{\boldsymbol{x}}^2 \tilde{\Psi}),
\label{equation in appendix: SDE constraints}
\end{aligned}
\end{equation}
such that $\Psi(\boldsymbol{x}, 0) \tilde{\Psi}(\boldsymbol{x}, 0) = p_{\text{data}}, \Psi(\boldsymbol{x}, T) \tilde{\Psi}(\boldsymbol{x}, T) = p_{\text{prior}}$. 

Similarly, we define the mutual transformation between clean and adversarial sample distributions using coupled SDEs. The forward SDE describes the transformation process from benign to adversarial example distributions corresponding to an attacker's process. For an untargeted PGD attack, this is represented as:
\begin{equation}
\begin{aligned}
\mathrm{d} \boldsymbol{x} = \alpha \mathrm{sign} (\nabla_{\boldsymbol{x}} \mathcal{L}(\theta; \boldsymbol{x}_t, y_{\text{true}})) \mathrm{d} t + \sigma \mathrm{d} \boldsymbol{w}_t , \quad \boldsymbol{x}_0 \sim p_{\text{ben}},
\label{equation in appendix: forward SDE of untargeted PGD attack}
\end{aligned}
\end{equation}
where $\alpha$ represents the attack step size, $\mathcal{L}(\theta; \boldsymbol{x}_t, y_{\text{true}})$ denotes the loss when the adversarial example $\boldsymbol{x}_t$ is classified by the model with parameters $\theta$ as the ground truth category $y_{\text{true}}$ at attack step $t$ (where $t \in [0, T]$), $\mathrm{d} \boldsymbol{w}_t$ denotes the Wiener process (Brownian motion) to express more general cases, and $\sigma$ is a constant. The adversarial example $\boldsymbol{x}_T$ is bound by $\epsilon$ in $l_p$ norm, i.e., $\|\boldsymbol{x}_T - \boldsymbol{x}_0 \|_p \leq \epsilon$.

According to Eq.~\ref{equation in appendix: forward SDE of untargeted PGD attack}, we obtain formulation of the drift function and diffusion term is
\begin{equation}
\begin{aligned}
f(\boldsymbol{x}_t, t) & = \alpha \mathrm{sign} (\nabla_{\boldsymbol{x}} \mathcal{L}(\theta; \boldsymbol{x}_t, y_{\text{true}})) \\
g(t) & = \sigma
\end{aligned}
\end{equation}

According to \citep{anderson1982reverse, song2020score}, the corresponding reverse-time SDE is:
\begin{equation}
\begin{aligned}
\mathrm{d}\boldsymbol{x} = [f(\boldsymbol{x}_t, t) - g^2(t) \nabla_{\boldsymbol{x}} \log p(\boldsymbol{x}) ] \mathrm{d}t + \sigma \mathrm{d} \tilde{\boldsymbol{w}}_t.
\end{aligned}
\end{equation}

Thus, we can derive the corresponding reverse SDE of Eq.~\ref{equation in appendix: forward SDE of untargeted PGD attack} as
\begin{equation}
\begin{aligned}
dx = [\alpha \mathrm{sign} (\nabla_{\boldsymbol{x}} \mathcal{L}(\theta; \boldsymbol{x}_t, y_{\text{true}})) - \sigma^2 \nabla \log p(\boldsymbol{x}_t)] \mathrm{d} t + \sigma \mathrm{d} \tilde{w}_t, \quad \boldsymbol{x}_T \sim p_{\mathrm{adv}},
\label{equation in appendix: reverse SDE of untargeted PGD attack}
\end{aligned}
\end{equation}
where $\log p(\boldsymbol{x})$ is the log-likelihood of the marginal distribution of benign samples, also known as the score function in Score SDE \citep{song2020score}. $\mathrm{d}\tilde{w}_t$ denotes the reverse-time Wiener process as time flows from $t=T$ to $t=0$.

Note that the coupled SDEs described in Eq.~\ref{equation in appendix: forward SDE of untargeted PGD attack} and Eq.~\ref{equation in appendix: reverse SDE of untargeted PGD attack} are special cases of Eq.~\ref{equation in appendix: SDE constraints} when $\nabla \log \Psi(\boldsymbol{x}_t, t) = 0 $ and $\nabla \log \tilde{\Psi}(\boldsymbol{x}_t, t) = \nabla \log p(\boldsymbol{x}_t)$.


Without loss of generality, other adversarial attack methods can typically be described using the forward SDE. Therefore, our proposed approach of modeling adversarial attacks and purification processes using coupled SDEs is universally applicable.


% definition

% schordinger bridge

% forward and reverse SDE

% special case

% targeted attack

% discussion of targeted attack, other type of attack as well as OOD transformation

\section{Detailed Proof of Purification Risk Function}
\label{appendix: proof of purification quality function}

We first define the joint distribution of the attack process described by the forward SDE in Eq.~\ref{equation: forward sde of untargeted attack} as $\mathcal{P}_{T+1}$, which spans $\mathbb{R}^{(T+1)\times d}$. Thus we have $\mathcal{P}_{0} = p_{\text{ben}}$ and $\mathcal{P}_{T} = p_{\text{adv}}$. For the purification process, we designate the joint distribution as $\mathcal{Q}_{T+1}$, also defined over $\mathbb{R}^{(T+1)\times d}$, with $\mathcal{Q}_{0} = p_{\text{adv}}$ and $\mathcal{Q}_{T} = p_{\text{pure}}$. $T$ represents the timestep, $p_{\text{ben}}$, $p_{\text{adv}}$ and $p_{\text{pure}}$ denote the distribution of benign, adversarial, and purified samples.

According to the results of \citet{leonard2014some}, we have 
\begin{equation}
\begin{aligned}
\mathrm{KL}(\mathcal{Q}_{0:T} | \mathcal{P}_{0:T}) & = \mathrm{KL}(\mathcal{Q}_{0, T} | \mathcal{P}_{0,T}) + \mathbb{E}_{\mathcal{Q}_{0,T}}[\mathrm{KL}(\mathcal{Q}_{1:T-1|0,T} | \mathcal{P}_{1:T-1|0,T})]. \\
% & \geq \mathrm{KL}(\mathcal{Q}_{0, T} | \mathcal{P}_{0,T}),
\end{aligned}
\end{equation}

Since we focus on the differences between the purified and benign examples rather than the entire purification path, we concentrate on $\mathrm{KL}(\mathcal{Q}_{0, T} \| \mathcal{P}_{0,T})$. Thus, we continue to derive:
\begin{equation}
\begin{aligned}
& \mathrm{KL}(\mathcal{Q}_{0, T} \| \mathcal{P}_{0,T}) \\
= & \ \mathrm{KL}(p(\boldsymbol{x}_{\mathrm{adv}}, \boldsymbol{x}_{\mathrm{pure}}) | p_{\mathrm{attack}}(\boldsymbol{x}_{\text{ben}}, \boldsymbol{x}_{\mathrm{adv}}))\\
= & \ \int \int p(\boldsymbol{x}_{\text{adv}}) p(\boldsymbol{x}_{\text{pure}} | \boldsymbol{x}_{\text{adv}}) \log \frac{p(\boldsymbol{x}_{\text{adv}}) p(\boldsymbol{x}_{\text{pure}} | \boldsymbol{x}_{\text{adv}})}{p(\boldsymbol{x}_{\text{ben}}) p(\boldsymbol{x}_{\text{adv}} | \boldsymbol{x}_{\text{ben}})} \, d\boldsymbol{x}_{\text{pure}} \, d\boldsymbol{x}_{\text{adv}} \\
= & \ \int p(\boldsymbol{x}_{\text{adv}}) \left( \int p(\boldsymbol{x}_{\text{pure}} | \boldsymbol{x}_{\text{adv}}) \log \frac{p(\boldsymbol{x}_{\text{pure}} | \boldsymbol{x}_{\text{adv}})}{p(\boldsymbol{x}_{\text{adv}} | \boldsymbol{x}_{\text{ben}})} \, d\boldsymbol{x}_{\text{pure}} \right) d\boldsymbol{x}_{\text{adv}} - \int p(\boldsymbol{x}_{\text{adv}}) \log \frac{p(\boldsymbol{x}_{\text{adv}})}{p(\boldsymbol{x}_{\text{ben}})} \, d\boldsymbol{x}_{\text{adv}} \\
= & \ \mathbb{E}_{\boldsymbol{x}_{\text{adv}}} \left[ \mathrm{KL}(p(\boldsymbol{x}_{\text{pure}} | \boldsymbol{x}_{\text{adv}}) \| p(\boldsymbol{x}_{\text{adv}} | \boldsymbol{x}_{\text{ben}})) \right] - \mathrm{KL}(p(\boldsymbol{x}_{\text{adv}}) \| p(\boldsymbol{x}_{\text{ben}})),
\end{aligned}
\end{equation}
where $p(\boldsymbol{x}_{\text{adv}}|\boldsymbol{x}_{\text{ben}})$ represents the conditional probability transitioning from $\boldsymbol{x}_{\text{ben}}$ to $\boldsymbol{x}_{\text{adv}}$ as described by the forward SDE, and $q(\boldsymbol{x}_{\text{pure}}|\boldsymbol{x}_{\text{adv}})$ represents the conditional probability transitioning from $\boldsymbol{x}_{\text{adv}}$ to $\boldsymbol{x}_{\text{pure}}$ as described by the reverse SDE. This formulation underscores the effectiveness of the purification process by measuring how well the adversarial transformations are reversed.

Given that the perturbations added by attackers are typically imperceptible, over a small time interval $\Delta t$, the solution of the forward SDE described by Eq.~\ref{equation: forward sde of untargeted attack} approximates a normal distribution:
\begin{equation}
\begin{aligned}
p({\boldsymbol{x}_{\text{adv}}|\boldsymbol{x}_{\text{ben}}}) \sim \mathcal{N}(\boldsymbol{x}_{\text{ben}} + \alpha \operatorname{sign}(\nabla_x \mathcal{L}) \Delta t, \sigma^2 \Delta t).
\end{aligned}
\end{equation}
Similarly, the reverse SDE process described by Eq.~\ref{equation: reverse sde of untargeted attack} approximates a normal distribution:
\begin{equation}
\begin{aligned}
p(\boldsymbol{x}_{\text{pure}}|\boldsymbol{x}_{\text{adv}}) \sim \mathcal{N}(\boldsymbol{x}_{\text{adv}} - (\alpha \operatorname{sign}(\nabla_{\boldsymbol{x}_{\text{adv}}} \mathcal{L}(\theta; \boldsymbol{x}_{\text{adv}}, y_{\text{true}}))) - \sigma^2 \nabla \log p(\boldsymbol{x}_{\text{adv}})) \Delta t, \sigma^2 \Delta t).
\end{aligned}
\end{equation}

The KL divergence formula for two normal distributions is:
\begin{equation}
\begin{aligned}
KL(\mathcal{N}(\mu_0, \Sigma_0) \parallel \mathcal{N}(\mu_1, \Sigma_1)) = \frac{1}{2} \left(\operatorname{tr}(\Sigma_1^{-1} \Sigma_0) + (\mu_1 - \mu_0)^T \Sigma_1^{-1} (\mu_1 - \mu_0) - k + \log\frac{\det \Sigma_1}{\det \Sigma_0}\right).
\end{aligned}
\end{equation}
When $\Sigma_0 = \Sigma_1 = \sigma^2 \Delta t$, the KL divergence reduced as
\begin{equation}
\begin{aligned}
KL(\mathcal{N}(\mu_0, \Sigma_0) \parallel \mathcal{N}(\mu_1, \Sigma_1)) = \frac{1}{2 \sigma^2 \Delta t} \|\mu_1 - \mu_0\|^2.
\end{aligned}
\end{equation}
Thus we have
\begin{equation}
\begin{aligned}
& \mathbb{E}_{\boldsymbol{x}_{\text{adv}}} \left[ \mathrm{KL}(p(\boldsymbol{x}_{\text{pure}} | \boldsymbol{x}_{\text{adv}}) \| p(\boldsymbol{x}_{\text{adv}} | \boldsymbol{x})) \right] \\
= & \ \mathbb{E}_{\boldsymbol{x}_{\text{adv}}} \left[ \nabla \log p(\boldsymbol{x}_{\text{ben}})^T \nabla \log p(\boldsymbol{x}_{\text{ben}}) \sigma^2 \Delta t \right].
\end{aligned}
\end{equation}
Thus we have proved the result in Eq.~\ref{equation: purification risk} that
\begin{equation}
\begin{aligned}
& \mathrm{KL}(\mathcal{Q}_{0, T} | \mathcal{P}_{0,T}) 
= \mathrm{KL}(p(\boldsymbol{x}_{\mathrm{adv}}, \boldsymbol{x}_{\text{pure}}) | p(\boldsymbol{x}, \boldsymbol{x}_{\mathrm{adv}}))\\
= & \ \mathbb{E}_{\boldsymbol{x}_{\text{adv}}} \left[ \mathrm{KL}(q(\boldsymbol{x}_{\text{pure}} | \boldsymbol{x}_{\text{adv}}) \| p(\boldsymbol{x}_{\text{adv}} | \boldsymbol{x}_{\text{ben}})) \right] - \mathrm{KL}(p(\boldsymbol{x}_{\text{adv}}) \| p(\boldsymbol{x})) \\
= & \frac{1}{2} \mathbb{E}_{\boldsymbol{x}_{\text{adv}}} \left[ \nabla \log p(\boldsymbol{x}_{\text{adv}})^T \nabla \log p(\boldsymbol{x}_{\text{adv}}) \sigma^2 \Delta t \right] - \mathrm{KL}(p(\boldsymbol{x}_{\text{adv}}) \| p(\boldsymbol{x}_{\text{ben}})).
\end{aligned}
\end{equation}


% \section{Proof of Purification Risk in Latent Space}
% \label{appendix: proof of purification risk in latent space}
% In this section, we make a further comparison of the purification risk between pixel space and latent space.

% According to the result in Eq.~\ref{equation: purification risk} that the risk of purification process $\mathcal{Q}$ is defined as
% \begin{equation}
% \begin{aligned}
% \mathcal{R}(\mathcal{Q}) = \mathbb{E}_{\boldsymbol{x}_{\text{adv}}} \left[ \nabla \log p(\boldsymbol{x})^T \nabla \log p(\boldsymbol{x}) \sigma^2 \Delta t \right],
% \end{aligned}
% \end{equation}
% where $\boldsymbol{x}_{\text{adv}}$ represents the adversarial examples, $\log p(\boldsymbol{x})$ is the log-likelihood of $\boldsymbol{x}$, $\sigma$ denotes the scale factor of Wiener process in Eq.~\ref{equation: forward sde of untargeted attack}, $\Delta t $ is the small time interval related to the perturbation.

% % 借用flow-based model对隐空间与原始空间

\begin{figure}[t]
\centering
% \captionsetup{labelfont={color=blue}}
\includegraphics[width=\linewidth]{figs/CLIPure_process.png} % Adjust the path and options as needed
\caption{Illustration of the process of CLIPure including the purification in latent space and zero-shot classification, detailed in Algorithm~\ref{algorithm: latent purification by CLIP and DaLLE2.DiffusionPrior}.}
\label{figure: CLIPure process}
\end{figure}

\section{More Experimental Settings}
\label{appendix: experimental settings}
\subsection{Datasets for Zero-Shot Performance}
\label{appendix: zero-shot datasets}
In order to evaluate the zero-shot performance of our CLIPure and adversarially trained CLIP model including FARE \citep{schlarmann2024robust} and TeCoA \citep{mao2022understanding}, we follow the settings and evaluation metrics used by FARE and outlined in the CLIP-benchmark\footnote{Available at \url{https://github.com/LAION-AI/CLIP_benchmark/}}. We evaluate the clean accuracy and adversarial robustness across 13 datasets against an $\ell_{\infty}$ threat model with $\epsilon=4/255$ and $\epsilon=2/255$. The datasets for zero-shot performance evaluation include CalTech101 \citep{griffin2007caltech}, StanfordCars \citep{krause20133d}, CIFAR10, CIFAR100 \citep{krizhevsky2009learning}, DTD \citep{cimpoi2014describing}, EuroSAT \citep{helber2019eurosat}, FGVC Aricrafts \citep{maji2013fine}, Flowers \citep{nilsback2008automated}, ImageNet-R \citep{hendrycks2021many}, ImageNet-S \citep{wang2019learning}, PCAM \citep{veeling2018rotation}, OxfordPets \citep{parkhi2012cats}, and STL10 \citep{coates2011analysis}.

\subsection{Adversarial Attaks}
\label{sec: detailed setting of attacks}
% \textbf{Adaptive Attack with AutoAttack.}
% Following the setup used by FARE \citep{schlarmann2024robust}, we employ AutoAttack's \citep{croce2020reliable} strongest white-box APGD for both targeted and untargeted attacks across 100 iterations, focusing on an $\ell_{\infty}$ threat model (typically $\epsilon=4/255$ or $\epsilon=8/255$) as well as $\ell_2$ threat model (typically $\epsilon=0.5$) for evaluation.
% We leverage \textbf{adaptive attack} with full access to the model parameters and inference strategies, including the purification mechanism to expose the model's vulnerabilities thoroughly. It means attackers can compute gradients against the entire CLIPure process according to the chain rule:
% % \begin{equation}
% % \begin{aligned}
% $\frac{\partial \mathcal{L}}{\partial \boldsymbol{x}} = \frac{\partial \mathcal{L}}{\partial \boldsymbol{z}^i_{\text{pure}}} \cdot \frac{\partial \boldsymbol{z}^i_{\text{pure}}}{\partial \boldsymbol{z}^i} \cdot \frac{\partial \boldsymbol{z}^i}{\partial \boldsymbol{x}}.$
% % \label{equation: adaptive attack}
% % \end{aligned}
% % \end{equation}

% Moreover, as some baseline purification methods are non-differentiable, we also compare our method under the \textbf{BPDA} (short for Backward Pass Differentiable Approximation) with \textbf{EOT} (Expectation Over Transformation)=20 setting \citep{hill2020stochastic} to ensure robust comparisons across a broad range of baselines, where EOT helps mitigate inaccuracies introduced by randomness. Additionally, we consider \textbf{latent-based attack} \citep{rombach2022high} that targets the CLIPure's latent space, ensuring a comprehensive evaluation of CLIPure's defense strategies. Due to space constraints, we represent the performance under BPDA+EOT and latent-based attack methods in Table~\ref{table: BPDA+EOT on CIFAR10} and Table~\ref{table: latent attack on CIFAR10} in Appendix~\ref{appendix: other attacks}.


% \subsection{Adversarial Robustness Against Other Attacks}
% \label{appendix: other attacks}

\begin{table}[t]
\caption{Performance comparison of defense methods on CIFAR-10 against BPDA with EOT-20 under $\ell_{\infty}$ ($\epsilon=8/255$) norm bound. We use underlining to highlight the best robustness for baselines, and bold font to denote the state-of-the-art (SOTA) across all methods.}
\label{table: BPDA+EOT on CIFAR10}
\begin{center}
\begin{tabular}{lcc}
% \hline 
% \multirow{4}{*}{\makecell[c]{\text{Adv.} \\ \text{Train}}} 
\toprule
\bf Method &\textbf{\makecell[c]{\text{Clean} \\ \text{Acc (\%)}}} & \textbf{\makecell[c]{\text{Robust} \\ \text{Acc (\%)}}} \\
\midrule
FARE \citep{schlarmann2024robust} & 77.7 & 8.5 \\
TeCoA \citep{mao2022understanding} & 79.6 & 10.0 \\
Purify - EBM \citep{hill2020stochastic} & 84.1 & 54.9 \\
LM - EDM \citep{chen2023robust} & 83.2 & 69.7 \\
ADP \citep{yoon2021adversarial} & 86.1 & 70.0 \\
RDC \citep{chen2023robust} & 89.9 & 75.7 \\
GDMP \citep{wang2022guided} & 93.5  & 76.2 \\
DiffPure \citep{nie2022diffusion} & 90.1 & \underline{81.4} \\
\textbf{Our CLIPure - Diff}  & 95.2 & \bf 94.6  \\
\textbf{Our CLIPure - Cos} & 95.6  & \bf 93.5  \\
\bottomrule
\end{tabular}
\end{center}
\end{table}

\textbf{BPDA with EOT.}
To compare our approach with purification methods that do not support gradient propagation, we conducted evaluations using the BPDA (Backward Pass Differentiable Approximation) \citep{athalye2018obfuscated} attack method, augmented with EOT=20 (Expectation over Transformation) to mitigate potential variability due to randomness. As depicted in Table~\ref{table: BPDA+EOT on CIFAR10}, our CLIPure-Cos model significantly outperforms traditional pixel-space purification methods based on generative models such as EBM \citep{hill2020stochastic}, DDPM \citep{chen2023robust}, and Score SDE \citep{nie2022diffusion}, despite being a discriminative model. Notably, our CLIPure demonstrates higher robust accuracy compared to the AutoAttack method under BPDA+EOT-20 attack. This is attributed to the fact that adaptive white-box attacks are precisely engineered to target the model, leading to a higher attack success rate.


\textbf{Latent-based Attack.}
\citet{shukla2023generating} introduced a latent-based attack method, utilizing generative models such as GANs \citep{goodfellow2014generative} to modify the latent space representations and generate adversarial samples. We applied this approach to the CIFAR-10 dataset's test set, and the results, displayed in Table~\ref{table: latent attack on CIFAR10}, reveal that unbounded attacks achieve a higher success rate than bounded attacks. Despite these aggressive attacks, our CLIPure method retains an advantage over baseline approaches. This is attributed to CLIPure leveraging the inherently rich and well-trained latent space of the CLIP model, which was trained on diverse and sufficient datasets, allowing it to effectively defend against unseen attacks without the need for additional training.


\begin{table}[t]
\caption{Performance comparison of defense methods on CIFAR-10 against unbounded latent-based attack \citep{shukla2023generating} using a generative adversarial network (GAN) \citep{goodfellow2014generative}. }
\label{table: latent attack on CIFAR10}
\begin{center}
\begin{tabular}{lcc}
% \hline 
% \multirow{4}{*}{\makecell[c]{\text{Adv.} \\ \text{Train}}} 
\toprule
\bf Method &\textbf{\makecell[c]{\text{Clean} \\ \text{Acc (\%)}}} & \textbf{\makecell[c]{\text{Robust} \\ \text{Acc (\%)}}} \\
\midrule
ResNet-18 & 80.2 & 12.4 \\
FARE \citep{schlarmann2024robust} & 77.7 & 61.7 \\
TeCoA \citep{mao2022understanding} & 79.6 & \underline{63.7} \\
\textbf{Our CLIPure - Diff}  & 95.2 & \bf 69.2 \\
\textbf{Our CLIPure - Cos} & 95.6  & \bf 70.8 \\
\bottomrule
\end{tabular}
\end{center}
\end{table}


\subsection{Purification Settings}
\label{section: purification settings}
For the blank template used in purification, we employ 80 diverse description templates combined with class names, such as ``a \textit{good} photo of $<$class-name$>$'' to enhance stability, following zero-shot classification strategies. Consistently, each class $c$'s text embedding, $\boldsymbol{z}^t_c$, as referred to in Eq. \ref{equation:zero-shot-classification}, is computed as the average text embedding combined with all templates.


For CLIPure-Diff, we use the off-the-shelf DiffusionPrior model from DaLLE 2 \citep{ramesh2022hierarchical}. During purification, following \citet{chen2023robust}, we estimate the log-likelihood at a single timestep and then perform a one-step purification. We conduct a total of 10 purification steps on image embeddings obtained from CLIP's image encoder, with a step size of 30, focusing on gradient ascent updates directly on the vector direction of image embeddings. Experiments indicate that timesteps in the range of 900 to 1000 yield better defense outcomes, hence we select this range for timestep selection.
Regarding CLIPure-Cos, we utilize the off-the-shelf CLIP model, similarly conducting 10 purification steps, each with a step size of 30.

\subsection{Baselines}
\label{sec: detailed baselines in appendix}
In this section, we discuss the detailed experimental settings of the baselines mentioned in Section~\ref{section: experimental settings}.
For the CIFAR-10 dataset results shown in Table~\ref{table: main result on CIFAR10}, we use the off-the-shelf Stable Diffusion model as provided by \citet{li2023your}, employing it as a zero-shot classifier. Our LM-StableDiffusion method adapts this model to a likelihood maximization approach. Other methods were applied directly according to the model in the original papers for CIFAR-10 without additional training.

For the ImageNet dataset experiments detailed in Table~\ref{table: main result on imagenet}, FARE \citep{schlarmann2024robust} and TeCoA \citep{mao2022understanding} involve testing checkpoints that were adversarially trained on the ImageNet dataset specifically for an $\ell_{\infty}$ threat model with $\epsilon=4/255$. The LM-DaLLE2.Decoder method utilizes the Decoder module from the ViT-L-14 model of DaLLE2 \citep{ramesh2022hierarchical} provided by OpenAI, adapted to the likelihood maximization method. DiffPure-DaLLE2.Decoder adapts the same Decoder module to the DiffPure \citep{nie2022diffusion} purification method. Other methods use models as provided and trained in the original papers.

Regarding the zero-shot performance across 13 datasets shown in Table~\ref{table: zero-shot results}, we opt for the adversarially trained models of TeCoA \citep{mao2022understanding} and FARE \citep{schlarmann2024robust} with an $\ell_{\infty}$ threat model at $\epsilon=4/255$, as they exhibited superior performance compared to those trained with $\epsilon=2/255$.

In the results presented in Table~\ref{table: BPDA+EOT on CIFAR10} and Table~\ref{table: latent attack on CIFAR10}, we conduct an adversarial evaluation using models as specified in the literature without any additional fine-tuning.


\color{black}{}
\section{More Experimental Results}
\label{appendix: more expxperimental results}



\begin{table}[t]
% \captionsetup{labelfont={color=blue}}
\caption{Performance of CLIPure-Cos based on various versions of CLIP \citep{radford2021learning}, EVA2-CLIP \citep{sun2023eva}, CLIPA \citep{li2024inverse}, and CoCa \citep{yu2205coca} under the $\ell_{\infty}$ threat model ($\epsilon=4/255$) on the ImageNet dataset. "Param" denotes the number of parameters. The prefix "RN" refers to ResNet-based \citep{he2016deep} methods, while "ViT" indicates Vision Transformer-based \citep{alexey2020image} approaches.}
\label{table: performance across backbones}
\begin{center}
\setlength{\tabcolsep}{4pt}
\begin{tabular}{ccl|cccc}
% \hline 
% \multirow{4}{*}{\makecell[c]{\text{Adv.} \\ \text{Train}}} 
\toprule
\multirow{2}{*}{\textbf{Model}} & \multirow{2}{*}{\textbf{Version}} & \multirow{2}{*}{\textbf{Param (M)}} & \multicolumn{2}{c}{\textbf{w/o Defense}}  & \multicolumn{2}{c}{\textbf{CLIPure}} \\
& & & Acc (\%) & Rob (\%) & Acc (\%) & Rob (\%) \\ 
\midrule
\multirow{8}{*}{\makecell[c]{\text{CLIP} \\ \text{ \citep{radford2021learning}}}} & RN50 & 102 & 59.7& 0.0& 60.0& 52.9\\
 &RN101 & 119       & 61.6        & 0.0               & 61.9             & 55.5          \\
&RN50x64         & 623       & 72.0        & 0.0               & 72.3             & 69.5          \\
&ViT-B-16        & 149       & 68.1        & 0.0               & 68.2             & 63.0          \\
&ViT-B-32        & 151       & 62.0        & 0.0               & 62.0             & 58.1          \\
&ViT-L-14        & 427       & 74.9        & 0.0               & 76.3             & 72.6          \\
&ViT-H-14        & 986       & 77.2        & 0.0               & 77.4             & 74.4          \\
&ViT-bigG-14    & 2539      & 80.4        & 0.0               & 80.4             & 77.6          \\
\hline
\multirow{2}{*}{\makecell[c]{\text{EVA2-CLIP} \\ \text{ \citep{sun2023eva}}}}& ViT-B-16      & 149       & 74.6        & 0.0               & 74.7             & 71.7          \\
&ViT-L-14      & 427       & 81.0        & 0.0               & 80.7             & 78.7          \\
\hline
\multirow{2}{*}{\makecell[c]{\text{CLIPA} \\ \text{ \citep{li2024inverse}}}}&ViT-L-14 & 414 & 79.0 & 0.0 & 79.0 & 77.2 \\
&ViT-H-14  & 968       & 81.8        & 0.0               & \bf 81.5             & \bf 79.3          \\
\hline
\makecell[c]{\text{CoCa} \\ \text{ \citep{yu2205coca}}} & ViT-B-32 & 253 &64.2&0.0&63.8&59.8\\
\bottomrule
\end{tabular}
\end{center}
\end{table}




\begin{table}[t]
\caption{Zero-shot performance on 13 datasets against AutoAttack under $\ell_{\infty}$ threat model with $\epsilon=2/255$ and $\epsilon=4/255$. FARE \citep{schlarmann2024robust} and TeCoA \citep{mao2022understanding} are trained on $\ell_{\infty}$ threat model with $\epsilon=4/255$ for its generally better performance than $\epsilon=4/255$. Underlined results indicate the best robustness among baselines, while bold text denotes state-of-the-art (SOTA) performance across all methods. The term \textcolor{blue}{increase} represents the percentage improvement in robustness compared to the best baseline method.}
\label{table: zero-shot results}
\begin{center}
\setlength{\tabcolsep}{1.5pt} % Default value: 6pt
\renewcommand{\arraystretch}{1.1} % Default value: 1
\begin{tabular}{cc|ccccccccccccc|l}
% \hline
% \multicolumn{2}{c}{\textbf{Eval.}} & \multicolumn{14}{c|}{\textbf{Zero-shot datasets}} & \multirow{1}{*}{\textbf{Average Zero-shot}} \\ \cline{3-15}
% \multicolumn{2}{c|}{\textbf{Eval.}}
\toprule
\multirow{6}{*}{Eval.}& \multirow{6}{*}{Method} & &&&&&&&&&&&&& \multirow{6}{*}{\makecell[c]{\text{Average} \\ \text{Zero-shot}}} \\
% & \multicolumn{13}{c}{\textbf{Datasets}} &  \multirow{7}{*}{\makecell[c]{\text{Average} \\ \text{Zero-shot}}}\\
% \hline
% \cline{3-15}
& & \rotatebox{90}{CalTech} & \rotatebox{90}{Cars} & \rotatebox{90}{CIFAR10} & \rotatebox{90}{CIFAR100} & \rotatebox{90}{DTD} & \rotatebox{90}{EuroSAT} & \rotatebox{90}{FGVC} & \rotatebox{90}{Flowers} & \rotatebox{90}{ImageNet-R} & \rotatebox{90}{ImageNet-S} & \rotatebox{90}{PCAM} & \rotatebox{90}{OxfordPets} & \rotatebox{90}{STL-10} &  \\ 
\hline
\multirow{5}{*}{\rotatebox{90}{clean}} & CLIP & 83.3 & 77.9 & 95.2 & 71.1 & 55.2 & 62.6 & 31.8 & 79.2 & 87.9 & 59.6 & 52.0 & 93.2 & 99.3 & 73.1 \\
% & TeCoA$^{2}$ & 80.7 & 50.1 & 87.5 & 60.7 & 44.4 & 26.1 & 14.0 & 51.8 & 80.1 & 58.4 & 49.9 & 80.0 & 96.1 & 60.0 \\
% & FARE$^{2}$  & 84.8 & 70.5 & 89.5 & 69.1 & 50.0 & 25.4 & 26.7 & 70.6 & 91.1 & 59.7 & 50.0 & 91.1 & 98.5 & 67.0 \\
& TeCoA & 78.4 & 37.9 & 79.6 & 50.3 & 38.0 & 22.5 & 11.8 & 38.4 & 74.3 & 54.2 & 50.0 & 76.1 & 93.4 & 54.2 \\
& FARE & 84.7 & 63.8 & 77.7 & 56.5 & 43.8 & 18.3 & 22.0 & 58.1 & 80.2 & 56.7 & 50.0 & 87.1 & 96.0 & 61.1 \\
% & \bf LMz(adv) & 82.7 & 78.6 & 95.7 & 72.8&55.7& 63.5& 33.2&79.4& 87.5 & 58.4 &51.8 & 92.5 & 99.5 & 73.2\\
% & \bf LMz(COUP) & &&95.7&&&&&&&&&&&\\
% & \bf LMx-Diff&  &&& &&& && & &&&\\
&\cellcolor{mygray}\bf CLIPure-Diff&\cellcolor{mygray}79.9&\cellcolor{mygray}75.5&\cellcolor{mygray}94.9&\cellcolor{mygray}63.8&\cellcolor{mygray}55.2&\cellcolor{mygray}58.2&\cellcolor{mygray}29.3&\cellcolor{mygray}75.0&\cellcolor{mygray}87.5&\cellcolor{mygray}55.9&\cellcolor{mygray}56.8&\cellcolor{mygray}90.4&\cellcolor{mygray}98.4&\cellcolor{mygray}70.8\\
&\cellcolor{mygray}\bf CLIPure-Cos&\cellcolor{mygray}82.9&\cellcolor{mygray}78.6&\cellcolor{mygray}95.6&\cellcolor{mygray}73.0&\cellcolor{mygray}55.4&\cellcolor{mygray}63.3&\cellcolor{mygray}33.2&\cellcolor{mygray}79.3&\cellcolor{mygray}87.7&\cellcolor{mygray}58.3&\cellcolor{mygray}52.0&\cellcolor{mygray}92.5&\cellcolor{mygray}99.6&\cellcolor{mygray}73.2\\
\hline
\multirow{5}{*}{\rotatebox{90}{$\ell_{\infty} = 2/255$}} & CLIP & 0.0 & 0.0 & 0.0 & 0.0 & 0.0 & 0.1 & 0.0 & 0.0 & 0.0 & 0.0 & 0.0 & 0.0 & 0.0 & 0.0 \\
% & TeCoA$^{2}$ & 70.2 & 22.2 & 63.7 & 35.0 & 27.0 & 12.8 & 5.8 & 27.6 & 58.8 & 45.2 & 40.0 & 69.7 & 88.7 & 43.6 \\
% & FARE$^{2}$ & 73.0 & 26.0 & 60.3 & 35.6 & 26.7 & 6.2 & 5.9 & 31.2 & 68.3 & 38.3 & 41.9 & 68.3 & 90.1 & 43.1 \\
& TeCoA & 69.7 & 17.9 & 59.7 & 33.7 & 26.5 & 8.0 & 5.0 & 24.1 & 59.2 & 43.0 & 48.8 & 68.0 & 86.7 & 42.3 \\
& FARE & 76.7 & 30.0 & 57.3 & 36.5 & 28.3 & 12.8 & 8.2 & 31.6 & 61.6 & 41.6 & 48.4 & 72.4 & 89.6 & \underline{45.9}  \\
% & \bf LMx-Diff&  & & &  & & &  &  & & & & &\\
& \cellcolor{mygray}\bf CLIPure-Diff &\cellcolor{mygray}75.1& \cellcolor{mygray}65.9&  \cellcolor{mygray}92.5&\cellcolor{mygray}52.6&\cellcolor{mygray}45.9&\cellcolor{mygray}41.5&\cellcolor{mygray}20.8 &\cellcolor{mygray}65.8&\cellcolor{mygray}86.5 &\cellcolor{mygray}49.8&\cellcolor{mygray}51.4&\cellcolor{mygray}86.2&\cellcolor{mygray}97.9&\cellcolor{mygray}64.0 \textcolor{blue}{$\uparrow$}\textcolor{blue}{39.4\%}\\
& \cellcolor{mygray}\bf CLIPure-Cos&\cellcolor{mygray} 80.8&\cellcolor{mygray}73.9&\cellcolor{mygray}93.0&\cellcolor{mygray}65.0&\cellcolor{mygray}50.7&\cellcolor{mygray}49.1&\cellcolor{mygray}28.2&\cellcolor{mygray}75.3&\cellcolor{mygray}85.4&\cellcolor{mygray}54.3&\cellcolor{mygray}49.1&\cellcolor{mygray}91.2&\cellcolor{mygray}99.5&\cellcolor{mygray}68.8 \textcolor{blue}{$\uparrow$}\textcolor{blue}{45.9\%}\\
\hline
\multirow{5}{*}{\rotatebox{90}{$\ell_{\infty} = 4/255$}} & CLIP & 0.0 & 0.0 & 0.0 & 0.0 & 0.0 & 0.0 & 0.0 & 0.0 & 0.0 & 0.0 & 0.0 & 0.0 & 0.0 & 0.0 \\
% & TeCoA$^{2}$ & 57.4 & 6.5 & 31.0 & 17.8 & 14.7 & 7.7 & 1.1 & 9.8 & 36.7 & 32.8 & 16.0 & 50.3 & 69.2 & 27.0 \\
% & FARE$^{2}$ & 46.6 & 4.8 & 25.9 & 13.9 & 11.7 & 0.5 & 0.6 & 7.1 & 22.5 & 22.5 & 17.2 & 27.9 & 61.7 & 20.5 \\
& TeCoA & 60.9 & 8.4 & 37.1 & 21.5 & 16.4 & 6.6 & 2.1 & 12.4 & 41.9 & 34.2 & 44.0 & 55.2 & 74.3 & 31.9 \\
& FARE & 64.1 & 12.7 & 34.6 & 20.2 & 17.3 & 11.1 & 2.6 & 12.5 & 40.6 & 30.9 & 48.3 & 50.7 & 74.4 & \underline{32.4} \\
% & \bf LMz-Diff(adv) & 78.6 & 65.9 & 94.9 & 62.3 & 48.8& 46.6&22.7&68.0& 81.9&48.9&51.4& 81.9 & 98.8 & 65.4 \\
% & \bf LMx-Diff&  & & &  & & &  &  & & & & &\\
&\cellcolor{mygray}\bf CLIPure-Diff&\cellcolor{mygray}74.1&\cellcolor{mygray}66.3&\cellcolor{mygray}90.6&\cellcolor{mygray}51.6&\cellcolor{mygray}45.2&\cellcolor{mygray}42.9&\cellcolor{mygray}\cellcolor{mygray}18.8&\cellcolor{mygray}65.8&\cellcolor{mygray}83.8&\cellcolor{mygray}48.8&\cellcolor{mygray}52.0&\cellcolor{mygray}85.7&\cellcolor{mygray}97.2&\cellcolor{mygray}63.3 \textcolor{blue}{$\uparrow$}\textcolor{blue}{95.4\%}\\
&\cellcolor{mygray}\bf CLIPure-Cos&\cellcolor{mygray}80.1&\cellcolor{mygray}72.2&\cellcolor{mygray}91.1&\cellcolor{mygray}59.1&\cellcolor{mygray}50.1&\cellcolor{mygray}48.4&\cellcolor{mygray}26.1&\cellcolor{mygray}74.8&\cellcolor{mygray}84.6&\cellcolor{mygray}52.4&\cellcolor{mygray}48.9&\cellcolor{mygray}91.0&\cellcolor{mygray}99.4&\cellcolor{mygray}67.4 \textcolor{blue}{$\uparrow$}\textcolor{blue}{108.0\%}\\
\bottomrule
\end{tabular}
\end{center}
\end{table}



\subsection{Case Study}
\label{section: case study}
\begin{figure}[t]
\centering
\includegraphics[width=0.9\linewidth]{figs/purification_process.png} % Adjust the path and options as needed
\caption{Purification process of our CLIPure model for an image of a 'valley' adversarially perturbed to be classified as 'stupa'. Using the DaLLE2 Decoder \citep{ramesh2022hierarchical}, we visualize the stages of image embedding purification. The text below each picture annotates the classification results corresponding to each timestep in the purification process.}
\label{figure: purification process shown by generated images}
\end{figure}

To visualize the purification path of CLIPure, we employ the Decoder of DaLLE2 \citep{ramesh2022hierarchical}, which models the generation process from image embeddings to images through a diffusion model. As shown in Figure~\ref{figure: purification process shown by generated images}, starting from an adversarial example initially classified as a 'stupa,' CLIPure modifies the semantic properties of the image embedding. By the second step, the image is purified to 'mountain,' closely aligning with the semantics of the original image, which also contains mountainous features. Subsequently, after multiple purification steps, the image embedding is accurately classified as the 'valley' category.

In Figure~\ref{figure: cos with word distribution of clean and adv}, we analyze the distribution of cosine similarity between the embeddings of clean images and adversarial examples with words to understand the semantics of image embeddings. In Figure~\ref{figure: cosine similarity with words on clean and adv samples}, we also highlight the specific top-ranked categories for the benign and adversarial examples. We observe that the cosine similarity distribution for words close to the clean image is stable and resembles the prior distribution shown in Figure~\ref{subfigure: vocab_distribution_prior}. In contrast, the distribution of cosine similarity for the adversarial examples shows anomalies, with the top-ranked adversarial labels displaying abnormally high cosine similarities. This could potentially offer new insights into adversarial sample detection and defense strategies.

\begin{figure}[t]
% \captionsetup{labelfont={color=blue}}
\centering
\includegraphics[width=0.6\linewidth]{figs/process_tsne.pdf} % Adjust the path and options as needed
\caption{Purification process of the adversarial example attacked from ground truth label 'airplane' to adversarial label 'truck' and purified by our CLIPure-Diff visualized by T-SNE. The image is randomly sampled from the CIFAR-10 dataset.}
\label{figure: purification process by T-SNE}
\end{figure}

Additionally, we illustrate the purification path using T-SNE. As depicted in Figure~\ref{figure: purification process by T-SNE}, an image sampled from CIFAR-10 originally categorized as 'airplane' was adversarially manipulated to resemble a 'truck,' making it an outlier. Through our CLIPure method, the image is purified towards a high-density area and ultimately reclassified accurately as the 'airplane' category.


% \textbf{Analysis on Semantically Similar Words.}
% \subsection{Analysis on Semantically Similar Words}


\subsection{Analysis from Textual Perspective}


\begin{figure}[t]
\centering
% \newlength{\commonheight}
% \setlength{\commonheight}{3.5cm}
\begin{subfigure}[b]{0.46\linewidth}
    \includegraphics[width=\linewidth]{figs/vocab_topwords_clean.pdf}
    \caption{clean sample}
    \label{subfigure: vocab_topwords_clean}
\end{subfigure}
% \hfill  % Add space between subfigures
\begin{subfigure}[b]{0.46\linewidth}  % Width set to one third of text width
    \includegraphics[width=\linewidth]{figs/vocab_topwords_adv.pdf}
    \caption{adversarial example}
    \label{subfigure: vocab_topwords_adv}
\end{subfigure}
\caption{Cosine similarity between word and image embeddings of (a) clean and (b) adversarial examples in Figure~\ref{figure: purification process shown by generated images} across different ranks. Blue dots denote 10,000 words sampled from Word2Vec vocabulary \citep{church2017word2vec}, and red dots denote words from the 1,000 ImageNet categories.}
\label{figure: cosine similarity with words on clean and adv samples}
\end{figure}

\begin{figure}[t]
\centering
% \newlength{\commonheight}
% \setlength{\commonheight}{3.5cm}
\begin{subfigure}[b]{0.49\linewidth}
    \includegraphics[width=\linewidth]{figs/vocab_11k_prior.pdf}
    \caption{prior probability}
    \label{subfigure: vocab_distribution_prior}
\end{subfigure}
% \hfill  % Add space between subfigures
\begin{subfigure}[b]{0.49\linewidth}  % Width set to one third of text width
    \includegraphics[width=\linewidth]{figs/vocab_topwords_noise.pdf}
    \caption{highlight top-ranked words}
    \label{subfigure: vocab_topwords_noise}
\end{subfigure}
\caption{(a)Cosine similarity between words and the image embeddings of a noisy example to illustrate the prior probability of the words. (b) Highlights the top-ranked words within the 1,000 ImageNet categories, emphasizing the most semantically similar words to the noisy image.}
\end{figure}


\begin{figure}[t]
\centering
% \newlength{\commonheight}
% \setlength{\commonheight}{3.5cm}
\begin{subfigure}[b]{0.49\linewidth}
    \includegraphics[width=\linewidth]{figs/vocab_11k_clean.pdf}
    \caption{clean sample}
    \label{subfigure: vocab_distribution_clean}
\end{subfigure}
% \hfill  % Add space between subfigures
\begin{subfigure}[b]{0.49\linewidth}  % Width set to one third of text width
    \includegraphics[width=\linewidth]{figs/vocab_11k_adv.pdf}
    \caption{adversarial example}
    \label{subfigure: vocab_distribution_adv}
\end{subfigure}
\caption{Cosine similarity between words and the image embedding of (a) clean sample and (b) adversarial example of the case shown in Figure~\ref{figure: purification process shown by generated images} to illustrate the top-ranked words.}
\label{figure: cos with word distribution of clean and adv}
\end{figure}

In this section, we provide additional analysis from a textual perspective as a supplement to the analysis. In Figure~\ref{figure: cosine similarity with words on clean and adv samples}, we take advantage of the text modality of the CLIP model to understand the semantics of clean and adversarial examples by matching them with closely related words. We selected words from ImageNet's categories and supplemented this with an additional 10,000 randomly drawn words from natural language vocabulary for a comprehensive list. We observe that the meanings of the category words ranking high are relatively similar. For clean samples, the distribution of cosine similarity across ranks is relatively stable, whereas the adversarial samples exhibit abnormally high cosine similarity for adversarial categories at top ranks. 
% Our vocabulary includes the 1,000 categories from ImageNet and 10,000 words randomly sampled from Word2Vec \citep{church2017word2vec}. 
Figure~\ref{subfigure: vocab_distribution_prior} displays the distribution of cosine similarities between the word embeddings and an image embedding generated from pure noise, across various ranks. We aim to use these similarities to represent the prior probabilities of the words. The results indicate that most samples cluster around a cosine similarity of approximately 0.15, with only a few extreme values either much higher or lower, suggesting a normal distribution pattern. This indicates that a majority of the samples have moderate prior probabilities, contrasting with the long-tail distribution often observed in textual data. In Figure~\ref{subfigure: vocab_topwords_noise}, we highlight the top-ranked words in the vocabulary that are closest to the noisy image embedding. Words like ``television'', which appear frequently in image data, rank high, aligning with our expectations.
This abnormal phenomenon in adversarial samples could potentially inspire adversarial detection and more robust adversarial defense methods.


\subsection{Combining CLIPure with Other Strategies}
\label{section: combination}
In this section, we conduct experiments that combine orthogonal strategies, including adversarial training \citep{schlarmann2024robust} and pixel space purification \citep{chen2023robust}, as well as incorporating classifier confidence guidance.
We carry out evaluations using a sample of 256 images from the ImageNet test set. The experimental setup in this section is aligned with the same configuration as presented in Table~\ref{table: main result on imagenet}, detailed in Section~\ref{section: experimental settings}.


\textbf{Combination with Adversarial Training (CLIPure+AT).}
Orthogonal adversarial training (AT) methods like FARE \citep{schlarmann2024robust} has fintuned CLIP on ImageNet. We combine CLIPure and FARE by purifying image embedding in FARE's latent space and classification via FARE. The experimental results, depicted in Fig.~\ref{subfigure: combination}, indicate that the CLIPure-Diff+AT method enhances adversarial robustness in CLIPure-Diff. This improvement could be attributed to FARE providing more accurate likelihood estimates, as it is specifically trained on adversarial examples. When combined with CLIPure-Cos, the outcomes were comparable, likely because the inherent robustness of CLIPure-Cos is already high, leaving limited scope for further enhancement.

\textbf{Combination with Pixel Space Purification (CLIPure+LM).}
Purification in latent space and pixel space occurs at different stages of processing. Pixel space purification acts directly on the input samples, while latent space purification operates on the latent vectors encoded from the input picture. By combining both methods, we first purify the input samples in pixel space, then pass the purified samples through an encoder to obtain latent vectors, which are further purified in latent space. We mark this combination as CLIPure+LM. For the Likelihood Maximization (LM) method, we chose LM-EDM \citep{chen2023robust} due to its strong performance across pixel space purification methods.

Experimental results, as shown in Fig.~\ref{subfigure: combination}, indicate that combining with LM leads to a decrease in both clean accuracy and robustness. This decline may be associated with the challenges of information loss inherent to pixel space purification. Since LM operates directly on the pixels of the image, it is highly sensitive to the degree of purification. However, the adversarial perturbation varies across samples, potentially leading to over-purification thus a decrease in performance. In contrast, purification in latent space with polar coordinates simply involves changing vector directions—altering semantics without reducing semantic information. Thus, it avoids the problem of semantic information loss.


\textbf{Combination with Classifier Guidance}.
\begin{figure}[t]
\centering
% \newlength{\commonheight}
% \setlength{\commonheight}{3.5cm}
\begin{subfigure}[b]{0.53\linewidth}  % Width set to one third of text width
    \includegraphics[width=\linewidth]{figs/combination.pdf}
    \caption{Combination methods}
    \label{subfigure: combination}
\end{subfigure}
% \hfill  % Add space between subfigures
\begin{subfigure}[b]{0.39\linewidth}
    \includegraphics[width=\linewidth]{figs/guidance_lmd.pdf}
    \caption{Impact of guidance weight}
    \label{subfigure: guidance}
\end{subfigure}
\caption{(a) Accuracy and robustness against AutoAttack with $\ell_{\infty}=4/255$ on ImageNet of CLIPure, CLIPure combined with Adversarial Training (AT), and CLIPure combined with pixel space Likelihood Maximization (LM). (b) Performance on ImageNet against AutoAttack with $\ell_{\infty}=4/255$ incorporating with classifier confidence guidance.}
\end{figure}
In Eq.~\ref{equation: reverse sde of untargeted attack}, we note that the reverse-time SDE associated with the attack method includes not only a likelihood maximization purification term but also a classifier guidance term, represented as $\nabla_{\boldsymbol{x}} \mathcal{L}(\theta; \boldsymbol{x}_t, y_{\text{true}})$. This approach aligns with the classifier confidence guidance proposed by \citet{zhang2024classifier}. We incorporate this classifier guidance term in two versions of CLIPure (CLIPure-Diff and CLIPure-Cos) as described in Algorithm~\ref{algorithm: latent purification by CLIP and DaLLE2.DiffusionPrior}. Since the ground truth label $y_{\text{true}}$ is unknown, we employ the predicted label by the CLIP model as a proxy for $y_{\text{true}}$. To address potential inaccuracies in early estimations, classifier guidance was only applied in the final 5 steps of the 10-step purification process.

We evaluate the performance varied with the weight of the guidance term, as depicted in Fig.~\ref{subfigure: guidance}. It shows that a guidance weight of $10^{-4}$ led to the largest improvement in robustness: CLIPure-Diff achieved a robustness of 67.2\% (an increase of +2.2\% over scenarios without guidance), and CLIPure-Cos attained a robustness of 73.1\% (an improvement of +0.5\% over no guidance).


\subsection{T-SNE visualization}
In Figure~\ref{subfigure: polar plot of embeddings}, we display the distribution of image embeddings $\boldsymbol{z}^i$ for both benign samples and adversarial examples from a subset of 30 samples on the CIFAR-10 dataset, after dimensionality reduction using t-SNE. The figure also includes embeddings for 10 category-associated texts (e.g., ``a photo of a dog.'') $\boldsymbol{z}^t$, as well as the text embedding for a blank template ``a photo of a .''.

We observe distinct clustering of text vectors, clean image embeddings, and adversarial image embeddings in the space. The blank template, which is used for purification, is positioned at the center of the category text clusters, representing a general textual representation. This arrangement demonstrates that clean and adversarial sample distributions occupy distinct regions in the embedding space, providing a foundational basis for the effectiveness of adversarial sample purification.



\subsection{Hyperparameters}

\subsubsection{Impact of Step Size}

\begin{figure}[t]
\centering
% \newlength{\commonheight}
% \setlength{\commonheight}{3.5cm}
\begin{subfigure}[b]{0.4\linewidth}
    \includegraphics[width=\linewidth]{figs/polar_plot.pdf}
    \caption{Embeddings in CLIP's latent space}
    \label{subfigure: polar plot of embeddings}
\end{subfigure}
\hfill  % Add space between subfigures
\begin{subfigure}[b]{0.54\linewidth}  % Width set to one third of text width
    \includegraphics[width=\linewidth]{figs/hyperparam_stepsize.pdf}
    \caption{Impact of step size}
    \label{subfigure: hyperparam}
\end{subfigure}
\caption{(a) Visualization of image and text embeddings reduced via PCA and plotted in polar coordinates using t-SNE. $\boldsymbol{z}^i$ and $\boldsymbol{z}^t$ represent the image and text embeddings obtained through CLIP, respectively. Since vector direction signifies semantics, we depict the samples in polar coordinates to emphasize directional properties. (b) Impact of step size during a 10-step purification process using CLIPure-Cos on 1000 ImageNet samples against AutoAttack with $\ell_{\infty}$ threat model ($\epsilon=4/255$).}
\end{figure}

% \begin{figure}[t]
% \centering
% \includegraphics[width=0.75\linewidth]{figs/hyperparam_stepsize.pdf} % Adjust the path and options as needed
% \caption{hyperparameter.}
% \label{figure: hyperparameter stepsize}
% \end{figure}


\begin{figure}[t]
\centering
% \captionsetup{labelfont={color=blue}}
% \newlength{\commonheight}
% \setlength{\commonheight}{3.5cm}
\begin{subfigure}[b]{0.49\linewidth}
    \includegraphics[width=\linewidth]{figs/CLIPure_step.pdf}
    % \caption{\color{blue}{prior probability}}
    \label{subfigure: CLIPure_step}
\end{subfigure}
% \hfill  % Add space between subfigures
\begin{subfigure}[b]{0.49\linewidth}  % Width set to one third of text width
    \includegraphics[width=\linewidth]{figs/CLIPure_eps.pdf}
    % \caption{\color{blue}{highlight top-ranked words}}
    \label{subfigure: CLIPure_eps}
\end{subfigure}
\caption{Clean accuracy (marked as "Acc") and robust accuracy (marked as "Rob") across different (Left) purification steps and (Right) attack budgets. "Diff" indicates CLIPure-Diff while "Cos" represents CLIPure-Cos.}
\label{figure: purification step and epsilon}
\end{figure}

We conducted experiments to assess the impact of different step sizes on the effectiveness of purification. Figure~\ref{subfigure: hyperparam} shows how the clean accuracy and robustness against AutoAttack ($\ell_\infty=4/255$) of our CLIPure-Cos method vary with changes in step size. These experiments were carried out on 1,000 samples from the ImageNet test set with 10-step purification. The results indicate that the performance of the method is relatively stable across different step sizes, consistently demonstrating strong effectiveness.

Limited to the computational complexity of CLIPure-Diff, we opt not to conduct a parameter search for this model. Instead, we directly apply the optimal parameters found for CLIPure-Cos.


\subsubsection{Impact of Purification Step}

As illustrated in Figure~\ref{figure: purification step and epsilon} (Left), we assess the performance of various purification steps in CLIPure-Cos and CLIPure-Diff. The results indicate that insufficient purification while maintaining acceptable clean accuracy, results in low robustness. As the number of purification steps increases, robustness gradually improves. Continuing to increase the purification steps stabilizes both accuracy and robustness, demonstrating a balance between the two metrics as the process evolves.




\subsection{Performance Across Attack Budget}
To further explore the adversarial defense capabilities of CLIPure, we evaluate the robustness of CLIPure-Diff and CLIPure-Cos against various attack budgets, following the settings used in Table~\ref{table: main result on imagenet}. As shown in Figure~\ref{figure: purification step and epsilon} (Right), CLIPure-Diff demonstrates superior clean accuracy (at $\epsilon$=0); however, its robustness decreases as the intensity of the attacks increases. In contrast, CLIPure-Cos exhibits a stronger ability to withstand adversarial perturbations.





% \begin{figure}[t]
% \centering
% % \newlength{\commonheight}
% % \setlength{\commonheight}{3.5cm}
% \begin{subfigure}[b]{0.4\linewidth}  % Width set to one third of text width
%     \includegraphics[width=\linewidth]{figs/polar_plot.pdf}
%     \caption{Embedding of CLIP in Polar Coordinates}
%     \label{subfigure: polar plot of embeddings}
% \end{subfigure}
% % \hfill  % Add space between subfigures
% % \begin{subfigure}[b]{0.36\linewidth}
% %     \includegraphics[width=\linewidth]{figs/distribution_pz_CLIP.pdf}
% %     \caption{p(z) Estimated by CLIP}
% %     \label{subfigure: log-likelihood distribution of pz by CLIP}
% % \end{subfigure}
% \caption{Subfigure captions}
% \end{figure}






\end{document}

