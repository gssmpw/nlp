\section{Related Work}
\subsection{AI-assisted Group Decision-making}
As AI advances, researchers have increasingly explored AI-enhanced methods to improve group decision-making processes. Although various existing works have focused on individuals interacting with AI ____, there is growing interest in understanding group-level engagement with AI agents ____. For example, Zheng et al. found that even when AI is granted nominal equality in decision-making, it often remains peripheral because of its limited capacity to navigate social nuances ____. At the same time, Chiang et al. observed that groups can over-rely on AI-generated inputs ____ Meanwhile, initiatives to support minority voices directly within small-group settings remain relatively scarce ____: many interventions risk singling out underrepresented members or glossing over the unique needs of individuals ____. Consequently, current AI systems face challenges such as over-reliance on AI suggestions, limited handling of complex group dynamics, and a lack of subtle support mechanisms for marginalized participants  ____. To address these gaps, we propose an LLM-powered devil’s advocate approach that strategically represents minority perspectives, encourage critical thinking, and promotes more inclusive group decision-making without increasing discomfort for minority contributors.



\begin{figure*}[]
  \centering
  \includegraphics[width=1.0\textwidth]{figures/systemImplementation_IUI_v2.pdf}
  \caption{System Overview and Example Task Scenario. The figure illustrates a team leader promotion decision task, where participants discuss candidate qualifications in a chat interface. Minority members can privately share dissenting views via direct messages(DM) to the system, which reformulates and presents them as AI-mediated messages. If there is no DM with an opposing opinion, the system will send out a counterargument that it has generated on its own. The system architecture consists of a chat interface, database, and server, processing both public discussions and private DMs through four key agents: (A) Summary Agent for analyzing public opinion, (A') Paraphrase Agent for rephrasing minority views, (B) Conversation Agent for generating contextual counterarguments, and (C) AI Duplicate Checker for ensuring message novelty via cosine-similarity comparison.}
  \Description{This figure illustrates a system for group decision-making in a team leader promotion task, where participants discuss candidate qualifications in a chat interface. Minority members can privately share dissenting views via direct messages (DMs) to the AI, which reformulates and presents them as AI-generated insights to reduce social pressure. The system consists of a chat interface, database, and server with four AI agents: (A) Summary Agent for analyzing public opinion, (A’) Paraphrase Agent for anonymizing minority views, (B) Conversation Agent for generating counterarguments, and (C) AI Duplicate Checker for ensuring message novelty. If no dissenting DM is provided, the AI generates its own counterargument, fostering diverse perspectives and mitigating conformity bias in group discussions.}
  \label{fig:system}
\end{figure*}

\subsection{AI-mediated Communication}
AI-mediated communication (AIMC), where AI systems modify, augment, or generate messages for communicators, now pervades various contexts as AI technology advances ____. We identify four distinct AIMC patterns (\autoref{fig:communicationPattern}): First, humans can request and relay AI-generated content ____. Second, humans can selectively share AI's insights or viewpoints ____. Third, AI can reformulate and present human-provided messages (our system's approach). Fourth, AI can directly solicit and share input from multiple participants ____. These patterns distribute power and authorship differently among human communicators, AI systems, and message recipients. However, research has rarely explored how AIMC might address power imbalances and minority influence in group settings. Our proposed LLM-driven Devil's Advocate system adopts the third AIMC pattern to amplify underrepresented perspectives, reduce social influence biases, and foster more balanced discussions. However, as AI takes a more active role in content creation, AIMC approaches raise important concerns about user agency and authentic communication ____. Future research must carefully examine these ethical implications to avoid potential negative effects on group dynamics.