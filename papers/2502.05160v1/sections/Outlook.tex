\section{Outlook}

While there is a growing number of open-source libraries providing implementations of PQC algorithms, very few libraries providing the tools for empirically testing the resilience of PQC implementations exist to date. The latter would however be necessary for further gaining confidence in the security of these new methods for which there is less experience regarding implementations and cryptanalysis compared to established schemes. Accordingly, in a joint statement from partners from 18 EU member states it is recommended to ``deploy PQC in hybrid
solutions for most use-cases, i.e. combining a deployed cryptographic scheme with PQC in such a
way that the combination remains secure even if one of its components is broken'' \cite{BSI}.

Lattice-based cryptography can be challenged by a variety of attackers, e.g., classical lattice reduction algorithms (LLL, BKZ) and quantum-accelerated versions thereof \cite{prokop2024grover}, hybrid quantum-classical optimization algorithms (QAOA) \cite{koppl2024resilience}, or potentially emerging new quantum algorithms. Robust testing of lattice-based cryptography requires suitable software libraries that are optimized for integrated classical/quantum high-performance computing. 
The open-source high-level quantum programming framework Eclipse Qrisp \cite{Qrisp} -- with its recent integration of the high-performance computing library JAX -- bridges the gap for enabling researchers to conduct experiments using hybrid classical/quantum high-performance computing in the era of fault-tolerant quantum computing. 



