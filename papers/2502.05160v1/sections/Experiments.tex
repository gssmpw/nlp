\section{Experiments}
\label{sec:Experiments}

This section shows experiments for solving LWE instances with LLL and BKZ lattice reduction algorithms. The LWE instances were generated based on the NIST FIPS 203 standard with varying parameters. Depending on the security parameter, the NIST standard uses block matrices of size $k=2,3,4$, with each block sized $n = 256$, and modulus $q=3329$. For simplicity, we will only consider block matrices of size $k=1$ but will vary the block size $n$ and modulus $q$. The secret vector $s$ and error vector $e$ in $\{-2,\dotsc,2\}^n$ are drawn from a centered binomial distribution with mean $0$. The LWE instances are transformed into shortest vector problems, and the aforementioned lattice reductions techniques are applied.

The experiments were performed on a dual Intel Xeon E5-2695 (2.10 GHz) 18 core processor.

\subsection{Results for LLL algorithm}
\label{sec:Experiments:LLL}

For the experiments with the LLL algorithm we utilize the implementation in \texttt{fpylll} \cite{FPYLLL}. We choose the LLL parameter $\delta=0.99$ for all experiments. Experiments were conducted for security parameters $n=8k$ for $k=1,\dotsc,10$, and $q\in\{3,17,71,277,401,521,1031,3329\}$. For each combination of parameters $(n, q)$, $128$ LWE instances were generated. Based on these 128 instances, the probability that the LLL algorithm recovers the secret was calculated. For $q\in\{71,401,3329\}$ the results are shown in Figure \ref{fig:lll_1}. The results indicate that 1) for fixed modulus $q$ the success probability decreases exponentially with the key length $n$, and 2) for fixed key length $n$ the success probability grows with increasing modulus $q$. The first observation is as expected, as the search space grows exponentially with larger key size $n$.

The latter phenomenon is further studied in Figure \ref{fig:lll_2}. This Figure shows the maximum key length $n$ for which the probability that the LLL algorithm recovers the secret is at least $50\%$ for varying modulus $q$. The results indicate that this key length $n$ grows logarithmically with the modulus $q$. While this observation may be counter intuitive, recall that the size of the search space for the underlying LWE instance does only depend on key size $n$ and not on the modulus $q$ since $s,e\in\{-2,\dotsc,2\}^n$. The observations are further consistent with results in \cite{Laine15} establishing that success can be guaranteed with high probability for narrow error distributions and $\log_2 q>2n$. Similar to \cite{Laine15}, the results show that, in practice, significantly smaller moduli $q$ are vulnerable.

\begin{figure}[htbp]
\includegraphics[scale=1.1]{graphics/lll-res.pdf}
\caption{The probability that the LLL algorithm recovers the secret for varying key length $n$ and modulus $q\in\{71,401,3329\}$. The curves were fitted using a sigmoid function $p_{\rho,\sigma}(n)=1-(1+\exp(\rho-\sigma n))^{-1}$ with parameters $\rho=9.32$, $\sigma=0.35$ for $q=71$, $\rho=17.00$, $\sigma=0.36$ for $q=401$, and $\rho=26.94$, $\sigma=0.38$ for $q=3329$.}
\label{fig:lll_1}
\end{figure}

\begin{figure}[htbp]
\includegraphics[scale=1.05]{graphics/lll-solve_prob.pdf}
\caption{The maximum key length $n$ for which the probability that the LLL algorithm recovers the secret is at least $50\%$ for varying modulus $q$. The values $n$ are computed as the ratio of the fit parameters $\rho/\sigma$.}
\label{fig:lll_2}
\end{figure}


\subsection{Results for BKZ algorithm}
\label{sec:Experiments:BKZ}

For the experiments with the BKZ algorithm we utilize the implementation of \texttt{BKZ 2.0} in \texttt{fpylll}. We choose the LLL parameter $\delta=0.99$ for all experiments. Experiments were conducted for security parameters $n=8k$ for $k=1,\dotsc,14$, and $q\in\{71,401,3329\}$, and for varying block size $\beta=8l$ for $l=1,\dotsc,7$. For each key length $n\leq 88$, $512$ LWE instances were generated, and for each $n>88$, the sample size was reduced to $256$ LWE instances due to the larger computational costs.  
Additionally, for parameters $(n,\beta)$ with $n>88$, the sample size was further reduced. Instances that encountered errors due to insufficient floating-point precision during the BKZ calculations were discarded. For $n=112$ and $\beta\geq16$ the BKZ algorithm failed in almost all cases. For each $(n,\beta)$ the success probability of the BKZ algorithm was calculated. The results are shown in Figure \ref{fig:bkz}. As expected, we observe that the success probability for fixed key length $n$ increases with the block size $\beta$. 
Notably, for $\beta=56$ the success probability for relatively large keys $n=104$ ($q=401$) is still at $28\%$. 
Similar to the experiments utilizing the LLL algorithm, we notice that for fixed key length $n$, the success probability is higher for larger modulus $q$.
Moreover, the average runtime of the BKZ algorithm is shown in Figure \ref{fig:bkz_runtime} in a logarithmic plot. As expected, it shows that the average runtime grows exponentially with the bock size $\beta$.


%\begin{figure}[htbp]
%\includegraphics[scale=0.95]{bachelor_thesis/bkz-fig-%401_big.pdf}
%\caption{The probability that the BKZ algorithm recovers the secret for varying key length $n$ and block size $\beta$, and modulus $q=401$.}
%\label{fig:bkz}
%\end{figure}

\begin{figure}[htbp]
\includegraphics[scale=0.615]{graphics/bkz-time-401-legend-outside.pdf}
\caption{The average runtime for the BKZ algorithm for varying block size $\beta$ and key size $n$, and modulus $q=401$.}
\label{fig:bkz_runtime}
\end{figure}

\begin{figure*}[htbp]
\centering
\subfigure[Modulus $q=71$]{
    \label{fig:bkz_71}
    \includegraphics[scale=0.95]{graphics/bkz-fig-71.pdf}
}
\subfigure[Modulus $q=401$]{
    \label{fig:bkz_401}
    \includegraphics[scale=0.95]{graphics/bkz-fig-401.pdf}
}

\subfigure[Modulus $q=3329$]{
    \label{fig:bkz_3329}
    \includegraphics[scale=0.95]{graphics/bkz-fig-3329.pdf}
}

\caption{The probability that the BKZ algorithm recovers the secret for varying key length $n$ and block size $\beta$, and moduli $q=71$ ((a) top left), $q=401$ ((b) top right), and $q=3329$ ((c) bottom).}
\label{fig:bkz}

%\caption{The probability that the BKZ algorithm recovers the secret for varying key length $n$ and block size $\beta$, and moduli $q=71$ (Figure \ref{fig:bkz_71}), $q=401$ (Figure \ref{fig:bkz_401}), and $q=3329$ (Figure \ref{fig:bkz_3329}).}
%\label{fig:bkz}
\end{figure*}


\subsection{Conclusions and implications for lattice-based cryptography}

\begin{itemize}
    \item The results presented is this study underscore the importance of an appropriate choice of security parameters in latticed-based cryptography. In particular, the findings in Section \ref{sec:Experiments:LLL} indicate that an LWE instance can be solved by LLL even for large key size $n$ if the modulus $q$ is chosen too large. A similar behavior was observed for the BKZ algorithm.

    \item There are only a few open-source libraries available for lattice reduction algorithms. Although \texttt{fpylll} receives regular updates, its documentation remains insufficiently detailed and the selection of hyperparameters is particularly unclear, leading to potential errors and making it difficult to compare results. 

    \item Related work providing theoretical and experimental analysis of the cost for solving SVPs with lattice reduction algorithms is given by \cite{Albrecht17},\cite{Laine15},\cite{Schneider10}.  Essentially, these works share the conclusions that: 1) Empirically observed results are better than what is expected form theoretical analysis. 2) There is an insufficient availability of state-of-the art software libraries for testing attacks on SVPs. 
\end{itemize}

  



