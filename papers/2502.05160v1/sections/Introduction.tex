\section{Introduction}

Due to significant progress that has been made to improve both quantum hardware and algorithms, new concepts for public-key cryptosystems have to be developed. The goal is to design cryptosystems or cryptographic algorithms that are resistant to attacks using quantum computers. In literature, such algorithms are denoted as Post-quantum cryptography (PQC) algorithms \cite{kumar2020post,bernstein17}. 

One of the different PQC algorithms that have been investigated so far is the key encapsulation algorithm described in the NIST report FIPS 203 \cite{NIST203}. This algorithm is based on a mathematical problem given by a linear system of equations over a modular ring, where he matrix of this system is sampled from a uniform distribution. The right-hand side and matrix of this linear system of equations form the public key of the key encapsulation algorithm. However, in addition to that, a perturbation or error vector is sampled from a Gaussian distribution and added to the right-hand side of the linear system of equations. Depending on the standard deviation, the absolute values within the error vectors are usually bounded by $2$ or $3$. As a consequence, the public key cannot be used to recover the secret key, which is the solution of the overall system. This type of problem is known as learning with errors problem (LWE problem) \cite{Albrecht17}. 

Due to the significance of LWE problems several solution strategies have been developed \cite{Peikert16}. One of them is to reformulate the LWE problem as an optimization problem whose solution vector consists of the secret key and the error vector. Moreover, it is assumed that both vectors have components with small absolute values. For this reason, those problems are referred to as short vector problems (SVPs) \cite{uemura2021shortest}. Contrary to an LWE problem, the solutions of a SVP are given by vectors over the ring of integers. This facilitates the design of solution algorithms, since in terms of a SVP no modular arithmetics is required anymore. It is sufficient to use standard arithmetics in integer rings. 

However, up to now, there are no efficient algorithms for solving SVPs of relevant size. To facilitate the solution of an SVP, algorithms like the Lenstra–Lenstra–Lovász basis reduction algorithm (LLL algorithm) or the Block Korkine–Zolotarev algorithm (BKZ algorithm) are utilized \cite{Schneider10,Xia22,LLL82,koppl2024resilience,li2025complete}. These algorithms consider the solutions of SVPs as elements of a lattice and transform the basis of this lattice such that the new basis is almost orthogonal and consisting of short vectors. By means of such a basis, the solution of a SVP can be computed or approximated in a more convenient way. In some cases, the basis can contain the solution of the SVP itself. Quite often LLL and BKZ are combined with other solvers for discrete optimization problems \cite{prokop2024grover,Lv22}.

In this work, we apply the LLL and BKZ to LWE problems of different sizes. In addition to that the modulus of the underlying modular rings is varied. For each setting, the probability of solving the corresponding SVP is computed. Our goal is to determine for which problem sizes and moduli lattice reduction algorithms can solve a given SVP. As a consequence one can derive criteria, which can be utilized to decide when it is required to combine lattice reduction algorithms with further solvers. The remainder of this work is organized as follows: In Section \ref{sec:SVPs}, we describe the fundamentals of lattice-based cryptography and SVPs. The following section contains a short description of the LLL and BKZ algorithm. In Section \ref{sec:Experiments}, the results of our numerical tests are summarized. The paper is concluded by a short outlook.
