\documentclass[conference]{IEEEtran}
\IEEEoverridecommandlockouts
% The preceding line is only needed to identify funding in the first footnote. If that is unneeded, please comment it out.
\usepackage{cite}
\usepackage{amsmath,amssymb,amsfonts}
\usepackage{algorithmic}
\usepackage{graphicx}
\usepackage{textcomp}
\usepackage{xcolor}
\usepackage{braket}
\usepackage[FIGTOPCAP]{subfigure}
%\usepackage{subcaption}
%\usepackage{hyperref}
\def\BibTeX{{\rm B\kern-.05em{\sc i\kern-.025em b}\kern-.08em
    T\kern-.1667em\lower.7ex\hbox{E}\kern-.125emX}}

%%%%%%%%%%%%%%%%%%%%%%%%%%%%%%%%%%%%%%%%%%%%%%%%%%%%%%%%%%%%
% Math commands
\newcommand{\R}{\mathbb R}
\newcommand{\Z}{\mathbb Z}
\newcommand{\N}{\mathbb N}
\newcommand{\Q}{\mathbb Q}
\newcommand{\C}{\mathbb C}
%\newcommand{\linebreakand}{%
%  \end{@IEEEauthorhalign}
%  \hfill\mbox{}\par
% \mbox{}\hfill\begin{@IEEEauthorhalign}
%}

%%%%%%%%%%%%%%%%%%%%%%%%%%%%%%%%%%%%%%%%%%%%%%%%%%%%%%%%%%%%
% Colors
\definecolor{refblue}{RGB}{102, 102, 153}
\definecolor{stringgreen}{RGB}{80, 107, 65}
\definecolor{keywordblue}{RGB}{64, 89, 245}
\definecolor{commentbrown}{RGB}{59, 35, 0}

%%%%%%%%%%%%%%%%%%%%%%%%%%%%%%%%%%%%%%%%%%%%%%%%%%%%%%%%%%%%
% Listings
\usepackage{listings}
\lstdefinestyle{lststyle}{
  commentstyle=\color{commentbrown},
  keywordstyle=\color{keywordblue},
  numberstyle=\tiny\color{gray},
  stringstyle=\color{stringgreen},
  basicstyle=\ttfamily\footnotesize,
  breakatwhitespace=false,
  breaklines=true,
  numbers=none,
  captionpos=b,
  frame=lines,
  keepspaces=true,
  numbers=left,
  numbersep=5pt,
  showspaces=false,
  showstringspaces=false,
  showtabs=false,
  tabsize=1,
  columns=fullflexible 
}
\lstset{style=lststyle}

%%%%%%%%%%%%%%%%%%%%%%%%%%%%%%%%%%%%%%%%%%%%%%%%%%%%%%%%%%%%
    
\begin{document}

\title{A parameter study for LLL and BKZ with application to shortest vector problems}

%\author{
%\IEEEauthorblockN{1\textsuperscript{st} Tobias Köppl}
%\IEEEauthorblockA{Fraunhofer Institute (FOKUS) \\
%Berlin, Germany \\
%tobias.koeppl@fokus.fraunhofer.de \\
%ORCID: 0000-0003-3548-2807}
%\and
%\IEEEauthorblockN{2\textsuperscript{nd} René Zander}
%\IEEEauthorblockA{Fraunhofer Institute (FOKUS) \\
%Berlin, Germany \\
%rene.zander@fokus.fraunhofer.de}
%\and
%\IEEEauthorblockN{3\textsuperscript{rd} Louis Henkel}
%\IEEEauthorblockA{Fraunhofer Institute (FOKUS) \\
%Berlin, Germany \\
%louis.henkel@fokus.fraunhofer.de}
%\and
%\IEEEauthorblockN{4\textsuperscript{th} Nikolay Tcholtchev}
%\IEEEauthorblockA{RheinMain University of \\ Applied Sciences, \& \\ %Fraunhofer Institute (FOKUS) \\
%Berlin, Germany \\
%nikolay.tcholtchev@fokus.fraunhofer.de}    
%}

\author{\IEEEauthorblockN{Tobias Köppl\IEEEauthorrefmark{1}, René Zander\IEEEauthorrefmark{2}, Louis Henkel\IEEEauthorrefmark{3}, Nikolay Tcholtchev\IEEEauthorrefmark{4}}
\IEEEauthorblockA{\IEEEauthorrefmark{1}Fraunhofer Institute FOKUS\\
Berlin, Germany\\
Email: tobias.koeppl@fokus.fraunhofer.de}
\IEEEauthorblockA{\IEEEauthorrefmark{2}Fraunhofer Institute FOKUS\\
Berlin, Germany\\
Email: rene.zander@fokus.fraunhofer.de}
\IEEEauthorblockA{\IEEEauthorrefmark{3}Fraunhofer Institute FOKUS\\
Berlin, Germany\\
Email: louis.henkel@fokus.fraunhofer.de}
\IEEEauthorblockA{\IEEEauthorrefmark{4}RheinMain University of Applied Sciences \& Fraunhofer Institute FOKUS\\
Berlin, Germany\\
Email: nikolay.tcholtchev@fokus.fraunhofer.de}
}

\maketitle
\begin{abstract}
In this work, we study the solution of shortest vector problems (SVPs) arising in terms of learning with error problems (LWEs). LWEs are linear systems of equations over a modular ring, where a perturbation vector is added to the right-hand side. This type of problem is of great interest, since LWEs have to be solved in order to be able to break lattice-based cryptosystems as the Module-Lattice-Based Key-Encapsulation Mechanism published by NIST in 2024. Due to this fact, several classical and quantum-based algorithms have been studied to solve SVPs. Two well-known algorithms that can be used to simplify a given SVP are the Lenstra–Lenstra–Lovász (LLL) algorithm and the Block Korkine–Zolotarev (BKZ) algorithm. LLL and BKZ construct bases that can be used to compute or approximate solutions of the SVP. We study the performance of both algorithms for SVPs with different sizes and modular rings. Thereby, application of LLL or BKZ to a given SVP is considered to be successful if they produce bases containing a solution vector of the SVP.
\end{abstract}
\ \\ \\
\begin{IEEEkeywords}
\;NIST report 203, lattice-based cryptography, LWE problems, lattice reduction algorithms, LLL, BKZ
\end{IEEEkeywords}

% 
% 
The widespread integration of communication networks and smart devices in modern control systems has increased the vulnerability of industrial systems to online cyber-attacks, e.g., Industroyer, Blackenergy, etc \citep{osti_1505628}.
% Modern control systems have seen a large push to include communication networks and smart devices to increase performance, made possible by improvements in communication device cost and energy consumption. This trend has been coupled with the usage of open-standard communication protocols among industrial control systems, making them vulnerable to online cyber-attacks such as Industroyer, Blackenergy, etc \citep{osti_1505628}. 
To counter this, methods have been developed to improve security by achieving attack detection, mitigation, and monitoring, among others \citep{sandberg2022secure}. This paper focuses on active attack diagnosis to mitigate stealthy attacks. 
%
%\subsection{Literature review}

Active diagnosis techniques rely on the inclusion of additional moduli to control systems
% inclusion within the control system of additional moduli 
to alter the behavior of the system compared to information known by the attacker. 
For instance, the concept of additive watermarking was introduced in \cite{mo2015physical}, where noise signals of known mean and variance are added at the plant and compensated for it at the controller. 
This compensation, however, is not exact, causing some performance degradation. Thus, trade-offs between performance and detectability  are necessary \citep{zhu2023detection}.
% A later work \citep{zhu2023detection} designs the watermark signal by trading performance for detection. Thus, although additive watermarking serves as a good detection scheme, they endure performance losses even in the nominal case. 

In encrypted control \citep{darup2021encrypted}, the sensor data is encrypted, sent to the controller, and then operated on directly. Encrypted input signals are sent back to the plant for decryption. Although encryption is widespread in IT security, in control systems it presents some concerns, such as the introduction of time delays \citep{stabile2024verifiable}, while it may present inherent weaknesses \citep{alisic2023model}.
% they are not preferred as they introduce time delays \citep{stabile2024verifiable} which can cause instability, and some encryption schemes can be very weak  \citep{alisic2023model}. 

In moving target defense \citep{griffioen2020moving}, the plant is augmented with fictitious dynamics, known to the controller. The plant output is transmitted to the controller along with the fictitious states over a network under attack. 
The additional measurements then aide in the detection of attacks. 
This comes at the cost of higher communication bandwidth needs, which increases rapidly with the dimension of the augmented systems.
% Since the dynamics of the fictitious dynamics are exactly known to the controller, the attack is detected easily. However, when the scale of the system increases, the communication bandwidth used by moving the target defense approach increases rapidly. 

Other recently proposed works include two-way coding \citep{fang2019two}, a weak encryuption technique, and dynamic masking \citep{abdalmoaty2023privacy}, which enhances privacy as well as security, have been shown to be effective against zero-dynamics attacks.
% Two-way coding \citep{fang2019two} and dynamic masking \citep{abdalmoaty2023privacy} are other recently proposed approaches. Two-way coding is another form of weak encryption technique whilst dynamic masking proposes an architecture that enhances both privacy and security. These schemes are shown to be effective against zero dynamics attacks but remain to be studied for other classes of attacks. 
% Recent extensions include \citep{mukherjee2021secure,ramos2024privacy}.
% Some other works which are related are \citep{mukherjee2021secure}, an extension of \cite{fang2019two}. The work \citep{ramos2024privacy} is an extension of moving target defense for multi-agent systems. 
Furthermore, filtering techniques for attack detection are proposed by \cite{murguia2020security,hashemi2022codesign,escudero2023safety}, while not focusing on stealthy attacks.
% The works \citep{murguia2020security,hashemi2022codesign,escudero2023safety} develop filtering techniques to guarantee safety, without being focused on stealthy covert attacks.

Multiplicative watermarking (mWM) has been proposed by the authors as a diagnosis technique \citep{ferrari2020switching}. mWM consists of a pair of filters on each communication channel between the plant and its controller; the scheme is affine to weak encryption, whereby ``encoding'' and ``decoding'' are done by changing signals' dynamic characteristics through inverse pairs of filters. This enables original signals to be recovered exactly, and thus does not lead to performance degradation.
% A multiplicative watermark is an affine to a weak encryption technique, through which the signal is ``encoded'' by a filter, changing its dynamic behavior. The use of inverse pairs means that the original signal can be recovered, through ``decoding'' via an inverse filter. As such, differently to techniques based on additive watermarking, no performance is lost due to the injection of noise, and there are no bandwidth limitations.

%\subsection{Contributions}
One of the critical features of multiplicative watermarking is that to detect stealthy attacks, the mWM filter parameters must be switched over time. In this paper, an algorithm to optimally design the mWM parameters after a switching event is presented, enhancing detection performance, without changing the switching time.
% This is done without changing the switching time, which is taken as given.

\textcolor{black}{
To formalize the filter design problem, we suppose the defender is interested in optimal performance against adversaries injecting covert attacks with matched system parameters \citep{smith2015covert}, including the mWM parameters prior to the switch. This scenario represents a worst case where malicious agents can take full control of the system while remaining undetected.
Thus, the attack strategy is explicitly included within the formulation of the closed-loop system, and the mWM filters are chosen by solving an optimization problem minimizing the attack-energy-constrained output-to-output gain (AEC-OOG) \citep{anand2023risk}, a variation of the output-to-output gain proposed in  \cite{teixeira2015strategic}.
}
The main contributions of this paper are:
% We consider an adversary injecting a covert attack with matched system parameters \citep{smith2015covert}, i.e., an attacker with full knowledge of the control system parameters, including those of the mWM filters before the switch. This scenario is taken as a worst case, as it has been shown that this class of attacks can be made stealthy. To quantitatively define a cost, the output-to-output gain (OOG) \citep{teixeira2015strategic} is leveraged,
% a metric introduced to evaluate the impact of an additive attack in a control system. %Specifically, OOG evaluates the worst-case performance loss that an attacker injecting an undetectable attack can obtain. 
% Here, the maximum performance loss caused by a stealthy adversary with limited energy is taken, the attack-energy-constrained OOG (AEC-OOG) \citep{anand2023risk}. The main contributions of this paper are:
\begin{enumerate}
%[label=\alph*.]
\item The problem of optimally designing the switching mWM filters is formulated as an optimization problem, with the AEC-OOG is taken as the objective;%where the AEC-OOG is taken as the impact metric; 
\item The worst-case scenario of a covert attack with exact knowledge of plant and mWM filter parameters is embedded within the design problem;
% The optimization problem is defined to incorporate the worst-case scenario of a covert attack with exact knowledge of plant and mWM filter parameters;
\item The feasibility of the optimization problem is shown to be dependent only on stability conditions; 
\item A solution scheme is proposed to promote randomization of the mWM filter parameters such that an eavesdropping adversary cannot remain stealthy.
\end{enumerate} 

This builds on the results of \cite{ferrari2020switching}, where the focus was on the design of the switching protocols, rather than the parameters themselves.
Compared to previous work \citep{gallo2021design}, this paper introduces an optimization problem which is always feasible (thanks to the use of AEC-OOG in the objective), while also considering a more sophisticated class of covert attacks, where the presence of watermark is known to the adversary. 
Moreover, this paper poses a different objective than \citep{zhang2023hybrid}; indeed, while \citep{zhang2023hybrid} provided a design strategy to ensure certain privacy properties, in this paper we address the problem of optimal parameter design following a switching event.


%\subsection{Organization}
The rest of the paper is organized as follows. 
After formulating the problem in Section~\ref{sec:PF}, we propose our design algorithm in Section~\ref{sec:main}, and analyze its properties. It is then evaluated through a numerical example in Section~\ref{sec:NE}, and concluding remarks are given Section~\ref{sec:Con}.
% We provide the problem background in Section~\ref{sec:PF}. We formulate the design problem in Section~\ref{sec:main}, together with an analysis of its properties. The proposed algorithm is evaluated through a numerical example in Section \ref{sec:NE}. Concluding remarks are offered in Section \ref{sec:Con}.

\section{Lattice-based cryptography and SVPs}
\label{sec:SVPs}

In this section, we recall the fundamentals of the key generator described in the NIST report FIPS 203 \cite{NIST203}. Since this cryptosystem is a symmetric public key system, the key generator has to produce both a private and a public key for encryption and decryption. The public key consists of vectors 
$$
\hat{t}_l \in \mathbb{Z}_q^n, l \in \left\{1,\ldots,k\right\},
$$
which can be compiled to a single vector $t \in \mathbb{Z}_q^{nk}$:
$$
t = \left( \hat{t}_1,\ldots, \hat{t}_k \right)^T
$$
and further vectors
$$
a^{(ij)} = \left(a_1^{(ij)},\ldots,a_n^{(ij)} \right)^T \in \mathbb{Z}_q^{n}, i,j \in \left\{1,\ldots,k\right\},
$$
where $n \in \mathbb{N}$ is the characteristic key length of the public key. For $k$ it holds according to \cite{NIST203}: $k \in \left\{2,3,4\right\}$ and $q \in \mathbb{N}$ is the modulus of the ring $\mathbb{Z}_q$. The vectors $a^{(ij)}$ are sampled from a uniform distribution. Based on $a^{(ij)}$, we construct matrices $A_{ij} \in \mathbb{Z}_q^{n \times n}$:
$$
A_{ij} = 
\begin{pmatrix} a_1^{(ij)} & -a_{n}^{(ij)} & \ldots & -a_2^{(ij)} \\ 
                a_2^{(ij)} &  a_1^{(ij)}     & \ldots & -a_3^{(ij)} \\
                \vdots     &  \vdots         &   & \vdots \\
                a_{n}^{(ij)} & a_{n-1}^{(ij)}  & \ldots & a_{1}^{(ij)}
\end{pmatrix}
$$
The matrix $A_{ij}$ is a T\"oplitz matrix, i.e., on each diagonal from top to bottom is the same element \cite{Gray06}. Each column vector is generated by a cyclic permutation of the components in $a^{(ij)}$ and to the entries in the upper triangular part a minus sign is assigned. All the submatrices can assigned to a further matrix $A \in \mathbb{Z}_q^{nk \times nk}$:
$$
A= \begin{pmatrix} A_{11} & \cdots & A_{1k} \\ \vdots & \ddots &  \vdots \\  A_{k1} & \cdots & A_{kk}\end{pmatrix}.
$$
Next, we sample the secret vector or private key $s \in \mathbb{Z}_q^{nk}$ as well as the error vector $e \in \mathbb{Z}_q^{nk}$ from a discrete Gaussian distribution such that the absolute values of $s$ and $e$ are bounded by $2$ or $3$. $s$ and $e$ have the following structure:
\begin{align*}
s &= \left( \hat{s}_1, \ldots, \hat{s}_k \right)^T,\;\hat{s}_l \in \mathbb{Z}_q^{n}, \\
e &= \left( \hat{e}_1, \ldots, \hat{e}_k \right)^T,\;\hat{e}_l \in \mathbb{Z}_q^{n}.
\end{align*}
Using $A$, $s$ and $e$, the missing part of the public key $t$ can be determined:
\begin{equation}
    \label{eq:LWE1}
t = A \cdot s + e,
\end{equation}
where all the arithmetic operations have to be performed in the ring $\mathbb{Z_q}$. For a potential attacker, the matrix $A$ and the vector $t$ are known, while the vectors $s$ and $e$ have to be determined. This problem is know as LWE problem. Since a LWE problem is not well-posed, \eqref{eq:LWE1} is reformulated. In the remainder of this section, we present the most important steps. $\mathbf{0}$ denotes either a vector filled with zeros or a matrix filled with zeros.
\ \\
\begin{itemize}
    \item[]\textbf{Step (1):} In a first step the lattice 
    $$
    \Lambda = \left\{ \left. x \in \mathbb{Z}^{2m+1} \right| x^T \left( \left. A \right| \left. I_m \right| -t \right) \equiv \mathbf{0} \mod q \right\},
    $$
    is constructed, where $I_m$ is the identity matrix with $m$ rows and columns. Obviously, it holds that the vector
    $$
    y = \left( \left. s \right| \left. e \right| 1 \right)^T \in \mathbb{Z}^{2m+1}
    $$
    is contained in $\Lambda$. For convenience, we set: $m=kn$
    \ \\ 
    \item[]\textbf{Step (2):} The rows of the matrix 
    \begin{equation}
    \label{eq:LWE2}
    B = \begin{pmatrix} I_m & -A^T & \mathbf{0} \\ \mathbf{0} & q I_m & \mathbf{0} \\  \mathbf{0} & t^T & 1
    \end{pmatrix}
    \end{equation}
    form a basis of the lattice $\Lambda$, provided that computing linear combinations of the rows integer valued coefficients are used. Compiling the matrix $B$ in this way is also known as Bai and Galbraith’s embedding \cite{Albrecht17}. 
    \ \\
    \item[]\textbf{Step (3):} We compute $y$ by solving the following optimization problem (shortest vector problem, SVP):
    \begin{equation}
    \label{eq:LWE3}
    z = \text{argmin}_{x \in \mathbb{Z}^{2m+1} \setminus \left\{\mathbf{0}\right\}} \left\| x^T B \right\|_2.
    \end{equation}
    After that, we set:
    $$
    z^T B = y^T \Leftrightarrow y = B^T \cdot z.
    $$
    Solving an SVP is based on the fact that $s$ and $e$ are sampled from a discrete Gaussian distribution with a small standard deviation. 
\end{itemize}

\section{Lattice reduction algorithms}

In this section, we briefly discuss classical methods for lattice reduction. Let $$\Lambda(B)=\left\{\sum_{i=1}^nx_ib_i\mid x_i\in\Z\right\}$$ be a lattice with basis $B=(b_1,\dotsc,b_n)$ for linearly independent vectors $b_i\in\R^d$ for $n\leq d$. Here, we consider the full-dimensional case $n=d$. Such bases are not unique as lattices are invariant under unimodular transformations, i.e., $\Lambda(B)=\Lambda(BU)$ for all $U\in\mathrm{GL}_n(\Z)$ with $\det(U)=\pm1$. In particular, every lattice of dimension $n>1$ has infinitely many bases. We denote the length of a shortest vector of the lattice $\Lambda$ by $\lambda_1(\Lambda)$.

The $\gamma$-SVP problem asks for finding an approximate shortest non-zero vector in the lattice, i.e., a vector $v\in\Lambda\setminus\{0\}$ with $\|v\|\leq\gamma\lambda_1(\Lambda)$, where $\gamma\geq1$ is the approximation factor. Breaking an LWE based cryptosystem requires the ability of finding a shortest vector up to an approximation factor that is linear in the lattice dimension \cite{Peikert16}. 

Classical lattice reduction algorithms such as Lenstra-Lenstra-Lovász (LLL) \cite{LLL82} and Block Korkin-Zolotarev (BKZ) \cite{Schnorr94} and variants thereof can be applied to obtain a basis $B'=(b_1',\dotsc,b_n')$ of relatively short and nearly orthogonal vectors. Here, ``relatively short'' means that the resulting bases are guaranteed to contain a vector of length $\gamma\cdot\lambda_1(\Lambda)$ up to an approximation factor $\gamma\in\R$ \cite{Schneider10}.

The LLL algorithm \cite{LLL82} has a polynomial runtime with respect to the lattice dimension $n$, and is guaranteed to solve $\gamma$-SVP for approximation factors exponential in the lattice dimension $n$: It is parametrized by $\delta\in(1/4,1]$ which appears in the Lovász condition that controls the strictness of the reduction process, i.e. how short and orthogonal the vectors of the reduced basis are. Thereby, it balances the trade-off between computational efficiency and the quality of the reduced basis.
The LLL algorithm produces a basis whose first vector satisfies
$$ \|b_1\|\leq (\delta-1/4)^{\frac{1-n}{2}}\lambda_1(\Lambda) $$
In practice, the LLL algorithm finds much shorter vectors than guaranteed by this worse-case bound \cite{Schnorr94}.

The BKZ algorithm \cite{Schnorr94} combines enumeration based methods with LLL, and is parametrized by the block size parameter $\beta$ and the LLL parameter $\delta$. It produces bases whose first vector satisfies 
$$ \|b_1\|\leq (\gamma_{\beta})^{\frac{n-1}{\beta-1}}\lambda_1(\Lambda) $$
where $\gamma_{\beta}$ is the Hermite constant in dimension $\beta$ \cite{Schnorr94}.
The running time scales exponentially in the bock size $\beta$ for the enumeration step.
Therefore, it offers a trade-off between the running time and the quality of the output basis.




In this section, we empirically compare the proposed algorithm on both sequence windows and time windows with existing methods.
\paragraph{Datasets} For the sequence-based model, we used two synthetic datasets and two cross-language datasets. The statistics of the datasets are provided in Table \ref{table:statistics}:

\begin{table}[t]
    \centering
    \caption{The statistics of the datasets. The datasets satisfy $1 \leq \|\vx\|\|\vy\| \leq R $.}
    \label{table:statistics}
    \begin{tabular}{|c|c|c|c|c|c|}
    \hline
        Dataset & $n$ & $m_x$ & $m_y$ & $N$ & $R$ \\ \hline
        SYNTHETIC(1) & 100,000 & 1,000 & 2,000 & 50,000 & 65 \\ \hline
        SYNTHETIC(2) & 100,000 & 1,000 & 2,000 & 50,000 & 724 \\ \hline
        APR & 23,235 & 28,017 & 42,833 & 10,000 & 773 \\ \hline
        PAN11 & 88,977 & 5,121 & 9,959 & 10,000 & 5,548 \\ \hline
        EURO & 475,834 & 7,247 & 8,768 & 100,000 & 107,840 \\ \hline
    \end{tabular}
\end{table}

\begin{itemize}
    \item Synthetic: The elements of the two synthetic datasets are initially uniformly sampled from the range (0,1), then multiplied by a coefficient to adjust the maximum column squared norm $R$. The X matrix has 1,000 rows, and the Y matrix has 2,000 rows, each with 100,000 columns. The window size is set to 50,000.
    \item APR: The Amazon Product Reviews (APR) dataset is a publicly available collection containing product reviews and related information from the Amazon website. This dataset consists of millions of sentences in both English and French. We structured it into a review matrix where the X matrix has 28,017 rows, and the Y matrix has 42,833 rows, with both matrices sharing 23,235 columns. The window size is 10,000.
    \item PAN11: PANPC-11 (PAN11) is a dataset designed for text analysis, particularly for tasks such as plagiarism detection, author identification, and near-duplicate detection. The dataset includes texts in English and French. The X and Y matrices contain 5,121 and 9,959 rows, respectively, with both matrices having 88,977 columns. The window size is 10,000.
\end{itemize}
We evaluate the time-based model on another real-world dataset:
\begin{itemize}
    \item EURO: The Europarl (EURO) dataset is a widely used multilingual parallel corpus, comprising the proceedings of the European Parliament. We selected a subset of its English and French text portions. The X and Y matrices contain 7,247 and 8,768 rows, respectively, and both matrices share 475,834 columns. Timestamps are generated using the $Poisson$ $Arrival$ $Process$ with a rate parameter of $\lambda=2$. The window size is set to 100,000, with approximately 30,000 columns of data on average in each window.
\end{itemize}

\paragraph{Setup} For the sequence-based model, we compare the proposed hDS-COD and  aDS-COD with EH-COD~\cite{yao2024approximate} and DI-COD~\cite{yao2024approximate}. We do not consider the Sampling algorithm as a baseline, as its performance is inferior to that of EH-COD and DI-CID, as demonstrated in \cite{yao2024approximate}. %The hDS-COD is adjusted by the parameter $\ell$ and the maximum number of levels $L = \log{R}$, where $R$ is the prior estimate of the maximum squared column norm of the dataset. DI-COD similarly requires a prior estimate of $R$ to limit the maximum number of levels $L = \log{(R/\varepsilon})$. In contrast, aDS-COD and EH-COD do not require an estimate of $R$; their error-space balance is controlled by the parameter $\ell = \frac{1}{\varepsilon}$. 
For the time-based model, we compare the proposed hDS-COD and  aDS-COD with EH-COD and the Sampling algorithm since DI-COD cannot be applied to time-based sliding window model. To achieve the same error bound, the maximum number of levels for hDS-COD is set to $L = \log{(\varepsilon NR)}$, and the initial threshold for aDS-COD is set to $1$.

Our experiments aim to illustrate the trade-offs between space and approximation errors. The x-axis represents two metrics for space: final sketch size and total space cost. The final sketch size refers to the number of columns in the result sketches $\mA$ and $\mB$ generated by the algorithm, representing a compression ratio. The total space cost refers to the maximum space required during the algorithm's execution, measured by the number of columns.We evaluate the approximation performance of all algorithms based on correlation errors $\operatorname{corr-err}(\mathbf{X}_W \mathbf{Y}_W^\top, \mathbf{A} \mathbf{B}^\top)$, which is reflected on the y-axis. Every 1,000 iterations, all algorithms query the window and record the average and maximum errors across all sampled windows.

The experiments for all algorithms were conducted using MATLAB (R2023a), with all algorithms running on a Windows server equipped with 32GB of memory and a single processor of Intel i9-13900K.

\paragraph{Performance} Figure \ref{fig:error vs l} and Figure \ref{fig:error vs space} illustrate the space efficiency comparison of the algorithms on sequence-based datasets. Panels (a-d) show the average errors across all sampled windows, while panels (e-h) display the maximum errors.

Figure \ref{fig:error vs l} evaluates the compression effect of the final sketch. The hDS-COD, aDS-COD, and EH-COD show similar compression performances. But the DS series is more stable, particularly on the synthetic datasets, where they significantly outperform EH-COD and DI-COD. The performance of hDS-COD and aDS-COD is nearly the same, indicating that the adaptive threshold trick in aDS-COD does not have a noticeable negative impact on it, maintaining the same error as hDS-COD.

Figure \ref{fig:error vs space} measures the total space cost of the algorithms. hDS-COD and aDS-COD show a significant advantage over existing methods, as they can achieve the  $\varepsilon$-approximation error with much less space. For the same space cost, the correlation errors of hDS-COD and aDS-COD are much smaller than those of EH-COD and DI-COD. Also, aDS-COD has better space efficiency than hDS-COD because aDS only uses a single-level structure while hDS requires $\log R+1$ levels. We find that hDS-COD requires more space on  SYNTHETIC(2) dataset compared to SYNTHETIC(1) dataset. This phenomenon occurs because SYNTHETIC(2) dataset has a larger $R$, which confirms the dependence on $R$ as stated in Theorem~\ref{thm:hds}. 

Figure \ref{fig:time-based} compares the performance of algorithms on time-based windows. Panels (a) and (b) present the error against the final sketch size, which show that our aDS-COD and hDS-COD algorithms enjoy similar performance as EH-COD and significantly outperform the sampling algorithm. On the other hand, as shown in panels (c) and (d), our methods outperform baselines in terms of total space cost.


\section{Outlook}

While there is a growing number of open-source libraries providing implementations of PQC algorithms, very few libraries providing the tools for empirically testing the resilience of PQC implementations exist to date. The latter would however be necessary for further gaining confidence in the security of these new methods for which there is less experience regarding implementations and cryptanalysis compared to established schemes. Accordingly, in a joint statement from partners from 18 EU member states it is recommended to ``deploy PQC in hybrid
solutions for most use-cases, i.e. combining a deployed cryptographic scheme with PQC in such a
way that the combination remains secure even if one of its components is broken'' \cite{BSI}.

Lattice-based cryptography can be challenged by a variety of attackers, e.g., classical lattice reduction algorithms (LLL, BKZ) and quantum-accelerated versions thereof \cite{prokop2024grover}, hybrid quantum-classical optimization algorithms (QAOA) \cite{koppl2024resilience}, or potentially emerging new quantum algorithms. Robust testing of lattice-based cryptography requires suitable software libraries that are optimized for integrated classical/quantum high-performance computing. 
The open-source high-level quantum programming framework Eclipse Qrisp \cite{Qrisp} -- with its recent integration of the high-performance computing library JAX -- bridges the gap for enabling researchers to conduct experiments using hybrid classical/quantum high-performance computing in the era of fault-tolerant quantum computing. 





\section*{Acknowledgment}
This work was supported by the EU Horizon Europe Framework Program under Grant Agreement no 101119547 (PQ-REACT).
In terms of PQ-REACT methods and tools for benchmarking PQC algorithms are investigated. In particular, testing the resilience of PQC algorithms against various attacks is a central pillar and part of ongoing work in the PQ-REACT project.

\begin{thebibliography}{00}

\bibitem{BSI} Bundesamt für Informationssicherheit, ``Securing Tomorrow, Today: Trasitioning to Post-Quantum Cryptography,'' A joint statement from partners from 18 EU member states, https://www.bsi.bund.de/SharedDocs/Downloads/EN/BSI/Crypto/PQC-joint-statement.pdf?\_\_blob=publicationFile\&v=5, Accessed: 02.02.2025, 2024.

\bibitem{NIST203} National Institute of Standards and Technology,
``FIPS 203, Module-Lattice-Based Key-Encapsulation Mechanism Standard,''
https://csrc.nist.gov/pubs/fips/203/ipd, 2024.

\bibitem{Albrecht17} M. R. Albrecht, G. Göpfert, F. Virdia, T. Wunderer, 
``Revisiting the Expected Cost of Solving uSVP and Applications to LWE,'' Advances in Cryptology -- ASIACRYPT 2017, Springer International Publishing, pp. 297--322, 2017.

\bibitem{bernstein17} D. Bernstein and T. Lange, ``Post-quantum cryptography,'' Nature, vol. 549, pp. 188--194, 2017.

\bibitem{Goepfert17} F. G\"opfert, C. van Vredendaal and T.  Wunderer, ``A hybrid lattice basis reduction and quantum search attack on LWE,'' In Post-Quantum Cryptography: 8th International Workshop, PQCrypto 2017, Proceedings 8 (pp. 184--202). Springer International Publishing, 2017.

\bibitem{Gray06} R. Gray, ``Toeplitz and circulant matrices: A review,'' Foundations and Trends® in Communications and Information Theory, vol. 2(3), pp. 155--239, 2006.

\bibitem{Grover96} L. K. Grover, 
``A fast quantum mechanical algorithm for database search,'' 
Proceedings of the twenty-eighth annual ACM symposium on Theory of computing, pp. 212--219, 1996.

\bibitem{koppl2024resilience} T. K{\"o}ppl, R. Zander and N. Tcholtchev, ``Resilience of lattice-based Cryptosystems to Quantum Attacks,'' 2024 IEEE Symposium on Computers and Communications (ISCC), pp. 1--6, 2024.

\bibitem{kumar2020post} M. Kumar and P. Pattnaik, ``Post quantum cryptography (PQC)--An overview,'' 2020 IEEE High Performance Extreme Computing Conference (HPEC), pp. 1--9, 2020.

\bibitem{Laine15} K. Laine, K. Lauter,
``Key Recovery for LWE in Polynomial Time,'' Cryptology ePrint Archive, Paper 2015/176, 2015.

\bibitem{LLL82} A. Lenstra, H. Lenstra and L. Lovász, ``Factoring polynomials with rational coefficients,'' Mathematische Annalen, vol. 261, pp. 515--534, 1982.

\bibitem{Lv22} L. Lv et al.,
``Using Variational Quantum Algorithm to Solve the LWE Problem,''
Entropy, vol. 24, pp. 1428, 2022.

\bibitem{li2025complete} J. Li and P. Nguyen, ``A complete analysis of the BKZ lattice reduction algorithm,'' Journal of Cryptology, 38, 1, pp. 1--58, Springer, 2025.

\bibitem{Peikert16} C. Peikert, 
``A decade of lattice cryptography,'' 
Foundations and trends{\textregistered} in theoretical computer science, vol. 10, pp. 283--424, 2016.

\bibitem{prokop2024grover} M. Prokop, P. Wallden and D. Joseph, ``Grover's oracle for the Shortest Vector Problem and its application in hybrid classical-quantum solver,'' IEEE Transactions on Quantum Engineering, IEEE Transactions on Quantum Engineering, 2024.

\bibitem{Schneider10} M. Schneider, J. Buchmann,
``Extended Lattice Reduction Experiments Using the BKZ Algorithm,''
Lecture Notes in Informatics (LNI), Proceedings -- Series of the Gesellschaft fur Informatik (GI), pp. 241--252, 2010.

\bibitem{Schnorr94} C. P. Schnorr, M. Euchner,
``Lattice basis reduction: Improved practical algorithms and solving subset sum problems,''
Mathematical Programming, vol. 66, pp. 181-189, 1994.

\bibitem{Qrisp} R. Seidel, S. Bock, R. Zander, M. Petrič, N. Steinmann, N. Tcholtchev, M. Hauswirth,
``Qrisp: A Framework for Compilable High-Level Programming of Gate-Based Quantum Computers,''
arXiv:2406.14792, 2024.

\bibitem{Shor97} P. W. Shor, 
``Polynomial-Time Algorithms for Prime Factorization and Discrete Logarithms on a Quantum Computer,'' 
SIAM Journal on Computing, vol. 26, pp. 1484--1509, 1997.

\bibitem{uemura2021shortest} S. Uemura, K. Fukushima, S. Kiyomoto, M. Kudo and T. Takagi, ``Shortest Vectors in Lattices of Bai-Galbraith’s Embedding Attack on the LWE Problem,'' Advances in Information and Computer Security: 16th International Workshop on Security, IWSEC 2021, Virtual Event, September 8--10, 2021, Proceedings 16, pp. 23--41, Springer, 2021.

\bibitem{Xia22} W. Xia, L. Wang, D. Gu and B. Wang, ``Improved Progressive BKZ with Lattice Sieving and a Two-Step Mode for Solving uSVP,'' Cryptology ePrint Archive, Paper 2022/1343, 2022.

\bibitem{FPYLLL} The FPLLL Developers, ``fpylll, a Python wrapper for the fplll lattice reduction library, Version 0.6.1,'' https://github.com/fplll/fpylll, 2024.

\bibitem{Sage} The Sage Developers, ``SageMath, the Sage Mathematics Software System, Version 10.0,''  https://www.sagemath.org, 2024.

\end{thebibliography}

\end{document}
