\begin{abstract}
% \red{Despite their remarkable performance, modern Diffusion Transformer (DiT) models are often hindered by their substantial resource requirements, stemming from the fixed and large amount of compute needed for each denoising step. We propose to change this static paradigm and instead dynamically adjust the compute for each denoising step. We present a simple and sample-efficient framework to adapt pre-trained DiT models into \emph{flexible} ones --- which we coin~\flexidit --- that can process inputs at different granularities and generate samples at different compute budgets. Our framework is general and largely agnostic to the input and conditioning modality. Indeed, we showcase how our single \emph{flexible} model reduces FLOPs by up to 40\% with no drop in performance for both class-conditioned and text-conditioned image generation. For video generation, we demonstrate how up to 80\% of the compute can be removed with no drop in performance.}
Despite their remarkable performance, modern Diffusion Transformers (DiTs) are hindered by substantial resource requirements during inference, stemming from the fixed and large amount of compute needed for each denoising step. In this work, we revisit the conventional static paradigm that allocates a fixed compute budget per denoising iteration and propose a dynamic strategy instead. Our simple and sample-efficient framework enables pre-trained DiT models to be converted into \emph{flexible} ones --- dubbed~\flexidit --- allowing them to process inputs at varying compute budgets. We demonstrate how a single \emph{flexible} model can generate images without any drop in quality, while reducing the required FLOPs by more than $40$\% compared to their static counterparts, for both class-conditioned and text-conditioned image generation. Our method is general and agnostic to input and conditioning modalities. We show how our approach can be readily extended for video generation, where~\flexidit~models generate samples with up to $75$\% less compute without compromising performance.
\end{abstract}
