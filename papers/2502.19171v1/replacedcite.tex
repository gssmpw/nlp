\section{Background And Related Work}
\label{sec:rel}
The development of \system is grounded in related research on challenges of citizens looking to pursue urban gardening, HCI approaches aiming to enhance gardening practices, and HRI with PARs. 
\subsection{Challenges Related To Urban Gardening}\label{subsec:rel-challenges} Urban gardening encompasses all the practices related to growing food within and near cities, from inner city allotments and community gardens to periurban off-ground cultivation____. Going by this definition, practices such as backyard, allotment, rooftop, balcony, and community gardening are included under the broader umbrella term of urban gardening. Engagement in urban gardening is connected to numerous benefits, such as enhancing well-being and food resilience____. Much research on urban gardening focuses on understanding the motivations, strategies, goals, and challenges of citizens who actively engage in urban gardening or intend to do so (e.g.,____). Prior work has identified a wide range of motivations, including practical, intrinsic, and aesthetic factors____. According to ____, practical motivations often center around food production and promoting biodiversity, while intrinsic motivations typically involve personal pleasure and enjoyment throughout the growing process. Additionally, aesthetic motivations encompass the desire to shape one’s environment and are known as key driving factors. Previous research has demonstrated that, despite strong motivations, the ability to act on intentions to cultivate an urban garden is often accompanied by various challenges____. A common issue is the lack of accessible spaces for urban gardening____. Previous studies have additionally noted the unequal distribution of green spaces between lower- and higher-income neighborhoods in large cities____. Gardening also demands knowledge of crop seasonality, the required frequency of plant care tasks, and the ability to assess plant health throughout the growth cycle. Lacking such knowledge has been shown to impede crop cultivation success____, which can, in turn, diminish motivation, especially for novice gardeners____. For those aiming to engage in urban gardening consistently, integrating this practice into their daily routines is a key consideration. Grassroots initiatives like urban community gardens, where individuals share gardening spaces, aim to reduce barriers and foster social connection____. Research indicates that interest in these gardens has increased recently despite a temporary decline during the COVID-19 pandemic____. Community gardens provide opportunities for members to share knowledge, support those new to gardening, and manage tasks collaboratively, addressing challenges such as the lack of private green space and limited gardening experience. However, community gardens are always accessible____.

\subsection{Supporting Gardening Through Technology}\label{subsec:tech-garden-rel}
In HCI research, various works have focused on understanding and supporting gardeners, not just in urban settings. Research on urban gardening often includes ethnographic studies of practices, traditions, and challenges in private and community gardens (e.g.,____). A key focus of this research has been exploring how technology can be introduced to better support gardeners in their activities____. Understanding where and when technology may enhance gardening or any nature engagement experience is crucial as the introduction of technology also has the potential to diminish nature experiences____.
Since gardening encompasses activities and experiences that go beyond the cultivation of plants, technological support should extend to various aspects beyond the gardening process itself. 
In a literature mapping,____ show that technical approaches surrounding urban gardening primarily aim to teach gardening skills, support the connection and coordination between gardeners, and reduce resource waste. Technologies used to support urban gardeners and gardening communities frequently fall under the broader category of IoT technologies____, including smart irrigation systems____ and sensory toolkits____ to monitor plant health markers. For instance, GrowKit____ or WeSense____ leverage smart sensors to educate users on plant health.  With "Connected Roots", ____ demonstrated how automated irrigation systems linked across multiple units in a residential building can facilitate interactions among residents interested in gardening. This approach exemplifies how automation can be used to assign new social value to a typically repetitive gardening task. When and where to incorporate technology to support gardening experiences has also been a research concern in the past. 
Additionally, a growing body of research explores ways to strengthen human-nature relationships____. For example, ____ demonstrated using camera traps to help citizens observe and reflect on backyard ecosystems, while ____ proposed using technology-mediated auditory experiences to raise awareness of local bird species and foster a sense of connectedness with local biodiversity____. These works do not directly address the topic of urban gardening in the sense of crop cultivation. However, they focus on non-human actors and habitats users may create through gardening.\\ 

In summary, prior work investigating how the introduction of technology may support the gardening endeavors of urban dwellers has focused on enhancing gardening skills, facilitating social interactions and coordination between gardeners, promoting sustainability through resource-efficient practices and monitoring tools, and aspects that go beyond the process of plant cultivation. In the context of urban gardening, technology has been successfully employed to enhance gardeners' capabilities. However, as highlighted in the previous section, many users express interest in gardening but are hindered by practical barriers. Addressing the needs of these individuals requires shifting from \textit{enhancing capabilities} to \textit{creating opportunities} through technology. 

\subsection{Collaboration with Agriculture Robots}\label{subsec:rel-robots}
Previous HCI research on enhancing gardening capabilities has primarily used traditional smart gardening devices. However, recent advances in precision agriculture enable technology to take a more active role in gardening. PARs are autonomous or semi-autonomous systems designed to perform tasks such as planting, watering, weeding, and monitoring crop health____. The scale at which PARs are deployed, whether in small gardens or large fields, affects their size, functionality, and user interaction. Larger PARs manage extensive farm operations____, while smaller ones are suited for local urban settings____. Small-scale consumer PARs such as FarmBot are often fixed in private backyards and do not require significant movement outside the designated field. Remote controllability allows users to interact with these PARs on-demand, whether nearby or at a distance____. Control methods for PARs vary depending on task complexity and environmental conditions____. Fully autonomous PARs handle simple tasks like irrigation, while more complex tasks use semi-autonomous control, where human intervention is needed for decision-making____. High-precision tasks, such as pruning or inspection, rely on teleoperation, with human operators remotely guiding the PAR step by step. Similar mixed-initiative approaches are commonly found in HRI research____.
As previously mentioned, crop cultivation requires consistent management of gardening tasks. PARs are able to execute tasks like watering____, pruning____, harvesting____, monitoring____, and mapping____, often necessitating specialized hardware____. Small-scale PARs, like those designed for individual consumers, are often built to handle various gardening tasks, prioritizing user convenience. The aforementioned factors (i.e., scale, interaction proximity, tooling) additionally influence how PARs visualize information for the user. Effective information communication is important for maintaining situation awareness, trust, and acceptance____ across diverse tasks, settings, and interaction strategies. Meta-analyses from HRI and HCI highlight the importance of minimalism and simplicity, ensuring consistency while delivering only relevant information____. The level of detail is largely influenced by the task, control strategy, and user expertise. For instance, users of commercial PARs may require less detailed information than remote operators managing large-scale farming tasks with drones.\\ 

In urban gardening, using PARs for collaborative interaction introduces novel concepts, such as fully remote engagement, due to the broad range of tasks PARs can manage. Prior work, such as ____, comprehensively reviewed the literature on technology-mediated nature engagement, finding that approaches vary across the dimensions of distance and directness. In distant settings, engagement often involves interactive videos____, abstract representations____, or computer-generated depictions____. Shared PARs represent a novel form of distant nature engagement, where remote engagement with a robotic actor leads to tangible physical changes in the environment. Further, discourse about the effects of PARs deployed at scale in future cities is already emerging (cf.____). Therefore, investigating interaction with PARs for urban gardening could open new research spaces and facilitate the design of novel urban greening strategies. The following sections detail how \system adopts this approach and addresses common barriers to urban gardening participation.