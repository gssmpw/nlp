\section{Related Research\label{lab:related}
}

MANTRA (Metaverse ready Architectures for Open Transport) is a group ____ with the goal of constructing an end-to-end reference network architecture. This facilitates the transition from aggregated to vendor-independent disaggregated open network architectures. As an implicit result, a new generation of IPoWDM networks with switches carrying 400G coherent pluggable transceivers is emerging rapidly. This architecture lies in the core of the current research which enables the efficient communication between SDN controllers and core optical routing devices (Figure \ref{figEdgecore}).

Following the open initiatives, vendor lock-in concerning transceiver devices is eliminated ____ by utilising standards like OpenConfig and OpenROADM. Transmission modes, and others attributes as well, are exposed with YANG models ____, so the SDN controller can deploy algorithms that aim at efficient mode selection during network's operational state. This enables the design of software agents handling the communication between entities in the control plane, similar to the current research.

Relying on MANTRA, conventional functionality can adapt to the future network requirements. Failure recovery is important aspect for adoption since it is not strictly defined in this standard and increases the quality of service. Some related methods aiming at failure recovery and can be used within MANTRA are presented and evaluated ____. Tuning performance of pluggable coherent devices is evaluated as well, along with recovery time with real traffic and estimation of the channel bandwidth.

Effective cooperation of packet-optical nodes which are managed by the BGP and OSPF protocols and a hierarchical control architecture, is demonstrated ____. It aims at the orchestrated provisioning and soft failure recovery in a metro network topology. The presented perspective is based on a variation from MANTRA where the OptCTL configures the optical domain and the coherent pluggables, while the PacketCTL configures the BGP and OSPF instances of the packet-optical nodes.

SDN solutions with communicating entities (like in current research) which coordinate and control modern pluggable transceivers in a multi-layer network consisting of whiteboxes are presented ____. These two approaches, i.e., the exclusive and the shared one, are evaluated according to defined workflow experiments. Included in the results are, the end-to-end connection setup time, the optical intent setup time, the discovery time of the new link in the packet domain, and finally, the packet intent setup time.