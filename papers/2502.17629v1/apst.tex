% !TeX root = main.tex 
%!TEX root = main.tex

\newcommand{\Nforward}[1]{\ensuremath{N^{>}_{\pi}(#1)}}
\newcommand{\degForward}[1]{\ensuremath{\deg^{>}_{\pi}(#1)}}



\section{Asymmetric Palette Sparsification}\label{thm:apst}

We present our APST in a more general form that allows for coloring vertices from arbitrary palettes of size proportional to their individual degrees---instead of the same original palette of $\set{1,2,\ldots,\Delta+1}$ described earlier---, namely, the $(\deg+1)$-list coloring problem. We then obtain the APST for $(\Delta+1)$ coloring as a simple corollary of this result. We remark that a PST version of $(\deg+1)$-list coloring (by sampling $\polylog{(n)}$ colors per vertex in the worst-case) 
has been previously established in~\cite{HalldorssonKNT22}; see also~\cite{AlonA20}. 

%% Generalization deg+1 APST Theorem--------------------
\begin{theorem}[\textbf{Asymmetric Palette Sparsification Theorem (for ($\deg+1$)-list coloring)}]\label{thm:apst-deg+1}
	Let $G=(V,E)$ be any $n$-vertex graph together with lists $S(v)$ with $\deg(v)+1$ arbitrary colors for each vertex $v\in V$.
	Sample a random permutation $\pi: V \rightarrow [n]$ uniformly and define 
	\[
	\ell(v) := \min \left( \deg(v)+1\, , \,  \frac{40\,n\ln{n}}{\pi(v)}\right),
	\]
	for every $v \in V$ as the size of the list of colors to be sampled for vertex $v$. Then,  
	\begin{itemize}
		\item \emph{\textbf{List sizes:}} Deterministically, $\sum_{v \in V} \ell(v) = O(n\log^2\!{(n)})$, and for any fixed vertices $u \neq v \in V$, 
		\[
		\expect{\ell(v)} = O(\log^2\!{(n)}) \quad \text{and} \quad \expect{\ell(u) \cdot \ell(v)} = O(\log^{4}\!{(n)}). 
		\] 
		\item \emph{\textbf{Colorability:}}
		If for every vertex $v \in V$ we sample a list $L(v)$ of $\ell(v)$ colors from $S(v)$ uniformly and independently, then, with high probability (over the randomness of $\ell$ and sampled lists) the greedy 
		algorithm that iterates over vertices $v \in V$ in the increasing order of $\ell(v)$ finds a proper list-coloring of $G$ from the lists 
		$\set{L(v) \mid v \in V}$. 
	\end{itemize}
	
\end{theorem}
	\Cref{thm:apst-deg+1} follows immediately from~\Cref{clm:apst-list-sizes} and~\Cref{clm:apst-colorability} proven below. 
	\begin{lemma}[\textbf{List sizes}]\label{clm:apst-list-sizes}
		Deterministically, $\sum_{v \in V} \ell(v) = O(n\log^2\!{(n)})$, and for all $u \neq v \in V$, 
		\[
		\expect{\ell(v)} = O(\log^2\!{(n)}) \qquad \text{and} \qquad \expect{\ell(u) \cdot \ell(v)} =O(\log^{4}\!{(n)}). 
		\]
	\end{lemma}
	\begin{proof}
		By the definition of $\ell(v)$, 
		\[
		\sum_{v \in V} \ell(v) \leq \sum_{v \in V} \frac{40\,n\ln{n}}{\pi(v)} = \sum_{i=1}^{n} \frac{40\,n\ln{n}}{i} = O(n\log^2{n}),
		\]
		by the sum of the Harmonic series. This proves the deterministic bound on the total size of the lists as well as the first inequality for the expected size of each list, 
		given the distribution of list sizes is symmetric across vertices. For the second inequality, we have
		\[
		\expect{\ell(u) \cdot \ell(v)}  \leq \sum_{i \neq j} \Prob(\pi(v)=i \, \wedge \, \pi(u)=j) \cdot \frac{(40\,n\ln{(n)})^2}{i\cdot j} = \frac{1600\,n^2\ln^2{(n)}}{{{n}\choose{2}}} \cdot \sum_{i =1}^{n} \sum_{j=1}^{i-1} \frac{1}{i \cdot j} = O(\log^4(n)),
		\]
		again, by the sum of the Harmonic series. This concludes the proof. \Qed{clm:apst-list-sizes}
		
	\end{proof}
	
	Establishing the colorability property is the main part of the proof. We first need some notation. 
	Iterating over vertices in the increasing order of $\ell(\cdot)$ is the same as the decreasing order of $\pi(\cdot)$. 
	With this in mind, for any vertex $v \in V$, define $\Nback{v}$ as the neighbors $u$ of $v$ with $\pi(u) < \pi(v)$, namely, the ones that are processed 
	\emph{after} $v$ by the greedy algorithm. Let  $\degback{v}:= \card{\Nback{v}}$. 
	
	An easy observation is that when coloring any vertex $v$ by the greedy algorithm, there are still at least $\degback{v}+1$ colors not used in the neighborhood of $v$. 
	Thus, lower bounding $\degback{v}$ allows us to later prove that $L(v)$ contains an available color for $v$ with high probability.  
	We only need such a bound for the subset of vertices with $\ell(v) < \deg(v)+1$ as captured in the following claim. 
	
	\begin{claim}\label{clm:apst-a(v)}
		For all $v \in V$ with $\pi(v) > 40n\ln(n)/(\deg(v)+1)$, 
		\[
		\Pr\paren{\degback{v} < \frac{\deg(v) \cdot \pi(v)}{2n}} \leq n^{-2.4}.
		\]
	\end{claim}
	\begin{proof}
		Fix any vertex $v$ as above and condition on $\pi(v) = i+1$ for some $i \in \set{0,1,\ldots,n-1}$. 
		For $j \in [\deg(v)]$, define $X_j \in \set{0,1}$ as the 
		indicator random variable which is $1$ iff the $j$-th neighbor of $v$ appears in the first $i$ vertices of $\pi$. 
		For $X$ being the sum of these $X_j$-variables (and implicitly conditioned on $\pi(v) = i+1$ to avoid cluttering the equations), we have, 
		\[
		\expect{\degback{v} \mid \pi(v)=i+1} = \expect{X} = \sum_{j=1}^{\deg(v)} \expect{X_j} = \deg(v) \cdot \frac{i}{n-1},
		\]
		as each neighbor of $v$ appears in the first $i$ position of $\pi$ with probability $i/(n-1)$ conditioned on $\pi(v) = i+1$.  
		Random variable $X$ is distributed as a hypergeometric random variable with parameters $N=n-1$, $K = i$, and $M = \deg(v)$. Thus, by~\Cref{thm:book} for the parameter
		\begin{align}
		t &:=  \expect{X} - \frac{\deg(v)\cdot (i+1)}{2n} = \frac{\deg(v) \cdot i}{n-1} - \frac{\deg(v)\cdot (i+1)}{2n} \geq  (1-o(1)) \cdot \frac{\expect{X}}{2}  \tag{by the value of $\expect{X}$ calculated above and given $i = \omega(1)$ in the claim statement}
		\end{align} 
		we have, 
		\begin{align*}
			\Pr\paren{X \leq \frac{\deg(v)\cdot (i+1)}{2n}} &= \Pr\paren{X \leq \expect{X} - t} \leq \exp\paren{-\frac{t^2}{2\expect{X}}} \leq \exp\paren{-(1-o(1)) \cdot \frac{\expect{X}}{8}} \tag{by the lower bound on $t$ established above} \\
			&= \exp\paren{-(1-o(1)) \cdot \frac{\deg(v) \cdot i}{8 \cdot (n-1)}} \leq \exp\paren{-(1-o(1)) \cdot \frac{40\ln{n}}{8}} \leq n^{-2.4},
		\end{align*}
		where in the second line we used the lower bound on $i = \pi(v)-1$ in the claim statement. 
		Given $\degback{v} = X$ and the bound holds for all choices of $\pi(v)$, we can conclude the proof. \Qed{clm:apst-a(v)}
		
	\end{proof}
	\begin{lemma}[\textbf{Colorability}]\label{clm:apst-colorability}
		With high probability, when coloring each vertex $v \in V$ in the greedy algorithm, 
		at least one of the colors sampled in $L(v)$ has not been used in the neighborhood of $v$; that is, the greedy algorithm can color this vertex. 
	\end{lemma}
	\begin{proof}	
		We condition on the choice of $\pi$ and by union bound obtain that with high probability, the complement of the event in~\Cref{clm:apst-a(v)} holds for all vertices. 
		For any vertex $v \in V$, we say a color in $S(v)$ is \textbf{available} to $v$ iff it is not assigned to any neighbor of $v$ by the time we process $v$ in the greedy algorithm. 
		Let $a(v)$ denote the number of available colors and note that $a(v) \geq  \degback{v}+1$. 
		
		In the following, we fix a vertex $v \in V$ and further condition on the randomness of $L(u)$ for all vertices $u$ with $\pi(u) > \pi(v)$. The conditionings so far fix the set of available colors and the values of $a(v)$ and $\ell(v)$, but $L(v)$ is still a random $\ell(v)$-subset of the $\deg(v)+1$ colors in $S(v)$. If $\pi(v) \leq 40n\ln(n)/(\deg(v)+1)$, we have $\ell(v) = \deg(v)+1$ which means $L(v) = S(v)$ and thus there exists an available color in $L(v)$, proving the claim for this vertex. 
		
		For the remaining vertices with $\pi(v) > 40n\ln(n)/(\deg(v)+1)$, we can apply~\Cref{clm:apst-a(v)} to have, 
		\begin{align*}
			\Pr\paren{\text{no available color of $v$ is sampled in $L(v)$}} &\leq (1-\frac{a(v)}{\deg(v)+1})^{\ell(v)} \leq \exp\paren{-\frac{a(v) \cdot \ell(v)}{\deg(v)+1}} \\ 
			& \leq \exp\paren{-\frac{\deg(v) \cdot \pi(v)}{2n} \cdot \frac{40n\ln{n}}{\pi(v)} \cdot \frac{1}{\deg(v)+1}} \leq n^{-5}.
		\end{align*}
		Thus, with high probability, we can color $v$ from $L(v)$ in the greedy algorithm. Taking a union bound over all vertices concludes the proof. 
		\Qed{clm:apst-colorability}
	\end{proof}
	

We obtain our  APST for $(\Delta+1)$ coloring described earlier as a direct corollary of~\Cref{thm:apst-deg+1}. 

\begin{corollary}[\textbf{Asymmetric Palette Sparsification Theorem (for ($\Delta+1$) coloring)}]\label{thm:apst} ~ \\
	Let $G=(V,E)$ be any $n$-vertex graph with maximum degree $\Delta$. 
	Sample a random permutation $\pi: V \rightarrow [n]$ uniformly and define 
	\[
	\ell(v) := \min \left( \deg(v)+1\, , \,  \frac{40\,n\ln{n}}{\pi(v)}\right),
	\]
	for every $v \in V$ as the size of the lists of colors to be sampled for vertex $v$. Then,  
	\begin{itemize}
		\item \emph{\textbf{List sizes:}} Deterministically, $\sum_{v \in V} \ell(v) = O(n\log^2\!{(n)})$, and for any fixed vertices $u \neq v \in V$, 
		\[
		\expect{\ell(v)} = O(\log^2\!{(n)}) \quad \text{and} \quad \expect{\ell(u) \cdot \ell(v)} = O(\log^{4}\!{(n)}). 
		\] 
		\item \emph{\textbf{Colorability:}} If for every vertex $v \in V$, we sample a list $L(v)$ of $\ell(v)$ colors from $[\Delta+1]$ uniformly and independently, then, with high probability (over the randomness of $\ell$ and sampled lists) the greedy 
		algorithm that iterates over vertices $v \in V$ in the increasing order of $\ell(v)$ finds a proper list-coloring of $G$ from the lists 
		$\set{L(v) \mid v \in V}$. 
	\end{itemize}
%%	The distribution over $\ell: V \rightarrow \IN$ is as follows. 
\end{corollary}
\begin{proof}
\noindent 
	For the \textbf{List Size} property, the distribution of $\set{\ell(v) \mid v \in V}$ is identical to~\Cref{thm:apst-deg+1} and thus we obtain this property immediately. 

	For the \textbf{Colorability} property, assume we first sample $\deg(v)+1$ colors from $[\Delta+1]$ uniformly and put them in a list $S(v)$, and then pick $\ell(v)$ colors from $S(v)$ uniformly. The set $L(v)$ is now
	distributed as a random $\ell(v)$-subset of $[\Delta+1]$ (matching the requirement of~\Cref{thm:apst}) as well as a random $\ell(v)$-subset of $S(v)$ (matching the requirement of~\Cref{thm:apst-deg+1}). 
	We can thus apply~\Cref{thm:apst-deg+1} and obtain that with high probability, vertices of $G$ can be colored from $L(v) \subseteq S(v) \subseteq [\Delta+1]$, namely, a $(\Delta+1)$ vertex coloring of $G$. 
	\Qed{thm:apst}
\end{proof}

	
