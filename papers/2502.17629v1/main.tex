\documentclass[11pt]{article}

\usepackage{amsthm}
\usepackage{graphicx} % support the \includegraphics command and options
\usepackage{array} % for better arrays (eg matrices) in maths

\usepackage{amsmath, amssymb, amsfonts, verbatim}
\usepackage{hyphenat,epsfig,subcaption,multirow}
\usepackage[font=small,labelfont=bf]{caption}
\usepackage{nicefrac}
\usepackage{paralist}


\usepackage{tocloft}
\renewcommand{\cftsecfont}{\normalfont }  


\usepackage[usenames,dvipsnames]{xcolor}
\usepackage[ruled]{algorithm2e}
\renewcommand{\algorithmcfname}{Algorithm}
%\SetVlineSkip
%\renewcommand{\SetVlineSkip}

\DeclareFontFamily{U}{mathx}{\hyphenchar\font45}
\DeclareFontShape{U}{mathx}{m}{n}{
      <5> <6> <7> <8> <9> <10>
      <10.95> <12> <14.4> <17.28> <20.74> <24.88>
      mathx10
      }{}
\DeclareSymbolFont{mathx}{U}{mathx}{m}{n}
\DeclareMathSymbol{\bigtimes}{1}{mathx}{"91}

\usepackage{tcolorbox}
\tcbuselibrary{skins,breakable}
\tcbset{enhanced jigsaw}

\usepackage[normalem]{ulem}
\usepackage[compact]{titlesec}

\definecolor{DarkRed}{rgb}{0.5,0.1,0.1}
\definecolor{DarkBlue}{rgb}{0.1,0.1,0.5}


\usepackage{nameref}
\definecolor{ForestGreen}{rgb}{0.1333,0.5451,0.1333}
%\definecolor{DarkRed}{rgb}{0.8,0,0}
\definecolor{Red}{rgb}{0.9,0,0}
\usepackage[linktocpage=true,
	pagebackref=true,colorlinks,
	linkcolor=DarkRed,citecolor=ForestGreen,
	bookmarks,bookmarksopen,bookmarksnumbered]
	{hyperref}
\usepackage[noabbrev,nameinlink]{cleveref}
\crefname{property}{property}{Property}
\creflabelformat{property}{(#1)#2#3}
\crefname{equation}{eq}{Eq}
\creflabelformat{equation}{(#1)#2#3}
%\crefrangeformat{equation}{Eqs.(#1)#4-#5(#2)#6}

\usepackage{bm}
\usepackage{url}
\usepackage{xspace}
\usepackage[mathscr]{euscript}

\usepackage{tikz}
\usetikzlibrary{arrows}
\usetikzlibrary{arrows.meta}
\usetikzlibrary{shapes}
\usetikzlibrary{backgrounds}
\usetikzlibrary{positioning}
\usetikzlibrary{decorations.markings}
\usetikzlibrary{decorations.pathreplacing} % needed for drawing braces
\usetikzlibrary{patterns}
\usetikzlibrary{calc}
\usetikzlibrary{fit}
\usetikzlibrary{decorations}

\usepackage[framemethod=TikZ]{mdframed}

\usepackage[noend]{algpseudocode}
\makeatletter
\def\BState{\State\hskip-\ALG@thistlm}
\makeatother

\usepackage{cite}
\usepackage{enumitem}
\setlist[itemize]{leftmargin=20pt}
\setlist[enumerate]{leftmargin=20pt}


\usepackage[margin=1in]{geometry}


\usepackage{thmtools}
\usepackage{thm-restate}

\newtheorem{theorem}{Theorem}
\newtheorem{lemma}{Lemma}[section]
\newtheorem{proposition}[lemma]{Proposition}
\newtheorem{corollary}[lemma]{Corollary}
\newtheorem{claim}[lemma]{Claim}
\newtheorem{fact}[lemma]{Fact}
\newtheorem{assumption}[lemma]{Assumption}
\newtheorem{invariant}[lemma]{Invariant}

\newtheorem{conj}[lemma]{Conjecture}
\newtheorem{definition}[lemma]{Definition}
\newtheorem{problem}{Problem}

\newtheorem*{claim*}{Claim}
\newtheorem*{assumption*}{Assumption}
\newtheorem*{proposition*}{Proposition}
\newtheorem*{lemma*}{Lemma}

\newtheorem{observation}[lemma]{Observation}
\newtheorem{property}{Property}


\newtheorem*{theorem*}{Theorem}

\crefname{lemma}{Lemma}{Lemmas}
\crefname{claim}{claim}{claims}
\crefname{property}{Property}{Properties}
\crefname{invariant}{Invariant}{Invariants}

\newtheorem{mdresult}{Result}
\newenvironment{result}{\begin{mdframed}[backgroundcolor=lightgray!40,topline=false,rightline=false,leftline=false,bottomline=false,innertopmargin=5pt]}{\end{mdframed}}

%\newtheorem{result}[mdresult]{Result}


%\theoremstyle{definition}
\newtheorem{remark}[lemma]{Remark}

\theoremstyle{definition}

\newtheorem{mdproblem}{Problem}
\newenvironment{Problem}{\begin{mdframed}[backgroundcolor=ForestGreen!15,topline=false,bottomline=false, innerbottommargin=12pt,innertopmargin=12pt]\begin{problem}}{\end{problem}\end{mdframed}}
\newtheorem*{mdproblem*}{Problem}
\newenvironment{Problem*}{\begin{mdframed}[hidealllines=false,innerleftmargin=10pt,backgroundcolor=gray!10,innertopmargin=5pt,innerbottommargin=5pt,roundcorner=10pt]\begin{mdproblem*}}{\end{mdproblem*}\end{mdframed}}
\newtheorem{mddefinition}[lemma]{Definition}
\newenvironment{Definition}{\begin{mdframed}[hidealllines=false,innerleftmargin=10pt,backgroundcolor=white!10,innertopmargin=5pt,innerbottommargin=5pt,roundcorner=10pt]\begin{mddefinition}}{\end{mddefinition}\end{mdframed}}
\newtheorem*{mddefinition*}{Definition}
\newenvironment{Definition*}{\begin{mdframed}[hidealllines=false,innerleftmargin=10pt,backgroundcolor=white!10,innertopmargin=5pt,innerbottommargin=5pt,roundcorner=10pt]\begin{mddefinition*}}{\end{mddefinition*}\end{mdframed}}
\newtheorem{mdremark}{Remark}
\newenvironment{Remark}{\begin{mdframed}[hidealllines=false,innerleftmargin=10pt,backgroundcolor=gray!10,innertopmargin=5pt,innerbottommargin=5pt,roundcorner=10pt]\begin{mdremark}}{\end{mdremark}\end{mdframed}}

\newenvironment{ourbox}{\begin{mdframed}[hidealllines=false,innerleftmargin=10pt,backgroundcolor=white!10,innertopmargin=2pt,innerbottommargin=5pt,roundcorner=10pt]}{\end{mdframed}}


\newtheorem{mdalgorithm}{Algorithm}
\newenvironment{Algorithm}{\begin{ourbox}\begin{mdalgorithm}}{\end{mdalgorithm}\end{ourbox}}

\allowdisplaybreaks



\DeclareMathOperator*{\argmax}{arg\,max}

\renewcommand{\qed}{\nobreak \ifvmode \relax \else
      \ifdim\lastskip<1.5em \hskip-\lastskip
      \hskip1.5em plus0em minus0.5em \fi \nobreak
      \vrule height0.75em width0.5em depth0.25em\fi}

\newcommand{\Qed}[1]{\rlap{\qed$_{\textnormal{~~\Cref{#1}}}$}}

\setlength{\parskip}{3pt}

\newcommand{\logstar}[1]{\ensuremath{\log^{*}\!{#1}}}
\newcommand{\logr}[1]{\ensuremath{\log^{(#1)}}}
\renewcommand{\leq}{\leqslant}
\renewcommand{\geq}{\geqslant}
\renewcommand{\le}{\leq}
\renewcommand{\ge}{\geq}


\newcommand{\thought}[1]{{\color[rgb]{0.2,0.39,0.66}(#1)}}
\newcommand{\todo}[1]{{\color[rgb]{1.0,0.0,0.0}(#1)}}
\newcommand{\hsh}[1]{{\color{green!50!black} Henrik: #1}}
\newcommand{\st}[1]{{\color{red!50!black} Sebastian: #1}}

\newcommand{\ulm}[1]{_{\scaleto{\mathrm{#1}}{3pt}}}
\newcommand\at[2]{\left.#1\right|_{#2}}











\newtheorem{assumption}{Assumption}

\DeclareMathOperator*{\argmax}{arg\,max}
\DeclareMathOperator*{\argmin}{arg\,min}

\newcommand{\swname}[1]{\texttt{#1}}
\newcommand{\ie}{i\/.\/e\/.,\/~}
\newcommand{\eg}{e\/.\/g\/.,\/~}
\newcommand{\cf}{cf\/.\/~}

\newcommand{\fig}{Fig\/.\/~}
\newcommand{\defn}{Def\/.\/~}
\newcommand{\sect}{Sec\/.\/~}
\newcommand{\tabl}{Tab\/.\/~}
\newcommand{\algo}{Algorithm~}
\newcommand{\theo}{Theorem~}

\newcommand{\bnnl}{3 hidden layers}
\newcommand{\bnnn}{50 neurons}
\newcommand{\bnna}{tanh activations}

\newcommand{\capt}[1]{\mdseries{\emph{#1}}}

\newcommand{\videolink}{at \url{https://youtu.be/_d7AqTRjz6g}}
\newcommand{\codelink}{\url{https://github.com/wheelbot/mini-wheelbot}}

\newcommand{\fakepar}[1]{\vspace{0mm}\noindent\textbf{#1.}}

\newcommand{\needref}{\textcolor{red}{[REF]}}

\newcommand{\plotfontsize}{9pt}


\title{Simple Sublinear Algorithms for $(\Delta+1)$ Vertex Coloring \\ via Asymmetric Palette Sparsification\footnote{A preliminary version of this work appeared in \emph{Symposium on Simplicity in Algorithms (SOSA 2025)}. \smallskip}}
\author{Sepehr Assadi\footnote{(sepehr@assadi.info) Cheriton School of Computer Science, University of Waterloo. 
Supported in part by a  Sloan Research Fellowship, an NSERC
Discovery Grant, a University of Waterloo startup grant, and a Faculty of Math Research Chair grant.} \and Helia Yazdanyar\footnote{(hyazdanyar@uwaterloo.ca) Cheriton School of Computer Science, University of Waterloo.}}

\date{}

\begin{document}
\maketitle


\begin{abstract}
Retrieval-Augmented Generation (RAG) is often used with Large Language Models (LLMs) to infuse domain knowledge or user-specific information. In RAG, given a user query, a retriever extracts chunks of relevant text from a knowledge base. These chunks are sent to an LLM as part of the input prompt. Typically, any given chunk is repeatedly retrieved across user questions. However, currently, for every question, attention-layers in LLMs fully compute the key values (KVs) repeatedly for the input chunks, as state-of-the-art methods cannot reuse KV-caches when chunks appear at arbitrary locations with arbitrary contexts. Naive reuse leads to output quality degradation.  This leads to potentially redundant computations on expensive GPUs and increases latency. In this work, we propose \sys, a system for managing and reusing precomputed KVs corresponding to the text chunks (we call \textit{chunk-caches}) in RAG-based systems. We present how to identify \hl{\textit{chunk-caches} that are reusable}, how to efficiently perform a small fraction of recomputation to \textit{fix} the cache to maintain output quality, and how to efficiently store and evict \textit{chunk-caches} in the hardware for maximizing reuse while masking any overheads. With real production workloads as well as synthetic datasets, we show that \sys reduces redundant computation by \textbf{51\%} over SOTA prefix-caching and \textbf{75\%} over full recomputation.
\hl{Additionally, with continuous batching on a real production workload, we get a \textbf{1.6$\times$} speedup in throughput and a \textbf{2$\times$} reduction in end-to-end response latency over prefix-caching while maintaining quality, for both the \llama-3-8B and \llama-3-70B models. 
}
\end{abstract}






\clearpage

\section{Introduction}
\label{sec:intro}

\begin{figure*}[tb]
    \centering
    \includegraphics[width=0.848\linewidth]{figs/circuitnn.pdf} 
    \caption{Illustration of differentiable CircuitNN. CircuitNN is designed based on differentiable NAND gates. After DAS is guided by PI and PO pairs of the truth table, CircuitNN can get the precise circuit architecture logic equivalent to the truth table.}
    \label{fig:circuitnn}
\end{figure*}

% 1. Describe the importance of logic synthesis
% 2. Existing Problems
% (a) Neural Architecture Search: Unstable, Predefined Setting, etc.
% (b) Circuit Generation: Probabilistic Model, Logic Equivalence

With the rapid advancement of technology, the scale of integrated circuits (ICs) has expanded exponentially. 
This expansion has introduced significant challenges in chip manufacturing, particularly concerning power and area metrics.
A primary objective in IC design is achieving the same circuit function with fewer transistors, thereby reducing power usage and area occupancy.

Logic synthesis~\cite{hachtel2005logicsynth}, a critical step in electronic design automation (EDA), transforms behavioral-level circuit designs into optimized gate-level circuits, ultimately yielding the final IC layout. 
The primary goal of logic synthesis is to identify the physical implementation with the fewest gates for a given circuit function. 
This task constitutes a challenging NP-hard combinatorial optimization problem. 
Current logic synthesis tools~\cite{brayton2010abc, wolf2013yosys} rely on human-designed heuristics, often leading to sub-optimal outcomes.

Differentiable architecture search (DAS) techniques~\cite{liu2018darts, chu2020darts} offer novel perspectives on addressing challenges in this problem.
Circuit functions can be represented through truth tables, which map binary inputs to their corresponding outputs. 
Truth tables provide a precise representation of input-output relationships, ensuring the design of functionally equivalent circuits.
Inspired by this, researchers~\cite{deepmind2024ai4sys, wang2024tnet} have begun exploring the application of DAS to synthesize circuits directly from truth tables.
Specifically, \citet{deepmind2024ai4sys} proposed CircuitNN, a framework that learns differentiable connection structures with logic gates, enabling the automatic generation of logic circuits from truth tables.
This approach significantly reduces the complexity of traditional circuit generation. 
Building on this, \citet{wang2024tnet} introduced T-Net, a triangle-shaped variant of CircuitNN, incorporating regularization techniques to enhance the efficiency of DAS.

Despite these advancements, several challenges remain. 
The computational complexity of DAS grows quadratically with the number of gates, posing scalability issues.
Although triangle-shaped architecture~\cite{wang2024tnet} partially mitigates this problem, redundancy persists. 
%Additionally, DAS is susceptible to converging to local optima, limiting the ability to search architectures that satisfy the given truth tables~\cite{liu2018darts}. 
%Furthermore, hyperparameters (network depth and layer width) require extensive searches, introducing complexity and prolonging the synthesis process. 
Additionally, DAS is susceptible to converging to local optima~\cite{liu2018darts} and hyperparameters (network depth and layer width) require extensive searches. 
The challenges arise from the vast search space in DAS. 
% Even with predefined settings for CircuitNN, finding a configuration that meets the truth table requires extensive trial and error during the DAS process. 
Intuitively, limiting the search space through predefined parameters (network depth, gates per layer, and connection probabilities) can significantly reduce the complexity.

Recent advances~\cite{openai2023gpt4, abramson2024alphafold3, esser2024sd3, li2024mar} in conditional generative models have demonstrated remarkable performance across language, vision, and graph generation tasks. 
Motivated by these developments, we propose a novel approach to circuit generation that generates preliminary circuit structures to guide DAS in generating refined circuits matching specified truth tables. 
Firstly, we introduce CircuitVQ, a tokenizer with a discrete codebook for circuit tokenization. 
Built upon our Circuit AutoEncoder framework~\cite{hou2022graphmae,li2023maskgae,wu2025mgvga}, CircuitVQ is trained through a circuit reconstruction task. 
Specifically, the CircuitVQ encoder encodes input circuits into discrete tokens using a learnable codebook, while the decoder reconstructs the circuit adjacency matrix based on these tokens.
Subsequently, the CircuitVQ encoder serves as a circuit tokenizer for CircuitAR pretraining, which employs a masked autoregressive modeling paradigm~\cite{chang2022maskgit, li2023mage}. 
In this process, the discrete codes function as supervision signals. 
After training, CircuitAR can generate discrete tokens progressively, which can be decoded into initial circuit structures by the decoder of the CircuitVQ. 
These prior insights can guide DAS in producing refined circuits that match the target truth tables precisely.

Our key contributions can be summarized as follows:
\begin{itemize}
\item We introduce CircuitVQ, a circuit tokenizer that facilitates graph autoregressive modeling for circuit generation, based on our Circuit AutoEncoder framework;
\item Develop CircuitAR, a model trained using masked autoregressive modeling, which generates initial circuit structures conditioned on given truth tables;
\item Propose a refinement framework that integrates differentiable architecture search to produce functionally equivalent circuits guided by target truth tables;
\item Comprehensive experiments demonstrating the scalability and capability emergence of our CircuitAR and the superior performance of the proposed circuit generation approach.
\end{itemize}

% Motivation
% (a) Diffusion (Vision, Graph), Autoregressive (Language, Vision)
% (b) Circuit Generation for Predefined Setting
% (c) Neural Architecture Search for Strict Logic Equivalence

% Contribution
% (a) Circuit Tokenizer (new transformer arch, training strategy)
% (b) CircuitAR (train and gen strategies, post-ar strategy)
% (c) Extensive Evaluation including BitD (Bit Distance) for Scalability


\section{Preliminary}

\paragraph{Notation} Consider a sentence of $T$ tokens $\vx=\{\vx_1,\ldots, \vx_T\}\in\gX$, and let $P$ be the unknown target language distribution, $\tilde P(\vx)$ be the empirical distribution of the training data (which is an approximation of $P$), and $Q$ be the distribution of our model at hand. Since our paper is also closely related to RLHF, we will also use $\pi$ to represent the distributions. In particular, we sometimes write $\pi_\theta$ for a distribution that is parameterized by $\theta$, where $\theta$ is usually the set of trainable parameters of the LLM; we write $\pr$ for a reference distribution that should be clear given the context. The next token prediction loss is minimizing the forward-KL between $P$ and $Q$. 






% !TeX root = main.tex 
%!TEX root = main.tex

\newcommand{\Nforward}[1]{\ensuremath{N^{>}_{\pi}(#1)}}
\newcommand{\degForward}[1]{\ensuremath{\deg^{>}_{\pi}(#1)}}



\section{Asymmetric Palette Sparsification}\label{thm:apst}

We present our APST in a more general form that allows for coloring vertices from arbitrary palettes of size proportional to their individual degrees---instead of the same original palette of $\set{1,2,\ldots,\Delta+1}$ described earlier---, namely, the $(\deg+1)$-list coloring problem. We then obtain the APST for $(\Delta+1)$ coloring as a simple corollary of this result. We remark that a PST version of $(\deg+1)$-list coloring (by sampling $\polylog{(n)}$ colors per vertex in the worst-case) 
has been previously established in~\cite{HalldorssonKNT22}; see also~\cite{AlonA20}. 

%% Generalization deg+1 APST Theorem--------------------
\begin{theorem}[\textbf{Asymmetric Palette Sparsification Theorem (for ($\deg+1$)-list coloring)}]\label{thm:apst-deg+1}
	Let $G=(V,E)$ be any $n$-vertex graph together with lists $S(v)$ with $\deg(v)+1$ arbitrary colors for each vertex $v\in V$.
	Sample a random permutation $\pi: V \rightarrow [n]$ uniformly and define 
	\[
	\ell(v) := \min \left( \deg(v)+1\, , \,  \frac{40\,n\ln{n}}{\pi(v)}\right),
	\]
	for every $v \in V$ as the size of the list of colors to be sampled for vertex $v$. Then,  
	\begin{itemize}
		\item \emph{\textbf{List sizes:}} Deterministically, $\sum_{v \in V} \ell(v) = O(n\log^2\!{(n)})$, and for any fixed vertices $u \neq v \in V$, 
		\[
		\expect{\ell(v)} = O(\log^2\!{(n)}) \quad \text{and} \quad \expect{\ell(u) \cdot \ell(v)} = O(\log^{4}\!{(n)}). 
		\] 
		\item \emph{\textbf{Colorability:}}
		If for every vertex $v \in V$ we sample a list $L(v)$ of $\ell(v)$ colors from $S(v)$ uniformly and independently, then, with high probability (over the randomness of $\ell$ and sampled lists) the greedy 
		algorithm that iterates over vertices $v \in V$ in the increasing order of $\ell(v)$ finds a proper list-coloring of $G$ from the lists 
		$\set{L(v) \mid v \in V}$. 
	\end{itemize}
	
\end{theorem}
	\Cref{thm:apst-deg+1} follows immediately from~\Cref{clm:apst-list-sizes} and~\Cref{clm:apst-colorability} proven below. 
	\begin{lemma}[\textbf{List sizes}]\label{clm:apst-list-sizes}
		Deterministically, $\sum_{v \in V} \ell(v) = O(n\log^2\!{(n)})$, and for all $u \neq v \in V$, 
		\[
		\expect{\ell(v)} = O(\log^2\!{(n)}) \qquad \text{and} \qquad \expect{\ell(u) \cdot \ell(v)} =O(\log^{4}\!{(n)}). 
		\]
	\end{lemma}
	\begin{proof}
		By the definition of $\ell(v)$, 
		\[
		\sum_{v \in V} \ell(v) \leq \sum_{v \in V} \frac{40\,n\ln{n}}{\pi(v)} = \sum_{i=1}^{n} \frac{40\,n\ln{n}}{i} = O(n\log^2{n}),
		\]
		by the sum of the Harmonic series. This proves the deterministic bound on the total size of the lists as well as the first inequality for the expected size of each list, 
		given the distribution of list sizes is symmetric across vertices. For the second inequality, we have
		\[
		\expect{\ell(u) \cdot \ell(v)}  \leq \sum_{i \neq j} \Prob(\pi(v)=i \, \wedge \, \pi(u)=j) \cdot \frac{(40\,n\ln{(n)})^2}{i\cdot j} = \frac{1600\,n^2\ln^2{(n)}}{{{n}\choose{2}}} \cdot \sum_{i =1}^{n} \sum_{j=1}^{i-1} \frac{1}{i \cdot j} = O(\log^4(n)),
		\]
		again, by the sum of the Harmonic series. This concludes the proof. \Qed{clm:apst-list-sizes}
		
	\end{proof}
	
	Establishing the colorability property is the main part of the proof. We first need some notation. 
	Iterating over vertices in the increasing order of $\ell(\cdot)$ is the same as the decreasing order of $\pi(\cdot)$. 
	With this in mind, for any vertex $v \in V$, define $\Nback{v}$ as the neighbors $u$ of $v$ with $\pi(u) < \pi(v)$, namely, the ones that are processed 
	\emph{after} $v$ by the greedy algorithm. Let  $\degback{v}:= \card{\Nback{v}}$. 
	
	An easy observation is that when coloring any vertex $v$ by the greedy algorithm, there are still at least $\degback{v}+1$ colors not used in the neighborhood of $v$. 
	Thus, lower bounding $\degback{v}$ allows us to later prove that $L(v)$ contains an available color for $v$ with high probability.  
	We only need such a bound for the subset of vertices with $\ell(v) < \deg(v)+1$ as captured in the following claim. 
	
	\begin{claim}\label{clm:apst-a(v)}
		For all $v \in V$ with $\pi(v) > 40n\ln(n)/(\deg(v)+1)$, 
		\[
		\Pr\paren{\degback{v} < \frac{\deg(v) \cdot \pi(v)}{2n}} \leq n^{-2.4}.
		\]
	\end{claim}
	\begin{proof}
		Fix any vertex $v$ as above and condition on $\pi(v) = i+1$ for some $i \in \set{0,1,\ldots,n-1}$. 
		For $j \in [\deg(v)]$, define $X_j \in \set{0,1}$ as the 
		indicator random variable which is $1$ iff the $j$-th neighbor of $v$ appears in the first $i$ vertices of $\pi$. 
		For $X$ being the sum of these $X_j$-variables (and implicitly conditioned on $\pi(v) = i+1$ to avoid cluttering the equations), we have, 
		\[
		\expect{\degback{v} \mid \pi(v)=i+1} = \expect{X} = \sum_{j=1}^{\deg(v)} \expect{X_j} = \deg(v) \cdot \frac{i}{n-1},
		\]
		as each neighbor of $v$ appears in the first $i$ position of $\pi$ with probability $i/(n-1)$ conditioned on $\pi(v) = i+1$.  
		Random variable $X$ is distributed as a hypergeometric random variable with parameters $N=n-1$, $K = i$, and $M = \deg(v)$. Thus, by~\Cref{thm:book} for the parameter
		\begin{align}
		t &:=  \expect{X} - \frac{\deg(v)\cdot (i+1)}{2n} = \frac{\deg(v) \cdot i}{n-1} - \frac{\deg(v)\cdot (i+1)}{2n} \geq  (1-o(1)) \cdot \frac{\expect{X}}{2}  \tag{by the value of $\expect{X}$ calculated above and given $i = \omega(1)$ in the claim statement}
		\end{align} 
		we have, 
		\begin{align*}
			\Pr\paren{X \leq \frac{\deg(v)\cdot (i+1)}{2n}} &= \Pr\paren{X \leq \expect{X} - t} \leq \exp\paren{-\frac{t^2}{2\expect{X}}} \leq \exp\paren{-(1-o(1)) \cdot \frac{\expect{X}}{8}} \tag{by the lower bound on $t$ established above} \\
			&= \exp\paren{-(1-o(1)) \cdot \frac{\deg(v) \cdot i}{8 \cdot (n-1)}} \leq \exp\paren{-(1-o(1)) \cdot \frac{40\ln{n}}{8}} \leq n^{-2.4},
		\end{align*}
		where in the second line we used the lower bound on $i = \pi(v)-1$ in the claim statement. 
		Given $\degback{v} = X$ and the bound holds for all choices of $\pi(v)$, we can conclude the proof. \Qed{clm:apst-a(v)}
		
	\end{proof}
	\begin{lemma}[\textbf{Colorability}]\label{clm:apst-colorability}
		With high probability, when coloring each vertex $v \in V$ in the greedy algorithm, 
		at least one of the colors sampled in $L(v)$ has not been used in the neighborhood of $v$; that is, the greedy algorithm can color this vertex. 
	\end{lemma}
	\begin{proof}	
		We condition on the choice of $\pi$ and by union bound obtain that with high probability, the complement of the event in~\Cref{clm:apst-a(v)} holds for all vertices. 
		For any vertex $v \in V$, we say a color in $S(v)$ is \textbf{available} to $v$ iff it is not assigned to any neighbor of $v$ by the time we process $v$ in the greedy algorithm. 
		Let $a(v)$ denote the number of available colors and note that $a(v) \geq  \degback{v}+1$. 
		
		In the following, we fix a vertex $v \in V$ and further condition on the randomness of $L(u)$ for all vertices $u$ with $\pi(u) > \pi(v)$. The conditionings so far fix the set of available colors and the values of $a(v)$ and $\ell(v)$, but $L(v)$ is still a random $\ell(v)$-subset of the $\deg(v)+1$ colors in $S(v)$. If $\pi(v) \leq 40n\ln(n)/(\deg(v)+1)$, we have $\ell(v) = \deg(v)+1$ which means $L(v) = S(v)$ and thus there exists an available color in $L(v)$, proving the claim for this vertex. 
		
		For the remaining vertices with $\pi(v) > 40n\ln(n)/(\deg(v)+1)$, we can apply~\Cref{clm:apst-a(v)} to have, 
		\begin{align*}
			\Pr\paren{\text{no available color of $v$ is sampled in $L(v)$}} &\leq (1-\frac{a(v)}{\deg(v)+1})^{\ell(v)} \leq \exp\paren{-\frac{a(v) \cdot \ell(v)}{\deg(v)+1}} \\ 
			& \leq \exp\paren{-\frac{\deg(v) \cdot \pi(v)}{2n} \cdot \frac{40n\ln{n}}{\pi(v)} \cdot \frac{1}{\deg(v)+1}} \leq n^{-5}.
		\end{align*}
		Thus, with high probability, we can color $v$ from $L(v)$ in the greedy algorithm. Taking a union bound over all vertices concludes the proof. 
		\Qed{clm:apst-colorability}
	\end{proof}
	

We obtain our  APST for $(\Delta+1)$ coloring described earlier as a direct corollary of~\Cref{thm:apst-deg+1}. 

\begin{corollary}[\textbf{Asymmetric Palette Sparsification Theorem (for ($\Delta+1$) coloring)}]\label{thm:apst} ~ \\
	Let $G=(V,E)$ be any $n$-vertex graph with maximum degree $\Delta$. 
	Sample a random permutation $\pi: V \rightarrow [n]$ uniformly and define 
	\[
	\ell(v) := \min \left( \deg(v)+1\, , \,  \frac{40\,n\ln{n}}{\pi(v)}\right),
	\]
	for every $v \in V$ as the size of the lists of colors to be sampled for vertex $v$. Then,  
	\begin{itemize}
		\item \emph{\textbf{List sizes:}} Deterministically, $\sum_{v \in V} \ell(v) = O(n\log^2\!{(n)})$, and for any fixed vertices $u \neq v \in V$, 
		\[
		\expect{\ell(v)} = O(\log^2\!{(n)}) \quad \text{and} \quad \expect{\ell(u) \cdot \ell(v)} = O(\log^{4}\!{(n)}). 
		\] 
		\item \emph{\textbf{Colorability:}} If for every vertex $v \in V$, we sample a list $L(v)$ of $\ell(v)$ colors from $[\Delta+1]$ uniformly and independently, then, with high probability (over the randomness of $\ell$ and sampled lists) the greedy 
		algorithm that iterates over vertices $v \in V$ in the increasing order of $\ell(v)$ finds a proper list-coloring of $G$ from the lists 
		$\set{L(v) \mid v \in V}$. 
	\end{itemize}
%%	The distribution over $\ell: V \rightarrow \IN$ is as follows. 
\end{corollary}
\begin{proof}
\noindent 
	For the \textbf{List Size} property, the distribution of $\set{\ell(v) \mid v \in V}$ is identical to~\Cref{thm:apst-deg+1} and thus we obtain this property immediately. 

	For the \textbf{Colorability} property, assume we first sample $\deg(v)+1$ colors from $[\Delta+1]$ uniformly and put them in a list $S(v)$, and then pick $\ell(v)$ colors from $S(v)$ uniformly. The set $L(v)$ is now
	distributed as a random $\ell(v)$-subset of $[\Delta+1]$ (matching the requirement of~\Cref{thm:apst}) as well as a random $\ell(v)$-subset of $S(v)$ (matching the requirement of~\Cref{thm:apst-deg+1}). 
	We can thus apply~\Cref{thm:apst-deg+1} and obtain that with high probability, vertices of $G$ can be colored from $L(v) \subseteq S(v) \subseteq [\Delta+1]$, namely, a $(\Delta+1)$ vertex coloring of $G$. 
	\Qed{thm:apst}
\end{proof}

	


% !TeX root = main.tex 
%!TEX root = main.tex

\newcommand{\LL}{\ensuremath{\mathcal{L}}}



\section{Sublinear Algorithms from Asymmetric Palette Sparsification}\label{sec:asymmetric} 

We now show how our asymmetric palette sparsification theorem in~\Cref{thm:apst} can be used to obtain sublinear algorithms for $(\Delta+1)$ vertex coloring. These algorithms are more or less identical 
to the (exponential-time) sublinear algorithms of~\cite{AssadiCK19} from their original palette sparsification theorem and we claim no novelty in this part\footnote{We do note that however, the time-efficient algorithms of~\cite{AssadiCK19} are considerably more complex. They first need to find a so-called sparse-dense decomposition of the input graph via sublinear algorithms. Then, this decomposition is used to color the final graph from the sampled colors using an algorithmic version of the proof of their palette sparsification theorem which in particular requires a highly non-greedy and not-so-simple approach.}. Instead, we merely point out how the ``asymmetry'' in list-sizes in~\Cref{thm:apst} 
does not weaken the performance of the resulting sublinear algorithms beyond some $\polylog{(n)}$-factors, but instead leads to much simpler post-processing algorithms for finding the coloring of the graph from the sampled lists. 

\begin{theorem}\label{thm:sublinear}
	There exist randomized sublinear algorithms that given any graph $G=(V,E)$ with maximum degree $\Delta$ with high probability output a $(\Delta+1)$ vertex coloring of $G$ using: 
	\begin{itemize}
		\item \emph{\textbf{Graph streaming:}} a single pass over the edges of $G$ in any order and $\Ot(n)$ space; 
		\item \emph{\textbf{Sublinear time:}} $\Ot(n^{1.5})$ time and non-adaptive queries to adjacency list and matrix of $G$;
		\item \emph{\textbf{Massively parallel computation (MPC):}} $O(1)$ rounds with machines of $\Ot(n)$ memory.  
	\end{itemize}
\end{theorem}

As stated earlier, the proof of~\Cref{thm:sublinear} follows the same exact strategy as the (exponential-time) sublinear algorithms of~\cite{AssadiCK19}. To do so, we need the following definition. 

\paragraph{Conflict graphs.} Let $G=(V,E)$ be any graph with maximum degree $\Delta$ and $\LL := \set{L(v) \mid v \in V}$ be a set of lists of colors sampled for vertices of $G$ according to the distribution of~\Cref{thm:apst}. 
We define the \textbf{conflict graph} $G_\LL = (V, E_{\LL})$ of $G$ and $\LL$ as the spanning subgraph of $G$ consisting of all edges $(u,v) \in E$ such that the sampled lists $L(u)$ and $L(v)$ intersect with each other. 

The following observation allows us to use conflict graphs in our sublinear algorithms as a proxy to the original graph $G$. 

\begin{observation}\label{obs:conflict-graph-use}
	The greedy algorithm in~\Cref{thm:apst} outputs the same exact coloring when run over the conflict graph $G_{\LL}$ instead of the original graph $G$. 
\end{observation}
\begin{proof}
	The only edges that affect the greedy algorithm of~\Cref{thm:apst} are edges $(u,v) \in E$ such that $L(u) \cap L(v)$ is non-empty. These edges are identical in $G$ and $G_{\LL}$. 
\end{proof}

The following easy claim also allows us to bound the size of the conflict graphs. 

\begin{claim}\label{clm:conflict-graph-size}
	The list of colors $\LL$ consists of $O(n\log^2{n})$ colors deterministically and the expected number of edges in $G_\LL=(V,E_{\LL})$ is $\Exp\card{E_{\LL}} = O(n\log^4{n})$. 
\end{claim}
\begin{proof}
	By the list sizes property of \Cref{thm:apst}, we already know that  $\LL$ consists of $O(n\log^2{n})$ colors deterministically. For the second part, for any edge $(u,v)$, we have, 
	\[
		\Pr\paren{\text{$(u,v)$ is in $E_{\LL}$}} \leq \Exp_{\ell(u),\ell(v)}\bracket{\Pr\paren{L(u) \cap L(v) \neq \emptyset \mid \ell(u),\ell(v)}} \leq \expect{\frac{\ell(u) \cdot \ell(v)}{\Delta+1}} = \frac{O(\log^4{(n)})}{\Delta},
	\]
	where the first inequality is by the law of conditional expectation (by conditioning on list-sizes first), the second is by union bound, and the third is by the list-size properties of~\Cref{thm:apst}. Since the total
	number of edges in $G$ is at most $n\Delta/2$, we can conclude the proof. 	
\end{proof}

We now prove~\Cref{thm:sublinear} for each family of sublinear algorithms separately. In the following, we prove the resource guarantees of the algorithms only in expectation instead of deterministically. However, using the standard trick of running $O(\log{n})$ copies of the algorithm in parallel, terminating any copy that uses more than twice the expected resources, and returning the answer of any of the remaining ones, we obtain the desired
algorithms in~\Cref{thm:sublinear} as well (this reduction only increases space/query/memory of algorithms with an $O(\log{n})$ multiplicative factor and the error probability with a $1/\poly(n)$ additive factor). 

\paragraph{Graph streaming.} At the beginning of the stream, sample the colors $\LL$ using~\Cref{thm:apst} and during the stream, only store the edges that belong to $G_{\LL}$. In the end, 
run the greedy algorithm on $G_{\LL}$ and return the coloring.~\Cref{thm:apst} ensures that with high probability $G$ is (list-)colorable from the sampled lists which leads to a $(\Delta+1)$ coloring of the entire graph. 
\Cref{obs:conflict-graph-use} ensures that we only need to work with $G_{\LL}$ at the end of the stream and not all of $G$, and~\Cref{clm:conflict-graph-size} bounds the space of the algorithm with $\Ot(n)$ space in expectation. 

We can implement this algorithm in dynamic streams also by recovering the conflict graph using a sparse recovery sketch instead of explicitly storing each of its edges in the stream. See~\cite{AhnGM12} for more on dynamic streams. 

\paragraph{Sublinear time.} Sample the colors $\LL$ using~\Cref{thm:apst} and query the edges between \emph{all} pairs of vertices $u \neq v \in V$ with $L(u) \cap L(v)$ being non-empty to find the edges of $G_{\LL}$. The same analysis as in~\Cref{clm:conflict-graph-size}
applied to the ${{n}\choose{2}}$ vertex-pairs (instead of $\leq n\Delta/2$ edges), ensures that the expected number of queries is $\Ot(n^2/\Delta)$. We can then color $G_{\LL}$ using the greedy algorithm in $\Ot(n)$ time. The correctness follows 
from~\Cref{thm:apst} and \Cref{obs:conflict-graph-use} as before. This algorithm only requires adjacency matrix access to $G$ and we run it when $\Delta \geq \sqrt{n}$ to obtain $\Ot(n\sqrt{n})$ time/query algorithm. When $\Delta \leq \sqrt{n}$, we instead run the standard greedy algorithm using $O(n\Delta) = \Ot(n\sqrt{n})$ 
time and queries to the adjacency list of $G$ instead. 

\paragraph{MPC.} The algorithm is almost identical to the semi-streaming one. Suppose we have access to public randomness. Then, we can sample the lists $\LL$ publicly, and each machine that has an 
edge in $G_{\LL}$ can send it to a designated machine. This way, a single machine receives all of $G_{\LL}$ and can color it greedily. The correctness follows from~\Cref{thm:apst} and~\Cref{obs:conflict-graph-use} and the memory needed for this designated machine will be $\Ot(n)$ in expectation by~\Cref{clm:conflict-graph-size}. 
Finally, we can remove the public randomness by having one machine do the sampling first on its own and share it with all the remaining machines in $O(1)$ rounds using the standard MPC primitives of search and sort. 


 

\section*{Acknowledgement} 
\addcontentsline{toc}{section}{Acknowledgement}

We would like to thank Soheil Behnezhad, Yannic Maus, and Ronitt Rubinfeld for helpful discussions and their encouragement toward finding a simpler sublinear algorithms for $(\Delta+1)$ vertex coloring. 
We are also thankful to Chris Trevisan for pointing out a shorter proof of~\Cref{lem:warmup-RHS-suffices} included in this version of the paper. Sepehr Assadi is additionally grateful to Yu Chen and Sanjeev Khanna for their prior collaboration in~\cite{AssadiCK19} that formed the backbone of
this work.

Part of this work was conducted while the first named author was visiting the Simons Institute for the Theory of Computing as part of the Sublinear Algorithms program. 

\bibliographystyle{alpha}
\bibliography{general}

% !TeX root = main.tex 
%!TEX root = main.tex


\clearpage

\appendix
\section{Appendix: A Self-contained Sublinear Time Algorithm}\label{sec:warm-up}

In addition to our main result, we also first present a very simple sublinear time algorithm for $(\Delta+1)$ coloring in the following theorem. This algorithm has recently and independently been discovered in~\cite{ferber2025improved} 
and has also appeared as part of lecture notes for different courses~\cite{Lec24,Lec25,HW} at this point. Finally, this algorithm can also be seen as a simple adjustment to the sublinear time $(\Delta+o(\Delta))$ coloring algorithm of~\cite{MorrisS21}. 


\begin{theorem}\label{thm:warm-up}
	There is a randomized algorithm that given any graph $G$ with maximum degree $\Delta$ via adjacency list and matrix access, 
	outputs a $(\Delta+1)$ coloring of $G$ in  $O(n \cdot \sqrt{n\log{n}})$ expected time. 
\end{theorem}

We note that unlike our sublinear time algorithm in~\Cref{thm:sublinear} which was \emph{non-adaptive}, namely, made all its queries in advance before seeing the answer to them, the current algorithm is adaptive and needs to 
receive the answer to each query before deciding its next query. 

The algorithm in~\Cref{thm:warm-up} is quite similar to the standard greedy algorithm for $(\Delta+1)$ coloring. It iterates over the vertices
and color each one greedily by finding a color not used in the neighbors' of this vertex yet (which exists by pigeonhole principle). However, unlike the greedy algorithm, $(i)$ it crucially needs to iterate
over the vertices in a random order, and, $(ii)$ instead of 
iterating over the neighbors of the current vertex to find an available color, it samples a color randomly for this vertex and then iterates over all vertices with this color to make sure they are not neighbor to the current vertex. 
Formally, the algorithm is as follows. 

\begin{Algorithm}\label{alg:warmup}
	An (adaptive) sublinear time algorithm for $(\Delta+1)$ vertex coloring. 
	
	\begin{enumerate}
		\item Let $C_1,C_2,\ldots,C_{\Delta+1}$ be the \textbf{color classes} to be output at the end, initially set to empty. 
		\item Pick a permutation $\pi$ of vertices in $V$ uniformly at random.
		\item For $v \in V$ in the order of the permutation $\pi$: 
		\begin{enumerate}
			\item\label{line:reset} Sample $c \in [\Delta+1]$ uniformly at random. 
			\item For every vertex $u \in C_c$, check if $(u,v)$ is an edge in $G$; if \emph{Yes}, restart from Line~\eqref{line:reset}. 
			\item If the algorithm reaches this step, color the vertex $v$ with $c$ and add $v$ to $C_c$. 
		\end{enumerate}
	\end{enumerate}
	
\end{Algorithm}

It is easy to see that this algorithm never outputs a wrong coloring (namely, if it ever terminates, its answer is always correct). Any new vertex colored does not create a conflict with previously colored vertices (given the algorithm explicitly checks to not color $v$ with a color $c$ if one of its neighbors is already colored $c$) and thus at the end, there cannot be any monochromatic edge in the graph. The interesting part of the analysis is to show that the algorithm terminates quickly enough, which is captured by the following lemma. 

\begin{lemma}\label{lem:warmup-runtime}
	For any input graph $G=(V,E)$, the expected runtime of~\Cref{alg:warmup} is 
	\[
	O(\frac{n^2}{\Delta} \cdot \log{\Delta}) ~\text{time}. 
	\] 
\end{lemma}

\noindent
To continue, we need to set up some notation.  For any vertex $v \in V$, we define the random variable $X_v$ as the number of $(u,v)$ queries checked by the algorithm in the for-loop of coloring $v$. Additionally, for any $v \in V$ and permutation $\pi$ picked over $V$, define $\Nback{v}$ as the neighbors $u$ of $v$ with $\pi(u) < \pi(v)$, namely, the ones that are colored \emph{before} $v$ by Algorithm~\Cref{alg:warmup}. Let $\degback{v}:= \card{\Nback{v}}$. We start with the basic observation that the runtime of~\Cref{alg:warmup} can be stated in terms of the variables $\set{X_v}_{v \in V}$. 

\begin{observation}\label{obs:warmup-Xv}
	The expected runtime of~\Cref{alg:warmup} is $O(\sum_{v \in V} \expect{X_v})$. 
\end{observation}
\begin{proof}
	By definition, $X_v$ is the number of queries checked in the for-loop of coloring $v$. The algorithm repeats this for-loop for all $v \in V$, so the expected runtime is the number of all queries checked in this algorithm which is proportional to $\expect{\sum_{v\in V} X_v}$. Applying linearity of expectation concludes the proof. 
\end{proof}

Our task is now to bound each of $\expect{X_v}$ for $v \in V$ to bound the runtime of the algorithm using~\Cref{obs:warmup-Xv}. 

\begin{lemma}\label{lem:warmup-Xv-bound}
	For any vertex $v \in V$ and any choice of the permutation $\pi$:  
	\[
	\expect{X_v \mid \pi} \leq \frac{n}{\Delta+1-\degback{v}}.
	\]
\end{lemma}

To prove~\Cref{lem:warmup-Xv-bound}, we first need the following claim. 

\begin{claim}\label{clm:warmup-Xv-step}
	Fix any vertex $v \in V$, any choice of the permutation $\pi$, and any assignment of colors $C(u_1),C(u_2),\ldots$ by~\Cref{alg:warmup} to all vertices that appear before $v$ in $\pi$. 
	Then, 
	\[
	\expect{X_v\mid \pi} = \frac{\expect{\card{C_c}\mid \pi}}{\Pr\paren{\text{$c$ does not appear in $\Nback{v}$} \mid \pi, C(u_1),C(u_2),\ldots}},
	\]
	where in the RHS, both the expectation and the probability are taken with respect to a color $c$ chosen uniformly at random from $[\Delta+1]$. 
\end{claim}
\begin{proof}
	Define the colors $B(v)$ as the set of colors that appear in $\Nback{v}$, that is 
	\[
	B(v) := \set{c \in [\Delta+1] \mid  \text{there exists $u\in\Nback{v}$ with $c(u)=c$}}.
	\]
	
	For every color $c$, if $c$ is in $B(v)$ then $v$ cannot be colored by $c$, and otherwise it can. The probability of picking each color $c$ is $1/(\Delta +1)$. 
	For $c \notin B(v)$, the number of needed queries before coloring $v$ is $\card{C_c}$. 
	For $c \in B(v)$, the algorithm first needs to check up to $\card{C_c}$ queries to know this color is not available to $v$, and then it simply needs to repeat the same exact process.
	As such, 
	\[
	\expect{X_v\mid \pi}  = \sum_{c \notin B(v)} \frac{1}{\Delta+1}\card{C_{c}} + \sum_{c \in B(v)} \frac{1}{\Delta+1}(\card{C_{c}}+ \expect{X_v\mid \pi} ) = \expect{\card{C_c}\mid \pi} + \frac{\card{B(v)}}{\Delta +1} \cdot \expect{X_v\mid \pi}.
	\]
	
	We can also define the probability of $c$ not appearing in $\Nback{v}$ in terms of  $\card{B(v)}$ as below:
	
	\[
	\Pr\paren{\text{$c$ does not appear in $\Nback{v}$} \mid \pi, C(u_1),C(u_2),\ldots} = 1 - \frac{\card{B(v)}}{\Delta+1}. 
	\]
	
	By solving the recursive equation above, we will get that
	\[
	\expect{X_v\mid \pi} =\frac{\expect{\card{C_c}\mid \pi}}{\Pr\paren{\text{$c$ does not appear in $\Nback{v}$} \mid \pi, C(u_1),C(u_2),\ldots}}. \Qed{clm:warmup-Xv-step}
	\]
	
\end{proof}

Using~\Cref{clm:warmup-Xv-step}, we can conclude the proof of~\Cref{lem:warmup-Xv-bound}. 
\begin{proof}[Proof of~\Cref{lem:warmup-Xv-bound}]
	Each color $c' \in [\Delta+1]$ is chosen with probability $1/(\Delta+1)$ in Line~\eqref{line:reset} of the algorithm. Thus, 
	\[
	\expect{\card{C_c} \mid \pi} = \sum_{c'\in [\Delta +1]}{\Pr\paren{c=c' \mid \pi}\cdot \card{C_{c'}} } \leq \frac{n}{\Delta +1}, 
	\]
	as the sets $\set{C_{c'} \mid c' \in [\Delta+1]}$ are disjoint and partition the already-colored vertices which are at most $n$ vertices. 
	At this point, at most $\degback{v}$ colors have been used in the neighborhood of $v$ and thus cannot be used to color $v$. As such,
	\[
	\Pr\paren{\text{$c$ does not appear in $\Nback{v}$} \mid \pi, C(u_1),C(u_2),\ldots} \geq 1-\frac{\degback{v}}{\Delta +1}.
	\]
	Using~\Cref{clm:warmup-Xv-step} and the above bounds, we conclude
	\begin{align*}
		\expect{X_v \mid \pi } &= \frac{\expect{\card{C_c} \mid \pi}}{\Pr\paren{\text{$c$ does not appear in $\Nback{v}$} \mid \pi, C(u_1),C(u_2),\ldots}} \\
		&\leq \frac{n}{\Delta +1 -\degback{v}}.\Qed{lem:warmup-Xv-bound}
	\end{align*}
	
\end{proof}

By~\Cref{lem:warmup-Xv-bound} (and~\Cref{obs:warmup-Xv}), for any choice of the permutation $\pi$ in~\Cref{alg:warmup},  
\begin{align}
	\expect{\text{runtime of~\Cref{alg:warmup}} \mid \pi} = O(1) \cdot \sum_{v \in V} \frac{n}{\Delta+1-\degback{v}}.  \label{eq:warmup-suffices}
\end{align}

We now consider the randomness of  $\pi$ to bound the RHS above in expectation over $\pi$. 
\begin{lemma}\label{lem:warmup-RHS-suffices}
	We have, 
	\[
	\Exp_{\pi}\bracket{\sum_{v \in V} \frac{n}{\Delta+1-\degback{v}}} = O(\frac{n^2}{\Delta} \cdot \log{\Delta}).
	\] 
\end{lemma}
\begin{proof}
	Using the linearity of expectation, we have
	\[
	\Exp_{\pi}\bracket{\sum_{v \in V} \frac{n}{\Delta+1-\degback{v}}} =n \cdot \sum_{v\in V} \Exp_{\pi}\bracket{\frac{1}{\Delta+1-\degback{v}}}.
	\]
	
	For each permutation $\pi$, $\degback{v}$ depends on $v$'s relative position in the permutation $\pi$ with respect to its neighbors and it can vary from $0$ to $\deg(v)$. Each of these positions happens with the same probability ${1}/({\deg(v)+1})$. 
	Hence, for every vertex $v\in V$,
	\begin{align*}
		\Exp_{\pi}\bracket{\frac{1}{\Delta+1-\degback{v}}} &= \sum_{d=0}^{\deg(v)}\frac{1}{\deg(v)+1}\cdot \frac{1}{\Delta +1-d} \leq \sum_{d=0}^{\Delta}\frac{1}{\Delta+1}\cdot \frac{1}{\Delta +1-d} = O(\frac{\log{\Delta}}{\Delta}),
	\end{align*}
	where the inequality holds because if we let $A := \set{n/(\Delta + 1 - d)}_{d=0}^{\Delta}$, then, in the LHS, we are taking the average of the smallest $\deg(v) + 1$ numbers in $A$, whereas in the RHS we are taking the average of all of $A$.
	Plugging in this bound in the equation above concludes the proof. 
\end{proof}

\Cref{lem:warmup-runtime} now follows immediately from \Cref{eq:warmup-suffices} and \Cref{lem:warmup-RHS-suffices}. We can now use this to wrap up the proof of~\Cref{thm:warm-up}. 

\begin{proof}[Proof of~\Cref{thm:warm-up}]
	
	First, we consider the case $\Delta \geq \sqrt{n\log n}$.  In this case, we use \Cref{alg:warmup}, which relies on the adjacency matrix to access the input graph.
	By~\Cref{lem:warmup-runtime} we know that this algorithm has expected runtime $O(\frac{n^2}{\Delta} \cdot \log{\Delta})$ which is $O(n\sqrt{n\cdot \log(n)})$ by the lower bound on $\Delta$.
	
	
	If $\Delta < \sqrt{n\log n}$ we can use the standard deterministic greedy algorithm for vertex coloring which has linear runtime of $O(n\Delta)$. This algorithm uses an adjacency list to access the graph. 
	This is again $O(n\sqrt{n\log{n}})$ by the upper bound on $\Delta$ in this case, concluding the proof.
\end{proof}



\end{document}
