\documentclass[11pt]{article}

\usepackage{amsthm}
\usepackage{graphicx} % support the \includegraphics command and options
\usepackage{array} % for better arrays (eg matrices) in maths

\usepackage{amsmath, amssymb, amsfonts, verbatim}
\usepackage{hyphenat,epsfig,subcaption,multirow}
\usepackage[font=small,labelfont=bf]{caption}
\usepackage{nicefrac}
\usepackage{paralist}


\usepackage{tocloft}
\renewcommand{\cftsecfont}{\normalfont }  


\usepackage[usenames,dvipsnames]{xcolor}
\usepackage[ruled]{algorithm2e}
\renewcommand{\algorithmcfname}{Algorithm}
%\SetVlineSkip
%\renewcommand{\SetVlineSkip}

\DeclareFontFamily{U}{mathx}{\hyphenchar\font45}
\DeclareFontShape{U}{mathx}{m}{n}{
      <5> <6> <7> <8> <9> <10>
      <10.95> <12> <14.4> <17.28> <20.74> <24.88>
      mathx10
      }{}
\DeclareSymbolFont{mathx}{U}{mathx}{m}{n}
\DeclareMathSymbol{\bigtimes}{1}{mathx}{"91}

\usepackage{tcolorbox}
\tcbuselibrary{skins,breakable}
\tcbset{enhanced jigsaw}

\usepackage[normalem]{ulem}
\usepackage[compact]{titlesec}

\definecolor{DarkRed}{rgb}{0.5,0.1,0.1}
\definecolor{DarkBlue}{rgb}{0.1,0.1,0.5}


\usepackage{nameref}
\definecolor{ForestGreen}{rgb}{0.1333,0.5451,0.1333}
%\definecolor{DarkRed}{rgb}{0.8,0,0}
\definecolor{Red}{rgb}{0.9,0,0}
\usepackage[linktocpage=true,
	pagebackref=true,colorlinks,
	linkcolor=DarkRed,citecolor=ForestGreen,
	bookmarks,bookmarksopen,bookmarksnumbered]
	{hyperref}
\usepackage[noabbrev,nameinlink]{cleveref}
\crefname{property}{property}{Property}
\creflabelformat{property}{(#1)#2#3}
\crefname{equation}{eq}{Eq}
\creflabelformat{equation}{(#1)#2#3}
%\crefrangeformat{equation}{Eqs.(#1)#4-#5(#2)#6}

\usepackage{bm}
\usepackage{url}
\usepackage{xspace}
\usepackage[mathscr]{euscript}

\usepackage{tikz}
\usetikzlibrary{arrows}
\usetikzlibrary{arrows.meta}
\usetikzlibrary{shapes}
\usetikzlibrary{backgrounds}
\usetikzlibrary{positioning}
\usetikzlibrary{decorations.markings}
\usetikzlibrary{decorations.pathreplacing} % needed for drawing braces
\usetikzlibrary{patterns}
\usetikzlibrary{calc}
\usetikzlibrary{fit}
\usetikzlibrary{decorations}

\usepackage[framemethod=TikZ]{mdframed}

\usepackage[noend]{algpseudocode}
\makeatletter
\def\BState{\State\hskip-\ALG@thistlm}
\makeatother

\usepackage{cite}
\usepackage{enumitem}
\setlist[itemize]{leftmargin=20pt}
\setlist[enumerate]{leftmargin=20pt}


\usepackage[margin=1in]{geometry}


\usepackage{thmtools}
\usepackage{thm-restate}

\newtheorem{theorem}{Theorem}
\newtheorem{lemma}{Lemma}[section]
\newtheorem{proposition}[lemma]{Proposition}
\newtheorem{corollary}[lemma]{Corollary}
\newtheorem{claim}[lemma]{Claim}
\newtheorem{fact}[lemma]{Fact}
\newtheorem{assumption}[lemma]{Assumption}
\newtheorem{invariant}[lemma]{Invariant}

\newtheorem{conj}[lemma]{Conjecture}
\newtheorem{definition}[lemma]{Definition}
\newtheorem{problem}{Problem}

\newtheorem*{claim*}{Claim}
\newtheorem*{assumption*}{Assumption}
\newtheorem*{proposition*}{Proposition}
\newtheorem*{lemma*}{Lemma}

\newtheorem{observation}[lemma]{Observation}
\newtheorem{property}{Property}


\newtheorem*{theorem*}{Theorem}

\crefname{lemma}{Lemma}{Lemmas}
\crefname{claim}{claim}{claims}
\crefname{property}{Property}{Properties}
\crefname{invariant}{Invariant}{Invariants}

\newtheorem{mdresult}{Result}
\newenvironment{result}{\begin{mdframed}[backgroundcolor=lightgray!40,topline=false,rightline=false,leftline=false,bottomline=false,innertopmargin=5pt]}{\end{mdframed}}

%\newtheorem{result}[mdresult]{Result}


%\theoremstyle{definition}
\newtheorem{remark}[lemma]{Remark}

\theoremstyle{definition}

\newtheorem{mdproblem}{Problem}
\newenvironment{Problem}{\begin{mdframed}[backgroundcolor=ForestGreen!15,topline=false,bottomline=false, innerbottommargin=12pt,innertopmargin=12pt]\begin{problem}}{\end{problem}\end{mdframed}}
\newtheorem*{mdproblem*}{Problem}
\newenvironment{Problem*}{\begin{mdframed}[hidealllines=false,innerleftmargin=10pt,backgroundcolor=gray!10,innertopmargin=5pt,innerbottommargin=5pt,roundcorner=10pt]\begin{mdproblem*}}{\end{mdproblem*}\end{mdframed}}
\newtheorem{mddefinition}[lemma]{Definition}
\newenvironment{Definition}{\begin{mdframed}[hidealllines=false,innerleftmargin=10pt,backgroundcolor=white!10,innertopmargin=5pt,innerbottommargin=5pt,roundcorner=10pt]\begin{mddefinition}}{\end{mddefinition}\end{mdframed}}
\newtheorem*{mddefinition*}{Definition}
\newenvironment{Definition*}{\begin{mdframed}[hidealllines=false,innerleftmargin=10pt,backgroundcolor=white!10,innertopmargin=5pt,innerbottommargin=5pt,roundcorner=10pt]\begin{mddefinition*}}{\end{mddefinition*}\end{mdframed}}
\newtheorem{mdremark}{Remark}
\newenvironment{Remark}{\begin{mdframed}[hidealllines=false,innerleftmargin=10pt,backgroundcolor=gray!10,innertopmargin=5pt,innerbottommargin=5pt,roundcorner=10pt]\begin{mdremark}}{\end{mdremark}\end{mdframed}}

\newenvironment{ourbox}{\begin{mdframed}[hidealllines=false,innerleftmargin=10pt,backgroundcolor=white!10,innertopmargin=2pt,innerbottommargin=5pt,roundcorner=10pt]}{\end{mdframed}}


\newtheorem{mdalgorithm}{Algorithm}
\newenvironment{Algorithm}{\begin{ourbox}\begin{mdalgorithm}}{\end{mdalgorithm}\end{ourbox}}

\allowdisplaybreaks



\DeclareMathOperator*{\argmax}{arg\,max}

\renewcommand{\qed}{\nobreak \ifvmode \relax \else
      \ifdim\lastskip<1.5em \hskip-\lastskip
      \hskip1.5em plus0em minus0.5em \fi \nobreak
      \vrule height0.75em width0.5em depth0.25em\fi}

\newcommand{\Qed}[1]{\rlap{\qed$_{\textnormal{~~\Cref{#1}}}$}}

\setlength{\parskip}{3pt}

\newcommand{\logstar}[1]{\ensuremath{\log^{*}\!{#1}}}
\newcommand{\logr}[1]{\ensuremath{\log^{(#1)}}}
\renewcommand{\leq}{\leqslant}
\renewcommand{\geq}{\geqslant}
\renewcommand{\le}{\leq}
\renewcommand{\ge}{\geq}



%
\setlength\unitlength{1mm}
\newcommand{\twodots}{\mathinner {\ldotp \ldotp}}
% bb font symbols
\newcommand{\Rho}{\mathrm{P}}
\newcommand{\Tau}{\mathrm{T}}

\newfont{\bbb}{msbm10 scaled 700}
\newcommand{\CCC}{\mbox{\bbb C}}

\newfont{\bb}{msbm10 scaled 1100}
\newcommand{\CC}{\mbox{\bb C}}
\newcommand{\PP}{\mbox{\bb P}}
\newcommand{\RR}{\mbox{\bb R}}
\newcommand{\QQ}{\mbox{\bb Q}}
\newcommand{\ZZ}{\mbox{\bb Z}}
\newcommand{\FF}{\mbox{\bb F}}
\newcommand{\GG}{\mbox{\bb G}}
\newcommand{\EE}{\mbox{\bb E}}
\newcommand{\NN}{\mbox{\bb N}}
\newcommand{\KK}{\mbox{\bb K}}
\newcommand{\HH}{\mbox{\bb H}}
\newcommand{\SSS}{\mbox{\bb S}}
\newcommand{\UU}{\mbox{\bb U}}
\newcommand{\VV}{\mbox{\bb V}}


\newcommand{\yy}{\mathbbm{y}}
\newcommand{\xx}{\mathbbm{x}}
\newcommand{\zz}{\mathbbm{z}}
\newcommand{\sss}{\mathbbm{s}}
\newcommand{\rr}{\mathbbm{r}}
\newcommand{\pp}{\mathbbm{p}}
\newcommand{\qq}{\mathbbm{q}}
\newcommand{\ww}{\mathbbm{w}}
\newcommand{\hh}{\mathbbm{h}}
\newcommand{\vvv}{\mathbbm{v}}

% Vectors

\newcommand{\av}{{\bf a}}
\newcommand{\bv}{{\bf b}}
\newcommand{\cv}{{\bf c}}
\newcommand{\dv}{{\bf d}}
\newcommand{\ev}{{\bf e}}
\newcommand{\fv}{{\bf f}}
\newcommand{\gv}{{\bf g}}
\newcommand{\hv}{{\bf h}}
\newcommand{\iv}{{\bf i}}
\newcommand{\jv}{{\bf j}}
\newcommand{\kv}{{\bf k}}
\newcommand{\lv}{{\bf l}}
\newcommand{\mv}{{\bf m}}
\newcommand{\nv}{{\bf n}}
\newcommand{\ov}{{\bf o}}
\newcommand{\pv}{{\bf p}}
\newcommand{\qv}{{\bf q}}
\newcommand{\rv}{{\bf r}}
\newcommand{\sv}{{\bf s}}
\newcommand{\tv}{{\bf t}}
\newcommand{\uv}{{\bf u}}
\newcommand{\wv}{{\bf w}}
\newcommand{\vv}{{\bf v}}
\newcommand{\xv}{{\bf x}}
\newcommand{\yv}{{\bf y}}
\newcommand{\zv}{{\bf z}}
\newcommand{\zerov}{{\bf 0}}
\newcommand{\onev}{{\bf 1}}

% Matrices

\newcommand{\Am}{{\bf A}}
\newcommand{\Bm}{{\bf B}}
\newcommand{\Cm}{{\bf C}}
\newcommand{\Dm}{{\bf D}}
\newcommand{\Em}{{\bf E}}
\newcommand{\Fm}{{\bf F}}
\newcommand{\Gm}{{\bf G}}
\newcommand{\Hm}{{\bf H}}
\newcommand{\Id}{{\bf I}}
\newcommand{\Jm}{{\bf J}}
\newcommand{\Km}{{\bf K}}
\newcommand{\Lm}{{\bf L}}
\newcommand{\Mm}{{\bf M}}
\newcommand{\Nm}{{\bf N}}
\newcommand{\Om}{{\bf O}}
\newcommand{\Pm}{{\bf P}}
\newcommand{\Qm}{{\bf Q}}
\newcommand{\Rm}{{\bf R}}
\newcommand{\Sm}{{\bf S}}
\newcommand{\Tm}{{\bf T}}
\newcommand{\Um}{{\bf U}}
\newcommand{\Wm}{{\bf W}}
\newcommand{\Vm}{{\bf V}}
\newcommand{\Xm}{{\bf X}}
\newcommand{\Ym}{{\bf Y}}
\newcommand{\Zm}{{\bf Z}}

% Calligraphic

\newcommand{\Ac}{{\cal A}}
\newcommand{\Bc}{{\cal B}}
\newcommand{\Cc}{{\cal C}}
\newcommand{\Dc}{{\cal D}}
\newcommand{\Ec}{{\cal E}}
\newcommand{\Fc}{{\cal F}}
\newcommand{\Gc}{{\cal G}}
\newcommand{\Hc}{{\cal H}}
\newcommand{\Ic}{{\cal I}}
\newcommand{\Jc}{{\cal J}}
\newcommand{\Kc}{{\cal K}}
\newcommand{\Lc}{{\cal L}}
\newcommand{\Mc}{{\cal M}}
\newcommand{\Nc}{{\cal N}}
\newcommand{\nc}{{\cal n}}
\newcommand{\Oc}{{\cal O}}
\newcommand{\Pc}{{\cal P}}
\newcommand{\Qc}{{\cal Q}}
\newcommand{\Rc}{{\cal R}}
\newcommand{\Sc}{{\cal S}}
\newcommand{\Tc}{{\cal T}}
\newcommand{\Uc}{{\cal U}}
\newcommand{\Wc}{{\cal W}}
\newcommand{\Vc}{{\cal V}}
\newcommand{\Xc}{{\cal X}}
\newcommand{\Yc}{{\cal Y}}
\newcommand{\Zc}{{\cal Z}}

% Bold greek letters

\newcommand{\alphav}{\hbox{\boldmath$\alpha$}}
\newcommand{\betav}{\hbox{\boldmath$\beta$}}
\newcommand{\gammav}{\hbox{\boldmath$\gamma$}}
\newcommand{\deltav}{\hbox{\boldmath$\delta$}}
\newcommand{\etav}{\hbox{\boldmath$\eta$}}
\newcommand{\lambdav}{\hbox{\boldmath$\lambda$}}
\newcommand{\epsilonv}{\hbox{\boldmath$\epsilon$}}
\newcommand{\nuv}{\hbox{\boldmath$\nu$}}
\newcommand{\muv}{\hbox{\boldmath$\mu$}}
\newcommand{\zetav}{\hbox{\boldmath$\zeta$}}
\newcommand{\phiv}{\hbox{\boldmath$\phi$}}
\newcommand{\psiv}{\hbox{\boldmath$\psi$}}
\newcommand{\thetav}{\hbox{\boldmath$\theta$}}
\newcommand{\tauv}{\hbox{\boldmath$\tau$}}
\newcommand{\omegav}{\hbox{\boldmath$\omega$}}
\newcommand{\xiv}{\hbox{\boldmath$\xi$}}
\newcommand{\sigmav}{\hbox{\boldmath$\sigma$}}
\newcommand{\piv}{\hbox{\boldmath$\pi$}}
\newcommand{\rhov}{\hbox{\boldmath$\rho$}}
\newcommand{\upsilonv}{\hbox{\boldmath$\upsilon$}}

\newcommand{\Gammam}{\hbox{\boldmath$\Gamma$}}
\newcommand{\Lambdam}{\hbox{\boldmath$\Lambda$}}
\newcommand{\Deltam}{\hbox{\boldmath$\Delta$}}
\newcommand{\Sigmam}{\hbox{\boldmath$\Sigma$}}
\newcommand{\Phim}{\hbox{\boldmath$\Phi$}}
\newcommand{\Pim}{\hbox{\boldmath$\Pi$}}
\newcommand{\Psim}{\hbox{\boldmath$\Psi$}}
\newcommand{\Thetam}{\hbox{\boldmath$\Theta$}}
\newcommand{\Omegam}{\hbox{\boldmath$\Omega$}}
\newcommand{\Xim}{\hbox{\boldmath$\Xi$}}


% Sans Serif small case

\newcommand{\Gsf}{{\sf G}}

\newcommand{\asf}{{\sf a}}
\newcommand{\bsf}{{\sf b}}
\newcommand{\csf}{{\sf c}}
\newcommand{\dsf}{{\sf d}}
\newcommand{\esf}{{\sf e}}
\newcommand{\fsf}{{\sf f}}
\newcommand{\gsf}{{\sf g}}
\newcommand{\hsf}{{\sf h}}
\newcommand{\isf}{{\sf i}}
\newcommand{\jsf}{{\sf j}}
\newcommand{\ksf}{{\sf k}}
\newcommand{\lsf}{{\sf l}}
\newcommand{\msf}{{\sf m}}
\newcommand{\nsf}{{\sf n}}
\newcommand{\osf}{{\sf o}}
\newcommand{\psf}{{\sf p}}
\newcommand{\qsf}{{\sf q}}
\newcommand{\rsf}{{\sf r}}
\newcommand{\ssf}{{\sf s}}
\newcommand{\tsf}{{\sf t}}
\newcommand{\usf}{{\sf u}}
\newcommand{\wsf}{{\sf w}}
\newcommand{\vsf}{{\sf v}}
\newcommand{\xsf}{{\sf x}}
\newcommand{\ysf}{{\sf y}}
\newcommand{\zsf}{{\sf z}}


% mixed symbols

\newcommand{\sinc}{{\hbox{sinc}}}
\newcommand{\diag}{{\hbox{diag}}}
\renewcommand{\det}{{\hbox{det}}}
\newcommand{\trace}{{\hbox{tr}}}
\newcommand{\sign}{{\hbox{sign}}}
\renewcommand{\arg}{{\hbox{arg}}}
\newcommand{\var}{{\hbox{var}}}
\newcommand{\cov}{{\hbox{cov}}}
\newcommand{\Ei}{{\rm E}_{\rm i}}
\renewcommand{\Re}{{\rm Re}}
\renewcommand{\Im}{{\rm Im}}
\newcommand{\eqdef}{\stackrel{\Delta}{=}}
\newcommand{\defines}{{\,\,\stackrel{\scriptscriptstyle \bigtriangleup}{=}\,\,}}
\newcommand{\<}{\left\langle}
\renewcommand{\>}{\right\rangle}
\newcommand{\herm}{{\sf H}}
\newcommand{\trasp}{{\sf T}}
\newcommand{\transp}{{\sf T}}
\renewcommand{\vec}{{\rm vec}}
\newcommand{\Psf}{{\sf P}}
\newcommand{\SINR}{{\sf SINR}}
\newcommand{\SNR}{{\sf SNR}}
\newcommand{\MMSE}{{\sf MMSE}}
\newcommand{\REF}{{\RED [REF]}}

% Markov chain
\usepackage{stmaryrd} % for \mkv 
\newcommand{\mkv}{-\!\!\!\!\minuso\!\!\!\!-}

% Colors

\newcommand{\RED}{\color[rgb]{1.00,0.10,0.10}}
\newcommand{\BLUE}{\color[rgb]{0,0,0.90}}
\newcommand{\GREEN}{\color[rgb]{0,0.80,0.20}}

%%%%%%%%%%%%%%%%%%%%%%%%%%%%%%%%%%%%%%%%%%
\usepackage{hyperref}
\hypersetup{
    bookmarks=true,         % show bookmarks bar?
    unicode=false,          % non-Latin characters in AcrobatÕs bookmarks
    pdftoolbar=true,        % show AcrobatÕs toolbar?
    pdfmenubar=true,        % show AcrobatÕs menu?
    pdffitwindow=false,     % window fit to page when opened
    pdfstartview={FitH},    % fits the width of the page to the window
%    pdftitle={My title},    % title
%    pdfauthor={Author},     % author
%    pdfsubject={Subject},   % subject of the document
%    pdfcreator={Creator},   % creator of the document
%    pdfproducer={Producer}, % producer of the document
%    pdfkeywords={keyword1} {key2} {key3}, % list of keywords
    pdfnewwindow=true,      % links in new window
    colorlinks=true,       % false: boxed links; true: colored links
    linkcolor=red,          % color of internal links (change box color with linkbordercolor)
    citecolor=green,        % color of links to bibliography
    filecolor=blue,      % color of file links
    urlcolor=blue           % color of external links
}
%%%%%%%%%%%%%%%%%%%%%%%%%%%%%%%%%%%%%%%%%%%



\title{Simple Sublinear Algorithms for $(\Delta+1)$ Vertex Coloring \\ via Asymmetric Palette Sparsification\footnote{A preliminary version of this work appeared in \emph{Symposium on Simplicity in Algorithms (SOSA 2025)}. \smallskip}}
\author{Sepehr Assadi\footnote{(sepehr@assadi.info) Cheriton School of Computer Science, University of Waterloo. 
Supported in part by a  Sloan Research Fellowship, an NSERC
Discovery Grant, a University of Waterloo startup grant, and a Faculty of Math Research Chair grant.} \and Helia Yazdanyar\footnote{(hyazdanyar@uwaterloo.ca) Cheriton School of Computer Science, University of Waterloo.}}

\date{}

\begin{document}
\maketitle


\begin{abstract}  
Test time scaling is currently one of the most active research areas that shows promise after training time scaling has reached its limits.
Deep-thinking (DT) models are a class of recurrent models that can perform easy-to-hard generalization by assigning more compute to harder test samples.
However, due to their inability to determine the complexity of a test sample, DT models have to use a large amount of computation for both easy and hard test samples.
Excessive test time computation is wasteful and can cause the ``overthinking'' problem where more test time computation leads to worse results.
In this paper, we introduce a test time training method for determining the optimal amount of computation needed for each sample during test time.
We also propose Conv-LiGRU, a novel recurrent architecture for efficient and robust visual reasoning. 
Extensive experiments demonstrate that Conv-LiGRU is more stable than DT, effectively mitigates the ``overthinking'' phenomenon, and achieves superior accuracy.
\end{abstract}  

\clearpage

\section{Introduction}


\begin{figure}[t]
\centering
\includegraphics[width=0.6\columnwidth]{figures/evaluation_desiderata_V5.pdf}
\vspace{-0.5cm}
\caption{\systemName is a platform for conducting realistic evaluations of code LLMs, collecting human preferences of coding models with real users, real tasks, and in realistic environments, aimed at addressing the limitations of existing evaluations.
}
\label{fig:motivation}
\end{figure}

\begin{figure*}[t]
\centering
\includegraphics[width=\textwidth]{figures/system_design_v2.png}
\caption{We introduce \systemName, a VSCode extension to collect human preferences of code directly in a developer's IDE. \systemName enables developers to use code completions from various models. The system comprises a) the interface in the user's IDE which presents paired completions to users (left), b) a sampling strategy that picks model pairs to reduce latency (right, top), and c) a prompting scheme that allows diverse LLMs to perform code completions with high fidelity.
Users can select between the top completion (green box) using \texttt{tab} or the bottom completion (blue box) using \texttt{shift+tab}.}
\label{fig:overview}
\end{figure*}

As model capabilities improve, large language models (LLMs) are increasingly integrated into user environments and workflows.
For example, software developers code with AI in integrated developer environments (IDEs)~\citep{peng2023impact}, doctors rely on notes generated through ambient listening~\citep{oberst2024science}, and lawyers consider case evidence identified by electronic discovery systems~\citep{yang2024beyond}.
Increasing deployment of models in productivity tools demands evaluation that more closely reflects real-world circumstances~\citep{hutchinson2022evaluation, saxon2024benchmarks, kapoor2024ai}.
While newer benchmarks and live platforms incorporate human feedback to capture real-world usage, they almost exclusively focus on evaluating LLMs in chat conversations~\citep{zheng2023judging,dubois2023alpacafarm,chiang2024chatbot, kirk2024the}.
Model evaluation must move beyond chat-based interactions and into specialized user environments.



 

In this work, we focus on evaluating LLM-based coding assistants. 
Despite the popularity of these tools---millions of developers use Github Copilot~\citep{Copilot}---existing
evaluations of the coding capabilities of new models exhibit multiple limitations (Figure~\ref{fig:motivation}, bottom).
Traditional ML benchmarks evaluate LLM capabilities by measuring how well a model can complete static, interview-style coding tasks~\citep{chen2021evaluating,austin2021program,jain2024livecodebench, white2024livebench} and lack \emph{real users}. 
User studies recruit real users to evaluate the effectiveness of LLMs as coding assistants, but are often limited to simple programming tasks as opposed to \emph{real tasks}~\citep{vaithilingam2022expectation,ross2023programmer, mozannar2024realhumaneval}.
Recent efforts to collect human feedback such as Chatbot Arena~\citep{chiang2024chatbot} are still removed from a \emph{realistic environment}, resulting in users and data that deviate from typical software development processes.
We introduce \systemName to address these limitations (Figure~\ref{fig:motivation}, top), and we describe our three main contributions below.


\textbf{We deploy \systemName in-the-wild to collect human preferences on code.} 
\systemName is a Visual Studio Code extension, collecting preferences directly in a developer's IDE within their actual workflow (Figure~\ref{fig:overview}).
\systemName provides developers with code completions, akin to the type of support provided by Github Copilot~\citep{Copilot}. 
Over the past 3 months, \systemName has served over~\completions suggestions from 10 state-of-the-art LLMs, 
gathering \sampleCount~votes from \userCount~users.
To collect user preferences,
\systemName presents a novel interface that shows users paired code completions from two different LLMs, which are determined based on a sampling strategy that aims to 
mitigate latency while preserving coverage across model comparisons.
Additionally, we devise a prompting scheme that allows a diverse set of models to perform code completions with high fidelity.
See Section~\ref{sec:system} and Section~\ref{sec:deployment} for details about system design and deployment respectively.



\textbf{We construct a leaderboard of user preferences and find notable differences from existing static benchmarks and human preference leaderboards.}
In general, we observe that smaller models seem to overperform in static benchmarks compared to our leaderboard, while performance among larger models is mixed (Section~\ref{sec:leaderboard_calculation}).
We attribute these differences to the fact that \systemName is exposed to users and tasks that differ drastically from code evaluations in the past. 
Our data spans 103 programming languages and 24 natural languages as well as a variety of real-world applications and code structures, while static benchmarks tend to focus on a specific programming and natural language and task (e.g. coding competition problems).
Additionally, while all of \systemName interactions contain code contexts and the majority involve infilling tasks, a much smaller fraction of Chatbot Arena's coding tasks contain code context, with infilling tasks appearing even more rarely. 
We analyze our data in depth in Section~\ref{subsec:comparison}.



\textbf{We derive new insights into user preferences of code by analyzing \systemName's diverse and distinct data distribution.}
We compare user preferences across different stratifications of input data (e.g., common versus rare languages) and observe which affect observed preferences most (Section~\ref{sec:analysis}).
For example, while user preferences stay relatively consistent across various programming languages, they differ drastically between different task categories (e.g. frontend/backend versus algorithm design).
We also observe variations in user preference due to different features related to code structure 
(e.g., context length and completion patterns).
We open-source \systemName and release a curated subset of code contexts.
Altogether, our results highlight the necessity of model evaluation in realistic and domain-specific settings.






\section{Preliminaries}
\label{sec:prelim}
\label{sec:term}
We define the key terminologies used, primarily focusing on the hidden states (or activations) during the forward pass. 

\paragraph{Components in an attention layer.} We denote $\Res$ as the residual stream. We denote $\Val$ as Value (states), $\Qry$ as Query (states), and $\Key$ as Key (states) in one attention head. The \attlogit~represents the value before the softmax operation and can be understood as the inner product between  $\Qry$  and  $\Key$. We use \Attn~to denote the attention weights of applying the SoftMax function to \attlogit, and ``attention map'' to describe the visualization of the heat map of the attention weights. When referring to the \attlogit~from ``$\tokenB$'' to  ``$\tokenA$'', we indicate the inner product  $\langle\Qry(\tokenB), \Key(\tokenA)\rangle$, specifically the entry in the ``$\tokenB$'' row and ``$\tokenA$'' column of the attention map.

\paragraph{Logit lens.} We use the method of ``Logit Lens'' to interpret the hidden states and value states \citep{belrose2023eliciting}. We use \logit~to denote pre-SoftMax values of the next-token prediction for LLMs. Denote \readout~as the linear operator after the last layer of transformers that maps the hidden states to the \logit. 
The logit lens is defined as applying the readout matrix to residual or value states in middle layers. Through the logit lens, the transformed hidden states can be interpreted as their direct effect on the logits for next-token prediction. 

\paragraph{Terminologies in two-hop reasoning.} We refer to an input like “\Src$\to$\brga, \brgb$\to$\Ed” as a two-hop reasoning chain, or simply a chain. The source entity $\Src$ serves as the starting point or origin of the reasoning. The end entity $\Ed$ represents the endpoint or destination of the reasoning chain. The bridge entity $\Brg$ connects the source and end entities within the reasoning chain. We distinguish between two occurrences of $\Brg$: the bridge in the first premise is called $\brga$, while the bridge in the second premise that connects to $\Ed$ is called $\brgc$. Additionally, for any premise ``$\tokenA \to \tokenB$'', we define $\tokenA$ as the parent node and $\tokenB$ as the child node. Furthermore, if at the end of the sequence, the query token is ``$\tokenA$'', we define the chain ``$\tokenA \to \tokenB$, $\tokenB \to \tokenC$'' as the Target Chain, while all other chains present in the context are referred to as distraction chains. Figure~\ref{fig:data_illustration} provides an illustration of the terminologies.

\paragraph{Input format.}
Motivated by two-hop reasoning in real contexts, we consider input in the format $\bos, \text{context information}, \query, \answer$. A transformer model is trained to predict the correct $\answer$ given the query $\query$ and the context information. The context compromises of $K=5$ disjoint two-hop chains, each appearing once and containing two premises. Within the same chain, the relative order of two premises is fixed so that \Src$\to$\brga~always precedes \brgb$\to$\Ed. The orders of chains are randomly generated, and chains may interleave with each other. The labels for the entities are re-shuffled for every sequence, choosing from a vocabulary size $V=30$. Given the $\bos$ token, $K=5$ two-hop chains, \query, and the \answer~tokens, the total context length is $N=23$. Figure~\ref{fig:data_illustration} also illustrates the data format. 

\paragraph{Model structure and training.} We pre-train a three-layer transformer with a single head per layer. Unless otherwise specified, the model is trained using Adam for $10,000$ steps, achieving near-optimal prediction accuracy. Details are relegated to Appendix~\ref{app:sec_add_training_detail}.


% \RZ{Do we use source entity, target entity, and mediator entity? Or do we use original token, bridge token, end token?}





% \paragraph{Basic notations.} We use ... We use $\ve_i$ to denote one-hot vectors of which only the $i$-th entry equals one, and all other entries are zero. The dimension of $\ve_i$ are usually omitted and can be inferred from contexts. We use $\indicator\{\cdot\}$ to denote the indicator function.

% Let $V > 0$ be a fixed positive integer, and let $\vocab = [V] \defeq \{1, 2, \ldots, V\}$ be the vocabulary. A token $v \in \vocab$ is an integer in $[V]$ and the input studied in this paper is a sequence of tokens $s_{1:T} \defeq (s_1, s_2, \ldots, s_T) \in \vocab^T$ of length $T$. For any set $\mathcal{S}$, we use $\Delta(\mathcal{S})$ to denote the set of distributions over $\mathcal{S}$.

% % to a sequence of vectors $z_1, z_2, \ldots, z_T \in \real^{\dout}$ of dimension $\dout$ and length $T$.

% Let $\mU = [\vu_1, \vu_2, \ldots, \vu_V]^\transpose \in \real^{V\times d}$ denote the token embedding matrix, where the $i$-th row $\vu_i \in \real^d$ represents the $d$-dimensional embedding of token $i \in [V]$. Similarly, let $\mP = [\vp_1, \vp_2, \ldots, \vp_T]^\transpose \in \real^{T\times d}$ denote the positional embedding matrix, where the $i$-th row $\vp_i \in \real^d$ represents the $d$-dimensional embedding of position $i \in [T]$. Both $\mU$ and $\mP$ can be fixed or learnable.

% After receiving an input sequence of tokens $s_{1:T}$, a transformer will first process it using embedding matrices $\mU$ and $\mP$ to obtain a sequence of vectors $\mH = [\vh_1, \vh_2, \ldots, \vh_T] \in \real^{d\times T}$, where 
% \[
% \vh_i = \mU^\transpose\ve_{s_i} + \mP^\transpose\ve_{i} = \vu_{s_i} + \vp_i.
% \]

% We make the following definitions of basic operations in a transformer.

% \begin{definition}[Basic operations in transformers] 
% \label{defn:operators}
% Define the softmax function $\softmax(\cdot): \real^d \to \real^d$ over a vector $\vv \in \real^d$ as
% \[\softmax(\vv)_i = \frac{\exp(\vv_i)}{\sum_{j=1}^d \exp(\vv_j)} \]
% and define the softmax function $\softmax(\cdot): \real^{m\times n} \to \real^{m \times n}$ over a matrix $\mV \in \real^{m\times n}$ as a column-wise softmax operator. For a squared matrix $\mM \in \real^{m\times m}$, the causal mask operator $\mask(\cdot): \real^{m\times m} \to \real^{m\times m}$  is defined as $\mask(\mM)_{ij} = \mM_{ij}$ if $i \leq j$ and  $\mask(\mM)_{ij} = -\infty$ otherwise. For a vector $\vv \in \real^n$ where $n$ is the number of hidden neurons in a layer, we use $\layernorm(\cdot): \real^n \to \real^n$ to denote the layer normalization operator where
% \[
% \layernorm(\vv)_i = \frac{\vv_i-\mu}{\sigma}, \mu = \frac{1}{n}\sum_{j=1}^n \vv_j, \sigma = \sqrt{\frac{1}{n}\sum_{j=1}^n (\vv_j-\mu)^2}
% \]
% and use $\layernorm(\cdot): \real^{n\times m} \to \real^{n\times m}$ to denote the column-wise layer normalization on a matrix.
% We also use $\nonlin(\cdot)$ to denote element-wise nonlinearity such as $\relu(\cdot)$.
% \end{definition}

% The main components of a transformer are causal self-attention heads and MLP layers, which are defined as follows.

% \begin{definition}[Attentions and MLPs]
% \label{defn:attn_mlp} 
% A single-head causal self-attention $\attn(\mH;\mQ,\mK,\mV,\mO)$ parameterized by $\mQ,\mK,\mV \in \real^{{\dqkv\times \din}}$ and $\mO \in \real^{\dout\times\dqkv}$ maps an input matrix $\mH \in \real^{\din\times T}$ to
% \begin{align*}
% &\attn(\mH;\mQ,\mK,\mV,\mO) \\
% =&\mO\mV\layernorm(\mH)\softmax(\mask(\layernorm(\mH)^\transpose\mK^\transpose\mQ\layernorm(\mH))).
% \end{align*}
% Furthermore, a multi-head attention with $M$ heads parameterized by $\{(\mQ_m,\mK_m,\mV_m,\mO_m) \}_{m=1}^M$ is defined as 
% \begin{align*}
%     &\Attn(\mH; \{(\mQ_m,\mK_m,\mV_m,\mO_m) \}_{m\in[M]}) \\ =& \sum_{m=1}^M \attn(\mH;\mQ_m,\mK_m,\mV_m,\mO_m) \in \real^{\dout \times T}.
% \end{align*}
% An MLP layer $\mlp(\mH;\mW_1,\mW_2)$ parameterized by $\mW_1 \in \real^{\dhidden\times \din}$ and $\mW_2 \in \real^{\dout \times \dhidden}$ maps an input matrix $\mH = [\vh_1, \ldots, \vh_T] \in \real^{\din \times T}$ to
% \begin{align*}
%     &\mlp(\mH;\mW_1,\mW_2) = [\vy_1, \ldots, \vy_T], \\ \text{where } &\vy_i = \mW_2\nonlin(\mW_1\layernorm(\vh_i)), \forall i \in [T].
% \end{align*}

% \end{definition}

% In this paper, we assume $\din=\dout=d$ for all attention heads and MLPs to facilitate residual stream unless otherwise specified. Given \Cref{defn:operators,defn:attn_mlp}, we are now able to define a multi-layer transformer.

% \begin{definition}[Multi-layer transformers]
% \label{defn:transformer}
%     An $L$-layer transformer $\transformer(\cdot): \vocab^T \to \Delta(\vocab)$ parameterized by $\mP$, $\mU$, $\{(\mQ_m^{(l)},\mK_m^{(l)},\mV_m^{(l)},\mO_m^{(l)})\}_{m\in[M],l\in[L]}$,  $\{(\mW_1^{(l)},\mW_2^{(l)})\}_{l\in[L]}$ and $\Wreadout \in \real^{V \times d}$ receives a sequence of tokens $s_{1:T}$ as input and predict the next token by outputting a distribution over the vocabulary. The input is first mapped to embeddings $\mH = [\vh_1, \vh_2, \ldots, \vh_T] \in \real^{d\times T}$ by embedding matrices $\mP, \mU$ where 
%     \[
%     \vh_i = \mU^\transpose\ve_{s_i} + \mP^\transpose\ve_{i}, \forall i \in [T].
%     \]
%     For each layer $l \in [L]$, the output of layer $l$, $\mH^{(l)} \in \real^{d\times T}$, is obtained by 
%     \begin{align*}
%         &\mH^{(l)} =  \mH^{(l-1/2)} + \mlp(\mH^{(l-1/2)};\mW_1^{(l)},\mW_2^{(l)}), \\
%         & \mH^{(l-1/2)} = \mH^{(l-1)} + \\ & \quad \Attn(\mH^{(l-1)}; \{(\mQ_m^{(l)},\mK_m^{(l)},\mV_m^{(l)},\mO_m^{(l)}) \}_{m\in[M]}), 
%     \end{align*}
%     where the input $\mH^{(l-1)}$ is the output of the previous layer $l-1$ for $l > 1$ and the input of the first layer $\mH^{(0)} = \mH$. Finally, the output of the transformer is obtained by 
%     \begin{align*}
%         \transformer(s_{1:T}) = \softmax(\Wreadout\vh_T^{(L)})
%     \end{align*}
%     which is a $V$-dimensional vector after softmax representing a distribution over $\vocab$, and $\vh_T^{(L)}$ is the $T$-th column of the output of the last layer, $\mH^{(L)}$.
% \end{definition}



% For each token $v \in \vocab$, there is a corresponding $d_t$-dimensional token embedding vector $\embed(v) \in \mathbb{R}^{d_t}$. Assume the maximum length of the sequence studied in this paper does not exceed $T$. For each position $t \in [T]$, there is a corresponding positional embedding  









% !TeX root = main.tex 
%!TEX root = main.tex

\newcommand{\Nforward}[1]{\ensuremath{N^{>}_{\pi}(#1)}}
\newcommand{\degForward}[1]{\ensuremath{\deg^{>}_{\pi}(#1)}}



\section{Asymmetric Palette Sparsification}\label{thm:apst}

We present our APST in a more general form that allows for coloring vertices from arbitrary palettes of size proportional to their individual degrees---instead of the same original palette of $\set{1,2,\ldots,\Delta+1}$ described earlier---, namely, the $(\deg+1)$-list coloring problem. We then obtain the APST for $(\Delta+1)$ coloring as a simple corollary of this result. We remark that a PST version of $(\deg+1)$-list coloring (by sampling $\polylog{(n)}$ colors per vertex in the worst-case) 
has been previously established in~\cite{HalldorssonKNT22}; see also~\cite{AlonA20}. 

%% Generalization deg+1 APST Theorem--------------------
\begin{theorem}[\textbf{Asymmetric Palette Sparsification Theorem (for ($\deg+1$)-list coloring)}]\label{thm:apst-deg+1}
	Let $G=(V,E)$ be any $n$-vertex graph together with lists $S(v)$ with $\deg(v)+1$ arbitrary colors for each vertex $v\in V$.
	Sample a random permutation $\pi: V \rightarrow [n]$ uniformly and define 
	\[
	\ell(v) := \min \left( \deg(v)+1\, , \,  \frac{40\,n\ln{n}}{\pi(v)}\right),
	\]
	for every $v \in V$ as the size of the list of colors to be sampled for vertex $v$. Then,  
	\begin{itemize}
		\item \emph{\textbf{List sizes:}} Deterministically, $\sum_{v \in V} \ell(v) = O(n\log^2\!{(n)})$, and for any fixed vertices $u \neq v \in V$, 
		\[
		\expect{\ell(v)} = O(\log^2\!{(n)}) \quad \text{and} \quad \expect{\ell(u) \cdot \ell(v)} = O(\log^{4}\!{(n)}). 
		\] 
		\item \emph{\textbf{Colorability:}}
		If for every vertex $v \in V$ we sample a list $L(v)$ of $\ell(v)$ colors from $S(v)$ uniformly and independently, then, with high probability (over the randomness of $\ell$ and sampled lists) the greedy 
		algorithm that iterates over vertices $v \in V$ in the increasing order of $\ell(v)$ finds a proper list-coloring of $G$ from the lists 
		$\set{L(v) \mid v \in V}$. 
	\end{itemize}
	
\end{theorem}
	\Cref{thm:apst-deg+1} follows immediately from~\Cref{clm:apst-list-sizes} and~\Cref{clm:apst-colorability} proven below. 
	\begin{lemma}[\textbf{List sizes}]\label{clm:apst-list-sizes}
		Deterministically, $\sum_{v \in V} \ell(v) = O(n\log^2\!{(n)})$, and for all $u \neq v \in V$, 
		\[
		\expect{\ell(v)} = O(\log^2\!{(n)}) \qquad \text{and} \qquad \expect{\ell(u) \cdot \ell(v)} =O(\log^{4}\!{(n)}). 
		\]
	\end{lemma}
	\begin{proof}
		By the definition of $\ell(v)$, 
		\[
		\sum_{v \in V} \ell(v) \leq \sum_{v \in V} \frac{40\,n\ln{n}}{\pi(v)} = \sum_{i=1}^{n} \frac{40\,n\ln{n}}{i} = O(n\log^2{n}),
		\]
		by the sum of the Harmonic series. This proves the deterministic bound on the total size of the lists as well as the first inequality for the expected size of each list, 
		given the distribution of list sizes is symmetric across vertices. For the second inequality, we have
		\[
		\expect{\ell(u) \cdot \ell(v)}  \leq \sum_{i \neq j} \Prob(\pi(v)=i \, \wedge \, \pi(u)=j) \cdot \frac{(40\,n\ln{(n)})^2}{i\cdot j} = \frac{1600\,n^2\ln^2{(n)}}{{{n}\choose{2}}} \cdot \sum_{i =1}^{n} \sum_{j=1}^{i-1} \frac{1}{i \cdot j} = O(\log^4(n)),
		\]
		again, by the sum of the Harmonic series. This concludes the proof. \Qed{clm:apst-list-sizes}
		
	\end{proof}
	
	Establishing the colorability property is the main part of the proof. We first need some notation. 
	Iterating over vertices in the increasing order of $\ell(\cdot)$ is the same as the decreasing order of $\pi(\cdot)$. 
	With this in mind, for any vertex $v \in V$, define $\Nback{v}$ as the neighbors $u$ of $v$ with $\pi(u) < \pi(v)$, namely, the ones that are processed 
	\emph{after} $v$ by the greedy algorithm. Let  $\degback{v}:= \card{\Nback{v}}$. 
	
	An easy observation is that when coloring any vertex $v$ by the greedy algorithm, there are still at least $\degback{v}+1$ colors not used in the neighborhood of $v$. 
	Thus, lower bounding $\degback{v}$ allows us to later prove that $L(v)$ contains an available color for $v$ with high probability.  
	We only need such a bound for the subset of vertices with $\ell(v) < \deg(v)+1$ as captured in the following claim. 
	
	\begin{claim}\label{clm:apst-a(v)}
		For all $v \in V$ with $\pi(v) > 40n\ln(n)/(\deg(v)+1)$, 
		\[
		\Pr\paren{\degback{v} < \frac{\deg(v) \cdot \pi(v)}{2n}} \leq n^{-2.4}.
		\]
	\end{claim}
	\begin{proof}
		Fix any vertex $v$ as above and condition on $\pi(v) = i+1$ for some $i \in \set{0,1,\ldots,n-1}$. 
		For $j \in [\deg(v)]$, define $X_j \in \set{0,1}$ as the 
		indicator random variable which is $1$ iff the $j$-th neighbor of $v$ appears in the first $i$ vertices of $\pi$. 
		For $X$ being the sum of these $X_j$-variables (and implicitly conditioned on $\pi(v) = i+1$ to avoid cluttering the equations), we have, 
		\[
		\expect{\degback{v} \mid \pi(v)=i+1} = \expect{X} = \sum_{j=1}^{\deg(v)} \expect{X_j} = \deg(v) \cdot \frac{i}{n-1},
		\]
		as each neighbor of $v$ appears in the first $i$ position of $\pi$ with probability $i/(n-1)$ conditioned on $\pi(v) = i+1$.  
		Random variable $X$ is distributed as a hypergeometric random variable with parameters $N=n-1$, $K = i$, and $M = \deg(v)$. Thus, by~\Cref{thm:book} for the parameter
		\begin{align}
		t &:=  \expect{X} - \frac{\deg(v)\cdot (i+1)}{2n} = \frac{\deg(v) \cdot i}{n-1} - \frac{\deg(v)\cdot (i+1)}{2n} \geq  (1-o(1)) \cdot \frac{\expect{X}}{2}  \tag{by the value of $\expect{X}$ calculated above and given $i = \omega(1)$ in the claim statement}
		\end{align} 
		we have, 
		\begin{align*}
			\Pr\paren{X \leq \frac{\deg(v)\cdot (i+1)}{2n}} &= \Pr\paren{X \leq \expect{X} - t} \leq \exp\paren{-\frac{t^2}{2\expect{X}}} \leq \exp\paren{-(1-o(1)) \cdot \frac{\expect{X}}{8}} \tag{by the lower bound on $t$ established above} \\
			&= \exp\paren{-(1-o(1)) \cdot \frac{\deg(v) \cdot i}{8 \cdot (n-1)}} \leq \exp\paren{-(1-o(1)) \cdot \frac{40\ln{n}}{8}} \leq n^{-2.4},
		\end{align*}
		where in the second line we used the lower bound on $i = \pi(v)-1$ in the claim statement. 
		Given $\degback{v} = X$ and the bound holds for all choices of $\pi(v)$, we can conclude the proof. \Qed{clm:apst-a(v)}
		
	\end{proof}
	\begin{lemma}[\textbf{Colorability}]\label{clm:apst-colorability}
		With high probability, when coloring each vertex $v \in V$ in the greedy algorithm, 
		at least one of the colors sampled in $L(v)$ has not been used in the neighborhood of $v$; that is, the greedy algorithm can color this vertex. 
	\end{lemma}
	\begin{proof}	
		We condition on the choice of $\pi$ and by union bound obtain that with high probability, the complement of the event in~\Cref{clm:apst-a(v)} holds for all vertices. 
		For any vertex $v \in V$, we say a color in $S(v)$ is \textbf{available} to $v$ iff it is not assigned to any neighbor of $v$ by the time we process $v$ in the greedy algorithm. 
		Let $a(v)$ denote the number of available colors and note that $a(v) \geq  \degback{v}+1$. 
		
		In the following, we fix a vertex $v \in V$ and further condition on the randomness of $L(u)$ for all vertices $u$ with $\pi(u) > \pi(v)$. The conditionings so far fix the set of available colors and the values of $a(v)$ and $\ell(v)$, but $L(v)$ is still a random $\ell(v)$-subset of the $\deg(v)+1$ colors in $S(v)$. If $\pi(v) \leq 40n\ln(n)/(\deg(v)+1)$, we have $\ell(v) = \deg(v)+1$ which means $L(v) = S(v)$ and thus there exists an available color in $L(v)$, proving the claim for this vertex. 
		
		For the remaining vertices with $\pi(v) > 40n\ln(n)/(\deg(v)+1)$, we can apply~\Cref{clm:apst-a(v)} to have, 
		\begin{align*}
			\Pr\paren{\text{no available color of $v$ is sampled in $L(v)$}} &\leq (1-\frac{a(v)}{\deg(v)+1})^{\ell(v)} \leq \exp\paren{-\frac{a(v) \cdot \ell(v)}{\deg(v)+1}} \\ 
			& \leq \exp\paren{-\frac{\deg(v) \cdot \pi(v)}{2n} \cdot \frac{40n\ln{n}}{\pi(v)} \cdot \frac{1}{\deg(v)+1}} \leq n^{-5}.
		\end{align*}
		Thus, with high probability, we can color $v$ from $L(v)$ in the greedy algorithm. Taking a union bound over all vertices concludes the proof. 
		\Qed{clm:apst-colorability}
	\end{proof}
	

We obtain our  APST for $(\Delta+1)$ coloring described earlier as a direct corollary of~\Cref{thm:apst-deg+1}. 

\begin{corollary}[\textbf{Asymmetric Palette Sparsification Theorem (for ($\Delta+1$) coloring)}]\label{thm:apst} ~ \\
	Let $G=(V,E)$ be any $n$-vertex graph with maximum degree $\Delta$. 
	Sample a random permutation $\pi: V \rightarrow [n]$ uniformly and define 
	\[
	\ell(v) := \min \left( \deg(v)+1\, , \,  \frac{40\,n\ln{n}}{\pi(v)}\right),
	\]
	for every $v \in V$ as the size of the lists of colors to be sampled for vertex $v$. Then,  
	\begin{itemize}
		\item \emph{\textbf{List sizes:}} Deterministically, $\sum_{v \in V} \ell(v) = O(n\log^2\!{(n)})$, and for any fixed vertices $u \neq v \in V$, 
		\[
		\expect{\ell(v)} = O(\log^2\!{(n)}) \quad \text{and} \quad \expect{\ell(u) \cdot \ell(v)} = O(\log^{4}\!{(n)}). 
		\] 
		\item \emph{\textbf{Colorability:}} If for every vertex $v \in V$, we sample a list $L(v)$ of $\ell(v)$ colors from $[\Delta+1]$ uniformly and independently, then, with high probability (over the randomness of $\ell$ and sampled lists) the greedy 
		algorithm that iterates over vertices $v \in V$ in the increasing order of $\ell(v)$ finds a proper list-coloring of $G$ from the lists 
		$\set{L(v) \mid v \in V}$. 
	\end{itemize}
%%	The distribution over $\ell: V \rightarrow \IN$ is as follows. 
\end{corollary}
\begin{proof}
\noindent 
	For the \textbf{List Size} property, the distribution of $\set{\ell(v) \mid v \in V}$ is identical to~\Cref{thm:apst-deg+1} and thus we obtain this property immediately. 

	For the \textbf{Colorability} property, assume we first sample $\deg(v)+1$ colors from $[\Delta+1]$ uniformly and put them in a list $S(v)$, and then pick $\ell(v)$ colors from $S(v)$ uniformly. The set $L(v)$ is now
	distributed as a random $\ell(v)$-subset of $[\Delta+1]$ (matching the requirement of~\Cref{thm:apst}) as well as a random $\ell(v)$-subset of $S(v)$ (matching the requirement of~\Cref{thm:apst-deg+1}). 
	We can thus apply~\Cref{thm:apst-deg+1} and obtain that with high probability, vertices of $G$ can be colored from $L(v) \subseteq S(v) \subseteq [\Delta+1]$, namely, a $(\Delta+1)$ vertex coloring of $G$. 
	\Qed{thm:apst}
\end{proof}

	


\subsection{Efficient Computing $\Esim$}

\begin{algorithm}[htbp]
\caption{{\textsc{ConstructApproximateEsim}}$(G^+=(V, E^+), \phi)$\label{alg:main}\\
\textbf{Input}: A graph $G^{+}$ and a parameter $\phi$. \\
\textbf{Output}: A clustering $\mathcal{F} = \{L_i\}_{i=1}^{|\mathcal{L}|}$ of $\Vhigh$ such that $\mathcal{F} = \mathcal{L}$. }
%, where $x_{uv}$ is the positive edge value and $z_e$ is the negative edge value. 
%\\
%\textbf{Auxiliary Information}: $E^-:={\binom{V}{2}}\setminus E^+$;. 
%$c = \exp(\epsilon)$, where $e$ is the Euler's number.
\label{alg:constructesim}
\begin{algorithmic}[1]
\Function {\textsc{ConstructApproximateEsim}}{$G^+=(V, E^+), \phi$}
    \State $\Esim \leftarrow \{ \}$
    \For{$i \in [\log \big(\phi \epsilon^2 \big ), \log n]$}
    \State Sample each node from $V$ with probability $\min(\frac{\log n}{2^i}, 1)$, let the sampled set be $S(i)$
    \EndFor
    \For{$uv \in E^+$ such that $|d(u) - d(v)| \leq 2\phi$}
        \State Let $j \leftarrow \lceil \log \max(d(u), d(v)) + \log \epsilon \rceil$
        \If{$\max(d(u), d(v)) \leq \phi$}
            \State $\Esim \leftarrow \Esim \cup \{ uv \}$
        \ElsIf{$\max(d(u), d(v)) \leq 10\phi$ and $|S(u,  \log  \big(\phi \epsilon^2 \big )) \Delta S(v, \log  \big(\phi \epsilon^2 \big ))| \leq 2 (1 + \epsilon) \phi$}
        \State $\Esim \leftarrow \Esim \cup \{ uv \}$
        \ElsIf{$|S(u, j) \Delta S(v, j)| \leq 4\phi$}
        \State $\Esim \leftarrow \Esim \cup \{ uv \}$
        \EndIf
    \EndFor
    \Return $\Gsim = (V, \Esim)$
\EndFunction
\end{algorithmic}
\end{algorithm}

\begin{lemma}
Suppose that $\OPT \leq \phi$. \Cref{alg:constructesim} outputs a clustering $\mathcal{F}$ such that $\mathcal{F} = \mathcal{L}$.
\end{lemma}

\section*{Acknowledgement} 
\addcontentsline{toc}{section}{Acknowledgement}

We would like to thank Soheil Behnezhad, Yannic Maus, and Ronitt Rubinfeld for helpful discussions and their encouragement toward finding a simpler sublinear algorithms for $(\Delta+1)$ vertex coloring. 
We are also thankful to Chris Trevisan for pointing out a shorter proof of~\Cref{lem:warmup-RHS-suffices} included in this version of the paper. Sepehr Assadi is additionally grateful to Yu Chen and Sanjeev Khanna for their prior collaboration in~\cite{AssadiCK19} that formed the backbone of
this work.

Part of this work was conducted while the first named author was visiting the Simons Institute for the Theory of Computing as part of the Sublinear Algorithms program. 

\bibliographystyle{alpha}
\bibliography{general}

% !TeX root = main.tex 
%!TEX root = main.tex


\clearpage

\appendix
\section{Appendix: A Self-contained Sublinear Time Algorithm}\label{sec:warm-up}

In addition to our main result, we also first present a very simple sublinear time algorithm for $(\Delta+1)$ coloring in the following theorem. This algorithm has recently and independently been discovered in~\cite{ferber2025improved} 
and has also appeared as part of lecture notes for different courses~\cite{Lec24,Lec25,HW} at this point. Finally, this algorithm can also be seen as a simple adjustment to the sublinear time $(\Delta+o(\Delta))$ coloring algorithm of~\cite{MorrisS21}. 


\begin{theorem}\label{thm:warm-up}
	There is a randomized algorithm that given any graph $G$ with maximum degree $\Delta$ via adjacency list and matrix access, 
	outputs a $(\Delta+1)$ coloring of $G$ in  $O(n \cdot \sqrt{n\log{n}})$ expected time. 
\end{theorem}

We note that unlike our sublinear time algorithm in~\Cref{thm:sublinear} which was \emph{non-adaptive}, namely, made all its queries in advance before seeing the answer to them, the current algorithm is adaptive and needs to 
receive the answer to each query before deciding its next query. 

The algorithm in~\Cref{thm:warm-up} is quite similar to the standard greedy algorithm for $(\Delta+1)$ coloring. It iterates over the vertices
and color each one greedily by finding a color not used in the neighbors' of this vertex yet (which exists by pigeonhole principle). However, unlike the greedy algorithm, $(i)$ it crucially needs to iterate
over the vertices in a random order, and, $(ii)$ instead of 
iterating over the neighbors of the current vertex to find an available color, it samples a color randomly for this vertex and then iterates over all vertices with this color to make sure they are not neighbor to the current vertex. 
Formally, the algorithm is as follows. 

\begin{Algorithm}\label{alg:warmup}
	An (adaptive) sublinear time algorithm for $(\Delta+1)$ vertex coloring. 
	
	\begin{enumerate}
		\item Let $C_1,C_2,\ldots,C_{\Delta+1}$ be the \textbf{color classes} to be output at the end, initially set to empty. 
		\item Pick a permutation $\pi$ of vertices in $V$ uniformly at random.
		\item For $v \in V$ in the order of the permutation $\pi$: 
		\begin{enumerate}
			\item\label{line:reset} Sample $c \in [\Delta+1]$ uniformly at random. 
			\item For every vertex $u \in C_c$, check if $(u,v)$ is an edge in $G$; if \emph{Yes}, restart from Line~\eqref{line:reset}. 
			\item If the algorithm reaches this step, color the vertex $v$ with $c$ and add $v$ to $C_c$. 
		\end{enumerate}
	\end{enumerate}
	
\end{Algorithm}

It is easy to see that this algorithm never outputs a wrong coloring (namely, if it ever terminates, its answer is always correct). Any new vertex colored does not create a conflict with previously colored vertices (given the algorithm explicitly checks to not color $v$ with a color $c$ if one of its neighbors is already colored $c$) and thus at the end, there cannot be any monochromatic edge in the graph. The interesting part of the analysis is to show that the algorithm terminates quickly enough, which is captured by the following lemma. 

\begin{lemma}\label{lem:warmup-runtime}
	For any input graph $G=(V,E)$, the expected runtime of~\Cref{alg:warmup} is 
	\[
	O(\frac{n^2}{\Delta} \cdot \log{\Delta}) ~\text{time}. 
	\] 
\end{lemma}

\noindent
To continue, we need to set up some notation.  For any vertex $v \in V$, we define the random variable $X_v$ as the number of $(u,v)$ queries checked by the algorithm in the for-loop of coloring $v$. Additionally, for any $v \in V$ and permutation $\pi$ picked over $V$, define $\Nback{v}$ as the neighbors $u$ of $v$ with $\pi(u) < \pi(v)$, namely, the ones that are colored \emph{before} $v$ by Algorithm~\Cref{alg:warmup}. Let $\degback{v}:= \card{\Nback{v}}$. We start with the basic observation that the runtime of~\Cref{alg:warmup} can be stated in terms of the variables $\set{X_v}_{v \in V}$. 

\begin{observation}\label{obs:warmup-Xv}
	The expected runtime of~\Cref{alg:warmup} is $O(\sum_{v \in V} \expect{X_v})$. 
\end{observation}
\begin{proof}
	By definition, $X_v$ is the number of queries checked in the for-loop of coloring $v$. The algorithm repeats this for-loop for all $v \in V$, so the expected runtime is the number of all queries checked in this algorithm which is proportional to $\expect{\sum_{v\in V} X_v}$. Applying linearity of expectation concludes the proof. 
\end{proof}

Our task is now to bound each of $\expect{X_v}$ for $v \in V$ to bound the runtime of the algorithm using~\Cref{obs:warmup-Xv}. 

\begin{lemma}\label{lem:warmup-Xv-bound}
	For any vertex $v \in V$ and any choice of the permutation $\pi$:  
	\[
	\expect{X_v \mid \pi} \leq \frac{n}{\Delta+1-\degback{v}}.
	\]
\end{lemma}

To prove~\Cref{lem:warmup-Xv-bound}, we first need the following claim. 

\begin{claim}\label{clm:warmup-Xv-step}
	Fix any vertex $v \in V$, any choice of the permutation $\pi$, and any assignment of colors $C(u_1),C(u_2),\ldots$ by~\Cref{alg:warmup} to all vertices that appear before $v$ in $\pi$. 
	Then, 
	\[
	\expect{X_v\mid \pi} = \frac{\expect{\card{C_c}\mid \pi}}{\Pr\paren{\text{$c$ does not appear in $\Nback{v}$} \mid \pi, C(u_1),C(u_2),\ldots}},
	\]
	where in the RHS, both the expectation and the probability are taken with respect to a color $c$ chosen uniformly at random from $[\Delta+1]$. 
\end{claim}
\begin{proof}
	Define the colors $B(v)$ as the set of colors that appear in $\Nback{v}$, that is 
	\[
	B(v) := \set{c \in [\Delta+1] \mid  \text{there exists $u\in\Nback{v}$ with $c(u)=c$}}.
	\]
	
	For every color $c$, if $c$ is in $B(v)$ then $v$ cannot be colored by $c$, and otherwise it can. The probability of picking each color $c$ is $1/(\Delta +1)$. 
	For $c \notin B(v)$, the number of needed queries before coloring $v$ is $\card{C_c}$. 
	For $c \in B(v)$, the algorithm first needs to check up to $\card{C_c}$ queries to know this color is not available to $v$, and then it simply needs to repeat the same exact process.
	As such, 
	\[
	\expect{X_v\mid \pi}  = \sum_{c \notin B(v)} \frac{1}{\Delta+1}\card{C_{c}} + \sum_{c \in B(v)} \frac{1}{\Delta+1}(\card{C_{c}}+ \expect{X_v\mid \pi} ) = \expect{\card{C_c}\mid \pi} + \frac{\card{B(v)}}{\Delta +1} \cdot \expect{X_v\mid \pi}.
	\]
	
	We can also define the probability of $c$ not appearing in $\Nback{v}$ in terms of  $\card{B(v)}$ as below:
	
	\[
	\Pr\paren{\text{$c$ does not appear in $\Nback{v}$} \mid \pi, C(u_1),C(u_2),\ldots} = 1 - \frac{\card{B(v)}}{\Delta+1}. 
	\]
	
	By solving the recursive equation above, we will get that
	\[
	\expect{X_v\mid \pi} =\frac{\expect{\card{C_c}\mid \pi}}{\Pr\paren{\text{$c$ does not appear in $\Nback{v}$} \mid \pi, C(u_1),C(u_2),\ldots}}. \Qed{clm:warmup-Xv-step}
	\]
	
\end{proof}

Using~\Cref{clm:warmup-Xv-step}, we can conclude the proof of~\Cref{lem:warmup-Xv-bound}. 
\begin{proof}[Proof of~\Cref{lem:warmup-Xv-bound}]
	Each color $c' \in [\Delta+1]$ is chosen with probability $1/(\Delta+1)$ in Line~\eqref{line:reset} of the algorithm. Thus, 
	\[
	\expect{\card{C_c} \mid \pi} = \sum_{c'\in [\Delta +1]}{\Pr\paren{c=c' \mid \pi}\cdot \card{C_{c'}} } \leq \frac{n}{\Delta +1}, 
	\]
	as the sets $\set{C_{c'} \mid c' \in [\Delta+1]}$ are disjoint and partition the already-colored vertices which are at most $n$ vertices. 
	At this point, at most $\degback{v}$ colors have been used in the neighborhood of $v$ and thus cannot be used to color $v$. As such,
	\[
	\Pr\paren{\text{$c$ does not appear in $\Nback{v}$} \mid \pi, C(u_1),C(u_2),\ldots} \geq 1-\frac{\degback{v}}{\Delta +1}.
	\]
	Using~\Cref{clm:warmup-Xv-step} and the above bounds, we conclude
	\begin{align*}
		\expect{X_v \mid \pi } &= \frac{\expect{\card{C_c} \mid \pi}}{\Pr\paren{\text{$c$ does not appear in $\Nback{v}$} \mid \pi, C(u_1),C(u_2),\ldots}} \\
		&\leq \frac{n}{\Delta +1 -\degback{v}}.\Qed{lem:warmup-Xv-bound}
	\end{align*}
	
\end{proof}

By~\Cref{lem:warmup-Xv-bound} (and~\Cref{obs:warmup-Xv}), for any choice of the permutation $\pi$ in~\Cref{alg:warmup},  
\begin{align}
	\expect{\text{runtime of~\Cref{alg:warmup}} \mid \pi} = O(1) \cdot \sum_{v \in V} \frac{n}{\Delta+1-\degback{v}}.  \label{eq:warmup-suffices}
\end{align}

We now consider the randomness of  $\pi$ to bound the RHS above in expectation over $\pi$. 
\begin{lemma}\label{lem:warmup-RHS-suffices}
	We have, 
	\[
	\Exp_{\pi}\bracket{\sum_{v \in V} \frac{n}{\Delta+1-\degback{v}}} = O(\frac{n^2}{\Delta} \cdot \log{\Delta}).
	\] 
\end{lemma}
\begin{proof}
	Using the linearity of expectation, we have
	\[
	\Exp_{\pi}\bracket{\sum_{v \in V} \frac{n}{\Delta+1-\degback{v}}} =n \cdot \sum_{v\in V} \Exp_{\pi}\bracket{\frac{1}{\Delta+1-\degback{v}}}.
	\]
	
	For each permutation $\pi$, $\degback{v}$ depends on $v$'s relative position in the permutation $\pi$ with respect to its neighbors and it can vary from $0$ to $\deg(v)$. Each of these positions happens with the same probability ${1}/({\deg(v)+1})$. 
	Hence, for every vertex $v\in V$,
	\begin{align*}
		\Exp_{\pi}\bracket{\frac{1}{\Delta+1-\degback{v}}} &= \sum_{d=0}^{\deg(v)}\frac{1}{\deg(v)+1}\cdot \frac{1}{\Delta +1-d} \leq \sum_{d=0}^{\Delta}\frac{1}{\Delta+1}\cdot \frac{1}{\Delta +1-d} = O(\frac{\log{\Delta}}{\Delta}),
	\end{align*}
	where the inequality holds because if we let $A := \set{n/(\Delta + 1 - d)}_{d=0}^{\Delta}$, then, in the LHS, we are taking the average of the smallest $\deg(v) + 1$ numbers in $A$, whereas in the RHS we are taking the average of all of $A$.
	Plugging in this bound in the equation above concludes the proof. 
\end{proof}

\Cref{lem:warmup-runtime} now follows immediately from \Cref{eq:warmup-suffices} and \Cref{lem:warmup-RHS-suffices}. We can now use this to wrap up the proof of~\Cref{thm:warm-up}. 

\begin{proof}[Proof of~\Cref{thm:warm-up}]
	
	First, we consider the case $\Delta \geq \sqrt{n\log n}$.  In this case, we use \Cref{alg:warmup}, which relies on the adjacency matrix to access the input graph.
	By~\Cref{lem:warmup-runtime} we know that this algorithm has expected runtime $O(\frac{n^2}{\Delta} \cdot \log{\Delta})$ which is $O(n\sqrt{n\cdot \log(n)})$ by the lower bound on $\Delta$.
	
	
	If $\Delta < \sqrt{n\log n}$ we can use the standard deterministic greedy algorithm for vertex coloring which has linear runtime of $O(n\Delta)$. This algorithm uses an adjacency list to access the graph. 
	This is again $O(n\sqrt{n\log{n}})$ by the upper bound on $\Delta$ in this case, concluding the proof.
\end{proof}



\end{document}
