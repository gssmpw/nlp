% !TeX root = main.tex 
%!TEX root = main.tex

\begin{abstract}

\smallskip

The \textbf{palette sparsification theorem (PST)} of Assadi, Chen, and Khanna (SODA 2019) states that in every graph $G$ with maximum degree $\Delta$, sampling a list of $O(\log{n})$ colors from 
$\{1,\ldots,\Delta+1\}$ for 
every vertex independently and uniformly, with high probability, allows for finding a $(\Delta+1)$ vertex coloring of $G$ by coloring each vertex only from its sampled list. PST naturally leads to a host of \textbf{sublinear algorithms} 
for $(\Delta+1)$ vertex coloring, including in semi-streaming, sublinear time, and MPC models, which are all proven to be nearly optimal, and in the case of the former two are the only known sublinear algorithms for this problem. 

\smallskip

While being a quite natural and simple-to-state theorem, PST suffers from two drawbacks. Firstly, all its known proofs require technical arguments that rely on sophisticated graph decompositions and probabilistic arguments. 
Secondly, finding the coloring of the graph from the sampled lists in an efficient manner requires a considerably complicated algorithm. 

\smallskip

We show that a natural \emph{weakening} of PST addresses both these drawbacks while still leading to sublinear algorithms of similar quality (up to polylog factors). 
In particular, we prove an \textbf{asymmetric palette sparsification theorem (APST)} that allows for list sizes of the vertices to have different sizes and only bounds the \emph{average} size of these lists. 
The benefit of this weaker requirement is that we can now easily show the graph can be $(\Delta+1)$ colored from the sampled lists using the standard greedy coloring algorithm. This way, we can recover 
nearly-optimal bounds for $(\Delta+1)$ vertex coloring in all the aforementioned models using algorithms that are much simpler to implement and analyze. 

\smallskip

\tableofcontents

\end{abstract}
