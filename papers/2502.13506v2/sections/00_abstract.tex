
\begin{abstract}
Negation is a fundamental aspect of human communication, yet it remains a challenge for Language Models (LMs) in Information Retrieval (IR). Despite the heavy reliance of modern neural IR systems on LMs, little attention has been given to their handling of negation. In this study, we reproduce and extend the findings of NevIR, a benchmark study that revealed most IR models perform at or below the level of random ranking when dealing with negation. We replicate NevIR’s original experiments and evaluate newly developed state-of-the-art IR models. Our findings show that a recently emerging category---listwise Large Language Model (LLM) rerankers—outperforms other models but still underperforms human performance. Additionally, we leverage ExcluIR, a benchmark dataset designed for exclusionary queries with extensive negation, to assess the generalizability of negation understanding. Our findings suggest that fine-tuning on one dataset does not reliably improve performance on the other, indicating notable differences in their data distributions. Furthermore,
we observe that only cross-encoders and listwise LLM rerankers achieve reasonable performance across both negation tasks.
\end{abstract}
