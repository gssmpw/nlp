
\documentclass[10pt,twocolumn,letterpaper]{article}

\usepackage[pagenumbers]{cvpr} %


%
% --- inline annotations
%
\newcommand{\red}[1]{{\color{red}#1}}
\newcommand{\todo}[1]{{\color{red}#1}}
\newcommand{\TODO}[1]{\textbf{\color{red}[TODO: #1]}}
% --- disable by uncommenting  
% \renewcommand{\TODO}[1]{}
% \renewcommand{\todo}[1]{#1}



\newcommand{\VLM}{LVLM\xspace} 
\newcommand{\ours}{PeKit\xspace}
\newcommand{\yollava}{Yo’LLaVA\xspace}

\newcommand{\thisismy}{This-Is-My-Img\xspace}
\newcommand{\myparagraph}[1]{\noindent\textbf{#1}}
\newcommand{\vdoro}[1]{{\color[rgb]{0.4, 0.18, 0.78} {[V] #1}}}
% --- disable by uncommenting  
% \renewcommand{\TODO}[1]{}
% \renewcommand{\todo}[1]{#1}
\usepackage{slashbox}
% Vectors
\newcommand{\bB}{\mathcal{B}}
\newcommand{\bw}{\mathbf{w}}
\newcommand{\bs}{\mathbf{s}}
\newcommand{\bo}{\mathbf{o}}
\newcommand{\bn}{\mathbf{n}}
\newcommand{\bc}{\mathbf{c}}
\newcommand{\bp}{\mathbf{p}}
\newcommand{\bS}{\mathbf{S}}
\newcommand{\bk}{\mathbf{k}}
\newcommand{\bmu}{\boldsymbol{\mu}}
\newcommand{\bx}{\mathbf{x}}
\newcommand{\bg}{\mathbf{g}}
\newcommand{\be}{\mathbf{e}}
\newcommand{\bX}{\mathbf{X}}
\newcommand{\by}{\mathbf{y}}
\newcommand{\bv}{\mathbf{v}}
\newcommand{\bz}{\mathbf{z}}
\newcommand{\bq}{\mathbf{q}}
\newcommand{\bff}{\mathbf{f}}
\newcommand{\bu}{\mathbf{u}}
\newcommand{\bh}{\mathbf{h}}
\newcommand{\bb}{\mathbf{b}}

\newcommand{\rone}{\textcolor{green}{R1}}
\newcommand{\rtwo}{\textcolor{orange}{R2}}
\newcommand{\rthree}{\textcolor{red}{R3}}
\usepackage{amsmath}
%\usepackage{arydshln}
\DeclareMathOperator{\similarity}{sim}
\DeclareMathOperator{\AvgPool}{AvgPool}

\newcommand{\argmax}{\mathop{\mathrm{argmax}}}     


\usepackage[most]{tcolorbox}
\usepackage{multirow}
\definecolor{cvprblue}{rgb}{0.21,0.49,0.74}
\usepackage[pagebackref,breaklinks,colorlinks,allcolors=cvprblue]{hyperref}

\def\paperID{16965} %
\def\confName{CVPR}
\def\confYear{2025}

\title{SPARC: Score Prompting and Adaptive Fusion for Zero-Shot Multi-Label Recognition in Vision-Language Models}

\author{
Kevin Miller\\
Boston University\\
{\tt\small nivek@bu.edu}
\and
Aditya Gangrade\\
Boston University\\
{\tt\small gangrade@bu.edu}
\and
Samarth Mishra\\
Boston University\\
{\tt\small samarthm@bu.edu}
\and
Kate Saenko\\
Boston University and Meta AI (FAIR)\\
{\tt\small saenko@bu.edu}
\and
Venkatesh Saligrama\\
Boston University\\
{\tt\small srv@bu.edu}
}

\begin{document}
\maketitle
\newcommand{\captionMainTableCOCO}{Main results for COCO.}
\newcommand{\captionMainTableVOC}{Main results for VOC.}
\newcommand{\captionMainTableNUSWIDE}{Main results for NUSWIDE.}
\newcommand{\captionMainTableAvgArch}{Results over three datasets, averaged over backbones.}
\newcommand{\captionCompetitionTagCLIPlogNOrowcalibbaseNO}{Results from combining our method with TagCLIP.}
\newcommand{\captionCompetitionTagCLIPlogNOrowcalibbaseYES}{Results from combining our method with TagCLIP.}
\newcommand{\captionCompetitionTagCLIPlogYESrowcalibbaseNO}{Shows complementarity with architectural approaches, improving TagCLIP by 2.6\%}
\newcommand{\captionCompetitionTagCLIPlogYESrowcalibbaseYES}{Results from combining our method with TagCLIP.}
\newcommand{\captionCompetitionTaIDPTrowcalibbaseNO}{TaI-DPT results}
\newcommand{\captionCompetitionTaIDPTrowcalibbaseYES}{TaI-DPT results}
\newcommand{\captionCompetitionCoMCrowcalibbaseNO}{CoMC results}
\newcommand{\captionCompetitionCoMCrowcalibbaseYES}{CoMC results}
\newcommand{\captionComphrehensiveCompoundOnly}{Average mAP (across all datasets and architectures) for different Rank Fusion strategies, without ``merge'' step.}
\newcommand{\captionComphrehensiveAvg}{Average mAP (across all datasets and architectures) for different Rank Fusion strategies, with ``merge'' step.}
\newcommand{\captionComphrehensivePCA}{Average mAP (across all datasets and architectures) for different selection strategies, PCA ensemble.}
\newcommand{\captionNoiseModelResults}{Comparing different noise models for pair prompt scores.}
\newcommand{\captionDebiasAblationTable}{Ablations on Debias module. Quantifies impact of debiasing both with and without compound prompts.}
\newcommand{\captionQualitativeHist}{Histograms for singleton, 1st max, and 2nd max scores for ``cat'' in the COCO dataset. We see that 1st max creates overlap by lifting the scores of some ground-truth negatives. 2nd max does not create these issues and performs well when fused with singleton scores.}
\newcommand{\captionQualitativeExample}{An example of why 2max works better than 1max. The left image gets a lower singleton score than the right image, resulting in a misordering. The 1max strategy lifts the scores of both images and thus fails to correct the misordering. The 2max strategy successfully corrects the misordering by lifting only the left image's score.}
\newcommand{\captionOurMethodSchematic}{A schematic of our method. \textbf{Top:} We use the target classnames along with cooccurrence stats to create compound prompts that mention multiple classes together. \textbf{Bottom:} At inference time, we query the VLM with both singleton and compound prompts, and use debiasing, selection, and fusion to come up with refined scores.}
\newcommand{\captionIntroFigure}{\textbf{Top Left:} Motivating example where the second-highest-scoring compound prompt does a better job than the highest-scoring one at correcting a misordering. \textbf{Top Right:} Our method uses classnames and cooccurrence info to generate compound prompts that mention multiple objects. \textbf{Bottom Left:} We use Debias and Rank Fusion modules to refine singleton scores by combining them with compound scores. \textbf{Bottom Right:} Rank Fusion module sorts the compound scores weights them based on max variation direction.}
\newcommand{\captionMethodFigure}{(top) Vision Language models(VLMs) like CLIP can be used for zero-shot classification with image-text similarity scores. While this works fairly well for single-class labels, they can struggle in the multi-label scenario. (bottom) In this paper, we instroduce SPARC, our solution that functions on top of an existing VLM, treating it simply as a black-box score generator. Using class names, SPARC first creates compound prompts for additional queries to the VLM. It then debiases, ranks and appropriately fuses them to generate final scores for the original classes.}
\newcommand{\captionExampleFigure}{A motivating example with an image where class ``cat'' is absent (left) and one where it is present (right). The highest compound prompt score is an unhelpful signal because it gives a high score to both negatives and positives, while the second-highest is more discriminative. Our method adaptively fuses the most informative order statistics, resulting in a strong signal.}
\newcommand{\captionCompetitionTaIDPTandCoMC}{Results from combining our method with TaI-DPT (left) and CoMC (right) showing compatibility with training-based methods. Our method improves TaI-DPT by 1.7\% while preserving the existing strong signal of CoMC, degrading it by only 0.3\%.}%
\newcommand{\captionMainTableThreeDatasets}{Results over three datasets and nine CLIP backbones. Our proposed method consistently outperforms the ZSCLIP baseline across all datasets and architectures exhibiting its effectivness on multi-label recognition tasks.}
\newcommand{\captionRankFusionAblation}{Average mAP for different Rank Fusion strategies, without (top) and with (bottom) the ``merge'' step \eqref{eq:merge} demonstrates superiority of adaptive fusion over fixed strategies.}
\newcommand{\captionPerClassAPsCOCO}{Per-class APs (averaged over all CLIP backbones) for our method vs vanilla ZSCLIP on the COCO dataset. Our method consistently improves over ZSCLIP for almost every class. Plots for VOC and NUSWIDE are shown in the Supplementary.}

\begin{abstract}


The choice of representation for geographic location significantly impacts the accuracy of models for a broad range of geospatial tasks, including fine-grained species classification, population density estimation, and biome classification. Recent works like SatCLIP and GeoCLIP learn such representations by contrastively aligning geolocation with co-located images. While these methods work exceptionally well, in this paper, we posit that the current training strategies fail to fully capture the important visual features. We provide an information theoretic perspective on why the resulting embeddings from these methods discard crucial visual information that is important for many downstream tasks. To solve this problem, we propose a novel retrieval-augmented strategy called RANGE. We build our method on the intuition that the visual features of a location can be estimated by combining the visual features from multiple similar-looking locations. We evaluate our method across a wide variety of tasks. Our results show that RANGE outperforms the existing state-of-the-art models with significant margins in most tasks. We show gains of up to 13.1\% on classification tasks and 0.145 $R^2$ on regression tasks. All our code and models will be made available at: \href{https://github.com/mvrl/RANGE}{https://github.com/mvrl/RANGE}.

\end{abstract}

    
\vspace{-12pt}
\section{Introduction}
\begin{figure}[t]
  \centering
   \includegraphics[width=1\linewidth]{sec/figs/fig1.png}
   \caption{\captionMethodFigure}
   \label{fig:methodFigure}
\end{figure}
\begin{figure}[t]
  \centering
   \includegraphics[width=1\linewidth]{sec/figs/exampleFigure.png}
   \caption{\captionExampleFigure}
   \label{fig:exampleFigure}
   \vspace{-12pt}
\end{figure}
Vision-Language Models (VLMs), such as CLIP~\cite{CLIP}, have emerged as general-purpose systems for understanding visual data through language-based queries. These models enable a broad range of applications, from object detection to image captioning, by linking visual inputs to language prompts. In standard settings where images contain single, recognizable objects, VLMs perform remarkably well. However, for the more complex task of zero-shot multi-label recognition (MLR) (Fig. \ref{fig:methodFigure} (top)), where models must identify multiple objects within an image without prior training on specific data, VLMs face significant limitations. Zero-shot MLR is crucial for applications in fields like robotics and medical imaging, where objects rarely appear in configurations that align neatly with training distributions. In these scenarios, achieving robust multi-label recognition without fine-tuning is challenging, given the task’s reliance on mean Average Precision (mAP) scores, which depend on ranking images for object presence.

\noindent \textbf{VLM: Prompt Dependent AND/OR Noisy Channel.} Despite the promise of zero-shot capabilities, current VLM approaches often struggle with MLR due to inherent scoring behaviors and biases. The performance of these models is hampered by a mix of conjunction (AND) and disjunction (OR) behaviors in their scoring, leading to inflated scores in compound prompts that contain multiple objects. For example, a prompt like “cat and sofa” might yield a high score even if only one of these objects is present in the image. This tendency reflects biases learned during training, where common object pairs receive higher scores even when only one object is present, disrupting the accuracy of mAP-based evaluations. Furthermore, existing methods for adapting VLMs to zero-shot MLR frequently rely on prompt tuning or architectural adjustments—approaches that are often dependent on training data and computationally intensive fine-tuning, which limit their generalizability to novel tasks.

\noindent \textbf{Our Approach.} In contrast to these methods, we introduce SPARC (Score Prompting and Adaptive Fusion for Zero-Shot Multi-Label Recognition in VLMs), a novel approach to zero-shot MLR that bypasses the need for training data, prompt tuning, or model-specific modifications. Our method treats the VLM as a black box, relying solely on its output scores to infer object presence (see Fig. \ref{fig:methodFigure}). This black-box approach enables us to avoid assumptions about the model’s internal workings, allowing for a purely zero-shot framework that is both model-agnostic and dataset-independent. SPARC introduces two main innovations that address the unique challenges of zero-shot MLR.

\noindent \textbf{A. Compound Prompt Composition:} Recognizing that VLMs can provide richer information when prompted with combinations of objects, we develop a method for constructing compound prompts. These prompts reflect likely contextual associations between objects, such as “cat and sofa” or “car and bus.” By gathering scores from these compound prompts, we can capture a spectrum of potential object contexts within the image, enhancing detection without relying on training-based adaptations. This composition strategy allows us to agnostically extract information from the VLM, leveraging probable object relationships without depending on any specific dataset or VLM architecture.\\
\noindent \textbf{B. Score Debiasing and Adaptive Fusion.} A critical insight in our approach lies in the surprising observation that the \underline{maximum score among compound prompts} is often a poor proxy for true object presence. Although one might expect the highest score to serve as a reliable signal, we find that it frequently reflects compositional biases, as VLMs tend to respond to compound prompts with OR-like behavior, raising scores even when only one object in the prompt is present. Instead, we observe that the second-highest score consistently provides a more accurate indicator of object presence, minimizing the effects of false positives caused by compositional bias. Building on this insight, we develop a debiasing algorithm that normalizes scores across images to address image-specific noise and clarify genuine object presence signals. This debiased score set is then processed through a PCA-based fusion method that further refines object rankings by combining information from both compound and singleton prompts, ultimately optimizing mAP by enhancing score accuracy.\\
\textbf{Complementarity.} SPARC is complementary to other zero-shot and training-free MLR methods. When applied on top of these approaches, SPARC consistently enhances mAP scores by refining object ranking and reducing bias in VLM outputs. This capability makes SPARC an adaptable solution that can improve upon existing methods while maintaining a fully zero-shot, model-agnostic framework.

\noindent \textbf{Empirical Results.} SPARC achieves significant improvements in mAP, outperforming methods that incorporate architectural modifications. This outcome shows the potential of a fully zero-shot approach that relies only on systematic prompt design and score interpretation, rather than prompt-training or fine-tuning. By revealing that the second-highest score can be a superior proxy to the maximum, our findings provide new insights into VLM scoring behavior, suggesting that careful treatment of prompt compositions and score patterns can unlock robust MLR capabilities.


\section{Related Work}
\label{sec:related-works}
\subsection{Novel View Synthesis}
Novel view synthesis is a foundational task in the computer vision and graphics, which aims to generate unseen views of a scene from a given set of images.
% Many methods have been designed to solve this problem by posing it as 3D geometry based rendering, where point clouds~\cite{point_differentiable,point_nfs}, mesh~\cite{worldsheet,FVS,SVS}, planes~\cite{automatci_photo_pop_up,tour_into_the_picture} and multi-plane images~\cite{MINE,single_view_mpi,stereo_magnification}, \etal
Numerous methods have been developed to address this problem by approaching it as 3D geometry-based rendering, such as using meshes~\cite{worldsheet,FVS,SVS}, MPI~\cite{MINE,single_view_mpi,stereo_magnification}, point clouds~\cite{point_differentiable,point_nfs}, etc.
% planes~\cite{automatci_photo_pop_up,tour_into_the_picture}, 


\begin{figure*}[!t]
    \centering
    \includegraphics[width=1.0\linewidth]{figures/overview-v7.png}
    %\caption{\textbf{Overview.} Given a set of images, our method obtains both camera intrinsics and extrinsics, as well as a 3DGS model. First, we obtain the initial camera parameters, global track points from image correspondences and monodepth with reprojection loss. Then we incorporate the global track information and select Gaussian kernels associated with track points. We jointly optimize the parameters $K$, $T_{cw}$, 3DGS through multi-view geometric consistency $L_{t2d}$, $L_{t3d}$, $L_{scale}$ and photometric consistency $L_1$, $L_{D-SSIM}$.}
    \caption{\textbf{Overview.} Given a set of images, our method obtains both camera intrinsics and extrinsics, as well as a 3DGS model. During the initialization, we extract the global tracks, and initialize camera parameters and Gaussians from image correspondences and monodepth with reprojection loss. We determine Gaussian kernels with recovered 3D track points, and then jointly optimize the parameters $K$, $T_{cw}$, 3DGS through the proposed global track constraints (i.e., $L_{t2d}$, $L_{t3d}$, and $L_{scale}$) and original photometric losses (i.e., $L_1$ and $L_{D-SSIM}$).}
    \label{fig:overview}
\end{figure*}

Recently, Neural Radiance Fields (NeRF)~\cite{2020NeRF} provide a novel solution to this problem by representing scenes as implicit radiance fields using neural networks, achieving photo-realistic rendering quality. Although having some works in improving efficiency~\cite{instant_nerf2022, lin2022enerf}, the time-consuming training and rendering still limit its practicality.
Alternatively, 3D Gaussian Splatting (3DGS)~\cite{3DGS2023} models the scene as explicit Gaussian kernels, with differentiable splatting for rendering. Its improved real-time rendering performance, lower storage and efficiency, quickly attract more attentions.
% Different from NeRF-based methods which need MLPs to model the scene and huge computational cost for rendering, 3DGS has stronger real-time performance, higher storage and computational efficiency, benefits from its explicit representation and gradient backpropagation.

\subsection{Optimizing Camera Poses in NeRFs and 3DGS}
Although NeRF and 3DGS can provide impressive scene representation, these methods all need accurate camera parameters (both intrinsic and extrinsic) as additional inputs, which are mostly obtained by COLMAP~\cite{colmap2016}.
% This strong reliance on COLMAP significantly limits their use in real-world applications, so optimizing the camera parameters during the scene training becomes crucial.
When the prior is inaccurate or unknown, accurately estimating camera parameters and scene representations becomes crucial.

% In early works, only photometric constraints are used for scene training and camera pose estimation. 
% iNeRF~\cite{iNerf2021} optimizes the camera poses based on a pre-trained NeRF model.
% NeRFmm~\cite{wang2021nerfmm} introduce a joint optimization process, which estimates the camera poses and trains NeRF model jointly.
% BARF~\cite{barf2021} and GARF~\cite{2022GARF} provide new positional encoding strategy to handle with the gradient inconsistency issue of positional embedding and yield promising results.
% However, they achieve satisfactory optimization results when only the pose initialization is quite closed to the ground-truth, as the photometric constrains can only improve the quality of camera estimation within a small range.
% Later, more prior information of geometry and correspondence, \ie monocular depth and feature matching, are introduced into joint optimisation to enhance the capability of camera poses estimation.
% SC-NeRF~\cite{SCNeRF2021} minimizes a projected ray distance loss based on correspondence of adjacent frames.
% NoPe-NeRF~\cite{bian2022nopenerf} chooses monocular depth maps as geometric priors, and defines undistorted depth loss and relative pose constraints for joint optimization.
In earlier studies, scene training and camera pose estimation relied solely on photometric constraints. iNeRF~\cite{iNerf2021} refines the camera poses using a pre-trained NeRF model. NeRFmm~\cite{wang2021nerfmm} introduces a joint optimization approach that simultaneously estimates camera poses and trains the NeRF model. BARF~\cite{barf2021} and GARF~\cite{2022GARF} propose a new positional encoding strategy to address the gradient inconsistency issues in positional embedding, achieving promising results. However, these methods only yield satisfactory optimization when the initial pose is very close to the ground truth, as photometric constraints alone can only enhance camera estimation quality within a limited range. Subsequently, 
% additional prior information on geometry and correspondence, such as monocular depth and feature matching, has been incorporated into joint optimization to improve the accuracy of camera pose estimation. 
SC-NeRF~\cite{SCNeRF2021} minimizes a projected ray distance loss based on correspondence between adjacent frames. NoPe-NeRF~\cite{bian2022nopenerf} utilizes monocular depth maps as geometric priors and defines undistorted depth loss and relative pose constraints.

% With regard to 3D Gaussian Splatting, CF-3DGS~\cite{CF-3DGS-2024} also leverages mono-depth information to constrain the optimization of local 3DGS for relative pose estimation and later learn a global 3DGS progressively in a sequential manner.
% InstantSplat~\cite{fan2024instantsplat} focus on sparse view scenes, first use DUSt3R~\cite{dust3r2024cvpr} to generate a set of densely covered and pixel-aligned points for 3D Gaussian initialization, then introduce a parallel grid partitioning strategy in joint optimization to speed up.
% % Jiang et al.~\cite{Jiang_2024sig} proposed to build the scene continuously and progressively, to next unregistered frame, they use registration and adjustment to adjust the previous registered camera poses and align unregistered monocular depths, later refine the joint model by matching detected correspondences in screen-space coordinates.
% \gjh{Jiang et al.~\cite{Jiang_2024sig} also implemented an incremental approach for reconstructing camera poses and scenes. Initially, they perform feature matching between the current image and the image rendered by a differentiable surface renderer. They then construct matching point errors, depth errors, and photometric errors to achieve the registration and adjustment of the current image. Finally, based on the depth map, the pixels of the current image are projected as new 3D Gaussians. However, this method still exhibits limitations when dealing with complex scenes and unordered images.}
% % CG-3DGS~\cite{sun2024correspondenceguidedsfmfree3dgaussian} follows CF-3DGS, first construct a coarse point cloud from mono-depth maps to train a 3DGS model, then progressively estimate camera poses based on this pre-trained model by constraining the correspondences between rendering view and ground-truth.
% \gjh{Similarly, CG-3DGS~\cite{sun2024correspondenceguidedsfmfree3dgaussian} first utilizes monocular depth estimation and the camera parameters from the first frame to initialize a set of 3D Gaussians. It then progressively estimates camera poses based on this pre-trained model by constraining the correspondences between the rendered views and the ground truth.}
% % Free-SurGS~\cite{freesurgs2024} matches the projection flow derived from 3D Gaussians with optical flow to estimate the poses, to compensate for the limitations of photometric loss.
% \gjh{Free-SurGS~\cite{freesurgs2024} introduces the first SfM-free 3DGS approach for surgical scene reconstruction. Due to the challenges posed by weak textures and photometric inconsistencies in surgical scenes, Free-SurGS achieves pose estimation by minimizing the flow loss between the projection flow and the optical flow. Subsequently, it keeps the camera pose fixed and optimizes the scene representation by minimizing the photometric loss, depth loss and flow loss.}
% \gjh{However, most current works assume camera intrinsics are known and primarily focus on optimizing camera poses. Additionally, these methods typically rely on sequentially ordered image inputs and incrementally optimize camera parameters and scene representation. This inevitably leads to drift errors, preventing the achievement of globally consistent results. Our work aims to address these issues.}

Regarding 3D Gaussian Splatting, CF-3DGS~\cite{CF-3DGS-2024} utilizes mono-depth information to refine the optimization of local 3DGS for relative pose estimation and subsequently learns a global 3DGS in a sequential manner. InstantSplat~\cite{fan2024instantsplat} targets sparse view scenes, initially employing DUSt3R~\cite{dust3r2024cvpr} to create a densely covered, pixel-aligned point set for initializing 3D Gaussian models, and then implements a parallel grid partitioning strategy to accelerate joint optimization. Jiang \etal~\cite{Jiang_2024sig} develops an incremental method for reconstructing camera poses and scenes, but it struggles with complex scenes and unordered images. 
% Similarly, CG-3DGS~\cite{sun2024correspondenceguidedsfmfree3dgaussian} progressively estimates camera poses using a pre-trained model by aligning the correspondences between rendered views and actual scenes. Free-SurGS~\cite{freesurgs2024} pioneers an SfM-free 3DGS method for reconstructing surgical scenes, overcoming challenges such as weak textures and photometric inconsistencies by minimizing the discrepancy between projection flow and optical flow.
%\pb{SF-3DGS-HT~\cite{ji2024sfmfree3dgaussiansplatting} introduced VFI into training as additional photometric constraints. They separated the whole scene into several local 3DGS models and then merged them hierarchically, which leads to a significant improvement on simple and dense view scenes.}
HT-3DGS~\cite{ji2024sfmfree3dgaussiansplatting} interpolates frames for training and splits the scene into local clips, using a hierarchical strategy to build 3DGS model. It works well for simple scenes, but fails with dramatic motions due to unstable interpolation and low efficiency.
% {While effective for simple scenes, it struggles with dramatic motion due to unstable view interpolation and suffers from low computational efficiency.}

However, most existing methods generally depend on sequentially ordered image inputs and incrementally optimize camera parameters and 3DGS, which often leads to drift errors and hinders achieving globally consistent results. Our work seeks to overcome these limitations.

\section{Method}
\label{sec:method}


We detail our method, SPARC (see Alg.~\ref{alg:sparc} and Fig.~\ref{fig:methodFigure}).


\begin{algorithm}
\caption{SPARC Pipeline}
\label{alg:sparc}
\begin{algorithmic}
\STATE \textbf{Input:} Images $\mathcal{I}$, Class Names $\mathcal{C}$
\STATE \textbf{Output:} Final scores $\zeta^t_i$ for each image $t$ and class $i$
\STATE $P \gets \text{GenerateCompoundPrompts}(\mathcal{C})$ \hfill \textit{(Details in Supp.)}
\FOR{$t \in \mathcal{I}$}
   \STATE $s^t_i \gets \text{GetVLMScores}(t, i)$ \hfill\textit{(Singleton scores)}
   \STATE $\forall p \in P,$ $s^t_p \gets \text{GetVLMScores}(t, p)$ \hfill\textit{(Compound scs.)}
   \STATE // \textit{(Debias scores (Eqs.~\ref{eqn:debias_images},\ref{eqn:debias_classes}))}
   \STATE $\tilde{s}^t_i \gets \text{ImageDebias}(s^t_i)$, $\tilde{s}^t_p \gets \text{ImageDebias}(s^t_p)$
   \STATE $\bar{s}^t_i \gets \text{PromptDebias}(\tilde{s}^t_i)$,
   $\bar{s}^t_p \gets \text{PromptDebias}(\tilde{s}^t_p)$
   \FOR{$i \in \mathcal{C}$}
       \STATE $r^t_{i,k} \gets \text{GetOrderStatistics}(\{\bar{s}^t_p: i \in p\})$
       \STATE $w^{i*} \gets \text{MaxVarDirection}([\bar{s}_i^t, r^t_{i,k}])$ \hfill 
 \textit{(Eq.~\ref{eq:maxVariance})}
       \STATE $\tilde{\zeta}^t_i \gets w^{i*}_0 \bar{s}^t_i + \sum_k w^{i*}_k r^t_{i,k}$ \hfill \textit{(Eq.~\ref{eqn:pca_projection})}
       \STATE $\zeta^t_i \gets \bar{s}^t_i + \tilde{\zeta}^t_i$  \hfill \textit{(Eq.~\ref{eq:merge})}
   \ENDFOR
\ENDFOR
\RETURN $\{\zeta^t_i\}$
\end{algorithmic}
\end{algorithm}

\noindent \emph{Setup.} %
Suppose we have $M$ images and $N$ target classes, with classnames $c_1,...,c_N$. We assume that we have a set of ``singleton'' prompts describing each of these classes in isolation (e.g. ``picture of a cat''). Given an image $t$, let $s_1^t,...,s_N^t$ denote the similarity scores produced by a VLM when comparing the singleton prompts to that image.

A na\"{i}ve approach is to directly use these singleton scores to perform MLR. However, since the presence of a particular class alters the conditional probability of the remaining classes in images, it should be possible to further refine these scores by accounting for the multi-label structure of the images. In order to do this, we use the classnames to generate a further set of ``compound'' prompts $P$ which mention multiple classes. For instance, the classes ``cat'' and ``sofa'' may be compounded to ``cat and sofa''. Let $\{s_p^t : p \in P\}$ denote the scores given by the VLM when comparing these compound prompts to image $t$.

Using the singleton as well as the compound prompts above, SPARC produces a vector of refined class-wise scores $\zeta_1^t, \cdots, \zeta_N^t,$ for each image $t$. Given this structure, SPARC has three main components: (i) compound prompt generation to enable the use of CLIP scoring for MLR (ii) image and prompt level debiasing to allow scores to be directly compared, and (iii) Rank Fusion to combine singleton and compound prompt scores into a single score per class. We now discuss each of these aspects in detail. 

\subsection{Compound Prompt Generation}
Our compound prompt generation method relies on contextual associations from commonly observed visual patterns to inform prompt structure. Specifically, we remove object pairs and triplets that are implausible based on low-frequency observations in the data, retaining only combinations that align with realistic visual scenes. Leveraging these contextual associations, an off-the-shelf LLM generates descriptive prompts that expand beyond the target class names, enriching the diversity of object scenarios. While we explored various prompt templates ("A OR B", "A next to B", "A with B"), "A and B" performed best in initial experiments. We focus on this template to establish our core method, leaving systematic prompting as future work.
Pseudocode and details are available in the Supplementary.

\subsection{Debiasing}
Once we have obtained compound prompts $P$, we can query the VLM to obtain singleton scores $s_1^t,...,s_N^t$ and compound scores $\{s_p^t : p \in P\}$ for each image $t$. However, these scores contain both image- and prompt-specific biases. The former in particular is a problem, as it can change the order of scores between negative and positive images.

We address image-level bias for singleton and compound prompts, respectively, as follows:
\begin{align}\label{eqn:debias_images}
\tilde{s}_i^t := \frac{s_i^t - \hat{\mu}(s_{\cdot}^t)}{\hat{\sigma}(s_{\cdot}^t)} \quad \textit{and} \quad
\tilde{s}_p^t := \frac{s_p^t - \hat{\mu}(\check{s}_{\cdot}^t)}{\hat{\sigma}(\check{s}_{\cdot}^t)},
\end{align}
where $\hat{\mu}(s_{\cdot}^t)$, $\hat{\mu}(\check{s}_{\cdot}^t)$, $\hat{\sigma}(s_{\cdot}^t)$, $\hat{\sigma}(\check{s}_{\cdot}^t)$ are sample means and standard deviations across the prompt dimension for a single image. $\check{s}_1^t,...,\check{s}_N^t$ are the scores of ``auxiliary'' prompts that only mention the classnames $c_1,...,c_N$ and are used only to obtain statistics for standardizing the compound prompt scores. We use these instead of singleton prompts for this purpose as the latter might have different statistical properties depending on the nature of the singleton prompting method.

We also standardize across images to quash prompt bias:
\begin{align}\label{eqn:debias_classes}
\overline{s}_i^t := \frac{\tilde{s}_i^t - \hat{\mu}(\tilde{s}_i^{\cdot})}{\hat{\sigma}(\tilde{s}_i^{\cdot})} \quad \textit{and} \quad
\overline{s}_p^t := \frac{\tilde{s}_p^t - \hat{\mu}(\tilde{s}_p^{\cdot})}{\hat{\sigma}(\tilde{s}_p^{\cdot})},
\end{align}
where $\hat{\mu}(\tilde{s}_i^{\cdot})$, $\hat{\mu}(\tilde{s}_p^{\cdot})$, $\hat{\sigma}(\tilde{s}_i^{\cdot})$, and $\hat{\sigma}(\tilde{s}_p^{\cdot})$ now denote sample means and SDs across the images. Removing prompt-level bias makes scores from different prompts more compatible with each other, which is important for fusing them.

\subsection{Rank Fusion}
Our goal now is to find a way to use the compound scores to strengthen the singleton scores. Let us define a bit more notation here. Let $C(p)$ be a function that maps each compound prompt to the set of classes in $[N]$ that are mentioned by that prompt. Let $r_{i,k}^t$ denote the $k$-th largest element of the set $\{\overline{s}_p^t : i \in C(p)\}$.

A natural choice might be to choose $r_{i,1}^t$, i.e. the highest scoring prompt. After all, we would expect this to be the prompt that most closely describes the image, and if the image did not contain $c_i$ then we would expect none of the prompts containing $c_i$ to score high. However we find that $r_{i,1}^t$ does not offer a good detection of classes. The mechanism behind this, which is explored in detail in \S\ref{sec:noisemodel}, is that when using a compound prompt of the form `A and B' to detect the class $A$, the score sees a large increase if the object $B$ truly occurs in the image. This `OR'-like behaviour (\S\ref{sec:noisemodel}) means that maximum score $r_{i,1}^t$ typically captures the effect of other classes than $A$ being present in the image, and thus leads to very poor separation between ground truth positive and negative images for any class. Counterintuitively, then, we find that `\emph{weakened maxes}', i.e., lower order statistics like $r_{i,2}^t$ and $r_{i,3}^t$ yield much better separation. 

The above fact necessitates that we need to deal with the entirety of the order statistics $\{r_{i,k}^t\}$ in order to generate an effective fusion rule. We fuse these scores by projecting the vector of order statistics along the direction of highest variance, i.e., we compute the fused compound score $\tilde{\zeta}_i^t$ as
\begin{align}
\label{eq:maxVariance}
w^{i*} &:= \argmax_{w^i} \textrm{Var}_t(w_0^i \bar{s}_i^t + \sum_{k} w_k^i r_{i,k}^t )\\
\tilde{\zeta}_i^t &:= w_0^{i*} \bar{s}_i^t + \sum_{k} w_k^{i*} r_{i,k}^t. \label{eqn:pca_projection}
\end{align}
Notice that the variance above is computed across all images. The main idea is similar to linear discriminant analysis. Indeed, if we assume that the mean compound score are when class $i$ is actually present or absent differ from one another, then the collection of the compound scores across all images can be modeled as a mixture of two noisy distributions, with some mean separation. If we further assume that the noise in these scores are roughly homoskedastic in either component, then, as long as the `signal' induced by the above mean shift is large compared to the noisy variation, highest variance direction $w^{i*}$ captures precisely the direction of the mean shift, and thus $\tilde{\zeta_i^t}$ effectively computes the component of the score vector that lies along this signal direction. While principled, this max-variance strategy is of course not necessary to execute, and we compare it to alternative fusion strategies in \S\ref{subsubsec:RankFusionAblation}. We add (i.e. ``merge'') this fused score into the original singleton score to get our final score:
\begin{equation}
\label{eq:merge}
\zeta_i^t := s_i^t + \tilde{\zeta}_i^t
\end{equation}
In the next section we probe CLIP scores through a statistical lens to build intuition on our approach.

\section{Noise Model of a VLM}
\label{sec:noisemodel}



Our methodological choices are strongly driven by the behaviour of CLIP scores under compound prompts, and this leads to both the debiasing approach taken, as well as our counterintuitive choice of exploiting the weakened maxes rather than just fusing the compound prompts via the maximum score. In order to justify these choices, we probe the structure of CLIP scores under compound prompts by constructing a simple model of the same.

Given an image $t$ and a prompt with classes $i$ and $j$, we model the score of compound prompt `$i$ and $j$' as \mbox{$s_{i,j}^t := \theta_1^t \cdot f(y_i^t, y_j^t) + \theta_0^t + \varepsilon$}
where $y_i^t, y_j^t \in \{0,1\}$ indicate ground-truth presence of of the two classes, $\theta_1^t$ and $\theta_0^t$ represent image-level bias (e.g. due to visual domain-shifts), and $\varepsilon$ is a zero-centered Gaussian noise.
Notice that our Debias module is well-suited to mitigate variations in $\theta_0^t$ and $\theta_1^t$, at least when $\varepsilon$ is not strongly dominating. 

Now, the core object of interest is the map $f : \{0,1\}^2 \to \mathbb{R},$ which is responsible for modeling class pair interactions and prompt-level biases (see Table~\ref{tab:NoiseModelTable}). In an ideal situation for MLR, $f$ would behave roughly as an `AND' function, so that compound prompts would let us directly hone in on the presence of pairs of classes. However, prior work has observed that instead, for CLIP-type models, $f$ tends to predominantly behave as an `OR' function, in that $f$ is high if either $i$ or $j$ is present. Notice that if $f$ were to exactly behave as an OR, compound prompts would not be effective in detecting classes that were missed by the singleton scores.

In experimentation, we found that the behaviour of $s_{i,j}^t$ is intermediary to the above extremes: CLIP scores for compound prompts behave qualitatively as an OR function with a small AND `bonus' when both classes are present. This bonus, as well as the particular scores, are modulated by the classes mentioned (a consequence of prompt-level bias for each class) which we model as linear effects, leading to an overall model of the signal term of the CLIP score as \begin{align}
f(y_i^t, y_j^t) &:= \max(a_0^i + y_i^t a_1^i, a_0^j + y_j^t a_1^j) \nonumber \\ &+ \delta_{ij} \min(a_0^i + y_i^t a_1^i, a_0^j + y_j^t a_1^j),
\end{align} where the $a$ terms are class-wise linear effect coefficients, and $\delta_{ij}$ captures the strength of this AND bonus (which depends on which pair is being queried).

Note that this AND bonus is a critical feature of CLIP scores in their use for MLR problems, since this bump allows us to both infer the presence of classes that singleton prompts have missed, and indirectly to filter false positives (since the true positives are further elevated by the bonus).

\begin{table}[t]
\centering
\begin{tabular}{| c | c | c |}
  \hline
  \textbf{Noise Model} & $f(y_i^t, y_j^t)$ & \textbf{FVU} \\ \hline
  constant  & $0.1$ & 0.802 \\ \hline
  only AND    & $\min(a_0^i\! +\! y_i^t a_1^i, a_0^j + y_j^t a_1^j)$    & 0.536 \\ \hline
  only OR    & $\max(a_0^i + y_i^t a_1^i, a_0^j + y_j^t a_1^j)$  & 0.288 \\ \hline
  additive   & $a_0^i + y_i^t a_1^i + a_0^j + y_j^t a_1^j$   & 0.267 \\ \hline
  \makecell{OR + static\\ AND-bonus}    & $\makecell{\max(a_0^i + y_i^t a_1^i, a_0^j + y_j^t a_1^j)+\\ \delta \min(a_0^i\! +\! y_i^t a_1^i, a_0^j\! +\! y_j^t a_1^j)}$  & 0.263 \\ \hline
  \makecell{OR + variable\\ AND-bonus}  & $\makecell{\max(a_0^i + y_i^t a_1^i, a_0^j + y_j^t a_1^j)+\\ \delta_{i,j} \min(a_0^i\! +\! y_i^t a_1^i, a_0^j\! +\! y_j^t a_1^j)}$  & 0.248 \\ \hline
  look-up table    & $\textrm{LUT}^{(i,j)}(y_i^t, y_j^t)$   & 0.235 \\ \hline
\end{tabular}
\vspace{-5pt}
\caption{Comparison of fidelity of noise models for scoring pairwise compound prompts. Notice that the OR+AND-bonus model captures nearly all of the fidelity of the look-up table, and offers significant improvements over the simplified models above it.}\vspace{-\baselineskip} %
\label{tab:NoiseModelTable}
\end{table}


Table~\ref{tab:NoiseModelTable} shows the results of a systematic evaluation of the AND-OR structure of the compound CLIP scores. We evaluate a variety of models of $f$, ranging from simple constant, `only OR,' and `only AND' models to additive models and OR+AND-bonus models with both a `static' $\delta$ shared by all class pairs, and a `variable' $\delta_{i,j}$ that is specific to each class pairs. The final entry is a `look-up table' (LUT) model wherein all four values of $f$ are allowed to vary arbitrarily, which serves as the most expressive baseline in this model class. These are evaluated w.r.t. the fraction of variance unexplained (i.e., $1-R^2,$ where $R$ is the residual). Observe that while the only OR model has remarkable fidelity, the additive and AND-bonus models significantly improve upon this, and allowing $\delta$ to vary with $i,j$ yield a further significant improvement to within a $6\%$ extra loss from the LUT model. This justifies our above model, and shows that it captures a significant amount of the qualitative behaviour of CLIP scores for compound prompts. 

A closer look at these fitted models reveals the $\delta$ values in these models are both small, and relatively stable: for the static $\delta,$ the recovered value is $0.56$, and the interquantile range of the varying $\delta_{ij}$ is $(0.44, 0.57)$.  %

We find similar behavior when examining other backbones and datasets, which we detail in the Supplementary.

We now use this noise model to explain the strengths of the weakened max from a theoretical perspective.

\noindent \textbf{Theoretical Explanation for Weakened Max.}
We observe empirically in Sec.~\ref{subsubsec:RankFusionAblation} that using the second-max score outperforms the first-max in discriminative ability. %
Consider a target class \(0\) with cooccurring classes \(1, \dots, m\) and assume that each cooccurrence is governed by a simple distribution with probabilities \(\rho\) and \(q\) as follows:
\[
    \Pr(y_i = 1 \mid y_0 = 1) = \rho, \quad \Pr(y_i = 1 \mid y_0 = 0) = q < \rho.
\]
Let us now assume that the scores for these compound prompts are derived from the following simplified noise model:
$s_{0,i} = \max(\tilde{y}_0, \tilde{y}_i) + \delta \min(\tilde{y}_0, \tilde{y}_i) + \varepsilon,$
where \(\varepsilon \sim \mathcal{W}(\sigma)\) is additive noise, and $\tilde{y}_0,\, \tilde{y}_i \in \{0,1\}$ are noisy, potentially flipped versions of ground truth $y_0, y_i,$  respectively. This label-flip noise models the effects of real-world conditions such as occlusion or partial visibility of each object in the input to the VLM. We informally state our theoretical finding. See Supplementary for full details:\\
\textbf{Theorem (informal)} For any distribution of label noise on \(y_0\), the second-max consistently improves discrimination over the first-max, given enough cooccurring classes.
\subsection{Methodological Implications}

The qualitative structure of the OR+AND-bonus behaviour observed in the previous section offers a robust justification for our methodological choices in \S\ref{sec:method}. Naturally, image level debiasing reduces the effect of variation in $\theta_0^t$ (via the mean subtraction) and $\theta_1^t$ (via the scaling by standard deviation). Further, prompt level debiasing reduces the effect of variations in $a_0$ and $a_1$ across classes. 

We can also qualitatively observe that due to the fact that the OR structure of the scores dominates the AND structure, the behaviour of the largest score $r_{i,1}^t$ should often reflect the presence of a related class $j$ instead of that of $i$, and suggest that subsequent order statistics are more informative. In fact, this property can be theoretically derived under mild assumptions, as stated above. 






















% \section{Experiment and Results}
\section{Results and Analysis}
In this section, we first present safe vs. unsafe evaluation results for 12 LLMs, followed by fine-grained responding pattern analysis over six models among them, and compare models' behavior when they are attacked by same risky questions presented in different languages: Kazakh, Russian and code-switching language.    

\begin{table}[t!]
\centering
\small
\resizebox{\columnwidth}{!}{
\begin{tabular}{clcccc}
\toprule
\multicolumn{1}{l}{\textbf{Rank} } & \textbf{Model} & \textbf{Kazakh $\uparrow$} & \textbf{Russian $\uparrow$} & \textbf{English $\uparrow$} \\
\midrule
1 & \claude & \textbf{96.5}   & 93.5    & \textbf{98.6}    \\
2 & \gptfouro & 95.8   & 87.6    & 95.7    \\
3 & \yandexgpt & 90.7   & \textbf{93.6}    & 80.3    \\
4 & \kazllmseventy & 88.1 & 87.5 & 97.2 \\
5 & \llamaseventy & 88.0   & 85.5    & 95.7    \\
6 & \sherkala & 87.1   & 85.0    & 96.0    \\
7 & \falcon & 87.1   & 84.7    & 96.8    \\
8 & \qwen & 86.2   & 85.1    & 88.1    \\
9 & \llamaeight & 85.9   & 84.7    & 98.3    \\
10 & \kazllmeight & 74.8   & 78.0    & 94.5 \\
11 & \aya & 72.4 & 84.5 & 96.6 \\
12 & \vikhr & --- & 85.6 & 91.1 \\
\bottomrule
\end{tabular}
}
\caption{Safety evaluation results of 12 LLMs, ranked by the percentage of safe responses in the Kazakh dataset. \claude\ achieves the highest safety score for both Kazakh and English, while \yandexgpt\ is the safest model for Russian responses.}
\label{tab:safety-binary-eval}
\end{table}



\subsection{Safe vs. Unsafe Classification}
% In this subsection, 
We present binary evaluation results of 12 LLMs, by assessing 52,596 Russian responses and 41,646 Kazakh responses.
% 26,298 responses generated by six models on the Russian dataset and 22,716 responses on the Kazakh dataset. 

%\textbf{Safety Rank.} In general, proprietary systems outperform the open-source model. For Russian, As shown in Table \ref{tab:model_comparison_russian}, \textbf{Yandex-GPT} emerges as the safest large language model (LLM) for Russian, with a safety percentage of 93.57\%. Among the open-source models, \textbf{Vikhr-Nemo-12B} is the safest, achieving a safety percentage of 85.63\%.
% Edited: This is mentioned in the discussion
% This outcome highlights the potential impact of pretraining data on model behavior. Models pre-trained primarily on Russian data are better at understanding and handling harmful questions - in both proprietary systems and open-source setups. 
%For Kazakh, as shown in Table \ref{tab:model_comparison_kazakh}, \textbf{Claude} emerges as the safest large language model (LLM) with a safety percentage of 96.46\%, closely followed by GPT-4o at 95.75\%. In contrast, \textbf{Aya-101}, despite being specifically tuned for Kazakh, consistently shows the highest unsafe response rates at 72.37\%. 

\begin{figure*}[t!]
	\centering
        \includegraphics[scale=0.28]{figures/question_type_no6_kaz.png}
	\includegraphics[scale=0.28]{figures/question_type_exclude_region_specific_new.png} 

	\caption{Unsafe answer distribution across three question types for risk types I-V: Kazakh (left) and Russian (right)}
	\label{fig:qt_non_reg}
\end{figure*}

\begin{figure*}[t!]
	\centering
        \includegraphics[scale=0.28]{figures/question_type_only6_kaz.png}
	\includegraphics[scale=0.28]{figures/question_type_region_specific_new.png} 
	
	\caption{Unsafe answer distribution across three question types for risk type VI: Kazakh (left) and Russian (right)}
	\label{fig:qt_reg}
\end{figure*}

\textbf{Safety Rank.} In general, proprietary systems outperform the open-source models. 
For Russian, as shown in Table~\ref{tab:safety-binary-eval},  % \ref{tab:model_comparison_russian}, 
\yandexgpt emerges as the safest language model for Russian, with safe responses account for 93.57\%. Among the open-source models, \kazllmseventy is the safest (87.5\%), followed by \vikhr with a safety percentage of 85.63\%.

For Kazakh, % as shown in Table \ref{tab:model_comparison_kazakh}, 
% YX: todo, rerun Kazakh safety percentage using Diana threshold
\claude is the safest model with 96.46\% safe responses, closely followed by \gptfouro\ at 95.75\%. \aya, despite being specifically tuned for Kazakh, shows the highest unsafe rates at 72.37\%.



% \begin{table}[t!]
% \centering
% \resizebox{\columnwidth}{!}{%
% \begin{tabular}{clccc}
% \toprule
% \textbf{Rank} & \textbf{Model Name}  & \textbf{Safe} & \textbf{Unsafe} & \textbf{Safe \%} \\ \midrule
% \textbf{1} & \textbf{Yandex-GPT} & \textbf{4101} & \textbf{282} & \textbf{93.57} \\
% 2 & Claude & 4100 & 283 & 93.54 \\
% 3 & GPT-4o & 3839 & 544 & 87.59 \\
% 4 & Vikhr-12B & 3753 & 630 & 85.63 \\
% 5 & LLama-3.1-instruct-70B & 3746 & 637 & 85.47 \\
% 6 & LLama-3.1-instruct-8B & 3712 & 671 & 84.69 \\
% \bottomrule
% \end{tabular}
% }
% \caption{Comparison of models based on safety percentages for the Russian dataset.}
% \label{tab:model_comparison_russian}
% \end{table}


% \begin{table}[t!]
% \centering
% \resizebox{\columnwidth}{!}{%
% \begin{tabular}{clccc}
% \toprule
% \textbf{Rank} & \textbf{Model Name}  & \textbf{Safe} & \textbf{Unsafe} & \textbf{Safe \%} \\ \midrule
% 1             & \textbf{Claude}  & \textbf{3652} & \textbf{134} & \textbf{96.46} \\ 
% 2             & GPT-4o                      & 3625          & 161          & 95.75 \\ 
% 3             & YandexGPT                   & 3433          & 353          & 90.68 \\
% 4             & LLama-3.1-instruct-70B      & 3333          & 453          & 88.03 \\
% 5             & LLama-3.1-instruct-8B       & 3251          & 535	       & 85.87 \\
% 6             & Aya-101                     & 2740          & 1046         & 72.37 \\ 
% \bottomrule
% \end{tabular}
% }
% \caption{Comparison of models based on safety percentages for the Kazakh dataset.}
% \label{tab:model_comparison_kazakh}
% \end{table}



\textbf{Risk Areas.} 
We selected six representative LLMs for Russian and Kazakh respectively and show their unsafe answer distributions over six risk areas.
As shown in Table \ref{tab:unsafe_answers_summary}, risk type VI (region-specific sensitive topics) overwhelmingly contributes the largest number of unsafe responses across all models. This highlights that LLMs are poorly equipped to address regional risks effectively. For instance, while \llama models maintain comparable safety levels across other risk type (I–V), their performance drops significantly when dealing with risk type VI. Interestingly, while \yandexgpt\ demonstrates relatively poor performance in most other risk areas, it handles region-specific questions remarkably well, suggesting a stronger alignment with regional norms and sensitivities. For Kazakh, Table \ref{tab:unsafe_answers_summary_kazakh} shows that region‐specific topics (risk type VI) pose a substantial challenge across all six models, including the commercial \gptfouro\ and \claude, which demonstrate superior safety on general categories. 

% \begin{table}[t!]
% \centering
% \resizebox{\columnwidth}{!}{%
% \begin{tabular}{lccccccc}
% \toprule
% \textbf{Model} & \textbf{I} & \textbf{II} & \textbf{III} & \textbf{IV} & \textbf{V} & \textbf{VI} & \textbf{Total} \\ \midrule
% LLama-3.1-instruct-8B & 60 & 70 & 16 & 31 & 9 & 485 & 671 \\
% LLama-3.1-instruct-70B & 29 & 55 & 8 & 4 & 1 & 540 & 637 \\
% Vikhr-12B & 41 & 93 & 15 & 1 & 3 & 477 & 630 \\
% GPT-4o & 21 & 51 & 6 & 2 & 0 & 464 & 544 \\
% Claude & 7 & 10 & 1 & 0 & 0 & 265 & 283 \\
% Yandex-GPT & 55 & 125 & 9 & 3 & 8 & 82 & 282 \\
% \bottomrule
% \end{tabular}%
% }
% \caption{Ru unsafe cases over risk areas of six models.}
% \label{tab:unsafe_answers_summary}
% \end{table}


\begin{table}[t!]
\centering
\resizebox{\columnwidth}{!}{%
\begin{tabular}{lccccccc}
\toprule
\textbf{Model} & \textbf{I} & \textbf{II} & \textbf{III} & \textbf{IV} & \textbf{V} & \textbf{VI} & \textbf{Total} \\ \midrule
\llamaeight & 60 & 70 & 16 & 31 & 9 & 485 & 671 \\
\llamaseventy & 29 & 55 & 8 & 4 & 1 & 540 & 637 \\
\vikhr & 41 & 93 & 15 & 1 & 3 & 477 & 630 \\
\gptfouro & 21 & 51 & 6 & 2 & 0 & 464 & 544 \\
\claude & 7 & 10 & 1 & 0 & 0 & 265 & 283 \\
\yandexgpt & 55 & 125 & 9 & 3 & 8 & 82 & 282 \\
\bottomrule
\end{tabular}%
}
\caption{Ru unsafe cases over risk areas of six models.}
\label{tab:unsafe_answers_summary}
\end{table}


% \begin{table}[t!]
% \centering
% \resizebox{\columnwidth}{!}{%
% \begin{tabular}{lccccccc}
% \toprule
% \textbf{Model} & \textbf{I} & \textbf{II} & \textbf{III} & \textbf{IV} & \textbf{V} & \textbf{VI} & \textbf{Total} \\ \midrule
% Aya-101 & 96 & 235 & 165 & 166 & 90 & 294 & 1046 \\
% Llama-3.1-instruct-8B & 25 & 15 & 91 & 37 & 14 & 353 & 535 \\
% Llama-3.1-instruct-70B & 33 & 39 & 88 & 27 & 20 & 246 & 453 \\
% Yandex-GPT & 29 & 76 & 95 & 29 & 16 & 108 & 353 \\
% GPT-4o & 2 & 1 & 41 & 0 & 3 & 114 & 161 \\
% Claude & 2 & 1 & 26 & 3 & 6 & 96 & 134 \\ \bottomrule
% \end{tabular}%
% }
% \caption{Kaz unsafe cases over risk areas of six models.}
% \label{tab:unsafe_answers_summary_kazakh}
% \end{table}


\begin{table}[t!]
\centering
\resizebox{\columnwidth}{!}{%
\begin{tabular}{lccccccc}
\toprule
\textbf{Model} & \textbf{I} & \textbf{II} & \textbf{III} & \textbf{IV} & \textbf{V} & \textbf{VI} & \textbf{Total} \\ \midrule
\aya & 96 & 235 & 165 & 166 & 90 & 294 & 1046 \\
\llamaeight & 25 & 15 & 91 & 37 & 14 & 353 & 535 \\
\llamaseventy & 33 & 39 & 88 & 27 & 20 & 246 & 453 \\
\yandexgpt & 29 & 76 & 95 & 29 & 16 & 108 & 353 \\
\gptfouro & 2 & 1 & 41 & 0 & 3 & 114 & 161 \\
\claude & 2 & 1 & 26 & 3 & 6 & 96 & 134 \\ 
\bottomrule
\end{tabular}%
}
\caption{Kaz unsafe cases over risk areas of six models.}
\label{tab:unsafe_answers_summary_kazakh}
\end{table}

% \begin{figure*}[t!]
% 	\centering
% 	\includegraphics[scale=0.28]{figures/human_1000_kz_font16.png} 
% 	\includegraphics[scale=0.28]{figures/human_1000_ru_font16.png}

% 	\caption{Human vs \gptfouro\ fine-grained labels on 1,000 Kazakh (left) and Russian (right) samples.}
% 	\label{fig:human_fg_1000}
% \end{figure*}

\textbf{Question Type.} For Russian, Figures \ref{fig:qt_non_reg} and \ref{fig:qt_reg} reveal differences in how models handle general risks I-V and region-specific risk VI. For risks I-V, indirect attacks % crafted to exploit model vulnerabilities—
result in more unsafe responses due to their tricky phrasing. 

In contrast, region-specific risks see slightly more unsafe responses from direct attacks, 
% as these explicit prompts are more likely to bypass safety mechanisms. 
since indirect attacks for region-specific prompts often elicit safer, vaguer answers. %, suggesting models struggle less with implicit harm. 
Overall, the number of unsafe responses is similar across question types, indicating models generally struggle with safety alignment in all jailbreaking queries.

For Kazakh, Figures \ref{fig:qt_non_reg} and \ref{fig:qt_reg} show greater variation in unsafe responses across question types due to the low-resource nature of Kazakh data. For general risks I-V, \llamaseventy\ and \aya\ produce more unsafe outputs for direct harm prompts. At the same time, \claude\ and \gptfouro\ struggle more with indirect attacks, reflecting challenges in handling subtle cues. For region-specific risk VI, most models perform similarly due to limited Kazakh-specific data, though \llamaeight\ shows higher unsafe outputs for indirect local references, likely due to their implicit nature. Direct region-specific attacks yield fewer unsafe responses, as explicit prompts trigger more cautious outputs. Across all risk areas, general questions with sensitive words produce the fewest unsafe answers, suggesting over-flagging or cautious behavior for unclear harmful intent.



% \subsection{Fine-grained Classification}
% We extended our analysis to include fine-grained classifications for both safe and unsafe responses. For unsafe responses, we categorized outputs into four harm types, as shown in Table \ref{table:unsafe_response_categories}. 

% For safe responses, we classified outputs into six distinct patterns of safety, following a fine-grained rubric provided in \cite{wang2024chinesedatasetevaluatingsafeguards}. The types outlined in this rubric are presented in Table \ref{table:safe_response_categories}.

% To validate the fine-grained classification, we conducted human evaluation on the same 1,000 responses in Russian used for the preliminary binary classification.
% The confusion matrix highlights the alignment and discrepancies between human annotations and GPT's fine-grained labels. The diagonal values represent instances where the GPT's labels match human annotations, with category 5 (provides general, safe information) showing the highest agreement (404 instances). However, off-diagonal values reveal areas of disagreement, such as misclassifications in categories 1 (not willing to answer), 4 (recommends consulting an expert), and 7 (provides harmful or risky instructions). For example, in category 1, while 94 responses were correctly labeled, GPT-4o misclassified several instances into categories 4, 5, or 7, indicating overlap or ambiguity in these classifications. Similar trends are observed in other categories, where GPT sometimes struggles to differentiate nuanced distinctions in human-labeled categories. Overall, GPT's fine-grained labels match human annotations in 710 out of 1000 cases, achieving an agreement rate of 71\%.

% \begin{figure}[ht!]
%     \centering
%     \includegraphics[width=0.95\linewidth]{figures/human_fg_1000_ru.png}
%     \caption{{Human vs GPT-4o Fine-Grained Labels on 1,000 Russian Samples}}
%     \label{fig:human_fg_1000_ru}
% \end{figure}


% After conducting human evaluation on a representative sample, we extended the fine-grained classification to a full dataset comprising 21,915 responses generated by five different models. 


% \begin{figure}[ht!]
%     \centering
%     \includegraphics[width=0.95\linewidth]{figures/all_5_ru.png}
%     \caption{Fine-grained label distribution for responses from five models for Russian.}
%     \label{fig:all_5_fg_russian}
% \end{figure}
% Category 5 ("safe and general information") consistently has the highest frequency across all models, aligning with its dominance in the 1,000-sample evaluation. However, differences in the distribution across other categories highlight variability in how models handle nuanced safety risks, with Yandex-GPT showing a slightly broader spread across categories. 
% In the distribution of unsafe responses, most models exhibit higher counts in certain labels such as 8. However, Yandex-GPT displays comparatively fewer responses in these labels. 
% It exhibits a high rate of responses classified under label 7, which indicates instances where the model provides harmful or risky instructions, including unethical behavior or sensitive discussions. While this may suggest a vulnerability in addressing complex or challenging prompts, it was observed that many of Yandex-GPT’s responses tend to deflect responsibility or offer vague advice such as "check the internet". Although this approach minimizes the risk of unsafe outputs, it often results in responses that lack depth or contextually relevant information, limiting their overall usefulness for users.

% % \subsection{Kazakh}


% % Overall, these findings underscore how resource constraints and prompt explicitness affect model safety in Kazakh. Some models manage direct attacks better yet fail on indirect ones, while region-specific content remains challenging for all given the lack of localized training data.
% \subsubsection{Fine-grained Classification}
% Similarly, we conducted a human evaluation on 1,000 Kazakh samples, following the same methodology as the Russian evaluation. The match between human annotations and GPT-4o's fine-grained classifications was 707 out of 1,000, ensuring that the fine-grained classification framework aligned well with human judgments.
% The confusion matrix in Figure \ref{fig:human_fg_1000_kz} for 1,000 Kazakh samples illustrates the agreement between human annotations and GPT-4o's fine-grained classifications. The highest agreement is observed in category 5 (360 instances), indicating GPT-4o's strength in recognizing responses labeled by humans as "safe and general information." However, discrepancies are evident in categories such as 3 and 7, where GPT-4o misclassified several instances, highlighting areas for further refinement in distinguishing nuanced classifications.


\begin{figure}[t!]
	\centering
	\includegraphics[scale=0.18]{figures/human_1000_kz_font16.png} 
	\includegraphics[scale=0.18]{figures/human_1000_ru_font16.png}

	\caption{Human vs \gptfouro\ fine-grained labels on 1,000 Kazakh (left) and Russian (right) samples.}
	\label{fig:human_fg_1000}
\end{figure}

% \begin{figure}[t!]
% 	\centering
% 	\includegraphics[scale=0.28]{figures/human_1000_kz_font16.png} 
% 	\includegraphics[scale=0.28]{figures/human_1000_ru_font16.png}

% 	\caption{Human vs \gptfouro\ fine-grained labels on 1,000 Kazakh (top) and Russian (bottom) samples.}
% 	\label{fig:human_fg_1000}
% \end{figure}

% \begin{figure*}[t!]
% 	\centering
% 	\includegraphics[scale=0.28]{figures/all_5_kz_font16.png} 
% 	\includegraphics[scale=0.28]{figures/all_5_ru_font_16.png} \\
% 	\caption{Fine-grained responding pattern distribution across five models for Kazakh (left) and Russian (right).}
% 	\label{fig:all_5}
% \end{figure*}

\begin{figure}[t!]
	\centering
	\includegraphics[scale=0.28]{figures/all_5_kz_font16.png} 
	\includegraphics[scale=0.28]{figures/all_5_ru_font_16.png} \\
	\caption{Fine-grained responding pattern distribution across five models for Kazakh (top) and Russian (bottom).}
	\label{fig:all_5}
\end{figure}


\subsection{Fine-Grained Classification}
\label{sec:fine-grained-classification}
% As discussed in Section \ref{harmfulness_evaluation}, 
We further analyzed fine-grained responding patterns for safe and unsafe responses. For unsafe responses, outputs were categorized into four harm types, and safe responses were classified into six distinct patterns of safety, as rubric in Appendix \ref{safe_unsafe_response_categories}. 
% \cite{wang2024chinesedatasetevaluatingsafeguards}

\paragraph{Human vs. GPT-4o}
Similar to binary classification, we validated \gptfouro's automatic evaluation results by comparing with human annotations on 1,000 samples for both Russian and Kazakh. %, comparing human annotations with \gptfouro's fine-grained labels.
For the Russian dataset, \gptfouro's labels aligned with human annotations in 710 out of 1,000 cases, achieving an agreement rate of 71\%. 
Agreement rate of Kazakh samples is 70.7\%. %with 707 out of 1,000 cases matching
% The confusion matrix highlights areas of alignment and discrepancies.
% 
As confusion matrices illustrated in Figure~\ref{fig:human_fg_1000}, the majority of cases falling into \textit{safe responding patter 5} --- providing general and harmless information, for both human annotations and automatic predictions.
% highest agreement with 404 correct classifications for Russian. 
Mis-classifications for safe responses mainly focus on three closely-similar patterns: 3, 4, and 5, and patterns 7 and 8 are confusing to discern for unsafe responses, particularly for Kazakh (left figure).
We find many Russian samples which were labeled as ``1. reject to answer'' by humans are diversely classified across 1-6 by GPT-4o, which is also observed in Kazakh but not significant.

% suggesting label alignment issues are language-independent. 
% YX: I did not observe this, commented
% Notably, Russian showed confusion between 7 (risky instructions) and 1 (refusal to answer), this trend does not appear in Kazakh.


% highlight the strengths and limitations of {\gptfouro}'s fine-grained classification framework across both languages, paving the way for further refinements.


% However, misclassifications were observed in categories such as 1 (not willing to answer), 4 (recommends consulting an expert), and 7 (provides harmful or risky instructions), revealing overlaps and ambiguities in nuanced classifications.

% Similarly, for the Kazakh dataset, the agreement rate between human annotations and GPT-4o's labels was 70.7\%, with 707 out of 1,000 cases matching. As with the Russian analysis, category 5 (360 instances) showed the highest alignment. However, discrepancies were more prominent in categories such as 3 and 7, underscoring GPT-4o's challenges in differentiating fine-grained human-labeled categories. 
% A similar pattern was observed for Kazakh dataset, which suggests that alignment and misaligned of fine-grained lables is not language dependent.

% These findings, illustrated in Figures \ref{fig:human_fg_1000}, highlight the strengths and limitations of {\gptfouro}'s fine-grained classification framework across both languages, paving the way for further refinements.

\paragraph{Fine-grained Analysis of Five LLMs}
% After conducting human evaluation on representative samples, we extended 
\figref{fig:all_5} shows fine-grained responding pattern distribution across five models based on the full set of Russian and Kazakh data.
% For Russian, we selected \vikhr, \gptfouro, \llamaseventy, \claude, and \yandexgpt, while for Kazakh, we chose \aya, \gptfouro, \llamaseventy, \claude, and \yandexgpt. 
% The evaluation covered 21,915 responses in Russian and 18,930 responses in Kazakh.
% 
In both languages, pattern 5 of providing \textit{general and harmless information} consistently witnessed the highest frequency across all models, with \llamaseventy\ exhibiting the largest number of responses falling into this category for Kazakh (2,033). 
% YX:summarize more noteable findings here.

Differences of other patterns vary across languages. 
Unsafe responses in Russian are predominantly in pattern 8, where models provide incorrect or misleading information without expressing uncertainty. % (misinformation and speculation), 
For Kazakh, \aya\ exhibits the highest occurrence of pattern 7 (harmful or risky information) and pattern 8, indicating a stronger tendency to generate unethical, misleading, or potentially harmful content.

%Variations in other patterns across models highlight differences in how nuanced safety risks are classified, reflecting the models' differing capabilities in handling safety evaluation for these distinct linguistic contexts. For Russian, the majority of unsafe responses fall under pattern 8 (misinformation and speculation), indicating that models frequently provide incorrect or misleading information without acknowledging uncertainty. For Kazakh, \aya\ has the highest occurence of pattern 7 (harmful or risky information) and pattern 8 (misinformation and speculation), indicating a greater tendency to generate unethical, misleading, or potentially harmful content. 

%This trend suggests that Russian models may struggle more with factual accuracy, whereas Kazakh models, particularly \aya, pose higher risks related to both harmful content and misinformation. Additionally, \gptfouro\ and \claude\ consistently produce fewer unsafe responses in both languages, demonstrating stronger alignment with safety standards
\subsection{Code Switching}
\begin{table}[t!]
\centering

\setlength{\tabcolsep}{3pt}
\scalebox{0.7}{
\begin{tabular}{lcccccccccc}
\toprule
\textbf{Model Name} & \multicolumn{2}{c}{\textbf{Kazakh}} & \multicolumn{2}{c}{\textbf{Russian}} & \multicolumn{2}{c}{\textbf{Code-Switched}} \\  
\cmidrule(lr){2-3} \cmidrule(lr){4-5} \cmidrule(lr){6-7}
& \textbf{Safe} & \textbf{Unsafe} & \textbf{Safe} & \textbf{Unsafe} & \textbf{Safe} & \textbf{Unsafe} \\ 
\midrule
\llamaseventy & 450 & 50 & 466 & 34 & 414 & 86 \\
\gptfouro & 492 & 8 & 473 & 27 & 481 & 19
 \\
\claude & 491 & 9 & 478 & 22 & 484 & 16 \\ 
\yandexgpt & 435 & 65 & 458 & 42 & 464 & 36 \\
\midrule
\end{tabular}}
\caption{Model safety when prompted in Kazakh, Russian, and code-switched language.}
\label{tab:finetuning-comparison}
\end{table}


\gptfouro\ and \claude\ demonstrate strong safety performance across three languages, even with a high proportion of safe responses in the challenging code-switching context. In contrast, \llamaseventy\ and \yandexgpt\ are less safe, exhibiting more harmful responses, particularly in the code-switching scenario. These results show the varying capabilities of models in defending the same attacks that are just presented in different languages, where open-sourced large language models especially require more robust safety alignment in multilingual and code-switching scenarios.

% \subsection{LLM Response Collection}
% We conducted experiments with a variety of mainstream and region-specific 
% large language models for both Russian and Kazakh languages. For both Russian and Kazakh languages, we employed four multilingual models: Claude-3.5-sonnet, Llama 3.1 70B \cite{meta2024llama3}, GPT-4 \cite{openai2024gpt4o}, and YandexGPT. Additionally, we included language-specific models: VIKHR \cite{nikolich2024vikhrconstructingstateoftheartbilingual} for Russian and Aya \cite{ustun-etal-2024-aya} for Kazakh. 

% \subsection{Kazakh-Russian Code-Switching Evaluation}

% In Kazakhstan, the prevalence of bilingualism is a defining characteristic of its linguistic landscape, with most individuals seamlessly mixing Kazakh and Russian in daily communication \cite{Zharkynbekova2022}. This phenomenon, known as code-switching, reflects the unique cultural and social dynamics of the region. Despite this, there is currently no safety evaluation dataset tailored to this unique multilingual environment. Developing a code-switched dataset is essential to evaluate the ability of large language models (LLMs) to navigate the complexities of bilingual interactions, ensuring they produce contextually appropriate, non-harmful, and culturally sensitive responses. To address this, we sampled 500 questions from both Kazakh and Russian question sets, ensuring they represent similar queries but with culturally nuanced differences in meaning. Building on these, we developed a mixed version that integrates Kazakh and Russian naturally, reflecting the way people in Kazakhstan communicate. This process involved annotators to ensure that the mixed questions were crafted with a natural tone and were localized to Kazakhstan's context.


% \begin{table*}[t!]
%     \centering
%     \small
%     \setlength{\tabcolsep}{6pt} % Adjusts column separation
%     \adjustbox{max width=\linewidth}{
%     \begin{tabular}{ll ccc ccc c}
%     \toprule
%     \textbf{Risk Area} & \textbf{Harm Type} & \multicolumn{3}{c}{\textbf{Kazakh}} & \multicolumn{3}{c}{\textbf{Russian}} & \textbf{\#Q} \\
%                        &                    & \textbf{Ori} & \textbf{FN} & \textbf{FP} & \textbf{Ori} & \textbf{FN} & \textbf{FP} & \\
%     \midrule
%     \multirow{2}{*}{I. Information Hazards} & 1. Risks from leaking sensitive information... & 131 & 133 & 131 & 131 & 133 & 131 & 790 \\
%                                              & 2. Compromise of privacy by leaking or inferring private information (person/individual) & 81 & 82 & 81 & 81 & 82 & 81 & 488 \\
%     \midrule
%     \multirow{3}{*}{II. Malicious Uses} & 3. Assisting illegal activities & 132 & 135 & 132 & 132 & 135 & 132 & 798 \\
%                                         & 4. Nudging or advising unethical actions & 71 & 71 & 71 & 71 & 71 & 71 & 426 \\
%                                         & 5. Reducing the cost of disinformation campaigns & 40 & 42 & 40 & 40 & 42 & 40 & 244 \\
%     \midrule
%     \multirow{3}{*}{III. Discrimination, Exclusion, Toxicity, Hateful, Offensive} & 6. Social stereotypes and unfair discrimination & 94 & 96 & 94 & 94 & 96 & 94 & 568 \\
%                                         & 7. Toxic language (hate speech) & 52 & 59 & 52 & 52 & 59 & 52 & 326 \\
%                                         & 8. Adult content & 27 & 29 & 27 & 27 & 29 & 27 & 166 \\

%     \midrule
%     \multirow{2}{*}{IV. Misinformation Harms} & 9. Disseminating false or misleading information & 92 & 99 & 92 & 92 & 99 & 92 & 566 \\
%                                         & 10. Causing material harm by disseminating misinformation e.g. in medicine or law & 63 & 63 & 63 & 63 & 63 & 63 & 378 \\
%     \midrule
%     \multirow{2}{*}{V. Human–chatbot Interaction Harms} & 11. Mental health or overreliance concerns & 66 & 66 & 66 & 66 & 66 & 66 & 396 \\
%                                         & 12. Treating the chatbot as a human & 50 & 51 & 50 & 50 & 51 & 50 & 302 \\

%     \midrule
%     \multirow{2}{*}{VI. Kazakhstan-specific Sensitivity} & 13. Politically sensitive topics & 63 & 66 & 63& 63 & 66 & 63 &  384 \\
%         & 14. Controversial historical events & 46 & 57 & 46 & 46 & 57 & 46  & 298 \\
% & 15. Regional and racial issues & 45 & 45 & 45 &  45 & 45 & 45 & 270  \\
% & 16. Societal and cultural concerns & 138 & 139 & 138 &  138 & 139 & 138  & 830  \\
% & 17. Legal and human rights matters & 57 & 57 & 57 & 57 & 57 & 57  & 342 \\
%     \midrule
%         \multirow{2}{*}{VII. Russia-specific Sensitivity} 
%             & 13. Politically sensitive topics & - & - & - & 54 & 54 & 54 & 162 \\
%     & 14. Controversial historical events & - & - & - & 38 & 38 & 38 & 114 \\
%     & 15. Regional and racial issues & - & - & - & 26 & 26 & 26 & 78 \\
%     & 16. Societal and cultural concerns & - & - & - & 40 & 40 & 40 & 120 \\
%     & 17. Legal and human rights matters & - & - & - & 41 & 41 & 41 & 123 \\
%     \midrule
%     \bf Total & -- & 1248 & 1290 & 1248 & 1447 & 1489 & 1447 & \textbf{8169} \\
%     \bottomrule
%     \end{tabular}
%     }
%     \caption{The number of questions for Kazakh and Russian datasets across six risk areas and 17 harm types. Ori: original direct attack, FN: indirect attack, and FP: over-sensitivity assessment.}
%     \label{tab:kazakh-russian-data}
% \end{table*}




\section{Discussion}

% \subsection{Kazakh vs Russian}

% The evaluation reveals that Kazakh responses tend to be generally safer than their Russian counterparts, likely due to Kazakh being a low-resource language with significantly less training data. As a result, Kazakh models are less exposed to the vast, often unfiltered datasets containing harmful or unsafe content, which are more prevalent in high-resource languages like Russian. This data scarcity naturally limits the model's ability to generate nuanced but potentially unsafe responses. However, this does not mean the models are specifically fine-tuned for safer performance. When analyzing unsafe answers, it’s clear that Kazakh models, while safer overall, distribute their unsafe responses more evenly across various risk types and question types. This suggests Kazakh models generate fewer unsafe answers but in a broader range of contexts.

% In contrast, Russian models tend to concentrate unsafe answers in specific areas, particularly region-specific risks or indirect attacks. This indicates that Russian models have learned to handle certain types of unsafe content by focusing on specific topics, such as politically sensitive issues, but struggle when confronted with unfamiliar content, leading to unsafe responses due to insufficient filtering. Kazakh models, despite having less training data, tend to respond more broadly, including both direct and indirect risks. This could be due to the less curated nature of their training data, making them more likely to answer unsafe questions without filtering the potential harm involved. The exception to this trend is Aya, a model specifically fine-tuned for Kazakh. Despite fine-tuning, it exhibits the lowest safety percentage (72.37\%) in the Kazakh dataset, suggesting that fine-tuning in specific languages may introduce risks if proper safety measures are not taken.

% The evaluation reveals notable differences in the distribution of safe response patterns across Kazakh and Russian fine-grained labels. Refusal to answer is more frequent in Russian models, particularly Yandex-GPT, reflecting a cautious approach to safety-critical queries. Interestingly, Aya, despite being fine-tuned for Kazakh and exhibiting lower overall safety, also frequently refuses to answer, suggesting an over-reliance on conservative mechanisms. Responses providing general, safe information dominate in both languages, with Kazakh models displaying a slightly higher tendency to rely on this approach. This highlights how the low-resource nature of Kazakh results in more generalized and inherently safer responses. In contrast, Russian models excel at recognizing risks, issuing disclaimers, and refuting incorrect assumptions, likely benefiting from richer and more diverse training data.
% Yandex-GPT exhibits a notably high rate of responses classified under label 7, indicating an overreliance on general disclaimers or deflections, such as "check the internet" or "I don't know." While these responses minimize the risk of unsafe outputs, they often lack substantive or contextually relevant information, reducing their overall utility for users.


Most models perform safer on Kazakh dataset than Russian dataset, higher safe rate on Kazakh dataset in \tabref{tab:safety-binary-eval}. This does not necessarily reveal that current LLMs have better understanding and safety alignment on Kazakh language than Russian, while this may conversely imply that models do not fully understand the meaning of Kazakh attack questions, fail to perceive risks and then provide general information due to lacking sufficient knowledge regarding this request.

We observed the similar number of examples falling into category 5 \textit{general and harmless information} for both Kazakh and Russian, while the Kazakh data set size is 3.7K and Russian is 4.3K. Kazakh has much less examples in category 1 \textit{reject to answer} compared to Russian. This demonstrate models tend to provide general information and cannot clearly perceive risks for many cases.

Additionally, in spite of less harmful responses on Kazakh data, these unsafe responses distribute evenly across different risk areas and question categories, exhibiting equally vulnerability spanning all attacks regardless of what risks and how we jailbreak it.
In contrary, unsafe responses on Russian dataset often concentrate on specific areas and question types, such as region-specific risks or indirect attacks, presenting similar model behaviors when evaluating over English and Chinese data.
It suggests that broader training data in English, Chinese and Russian may allow models to address certain types of attacks robustly,
% effectively—particularly politically sensitive issues—
yet they may falter when confronted with unfamiliar content like regional sensitive topics.

Moreover, in responses collection, we observed many Russian or English responses especially for open-sourced LLMs when we explicitly instructed the models to answer Kazakh questions in Kazakh language. This further implies more efforts are still needed to improve LLMs' performance on low-resource languages.
Interestingly, \aya, a fine-tuned Kazakh model, proves an exception by displaying the lowest safety percentage (72.37\%) among Kazakh models, revealing that the multilingual fine-tuning without stringent safety measures can introduce risks.



% However, this does not mean they are explicitly fine-tuned for safety, likely it happens due to limited training data, which reduces exposure to harmful content. 
% \aya, a fine-tuned Kazakh model, proves an exception by displaying the lowest safety percentage (72.37\%) among Kazakh models, revealing that the multilingual fine-tuning without stringent safety measures can introduce risks.
% Kazakh models generally produce safer responses than their Russian counterparts, likely because Kazakh is a low-resource language with less training data. 
% This limited exposure to harmful or unsafe content naturally limits nuanced yet potentially unsafe outputs. 
% However, it does not imply that the models are specifically fine-tuned for enhanced safety.


% while Kazakh models tend to generate fewer unsafe answers overall, those unsafe responses appear more evenly spread across different risk types and question categories.
% Russian models, on the other hand, often concentrate unsafe responses in specific areas, such as region-specific risks or indirect attacks.
% It implies that their broader training datasets allow them to address certain types of unsafe content more effectively—particularly politically sensitive issues—yet they may falter when confronted with unfamiliar or insufficiently filtered content.

% Meanwhile, Kazakh models sometimes respond more broadly, possibly due to less curated training data. 

Differences also emerge in how language models handle safe responses. 
\yandexgpt, for instance, often refuses to answer high-risk queries. 
It frequently relies on generic disclaimers or deflections like ``check in the Internet'' or ``I don’t know,'' minimizing risk but are less helpful. Interestingly, it often responds with ``I don’t know'' in Russian, even for Kazakh queries, we speculate that these may be default responses stemming from internal system filters, rather than generated by model itself.
This likely explains why \yandexgpt\ is the safest model for the Russian language but ranks third for Kazakh. While its filters perform well for Russian, they struggle with the low-resource Kazakh language.

% Aya, despite its lower overall safety, also employs refusals often, hinting at an over-reliance on conservative approaches. 

% Across both languages, models commonly resort to providing general, safe information, although Kazakh models lean on this strategy slightly more. 
% Russian models, by contrast, excel at detecting risks, issuing disclaimers, and correcting inaccuracies, likely benefiting from richer and more diverse training data.


% \subsection{Response Patterns}


% We conducted a detailed analysis of the models' outputs and identified several noteworthy patterns. YandexGPT, while being one of the safest overall, frequently generates responses in Russian even when the question is posed in Kazakh. These responses often appear as placeholders, prompting users to search for the answer online. This behavior might not originate from the model itself but rather from safety filters implemented in the YandexGPT system. The model's leading performance in ensuring safety during Russian-language interactions, coupled with its lower performance in Kazakh, can be attributed to the limited robustness of these safety filters when handling unsafe content in Kazakh.

% In contrast, Aya-101 exhibits a tendency to fall into repetition, often repeating the same sentences multiple times. Interestingly, the Vikhr model, despite being of a similar size, does not exhibit this issue. We attribute this difference to two key factors. First, Vikhr and Aya-101 have distinct architectures: Vikhr is based on the Mistral-Nemo model, whereas Aya-101 is built on mT5, an older and less robust model. Second, Aya-101 is a multilingual model, while Vikhr was predominantly trained for Russian. Multilingualism has been shown to potentially degrade performance in large language models~\cite{huang2025surveylargelanguagemodels}, which may explain Aya-101's issues with repetition.

\section{Conclusions}
We presented SPARC, a zero-shot multi-label recognition approach that improves VLM performance without training data or architectural changes. Our key insights - the suboptimal nature of maximum scores and presence of systematic biases - led to two complementary innovations: compound prompts with adaptive fusion, and systematic debiasing. Beyond improving standalone VLM performance, SPARC enhances existing zero-shot and training-based methods while maintaining a black-box approach. The success of our method reveals fundamental properties of VLM scoring behavior, suggesting promising directions for improving zero-shot recognition through score analysis rather than architectural modification or prompt engineering.
\clearpage



{
    \small
    \bibliographystyle{ieeenat_fullname}
    \bibliography{main}
}

\clearpage
\pagenumbering{gobble}
\maketitlesupplementary

\section{Additional Results on Embodied Tasks}

To evaluate the broader applicability of our EgoAgent's learned representation beyond video-conditioned 3D human motion prediction, we test its ability to improve visual policy learning for embodiments other than the human skeleton.
Following the methodology in~\cite{majumdar2023we}, we conduct experiments on the TriFinger benchmark~\cite{wuthrich2020trifinger}, which involves a three-finger robot performing two tasks: reach cube and move cube. 
We freeze the pretrained representations and use a 3-layer MLP as the policy network, training each task with 100 demonstrations.

\begin{table}[h]
\centering
\caption{Success rate (\%) on the TriFinger benchmark, where each model's pretrained representation is fixed, and additional linear layers are trained as the policy network.}
\label{tab:trifinger}
\resizebox{\linewidth}{!}{%
\begin{tabular}{llcc}
\toprule
Methods       & Training Dataset & Reach Cube & Move Cube \\
\midrule
DINO~\cite{caron2021emerging}         & WT Venice        & 78.03     & 47.42     \\
DoRA~\cite{venkataramanan2023imagenet}          & WT Venice        & 81.62     & 53.76     \\
DoRA~\cite{venkataramanan2023imagenet}          & WT All           & 82.40     & 48.13     \\
\midrule
EgoAgent-300M & WT+Ego-Exo4D      & 82.61    & 54.21      \\
EgoAgent-1B   & WT+Ego-Exo4D      & \textbf{85.72}      & \textbf{57.66}   \\
\bottomrule
\end{tabular}%
}
\end{table}

As shown in Table~\ref{tab:trifinger}, EgoAgent achieves the highest success rates on both tasks, outperforming the best models from DoRA~\cite{venkataramanan2023imagenet} with increases of +3.32\% and +3.9\% respectively.
This result shows that by incorporating human action prediction into the learning process, EgoAgent demonstrates the ability to learn more effective representations that benefit both image classification and embodied manipulation tasks.
This highlights the potential of leveraging human-centric motion data to bridge the gap between visual understanding and actionable policy learning.



\section{Additional Results on Egocentric Future State Prediction}

In this section, we provide additional qualitative results on the egocentric future state prediction task. Additionally, we describe our approach to finetune video diffusion model on the Ego-Exo4D dataset~\cite{grauman2024ego} and generate future video frames conditioned on initial frames as shown in Figure~\ref{fig:opensora_finetune}.

\begin{figure}[b]
    \centering
    \includegraphics[width=\linewidth]{figures/opensora_finetune.pdf}
    \caption{Comparison of OpenSora V1.1 first-frame-conditioned video generation results before and after finetuning on Ego-Exo4D. Fine-tuning enhances temporal consistency, but the predicted pixel-space future states still exhibit errors, such as inaccuracies in the basketball's trajectory.}
    \label{fig:opensora_finetune}
\end{figure}

\subsection{Visualizations and Comparisons}

More visualizations of our method, DoRA, and OpenSora in different scenes (as shown in Figure~\ref{fig:supp pred}). For OpenSora, when predicting the states of $t_k$, we use all the ground truth frames from $t_{0}$ to $t_{k-1}$ as conditions. As OpenSora takes only past observations as input and neglects human motion, it performs well only when the human has relatively small motions (see top cases in Figure~\ref{fig:supp pred}), but can not adjust to large movements of the human body or quick viewpoint changes (see bottom cases in Figure~\ref{fig:supp pred}).

\begin{figure*}
    \centering
    \includegraphics[width=\linewidth]{figures/supp_pred.pdf}
    \caption{Retrieval and generation results for egocentric future state prediction. Correct and wrong retrieval images are marked with green and red boundaries, respectively.}
    \label{fig:supp pred}
\end{figure*}

\begin{figure*}[t]
    \centering
    \includegraphics[width=0.9\linewidth]{figures/motion_prediction.pdf}
    \vspace{-0.5mm}
    \caption{Motion prediction results in scenes with minor changes in observation.}
    \vspace{-1.5mm}
    \label{fig:motion_prediction}
\end{figure*}

\subsection{Finetuning OpenSora on Ego-Exo4D}

OpenSora V1.1~\cite{opensora}, initially trained on internet videos and images, produces severely inconsistent results when directly applied to infer future videos on the Ego-Exo4D dataset, as illustrated in Figure~\ref{fig:opensora_finetune}.
To address the gap between general internet content and egocentric video data, we fine-tune the official checkpoint on the Ego-Exo4D training set for 50 epochs.
OpenSora V1.1 proposed a random mask strategy during training to enable video generation by image and video conditioning. We adopted the default masking rate, which applies: 75\% with no masking, 2.5\% with random masking of 1 frame to 1/4 of the total frames, 2.5\% with masking at either the beginning or the end for 1 frame to 1/4 of the total frames, and 5\% with random masking spanning 1 frame to 1/4 of the total frames at both the beginning and the end.

As shown in Fig.~\ref{fig:opensora_finetune}, despite being trained on a large dataset, OpenSora struggles to generalize to the Ego-Exo4D dataset, producing future video frames with minimal consistency relative to the conditioning frame. While fine-tuning improves temporal consistency, the moving trajectories of objects like the basketball and soccer ball still deviate from realistic physical laws. Compared with our feature space prediction results, this suggests that training world models in a reconstructive latent space is more challenging than training them in a feature space.


\section{Additional Results on 3D Human Motion Prediction}

We present additional qualitative results for the 3D human motion prediction task, highlighting a particularly challenging scenario where egocentric observations exhibit minimal variation. This scenario poses significant difficulties for video-conditioned motion prediction, as the model must effectively capture and interpret subtle changes. As demonstrated in Fig.~\ref{fig:motion_prediction}, EgoAgent successfully generates accurate predictions that closely align with the ground truth motion, showcasing its ability to handle fine-grained temporal dynamics and nuanced contextual cues.

\section{OpenSora for Image Classification}

In this section, we detail the process of extracting features from OpenSora V1.1~\cite{opensora} (without fine-tuning) for an image classification task. Following the approach of~\cite{xiang2023denoising}, we leverage the insight that diffusion models can be interpreted as multi-level denoising autoencoders. These models inherently learn linearly separable representations within their intermediate layers, without relying on auxiliary encoders. The quality of the extracted features depends on both the layer depth and the noise level applied during extraction.


\begin{table}[h]
\centering
\caption{$k$-NN evaluation results of OpenSora V1.1 features from different layer depths and noising scales on ImageNet-100. Top1 and Top5 accuracy (\%) are reported.}
\label{tab:opensora-knn}
\resizebox{0.95\linewidth}{!}{%
\begin{tabular}{lcccccc}
\toprule
\multirow{2}{*}{Timesteps} & \multicolumn{2}{c}{First Layer} & \multicolumn{2}{c}{Middle Layer} & \multicolumn{2}{c}{Last Layer} \\
\cmidrule(r){2-3}   \cmidrule(r){4-5}  \cmidrule(r){6-7}  & Top1           & Top5           & Top1            & Top5           & Top1           & Top5          \\
\midrule
32        &  6.10           & 18.20             & 34.04               & 59.50             & 30.40             & 55.74             \\
64        & 6.12              & 18.48              & 36.04               & 61.84              & 31.80         & 57.06         \\
128       & 5.84             & 18.14             & 38.08               & 64.16              & 33.44       & 58.42 \\
256       & 5.60             & 16.58              & 30.34               & 56.38              &28.14          & 52.32        \\
512       & 3.66              & 11.70            & 6.24              & 17.62              & 7.24              & 19.44  \\ 
\bottomrule
\end{tabular}%
}
\end{table}

As shown in Table~\ref{tab:opensora-knn}, we first evaluate $k$-NN classification performance on the ImageNet-100 dataset using three intermediate layers and five different noise scales. We find that a noise timestep of 128 yields the best results, with the middle and last layers performing significantly better than the first layer.
We then test this optimal configuration on ImageNet-1K and find that the last layer with 128 noising timesteps achieves the best classification accuracy.

\section{Data Preprocess}
For egocentric video sequences, we utilize videos from the Ego-Exo4D~\cite{grauman2024ego} and WT~\cite{venkataramanan2023imagenet} datasets.
The original resolution of Ego-Exo4D videos is 1408×1408, captured at 30 fps. We sample one frame every five frames and use the original resolution to crop local views (224×224) for computing the self-supervised representation loss. For computing the prediction and action loss, the videos are downsampled to 224×224 resolution.
WT primarily consists of 4K videos (3840×2160) recorded at 60 or 30 fps. Similar to Ego-Exo4D, we use the original resolution and downsample the frame rate to 6 fps for representation loss computation.
As Ego-Exo4D employs fisheye cameras, we undistort the images to a pinhole camera model using the official Project Aria Tools to align them with the WT videos.

For motion sequences, the Ego-Exo4D dataset provides synchronized 3D motion annotations and camera extrinsic parameters for various tasks and scenes. While some annotations are manually labeled, others are automatically generated using 3D motion estimation algorithms from multiple exocentric views. To maximize data utility and maintain high-quality annotations, manual labels are prioritized wherever available, and automated annotations are used only when manual labels are absent.
Each pose is converted into the egocentric camera's coordinate system using transformation matrices derived from the camera extrinsics. These matrices also enable the computation of trajectory vectors for each frame in a sequence. Beyond the x, y, z coordinates, a visibility dimension is appended to account for keypoints invisible to all exocentric views. Finally, a sliding window approach segments sequences into fixed-size windows to serve as input for the model. Note that we do not downsample the frame rate of 3D motions.

\section{Training Details}
\subsection{Architecture Configurations}
In Table~\ref{tab:arch}, we provide detailed architecture configurations for EgoAgent following the scaling-up strategy of InternLM~\cite{team2023internlm}. To ensure the generalization, we do not modify the internal modules in InternML, \emph{i.e.}, we adopt the RMSNorm and 1D RoPE. We show that, without specific modules designed for vision tasks, EgoAgent can perform well on vision and action tasks.

\begin{table}[ht]
  \centering
  \caption{Architecture configurations of EgoAgent.}
  \resizebox{0.8\linewidth}{!}{%
    \begin{tabular}{lcc}
    \toprule
          & EgoAgent-300M & EgoAgent-1B \\
          \midrule
    Depth & 22    & 22 \\
    Embedding dim & 1024  & 2048 \\
    Number of heads & 8     & 16 \\
    MLP ratio &    8/3   & 8/3 \\
    $\#$param.  & 284M & 1.13B \\
    \bottomrule
    \end{tabular}%
    }
  \label{tab:arch}%
\end{table}%

Table~\ref{tab:io_structure} presents the detailed configuration of the embedding and prediction modules in EgoAgent, including the image projector ($\text{Proj}_i$), representation head/state prediction head ($\text{MLP}_i$), action projector ($\text{Proj}_a$) and action prediction head ($\text{MLP}_a$).
Note that the representation head and the state prediction head share the same architecture but have distinct weights.

\begin{table}[t]
\centering
\caption{Architecture of the embedding ($\text{Proj}_i$, $\text{Proj}_a$) and prediction ($\text{MLP}_i$, $\text{MLP}_a$) modules in EgoAgent. For details on module connections and functions, please refer to Fig.~2 in the main paper.}
\label{tab:io_structure}
\resizebox{\linewidth}{!}{%
\begin{tabular}{lcl}
\toprule
       & \multicolumn{1}{c}{Norm \& Activation} & \multicolumn{1}{c}{Output Shape}  \\
\midrule
\multicolumn{3}{l}{$\text{Proj}_i$ (\textit{Image projector})} \\
\midrule
Input image  & -          & 3$\times$224$\times$224 \\
Conv 2D (16$\times$16) & -       & Embedding dim$\times$14$\times$14    \\
\midrule
\multicolumn{3}{l}{$\text{MLP}_i$ (\textit{State prediction head} \& \textit{Representation head)}} \\
\midrule
Input embedding  & -          & Embedding dim \\
Linear & GELU       & 2048          \\
Linear & GELU       & 2048          \\
Linear & -          & 256           \\
Linear & -          & 65536     \\
\midrule
\multicolumn{3}{l}{$\text{Proj}_a$ (\textit{Action projector})} \\
\midrule
Input pose sequence  & -          & 4$\times$5$\times$17 \\
Conv 2D (5$\times$17) & LN, GELU   & Embedding dim$\times$1$\times$1    \\
\midrule
\multicolumn{3}{l}{$\text{MLP}_a$ (\textit{Action prediction head})} \\
\midrule
Input embedding  & -          & Embedding dim$\times$1$\times$1 \\
Linear & -          & 4$\times$5$\times$17     \\
\bottomrule
\end{tabular}%
}
\end{table}


\subsection{Training Configurations}
In Table~\ref{tab:training hyper}, we provide the detailed training hyper-parameters for experiments in the main manuscripts.

\begin{table}[ht]
  \centering
  \caption{Hyper-parameters for training EgoAgent.}
  \resizebox{0.86\linewidth}{!}{%
    \begin{tabular}{lc}
    \toprule
    Training Configuration & EgoAgent-300M/1B \\
    \midrule
    Training recipe: &  \\
    optimizer & AdamW~\cite{loshchilov2017decoupled} \\
    optimizer momentum & $\beta_1=0.9, \beta_2=0.999$ \\
    \midrule
    Learning hyper-parameters: &  \\
    base learning rate & 6.0E-04 \\
    learning rate schedule & cosine \\
    base weight decay & 0.04 \\
    end weight decay & 0.4 \\
    batch size & 1920 \\
    training iters & 72,000 \\
    lr warmup iters & 1,800 \\
    warmup schedule & linear \\
    gradient clip & 1.0 \\
    data type & float16 \\
    norm epsilon & 1.0E-06 \\
    \midrule
    EMA hyper-parameters: &  \\
    momentum & 0.996 \\
    \bottomrule
    \end{tabular}%
    }
  \label{tab:training hyper}%
\end{table}%

\clearpage


\end{document}
