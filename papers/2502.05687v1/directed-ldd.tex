
\documentclass[letterpaper,11pt]{article}

\usepackage{style}
\usepackage{shortcuts}

\begin{document}
	
\begin{center}
	\begin{minipage}[H]{14.5cm} 
		
		\begin{center}
		{\huge \bf Near-Optimal Directed\\[.5ex]Low-Diameter Decompositions}
			\end{center}
		\vspace{1cm}
		
		{\large \textbf{Karl Bringmann}\footnote{This work is part of the project TIPEA that has received funding from the European Research Council (ERC) under the European Unions Horizon 2020 research and innovation programme (grant agreement No.~850979).} -- Saarland University and Max Planck Institute for Informatics, Saarland Informatics Campus, Germany} \vspace{1mm}\\
		{\large \textbf{Nick Fischer\footnote{Partially funded by the Ministry of Education and Science of Bulgaria's support for INSAIT as part of the Bulgarian National Roadmap for Research Infrastructure.}} -- INSAIT, Sofia University "St. Kliment Ohridski", Bulgaria} \vspace{1mm}\\
		{\large \textbf{Bernhard Haeupler\footnote{Partially funded by the Ministry of Education and Science of Bulgaria's support for INSAIT as part of the Bulgarian National Roadmap for Research Infrastructure and through the European Research Council (ERC) under the European Union's Horizon 2020 research and innovation program (ERC grant agreement 949272).}} -- INSAIT, Sofia University "St. Kliment Ohridski", Bulgaria \& ETH Zürich, Switzerland} \vspace{1mm}\\
		{\large \textbf{Rustam Latypov\footnote{Supported by the Research Council of Finland (grant 334238), The Finnish Foundation for Technology Promotion and The Nokia Foundation}} -- Aalto University, Finland} \vspace{1mm}\\


		\begin{abstract}
			\noindent
			Low Diameter Decompositions (LDDs) are invaluable tools in the design of combinatorial graph algorithms. While  historically they have been applied mainly to undirected graphs, in the recent breakthrough for the negative-length Single Source Shortest Path problem, Bernstein, Nanongkai, and Wulff-Nilsen [FOCS '22] extended the use of LDDs to directed graphs for the first time. Specifically, their LDD deletes each edge with probability at most \makebox{$O(\frac{1}{D} \cdot \log^2 n)$}, while ensuring that each strongly connected component in the remaining graph has a (weak) diameter of at most $D$.
			
			In this work, we make further advancements in the study of directed LDDs. We reveal a natural and intuitive (in hindsight) connection to Expander Decompositions, and leveraging this connection along with additional techniques, we establish the existence of an LDD with an edge-cutting probability of \makebox{$O(\frac{1}{D} \cdot \log n \log\log n)$}. This improves the previous bound by nearly a logarithmic factor and closely approaches the lower bound of \makebox{$\Omega(\frac{1}{D} \cdot \log n)$}. With significantly more technical effort, we also develop two efficient algorithms for computing our LDDs: a deterministic algorithm that runs in time~\smash{$\widetilde\Order(m \poly(D))$} and a randomized algorithm that runs in near-linear time~\smash{$\widetilde\Order(m)$}.

			We believe that our work provides a solid conceptual and technical foundation for future research relying on directed LDDs, which will undoubtedly follow soon.
		\end{abstract}
		
		\end{minipage}
	
\end{center}

\thispagestyle{empty}

%\newpage
%\thispagestyle{empty}
%\tableofcontents

\newpage
\pagenumbering{arabic}

\section{Introduction}


\begin{figure}[t]
\centering
\includegraphics[width=0.6\columnwidth]{figures/evaluation_desiderata_V5.pdf}
\vspace{-0.5cm}
\caption{\systemName is a platform for conducting realistic evaluations of code LLMs, collecting human preferences of coding models with real users, real tasks, and in realistic environments, aimed at addressing the limitations of existing evaluations.
}
\label{fig:motivation}
\end{figure}

\begin{figure*}[t]
\centering
\includegraphics[width=\textwidth]{figures/system_design_v2.png}
\caption{We introduce \systemName, a VSCode extension to collect human preferences of code directly in a developer's IDE. \systemName enables developers to use code completions from various models. The system comprises a) the interface in the user's IDE which presents paired completions to users (left), b) a sampling strategy that picks model pairs to reduce latency (right, top), and c) a prompting scheme that allows diverse LLMs to perform code completions with high fidelity.
Users can select between the top completion (green box) using \texttt{tab} or the bottom completion (blue box) using \texttt{shift+tab}.}
\label{fig:overview}
\end{figure*}

As model capabilities improve, large language models (LLMs) are increasingly integrated into user environments and workflows.
For example, software developers code with AI in integrated developer environments (IDEs)~\citep{peng2023impact}, doctors rely on notes generated through ambient listening~\citep{oberst2024science}, and lawyers consider case evidence identified by electronic discovery systems~\citep{yang2024beyond}.
Increasing deployment of models in productivity tools demands evaluation that more closely reflects real-world circumstances~\citep{hutchinson2022evaluation, saxon2024benchmarks, kapoor2024ai}.
While newer benchmarks and live platforms incorporate human feedback to capture real-world usage, they almost exclusively focus on evaluating LLMs in chat conversations~\citep{zheng2023judging,dubois2023alpacafarm,chiang2024chatbot, kirk2024the}.
Model evaluation must move beyond chat-based interactions and into specialized user environments.



 

In this work, we focus on evaluating LLM-based coding assistants. 
Despite the popularity of these tools---millions of developers use Github Copilot~\citep{Copilot}---existing
evaluations of the coding capabilities of new models exhibit multiple limitations (Figure~\ref{fig:motivation}, bottom).
Traditional ML benchmarks evaluate LLM capabilities by measuring how well a model can complete static, interview-style coding tasks~\citep{chen2021evaluating,austin2021program,jain2024livecodebench, white2024livebench} and lack \emph{real users}. 
User studies recruit real users to evaluate the effectiveness of LLMs as coding assistants, but are often limited to simple programming tasks as opposed to \emph{real tasks}~\citep{vaithilingam2022expectation,ross2023programmer, mozannar2024realhumaneval}.
Recent efforts to collect human feedback such as Chatbot Arena~\citep{chiang2024chatbot} are still removed from a \emph{realistic environment}, resulting in users and data that deviate from typical software development processes.
We introduce \systemName to address these limitations (Figure~\ref{fig:motivation}, top), and we describe our three main contributions below.


\textbf{We deploy \systemName in-the-wild to collect human preferences on code.} 
\systemName is a Visual Studio Code extension, collecting preferences directly in a developer's IDE within their actual workflow (Figure~\ref{fig:overview}).
\systemName provides developers with code completions, akin to the type of support provided by Github Copilot~\citep{Copilot}. 
Over the past 3 months, \systemName has served over~\completions suggestions from 10 state-of-the-art LLMs, 
gathering \sampleCount~votes from \userCount~users.
To collect user preferences,
\systemName presents a novel interface that shows users paired code completions from two different LLMs, which are determined based on a sampling strategy that aims to 
mitigate latency while preserving coverage across model comparisons.
Additionally, we devise a prompting scheme that allows a diverse set of models to perform code completions with high fidelity.
See Section~\ref{sec:system} and Section~\ref{sec:deployment} for details about system design and deployment respectively.



\textbf{We construct a leaderboard of user preferences and find notable differences from existing static benchmarks and human preference leaderboards.}
In general, we observe that smaller models seem to overperform in static benchmarks compared to our leaderboard, while performance among larger models is mixed (Section~\ref{sec:leaderboard_calculation}).
We attribute these differences to the fact that \systemName is exposed to users and tasks that differ drastically from code evaluations in the past. 
Our data spans 103 programming languages and 24 natural languages as well as a variety of real-world applications and code structures, while static benchmarks tend to focus on a specific programming and natural language and task (e.g. coding competition problems).
Additionally, while all of \systemName interactions contain code contexts and the majority involve infilling tasks, a much smaller fraction of Chatbot Arena's coding tasks contain code context, with infilling tasks appearing even more rarely. 
We analyze our data in depth in Section~\ref{subsec:comparison}.



\textbf{We derive new insights into user preferences of code by analyzing \systemName's diverse and distinct data distribution.}
We compare user preferences across different stratifications of input data (e.g., common versus rare languages) and observe which affect observed preferences most (Section~\ref{sec:analysis}).
For example, while user preferences stay relatively consistent across various programming languages, they differ drastically between different task categories (e.g. frontend/backend versus algorithm design).
We also observe variations in user preference due to different features related to code structure 
(e.g., context length and completion patterns).
We open-source \systemName and release a curated subset of code contexts.
Altogether, our results highlight the necessity of model evaluation in realistic and domain-specific settings.





\section{Overview}

\revision{In this section, we first explain the foundational concept of Hausdorff distance-based penetration depth algorithms, which are essential for understanding our method (Sec.~\ref{sec:preliminary}).
We then provide a brief overview of our proposed RT-based penetration depth algorithm (Sec.~\ref{subsec:algo_overview}).}



\section{Preliminaries }
\label{sec:Preliminaries}

% Before we introduce our method, we first overview the important basics of 3D dynamic human modeling with Gaussian splatting. Then, we discuss the diffusion-based 3d generation techniques, and how they can be applied to human modeling.
% \ZY{I stopp here. TBC.}
% \subsection{Dynamic human modeling with Gaussian splatting}
\subsection{3D Gaussian Splatting}
3D Gaussian splatting~\cite{kerbl3Dgaussians} is an explicit scene representation that allows high-quality real-time rendering. The given scene is represented by a set of static 3D Gaussians, which are parameterized as follows: Gaussian center $x\in {\mathbb{R}^3}$, color $c\in {\mathbb{R}^3}$, opacity $\alpha\in {\mathbb{R}}$, spatial rotation in the form of quaternion $q\in {\mathbb{R}^4}$, and scaling factor $s\in {\mathbb{R}^3}$. Given these properties, the rendering process is represented as:
\begin{equation}
  I = Splatting(x, c, s, \alpha, q, r),
  \label{eq:splattingGA}
\end{equation}
where $I$ is the rendered image, $r$ is a set of query rays crossing the scene, and $Splatting(\cdot)$ is a differentiable rendering process. We refer readers to Kerbl et al.'s paper~\cite{kerbl3Dgaussians} for the details of Gaussian splatting. 



% \ZY{I would suggest move this part to the method part.}
% GaissianAvatar is a dynamic human generation model based on Gaussian splitting. Given a sequence of RGB images, this method utilizes fitted SMPLs and sampled points on its surface to obtain a pose-dependent feature map by a pose encoder. The pose-dependent features and a geometry feature are fed in a Gaussian decoder, which is employed to establish a functional mapping from the underlying geometry of the human form to diverse attributes of 3D Gaussians on the canonical surfaces. The parameter prediction process is articulated as follows:
% \begin{equation}
%   (\Delta x,c,s)=G_{\theta}(S+P),
%   \label{eq:gaussiandecoder}
% \end{equation}
%  where $G_{\theta}$ represents the Gaussian decoder, and $(S+P)$ is the multiplication of geometry feature S and pose feature P. Instead of optimizing all attributes of Gaussian, this decoder predicts 3D positional offset $\Delta{x} \in {\mathbb{R}^3}$, color $c\in\mathbb{R}^3$, and 3D scaling factor $ s\in\mathbb{R}^3$. To enhance geometry reconstruction accuracy, the opacity $\alpha$ and 3D rotation $q$ are set to fixed values of $1$ and $(1,0,0,0)$ respectively.
 
%  To render the canonical avatar in observation space, we seamlessly combine the Linear Blend Skinning function with the Gaussian Splatting~\cite{kerbl3Dgaussians} rendering process: 
% \begin{equation}
%   I_{\theta}=Splatting(x_o,Q,d),
%   \label{eq:splatting}
% \end{equation}
% \begin{equation}
%   x_o = T_{lbs}(x_c,p,w),
%   \label{eq:LBS}
% \end{equation}
% where $I_{\theta}$ represents the final rendered image, and the canonical Gaussian position $x_c$ is the sum of the initial position $x$ and the predicted offset $\Delta x$. The LBS function $T_{lbs}$ applies the SMPL skeleton pose $p$ and blending weights $w$ to deform $x_c$ into observation space as $x_o$. $Q$ denotes the remaining attributes of the Gaussians. With the rendering process, they can now reposition these canonical 3D Gaussians into the observation space.



\subsection{Score Distillation Sampling}
Score Distillation Sampling (SDS)~\cite{poole2022dreamfusion} builds a bridge between diffusion models and 3D representations. In SDS, the noised input is denoised in one time-step, and the difference between added noise and predicted noise is considered SDS loss, expressed as:

% \begin{equation}
%   \mathcal{L}_{SDS}(I_{\Phi}) \triangleq E_{t,\epsilon}[w(t)(\epsilon_{\phi}(z_t,y,t)-\epsilon)\frac{\partial I_{\Phi}}{\partial\Phi}],
%   \label{eq:SDSObserv}
% \end{equation}
\begin{equation}
    \mathcal{L}_{\text{SDS}}(I_{\Phi}) \triangleq \mathbb{E}_{t,\epsilon} \left[ w(t) \left( \epsilon_{\phi}(z_t, y, t) - \epsilon \right) \frac{\partial I_{\Phi}}{\partial \Phi} \right],
  \label{eq:SDSObservGA}
\end{equation}
where the input $I_{\Phi}$ represents a rendered image from a 3D representation, such as 3D Gaussians, with optimizable parameters $\Phi$. $\epsilon_{\phi}$ corresponds to the predicted noise of diffusion networks, which is produced by incorporating the noise image $z_t$ as input and conditioning it with a text or image $y$ at timestep $t$. The noise image $z_t$ is derived by introducing noise $\epsilon$ into $I_{\Phi}$ at timestep $t$. The loss is weighted by the diffusion scheduler $w(t)$. 
% \vspace{-3mm}

\subsection{Overview of the RTPD Algorithm}\label{subsec:algo_overview}
Fig.~\ref{fig:Overview} presents an overview of our RTPD algorithm.
It is grounded in the Hausdorff distance-based penetration depth calculation method (Sec.~\ref{sec:preliminary}).
%, similar to that of Tang et al.~\shortcite{SIG09HIST}.
The process consists of two primary phases: penetration surface extraction and Hausdorff distance calculation.
We leverage the RTX platform's capabilities to accelerate both of these steps.

\begin{figure*}[t]
    \centering
    \includegraphics[width=0.8\textwidth]{Image/overview.pdf}
    \caption{The overview of RT-based penetration depth calculation algorithm overview}
    \label{fig:Overview}
\end{figure*}

The penetration surface extraction phase focuses on identifying the overlapped region between two objects.
\revision{The penetration surface is defined as a set of polygons from one object, where at least one of its vertices lies within the other object. 
Note that in our work, we focus on triangles rather than general polygons, as they are processed most efficiently on the RTX platform.}
To facilitate this extraction, we introduce a ray-tracing-based \revision{Point-in-Polyhedron} test (RT-PIP), significantly accelerated through the use of RT cores (Sec.~\ref{sec:RT-PIP}).
This test capitalizes on the ray-surface intersection capabilities of the RTX platform.
%
Initially, a Geometry Acceleration Structure (GAS) is generated for each object, as required by the RTX platform.
The RT-PIP module takes the GAS of one object (e.g., $GAS_{A}$) and the point set of the other object (e.g., $P_{B}$).
It outputs a set of points (e.g., $P_{\partial B}$) representing the penetration region, indicating their location inside the opposing object.
Subsequently, a penetration surface (e.g., $\partial B$) is constructed using this point set (e.g., $P_{\partial B}$) (Sec.~\ref{subsec:surfaceGen}).
%
The generated penetration surfaces (e.g., $\partial A$ and $\partial B$) are then forwarded to the next step. 

The Hausdorff distance calculation phase utilizes the ray-surface intersection test of the RTX platform (Sec.~\ref{sec:RT-Hausdorff}) to compute the Hausdorff distance between two objects.
We introduce a novel Ray-Tracing-based Hausdorff DISTance algorithm, RT-HDIST.
It begins by generating GAS for the two penetration surfaces, $P_{\partial A}$ and $P_{\partial B}$, derived from the preceding step.
RT-HDIST processes the GAS of a penetration surface (e.g., $GAS_{\partial A}$) alongside the point set of the other penetration surface (e.g., $P_{\partial B}$) to compute the penetration depth between them.
The algorithm operates bidirectionally, considering both directions ($\partial A \to \partial B$ and $\partial B \to \partial A$).
The final penetration depth between the two objects, A and B, is determined by selecting the larger value from these two directional computations.

%In the Hausdorff distance calculation step, we compute the Hausdorff distance between given two objects using a ray-surface-intersection test. (Sec.~\ref{sec:RT-Hausdorff}) Initially, we construct the GAS for both $\partial A$ and $\partial B$ to utilize the RT-core effectively. The RT-based Hausdorff distance algorithms then determine the Hausdorff distance by processing the GAS of one object (e.g. $GAS_{\partial A}$) and set of the vertices of the other (e.g. $P_{\partial B}$). Following the Hausdorff distance definition (Eq.~\ref{equation:hausdorff_definition}), we compute the Hausdorff distance to both directions ($\partial A \to \partial B$) and ($\partial B \to \partial A$). As a result, the bigger one is the final Hausdorff distance, and also it is the penetration depth between input object $A$ and $B$.


%the proposed RT-based penetration depth calculation pipeline.
%Our proposed methods adopt Tang's Hausdorff-based penetration depth methods~\cite{SIG09HIST}. The pipeline is divided into the penetration surface extraction step and the Hausdorff distance calculation between the penetration surface steps. However, since Tang's approach is not suitable for the RT platform in detail, we modified and applied it with appropriate methods.

%The penetration surface extraction step is extracting overlapped surfaces on other objects. To utilize the RT core, we use the ray-intersection-based PIP(Point-In-Polygon) algorithms instead of collision detection between two objects which Tang et al.~\cite{SIG09HIST} used. (Sec.~\ref{sec:RT-PIP})
%RT core-based PIP test uses a ray-surface intersection test. For purpose this, we generate the GAS(Geometry Acceleration Structure) for each object. RT core-based PIP test takes the GAS of one object (e.g. $GAS_{A}$) and a set of vertex of another one (e.g. $P_{B}$). Then this computes the penetrated vertex set of another one (e.g. $P_{\partial B}$). To calculate the Hausdorff distance, these vertex sets change to objects constructed by penetrated surface (e.g. $\partial B$). Finally, the two generated overlapped surface objects $\partial A$ and $\partial B$ are used in the Hausdorff distance calculation step.
\section{Preliminaries} \label{sec:prelims}
Before diving into the technical results, we state the basic graph notations used throughout the paper and recap the new non-standard definitions we have introduced throughout \Cref{sec:overview}.

\paragraph{Graphs.}
Throughout we consider directed simple graphs $G = (V, E)$, where $E \subseteq V^2$, with $n = |V|$ nodes and $m = |E|$ edges. The edges of the graph can be associated with some value: a length $\ell(e)$ or a capacity/cost $c(e)$, all of which we require to be positive. For any $U \subseteq V$, we write $\overline U = V \setminus U$. Let $G[U]$ be the subgraph induced by $U$. We denote with $\delta^{+}(U)$ the set of edges that have their starting point in $U$ and endpoint in~$\overline U$. We define $\delta^{-}(U)$ symmetrically. We also sometimes write $c(S) = \sum_{e \in S} c(e)$ (for a set of edges $S$) or $c(U, W) = \sum_{e \in E \cap (U \times W)} c(e)$ and $c(U) = c(U, U)$ (for sets of nodes $U, W$).

The distance between two nodes $v$ and $u$ is written $d_G(v,u)$ (throughout we consider only the \emph{length} functions to be relevant for distances). We may omit the subscript if it is clear from the context. The diameter of the graph is the maximum distance between any pair of nodes. For a subgraph $G'$ of $G$ we occasionally say that~$G'$ has \emph{weak diameter} $D$ if for all pairs of nodes $u, v$ in~$G'$, we have $d_G(u, v), d_G(v, u) \leq D$. A strongly connected component in a directed graph $G$ is a subgraph where for every pair of nodes $v,u$ there is a path from $v$ to $u$ and vise versa. Finally, for a radius $r \geq 0$ we write $B^+(v, r) = \set{x \in V : d_G(v, x) \leq r}$ and $B^-(v, r) = \set{y \in V : d_G(y, v) \leq r}$.


\paragraph{Polynomial Bounds.}
For graphs with edge lengths (or capacities), we assume that they are positive and the maximum edge length is bounded by $\poly(n)$. This is only for the sake of simplicity in \cref{sec:ldd-expander,sec:ldd-deterministic} (where in the more general case that all edge lengths are bounded by some threshold $W$ some logarithmic factors in $n$ become $\log (nW)$ instead), and is not necessary for our strongest LDD developed in \cref{sec:ldd-fast}.

\paragraph{Expander Graphs.}
Let $G = (V, E, \ell, c)$ be a directed graph with positive edge capacities $c$ and positive unit edge lengths $\ell$. We define the \emph{volume $\vol(U)$} by
\begin{equation*}
	\vol(U) = c(U, V) = \sum_{e \in E \cap (U \times V)} c(e),
\end{equation*}
and set $\minvol(U) = \min\set{\vol(U), \vol(\overline U)}$ where $\overline U = V \setminus U$. A node set $U$ naturally corresponds to a cut $(U, \overline U)$. The \emph{sparsity} (or \emph{conductance}) of $U$ is defined by
\begin{equation*}
	\phi(U) = \frac{c(U, \overline U)}{\minvol(U)}.
\end{equation*}
In the special cases that $U = \emptyset$ we set $\phi(U) = 1$ and in the special case that $U \neq \emptyset$ but $\vol(U) = 0$, we set $\phi(U) = 0$.
We say that $U$ is \emph{$\phi$-sparse} if $\phi(U) \leq \phi$. We say that a directed graph is a $\phi$-expander if it does not contain a $\phi$-sparse cut $U \subseteq V$. 
We define the \emph{lopsided sparsity} of $U$ as
\begin{equation*}
	\psi(U) = \frac{c(U, \overline U)}{\minvol(U) \cdot \log \frac{\vol(V)}{\minvol(U)}},
\end{equation*}
(with similar special cases), and we similarly say that $U$ is \emph{$\psi$-lopsided sparse} if $\psi(U) \leq \psi$. Finally, we call a graph a \emph{$\psi$-lopsided expander} if it does not contain a $\psi$-lopsided sparse cut $U \subseteq V$.

\section{Near-Optimal LDDs via Expander Decompositions} \label{sec:ldd-expander}

In this section, we show the existence of a near-optimal LDD, thereby proving our first main theorem: 

\thmMainExistential*

We introduce some technical lemmas in order to build up the framework for the proof of \Cref{thm:main-existential}, which can be found in the end of the section.


\subsection{Reduction to Cost-Minimizers}

\mwu*

\begin{proof}
	Let $G = (V, E)$ denote the graph for which we are supposed to design the LDD. Let us also introduce edge costs that are initially defined as $c(e) = 1$ for all edges. We will now repeatedly call the cost-minimizer, obtain a set of cut edges $S \subseteq E$, and then update the edge costs by $c(e) \gets 2 \cdot c(e)$ for all $e \in S$. We stop the process after $R = \log |E| \cdot D/L$ iterations, and let $\mathcal S$ denote the collection of all~$R$ sets~$S$ that we have obtained throughout. We claim that the uniform distribution on $\mathcal S$ is the desired LDD for $G$.
	
	It is clear that all for all sets $S \in \mathcal S$ the diameter condition is satisfied. We show that additionally for all edges $e \in E$ we have that
	\begin{equation*}
		\Pr(e \in S) \leq \frac{10L}{D},
	\end{equation*}
	for $S$ sampled uniformly from $\mathcal S$. Suppose otherwise, then in particular we have increased the cost of $e$ to at least
	\begin{equation*}
		c(e) \geq 2^{\frac{10L}{D} \cdot R} = 2^{10 \log |E|} = |E|^{10}.
	\end{equation*}
	On the other hand, let $c'$ denote the adapted costs after running the process for one iteration. Then the total cost increase is
	\begin{equation*}
		\sum_{e \in E} c'(e) - c(e) = \sum_{e \in S} c'(e) - c(e) = \sum_{e \in S} c(e) = c(S) \leq c(E) \cdot \frac{L}{D}.
	\end{equation*}
	That is, with every step of the process the total cost increases by a factor of $(1 + \frac{L}{D})$ and thus the total cost when the process stops is bounded by
	\begin{equation*}
		|E| \cdot \parens*{1 + \frac{L}{D}}^R \leq |E| \cdot e^{\frac{L}{D} \cdot R} = |E| \cdot e^{\log |E|} \leq |E|^3,
	\end{equation*}
	leading to a contradiction. The same argument shows that all costs are bounded by $|E|^3 \leq |V|^6$ throughout.
\end{proof}

\lemLexpDecomp*

\begin{proof}
Consider the following algorithm: If there is no $\psi$-lopsided sparse cut then the graph is a $\psi$-lopsided expander by definition and we stop. Otherwise, there exists a $\psi$-lopsided sparse cut~$(U, \overline U)$. We then distinguish two cases: If $\vol(U) \leq \vol(\overline U)$ then we remove all edges from~$U$ to~$\overline U$, and otherwise we remove all edges from~$\overline U$ to~$U$ (in both cases placing these edges in~$S$). Then we recursively continue on all strongly connected components in the remaining graph~$G \setminus S$. 

It is clear that all strongly connected components in the remaining graph $G \setminus S$ are $\psi$-lopsided expanders, but it remains to show that we cut edges with total capacity at most $c(E) \cdot \psi \log c(E)$. Imagine that initially we associate to each edge $e$ a \emph{potential} of~\makebox{$c(e) \cdot \log c(E)$}. The total initial potential is thus $\sum_e c(e) \log c(E) = c(E) \log c(E)$. Throughout the procedure we maintain the invariant that each edge holds a potential of at least $c(e) \log \vol(C)$, where $C$ is the strongly connected component containing edge $e$. Focus on any recursion step and its current strongly connected component~$C$, and let $C = U \sqcup \overline U$ denote the current $\psi$-lopsided sparse cut. Assume first that $\vol(U) \leq \vol(\overline U)$. Observe that an edge $e \in U$ suddenly needs to hold a potential of $c(e)\log c(U)$ instead of $c(e)\log c(C)$. Hence, the amount of freed potential in $U$ is at least
\begin{align*}
	\sum_{e \in E \cap (U \times V)} c(e) (\log \vol(C) - \log \vol(U)) &=
	\sum_{e \in E \cap (U \times V)} c(e)  \cdot \log \frac{\vol(C)}{\vol(U)} = \vol(U) \cdot \log \frac{\vol(C)}{\vol(U)}.
\end{align*}
On the other hand, since $(U, \overline U)$ is a $\psi$-lopsided sparse cut we have that
\begin{equation*}
	\psi \geq \psi(U) = \frac{c(U, \overline U)}{\minvol(U) \cdot \log \frac{\vol(C)}{\minvol(U)}} = \frac{c(U, \overline U)}{\vol(U) \cdot \log \frac{\vol(C)}{\vol(U)}}.
\end{equation*}
Putting these together, this means any cut edge $e$ from $U$ to $\overline U$ can get ``paid'' a potential of~\smash{$c(e) \cdot \psi^{-1}$} while still maintaining the potential invariant. (Note that here we only exploit the potential freed by the smaller side of the cut $U$, and forget about the overshoot potential in the larger side $\overline U$.) A symmetric argument applies when $\vol(U) < \vol(\overline U)$.

All in all, we start with a total potential of $c(E) \log c(E)$ and pay for each cut edge $e \in S$ with a potential of at least~\smash{$c(e) \cdot \psi^{-1}$}. This implies that $c(E) \log c(E) \geq c(S) \cdot \psi^{-1}$ and the claim follows.
\end{proof}

To prove that lopsided expanders have small diameter, we first establish the following technical lemma. 

\begin{lemma} \label{lem:lopsided-expansion}
Let $G = (V, E, c)$ be a directed graph and let $\psi > 0$. For any node $v \in V$ there is some radius $R = \Order(\psi^{-1} \log\log \vol(V) + \log \vol(V))$ such that one of the following two properties holds:
\begin{itemize}
	\item $\vol(B^+(v, R)) \geq \frac{1}{2} \cdot \vol(V)$, or
	\item $\psi(B^+(v, r)) \leq \psi$ for some $0 \leq r \leq R$.
\end{itemize}
\end{lemma}
\begin{proof}
We write $\Delta_i = \ceil{\frac{1}{i \psi}}$ and define the radii~\smash{$1 = r_{\ceil{\log \vol(V)}} \leq \dots \leq r_1$} by $r_i = r_{i+1} + \Delta_i$. We prove by induction that~\smash{$\vol(B^+(v, r_i)) \geq 2^{-i} \cdot \vol(V)$}, or alternatively that we find a sparse cut. This is clearly true in the base case for $i = \ceil{\log \vol(V)}$: Either $\vol(B^+(v, 1)) \geq 1$ or $v$ is an isolated node and therefore $\psi(B^+(v, r)) = 0$.

For the inductive case, suppose for the sake of contradiction that $\vol(B^+(v, r_i)) < 2^{-i} \cdot \vol(V)$. By induction we know however that $\vol(B^+(v, r_{i+1})) \geq 2^{-i-1} \cdot \vol(V)$. It follows there is some radius $r_{i+1} \leq r < r_i = r_{i+1} + \Delta_i$ such that
\begin{equation*}
	\frac{\vol(B^+(v, r + 1))}{\vol(B^+(v, r))} \leq 2^{1/\Delta_i} \leq 1 + \frac{1}{\Delta_i}.
\end{equation*}
It follows that
\begin{equation*}
	c(B^+(v, r), \overline{B^+(v, r)}) = \vol(B^+(v, r + 1)) - \vol(B^+(v, r)) \leq \frac{\vol(B^+(v, r))}{\Delta_i}.
\end{equation*}
Therefore the cut induced by $B^+(v, r)$ has lopsided sparsity
\begin{align*}
	\psi(B^+(v, r)) &= \frac{c(B^+(v, r), \overline{B^+(v, r)})}{\minvol(B^+(v, r)) \log \frac{\vol(V)}{\minvol(B^+(v, r))}} \\
	&\leq \frac{c(B^+(v, r), \overline{B^+(v, r)})}{\vol(B^+(v, r)) \log \frac{\vol(V)}{\vol(B^+(v, r))}} \\
	&\leq \frac{1}{\Delta_i \cdot \log \frac{\vol(V)}{\vol(B^+(v, r))}} \\
	&\leq \frac{1}{\Delta_i \cdot \log \frac{\vol(V)}{\vol(B^+(v, r_i))}} \\
	&\leq \frac{1}{\Delta_i \cdot \log(2^i)} \\
	&\leq \psi.
\end{align*}
Here, in the second step we have used that $\minvol(B^+(v, r)) = \vol(B^+(v, r))$ as in the opposite case we have $\vol(B^+(v, r)) \geq \frac{1}{2} \cdot \vol(V)$ which also proves the claim. This finally leads to a contradiction since we assume that the graph is a $\psi$-lopsided expander and thus does not contain $\psi$-lopsided sparse cuts.

In summary, the induction shows that $\vol(B^+(v, r_1)) \geq \frac{1}{2} \cdot \vol(V)$ and thus we may choose $R = r_1$. To prove that $R$ is as claimed, consider the following calculation:
\begin{align*}
	R &= 2 + \sum_{i=1}^{\ceil{\log \vol(V)}} \Delta_i \\
	&= 2 + \sum_{i=1}^{\ceil{\log \vol(V)}} \ceil*{\frac{1}{i \psi}} \\
	&\leq 2 + \ceil{\log \vol(V)} + \sum_{i=1}^{\ceil{\log \vol(V)}} \frac{1}{i \psi} \\
	&\leq \Order(\log\vol(V) + \psi^{-1} \log\log\vol(V)),
\end{align*}
using the well-known fact that the harmonic numbers are bounded by $\sum_{k=1}^n 1/k = \Order(\log n)$.
\end{proof}

One can easily strengthen the lemma as follows. This insight will play a role in the next \cref{sec:ldd-deterministic} in the construction of the deterministic algorithm.

\begin{lemma} \label{lem:lopsided-expansion-boosted}
Let $G = (V, E, c)$ be a directed graph and let $\psi > 0$ and $0 < \alpha < 1$. For any node $v \in V$ there is some radius $R = \Order(\psi^{-1} \log\log \vol(V) + \psi^{-1} \alpha^{-1} + \log \vol(V))$ such that one of the following two properties holds:
\begin{itemize}
	\item $\vol(B^+(v, R)) \geq (1 - \alpha) \cdot \vol(V)$, or
	\item $\psi(B^+(v, r)) \leq \psi$ for some $0 \leq r \leq R$.
\end{itemize}
\end{lemma}
\begin{proof}
Applying the previous lemma with parameter $\psi$ yields $R' = \Order(\psi^{-1} \log\log \vol(V) + \log \vol(V))$ such that either $\vol(B^+(v, R')) \geq \frac{1}{2} \cdot \vol(V)$, or $\psi(B^+(v, r)) \leq \psi$ for some $0 \leq r \leq R'$. In the latter case we are immediately done, so suppose that we are in the former case.

Let $\Delta = \ceil{2 \alpha^{-1} \psi^{-1}}$ and let $R = R' + \Delta$. If $\vol(B^+(v, R)) \geq (1 - \alpha) \cdot \vol(V)$ then we have shown the first case and are done. So suppose that otherwise $\vol(B^+(v, R)) \leq (1 - \alpha) \cdot \vol(V)$. Then due to the trivial bound $\vol(B^+(v, R)) \leq \vol(V)$, there is some radius $R' \leq r \leq R = R' + \Delta$ with
\begin{equation*}
	\frac{\vol(B^+(v, r + 1))}{\vol(B^+(v, r))} \leq 2^{1/\Delta} \leq 1 + \frac{1}{\Delta},
\end{equation*}
and hence,
\begin{equation*}
	c(B^+(v, r), \overline{B^+(v, r)}) = \vol(B^+(v, r + 1)) - \vol(B^+(v, r)) \leq \frac{\vol(B^+(v, r))}{\Delta}.
\end{equation*}
For this radius $r$ it further holds that $\vol(B^+(v, r)) \leq (1 - \alpha) \cdot \vol(V)$ and thus $\vol(\overline{B^+(v, r)}) \geq \alpha \cdot \vol(V)$. In particular, we have that $\minvol(B^+(v, r)) \geq \frac{\alpha}{2} \cdot \vol(V)$. Putting these statements together, we have that
\begin{align*}
	\psi(B^+(v, r))
	&= \frac{c(B^+(v, r), \overline{B^+(v, r)})}{\minvol(B^+(v, r)) \log \frac{\vol(V)}{\minvol(B^+(v, r))}} \\
	&\leq \frac{\vol(B^+(v, r))}{\Delta \cdot \frac{\alpha}{2} \cdot \vol(B^+(v, r)) \log \frac{\vol(V)}{\minvol(B^+(v, r))}} \\
	&\leq \frac{1}{\Delta \cdot \frac{\alpha}{2}} \\
	&\leq \psi,
\end{align*}
witnessing indeed the desired sparse lopsided cut.
\end{proof}

\lemLexpDiam*

\begin{proof}
	Take an arbitrary pair of nodes $v, u$. Applying \cref{lem:lopsided-expansion-boosted} with parameters $\psi$ and $\alpha = \frac{1}{4}$, say, yields a radius $R = \Order(\psi^{-1} \log\log \vol(V) + \log \vol(V))$ such that
	\begin{equation*}
		\vol(B^+(v, R)) \geq \frac{3}{4} \cdot \vol(V),
	\end{equation*}
	and symmetrically,
	\begin{equation*}
		\vol(B^-(u, R)) \geq \frac{3}{4} \cdot \vol(V).
	\end{equation*}
	Therefore, there is some edge $e = (x, y)$ contributing to both of these volumes. Thus $x \in B^+(v, R)$ and $y \in B^-(u, R)$. It follows that
	\begin{equation*}
		d_G(u, v) \leq d_G(u, x) + d_G(x, y) +  d_G(y, u) \leq R + 1 + R = \Order(\psi^{-1} \log\log \vol(V) + \log \vol(V)).
	\end{equation*}
	Since the nodes $u, v$ were chosen arbitrarily this establishes the claimed diameter bound.
\end{proof}

\begin{proof}[Proof of \cref{thm:main-existential}]
Let $G = (V, E, \ell)$ be a directed graph with positive edge lengths. We show that there is an LDD with loss $\Order(\log n \log\log n)$ for $G$. We first deal with two trivial cases: First, if $D \leq \log n / \gamma$ (for some constant $\gamma > 0$ to be determined later) then we simply remove all edges and stop. Second, we remove all edges with length more than $D$ from the graph. In both cases edges can be deleted with probability $1$ without harm.

Next, we transform the graph into $G'$ by replacing each $e$ by a path of $\ell(e)$ unit-length edges. In the following it suffices to design an LDD for the augmented graph; if that LDD cuts any of the edges along the path corresponding to an original edge $e$ we will cut $e$ entirely. An LDD with loss~$L$ in the augmented graph will thus delete an original edge with probability at most $\ell(e) \cdot \frac{L}{D}$ by a union bound. All in all, this transformation blows up the number of nodes and edges in the graph by a factor of at most $D$ (since we removed edges with larger length). Recall that we throughout assume that $D \leq n^c$, for some constant $c$, and thus $|V'| \leq n^{\Order(1)}$.

By \cref{lem:mwu} we further reduce the existence of an LDD of $G'$ to the following cost-minimizer task: View $G'$ as an edge-capacitated graph $G' = (V', E', c)$ for some capacities~\makebox{$c : E' \to [|V'|^{10}]$}. In particular, under this capacity function $G'$ has volume $\vol(V') \leq |V'|^2 \cdot |V'|^{10} = |V'|^{12} = n^{\Order(1)}$. The goal is to delete edges $S \subseteq E'$ in $G'$ so all remaining strongly connected components have (weak) diameter at most $D$, and the total cost of all deleted edges is only $c(S) \leq c(E) \cdot \frac{L}{D}$.

Finally, we apply the Lopsided Expander Decomposition from \cref{lem:lexp-decomp} on $G'$. Specifically, we define
\begin{equation*}
	\psi = \frac{\log\log \vol(V')}{\epsilon D}
\end{equation*}
for some constant $\epsilon > 0$ to be determined later. The Expander Decomposition then cuts edges $S \subseteq E'$ so that each remaining strongly connected component is $\psi$-lopsided expander. Thus, by \cref{lem:lexp-diam} each strongly connected component has diameter
\begin{equation*}
	\Order(\psi^{-1} \log\log \vol(V') + \log \vol(V')) = \Order(\epsilon D + \gamma D).
\end{equation*}
By choosing the constants $\epsilon$ and $\gamma$ to be sufficiently small, the diameter bound becomes $D$ as desired. Moreover, \cref{lem:exp-decomp} guarantees that we cut edges of total capacity
\begin{equation*}
	c(S) \leq c(E') \cdot \psi \log \vol(V') \leq c(E') \cdot \frac{\log \vol(V') \log\log \vol(V')}{\epsilon D},
\end{equation*}
which becomes $\frac{L}{D}$ by choosing
\begin{equation*}
	L = \frac{\log \vol(V') \log\log \vol(V')}{\epsilon} \leq \frac{\log |V'|^{12} \log\log |V'|^{12}}{\epsilon} = \Order(\log n \log\log n)
\end{equation*}
as planned.
\end{proof}
\section{Near-Optimal LDDs Deterministically} \label{sec:ldd-deterministic}
In this section, we present the deterministic algorithm for computing a near-optimal LDD, thereby proving our second main theorem:

\thmMainDet*

To this end, we utilize many of the same building blocks we have already introduced in \Cref{sec:ldd-expander}. In particular, we follow the general framework of Multiplicative Weights Update to reduce the computation of an LDD to solving the cost-minimizing task. The full proof of \Cref{thm:main-det} can be found in the end of the section. The following lemma restates the MWU method algorithmically; we omit a proof as it follows exactly the proof of \cref{lem:mwu} in \Cref{sec:ldd-expander}. 

\begin{lemma}[Algorithmic Multiplicative Weight Update]
Let $G = (V, E)$ be a directed graph and let $D \geq 1$. Suppose that there is an algorithm $\mathcal A$ that, given $G$, $D$ and a cost function $c : E \to [|V|^{10}]$, computes a set of edges $S \subseteq E$ satisfying the following properties:
\begin{itemize}
	\item For any two nodes $u, v \in V$ that are part of the same strongly connected component in $G \setminus S$, we have $d_G(u, v) \leq D$ and $d_G(v, u) \leq D$.
	\item $c(S) \leq c(E) \cdot \frac{L}{D}$.
\end{itemize}
Then there is a deterministic algorithm to compute an LDD with loss $\Order(L)$ for $G$ (i.e., we compute the full support of a uniform distribution over $\Order(D \log n)$ cut sets). It runs in time~\smash{$\widetilde\Order(m D)$} and issues $\Order(D \log n)$ oracle calls to $\mathcal A$.
\end{lemma}

For the remainder of this section we will therefore focus on the same cost minimizer setting: Given a directed graph $G = (V, E, c)$ with edge capacities (and unit lengths), the goal is select a set of cut edges $S \subseteq E$ such that $c(S) \leq c(E) \cdot \Order(\frac{1}{D} \cdot \log n \log\log n)$ and such that all strongly connected components in the remaining graph $G \setminus S$ have (weak) diameter at most $D$.

The following lemma is a consequence of the lopsided expander machinery set up before:

\begin{lemma}[Finding Sparse Cuts] \label{lem:sparse-cut-det}
Let $G = (V, E, c)$ be a directed graph, let $D \geq \log \vol(V)$ and let $v \in V$. Then there is some $\psi = \Order(\frac{1}{D} \cdot \log\log \vol(V))$ and an algorithm to determine which of the following cases applies:
\begin{enumerate}[label=(\roman*)]
	\item There is a radius $0 \leq r \leq D$ with $\psi(B^+(v, r)) \leq \psi$ and $c(B^+(v, r)) \leq 0.95 \cdot \vol(V)$.\\(In this case the algorithm runs in linear time in the number of edges incident to $B^+(v, r)$.)
	\item Or, there is a radius $0 \leq r \leq D$ such that $\psi(\overline{B^-(v, r)}) \leq \psi$ and $c(B^-(v, r)) \leq 0.95 \cdot \vol(V)$.\\(In this case the algorithm runs in linear time in the number of edges incident to $B^-(v, r)$.)
	\item Or, $c(B^+(v, D) \cap B^-(v, D)) \geq 0.9 \cdot \vol(V)$.\\(In this case the algorithm runs in time~\smash{$\Order(m)$}.)
\end{enumerate}
\end{lemma}
\begin{proof}
Existentially the statement follows from \cref{lem:lopsided-expansion-boosted} applied with parameters $\psi = \Theta(\frac{1}{D} \cdot \log\log \vol(V))$ (with an appropriately large hidden constant) and $\alpha = 0.05$. This lemma states that $\psi(B^+(v, r)) \leq \psi$ for some radius $0 \leq r \leq D - 1$ (proving (i)) or that~\smash{$\vol(B^+(v, D - 1)) \geq 0.95 \vol(V)$}. Applying the same statement to the reverse graph similarly gives that $\psi(\overline{B^-(v, r)}) \leq \psi$ for some radius $0 \leq r \leq D - 1$ (proving (ii)) or that~\smash{$c(\overline{B^-(v, D - 1)}, B^-(v, D - 1)) \geq 0.95 \cdot \vol(V)$}. In the only remaining case we thus have both
\begin{align*}
	c(B^+(v, D - 1), V) \geq 0.95 \cdot \vol(V)\quad\text{and}\quad c(\overline{B^-(v, D - 1)}, V) \geq 0.95 \cdot \vol(V).
\end{align*}
Combining both statements we obtain that edges of total capacity at least $0.9 \cdot \vol(V)$ must lie in $B^+(v, D - 1) \times B^-(v, D - 1)$. In particular, it follows that $c(B^+(v, D) \cap B^-(v, D)) \geq 0.9 \cdot \vol(V)$ thereby proving (iii).

To make the lemma algorithmic, we simultaneously grow an out-ball $B^+(v, r^+)$ and an in-ball~$B^-(v, r^-)$ around the node $v$. Explicitly, we start with $r^+ = 0$ and increase $r^+$ step by step to compute $B^+(v, r^+)$ (with breadth-first search). We can, without overhead, keep track of the current volume of $B^+(v, r^+)$. If we at some point encounter that $B^+(v, r)$ is $\psi$-lopsided sparse, then we stop and report output (i). Similarly, we start with $r^- = 0$ and step by step explore $B^-(v, r^-)$. If at some point $B^-(v, r^-)$ becomes a $\psi$-lopsided sparse cut, we stop and report answer (ii). We interleave these two computations so that when we output (i) the overhead of exploring the in-ball~$B^-(v, r^-)$ incurs only a constant factor in the running time, and similarly for (ii). In the remaining case where we have not encountered a sparse cut in the graph before reaching $r^+ = r^- = D$, we report (iii). In this case we indeed spend time at most $\Order(m)$.
\end{proof}

Having established \cref{lem:sparse-cut-det}, now consider the algorithm in \cref{alg:det}. In summary, it runs in two phases. In Phase (I) we first repeatedly select a node $v$ and attempt to cut a lopsided sparse cut around $v$ (i.e., we cut the edges in $\delta^+(B^+(v, r))$ for some radius $r$). We only execute these cuts, however, until we find a node $z$ for which $c(B^+(z, D') \cap B^-(z, D')) \geq 0.9 \cdot \vol(V)$---that is, both the radius-$D'$ out- and in-balls of $z$ make up for a big constant fraction of the entire graph. We call $z$ a \emph{center} node and move on to phase (II). In this phase we repeat the same steps as in Phase (I), but we only choose nodes $v$ that have distance at least $2D'$ (in one direction or the other) to the center $z$. The intuition is that we can never find a second node $z'$ which equally makes up for the entire graph, as then $z$ and $z'$ would have to be connected by a short path. In the remainder of this section we formally analyze \cref{alg:det}.

\begin{algorithm}[t]
	\caption{The deterministic near-optimal LDD, see \Cref{thm:main-det}.} \label{alg:det}
	\begin{enumerate}[label=\arabic*.]
		\item[(I)] Repeat the following steps: Take an arbitrary node $v \in V$ and apply \cref{lem:sparse-cut-det} with parameter \smash{$D' = \floor{\frac{D}{4}}$}. Depending on the output execute the following steps:
		\begin{enumerate}[label=(\roman*)]
			\item Cut all edges in $\delta^+(B^+(v, r))$, recurse on the induced graph $G[B^+(v, r)]$, then remove all nodes in $B^+(v, r)$ from the graph.
			\item Cut all edges $\delta^-(B^-(v, r))$, recurse on the induced graph $G[B^-(v, r)]$, then remove all nodes in $B^-(v, r)$ from the graph.
			\item Remember $z \gets v$ (called the \emph{center} node) and continue with Phase (II).
		\end{enumerate}
		\item[(II)] Compute the sets
		\begin{align*}
			X &= B^+(z, D') \cap B^-(z, D'), \\
			Y &= B^+(z, 2D') \cap B^-(z, 2D').
		\end{align*}
		Then repeat the following steps while there still exists nodes in $V \setminus Y$: Take an arbitrary node~\makebox{$u \in V \setminus Y$} and apply \cref{lem:sparse-cut-det} with parameter~$D'$. Depending on the output execute the following steps:
		\begin{enumerate}[label=(\roman*)]
			\item Cut all edges in $\delta^+(B^+(v, r))$, recurse on the induced graph $G[B^+(v, r)]$, then remove all nodes in $B^+(v, r)$ from the graph.
			\item Cut all edges $\delta^-(B^-(v, r))$, recurse on the induced graph $G[B^-(v, r)]$, then remove all nodes in $B^-(v, r)$ from the graph.
		\end{enumerate}
	\end{enumerate}
\end{algorithm}

\begin{lemma}[Total Cost of \cref{alg:det}] \label{lem:ldd-det-cost}
Let $S \subseteq E$ denote the set of edges cut by \cref{alg:det}. Then $c(S) \leq c(E) \cdot \Order(\frac{1}{D} \cdot \log \vol(V) \log\log \vol(V))$.
\end{lemma}
\begin{proof}
Observe that the algorithm only cuts the edges of $\psi$-lopsided sparse cuts for some parameter $\psi = \Order(\frac{1}{D} \cdot \log\log \vol(V))$. Indeed, the algorithm only cuts edges from $B^+(v, r)$ to~\smash{$\overline{B^+(v, r)}$} in subcase~(i), and edges from~\smash{$\overline{B^-(v, r)}$} to $B^-(v, r)$ in subcase~(ii). In both these cases \cref{lem:sparse-cut-det} gives exactly the guarantee that these respective cuts are $\psi$-lopsided sparse.

With this in mind, we can apply exactly the same potential argument as in the proof of \cref{lem:exp-decomp}. To avoid repetitions, we only give a quick reminder here: We initially associate to each edge a potential of~\smash{$c(e) \cdot \log\vol(V)$}. Then, following the calculations as in \cref{lem:exp-decomp}, we can free a potential of at least $c(e) / \psi$ for each cut edge, proving that $c(S) \leq c(E) \cdot \psi \log\vol(V)$ as claimed.
\end{proof}

\begin{lemma}[Well-Definedness of \cref{alg:det}] \label{lem:ldd-det-well-defined}
While executing Phase~(II) of \cref{alg:det}, the subcase~(iii) never happens.
\end{lemma}
\begin{proof}
Let $G = (V, E)$ denote the graph at the transition from Phase (I) to Phase (II). Suppose for contradiction that during the execution of Phase (II) we find a node which falls into case~(iii). That is, let $G' = (V', E')$ denote the graph remaining at this point in the execution of the algorithm, and suppose that there is a node $v \in V'$ for which $\vol(G'[B^+(v, D') \cap B^-(v, D')]) \geq 0.9 \cdot \vol(G')$. Since we have picked~\makebox{$v \not\in Y$}, we either have that $d_G(z, v) > 2D'$ or that $d_G(v, z) > 2D'$; focus on the former case. Then for all~\makebox{$x \in X$}, we have $d_G(x, v) > D'$ (as otherwise~\makebox{$d_G(z, v) \leq d_G(z, x) + d_G(x, v) \leq 2D'$}), and thus $X$ and $B^-(v, D')$ are disjoint. But this leads to a contradiction as supposedly both $\vol(G[X]) \geq 0.9 \vol(G)$ and $\vol(G[B^-(v, D')]) \geq 0.9 \vol(G') \geq 0.9 \cdot 0.9 \cdot \vol(G) \geq 0.8 \vol(G)$, leading to a total volume of more than $\vol(G)$. Here in the last step we have used that invariantly $\vol(G') \geq 0.9 \cdot \vol(G)$ since the algorithm can never cut nodes from $X$. The remaining case is symmetric.
\end{proof}

\begin{lemma}[Correctness of \cref{alg:det}] \label{lem:ldd-det-correctness}
Let $S \subseteq E$ denote the set of edges cut by \cref{alg:det}. Then for any two nodes $u, w$ in the same strongly connected component in $G \setminus S$, it holds that $d_G(u, w) \leq D$ and $d_G(w, u) \leq D$.
\end{lemma}
\begin{proof}
We prove the statement by induction. The base case is clear for an appropriate implementation of constant-size graphs. For the inductive step, consider an execution of \cref{alg:det} and an arbitrary pair of nodes $u, w$. Whenever the algorithm cuts a ball $B^+(v, r)$ and we have $u \in B^+(v, r)$ and $w \not\in B^+(v, r)$, then $u$ and $w$ do not end up in the same strongly connected component in~$G \setminus S$. If instead both $u, w \in B^+(v, r)$ then the claim follows by induction as the algorithm recurses on the subgraph induced by $B^+(v, r)$. The only remaining case is if both $u$ and $w$ are never cut during the execution of the algorithm. Clearly the algorithm cannot have terminated after Phase (I) (as then there would be no nodes left in the graph), so the algorithm has reached Phase (II). Moreover, we have $u, w \in Y$ as otherwise the algorithm would not have terminated yet. But then by definition, we have $d_G(v, u) \leq 2D'$, $d_G(u, v) \leq 2D'$, $d_G(w, v) \leq 2D'$ and $d_G(v, w) \leq 2D'$. Putting all these together we have that $d_G(u, w) \leq 4D' \leq D$ and $d_G(w, u) \leq 4D' \leq D$ as claimed.
\end{proof}

\begin{lemma}[Running Time of \cref{alg:det}] \label{lem:ldd-det-time}
\cref{alg:det} runs in time $\Order(m \log \vol(V))$.
\end{lemma}
\begin{proof}
Consider one execution of \cref{alg:det}. We spend time $\Order(m)$ once to compute the sets $X$ and $Y$ at the beginning of Phase (II). Other than that, all steps only run in local fragments of the graph. Specifically, whenever the algorithm cuts a ball $B^+(v, r)$ we spend time $\Order(|\delta^+(B^+(v, r))|)$ by \cref{lem:sparse-cut-det} (i.e., time proportional to the number of edges incident to $B^+(v, r)$), but then we delete all nodes in $B^+(v, r)$ (and thereby also all edges in $\delta^+(B^+(v, r))$). We can thus express the running time $T(m)$ by the recurrence
\begin{equation*}
	T(m, C) \leq \Order(m) + \sum_i T(m_i, C_i),
\end{equation*}
where $m_i$ is the number of edges and $C_i$ is the total capacity of the $i$-th recursive call. Clearly we have that $\sum_i m_i \leq m$. Moreover, we only cut balls with $c(B^+(v, r)) \leq 0.95 \cdot \vol(V)$ by \cref{lem:sparse-cut-det} (and similarly for the in-balls $B^-(v, r)$). This drop in capacity bounds the recursion depth by $\Order(\log \vol(V))$ and so the recursion solves to $\Order(m \log |V|)$.
\end{proof}

This completes the analysis of \cref{alg:det} and puts us in the position of completing the proof of \cref{thm:main-det}.

\begin{proof}[Proof of \cref{thm:main-det}]
As before, we first turn the given graph into a unit-length graph which only blows up the number of nodes and edges by a factor of $D$ (all edges with length $> D$ can anyways be removed for free). Then, by \cref{lem:mwu} it suffices to design a deterministic algorithm for the cost minimization problem (indeed, this algorithm is then turned into a deterministic LDD algorithm by \cref{lem:mwu}, at the cost of another factor-$\Order(D \log n)$ blow-up). Finally, we can assume that~\makebox{$D \geq \log \vol(V)$} in the remaining task, as otherwise it is within our budget to simply remove all edges.

To solve the cost minimization problem, we run \cref{alg:det}. Let $S \subseteq E$ denote the edges cut by \cref{alg:det}. Then, by \cref{lem:ldd-det-correctness} indeed the remaining graph $G \setminus S$ has (weak) diameter at most~$D$. By \cref{lem:ldd-det-cost} the total capacity of the cut edges is $c(S) \leq c(E) \cdot \Order(\frac{1}{D} \cdot \log\vol(V) \log\log\vol(V))$. And by \cref{lem:ldd-det-time} the running time is near-linear in the number of edges. All three properties are as required by \cref{lem:mwu}, ultimately leading to an LDD algorithm with loss $\Order(\log n \log\log n)$ (using that $\log \vol(V) = \Order(\log n)$) and in time $\widetilde\Order(m D^2)$.
\end{proof}
\section{Near-Optimal LDDs in Near-Linear Time} \label{sec:ldd-fast}
In this section we show that near-optimal LDDs can even be computed in near-linear (i.e., near-optimal) running time. Formally, the goal of this section is to prove the last of our main theorems:

\thmMainFast*

We structure this section as follows: In \cref{sec:ldd-fast:sec:cutting} we first prove an important cutting lemma (\cref{lem:cut-light}), which we then apply in \cref{sec:ldd-fast:sec:algo} to prove \cref{thm:main-fast}.

\subsection{Cutting Lemma} \label{sec:ldd-fast:sec:cutting}
We rely on the following result due to Cohen that we can approximate efficiently the sizes of out-balls $\Bout(v, r)$ simultaneously for all nodes $v$. Note that we can equally approximate the size of the in-balls $\Bin(v, r)$ by applying \cref{lem:cohen} on the reverse graph.

\begin{lemma}[Cohen's Algorithm~{{\cite[Theorem~5.1]{Cohen97}}}] \label{lem:cohen}
Let $G$ be a directed weighted graph, let $r \geq 0$ and~\makebox{$\epsilon > 0$}. There is an algorithm that runs in time $\Order(m \epsilon^{-2} \log^3 n)$ and computes approximations~$b(v)$ satisfying with high probability that
\begin{equation*}
    (1 - \epsilon) |\Bout(v, r)| \leq b(v) \leq (1 + \epsilon) |\Bout(v, r)|.
\end{equation*}
\end{lemma}

The goal of this subsection is to establish the following lemma:

\begin{lemma}[Cutting Light Nodes] \label{lem:cut-light}
Let $G = (V, E)$ be a directed weighted graph, and consider the parameters $\delta > 0$, $0 \leq r_0 < r_1$, $1 \leq s_1 < s_0 \leq m$. There is an algorithm that computes a set of cut edges $S \subseteq E$ and sets of remaining vertices $R \subseteq V$ such that:
\begin{enumerate}[label=(\roman*)]
    \item Each strongly connected component in $G \setminus S$ either only contains nodes from $R$ or only nodes from $V \setminus R$. In the latter case it contains at most $\frac32 \cdot \frac{m}{s_1}$ edges.
    \item For every node $v \in R$ with $|\Bout_G(v, r_1)| \leq \frac{m}{s_1}$, it holds that $|\Bout_{G[R]}(v, r_0)| \leq \frac{m}{s_0}$.
    \item None of the edges in $S$ has its source in $R$ (i.e., $S \subseteq (V \setminus R) \times V$). Moreover, for each edge $e = (x, y) \in E$ it holds that:
    \begin{equation*}
        \Pr(e \in S \mid x \not\in R) \leq \frac{w_e \ln(2s_0 / \delta)}{r_1 - r_0}.
    \end{equation*}
\end{enumerate}
With probability at most $\delta$ the algorithm instead returns ``fail'', and with probability at most~$\frac{1}{\poly(n)}$ the algorithm returns an arbitrary answer. The running time is~\smash{$\Order(m \log^4 n)$}.
\end{lemma}

\begin{proof}
Consider the pseudocode in \cref{alg:cut-light}. Throughout we maintain a set of cut edges~\makebox{$S \subseteq E$} that is initially empty. We classify nodes as \emph{good} or \emph{bad} depending on whether their out-balls~$\Bout(v, r_0)$ and~$\Bout(v, r_1)$ satisfy certain size requirements. Then, in several iterations the algorithm samples a good node $v$ and attempts to \emph{cut} around $v$ in the sense that we sample a radius~\smash{$r_0 \leq r \leq r_1$}, include the boundary edges $\delta^+(\Bout(v, r))$ into $S$, and then remove $\Bout(v, r)$ from the graph.

\begin{algorithm}[t]
\caption{Implements the cutting procedure from \cref{lem:cut-light}.} \label{alg:cut-light}
\begin{enumerate}[label=\arabic*.]
    \item Run Cohen's algorithm (\cref{lem:cohen} with parameter $\epsilon = \frac18$) on $G$ to compute approximations~$b_1(v)$ satisfying that $\tfrac78 \cdot |\Bout(v, r_1)| \leq b_1(v) \leq \tfrac98 \cdot |\Bout(v, r_1)|$. Mark all nodes $v$ satisfying that $b_1(v) \leq \frac98 \cdot \frac{m}{s_1}$ as \emph{good} and all other nodes as \emph{bad.}
    \item Repeat the following steps $\log n + 1$ times:
    \begin{enumerate}[label=2.\arabic*.]
        \item Repeat the following steps $s_0 \cdot 100 \log n$ times:
        \begin{enumerate}[label=2.1.\arabic*.]
            \item Among the nodes that are marked good, pick a uniformly random node $v$.
            \item Test whether $|\Bout(v, r_0)| \geq \frac{1}{2} \cdot \frac{m}{s_0}$, and otherwise continue with the next iteration of step (b).
            \item Sample $r \sim r_0 + \geom(p)$ (where~\smash{$p = \frac{\ln(2s_0 / \delta)}{r_1 - r_0}$}) and compute $B^+(v, r)$.
            \item If $r > r_1$, then return ``fail''.
            \item Update $S \gets S \cup \delta^+(B^+(v, r))$, mark the nodes in $B^+(v, r)$ as bad, and remove these nodes from $G$.
        \end{enumerate}
        \item Run Cohen's algorithm (\cref{lem:cohen} with parameter $\epsilon = \frac18$) on $G$ to compute approximations $b_0(v)$ satisfying that $\tfrac78 \cdot |\Bout(v, r_0)| \leq b_0(v) \leq \tfrac98 \cdot |\Bout(v, r_0)|$. Mark all nodes $v$ satisfying that $b_0(v) < \frac78 \cdot \frac{m}{s_0}$ as bad.
    \end{enumerate}
    \item Let $R$ denote the set of remaining nodes in $G$, and return $(S, R)$.
\end{enumerate}
\end{algorithm}

In the following we formally analyze \cref{alg:cut-light}. Throughout we often condition on the event that Cohen's algorithm is correct as this happens with high probability. (Note however that we cannot necessarily detect that Cohen's algorithm has erred, and thus the algorithm may return an arbitrary answer in this rare case.)

\paragraph{Correctness of (i).}
Note that whenever the algorithm cuts a ball $\Bout(v, r)$ from the graph~$G$ and adds the boundary edges $\delta^+(\Bout(v, r))$ to $S$, it makes sure that the nodes inside and outside~$\Bout(v, r)$ lie in different strongly connected components in $G \setminus S$. As $R$ is defined to be the left-over nodes that are never cut, and as we only cut around \emph{good} nodes $v$, it suffices to prove the following claim: With high probability, at any time during the execution of the algorithm it holds that $|\Bout_G(v, r_1)| \leq \frac32 \cdot \frac{m}{s_1}$ for all good nodes $v$.

To this end, observe that we only mark nodes as good in the first step, and in this step we only mark nodes as good when $b_1(v) \leq \frac98 \cdot \frac{m}{s_1}$. By the guarantee of \cref{lem:cohen}, we therefore initially have that
\begin{equation*}
    |\Bout_G(v, r_1)| \leq \frac98 \cdot b_1(v) \leq \frac{9^2}{8^2} \cdot \frac{m}{s_1} \leq \frac{3}{2} \cdot \frac{m}{s_1}.
\end{equation*}
Finally, as we only remove nodes and edges from $G$, the size of $|\Bout_G(v, r_1)|$ can only decrease throughout the remaining execution.

\paragraph{Correctness of (ii).}
In order to prove the correctness of (ii), we crucially rely on the following claim:

\begin{claim}
When the algorithm terminates (without returning ``fail''), with high probability all nodes are marked bad.
\end{claim}
\begin{proof}
The idea is to prove that with each iteration of step 2 the number of good nodes halves. As the algorithm runs $\log n + 1$ iterations of step 2, the claim then follows.

Focus on any iteration of step 2. Let $M_0$ denote the initial set of good nodes, let $M_1, \dots, M_k$ denote the subsets of nodes that are still marked good after the respective $k = s_0 \cdot 100 \log n$ iterations of step~2.1, and let $M^*$ denote the good nodes after executing step~2.2. We clearly have that~\makebox{$M_0 \supseteq M_1 \supseteq \dots \supseteq M_k \supseteq M^*$}. Our goal is to show that $|M^*| \leq \frac12 \cdot |M_0|$. The claim is clear if~\makebox{$|M_k| \leq \frac12 \cdot |M_0|$}, so assume otherwise. We argue that after completing step~2.1, with high probability at least half of the nodes $v \in M_k$ satisfy that $|\Bout_G(v, r_0)| < \tfrac12 \cdot \tfrac{m}{s_0}$. This implies the claim as then we will mark at least half of the nodes in $M_k$ as bad while executing step 2.2. To prove this, suppose that instead more than half of the nodes $v \in M_k$ satisfy that~\smash{$|\Bout_G(v, r_0)| \geq \tfrac12 \cdot \tfrac{m}{s_0}$}. In particular, there are least $\frac{|M_k|}{2} \geq \frac{|M_0|}{4}$ such nodes. But this means that every iteration of step 2.1 \emph{succeeds} (in the sense that the algorithm samples a node $v$ passing the condition in 2.1.2) with probability at least $\frac14$. As there are $k = s_0 \cdot 100 \log n$ repetitions, with high probability at least, say,~$10 s_0$ repetitions succeed. However, every time this happens the algorithm either removes at least $\frac12 \cdot \frac{m}{s_0}$ edges from the graph or fails and stops. Thus, the total number of repetitions can be at most $2 s_0$, leading to a contradiction.
\end{proof}

With this claim in mind, we return to the correctness of Property (ii). Each node $v \in R$ has been classified as bad at some point during the execution. This could have happened in step 1 (if~\makebox{$b_1(v) > \frac98 \cdot \frac{m}{s_1}$}), or in step 2.2 (if $b_0 < \frac78 \cdot \frac{m}{s_0}$), but not in step 2.1.5 as otherwise $v$ would not have ended up in $R$. Conditioning on the correctness of Cohen's algorithm, in the former case we have that
\begin{equation*}
    |\Bout_G(v, r_1)| \geq \frac89 \cdot b_1(v) > \frac89 \cdot \frac98 \cdot \frac{m}{s_1} = \frac{m}{s_1} 
\end{equation*}
(where $G$ is the initial graph). In the latter case we have that
\begin{equation*}
    |\Bout_G(v, r_0)| \leq \frac87 \cdot b_0(v) < \frac87 \cdot \frac78 \cdot \frac{m}{s_0} = \frac{m}{s_0}
\end{equation*}
(where $G$ is the current graph), and thus also~\smash{$|\Bout_{G[R]}(v, r_0)| < \frac{m}{s_0}$} for the set $R$ returned by the algorithm. The statement~(ii) follows.

\paragraph{Correctness of (iii).}
It is clear that the source nodes of all cut edges in $S$ cannot be in~$R$. Let $e = (x, y) \in E$ denote an arbitrary edge; we show that the probability~\makebox{$\Pr(e \in S \mid x \not\in R)$} is as claimed. Note that we can only have~\makebox{$x \not\in R$} if there is some iteration of step 2 that selects a node~$v$ and samples radius $r \sim r_0 + \geom(p)$ such that~\makebox{$x \in \Bout(v, r)$}. In this iteration we might include~$e$ into $S$ if it happens that~\makebox{$e \in \delta^+(\Bout(v, r))$}. After this iteration, however, we remove the edge~$e$ from the graph and the algorithm will never attempt to include $e$ into $S$ later on. Therefore, we can upper bound the desired probability by
\begin{flalign*}
    &\Pr(e \in S \mid x \not\in R) \\
    &\qquad\qquad\leq \max_{G, v} \Pr_{r \sim r_0 + \geom(p)}(e \in \delta^+_G(\Bout_G(v, r)) \mid x \in \Bout_G(v, r)) \\
    &\qquad\qquad= \max_{G, v} \Pr_{r \sim \geom(p)}(d_G(v, y) > r_0 + r \mid d_G(v, x) \leq r_0 + r) \\
    &\qquad\qquad \leq \max_{G, v} \Pr_{r \sim \geom(p)}(d_G(v, x) + w_e > r_0 + r \mid d_G(v, x) \leq r_0 + r)
\intertext{We will now exploit the memorylessness of the geometric distribution: Conditioning on the event that $r \geq d_G(v, x) - r_0$, the random variable $r - (d_G(v, x) - r_0)$ behaves like a geometric random variable from the same original distribution (assuming that $d_G(v, x) - r_0 \geq 0$; otherwise we can simply drop the condition). Therefore:}
    &\qquad\qquad \leq \Pr_{r \sim \geom(p)}(r < w_e) \\
    &\qquad\qquad \leq p \cdot w_e.
\end{flalign*}
Recall that~\smash{$p = \frac{\ln(2s_0 / \delta)}{r_1 - r_0}$}, so the correctness follows.

\paragraph{Failure Probability.}
Next, we analyze that the algorithm returns ``fail''. Recall that this only happens if in steps 2.1.3 and 2.1.4 we sample a radius $r \sim r_0 + \geom(p)$ that satisfies~$r > r_1$. This event happens with probability
\begin{align*}
    \Pr_{r \sim r_0 + \geom(p)}(r > r_1) &= \Pr_{r \sim \geom(p)}(r > r_1 - r_0) \\
    &\leq (1 - p)^{r_1 - r_0} \\
    &\leq \exp(-p(r_1 - r_0)) \\
    &= \exp(-\ln(2s_0) / \delta) \\
    &= \frac{\delta}{2s_0}.
\end{align*}
Whenever the algorithm does not return ``fail'' in step 2.1.4, we are guaranteed to remove at least~$\frac{1}{2} \cdot \frac{m}{s_0}$ edges from the graph $G$ in step~2.1.5. In particular, there can be at most $2s_0$ repetitions of step~2.1.5 and thus at most trials of step 2.1.4. By a union bound it follows that the algorithm indeed returns ``fail'' with probability at most $\delta$.

\paragraph{Running Time.}
Finally, we analyze the running time. We run \cref{lem:cohen} once in step~1, and $\Order(\log n)$ times in step 2.2. Each call runs in time $\Order(m \log^3 n)$, so the total time spent for \cref{lem:cohen} is~$\Order(m \log^4 n)$. There are $\Order(s_0 \log^2 n)$ iterations of step 2.1. In each iteration, we spend time $\Order(1)$ for steps 2.1.1 and 2.1.4. For step 2.1.2 we spend time $\Order(\frac{m}{s_0} \log n)$---indeed, we can test whether~\smash{$|\Bout(v, r_0)| \geq \frac{1}{2} \cdot \frac{m}{s_0}$} by Dijkstra's algorithm, and if the time budget is exceeded it is clear that the bound holds. In steps 2.1.3 and 2.1.5 we spend time proportional to the size of $\Bout(v, r)$, but since we remove the nodes and edges in $\Bout(v, r)$ from $G$ afterwards in total we spend only time $\Order(m \log n)$ in these steps. Therefore, the total running time spend in step~2.1 is~\smash{$\Order(s_0 \log^2 n \cdot \frac{m}{s_0} \log n + m \log n) = \Order(m \log^3 n)$}.
\end{proof}

\subsection{Near-Linear-Time LDD} \label{sec:ldd-fast:sec:algo}
We are ready to state the LDD algorithm. Throughout this section, we fix the following parameters:
\begin{itemize}
    \item $L = \ceil{\log\log m} + 1$,
    \item \smash{$\delta = \frac{1}{\log^{10} m}$},
    \item $r_0 := 0$ and \smash{$r_\ell := r_{\ell-1} + \frac{D}{2^{L-\ell+3}} + \frac{D}{4 L}$} (for $1 \leq \ell \leq L$),
    \item \smash{$s_\ell := \min(2^{2^{L-\ell}}, m+1)$} (for $0 \leq \ell \leq L$).
\end{itemize}
With these parameters in mind, consider \cref{alg:ldd-fast}. The algorithm first deletes all edges with sufficiently large weight. Then it proceeds in $L$ iterations where in each iteration we apply \cref{lem:cut-light} to cut some edges in the graph (or reverse graph). If any of these calls returns ``fail'', then the entire algorithm restarts. In the following \cref{lem:ldd-fast-correctness,lem:ldd-fast-prob,lem:ldd-fast-time} we will show that this procedure correctly implements the algorithm claimed in \cref{thm:main-fast}.

\begin{algorithm}[t]
\caption{The near-linear-time near-optimal LDD, see \cref{thm:main-fast}.} \label{alg:ldd-fast}
\begin{enumerate}
    \item Initially let $S \subseteq E$ be the set of edges of weight at least $\frac{D}{4L}$, and remove these edges from $G$
    \item For $\ell \gets L, \dots, 1$:
    \begin{enumerate}
        \item[2.1.] Run \cref{lem:cut-light} on $G$ with parameters $\delta, r_{\ell-1}, r_\ell, s_{\ell-1}, s_{\ell}$, and let $S^+_\ell, R^+_\ell$ denote the resulting sets.
        \item[2.2.] Compute the strongly connected components in $(G \setminus S^+_\ell)[V \setminus R^+_\ell]$. Recur on each such component and add the recursively computed cut edges to $S$.
        \item[2.3.] Update $G \gets G[R^+_\ell]$ and $S \gets S \cup S^+_\ell$.
        \item[2.4.] Run \cref{lem:cut-light} on $\rev(G)$ with parameters $\delta, r_{\ell-1}, r_\ell, s_{\ell-1}, s_{\ell}$, and let $S^-_\ell, R^-_\ell$ denote the resulting sets.
        \item[2.5.] Compute the strongly connected components in $(G \setminus S^-_\ell)[V \setminus R^-_\ell]$. Recur on each such component and add the recursively computed cut edges to $S$.
        \item[2.6.] Update $G \gets G[R^-_\ell]$ and $S \gets S \cup S^-_\ell$.
    \end{enumerate}
    \item If any of the $2L$ previous calls to \cref{lem:cut-light} returns ``fail'', then we restart the entire algorithm (i.e., we reset $G$ to be the given graph, we reset $S \gets \emptyset$, and we start the execution from step~1).
\end{enumerate}
\end{algorithm}

\begin{lemma}[Correctness of \cref{alg:ldd-fast}] \label{lem:ldd-fast-correctness}
Let $u, v$ be two nodes lying in the same strongly connected component in $G \setminus S$. Then $d_G(u, v) \leq D$.
\end{lemma}
\begin{proof}
For clarity, let us denote the original (given) graph $G$ by $G_0$, and let $G$ denote the graph that is being manipulated by the algorithm. We prove the claim by induction. In the base case (when the graph has constant size) the statement can easily be enforced, so focus on the inductive case. Let $u, v$ be two arbitrary nodes.

For any iteration $\ell$, by \cref{lem:cut-light}~(i) it is clear that if $u \in R^+_\ell$ and $v \not\in R^+_\ell$ (or similarly for~$R^-_\ell$), then $u$ and $v$ cannot end up in the same strongly connected component in $G \setminus S$. If both $u, v \not\in R^+_\ell$ (or similarly for $R^-_\ell$) then the claim follows by induction since we recur on $(G \setminus S^+_\ell)[V \setminus R^+_\ell]$. This leaves as the only remaining case that $u, v \in R^+_\ell$. In particular, after completing all $L$ iterations, the only nodes left to consider are the nodes that are still contained in $G$ when the algorithm terminates. Let us call a node $v$ \emph{heavy} if
\begin{equation*}
    |\Bout_{G_0}(v, \tfrac D2)| \geq \frac{m}{2} \qquad\text{and}\qquad |\Bin_{G_0}(v, \tfrac D2)| \geq \frac{m}{2}.
\end{equation*}
In the following paragraph we prove that the only nodes $v$ remaining in $G$ are either (i) heavy or (ii) have no out- or no in-neighbors. In case (ii) clearly $v$ forms a strongly connected component in~$G \setminus S$ on its own. For (i), we claim that for any pair $u, v$ of heavy nodes it holds that $d_{G_0}(u, v) \leq D$---indeed, the out-ball~\smash{$\Bout_{G_0}(u, \frac D2)$} and the in-ball~\smash{$\Bin_{G_0}(v, \frac D2)$} necessarily intersect. It similarly holds that $d_{G_0}(v, u) \leq D$.

To prove the missing claim, let $v$ be a non-heavy node; we show that if $v$ remains in the graph~$G$ then it has no out- or no in-neighbors. By induction we show that
\begin{equation*}
    |\Bout_G(v, r_\ell)| \leq \frac{m}{s_\ell} \qquad\text{or}\qquad |\Bin_G(v, r_\ell)| \leq \frac{m}{s_\ell}.
\end{equation*}
Initially, for $\ell = L$, this is clearly true using that $v$ is not heavy, that $s_L = 2$, and that
\begin{equation*}
    r_L = \sum_{\ell=1}^L \parens*{\frac{D}{2^{L-\ell+3}} + \frac{D}{4L}} \leq \frac{D}{8} \cdot \sum_{k=0}^\infty \frac{1}{2^k} + \frac{D}{4} = \frac{D}{4} + \frac{D}{4} = \frac{D}{2}.
\end{equation*}
So now assume inductively that the claim holds for some $\ell \leq L$. Then \cref{lem:cut-light}~(ii) guarantees that
\begin{equation*}
    |\Bout_{G[R^+_\ell]}(v, r_{\ell-1})| \leq \frac{m}{s_{\ell-1}} \qquad\text{or}\qquad |\Bin_{G[R^-_\ell]}(v, r_{\ell-1})| \leq \frac{m}{s_{\ell-1}}.
\end{equation*}
Since we update $G \gets G[R^+_\ell]$ and $G \gets G[R^-_\ell]$, this is exactly as desired. In iteration $\ell = 1$ we thus have that 
\begin{equation*}
    |\Bout_G(v, r_1)| \leq \frac{m}{s_1} < 1 \qquad\text{or}\qquad |\Bin_G(v, r_1)| \leq \frac{m}{s_1} < 1,
\end{equation*}
that is, $v$ has no outgoing or incoming edges of weight less than $r_1$. Recall that initially we remove all edges of weight larger than $\frac{D}{4L} \leq r_1$, and therefore $v$ truly has no outgoing or incoming edges. This completes the proof.
\end{proof}

\begin{lemma}[Edge Cutting Probability of \cref{alg:ldd-fast}] \label{lem:ldd-fast-prob}
For any edge $e \in E$, we have
\begin{equation*}
    \Pr(e \in S) \leq \Order\parens*{\frac{w_e}{D} \cdot \log n \log\log n + \frac{1}{\poly(n)}}.
\end{equation*}
\end{lemma}
\begin{proof}
Let $e = (x, y) \in E$ be an arbitrary edge. The algorithm certainly cuts $e$ if it has weight at least $\frac{D}{4L}$, but in this case the right-hand side exceeds $1$. Otherwise, it only cuts edges via \cref{lem:cut-light} or via recursive calls. Since \cref{lem:cut-light} never cuts edges whose source lies in the returned set~$R$, each edge becomes relevant only in the first iteration $\ell$ when $x \in R_\ell^+$, or~$x \not\in R_\ell^+$ and~\makebox{$x \in R_\ell^-$}. Let us focus on the former case; the latter is analogous. Conditioning on the event that $x \in R_\ell^+$, the edge~$e$ can be cut in three ways:
\begin{enumerate}
    \item $e$ is cut by the call to \cref{lem:cut-light} (i.e., $e \in S^+_\ell$),
    \item $e$ is cut in the recursive call, or
    \item the algorithm fails and cuts $e$ during the restart.
\end{enumerate}
In particular, note that the edge cannot be cut in later iterations $\ell' < \ell$, as then we have already removed $x$ from the current graph $G$.

The probability of the first event is at most as follows, using \cref{lem:cut-light}~(iii):
\begin{align*}
    \Pr(e \in S^+_\ell \mid x \not\in R^+_\ell)
    &\leq \frac{w_e \ln(2s_{\ell-1} / \delta)}{r_\ell - r_{\ell-1}} \\
    &\leq \frac{w_e \ln(2^{2^{L - \ell + 1}} \cdot (\log m)^{10})}{\frac{D}{2^{L-\ell+3}} + \frac{D}{4L}} \\
    &\leq \frac{w_e}{D} \cdot \frac{2^{L - \ell + 1} + 10L}{\max\set*{\frac{1}{2^{L-\ell+3}}, \frac{1}{4L}}} \\
    &\leq \frac{w_e}{D} \cdot (2^{L-\ell+3} + 10L) \cdot \min\set*{2^{L-\ell+3}, 10L} \\
    &\leq \frac{w_e}{D} \cdot 2^{L-\ell} \cdot 160L.
\end{align*}
In the last step we used that $(a + b) \min\set{a, b} \leq 2 \max\set{a, b} \min\set{a, b} = 2ab$.

To analyze the events 2 and 3, let us write $p(m) = \Pr(e \in S)$ (where $m$ is the number of edges in the input). Note that each recursive call in the $\ell$-th iteration is on a graph with at most~\smash{$\frac{3}{2} \cdot \frac{m}{s_\ell} \leq \frac32 \cdot \frac{m}{2^{2^{L-\ell}}}$} edges by \cref{lem:cut-light}~(i). Therefore, the probability of event 2 is at most~\smash{$p(\frac32 \cdot \frac{m}{2^{2^{L-\ell}}})$}.

Next, focus on event 3. Each call to \cref{lem:cut-light} returns ``fail'' with probability at most~$\delta$, and by a union bound over the at most $2L$ calls the algorithm restarts with probability at most $2L\delta$. If the algorithm restarts, then clearly its randomness is independent of the previous iterations and thus it cuts $e$ with probability $p(m)$. Hence, the probability of the event 3 is at most $p(m) \cdot 2L\delta$.

By a union bound over these three probabilities we obtain that
\begin{flalign*}
    p(m) &\leq \max_\ell \parens*{\frac{w_e}{D} \cdot 2^{L-\ell} \cdot 160L + p\parens*{\frac32 \cdot \frac{m}{2^{2^{L-\ell}}}} + p(m) \cdot 2L \delta},
\end{flalign*}
or equivalently,
\begin{flalign*}
    p(m) &\leq \frac{1}{1 - 2L\delta} \cdot \max_\ell \parens*{\frac{w_e}{D} \cdot 2^{L-\ell} \cdot 160L + p\parens*{\frac32 \cdot \frac{m}{2^{2^{L-\ell}}}}} \\
    &\leq \parens*{1 + 4L\delta} \cdot \max_\ell \parens*{\frac{w_e}{D} \cdot 2^{L-\ell} \cdot 160L + p\parens*{\frac32 \cdot \frac{m}{2^{2^{L-\ell}}}}}.
\end{flalign*}
This recurrence solves to
\begin{flalign*}
    p(m) &\leq (1 + 4L\delta)^{\log_{4/3}(m)} \cdot \frac{w_e}{D} \cdot 320L \cdot \log m,
\end{flalign*}
as can easily be verified as follows; here we will bound~\smash{$\frac32 \cdot \frac{m}{2^{2^{L-\ell}}}$} on the one hand by $\leq \frac34 m$ and on the other hand by~\smash{$\leq \frac{m}{2^{2^{L-\ell-1}}}$}:
\begin{flalign*}
    p(m) &\leq \parens*{1 + 4L\delta} \cdot \max_\ell \parens*{\frac{w_e}{D} \cdot 2^{L-\ell} \cdot 160L + p\parens*{\frac32 \cdot \frac{m}{2^{2^{L-\ell}}}}} \\
    &\leq \parens*{1 + 4L\delta} \cdot \max_\ell \parens*{\frac{w_e}{D} \cdot 2^{L-\ell} \cdot 160L + (1 + 4L\delta)^{\log_{4/3}(m)-1} \cdot \frac{w_e}{D} \cdot 320L \cdot \log \frac{m}{2^{2^{L-\ell-1}}}} \\
    &\leq \parens*{1 + 4L\delta}^{\log_{4/3}(m)} \cdot \max_\ell \parens*{\frac{w_e}{D} \cdot 2^{L-\ell} \cdot 160L + \frac{w_e}{D} \cdot 320L \cdot \parens*{\log m - 2^{L-\ell-1}}} \vphantom{\parens*{\frac{m}{2^{2^{L-\ell-1}}}}} \\
    &= \parens*{1 + 4L\delta}^{\log_{4/3}(m)} \cdot \frac{w_e}{D} \cdot 320L \cdot \log m. \vphantom{\parens*{\frac{m}{2^{2^{L-\ell-1}}}}}
\end{flalign*}
Finally, we plug in the chosen values for $L \leq \Order(\log\log m)$ and $\delta = (\log m)^{-10}$ to obtain that
\begin{align*}
    p(m) &\leq \parens*{1 + \frac{\Order(\log\log m)}{(\log m)^{10}}}^{\log_{4/3}(m)} \cdot \Order\parens*{\frac{w_e}{D} \cdot \log m \log\log m} \\
    &\leq \exp\parens*{\frac{\Order(\log\log m)}{(\log m)^{10}} \cdot \log_{4/3}(m)} \cdot \Order\parens*{\frac{w_e}{D} \cdot \log m \log\log m} \\
    &\leq \Order\parens*{\frac{w_e}{D} \cdot \log m \log\log m},
\end{align*}
as claimed. A final detail is that all previous bounds condition on the high-probability event that none of the calls to \cref{lem:cut-light} returns an invalid answer (but not ``fail''). To take this into account, the edge cutting probability increases additively by $\frac{1}{\poly(n)}$ (for an arbitrarily large polynomial).
\end{proof}

\begin{lemma}[Running Time of \cref{alg:ldd-fast}] \label{lem:ldd-fast-time}
The expected running time is $\Order(m \log^5 n \log\log n)$.
\end{lemma}
\begin{proof}
Ignoring the cost of recursive calls, one execution of \cref{alg:ldd-fast} involves $\Order(L)$ calls to \cref{lem:cut-light} each of which runs in time $\Order(m \log^4 n)$. (The additional computations such as computing strongly connected components runs in linear time.) To take the recursive calls and restarts into account, let~$T(m)$ denote the running time given $m$ edges. We recurse on subgraphs with, say,~$m_1, \dots, m_r$ edges where $m_i \leq \frac34 m$ and $\sum_{i=1}^r m_i \leq m$. Moreover, the probability that the algorithm restarts is at most $2L \delta \leq (\log m)^{-9}$. Thus, we can bound
\begin{equation*}
    T(m) = \sum_{i=1}^r T(m_i) + 2L\delta T(m) + \Order(m \log^4 n \log\log n),
\end{equation*}
which, similarly to the analysis in \cref{lem:ldd-fast-prob}, can be solved to $T(m) \leq \Order(m \log^5 n \log\log n)$.
\end{proof}

We remark that we have not attempted to optimize the log-factors in the running time here and it is likely that the overhead can be reduced. An obvious improvement to shave one log-factor from the running time is to employ a faster priority queue in Dijkstra's algorithm, e.g.\ as developed by Thorup~\cite{Thorup03}.
\bibliographystyle{plainurl}
\bibliography{ref}

\end{document}
