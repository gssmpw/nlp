\section{Near-Optimal LDDs Deterministically} \label{sec:ldd-deterministic}
In this section, we present the deterministic algorithm for computing a near-optimal LDD, thereby proving our second main theorem:

\thmMainDet*

To this end, we utilize many of the same building blocks we have already introduced in \Cref{sec:ldd-expander}. In particular, we follow the general framework of Multiplicative Weights Update to reduce the computation of an LDD to solving the cost-minimizing task. The full proof of \Cref{thm:main-det} can be found in the end of the section. The following lemma restates the MWU method algorithmically; we omit a proof as it follows exactly the proof of \cref{lem:mwu} in \Cref{sec:ldd-expander}. 

\begin{lemma}[Algorithmic Multiplicative Weight Update]
Let $G = (V, E)$ be a directed graph and let $D \geq 1$. Suppose that there is an algorithm $\mathcal A$ that, given $G$, $D$ and a cost function $c : E \to [|V|^{10}]$, computes a set of edges $S \subseteq E$ satisfying the following properties:
\begin{itemize}
	\item For any two nodes $u, v \in V$ that are part of the same strongly connected component in $G \setminus S$, we have $d_G(u, v) \leq D$ and $d_G(v, u) \leq D$.
	\item $c(S) \leq c(E) \cdot \frac{L}{D}$.
\end{itemize}
Then there is a deterministic algorithm to compute an LDD with loss $\Order(L)$ for $G$ (i.e., we compute the full support of a uniform distribution over $\Order(D \log n)$ cut sets). It runs in time~\smash{$\widetilde\Order(m D)$} and issues $\Order(D \log n)$ oracle calls to $\mathcal A$.
\end{lemma}

For the remainder of this section we will therefore focus on the same cost minimizer setting: Given a directed graph $G = (V, E, c)$ with edge capacities (and unit lengths), the goal is select a set of cut edges $S \subseteq E$ such that $c(S) \leq c(E) \cdot \Order(\frac{1}{D} \cdot \log n \log\log n)$ and such that all strongly connected components in the remaining graph $G \setminus S$ have (weak) diameter at most $D$.

The following lemma is a consequence of the lopsided expander machinery set up before:

\begin{lemma}[Finding Sparse Cuts] \label{lem:sparse-cut-det}
Let $G = (V, E, c)$ be a directed graph, let $D \geq \log \vol(V)$ and let $v \in V$. Then there is some $\psi = \Order(\frac{1}{D} \cdot \log\log \vol(V))$ and an algorithm to determine which of the following cases applies:
\begin{enumerate}[label=(\roman*)]
	\item There is a radius $0 \leq r \leq D$ with $\psi(B^+(v, r)) \leq \psi$ and $c(B^+(v, r)) \leq 0.95 \cdot \vol(V)$.\\(In this case the algorithm runs in linear time in the number of edges incident to $B^+(v, r)$.)
	\item Or, there is a radius $0 \leq r \leq D$ such that $\psi(\overline{B^-(v, r)}) \leq \psi$ and $c(B^-(v, r)) \leq 0.95 \cdot \vol(V)$.\\(In this case the algorithm runs in linear time in the number of edges incident to $B^-(v, r)$.)
	\item Or, $c(B^+(v, D) \cap B^-(v, D)) \geq 0.9 \cdot \vol(V)$.\\(In this case the algorithm runs in time~\smash{$\Order(m)$}.)
\end{enumerate}
\end{lemma}
\begin{proof}
Existentially the statement follows from \cref{lem:lopsided-expansion-boosted} applied with parameters $\psi = \Theta(\frac{1}{D} \cdot \log\log \vol(V))$ (with an appropriately large hidden constant) and $\alpha = 0.05$. This lemma states that $\psi(B^+(v, r)) \leq \psi$ for some radius $0 \leq r \leq D - 1$ (proving (i)) or that~\smash{$\vol(B^+(v, D - 1)) \geq 0.95 \vol(V)$}. Applying the same statement to the reverse graph similarly gives that $\psi(\overline{B^-(v, r)}) \leq \psi$ for some radius $0 \leq r \leq D - 1$ (proving (ii)) or that~\smash{$c(\overline{B^-(v, D - 1)}, B^-(v, D - 1)) \geq 0.95 \cdot \vol(V)$}. In the only remaining case we thus have both
\begin{align*}
	c(B^+(v, D - 1), V) \geq 0.95 \cdot \vol(V)\quad\text{and}\quad c(\overline{B^-(v, D - 1)}, V) \geq 0.95 \cdot \vol(V).
\end{align*}
Combining both statements we obtain that edges of total capacity at least $0.9 \cdot \vol(V)$ must lie in $B^+(v, D - 1) \times B^-(v, D - 1)$. In particular, it follows that $c(B^+(v, D) \cap B^-(v, D)) \geq 0.9 \cdot \vol(V)$ thereby proving (iii).

To make the lemma algorithmic, we simultaneously grow an out-ball $B^+(v, r^+)$ and an in-ball~$B^-(v, r^-)$ around the node $v$. Explicitly, we start with $r^+ = 0$ and increase $r^+$ step by step to compute $B^+(v, r^+)$ (with breadth-first search). We can, without overhead, keep track of the current volume of $B^+(v, r^+)$. If we at some point encounter that $B^+(v, r)$ is $\psi$-lopsided sparse, then we stop and report output (i). Similarly, we start with $r^- = 0$ and step by step explore $B^-(v, r^-)$. If at some point $B^-(v, r^-)$ becomes a $\psi$-lopsided sparse cut, we stop and report answer (ii). We interleave these two computations so that when we output (i) the overhead of exploring the in-ball~$B^-(v, r^-)$ incurs only a constant factor in the running time, and similarly for (ii). In the remaining case where we have not encountered a sparse cut in the graph before reaching $r^+ = r^- = D$, we report (iii). In this case we indeed spend time at most $\Order(m)$.
\end{proof}

Having established \cref{lem:sparse-cut-det}, now consider the algorithm in \cref{alg:det}. In summary, it runs in two phases. In Phase (I) we first repeatedly select a node $v$ and attempt to cut a lopsided sparse cut around $v$ (i.e., we cut the edges in $\delta^+(B^+(v, r))$ for some radius $r$). We only execute these cuts, however, until we find a node $z$ for which $c(B^+(z, D') \cap B^-(z, D')) \geq 0.9 \cdot \vol(V)$---that is, both the radius-$D'$ out- and in-balls of $z$ make up for a big constant fraction of the entire graph. We call $z$ a \emph{center} node and move on to phase (II). In this phase we repeat the same steps as in Phase (I), but we only choose nodes $v$ that have distance at least $2D'$ (in one direction or the other) to the center $z$. The intuition is that we can never find a second node $z'$ which equally makes up for the entire graph, as then $z$ and $z'$ would have to be connected by a short path. In the remainder of this section we formally analyze \cref{alg:det}.

\begin{algorithm}[t]
	\caption{The deterministic near-optimal LDD, see \Cref{thm:main-det}.} \label{alg:det}
	\begin{enumerate}[label=\arabic*.]
		\item[(I)] Repeat the following steps: Take an arbitrary node $v \in V$ and apply \cref{lem:sparse-cut-det} with parameter \smash{$D' = \floor{\frac{D}{4}}$}. Depending on the output execute the following steps:
		\begin{enumerate}[label=(\roman*)]
			\item Cut all edges in $\delta^+(B^+(v, r))$, recurse on the induced graph $G[B^+(v, r)]$, then remove all nodes in $B^+(v, r)$ from the graph.
			\item Cut all edges $\delta^-(B^-(v, r))$, recurse on the induced graph $G[B^-(v, r)]$, then remove all nodes in $B^-(v, r)$ from the graph.
			\item Remember $z \gets v$ (called the \emph{center} node) and continue with Phase (II).
		\end{enumerate}
		\item[(II)] Compute the sets
		\begin{align*}
			X &= B^+(z, D') \cap B^-(z, D'), \\
			Y &= B^+(z, 2D') \cap B^-(z, 2D').
		\end{align*}
		Then repeat the following steps while there still exists nodes in $V \setminus Y$: Take an arbitrary node~\makebox{$u \in V \setminus Y$} and apply \cref{lem:sparse-cut-det} with parameter~$D'$. Depending on the output execute the following steps:
		\begin{enumerate}[label=(\roman*)]
			\item Cut all edges in $\delta^+(B^+(v, r))$, recurse on the induced graph $G[B^+(v, r)]$, then remove all nodes in $B^+(v, r)$ from the graph.
			\item Cut all edges $\delta^-(B^-(v, r))$, recurse on the induced graph $G[B^-(v, r)]$, then remove all nodes in $B^-(v, r)$ from the graph.
		\end{enumerate}
	\end{enumerate}
\end{algorithm}

\begin{lemma}[Total Cost of \cref{alg:det}] \label{lem:ldd-det-cost}
Let $S \subseteq E$ denote the set of edges cut by \cref{alg:det}. Then $c(S) \leq c(E) \cdot \Order(\frac{1}{D} \cdot \log \vol(V) \log\log \vol(V))$.
\end{lemma}
\begin{proof}
Observe that the algorithm only cuts the edges of $\psi$-lopsided sparse cuts for some parameter $\psi = \Order(\frac{1}{D} \cdot \log\log \vol(V))$. Indeed, the algorithm only cuts edges from $B^+(v, r)$ to~\smash{$\overline{B^+(v, r)}$} in subcase~(i), and edges from~\smash{$\overline{B^-(v, r)}$} to $B^-(v, r)$ in subcase~(ii). In both these cases \cref{lem:sparse-cut-det} gives exactly the guarantee that these respective cuts are $\psi$-lopsided sparse.

With this in mind, we can apply exactly the same potential argument as in the proof of \cref{lem:exp-decomp}. To avoid repetitions, we only give a quick reminder here: We initially associate to each edge a potential of~\smash{$c(e) \cdot \log\vol(V)$}. Then, following the calculations as in \cref{lem:exp-decomp}, we can free a potential of at least $c(e) / \psi$ for each cut edge, proving that $c(S) \leq c(E) \cdot \psi \log\vol(V)$ as claimed.
\end{proof}

\begin{lemma}[Well-Definedness of \cref{alg:det}] \label{lem:ldd-det-well-defined}
While executing Phase~(II) of \cref{alg:det}, the subcase~(iii) never happens.
\end{lemma}
\begin{proof}
Let $G = (V, E)$ denote the graph at the transition from Phase (I) to Phase (II). Suppose for contradiction that during the execution of Phase (II) we find a node which falls into case~(iii). That is, let $G' = (V', E')$ denote the graph remaining at this point in the execution of the algorithm, and suppose that there is a node $v \in V'$ for which $\vol(G'[B^+(v, D') \cap B^-(v, D')]) \geq 0.9 \cdot \vol(G')$. Since we have picked~\makebox{$v \not\in Y$}, we either have that $d_G(z, v) > 2D'$ or that $d_G(v, z) > 2D'$; focus on the former case. Then for all~\makebox{$x \in X$}, we have $d_G(x, v) > D'$ (as otherwise~\makebox{$d_G(z, v) \leq d_G(z, x) + d_G(x, v) \leq 2D'$}), and thus $X$ and $B^-(v, D')$ are disjoint. But this leads to a contradiction as supposedly both $\vol(G[X]) \geq 0.9 \vol(G)$ and $\vol(G[B^-(v, D')]) \geq 0.9 \vol(G') \geq 0.9 \cdot 0.9 \cdot \vol(G) \geq 0.8 \vol(G)$, leading to a total volume of more than $\vol(G)$. Here in the last step we have used that invariantly $\vol(G') \geq 0.9 \cdot \vol(G)$ since the algorithm can never cut nodes from $X$. The remaining case is symmetric.
\end{proof}

\begin{lemma}[Correctness of \cref{alg:det}] \label{lem:ldd-det-correctness}
Let $S \subseteq E$ denote the set of edges cut by \cref{alg:det}. Then for any two nodes $u, w$ in the same strongly connected component in $G \setminus S$, it holds that $d_G(u, w) \leq D$ and $d_G(w, u) \leq D$.
\end{lemma}
\begin{proof}
We prove the statement by induction. The base case is clear for an appropriate implementation of constant-size graphs. For the inductive step, consider an execution of \cref{alg:det} and an arbitrary pair of nodes $u, w$. Whenever the algorithm cuts a ball $B^+(v, r)$ and we have $u \in B^+(v, r)$ and $w \not\in B^+(v, r)$, then $u$ and $w$ do not end up in the same strongly connected component in~$G \setminus S$. If instead both $u, w \in B^+(v, r)$ then the claim follows by induction as the algorithm recurses on the subgraph induced by $B^+(v, r)$. The only remaining case is if both $u$ and $w$ are never cut during the execution of the algorithm. Clearly the algorithm cannot have terminated after Phase (I) (as then there would be no nodes left in the graph), so the algorithm has reached Phase (II). Moreover, we have $u, w \in Y$ as otherwise the algorithm would not have terminated yet. But then by definition, we have $d_G(v, u) \leq 2D'$, $d_G(u, v) \leq 2D'$, $d_G(w, v) \leq 2D'$ and $d_G(v, w) \leq 2D'$. Putting all these together we have that $d_G(u, w) \leq 4D' \leq D$ and $d_G(w, u) \leq 4D' \leq D$ as claimed.
\end{proof}

\begin{lemma}[Running Time of \cref{alg:det}] \label{lem:ldd-det-time}
\cref{alg:det} runs in time $\Order(m \log \vol(V))$.
\end{lemma}
\begin{proof}
Consider one execution of \cref{alg:det}. We spend time $\Order(m)$ once to compute the sets $X$ and $Y$ at the beginning of Phase (II). Other than that, all steps only run in local fragments of the graph. Specifically, whenever the algorithm cuts a ball $B^+(v, r)$ we spend time $\Order(|\delta^+(B^+(v, r))|)$ by \cref{lem:sparse-cut-det} (i.e., time proportional to the number of edges incident to $B^+(v, r)$), but then we delete all nodes in $B^+(v, r)$ (and thereby also all edges in $\delta^+(B^+(v, r))$). We can thus express the running time $T(m)$ by the recurrence
\begin{equation*}
	T(m, C) \leq \Order(m) + \sum_i T(m_i, C_i),
\end{equation*}
where $m_i$ is the number of edges and $C_i$ is the total capacity of the $i$-th recursive call. Clearly we have that $\sum_i m_i \leq m$. Moreover, we only cut balls with $c(B^+(v, r)) \leq 0.95 \cdot \vol(V)$ by \cref{lem:sparse-cut-det} (and similarly for the in-balls $B^-(v, r)$). This drop in capacity bounds the recursion depth by $\Order(\log \vol(V))$ and so the recursion solves to $\Order(m \log |V|)$.
\end{proof}

This completes the analysis of \cref{alg:det} and puts us in the position of completing the proof of \cref{thm:main-det}.

\begin{proof}[Proof of \cref{thm:main-det}]
As before, we first turn the given graph into a unit-length graph which only blows up the number of nodes and edges by a factor of $D$ (all edges with length $> D$ can anyways be removed for free). Then, by \cref{lem:mwu} it suffices to design a deterministic algorithm for the cost minimization problem (indeed, this algorithm is then turned into a deterministic LDD algorithm by \cref{lem:mwu}, at the cost of another factor-$\Order(D \log n)$ blow-up). Finally, we can assume that~\makebox{$D \geq \log \vol(V)$} in the remaining task, as otherwise it is within our budget to simply remove all edges.

To solve the cost minimization problem, we run \cref{alg:det}. Let $S \subseteq E$ denote the edges cut by \cref{alg:det}. Then, by \cref{lem:ldd-det-correctness} indeed the remaining graph $G \setminus S$ has (weak) diameter at most~$D$. By \cref{lem:ldd-det-cost} the total capacity of the cut edges is $c(S) \leq c(E) \cdot \Order(\frac{1}{D} \cdot \log\vol(V) \log\log\vol(V))$. And by \cref{lem:ldd-det-time} the running time is near-linear in the number of edges. All three properties are as required by \cref{lem:mwu}, ultimately leading to an LDD algorithm with loss $\Order(\log n \log\log n)$ (using that $\log \vol(V) = \Order(\log n)$) and in time $\widetilde\Order(m D^2)$.
\end{proof}