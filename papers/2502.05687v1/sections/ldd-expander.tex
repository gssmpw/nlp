\section{Near-Optimal LDDs via Expander Decompositions} \label{sec:ldd-expander}

In this section, we show the existence of a near-optimal LDD, thereby proving our first main theorem: 

\thmMainExistential*

We introduce some technical lemmas in order to build up the framework for the proof of \Cref{thm:main-existential}, which can be found in the end of the section.


\subsection{Reduction to Cost-Minimizers}

\mwu*

\begin{proof}
	Let $G = (V, E)$ denote the graph for which we are supposed to design the LDD. Let us also introduce edge costs that are initially defined as $c(e) = 1$ for all edges. We will now repeatedly call the cost-minimizer, obtain a set of cut edges $S \subseteq E$, and then update the edge costs by $c(e) \gets 2 \cdot c(e)$ for all $e \in S$. We stop the process after $R = \log |E| \cdot D/L$ iterations, and let $\mathcal S$ denote the collection of all~$R$ sets~$S$ that we have obtained throughout. We claim that the uniform distribution on $\mathcal S$ is the desired LDD for $G$.
	
	It is clear that all for all sets $S \in \mathcal S$ the diameter condition is satisfied. We show that additionally for all edges $e \in E$ we have that
	\begin{equation*}
		\Pr(e \in S) \leq \frac{10L}{D},
	\end{equation*}
	for $S$ sampled uniformly from $\mathcal S$. Suppose otherwise, then in particular we have increased the cost of $e$ to at least
	\begin{equation*}
		c(e) \geq 2^{\frac{10L}{D} \cdot R} = 2^{10 \log |E|} = |E|^{10}.
	\end{equation*}
	On the other hand, let $c'$ denote the adapted costs after running the process for one iteration. Then the total cost increase is
	\begin{equation*}
		\sum_{e \in E} c'(e) - c(e) = \sum_{e \in S} c'(e) - c(e) = \sum_{e \in S} c(e) = c(S) \leq c(E) \cdot \frac{L}{D}.
	\end{equation*}
	That is, with every step of the process the total cost increases by a factor of $(1 + \frac{L}{D})$ and thus the total cost when the process stops is bounded by
	\begin{equation*}
		|E| \cdot \parens*{1 + \frac{L}{D}}^R \leq |E| \cdot e^{\frac{L}{D} \cdot R} = |E| \cdot e^{\log |E|} \leq |E|^3,
	\end{equation*}
	leading to a contradiction. The same argument shows that all costs are bounded by $|E|^3 \leq |V|^6$ throughout.
\end{proof}

\lemLexpDecomp*

\begin{proof}
Consider the following algorithm: If there is no $\psi$-lopsided sparse cut then the graph is a $\psi$-lopsided expander by definition and we stop. Otherwise, there exists a $\psi$-lopsided sparse cut~$(U, \overline U)$. We then distinguish two cases: If $\vol(U) \leq \vol(\overline U)$ then we remove all edges from~$U$ to~$\overline U$, and otherwise we remove all edges from~$\overline U$ to~$U$ (in both cases placing these edges in~$S$). Then we recursively continue on all strongly connected components in the remaining graph~$G \setminus S$. 

It is clear that all strongly connected components in the remaining graph $G \setminus S$ are $\psi$-lopsided expanders, but it remains to show that we cut edges with total capacity at most $c(E) \cdot \psi \log c(E)$. Imagine that initially we associate to each edge $e$ a \emph{potential} of~\makebox{$c(e) \cdot \log c(E)$}. The total initial potential is thus $\sum_e c(e) \log c(E) = c(E) \log c(E)$. Throughout the procedure we maintain the invariant that each edge holds a potential of at least $c(e) \log \vol(C)$, where $C$ is the strongly connected component containing edge $e$. Focus on any recursion step and its current strongly connected component~$C$, and let $C = U \sqcup \overline U$ denote the current $\psi$-lopsided sparse cut. Assume first that $\vol(U) \leq \vol(\overline U)$. Observe that an edge $e \in U$ suddenly needs to hold a potential of $c(e)\log c(U)$ instead of $c(e)\log c(C)$. Hence, the amount of freed potential in $U$ is at least
\begin{align*}
	\sum_{e \in E \cap (U \times V)} c(e) (\log \vol(C) - \log \vol(U)) &=
	\sum_{e \in E \cap (U \times V)} c(e)  \cdot \log \frac{\vol(C)}{\vol(U)} = \vol(U) \cdot \log \frac{\vol(C)}{\vol(U)}.
\end{align*}
On the other hand, since $(U, \overline U)$ is a $\psi$-lopsided sparse cut we have that
\begin{equation*}
	\psi \geq \psi(U) = \frac{c(U, \overline U)}{\minvol(U) \cdot \log \frac{\vol(C)}{\minvol(U)}} = \frac{c(U, \overline U)}{\vol(U) \cdot \log \frac{\vol(C)}{\vol(U)}}.
\end{equation*}
Putting these together, this means any cut edge $e$ from $U$ to $\overline U$ can get ``paid'' a potential of~\smash{$c(e) \cdot \psi^{-1}$} while still maintaining the potential invariant. (Note that here we only exploit the potential freed by the smaller side of the cut $U$, and forget about the overshoot potential in the larger side $\overline U$.) A symmetric argument applies when $\vol(U) < \vol(\overline U)$.

All in all, we start with a total potential of $c(E) \log c(E)$ and pay for each cut edge $e \in S$ with a potential of at least~\smash{$c(e) \cdot \psi^{-1}$}. This implies that $c(E) \log c(E) \geq c(S) \cdot \psi^{-1}$ and the claim follows.
\end{proof}

To prove that lopsided expanders have small diameter, we first establish the following technical lemma. 

\begin{lemma} \label{lem:lopsided-expansion}
Let $G = (V, E, c)$ be a directed graph and let $\psi > 0$. For any node $v \in V$ there is some radius $R = \Order(\psi^{-1} \log\log \vol(V) + \log \vol(V))$ such that one of the following two properties holds:
\begin{itemize}
	\item $\vol(B^+(v, R)) \geq \frac{1}{2} \cdot \vol(V)$, or
	\item $\psi(B^+(v, r)) \leq \psi$ for some $0 \leq r \leq R$.
\end{itemize}
\end{lemma}
\begin{proof}
We write $\Delta_i = \ceil{\frac{1}{i \psi}}$ and define the radii~\smash{$1 = r_{\ceil{\log \vol(V)}} \leq \dots \leq r_1$} by $r_i = r_{i+1} + \Delta_i$. We prove by induction that~\smash{$\vol(B^+(v, r_i)) \geq 2^{-i} \cdot \vol(V)$}, or alternatively that we find a sparse cut. This is clearly true in the base case for $i = \ceil{\log \vol(V)}$: Either $\vol(B^+(v, 1)) \geq 1$ or $v$ is an isolated node and therefore $\psi(B^+(v, r)) = 0$.

For the inductive case, suppose for the sake of contradiction that $\vol(B^+(v, r_i)) < 2^{-i} \cdot \vol(V)$. By induction we know however that $\vol(B^+(v, r_{i+1})) \geq 2^{-i-1} \cdot \vol(V)$. It follows there is some radius $r_{i+1} \leq r < r_i = r_{i+1} + \Delta_i$ such that
\begin{equation*}
	\frac{\vol(B^+(v, r + 1))}{\vol(B^+(v, r))} \leq 2^{1/\Delta_i} \leq 1 + \frac{1}{\Delta_i}.
\end{equation*}
It follows that
\begin{equation*}
	c(B^+(v, r), \overline{B^+(v, r)}) = \vol(B^+(v, r + 1)) - \vol(B^+(v, r)) \leq \frac{\vol(B^+(v, r))}{\Delta_i}.
\end{equation*}
Therefore the cut induced by $B^+(v, r)$ has lopsided sparsity
\begin{align*}
	\psi(B^+(v, r)) &= \frac{c(B^+(v, r), \overline{B^+(v, r)})}{\minvol(B^+(v, r)) \log \frac{\vol(V)}{\minvol(B^+(v, r))}} \\
	&\leq \frac{c(B^+(v, r), \overline{B^+(v, r)})}{\vol(B^+(v, r)) \log \frac{\vol(V)}{\vol(B^+(v, r))}} \\
	&\leq \frac{1}{\Delta_i \cdot \log \frac{\vol(V)}{\vol(B^+(v, r))}} \\
	&\leq \frac{1}{\Delta_i \cdot \log \frac{\vol(V)}{\vol(B^+(v, r_i))}} \\
	&\leq \frac{1}{\Delta_i \cdot \log(2^i)} \\
	&\leq \psi.
\end{align*}
Here, in the second step we have used that $\minvol(B^+(v, r)) = \vol(B^+(v, r))$ as in the opposite case we have $\vol(B^+(v, r)) \geq \frac{1}{2} \cdot \vol(V)$ which also proves the claim. This finally leads to a contradiction since we assume that the graph is a $\psi$-lopsided expander and thus does not contain $\psi$-lopsided sparse cuts.

In summary, the induction shows that $\vol(B^+(v, r_1)) \geq \frac{1}{2} \cdot \vol(V)$ and thus we may choose $R = r_1$. To prove that $R$ is as claimed, consider the following calculation:
\begin{align*}
	R &= 2 + \sum_{i=1}^{\ceil{\log \vol(V)}} \Delta_i \\
	&= 2 + \sum_{i=1}^{\ceil{\log \vol(V)}} \ceil*{\frac{1}{i \psi}} \\
	&\leq 2 + \ceil{\log \vol(V)} + \sum_{i=1}^{\ceil{\log \vol(V)}} \frac{1}{i \psi} \\
	&\leq \Order(\log\vol(V) + \psi^{-1} \log\log\vol(V)),
\end{align*}
using the well-known fact that the harmonic numbers are bounded by $\sum_{k=1}^n 1/k = \Order(\log n)$.
\end{proof}

One can easily strengthen the lemma as follows. This insight will play a role in the next \cref{sec:ldd-deterministic} in the construction of the deterministic algorithm.

\begin{lemma} \label{lem:lopsided-expansion-boosted}
Let $G = (V, E, c)$ be a directed graph and let $\psi > 0$ and $0 < \alpha < 1$. For any node $v \in V$ there is some radius $R = \Order(\psi^{-1} \log\log \vol(V) + \psi^{-1} \alpha^{-1} + \log \vol(V))$ such that one of the following two properties holds:
\begin{itemize}
	\item $\vol(B^+(v, R)) \geq (1 - \alpha) \cdot \vol(V)$, or
	\item $\psi(B^+(v, r)) \leq \psi$ for some $0 \leq r \leq R$.
\end{itemize}
\end{lemma}
\begin{proof}
Applying the previous lemma with parameter $\psi$ yields $R' = \Order(\psi^{-1} \log\log \vol(V) + \log \vol(V))$ such that either $\vol(B^+(v, R')) \geq \frac{1}{2} \cdot \vol(V)$, or $\psi(B^+(v, r)) \leq \psi$ for some $0 \leq r \leq R'$. In the latter case we are immediately done, so suppose that we are in the former case.

Let $\Delta = \ceil{2 \alpha^{-1} \psi^{-1}}$ and let $R = R' + \Delta$. If $\vol(B^+(v, R)) \geq (1 - \alpha) \cdot \vol(V)$ then we have shown the first case and are done. So suppose that otherwise $\vol(B^+(v, R)) \leq (1 - \alpha) \cdot \vol(V)$. Then due to the trivial bound $\vol(B^+(v, R)) \leq \vol(V)$, there is some radius $R' \leq r \leq R = R' + \Delta$ with
\begin{equation*}
	\frac{\vol(B^+(v, r + 1))}{\vol(B^+(v, r))} \leq 2^{1/\Delta} \leq 1 + \frac{1}{\Delta},
\end{equation*}
and hence,
\begin{equation*}
	c(B^+(v, r), \overline{B^+(v, r)}) = \vol(B^+(v, r + 1)) - \vol(B^+(v, r)) \leq \frac{\vol(B^+(v, r))}{\Delta}.
\end{equation*}
For this radius $r$ it further holds that $\vol(B^+(v, r)) \leq (1 - \alpha) \cdot \vol(V)$ and thus $\vol(\overline{B^+(v, r)}) \geq \alpha \cdot \vol(V)$. In particular, we have that $\minvol(B^+(v, r)) \geq \frac{\alpha}{2} \cdot \vol(V)$. Putting these statements together, we have that
\begin{align*}
	\psi(B^+(v, r))
	&= \frac{c(B^+(v, r), \overline{B^+(v, r)})}{\minvol(B^+(v, r)) \log \frac{\vol(V)}{\minvol(B^+(v, r))}} \\
	&\leq \frac{\vol(B^+(v, r))}{\Delta \cdot \frac{\alpha}{2} \cdot \vol(B^+(v, r)) \log \frac{\vol(V)}{\minvol(B^+(v, r))}} \\
	&\leq \frac{1}{\Delta \cdot \frac{\alpha}{2}} \\
	&\leq \psi,
\end{align*}
witnessing indeed the desired sparse lopsided cut.
\end{proof}

\lemLexpDiam*

\begin{proof}
	Take an arbitrary pair of nodes $v, u$. Applying \cref{lem:lopsided-expansion-boosted} with parameters $\psi$ and $\alpha = \frac{1}{4}$, say, yields a radius $R = \Order(\psi^{-1} \log\log \vol(V) + \log \vol(V))$ such that
	\begin{equation*}
		\vol(B^+(v, R)) \geq \frac{3}{4} \cdot \vol(V),
	\end{equation*}
	and symmetrically,
	\begin{equation*}
		\vol(B^-(u, R)) \geq \frac{3}{4} \cdot \vol(V).
	\end{equation*}
	Therefore, there is some edge $e = (x, y)$ contributing to both of these volumes. Thus $x \in B^+(v, R)$ and $y \in B^-(u, R)$. It follows that
	\begin{equation*}
		d_G(u, v) \leq d_G(u, x) + d_G(x, y) +  d_G(y, u) \leq R + 1 + R = \Order(\psi^{-1} \log\log \vol(V) + \log \vol(V)).
	\end{equation*}
	Since the nodes $u, v$ were chosen arbitrarily this establishes the claimed diameter bound.
\end{proof}

\begin{proof}[Proof of \cref{thm:main-existential}]
Let $G = (V, E, \ell)$ be a directed graph with positive edge lengths. We show that there is an LDD with loss $\Order(\log n \log\log n)$ for $G$. We first deal with two trivial cases: First, if $D \leq \log n / \gamma$ (for some constant $\gamma > 0$ to be determined later) then we simply remove all edges and stop. Second, we remove all edges with length more than $D$ from the graph. In both cases edges can be deleted with probability $1$ without harm.

Next, we transform the graph into $G'$ by replacing each $e$ by a path of $\ell(e)$ unit-length edges. In the following it suffices to design an LDD for the augmented graph; if that LDD cuts any of the edges along the path corresponding to an original edge $e$ we will cut $e$ entirely. An LDD with loss~$L$ in the augmented graph will thus delete an original edge with probability at most $\ell(e) \cdot \frac{L}{D}$ by a union bound. All in all, this transformation blows up the number of nodes and edges in the graph by a factor of at most $D$ (since we removed edges with larger length). Recall that we throughout assume that $D \leq n^c$, for some constant $c$, and thus $|V'| \leq n^{\Order(1)}$.

By \cref{lem:mwu} we further reduce the existence of an LDD of $G'$ to the following cost-minimizer task: View $G'$ as an edge-capacitated graph $G' = (V', E', c)$ for some capacities~\makebox{$c : E' \to [|V'|^{10}]$}. In particular, under this capacity function $G'$ has volume $\vol(V') \leq |V'|^2 \cdot |V'|^{10} = |V'|^{12} = n^{\Order(1)}$. The goal is to delete edges $S \subseteq E'$ in $G'$ so all remaining strongly connected components have (weak) diameter at most $D$, and the total cost of all deleted edges is only $c(S) \leq c(E) \cdot \frac{L}{D}$.

Finally, we apply the Lopsided Expander Decomposition from \cref{lem:lexp-decomp} on $G'$. Specifically, we define
\begin{equation*}
	\psi = \frac{\log\log \vol(V')}{\epsilon D}
\end{equation*}
for some constant $\epsilon > 0$ to be determined later. The Expander Decomposition then cuts edges $S \subseteq E'$ so that each remaining strongly connected component is $\psi$-lopsided expander. Thus, by \cref{lem:lexp-diam} each strongly connected component has diameter
\begin{equation*}
	\Order(\psi^{-1} \log\log \vol(V') + \log \vol(V')) = \Order(\epsilon D + \gamma D).
\end{equation*}
By choosing the constants $\epsilon$ and $\gamma$ to be sufficiently small, the diameter bound becomes $D$ as desired. Moreover, \cref{lem:exp-decomp} guarantees that we cut edges of total capacity
\begin{equation*}
	c(S) \leq c(E') \cdot \psi \log \vol(V') \leq c(E') \cdot \frac{\log \vol(V') \log\log \vol(V')}{\epsilon D},
\end{equation*}
which becomes $\frac{L}{D}$ by choosing
\begin{equation*}
	L = \frac{\log \vol(V') \log\log \vol(V')}{\epsilon} \leq \frac{\log |V'|^{12} \log\log |V'|^{12}}{\epsilon} = \Order(\log n \log\log n)
\end{equation*}
as planned.
\end{proof}