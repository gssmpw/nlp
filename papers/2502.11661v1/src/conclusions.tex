\section{Discussion and Future Work}
In this work, we analyzed the complexity of contract design with single-dimensional types. Despite the recent reduction from multi-dimensional to single-dimensional types by \citet{castiglioni2025reduction} suggested that efficient algorithms might be unattainable, we demonstrate that the problem admits a PTAS when considering additive approximations. Moreover, we show that this result cannot be further improved as the problem does not admit a $\FPTAS$ if $\PolyClass\neq\NP$. This suggests that, to some extent, contract design in the single-dimensional setting is computationally easier than in the multi-dimensional case.
%

%Aside from this computational perspective, an information perspective also corroborated that single-dimensional contract design is simpler than its multi-dimensional counterpart. 

Beyond this computational perspective, an information-theoretic viewpoint also supports the idea that single-dimensional contract design is simpler than its multi-dimensional counterpart.
In an online learning setting, we derive a regret bound of $\widetilde O(\poly(\mathcal I)T^{3/4})$ and a sample complexity bound $\widetilde O\left(\frac{\poly(\mathcal I)}{\eta^4}\right)$, where $\eta$ represents the additive error and $\mathcal I$ denotes the instance size.
%
This stands in contrast to the multi-dimensional case, where regret bounds are either exponential in the instance size or nearly linear.

Several interesting directions remain open.
First, it would be interesting to extend our results to \emph{menus of contracts}. 
%For such a mechanism, the question is also open to what concerns the multiplicative approximation.
While we believe that our $\PTAS$ can be fairly easily extended to menus of deterministic contracts, we don't see a clear connection between our hardness result and the impossibility of designing an $\FPTAS$ for menus of contracts. 
%
Another promising direction is to explore the online setting further, providing both lower bounds (on regret or sample complexity) and positive results under weaker or alternative assumptions. 
A central question is whether the bounded density assumption is truly necessary, or other properties related to the single-dimensional structure of the problem can be exploited. 

%would be of interest. Ultimately, the question is whether the problem possesses additional structure other than the almost-linear structure that it exhibits in the utility space.