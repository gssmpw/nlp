\subsection{Menus of Randomized Contracts}\label{sec:preliminaries_contracts}

\mat{non modificato da qua. Copiato da un altro paper solo per avere un po di notazione}

We study Bayesian principal-agent problems in which there is an additional \emph{type-reporting} stage in which the principal proposes to the agent a menu of randomized contracts to choose from.

A \emph{randomized contract} is a probability distribution $\gamma$ over $\mathbb{R}^m_+$, \emph{i.e.}, over the set of vectors $p \in \mathbb{R}^m_+$ representing all the possible contracts.
% defined by a payment $p_\omega \geq 0$ for each outcome $\omega \in \Omega$.
%
We use $p \sim \gamma$ to denote that the (random) contract $p$ is distributed according to $\gamma$, and write $\mathbb{E}_{p \sim \gamma} [\cdot]$ to indicate the expectation taken with respect to the randomness of $p$.
%
We denote by $\supp (\gamma)$ the support of $\gamma$.
%
When $\gamma$ has a finite support, \emph{i.e.}, $|\supp (\gamma)| < \infty$, we let $\gamma_p $ be the probability that $\gamma$ assigns to contract $p \in \mathbb{R}^m_+$.


A \emph{menu of randomized contracts} is defined by a tuple $\Gamma = \left( \gamma^\theta \right)_{\theta \in \Theta}$ specifying a probability distribution $\gamma^\theta$ over $\mathbb{R}^m_+$ for each agent's type $\theta \in \Theta$.


The interaction between the principal and an agent of type $\theta \in \Theta$ goes as follows:
%
\begin{itemize}
	\item[(i)] the principal publicly commits to a menu $\Gamma = \left( \gamma^\theta \right)_{\theta \in \Theta}$ of randomized contracts;
	%
	\item[(ii)] the agent reports a type $\hat \theta \in \Theta$ to the principal, possibly different from the true type $\theta$;
	% so as to maximize their expected utility $\mathbb{E}_{p \sim \gamma^{\theta'}} \left[ \sum_{\omega \in \Omega} F_{\theta, b^\theta(p), \omega} \, p_\omega - c_{\theta, b^\theta(p)}   \right]$.
	%
	\item[(iii)] the principal draws a contract $p \sim \gamma^{\hat \theta}$ and communicates it to the agent;
	%
	\item[(iv)] the agent plays the best-response action $b^\theta(p)$.
\end{itemize}



The goal of the principal is to commit to a utility-maximizing menu of randomized contracts, selecting among those that are \emph{dominant-strategy incentive compatible} (DSIC).\footnote{Notice that, by a revelation-principle-style argument (see the book by~\citet{shoham2008multiagent} for some examples of these kind of arguments), focusing on DSIC menus of contracts is w.l.o.g. when looking for a principal-optimal menu.}
%
Formally, a menu $\Gamma = \left( \gamma^\theta \right)_{\theta \in \Theta}$ of randomized contracts is DSIC if the following holds:
\begin{equation}\label{eq:dsic}
	\mathbb{E}_{p \sim \gamma^\theta} \left[  \sum_{\omega \in \Omega} F_{\theta, b^\theta(p), \omega} \, p_\omega - c_{\theta, b^\theta(p)} \right]  \geq \mathbb{E}_{p \sim \gamma^{\hat \theta}} \left[  \sum_{\omega \in \Omega} F_{\theta, b^\theta(p),\omega} \, p_\omega - c_{\theta, b^\theta(p)}  \right]    \forall \hat \theta \neq \theta \in \Theta.
\end{equation}
%
Intuitively, the conditions above guarantee that the expected utility of an agent of type $\theta \in \Theta$ under the randomized contract $\gamma^\theta$ is greater than or equal to that obtained under $\gamma^{\hat \theta}$, for any $\hat \theta \neq \theta \in \Theta$.
%
This ensures that the agent is always better off reporting their true type to the principal.
%
Then, the principal's goal is to find a menu $\Gamma = \left( \gamma^\theta \right)_{\theta \in \Theta}$ that is optimal for the following problem:
\begin{equation}\label{eq:problem}
	\max_{\Gamma = \left( \gamma^\theta \right)_{\theta \in \Theta}} \quad \sum_{\theta \in \Theta} \mu_\theta \, \mathbb{E}_{p \sim \gamma^\theta} \left[  \sum_{\omega \in \Omega} F_{\theta, b^\theta(p), \omega} \, r_\omega -\sum_{\omega \in \Omega} F_{\theta, b^\theta(p), \omega} \, p_\omega \right] \quad \textnormal{s.t.} \quad \textnormal{Equation~\eqref{eq:dsic}},
\end{equation}
%
whose objective is the principal's expected utility for a DSIC menu of randomized contracts.


We also consider the case, already investigated by~\citet{guruganesh2021contracts}, in which the menu of contracts is made by non-randomized contracts.
%
Formally, we denote a \emph{menu of deterministic contracts} by a tuple $P = \left(  p^\theta \right)_{\theta \in \Theta}$, containing a contract $p^\theta \in \mathbb{R}^m_+$ for each agent's type $\theta \in \Theta$.
%
In this case, the principal does \emph{not} draw a contract in step (iii) of the interaction, but they simply adopt contract $p^{\hat \theta}$, where $\hat \theta \in \Theta$ is the type reported by the agent.
%
Thus, all the definitions above can be specialized to the case of menus of deterministic contracts by dropping expectations. 