\section{Omitted Proofs from \Cref{sec:PTAS}: Existence of a $\PTAS$}

\lemmadiscretization*

\begin{proof}
	For the first part, we simply use the law of total expectation 
	\begin{align*}
		\mathbb{E}_{\theta\sim\Gamma} \left[ \Up(p, \rho(\theta))\right] &= \sum_{\Pcal_i\in\Pcal}\mathbb{E}_{\theta\sim\Gamma} \left[ \Up(p, \rho(\theta))\Big\vert\theta\in\Pcal_i\right]\mathbb{P}_{\theta\sim\Gamma}[\theta\in\Pcal_i]\\
		%&=\sum_{\theta_D\in\Theta_\epsilon}\int_{\theta\in[\theta_D-\epsilon/2,\theta_D+\epsilon/2]} \mu_{\theta} \Up(p, b^{\theta_D}(p))d\theta\\
		&=\sum_{\Pcal_i\in\Pcal}\Up(p, \rho(\theta_i))\mathbb{P}_{\theta\sim\Gamma}[\theta\in\Pcal_i]\\
		&=\sum_{\Pcal_i\in\Pcal}\gamma_{\theta_i}\Up(p, \rho(\theta_i)).
	\end{align*}
%
For the second part, take any $i\in[k]$ and $\theta,\tilde \theta\in \Pcal_i$. Then, for all actions $a\neq \rho(\theta)$ it holds:
	\begin{align*}
		\Ua_{\theta}(\rho(\theta),p)&:=\sum_{\omega\in\Omega} F_{\rho(\theta),\omega}p_\omega-{ \theta c_{\rho(\theta)}} \\
		&= \sum_{\omega\in\Omega} F_{\rho(\theta),\omega}p_\omega- \tilde\theta c_{\rho(\theta)}+(\tilde\theta-\theta)c_{\rho(\theta)}\\
		&\ge \sum_{\omega\in\Omega} F_{a,\omega}p_\omega- \tilde\theta c_{a}+(\tilde\theta-\theta)c_{b^{\tilde\theta}(p)}\tag{By definition $\Bcal^{\tilde \theta}(p)$}\\
		&=\sum_{\omega\in\Omega} F_{a,\omega}p_\omega-{ \theta c_{a}}+(\tilde\theta-\theta)(c_{b^{\tilde\theta}(p)}-c_{a})\\
		&\ge \sum_{\omega\in\Omega} F_{a,\omega}p_\omega-{ \theta c_{a}} -\diam(\Pcal_i)\tag{Cauchy-Schwarz}\\
		&=\Ua_{\theta}(p,a)-\diam(\Pcal_i).
	\end{align*}
	This implies $\rho(\theta)\in\Bcal_{\diam(\Pcal_i)}^\theta(p)$, concluding the proof.
\end{proof}



\section{Impossibility of a $\FPTAS$}\label{app:reduction}
In this section, we show the missing details from the reduction of \Cref{sec:hardness}.
\subsection{Missing proofs for \Cref{thm:reduction}: If Analysis}
\begin{lemma}\label{lem:BR_IF}
    Let $\Scal^\star\subset \Scal$ be a cover of $E$, and for each $S\in \Scal^\star$, set $p_{\omega_S}=\frac{1}{n}$.
	while, $p_{\omega^\star}=p_{\bar \omega}=0$ and $p_{\omega_S}=0$ for all $S\notin \Scal^\star$. Then the best response of any type $\theta_i$ for $i\in E$ is in the set $\{a_{i,S}\}_{S\in\Scal^\star}$.
\end{lemma}

\begin{proof}
        Let $\tilde a_i$ be any action in the set $\{a_{i,S}\}_{S\in\Scal^\star}$.
        The utility of playing an action $a_{j, S}$ (for $j\in S$ and $S\in\Scal$) is
	%
	\begin{align*}
		\sum_{\omega \in \Omega} F_{ a_{j, S}, \omega } p_\omega - \theta_i \cdot c_{a_{j, S}} &= \sum_{S'\in\Scal^\star} F_{ a_{j, S}, {\omega_{S'}} } p_{\omega_{S'}} - \frac{i}{n} \cdot c_{a_{j, S}}\\
        & = F_{ a_{j, S}, {\omega_{S}} } p_{\omega_{S}} - \frac{i}{n} \cdot c_{a_{j, S}}  \\
        & =  F_{ a_{j, S}, {\omega_{S}} } p_{\omega_{S}} - \frac{i}{4j^2n}\mu  \\
        &= \mathbb{I}(S \in S^\star) \frac{1}{2in}\mu -  \frac{i}{4j^2n}\mu. 
	\end{align*}
 
    Hence, the agent's utility playing action $\tilde a_{i}$ is 
    \[
    \mathbb{I}(S \in S^\star) \frac{1}{2in}\mu-  \frac{i}{4i^2n}\mu = \frac{1}{2in}\mu-  \frac{1}{4in}\mu =   \frac{1}{4in}\mu.
    \]

    Now, we show that the utility playing any other action is not larger.
    Playing any other action $a_{j,S'}$ the agent's utility is at most
    \[
        \mathbb{I}(S' \in \Scal^\star) \frac{1}{2jn}\mu -  \frac{i}{4j^2n}\mu \le \frac{1}{2jn}\mu -  \frac{i}{4j^2n}\mu=
        \frac{1}{2jn}\mu \left(1-\frac{i}{2j}\right). 
    \]
    Then, observing that the function $j\mapsto \frac{1}{j}(1-\frac{i}{2j})$ is increasing for $j\le i$ and decreasing for $j\ge i$, we conclude that the maximum incurs at $j=i$ with utility $\frac{1}{4in}\mu$.
    Hence, the agent's utility playing $a_{j,S'}$ us at most
    \[ \frac{1}{2jn}\mu \left(1-\frac{i}{2j}\right) \le \frac{1}{4in}\mu. \]
	
   Now, we consider the agent's utility playing an action $\bar a_{j, S}$, $j\in E$ and $S\in\Scal^\star$ (we can consider only $S\in\Scal^\star$, since, if $S\notin\Scal^\star$, then the agent utility is strictly negative):
	\begin{align*}
		\sum_{\omega \in \Omega} F_{ \bar a_{j, S}, \omega } p_\omega - \frac{i}{n} \cdot c_{\bar a_{j, S}} &=  F_{ \bar a_{j, S}, {\omega_{S}}} p_{\omega_S}  - \frac{i}{n} \cdot c_{\bar a_{j, S}}\\
		&\le \frac{1}{n} F_{ \bar a_{j, S}, {\omega_{S}}} -\frac{i}{4j^2n}\mu(1-\eta) \\
		&=\frac{1}{2jn}\left(1-\frac\eta2\right)\mu-\frac{i}{4j^2n}\mu(1-\eta) \\
	&=\frac{1}{2jn}\mu\left(1-\frac{\eta}{2}-\frac{i}{2j}(1-\eta)\right).
	\end{align*}
	
	Then, we can observe that $j\mapsto \frac{1}{j}\left(1-\frac{\eta}{2}-\frac{i}{2j}(1-\eta)\right)$ is decreasing for $j\le 2i\frac{1-\eta}{2-\eta}$ and increasing otherwise. Since $2i\frac{1-\eta}{2-\eta}\in[i-1, i]$ for $\eta$ small enough, we only need to show that the agent's utility for actions $\bar a_{i, S}$ and $\bar a_{i-1, S}$ are suboptimal.
	%
	Straightforward calculations shows that the value of action $\bar a_{i-1, S}$ is at most $\frac{1}{2(i-1)^2n}\mu\left[\frac{i}{2}+\frac{\eta}{2}-1
    \right]$, while the value of action $\bar a_{i, S}$ is $\frac{1}{4in}\mu$. Then, if $\eta$ is small enough it holds $\frac{1}{2(i-1)^2n}\mu\left[\frac{i}{2}+\frac{\eta}{2}-1
    \right] <\frac{1}{4in}\mu$. Thus $\bar a_{i, S}$ with $S\in\Scal^\star$ is optimal among all actions $\{\bar a_{j, S}\}_{j, S}$ and provides agent's utility $\frac{1}{4in}\mu$.
	
    Finally, consider action $a^\star$. The agent's utility playing this action would be
	\begin{align*}
		\sum_{\omega\in \Omega} F_{a^\star,\omega}p_\omega - \frac{i}{n}\cdot c_{a^\star}  = \frac{1}{n}\sum_{S\in \Scal^\star} F_{a^\star,\omega_S} - \frac{i}{n}
		\le \frac{\epsilon m}{n} - \frac{i}{n}
		\le 0.
	\end{align*}
    Hence, we can finally conclude that $\tilde a_i$ is a best response for type $\theta_i$.
    % We conclude the analysis for type $\theta_i$ noticing that the principal's utility when the type is $\theta_i:=a_{i,S}$ is
    % \begin{align*}
    %     \sum_{\omega\in \Omega} F_{\tilde a_i,\omega}(r_\omega-p_\omega)=F_{a_{i,S},\omega^\star}r_{\omega^\star}-F_{a_{i,S},\omega_S}p_{\omega_S}
    %     =\frac{1}{in}\mu-\frac{1}{2in}\mu
    %     =\frac{1}{2in}\mu.
    % \end{align*}
\end{proof}


\begin{lemma}\label{lem:BR_IF2}
    Under the same conditions of \Cref{lem:BR_IF}, the best response of type $\theta_0$ is $a^\star$.
\end{lemma}
\begin{proof}
    %Now, we focus on the best response of type $\theta_0$.
    An agent of type $\theta_0$ doesn't pay any cost for playing actions, and hence their best response simply maximizes the expected payment. 
    The expected payment playing $a^\star$ would be at least $\frac{k\epsilon}{n}$ which is greater than $\mu/n$. On the other hand, the value of playing action $a_{i, S}$ or $\bar a_{i,S}$ (with $i\in S$) would be at most $\frac{1}{2in}\mu$ and $\frac{1}{2in}\mu(1-\eta/2)$ which are both strictly less then $\mu/n$. Thus, we conclude that an agent of type $\theta_0$ will play action $a^\star$.
\end{proof}


\subsection{Missing Proofs for \Cref{thm:reduction}: Only if Analysis}


\begin{lemma}\label{lem:onlyif1}
   The principal utility for types $i\in E_1$ is at most $\frac{1}{2in}\mu+\frac{1}{i}p_{\omega^\star}\mu\left(\frac{2}{\eta}-1\right)$.
\end{lemma}
\begin{proof}
    	We start analyzing the constraints that the best responses of types $\theta_i$, $i \in E_1$, impose on the contract.
	Consider a type $\theta_i$, $i \in E_1$, and let $a_{i,S}$ be their best response. Notice that $i \in S$.
	%
	Then, from the IC constraint with respect to action $\bar a_{i,S}$ it holds
	%
	\[
	p_{\omega_{S}}\frac{1}{2i}\mu + p_{\omega^\star}  \frac{1}{i}\mu+\left(1-\frac{3}{2i}\mu\right)p_{\bar\omega}- \frac{1}{4in}\mu \ge  p_{\omega_{S}}\frac{1-\frac{\eta}{2}}{2i} \mu +\left(1-\frac{1-\frac{\eta}{2}}{2i}\mu\right)p_{\bar\omega}- \frac{1}{4in} (1-\eta) \mu,
	\]
	implying
	\begin{align}\label{eq:largeE1}
	    p_{\omega_{S}}\ge \frac{1}{n}-\frac{4}{\eta}p_{\omega^\star}.
	\end{align}
%	
	From this inequality it follows that the principal utility when the agent's type is $\theta_i$, $i\in E_1$ is at most
	% \ma{giusto buttare via $\bar \omega$? farà schifo come outcome eh, però magari bisogna dismostrare che $p_{\bar\omega}=0$}
	\begin{align*}
	&F_{a_{i,S},\omega_S}(r_{\omega_S}-p_{\omega_S})+F_{a_{i,S}}(r_{\omega^\star}-p_{\omega^\star})+F_{a_{i,S},\bar \omega}(r_{\bar \omega}-p_{\bar\omega})\nonumber\\
        &\hspace{6cm}\le-\frac{1}{2i}\mu(\frac{1}{n}-4p_{\omega^\star}/\eta)+\frac{1}{i}\mu(\frac{1}{n}-p_{\omega^\star})\nonumber\\
	&\hspace{6cm}=\frac{1}{2in}\mu+\frac{1}{i}p_{\omega^\star}\mu\left(\frac{2}{\eta}-1\right),\label{eq:boundE1}
	\end{align*}
    as desired.
\end{proof}

\begin{lemma}\label{lem:onlyif2}
   The principal utility for types $i\in E_2$ is at most $\mu\left(\frac{1}{2in}-\frac{1}{8n^4}+\frac{2}{\eta}p_{\omega^\star}\right)$.
\end{lemma}
\begin{proof}
    	Now, consider an agent's type $\theta_i$, $i\in E_2$, and let $a_{j,S}$, $j \neq i$ be the action played by the agent. 
	Notice that $j \in S$. Then, IC constraint with respect to action $\bar a_{j,S}$ reads
	\[
	p_{\omega_{S}} \frac{1}{2j} \mu + \frac{1}{j}p_{\omega^\star}\mu+(1-\frac{3}{2j}\mu)p_{\bar\omega}- \frac{i}{4j^2n}\mu \ge p_{\omega_{S}} \frac{1}{2j} (1-\eta/2) \mu+\left(1-\frac{1-\frac{\eta}{2}}{10j}\mu\right)p_{\bar\omega}- \frac{i}{4j^2n} (1-\eta) \mu,
	\]
	which implies that
	%\[ p_{\omega_{S}} \frac{\eta}{20j} \ge  \frac{i\eta}{20j^2} - \frac{1}{5j} p_{\omega^\star} \]
	%and
	\[
	p_{\omega_{S}} \ge \frac{i}{jn}-  \frac{4}{\eta} p_{\omega^\star}. 
	\]
	
	Hence, the principal's expected utility when the realized type is a $\theta_i$, $i \in E_2$, is at most:
	\begin{align*}
		F_{a_{j,S},\omega_S}(r_{\omega_S}-p_{\omega_S})+F_{a_{j,S},\omega^\star}(r_{\omega^\star}-p_{\omega^\star})+F_{a_{j,S},\bar \omega}(r_{\bar \omega}-p_{\bar\omega})&\le-\frac{1}{2j}\mu\left( \frac{i}{jn}- p_{\omega^\star} \frac{4}{\eta}\right)+\frac{1}{jn}\mu\\
		&=\mu\left(\frac{1}{jn}-\frac{i}{2j^2n}+\frac{2}{j\eta}p_{\omega^\star}\right)\\
		&\le\mu\left(\frac{1}{jn}-\frac{i}{2j^2n}+\frac{2}{\eta}p_{\omega^\star}\right)
	\end{align*}

	Now consider the term $\frac{1}{j}-\frac{i}{2j^2}$.
    This term over $E$ is maximized by $j=i-1$ or $j=i+1$ (recall that we cannot have that $j=i$ as $i\in E_2$). 
    For $j=i-1$, we can obtain an upper bound on this quantity observing that
	\begin{align*}
		\frac{1}{(i-1)}-\frac{i}{2(i-1)^2}&=\frac{1}{2i}-\frac{1}{2(i-1)^2i}\\
		&\le \frac{1}{2i} -\frac{1}{2(n-1)^2n}\tag{$x\mapsto -\frac{1}{(x-1)^2x}$ is increasing for $x\ge 1$}\\
		&\le \frac{1}{2i}-\frac{1}{2n^3}
	\end{align*}
    For $j=i+1$, we can obtain an upper bound on this quantity observing that
	\begin{align*}
		\frac{1}{(i+1)}-\frac{i}{2(i+1)^2}&=\frac{1}{2i}-\frac{1}{2(i+1)^2i}\\
		&\le \frac{1}{2i} -\frac{1}{2(n+1)^2n}\tag{$x\mapsto -\frac{1}{(x+1)^2x}$ is increasing for $x\ge 1$}\\
		&\le\frac{1}{2i} -\frac{1}{8n^3}
	\end{align*}
	
	
	
	Hence, the principal's expected utility when the realized type is a $\theta=i \in E_2$ is at most:
	\begin{align*}\label{eq:boundE2}
	\mu\left(\frac{1}{2in}-\frac{1}{8n^4}+\frac{2}{\eta}p_{\omega^\star}\right),
	\end{align*}
    as stated.    
\end{proof}

\begin{lemma}\label{lem:onlyif3}
   The principal utility for types $i\in E_3$ is at most $0$.
\end{lemma}

\begin{proof}
    	Now consider any type $i\in E_3$. This agent plays an action $\bar a_{j, S}$ for some $S$ and $j\in S$, actions $a^\star$ or action $a_0$.
	Clearly when actions $\bar a_{j, S}$ and $a_0$ are played the principal's utility is at most zero, as the only outcomes reachable from these actions are $\omega_S$ and $\bar\omega$ which have reward $0$. Moreover, the principal utility is at most zero also for action $a^\star$. Indeed, consider the IC constraints of action $a^\star$ with respect to $a_0$:
	\[
	\sum_{S\in\Scal}F_{a^\star,\omega_S}p_{\omega_S}-\frac{i}{n} \cdot c_{a^\star} + F_{a^\star,\omega^\star} p_{\omega^\star}\ge 0,
	\]
	which implies that $\epsilon\sum_{S\in\Scal}p_{\omega_S}+(1-m\epsilon)p_{\omega^\star}\ge \frac{i}{n}$. Then, the principal utility, when action $a^\star$ is played is 
	\begin{align*}
		\sum_{S\in\Scal} F_{a^\star,\omega_S}(r_{\omega_S}-p_{\omega_S})+F_{a^\star,\omega^\star}(r_{\omega^\star}-p_{\omega^\star})=&-\epsilon\sum_{S\in\Scal}p_{\omega_S}+(1-m\epsilon)\left(\frac{1}{n}-p_{\omega^\star}\right)\\
		&\le \frac{1-m\epsilon}{n}- \frac{i}{n}.
	\end{align*}
	and thus, the principal utility is negative also for action $a^\star$.
    Hence, we can conclude that when the realized type is $\theta_i$, $i\in E_3$, the principal's utility is at most zero.
 %   \begin{align}\label{eq:boundE3}
 %   \Up(p,\theta_i)\le 0
 %   \end{align}
	%
	%\ma{aggiongiere $a_0$ con costo zero che manda determ in $\bar \omega$}. \ma{dici che tutte queste azoni danno utilità negativa ma $a^\star$ mi fa fare un sacco di soldi invece... Bisogna far vedere che non la giocano. O meglio che se la giocassero io sto spendendo troppo per fargliela giocare visto che pagano 1 per giocarla. O magari mi sono rincoglionito...}
\end{proof}


\begin{lemma}\label{lem:onlyif4}
    When the realized type is $\theta_0$ the principal utility is $-\epsilon\sum_{S\in\Scal}p_{\omega_S}+(1-m\epsilon)(\frac{1}{n}-p_{\omega^\star})$.
\end{lemma}
\begin{proof}
    It is easy to see that type $\theta_0=0$ best responds with action $a^\star$  to any contract $p$ with expected principal's utility at least $\ell-n^{15}m^{-1}$. 
    Indeed, the maximum expected reward obtaining without playing $a^\star$ is obtained playing an action $a_{1,S}$ with expected reward
    \[ 
    \frac{1}{n} \mu \le \ell-n^{15}m^{-1}< \ell-n^{15}m^{-1}.  
    \]
%
    Hence, we conclude computing the principal's expected utility when the agent's type is $\theta_0$ playing action $a^\star$:
    
	\begin{align*}\label{eq:bound0}
	-\epsilon\sum_{S\in\Scal}p_{\omega_S}+(1-m\epsilon)\left(\frac{1}{n}-p_{\omega^\star}\right).
	\end{align*}
    as desired.
\end{proof}


\begin{lemma}\label{lem:onlyif5}
Define 	$\bar \Scal:=\{S\in\Scal: p_{\omega_S}\ge\frac{1}{n}-\frac{4}{\eta}p_{\omega^\star}\}$, then the principal utility is at most
    \[
        \frac{1-\rho}{n}\mu\left[\sum_{i\in E_1}\frac{1}{2in}+\sum_{i\in E_2}\left(\frac{1}{2in}-\frac{1}{8n^4}\right)\right]+\frac{1}{n}\rho(1-m\epsilon)-\frac{1}{n}\epsilon\rho|\bar\Scal|.
    \]
\end{lemma}

\begin{proof}
        Now, we combine the bound on the principal's utility for the different types.
    Let define the set $\bar \Scal:=\{S\in\Scal: p_{\omega_S}\ge\frac{1}{n}-\frac{4}{\eta}p_{\omega^\star}\}$. Notice that by \Cref{eq:largeE1}, for each element $i \in E_1$ it hold $i \in \bigcup_{S\in \bar \Scal} S$. Combining \Cref{lem:onlyif1}, \Cref{lem:onlyif2} and \Cref{lem:onlyif3}, we can upperbound the  principal's expected  utility with
    \begin{align}
    &\frac{1-\rho}{n}\mu\left[\sum_{i\in E_1}\right(\frac{1}{2in}+\frac{1}{i}p_{\omega^\star}(\frac{2}{\eta}-1)\left)+\sum_{i\in E_2}\left(\frac{1}{2in}-\frac{1}{8n^4}+\frac{2}{\eta}p_{\omega^\star}\right)\right]\nonumber\\
    &\hspace{6cm}+\rho(1-m\epsilon)\left(\frac{1}{n}-p_{\omega^\star}\right)-\epsilon\rho\sum_{S\in\Scal}p_{\omega_S}\nonumber\\
    & \le \frac{1-\rho}{n}\mu\left[\sum_{i\in E_1}\right(\frac{1}{2in}+\frac{1}{i}p_{\omega^\star}(\frac{2}{\eta}-1)\left)+\sum_{i\in E_2}\left(\frac{1}{2in}-\frac{1}{8n^4}+\frac{2}{\eta}p_{\omega^\star}\right)\right]\nonumber\\
    &\hspace{6cm}+\rho(1-m\epsilon)\left(\frac{1}{n}-p_{\omega^\star}\right)-\epsilon\rho |\bar\Scal|\left(\frac{1}{n}-\frac{4}{\eta}p_{\omega^\star}\right) \label{eq:upperbound}
    \end{align}
    where in the inequality we used that $\sum_{S\in\Scal} p_{\omega_S}\ge |\bar\Scal|(\frac{1}{n}-\frac{4}{\eta}p_{\omega^\star})$ by the definition of $\bar \Scal$.
    %
    Then, we show that this upperbound on the principal's utility is maximized for $p_{\omega^\star}=0$. Indeed, in the expression the coefficient of $p_{\omega^\star}$ is 
	\[
	\frac{1-\rho}{n}\mu\left[\sum_{i\in E_1}\frac{1}{i}\left(\frac{2}{\eta}-1\right)+\sum_{i\in E_2}\frac{2}{\eta}\right]-\rho(1-m\epsilon)+\frac{4}{\eta}\epsilon\rho|\bar\Scal|.
	\]
	The following calculations shows that this is negative, implying that the expression is maximized for $p_{\omega^\star}=0$:
	\begin{align*}
		\frac{1-\rho}{n}\mu&\left[\sum_{i\in E_1}\frac{1}{i}\left(\frac{2}{\eta}-1\right)+\sum_{i\in E_2}\frac{2}{\eta}\right]-\rho(1-m\epsilon)+\frac{4}{\eta}\epsilon\rho|\bar\Scal|\\
		&\le \frac{1-\rho}{n}\mu\sum_{i\in E_1\cup E_2}\frac{2}{\eta} - \rho(1-m\epsilon)+4\frac{\rho \epsilon m}{\eta}\\
		&\le \frac{1-\rho}{n}\mu\frac{2}{\eta}n-\rho(1-m\epsilon)+4\frac{\rho \epsilon m}{\eta}\\
		&\le \frac{2}{\eta}\mu-\rho(1-m\epsilon)+4\frac{\rho \epsilon m}{\eta}\\
		&\le 2\frac{\mu}{\eta}-\frac{\rho}{2}+4\frac{\rho \epsilon m}{\eta}\\
		&=2\frac{n^2}{n^9m}-\frac{1}{2n^6}+4\frac{mn^2}{n^6n^8m}\\
		&=2\frac{1}{n^7}-\frac{1}{2n^6}+4\frac{1}{n^{12}},
	\end{align*}
	
	which is negative for $n\ge 2$.
	
	Thus, setting $p_{\omega^\star}=0$ in \Cref{eq:upperbound}, the principal utility is upper bounded by
	\[
	\frac{1-\rho}{n}\mu\left[\sum_{i\in E_1}\frac{1}{2in}+\sum_{i\in E_2}\left(\frac{1}{2in}-\frac{1}{8n^4}\right)\right]+\frac{1}{n}\rho(1-m\epsilon)-\frac{1}{n}\epsilon\rho|\bar\Scal|,
	\]
    concluding the proof.
\end{proof}


\section{Omitted Proofs from \Cref{sec:learning}: Learning Optimal Contracts}

\lemmaClose*

\begin{proof}
	Consider any contract $p\in[0,1]^m$. Then, for each action $a\in A$, let $\Theta(p,a)= \{\theta \in \Theta: b^{\theta}(p)=a\}$ which is the set of types that play $a$ as a best-response to $p$. 
    Let $\widehat \Theta(p)\subseteq \Theta_\epsilon$ be the set of $\theta \in \Theta_\epsilon$ such that in the interval $(\theta-\epsilon/2, \theta+\epsilon/2)$ there are multiple best-responses, \ie there are two actions $a\neq a'$ such that $\Theta(p,a)\cap (\theta-\epsilon/2,\theta+\epsilon/2]\neq\emptyset$ and $\Theta(p,a') \cap (\theta-\epsilon/2,\theta+\epsilon/2]\neq \emptyset$.
	%
	Then, consider any $\theta\in\widehat\Theta(p)$ and observe that by \Cref{ass:boundedDensity} it holds
    \begin{align*}
        \left| \gamma_\theta \Up(p,\theta)-\int_{\theta-\epsilon/2}^{\theta+\epsilon/2} f_\Gamma(\tilde \theta) \Up(p,\tilde \theta)d\tilde\theta \right|&= \left| \int_{\theta-\epsilon/2}^{\theta+\epsilon/2} f_\Gamma(\tilde \theta) \left(\Up(p, \theta)-\Up(p,\tilde \theta)\right)d\tilde\theta \right|\\
        &\le \int_{\theta-\epsilon/2}^{\theta+\epsilon/2}f_\Gamma(\tilde \theta) \left|\Up(p, \theta)-\Up(p,\tilde \theta)\right|d\tilde\theta\\
        &\le 2\beta\epsilon.
    \end{align*}
%
    On the other hand, for any type $\theta\in\Theta_\epsilon\setminus\widehat\Theta(p)$ we have by definition that all the types in the interval $(\theta-\epsilon/2,\theta+\epsilon/2]$ have the same best response and thus
    \[
    \gamma_\theta\Up(p,\theta)=\int_{\theta-\epsilon/2}^{\theta+\epsilon/2} f_\Gamma(\tilde\theta) \Up(p,\tilde\theta)d\tilde\theta.
    \]
    Combining the two, we get that
    \begin{align*}
        \left|\mathbb{E}_{\theta\sim\Gamma}[\Up(p,\theta)]-\sum_{\theta\in\Theta_\epsilon}\gamma_\theta \Up(p,\theta)\right|&\le \sum_{\theta\in\Theta_\epsilon}\left|\gamma_\theta\Up(p,\theta)-\int\limits_{\theta-\epsilon/2}^{\theta+\epsilon/2}f_\Gamma(\tilde\theta)\Up(p,\tilde\theta)d\tilde\theta\right|\\
        &\le \sum_{\theta\in\widehat\Theta(p)}\left|\gamma_\theta\Up(p,\theta)-\int\limits_{\theta-\epsilon/2}^{\theta+\epsilon/2}f_\Gamma(\tilde\theta)\Up(p,\tilde\theta)d\tilde\theta\right|\\
        &\le 2|\widehat\Theta(p)|\beta\epsilon.
    \end{align*}

    Now we argue that $|\widehat\Theta(p)|$ is at most $n$. It is well known~\citep{alon2021contracts} that $\Theta(p,a)$ is an interval for each $a \in A$, and thus the number of intersections between such intervals is at most $n-1$, which is on to itself an upper-bound on $|\widehat\Theta(p)|$.
\end{proof}

\subsection{Proof of \Cref{thm:sample}}

\samplecomplexity*

\begin{proof}
    The proof essentially delves into the analysis of regret guarantees of $\PE$ presented in \citet{lattimore2020learning}. The algorithm $\PE$ works by partitioning the time horizon into geometrically increasing blocks each of length $T_\ell=2^{\ell-1}\lceil 4d\log\log d+16\rceil$. In each window each arm $x\in X$ is pulled $T_\ell\cdot\rho(x)$ times where $\rho\in\Delta_X$ is the optimal design, which has support at most $4d \log \log d + 16$ \citep{todd2016minimum}. The rewards observed are used to compute an estimate $\widehat \phi_\ell$ of $\phi$ as
    \[
    \widehat \phi_\ell=G_\ell^{-1}\sum_{k=T_{\ell}+1}^{T_{\ell+1}} \mu_k e_{x_t}\quad\text{where}\quad G_\ell=\sum_{x\in X} \lceil m_\ell\rho(x)\rceil x_tx_t^\top,
    \]
    and where $e_x\in\Reals^{k}$ is the vector with all zeros except the component associated with the arm $x$ (according to whatever fixed ordering).
    Then $\widehat \phi_\ell$ is used to discard provably suboptimal arms and maintain a set of optimal arms $X_t$ is updated at $T_l$
    \[
    X_{\ell}=\left\{x\in X_{{\ell-1}}:\max_{y\in X_{{\ell-1}}} (y-x)^\top \widehat\phi_\ell\le2\sqrt{\frac{4d}{T_\ell}\log\left(\frac{k}{\delta_\ell}\right)}\right\},
    \]
    where $\delta_\ell$ will be set in the following.
    Now we can use that key claim that with high probability $\widehat \phi_\ell$ is a good estimate of $\phi$.
    \begin{claim}[{\cite[Appendix~D]{lattimore2020learning}}]\label{claim:estimation}
    For any $\delta_\ell\in(0,1)$, the event 
    \[
    \Ecal_\ell=\left\{|x^\top(\widehat \phi_\ell-\phi)|\le 2\alpha\sqrt{d}+\sqrt{\frac{4d}{T_\ell}\log\left(\frac{k}{\delta_\ell}\right)}\quad\forall x\in X\right\}
    \]
    holds with probability at least $1-\delta_\ell$.
    \end{claim}
    Now define $x^\star=\max_{x\in X}\mu(x)$ and $\widehat x_\ell=\max_{x\in X_\ell}x^\top\widehat \phi_\ell$ and consider now any arm $x\in X_\ell$ the following inequalities
    \begin{align}
        2\sqrt{\frac{4d}{T_\ell}\log\left(\frac{k}{\delta_\ell}\right)}&\ge (\widehat x_\ell-x)^\top \widehat \phi_\ell\tag{Since $x\in X_\ell$}\\
        &\ge (x^\star-x)^\top \widehat \phi_\ell\tag{Definition of $\widehat x_\ell$}\\
        &\ge (x^\star-x)^\top \phi - 4\alpha\sqrt{d}-2\sqrt{\frac{4d}{T_\ell}\log\left(\frac{k}{\delta_\ell}\right)}\tag{Under event $\Ecal_\ell$}
    \end{align}

    Now, we can use the equation above to obtain a bound on the sub-optimality of any arm $x\in X_\ell$:
    \begin{align*}
        \mu(x)&\ge x^\top \phi-\alpha\\
        &\ge x^{\star,\top} \phi-\alpha- 4\alpha\sqrt{d}-4\sqrt{\frac{4d}{T_\ell}\log\left(\frac{k}{\delta_\ell}\right)}\\
        &\ge \mu(x^\star)-2\alpha- 4\alpha\sqrt{d}-4\sqrt{\frac{4d}{T_\ell}\log\left(\frac{k}{\delta_\ell}\right)}
    \end{align*}

    The event $\Ecal=\cup_{\ell=1}^{+\infty}\Ecal_{\ell}$ holds with probability at least $1-\sum_{\ell=1}^{+\infty}\delta_\ell$. Thus, by fixing $\delta_\ell = \frac{6\delta}{\pi^2\ell^2}$, we have that $\Ecal$ holds with probability at least $1-\delta$.

    Now we need to find the best lowest $\ell$ such that $\mu(x)\ge \mu(x^\star)-\eta$, or equivalently such that
    \begin{equation}\label{eq:tmp7}    
    2\alpha+ 4\alpha\sqrt{d}+4\sqrt{\frac{4d}{T_\ell}\log\left(\frac{k\pi^2\ell^2}{6\delta}\right)}\le \eta
    \end{equation}
    which is implied by the more stringent inequality
    \(
    6\alpha\sqrt{d}+4\sqrt{\frac{1}{2^\ell}\log\left(\frac{k\pi^2\ell^2}{6\delta}\right)}\le \eta.
    \)
    Now, we can use a standard technical claim (\emph{e.g.}, similar to \citet[Lemma~12]{jonsson2020planning}) that we can use to solve the equation above.
    \begin{claim}
        Define $D(c,\ell)=\sqrt{\frac{\log(c\cdot\ell^2)}{2^\ell}}$ for some $c>0$. Given any $z>0$, if $\ell\ge\log\left(\frac{2}{z^2}\log\left(\frac{c}{z^2}\right)\right)$ then $D(c,\ell)\le z$.
    \end{claim}
 
    We can use the claim above with $z=\frac{\eta-6\alpha\sqrt{d}}{4}$ and $c=\frac{k\pi^2}{6\delta}$ and thus we known that if 
    \[
    \ell\ge \ell^\star\coloneqq\log\left(\frac{32}{(\eta-6\alpha\sqrt{d})^2}\log\left(\frac{8k\pi^2}{3\delta (\eta-6\alpha\sqrt{d})^2}\right)\right)
    \]
    then \Cref{eq:tmp7} is satisfied. If we are going to use $\ell^\star$ blocks then we are actually using
    $\sum_{\ell=1}^{\ell^\star} T_\ell$ samples.
    \begin{align*}
        \sum_{\ell=1}^{\ell^\star} T_\ell &=\lceil4d\log\log d+16\rceil\sum_{\ell=1}^{\ell^\star}2^{\ell-1}\\
        &\le\lceil4d\log\log d+16\rceil 2^{\ell^\star}\\
        &\le\lceil4d\log\log d+16\rceil\frac{32}{(\eta-6\alpha\sqrt{d})^2}\log\left(\frac{8k\pi^2}{3\delta (\eta-6\alpha\sqrt{d})^2}\right).
        %&\le C\frac{d\log\log d}{(\eta-6\alpha\sqrt{d})^2}\log\left(\frac{k}{\delta}\right)
    \end{align*}
    %for some absolute constant $C=O(1)$.
    This proves that the sample complexity of misspecified linear bandits is $\tilde O\left(\frac{d}{(\eta-6\alpha\sqrt{d})^2}\log\left(\frac{k}{\delta}\right)\right)$.

    %This is an upper bound on the sample complexity of best-arm identification in misspecified linear bandits.
    %
    When we reduce from our problem of contract design we can thus choose appropriately the discretization $\epsilon$. As per \Cref{lem:valClose}, for a discretization $\epsilon$ we have a misspecification of $\alpha = 2\beta n \epsilon$, a number of actions which is $k=|\Pcal|=\poly((n,m,1/\epsilon)^m)$ and a dimension of $d=|\Theta_\epsilon|=1/\epsilon$. By setting $\epsilon$ such that $6\cdot 2\beta n\epsilon\sqrt{d}=6\cdot 2\beta n\sqrt{\epsilon}=\frac\eta2$ we get $\epsilon = (\frac{\eta}{24\beta n})^2$, and we obtain a sample complexity of 
    \[
        C\frac{(\beta n)^2m\log\log(\beta n/\eta)}{\eta^4}\log\left(\frac{\poly(m,n)}{\delta}\right),
    \]
    which is $\tilde O\left(\frac{\poly(\beta,n,m)}{\eta^4}\log\left(\frac1\delta\right)\right)$ as stated.
\end{proof}
