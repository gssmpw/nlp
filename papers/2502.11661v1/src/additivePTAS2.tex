\section{An Additive $\PTAS$ for Single-Dimensional Costs and Continuous-Types}\label{sec:PTAS}

In this section, we show that the problem admits an additive $\PTAS$. 
This is in contrast to the multi-dimensional problem, that does not admits an additive $\PTAS$, unless $\PolyClass=\NP$ (see \cite{guruganesh2021contracts} and \cite[Theorem~7 for $\rho=1$,]{castiglioni2022bayesian}). 
%Notice that such result is a Corollary of \citet[Theorem~7]{castiglioni2022bayesian} setting $\rho=1$.

While in instances with multi-dimensional costs, all types might have very different incentives, our result relies on the simple observation that in single-dimensional instances, similar types share similar incentives and behavior.
Our algorithm exploits this feature and groups agents with similar types into $k$ buckets.
Then, replacing each bucket with a representative type, we can employ algorithms for discrete types such as the one proposed by ~\citet{guruganesh2021contracts} and \citet{castiglioni2022bayesian} to solve the discretized problem. Such algorithms run in time $\poly(n^k, m)$ and therefore in polynomial time when the number of buckets $k$ is constant.
%
The core of the proof is to manage the error with respect to $\OPT$ in terms of the number of buckets $k$. In particular, we will show that when we group two agents of type $\theta$ and $\theta'$ together, we make an error of order $\poly(|\theta-\theta'|)$. Thus, to guarantee an $\epsilon=O(1)$ additive error, we can set $k=\poly(1/\epsilon)$, resulting in a $\PTAS$.

More specifically, our algorithm works as follows. 
Given an instance of single-dimensional Bayesian contract design (possibly with continuous types), we build an instance of contract design with discretized types as follows.
Given a parameter $\delta$ to be set in the following, let $\theta_i=(i-1/2) \delta$ for each $i \in \lceil 1/\delta\rceil$.
Then, we consider the set of types $\Theta_{\delta}:=\{\theta_i\}_{i\in\lceil 1/\delta\rceil}$ each realizing with probability 
\begin{align*}
    \gamma_{\theta_i}\coloneqq \begin{cases}
    \mathbb{P}_{\theta\sim\Gamma}[\theta\in{(\theta_i-\delta/2,\theta_i+\delta/2]}]&\text{if $i>1$}\\
    \mathbb{P}_{\theta\sim\Gamma}[\theta\in{[0,\delta/2]}]&\text{if $i=1$}
    \end{cases}.
\end{align*}
%$\gamma_{\theta_i}:=\mathbb{P}_{\theta\sim\Gamma}[\theta\in{(\theta_i-\delta/2,\theta_i+\delta/2]}]$ if $i>1$ and $\gamma_{\theta_1}:=\mathbb{P}_{\theta\sim\Gamma}[\theta\in{[0,\delta]}]$.
Here, we do not require any particular assumption on how the distribution is represented. Computationally, we only make the mild assumption to have access to an oracle, returning the probability that $\theta\sim \Gamma$ belongs to a given interval.\footnote{It will be clear from our proof that our algorithm is robust to approximations in the probabilities returned by the oracle.}
As we already observed, this discretized problem with types supported on $\Theta_\delta$ can be solved in polynomial time for any constant $\delta>0$~\citep{guruganesh2021contracts, castiglioni2022bayesian}.
Let $\tilde p$ be the contract computed on the discretized problem.
Then, our algorithm returns as a solution the contract $p= \tilde p +\alpha (r-\tilde p)$, where $\alpha$ will be set in the following. Intuitively, this step makes the contract robust to the use of approximate types. 

Formally, we obtain the following guarantees. 
\begin{theorem}\label{thm:PTAS}
    There is an algorithm that for any $\epsilon>0$ runs in time $\poly(n^{{1}/{\epsilon}},m)$ and finds a contract $p\in\Reals^m_{\ge 0}$ such that $\Up(p)\ge \OPT-O(\epsilon)$.
\end{theorem}


Technically, our result is based on two main observations:
\begin{itemize}
\item If the payment vector $p= \tilde p +\alpha(r-\tilde p)$ is applied to the original instance of the problem, it yields a principal's utility close to the utility of contract $\tilde p$ applied to the discretized instance;
\item We relate the utility achieved by an optimal contract in the original and the discretized instance, showing that the latter is only slightly smaller than the former.
\end{itemize}
The theorem follows from the combination of these two results.
%Then we consider a cost vector $\tilde c_{\theta, a}=\theta\cdot c_a$ for each $\theta\in\Theta_\delta$ and $a\in A$.
%\mat{Perchè vuoi vedere l'istanza come multidimensionale? check se usiamo $c_{\theta, a}$  o $c_{a,\theta}$, ora non è consistente con i preliminari. }



We start the formal analysis providing some thecnical lemmas. First, we notice that an optimal contract for the discretized instance that incentivizes action $a_i$ for a type $\theta_i\in \Theta_\delta$, approximately incentivizes the same action for types close to $\theta_i$ in the original instance.
As a tool to exploit this $\delta$-best response, we will use the well-known result of \citet{dutting2021complexity} that relates IC and approximate-IC contracts. We recall the statement of this result in the following lemma.
\begin{lemma}[{\citep[Proposition~2.4]{dutting2021complexity}}]\label{lem:linearization}
Let $\rho:\Theta\to A$ be a generic mapping from types to actions, specifying the action $\rho(\theta)$ taken by each type $\theta\in\Theta$, and let $p\in\Reals^m_{\ge 0}$ satisfy $\rho(\theta)\in \Bcal^{\theta}_{\epsilon}(p)$ for all $\theta\in\Theta$. Then, $\tilde p:=p+\alpha(r-p)$ guarantees $\Up(\tilde p,\theta)\ge \Up(p,\rho(\theta))-\left(\frac{\epsilon}{\alpha}+\alpha\right)$.
\end{lemma}

Next, we must establish a connection between the discretized and continuous instances of the problem. 
%Our second technical lemma helps us relate the discretized and continuous instance and has two main implications. 
We do that through the following lemma, which has two main implications. 
First, it implies that if for each type $\theta_i\in \Theta_\delta$ the  action played (which may not be IC) does not change across the interval $(\theta_i-\delta/2, \theta_i+\delta/2]$, then the principal's utility in the original and discretized instances coincide.
Second, it shows that the best response of a type $\theta_i \in \Theta_\delta$ is still an approximate best response for each type in $(\theta_i-\delta/2, \theta_i+\delta/2]$.
In the following lemma, we prove a slightly more general result for arbitrary partition of $\Theta=[0,1]$.

\begin{restatable}{lemma}{lemmadiscretization}\label{lem:discretization}
    Let $\Pcal=\{\Pcal_1,\ldots,\Pcal_k\}$ be any finite partition of $[0,1]$. Moreover let $\hat\Theta:=\{\theta_1,\ldots,\theta_k\}$ be such that $\theta_i\in\Pcal_i$ for all $i\in[k]$, and let $\rho:\Theta\to A$ be such that $\rho(\theta)=\rho(\theta_i)$ for each $\Pcal_i$ and $\theta\in\Pcal_i$. Then, for any contract $p\in\Reals^m_{\ge 0}$ it holds
    \[
        \mathbb{E}_{\theta\sim\Gamma} \left[\Up(p, \rho(\theta))\right] = \sum_{\theta_i\in\hat\Theta} \gamma_{\theta_i} \Up(p, \rho(\theta_i)),
    \]
    where $\gamma_{\theta_i}\coloneqq\mathbb{P}_{\theta\sim\Gamma}[\theta\in \Pcal_i]$.
    %
    Moreover, if for any $\theta\in\Pcal_i$, $i \in [k]$, we have $\rho(\theta)\in \Bcal^{\tilde\theta}(p)$ for some $\tilde \theta\in \Pcal_{i}$, then $\rho(\theta)\in\Bcal^{\theta}_{\diam(\Pcal_i)}(p)$ for all $\theta\in\Pcal_i$, where $\diam(\Pcal_i)\coloneqq\sup_{\theta,\theta'\in\Pcal_i}|\theta-\theta'|$.
    %\ac{ok?} \mat{potrebbe ma i quantificatori sono troppo incasinati. Ci ho messo una vita a parsarla.}\ma{la second apart non la riesco a capire io}\ac{boh a me sembra più chiaro così di come era prima. comunque è commentato, potete fare revert}
    %Moreover, if, for all $i \in [k]$ and $\theta, \tilde \theta\in \Pcal_{i}$,  $\rho(\theta_i)\in \Bcal^{\tilde\theta}(p)$ then $\rho(\theta)\in\Bcal^{\theta}_{\diam(\Pcal_i)}(p)$, where $\diam(\Pcal_i):=\sup_{\theta_1,\theta_2\in\Pcal_i}|\theta_1-\theta_2|$.
\end{restatable}

%For he partition of $[0,1]$ that we can consider is a simple uniform gird on $[0,1]$,
For instance, we can consider the set $\Theta_\delta$ defined above together with the natural partition of $[0,1]$ given by
\(\{[0,\delta],(\delta,2\delta],\ldots(\lfloor\frac1\delta\rfloor\delta,1]\}\).
%
The first part of \Cref{lem:discretization} states that if $\rho:\Theta\to A$ is a constant function on each partition $\Pcal_i$, then the values under the original and discretized instance are the same. The second part of the statement states that if the function $\rho$ recommends to an agent of type $\theta\in\Pcal_i$ an action that is a best response to a type $\tilde\theta\in\Pcal_i$, then $\rho(\theta)$ is $\diam(\Pcal_i)$-IC for any agent with type belonging to $\Pcal_i$.

%With this choice for any contract $p\in\Reals^m_{\ge 0}$ we can define $\rho^\delta_p(\theta):=b^{\Pi_{\Theta_\delta}(\theta)}(p)$, where $\Pi_A(x)$ is the $\ell_2$-projection of $x$ onto a set $A$. Clearly, the diameter of this partition is $\delta$.

Equipped with the lemma above, we can establish the existence of a $\PTAS$.


\begin{proof}[Proof of \Cref{thm:PTAS}]
    Let $\tilde p$ be the optimal contract of the discrete-type instance defined over types $\Theta_\delta$ and with distribution $\{\gamma_\theta\}_{\theta \in \Theta_\delta}$, and let $p\coloneqq \tilde p+\alpha(r-\tilde p)$, where $\delta>0$ and $\alpha>$ will be defined in the following.
    %
    By using the algorithm of \citet[Theorem~8]{castiglioni2022bayesian} or \citet[Lemma~2]{guruganesh2021contracts}, we can build $\tilde p$ in time $\poly(n^{|\Theta_\delta|}, m)$, where $|\Theta_\delta|=\lfloor\frac{1}{\delta}\rfloor$.

    First, we relate the principal's utility when playing $p$ in the original instance to their utility when playing $\tilde p$ in the discretized instance.
    %
    Let $\rho_{\tilde p}^\delta(\theta)$ be the function that assigns to each type $\theta\in [0,1]$ the best response played by the closest type in $\Theta_\delta$. Formally, for all $\theta\in[0,1]$, taking  $\theta_i\in \Theta_\delta$ such that $\theta \in(\theta_i-\delta/2,\theta_i+\delta/2]$, we have that $\rho_{\tilde p}^\delta(\theta) = b^{\theta_i}(\tilde p)$.
%
    Then, by the second part of \Cref{lem:discretization}, for each $\theta \in [0,1]$ and  $\theta_i \in \Theta_\delta$  such that $\theta \in(\theta_i-\delta/2,\theta_i+\delta/2]$, it holds
    \[ \rho_{\tilde p}^\delta(\theta)\in \Bcal^{\theta_i}(\tilde p) \subseteq \Bcal_{\delta}^{\theta}(\tilde p).   \]
    %where $\theta_i \in \Theta_\delta$ is such that $\theta \in(\theta_i-\delta/2,\theta_i+\delta/2]$.
    %
    Thus, by \Cref{lem:linearization}, we can readily say that for each $\theta\in\Theta$ it holds
	\begin{align*}
	\Up(p,\theta)+\left(\frac{\delta}{\alpha}+\alpha\right)\ge \Up(\tilde p, \rho^\delta_{\tilde p}(\theta)).
	\end{align*}
%
	By taking expectations over $\Gamma$ we find that
	\begin{align}\label{eq:ptastmp1}
	\Up(p)+\left(\frac{\delta}{\alpha}+\alpha\right)\ge \mathbb{E}_{\theta\sim\Gamma} \left[\Up(\tilde p, \rho^\delta_{\tilde p}(\theta))\right].
	\end{align}

    Now, we have to relate the optimal value of the discretized instance to the optimal value of the original one, showing that these are not too dissimilar.
    
    Instrumentally to this goal, we can apply the first part of \Cref{lem:discretization} to the right-hand side of \Cref{eq:ptastmp1}, obtaining:
	\begin{align}\label{eq:ptasAdded}
	\mathbb{E}_{\theta\sim\Gamma} \left[\Up(\tilde p, \rho^\delta_{\tilde p}(\theta))\right]= \sum_{\theta_i\in\Theta_{\delta}} \gamma_{\theta_i} \Up(\tilde p, \theta_i).
	\end{align}
	Now notice that $\tilde p$ is the optimum with respect to the discretized instance and thus  
    \begin{align}\label{eq:ptasAdded2}
    \sum_{\theta_i\in\Theta_{\delta}} \gamma_{\theta_i} \Up(\tilde p, \theta_i)\ge \sum_{\theta_i\in\Theta_{\delta}} \gamma_{\theta_i} \Up(\bar p, \theta_i)\quad \forall \bar p\in \Reals^m_{\ge 0}.
    \end{align}
	In particular, we can take $\bar p=p^{\star}_k+\alpha(r-p^{\star}_k)$, where $p^{\star}_k$ is any contract such that $\Up(p^{\star}_k)\ge \OPT-\frac{1}{k}$ where $k\in\Naturals$ (we observe that a contract that attains the supremum might not exist).
	%

        Now, for $i\in \lfloor \frac{1}{\delta}\rfloor$ and $k\in\Naturals$, we define the set of possible best responses of types in the interval $\Pcal_i:=(\theta_i-\delta/2,\theta_i+\delta/2]$ for contract $p^\star_k$ as 
        \[
        \mathcal{A}_{k,i}=\bigcup_{\theta\in\Pcal_i} \Bcal^{\theta}(p^\star_k).
        \]
        Then, we define a function $\rho:\Theta\to A$ so that, for any $\theta\in\Pcal_i$, it holds
        \[
	\rho(\theta) = \arg\max_{a\in\mathcal{A}_{k,i}} \Up( p^\star_k,a).
	\]
	% Now, we define a function $\rho:\Theta\to A$ as follows. For each $\theta\in \Theta$, let $i\in \lfloor \frac{1}{\delta}\rfloor$ the index such that $\theta\in \Pcal_i:=(\theta_i-\delta/2,\theta_i+\delta/2]$ and $\mathcal{A}_{k,i}=\cup_{\theta\in\Pcal_i} \Bcal^{\theta}(p^\star_k)$ the set of possible best responses for the types in the interval.  Then, we set
	% \[
	% \rho(\theta) = \arg\max_{a\in\mathcal{A}_{k,i}} \Up( p^\star_k,a).
	% \]
   Intuitively, $\rho(\theta)$ is the action that maximizes the principal's utility among all actions that are IC for a type $\theta'$ that is in the same partition as $\theta$.
    Now, thanks to the second part of \Cref{lem:discretization}, $\rho(\theta)\in\Bcal_\delta^{\theta}(p^\star_k)$.
    Hence, we can apply \Cref{lem:linearization} and take the expectation over $\Gamma$ obtaining
	\begin{align}\label{eq:ptastmp2}
    %\Up(p)+\left(\frac\delta\alpha+\alpha\right)\ge
    \sum_{\theta_i\in\Theta_{\delta}}\gamma_{ \theta_i}\Up(\bar p, \theta_i)\ge \sum_{\theta_i\in\Theta_{\delta}}\gamma_{\theta_i}\Up(p_k^\star, \rho(\theta_i))-\left(\frac\delta\alpha+\alpha\right).
	\end{align}
	%
	Since $\rho(\theta)$ is constant on each partition, we can use the first part of \Cref{lem:discretization} to get that
	\[
	\sum_{\theta_i\in\Theta_{\delta} }\gamma_{\theta_i}\Up(p_k^\star, \rho(\theta_i))=\mathbb{E}_{\theta\sim\Gamma} \left[\Up(p_k^\star, \rho(\theta))\right].
	\]
%	
	Now, consider each term $\Up(p_k^\star, \rho(\theta))$ separately. By our construction of $\rho(\theta)$ it is easy to see that 
    \begin{align}\label{eq:PTASFinal}
    \Up(p^\star_k, \rho(\theta))\ge \Up(p_k^\star, \theta).
    \end{align}
    Combining, \Cref{eq:ptastmp1,eq:ptastmp2,eq:ptasAdded,eq:ptasAdded2,eq:PTASFinal} we get that \(\Up(p)\ge\mathbb{E}_{\theta\sim\Gamma}[\Up(p^\star_k,\theta)]-2\left(\frac\delta\alpha+\alpha\right)\). Therefore,
	\begin{align*}
	\Up(p)&
    %\ge \mathbb{E}_{\theta\sim\Gamma} \left[\Up(\tilde p, \rho^\delta_{\tilde p}(\theta))\right] -\left(\frac\delta\alpha+\alpha\right) \tag{\Cref{eq:ptastmp1}}\\
    %&= \sum_{\theta_i\in\Theta_{\delta}} \gamma_{\theta_i} \Up(\tilde p, \theta_i) -\left(\frac\delta\alpha+\alpha\right) \tag{\Cref{eq:ptasAdded}}\\
    %&\ge \sum_{\theta_i\in\Theta_{\delta}} \gamma_{\theta_i} \Up(\bar p, \theta_i) -\left(\frac\delta\alpha+\alpha\right) \tag{\Cref{eq:ptasAdded2}}\\
    %&\ge \sum_{\theta_i\in\Theta_{\delta}}\gamma_{\theta_i}\Up(p_k^\star, \rho(\theta_i)) -2\left(\frac\delta\alpha+\alpha\right) \tag{\Cref{eq:ptastmp2}}\\
    \ge \mathbb{E}_{\theta\sim\Gamma}[\Up(p^\star_k,\theta)]-2\left(\frac\delta\alpha+\alpha\right) \\
    &=\Up(p^\star_k)-2\left(\frac\delta\alpha+\alpha\right)\ge \OPT-\frac{1}{k}-2\left(\frac\delta\alpha+\alpha\right).
	\end{align*}
    Choosing $\alpha=\sqrt{\delta}=\epsilon^2/16$ and taking the limit $k\to+\infty$ concludes the proof.	
\end{proof}

\begin{comment}
\ma{======OLD======}

The idea of the algorithm is fairly simple, with a slightly less simple analysis of which we try to give the main intuition here. The main technical tool is the well know relationship between IC and $\delta$-IC contracts first resented in \cite{dutting2021complexity}.
We will combine this now staple feature of algorithmic contracts design with the simple observation that the set of types $\Theta$ admits efficient covers. More formally we will build a discretized instance of our problem that has roughly $O(1/\delta)$ many types. Previous works~\cite{castiglioni2022bayesian, guruganesh2021contracts} provide algorithms (for the multi-dimensional setting) that are exponential only in the number of types. Moreover, it is easy to see that contracts that are IC for the discretized multi-dimensional setting are $\delta$-IC for the original single-dimensional setting. Here comes into play the relation in principal utility between IC and $\delta$-IC contracts of \cite{dutting2021complexity}. We will restate this result in \Cref{lem:linearization} for completeness.




The space of possible types $\Theta$ is small 

For any $\delta>0$ consider the set $\tilde\Theta_\delta=\{k\delta\}_{k=0}^{\lfloor 1/\delta\rfloor}$ and compute the probability of each type $\theta \in \tilde\Theta_\delta$ as $ \gamma_{\theta}:=\int_{\theta':|\theta'-\theta|\le\delta/2}\Gamma(\theta') d\theta'$ for all $\theta\in \tilde\Theta_\delta$.
\footnote{Computationally we assume to have a oracle that for each $[a,b]\subset[0,1]$ computes $\int_{a}^b\Gamma(\theta)d\theta$.}
%
Also consider the discrete set of costs $\tilde c_{\theta, a}:=\theta\cdot c_a$ for each $\theta\in\tilde \Theta_\delta$. Now consider the bayesian contract design problem defined by the set of types $\tilde \Theta_{\delta}$ with distribution $\{ \gamma_{\theta}\}_{\theta\in\tilde\Theta_\delta}$, costs $\{\tilde c_{a,\theta}\}_{\theta\in\tilde\Theta_\delta,a\in A}$, the set of outcomes, outcome distribution and rewards of the principal are left unchanged. We will refer to the instance built in this way as a discretized instance. \citet[Theorem~8]{castiglioni2022bayesian} gives an algorithm with running time $\poly(n^{1/\delta}, m)$ that finds the optimal contract $\tilde p$ on the discretized instance. Then we can build efficiently a new contract $p$ from $\tilde p$ and prove that this gives an utility grater then $\OPT-O(\sqrt{\delta})$. First we need two technical lemmas. The first is a fundamental result about the stabilization of 

\begin{lemma}[{\citep[Proposition~2.4]{dutting2021complexity}}]\label{lem:linearization}
	Let $\rho:\Theta\to A$ define a function that prescribe the action $\rho(\theta)$ that an agent of type $\theta\in\Theta$ takes. Let $p\in\Reals^m_{\ge 0}$ be such that $\rho(\theta)\in \Bcal^{\theta}_{\epsilon}(p)$ for all $\theta\in\Theta$, then $\tilde p:=p+\alpha(r-p)$ is such that $\Up(\tilde p,\theta)\ge \Up(p,\rho(\theta))-\left(\frac{\epsilon}{\alpha}+\alpha\right)$.
\end{lemma}

% \begin{proof}
% 	First we prove that
% 	\begin{equation}\label{eq:tmp_lineariz_1}
% 		\Ua_\theta(\tilde p, \theta)\ge \Ua_{\tilde\theta}(p, b^\theta(\tilde p)) - \epsilon + \alpha\cdot \Up(p,\rho(\theta)).
% 	\end{equation}
% 	This easily follow from the following
% 	\begin{align*}
% 		\Ua_\theta(\tilde p, \theta)&=\sum_{\omega\in\Omega} F_{b^\theta(\tilde p),\omega} \tilde p_\omega-\tilde\theta c_{b^\theta(\tilde p)}\\
% 		&\ge \sum_{\omega\in\Omega} F_{\rho(\theta),\omega} \tilde p_\omega-\tilde\theta c_{\rho(\theta)}\tag{Definition of $b^\theta(\tilde p)$}\\
% 		&=\sum_{\omega\in\Omega} F_{\rho(\theta),\omega} p_\omega-\tilde\theta c_{\rho(\theta)} + \alpha\cdot \Up(p,\rho(\theta))\tag{Definition of $\tilde p$}\\
% 		&=\Ua_{\tilde\theta}(p, \rho(\theta))+ \alpha\cdot \Up(p,\rho(\theta))\\
% 		&\ge \Ua_{\tilde\theta}(p, b^\theta(\tilde p)) - \epsilon + \alpha\cdot \Up(p,\rho(\theta))\tag{Since $\rho(\theta)\in \Bcal^{\theta,\epsilon}_p$}.
% 	\end{align*}
% 	
% 	Now, for any $a\in A$, it is easy to rewrite $\Ua_\theta(\tilde p, a)$ as 
% 	\[
% 	\Ua_\theta(\tilde p, a) = \Ua_\theta( p, a)+\alpha\cdot \Up( p, a).
% 	\]
% 	
% 	Thus by picking $a=b^\theta(\tilde p)$ and combining it with \Cref{eq:tmp_lineariz_1}we can conclude that
% 	\[
% 	 \Ua_\theta( p, b^\theta(\tilde p))+\alpha\cdot \Up( p, b^\theta(\tilde p))\ge \Ua_{\tilde\theta}(p, b^\theta(\tilde p)) - \epsilon + \alpha\cdot \Up(p,\rho(\theta))
% 	\]
% 	%
% 	and by rearranging it we obtain
% 	\begin{align}\label{eq:tmp_lineariz_2}
% 	\Up( p, b^\theta(\tilde p))\ge \Up(p,\rho(\theta))-\frac{\epsilon}{\alpha}.
% 	\end{align}
% 	
% 	Moreover for each action $a\in A$ we have the following relation between the principal utility of contract $\tilde p$ and contract $p$:  
% 	\[
% 	\Up(\tilde p, a):=\sum_{\omega\in\Omega} F_{a,\omega}(\tilde p_{\omega}-r_\omega)=(1-\alpha)\sum_{\omega\in\Omega} F_{a,\omega}( p_{\omega}-r_\omega)=(1-\alpha)\Up(p, a),
% 	\]
% 	and thus, by taking $a=b^{\theta}(\tilde p)$ and combining it with \Cref{eq:tmp_lineariz_2}, we can conclude that
% 	\[
% 	\Up(\tilde p,\theta)\ge (1-\alpha)\left( \Up(p,\rho(\theta))-\frac{\epsilon}{\alpha}\right)\ge \Up(p,\rho(\theta))-\left(\frac{\epsilon}{\alpha}+\alpha\right),
% 	\]
% 	where the last inequality holds since $\Up(p,a)\le 1$ for all $a\in A$.	
% \end{proof}

\begin{lemma}\label{lem:discretization}
    Let $\Pcal=\{\Pcal_1,\ldots,\Pcal_k\}$ be any finite partition of $[0,1]$ and any $\hat\Theta:=\{\theta_1,\ldots,\theta_k\}$ such that $\theta_i\in\Pcal_i$ for $i\in[k]$. Then consider the function $\rho:\Theta\to A$ such that for each $\Pcal_i$ and $\theta\in\Pcal_i$ we have that $\rho(\theta)=\rho(\theta_i)$. Then for any contract $p\in\Reals^m_{\ge 0}$ we have
	\[
	\mathbb{E}_{\theta\sim\Gamma} \left[\Up(p, \rho(\theta))\right] = \sum_{\theta_i\in\hat\Theta} \gamma_{\theta_i} \Up(p, \rho(\theta_i)),
	\]
	where $\gamma_{\theta_i}:=\mathbb{P}_{\theta\sim\Gamma}[\theta\in \Pcal_i]$.
	
	Moreover if $\rho(\theta_i)=b^{\tilde\theta}(p)$ for some $\tilde\theta\in\Pcal_i$ then $\rho(\theta)\in\Bcal^{\theta}_{\diam(\Pcal_i)}(p)$ for all $\theta\in\Theta$, where $\diam(\Pcal):=\sup_{\theta_1,\theta_2\in\Pcal_i}|\theta_1-\theta_2|$
\end{lemma}

A typical partition of $[0,1]$ that we can consider is a simple $\delta$-net of $[0,1]$ as $\{[0,\delta],(\delta,2\delta],\ldots(\lfloor\frac1\delta\rfloor\delta,1]\}$ and $\Theta_{\delta}:=\{i\cdot\delta/2\}_{i\in\lfloor\frac1\delta\rfloor}$. With this choice for any contract $p\in\Reals^m_{\ge 0}$ we can define $\rho^\delta_p(\theta)$
\[
\rho^\delta_p(\theta):=b^{\Pi_{\Theta_\delta}(\theta)}(p),
\]
where $\Pi_A(x)$ is the $\ell_2$-projection of $x$ onto a set $A$. Clearly the diameter of this partition is $\delta$.

\begin{theorem}
    There is an algorithm that for any $\epsilon>0$ runs in time $\poly(n^{{1}/{\epsilon}},m)$ and finds a contract $p\in\Reals^m_{\ge 0}$ such that $\Up(p)\ge \OPT-O(\epsilon)$.
\end{theorem}

\begin{proof}
	Let $\tilde p$ be the optimal contract of the discretized instance and $p=\tilde p+\alpha(r-\tilde p)$. By using the algorithm of \citet[Theorem~8]{castiglioni2022bayesian}, we can build $\tilde p$ in $\poly(n^{|\Theta_\delta|}, m)$ time, where $|\Theta_\delta|=\lfloor\frac{1}{\delta}\rfloor$.
	
	By combining \Cref{lem:linearization} and the second statement of \Cref{lem:discretization}, we can readily say that for each $\theta\in\Theta$ it holds that for all $\theta\in\Theta$ we have
	\[
	\Up(p,b^\theta(p))\ge \Up(\tilde p, \rho^\epsilon_{\tilde p}(\theta))-\left(\frac{\epsilon}{\alpha}+\alpha\right)
	\]
	and by integrating on $\Theta$ with respect $\mu_\theta$ we find that
	\[
	\Up(p)+\left(\frac{\epsilon}{\alpha}+\alpha\right)\ge \int_{\Theta}\mu_\theta \Up(\tilde p, \rho^\epsilon_{\tilde p}(\theta))d\theta.
	\]
	
	Then we can analyze the right hand side with the first statement of \Cref{lem:discretization} and find that
	\[
	\Up(p)+\left(\frac{\epsilon}{\alpha}+\alpha\right)\ge \sum_{\tilde\theta\in\Theta_{\epsilon}} \tilde\mu_{\tilde\theta} \Up(\tilde p, b^{\tilde\theta}(\tilde p)).
	\]
	Now notice that by definition $\tilde p$ is the optimum with respect to the discrete instance and thus we can consider any other contract $\bar p$ and $\sum_{\tilde\theta\in\Theta_{\epsilon}} \tilde\mu_{\tilde\theta} \Up(\tilde p, b^{\tilde\theta}(\tilde p))\ge \sum_{\tilde\theta\in\Theta_{\epsilon}} \tilde\mu_{\tilde\theta} \Up(\bar p, b^{\tilde\theta}(\bar p)).$ %Any choice of $\bar p$ and using again the first statement of \Cref{lem:discretization}, let us thus conclude that
	\[
	\Up(p)+\left(\frac\epsilon\alpha+\alpha\right)\ge \sum_{\tilde\theta\in\Theta_{\epsilon}} \tilde\mu_{\tilde\theta} \Up(\bar p, b^{\tilde\theta}(\bar p)).
	\]
	In particular we can take $\bar p=p^{\star}_k+\alpha(r-p^{\star}_k)$, where $p^{\star}_k$ is any contract such that $\Up(p^{\star}_k)\ge \OPT-\frac{1}{k}$ where $k\in\Naturals$ (note that a contract that attains the supremum might not exist).
	%
	Now we define a function $\rho:\Theta\to A$ as follows:
	\[
	\rho(\theta) = \arg\max_{a\in\cup_{\theta':|\theta-\theta'|\le\epsilon/2} b^{\theta'}(p^\star_k)} \Up(a, p^\star_k).
	\]
	
	We claim that $\rho(\theta)\in\Bcal^{\theta,\epsilon}_{p^\star_k}$, and thus we can apply \Cref{lem:linearization} and, after integration over $\Theta$ with respect to $\tilde\mu_\theta$, we can conclude that 
	\[
		\Up(p)+\left(\frac\epsilon\alpha+\alpha\right)\ge\sum_{\tilde\theta\in\Theta_{\epsilon}}\tilde\mu_{\theta}\Up(\bar p, b^{\tilde\theta}(\bar p))\ge \sum_{\tilde\theta\in\Theta_{\epsilon}}\tilde\mu_{\tilde\theta}\Up(p_k^\star, \rho(\tilde\theta))-\left(\frac\epsilon\alpha+\alpha\right).
	\]
	
	Since $\rho(\theta)$ is constant on each $\theta\in[\Pi_{\Theta_\epsilon}(\theta)-\epsilon/2,\Pi_{\Theta_\epsilon}(\theta)+\epsilon/2]$ we get that
	\[
	\sum_{\tilde\theta\in\Theta_{\epsilon}}\tilde\mu_{\tilde\theta}\Up(p_k^\star, \rho(\tilde\theta))=\int_\Theta\mu_\theta \Up(p_k^\star, \rho(\theta))d\theta
	\]
	
	Now consider each term $\Up(p_k^\star, \rho(\theta))$. By our construction of $\rho(\theta)$ it is easy to see that $\Up(p^\star_k, \rho(\theta))\ge \Up(p_k^\star, b^{\theta}(p_k^\star))$ and thus:
	\[
	\Up(p)\ge \int_\Theta \Up(p^\star_k,b^{\theta}(p^\star_k))d\theta-2\left(\frac\epsilon\alpha+\alpha\right)=\Up(p^\star_k)-2\left(\frac\epsilon\alpha+\alpha\right)\ge \OPT-\frac{1}{k}-2\left(\frac\epsilon\alpha+\alpha\right).
	\]
	Choosing $\alpha=\sqrt{\epsilon}$ and taking the limit $k\to+\infty$.

	
\end{proof}





\end{comment}