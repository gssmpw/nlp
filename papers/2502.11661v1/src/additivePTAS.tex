\section{additive PTAS}

Discretizzare e fare un menu diretto deterministico (epsonenziale nel numero di tipi) e poi usare linearizzazione.

Here, we show that the problem admits an additive PTAS.

Divide the space of types in segments and find an optimal contract/menu $\epsilon$-IC with a constant number of types.

Robustify the solution.



\section{hardness}

\begin{theorem}
	No additive FPTAS.
\end{theorem}

\begin{proof}
	We reduce from set cover.
	%
	Given a set of elements $E$ and a set of sets $\Scal \subset 2^E$, we reduce from the decision problem of determining if there exists a set cover of size $k$, i.e., a set $\Scal^*\subseteq \Scal$ of size $k$ such that 
	$\cup_{S \in \Scal^*} S=E$. This problem is well known to be NP-Hard \cite{feige..}.
	
	We show that if there exists a set cover of size $k$, then principal's utility from the optimal contract is at least $\gamma$ where $\gamma$ will be defined in the following, while if all the set covers have size at least $k+1$ then the utility is at most $\gamma-|E|^{-14}|S|^{-1}$.
	
	\textbf{Construction.}
	%
	Let rename the elements of $E$ as $1,2,\ldots, n$, where $|E|=n$.
	Let $|\mathcal{S}|=m$.
	Moreover, let define  $\rho=n^{-6}$, $\eta=n^{-2}$, $\epsilon=n^{-8}m^{-1}$, and $\ell=n^{-9}m^{-1}$
	%
	%Moreover, when we write $(i,j)=e\in E$, we always assume that $i>j$.
	The instance includes an outcome $\omega_{S}$ with reward $r_{\omega_S}=0$ for each $S \in \Scal$.
	Moreover, it includes an outcome $\omega^*$ with $r_{\omega^*}=0$ and an outcome $\bar \omega$ with $r_{\bar \omega}=0$.
	
	The action available to the agent are the following.
	For each set $S\in \Scal$ and $i\in S$, an action $a_{i,S}$ with $F_{a_{i,S},\omega_S}=\frac{1}{10i}\ell$,  $F_{a_{i,S}, \omega^*}=\frac{1}{5i}\ell$, and $F_{a_{i,S},\bar \omega}=1-\frac{1}{10i}\ell-\frac{1}{5i}\ell$, where we recall that $i\in \mathbb{N}$ is an integer denoting an element.
	The cost of the action is $c_{a_{i,S}}=\frac{1}{20i^2}\ell$. 
	For each set $S$ and each $i\in S$, an action $\bar a_{i,S}$ with $F_{\bar a_{i,S},\omega_S}=\frac{1}{10i}(1-\eta/2)\ell$, $F_{\bar a_{i,S},\bar \omega}=1-\frac{1}{10i}(1-\eta/2)\ell$, and $c_{\bar a_{i,S}}=\frac{1}{20i^2} (1-\eta)\ell$.
	An action $a^*$ with cost $c_{a^*}=1$, $F_{a^*,\omega_S}=\epsilon$ for all $S\in \mathcal{S}$ and $F_{a^*,\omega^*}=1-m\epsilon$.
	
	Finally, the agent's types include a type $\theta=i$ for each $i\in E$, and an additional type $\theta=0$.
	\mat{ricordare di normalizzare i tipi tra 0 e 1se vogliamo.}
	
	The distribution is defined as follows: $\mu_{i}=\frac{1-\rho}{n}$ for each $i \in E$ and $\mu_0=\rho$.
	
	
	
	
	\textbf{if.}
	Suppose that there exists a set cover $\Scal^*$ of size $k$.
	We build a contract with utility at least 
	\[\gamma=	 \frac{1-\rho}{n} \ell \sum_{i} \frac{1}{10i} + \rho (1-m\epsilon) - \rho \epsilon |\Scal^*|  \]
	as follow.
	For each set $S\in \Scal^*$, set
	$p_{\omega_S}=1$.
	Moreover, let $p_{\omega^*}=p_{\bar \omega}=0$ and $p_{\omega_S}=0$ for all $S\notin \Scal^*$.
	
	Then, we compute the best responses for each possible agent's type.
	Consider a type $\theta=i$, $i \in V$.
	Since $\Scal$ is a set cover, there exists a set $S \in \Scal^*$ such that $i \in S$.
	We show that $a_{i,S}$ is a best response for the agent of type $i$.
	For an agent of type $i$, the utility of playing an action $a_{j, S}$ is
	\[
	 \sum_{\omega \in \Omega} F_{ a_{j, S}, \omega } p_\omega - i \cdot c_{a_{j, S}} 
	 \]
	
	
	The agent's utility playing action $a_{i,S}$ is 
	\[ \frac{1}{10i}\ell p_{\omega_{S}} - \frac{1}{20i}\ell=\frac{1}{20i}\ell. \]
	
	We show that the utility playing any other action is not higher.
	The agent utility playing an action $a_{j,S'}$, $S'\in \Scal$, $j \in S'$  is at most 
	\[ \frac{1}{10j}\ell p_{\omega_{S'}} -\frac{i}{20j^2}\ell\le \frac{1}{10j}\ell -\frac{i}{20j^2}\ell\le\frac{1}{20i}\ell. \]
	\mat{trovare modo  elegante per dimostrarlo. foglio martino?}
	
	Playing an action ${\bar a_{j,S'}}$, $S'\in \Scal$, $j \in S'$ , the agent's utility is at most
	\[ \frac{1}{10j} (1-\eta/2)\ell p_{\omega_{S'}}-\frac{i}{20j^2} (1-\eta) \ell\le  \frac{1}{10j} (1-\eta/2)\ell -\frac{i}{20j^2} (1-\eta) \ell \]
	The utility is maximized for $j=i$ getting utility $1/i$.
	Indeed, the maximum is at
	\[ j= i \frac{1-\eta}{1-\eta/2}  \]
	and the agent utility is 
	\[\frac{1}{10i} (1-\eta/2)\ell-\frac{1}{20i} (1-\eta) \ell=\frac{1}{20i}\ell \]	
	\mat{dimostrare ma anche qui ho pochi dubbi.}
	
	Finally, it is easy to see that the agent's utility playing action $a^*$ is negative.
	Hence, the best response of type $\theta=i$ is $a_{i,S}$.\footnote{dire che questa è quella che rompe i pareggi in favore del principal?}
	
	Finally, consider the type $\theta=0$.
	Taking action $a^*$, the agent's utility is
	\[|\Scal^*|\epsilon\ge \epsilon > \ell, \]
	while the utility taking any other action is at most $\ell$.
	
	Hence, the expected principal's utility is
	\[  \frac{1-\rho}{n} \ell \sum_{i} [\frac{1}{5i}- \frac{1}{10i}] + \rho (1-m\epsilon - \epsilon |\Scal^*|)= \frac{1-\rho}{n} \ell \sum_{i} \frac{1}{10i} + \rho (1-m\epsilon - \epsilon |\Scal^*|) =\gamma . \]
	
	\textbf{only if.}
	Suppose that all set covers have size at least $k+1$.
	%
	We show that any contract has utility at most $\gamma-n^{14}m^{-1}$.
	
	Let $p$ be a contract.
	% First, we show that $p_{\omega^*}<\gamma$.
	%Otherwise, we get that the expected utility is at most
	%\begin{align*}
	%	\rho n + (1-n\rho) (1-|E|\epsilon) - (1-n\rho) (1-|E|\epsilon) p_{\omega^{\star}} \ge \rho \sum_{i} (\frac{1}{2}+\frac{1}{10i}) + (1-n\rho) (1-|E|\epsilon) - (1-n\rho)\epsilon |E^*| -\nu		
	%\end{align*}
	%implying
	%\begin{align}
	%	p_{\omega^{\star}} \le \frac{1}{(1-n\rho) (1-|E|\epsilon)} [\rho n + \epsilon |E| + \nu   ]\le 2 [n^{-6}+n^{-8}+n^{-6}] \le 6 n^{-6} 
	%\end{align}
	We split the set of elements $E$ in three sets.
	$E_1\subseteq E$ includes the elements $i$ such that the best response of type $i$ to contract $p$ is $a_{i,S}$, for an $S \in \Scal$.
	%
	$E_2\subseteq E$ includes  the elements $i$ such that the best response of type $i$ is $a_{j,S}$ with $j\neq i$ and $S \in \Scal$.
	%
	Finally, $E_3=E \setminus (E_1\cup E_2)$.
	
	We start analyzing the constraints that the best responses of types $\theta \in E_1$ impose on the contract.
	Consider a type $\theta=i \in E_1$, and let $a_{i,S}$ be their best response. Notice that $i\in S$.
	%
	Then, from the IC constraint with respect to action $\bar a_{i,S}$ it holds
	
	\[p_{\omega_{S}}\frac{1}{10i}\ell + p_{\omega^*}  \frac{1}{5i}\ell- \frac{1}{20i}\ell \ge  p_{\omega_{S}}\frac{1}{10i} (1-\eta/2)\ell - \frac{1}{20i} (1-\eta) \ell \]
	implying
	\[p_{\omega_{S}}\ge 1-4  p_{\omega^*}/\eta.\]
	
	Now, consider an agent's type $\theta=i\in E_2$, and let $a_{j,S}$, $j \neq i$ be the action played by the agent. 
	Notice that $j \in S$. Then, the IC constraint with respect to action $\bar a_{j,S}$ implies
	\[ p_{\omega_{S}} \frac{1}{10j} \ell + \frac{1}{5j}p_{\omega^*}\ell- \frac{i}{20j^2}\ell \ge p_{\omega_{S}} \frac{1}{10j} (1-\eta/2) \ell- \frac{i}{20j^2} (1-\eta) \ell \]
	and
	%\[ p_{\omega_{S}} \frac{\eta}{20j} \ge  \frac{i\eta}{20j^2} - \frac{1}{5j} p_{\omega^*} \]
	%and
	\[ p_{\omega_{S}} \ge \frac{i}{j}- p_{\omega^*} \frac{4}{\eta}. \]
	
	Hence, the principal's expected utility when the realized type is a $\theta=i \in E_2$ is at most:
	\[ \frac{1}{5j} (1-p_{\omega_{S}}) - \frac{i}{10j^2} +  \frac{1}{10j} p_{\omega^*} \frac{4}{\eta}\le \frac{1}{5j} - \frac{i}{10j^2} +  \frac{1}{10j} p_{\omega^*} \frac{4}{\eta}
	\le \frac{1}{10i}-\frac{1}{40n^3}  + \frac{2}{5j\eta}p_{\omega^*}  \]
	\mat{da dimostrare, ho checkato su wolfram}
	
	Finally, consider the set of agent types $\theta=i\in E_3$. These agent's types play an action $\bar a_{j,S}$, $j\in E$, $S\in \Scal$ or action $a^*$, or \mat{or void action (aggiungere al modello)}. It is easy to see that the principal's utility is at most $0$ for all these possible actions. 
	
	Finally, type $\theta=0$ plays action $a^*$ otherwise the utility is clearly below $\gamma-n^{14}m^{-1}$.
	
	
	
	Let $\bar \Scal$ be the sets $S$ such that $p_{\omega_S}\ge 1- 4 p_{\omega^*}$.
	Then, the principal's utility is at most 
	\[ \frac{1-\rho}{n} \ell \left[\sum_{i \in E_1}( \frac{1}{5i} (1-p_{\omega^{*}}) - \frac{1}{10i}( 1-4p_{\omega^*}/\eta))  + \sum_{i \in E \setminus E_1} (\frac{1}{10i}-\frac{1}{40n^3}  + \frac{2}{5j\eta}p_{\omega^*})\right] + \rho (1-m\epsilon) (1-p_{\omega^*}) - \rho \epsilon |\bar\Scal|(1-4p_{\omega^*}/\eta)\]
	\[\le \frac{1-\rho}{n} \ell \left[\sum_{i \in E_1}( \frac{1}{5i}  - \frac{1}{10i}( 1-4p_{\omega^*}/\eta))  + \sum_{i \in E \setminus E_1} (\frac{1}{10i}-\frac{1}{40n^3}  + \frac{2}{5j\eta}p_{\omega^*})\right] + \rho (1-m\epsilon) (1-p_{\omega^*}) - \rho \epsilon |\bar\Scal|(1-4p_{\omega^*}/\eta)   \]
	
	
	Then, since
	\[ \frac{1-\rho}{n} \ell  n \frac{2}{5} \frac{1}{\eta} - \rho (1-m\epsilon) + 4 \rho \epsilon m \frac{1}{\eta} \le  \ell/\eta -\rho/2 + 4 \rho \epsilon m/\eta=n^{-7}m^{-1} - n^{-6}/2 + 4 n^{-12}<0 \]
	for $n$ large enough, it is the case that
	
	\[\arg \max_{p_{\omega^*}}  \frac{1-\rho}{n} \ell [\sum_{i \in V^*}( \frac{1}{5i}  - \frac{1}{10i}( 1-4p_{\omega^*}/\eta))  + \sum_{i \notin V^*} (\frac{1}{10i}-\frac{1}{40n^3}  + \frac{2}{5j\eta}p_{\omega^*})] + \rho (1-m\epsilon) (1-p_{\omega^*}) - \rho \epsilon |\bar\Scal|(1-4p_{\omega^*}/\eta)= 0,\]
	
	Hence, setting $p_{\omega^*}=0$ the principal's utility is at most
	\begin{align}
		&\frac{1-\rho}{n} \ell \left[\sum_{i \in E_1}( \frac{1}{5i}  - \frac{1}{10i})  + \sum_{i \in E \setminus E_1} (\frac{1}{10i}-\frac{1}{40n^3}) \right] + \rho (1-m\epsilon) - \rho \epsilon |\bar\Scal|\\
		&=\frac{1-\rho}{n} \ell \left[\sum_{i \in E_1} \frac{1}{10i} + \sum_{i \in E \setminus E_1} (\frac{1}{10i}-\frac{1}{40n^3}) \right] + \rho (1-m\epsilon) - \rho \epsilon |\bar\Scal|\\
		& \le \frac{1-\rho}{n} \ell \sum_{i \in E} \frac{1}{10i}    + \rho (1-m\epsilon) - \rho \epsilon (|\bar\Scal|+ |E\setminus E_1|) 
	\end{align}
	where the last inequality follows since 
	\[\frac{1-\rho}{n} \ell \frac{1}{40n^3}\ge n^{-13} m^{-1}/80 \ge n^{-14} m^{-1}  = \rho \epsilon. \]
	for $n$ large enough.
	
	Let define the function $s:E\rightarrow\Scal$ that assign to each element $i \in E$ a set $S \in \Scal$ such that $i\in S$.
	Now, notice that there exists a set cover of size $|\bar\Scal|+ |E\setminus E_1|$ taking the set $S=\bar\Scal\cup (\cup_{v \in V\setminus V^*} s(v))$. Indeed, all the elements in $E_1$ are covered by a set in $\Scal$ by definition, while $\cup_{E \in E \setminus E_1} s(v)$ clearly covers all the elements not in $E_1$.
	
	Thus,the principal's utility is at most
	\[ \frac{1-\rho}{n} \ell \sum_{i \in V} \frac{1}{10i}    + \rho (1-m\epsilon) - \rho \epsilon (k+1)\le \gamma - \rho \epsilon= \gamma- n^{-14} m^{-1}, \]
	concluding the proof.
	
	
	
	
	
\end{proof}

\section{martino version}

\section{additive PTAS}


Discretizzare e fare un menu diretto deterministico (epsonenziale nel numero di tipi) e poi usare linearizzazione.


Here, we show that the problem admits an additive PTAS. The algorithm works as follows

For any $\epsilon>0$ consider the set $\tilde\Theta_\epsilon=\{k\epsilon\}_{k=0}^{\lfloor 1/\epsilon\rfloor}$ and compute the probability of each type $\theta \in \tilde\Theta_\epsilon$ as $\tilde \mu_{\theta}:=\int_{\theta-\epsilon/2}^{\theta+\epsilon/2}\mu_\theta d\theta$ for all $\theta\in \tilde\Theta_\epsilon$.\ma{va approssimato forse o fare assunzioni su come mi viene data $\mu_\theta$}. \mat{la cosa più sensata mi pare avere un oracolo che di calcola l'intergrale dato l'intervallo}. Also consider the discrete set of costs $\tilde c_{\theta, a}:=\theta\cdot c_a$ for each $\theta\in\tilde \Theta_\epsilon$. Now consider the bayesian contract design problem defined by the set of types $\tilde \Theta_{\epsilon}$ with distribution $\{\tilde \mu_{\theta}\}_{\theta\in\tilde\Theta_\epsilon}$, costs $\{\tilde c_{a,\theta}\}_{\theta\in\tilde\Theta_\epsilon,a\in A}$, the set of outcomes, outcome distribution and rewards of the principal are left unchanged. \citet[Theorem~8]{castiglioni2022bayesian} gives an algorithm with running time $\poly(n^{1/\epsilon}, m)$ that finds the optimal contract $\tilde p$.

\begin{lemma}
		
\end{lemma}

\begin{lemma}
	A contract $p$ is such that for every $\tilde \theta\in\left[\theta-\frac{\epsilon}{2},\theta+\frac{\epsilon}{2}\right]$ we have that 
	\[
	\sum_{\omega\in\Omega} F_{b^{ \theta}(p),\omega}p_\omega-{\tilde \theta c_{b^{ \theta}(p)}} \ge 	\sum_{\omega\in\Omega} F_{a,\omega}p_\omega-{\tilde \theta c_a }-O(\epsilon)\quad\forall a\in A.
	\]
\end{lemma}

\begin{proof}
	\begin{align}
		\sum_{\omega\in\Omega} F_{b^{ \theta}(p),\omega}p_\omega-{ \theta c_{b^{ \theta}(p)}} &\ge \sum_{\omega\in\Omega} F_{a,\omega}p_\omega-{ \theta c_{a}} \tag{definition of $b^\theta(p)$}\\
		&=\sum_{\omega\in\Omega} F_{a,\omega}p_\omega-{ \tilde\theta c_{a}} +(\tilde\theta-\theta)c_{a}\\
		&\ge \sum_{\omega\in\Omega} F_{a,\omega}p_\omega-{ \tilde\theta c_{a}} -\epsilon/2.
	\end{align}
\end{proof}










\section{hardness}


\begin{theorem}
	No additive FPTAS.
\end{theorem}


\begin{proof}
	We reduce from set cover.
	%
	Given a set of elements $E$ and a set of sets $\Scal \subset 2^E$, we reduce from the decision problem of determining if there exists a set cover of size $k$, i.e., a set $\Scal^*\subseteq \Scal$ of size $k$ such that 
	$\cup_{S \in \Scal^*} S=E$. This problem is well known to be NP-Hard \cite{feige..}.
	
	We show that if there exists a set cover of size $k$, then principal's utility from the optimal contract is at least $\gamma$ where $\gamma$ will be defined in the following, while if all the set covers have size at least $k+1$ then the utility is at most $\gamma-|E|^{-14}|S|^{-1}$.

%	\textbf{Construction.}
%	%
%	Let rename the elements of $E$ as $1,2,\ldots, n$, where $|E|=n$.
%	Let $|\mathcal{S}|=m$.
%	Moreover, let define  $\rho=n^{-6}$, $\eta=n^{-2}$, $\epsilon=n^{-8}m^{-1}$, and $\ell=n^{-9}m^{-1}$
%	%
%	%Moreover, when we write $(i,j)=e\in E$, we always assume that $i>j$.
%	The instance includes an outcome $\omega_{S}$ with reward $r_{\omega_S}=0$ for each $S \in \Scal$.
%	Moreover, it includes an outcome $\omega^*$ with $r_{\omega^*}=0$ and an outcome $\bar \omega$ with $r_{\bar \omega}=0$. \ma{reward sono tutti zero. tbc}
%	
%	The action available to the agent are the following. \ma{$\mu$ non è definito}
%	For each set $S\in \Scal$ and $i\in S$, an action $a_{i,S}$ with $F_{a_{i,S},\omega_S}=\frac{1}{10i}\mu$,  $F_{a_{i,S}, \omega^*}=\frac{1}{5i}\mu$, and $F_{a_{i,S},\bar \omega}=1-\frac{1}{10i}\mu-\frac{1}{5i}\mu$, where we recall that $i\in \mathbb{N}$ is an integer denoting an element.
%	The cost of the action is $c_{a_{i,S}}=\frac{1}{20i^2}\mu$. 
%	For each set $S$ and each $i\in S$, an action $\bar a_{i,S}$ with $F_{\bar a_{i,S},\omega_S}=\frac{1}{10i}(1-\eta/2)\mu$, $F_{\bar a_{i,S},\bar \omega}=1-\frac{1}{10i}(1-\eta/2)\mu$, and $c_{\bar a_{i,S}}=\frac{1}{20i^2} (1-\eta)\mu$.
%	An action $a^*$ with cost $c_{a^*}=1$, $F_{a^*,\omega_S}=\epsilon$ for all $S\in \mathcal{S}$ and $F_{a^*,\omega^*}=1-m\epsilon$.
%	
%	
%	Finally, the agent's types include a type $\theta=i$ for each $i\in E$, and an additional type $\theta=0$.
%	\mat{ricordare di normalizzare i tipi se vogliamo.}
%	
%	The distribution is defined as follows: $\mu_{i}=\frac{1-\rho}{n}$ for each $i \in E$ and $\mu_0=\rho$.
	
	\paragraph{Construction}
	\ma{new construction}
	We identify $E$ with the set $\{1,2,\ldots, n\}$, where $|E|=n$, moreover let $m=|\mathcal{S}|$.
	For ease of presentation, we define  $\rho=n^{-6}$, $\eta=n^{-2}$, $\epsilon=n^{-8}m^{-1}$, and $\ell=n^{-9}m^{-1}$ \ma{$=\mu$?}. \mat{corretto ma dovrei aver rimosso tutti i mu... forse sono rimasti qui ma non sopra.}
	\begin{itemize}
		%
		\item Outcomes
		\begin{itemize}
		\item For each $S \in \Scal$ there is an outcome $\omega_{S}$ 
		%
		\item We define two additional outcomes $\omega^*$ and $\bar \omega$,
		\end{itemize}
		
		\item Rewards
		\begin{itemize}
		\item  For each $S \in \Scal$, reward $r_{\omega_S}=0$. 
		\item $r_{\omega^*}=1$ and $r_{\bar \omega}=0$. 
		\end{itemize}
		
		\item Actions
		\begin{itemize}
			\item For each set $S\in \Scal$ and $i\in S$, we define an action $a_{i,S}$
			\item Moreover, for each set $S\in\Scal$ and each $i\in S$, an action $\bar a_{i,S}$
			\item Additional action $a^*$
			\item $a_0$
		\end{itemize}
		
		\item Action's cost
		\begin{itemize}
			\item $c_{a_{i, S}}=\frac{1}{20 i^2}\mu$ for each $S\in\Scal$ and $i\in S$
			\item $c_{\bar a_{i, S}}=\frac{1}{20 i^2}\mu(1-\eta)$ for each $S\in\Scal$ and $i\in S$
			\item $c_{a^*}=1$
			\item $c_{a_0}=0$
		\end{itemize}
		
		\item Outcome distribution
		\begin{itemize}
		\item For each set $S\in \Scal$ and $i\in S$, $F_{a_{i,S},\omega_S}=\frac{1}{10i}\mu$,  $F_{a_{i,S}, \omega^*}=\frac{1}{5i}\mu$, and $F_{a_{i,S},\bar \omega}=1-\frac{1}{10i}\mu-\frac{1}{5i}\mu$.
		\item \ma{add that $F_{a_{i,S},\omega_{S'}}=0$ for $S\neq S'$?} \mat{dato per scontato ma si puo aggiungere}
		\item For each set $S\in\Scal$ and each $i\in S$, $F_{\bar a_{i,S},\omega_S}=\frac{1}{10i}(1-\frac\eta2)\mu$, $F_{\bar a_{i,S},\bar \omega}=1-\frac{1}{10i}(1-\frac\eta2)\mu$, and $c_{\bar a_{i,S}}=\frac{1}{20i^2} (1-\eta)\mu$.
		\item $F_{a^*,\omega_S}=\epsilon$ for all $S\in\Scal$ and $F_{a^*,\omega^*}=1-m\epsilon$.
		\item $F_{a_0, \bar\omega}=1$
		\end{itemize}
		
		\item Types \mat{ricordare di normalizzare i tipi se vogliamo.}
		\begin{itemize}
			\item The agent's types include a type $\theta=i$ for each $i\in E$, and an additional type $\theta=0$. 
		\end{itemize}
		
		\item Type distribution
		\begin{itemize}
			\item The distribution is defined as follows: $\mu_{i}=\frac{1-\rho}{n}$ for each $i \in E$ and $\mu_0=\rho$.
		\end{itemize}

\mat{in generale credo si possa sostituire 5 10 20 con 1 2 4}
		
%		\item The set of actions available to the agent are the following. \ma{$\mu$ non è definito}
%		For each set $S\in \Scal$ and $i\in S$, we define an action $a_{i,S}$ with $F_{a_{i,S},\omega_S}=\frac{1}{10i}\mu$,  $F_{a_{i,S}, \omega^*}=\frac{1}{5i}\mu$, and $F_{a_{i,S},\bar \omega}=1-\frac{1}{10i}\mu-\frac{1}{5i}\mu$.
%		%
%		
%		\item For each set $S\in\Scal$ and $i\in S$, the cost of action $a_{i,S}$ is $c_{a_{i,S}}=\frac{1}{20i^2}\mu$.
%		
%		\item Moreover, for each set $S\in\Scal$ and each $i\in S$, an action $\bar a_{i,S}$ with $F_{\bar a_{i,S},\omega_S}=\frac{1}{10i}(1-\frac\eta2)\mu$, $F_{\bar a_{i,S},\bar \omega}=1-\frac{1}{10i}(1-\frac\eta2)\mu$, and $c_{\bar a_{i,S}}=\frac{1}{20i^2} (1-\eta)\mu$.
%		
%		\item An action $a^*$ with cost $c_{a^*}=1$, $F_{a^*,\omega_S}=\epsilon$ for all $S\in \mathcal{S}$ and $F_{a^*,\omega^*}=1-m\epsilon$.
%		\item The agent's types include a type $\theta=i$ for each $i\in E$, and an additional type $\theta=0$. 	\mat{ricordare di normalizzare i tipi se vogliamo.}
%		\item The distribution is defined as follows: $\mu_{i}=\frac{1-\rho}{n}$ for each $i \in E$ and $\mu_0=\rho$.
	\end{itemize}
	
	
	
	
	
	\textbf{if.}
	Suppose that there exists a set cover $\Scal^*$ of size less then $k$.
	We build a contract with utility at least 
	\[\gamma=	 \frac{1-\rho}{n} \ell \sum_{i} \frac{1}{10i} + \rho (1-m\epsilon) - \rho \epsilon |\Scal^*|  \]
	as follow.
	For each set $S\in \Scal^*$, set
	$p_{\omega_S}=1$.
	Moreover, let $p_{\omega^*}=p_{\bar \omega}=0$ and $p_{\omega_S}=0$ for all $S\notin \Scal^*$.
	
	Then, we compute the best responses for each possible agent's type.
	
	For an agent of type $i\ge 1$, the utility of playing an action $a_{j, S}$ (for $j\in S$ and $S\in\Scal$) is
	
	\begin{align}
		\sum_{\omega \in \Omega} F_{ a_{j, S}, \omega } p_\omega - i \cdot c_{a_{j, S}} &= \sum_{S'\in\Scal^*} F_{ a_{j, S}, {\omega_{S'}} } p_{\omega_{S'}} - i \cdot c_{a_{j, S}}\\
		&=\sum_{S'\in\Scal^*} F_{ a_{j, S}, {\omega_{S'}} }- i \cdot c_{a_{j, S}}\\
		&=\sum_{S'\in\Scal^*} F_{ a_{j, S}, {\omega_{S'}} }-i \frac{1}{20 j^2}\mu
	\end{align}
	
	If $S\notin \Scal^*$ then the utility of playing $a_{j, S}$ is negative (since $F_{a_{j,S},\omega_{S'}}\neq 0$ if and only if $S=S'$), thus we can assume that $S\in \Scal^*$. Then, continuing from the above, the following holds
	
	\begin{align}
		\sum_{\omega \in \Omega} F_{ a_{j, S}, \omega } p_\omega - i \cdot c_{a_{j, S}} &=\sum_{S'\in\Scal^*} \mathbb{I}(S'=S)\frac{1}{10j}\mu- i\frac{1}{20 j^2}\mu\\
		&=\frac{1}{10j}\mu(1- \frac{i}{2j})
	\end{align}
	
	Now we state that $a_{j, S} = a_{i, S}$ (since $\Scal^*$ is a set cover we know that, for each type $i$, there is a set $S\in \Scal^*$ such that $i\in S$) is the action that maximizes $\frac{1}{10j}\mu(1- \frac{i}{2j})$. This is easily proven by observing that the function $j\mapsto \frac{1}{j}(1-\frac{i}{2j})$ is increasing for $j\le i$ and decreasing for $j\ge i$, thus the maximum incurs at $j=i$ and has utility $\frac{1}{20i}\mu$.
	
	When considering an agent of type $i$, and an action $\bar a_{j, S}$ (for $j\in S$ and $S\in\Scal$), similarly to the case above, we can consider only the case in which $S\in\Scal^*$. Thus we have that the utility of an agent of type $i$, that plays  an action $\bar a_{j, S}$ with $S\in\Scal^*$ is:
	\begin{align}
		\sum_{\omega \in \Omega} F_{ \bar a_{j, S}, \omega } p_\omega - i \cdot c_{\bar a_{j, S}} &= \sum_{S'\in\Scal^*} F_{ \bar a_{j, S}, {\omega_{S'}} }  - i \cdot c_{\bar a_{j, S}}\\
		&=\sum_{S'\in\Scal^*} F_{ \bar a_{j, S}, {\omega_{S'}} }  -i\frac{1}{20j^2}\mu(1-\eta) \\
		&=\frac{1}{10j}(1-\frac\eta2)\mu-i\frac{1}{20j^2}\mu(1-\eta) \\
		&=\frac{1}{10j}\mu\left(1-\frac{\eta}{2}-\frac{i}{2j}(1-\eta)\right).
 	\end{align}

	The we can observe that $j\mapsto \frac{1}{j}\left(1-\frac{\eta}{2}-\frac{i}{2j}(1-\eta)\right)$ is decreasing for $j\le 2i\frac{1-\rho}{2-\rho}$ and increasing otherwise. Since $2i\frac{1-\rho}{2-\rho}\in[i-1, i]$ we only need to compare the utility of $\bar a_{i, S}$ and $\bar a_{i-1, S}$.
	
	Simple calculations shows that the value of action $\bar a_{i-1, S}$ is $\frac{1}{10(i-1)^2}\mu\left[\frac{\eta}{2}(i+1)-1\right]$, while the value of action $\bar a_{i, S}$ is $\frac{1}{20i}\mu$. Straightforward calculations shows that if $\eta\in[0,2(\sqrt{2}-1))$ then $\frac{1}{10(i-1)^2}\mu\left[\frac{\eta}{2}(i+1)-1\right]<\frac{1}{20i}\mu$ and thus $\bar a_{i, S}$ is optimal among all actions $\{\bar a_{i, S}\}_{i, S}$ and gives utility of $\frac{1}{20i}\mu$. However, break ties in favor of the principal, the action $a_{i, S}$ is preferred over action $\bar a_{i, S}$.
	
	
	Indeed the value for the principal of an agent of type $i\ge 0$ playing $a_{i,S}$ (with $S\in \Scal^*$ and $i\in S$) is
	\begin{align}
		\sum_{\omega} F_{a_{i,S},\omega}(r_\omega-p_\omega)&=F_{a_{i,S},\omega^*}r_{\omega^*}-F_{a_{i,S},\omega_S}p_{\omega_S}\\
		&=\frac{1}{5i}\mu-\frac{1}{10i}\mu\\
		&=\frac{1}{5i}\mu
	\end{align}
	while the principal's value when playing $\bar a_{i, S}$ is negative, as the only outcome reachable from playing $\bar a_{i, S}$ are $\omega_S$ and $\bar \omega$ which both have reward of zero.
	
	Now consider that an agent of type $i\ge 1$ plays the action $a^*$. The perceived utility will be
	\begin{align}
		\sum_{\omega\in \Omega} F_{a^*,\omega}p_\omega - i\cdot c_{a^*} & = \sum_{S\in \Scal} F_{a^*,\omega_S} - i\\
		&=\epsilon m - i\\
		&\le 0
	\end{align}
	and thus for each type $i$, the action played would be any action $a_{i,S}$ with $S\in\Scal^*$ and $i\in S$.
	
	Now we have to consider the action played by a type $i=0$. Clearly, an agent of this type wont pay any cost for playing actions. The utility of playing $a^*$ would be at least $m\epsilon$ which is greater then $\mu$. On the other hand the value of playing action $a_{i, S}$ and $\bar a_{i,S}$ (with $i\in S$) would be $\frac{1}{10i}\mu$ and $\frac{1}{10i}\mu(1-\eta/2)$ which are both strictly less then $\mu$. Thus an agent of type $i=0$ will be playing action $a^*$.
	
	Now that we computed the action played by each type we can compute the principal' utility of such contract.
	\begin{align}
		&\sum_{i\ge 1} \mu_i \sum_{\omega\in\Omega}F_{a_{i, S},\omega}(r_\omega-p_\omega) + \mu_0 \sum_{\omega\in\Omega }F_{a^*,\omega}(r_\omega - p_\omega)\\
		&=\sum_{i\ge 1} \frac{1-\rho}{n} \left[F_{a_{i,S},\omega_S}(r_{\omega_S}-p_{\omega_s})+F_{a_{i,S},\omega^*}(r_{\omega^*}-p_{\omega^*})\right]
		 \\
		 &+ \rho(F_{a^*,\omega^*}(r_{\omega^*}-p_{\omega^*})+\sum_{S\in\Scal}F_{a^*,\omega_S}(r_{\omega_S}-p_{\omega_S}))\\
		 &=\frac{1-\rho}{n}\sum_{i\ge 1}\left[\frac{1}{10i}\mu(0-1)+\frac{1}{5i}\mu(1-0)\right]+\rho((1-m\epsilon)(1-0)+|\Scal^*|\epsilon(0-1))\\
		&=\frac{1-\rho}{n} \sum_{i\ge 1} \frac{1}{10i}+\rho(1-m\epsilon-\epsilon|\Scal^*|)=\gamma\\
		&=\frac{1-\rho}{n} \sum_{i\ge 1}\frac{1}{10i}+\rho(1-m\epsilon-\epsilon k):=\gamma
	\end{align}
	
%	\ma{OLD}
%	Consider a type $\theta=i$, $i \in S$.
%	Since $\Scal$ is a set cover, there exists a set $S \in \Scal^*$ such that $i \in S$.
%	We show that $a_{i,S}$ is a best response for the agent of type $i$.
%	The agent's utility playing action $a_{i,S}$ is 
%	\[ \frac{1}{10i}\ell p_{\omega_{S}} - \frac{1}{20i}\ell=\frac{1}{20i}\ell. \]
%	
%	We show that the utility playing any other action is not higher.
%	The agent utility playing an action $a_{j,S'}$, $S'\in \Scal$, $j \in S'$  is at most 
%	\[ \frac{1}{10j}\ell p_{\omega_{S'}} -\frac{i}{20j^2}\ell\le \frac{1}{10j}\ell -\frac{i}{20j^2}\ell\le\frac{1}{20i}\ell. \]
%	\mat{trovare modo  elegante per dimostrarlo. foglio martino?}
%	
%	Playing an action ${\bar a_{j,S'}}$, $S'\in \Scal$, $j \in S'$ , the agent's utility is at most
%	\[ \frac{1}{10j} (1-\eta/2)\ell p_{\omega_{S'}}-\frac{i}{20j^2} (1-\eta) \ell\le  \frac{1}{10j} (1-\eta/2)\ell -\frac{i}{20j^2} (1-\eta) \ell \]
%	The utility is maximized for $j=i$ getting utility $1/i$.
%	Indeed, the maximum is at
%	\[ j= i \frac{1-\eta}{1-\eta/2}  \]
%	and the agent utility is 
%	\[\frac{1}{10i} (1-\eta/2)\ell-\frac{1}{20i} (1-\eta) \ell=\frac{1}{20i}\ell \]	
%	\mat{dimostrare ma anche qui ho pochi dubbi.}
%	
%	Finally, it is easy to see that the agent's utility playing action $a^*$ is negative.
%	Hence, the best response of type $\theta=i$ is $a_{i,S}$.\footnote{dire che questa è quella che rompe i pareggi in favore del principal?}
%	
%	Finally, consider the type $\theta=0$.
%	Taking action $a^*$, the agent's utility is
%	\[|\Scal^*|\epsilon\ge \epsilon > \ell, \]
%	while the utility taking any other action is at most $\ell$.
%	
%	Hence, the expected principal's utility is
%	\[  \frac{1-\rho}{n} \ell \sum_{i} [\frac{1}{5i}- \frac{1}{10i}] + \rho (1-m\epsilon - \epsilon |\Scal^*|)= \frac{1-\rho}{n} \ell \sum_{i} \frac{1}{10i} + \rho (1-m\epsilon - \epsilon |\Scal^*|) =\gamma . \]
%	
	\textbf{only if.}
	Suppose that all set covers have size at least $k+1$.
	%
	We show that any contract has utility at most $\gamma-n^{14}m^{-1}$.
	
	Let $p\in\Reals^{|\Omega|}_{\ge 0}$ be a contract.
	% First, we show that $p_{\omega^*}<\gamma$.
	%Otherwise, we get that the expected utility is at most
	%\begin{align*}
	%	\rho n + (1-n\rho) (1-|E|\epsilon) - (1-n\rho) (1-|E|\epsilon) p_{\omega^{\star}} \ge \rho \sum_{i} (\frac{1}{2}+\frac{1}{10i}) + (1-n\rho) (1-|E|\epsilon) - (1-n\rho)\epsilon |E^*| -\nu		
	%\end{align*}
	%implying
	%\begin{align}
	%	p_{\omega^{\star}} \le \frac{1}{(1-n\rho) (1-|E|\epsilon)} [\rho n + \epsilon |E| + \nu   ]\le 2 [n^{-6}+n^{-8}+n^{-6}] \le 6 n^{-6} 
	%\end{align}
	We partition the set of elements $E$ in three sets.
	$E_1\subseteq E$ includes the elements $i$ such that the best response of type $i$ to contract $p$ is $a_{i,S}$, for some $S \in \Scal$.
	%
	$E_2\subseteq E$ includes  the elements $i$ such that the best response of type $i$ is $a_{j,S}$ with $j\neq i$ and $S \in \Scal$.
	%
	Finally, $E_3=E \setminus (E_1\cup E_2)$.
	
	We start analyzing the constraint that the best response of types $\theta \in E_1$ impose on the contract.
	Consider a type $\theta=i \in E_1$, and let $a_{i,S}$ be their best response. Notice that $i\in S$.
	%
	Then, from the IC constraint with respect to action $\bar a_{i,S}$ it holds
	
	\[
	p_{\omega_{S}}\frac{1}{10i}\mu + p_{\omega^*}  \frac{1}{5i}\mu+(1-\frac{3}{10i}\mu)p_{\bar\omega}- \frac{1}{20i}\mu \ge  p_{\omega_{S}}\frac{1-\frac{\eta}{2}}{10i} \mu +(1-\frac{1-\frac{\eta}{2}}{10i}\mu)p_{\bar\omega}- \frac{1}{20i} (1-\eta) \mu 
	\]
	implying that
	\[
	p_{\omega_{S}}\ge 1-\frac{4}{\eta}p_{\omega^*}.
	\]
	From this follows that the principal utility against an agent of type $i\in E_1$ is at most
	% \ma{giusto buttare via $\bar \omega$? farà schifo come outcome eh, però magari bisogna dismostrare che $p_{\bar\omega}=0$}
	\begin{align}
	&F_{a_{i,S},\omega_S}(r_{\omega_S}-p_{\omega_S})+F_{a_{i,S},\omega^*}(r_{\omega^*}-p_{\omega^*})+F_{a_{i,S},\bar \omega}(r_{\bar \omega}-p_{\bar\omega})\\
	&\le-\frac{1}{10i}\mu(1-4p_{\omega^*}/\eta)+\frac{1}{5i}\mu(1-p_{\omega^*})\\
	&=\frac{1}{10i}\mu+\frac{1}{5i}p_{\omega^*}\mu(\frac{2}{\eta}-1)
	\end{align}
	
	Now, consider an agent's type $\theta=i\in E_2$, and let $a_{j,S}$, $j \neq i$ be the action played by the agent. 
	Notice that $j \in S$. Then, IC constraint with respect to action $\bar a_{j,S}$ reads
	\[
	p_{\omega_{S}} \frac{1}{10j} \mu + \frac{1}{5j}p_{\omega^*}\mu+(1-\frac{3}{10j}\mu)p_{\bar\omega}- \frac{i}{20j^2}\mu \ge p_{\omega_{S}} \frac{1}{10j} (1-\eta/2) \mu+(1-\frac{1-\frac{\eta}{2}}{10j}\mu)p_{\bar\omega}- \frac{i}{20j^2} (1-\eta) \mu
	\]
	which implies that
	%\[ p_{\omega_{S}} \frac{\eta}{20j} \ge  \frac{i\eta}{20j^2} - \frac{1}{5j} p_{\omega^*} \]
	%and
	\[
	p_{\omega_{S}} \ge \frac{i}{j}- p_{\omega^*} \frac{4}{\eta}. 
	\]
	
	Hence, the principal's expected utility when the realized type is a $\theta=i \in E_2$ is at most:
	\begin{align}
		&F_{a_{j,S},\omega_S}(r_{\omega_S}-p_{\omega_S})+F_{a_{j,S},\omega^*}(r_{\omega^*}-p_{\omega^*})+F_{a_{j,S},\bar \omega}(r_{\bar \omega}-p_{\bar\omega})\\
		&\le-\frac{1}{10j}\mu\left( \frac{i}{j}- p_{\omega^*} \frac{4}{\eta}\right)+\frac{1}{5j}\mu\\
		&=\mu\left(\frac{1}{5j}-\frac{i}{10j^2}+\frac{2}{5j\eta}p_{\omega^*}\right)\\
		&\le\mu\left(\frac{1}{5j}-\frac{i}{10j^2}+\frac{2}{5\eta}p_{\omega^*}\right)
	\end{align}
	
	Now consider the first term $\frac{1}{5j}-\frac{i}{10j^2}$, this is maximized for $j=i-1$ (recall that we cannot have that $j=i$ as $i\in E_2$). To obtain an upper bound on this quantity observe that
	\begin{align}
		\frac{1}{5(i-1)}-\frac{i}{10(i-1)^2}-\frac{1}{10i}&=-\frac{1}{10(i-1)^2i}\\
		&\le -\frac{1}{10(n-1)^2n}\tag{$x\mapsto -\frac{1}{(x-1)^2x}$ is increasing for $x\ge 1$}\\
		&\le-\frac{1}{10n^3}
	\end{align}

	
	
	Hence, the principal's expected utility when the realized type is a $\theta=i \in E_2$ is at most:
	\[
		\mu\left(\frac{1}{10i}-\frac{1}{10n^3}+\frac{2}{5\eta}p_{\omega^*}\right).
	\]
	
	Now consider any type $i\in E_3$. These agents play an action $\bar a_{j, S}$ for some $S$ and $j\in S$ or actions $a^*$ or $a_0$.
	Clearly when actions $\bar a_{j, S}$ and $a_0$ are played the principal utility is at most zero, as the only outcomes reachable from these actions are $\omega_S$ and $\bar\omega$ which have reward $0$. The principal utility is at most zero also for action $a^*$. Indeed consider the IC constraints of action $a^*$ with respect to $a_0$:
	\[
	\sum_{S\in\Scal}F_{a^*,\omega_S}p_{\omega_S}-i\cdot c_{a^*} + F_{a^*,\omega^*} p_{\omega^*}\ge 0
	\]
	which implies that $\epsilon\sum_{S\in\Scal}p_{\omega_S}+(1-m\epsilon)p_{\omega^*}\ge i$. The principal utility, when action $a^*$ is played, is 
	\begin{align}
		\sum_{S\in\Scal} F_{a^*,\omega_S}(r_{\omega_S}-p_{\omega_S})+F_{a^*,\omega^*}(r_{\omega^*}-p_{\omega^*})=&-\epsilon\sum_{S\in\Scal}p_{\omega_S}+(1-m\epsilon)(1-p_{\omega^*})\\
		&\le i-m\epsilon.
	\end{align}
	and thus the principal utility is negative also for action $a^*$.
	
	%\ma{aggiongiere $a_0$ con costo zero che manda determ in $\bar \omega$}. \ma{dici che tutte queste azoni danno utilità negativa ma $a^*$ mi fa fare un sacco di soldi invece... Bisogna far vedere che non la giocano. O meglio che se la giocassero io sto spendendo troppo per fargliela giocare visto che pagano 1 per giocarla. O magari mi sono rincoglionito...}
	
	
	\ma{tbc: } Finally, type $\theta=0$ plays action $a^*$ otherwise the utility is clearly below $\gamma-n^{14}m^{-1}$. Moreover when an agent of type $0$ plays the action $a^*$ the utility of the principal is 
	\[
	-\epsilon\sum_{S\in\Scal}p_{\omega_S}+(1-m\epsilon)(1-p_{\omega^*})
	\]
	
	Define the following set $\bar \Scal:=\{S\in\Scal: p_{\omega_S}\ge1-\frac{4}{\eta}p_{\omega^*}\}$. Combining the upper bounds above on the principal utility we can say that the principal utility is at most
	\[
	\frac{1-\rho}{n}\mu\left[\sum_{i\in E_1}\right(\frac{1}{10i}+\frac{1}{5i}p_{\omega^*}(\frac{2}{\eta}-1)\left)+\sum_{i\in E_2}\left(\frac{1}{10i}-\frac{1}{10n^3}+\frac{2}{5\eta}p_{\omega^*}\right)\right]+\rho(1-m\epsilon)(1-p_{\omega^*})-\epsilon\rho\sum_{S\in\Scal}p_{\omega_S}.
	\]
	Moreover, we can notice that $\sum_{S\in\Scal} p_{\omega_S}\ge |\bar\Scal|(1-\frac{4}{\eta}p_{\omega^*})$, and thus the principal utility is at most an expression in which the coefficient of $p_\omega^*$ is 
	\[
	\frac{1-\rho}{n}\mu\left[\sum_{i\in E_1}\frac{1}{5i}(\frac{2}{\eta}-1)+\sum_{i\in E_2}\frac{2}{5\eta}\right]-\rho(1-m\epsilon)+\frac{4}{\eta}\epsilon\rho|\bar\Scal|.
	\]
	The following calculations shows that this is at most zero:
	\begin{align}
			\frac{1-\rho}{n}\mu&\left[\sum_{i\in E_1}\frac{1}{5i}(\frac{2}{\eta}-1)+\sum_{i\in E_2}\frac{2}{5\eta}\right]-\rho(1-m\epsilon)+\frac{4}{\eta}\epsilon\rho|\bar\Scal|\\
			&\le \frac{1-\rho}{n}\mu\sum_{i\in E_1\cup E_2}\frac{2}{5\eta}-\rho(1-m\epsilon)+4{\rho \epsilon m}{\eta}\\
			&\le \frac{1-\rho}{n}\mu\frac{2}{5\eta}n-\rho(1-m\epsilon)+4\frac{\rho \epsilon m}{\eta}\\
			&=\frac{2}{5\eta}\mu-\rho(1-m\epsilon)+4\frac{\rho \epsilon m}{\eta}\\
			&\le \frac{2}{5}\frac{\mu}{\eta}-\frac{\rho}{2}+4\frac{\rho \epsilon m}{\eta}\\
			&=\frac{2}{5}\frac{n^2}{n^9m}-\frac{1}{2n^6}+4\frac{mn^2}{n^6n^8m}\\
			&=\frac{2}{5}\frac{1}{n^7}-\frac{1}{2n^6}+4\frac{1}{n^{12}}
	\end{align}
	
	which is negative for $n\ge 2$.
	
	Thus any contract with $p_{\omega^*}=0$ achieves larger utility then any contract with $p_{\omega^*}>0$. Thus the principal utility is upper bounded by
	\[
	\frac{1-\rho}{n}\mu\left[\sum_{i\in E_1}\frac{1}{10i}+\sum_{i\in E_2}\left(\frac{1}{10i}-\frac{1}{10n^3}\right)\right]+\rho(1-m\epsilon)-\epsilon\rho|\bar\Scal|.
	\]
	Now observe that 
	\begin{align}
		\frac{1-\rho}{n}\frac{1}{10n^3}&\ge \frac{\mu}{20n^4}\\
		&=\frac{1}{20n^13m}\\
		&=\frac{1}{n^14m}\tag{$n\ge O(1)$}\\
		&=\rho\epsilon
	\end{align}
	
	and thus we can find that the principal utility is upper bounded by
	\[
	\frac{1-\rho}{n}\mu\sum_{i\in E}\frac{1}{10i}+\rho(1-m\epsilon)-\epsilon\rho(|\bar\Scal|+|E\setminus E_1|).
	\]
	Now we complete the set $\bar S$ to make it a set cover of $E$ in the following way. For each element $v$ of $E\setminus E_1$ we pick a set $s(v)\in\Scal$ such that $v\in s(v)$ and we claim that $\Scal^*:=\bar\Scal\cup( \cup_{v\in E\setminus E_1}s(v))$ is a cover of $E$, since by construction the set $\bar S$ is a cover of $E_1$ and $\cup_{v\in E\setminus E_1}s(v)$ is a cover of $E\setminus E_1$. Thus $\Scal^*$ is a cover of $E$ and thus has size at least $k+1$, also note that $|\cup_{v\in E\setminus E_1}s(v)|\le |E\setminus E_1|$. The proof is concluded by observing that $|\bar \Scal|+|E\setminus E_1|\ge |\Scal^*|\ge k+1$ and thus the principal utility is upper bounded by
	\[
	\frac{1-\rho}{n}\mu\sum_{i\in E}\frac{1}{10i}+\rho(1-m\epsilon)-\epsilon\rho(k+1)=\gamma-\epsilon\rho=\gamma-\frac{1}{n^{14}m}.
	\]
	
%	\ma{==================OLD==================}
%	\[ 
%	\frac{1}{5j} - \frac{i}{10j^2} +  \frac{1}{10j} p_{\omega^*} \frac{4}{\eta}
%	\le \frac{1}{10i}-\frac{1}{40n^3}  + \frac{2}{5j\eta}p_{\omega^*}  
%	\]
%	\mat{da dimostrare, o checkato su wolfram}
%	
%	Finally, consider the set of agent types $\theta=i\in E_3$. These agent's plays or an action $\bar a_{j,S}$, $j\in E$, $S\in \Scal$, or action $a^*$, or \mat{or void action (aggiungere al modello)}. It is easy to see that the principal's utility is at most $0$ for all these possible actions. \ma{qui vedo un problemino}
%	
%	Finally, type $\theta=0$ plays action $a^*$ otherwise the utility is clearly below $\gamma-n^{14}m^{-1}$.
%
%
%
%Let $\bar \Scal$ be the sets $S$ such that $p_{\omega_S}\ge 1- 4 p_{\omega^*}$.
%Then, the principal's utility is at most 
%\[ \frac{1-\rho}{n} \ell \left[\sum_{i \in E_1}( \frac{1}{5i} (1-p_{\omega^{*}}) - \frac{1}{10i}( 1-4p_{\omega^*}/\eta))  + \sum_{i \in E \setminus E_1} (\frac{1}{10i}-\frac{1}{40n^3}  + \frac{2}{5j\eta}p_{\omega^*})\right] + \rho (1-m\epsilon) (1-p_{\omega^*}) - \rho \epsilon |\bar\Scal|(1-4p_{\omega^*}/\eta)\]
%\[\le \frac{1-\rho}{n} \ell \left[\sum_{i \in E_1}( \frac{1}{5i}  - \frac{1}{10i}( 1-4p_{\omega^*}/\eta))  + \sum_{i \in E \setminus E_1} (\frac{1}{10i}-\frac{1}{40n^3}  + \frac{2}{5j\eta}p_{\omega^*})\right] + \rho (1-m\epsilon) (1-p_{\omega^*}) - \rho \epsilon |\bar\Scal|(1-4p_{\omega^*}/\eta)   \]
%
%
%Then, since
%\[\frac{1-\rho}{n} \ell  n \frac{2}{5} \frac{1}{\eta} - \rho (1-m\epsilon) + 4 \rho \epsilon m \frac{1}{\eta} \le  \ell/\eta -\rho/2 + 4 \rho \epsilon m/\eta=n^{-7}m^{-1} - n^{-6}/2 + 4 n^{-12}<0 \]
%for $n$ large enough, it is the case that
%
%\[\arg \max_{p_{\omega^*}}  \frac{1-\rho}{n} \ell [\sum_{i \in V^*}( \frac{1}{5i}  - \frac{1}{10i}( 1-4p_{\omega^*}/\eta))  + \sum_{i \notin V^*} (\frac{1}{10i}-\frac{1}{40n^3}  + \frac{2}{5j\eta}p_{\omega^*})] + \rho (1-m\epsilon) (1-p_{\omega^*}) - \rho \epsilon |\bar\Scal|(1-4p_{\omega^*}/\eta)= 0,\]
%
%Hence, setting $p_{\omega^*}=0$ the principal's utility is at most
%\begin{align}
%	&\frac{1-\rho}{n} \ell \left[\sum_{i \in E_1}( \frac{1}{5i}  - \frac{1}{10i})  + \sum_{i \in E \setminus E_1} (\frac{1}{10i}-\frac{1}{40n^3}) \right] + \rho (1-m\epsilon) - \rho \epsilon |\bar\Scal|\\
%	&=\frac{1-\rho}{n} \ell \left[\sum_{i \in E_1} \frac{1}{10i} + \sum_{i \in E \setminus E_1} (\frac{1}{10i}-\frac{1}{40n^3}) \right] + \rho (1-m\epsilon) - \rho \epsilon |\bar\Scal|\\
%	& \le \frac{1-\rho}{n} \ell \sum_{i \in E} \frac{1}{10i}    + \rho (1-m\epsilon) - \rho \epsilon (|\bar\Scal|+ |E\setminus E_1|) 
%\end{align}
%where the last inequality follows since 
%\[\frac{1-\rho}{n} \ell \frac{1}{40n^3}\ge n^{-13} m^{-1}/80 \ge n^{-14} m^{-1}  = \rho \epsilon. \]
%for $n$ large enough.
%
%Let define the function $s:E\rightarrow\Scal$ that assign to each element $i \in E$ a set $S \in \Scal$ such that $i\in S$.
%Now, notice that there exists a set cover of size $|\bar\Scal|+ |E\setminus E_1|$ taking the set $S=\bar\Scal\cup (\cup_{v \in V\setminus V^*} s(v))$. Indeed, all the elements in $E_1$ are covered by a set in $\Scal$ by definition, while $\cup_{E \in E \setminus E_1} s(v)$ clearly covers all the elements not in $E_1$.
%
%Thus,the principal's utility is at most
%\[ \frac{1-\rho}{n} \ell \sum_{i \in V} \frac{1}{10i}    + \rho (1-m\epsilon) - \rho \epsilon (k+1)\le \gamma - \rho \epsilon= \gamma- n^{-14} m^{-1}, \]
%concluding the proof.
%
%
%
%

\end{proof}