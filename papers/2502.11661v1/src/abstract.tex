\begin{abstract}
% We study principal-agent settings in which the agent performs a hidden action and has a hidden type. We focus on the design of optimal contracts when the agent's type is drawn from a distribution.
% %
% Unlike the multi-parameter setting, where each agent is characterized by multiple features influencing their productivity, we consider the single-parameter setting introduced in contract design by \citet{alon2021contracts}. 
% %
% In this natural setting, the agent's type is described by a single number that is referred to as the cost per unit-of-effort.
% %
% In this simpler model, one might expect the problem to be algorithmically easier.
% %could hope that algorithmically, the problem becomes easier.
% %
% \citet{castiglioni2025reduction} recently showed a reduction from the multi-parameter setting to the single-parameter one which preserves multiplicative approximations. As a result, their reduction rules our efficient algorithms for single-parameter contract design under multiplicative approximations.
% %In particular, their reduction preserves multiplicative approximations, thus ruling out efficient algorithms for single-parameter contract design under multiplicative approximations.
% %
% However, whether single-dimensional costs are easier than multi-dimensional ones under additive approximations remains an open question.

% In this paper, we show that the answer to this question is nuanced. Indeed, we prove that the problem admits a $\PTAS$ for constant additive approximation, while we show a reduction ruling out additive $\FPTAS$, unless $\PolyClass=\NP$.
% %
% %Finally, we show that, under bounded density of the type distribution, the problem can be learned online. We show results for both regret and sample complexity using a reduction from our problems to misspecified linear bandits.
% Furthermore, under bounded density of the type distribution, we show that the problem is learnable in an online setting. Our results, based on a reduction to misspecified linear bandits, provide guarantees on both regret and sample complexity.



We study a Bayesian contract design problem in which a principal interacts with an unknown agent.
%
We consider the single-parameter uncertainty model introduced by \citet{alon2021contracts}, in which the agent's type is described by a single parameter, \emph{i.e.}, the cost per unit-of-effort.
%
Despite its simplicity, several works have shown that single-dimensional contract design is not necessarily easier than its multi-dimensional counterpart in many respects. Perhaps the most surprising result is the reduction by \citet{castiglioni2025reduction} from multi- to single-dimensional contract design. However, their reduction preserves only multiplicative approximations, leaving open the question of whether additive approximations are easier to obtain than multiplicative ones.


%a line of works showed that single-dimensional contract design is not easier than multi-dimensional one in many aspects. Possibly, the most surprising of these results is a reduction by \citet{castiglioni2025reduction} from multi- to single-dimensional contract design.
%However, their reduction only preserves multiplicative approximations. Their reduction left open the question of whether additive approximations are easier to obtain than multiplicative ones.
%In particular, their reduction preserves multiplicative approximations, thus ruling out efficient algorithms for single-parameter contract design under multiplicative approximations.
%
%However, whether single-dimensional contract design is easier than multi-dimensional one under additive approximations remains an open question.
%
%In this paper, we show that \mat{under some aspects} single-dimension contract design is easier than multi-dimensional one:
%

%In this paper, we show that --to some extent-- single-dimension contract design is easier than a multi-dimensional one. More precisely
%In this paper, we answer this question---to some extent---positively. 
In this paper, we give a positive answer---at least to some extent---to this question.
In particular, we provide an additive $\PTAS$ for these problems while also ruling out the existence of an additive $\FPTAS$. This, in turn, implies that no reduction from multi- to single-dimensional contracts can preserve additive approximations. Moreover, we show that single-dimensional contract design is fundamentally easier than its multi-dimensional counterpart from a learning perspective. Under mild assumptions, we show that optimal contracts can be learned efficiently, providing results on both regret and sample complexity.
\end{abstract}


