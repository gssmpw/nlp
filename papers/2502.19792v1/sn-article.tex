\documentclass[sn-mathphys-num]{sn-jnl}% Math and 
\usepackage[utf8]{inputenc}
\usepackage{a4wide}
\usepackage{amssymb}
\usepackage{amsmath}
\usepackage{amsthm}
\usepackage{hyperref}
\usepackage{breqn}
\usepackage{xcolor}
\usepackage{mathtools}
\usepackage[numbers]{natbib}
\usepackage{bm}
\usepackage{enumitem}
\usepackage{mathrsfs}
\usepackage{graphicx}
\usepackage{caption}
\usepackage{url}
\usepackage{placeins}
\usepackage[english]{babel}
\usepackage{array}
\usepackage{booktabs}
\usepackage{subcaption}
\usepackage{multirow}
\numberwithin{equation}{section}
\newcommand{\dm}[1]{{\displaystyle{#1}}}
% \newtheorem{theorem}{\bf Theorem}[section]
%\newtheorem{definition}{Definition}[section]
\newtheorem{corollary}{Corollary}[section]
%\newtheorem{proposition}{Proposition}[section]
\newtheorem{lemma}{Lemma}[section]
%\newtheorem{remark}{Remark}[section]
\theoremstyle{remark}
\newtheorem{exam}{\bf Example}[section]
%%%%%%%%%%%%%%%%%%
\def \vec{\mathrm v\mathrm e \mathrm c}
\def \R{{\mathbb R}}
\def \K{{\mathscr{K}}}
\def \C{{\mathscr{C}}}
\def \Ms{{\mathscr{M}}}
\def \sign{\mathrm{sign}}
\def \x{{\bm x}}
\def \b{{\bf b}}
\def \H{\mathcal H}
\def \T{\mathsf T}
\def\bmatrix#1{\left[\begin{matrix}
		#1
	\end{matrix}\right]}
\def \Lam{\Lambda}
\def \diag{\mathrm{diag}}
\def \M{{\mathcal M}}
\def \SY{{\mathcal S }}
\def \D{{\Delta}}
\def \r{{\bf r}}
\def \be{\bm{\chi}}
\def \L{\mathbf{L}}
\def  \0{\bf 0}
%%%%%%%%%%%%%%%%%%
\newcommand{\inp}[2]{\langle {#1} ,\,{#2} \rangle}
\newcommand{\nrm}[1]{|\!|\!| {#1} |\!|\!|}
\newcommand{\norm}[1]{\|{#1}\|}
\def \rank{\mathrm{rank}}
\def \tr{\mathrm{ trace}}
\def \det{\mathrm{det}}
\def \rev{\mathrm{rev}}
\def \rmn{\R^{m\times n}}
\def \ep{{\varepsilon}}
\def \Up{{\Upsilon}}
\def \x{\bm x}
\def \l{\bm l}
\def \y{\bm y}
\def \z{\bm z}
\def \T{{\bm\top}}
\def \vp{\bm \varphi}
\newcommand {\eproof}{\space{\ \vbox{\hrule\hbox{\vrule height1.3ex\hskip0.8ex\vrule}\hrule}}  \par}


\def\bmatrix#1{\left[ \begin{matrix} #1 \end{matrix} \right]}
\def \noin{\noindent}

%%%%%%%%%%%%%%%%%%%%%%%%%%% new commands %%%%%%%%%
\renewcommand{\thefootnote}{\fnsymbol{footnote}}
\def \R{{\mathbb R}}
\renewcommand\qedsymbol{$\blacksquare$}
%%%%%=============================================================================%%%%
%%%%  Remarks: This template is provided to aid authors with the preparation
%%%%  of original research articles intended for submission to journals published 
%%%%  by Springer Nature. The guidance has been prepared in partnership with 
%%%%  production teams to conform to Springer Nature technical requirements. 
%%%%  Editorial and presentation requirements differ among journal portfolios and 
%%%%  research disciplines. You may find sections in this template are irrelevant 
%%%%  to your work and are empowered to omit any such section if allowed by the 
%%%%  journal you intend to submit to. The submission guidelines and policies 
%%%%  of the journal take precedence. A detailed User Manual is available in the 
%%%%  template package for technical guidance.
%%%%%=============================================================================%%%%

%% as per the requirement new theorem styles can be included as shown below
\theoremstyle{thmstyleone}%
\newtheorem{theorem}{Theorem}%  meant for continuous numbers
%%\newtheorem{theorem}{Theorem}[section]% meant for sectionwise numbers
%% optional argument [theorem] produces theorem numbering sequence instead of independent numbers for Proposition
\newtheorem{proposition}[theorem]{Proposition}% 
%%\newtheorem{proposition}{Proposition}% to get separate numbers for theorem and proposition etc.

\theoremstyle{thmstyletwo}%
\newtheorem{example}{Example}%
\newtheorem{remark}{Remark}%

\theoremstyle{thmstylethree}%
\newtheorem{definition}{Definition}%

\raggedbottom
%%\unnumbered% uncomment this for unnumbered level heads

\begin{document}

\title[Partial Condition Numbers for Double Saddle Point Problems]{Partial Condition Numbers for Double Saddle Point Problems}

%%=============================================================%%
%% GivenName	-> \fnm{Joergen W.}
%% Particle	-> \spfx{van der} -> surname prefix
%% FamilyName	-> \sur{Ploeg}
%% Suffix	-> \sfx{IV}
%% \author*[1,2]{\fnm{Joergen W.} \spfx{van der} \sur{Ploeg} 
%%  \sfx{IV}}\email{iauthor@gmail.com}
%%=============================================================%%

\author*[1]{\fnm{Sk. Safique} \sur{Ahmad}}\email{safique@iiti.ac.in}

\author[1]{\fnm{Pinki} \sur{Khatun}}\email{pinki996.pk@gmail.com}
%\equalcont{These authors contributed equally to this work.}

% \author[1,2]{\fnm{Third} \sur{Author}}\email{iiiauthor@gmail.com}
% \equalcont{These authors contributed equally to this work.}

\affil*[1]{\orgdiv{Department of Mathematics}, \orgname{Indian Institute of Technology Indore}, \orgaddress{\street{Simrol}, \city{Indore}, \postcode{453552}, \state{Madhya Pradesh}, \country{India}}}



%%%%%%%%%%%%%%%%%%%%%%%%%%%%%%%%%%%%%%%%%
\abstract{This paper presents a unified framework for investigating the partial condition number (CN) of the solution of double saddle point problems (DSPPs) and provides closed-form expressions for it. This unified framework encompasses the well-known partial normwise CN (NCN), partial mixed CN (MCN) and partial componentwise CN (CCN) as special cases. Furthermore, we derive sharp upper bounds for the partial NCN, MCN and CCN, which are computationally efficient and free of expensive Kronecker products. By applying perturbations that preserve the structure of the block matrices of the DSPPs, we analyze the structured partial NCN, MCN and CCN  when the block matrices exhibit linear structures. By leveraging the relationship between DSPP and equality constrained indefinite least squares (EILS) problems, we recover the partial CNs for the EILS problem. Numerical results confirm the sharpness of the derived upper bounds and demonstrate their effectiveness in estimating the partial CNs. }

%%================================%%
%% Sample for structured abstract %%
%%================================%%



\keywords{Partial condition number, Perturbation analysis, Double	saddle point problems, Equality constrained indefinite least squares problems, Structured perturbations}

%%\pacs[JEL Classification]{D8, H51}

\pacs[MSC Classification]{15A12, 65F20, 65F35,  65F99}

\maketitle

\section{Introduction}
We consider the following linear system  with the double saddle point structure:
\begin{align}\label{eq1:SPP}
   \mathfrak{B}\bm{w}=\bf{b},~ 
\end{align}
 where  %$\mathfrak{B}\in \R^{l\times l}$ is given by 
 \begin{eqnarray}
     \mathfrak{B}=\bmatrix{A & B^{\bm\top} & \bf 0\\ B &-D & C^{\bm\top} \\ \bf 0 & C& E}; ~ \bm{w}=\bmatrix{\bm{x}\\ \bm{y} \\ \bm{z}};~ \bf{b}=\bmatrix{\mathbf{b}_1\\\mathbf{b}_2 \\ \mathbf{b}_3};
 \end{eqnarray}
 $A\in \R^{n\times n},$ $B\in \R^{m\times n},$ $C\in \R^{p\times m},$ $D\in \R^{m\times m},$ $E\in \R^{p\times p},$ $\mathbf{b}\in \R^{\l\times \l}$ and $\l=n+m+p.$ Then the linear system of the form \eqref{eq1:SPP} is known as the double saddle point problem (DSPP). We refer  $\mathfrak{B}$ as the double saddle point matrix. Conditions on the invetability for the matrix $\mathfrak{B}$ have been studied in \cite{beik2024, Greif2023}. 
 To ensure a unique solution to \eqref{eq1:SPP}, throughout the paper, we assume that $\mathfrak{B}$ is nonsingular.

Linear systems of the form \eqref{eq1:SPP} emerge in numerous computational science and technological applications, such as in finite element discretizations of coupled Stokes-Darcy flow equations \cite{stokesdarcy2019}, the equality constrained indefinite least square problem (EILS)  \cite{ILSE2003}, computational fluid dynamics \cite{CFD2005},  liquid crystal director modeling \cite{LCDM2014}, PDE-constraint optimization problems \cite{PDE-constrained2010}, and so on. Due to the versatile applicability of the DSPP \eqref{eq1:SPP}, their numerical solutions have garnered significant interest and have been extensively investigated in recent years. Several iterative methods and preconditioning techniques have been developed in recent times for solving DSPPs; see \cite{ Greif2023, HuangNA, Multiparameter2023, Pinki_PESS} and reference therein.

Perturbation analysis is one of the most useful tool in numerical analysis for assessing the sensitivity and stability of the solution of a problem obtained through numerical methods \cite{Higham2002}. The condition number (CN) quantifies the worst-case sensitivity of a solution to a small perturbation in the input data matrices. Additionally, CNs are vital in providing first-order error estimates for the relative error of a numerical solution.  The general framework of CNs was first introduced by \citet{Rice1966}. It primarily focuses on the normwise CN (NCN), where norms are used to quantify both the input perturbations and the resulting errors in the output data. However, a significant limitation of NCN is its inability to take into account the inherent structure of poorly scaled or sparse input data. To overcome this limitation, the mixed CN (MCN) and componentwise CN (CCN) have gained increasing attention in the literature \cite{Gohberg1993, rohn1989new, skeel1979scaling}. The former quantifies input perturbations componentwise and output errors using norms, while the later measures both input perturbations and output errors componentwise.

Perturbation analysis and CNs for standard saddle point problems (SPPs) have been extensively studied in the literature; see \cite{Xiangcnd2007, weiwei2009, meng2019condition, PinkiGSPP}. 
Although, these studies fail to leverage the three-by-three block structure of the coefficient matrix $\mathfrak{B}$. 
%
Moreover, in many applications, the solution components \(\x\), \(\y\) and \(\z\) of \(\bm{w}\) represent different physical entities. For instance, in PDE-constrained optimization problems \cite{PDE-constrained2010}, \(\x\) represents the desired state, \(\z\) corresponds to the control variable, and \(\y\) denotes the Lagrange multipliers. However,  the sensitivity of each solution component can vary; hence, analyzing the sensitivity of the individual solution components $\x,$ $\y$ and $\z$ or each solution component of $\bm{w}$ becomes crucial \cite{Cao2003}. The traditional CNs lack the ability to reveal the conditioning of a specific part of the solution.  To tackle this situation, CN of a linear function of the solution has been investigated in the literature, by introducing a matrix $\mathrm{L}$. This matrix enables us to select any particular solution component. The CN of a linear function of the solution is referred to as the partial CN (or the projected CN). For the first time, this class of CN was investigated in \cite{Cao2003} for the system of linear equations and later extensively studied for various problems in recent years, for instance, in linear least squares problems \cite{LLS2007}, weighted least squares problems \cite{WLS2017}, the indefinite least squares problems \cite{ILS2018, ILSE2022}, total least squares problems \cite{TLS2011}, generalized SPPs \cite{PinkiGSPP}.
%

In recent years, the structured CNs of various problems have been studied, emphasizing the preservation of the linear structure of the original matrices in the perturbation matrices; see, for example \cite{Rump2003, RumpII, cucker2007mixed2, PinkiGSPP}. %such as linear systems \cite{Rump2003, RumpII}, linear least square problems \cite{cucker2007mixed2}, generalized SPPs \cite{PinkiGSPP}.
 The block matrices \(A\), \(D\) and \(E\) often exhibit particular linear structures in various applications; see \cite{PDE-constrained2010, LCDM2014}. This makes it compelling to explore the structured partial CNs for the DSPP \eqref{eq1:SPP} by preserving the linear structures of the diagonal block matrices to their corresponding perturbation matrices.

%partial condition number
In this paper, we consider the CN of the linear function $\mathrm{L}[\x^{\bm\top},\y^{\bm\top},\z^{\bm\top}]^{\bm\top}$ of the solution $\bm{w}=[\x^{\bm\top},\y^{\bm\top},\z^{\bm\top}]^{\bm\top},$ where $\mathrm{L}\in \R^{k\times {\bm l}}$ $(k\leq {\bm l})$ or the partial CN of the solution $\bm{w}=[\x^{\bm\top},\y^{\bm\top},\z^{\bm\top}]^{\bm\top}$ of the DSPP \eqref{eq1:SPP}. Different choices of \( \mathrm{L} \) provide the flexibility to determine the CNs of various solution components of \( \bm{w} \). For example, by selecting \(\mathrm{L} = I_{\bm{l}}\), \([I_n~~ \mathbf{0}_{n \times (\bm{l}-n)}]\), or $[{\bf 0}~~I_m~~ \mathbf{0}_{m \times p}]$, we can determine the CN of \(\bm{w}\), \(\x\), or  \(\bm{y}\), respectively. Furthermore, our investigation presents a general framework that encompasses well-known CNs, such as NCN, MCN and CCN, as special cases.

The key contributions of the paper are highlighted as follows:
\begin{itemize}
    \item In this work, we explore a general form of partial CN, refer as unified partial CN, which has a versatile nature and provides a comprehensive framework encompassing the partial NCN, MCN and CCN of the solution of the DSPP.
    \item By considering structure-preserving perturbations on $A,$ $D$, and $E,$ when they retain some linear structures, we derive structured partial CNs for the DSPP.
    \item By exploring the connection between DSPPs and EILS problems, we demonstrate that our derived CN formula can be used to recover the partial CNs for the EILS problem.
    \item Numerical experiments demonstrate that the derived upper bounds provide sharp estimates of the partial CNs. Furthermore, the partial CNs offer precise estimates of the relative forward error in the solution.
\end{itemize}

%structure of the paper is devoted
The structure of the rest of the paper is as follows. Section \ref{sec2}  introduces a few notations, basic definitions, and preliminaries. Section \ref{sec:PCN} presents a unified framework partial CN of the solution of the DSPP \eqref{eq1:SPP}. Section \ref{sec:SPCN} focuses on the investigation of the structured partial CNs for the DSPP. In Section \ref{sec:EILS}, we discuss the partial CNs for EILS problems. Section \ref{sec:Numerical}
 consists of some numerical examples. Section \ref{sec:Conclusion} includes the concluding statements.
 
 \section{Notation and preliminaries}\label{sec2}
\subsection{Notation}
Throughout the paper, $\R^{m\times n}$ represents the set of all $m\times n$ real matrices. The symbols $\|\cdot \|_F$, $\|\cdot \|_2,$ and $\|\cdot \|_{\infty}$ stand for the  Frobenius norm, spectral norm, and infinity norm of the argument, respectively. The notation ${\bf 1}_n\in \R^n$ represents a vector with all entries equal to $1$. For any vector $z=[z_1,z_2,\ldots, z_n]^{\bm\top}\in \R^n,$ $\Theta_{z}\in \R^{n\times n}$ is the diagonal matrix with $i$-th diagonal entry $z_i.$  Following \cite{polarD}, for any vector $z\in \R^n,$ we define 
\begin{align}
   z^{\ddagger}=[z^{\ddagger}_1,z^{\ddagger}_2,\ldots, z^{\ddagger}_n]^{\bm\top},
\end{align}
 where \begin{align}
     z_i^{\ddagger}=\left\{\begin{array}{cc}
   \frac{1}{z_i},      &  z_i\neq 0, \\
      1,    &   z_i=0.
     \end{array}\right.
 \end{align}
 Moreover, the entrywise division of two vector $z,w\in \R^n$ is defined as follows:
 \begin{equation*}
     \frac{z}{w}=\Theta_{w^{\ddagger}}z.
 \end{equation*}
 For the matrices $X=[x_{ij}]\in \R^{m\times n}$ and $Y=[y_{ij}]\in \R^{m\times n},$ we define $X\odot Y:=[x_{ij}y_{ij}]\in \R^{m\times n}.$ Note that, for $z,w\in \R^{n},$ $\dm{\frac{z}{w}}=w^{\ddagger}\odot z.$   For  $X \in \R^{m\times n},$ we set $|X|:=[|x_{ij}|].$   For $X,\,Y\in\R^{m\times n},$  the notation $|X|~
\leq |Y|$ represents $|x_{ij}|~\leq |y_{ij}|$ for all $1\leq i\leq m$ and $1\leq j\leq n$. For  $X=[\x_1, \x_2,\ldots, \x_n]\in \R^{m\times n},$ where $\x_i\in \R^{m},$ $i=1,2,\ldots, n,$  we define $\vec(X):=[\x_1^{T}, \x_2^{T},\ldots, \x_n^{T}]^{T}\in\R^{mn}$. For given matrices $A_1,A_2,\ldots, A_n$, we use $$\vec(\mathbf{X}):=[\vec(A_1)^{\bm\top},~ \vec(A_2)^{\bm\top}, ~\ldots,~ \vec(A_n)^{\bm\top}]^{\bm\top},$$ where $\mathbf{X}=(A_1,A_2,\ldots, A_n).$ The Kronecker product \cite{kronecker1981} of  $X\in \R^{m\times n}$ and $Y\in \R^{p\times q}$ is defined as $X\otimes Y:= [x_{ij}Y]\in \R^{mp\times nq}.$ For matrices $A, B, X, Y$ and $Z$ with compatible dimensions, we have the following properties \cite{kronecker1981, kronecker2004}:
 \begin{align}\label{kron}
\left\{ \begin{array}{c}
 \vec(AZB) =  (B^{\bm\top}\otimes A)\vec(Z),   \\
   (X\otimes Y)^{\bm\top}=X^{\bm\top}\otimes Y^{\bm\top},   \\
   |X\otimes Y|=|X|\otimes |Y|,\\
   (A\otimes B)(X\otimes Y)=AX\otimes BY.\\
 \end{array}\right.
  \end{align}

%
 \subsection{Preliminaries}
%%%%%%%%%%%%%%%%%%%%%%%%%%%%%%%%%%%%%%%%%
At the beginning of this subsection, we introduce the concept of the general CN, referred to as the unified CN. 
%
\begin{definition}\cite{UCN2018}\label{def:ucn}
    Let $\Upsilon : \R^p \mapsto \R^q$ be a continuous mapping defined on an open set $\Omega_{\Upsilon}\subseteq \R^p.$ Then, the unified CN of $\Upsilon$ at $\bm v\in \Omega_{\Upsilon}$ is defined by 
    \begin{equation}
        \mathfrak{K}_{\Upsilon}(\bm v) = \lim_{\bm{\epsilon} \rightarrow 0}\sup_{0< \|\chi^{\ddagger} \odot \D \bm v\|_{\tau}\leq \bm{\epsilon}} \frac{\left\|\xi^{\ddagger}\odot \left(\Upsilon(\bm v+\D \bm v)-\Upsilon(\bm v)\right)\right\|_{\gamma}}{\|\chi^{\ddagger} \odot \D \bm v\|_{\tau}},
    \end{equation}
where $\xi \in \R^{q},$ ${\chi}\in \R^p$ are the parameters such that if some entry of $\chi$ is zero, then the corresponding entry of $\D \bm v$ must be zero, and $\|\,\cdot\,\|_{\tau}$ and $\|\,\cdot\,\|_{\gamma}$ are two vector norms defined on $\R^p$ and $\R^q,$ respectively.
\end{definition}

The parameters $\chi$ and $\xi$ in Definition \ref{def:ucn} can also be positive real numbers, rather than vectors. Note that, Definition \ref{def:ucn} leads to  the following bound:
\begin{equation}\label{eq:forward}
    \left\|\xi^{\ddagger}\odot (\Upsilon(\bm v+\D \bm v)-\Upsilon(\bm v))\right\|_{\gamma}\leq \mathfrak{K}_{\Upsilon}(\bm v) \|\chi^{\ddagger} \odot \D \bm v\|_{\tau}+\mathcal{O}(\|\chi^{\ddagger} \odot \D \bm v\|^2_{\tau}).
\end{equation}
Therefore, the forward error in the solution can be estimated using CNs.
%
\begin{remark}\label{re21}
   The unified CN described in Definition \ref{def:ucn} represents a broad generalization of various well-known CNs that have been explored in the literature. For example:
   \begin{itemize}
       \item   \textbf{NCN:} Consider $\tau=\gamma=2,$
       $\chi =\|\bm v\|_2{\bf 1}_p$ $\in \R^p$ with $\bm v\neq \0$ and $\xi=\|\Upsilon(\bm v)\|_2{\bf 1}_q\in \R^q$ with $\Upsilon(\bm v)\neq \0,$ then  we obtain the NCN, denoted by $ \mathfrak{K}^{(2)}_{\Upsilon}(\bm v).$

 \item    \textbf{MCN:} Consider $\tau=\gamma=\infty,$ $\chi=\bm v\neq \0,$  $\xi=\|\Upsilon(\bm v)\|_{\infty}{\bf 1}_q\in \R^q$ with $\Upsilon(\bm v)\neq \0,$ then  we obtain the MCN, denoted by $ \mathfrak{K}^{\infty}_{mix,\Upsilon}(\bm v).$

\item    \textbf{CCN:} Consider $\tau=\gamma=\infty,$ $\chi =\bm v\neq \0,$  and $\xi=\Upsilon(\bm v)\in \R^q$ with  $\Upsilon(\bm v)\neq \0,$ then  we obtain the CCN, denoted by $ \mathfrak{K}^{\infty}_{com,\Upsilon}(\bm v).$
  \end{itemize}
\end{remark}

%\section{First order perturbation bounds for DSPP}
Next, we present a key result that is essential for the following sections. To derive this, 
let $\mathbf{H}=(A,B,C,D,E)$ and we set $$\vec(\mathbf{H})=[\vec(A)^{\bm\top}, \vec(B)^{\bm\top}, \vec(C)^{\bm\top}, \vec(D)^{\bm\top}, \vec(E)^{\bm\top}]^{\bm\top}.$$
Consider $\D A, \D B, \D C, \D D, \D E,$ and $\D \mathbf{b}=[\D {\bf b}_1^{\T}, \D {\bf b}_2^{\T}, \D {\bf b}_3^{\T}]^{\T}$ are the perturbations on $A, B, C, D,E,$ and $\mathbf{b},$ respectively. Further, we denote
$$\D \mathfrak{B}=\bmatrix{\D A & \D B^{\bm\top} & \bf 0\\ \D B & -\D D & \D C^{\bm\top}\\ \bf 0 & \D C& \D E},$$ and assume that $\|\D \mathfrak{B}\|_2\leq \bm{\epsilon} \|\mathfrak{B}\|_2$ %$\|\D B\|_2\leq \bm{\epsilon} \|B\|_2,$ $\|\D C\|_2\leq \bm{\epsilon} \|C\|_2,$ $\|\D D\|_2\leq \bm{\epsilon} \|D\|_2,$ $\|\D E\|_2\leq \bm{\epsilon} \|E\|_2,$ 
and $\|\D {\bf b}\|_2\leq \bm{\epsilon} \|{\bf b}\|_2$. Then, we have the following perturbed DSSP: 
\begin{eqnarray}\label{eq:pert}
    \bmatrix{A+\D A & (B+\D B)^{\bm\top} & \bf 0\\ B+\D B &-(D+ \D D) & (C+ \D C)^{\bm\top} \\ \bf 0 & C+\D C& E +\D E} \bmatrix{\bm{x}+\D \bm{x}\\ \bm{y}+ \D \bm{y} \\ \bm{z} +\D \bm{z}}= \bmatrix{\mathbf{b}_1+\D \mathbf{b}_1\\ \mathbf{b}_2+\D \mathbf{b}_2 \\\mathbf{b}_3+\D \mathbf{b}_3},
\end{eqnarray}
which has the unique solution $\bmatrix{\bm{x}+\D \bm{x}\\ \bm{y}+ \D \bm{y} \\ \bm{z} +\D \bm{z}}$ when $\|\mathfrak{B}^{-1}\|_2\|\Delta \mathfrak{B}\|_2<1.$ 
%

Consequently, we obtain the following important result.
\begin{lemma}\label{lemma:sec2}
Suppose $[\x^{\T}, \y^{\T}, \z^{\T}]$ and $[(\x+\D \x)^{\T}, (\y+\D \y)^{\T}, (\z+\D \z)^{\T}]^{\T}$ are the unique solutions of the original DSPP \eqref{eq1:SPP} and perturbed DSPP \eqref{eq:pert}, respectively.
Then, the first-order perturbation expression of $\bmatrix{\D\bm{x}^{\T},\D\bm{y}^{\T}, \D\bm{z}^{\T}}^{\T}$ is given by 
\begin{eqnarray}\label{eq27}
  \bmatrix{\D\bm{x}\\ \D\bm{z}\\ \D\bm{y}}= -\mathfrak{B}^{-1} \bmatrix{ \mathcal{G} &  -I_{\bm l}}\bmatrix{\vec(\D \mathbf{H})\\ \D {\bf b}} + \mathcal{O}(\bm{\epsilon}^2),
\end{eqnarray}
 where \begin{equation}
     \mathcal{G}= \bmatrix{\bm{x}^{\bm\top}\otimes I_n & I_n\otimes \bm{y}^{\bm\top}& \bf 0& \bf 0& \bf 0\\ \bf 0& \bm{x}^{\bm\top}\otimes I_m & I_m\otimes \bm{z}^{\bm\top} &- \bm{y}^{\bm\top}\otimes I_m & \bf 0\\ \bf 0 & \bf 0& \bm{y}^{\bm\top}\otimes I_p & \bf 0 & \bm{z}^{\bm\top}\otimes I_p}\in \R^{\bm{l} 
  \times \bm{s}},
 \end{equation}
 $$\vec(\D \mathbf{H})=[\vec(\D A)^{\bm\top}, \vec(\D B)^{\bm\top}, \vec(\D C)^{\bm\top}, \vec(\D D)^{\bm\top}, \vec(\D E)^{\bm\top}]^{\bm\top},$$
 and ${\bm s}=(n^2+m^2+p^2+nm+mp).$
 \end{lemma}
 %
\proof
Combining \eqref{eq1:SPP} and \eqref{eq:pert}, we obtain
\begin{eqnarray}\label{eq29}
    \bmatrix{A & B^{\bm\top} & \bf 0\\ B& -D &C^{\bm\top}\\ \bf 0& C & E} \bmatrix{\D \bm{x}\\ \D \bm{y}\\ \D \bm{z}}= \bmatrix{\D \mathbf{b}_1\\ \D \mathbf{b}_2\\ \D \mathbf{b}_3}-\bmatrix{\D A \bm{x}+\D B^{\bm\top} \bm{y}\\ \D B \bm{x}- \D D \bm{y}+ \D C^{\bm\top}\bm{z}\\ \D C \bm{y}+ \D E \bm{z}}+\mathcal{O}(\bm{\epsilon}^2).
\end{eqnarray}
Thus, the proof follows by applying the vec operator on \eqref{eq29} and utilizing the properties listed in \eqref{kron}.
$\blacksquare$

 %%%%%%%%%%%%%%%%%%%%%%%%%%%%%%%%%%%%%%%%%%%%%%%%%%%%%%%%%%%%%%%%%%%%%%%%%%%%%%%%%%
 \section{ Partial Unified CNs for the DSPP }\label{sec:PCN}
This section primarily focuses on developing a unified framework for the partial CN for the solution $\bm{w}=[\x^{\bm\top},\y^{\bm\top},\z^{\bm\top}]^{\bm\top}$ of the DSPP \eqref{eq1:SPP}. As special cases, we also derive the compact formulae and computationally efficient upper bounds for the partial NCN, MCN and CCN.

%Let $\mathbf{H}=(A,B,C,D,E)$ and we set $$\vec(\mathbf{H})=[\vec(A)^{\bm\top}, \vec(B)^{\bm\top}, \vec(C)^{\bm\top}, \vec(D)^{\bm\top}, \vec(E)^{\bm\top}]^{\bm\top}.$$
To derive the  partial unified CN of the DSPP \eqref{eq1:SPP}, we define the following mapping:
\begin{align}\label{eq:map}
\nonumber \vp:\,& \R^{n\times n}\times  \R^{m\times n}\times  \R^{p\times m}\times  \R^{m\times m}\times  \R^{p\times p}\times \R^{\bm l} \rightarrow \R^k\\
%
&~~\vp(\mathbf{H}, \mathbf{b})=\mathrm{L}\bmatrix{\x\\ \y \\ \z}=\mathrm{L}\mathfrak{B}^{-1}\mathbf{b},
\end{align}
where $\mathrm{L}\in \R^{k\times \bm l} (k\leq \l).$
Following the Definition \ref{def:ucn}, we now define the partial unified CN for the DSPP using the mapping $\vp$ as follows.

% \begin{lemma}\label{frechet}
%     The map $\Phi$ defined in \eqref{eq:map} is Fr\(\Acute{e}\)chet differentiable at $\mathbf{H}$ and the $Fr\Acute{e}chet$ derivative at $\mathbf{H}$ is given by
%     \begin{equation}
%         \bf{d}\Phi\left(\mathbf{H}, \bf{b}\right)= -\mathbf{L}\bmatrix{\mathfrak{B}^{-1} \mathcal{G} &  -I_{\bm l}}.
%     \end{equation}
% \end{lemma}

\begin{definition}\label{def:PartialCN}
Suppose $\bm{w}=[\bm{x}^{\bm\top}, \bm{y}^{\bm\top}, \bm{z}^{\bm\top}]^{\bm\top}$ is the unique solution of the DSPP \eqref{eq1:SPP} and $\mathrm{L}\in \R^{k\times \bm{l}}.$ Consider the map $\vp$ defined as in \eqref{eq:map}. Then, the  partial unified  CN of $\bm{w}=[\bm{x}^{\bm\top}, \bm{y}^{\bm\top}, \bm{z}^{\bm\top}]^{\bm\top}$  with respect to (w.r.t.) $\mathrm{L}$ is defined as follows:
    \begin{eqnarray*}
         \mathfrak{K}_{\vp}(\mathbf{H}, \b; \mathrm{L}):=\lim_{\bm{\epsilon}\rightarrow 0} \sup_{0<\left\|\vec\left(\Psi^{\ddagger} \odot \Delta \mathbf{H},\, \be^{\ddagger} \odot \D \b\right)\right\|_{\tau}\leq \bm{\epsilon}} \frac{\left\|\xi^{\ddagger}_{\mathrm{L}}\odot \left( \bm{\varphi}(\mathbf{H}+\D \mathbf{H}, \mathbf{b}+ \D \mathbf{b})-\vp(\mathbf{H}, \b)\right) \right\|_\gamma}{\left\|\vec\left(\Psi^{\ddagger} \odot \Delta \mathbf{H}, \be^{\ddagger} \odot \D \b\right)\right\|_{\tau}},
    \end{eqnarray*}
    where $\xi_{\mathrm{L}}\in \R^{k},$ $\Psi=(\Psi_A, \Psi_B,\Psi_C, \Psi_D,\Psi_E),$ $\Psi_A\in \R^{n\times n},$ $\Psi_B\in \R^{m\times n},$ $\Psi_C\in \R^{p\times m},$ $\Psi_D\in \R^{m\times m},$ $\Psi_E\in \R^{p\times p}$ and $\bm{\chi} \in \R^{\bm l}$ are the parameters with the assumptions that %all the entries of $\Psi$ and $\be$ are nonzero.%
    if some entries of $\Psi$ and $\be$ are zero, then the corresponding entry of $\D \mathbf{H}$ and $\D {\bf b},$ respectively,  must be zero.
\end{definition}
%
\begin{remark}
In the context of  Remark \ref{re21}, to obtain the partial NCN, we consider  
$\xi_{\mathrm{L}}=\|\mathrm{L}[\bm x^{\bm\top}, \bm y^{\bm\top}, \bm z^{\bm\top}]^{\T}\|_2 {\bf 1}_{\bm l},$
for the partial MCN, we consider 
$\xi_{\mathrm{L}}=\|\mathrm{L}[\bm x^{\bm\top}, \bm y^{\bm\top}, \bm z^{\bm\top}]^{\T}\|_{\infty} {\bf 1}_{\bm l},$ and
for partial CCN, we consider $\xi_{\mathrm{L}}=\mathrm{L}[\bm x^{\bm\top}, \bm y^{\bm\top}, \bm z^{\bm\top}]^{\T}.$
\end{remark}
%
In the following theorem, we provide compact and closed-form expression for the partial unified CN.
%
\begin{theorem}\label{th1}
    Suppose $\bm{w}=[\bm{x}^{\bm\top}, \bm{y}^{\bm\top}, \bm{z}^{\bm\top}]^{\bm\top}$ is the unique solution of the DSPP \eqref{eq1:SPP} and $\mathrm{L}\in \R^{k\times \bm{l}}.$ Then, the partial unified CN of $\bm{w}=[\bm{x}^{\bm\top}, \bm{y}^{\bm\top}, \bm{z}^{\bm\top}]^{\bm\top}$ w.r.t. $\mathrm{L}$ is given by
    \begin{equation}
         \mathfrak{K}_{\vp}(\mathbf{H}, {\bf  b}; \mathrm{L}) = \left\|\Theta_{\xi^{\ddagger}_{\mathrm{L}}}\mathrm{L}\mathfrak{B}^{-1}\bmatrix{ \mathcal{G} &  -I_{\bm l}} \bmatrix{\Theta_{\vec(\Psi)}& \bf 0\\ \bf 0 & \Theta_{\be}}\right\|_{\tau,\gamma},
    \end{equation}
    where $\|\cdot~\|_{\tau,\gamma}$ is the matrix norm induced by vector norms $\|\cdot~\|_{\tau} $ and $\|\cdot~\|_{\gamma}.$
\end{theorem}
% 
\proof From the definition of the mapping $\vp$ in \eqref{eq:map} and Lemma \ref{lemma:sec2}, we get
\begin{align}
\nonumber    \vp(\mathbf{H}+\D \mathbf{H}, \b +\D \b)-\vp(\mathbf{H}, \b)&=\mathrm{L}\bmatrix{\bm{x}+\D\bm{x}\\ \bm{y}+\D\bm{y}\\ \bm{z}+\D\bm{z}}-\mathrm{L}\bmatrix{\x\\ \y \\ \z}\\ \nonumber%\mathrm{L}[(\bm{x}+\D\bm{x})^{\bm\top}, (\y+\D\bm{y})^{\bm\top}, (\z+ \D\bm{z})^{\bm\top}]^{\bm\top}-\mathrm{L}[\x^{\bm\top},\y^{\bm\top},\z^{\bm\top}]^{\bm\top}\\ \nonumber
    &=\mathrm{L} \bmatrix{\D\bm{x}\\ \D\bm{y}\\ \D\bm{z}}\\ \label{th1:eq1}
    &= -\mathrm{L}\mathfrak{B}^{-1}\bmatrix{ \mathcal{G} &  -I_{\bm l}}\bmatrix{\vec(\D \mathbf{H})\\ \D \b}+ \mathcal{O}(\bm{\epsilon}^2).
\end{align}
By considering the requirement on $\Psi$ and $\be,$ we have
\begin{equation}\label{th1:eq2}
    \bmatrix{\vec(\D \mathbf{H}) \\ \D {\bf b}}=\bmatrix{\Theta_{\vec(\Psi)}& \bf 0\\ \bf 0 & \Theta_{\be}}\bmatrix{\vec(\Psi^{\ddagger} \odot \D \mathbf{H}) \\ \bm{\chi}^{\ddagger} \odot \D {\bf b}}.
\end{equation}
Substituting \eqref{th1:eq2} into \eqref{th1:eq1} and from Definition \ref{def:PartialCN}, we obtain
\begin{align}
     \nonumber  \mathfrak{K}_{\vp}(\mathbf{H}, \b; \mathrm{L})& = \sup_{\left\|\vec\left(\Psi^{\ddagger} \odot \Delta \mathbf{H},\, \be^{\ddagger} \odot \D \b\right)\right\|_{\tau}\neq 0} \frac{\left\|\Theta_{\xi^{\ddagger}_{\mathrm{L}}}\mathrm{L} 
 \mathfrak{B}^{-1} \bmatrix{ \mathcal{G} &  -I_{\bm l}} \bmatrix{\Theta_{\vec(\Psi)}& \bf 0\\ \bf 0 & \Theta_{\be}} \bmatrix{\vec(\Psi^{\ddagger} \odot \D \mathbf{H})\\ \bm{\chi}^{\ddagger} \odot \D {\bf b}}\right\|_{\gamma}}{\left\|\vec\left(\Psi^{\ddagger} \odot \Delta \mathbf{H}, \be^{\ddagger} \odot \D \b\right)\right\|_{\tau}}\\
      &=\left\|\Theta_{\xi^{\ddagger}_{\mathrm{L}}}\mathrm{L} \mathfrak{B}^{-1}  \bmatrix{\mathcal{G} &  -I_{\bm l}} \bmatrix{\Theta_{\vec(\Psi)}& \bf 0\\ \bf 0 & \Theta_{\be}}\right\|_{\tau,\gamma}.
\end{align}
Hence, the proof is completed.
$\blacksquare$

Next, we derive various partial CNs by considering specific norms. In the following result, we focus on   when \(\tau = \gamma = 2\).
%
\begin{theorem}\label{th32}
    Consider $\tau=\gamma= 2$ and assuming that  $\Psi,$ $\be$ and $\xi_{\mathrm{L}}$ are positive real numbers, then the partial CN has the following forms:
    \begin{align}\label{eq1:th2}
          &  \mathfrak{K}^{(2)}_{\vp}(\mathbf{H}, \b; \mathrm{L}) = \frac{\left\|\mathrm{L} \mathfrak{B}^{-1}   \bmatrix{\Psi  \mathcal{G} &  -\be I_{\bm l}} \right\|_{2}}{\xi_{\mathrm{L}}},\\ \label{eq:th32}
          %
         & \widehat{ \mathfrak{K}}^{(2)}_{\vp}(\mathbf{H}, \b; \mathrm{L}) = \frac{\left\|\mathrm{L}\mathfrak{B}^{-1}({\Psi}^2 \mathcal{J}   +{\be}^2I_{\bm l}) (\mathfrak{B}^{-1})^{\T} \mathrm{L}^{\bm\top}\right\|^{1/2}_{2}}{\xi_{\mathrm{L}}},
         %
          % & \mathfrak{K}^2_{2\Phi}(\mathbf{H}, \b; \mathrm{L}) = \frac{\left\|\mathrm{L}({\Psi}^2\mathfrak{B}^{-1} \mathcal{J}  \mathfrak{B}^{-1} +{\be}^2I_{\bm l}) \mathrm{L}^{\bm\top}\right\|^{1/2}_{2}}{\xi_{\mathrm{L}}},
    \end{align}
    where $\mathcal{J}\in \R^{\bm{l}\times \bm{l}}$ is given by
   % $$\mathcal{J}=\bmatrix{\x^{\bm\top}\x\otimes I_n+I_n\otimes \y^{\bm\top}\y & \x\otimes \y^{\bm\top} & {\bf 0}\\ \x^{\bm\top}\otimes \y & \x^{\bm\top}\x\otimes I_m+I_m\otimes \z^{\bm\top}\z+ \y^{\bm\top}\y\otimes I_m & \y\otimes \z^{\bm\top}\\ \bf 0  & \y^{\bm\top}\otimes \z & \y^{\bm\top}\y\otimes I_p+ \z^{\bm\top}\z\otimes I_p}.$$
    $$\mathcal{J}=\bmatrix{(\|\x\|_2^2 +\|\y\|_2^2) I_n & \x \y^{\bm\top} & {\bf 0}\\ \y\x^{\bm\top}  & (\|\x\|_2^2 +\|\y\|_2^2+ \|\z\|_2^2)  I_m & \y\z^{\bm\top}\\ \bf 0  & \z\y^{\bm\top}  & (\|\y\|_2^2 +\|\z\|_2^2)  I_p}.$$
\end{theorem}
%
\proof Since $\tau=\gamma =2$ and $\Psi,$ $\be$ and $\xi_{\mathrm{L}}$ are positive real numbers, from Theorem \ref{th1}, we obtain \begin{eqnarray}\label{eq3:th2}
     \mathfrak{K}^{(2)}_{1,\vp}(\mathbf{H}, \b; \mathrm{L})= \frac{\left\|\mathrm{L}\mathfrak{B}^{-1}\bmatrix{\Psi \mathcal{G} &  -\be I_{\bm l}} \right\|_{2}}{\xi_{\mathrm{L}}}.
\end{eqnarray}
 Using the property that, for $Z\in \R^{m\times n},$ $\|Z\|_2 =\|ZZ^{\bm\top}\|_2^{1/2},$ we have 
 \begin{align}\label{eq2:th2}
     \nonumber \left\|\mathrm{L}\mathfrak{B}^{-1}\bmatrix{\Psi \mathcal{G} &  -\be I_{\bm l}} \right\|_{2}&= \left\| \mathrm{L}\mathfrak{B}^{-1} (\Psi^2\mathcal{G} \mathcal{G}^{\bm\top}+\be^2 I_{\bm l}) (\mathfrak{B}^{-1})^{\T}\mathrm{L}^{\bm\top}\right\|_2^{1/2}\\
     &= \left\| \mathrm{L}\mathfrak{B}^{-1} (\Psi^2\mathcal{J} +\be^2 I_{\bm l}) (\mathfrak{B}^{-1})^{\T}\mathrm{L}^{\bm\top}\right\|_2^{1/2}.
 \end{align}
 Hence, the proof follows by substituting \eqref{eq2:th2} into  \eqref{eq3:th2}.
$\blacksquare$

\begin{remark}
    Notably, the equivalent expression of $ \widehat{\mathfrak{K}}^{(2)}_{\Phi}(\mathbf{H}, \b; \mathrm{L})$ in \eqref{eq:th32} is free of computationally expensive Kronecker products. Moreover, the matrices in \eqref{eq1:th2} and \eqref{eq:th32} have dimensions \(k \times (\bm{l+s})\) and \(k \times k\), respectively. Hence, the expression in \eqref{eq:th32} significantly reduces the storage requirements.
\end{remark}
\begin{remark}
  The partial CN in Theorem \ref{th32} is a simplified version of the partial NCN for the solution of the DSPP \eqref{eq1:SPP}. The  NCN for ${\bm w}=[\x^{\T},  \y^{\T},\z^{\T  }]^{\T}, \x, \y$ and $\z$ can be obtained by considering $$\mathrm{L}= I_{\bm l},~ \bmatrix{I_n & {\bf 0}_{n\times (m+p)}}, ~\bmatrix{{\bf 0}_{m\times n}& I_m&{\bf 0}_{m\times p}} ~\text{and}~ \bmatrix{{\bf 0}_{p\times (n+m)}& I_p},$$ respectively, in Theorem \ref{th32}.
\end{remark}
%
In the next result, we provide an easily computable upper bound for the partial CN $   \mathfrak{K}^{(2)}_{\Phi}(\mathbf{H}, \b; \mathrm{L}).$
\begin{corollary}\label{cor31}
    Under the assumption of Theorem \ref{th32}, we have the following upper bound:
    \begin{eqnarray}
         \mathfrak{K}^{(2)}_{1,\vp}(\mathbf{H}, \b; \mathrm{L})\leq \mathfrak{K}^{(2),u}_{1,\vp}(\mathbf{H}, \b; \mathrm{L}):=\frac{\|\mathrm{L}\mathfrak{B}^{-1}\|_2}{\xi_{\mathrm{L}}}\left( \Psi\|\mathcal{J}\|^{1/2}_{2}+\be \right).
    \end{eqnarray}
\end{corollary}
%
\proof Using the properties of the spectral norm that for the matrices $X$ and $Y$ of appropriate sizes, $\left\|\bmatrix{X & Y} \right\|_2\leq \|X\|_2 + \|Y\|_2$ and $\|XY\|_2\leq \|X\|_2 \|Y\|_2,$ and from \eqref{eq1:th2}, we obtain 
\begin{align}\label{eq:coro31}
  \nonumber  \left\|\mathrm{L}\mathfrak{B}^{-1}\bmatrix{\Psi \mathcal{G} &  -\be I_{\bm l}} \right\|_{2} &\leq \Psi \| \mathrm{L}\mathfrak{B}^{-1}\mathcal{G}\|_2 +\be \| \mathrm{L}\mathfrak{B}^{-1}\|_2\\ \nonumber
    & \leq \Psi \| \mathrm{L}\mathfrak{B}^{-1}\|_2 \|\mathcal{G}\|_2 +\be \| \mathrm{L}\mathfrak{B}^{-1}\|_2\\
   % &= \Psi \| \mathrm{L}\mathfrak{B}^{-1}\|_2 \|\mathcal{G} \mathcal{G}^{\bm\top}\|^{1/2}_2 +\be \| \mathrm{L}\mathfrak{B}^{-1}\|_2\\
    &= \Psi \| \mathrm{L}\mathfrak{B}^{-1}\|_2 \|\mathcal{J}\|^{1/2}_2 +\be \| \mathrm{L}\mathfrak{B}^{-1}\|_2.
\end{align}
Hence, the proof follows from \eqref{eq1:th2} and \eqref{eq:coro31}. $\blacksquare$

In the following theorem, we investigate the partial CN for the DSPP when \(\tau = \gamma = \infty\), from which we derive the partial MCN and CCN.
%
\begin{theorem}\label{th33}
    When $\tau=\gamma=\infty,$ the partial CN is given as follows:
    \begin{align}
         \mathfrak{K}^{\infty}_{\vp}(\mathbf{H}, \b; \mathrm{L}) = \left\| |\Theta_{\xi^{\ddagger}_{\mathrm{L}}}| \left|\mathrm{L}\mathfrak{B}^{-1} \bmatrix{ \mathcal{G} &  -I_{\bm l}}\right| \bmatrix{\vec(|\Psi|)\\  |\be|}\right\|_{\infty}.
    \end{align}
    Moreover, we consider $\Psi=\mathbf{H},$ $\be ={\bf b}.$ Then, if we set $\xi_{\mathrm{L}}=\|\mathrm{L}[\bm{x}^{\bm\top}, \bm{y}^{\bm\top}, \bm{z}^{\bm\top}]^{\bm\top}\|_{\infty}{\bf{1}}_{\bm l},$  the partial MCN 
  is given by 
   \begin{align}\label{PMCN}
         \mathfrak{K}^{\infty}_{mix,  \vp}(\mathbf{H}, \b; \mathrm{L}) = \frac{\left\| \left|\mathrm{L}\mathfrak{B}^{-1}\bmatrix{ \mathcal{G} &  -I_{\bm l}}\right| \bmatrix{\vec(|\mathbf{H}|)\\ |{\bf b}| }\right\|_{\infty}}{\|\mathrm{L}[\bm{x}^{\bm\top}, \bm{y}^{\bm\top}, \bm{z}^{\bm\top}]^{\bm\top}\|_{\infty}}
    \end{align}
    and if we set $\xi_{\mathrm{L}}= \mathrm{L}[\bm{x}^{\bm\top}, \bm{y}^{\bm\top}, \bm{z}^{\bm\top}]^{\bm\top},$  the partial CCN is given by
    \begin{align}\label{PCCN}
         \mathfrak{K}^{\infty}_{com, \vp}(\mathbf{H}, \b; \mathrm{L}) = \left\|\frac{ \left|\mathrm{L} \mathfrak{B}^{-1} \bmatrix{ \mathcal{G} &  -I_{\bm l}}\right| \bmatrix{\vec(|\mathbf{H}|)\\ |{\bf b}| }}{\left|\mathrm{L}[\bm{x}^{\bm\top}, \bm{y}^{\bm\top}, \bm{z}^{\bm\top}]^{\bm\top}\right|}\right\|_{\infty}.
    \end{align}
\end{theorem}
\proof Consider $\tau=\gamma=\infty,$ then from Theorem \ref{th1}, we have
\begin{align*}
     \mathfrak{K}^{\infty}_{\Phi}(\mathbf{H}, \b; \mathrm{L})& = \left\| \Theta_{\xi^{\ddagger}_{\mathrm{L}}}\mathrm{L}\mathfrak{B}^{-1} \bmatrix{ \mathcal{G} &  -I_{\bm l}} \bmatrix{\Theta_{\vec(\Psi)}& \bf 0\\ \bf 0 & \Theta_{\be}}\right\|_{\infty}\\
    %
    &= \left\| |\Theta_{\xi^{\ddagger}_{\mathrm{L}}}|  \left|\mathfrak{B}^{-1} \bmatrix{ \mathcal{G} &  -I_{\bm l}}\right| \bmatrix{|\Theta_{\vec(\Psi)}|& \bf 0\\ \bf 0 & |\Theta_{\be}|} \right\|_{\infty}\\
    %
     &= \left\| |\Theta_{\xi^{\ddagger}_{\mathrm{L}}}|  \left|\mathfrak{B}^{-1} \bmatrix{ \mathcal{G} &  -I_{\bm l}}\right| \bmatrix{|\Theta_{\vec(\Psi)}|& \bf 0\\ \bf 0 & |\Theta_{\be}|} {\bf 1}_{s+ \bm{l}}\right\|_{\infty}\\
     %
     &= \left\| |\Theta_{\xi^{\ddagger}_{\mathrm{L}}}|  \left| \mathfrak{B}^{-1} \bmatrix{ \mathcal{G} &  -I_{\bm l}}\right| \bmatrix{\vec(|\Psi|)\\  |\be|} \right\|_{\infty}.
 \end{align*}
Rest of the proof followings considering $\Psi = \mathbf{H},$ $\be= {\bf b},$ $\xi_{\mathrm{L}}=\|\mathrm{L}[\bm{x}^{\bm\top}, \bm{y}^{\bm\top}, \bm{z}^{\bm\top}]^{\bm\top}\|_{\infty}{\bf{1}}_{\bm l} $ or $\xi_{\mathrm{L}}=\mathrm{L}[\bm{x}^{\bm\top}, \bm{y}^{\bm\top}, \bm{z}^{\bm\top}]^{\bm\top}.$
$\blacksquare$


\begin{remark}
  The MCN and CCN for ${\bm w}=[\x^{\T},  \y^{\T},\z^{\T  }]^{\T}, \x, \y$ and $\z$ can be obtained by considering $$\mathrm{L}= I_{\bm l},~ \bmatrix{I_n & {\bf 0}_{n\times (m+p)}}, ~\bmatrix{{\bf 0}_{m\times n}& I_m&{\bf 0}_{m\times p}} ~\text{and}~ \bmatrix{{\bf 0}_{p\times (n+m)}& I_p},$$ respectively, in \eqref{PMCN}  and \eqref{PCCN} of Theorem \ref{th33}.
\end{remark}
In the following result, we provide sharp upper bounds for the partial MCN and CCN obtained in Theorem \ref{th33}.
\begin{corollary}\label{coro32}
    Assume that the conditions in Theorem \ref{th33} hold. Then
    \begin{align*}
         \mathfrak{K}^{\infty}_{mix, \vp}(\mathbf{H}, \b; \mathrm{L})\leq  \mathfrak{K}^{\infty, u}_{mix, \vp}(\mathbf{H}, \b; \mathrm{L}) := \frac{\left\||\mathrm{L}\mathfrak{B}^{-1}| \left(|\mathcal{H}|+|{\bf b}|\right)\right\|_{\infty}}{\left\|\mathrm{L}[\bm{x}^{\bm\top}, \bm{y}^{\bm\top}, \bm{z}^{\bm\top}]^{\bm\top}\right\|_{\infty}}
    \end{align*}
    and \begin{align*}
         \mathfrak{K}^{\infty}_{com, \vp}(\mathbf{H}, \b; \mathrm{L})\leq  \mathfrak{K}^{\infty, u}_{com, \vp}(\mathbf{H}, \b; \mathrm{L}) : = \left\|\frac{|\mathrm{L}\mathfrak{B}^{-1}| \left(|\mathcal{H}|+|{\bf b}|\right)}{\mathrm{L}[\bm{x}^{\bm\top}, \bm{y}^{\bm\top}, \bm{z}^{\bm\top}]^{\bm\top}}\right\|_{\infty},
    \end{align*}
    where $\mathcal{H}=\bmatrix{|A||\x|+|B^{\bm\top}||\y|\\ |B||\x|+|D^{\bm\top}||\y|+|C^{\bm\top}||\z|\\ |C||\y|+|E||\z|}.$
\end{corollary}
%
\proof Utilizing the properties of Kronecker product in \eqref{kron}, we have 
\begin{align}\label{eq:coro1}
  \nonumber  &\left|\mathrm{L}\mathfrak{B}^{-1}\bmatrix{ \mathcal{G} ~&  -I_{\bm l}}\right| \bmatrix{\vec(|\mathbf{H}|)\\ |{\bf b}| }\leq |\mathrm{L} \mathfrak{B}^{-1}| \bmatrix{|\mathcal{G}| & I_{\bm l}} \bmatrix{\vec(|\mathbf{H}|)\\ |{\bf b}| } \\ \nonumber
    %
   &= |\mathrm{L} \mathfrak{B}^{-1}| (|\mathcal{G}| \vec(|\mathbf{H}|) + |{\bf b}| )\\ \nonumber
   %
   &=  |\mathrm{L} \mathfrak{B}^{-1}| \left(\bmatrix{ (|\x|^{\bm\top} \otimes I_n) \vec(|A|) + (I_n\otimes |\y|^{\bm\top}) \vec(|B|)\\ (|\x|^{\bm\top} \otimes I_m) \vec(|B|)+ (I_m\otimes |\z|^{\bm\top}) \vec(|C|) + (|\y|^{\bm\top} \otimes I_m) \vec(|D|) \\ (|\y|^{\bm\top} \otimes I_p) \vec(|C|) + (|\z|^{\bm\top} \otimes I_p) \vec(|E|)} + |{\bf b}|\right)\\  
   &= |\mathrm{L}\mathfrak{B}^{-1}| \left(\left|\bmatrix{|A||\x|+|B^{\bm\top}||\y|\\ |B||\x|+|D^{\bm\top}||\y|+|C^{\bm\top}||\z|\\ |C||\y|+|E||\z|}\right| + |{\bf b}|\right).
\end{align}
From \eqref{eq:coro1} and the expressions of partial MCN and CCN in Theorem \ref{th33}, we get 
\begin{eqnarray*}
    \mathfrak{K}^{\infty}_{mix, \vp}(\mathbf{H}, \b; \mathrm{L})\leq  \mathfrak{K}^{\infty, u}_{mix, \vp}(\mathbf{H}, \b; \mathrm{L}) ~ \text{and}~    \mathfrak{K}^{\infty}_{com, \vp}(\mathbf{H}, \b; \mathrm{L})\leq  \mathfrak{K}^{\infty, u}_{com, \vp}(\mathbf{H}, \b; \mathrm{L}).
\end{eqnarray*}
Hence, the proof follows.
$\blacksquare$

\section{Structured partial CNs}\label{sec:SPCN}
%
Consider three subspaces $\mathbb{S}_1\subseteq \R^{n\times n}$, $\mathbb{S}_2\subseteq \R^{m\times m}$ and 
     $\mathbb{S}_3\subseteq \R^{p\times p}$ consisting of three distinct linear structured matrices, such as symmetric. Suppose that the corresponding dimensions of the linear subspaces are $s,$ $r$ and $q,$ respectively. Let $A\in \mathbb{S}_1, $ $D\in \mathbb{S}_2$ and $E\in \mathbb{S}_3,$ then according to \cite{HIGHAM1992, TLS2011Li, Rump2003}, there exist unique generating vectors $\bm{a}\in \R^{s}$ $\bm{d}\in \R^{r}$ and $\bm{e}\in \R^{q}$ such that 
\begin{align}\label{eq:51}
    \vec(A)=\bm{\Phi}_{\mathbb{S}_1}\bm{a}, ~\vec(D)=\bm{\Phi}_{\mathbb{S}_2}\bm{d} ~~\text{and}~~ \vec(E)=\bm{\Phi}_{\mathbb{S}_3}\bm{e},
\end{align}
where $\bm{\Phi}_{\mathbb{S}_1}\in \R^{n^2\times s},$ $\bm{\Phi}_{\mathbb{S}_2}\in \R^{m^2\times r}$ and $\bm{\Phi}_{\mathbb{S}_3}\in \R^{p^2\times q}.$ These matrices are fixed for each specific structure and encapsulate the information corresponding to the linear structure of their respective subspaces.  

Let $\vec_{\mathbb{S}}(\mathbf{H})=[\bm{a}^{\bm\top}, \vec(B)^{\bm\top}, \vec(C)^{\bm\top}, \bm{d}^{\bm\top}, \bm{e}^{\bm\top}]^{\bm\top}.$ Then the  structured partial CN for solution $\bm{w}=[\x^{\bm\top},\y^{\bm\top},\z^{\bm\top}]^{\bm\top}$ of the DSPP \eqref{eq1:SPP}  w.r.t. $\mathrm{L}$ is given by
\begin{align}\label{def:SCN}
         \mathfrak{K}^{\mathbb{S}}_{\vp}(\mathbf{H}, \b; \mathrm{L}):=\lim_{\bm{\epsilon}\rightarrow 0} \sup_{\substack{0<\left\|\vec\left(\Psi^{\ddagger} \odot \Delta \mathbf{H},\, \be^{\ddagger} \odot \D \b\right)\right\|_{\tau}\leq \bm{\epsilon}\\ \D A\in\, \mathbb{S}_1,\, \D D\in\, \mathbb{S}_2,\, \D E \in\, \mathbb{S}_3} } \frac{\left\|\xi^{\ddagger}_{\mathrm{L}}\odot \left( \vp(\mathbf{H}+\D \mathbf{H}, \mathbf{b}+ \D \mathbf{b})-\vp(\mathbf{H}, \b)\right) \right\|_\gamma}{\left\|\vec\left(\Psi^{\ddagger} \odot \Delta \mathbf{H}, \be^{\ddagger} \odot \D \b\right)\right\|_{\tau}},
    \end{align}
    where $\xi_{\mathrm{L}}\in \R^{k},$ $\Psi=(\Psi_A, \Psi_B,\Psi_C, \Psi_D,\Psi_E),$ $\Psi_A\in \mathbb{S}_1,$ $\Psi_B\in \R^{m\times n},$ $\Psi_C\in \R^{p\times m},$ $\Psi_D\in \mathbb{S}_2,$ $\Psi_E\in \mathbb{S}_3$ and $\bm{\chi} \in \R^{\bm l}$.

    Since the matrices $\D A, \Psi_A\in \mathbb{S}_1, $ $\D D, \Psi_D\in \mathbb{S}_2$ and $\D E, \Psi_E\in \mathbb{S}_3,$ as in \eqref{eq:51}, we have 
    \begin{align}\label{eq:53}
  &  \vec(\D A)=\bm{\Phi}_{\mathbb{S}_1}\D \bm{a}, ~ \vec(\Psi_A)=\bm{\Phi}_{\mathbb{S}_1}\psi_A, ~\vec(\D D)=\bm{\Phi}_{\mathbb{S}_2}\D\bm{d}, \\
    & \vec(\Psi_D)=\bm{\Phi}_{\mathbb{S}_2}\psi_D,~ \vec(\D E)=\bm{\Phi}_{\mathbb{S}_3}\D \bm{e}, ~ \vec(\Psi_E)=\bm{\Phi}_{\mathbb{S}_3}\psi_E,
\end{align}
where $\D \bm{a},$  $\D \bm{d},$  $\D \bm{e},$ $\psi_D,$ $\psi_A,$ and $\psi_E$ are the unique generating vectors of $\D A,$ $\D D,$ $\D E,$ $\Psi_A,$  $\Psi_D,$  and $\Psi_E$, respectively. Note that, $\Psi_A^{\ddagger}\in \mathbb{S}_1,$ $\Psi_D^{\ddagger}\in \mathbb{S}_2$ and $\Psi_E^{\ddagger}\in \mathbb{S}_3$. Consequently, we obtain
\begin{align}\label{eq:55}
    \vec(\Psi_A^{\ddagger}\odot\D A)&=\bm{\Phi}_{\mathbb{S}_1} (\psi_A^{\ddagger}\odot \D\bm{a}),~~ \vec(\Psi_D^{\ddagger}\odot\D D)=\bm{\Phi}_{\mathbb{S}_2} (\psi_D^{\ddagger}\odot \D\bm{d})\\ \label{eq:56}
     &\vec(\Psi_E^{\ddagger}\odot\D E)=\bm{\Phi}_{\mathbb{S}_3} (\psi_E^{\ddagger}\odot \D\bm{e}).
\end{align}
Subsequently, we obtain the following result.
\begin{lemma}
Let $\D A, \Psi_A \in \mathbb{S}_1,$ $\D D, \Psi_D \in \mathbb{S}_2,$ $\D E, \Psi_E \in \mathbb{S}_3,$ $B, \Psi_B\in \R^{m\times n},$ $C, \Psi_C\in \R^{p\times m},$  and ${\bf b}, \be\in \R^{\bm l}.$ Then, we have 
    \begin{equation}\label{eq:57}
        \bmatrix{\vec(\Psi^{\ddagger} \odot \mathbf{H})\\ \be^{\ddagger} \odot \D {\bf b}}=\bmatrix{\bm{\Phi}_{\mathbb{S}} & \bf 0 \\ \bf 0 & I_{\bm l}}\bmatrix{\vec_{\mathbb{S}}(\Psi^{\ddagger} \odot H)\\ \be^{\ddagger} \odot \D {\bf b}},
    \end{equation}
    where $\vec_{\mathbb{S}}(\Psi^{\ddagger} \odot \mathbf{H})=[(\psi_A^{\ddagger} \odot \D\bm{a})^{\bm\top}, (\Psi_B^{\ddagger}\odot \vec(\D B))^{\bm\top}, (\Psi_C^{\ddagger}\odot\vec(\D C))^{\bm\top} , (\psi_D^{\ddagger}\odot \D\bm{d})^{\bm\top} , (\psi_D^{\ddagger}\odot  \D\bm{e})^{\bm\top}]^{\bm\top}$ and
    
    $$\bm{\Phi}_{\mathbb{S}}=\bmatrix{\bm{\Phi}_{\mathbb{S}_1} & \bf 0 & \bf 0 & \bf 0\\\bf 0 & I_{mn+mp} & \bf 0 & \bf 0\\ \bf 0  & \bf 0 &\bm{\Phi}_{\mathbb{S}_2} & \bf 0 \\ \bf 0  & \bf 0  & \bf 0 &\bm{\Phi}_{\mathbb{S}_3} }.$$
\end{lemma}
\proof The proof follows immediately using identities in \eqref{eq:55} and \eqref{eq:56}. $\blacksquare$

In the following theorem, we present closed-form expressions for the structured partial CN by considering $\tau=\gamma=2.$
\begin{theorem}\label{Th:SNCN}
    Let $A\in \mathbb{S}_1,$ $D\in \mathbb{S}_2,$ $E\in \mathbb{S}_3$ and $\mathrm{L}\in \R^{k\times \bm{l}}.$ Suppose that ${\bm w}=[\x^{\bm\top},\y^{\bm\top},\z^{\bm\top}]^{\bm\top}$ is the unique solution of DSPP \eqref{eq1:SPP} and consider $\tau=\gamma=2.$ Then the structured partial CN of ${\bm w}=[\x^{\bm\top},\y^{\bm\top},\z^{\bm\top}]^{\bm\top}$ w.r.t. $\mathrm{L}$ is given by \begin{eqnarray*}
\mathfrak{K}^{(2),\mathbb{S}}_{\vp}(\mathbf{H}, \b; \mathrm{L})= \left\|\Theta_{\xi^{\ddagger}_{\mathrm{L}}}\mathrm{L}\mathfrak{B}^{-1}\bmatrix{ \mathcal{G} &  -I_{\bm l}} \bmatrix{\Theta_{\vec(\Psi)}\bm{\Phi}_{\mathbb{S}}\mathfrak{D}^{-1}_{\mathbb{S}}& \bf 0\\ \bf 0 & \Theta_{\be}}\right\|_2,
    \end{eqnarray*}
    where $$\mathfrak{D}_{\mathbb{S}}=\bmatrix{\Theta_{\bm{u}_1} & \bf 0 & \bf 0 & \bf 0\\ \bf 0 & I_{mn+mp}& \bf 0 & \bf 0 \\  \bf 0& \bf 0 & \Theta_{\bm{u}_2}& \bf 0 \\ \bf 0& \bf 0& \bf 0& \Theta_{\bm{u}_3} },$$
    $$\bm{u}_1=[\|\bm{\Phi}_{\mathbb{S}_1}(:,1)\|_2,\ldots, \|\bm{\Phi}_{\mathbb{S}_1}(:,s)\|_2 ]^{\bm\top},~~\bm{u}_2=[\|\bm{\Phi}_{\mathbb{S}_2}(:,1)\|_2,\ldots, \|\bm{\Phi}_{\mathbb{S}_2}(:,r)\|_2 ]^{\bm\top},$$
$$\text{and}~~\bm{u}_3=[\|\bm{\Phi}_{\mathbb{S}_3}(:,1)\|_2,\ldots, \|\bm{\Phi}_{\mathbb{S}_3}(:,q)\|_2 ]^{\bm\top}.$$
\end{theorem}
%
\proof Taking $\tau=\gamma=2$ in \eqref{def:SCN}, and using \eqref{th1:eq2} and \eqref{th1:eq1}, we obtain
\begin{eqnarray*}
         \mathfrak{K}^{(2),\mathbb{S}}_{\vp}(\mathbf{H}, \b; \mathrm{L})= \sup_{\substack{\left\|\vec\left(\Psi^{\ddagger} \odot \Delta \mathbf{H},\, \be^{\ddagger} \odot \D \b\right)\right\|_{2}\neq 0\\ \D A\in\, \mathbb{S}_1,\, \D D\in\, \mathbb{S}_2,\, \D E \in\, \mathbb{S}_3} } \frac{\left\|\Theta_{\xi^{\ddagger}_{\mathrm{L}}}\mathrm{L} 
 \mathfrak{B}^{-1} \bmatrix{ \mathcal{G} &  -I_{\bm l}} \bmatrix{\Theta_{\vec(\Psi)}& \bf 0\\ \bf 0 & \Theta_{\be}} \bmatrix{\vec(\Psi^{\ddagger} \odot \D \mathbf{H})\\ \bm{\chi}^{\ddagger} \odot \D {\bf b}}\right\|_2}{\left\|\vec\left(\Psi^{\ddagger} \odot \Delta \mathbf{H}, \be^{\ddagger} \odot \D \b\right)\right\|_{2}}.
    \end{eqnarray*}
    Substituting \eqref{eq:57} into the above equation yields
  {\footnotesize   \begin{equation}\label{eq:58}
         \mathfrak{K}^{(2),\mathbb{S}}_{\vp}(\mathbf{H}, \b; \mathrm{L})= \sup_{\substack{\footnotesize\left\|\bmatrix{\bm{\Phi}_{\mathbb{S}} & \bf 0 \\ \bf 0 & I_{\bm l}}\bmatrix{\vec_{\mathbb{S}}(\Psi^{\ddagger} \odot \D\mathbf{H})\\ \be^{\ddagger} \odot \D {\bf b}}\right\|_{2}\neq 0\\ \D A\in\, \mathbb{S}_1,\, \D D\in\, \mathbb{S}_2,\, \D E \in\, \mathbb{S}_3} } \frac{\left\|\Theta_{\xi^{\ddagger}_{\mathrm{L}}}\mathrm{L} 
 \mathfrak{B}^{-1} \bmatrix{ \mathcal{G} &  -I_{\bm l}} \bmatrix{\Theta_{\vec(\Psi)}\bm{\Phi}_{\mathbb{S}} & \bf 0\\ \bf 0 & \Theta_{\be}} \bmatrix{\vec_{\mathbb{S}}(\Psi^{\ddagger} \odot \D\mathbf{H})\\ \be^{\ddagger} \odot \D {\bf b}}\right\|_2}{\left\|\bmatrix{\bm{\Phi}_{\mathbb{S}} & \bf 0 \\ \bf 0 & I_{\bm l}}\bmatrix{\vec_{\mathbb{S}}(\Psi^{\ddagger} \odot \D\mathbf{H})\\ \be^{\ddagger} \odot \D {\bf b}}\right\|_{2}}.
    \end{equation}
    }
   % where $\Lambda=\bmatrix{\bm{\Phi}_{\mathbb{S}} & \bf 0 \\ \bf 0 & I_{\bm l}}\bmatrix{\vec_{\mathbb{S}}(\Psi^{\ddagger} \odot \mathbf{H})\\ \be^{\ddagger} \odot \D {\bf b}}.$
    
\noindent    Utilizing the fact that the matrices $\bm{\Phi}_{\mathbb{S}_i}$ for $i=1,2,3,$   are column orthogonal \cite{TLS2011Li}, we get $\bm{\Phi}_{\mathbb{S}_i}^{\bm\top}\bm{\Phi}_{\mathbb{S}_i}=\Theta_{\bm{u}_i}^2,$
where $\Theta_{\bm{u}_i}$ for $i=1,2,3,$ are the diagonal matrices.  
Then
\begin{align}\label{eq:49}
   \nonumber \left\|\bmatrix{\bm{\Phi}_{\mathbb{S}} & \bf 0 \\ \bf 0 & I_{\bm l}}\bmatrix{\vec_{\mathbb{S}}(\Psi^{\ddagger} \odot \D\mathbf{H})\\ \be^{\ddagger} \odot \D {\bf b}}\right\|_{2}&=\left\|\bmatrix{\vec_{\mathbb{S}}(\Psi^{\ddagger} \odot \D\mathbf{H})\\ \be^{\ddagger} \odot \D {\bf b}}^{\bm\top}\bmatrix{\bm{\Phi}_{\mathbb{S}}^{\bm\top}\bm{\Phi}_{\mathbb{S}} & \bf 0 \\ \bf 0 & I_{\bm l}}\bmatrix{\vec_{\mathbb{S}}(\Psi^{\ddagger} \odot \D\mathbf{H})\\ \be^{\ddagger} \odot \D {\bf b}}\right\|_2^{1/2}\\
    &= \left\|\bmatrix{\mathfrak{D}_{\mathbb{S}} & \bf 0 \\ \bf 0 & I_{\bm l}}\bmatrix{\vec_{\mathbb{S}}(\Psi^{\ddagger} \odot \D\mathbf{H})\\ \be^{\ddagger} \odot \D {\bf b}}\right\|_2. 
\end{align}
    Observe that 
    \begin{align}\label{eq:59}
\bmatrix{\bm{\Phi}_{\mathbb{S}} & \bf 0 \\ \bf 0 & I_{\bm l}}\bmatrix{\vec_{\mathbb{S}}(\Psi^{\ddagger} \odot \D\mathbf{H})\\ \be^{\ddagger} \odot \D {\bf b}}= \bmatrix{\bm{\Phi}_{\mathbb{S}}\mathfrak{D}^{-1}_{\mathbb{S}} & \bf 0 \\ \bf 0 & I_{\bm l}}\bmatrix{\mathfrak{D}_{\mathbb{S}} & \bf 0 \\ \bf 0 & I_{\bm l}} \bmatrix{\vec_{\mathbb{S}}(\Psi^{\ddagger} \odot \D \mathbf{H})\\ \be^{\ddagger} \odot \D {\bf b}}.
    \end{align}
    Therefore, substituting \eqref{eq:49} and \eqref{eq:59} in \eqref{eq:58}, we obtain
    {\footnotesize \begin{align*}\label{eq:510}
         \mathfrak{K}^{(2),\mathbb{S}}_{\vp}(\mathbf{H}, \b; \mathrm{L})&= \sup_{\substack{\footnotesize\left\|\bmatrix{\mathfrak{D}_{\mathbb{S}}\vec_{\mathbb{S}}(\Psi^{\ddagger} \odot \D\mathbf{H})\\ \be^{\ddagger} \odot \D {\bf b}}\right\|_{2}\neq 0\\ \D A\in\, \mathbb{S}_1,\, \D D\in\, \mathbb{S}_2,\, \D E \in\, \mathbb{S}_3} } \frac{\left\|\Theta_{\xi^{\ddagger}_{\mathrm{L}}}\mathrm{L} 
 \mathfrak{B}^{-1} \bmatrix{ \mathcal{G} &  -I_{\bm l}} \bmatrix{\Theta_{\vec(\Psi)}\bm{\Phi}_{\mathbb{S}}\mathfrak{D}_{\mathbb{S}}^{-1} & \bf 0\\ \bf 0 & \Theta_{\be}} \bmatrix{\mathfrak{D}_{\mathbb{S}}\vec_{\mathbb{S}}(\Psi^{\ddagger} \odot \D \mathbf{H})\\ \be^{\ddagger} \odot \D {\bf b}}\right\|_2}{\left\|\bmatrix{\mathfrak{D}_{\mathbb{S}}\vec_{\mathbb{S}}(\Psi^{\ddagger} \odot \D \mathbf{H})\\ \be^{\ddagger} \odot \D {\bf b}}\right\|_{2}}\\
 %
 &=\normalsize\left\|\Theta_{\xi^{\ddagger}_{\mathrm{L}}}\mathrm{L} 
 \mathfrak{B}^{-1} \bmatrix{ \mathcal{G} &  -I_{\bm l}} \bmatrix{\Theta_{\vec(\Psi)}\bm{\Phi}_{\mathbb{S}}\mathfrak{D}_{\mathbb{S}}^{-1} & \bf 0\\ \bf 0 & \Theta_{\be}} \right\|_2.
    \end{align*}
    }
    Hence, the proof is completed.
$\blacksquare$
%%%%%%%%%%%%%%%%%%%%%%%%%%%%%%%%%%%%%%%%%%%%%%%%%%%%%%%%%%%%%%%%%%%%%%%%%%

Next, we consider $\tau=\gamma=\infty,$ and derive the structured partial MCN and CCN for the DSPP.
%
\begin{theorem}\label{Th:MCN}
    Let $A\in \mathbb{S}_1,$ $D\in \mathbb{S}_2,$ $E\in \mathbb{S}_3$ and $\mathrm{L}\in \R^{k\times \bm{l}}.$ Suppose that ${\bm w}=[\x^{\bm\top},\y^{\bm\top},\z^{\bm\top}]^{\bm\top}$ is the unique solution of DSPP \eqref{eq1:SPP}.   When $\tau=\gamma=\infty,$ the structured partial CN of the solution ${\bm w}=[\x^{\bm\top},\y^{\bm\top},\z^{\bm\top}]^{\bm\top}$ 
 w.r.t. $\mathrm{L}$ is given as follows:
    \begin{align}
         \mathfrak{K}^{\infty,\,\mathbb{S}}_{\vp}(\mathbf{H}, \b; \mathrm{L}) = \left\| |\Theta_{\xi^{\ddagger}_{\mathrm{L}}}| \left|\mathrm{L}\mathfrak{B}^{-1} \bmatrix{ \mathcal{G} &  -I_{\bm l}} \bmatrix{\bm{\Phi}_{\mathbb{S}} & \bf 0\\ \bf 0& I_{\bm l}}\right| \bmatrix{\vec_{\mathbb{S}}(|\Psi|)\\  |\be|}\right\|_{\infty}.
    \end{align}
    Moreover, we consider $\Psi=\mathbf{H},$ $\be ={\bf b}.$ Then, if we set $\xi_{\mathrm{L}}=\|\mathrm{L}[\bm{x}^{\bm\top}, \bm{y}^{\bm\top}, \bm{z}^{\bm\top}]^{\bm\top}\|_{\infty}{\bf{1}}_{\bm l},$  the structured  partial MCN 
  is given by 
   \begin{align}
         \mathfrak{K}^{\infty,\,\mathbb{S}}_{mix,  \vp}(\mathbf{H}, \b; \mathrm{L}) = \frac{\left\| \left|\mathrm{L}\mathfrak{B}^{-1}\bmatrix{ \mathcal{G} &  -I_{\bm l}} \bmatrix{\bm{\Phi}_{\mathbb{S}} & \bf 0\\ \bf 0& I_{\bm l}} \right| \bmatrix{\vec_{\mathbb{S}}(|\mathbf{H}|)\\ |{\bf b}| }\right\|_{\infty}}{\|\mathrm{L}[\bm{x}^{\bm\top}, \bm{y}^{\bm\top}, \bm{z}^{\bm\top}]^{\bm\top}\|_{\infty}}
    \end{align}
    and if we set $\xi_{\mathrm{L}}= \mathrm{L}[\bm{x}^{\bm\top}, \bm{y}^{\bm\top}, \bm{z}^{\bm\top}]^{\bm\top},$  the  structured partial CCN is given by
    \begin{align}
         \mathfrak{K}^{\infty,\, \mathbb{S}}_{com, \vp}(\mathbf{H}, \b; \mathrm{L}) = \left\|\frac{ \left|\mathrm{L} \mathfrak{B}^{-1} \bmatrix{ \mathcal{G} &  -I_{\bm l}} \bmatrix{\bm{\Phi}_{\mathbb{S}} & \bf 0\\ \bf 0& I_{\bm l}} \right| \bmatrix{\vec_{\mathbb{S}}(|\mathbf{H}|)\\ |{\bf b}| }}{\left|\mathrm{L}[\bm{x}^{\bm\top}, \bm{y}^{\bm\top}, \bm{z}^{\bm\top}]^{\bm\top}\right|}\right\|_{\infty}.
    \end{align}
\end{theorem}
%
\proof By construction of the matrices $\bm{\Phi}_{\mathbb{S}_1},$ $\bm{\Phi}_{\mathbb{S}_2}$ and $\bm{\Phi}_{\mathbb{S}_3},$ they have almost one nonzero element in each row. Thus, we get
\begin{equation}\label{eq:513}
 \left\| \bmatrix{\vec(\Psi^{\ddagger} \odot\D \mathbf{H})\\ \bm{\chi}^{\ddagger} \odot \D {\bf b}}\right\|_{\infty}= \left\|\bmatrix{\bm{\Phi}_{\mathbb{S}} & \bf 0\\ \bf 0 & I_{\bm l}}  \bmatrix{\vec_{\mathbb{S}}(\Psi^{\ddagger} \odot\D \mathbf{H})\\ \bm{\chi}^{\ddagger} \odot \D {\bf b}}\right\|_{\infty}= \left\| \bmatrix{\vec_{\mathbb{S}}(\Psi^{\ddagger} \odot\D \mathbf{H})\\ \bm{\chi}^{\ddagger} \odot \D {\bf b}} \right\|_{\infty}.
\end{equation}
%
By considering $\tau=\gamma=\infty$ on \eqref{def:SCN} and using \eqref{eq:513}, \eqref{th1:eq2} and \eqref{th1:eq1}, we obtain
 \begin{align*}
         \mathfrak{K}^{\infty,\mathbb{S}}_{\vp}(\mathbf{H}, \b; \mathrm{L})&= \sup_{\substack{\left\|\vec\left(\Psi^{\ddagger} \odot \Delta \mathbf{H},\, \be^{\ddagger} \odot \D \b\right)\right\|_{\infty}\neq 0\\ \D A\in\, \mathbb{S}_1,\, \D D\in\, \mathbb{S}_2,\, \D E \in\, \mathbb{S}_3} } \frac{\left\|\Theta_{\xi^{\ddagger}_{\mathrm{L}}}\mathrm{L} 
 \mathfrak{B}^{-1} \bmatrix{ \mathcal{G} &  -I_{\bm l}} \bmatrix{\bm{\Phi}_{\mathbb{S}} & \bf 0\\ \bf 0 & I_{\bm l}} \bmatrix{\vec_{\mathbb{S}}( \D \mathbf{H})\\  \D {\bf b}}\right\|_{\infty}}{\left\|\vec\left(\Psi^{\ddagger} \odot \Delta \mathbf{H}, \be^{\ddagger} \odot \D \b\right)\right\|_{\infty}}\\
 %
 &= \sup_{\footnotesize\substack{\left\|\bmatrix{\vec_{\mathbb{S}}(\Psi^{\ddagger} \odot \D \mathbf{H})\\ \bm{\chi}^{\ddagger} \odot \D {\bf b}}\right\|_{\infty}\neq 0\\ \D A\in\, \mathbb{S}_1,\, \D D\in\, \mathbb{S}_2,\, \D E \in\, \mathbb{S}_3} } \frac{\left\|\Theta_{\xi^{\ddagger}_{\mathrm{L}}}\mathrm{L} 
 \mathfrak{B}^{-1} \bmatrix{ \mathcal{G} &  -I_{\bm l}}  \bmatrix{\bm{\Phi}_{\mathbb{S}} & \bf 0\\ \bf 0 & I_{\bm l}} \bmatrix{\Theta_{\vec_{\mathbb{S}}(\Psi)}& \bf 0\\ \bf 0 & \Theta_{\be}}\bmatrix{\vec_{\mathbb{S}}(\Psi^{\ddagger} \odot\D \mathbf{H})\\ \bm{\chi}^{\ddagger} \odot \D {\bf b}}\right\|_{\infty}}{\left\|\bmatrix{\vec_{\mathbb{S}}(\Psi^{\ddagger} \odot \D \mathbf{H})\\ \bm{\chi}^{\ddagger} \odot \D {\bf b}}\right\|_{\infty}}\\
 %
 &=\left\|\Theta_{\xi^{\ddagger}_{\mathrm{L}}}\mathrm{L} 
 \mathfrak{B}^{-1} \bmatrix{ \mathcal{G} &  -I_{\bm l}} \bmatrix{\bm{\Phi}_{\mathbb{S}} & \bf 0\\ \bf 0 & I_{\bm l}}\bmatrix{\Theta_{\vec_{\mathbb{S}}(\Psi)}& \bf 0\\ \bf 0 & \Theta_{\be}} \right\|_{\infty}\\
 &=\left\||\Theta_{\xi^{\ddagger}_{\mathrm{L}}}| \left|\mathrm{L}\mathfrak{B}^{-1} \bmatrix{ \mathcal{G} &  -I_{\bm l}} \bmatrix{\bm{\Phi}_{\mathbb{S}} & \bf 0\\ \bf 0& I_{\bm l}}\right| \bmatrix{\vec_{\mathbb{S}}(|\Psi|)\\  |\be|}\right\|_{\infty}.
    \end{align*}

    The rest of the proof follows by considering $\Psi=\mathbf{H},$ $\be ={\bf b},$ $\xi_{\mathrm{L}}=\|\mathrm{L}[\bm{x}^{\bm\top}, \bm{y}^{\bm\top}, \bm{z}^{\bm\top}]^{\bm\top}\|_{\infty}{\bf{1}}_{\bm l}$ (or $\xi_{\mathrm{L}}= \mathrm{L}[\bm{x}^{\bm\top}, \bm{y}^{\bm\top}, \bm{z}^{\bm\top}]^{\bm\top}$). 
$\blacksquare $

%%%%%%%%
 
 \section{Deduction of Partial CNs for the EILS problem}\label{sec:EILS}
The EILS problem is an extension of the famous linear least squares problem, having linear constraints on unknown parameters. It  can be expressed as follows:
\begin{equation}\label{EILS}
    \min_{\y\in \R^m}(b-M\y)^{\bm\top}\mathbb{J}~(b-M\y)\quad \text{subjected to}\quad C\y=d,
\end{equation}
where $M\in \R^{n\times m} (n\geq m),$ $C\in \R^{p\times m},$ $b\in \R^n$ $d\in \R^p$ and the signature matrix $ \mathbb{J}$ given by
\begin{equation}
    \mathbb{J}=\bmatrix{I_{n_1} & \bf 0\\ \bf 0 & -I_{n_2}}, ~ n_1+n_2=n.
\end{equation}
When $\rank(C)=p$ and $\y^{\bm\top}(M^{\bm\top}\mathbb{J}M)\y>0$ for all nonzero $\y\in \text{null}(C),$ the EILS problem \eqref{EILS} has a unique solution. 
The solution of the EILS \eqref{EILS} problem also satisfies the following the augmented system \cite{ILSE2022}:
\begin{equation}\label{EILS2}
   \widehat{\mathfrak{B}}\bmatrix{\lambda \\ \x \\ \y}:= \bmatrix{\bf 0 & \bf 0 & C\\ \bf 0 & \mathbb{J} & M\\ C^{\bm\top} & M^{\bm\top}& \bf 0 }\bmatrix{\lambda \\ \x \\ \y}=\bmatrix{d\\b\\ \bf 0},
\end{equation}
where $\x=\mathbb{J}{\bm r},$ ${\bm r}=b-M\y$ and $\lambda=(CC^{\bm\top})^{-1}CM^{\bm\top}\mathbb{J}{\bm r}$ is the vector of Lagrange multipliers \cite{EILSAlgorithms}. 
Note that the system in \eqref{EILS2} can be equivalently transformed into 
\begin{align}\label{EILS3}
    \bmatrix{\mathbb{J} & M & \bf 0 \\ M^{\bm\top} & \bf 0 & C^{\bm\top}\\ \bf 0 & C & \bf 0}\bmatrix{\x \\ \y \\ \lambda}=\bmatrix{ b\\\bf 0\\ d }=:\widehat{\bf d}.
\end{align}
Observe that, the above system is in the form of DSPP \eqref{eq1:SPP} with $A=\mathbb{J},$ $B=M^{\bm\top},$ ${\bf b}=[b^{\bm\top}, {\bf 0}, d^{\bm\top}]^{\bm\top}$ and $\z=\lambda.$
% Therefore, finding the CNs for the EILS problem is equivalent to finding the CNs for the solution $\y$ of the transformed system \eqref{EILS3}.
%Thus, obtaining the CNs for the EILS problem reduces to finding the CNs for the solution \(\mathbf{y}\) of the transformed system \eqref{EILS3}.
Therefore, the task of assessing the conditioning of the EILS problem \eqref{EILS} can be achieved by determining the CNs for the solution $\y$ of DSPP \eqref{EILS3}.

Generally, the signature matrix $\mathbb{J}$ has no perturbation and as $D={\bf 0},$ $E={\bf 0}$ and ${\bf b}_2={\bf 0}$, we consider $\D A=\0$, $\D D=\0,$ $\D E=0$  and  $\D {\bf b}_2={\bf 0}$ in \eqref{eq:pert}. %Hence,  $\Psi_{A}=\0,$ $\Psi_{D}=\0,$ $\Psi_{E}=\0$ and we obtain the following result. 
Then, the perturbation expression in \eqref{eq27} reduces to
\begin{eqnarray}\label{eq45}
  \bmatrix{\D\bm{x}\\ \D\bm{z}\\ \D\bm{y}}= -\mathfrak{B}^{-1} \bmatrix{\widehat{\mathcal{G}} &  -I_{n+p}}\bmatrix{\vec(\D B^{\T})\\ \vec(\D C)\\ \D {\bf b}_1 \\ \D {\bf b}_3} + \mathcal{O}(\bm{\epsilon}^2),
\end{eqnarray}
 where \begin{equation}
    \widehat{\mathcal{G}}= \bmatrix{ \bm{y}^{\bm\top}\otimes I_n& \bf 0\\  I_m\otimes \bm{x}^{\bm\top} & I_m\otimes \bm{z}^{\bm\top} \\ \bf 0& \bm{y}^{\bm\top}\otimes I_p }\in \R^{\bm{l}\times \hat{\bm s}},
 \end{equation}
 $\hat{\bm s}=m(n+p).$

 %
Let $\widehat{\mathbf{H}}=(B^{\T},C),$ $\D\widehat{\mathbf{H}}=(\D B^{\T}, \D C),$ $\widehat{\bf b}=[{\bf b}_1^{\bm\top}, {\bf b}_3^{\bm\top} ]^{\bm\top}$ and $\D \widehat{\bf b}=[\D {\bf b}_1^{\bm\top}, \D{\bf b}_3^{\bm\top} ]^{\bm\top}$.
We define the following mapping:
\begin{align}\label{eq:map2}
\nonumber \widetilde{\vp}:\,&   \R^{m\times n}\times  \R^{p\times m}  \times \R^{n+p  } \rightarrow \R^k\\
%
&~~\widetilde{\vp}(\widehat{\mathbf{H}}, \widehat{\mathbf{b}})=\mathrm{L}\bmatrix{\x\\ \y \\ \z}=\mathrm{L}\mathfrak{B}^{-1} \bmatrix{{\bf b}_1\\ \0\\  {\bf b}_3},
\end{align}
where $\mathrm{L}\in \R^{k\times \bm l}.$ Using a similar method to the Theorem \ref{th1}, we have the following result.
 \begin{theorem}\label{Th41}
 Assume that  $[\bm{x}^{\bm\top}, \bm{y}^{\bm\top}, \bm{z}^{\bm\top}]^{\bm\top}$ is the unique solution of the DSPP \eqref{eq1:SPP} with $D=\0,$ $E=\0$,  ${\bf b}_2={\bf 0}$ and $\mathrm{L}\in \R^{k\times \bm{l}}.$ Then the partial unified CN of $[\bm{x}^{\bm\top}, \bm{y}^{\bm\top}, \bm{z}^{\bm\top}]^{\bm\top}$ w.r.t. $\mathrm{L}$ is given by
    \begin{align*}
         \mathfrak{K}_{\widetilde{\vp} }(\widehat{\mathbf{H}}, \b; \mathrm{L})&:=\lim_{\bm{\epsilon}\rightarrow 0} \sup_{0<\left\|\vec\left(\widehat{\Psi}^{\ddagger} \odot \Delta \widehat{\mathbf{H}},\, \widehat{\be}^{\ddagger} \odot \D \b\right)\right\|_{\tau}\leq \bm{\epsilon}} \frac{\left\|\xi^{\ddagger}_{\mathrm{L}}\odot \left( \widetilde{\bm{\varphi}}(\widehat{\mathbf{H}}+\D \widehat{\mathbf{H}}, \widehat{\mathbf{b}}+ \D \widehat{\mathbf{b}})-\widetilde{\vp}(\widehat{\mathbf{H}}, \widehat{\b})\right) \right\|_\gamma}{\left\|\vec\left(\widehat{\Psi}^{\ddagger} \odot \Delta \widehat{\mathbf{H}}, \widehat{\be}^{\ddagger} \odot \D \widehat{\b}\right)\right\|_{\tau}},\\
         %
         &\, = \left\|\Theta_{\xi^{\ddagger}_{\mathrm{L}}}\mathrm{L}\mathfrak{B}^{-1}\bmatrix{ \widehat{\mathcal{G}} &  -I_{n+p}} \bmatrix{\Theta_{\vec(\widehat{\Psi})}& \bf 0\\ \bf 0 & \Theta_{\widehat{\be}}}\right\|_{\tau,\gamma},
             \end{align*}
    where $\widehat{\Psi}= (\Psi_{B^{\T}},\Psi_C),$ $\Psi_{B^{\T}}\in \R^{n\times m}, \Psi_C\in \R^{p\times m}$ and $\widehat{\be}\in \R^{n+p}.$ %$\|\cdot~\|_{\tau,\gamma}$ is the matrix norm induced by vector norms $\|\cdot~\|_{\tau} $ and $\|\cdot~\|_{\gamma}.$
 \end{theorem}

\begin{remark}
Taking $\mathrm{L}=[{\0}_{k\times n} ~~ \mathrm{L}_1~ ~{\0}_{k\times p}],$ $\mathrm{L}_1\in \R^{k\times m},$ $A=\mathbb{J}$  in Theorem \ref{Th41}, and since $\mathfrak{B} = \Sigma \widehat{\mathfrak{B}} \Sigma^{-1},$ where
\begin{equation}
    \Sigma=\bmatrix{\0&I_n &\0  \\ \0 & \0 &I_m\\ I_p &\0& \0},
\end{equation}
using the formula for the inverse of $\widehat{\mathfrak{B}}$ given in \cite{EILS2010}, we obtain
\begin{align}
       \mathfrak{K}_{\widetilde{\vp} }(\widehat{\mathbf{H}}, \b; \mathrm{L})=\left\|\Theta_{\xi^{\ddagger}_{\mathrm{L}}} \mathrm{L}_1 \bmatrix{\Xi& \Lambda&  -(\mathcal{Q}\mathcal{M}\mathcal{Q})^{\dagger}MJ & \mathcal{B}_M } \bmatrix{\Theta_{\vec(\widehat{\Psi})}& \bf 0\\ \bf 0 & \Theta_{\widehat{\be}}} \right\|_{\tau,\gamma},
\end{align}
where $\Xi=\y^{\bm\top} \otimes (\mathcal{Q}\mathcal{M}\mathcal{Q})^{\dagger}M\mathbb{J}- (\mathcal{Q}\mathcal{M}\mathcal{Q})^{\dagger}\otimes \x^{\bm\top},$ $\Lambda=\y^{\bm\top}\otimes \mathcal{B}_{M}-(\mathcal{Q}\mathcal{M}\mathcal{Q})^{\dagger}\otimes \z^{\bm\top},$ $\mathcal{M}=M\mathbb{J}M^{\bm\top},$ $\mathcal{Q}=I_m-C^{\dagger}C$ and $\mathcal{B}_M=(I_m-\mathcal{Q}\mathcal{M}\mathcal{Q})^{\dagger}.$ Note that the partial CN expression is the same as derived in \cite{ILSE2022}.
    
\end{remark}

\section{Numerical experiments}\label{sec:Numerical}
In this part, we present some numerical examples to verify the reliability of the derived the partial NCN, MCN and CCN and their upper bounds for the DSPP. Additionally, we demonstrate their effectiveness in providing tight upper bounds for the relative forward error in solving the DSPP.

We construct the entrywise perturbation as follows:
\begin{align*}
   & \D A=10^{-s}\cdot \mathtt{randn}(n,n)\odot A,~  \D B=10^{-s}\cdot \mathtt{randn}(m,n)\odot B,~ \D C=10^{-s}\cdot \mathtt{randn}(p,m)\odot C,\\
    & \D D=10^{-s}\cdot \mathtt{randn}(m, m)\odot D,~  \D E=10^{-s}\cdot \mathtt{randn}(p,p)\odot E,~ \D{\bf b}=10^{-s}\cdot \mathtt{randn}(\bm{l}, 1)\odot{\bf b},
\end{align*}
where $\mathtt{randn}(m,n)$ denotes $m\times n$ the random matrices generated by MATLAB command $\mathtt{randn}$. Let $\bm{w}= [\x^{\T},\y^{\T},\z^{\bm\top}]^{\T} $ and $\widetilde{\bm{w}}=[\widetilde{\x}^{\T}, \widetilde{\y}^{\T}, \widetilde{\z}^{\T}]^{\T}$ be the unique solutions of the original DSPP and the perturbed DSPP, respectively. To estimate an upper bound for the forward error in the solution, the normwise, mixed and componentwise relative forward errors  are defined as follows:
\begin{align*}
		\bm{r}_k=\frac{\|\mathrm{L}\widetilde{\bm{w}}-\mathrm{L}\bm{w}\|_2}{\|\mathrm{L}\bm{w}\|_2}, \quad  \bm{r}_m=\frac{\|\mathrm{L}\widetilde{\bm{w}}-\mathrm{L}\bm{w}\|_{\infty}}{\|\mathrm{L}\bm{w}\|_{\infty}}  \quad \text{and} ~~\bm{r}_c=\left\|\frac{\mathrm{L}\widetilde{\bm{w}}-\mathrm{L}\bm{w}}{\mathrm{L}\bm{w}}\right\|_{\infty},
	\end{align*}
 respectively.
 %
 Define the following quantities:
\begin{equation}
    \bm{\epsilon}_1=\frac{\left\|\bmatrix{\D \mathfrak{B}& \D \mathbf{b}}\right\|_F}{\left\|\bmatrix{\mathfrak{B} & \mathbf{b}}\right\|_F}~~ \text{and}~~\bm{\epsilon}_2=\min\left\{\bm{\epsilon}: \left|\D \mathfrak{B}\right|\leq \bm{\epsilon}\left|\mathfrak{B} \right|,~ \left|\D \mathbf{b}\right|\leq \bm{\epsilon}\left|\mathbf{b}\right|\right\}.
\end{equation}
Thus from \eqref{eq:forward}, $\bm{\epsilon}_1\mathfrak{K}^{(2)}_{\vp}(\mathbf{H}, \b; \mathrm{L}),$ $\bm{\epsilon}_2 \mathfrak{K}^{\infty}_{mix, \vp}(\mathbf{H}, \b; \mathrm{L})$ and $\bm{\epsilon}_2 \mathfrak{K}^{\infty}_{com, \vp}(\mathbf{H}, \b; \mathrm{L})$ can be employed to estimate the relative forward errors $\bm{r}_k,$ $\bm{r}_m$ and $\bm{r}_c,$ respectively. We select the matrix \(\mathrm{L}\) as $$\mathrm{L}_0=I_{\bm l}, ~ \mathrm{L}_n=\bmatrix{ I_n & \mathbf{0}_{n\times (m+p)}},~ \mathrm{L}_m=\bmatrix{{\0}_{m\times n}& I_m & \mathbf{0}_{m\times p}} ~\text{and} ~ \mathrm{L}_p=\bmatrix{{\0}_{p\times (n+m)}& I_p }$$ to obtain the \textit{CNs} for $\bm{w}$, \(\x\), $\y$ and $\z$, respectively.

\vspace{2mm}
\begin{exam}\label{exam1}
     We consider the DSPP \eqref{eq1:SPP} taken from \cite{HuangNA} with
		\begin{eqnarray*}
			& A= \bmatrix{I_q\otimes J+J\otimes I_q &\bf 0\\ \bf 0&I_q\otimes J+ J\otimes I_q}\in \R^{2q^2\times 2 q^2}, ~~  \\
	  &B=\bmatrix{I_q\otimes Z& Z\otimes I_q}\in \R^{q^2\times 2q^2},~ C= Y\otimes Z\in \R^{q^2\times q^2},
		\end{eqnarray*}
		where $J=\frac{1}{(q+1)^2}\, \mathtt{tridiag}(-1,2,-1)\in \R^{q\times q},$ $ Z=\frac{1}{q+1}\,  \mathtt{tridiag}(0,1,-1)\in \R^{q\times q}$  and  $Y=\diag(1, q+1, \ldots, q^2-q+ 1)\in \R^{q\times q}.$ Further, we take $D=I_m$ and $E=I_p.$ The notation $ \mathtt{tridiag}(d_1,d_2,d_3)\in \R^{q\times q}$ denotes the 
 tridiagonal matrix with subdiagonal entries $d_1$, diagonal entries $d_2$ and superdiagonal entries $d_3$. Here, the dimension of the coefficient matrix $\mathfrak{B}$ of the DSPP is ${\bm l}=4q^2.$   We take ${\bf b}=\tt{randn}(n,1)\in \R^{\bm l}$. Further, we consider $\Psi=\|\mathfrak{B}\|_F$ and ${\bm \chi}=\|{\bf b}\|_2.$ We use Theorems \ref{th32}, \ref{th33} to compute partial  CNs and Corollaries \ref{cor31} and \ref{coro32} for their upper bounds. The numerical results for different choices for $\mathrm{L}$ and $q=4:2:16$ are presented in Tables \ref{tab1}-\ref{tab4}.
 %
  
%%%%%
 \begin{sidewaystable}
	\centering
		\caption{Comparison of the NCN, MCN, and CCN, and their upper bounds, with the corresponding relative errors for   \(\mathrm{L} = \mathrm{L}_0\) for Example \ref{exam1}.}
 % \begin{center}\scalebox{0.75}{
			\label{tab1}
  % \resizebox{16.5cm}{!}{
				\begin{tabular}{@{}cccccccccc@{}}
					\toprule
			$q$	& $\bm{r}_k$ &$\mathfrak{K}_2$	& $\mathfrak{K}_2^U$ & $\bm{r}_m$ &$\mathfrak{K}_m$& $\mathfrak{K}_m^U$ & $\bm{r}_c$ &  $\mathfrak{K}_c$& $\mathfrak{K}_c^U$ \\
  \midrule 
   4 &$1.1882E-08$&$5.1234E-06$&	$6.3888E-06$&	$1.9907E-08$&	$3.5552E-07$&	$3.8637E-07$&	$2.1532E-07$&	$6.9184E-06$&	$8.6300E-06$\\[1ex]
   $6$ &$2.0426E-08$&	$8.2886E-06$&	$1.6984E-05$&	$2.3603E-08$&	$3.9780E-07$&	$4.0885E-07$&	$3.7248E-07$&	$1.2135E-05$&	$1.2490E-05$\\[1ex]
     $8$ &$3.1951E-08$&	$1.9154E-05$&	$3.6586E-05$&	$4.1045E-08$&	$4.9025E-07$&	$4.9808E-07$&	$8.3715E-06$&	$1.3746E-04$&	$1.3964E-04$\\	[1ex]
    $10$ &$2.9187E-08$&	$5.5170E-05$&	$1.1519E-04$&	$4.2771E-08$&	$1.2319E-06$&	$1.2706E-06$&	$1.8721E-07$&	$1.9213E-05$&	$2.1515E-05$\\[1ex]
     $12$ &$2.3289E-08$&	$9.4616E-05$&	$1.7034E-04$&	$2.9289E-08$&	$1.0722E-06$&	$1.0798E-06$&	$4.5669E-07$&	$3.5850E-05$&	$3.8136E-05$\\[1ex]
   $14$ & $3.9431E-08$&	$1.6571E-04$&	$3.3889E-04$&	$3.9739E-08$&	$1.2469E-06$&	$1.2747E-06$&	$6.0883E-07$&	$3.1225E-05$&$	3.5558E-05$\\[1ex]
  $16$ &$3.9038E-08$&	$2.7496E-04$&	$5.6909E-04$&	$3.9932E-08$&	$1.8001E-06$&	$1.8356E-06$&	$4.2789E-06$&	$1.7386E-04$&$	1.8883E-04$\\[1ex]
   \bottomrule   
    \end{tabular} 
       %}
 % \end{center}
    %}
    \end{sidewaystable}
%%
%begin{landscape}
     \begin{sidewaystable}
			\centering
		\caption{Comparison of the NCN, MCN, and CCN, and their upper bounds, with the corresponding relative errors for   \(\mathrm{L} = \mathrm{L}_n\) for Example \ref{exam1}.}
			\label{tab2}
   %\resizebox{16.5cm}{!}{
				\begin{tabular}{cccccccccc}
					\toprule
			$q$	& $\bm{r}_k$ &$\mathfrak{K}_2$	& $\mathfrak{K}_2^U$ & $\bm{r}_m$ &$\mathfrak{K}_m$& $\mathfrak{K}_m^U$ & $\bm{r}_c$ &  $\mathfrak{K}_c$& $\mathfrak{K}_c^U$ \\
  \midrule 
   $4$ &$2.3210E-08$&	$1.7899E-06$&	$2.2325E-06$&	$4.4418E-08$	&$3.9328E-07$	&$3.9952E-07$&	$6.1112E-08$&	$7.7010E-07$	&$7.9196E-07$\\[1ex]
   $6$ &$2.2133E-08$&	$2.6980E-06$&	$7.6732E-06$&	$4.2425E-08$&	$8.8875E-07$&	$9.5134E-07$&	$5.3918E-08$&	$5.8823E-06$&	$7.3855E-06$\\[1ex]
     $8$ &$3.8470E-08$&	$4.8135E-06$&	$1.5701E-05$&	$8.7676E-08$&	$1.4710E-06$	&$1.5322E-06$	&$5.7409E-07$	&$5.8895E-05$&	$7.0437E-05$\\	[1ex]
    $10$ &$2.9894E-08$&	$9.6923E-06$&	$3.1204E-05$&	$5.9374E-08$	&$2.0857E-06$	&$2.1065E-06$&	$2.7350E-07$&	$9.5571E-05$	&$1.0598E-04$\\[1ex]
     $12$& $3.3113E-08$&$3.6464E-05$&	$4.9737E-05$&	$5.6822E-08$&	$3.0914E-06$&	$3.0928E-06$&	$4.5080E-07$&	$9.3747E-06$&	$9.4743E-06$\\[1ex]
   $14$ & $8.0906E-08$&	$2.4824E-05$	&$7.6132E-05$	&$1.5910E-07$&	$3.3988E-06$&	$3.4220E-06$&	$8.4585E-07$&	$2.5939E-05$&	$2.9725E-05$\\[1ex]
  $16$ &$4.4663E-08$&	$3.5262E-05$&	$8.8133E-05$	&$8.0314E-08$&	$4.7189E-06$&	$4.7301E-06$	&$4.5566E-07$	&$3.7542E-05$	&$3.9779E-05$\\[1ex]
   \bottomrule
    \end{tabular}
    %}
    %
    \end{sidewaystable}
  % \end{landscape}
    %%%
    
    \begin{sidewaystable}
	\footnotesize		\centering
		\caption{Comparison of the NCN, MCN, and CCN, and their upper bounds, with the corresponding relative errors for   \(\mathrm{L} = \mathrm{L}_m\) for Example \ref{exam1}.}
			\label{tab3}
  % \resizebox{16.5cm}{!}{
				\begin{tabular}{cccccccccc}
					\toprule
			$q$	& $\bm{r}_k$ &$\mathfrak{K}_2$	& $\mathfrak{K}_2^U$ & $\bm{r}_m$ &$\mathfrak{K}_m$& $\mathfrak{K}_m^U$ & $\bm{r}_c$ &  $\mathfrak{K}_c$& $\mathfrak{K}_c^U$  \\
  \midrule 
   $4$ &$1.4640E-08$&	$4.1828E-06$&	$8.1301E-06$	&$1.7030E-08$&	$3.2527E-07$&	$3.6699E-07$&	$6.3575E-08$	&$1.6374E-06$&	$1.7734E-06$\\[1ex]
   $6$& $3.8057E-08$&	$3.4032E-05$&	$5.4748E-05$&	$3.9292E-08$	&$8.5355E-07$	&$1.0120E-06$	&$6.8320E-08	$&$2.2698E-06$	&$2.3143E-06$\\[1ex]
     $8$ &$6.2425E-08$	&$1.1653E-04$&	$1.7568E-04$&	$5.3397E-08$&	$1.9931E-06$&	$2.1774E-06$&	$3.0610E-07$	&$1.3012E-05$	&$1.3193E-05$\\	[1ex]
    $10$ &$4.7836E-08$&	$1.5526E-04$&	$3.3655E-04$&	$5.1330E-08$&	$1.8322E-06$&	$2.2332E-06$&	$5.0917E-07$&	$2.8394E-05$&	$2.8957E-05$\\[1ex]
     $12$&$ 3.7645E-08$ &	$9.3405E-05$&	$4.7395E-04$	&$3.6702E-08$	&$1.1129E-06$&	$1.4824E-06$	&$7.7646E-08$	&$7.9413E-06$&	$8.8422E-06$\\[1ex]
   $14$ & $7.7970E-08$&	$1.6058E-04$&	$9.2172E-04$&	$5.7666E-08$&	$1.7296E-06$&	$2.1704E-06$&	$2.9851E-06$&	$7.6483E-05$	&$7.9562E-05$\\[1ex]
  $16$ &$4.7259E-08$&	$4.3435E-04$&	$1.5265E-03$&	$6.6721E-08$&	$2.2484E-06$	& $2.8853E-06$&	$2.9895E-07$&	$5.8651E-05$&	$6.0528E-05$\\[1ex]
   \bottomrule
     
   
    \end{tabular}
  %  }
    \end{sidewaystable}
    %

     \begin{sidewaystable}
		\centering
		\caption{Comparison of the NCN, MCN, and CCN, and their upper bounds, with the corresponding relative errors for   \(\mathrm{L} = \mathrm{L}_p\) for Example \ref{exam1}.}
			\label{tab4}
 %  \resizebox{16.5cm}{!}{
				\begin{tabular}{@{}cccccccccc@{}}
					\toprule
			$q$	& $\bm{r}_k$ &$\mathfrak{K}_2$	& $\mathfrak{K}_2^U$ & $\bm{r}_m$ &$\mathfrak{K}_m$& $\mathfrak{K}_m^U$ & $\bm{r}_c$ &  $\mathfrak{K}_c$& $\mathfrak{K}_c^U$ \\
  \midrule 
   $4$ &$7.9587E-09$	& $3.2626E-06$ &	$5.0845E-06$	& $9.2868E-09$	&$2.7450E-07$&	$2.8678E-07$&	$6.4428E-08$&	$5.9782E-07$	&$6.2833E-07$\\[1ex]
   $6$&$1.3913E-08$&$	9.6919E-06$&	$1.1609E-05$&	$1.8364E-08$&	$4.4130E-07$	&$4.4401E-07$&	$4.1485E-08$&	$1.1621E-06$	&$1.1751E-06$\\[1ex]
     $8$ &$2.1543E-08$&	$2.5173E-05$&	$4.6780E-05$	&$2.7719E-08$&	$7.9552E-07$&	$8.4231E-07$&	$4.9306E-08$&	$4.2242E-06$&	$4.4848E-06$\\	[1ex]
    $10$ &$3.2132E-08$&	$5.7696E-05$&	$1.1250E-04$&	$4.3437E-08$&	$1.0618E-06$	&$1.1161E-06$&	$8.3724E-08$&	$4.1932E-06$&	$4.3258E-06$\\[1ex]
     $12$&$1.6735E-08$&	$8.7652E-05$&	$1.4248E-04$&	$3.1936E-08$&	$1.0011E-06$&	$1.0138E-06$&	$6.2623E-08$&	$2.5824E-06$	&$2.5860E-06$\\[1ex]
   $14$ & $2.0065E-08$&	$1.5497E-04$&	$1.9181E-04$&	$2.2200E-08$&	$1.0421E-06$&	$1.0457E-06$&	$7.7378E-08$&	$4.1703E-06$&	$4.1741E-06$\\[1ex]
  $16$ &$1.0902E-08$&	$2.4487E-04$	&$3.5342E-04$	&$1.5817E-08$	&$1.6369E-06$&$	1.6456E-06$&	$1.0229E-07$&	$3.4527E-06$&	$3.4614E-06$\\[1ex]
   \bottomrule
     
   
    \end{tabular}
%    }
    \end{sidewaystable}
  For simplicity,  we denote:
    \begin{align*}
        &\mathfrak{K}_2=\bm{\epsilon}_1\mathfrak{K}^{(2)}_{\vp}(\mathbf{H}, \b; \mathrm{L}_m), ~\mathfrak{K}_2^U=\bm{\epsilon}_1\mathfrak{K}^{(2),u}_{\vp}(\mathbf{H}, \b; \mathrm{L}_m),~
        \mathfrak{K}_m=\bm{\epsilon}_2\mathfrak{K}^{\infty}_{mix,  \vp}(\mathbf{H}, \b; \mathrm{L}_m),\\
       & \mathfrak{K}_m^U=\bm{\epsilon}_2\mathfrak{K}^{\infty,u}_{mix,  \vp}(\mathbf{H}, \b; \mathrm{L}_m),~
        \mathfrak{K}_c=\bm{\epsilon}_2\mathfrak{K}^{\infty}_{com,  \vp}(\mathbf{H}, \b; \mathrm{L}_m),~
        \mathfrak{K}_c^U=\bm{\epsilon}_2\mathfrak{K}^{\infty,u}_{com,  \vp}(\mathbf{H}, \b; \mathrm{L}_m).
    \end{align*}
    
    The results in Tables \ref{tab1}-\ref{tab4} show that the upper bounds of the NCN, MCN and CCN provide very sharp estimates of the exact NCN, MCN and CCN, respectively. Additionally, it is observed that both the MCN and CCN, along with their upper bounds, are at most two orders of magnitude larger than the actual relative forward errors, offering more accurate estimates compared to the NCN and its upper bounds. These numerical results highlight the effectiveness of the proposed CNs and their corresponding upper bounds. 
\end{exam}


\begin{exam}\label{example2}
We consider the DSPP \eqref{eq1:SPP} with the block matrices given by
    \begin{align*}
 &A=[(2ZZ^T+\Sigma_1) \oplus \Sigma_2\oplus \Sigma_3]\in \R^{n\times n}, ~ B=\bmatrix{N&-I_{2\tilde{q}}& I_{2\tilde{q}}}\in \R^{m\times n},~ \\
     &D={\tt{toeplitz}}(\bm d)\in \R^{m\times m}, ~C=M, ~E={\tt{toeplitz}}(\bm e)\in \R^{p\times p}, \\
     &~\text{where}~  Z=[z_{ij}]\in \R^{\hat{q}\times \hat{q}}~ \text{with}~ z_{ij}=e^{-2((i/3)^2+(j/3)^2)}, ~\Sigma_1=I_{\hat{q}}, 
 \end{align*}
 \vspace{-3mm}
and $\Sigma_k=\diag(d_j^{(k)})\in \R^{2\tilde{q}\times 2\tilde{q}}, ~k=2,3,$ are diagonal matrices with  
$$ d_{j}^{(2)}=\left\{\begin{array}{cc}
    1,& \text{for} ~1\leq j\leq \tilde{q},  \\
    10^{-5}(j-\tilde{q})^2, & ~\text{for}~ \tilde{q}+1\leq j\leq 2\tilde{q},
\end{array}\right.$$
%\vspace{-3mm}
$d^{(3)}_j=10^{-5}(j+\tilde{q})^2$ for $1\leq j\leq 2\tilde{q},$ where $\tilde{q}=q^2$ and $\hat{q}=q(q+1).$ Further, $N=\bmatrix{\widehat{N}\otimes I_q\\ I_q\otimes \widehat{N}}\in \R^{2\tilde{q}\times \hat{q}},$ $\widehat{N}={\tt tridiag}(0,2,-1)\in \R^{q\times(q+1)},$ $M=\bmatrix{\widehat{M}\otimes I_q & I_q\otimes \widehat{M}}\in \R^{\hat{q}\times2\tilde{q}},$ 
%
$$\widehat{M}=\bmatrix{q+1& -\frac{q-1}{q}& \frac{q-2}{q}&\ldots &\frac{(-1)^{q-1}}{q}\\
-\frac{q-1}{q}& 2q+1& \ddots&\ddots&\vdots\\
\frac{q-2}{q}&\ddots &\ddots&\ddots&\frac{q-2}{q}\\
\vdots&\ddots&\ddots&\ddots&-\frac{q-1}{q}\\\frac{(-1)^{q-1}}{q}&\vdots&\frac{q-2}{q}& -\frac{q-1}{q}& q^2+1\\0 &0&\ldots& 0&1
}\in \R^{(q+1)\times q},$$
$\bm{d}=\mathtt{randn}(m,1)\in \R^m, \bm{e}=\mathtt{randn}(p,1)\in \R^p.$
Here, the notation $[A\oplus B]$ means the block diagonal matrix with diagonal blocks $A$ and $B,$ and $\mathtt{toeplitz}(\bm{d})$ denotes the Toeplitz matrix generated by the vector $\bm{d}.$ Moreover, we get ${\bm{l}}=8q^2+2q.$ The vector ${\bf d}\in\R^{\bm l}$ is chosen as in Example \ref{exam1}.


\begin{sidewaystable}
		\centering
		\caption{Comparison of the partial NCN, MCN, and CCN with the corresponding structured partial NCN, MCN and CCN for Example \ref{example2}.}
			\label{tab5}
  % \resizebox{16.5cm}{!}{
				\begin{tabular}{cccccccc}
					\toprule
			$\mathrm{L}$&$q$	 &$\mathfrak{K}^{(2)}_{\vp}(\mathbf{H}, \b; \mathrm{L})$	& $\mathfrak{K}^{(2),\mathbb{S}}_{\vp}(\mathbf{H}, \b; \mathrm{L})$ & $\mathfrak{K}^{\infty}_{mix,  \vp}(\mathbf{H}, \b; \mathrm{L})$& $\mathfrak{K}^{\infty,\mathbb{S}}_{mix,  \vp}(\mathbf{H}, \b; \mathrm{L})$ &  $\mathfrak{K}^{\infty}_{com,  \vp}(\mathbf{H}, \b; \mathrm{L})$ &  $\mathfrak{K}^{\infty,\mathbb{S}}_{com,  \vp}(\mathbf{H}, \b; \mathrm{L})$ \\
  \midrule 
\multirow{3}{*}{$\mathrm{L}_0$}  & $2$ &$9.0961E+03$&	$8.9842E+03$&	$1.1174E+02$&	$9.2003E+01$&	$1.0211E+03$	&$7.8013E+02$\\[1ex]
   &$3$ &$1.0996E+04$&	$8.6963E+03$&	$1.1735E+02$&	$9.1218E+01$&	$1.2372E+04$&	$9.5905E+03$	\\[1ex]
   &  $4$ &$2.6066E+04$&	$1.9983E+04$	&$3.5456E+02$&	$2.6052E+02$&	$2.5284E+05$	&$6.7124E+04$	\\	[1ex]
   \midrule
   \multirow{3}{*}{$\mathrm{L}_n$}  & $2$ &$9.1011E+03$&	$8.9891E+03$	&$1.1174E+02$	&$9.2003E+01$	&$2.9170E+02$&	$2.1175E+02$\\[1ex]
   &$3$ &	$1.0972E+04$	&$8.6781E+03$&	$1.1735E+02$&	$9.1218E+01$&	$1.2372E+04$&	$9.5905E+03$\\[1ex]
   &  $4$ &$2.6371E+04$&	$2.0138E+04$&	$3.5456E+02$	&$2.6052E+02$&	$2.7573E+04$	&$1.8454E+04$\\	[1ex]
   \midrule
   \multirow{3}{*}{$\mathrm{L}_m$}  & $2$ &$8.4606E+03$	&$8.3300E+03$&	$1.5279E+02$&	$1.1486E+02$	&$2.5276E+02$&	$1.5433E+02$\\[1ex]
   &$3$ &	$1.8511E+04$&	$1.4614E+04$	&$2.2517E+02$&	$1.6131E+02$&	$9.9621E+02$&	$5.3614E+02$\\[1ex]
   &  $4$ &$3.2837E+04$&	$2.7473E+04$	&$8.9167E+02$	&5.8061E+02&	$2.5284E+05$&	$6.7124E+04$\\	[1ex]
   \midrule
   \multirow{3}{*}{$\mathrm{L}_p$}  & $2$ &$1.1952E+04$	&$1.1786E+04$	&$2.1568E+02$	& $1.6042E+02$&	$1.0211E+03$	&$7.8013E+02$\\[1ex]
   &$3$ &$9.4332E+03$	&$7.4505E+03$	&$1.2141E+02$	&$8.3080E+01$	&$1.1802E+04$&	$8.2273E+03$	\\[1ex]
   &  $4$ &$1.8803E+04$	&$1.5754E+04$	&$7.0206E+02$	&$4.8744E+02$&	$7.4731E+03$&	$4.8946E+03$\\	[1ex]
     
   \bottomrule
     
   
    \end{tabular}
    %}
    \end{sidewaystable}

    The structured partial NCN is calculated using Theorem \ref{Th:SNCN}, while the structured partial MCN and CCN are computed from Theorem \ref{Th:MCN}. Numerical results for various choices of \(\mathrm{L}\) and \(q = 2, 3, 4\) are summarized in Table \ref{tab5}. We observed for all choices of $\mathrm{L},$ $\mathfrak{K}^{\infty,\mathbb{S}}_{mix,  \vp}(\mathbf{H}, \b; \mathrm{L})$ and $\mathfrak{K}^{\infty,\mathbb{S}}_{com,  \vp}(\mathbf{H}, \b; \mathrm{L})$  are almost one order smaller than the $\mathfrak{K}^{\infty}_{mix,  \vp}(\mathbf{H}, \b; \mathrm{L})$ and $\mathfrak{K}^{\infty}_{com,  \vp}(\mathbf{H}, \b; \mathrm{L})$, respectively. 
\end{exam}
%%%%%%%%%%%%%%%%%%
\section{Conclusion}\label{sec:Conclusion}
%%%%%%%%%%%%%%%%%%%%%%%%%%%%%%%%%%%%%%%%%%%%%%%%%%%%%%
This paper introduces a unified framework for investigating the partial CN for the solution of the DSPP. We derive compact formulas for the partial CNs, and by considering specific norms, we obtain the partial NCN, MCN and CCN for the solution of the DSPP. Additionally, we provide sharp upper bounds for the partial CNs that are free from costly Kronecker products. Moreover, we compute structured partial NCN, MCN, and CCN  by introducing perturbations that maintain the structure of the block matrices of the coefficient matrix. Using our theoretical findings and by leveraging the relationship between the EILS problems and the DSPP, we recover several previously established results for the EILS problems. Experimental results demonstrate that the derived upper bounds for the partial CNs provide tight estimates of the actual partial CNs. Furthermore, the proposed partial CNs and their upper bounds provide sharp error estimation for the solution, highlighting their effectiveness and reliability.
%
%%%%%%%%%%%%%%%%%%%%%%%%%%%%%%%%%%%%%%%%%%%%%%%%%

% \section{Results}\label{sec2}

% Sample body text. Sample body text. Sample body text. Sample body text. Sample body text. Sample body text. Sample body text. Sample body text.

% \section{This is an example for first level head---section head}\label{sec3}

% \subsection{This is an example for second level head---subsection head}\label{subsec2}

% \subsubsection{This is an example for third level head---subsubsection head}\label{subsubsec2}

% Sample body text. Sample body text. Sample body text. Sample body text. Sample body text. Sample body text. Sample body text. Sample body text. 

% \section{Equations}\label{sec4}

% Equations in \LaTeX\ can either be inline or on-a-line by itself (``display equations''). For
% inline equations use the \verb+$...$+ commands. E.g.: The equation
% $H\psi = E \psi$ is written via the command \verb+$H \psi = E \psi$+.

% For display equations (with auto generated equation numbers)
% one can use the equation or align environments:
% \begin{equation}
% \|\tilde{X}(k)\|^2 \leq\frac{\sum\limits_{i=1}^{p}\left\|\tilde{Y}_i(k)\right\|^2+\sum\limits_{j=1}^{q}\left\|\tilde{Z}_j(k)\right\|^2 }{p+q}.\label{eq1}
% \end{equation}
% where,
% \begin{align}
% D_\mu &=  \partial_\mu - ig \frac{\lambda^a}{2} A^a_\mu \nonumber \\
% F^a_{\mu\nu} &= \partial_\mu A^a_\nu - \partial_\nu A^a_\mu + g f^{abc} A^b_\mu A^a_\nu \label{eq2}
% \end{align}
% Notice the use of \verb+\nonumber+ in the align environment at the end
% of each line, except the last, so as not to produce equation numbers on
% lines where no equation numbers are required. The \verb+\label{}+ command
% should only be used at the last line of an align environment where
% \verb+\nonumber+ is not used.
% \begin{equation}
% Y_\infty = \left( \frac{m}{\textrm{GeV}} \right)^{-3}
%     \left[ 1 + \frac{3 \ln(m/\textrm{GeV})}{15}
%     + \frac{\ln(c_2/5)}{15} \right]
% \end{equation}
% The class file also supports the use of \verb+\mathbb{}+, \verb+\mathscr{}+ and
% \verb+\mathcal{}+ commands. As such \verb+\mathbb{R}+, \verb+\mathscr{R}+
% and \verb+\mathcal{R}+ produces $\mathbb{R}$, $\mathscr{R}$ and $\mathcal{R}$
% respectively (refer Subsubsection~\ref{subsubsec2}).

% \section{Tables}\label{sec5}

% Tables can be inserted via the normal table and tabular environment. To put
% footnotes inside tables you should use \verb+\footnotetext[]{...}+ tag.
% The footnote appears just below the table itself (refer Tables~\ref{tab1} and \ref{tab2}). 
% For the corresponding footnotemark use \verb+\footnotemark[...]+

% \begin{table}[h]
% \caption{Caption text}\label{tab1}%
% \begin{tabular}{@{}llll@{}}
% \toprule
% Column 1 & Column 2  & Column 3 & Column 4\\
% \midrule
% row 1    & data 1   & data 2  & data 3  \\
% row 2    & data 4   & data 5\footnotemark[1]  & data 6  \\
% row 3    & data 7   & data 8  & data 9\footnotemark[2]  \\
% \botrule
% \end{tabular}
% \footnotetext{Source: This is an example of table footnote. This is an example of table footnote.}
% \footnotetext[1]{Example for a first table footnote. This is an example of table footnote.}
% \footnotetext[2]{Example for a second table footnote. This is an example of table footnote.}
% \end{table}

% \noindent
% The input format for the above table is as follows:

% %%=============================================%%
% %% For presentation purpose, we have included  %%
% %% \bigskip command. Please ignore this.       %%
% %%=============================================%%
% \bigskip
% \begin{verbatim}
% \begin{table}[<placement-specifier>]
% \caption{<tablE-caption>}\label{<tablE-label>}%
% \begin{tabular}{@{}llll@{}}
% \toprule
% Column 1 & Column 2 & Column 3 & Column 4\\
% \midrule
% row 1 & data 1 & data 2	 & data 3 \\
% row 2 & data 4 & data 5\footnotemark[1] & data 6 \\
% row 3 & data 7 & data 8	 & data 9\footnotemark[2]\\
% \botrule
% \end{tabular}
% \footnotetext{Source: This is an example of table footnote. 
% This is an example of table footnote.}
% \footnotetext[1]{Example for a first table footnote.
% This is an example of table footnote.}
% \footnotetext[2]{Example for a second table footnote. 
% This is an example of table footnote.}
% \end{table}
% \end{verbatim}
% \bigskip
% %%=============================================%%
% %% For presentation purpose, we have included  %%
% %% \bigskip command. Please ignore this.       %%
% %%=============================================%%

% \begin{table}[h]
% \caption{Example of a lengthy table which is set to full textwidth}\label{tab2}
% \begin{tabular*}{\textwidth}{@{\extracolsep\fill}lcccccc}
% \toprule%
% & \multicolumn{3}{@{}c@{}}{Element 1\footnotemark[1]} & \multicolumn{3}{@{}c@{}}{Element 2\footnotemark[2]} \\\cmidrule{2-4}\cmidrule{5-7}%
% Project & Energy & $\sigma_{calc}$ & $\sigma_{expt}$ & Energy & $\sigma_{calc}$ & $\sigma_{expt}$ \\
% \midrule
% Element 3  & 990 A & 1168 & $1547\pm12$ & 780 A & 1166 & $1239\pm100$\\
% Element 4  & 500 A & 961  & $922\pm10$  & 900 A & 1268 & $1092\pm40$\\
% \botrule
% \end{tabular*}
% \footnotetext{Note: This is an example of table footnote. This is an example of table footnote this is an example of table footnote this is an example of~table footnote this is an example of table footnote.}
% \footnotetext[1]{Example for a first table footnote.}
% \footnotetext[2]{Example for a second table footnote.}
% \end{table}

% In case of double column layout, tables which do not fit in single column width should be set to full text width. For this, you need to use \verb+\begin{table*}+ \verb+...+ \verb+\end{table*}+ instead of \verb+\begin{table}+ \verb+...+ \verb+\end{table}+ environment. Lengthy tables which do not fit in textwidth should be set as rotated table. For this, you need to use \verb+\begin{sidewaystable}+ \verb+...+ \verb+\end{sidewaystable}+ instead of \verb+\begin{table*}+ \verb+...+ \verb+\end{table*}+ environment. This environment puts tables rotated to single column width. For tables rotated to double column width, use \verb+\begin{sidewaystable*}+ \verb+...+ \verb+\end{sidewaystable*}+.

% \begin{sidewaystable}
% \caption{Tables which are too long to fit, should be written using the ``sidewaystable'' environment as shown here}\label{tab3}
% \begin{tabular*}{\textheight}{@{\extracolsep\fill}lcccccc}
% \toprule%
% & \multicolumn{3}{@{}c@{}}{Element 1\footnotemark[1]}& \multicolumn{3}{@{}c@{}}{Element\footnotemark[2]} \\\cmidrule{2-4}\cmidrule{5-7}%
% Projectile & Energy	& $\sigma_{calc}$ & $\sigma_{expt}$ & Energy & $\sigma_{calc}$ & $\sigma_{expt}$ \\
% \midrule
% Element 3 & 990 A & 1168 & $1547\pm12$ & 780 A & 1166 & $1239\pm100$ \\
% Element 4 & 500 A & 961  & $922\pm10$  & 900 A & 1268 & $1092\pm40$ \\
% Element 5 & 990 A & 1168 & $1547\pm12$ & 780 A & 1166 & $1239\pm100$ \\
% Element 6 & 500 A & 961  & $922\pm10$  & 900 A & 1268 & $1092\pm40$ \\
% \botrule
% \end{tabular*}
% \footnotetext{Note: This is an example of table footnote this is an example of table footnote this is an example of table footnote this is an example of~table footnote this is an example of table footnote.}
% \footnotetext[1]{This is an example of table footnote.}
% \end{sidewaystable}

% %\section{Figures}\label{sec6}

% As per the \LaTeX\ standards you need to use eps images for \LaTeX\ compilation and \verb+pdf/jpg/png+ images for \verb+PDFLaTeX+ compilation. This is one of the major difference between \LaTeX\ and \verb+PDFLaTeX+. Each image should be from a single input .eps/vector image file. Avoid using subfigures. The command for inserting images for \LaTeX\ and \verb+PDFLaTeX+ can be generalized. The package used to insert images in \verb+LaTeX/PDFLaTeX+ is the graphicx package. Figures can be inserted via the normal figure environment as shown in the below example:

% %%=============================================%%
% %% For presentation purpose, we have included  %%
% %% \bigskip command. Please ignore this.       %%
% %%=============================================%%
% \bigskip
% \begin{verbatim}
% \begin{figure}[<placement-specifier>]
% \centering
% \includegraphics{<eps-file>}
% \caption{<figurE-caption>}\label{<figurE-label>}
% \end{figure}
% \end{verbatim}
% \bigskip
% %%=============================================%%
% %% For presentation purpose, we have included  %%
% %% \bigskip command. Please ignore this.       %%
% %%=============================================%%

% \begin{figure}[h]
% \centering
% \includegraphics[width=0.9\textwidth]{fig.eps}
% \caption{This is a widefig. This is an example of long caption this is an example of long caption  this is an example of long caption this is an example of long caption}\label{fig1}
% \end{figure}

% In case of double column layout, the above format puts figure captions/images to single column width. To get spanned images, we need to provide \verb+\begin{figure*}+ \verb+...+ \verb+\end{figure*}+.

% For sample purpose, we have included the width of images in the optional argument of \verb+\includegraphics+ tag. Please ignore this. 

% %\section{Algorithms, Program codes and Listings}\label{sec7}

% Packages \verb+algorithm+, \verb+algorithmicx+ and \verb+algpseudocodE+ are used for setting algorithms in \LaTeX\ using the format:

% %%=============================================%%
% %% For presentation purpose, we have included  %%
% %% \bigskip command. Please ignore this.       %%
% %%=============================================%%
% \bigskip
% \begin{verbatim}
% \begin{algorithm}
% \caption{<alg-caption>}\label{<alg-label>}
% \begin{algorithmic}[1]
% . . .
% \end{algorithmic}
% \end{algorithm}
% \end{verbatim}
% \bigskip
% %%=============================================%%
% %% For presentation purpose, we have included  %%
% %% \bigskip command. Please ignore this.       %%
% %%=============================================%%

% You may refer above listed package documentations for more details before setting \verb+algorithm+ environment. For program codes, the ``verbatim'' package is required and the command to be used is \verb+\begin{verbatim}+ \verb+...+ \verb+\end{verbatim}+. 

% Similarly, for \verb+listings+, use the \verb+listings+ package. \verb+\begin{lstlisting}+ \verb+...+ \verb+\end{lstlisting}+ is used to set environments similar to \verb+verbatim+ environment. Refer to the \verb+lstlisting+ package documentation for more details.

% A fast exponentiation procedure:

% \lstset{texcl=true,basicstyle=\small\sf,commentstyle=\small\rm,mathescape=true,escapeinside={(*}{*)}}
% \begin{lstlisting}
% begin
%   for $i:=1$ to $10$ step $1$ do
%       expt($2,i$);  
%       newline() od                (*\textrm{Comments will be set flush to the right margin}*)
% where
% proc expt($x,n$) $\equiv$
%   $z:=1$;
%   do if $n=0$ then exit fi;
%      do if odd($n$) then exit fi;                 
%         comment: (*\textrm{This is a comment statement;}*)
%         $n:=n/2$; $x:=x*x$ od;
%      { $n>0$ };
%      $n:=n-1$; $z:=z*x$ od;
%   print($z$). 
% end
% \end{lstlisting}

% \begin{algorithm}
% \caption{Calculate $y = x^n$}\label{algo1}
% \begin{algorithmic}[1]
% \Require $n \geq 0 \vee x \neq 0$
% \Ensure $y = x^n$ 
% \State $y \Leftarrow 1$
% \If{$n < 0$}\label{algln2}
%         \State $X \Leftarrow 1 / x$
%         \State $N \Leftarrow -n$
% \Else
%         \State $X \Leftarrow x$
%         \State $N \Leftarrow n$
% \EndIf
% \While{$N \neq 0$}
%         \If{$N$ is even}
%             \State $X \Leftarrow X \times X$
%             \State $N \Leftarrow N / 2$
%         \Else[$N$ is odd]
%             \State $y \Leftarrow y \times X$
%             \State $N \Leftarrow N - 1$
%         \EndIf
% \EndWhile
% \end{algorithmic}
% \end{algorithm}

% %%=============================================%%
% %% For presentation purpose, we have included  %%
% %% \bigskip command. Please ignore this.       %%
% %%=============================================%%
% \bigskip
% \begin{minipage}{\hsize}%
% \lstset{frame=single,framexleftmargin=-1pt,framexrightmargin=-17pt,framesep=12pt,linewidth=0.98\textwidth,language=pascal}% Set your language (you can change the language for each codE-block optionally)
% %%% Start your codE-block
% \begin{lstlisting}
% for i:=maxint to 0 do
% begin
% { do nothing }
% end;
% Write('Case insensitive ');
% Write('Pascal keywords.');
% \end{lstlisting}
% \end{minipage}

% %\section{Cross referencing}\label{sec8}

% Environments such as figure, table, equation and align can have a label
% declared via the \verb+\label{#label}+ command. For figures and table
% environments use the \verb+\label{}+ command inside or just
% below the \verb+\caption{}+ command. You can then use the
% \verb+\ref{#label}+ command to cross-reference them. As an example, consider
% the label declared for Figure~\ref{fig1} which is
% \verb+\label{fig1}+. To cross-reference it, use the command 
% \verb+Figure \ref{fig1}+, for which it comes up as
% ``Figure~\ref{fig1}''. 

% To reference line numbers in an algorithm, consider the label declared for the line number 2 of Algorithm~\ref{algo1} is \verb+\label{algln2}+. To cross-reference it, use the command \verb+\ref{algln2}+ for which it comes up as line~\ref{algln2} of Algorithm~\ref{algo1}.

% \subsection{Details on reference citations}\label{subsec7}

% Standard \LaTeX\ permits only numerical citations. To support both numerical and author-year citations this template uses \verb+natbib+ \LaTeX\ package. For style guidance please refer to the template user manual.

% Here is an example for \verb+\cite{...}+: \cite{bib1}. Another example for \verb+\citep{...}+: \citep{bib2}. For author-year citation mode, \verb+\cite{...}+ prints Jones et al. (1990) and \verb+\citep{...}+ prints (Jones et al., 1990).

% All cited bib entries are printed at the end of this article: \cite{bib3}, \cite{bib4}, \cite{bib5}, \cite{bib6}, \cite{bib7}, \cite{bib8}, \cite{bib9}, \cite{bib10}, \cite{bib11}, \cite{bib12} and \cite{bib13}.


% \section{Examples for theorem like environments}\label{sec10}

% For theorem like environments, we require \verb+amsthm+ package. There are three types of predefined theorem styles exists---\verb+thmstyleonE+, \verb+thmstyletwo+ and \verb+thmstylethreE+ 

% %%=============================================%%
% %% For presentation purpose, we have included  %%
% %% \bigskip command. Please ignore this.       %%
% %%=============================================%%
% \bigskip
% \begin{tabular}{|l|p{19pc}|}
% \hline
% \verb+thmstyleonE+ & Numbered, theorem head in bold font and theorem text in italic style \\\hline
% \verb+thmstyletwo+ & Numbered, theorem head in roman font and theorem text in italic style \\\hline
% \verb+thmstylethreE+ & Numbered, theorem head in bold font and theorem text in roman style \\\hline
% \end{tabular}
% \bigskip
% %%=============================================%%
% %% For presentation purpose, we have included  %%
% %% \bigskip command. Please ignore this.       %%
% %%=============================================%%

% For mathematics journals, theorem styles can be included as shown in the following examples:

% \begin{theorem}[Theorem subhead]\label{thm1}
% Example theorem text. Example theorem text. Example theorem text. Example theorem text. Example theorem text. 
% Example theorem text. Example theorem text. Example theorem text. Example theorem text. Example theorem text. 
% Example theorem text. 
% \end{theorem}

% Sample body text. Sample body text. Sample body text. Sample body text. Sample body text. Sample body text. Sample body text. Sample body text.

% \begin{proposition}
% Example proposition text. Example proposition text. Example proposition text. Example proposition text. Example proposition text. 
% Example proposition text. Example proposition text. Example proposition text. Example proposition text. Example proposition text. 
% \end{proposition}

% Sample body text. Sample body text. Sample body text. Sample body text. Sample body text. Sample body text. Sample body text. Sample body text.

% \begin{example}
% Phasellus adipiscing semper elit. Proin fermentum massa
% ac quam. Sed diam turpis, molestie vitae, placerat a, molestie nec, leo. Maecenas lacinia. Nam ipsum ligula, eleifend
% at, accumsan nec, suscipit a, ipsum. Morbi blandit ligula feugiat magna. Nunc eleifend consequat lorem. 
% \end{example}

% Sample body text. Sample body text. Sample body text. Sample body text. Sample body text. Sample body text. Sample body text. Sample body text.

% \begin{remark}
% Phasellus adipiscing semper elit. Proin fermentum massa
% ac quam. Sed diam turpis, molestie vitae, placerat a, molestie nec, leo. Maecenas lacinia. Nam ipsum ligula, eleifend
% at, accumsan nec, suscipit a, ipsum. Morbi blandit ligula feugiat magna. Nunc eleifend consequat lorem. 
% \end{remark}

% Sample body text. Sample body text. Sample body text. Sample body text. Sample body text. Sample body text. Sample body text. Sample body text.

% \begin{definition}[Definition sub head]
% Example definition text. Example definition text. Example definition text. Example definition text. Example definition text. Example definition text. Example definition text. Example definition text. 
% \end{definition}

% Additionally a predefined ``proof'' environment is available: \verb+\begin{proof}+ \verb+...+ \verb+\end{proof}+. This prints a ``Proof'' head in italic font style and the ``body text'' in roman font style with an open square at the end of each proof environment. 

% \begin{proof}
% Example for proof text. Example for proof text. Example for proof text. Example for proof text. Example for proof text. Example for proof text. Example for proof text. Example for proof text. Example for proof text. Example for proof text. 
% \end{proof}

% Sample body text. Sample body text. Sample body text. Sample body text. Sample body text. Sample body text. Sample body text. Sample body text.

% \begin{proof}[Proof of Theorem~{\upshape\ref{thm1}}]
% Example for proof text. Example for proof text. Example for proof text. Example for proof text. Example for proof text. Example for proof text. Example for proof text. Example for proof text. Example for proof text. Example for proof text. 
% \end{proof}

% \noindent
% For a quote environment, use \verb+\begin{quote}...\end{quote}+
% \begin{quote}
% Quoted text example. Aliquam porttitor quam a lacus. Praesent vel arcu ut tortor cursus volutpat. In vitae pede quis diam bibendum placerat. Fusce elementum
% convallis neque. Sed dolor orci, scelerisque ac, dapibus nec, ultricies ut, mi. Duis nec dui quis leo sagittis commodo.
% \end{quote}

% Sample body text. Sample body text. Sample body text. Sample body text. Sample body text (refer Figure~\ref{fig1}). Sample body text. Sample body text. Sample body text (refer Table~\ref{tab3}). 

% \section{Methods}\label{sec11}

% Topical subheadings are allowed. Authors must ensure that their Methods section includes adequate experimental and characterization data necessary for others in the field to reproduce their work. Authors are encouraged to include RIIDs where appropriate. 

% \textbf{Ethical approval declarations} (only required where applicable) Any article reporting experiment/s carried out on (i)~live vertebrate (or higher invertebrates), (ii)~humans or (iii)~human samples must include an unambiguous statement within the methods section that meets the following requirements: 

% \begin{enumerate}[1.]
% \item Approval: a statement which confirms that all experimental protocols were approved by a named institutional and/or licensing committee. Please identify the approving body in the methods section

% \item Accordance: a statement explicitly saying that the methods were carried out in accordance with the relevant guidelines and regulations

% \item Informed consent (for experiments involving humans or human tissue samples): include a statement confirming that informed consent was obtained from all participants and/or their legal guardian/s
% \end{enumerate}

% If your manuscript includes potentially identifying patient/participant information, or if it describes human transplantation research, or if it reports results of a clinical trial then  additional information will be required. Please visit (\url{https://www.nature.com/naturE-research/editorial-policies}) for Nature Portfolio journals, (\url{https://www.springer.com/gp/authors-editors/journal-author/journal-author-helpdesk/publishing-ethics/14214}) for Springer Nature journals, or (\url{https://www.biomedcentral.com/getpublished/editorial-policies\#ethics+and+consent}) for BMC.

% \section{Discussion}\label{sec12}

% Discussions should be brief and focused. In some disciplines use of Discussion or `Conclusion' is interchangeable. It is not mandatory to use both. Some journals prefer a section `Results and Discussion' followed by a section `Conclusion'. Please refer to Journal-level guidance for any specific requirements. 

% \section{Conclusion}\label{sec13}

% Conclusions may be used to restate your hypothesis or research question, restate your major findings, explain the relevance and the added value of your work, highlight any limitations of your study, describe future directions for research and recommendations. 

% In some disciplines use of Discussion or 'Conclusion' is interchangeable. It is not mandatory to use both. Please refer to Journal-level guidance for any specific requirements. 

% \backmatter

% \bmhead{Supplementary information}

% If your article has accompanying supplementary file/s please state so here. 

% Authors reporting data from electrophoretic gels and blots should supply the full unprocessed scans for key as part of their Supplementary information. This may be requested by the editorial team/s if it is missing.

% Please refer to Journal-level guidance for any specific requirements.

\bmhead{Acknowledgements}
During this work, Pinki Khatun was financially supported by the Council of Scientific $\&$ Industrial Research (CSIR) in New Delhi, India, in the form of a fellowship (File no. $09/1022(0098)/2020$-EMR-I).


\section*{Declarations}
\subsection*{Funding} 
Not applicable
\subsection*{Competing interests}
The authors have no competing interests
\subsection*{Ethics approval} 
Not applicable
\subsection*{Data availability} 
Not applicable
\subsection*{Materials availability} 
Not applicable
%\subsection*{Code availability} 
%Not applicable
\subsection*{Author contribution}
Both authors contributed equally to the conceptualization, writing, and review of this work.


% \begin{itemize}
% \item : Not applicable
% \item : 
% \item : Not applicable
% \item Consent for publication: Not applicable
% \item : Not applicable
% \item Materials availability: Not applicable
% \item Code availability: Not applicable
% \item 
% \end{itemize}
\bibliography{sn-bibliography}
%\bibliographystyle{abbrvnat}% common bib file
%% if required, the content of .bbl file can be included here once bbl is generated
%%\input sn-article.bbl


\end{document}
