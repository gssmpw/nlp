
\begin{table*}[ht]
    \centering
    \resizebox{\linewidth}{!}{
    \small  % 设置表格字体为小号
    \begin{tabular}{p{0.5in}|p{6.4in}}
        \toprule
        \centering Question & How many positive two-digit integers are factors of both 100 and 150? \\
        \midrule
        \centering SFT & Okay, so I need to figure out how many positive two-digit integers are factors of both 100 and 150. \textcolor{red}{hmm}, let's start by understanding the problem....So, the GCD of 100 and 150 is 50. That means all the common factors of 100 and 150 are the factors of 50. Let me list out the factors of 50 to make sure. The factors of 50 are 1, 2, 5, 10, 25, and 50. \textcolor{red}{wait}, that's six factors. But the problem asks for two-digit integers. From this list, the two-digit factors would be 25 and 50. So that's two numbers. But \textcolor{red}{wait}, hold on. Let me verify if 25 and 50 are indeed factors of both 100 and 150. For 25: 100 / 25 = 4, which is an integer. 150 / 25 = 6, which is also an integer. So yes, 25 is a common factor. For 50: 100 / 50 = 2, and 150 / 50 = 3. Both integers, so 50 is also a common factor. But \textcolor{red}{wait}, are there any other two-digit numbers that are factors of both 100 and 150? Let's check. The factors of 100 are 1, 2, 4, 5, 10, 20, 25, 50, 100. The factors of 150 are 1, 2, 3, 5, 6, 10, 15, 25, 30, 50, 75, 150. The common factors are 1, 2, 5, 10, 25, 50. So, the two-digit common factors are 25 and 50. That gives us two numbers. \textcolor{red}{wait}, but hold on. The problem says "positive two-digit integers". ... So, factors of 50 are 1, 2, 5, 10, 25, 50. So, two-digit ones are 25 and 50. Therefore, the answer is 2. But \textcolor{red}{wait}, let me check if there are any other numbers between 10 and 99 that divide both 100 and 150. For example, 10 divides both. 10 is two-digit. 20 divides 100 but 150 / 20 is 7.5, which is not integer. So 20 is out. 30 divides 150 but 100 / 30 is approximately 3.333, not integer. 15 divides 150 but 100 / 15 is about 6.666, not integer. 75 divides 150 but 100 / 75 is 1.333, not integer. So, indeed, only 25 and 50 are in the two-digit range. Therefore, the answer is 2. But \textcolor{red}{wait}, \textcolor{red}{wait} a second. Let me check 25 and 50 again. 25 divides both 100 and 150, yes. 50 divides both, yes. Are there any other numbers between 10 and 99 that divide both? Let me check 10. 10 divides 100 and 150. 10 is two-digit. But we already have 10, 25, 50. \textcolor{red}{wait}, 10 is a factor, but it's single-digit. So 10 is not included. So 25 and 50 are the only two. Therefore, the answer is 2. So, the number of positive two-digit integers that are factors of both 100 and 150 is 2. But just to be thorough, let me check 25: 100 / 25 = 4, 150 / 25 = 6. Both integers. 50: 100 / 50 = 2, 150 / 50 = 3. Both integers. So yes, they are factors. Are there any others? Let me check 30: 100 / 30 = 3.333, not integer. 40: 100 / 40 = 2.5, not integer. 45: 100 / 45 = 2.222, no. 55: 100 / 55 = 1.818, no. 60: 100 / 60 = 1.666, no. 65: 100 / 65 = 1.538, no. 70: 100 / 70 = 1.428, no. 75: 100 / 75 = 1.333, no. 80: 100 / 80 = 1.25, no. 85: 100 / 85 = 1.176, no. 90: 100 / 90 = 1.111, no. 95: 100 / 95 = 1.052, no. So, none of these two-digit numbers are factors of both 100 and 150. Therefore, only 25 and 50. Therefore, the answer is 2. So, the final answer is 2, which in boxed form is 2. \\
        \midrule
        \centering ThinkPO & Okay, so I need to figure out how many positive two-digit integers are factors of both 100 and 150. \textcolor{red}{hmm}, let's start by understanding the problem.... For prime 5, both have exponent 2, so min(2,2)=2. Prime 3 is only in 150, so we ignore it. So GCD is 2 * 5 * 5 = 2 * 25 = 50. \textcolor{red}{wait}, so the GCD of 100 and 150 is 50. ... \textcolor{red}{wait}, that's two numbers. But let me verify. \textcolor{red}{wait}, 50's divisors: 1, 2, 5, 10, 25, 50. Yep, that's all. So two-digit numbers here are 10 and 50. So the answer would be 2? But let me make sure I didn't miss anything. But \textcolor{red}{wait}, is 10 and 50 the only two-digit divisors of 50? Let's check. 50 divided by 10 is 5, which is integer. 50 divided by 50 is 1. So yes, those are the divisors. But \textcolor{red}{wait}, 5 is a single-digit number. So the two-digit divisors are 10 and 50. So that's two numbers. \textcolor{red}{wait}, but the problem says \"positive two-digit integers\", so 10 and 50. But \textcolor{red}{wait}, is 50 a two-digit number? Yes, 50 is two digits. But 10 is also two digits. So that's two numbers. So the answer is 2? \textcolor{red}{hmm}, but let me cross-verify this. Alternatively, maybe I should approach it another way. ...So from 10 to 99, the numbers that divide both 100 and 150 are 10, 25, and 50. \textcolor{red}{wait}, but 25 and 50 are two-digit numbers. So that's three numbers: 10, 25, 50. \textcolor{red}{wait}, but why did I think earlier that 25 is a two-digit number? It is. So three numbers. But then 10, 25, 50. So three numbers. But when I first thought of the GCD approach, I considered the divisors of 50, which are 1, 2, 5, 10, 25, 50. So only two two-digit divisors. But here, when checking manually, I found three. So there's a discrepancy. Which one is correct?\textcolor{red}{wait}, perhaps I made a mistake in the initial approach. Let's check again. The GCD of 100 and 150 is 50. So the common factors are the divisors of 50. Divisors of 50 are 1, 2, 5, 10, 25, 50. So yes, two-digit divisors are 10 and 50. So that's two numbers. But according to the manual check, there are three. So which is correct?\textcolor{red}{wait}, let's check 25. 25 is a divisor of both 100 and 150. 25 divides 100 because 25*4=100, and 25 divides 150 because 25*6=150. So 25 is indeed a common factor. But 25 is a two-digit number, right? 2 and 5. So yes, two digits. So 25 is a two-digit divisor. Then 10 and 50. So that's three numbers. But according to the GCD method, the common factors are exactly the divisors of 50, which are six numbers, but only two two-digit. However, when I checked manually, I found three two-digit divisors. So where's the mistake?\textcolor{red}{wait}, maybe I made a mistake in the GCD calculation. Let me check again. 100 factors: $2^2 * 5^2$. 150 factors: $2^1 * 3^1 * 5^2$. Common prime factors: $2^min(2,1)=1$, $5^min(2,2)=2$. So GCD is $2^1 *5^2=2*25=50$. So that's correct. So the common factors are exactly the divisors of 50. Therefore, divisors of 50 are 1,2,5,10,25,50. So the two-digit ones are 10,25,50. \textcolor{red}{wait}, that's three numbers. But according to GCD approach, divisors of GCD. So why the discrepancy? \textcolor{red}{wait}, but if the GCD is 50, then the common factors are exactly the divisors of 50. So that's six numbers, but only three are two-digit. So why did the manual check show three? But according to the list, divisors of 50 are 1,2,5,10,25,50. So two-digit are 10,25,50. That's three. So that's correct. But why did I think earlier that there were two? Because maybe I made a mistake in the first thought process. \textcolor{red}{wait}, no. \textcolor{red}{wait}, when I first considered the problem, I thought the answer is two. But then when I did the manual check, I found three. So which is correct? Let me verify each of the three numbers:10: 10 divides 100 (10*10=100) and 150 (10*15=150). So yes.25: 25*4=100, 25*6=150. Yes, so 25 divides both.50: 50*2=100, 50*3=150. Yes, so 50 divides both.So all three are valid. Therefore, the answer should be 3. But \textcolor{red}{wait}, the initial GCD approach said divisors of 50, which are six numbers, but only three are two-digit. However, when I checked the manual approach, I found three. Therefore, the answer is three? But I need to make sure there are no other two-digit numbers that divide both 100 and 150. \textcolor{red}{wait}, let's check 30. 100 divided by 30 is 3.333, not integer. .... But \textcolor{red}{wait}, let me check another way. Alternatively, since the GCD is 50, the common factors are the divisors of 50. So divisors of 50 are 1,2,5,10,25,50. So two-digit are 10,25,50. So three numbers. Therefore, the answer is 3. But why did I initially think it was two? Maybe I confused divisors of 50 with common divisors. But no, the common divisors are exactly the divisors of the GCD. So if GCD is 50, then common divisors are divisors of 50. Therefore, three numbers. \textcolor{red}{wait}, but in the initial thought process, I considered that the GCD is 50, so common factors are 1,2,5,10,25,50. Then two-digit are 10,25,50. Three numbers. Therefore, the answer is 3. But let me check once more. Let's take 25. 25 divides 100 and 150. 25*4=100, 25*6=150. Yes. 10: 10*10=100, 10*15=150. Yes. 50: 50*2=100, 50*3=150. Yes. So all three are valid. Therefore, the number of positive two-digit integers that are factors of both 100 and 150 is 3. But \textcolor{red}{wait}, hold on. Let me check 5. 5 is a divisor of both, but it's single-digit. 1, 2, 5 are also divisors but single-digit. So the only two-digit divisors are 10,25,50. So three numbers. Therefore, the final answer should be 3. But \textcolor{red}{wait}, when I first did the GCD approach, I thought it was two, but after manual checking, I found three. So why the discrepancy? ..... But \textcolor{red}{wait}, the initial problem says \"positive two-digit integers\", so numbers from 10 to 99. So 10,25,50 are within that range. So yes, three numbers. Therefore, after thorough checking, the answer is 3. So I need to make sure I didn't make a mistake in the initial GCD approach. But since GCD is 50, and common factors are divisors of 50, then three two-digit. So the answer is 3. But let me also check if 50 is indeed a two-digit number. Yes, 50 has two digits. 10 is two-digit. 25 is two-digit. So three numbers. Therefore, I think the correct answer is 3. So the final answer should be 3, boxed as 3\\
        \bottomrule
    \end{tabular}
    }
     \caption{One more examples of LLM's wrong outputs before ThinkPO and correct outputs after ThinkPO. ThinkPO helps models generate more reasoning-supportive discourse markers, such as wait, hmm, and other hesitation cues. }
    \label{fig:example2}
\end{table*}

