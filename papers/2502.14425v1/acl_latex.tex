% This must be in the first 5 lines to tell arXiv to use pdfLaTeX, which is strongly recommended.
\pdfoutput=1
% In particular, the hyperref package requires pdfLaTeX in order to break URLs across lines.

\documentclass[11pt]{article}

% Change "review" to "final" to generate the final (sometimes called camera-ready) version.
% Change to "preprint" to generate a non-anonymous version with page numbers.
\usepackage[preprint]{acl}

% Standard package includes
\usepackage{times}
\usepackage{latexsym}

% For proper rendering and hyphenation of words containing Latin characters (including in bib files)
\usepackage[T1]{fontenc}
% For Vietnamese characters
% \usepackage[T5]{fontenc}
% See https://www.latex-project.org/help/documentation/encguide.pdf for other character sets

% This assumes your files are encoded as UTF8
\usepackage[utf8]{inputenc}

% This is not strictly necessary, and may be commented out,
% but it will improve the layout of the manuscript,
% and will typically save some space.
\usepackage{microtype}

% This is also not strictly necessary, and may be commented out.
% However, it will improve the aesthetics of text in
% the typewriter font.
\usepackage{inconsolata}

%Including images in your LaTeX document requires adding
%additional package(s)
\usepackage{graphicx}

\usepackage{forest}
\usepackage{array}
\usepackage{multirow}
\usepackage{booktabs}
\usepackage{amssymb} % 加载 amssymb 包
% If the title and author information does not fit in the area allocated, uncomment the following
%
%\setlength\titlebox{<dim>}
%
% and set <dim> to something 5cm or larger.

\title{A Survey on Data Contamination for Large Language Models}

% Author information can be set in various styles:
% For several authors from the same institution:
% \author{Author 1 \and ... \and Author n \\
%         Address line \\ ... \\ Address line}
% if the names do not fit well on one line use
%         Author 1 \\ {\bf Author 2} \\ ... \\ {\bf Author n} \\
% For authors from different institutions:
% \author{Author 1 \\ Address line \\  ... \\ Address line
%         \And  ... \And
%         Author n \\ Address line \\ ... \\ Address line}
% To start a separate ``row'' of authors use \AND, as in
% \author{Author 1 \\ Address line \\  ... \\ Address line
%         \AND
%         Author 2 \\ Address line \\ ... \\ Address line \And
%         Author 3 \\ Address line \\ ... \\ Address line}

\author{Yuxing Cheng$^{1}$ \quad  Yi Chang$^{2,3,4}$ \quad Yuan Wu$^{2}$\footnotemark[1] \\
        $^{1}$College of Software, Jilin University \\ 
        $^{2}$School of Artificial Intelligence, Jilin University \\
        $^{3}$Engineering Research Center of Knowledge-Driven Human-Machine Intelligence, MOE, China \\
        $^{4}$International Center of Future Science, Jilin University\\
        chengyx0121@mails.jlu.edu.cn, yichang@jlu.edu.cn, yuanwu@jlu.edu.cn \\   
}

%\author{
%  \textbf{First Author\textsuperscript{1}},
%  \textbf{Second Author\textsuperscript{1,2}},
%  \textbf{Third T. Author\textsuperscript{1}},
%  \textbf{Fourth Author\textsuperscript{1}},
%\\
%  \textbf{Fifth Author\textsuperscript{1,2}},
%  \textbf{Sixth Author\textsuperscript{1}},
%  \textbf{Seventh Author\textsuperscript{1}},
%  \textbf{Eighth Author \textsuperscript{1,2,3,4}},
%\\
%  \textbf{Ninth Author\textsuperscript{1}},
%  \textbf{Tenth Author\textsuperscript{1}},
%  \textbf{Eleventh E. Author\textsuperscript{1,2,3,4,5}},
%  \textbf{Twelfth Author\textsuperscript{1}},
%\\
%  \textbf{Thirteenth Author\textsuperscript{3}},
%  \textbf{Fourteenth F. Author\textsuperscript{2,4}},
%  \textbf{Fifteenth Author\textsuperscript{1}},
%  \textbf{Sixteenth Author\textsuperscript{1}},
%\\
%  \textbf{Seventeenth S. Author\textsuperscript{4,5}},
%  \textbf{Eighteenth Author\textsuperscript{3,4}},
%  \textbf{Nineteenth N. Author\textsuperscript{2,5}},
%  \textbf{Twentieth Author\textsuperscript{1}}
%\\
%\\
%  \textsuperscript{1}Affiliation 1,
%  \textsuperscript{2}Affiliation 2,
%  \textsuperscript{3}Affiliation 3,
%  \textsuperscript{4}Affiliation 4,
%  \textsuperscript{5}Affiliation 5
%\\
%  \small{
%    \textbf{Correspondence:} \href{mailto:email@domain}{email@domain}
%  }
%}

\begin{document}

\maketitle
\renewcommand{\thefootnote}{\fnsymbol{footnote}}
\footnotetext[1]{Corresponding authors}

\begin{abstract}
Recent advancements in Large Language Models (LLMs) have demonstrated significant progress in various areas, such as text generation and code synthesis. However, the reliability of performance evaluation has come under scrutiny due to data contamination—the unintended overlap between training and test datasets. 
%This overlap is inclined to artificially inflate model metrics given that LLMs are trained on vast publicly scraped datasets, which are prone to overlap with test benchmarks. 
This overlap has the potential to artificially inflate model performance, as LLMs are typically trained on extensive datasets scraped from publicly available sources. These datasets often inadvertently overlap with the benchmarks used for evaluation, leading to an overestimation of the models' true generalization capabilities.
In this paper,
%benchmarks and lifecycle, 
we first examine the definition and impacts of data contamination. 
%and explore the consequences it has on the evaluation of LLMs. 
Secondly, we review methods for contamination-free evaluation, focusing on three strategies: data updating-based methods, data rewriting-based methods, and prevention-based methods. Specifically, we highlight dynamic benchmarks and LLM-driven evaluation methods. Finally, we categorize contamination detecting methods based on model information dependency: white-Box, gray-Box, and black-Box detection approaches. Our survey highlights the requirements for more rigorous evaluation protocols and proposes future directions for addressing data contamination challenges.
\end{abstract}

\section{Introduction}
\section{Introduction}
\label{section:introduction}

% redirection is unique and important in VR
Virtual Reality (VR) systems enable users to embody virtual avatars by mirroring their physical movements and aligning their perspective with virtual avatars' in real time. 
As the head-mounted displays (HMDs) block direct visual access to the physical world, users primarily rely on visual feedback from the virtual environment and integrate it with proprioceptive cues to control the avatar’s movements and interact within the VR space.
Since human perception is heavily influenced by visual input~\cite{gibson1933adaptation}, 
VR systems have the unique capability to control users' perception of the virtual environment and avatars by manipulating the visual information presented to them.
Leveraging this, various redirection techniques have been proposed to enable novel VR interactions, 
such as redirecting users' walking paths~\cite{razzaque2005redirected, suma2012impossible, steinicke2009estimation},
modifying reaching movements~\cite{gonzalez2022model, azmandian2016haptic, cheng2017sparse, feick2021visuo},
and conveying haptic information through visual feedback to create pseudo-haptic effects~\cite{samad2019pseudo, dominjon2005influence, lecuyer2009simulating}.
Such redirection techniques enable these interactions by manipulating the alignment between users' physical movements and their virtual avatar's actions.

% % what is hand/arm redirection, motivation of study arm-offset
% \change{\yj{i don't understand the purpose of this paragraph}
% These illusion-based techniques provide users with unique experiences in virtual environments that differ from the physical world yet maintain an immersive experience. 
% A key example is hand redirection, which shifts the virtual hand’s position away from the real hand as the user moves to enhance ergonomics during interaction~\cite{feuchtner2018ownershift, wentzel2020improving} and improve interaction performance~\cite{montano2017erg, poupyrev1996go}. 
% To increase the realism of virtual movements and strengthen the user’s sense of embodiment, hand redirection techniques often incorporate a complete virtual arm or full body alongside the redirected virtual hand, using inverse kinematics~\cite{hartfill2021analysis, ponton2024stretch} or adjustments to the virtual arm's movement as well~\cite{li2022modeling, feick2024impact}.
% }

% noticeability, motivation of predicting a probability, not a classification
However, these redirection techniques are most effective when the manipulation remains undetected~\cite{gonzalez2017model, li2022modeling}. 
If the redirection becomes too large, the user may not mitigate the conflict between the visual sensory input (redirected virtual movement) and their proprioception (actual physical movement), potentially leading to a loss of embodiment with the virtual avatar and making it difficult for the user to accurately control virtual movements to complete interaction tasks~\cite{li2022modeling, wentzel2020improving, feuchtner2018ownershift}. 
While proprioception is not absolute, users only have a general sense of their physical movements and the likelihood that they notice the redirection is probabilistic. 
This probability of detecting the redirection is referred to as \textbf{noticeability}~\cite{li2022modeling, zenner2024beyond, zenner2023detectability} and is typically estimated based on the frequency with which users detect the manipulation across multiple trials.

% version B
% Prior research has explored factors influencing the noticeability of redirected motion, including the redirection's magnitude~\cite{wentzel2020improving, poupyrev1996go}, direction~\cite{li2022modeling, feuchtner2018ownershift}, and the visual characteristics of the virtual avatar~\cite{ogawa2020effect, feick2024impact}.
% While these factors focus on the avatars, the surrounding virtual environment can also influence the users' behavior and in turn affect the noticeability of redirection.
% One such prominent external influence is through the visual channel - the users' visual attention is constantly distracted by complex visual effects and events in practical VR scenarios.
% Although some prior studies have explored how to leverage user blindness caused by visual distractions to redirect users' virtual hand~\cite{zenner2023detectability}, there remains a gap in understanding how to quantify the noticeability of redirection under visual distractions.

% visual stimuli and gaze behavior
Prior research has explored factors influencing the noticeability of redirected motion, including the redirection's magnitude~\cite{wentzel2020improving, poupyrev1996go}, direction~\cite{li2022modeling, feuchtner2018ownershift}, and the visual characteristics of the virtual avatar~\cite{ogawa2020effect, feick2024impact}.
While these factors focus on the avatars, the surrounding virtual environment can also influence the users' behavior and in turn affect the noticeability of redirection.
This, however, remains underexplored.
One such prominent external influence is through the visual channel - the users' visual attention is constantly distracted by complex visual effects and events in practical VR scenarios.
We thus want to investigate how \textbf{visual stimuli in the virtual environment} affect the noticeability of redirection.
With this, we hope to complement existing works that focus on avatars by incorporating environmental visual influences to enable more accurate control over the noticeability of redirected motions in practical VR scenarios.
% However, in realistic VR applications, the virtual environment often contains complex visual effects beyond the virtual avatar itself. 
% We argue that these visual effects can \textbf{distract users’ visual attention and thus affect the noticeability of redirection offsets}, while current research has yet taken into account.
% For instance, in a VR boxing scenario, a user’s visual attention is likely focused on their opponent rather than on their virtual body, leading to a lower noticeability of redirection offsets on their virtual movements. 
% Conversely, when reaching for an object in the center of their field of view, the user’s attention is more concentrated on the virtual hand’s movement and position to ensure successful interaction, resulting in a higher noticeability of offsets.

Since each visual event is a complex choreography of many underlying factors (type of visual effect, location, duration, etc.), it is extremely difficult to quantify or parameterize visual stimuli.
Furthermore, individuals respond differently to even the same visual events.
Prior neuroscience studies revealed that factors like age, gender, and personality can influence how quickly someone reacts to visual events~\cite{gillon2024responses, gale1997human}. 
Therefore, aiming to model visual stimuli in a way that is generalizable and applicable to different stimuli and users, we propose to use users' \textbf{gaze behavior} as an indicator of how they respond to visual stimuli.
In this paper, we used various gaze behaviors, including gaze location, saccades~\cite{krejtz2018eye}, fixations~\cite{perkhofer2019using}, and the Index of Pupil Activity (IPA)~\cite{duchowski2018index}.
These behaviors indicate both where users are looking and their cognitive activity, as looking at something does not necessarily mean they are attending to it.
Our goal is to investigate how these gaze behaviors stimulated by various visual stimuli relate to the noticeability of redirection.
With this, we contribute a model that allows designers and content creators to adjust the redirection in real-time responding to dynamic visual events in VR.

To achieve this, we conducted user studies to collect users' noticeability of redirection under various visual stimuli.
To simulate realistic VR scenarios, we adopted a dual-task design in which the participants performed redirected movements while monitoring the visual stimuli.
Specifically, participants' primary task was to report if they noticed an offset between the avatar's movement and their own, while their secondary task was to monitor and report the visual stimuli.
As realistic virtual environments often contain complex visual effects, we started with simple and controlled visual stimulus to manage the influencing factors.

% first user study, confirmation study
% collect data under no visual stimuli, different basic visual stimuli
We first conducted a confirmation study (N=16) to test whether applying visual stimuli (opacity-based) actually affects their noticeability of redirection. 
The results showed that participants were significantly less likely to detect the redirection when visual stimuli was presented $(F_{(1,15)}=5.90,~p=0.03)$.
Furthermore, by analyzing the collected gaze data, results revealed a correlation between the proposed gaze behaviors and the noticeability results $(r=-0.43)$, confirming that the gaze behaviors could be leveraged to compute the noticeability.

% data collection study
We then conducted a data collection study to obtain more accurate noticeability results through repeated measurements to better model the relationship between visual stimuli-triggered gaze behaviors and noticeability of redirection.
With the collected data, we analyzed various numerical features from the gaze behaviors to identify the most effective ones. 
We tested combinations of these features to determine the most effective one for predicting noticeability under visual stimuli.
Using the selected features, our regression model achieved a mean squared error (MSE) of 0.011 through leave-one-user-out cross-validation. 
Furthermore, we developed both a binary and a three-class classification model to categorize noticeability, which achieved an accuracy of 91.74\% and 85.62\%, respectively.

% evaluation study
To evaluate the generalizability of the regression model, we conducted an evaluation study (N=24) to test whether the model could accurately predict noticeability with new visual stimuli (color- and scale-based animations).
Specifically, we evaluated whether the model's predictions aligned with participants' responses under these unseen stimuli.
The results showed that our model accurately estimated the noticeability, achieving mean squared errors (MSE) of 0.014 and 0.012 for the color- and scale-based visual stimili, respectively, compared to participants' responses.
Since the tested visual stimuli data were not included in the training, the results suggested that the extracted gaze behavior features capture a generalizable pattern and can effectively indicate the corresponding impact on the noticeability of redirection.

% application
Based on our model, we implemented an adaptive redirection technique and demonstrated it through two applications: adaptive VR action game and opportunistic rendering.
We conducted a proof-of-concept user study (N=8) to compare our adaptive redirection technique with a static redirection, evaluating the usability and benefits of our adaptive redirection technique.
The results indicated that participants experienced less physical demand and stronger sense of embodiment and agency when using the adaptive redirection technique. 
These results demonstrated the effectiveness and usability of our model.

In summary, we make the following contributions.
% 
\begin{itemize}
    \item 
    We propose to use users' gaze behavior as a medium to quantify how visual stimuli influences the noticebility of redirection. 
    Through two user studies, we confirm that visual stimuli significantly influences noticeability and identify key gaze behavior features that are closely related to this impact.
    \item 
    We build a regression model that takes the user's gaze behavioral data as input, then computes the noticeability of redirection.
    Through an evaluation study, we verify that our model can estimate the noticeability with new participants under unseen visual stimuli.
    These findings suggest that the extracted gaze behavior features effectively capture the influence of visual stimuli on noticeability and can generalize across different users and visual stimuli.
    \item 
    We develop an adaptive redirection technique based on our regression model and implement two applications with it.
    With a proof-of-concept study, we demonstrate the effectiveness and potential usability of our regression model on real-world use cases.

\end{itemize}

% \delete{
% Virtual Reality (VR) allows the user to embody a virtual avatar by mirroring their physical movements through the avatar.
% As the user's visual access to the physical world is blocked in tasks involving motion control, they heavily rely on the visual representation of the avatar's motions to guide their proprioception.
% Similar to real-world experiences, the user is able to resolve conflicts between different sensory inputs (e.g., vision and motor control) through multisensory integration, which is essential for mitigating the sensory noise that commonly arises.
% However, it also enables unique manipulations in VR, as the system can intentionally modify the avatar's movements in relation to the user's motions to achieve specific functional outcomes,
% for example, 
% % the manipulations on the avatar's movements can 
% enabling novel interaction techniques of redirected walking~\cite{razzaque2005redirected}, redirected reaching~\cite{gonzalez2022model}, and pseudo haptics~\cite{samad2019pseudo}.
% With small adjustments to the avatar's movements, the user can maintain their sense of embodiment, due to their ability to resolve the perceptual differences.
% % However, a large mismatch between the user and avatar's movements can result in the user losing their sense of embodiment, due to an inability to resolve the perceptual differences.
% }

% \delete{
% However, multisensory integration can break when the manipulation is so intense that the user is aware of the existence of the motion offset and no longer maintains the sense of embodiment.
% Prior research studied the intensity threshold of the offset applied on the avatar's hand, beyond which the embodiment will break~\cite{li2022modeling}. 
% Studies also investigated the user's sensitivity to the offsets over time~\cite{kohm2022sensitivity}.
% Based on the findings, we argue that one crucial factor that affects to what extent the user notices the offset (i.e., \textit{noticeability}) that remains under-explored is whether the user directs their visual attention towards or away from the virtual avatar.
% Related work (e.g., Mise-unseen~\cite{marwecki2019mise}) has showcased applications where adjustments in the environment can be made in an unnoticeable manner when they happen in the area out of the user's visual field.
% We hypothesize that directing the user's visual attention away from the avatar's body, while still partially keeping the avatar within the user's field-of-view, can reduce the noticeability of the offset.
% Therefore, we conduct two user studies and implement a regression model to systematically investigate this effect.
% }

% \delete{
% In the first user study (N = 16), we test whether drawing the user's visual attention away from their body impacts the possibility of them noticing an offset that we apply to their arm motion in VR.
% We adopt a dual-task design to enable the alteration of the user's visual attention and a yes/no paradigm to measure the noticeability of motion offset. 
% The primary task for the user is to perform an arm motion and report when they perceive an offset between the avatar's virtual arm and their real arm.
% In the secondary task, we randomly render a visual animation of a ball turning from transparent to red and becoming transparent again and ask them to monitor and report when it appears.
% We control the strength of the visual stimuli by changing the duration and location of the animation.
% % By changing the time duration and location of the visual animation, we control the strengths of attraction to the users.
% As a result, we found significant differences in the noticeability of the offsets $(F_{(1,15)}=5.90,~p=0.03)$ between conditions with and without visual stimuli.
% Based on further analysis, we also identified the behavioral patterns of the user's gaze (including pupil dilation, fixations, and saccades) to be correlated with the noticeability results $(r=-0.43)$ and they may potentially serve as indicators of noticeability.
% }

% \delete{
% To further investigate how visual attention influences the noticeability, we conduct a data collection study (N = 12) and build a regression model based on the data.
% The regression model is able to calculate the noticeability of the offset applied on the user's arm under various visual stimuli based on their gaze behaviors.
% Our leave-one-out cross-validation results show that the proposed method was able to achieve a mean-squared error (MSE) of 0.012 in the probability regression task.
% }

% \delete{
% To verify the feasibility and extendability of the regression model, we conduct an evaluation study where we test new visual animations based on adjustments on scale and color and invite 24 new participants to attend the study.
% Results show that the proposed method can accurately estimate the noticeability with an MSE of 0.014 and 0.012 in the conditions of the color- and scale-based visual effects.
% Since these animations were not included in the dataset that the regression model was built on, the study demonstrates that the gaze behavioral features we extracted from the data capture a generalizable pattern of the user's visual attention and can indicate the corresponding impact on the noticeability of the offset.
% }

% \delete{
% Finally, we demonstrate applications that can benefit from the noticeability prediction model, including adaptive motion offsets and opportunistic rendering, considering the user's visual attention. 
% We conclude with discussions of our work's limitations and future research directions.
% }

% \delete{
% In summary, we make the following contributions.
% }
% % 
% \begin{itemize}
%     \item 
%     \delete{
%     We quantify the effects of the user's visual attention directed away by stimuli on their noticeability of an offset applied to the avatar's arm motion with respect to the user's physical arm. 
%     Through two user studies, we identified gaze behavioral features that are indicative of the changes in noticeability.
%     }
%     \item 
%     \delete{We build a regression model that takes the user's gaze behavioral data and the offset applied to the arm motion as input, then computes the probability of the user noticing the offset.
%     Through an evaluation study, we verified that the model needs no information about the source attracting the user's visual attention and can be generalizable in different scenarios.
%     }
%     \item 
%     \delete{We demonstrate two applications that potentially benefit from the regression model, including adaptive motion offsets and opportunistic rendering.
%     }

% \end{itemize}

\begin{comment}
However, users will lose the sense of embodiment to the virtual avatars if they notice the offset between the virtual and physical movements.
To address this, researchers have been exploring the noticing threshold of offsets with various magnitudes and proposing various redirection techniques that maintain the sense of embodiment~\cite{}.

However, when users embody virtual avatars to explore virtual environments, they encounter various visual effects and content that can attract their attention~\cite{}.
During this, the user may notice an offset when he observes the virtual movement carefully while ignoring it when the virtual contents attract his attention from the movements.
Therefore, static offset thresholds are not appropriate in dynamic scenarios.

Past research has proposed dynamic mapping techniques that adapted to users' state, such as hand moving speed~\cite{frees2007prism} or ergonomically comfortable poses~\cite{montano2017erg}, but not considering the influence of virtual content.
More specifically, PRISM~\cite{frees2007prism} proposed adjusting the C/D ratio with a non-linear mapping according to users' hand moving speed, but it might not be optimal for various virtual scenarios.
While Erg-O~\cite{montano2017erg} redirected users' virtual hands according to the virtual target's relative position to reduce physical fatigue, neglecting the change of virtual environments. 

Therefore, how to design redirection techniques in various scenarios with different visual attractions remains unknown.
To address this, we investigate how visual attention affects the noticing probability of movement offsets.
Based on our experiments, we implement a computational model that automatically computes the noticing probability of offsets under certain visual attractions.
VR application designers and developers can easily leverage our model to design redirection techniques maintaining the sense of embodiment adapt to the user's visual attention.
We implement a dynamic redirection technique with our model and demonstrate that it effectively reduces the target reaching time without reducing the sense of embodiment compared to static redirection techniques.

% Need to be refined
This paper offers the following contributions.
\begin{itemize}
    \item We investigate how visual attractions affect the noticing probability of redirection offsets.
    \item We construct a computational model to predict the noticing probability of an offset with a given visual background.
    \item We implement a dynamic redirection technique adapting to the visual background. We evaluate the technique and develop three applications to demonstrate the benefits. 
\end{itemize}



First, we conducted a controlled experiment to understand how users perceived the movement offset while subjected to various distractions.
Since hand redirection is one of the most frequently used redirections in VR interactions, we focused on the dynamic arm movements and manually added angular offsets to the' elbow joint~\cite{li2022modeling, gonzalez2022model, zenner2019estimating}. 
We employed flashing spheres in the user's field of view as distractions to attract users' visual attention.
Participants were instructed to report the appearing location of the spheres while simultaneously performing the arm movements and reporting if they perceived an offset during the movement. 
(\zhipeng{Add the results of data collection. Analyze the influence of the distance between the gaze map and the offset.}
We measured the visual attraction's magnitude with the gaze distribution on it.
Results showed that stronger distractions made it harder for users to notice the offset.)
\zhipeng{Need to rewrite. Not sure to use gaze distribution or a metric obtained from the visual content.}
Secondly, we constructed a computational model to predict the noticing probability of offsets with given visual content.
We analyzed the data from the user studies to measure the influence of visual attractions on the noticing probability of offsets.
We built a statistical model to predict the offset's noticing probability with a given visual content.
Based on the model, we implement a dynamic redirection technique to adjust the redirection offset adapted to the user's current field of view.
We evaluated the technique in a target selection task compared to no hand redirection and static hand redirection.
\zhipeng{Add the results of the evaluation.}
Results showed that the dynamic hand redirection technique significantly reduced the target selection time with similar accuracy and a comparable sense of embodiment.
Finally, we implemented three applications to demonstrate the potential benefits of the visual attention adapted dynamic redirection technique.
\end{comment}

% This one modifies arm length, not redirection
% \citeauthor{mcintosh2020iteratively} proposed an adaptation method to iteratively change the virtual avatar arm's length based on the primary tasks' performance~\cite{mcintosh2020iteratively}.



% \zhipeng{TO ADD: what is redirection}
% Redirection enables novel interactions in Virtual Reality, including redirected walking, haptic redirection, and pseudo haptics by introducing an offset to users' movement.
% \zhipeng{TO ADD: extend this sentence}
% The price of this is that users' immersiveness and embodiment in VR can be compromised when they notice the offset and perceive the virtual movement not as theirs~\cite{}.
% \zhipeng{TO ADD: extend this sentence, elaborate how the virtual environment attracts users' attention}
% Meanwhile, the visual content in the virtual environment is abundant and consistently captures users' attention, making it harder to notice the offset~\cite{}.
% While previous studies explored the noticing threshold of the offsets and optimized the redirection techniques to maintain the sense of embodiment~\cite{}, the influence of visual content on the probability of perceiving offsets remains unknown.  
% Therefore, we propose to investigate how users perceive the redirection offset when they are facing various visual attractions.


% We conducted a user study to understand how users notice the shift with visual attractions.
% We used a color-changing ball to attract the user's attention while instructing users to perform different poses with their arms and observe it meanwhile.
% \zhipeng{(Which one should be the primary task? Observe the ball should be the primary one, but if the primary task is too simple, users might allocate more attention on the secondary task and this makes the secondary task primary.)}
% \zhipeng{(We need a good and reasonable dual-task design in which users care about both their pose and the visual content, at least in the evaluation study. And we need to be able to control the visual content's magnitude and saliency maybe?)}
% We controlled the shift magnitude and direction, the user's pose, the ball's size, and the color range.
% We set the ball's color-changing interval as the independent factor.
% We collect the user's response to each shift and the color-changing times.
% Based on the collected data, we constructed a statistical model to describe the influence of visual attraction on the noticing probability.
% \zhipeng{(Are we actually controlling the attention allocation? How do we measure the attracting effect? We need uniform metrics, otherwise it is also hard for others to use our knowledge.)}
% \zhipeng{(Try to use eye gaze? The eye gaze distribution in the last five seconds to decide the attention allocation? Basically constructing a model with eye gaze distribution and noticing probability. But the user's head is moving, so the eye gaze distribution is not aligned well with the current view.)}

% \zhipeng{Saliency and EMD}
% \zhipeng{Gaze is more than just a point: Rethinking visual attention
% analysis using peripheral vision-based gaze mapping}

% Evaluation study(ideal case): based on the visual content, adjusting the redirection magnitude dynamically.

% \zhipeng{(The risk is our model's effect is trivial.)}

% Applications:
% Playing Lego while watching demo videos, we can accelerate the reaching process of bricks, and forbid the redirection during the manipulation.

% Beat saber again: but not make a lot of sense? Difficult game has complicated visual effects, while allows larger shift, but do not need large shift with high difficulty



\section{What is Data Contamination}
\label{sec:what}
\subsection{Definition}

%In recent years, an increasing number of studies are proposed to address data contamination, but there is no unified definition or standard methodology to summarize data contamination. \citet{brown2020language} first highlighted pre-training data contamination using an N-gram diagnostic method, showing how data contamination inflates model performance. \citet{hartmann2023sok} has explored how memorization of LLMs is linked to data contamination, as memorization and contamination both involve regurgitating pre-training data. \citet{schwarzschild2024rethinkingllmmemorizationlens} suggests strings can be considered memorized if they can be reproduced using a shorter prompt. \citet{karamolegkou-etal-2023-copyright} explores verbatim memorization, particularly with copyrighted materials. Our research extends this framework in two ways: (1) vulnerabilities across the LLMs' lifecycle (including pre-training, fine-tuning, and post-deployment contamination), and (2) risks to benchmark integrity (including data manipulation and potential label leakage).

In recent years, a growing body of research has emerged to address the issue of data contamination in LLMs. However, the field lacks a unified definition or standardized methodology to comprehensively summarize data contamination. \citet{brown2020language} was among the first to highlight pre-training data contamination, employing an N-gram diagnostic method to demonstrate how contamination artificially inflates model performance. \citet{hartmann2023sok} further explored the connection between LLM memorization and data contamination, noting that both phenomena involve the regurgitation of pre-training data. \citet{schwarzschild2024rethinkingllmmemorizationlens} proposed that strings can be considered memorized if they can be reproduced using a shorter prompt, while \citet{karamolegkou-etal-2023-copyright} investigated verbatim memorization, particularly in the context of copyrighted materials. Building on these foundational studies, our research extends the framework into two significant directions: (1) examining vulnerabilities across the entire lifecycle of LLMs, including pre-training, fine-tuning, and post-deployment contamination, and (2) addressing risks to benchmark integrity, such as data manipulation and potential label leakage.

\subsubsection{Phase-based Contamination}
%For phase-based contamination, recent research outlines stage-based contamination risks in LLM development: pre-training (test data leakage into corpora), fine-tuning (unintentional exposure), and post-deployment (bias absorption from real-world interactions).  \citet{sainz-etal-2023-nlp} systematically mapped contamination pathways across these critical phases, while \citet{balloccu-etal-2024-leak} introduced the concept of indirect data contamination, where human interactions during LLM training inadvertently introduce biases, even when explicit test data inclusion is excluded. Moreover, multimodal large language models (MLLMs) face amplified contamination challenges due to multi-modal data integration, complicating data integrity preservation. \citet{song2024textimagesleakedsystematic} proposed a bimodal taxonomy, distinguishing between unimodal contamination and cross-modal contamination, while developing traceability frameworks specifically for MLLMs. 

For phase-based contamination, recent research has identified stage-specific contamination risks throughout the lifecycle of LLMs: pre-training (where test data may leak into training corpora), fine-tuning (where models are unintentionally exposed to evaluation data), and post-deployment (where models absorb biases from real-world interactions). \citet{sainz-etal-2023-nlp} systematically mapped contamination pathways across these critical phases, while \citet{balloccu-etal-2024-leak} introduced the concept of indirect data contamination, highlighting how human interactions during LLM training can inadvertently introduce biases, even in the absence of explicit test data inclusion. Furthermore, multimodal large language models (MLLMs) face heightened contamination challenges due to the integration of diverse data modalities~\cite{yin2023survey}. \citet{song2024textimagesleakedsystematic} proposed a bimodal taxonomy, distinguishing between unimodal contamination and cross-modal contamination, and developed traceability frameworks tailored specifically for MLLMs.

\subsubsection{Benchmark-based Contamination}
%For benchmark-based contamination, previous papers have generally discussed it in two main ways. One approach primarily focuses on whether labels are leaked and whether samples are rewritten. The other approach classifies contamination at the instance level or dataset level. \citet{yang2023rethinkingbenchmarkcontaminationlanguage} considered simple rewording as contamination, including synonym substitution and the use of translation. \citet{yao-etal-2024-data} revealed cross-language contamination through option rewriting detection. In the context of code generation tasks, \citet{palavalli-etal-2024-taxonomy} established a systematic taxonomy that contamination is categorized into dataset-level (e.g., test data leakage or mixing) and instance-level (e.g., output masking, input/output rewriting, or augmentation). Further expanding on this, \citet{matton-etal-2024-leakage} identified three sources of contamination and proposed the LBPP benchmark as a countermeasure. \citet{fu2024does} defined data contamination at both instance and dataset levels, providing formal mathematical definitions for instance-level contamination (via membership inference attacks) and dataset-level contamination (both full and partial contamination).

For benchmark-based contamination, prior research has generally approached the issue from two primary perspectives. The first focuses on whether labels are leaked or whether samples are rewritten, while the second categorizes contamination at either the instance level or the dataset level. \citet{yang2023rethinkingbenchmarkcontaminationlanguage} considered even simple rewording—such as synonym substitution or translation—as a form of contamination. \citet{yao-etal-2024-data} further revealed cross-language contamination through the detection of option rewriting. In code generation tasks, \citet{palavalli-etal-2024-taxonomy} established a systematic taxonomy, categorizing contamination into dataset-level (e.g., test data leakage or mixing) and instance-level (e.g., output masking, input/output rewriting, or augmentation). Expanding on this, \citet{matton-etal-2024-leakage} identified three distinct sources of contamination and proposed the LBPP benchmark as a countermeasure. Additionally, \citet{fu2024does} provided formal mathematical definitions for contamination at both the instance and dataset levels, defining instance-level contamination through membership inference attacks and dataset-level contamination as either full or partial contamination.

\subsection{Impacts}

Data contamination critically undermines evaluation reliability and research validity. As \cite{sainz-etal-2023-nlp} demonstrated, benchmark overfitting can artificially inflate model performance and compromise scientific conclusions in NLP studies. \cite{singh2024evaluation} identified two principal analysis approaches: causal analysis through controlled retraining experiments, and post-hoc contamination inference via performance pattern examination without model retraining.

\subsubsection{Evidence Collection}
\label{subsubsec:evidence}
Initial contamination investigation focuses on temporal data analysis and adversarial detection methods. \citet{li2024task} proposed evaluating models on pre/post-training datasets with membership inference attacks, revealing contamination effects on zero/few-shot performance. \citet{riddell2024quantifyingcontaminationevaluatingcode} demonstrated performance inflation on seen HumanEval/MBPP samples, while \citet{cao2024concerneddatacontaminationassessing} validated contamination mitigation strategy through using the most recent benchmarks. \citet{jiang2024investigatingdatacontaminationpretraining} differentiated between text contamination (input samples) and true contamination (input-output pairs). \citet{liu-etal-2024-evaluating} exposed Chinese LLMs' superficial knowledge despite broad training exposure. \citet{sainz-etal-2023-nlp} highlighted that current evidence on contamination remains fragmented across publications and informal channels, suggesting that the prevalence of contamination may be significantly underestimated.

%\citet{sainz-etal-2023-nlp} indicated current evidence remains fragmented across publications and informal channels, suggesting underestimated contamination prevalence.

\subsubsection{Factors Discussion}
In this section, we discuss some factors influencing contamination. \citet{magar-schwartz-2022-data} found that exploitation of contaminated data is influenced by factors like model size, learning rate, and the position of contaminated data, suggesting that memorization does not always lead to exploitation. \citet{mehrbakhsh-etal-2024-confounders} designed GPT-4-generated templates to investigate how the complexity of test instances influences the contamination in Llama-2 7B, aiming to better understand how varying levels of difficulty and diversity in the templates can influence the model's performance.
\citet{singh2024evaluation} proposed a new contamination evaluation protocol, ConTAM, to explore how data contamination affects the evaluation results of LLMs, and provided a method to quantify the impact of contamination.

\subsubsection{Non-Contamination Scenarios}
%In this section, we discuss Non-Contamination scenarios. \citet{dekoninck2024constat} established a causal relationship between model performance improvement and data contamination, defining the cases where overlap between training and testing data exists while it does not lead to performance improvement as non-contamination. \citet{palavalli-etal-2024-taxonomy} clarified several phenomena that improve performance on downstream tasks but are not regarded as influenced by contamination, including language understanding, prior task understanding, and transductive learning. These phenomena contribute to better empirical results without violating the integrity of the task or model.
In this section, we explore non-contamination scenarios, where the overlap between training and testing data does not lead to performance improvement. \citet{dekoninck2024constat} established a causal relationship between model performance improvement and data contamination, explicitly defining cases where such overlap exists but does not enhance performance as non-contamination. Furthermore, \citet{palavalli-etal-2024-taxonomy} clarified several phenomena that improve performance on downstream tasks without being influenced by contamination. These include language understanding, prior task understanding, and transductive learning. These phenomena enhance empirical results while preserving the integrity of both the task and the model, distinguishing them from contamination-related performance gains.

\subsubsection{Quantifying contamination}
\label{chap:quantifying}
Contamination scoring mechanisms classify evaluation samples through threshold-based indices. We have summarized some common contamination detection methods in table \ref{tab:ngram}. For instance, \citet{brown2020language} used N-grams to evaluate contamination by checking whether each token in the tested sample appears in an n-gram from the pre-training corpus. \citet{chowdhery2023palm} calculated the contamination score based on the proportion of contaminated n-grams. In contrast, \citet{touvron2023llama2} introduced a method to align extensions between the testing samples and pre-training corpus, allowing mismatches in certain token positions using a "skip\_budget" hyperparameter. \citet{singh2024evaluation} further extended this method, focusing on the longest contaminated token span rather than all potential matches. \citet{riddell2024quantifyingcontaminationevaluatingcode} employed the Dolos toolkit \cite{maertens2022dolos} to measure semantic similarity by converting programs into abstract syntax trees (ASTs) and performing k-gram matching.





\section{How to Avoid Data Contamination}
\label{sec:where}
This section discusses methods to avoid data contamination in evaluation. First, to reduce risks, benchmarks are often constructed following three strategies: Data updating-based methods, Data rewriting-based methods, and prevention-based methods. Second, Dynamic evaluation generates adaptive samples using techniques like algorithmic composition, graph structures, randomization, and reasoning graphs, ensuring controlled complexity and diversity. Finally, LLM-as-a-evaluator eliminates contamination risks, making it a key for contamination-free evaluation.

\subsection{Benchmark Contamination-free Strategies}

Contamination-free benchmarking strategies ensure datasets stay up-to-date, preventing models from using outdated data. Rewriting construction combines human efforts like manual labeling with LLM-assisted techniques such as rephrasing to avoid contamination. Preventive measures involve technical defenses like encryption, access control, and de-contamination during inference to guarantee the reliability and fairness of LLM evaluation.

\subsubsection{Data Updating-based Methods}

Using the most recent data is intuitive for constructing contamination-free benchmarks, and some studies have proposed automatically collecting recent data to build questions.
LatestEval proposed an automated pipeline to dynamically generate contamination-free test sets from recent materials~\citep{li2024latestevaladdressingdatacontamination}. 
\citet{white2024livebench} introduced LiveBench, a dynamically updated benchmark that integrates tasks across math, coding, and reasoning with automated scoring to mitigate data contamination and evaluation biases. Similarly, \citet{jain2024livecodebench} developed LiveCodeBench, a code-specific benchmark that expands beyond HumanEval~\cite{chen2021codex} and MBPP~\citep{austin2021program} by assessing self-repair and prediction abilities while ensuring periodic updates. To evaluate LLMs' world knowledge, \citet{yu2023kola} introduced the KoLA benchmark, which combines stable knowledge sources (e.g., Wikipedia) with recent data to balance evaluation fairness and contamination prevention.
\citet{zhang-etal-2024-pretraining} introduced PatentMIA, crawling Chinese patent data from Google Patents. This dataset contains 5,000 patents with a publication date after March 1, 2024, and 5,000 patents published before January 1, 2023.
\citet{haimes2024benchmark} proposed to use retro-holdout datasets to detect public benchmark influence on model training and measure discrepancies between benchmark results and real-world performance.
\citet{fan2024nphardeval4vdynamicreasoningbenchmark} introduced NPHardEval4V-a dynamically updated benchmark to assess reasoning capabilities of MLLMs.
In code evaluation, EvoCodeBench is proposed to dynamically align with real-world code repositories to guarantee fair evaluation.

\subsubsection{Data Rewriting-based Methods}

This type of methods use data augmentation to remove contamination from benchmarks, with LLMs' superior rephrasing and verifying capabilities. Human intervention is also integrated into the rewriting process to create novel data with a distribution similar to the original data.
\citet{zhu-etal-2024-clean} proposed Clean-Eval to purify contaminated benchmarks by paraphrasing and back-translating data into semantically equivalent but lexically distinct forms.
\citet{zhao2024mmlu} proposed the MMLU-CF dataset, which is constructed by collecting diverse questions, cleaning data, sampling difficulty reasonably, checking data integrity with LLMs, and applying rewriting methods such as rephrasing questions and shuffling options to ensure the dataset remains contamination-free.
\citet{zhang2024careful} provided GSM1k, employing manual labeling, three-tier quality control, and leak prevention design to avoid data contamination. Through meticulous human construction, GSM1k achieves high similarity to GSM8k\cite{cobbe2021gsm8k} in style, difficulty, and human solve rates, while maintaining complete content independence.
Meanwhile, LLMs can serve as assistants for rewriting or generating questions. CLEVA is generated by non-repetitive sampling for each evaluation round. Each test sample is further enhanced with multiple data rewriting strategies before being used to assess LLMs, significantly mitigating the risk of data contamination~\citep{li-etal-2023-cleva}. \citet{ying2024automating} updated benchmarks with two strategies: style-preserving mimicry with LLMs and cognitive-level expansion using Bloom's taxonomy. Similarly, \citet{zhu2024dynamicevaluationlargelanguageMPA} proposed Multi-Principle Assessment (MPA), which utilizes LLM-based agents to automatically transform existing questions into new ones. \citet{wang2024benchmark} introduced a multi-agent framework to implement self-evolving benchmarks, which dynamically mutates question contexts and structures to update benchmarks.

\subsubsection{Prevention-based Methods}
Preventive measures focus on safeguarding test data integrity through technical and procedural controls. Core strategies include encrypting public test data with public-key cryptography, enforcing strict access permissions, and prohibiting derivative data creation. \citet{zhu2024inference} introduced Inference-Time Decontamination (ITD), a novel technique that identifies and rewrites potentially memorized responses during model inference. \citet{li2024c2levacomprehensivecontaminationfreelanguage} introduced C\textsuperscript{2}LEVA, a comprehensive bilingual benchmark with systematic contamination prevention mechanisms, which implements proactive measures such as test data rotation and enhanced encryption.

\subsection{Dynamic Evaluation}
\label{sec:dynamic evaluation}
%Dynamic approaches combat contamination through adaptive assessment frameworks. \citet{zhu2024dyvaldynamicevaluationlarge} pioneered DYVAL, a graph-based system generating evaluation samples through algorithmic composition, constraint application, and functional description. Its directed acyclic graph architecture enables multi-step reasoning tasks with controlled complexity. \citet{lei-etal-2024-s3eval} developed S3EVAL for SQL evaluation through randomized table-query pairs. This synthetic approach enables customizable task lengths and difficulty levels while systematically testing long-context reasoning. \citet{zhang2024darg} introduced the DARG method, which dynamically generates evaluation samples with controllable complexity and diversity through adaptive reasoning graphs and verifies the correctness of labels using tool-augmented LLMs.  \citet{srivastava2024functional} proposed functionalization-converting static QAs into parameterized code that generates infinite test variants. \citet{qian2024varbench} extended dynamic evaluation through controlled question key variable perturbation, to dynamically generate dataset.

Dynamic approaches address data contamination by leveraging adaptive assessment frameworks. \citet{zhu2024dyvaldynamicevaluationlarge} introduced DYVAL, a graph-based system that generates evaluation samples through algorithmic composition, constraint application, and functional descriptions. Its directed acyclic graph (DAG) architecture facilitates multi-step reasoning tasks with precisely controlled complexity. \citet{lei-etal-2024-s3eval} developed S3EVAL, a framework for SQL evaluation that utilizes randomized table-query pairs. This synthetic approach allows for customizable task lengths and difficulty levels, while systematically assessing long-context reasoning capabilities. \citet{zhang2024darg} proposed the DARG method, which dynamically generates evaluation samples with adjustable complexity and diversity using adaptive reasoning graphs. \citet{srivastava2024functional} introduced functionalization, a technique that transforms static question-answer pairs into parameterized code, enabling the generation of infinite test variants. \citet{qian2024varbench} further extended dynamic evaluation by perturbing key variables in questions, allowing for the dynamic generation of datasets with controlled variations.


\subsection{LLM-as-a-Evaluator}
\label{sec:llm driven evaluation}
Next-generation evaluation leverages LLMs themselves as assessment tools. 
LLMs are no longer just "artisans" of content generation; they have become "judges" of content quality. They can serve the roles of scoring, ranking, and selection. \citet{bai2024benchmarking} presented the "LM-as-Examiner" framework, generating questions and evaluating responses through reference-free analysis. \citet{yu-etal-2024-kieval} deployed LLMs as "Interactors" in structured multi-turn dialogues that probe model capabilities while minimizing contamination risks.
\citet{li2024treeevalbenchmarkfreeevaluationlarge} proposed TreeEval-a benchmark-free system where LLMs generate hierarchical question trees. This adaptive approach adjusts difficulty based on model performance, creating unique assessment paths that prevent data contamination.

\section{How to Detect Data Contamination}
\label{sec:how}

%The definition of data contamination detection is the process of determining, through a specific method, whether a given text or dataset has appeared in the training corpus of a particular model. Data contamination detection has emerged as a critical challenge in LLM evaluation. We categorize detection approaches into three paradigms based on model information accessibility. This taxonomy reveals an evolving detection landscape where white-box methods provide high precision but limited applicability, gray-box approaches balance practicality and effectiveness, and black-box technologies rely on heuristic assumptions (appendix \ref{sec:assumption detail}). As specialized methods continue to emerge, the community's awareness of the potential for data contamination to distort evaluation results is gradually increasing. Therefore, we provide descriptions of some contamination detection tools in Appendix \ref{sec:Data contamination Detector}.

The definition of data contamination detection refers to the process of determining, through a specific methodology, whether a given text or dataset has been included in the training corpus of a particular model. As LLMs continue to advance, data contamination detection has emerged as a critical challenge in model evaluation. Here, we categorize detection approaches into three paradigms based on the level of access to model information: white-box, gray-box, and black-box methods. This taxonomy highlights an evolving detection landscape. White-box methods, which leverage full access to model architectures or training data, offer high precision but are often limited in applicability. Gray-box approaches, which utilize partial model information, strike a balance between practicality and effectiveness. Black-box technologies rely on heuristic assumptions (detailed in Appendix \ref{sec:assumption detail}) and operate without access to internal model details. As specialized detection methods continue to emerge, the research community is increasingly recognizing the importance of data contamination to distort evaluation outcomes. To support this growing awareness, we provide detailed descriptions of several contamination detection tools in Appendix \ref{sec:Data contamination Detector}.

\subsection{White-Box Detection}
\begin{table*}[ht]
\centering
\begin{tabular}{|>{\centering\arraybackslash}p{2cm}|>{\centering\arraybackslash}p{4cm}|>{\centering\arraybackslash}p{4cm}|>{\centering\arraybackslash}p{3cm}|}
\hline
\textbf{Model}   & \textbf{Author} & \textbf{Method} & \textbf{Token-Level} \\ 
\hline
\textbf{GPT3} & \cite{brown2020language}   & n-gram (n=13)       & \texttimes    \\ \hline
\textbf{Palm} & \cite{chowdhery2023palm}    & n-gram        & \checkmark      \\ \hline
\textbf{Llama2} & \cite{touvron2023llama2} & extended n-gram      & \checkmark     \\ \hline
\textbf{GPT4} & \cite{openai2024gpt4technicalreport}  & n-gram (n=50)    & \texttimes  \\ \hline
\textbf{Phi-4} & \cite{abdin2024phi4technicalreport}  & hybrid n-gram   & \texttimes    \\ \hline
\end{tabular}
\caption{N-gram method used for contamination detection, Token-level refers to the standard for measuring contamination scores using tokens.}
\label{tab:ngram}
\end{table*}


White-box methods directly utilize model internals or training data to detect data contamination. When pre-training corpora are accessible, content overlap with evaluation datasets can be explicitly measured~\cite{elangovan-etal-2021-memorization}. Prominent LLMs including LLaMA2 \cite{touvron2023llama2}, PaLM \cite{chowdhery2023palm}, and GPT-4 \cite{achiam2023gpt} all emphasize the necessity of detecting pre-training/evaluation overlaps. The n-gram overlap method, prioritized for its computational efficiency and simplicity, has become a standard tool for detecting contamination. Comparative implementations of these n-gram based overlap detection strategies are systematically summarized in Table \ref{tab:ngram}.

Embeddings similarity compares texts via cosine similarity of their embeddings, capturing semantic relationships beyond lexical variations~\cite{reimers2019sentence}. \citet{lee2023platypus} used a similarity exclusion method based on embeddings, reducing dataset redundancy and filtering out duplicate data to ensure clean training data. To address sophisticated contamination forms, \cite{yang2023rethinkingbenchmarkcontaminationlanguage} introduced a hybrid approach combining embedding similarity search with GPT-4 powered semantic analysis. This detects paraphrased samples, enabling proactive benchmark decontamination.

For known model weights,
\citet{tu2024dicedetectingindistributioncontamination} proposed DICE to identify in-distribution contamination during fine-tuning by analyzing layer-specific activation patterns. This method trains contamination classifiers on sensitive intermediate layers, demonstrating a strong correlation between detection signals and performance inflation across multiple LLMs.


\subsection{Gray-Box Detection}
Gray-box approaches in membership inference attacks (MIAs) leverage partial model information such as token probabilities to distinguish training data from non-members. \citet{duan2024membership} systematically investigated the underwhelming MIA performance on LLMs, identifying three primary contributing factors: the massive scale of training datasets that complicates memorization patterns, the limited number of training iterations that reduce model overfitting, and the inherently fuzzy decision boundaries between member and non-member samples. To address these shortcomings, the MIN-K\% method established token-based effective methods using outlier token probabilities for pretraining data detection~\cite{shi2024detectingpretrainingdatalarge}. \citet{zhang2024minkimprovedbaselinedetecting} subsequently proposed Min-K\%++, theoretically grounding detection in local probability maxima identification. \citet{zhang-etal-2024-pretraining} proposed DC-PDD to employ corpus frequency divergences to reduce false positives. \citet{ye-etal-2024-data} introduced PAC, an MIA method that calculates polarization distances through input perturbations. \citet{zhang-etal-2024-pacost} developed PaCoST, which statistically compares model confidence on original test items versus distributionally-similar counterparts, to reveal widespread contamination across open-source models.

Alternative gray-box strategies, including perplexity-based memorization detection \cite{li2023estimatingcontaminationperplexityquantifying} and the adversarial compression ratio (ACR) metric \cite{schwarzschild2024rethinkingllmmemorizationlens}, quantify memorization through input-output token efficiency.

\subsection{Black-Box Detection}

\begin{table*}[ht]
\centering
\begin{tabular}{|c|c|c|c|}
\hline
\textbf{Method} & \textbf{Authors} & \textbf{Assumption} & \textbf{Certain Tasks}\\
\hline
\textbf{DCQ} & \cite{golchin2023data} & Verbatim-Memorization  & \texttimes \\
\hline
\textbf{Guided Instruction} & \cite{golchin2023time} & Verbatim-Memorization &\texttimes\\
\hline
\textbf{DE-COP} & \cite{duarte2024decopdetectingcopyrightedcontent} & Verbatim-Memorization &\texttimes\\
\hline
\textbf{CDD} & \cite{dong-etal-2024-generalization} & Output Distribution  &\texttimes \\
\hline
\textbf{TS-Guessing} & \cite{deng2023investigating} & Verbatim-Memorization  &\texttimes \\
\hline
\textbf{ATD} & \cite{ranaldi-etal-2024-investigating} & Verbatim-Memorization & \checkmark\\
\hline
\textbf{Data Archaeology} & \cite{chang-etal-2023-speak} & Verbatim-Memorization & \checkmark\\
\hline
\end{tabular}
\caption{Black-box contamination detection methods, details of the assumptions underlying these approaches can be found in Appendix \ref{sec:assumption detail}.}
\label{tab:contamination method assumption}
\end{table*}

Black-box methods operate without access to model internals, training corpus, and are often accompanied by limitations in computational resources.
Specifically, these methods heavily rely on certain assumptions shown in Appendix \ref{sec:definition of assumptions}.

%\citet{golchin2023data} designed a multiple-choice question framework where each question presents an original instance alongside three perturbed versions (words are replaced with contextually relevant synonyms) and one invalid option. If the LLM frequently selects the original instance, it suggests that the model may be influenced by data contamination. \citet{golchin2023time} proposed a detection method based on "guided instruction", which effectively identifies contamination in datasets through instance completion and heuristic evaluation, offering a robust solution for data contamination.


\citet{golchin2023data} introduced a multiple-choice question framework in which each question presents an original instance alongside three perturbed versions (where words are replaced with contextually relevant synonyms) and one invalid option. If the LLM consistently selects the original instance, this behavior may indicate the presence of data contamination. Building on this, \citet{golchin2023time} proposed a guided instruction-based detection method, which effectively identifies contamination in datasets through instance completion and heuristic evaluation.

%\citet{duarte2024decopdetectingcopyrightedcontent} developed DE-COP, a copyright detection framework using verbal vs. paraphrased multiple-choice probing. Its BookTection/arXivTection benchmarks reveal temporal training data patterns in commercial LLMs. Similarly, \citet{deng2023investigating} proposed TS-Guessing, a protocol testing model ability to reconstruct masked test elements. Its analysis uncovers subtle contamination in major benchmarks, particularly in instruction-tuned LLMs. \citet{dong-etal-2024-generalization} introduced CDD to detect contamination through output distribution peakedness analysis. Paired with TED mitigation technique, the CDD approach addresses both explicit and implicit contamination forms while maintaining evaluation validity.

\citet{duarte2024decopdetectingcopyrightedcontent} developed DE-COP, a copyright detection framework that employs verbal versus paraphrased multiple-choice probing. Using benchmarks such as BookTection and arXivTection, DE-COP reveals temporal patterns in the training data of commercial LLMs. Similarly, \citet{deng2023investigating} proposed TS-Guessing, a protocol designed to test a model's ability to reconstruct masked elements of test data. This approach uncovers subtle contamination in major benchmarks. Further advancing this line of research, \citet{dong-etal-2024-generalization} introduced CDD to identify contamination by analyzing the peakedness of output distributions. When paired with the TED mitigation technique, the CDD approach effectively addresses both explicit and implicit forms of contamination while preserving the validity of model evaluations.



As highlighted by \cite{ranaldi-etal-2024-investigating}, the Text-to-SQL task with GPT-3.5 involves data contamination, where the model is tasked with reconstructing masked column names using the table name, the remaining column names, and contextual information. Similarly, \citet{chang-etal-2023-speak} introduced a challenging cloze task and employed data archaeology to examine the memorization of passages from 571 novels by using LLMs.









\section{Future Directions}
\label{sec:future directions}
\subsection{LLM Unlearning Methods}
Unlearning techniques offer the potential to mitigate LLM privacy risks by erasing specific data elements. Future research should explore integrating contamination mitigation through targeted unlearning mechanisms that remove biases or leaked information from certain sources. This emerging field shows promise and fundamental challenges. For instance, \citet{shumailov2024ununlearning} claimed such data erasure may be fundamentally unachievable in current architecture.

\subsection{Enhancing Black-box Detection Methods}
Black-box detection methods require more attention as most LLMs are black-box models. Some existing contamination detection methods heavily rely on heuristic rules. \citet{fu2024does} categorized the assumptions of multiple detection methods and their validation status, demonstrating that some assumptions may be invalidated in multiple scenarios. In other words, the stability of the assumptions underlying current data contamination detection approaches remains uncertain. Given that black-box methods may have broad applicability, more research into their reliability and effectiveness is essential.
 
\subsection{Distinguishing Between Data Contamination and Generalization}
The ambiguity between contamination and generalization remains unresolved. A core paradox lies in why in-distribution (ID) data contamination can not be interpreted as an alternative of LLMs' generalization capability, given the intrinsic overlap between memorization and generalization in LLMs \cite{zhang2021understanding}. Despite growing attention to state-of-the-art LLMs, the community lacks a standard definition for distinctions between contamination and generalization.

\subsection{Community Effort for Data Contamination}
Previously, the community has made some efforts to collect evidence of contamination, as shown in appendix \ref{sec:Evidence Collection}. Furthermore, the data contamination prevention paradox manifests as an inverse relationship between protective efficacy and benchmark availability. While enhanced safeguards reduce contamination risks, they simultaneously constrain the usability of existing benchmarks through stringent data isolation. As a result, dynamic evaluation should become the mainstream approach, and such strategies should be embraced as a community consensus.

\subsection{Non-Benchmark Evaluation}
LLM-as-a-judge approaches (Section \ref{sec:llm driven evaluation}) confront reliability challenges from persistent model biases. Current implementations often yield assessment inconsistencies that diverge from human judgment standards. Future directions should prioritize developing adversarial testing frameworks and hybrid evaluation frameworks to bridge the alignment gap between automated scoring and human values.


\section{Conclusion}

\section{Conclusion}
\label{sec:Conclusion}
In this paper, we proposed a complete real-time planning and control approach for continuous, reliable, and fast online generation of dynamically feasible Bernstein trajectories and control for FW aircrafts. The generated trajectories span kilometers, navigating through multiple waypoints. By leveraging differential flatness equations for coordinated flight, we ensure precise trajectory tracking. Our approach guarantees smooth transitions from simulation to real-world applications, enabling timely field deployment. The system also features a user-friendly mission planning interface. Continuous replanning  maintains the rajectory curvature 
$\kappa$ within limits, preventing abrupt roll changes.

Future works will include the ability to add  a higher-level kinodynamic path planner to optimize waypoint spatial allocation and improve replanning success, and enhancing the trajectory-tracking algorithm by refining the aerodynamic coefficient estimation. 


%\section*{Acknowledgments}

\section{Limitations}
While we extensively cover various forms of data contamination, it is possible that new contamination mechanisms or models may not be fully captured in our analysis. Additionally, our focus is primarily on data contamination within the context of LLMs, and we may not have fully incorporated previous research on data contamination in other areas of machine learning.

%% The file named.bst is a bibliography style file for BibTeX 0.99c
{
\bibstyle{acl_natbib}
\bibliography{custom,anthology}
\nocite{li2024awesome}
}


\appendix
\subsection{Lloyd-Max Algorithm}
\label{subsec:Lloyd-Max}
For a given quantization bitwidth $B$ and an operand $\bm{X}$, the Lloyd-Max algorithm finds $2^B$ quantization levels $\{\hat{x}_i\}_{i=1}^{2^B}$ such that quantizing $\bm{X}$ by rounding each scalar in $\bm{X}$ to the nearest quantization level minimizes the quantization MSE. 

The algorithm starts with an initial guess of quantization levels and then iteratively computes quantization thresholds $\{\tau_i\}_{i=1}^{2^B-1}$ and updates quantization levels $\{\hat{x}_i\}_{i=1}^{2^B}$. Specifically, at iteration $n$, thresholds are set to the midpoints of the previous iteration's levels:
\begin{align*}
    \tau_i^{(n)}=\frac{\hat{x}_i^{(n-1)}+\hat{x}_{i+1}^{(n-1)}}2 \text{ for } i=1\ldots 2^B-1
\end{align*}
Subsequently, the quantization levels are re-computed as conditional means of the data regions defined by the new thresholds:
\begin{align*}
    \hat{x}_i^{(n)}=\mathbb{E}\left[ \bm{X} \big| \bm{X}\in [\tau_{i-1}^{(n)},\tau_i^{(n)}] \right] \text{ for } i=1\ldots 2^B
\end{align*}
where to satisfy boundary conditions we have $\tau_0=-\infty$ and $\tau_{2^B}=\infty$. The algorithm iterates the above steps until convergence.

Figure \ref{fig:lm_quant} compares the quantization levels of a $7$-bit floating point (E3M3) quantizer (left) to a $7$-bit Lloyd-Max quantizer (right) when quantizing a layer of weights from the GPT3-126M model at a per-tensor granularity. As shown, the Lloyd-Max quantizer achieves substantially lower quantization MSE. Further, Table \ref{tab:FP7_vs_LM7} shows the superior perplexity achieved by Lloyd-Max quantizers for bitwidths of $7$, $6$ and $5$. The difference between the quantizers is clear at 5 bits, where per-tensor FP quantization incurs a drastic and unacceptable increase in perplexity, while Lloyd-Max quantization incurs a much smaller increase. Nevertheless, we note that even the optimal Lloyd-Max quantizer incurs a notable ($\sim 1.5$) increase in perplexity due to the coarse granularity of quantization. 

\begin{figure}[h]
  \centering
  \includegraphics[width=0.7\linewidth]{sections/figures/LM7_FP7.pdf}
  \caption{\small Quantization levels and the corresponding quantization MSE of Floating Point (left) vs Lloyd-Max (right) Quantizers for a layer of weights in the GPT3-126M model.}
  \label{fig:lm_quant}
\end{figure}

\begin{table}[h]\scriptsize
\begin{center}
\caption{\label{tab:FP7_vs_LM7} \small Comparing perplexity (lower is better) achieved by floating point quantizers and Lloyd-Max quantizers on a GPT3-126M model for the Wikitext-103 dataset.}
\begin{tabular}{c|cc|c}
\hline
 \multirow{2}{*}{\textbf{Bitwidth}} & \multicolumn{2}{|c|}{\textbf{Floating-Point Quantizer}} & \textbf{Lloyd-Max Quantizer} \\
 & Best Format & Wikitext-103 Perplexity & Wikitext-103 Perplexity \\
\hline
7 & E3M3 & 18.32 & 18.27 \\
6 & E3M2 & 19.07 & 18.51 \\
5 & E4M0 & 43.89 & 19.71 \\
\hline
\end{tabular}
\end{center}
\end{table}

\subsection{Proof of Local Optimality of LO-BCQ}
\label{subsec:lobcq_opt_proof}
For a given block $\bm{b}_j$, the quantization MSE during LO-BCQ can be empirically evaluated as $\frac{1}{L_b}\lVert \bm{b}_j- \bm{\hat{b}}_j\rVert^2_2$ where $\bm{\hat{b}}_j$ is computed from equation (\ref{eq:clustered_quantization_definition}) as $C_{f(\bm{b}_j)}(\bm{b}_j)$. Further, for a given block cluster $\mathcal{B}_i$, we compute the quantization MSE as $\frac{1}{|\mathcal{B}_{i}|}\sum_{\bm{b} \in \mathcal{B}_{i}} \frac{1}{L_b}\lVert \bm{b}- C_i^{(n)}(\bm{b})\rVert^2_2$. Therefore, at the end of iteration $n$, we evaluate the overall quantization MSE $J^{(n)}$ for a given operand $\bm{X}$ composed of $N_c$ block clusters as:
\begin{align*}
    \label{eq:mse_iter_n}
    J^{(n)} = \frac{1}{N_c} \sum_{i=1}^{N_c} \frac{1}{|\mathcal{B}_{i}^{(n)}|}\sum_{\bm{v} \in \mathcal{B}_{i}^{(n)}} \frac{1}{L_b}\lVert \bm{b}- B_i^{(n)}(\bm{b})\rVert^2_2
\end{align*}

At the end of iteration $n$, the codebooks are updated from $\mathcal{C}^{(n-1)}$ to $\mathcal{C}^{(n)}$. However, the mapping of a given vector $\bm{b}_j$ to quantizers $\mathcal{C}^{(n)}$ remains as  $f^{(n)}(\bm{b}_j)$. At the next iteration, during the vector clustering step, $f^{(n+1)}(\bm{b}_j)$ finds new mapping of $\bm{b}_j$ to updated codebooks $\mathcal{C}^{(n)}$ such that the quantization MSE over the candidate codebooks is minimized. Therefore, we obtain the following result for $\bm{b}_j$:
\begin{align*}
\frac{1}{L_b}\lVert \bm{b}_j - C_{f^{(n+1)}(\bm{b}_j)}^{(n)}(\bm{b}_j)\rVert^2_2 \le \frac{1}{L_b}\lVert \bm{b}_j - C_{f^{(n)}(\bm{b}_j)}^{(n)}(\bm{b}_j)\rVert^2_2
\end{align*}

That is, quantizing $\bm{b}_j$ at the end of the block clustering step of iteration $n+1$ results in lower quantization MSE compared to quantizing at the end of iteration $n$. Since this is true for all $\bm{b} \in \bm{X}$, we assert the following:
\begin{equation}
\begin{split}
\label{eq:mse_ineq_1}
    \tilde{J}^{(n+1)} &= \frac{1}{N_c} \sum_{i=1}^{N_c} \frac{1}{|\mathcal{B}_{i}^{(n+1)}|}\sum_{\bm{b} \in \mathcal{B}_{i}^{(n+1)}} \frac{1}{L_b}\lVert \bm{b} - C_i^{(n)}(b)\rVert^2_2 \le J^{(n)}
\end{split}
\end{equation}
where $\tilde{J}^{(n+1)}$ is the the quantization MSE after the vector clustering step at iteration $n+1$.

Next, during the codebook update step (\ref{eq:quantizers_update}) at iteration $n+1$, the per-cluster codebooks $\mathcal{C}^{(n)}$ are updated to $\mathcal{C}^{(n+1)}$ by invoking the Lloyd-Max algorithm \citep{Lloyd}. We know that for any given value distribution, the Lloyd-Max algorithm minimizes the quantization MSE. Therefore, for a given vector cluster $\mathcal{B}_i$ we obtain the following result:

\begin{equation}
    \frac{1}{|\mathcal{B}_{i}^{(n+1)}|}\sum_{\bm{b} \in \mathcal{B}_{i}^{(n+1)}} \frac{1}{L_b}\lVert \bm{b}- C_i^{(n+1)}(\bm{b})\rVert^2_2 \le \frac{1}{|\mathcal{B}_{i}^{(n+1)}|}\sum_{\bm{b} \in \mathcal{B}_{i}^{(n+1)}} \frac{1}{L_b}\lVert \bm{b}- C_i^{(n)}(\bm{b})\rVert^2_2
\end{equation}

The above equation states that quantizing the given block cluster $\mathcal{B}_i$ after updating the associated codebook from $C_i^{(n)}$ to $C_i^{(n+1)}$ results in lower quantization MSE. Since this is true for all the block clusters, we derive the following result: 
\begin{equation}
\begin{split}
\label{eq:mse_ineq_2}
     J^{(n+1)} &= \frac{1}{N_c} \sum_{i=1}^{N_c} \frac{1}{|\mathcal{B}_{i}^{(n+1)}|}\sum_{\bm{b} \in \mathcal{B}_{i}^{(n+1)}} \frac{1}{L_b}\lVert \bm{b}- C_i^{(n+1)}(\bm{b})\rVert^2_2  \le \tilde{J}^{(n+1)}   
\end{split}
\end{equation}

Following (\ref{eq:mse_ineq_1}) and (\ref{eq:mse_ineq_2}), we find that the quantization MSE is non-increasing for each iteration, that is, $J^{(1)} \ge J^{(2)} \ge J^{(3)} \ge \ldots \ge J^{(M)}$ where $M$ is the maximum number of iterations. 
%Therefore, we can say that if the algorithm converges, then it must be that it has converged to a local minimum. 
\hfill $\blacksquare$


\begin{figure}
    \begin{center}
    \includegraphics[width=0.5\textwidth]{sections//figures/mse_vs_iter.pdf}
    \end{center}
    \caption{\small NMSE vs iterations during LO-BCQ compared to other block quantization proposals}
    \label{fig:nmse_vs_iter}
\end{figure}

Figure \ref{fig:nmse_vs_iter} shows the empirical convergence of LO-BCQ across several block lengths and number of codebooks. Also, the MSE achieved by LO-BCQ is compared to baselines such as MXFP and VSQ. As shown, LO-BCQ converges to a lower MSE than the baselines. Further, we achieve better convergence for larger number of codebooks ($N_c$) and for a smaller block length ($L_b$), both of which increase the bitwidth of BCQ (see Eq \ref{eq:bitwidth_bcq}).


\subsection{Additional Accuracy Results}
%Table \ref{tab:lobcq_config} lists the various LOBCQ configurations and their corresponding bitwidths.
\begin{table}
\setlength{\tabcolsep}{4.75pt}
\begin{center}
\caption{\label{tab:lobcq_config} Various LO-BCQ configurations and their bitwidths.}
\begin{tabular}{|c||c|c|c|c||c|c||c|} 
\hline
 & \multicolumn{4}{|c||}{$L_b=8$} & \multicolumn{2}{|c||}{$L_b=4$} & $L_b=2$ \\
 \hline
 \backslashbox{$L_A$\kern-1em}{\kern-1em$N_c$} & 2 & 4 & 8 & 16 & 2 & 4 & 2 \\
 \hline
 64 & 4.25 & 4.375 & 4.5 & 4.625 & 4.375 & 4.625 & 4.625\\
 \hline
 32 & 4.375 & 4.5 & 4.625& 4.75 & 4.5 & 4.75 & 4.75 \\
 \hline
 16 & 4.625 & 4.75& 4.875 & 5 & 4.75 & 5 & 5 \\
 \hline
\end{tabular}
\end{center}
\end{table}

%\subsection{Perplexity achieved by various LO-BCQ configurations on Wikitext-103 dataset}

\begin{table} \centering
\begin{tabular}{|c||c|c|c|c||c|c||c|} 
\hline
 $L_b \rightarrow$& \multicolumn{4}{c||}{8} & \multicolumn{2}{c||}{4} & 2\\
 \hline
 \backslashbox{$L_A$\kern-1em}{\kern-1em$N_c$} & 2 & 4 & 8 & 16 & 2 & 4 & 2  \\
 %$N_c \rightarrow$ & 2 & 4 & 8 & 16 & 2 & 4 & 2 \\
 \hline
 \hline
 \multicolumn{8}{c}{GPT3-1.3B (FP32 PPL = 9.98)} \\ 
 \hline
 \hline
 64 & 10.40 & 10.23 & 10.17 & 10.15 &  10.28 & 10.18 & 10.19 \\
 \hline
 32 & 10.25 & 10.20 & 10.15 & 10.12 &  10.23 & 10.17 & 10.17 \\
 \hline
 16 & 10.22 & 10.16 & 10.10 & 10.09 &  10.21 & 10.14 & 10.16 \\
 \hline
  \hline
 \multicolumn{8}{c}{GPT3-8B (FP32 PPL = 7.38)} \\ 
 \hline
 \hline
 64 & 7.61 & 7.52 & 7.48 &  7.47 &  7.55 &  7.49 & 7.50 \\
 \hline
 32 & 7.52 & 7.50 & 7.46 &  7.45 &  7.52 &  7.48 & 7.48  \\
 \hline
 16 & 7.51 & 7.48 & 7.44 &  7.44 &  7.51 &  7.49 & 7.47  \\
 \hline
\end{tabular}
\caption{\label{tab:ppl_gpt3_abalation} Wikitext-103 perplexity across GPT3-1.3B and 8B models.}
\end{table}

\begin{table} \centering
\begin{tabular}{|c||c|c|c|c||} 
\hline
 $L_b \rightarrow$& \multicolumn{4}{c||}{8}\\
 \hline
 \backslashbox{$L_A$\kern-1em}{\kern-1em$N_c$} & 2 & 4 & 8 & 16 \\
 %$N_c \rightarrow$ & 2 & 4 & 8 & 16 & 2 & 4 & 2 \\
 \hline
 \hline
 \multicolumn{5}{|c|}{Llama2-7B (FP32 PPL = 5.06)} \\ 
 \hline
 \hline
 64 & 5.31 & 5.26 & 5.19 & 5.18  \\
 \hline
 32 & 5.23 & 5.25 & 5.18 & 5.15  \\
 \hline
 16 & 5.23 & 5.19 & 5.16 & 5.14  \\
 \hline
 \multicolumn{5}{|c|}{Nemotron4-15B (FP32 PPL = 5.87)} \\ 
 \hline
 \hline
 64  & 6.3 & 6.20 & 6.13 & 6.08  \\
 \hline
 32  & 6.24 & 6.12 & 6.07 & 6.03  \\
 \hline
 16  & 6.12 & 6.14 & 6.04 & 6.02  \\
 \hline
 \multicolumn{5}{|c|}{Nemotron4-340B (FP32 PPL = 3.48)} \\ 
 \hline
 \hline
 64 & 3.67 & 3.62 & 3.60 & 3.59 \\
 \hline
 32 & 3.63 & 3.61 & 3.59 & 3.56 \\
 \hline
 16 & 3.61 & 3.58 & 3.57 & 3.55 \\
 \hline
\end{tabular}
\caption{\label{tab:ppl_llama7B_nemo15B} Wikitext-103 perplexity compared to FP32 baseline in Llama2-7B and Nemotron4-15B, 340B models}
\end{table}

%\subsection{Perplexity achieved by various LO-BCQ configurations on MMLU dataset}


\begin{table} \centering
\begin{tabular}{|c||c|c|c|c||c|c|c|c|} 
\hline
 $L_b \rightarrow$& \multicolumn{4}{c||}{8} & \multicolumn{4}{c||}{8}\\
 \hline
 \backslashbox{$L_A$\kern-1em}{\kern-1em$N_c$} & 2 & 4 & 8 & 16 & 2 & 4 & 8 & 16  \\
 %$N_c \rightarrow$ & 2 & 4 & 8 & 16 & 2 & 4 & 2 \\
 \hline
 \hline
 \multicolumn{5}{|c|}{Llama2-7B (FP32 Accuracy = 45.8\%)} & \multicolumn{4}{|c|}{Llama2-70B (FP32 Accuracy = 69.12\%)} \\ 
 \hline
 \hline
 64 & 43.9 & 43.4 & 43.9 & 44.9 & 68.07 & 68.27 & 68.17 & 68.75 \\
 \hline
 32 & 44.5 & 43.8 & 44.9 & 44.5 & 68.37 & 68.51 & 68.35 & 68.27  \\
 \hline
 16 & 43.9 & 42.7 & 44.9 & 45 & 68.12 & 68.77 & 68.31 & 68.59  \\
 \hline
 \hline
 \multicolumn{5}{|c|}{GPT3-22B (FP32 Accuracy = 38.75\%)} & \multicolumn{4}{|c|}{Nemotron4-15B (FP32 Accuracy = 64.3\%)} \\ 
 \hline
 \hline
 64 & 36.71 & 38.85 & 38.13 & 38.92 & 63.17 & 62.36 & 63.72 & 64.09 \\
 \hline
 32 & 37.95 & 38.69 & 39.45 & 38.34 & 64.05 & 62.30 & 63.8 & 64.33  \\
 \hline
 16 & 38.88 & 38.80 & 38.31 & 38.92 & 63.22 & 63.51 & 63.93 & 64.43  \\
 \hline
\end{tabular}
\caption{\label{tab:mmlu_abalation} Accuracy on MMLU dataset across GPT3-22B, Llama2-7B, 70B and Nemotron4-15B models.}
\end{table}


%\subsection{Perplexity achieved by various LO-BCQ configurations on LM evaluation harness}

\begin{table} \centering
\begin{tabular}{|c||c|c|c|c||c|c|c|c|} 
\hline
 $L_b \rightarrow$& \multicolumn{4}{c||}{8} & \multicolumn{4}{c||}{8}\\
 \hline
 \backslashbox{$L_A$\kern-1em}{\kern-1em$N_c$} & 2 & 4 & 8 & 16 & 2 & 4 & 8 & 16  \\
 %$N_c \rightarrow$ & 2 & 4 & 8 & 16 & 2 & 4 & 2 \\
 \hline
 \hline
 \multicolumn{5}{|c|}{Race (FP32 Accuracy = 37.51\%)} & \multicolumn{4}{|c|}{Boolq (FP32 Accuracy = 64.62\%)} \\ 
 \hline
 \hline
 64 & 36.94 & 37.13 & 36.27 & 37.13 & 63.73 & 62.26 & 63.49 & 63.36 \\
 \hline
 32 & 37.03 & 36.36 & 36.08 & 37.03 & 62.54 & 63.51 & 63.49 & 63.55  \\
 \hline
 16 & 37.03 & 37.03 & 36.46 & 37.03 & 61.1 & 63.79 & 63.58 & 63.33  \\
 \hline
 \hline
 \multicolumn{5}{|c|}{Winogrande (FP32 Accuracy = 58.01\%)} & \multicolumn{4}{|c|}{Piqa (FP32 Accuracy = 74.21\%)} \\ 
 \hline
 \hline
 64 & 58.17 & 57.22 & 57.85 & 58.33 & 73.01 & 73.07 & 73.07 & 72.80 \\
 \hline
 32 & 59.12 & 58.09 & 57.85 & 58.41 & 73.01 & 73.94 & 72.74 & 73.18  \\
 \hline
 16 & 57.93 & 58.88 & 57.93 & 58.56 & 73.94 & 72.80 & 73.01 & 73.94  \\
 \hline
\end{tabular}
\caption{\label{tab:mmlu_abalation} Accuracy on LM evaluation harness tasks on GPT3-1.3B model.}
\end{table}

\begin{table} \centering
\begin{tabular}{|c||c|c|c|c||c|c|c|c|} 
\hline
 $L_b \rightarrow$& \multicolumn{4}{c||}{8} & \multicolumn{4}{c||}{8}\\
 \hline
 \backslashbox{$L_A$\kern-1em}{\kern-1em$N_c$} & 2 & 4 & 8 & 16 & 2 & 4 & 8 & 16  \\
 %$N_c \rightarrow$ & 2 & 4 & 8 & 16 & 2 & 4 & 2 \\
 \hline
 \hline
 \multicolumn{5}{|c|}{Race (FP32 Accuracy = 41.34\%)} & \multicolumn{4}{|c|}{Boolq (FP32 Accuracy = 68.32\%)} \\ 
 \hline
 \hline
 64 & 40.48 & 40.10 & 39.43 & 39.90 & 69.20 & 68.41 & 69.45 & 68.56 \\
 \hline
 32 & 39.52 & 39.52 & 40.77 & 39.62 & 68.32 & 67.43 & 68.17 & 69.30  \\
 \hline
 16 & 39.81 & 39.71 & 39.90 & 40.38 & 68.10 & 66.33 & 69.51 & 69.42  \\
 \hline
 \hline
 \multicolumn{5}{|c|}{Winogrande (FP32 Accuracy = 67.88\%)} & \multicolumn{4}{|c|}{Piqa (FP32 Accuracy = 78.78\%)} \\ 
 \hline
 \hline
 64 & 66.85 & 66.61 & 67.72 & 67.88 & 77.31 & 77.42 & 77.75 & 77.64 \\
 \hline
 32 & 67.25 & 67.72 & 67.72 & 67.00 & 77.31 & 77.04 & 77.80 & 77.37  \\
 \hline
 16 & 68.11 & 68.90 & 67.88 & 67.48 & 77.37 & 78.13 & 78.13 & 77.69  \\
 \hline
\end{tabular}
\caption{\label{tab:mmlu_abalation} Accuracy on LM evaluation harness tasks on GPT3-8B model.}
\end{table}

\begin{table} \centering
\begin{tabular}{|c||c|c|c|c||c|c|c|c|} 
\hline
 $L_b \rightarrow$& \multicolumn{4}{c||}{8} & \multicolumn{4}{c||}{8}\\
 \hline
 \backslashbox{$L_A$\kern-1em}{\kern-1em$N_c$} & 2 & 4 & 8 & 16 & 2 & 4 & 8 & 16  \\
 %$N_c \rightarrow$ & 2 & 4 & 8 & 16 & 2 & 4 & 2 \\
 \hline
 \hline
 \multicolumn{5}{|c|}{Race (FP32 Accuracy = 40.67\%)} & \multicolumn{4}{|c|}{Boolq (FP32 Accuracy = 76.54\%)} \\ 
 \hline
 \hline
 64 & 40.48 & 40.10 & 39.43 & 39.90 & 75.41 & 75.11 & 77.09 & 75.66 \\
 \hline
 32 & 39.52 & 39.52 & 40.77 & 39.62 & 76.02 & 76.02 & 75.96 & 75.35  \\
 \hline
 16 & 39.81 & 39.71 & 39.90 & 40.38 & 75.05 & 73.82 & 75.72 & 76.09  \\
 \hline
 \hline
 \multicolumn{5}{|c|}{Winogrande (FP32 Accuracy = 70.64\%)} & \multicolumn{4}{|c|}{Piqa (FP32 Accuracy = 79.16\%)} \\ 
 \hline
 \hline
 64 & 69.14 & 70.17 & 70.17 & 70.56 & 78.24 & 79.00 & 78.62 & 78.73 \\
 \hline
 32 & 70.96 & 69.69 & 71.27 & 69.30 & 78.56 & 79.49 & 79.16 & 78.89  \\
 \hline
 16 & 71.03 & 69.53 & 69.69 & 70.40 & 78.13 & 79.16 & 79.00 & 79.00  \\
 \hline
\end{tabular}
\caption{\label{tab:mmlu_abalation} Accuracy on LM evaluation harness tasks on GPT3-22B model.}
\end{table}

\begin{table} \centering
\begin{tabular}{|c||c|c|c|c||c|c|c|c|} 
\hline
 $L_b \rightarrow$& \multicolumn{4}{c||}{8} & \multicolumn{4}{c||}{8}\\
 \hline
 \backslashbox{$L_A$\kern-1em}{\kern-1em$N_c$} & 2 & 4 & 8 & 16 & 2 & 4 & 8 & 16  \\
 %$N_c \rightarrow$ & 2 & 4 & 8 & 16 & 2 & 4 & 2 \\
 \hline
 \hline
 \multicolumn{5}{|c|}{Race (FP32 Accuracy = 44.4\%)} & \multicolumn{4}{|c|}{Boolq (FP32 Accuracy = 79.29\%)} \\ 
 \hline
 \hline
 64 & 42.49 & 42.51 & 42.58 & 43.45 & 77.58 & 77.37 & 77.43 & 78.1 \\
 \hline
 32 & 43.35 & 42.49 & 43.64 & 43.73 & 77.86 & 75.32 & 77.28 & 77.86  \\
 \hline
 16 & 44.21 & 44.21 & 43.64 & 42.97 & 78.65 & 77 & 76.94 & 77.98  \\
 \hline
 \hline
 \multicolumn{5}{|c|}{Winogrande (FP32 Accuracy = 69.38\%)} & \multicolumn{4}{|c|}{Piqa (FP32 Accuracy = 78.07\%)} \\ 
 \hline
 \hline
 64 & 68.9 & 68.43 & 69.77 & 68.19 & 77.09 & 76.82 & 77.09 & 77.86 \\
 \hline
 32 & 69.38 & 68.51 & 68.82 & 68.90 & 78.07 & 76.71 & 78.07 & 77.86  \\
 \hline
 16 & 69.53 & 67.09 & 69.38 & 68.90 & 77.37 & 77.8 & 77.91 & 77.69  \\
 \hline
\end{tabular}
\caption{\label{tab:mmlu_abalation} Accuracy on LM evaluation harness tasks on Llama2-7B model.}
\end{table}

\begin{table} \centering
\begin{tabular}{|c||c|c|c|c||c|c|c|c|} 
\hline
 $L_b \rightarrow$& \multicolumn{4}{c||}{8} & \multicolumn{4}{c||}{8}\\
 \hline
 \backslashbox{$L_A$\kern-1em}{\kern-1em$N_c$} & 2 & 4 & 8 & 16 & 2 & 4 & 8 & 16  \\
 %$N_c \rightarrow$ & 2 & 4 & 8 & 16 & 2 & 4 & 2 \\
 \hline
 \hline
 \multicolumn{5}{|c|}{Race (FP32 Accuracy = 48.8\%)} & \multicolumn{4}{|c|}{Boolq (FP32 Accuracy = 85.23\%)} \\ 
 \hline
 \hline
 64 & 49.00 & 49.00 & 49.28 & 48.71 & 82.82 & 84.28 & 84.03 & 84.25 \\
 \hline
 32 & 49.57 & 48.52 & 48.33 & 49.28 & 83.85 & 84.46 & 84.31 & 84.93  \\
 \hline
 16 & 49.85 & 49.09 & 49.28 & 48.99 & 85.11 & 84.46 & 84.61 & 83.94  \\
 \hline
 \hline
 \multicolumn{5}{|c|}{Winogrande (FP32 Accuracy = 79.95\%)} & \multicolumn{4}{|c|}{Piqa (FP32 Accuracy = 81.56\%)} \\ 
 \hline
 \hline
 64 & 78.77 & 78.45 & 78.37 & 79.16 & 81.45 & 80.69 & 81.45 & 81.5 \\
 \hline
 32 & 78.45 & 79.01 & 78.69 & 80.66 & 81.56 & 80.58 & 81.18 & 81.34  \\
 \hline
 16 & 79.95 & 79.56 & 79.79 & 79.72 & 81.28 & 81.66 & 81.28 & 80.96  \\
 \hline
\end{tabular}
\caption{\label{tab:mmlu_abalation} Accuracy on LM evaluation harness tasks on Llama2-70B model.}
\end{table}

%\section{MSE Studies}
%\textcolor{red}{TODO}


\subsection{Number Formats and Quantization Method}
\label{subsec:numFormats_quantMethod}
\subsubsection{Integer Format}
An $n$-bit signed integer (INT) is typically represented with a 2s-complement format \citep{yao2022zeroquant,xiao2023smoothquant,dai2021vsq}, where the most significant bit denotes the sign.

\subsubsection{Floating Point Format}
An $n$-bit signed floating point (FP) number $x$ comprises of a 1-bit sign ($x_{\mathrm{sign}}$), $B_m$-bit mantissa ($x_{\mathrm{mant}}$) and $B_e$-bit exponent ($x_{\mathrm{exp}}$) such that $B_m+B_e=n-1$. The associated constant exponent bias ($E_{\mathrm{bias}}$) is computed as $(2^{{B_e}-1}-1)$. We denote this format as $E_{B_e}M_{B_m}$.  

\subsubsection{Quantization Scheme}
\label{subsec:quant_method}
A quantization scheme dictates how a given unquantized tensor is converted to its quantized representation. We consider FP formats for the purpose of illustration. Given an unquantized tensor $\bm{X}$ and an FP format $E_{B_e}M_{B_m}$, we first, we compute the quantization scale factor $s_X$ that maps the maximum absolute value of $\bm{X}$ to the maximum quantization level of the $E_{B_e}M_{B_m}$ format as follows:
\begin{align}
\label{eq:sf}
    s_X = \frac{\mathrm{max}(|\bm{X}|)}{\mathrm{max}(E_{B_e}M_{B_m})}
\end{align}
In the above equation, $|\cdot|$ denotes the absolute value function.

Next, we scale $\bm{X}$ by $s_X$ and quantize it to $\hat{\bm{X}}$ by rounding it to the nearest quantization level of $E_{B_e}M_{B_m}$ as:

\begin{align}
\label{eq:tensor_quant}
    \hat{\bm{X}} = \text{round-to-nearest}\left(\frac{\bm{X}}{s_X}, E_{B_e}M_{B_m}\right)
\end{align}

We perform dynamic max-scaled quantization \citep{wu2020integer}, where the scale factor $s$ for activations is dynamically computed during runtime.

\subsection{Vector Scaled Quantization}
\begin{wrapfigure}{r}{0.35\linewidth}
  \centering
  \includegraphics[width=\linewidth]{sections/figures/vsquant.jpg}
  \caption{\small Vectorwise decomposition for per-vector scaled quantization (VSQ \citep{dai2021vsq}).}
  \label{fig:vsquant}
\end{wrapfigure}
During VSQ \citep{dai2021vsq}, the operand tensors are decomposed into 1D vectors in a hardware friendly manner as shown in Figure \ref{fig:vsquant}. Since the decomposed tensors are used as operands in matrix multiplications during inference, it is beneficial to perform this decomposition along the reduction dimension of the multiplication. The vectorwise quantization is performed similar to tensorwise quantization described in Equations \ref{eq:sf} and \ref{eq:tensor_quant}, where a scale factor $s_v$ is required for each vector $\bm{v}$ that maps the maximum absolute value of that vector to the maximum quantization level. While smaller vector lengths can lead to larger accuracy gains, the associated memory and computational overheads due to the per-vector scale factors increases. To alleviate these overheads, VSQ \citep{dai2021vsq} proposed a second level quantization of the per-vector scale factors to unsigned integers, while MX \citep{rouhani2023shared} quantizes them to integer powers of 2 (denoted as $2^{INT}$).

\subsubsection{MX Format}
The MX format proposed in \citep{rouhani2023microscaling} introduces the concept of sub-block shifting. For every two scalar elements of $b$-bits each, there is a shared exponent bit. The value of this exponent bit is determined through an empirical analysis that targets minimizing quantization MSE. We note that the FP format $E_{1}M_{b}$ is strictly better than MX from an accuracy perspective since it allocates a dedicated exponent bit to each scalar as opposed to sharing it across two scalars. Therefore, we conservatively bound the accuracy of a $b+2$-bit signed MX format with that of a $E_{1}M_{b}$ format in our comparisons. For instance, we use E1M2 format as a proxy for MX4.

\begin{figure}
    \centering
    \includegraphics[width=1\linewidth]{sections//figures/BlockFormats.pdf}
    \caption{\small Comparing LO-BCQ to MX format.}
    \label{fig:block_formats}
\end{figure}

Figure \ref{fig:block_formats} compares our $4$-bit LO-BCQ block format to MX \citep{rouhani2023microscaling}. As shown, both LO-BCQ and MX decompose a given operand tensor into block arrays and each block array into blocks. Similar to MX, we find that per-block quantization ($L_b < L_A$) leads to better accuracy due to increased flexibility. While MX achieves this through per-block $1$-bit micro-scales, we associate a dedicated codebook to each block through a per-block codebook selector. Further, MX quantizes the per-block array scale-factor to E8M0 format without per-tensor scaling. In contrast during LO-BCQ, we find that per-tensor scaling combined with quantization of per-block array scale-factor to E4M3 format results in superior inference accuracy across models. 

\end{document}
