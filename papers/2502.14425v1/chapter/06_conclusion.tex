%Our paper analyze three fundamental perspectives in data contamination research: first, defining data contamination from the perspectives of phases and benchmarks; second, exploring approaches for conducting contamination-free evaluations, with an emphasis on dynamic evaluation and LLM-based assessments; and third, exploring methods for detecting data contamination, providing a comprehensive analyse of techniques and limitations. Our work provides an introduction for researchers new to the field and a roadmap that highlights data contamination as a key issue in LLM evaluation. We also offer recommendations for improving contamination-aware evaluation systems in LLM development.

Our paper examines three fundamental perspectives in data contamination research: (1) defining data contamination through the lenses of phases and benchmarks; (2) exploring methodologies for conducting contamination-free evaluations, with a particular focus on dynamic evaluation and LLM-based assessment techniques; and (3) investigating methods for detecting data contamination, offering a comprehensive analysis of existing techniques and their limitations. This work serves as both an introductory guide for researchers new to the field and a roadmap that underscores data contamination as a critical challenge in LLM evaluation. Furthermore, we provide actionable recommendations for enhancing contamination-aware evaluation systems, aiming to foster more robust and reliable LLM development practices.