\label{sec:future directions}
\subsection{LLM Unlearning Methods}
Unlearning techniques offer the potential to mitigate LLM privacy risks by erasing specific data elements. Future research should explore integrating contamination mitigation through targeted unlearning mechanisms that remove biases or leaked information from certain sources. This emerging field shows promise and fundamental challenges. For instance, \citet{shumailov2024ununlearning} claimed such data erasure may be fundamentally unachievable in current architecture.

\subsection{Enhancing Black-box Detection Methods}
Black-box detection methods require more attention as most LLMs are black-box models. Some existing contamination detection methods heavily rely on heuristic rules. \citet{fu2024does} categorized the assumptions of multiple detection methods and their validation status, demonstrating that some assumptions may be invalidated in multiple scenarios. In other words, the stability of the assumptions underlying current data contamination detection approaches remains uncertain. Given that black-box methods may have broad applicability, more research into their reliability and effectiveness is essential.
 
\subsection{Distinguishing Between Data Contamination and Generalization}
The ambiguity between contamination and generalization remains unresolved. A core paradox lies in why in-distribution (ID) data contamination can not be interpreted as an alternative of LLMs' generalization capability, given the intrinsic overlap between memorization and generalization in LLMs \cite{zhang2021understanding}. Despite growing attention to state-of-the-art LLMs, the community lacks a standard definition for distinctions between contamination and generalization.

\subsection{Community Effort for Data Contamination}
Previously, the community has made some efforts to collect evidence of contamination, as shown in appendix \ref{sec:Evidence Collection}. Furthermore, the data contamination prevention paradox manifests as an inverse relationship between protective efficacy and benchmark availability. While enhanced safeguards reduce contamination risks, they simultaneously constrain the usability of existing benchmarks through stringent data isolation. As a result, dynamic evaluation should become the mainstream approach, and such strategies should be embraced as a community consensus.

\subsection{Non-Benchmark Evaluation}
LLM-as-a-judge approaches (Section \ref{sec:llm driven evaluation}) confront reliability challenges from persistent model biases. Current implementations often yield assessment inconsistencies that diverge from human judgment standards. Future directions should prioritize developing adversarial testing frameworks and hybrid evaluation frameworks to bridge the alignment gap between automated scoring and human values.

