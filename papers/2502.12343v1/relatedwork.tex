\section{Related Works}
Conventional communication system design primarily focuses on optimizing the trade-off between performance and computational complexity. For example, this can involve maximizing the throughput while keeping the computational complexity within reasonable limits. In the context of MIMO precoding, which is the main focus of this paper, the conventional approach typically aims to maximize a measure of spectral efficiency, such as sum rate \cite{Foad2016Hybrid}, weighted sum rate \cite{shi2011iteratively}, or min rate \cite{Manijeh2019maxmin}, while respecting regulatory and hardware constraints.


With this mindset, various precoding methods have been developed for multi-user MIMO systems. One of the simplest forms of linear precoding is MRT precoding \cite{Lo_1999_MRT}, where the precoder is designed to maximize the power of the intended signals at the receivers. While MRT algorithm is quite simple and computationally friendly, it is unable to handle inter-layer or inter-user interference. As a result, simple yet effective ZF type precoders, e.g., \cite{Hong_2013_ZF,Cornelis2023WSRZF}, are developed, in which interference under perfect channel state information can be completely canceled. However, in scenarios where the number of spatial degrees of freedom is insufficient or the channels are ill-conditioned due to high correlation, the ZF solution performs poorly as it aims to ensure complete interference nulling \cite{Peel2005RZF}. To address this drawback of ZF, a regularization term is introduced in the RZF precoding method. The regularization term is chosen to make a balance between interference mitigation and power enhancement \cite{Emil_RZF_2014,Peel2005RZF}. Another alternative approach that has attracted significant attention is WMMSE precoding \cite{shi2011iteratively}. In this method, the WSR problem is converted into a weighted MSE minimization problem, and a block coordinate descent method is employed to find the precoder. The iterative WMMSE methods are guaranteed to converge to a stationary solution, and numerous examples have reported their near-optimal performance \cite{Juseong2024AnalogFeed}.


Most of the aforementioned precoding methods were initially developed under SPC, which restricts the total power of all BS antennas to a given budget. This power budget is typically determined based on regulations governing the maximum EIRP which happens in the broadside direction \cite{Kim2012Globe}. However, in practice, each antenna branch, consisting of a PA, has its own individual power budget \cite{Zheng2007PAPC,Victor2019PA}. As a result, subsequent research has aimed to ensure that the designed precoders also satisfy UB-PAPC. In its simplest form, to satisfy UB-PAPC, we can scale down the precoder designed by the SPC precoding schemes \cite{Lee2013Scale}. However, this approach leaves a significant portion of the power budget unused at some antennas, resulting in potential performance degradation. This motivates the extension of the precoding designs under SPC to incorporate UB-PAPC, where the UB-PAPC is set based on the maximum power budget of each antenna branch.


For instance, some works (e.g., \cite{Wisel2008, Rui2010, Pham2018}) have focused on developing variants of ZF precoders under UB-PAPCs for MIMO systems. In particular, \cite{Wisel2008} demonstrates that by applying SDR, the ZF problem can be formulated as a convex problem and optimally solved. The authors of \cite{Wisel2008} further prove that the relaxed problem always has a rank-one solution and propose a method to recover the rank-one solution. In this paper, we show that it is possible to use the same SDR technique to find a flat ZF precoder. However, unlike \cite{Wisel2008} which only considers UB-PAPC, there is no guarantee that the solution of the relaxed problem is rank one. Nevertheless, in practice, we observe that in most cases, the solution is rank one. For the remaining cases, we propose a technique to make it feasible for the power constraints, although the ZF condition may no longer be imposed.

\changeb{While it is possible to employ convex optimization tools and packages to find a ZF precoder under PAPC constraints}, the practical computational complexity of such approaches may still be high. To reduce the computational complexity of finding the desired ZF solution, the subgradient and approximation methods have been proposed in the literature, e.g., \cite{Rui2010, Pham2018}. In this paper, instead of following these strategies, we propose a highly efficient iterative method called FRG-Flat ZF by exploiting the primal-dual approach. We show that the performance of FRG-Flat ZF can approach that of the more relaxed baselines only under SPC.

Given the urgent threat of climate change, it becomes increasingly important to minimize energy usage achieve sustainability and support future technologies \cite{Stefan_2023_Mag}. In the wireless communications industry, the development of greener design in physical layer technologies is crucial for sustainable growth in 6G networks and beyond \cite{Han2021Green}. This paper argues that the traditional approach of optimizing performance and computational complexity alone is insufficient for designing future wireless communication systems. Recognizing the need for sustainability, recent research has focused on energy efficiency as a third dimension for system optimization \cite{Lopez_2022_CST,Lozano_2023_EE}. This paper proposes the  concept of flat precoding as a method to reduce energy consumption and implementation costs.

%method to ensure power efficiency in hardware components, such as PAs, is maximized. The numerical results demonstrate that the proposed efficient flat precoding methods yield only marginal performance reduction, while significantly reducing energy consumption and implementation costs.
%%%%%%%%%%%%%%%%%%%%%%%%%%%%%%%%%%%
% I.C. Paper Organization and Notations
%%%%%%%%%%%%%%%%%%%%%%%%%%%%%%%%%%%