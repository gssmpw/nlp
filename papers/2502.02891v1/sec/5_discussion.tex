\section{Limitations and Discussion}


While our method offers versatile sculpting capabilities with fine-grained user control, it has limitations, particularly with highly oscillatory brush profiles. It handles sharp, abrupt changes well (e.g., box-like or linear brushes) but struggles with high-frequency oscillations like wavelets with multiple peaks and troughs, leading to blurring. This trade-off between resolution and complexity in the backbone neural network requires deeper, more expressive MLPs, increasing training and inference costs. Fig.~\ref{fig:brush} shows the same edit on an MLP with more layers. This edit is impossible with 3DNS, even with dense sampling, due to overlapping point influences. Beyond sharpness issues due to their continuous nature, MLP-based representations also struggle to preserve unedited regions—limited regularization causes bumpy artifacts, while stronger regularization suppresses edits. While our sculpting operator is agnostic to the underlying representation, future work could explore alternatives like wavelet-based MLPs or neural SDF intersections to improve sharpness and region preservation.

Secondly, very high-intensity brushes can cause abrupt, discontinuous changes in the geometry by inducing large updates to the network weights. This is less of a concern in typical sculpting workflows, where low to medium-intensity edits are generally preferred for achieving smooth, controlled modifications. Thus, while these limitations are worth noting, they do not significantly impede the utility of the framework for most sculpting tasks.

\begin{figure}[t]
  \centering
   \includegraphics[width=0.9\linewidth]{figs/resolution.pdf}

   \caption{Impact of model resolution on brush template fidelity. Low-resolution model (middle) blurs high-frequency brush oscillations, while the high-resolution model (right) preserves finer details.}
   \label{fig:brush}
   \vspace{-0.5cm}
\end{figure}