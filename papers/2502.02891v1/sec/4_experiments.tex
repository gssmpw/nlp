
\section{Experiments}

In this section, we evaluate our proposed neural sculpting framework with continuous stroke-based edits and custom brush profiles. We test the performance of our method on multiple 3D objects and compare it with the point-edit approach used in previous work, as well as with traditional mesh-based editing. We use Chamfer distance as the primary metric to compare the geometric differences between the original and edited models. Additionally, we compare the editing speed of our approach with 3DNS. 
\subsection{Dataset and Testing Setup}
For our experiments, we use the same dataset as \cite{tzathas20233d}. This dataset comprises six 3D shapes: a frog, bust, and pumpkin (sourced from TurboSquid \cite{turbo}), the Stanford Bunny \cite{turk1994zippered}, and two analytical models—a sphere with radius 0.6 and a torus with major radius 0.45 and minor radius 0.25. These serve as the base models for our edits, e.g., in Fig~\ref{fig:edits}. The mesh models were preprocessed by normalizing the coordinates to ensure they fit within a bounding box $[-1,1]$, with additional space reserved for editing. We follow the same preprocessing steps as in \cite{tzathas20233d} to make the comparison consistent. We represent these shapes using a neural SDF parameterized by an MLP, specifically using the SIREN architecture with 2 hidden layers and 128 neurons per layer.

\subsection{Editing Performance Metric}
To quantitatively assess the accuracy of our edits, we use the chamfer distance, a widely used metric in geometric processing tasks. Given two point clouds, \(A\) and \(B\), their Chamfer distance is defined as:

\begin{equation}
d_{\text{chamfer}}(A, B) = \frac{1}{|A|}\sum_{a \in A} \min_{b \in B} \|a - b\| + \frac{1}{|B|}\sum_{b \in B} \min_{a \in A} \|b - a\|
\end{equation}

This measures how close the edited surface is to the ground truth by summing the nearest point distances between the two point clouds. Chamfer distance is computed over the entire surface as well as within the interaction region where edits are applied.

\subsection{Editing Comparison}

In this section, we compare our method with three baselines:with three baselines: edits on a high-resolution ground truth mesh, a low-resolution mesh, and 3DNS's point-based editing extended to strokes.

For the ground truth comparison, we apply the brush profile to mesh vertices within a radius of curve segments, providing an ideal edit by directly deforming the mesh geometry. We also compare our method to a simplified mesh, which has approximately the same number of triangles as the neural SDF model’s parameters via quadratic decimation, ensuring a fair comparison in terms of representation complexity. Lastly, we extend the point-based editing approach from 3DNS, applying the brush at multiple points along the stroke, represented by a Catmull-Rom spline.

We compute the Chamfer distance over 100,000 samples between the edited surface and the ground truth mesh. For these experiments, the brush radius is 0.08 and its intensity is 0.06. For a fair comparison, a quintic brush profile is used which can be expressed in the 3DNS framework. 12 point samples are used along the stroke and the number of samples in the untouched region is set to 120,000. The models are fine-tuned for 100 epochs. The mean is computed over 10 independent edits and summarized in Table.~\ref{tab:chamfer}, considering both the interaction region and the entire surface. Our method clearly outperforms the other approaches.

Fig.~\ref{fig:comparison} compares the application of a brush stroke across different methods at a similar computational cost. Our approach smoothly approximates the target edit even with sparse sampling along the stroke direction, whereas 3DNS results in jagged deformations due to pointwise editing. The coarse mesh edit, on the other hand, is both noisy and inaccurate.  

Increasing the number of points along the stroke and offset directions improves edit fidelity, as shown by decreasing chamfer distance values in Table~\ref{tab:cd_values} for a fixed sphere edit. While sharp boundary features observed at the edges of the brush profile in Fig.~\ref{fig:comparison} ground truth edit are theoretically achievable with our method, their realization depends on whether the underlying neural representation can support such high-frequency details.

\begin{figure}[h]
    \centering
    % Adjust the width of each subfigure to fit within a single row
    \begin{subfigure}{0.24\columnwidth} % Adjust size to fit in one row
        \includegraphics[width=\linewidth]{figs/bunny00.png}
    \end{subfigure}
    \vline
    % Second image
    \begin{subfigure}{0.24\columnwidth}
        \includegraphics[width=\linewidth]{figs/bunny03.png}
    \end{subfigure}
    % Third image
    \begin{subfigure}{0.24\columnwidth}
        \includegraphics[width=\linewidth]{figs/bunny02.png}
    \end{subfigure}
    % Fourth image
    \begin{subfigure}{0.24\columnwidth}
        \includegraphics[width=\linewidth]{figs/bunny01.png}
    \end{subfigure}
    
    \caption{Example of a stroke applied using different methods. From left to right: ideal edit (ground truth), our method, 3DNS, and edit on a coarse mesh.}
    \label{fig:comparison}
    \vspace{-0.4cm}
\end{figure}

\subsection{Efficiency Comparison}
The primary performance metric for an editing framework is its speed and interactivity. Table \ref{tab:timing} compares the average time and speedup achieved with our tubular sampling method versus the 3DNS approach, for varying numbers of points sampled along the stroke. The models were fine-tuned for 50 epochs--sufficient to yield a decent edit, with 10,000 sample points used to regularize the model's untouched regions. Results are averaged over 100 iterations and were computed on an NVIDIA GeForce RTX 4070 8 GB GPU. Our tubular sampling method enables faster edits (under a second) by using fewer sample points in the interaction region. In contrast, the point-based approach slows significantly as point density along the stroke increases, making it ill-suited for interactive use.
