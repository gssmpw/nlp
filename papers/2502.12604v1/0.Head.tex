\documentclass[lettersize,journal]{IEEEtran}
\usepackage{graphicx}
\usepackage{subcaption}
\usepackage{multirow}
\usepackage{booktabs}
\usepackage[table]{xcolor}
\usepackage{color}
\usepackage{colortbl}
\usepackage{tabularx}
\usepackage{bm}
\usepackage{url}
\usepackage{stackengine}
\usepackage{float}
\usepackage{verbatim}
\usepackage{array}
\usepackage{cite}
\usepackage{algorithm}
\usepackage{algorithmic}
\usepackage[colorlinks,linkcolor=blue]{hyperref}
\usepackage{amsmath,amssymb} % define this before the line numbering.
\usepackage{amsfonts}
\usepackage{comment}
\usepackage{svg}
\usepackage{orcidlink}
\usepackage{textcomp}
\usepackage{stfloats}
\hyphenation{op-tical net-works semi-conduc-tor IEEE-Xplore}

\begin{document}
%\title{S2C: A Contrastive Difference Learning Framework for Unsupervised Change Detection in VHR Remote Sensing Images}

%\title{S2C: A Noise-Resistant Difference Learning Framework for Unsupervised Change Detection in Optical and Multimodal Remote Sensing Images}

\title{S2C: Learning Noise-Resistant Differences for Unsupervised Change Detection in Multimodal Remote Sensing Images}

\author{Lei~Ding\orcidlink{0000-0003-0653-8373}, Xibing~Zuo, Danfeng~Hong,~\IEEEmembership{Senior Member,~IEEE,}, Haitao~Guo, Jun~Lu, Zhihui~Gong and Lorenzo~Bruzzone\orcidlink{0000-0002-6036-459X},~\IEEEmembership{Fellow,~IEEE}~

\thanks{L. Ding is with the Information Engineering University, Zhengzhou, China, and also with the Aerospace Information Research Institute, Chinese Academy of Sciences, Beijing, China (E-mail: dinglei14@outlook.com).}

\thanks{Xibing~Zuo, Haitao~Guo, Jun~Lu, and Zhihui~Gong are with the Information Engineering University, Zhengzhou, China.}

\thanks{D. Hong is with the Aerospace Information Research Institute, Chinese Academy of Sciences, Beijing, 100094, China, and also with the School of Electronic, Electrical and Communication Engineering, University of Chinese Academy of Sciences, 100049 Beijing, China. (e-mail: hongdf@aircas.ac.cn)}

\thanks{L. Bruzzone is with the Department of Information Engineering and Computer Science, University of Trento, 38123 Trento, Italy (E-mail: lorenzo.bruzzone@unitn.it).}

\thanks{This document is funded by the Natural Science Foundation of China under Grant 42201443. It is also funded by the Henan Provincial Key Technologies R \& D Program under Grant 242102211047. (Corresponding author: Lei Ding.)}}

\markboth{Manuscript under review}%
{Shell \MakeLowercase{\textit{et al.}}: Bare Demo of IEEEtran.cls for IEEE Journals}

\maketitle

\begin{abstract}
Unsupervised Change Detection (UCD) in multimodal Remote Sensing (RS) images remains a difficult challenge due to the inherent spatio-temporal complexity within data, and the heterogeneity arising from different imaging sensors. Inspired by recent advancements in Visual Foundation Models (VFMs) and Contrastive Learning (CL) methodologies, this research aims to develop CL methodologies to translate implicit knowledge in VFM into change representations, thus eliminating the need for explicit supervision. To this end, we introduce a Semantic-to-Change (S2C) learning framework for UCD in both homogeneous and multimodal RS images. Differently from existing CL methodologies that typically focus on learning multi-temporal similarities, we introduce a novel triplet learning strategy that explicitly models temporal differences, which are crucial to the CD task. Furthermore, random spatial and spectral perturbations are introduced during the training to enhance robustness to temporal noise. In addition, a grid sparsity regularization is defined to suppress insignificant changes, and an IoU-matching algorithm is developed to refine the CD results. Experiments on four benchmark CD datasets demonstrate that the proposed S2C learning framework achieves significant improvements in accuracy, surpassing current state-of-the-art by over 31\%, 9\%, 23\%, and 15\%, respectively. It also demonstrates robustness and sample efficiency, suitable for training and adaptation of various Visual Foundation Models (VFMs) or backbone neural networks. The relevant code will be available at: \href{github.com/DingLei14/S2C}{github.com/DingLei14/S2C}.
\end{abstract}

\begin{IEEEkeywords}
Unsupervised Change Detection, Visual Foundation Model, Contrastive Learning, Remote Sensing
\end{IEEEkeywords}

%%%%%%%%%%%%%%%%%%%%%%%%%%%%%%%%%%%%%%%%%%
\section{Introduction}

In recent years, with advancements in generative models and the expansion of training datasets, text-to-speech (TTS) models \cite{valle, voicebox, ns3} have made breakthrough progress in naturalness and quality, gradually approaching the level of real recordings. However, low-latency and efficient dual-stream TTS, which involves processing streaming text inputs while simultaneously generating speech in real time, remains a challenging problem \cite{livespeech2}. These models are ideal for integration with upstream tasks, such as large language models (LLMs) \cite{gpt4} and streaming translation models \cite{seamless}, which can generate text in a streaming manner. Addressing these challenges can improve live human-computer interaction, paving the way for various applications, such as speech-to-speech translation and personal voice assistants.

Recently, inspired by advances in image generation, denoising diffusion \cite{diffusion, score}, flow matching \cite{fm}, and masked generative models \cite{maskgit} have been introduced into non-autoregressive (NAR) TTS \cite{seedtts, F5tts, pflow, maskgct}, demonstrating impressive performance in offline inference.  During this process, these offline TTS models first add noise or apply masking guided by the predicted duration. Subsequently, context from the entire sentence is leveraged to perform temporally-unordered denoising or mask prediction for speech generation. However, this temporally-unordered process hinders their application to streaming speech generation\footnote{
Here, “temporally” refers to the physical time of audio samples, not the iteration step $t \in [0, 1]$ of the above NAR TTS models.}.


When it comes to streaming speech generation, autoregressive (AR) TTS models \cite{valle, ellav} hold a distinct advantage because of their ability to deliver outputs in a temporally-ordered manner. However, compared to recently proposed NAR TTS models,  AR TTS models have a distinct disadvantage in terms of generation efficiency \cite{MEDUSA}. Specifically, the autoregressive steps are tied to the frame rate of speech tokens, resulting in slower inference speeds.  
While advancements like VALL-E 2 \cite{valle2} have boosted generation efficiency through group code modeling, the challenge remains that the manually set group size is typically small, suggesting room for further improvements. In addition,  most current AR TTS models \cite{dualsteam1} cannot handle stream text input and they only begin streaming speech generation after receiving the complete text,  ignoring the latency caused by the streaming text input. The most closely related works to SyncSpeech are CosyVoice2 \cite{cosyvoice2.0} and IST-LM \cite{yang2024interleaved}, both of which employ interleaved speech-text modeling to accommodate dual-stream scenarios. However, their autoregressive process generates only one speech token per step, leading to low efficiency.



To seamlessly integrate with  upstream LLMs and facilitate dual-stream speech synthesis, this paper introduces \textbf{SyncSpeech}, designed to keep the generation of streaming speech in synchronization with the incoming streaming text. SyncSpeech has the following advantages: 1) \textbf{low latency}, which means it begins generating speech in a streaming manner as soon as the second text token is received,
and
2) \textbf{high efficiency}, 
which means for each arriving text token, only one decoding step is required to generate all the corresponding speech tokens.

SyncSpeech is based on the proposed \textbf{T}emporal \textbf{M}asked generative \textbf{T}ransformer (TMT).
During inference, SyncSpeech adopts the Byte Pair Encoding (BPE) token-level duration prediction, which can access the previously generated speech tokens and performs top-k sampling. 
Subsequently, mask padding and greedy sampling are carried out based on  the duration prediction from the previous step. 

Moreover, sequence input is meticulously constructed to incorporate duration prediction and mask prediction into a single decoding step.
During the training process, we adopt a two-stage training strategy to improve training efficiency and model performance. First, high-efficiency masked pretraining is employed to establish a rough alignment between text and speech tokens within the sequence, followed by fine-tuning the pre-trained model to align with the inference process.

Our experimental results demonstrate that, in terms of generation efficiency, SyncSpeech operates at 6.4 times the speed of the current dual-stream TTS model for English and at 8.5 times the speed for Mandarin. When integrated with LLMs, SyncSpeech achieves latency reductions of 3.2 and 3.8 times, respectively, compared to the current dual-stream TTS model for both languages.
Moreover, with the same scale of training data, SyncSpeech performs comparably to traditional AR models in terms of the quality of generated English speech. For Mandarin, SyncSpeech demonstrates superior quality and robustness compared to current dual-stream TTS models. This showcases the potential of  SyncSpeech as a foundational model to integrate with upstream LLMs.


%%%%%%%%%%%%%%%%%%%%%%%%%%%%%%%%%%%%%%%%%%%%%%%%%%%
% Related Work
%%%%%%%%%%%%%%%%%%%%%%%%%%%%%%%%%%%%%%%%%%%%%%%%%%%
\section{Related Work}
\label{sec:related-work}




%\subsection{Question Answering using Sensor Data}
\textbf{Question Answering using Sensor Data}
The QA problem has been extensively studied across various domains, including text~\cite{rogers2023qa}, visual~\cite{schwenk2022okvqa}, medical~\cite{pal2022medmcqa}, and remote sensing~\cite{hu2023rsgpt}. In the sensor domain, early works \textit{AI Therapist}~\cite{nie2022conversational} and its successor CaiTI~\cite{nie2024llm} utilized smart home devices, such as Amazon Echo, to engage in conversations with users and assess mental well-being. 
DeepSQA~\cite{xing2021deepsqa} was the first to benchmark time-series sensor-based QA for human activity recognition. It introduced SQA-GEN, an automated QA generation tool that gathers 1-minute sensor readings and generates valid Q\&A pairs by exhaustively searching six pre-defined question templates. They mainly focused on quantitative questions including time query, counting and action compare.
The authors also evaluated traditional neural network models, including CNNs and LSTMs, finding that the ConvLSTM network with Compositional Attention achieved the highest QA accuracy.
%However, DeepSQA is limited by a restricted variety of questions and answers, formulating the problem as a classification task. In contrast, \Method supports natural question types and varied time lengths.

Recent contributions have pioneered the integration of LLMs for generating more intuitive answers and better interpreting sensor data. Englhardt \textit{et al.}~\cite{englhardt2024classification}, Health-LLM~\cite{kim2024health}, and DrHouse~\cite{yang2024drhouse} converted physiological data from wearable devices, such as heart rates and daily step counts, into text prompts for LLMs, enabling more sophisticated medical and healthcare diagnoses.
The latest Sensor2Text~\cite{chen2024sensor2text} and PrISM-Q\&A~\cite{arakawa2024prism} explored natural language interactions between users and wearable devices to understand and support daily activities using sensor data, such as advising on "What should I do next with this?" Both approaches utilized LLMs as their backbone and fed sensor embeddings into the models.

While promising, existing systems exhibit significant limitations in handling long duration and complex sensor data required for accurate answers. Table~\ref{tbl:related_works} highlights the key differences between \Method and existing QA systems. In summary, \Method greatly enhances capabilities of existing systems by (1) answering questions based on long-duration sensor data spanning weeks or months, compared to the short windows of seconds or minutes in previous systems~\cite{xing2021deepsqa,chen2024sensor2text,arakawa2024prism}, and (2) encoding high-dimensional time-series data to extract fine-grained activity details for reasoning, unlike prior systems that are limited to low-dimensional sensor data~\cite{englhardt2024classification,kim2024health,yang2024drhouse}.
%However, these works use coarse-grained sensor data which cannot be easily adapted to timeseries and detailed quantitative questions.
%\Method instead focuses on analyzing rich timeseries data.
%However, these works have two drawbacks: (1) they merely input text-formatted sensor data into LLM which may not work well for timeseries; (2) their usage of closed-source LLMs such as GPT-4 and external knowledge databases are challenging for edge devices. \Method addresses both drawbacks. 
%However, all above works focused on health-related aspects and used lower dimensional of data compared to raw sensor time series. Their approaches cannot be directly transferred to daily-life activity monitoring with raw time series sensors.

%fused low-dimensional sensors data such as from smart home sensors or wristband with LLMs to infer health status.
%infer mental health and assess the user's daily functioning using GPT-3. %Their approach integrated a GPT-3-based natural language processing core to interact with users as well as detect abnormal mental status, based on 37 dimensions suggested by the therapists.

%considering only the top 27 answers and 

%A major constraint lies in their language models, which predict outputs as a classification problem - such as a 0-2 mental health core in \textit{AI therapist} and only the top 27 answers considered as the ground-truth labels in DeepSQA. This design restricts their models to a highly constrained QA scenario that may not be applicable in real life.
%In contrast, the distinctiveness of \Method lies in its capacity to accommodate "arbitrary" and "unpredictable" questions, namely open-ended question answering.

%\subsection{LLMs for Multimodal Reasoning}
%\vspace{1mm}
\textbf{LLMs for Multimodal Reasoning}
%The surge of LLM has triggered a wave of innovations in text understanding and reasoning applications.
Recent works investigated multimodal LLMs that transform other data modalities into a sequence of tokens for LLM integration~\cite{zhang2023llama,lin2024vila}.
IMU2CLIP~\cite{moon-etal-2023-imu2clip} and TENT~\cite{zhou2023tent} employed contrastive pretraining to align text with various timeseries sensor signals.
%To enable this transformation, IMU2CLIP~\cite{moon-etal-2023-imu2clip} utilized contrastive pretraining to align IMU signals with text narration~\cite{grauman2022ego4d}. Similarly, TENT~\cite{zhou2023tent} employed contrastive pretraining to align text with a broader set of IoT sensor signals, including camera video, LiDAR, and mmWave data. Both approaches have limited sensor-specific reasoning capabilities as they used a frozen LLM without further tuning or did not utilize an LLM at all. 
Recent designs like AnyMAL~\cite{moon2023anymal} and OneLLM~\cite{han2024onellm} proposed fine-tuning multimodal LLMs to process up to eight different modalities, including IMU time series, for reasoning tasks.
However, direct integration of time series sensor data to LLMs is constrained by LLM's inherent weakness in handling long-context inputs~\cite{li2024long,gu2023mamba}, making them unsuitable for processing long-duration sensor signals. In contrast, \Method overcomes this limitation by incorporating a dedicated sensor data query stage, enabling it to handle long-term queries effectively.
%requires a large instruction dataset with well-aligned multimodal data, which is lacking in multimodal sensor-based QA. 
%that accept IMU signal inputs using carefully prepared instruction datasets.

%\begin{wraptable}{r}{0.55\textwidth}
%\vspace{-4mm}
\begin{table}
\small
\caption{Comparing {\Method} and existing QA systems.}
\label{tbl:related_works}
\vspace{-4mm}
\begin{center}
\begin{tabular}{ccc} % note: no vertical bars at all
\toprule
\textbf{Existing QA systems for sensor data} & \textbf{Long-duration} & \textbf{High-dimensional} \\
& \textbf{sensor data} & \textbf{time series sensors} \\
\midrule
DeepSQA~\cite{xing2021deepsqa}, Sensor2Text~\cite{chen2024sensor2text}, PrISM-Q\&A~\cite{arakawa2024prism} & \X & \Ch \\
Englhardt \textit{et al.}~\cite{englhardt2024classification}, Health-LLM~\cite{kim2024health}, DrHouse~\cite{yang2024drhouse}& \Ch & \X \\ \midrule
 \textbf{\Method (this work)} & \textbf{\Ch} & \textbf{\Ch} \\
\bottomrule
\end{tabular}
\end{center}
\vspace{-4mm}
\end{table}
%\end{wraptable}


\iffalse
\begin{figure*}[htbp]
    \centering
    \begin{subfigure}[b]{0.48\textwidth}
        \centering
        \includegraphics[width=\textwidth]{figs/frequency_query.png}
        \vspace{-6mm}
        \caption{Example question of frequency query.}
        \label{fig:daily-freq}
    \end{subfigure}
    \hfill
    \begin{subfigure}[b]{0.48\textwidth}
        \centering
        \includegraphics[width=\textwidth]{figs/time_query.png}
        \vspace{-6mm}
        \caption{Example question of time query.}
        \label{fig:daily-time}
    \end{subfigure}
    
    \begin{subfigure}[b]{0.48\textwidth}
        \centering
        \includegraphics[width=\textwidth]{figs/day_query.png}
        \vspace{-6mm}
        \caption{Example question of day query.}
        \label{fig:weekly-day}
    \end{subfigure}
    \hfill
    \begin{subfigure}[b]{0.48\textwidth}
        \centering
        \includegraphics[width=\textwidth]{figs/activity_query.png}
        \vspace{-6mm}
        \caption{Example question of activity query.}
        \label{fig:weekly-activity}
    \end{subfigure}
    \vspace{-4mm}
    \caption{Example QA pairs that we collect from AMT. (a) and (b) are generated from daily graph, while (c) and (d) are generated from weekly graph.}
    \label{fig:example_qas}
    \vspace{-4mm}
\end{figure*}
\fi





\iffalse
Limited prior research has studied finetuning LLMs for specific reasoning tasks. Based on IMU2CLIP, AnyMAL~\cite{moon2023anymal} further finetuned the LLM by training projection layers or applying Low-Rank Adaptation (LoRA).
Nevertheless, the general fine-tuning approach of AnyMAL struggles to accurately address sensor-specific queries, such as those pertaining to specific times and locations. These challenges are effectively tackled in \Method through the introduction of two novel fine-tuning techniques, setting \Method apart from existing methodologies.
\fi

%\subsection{Human Activity Monitoring}
%\vspace{1mm}
\textbf{Mobile Systems for Daily Life Monitoring}
Researchers have explored a variety of sensing technologies for monitoring human lives, including built-in sensors on smart phones~\cite{zhang2020pdlens}, cameras~\cite{radu2019vision2sensor}, Wi-Fi signals~\cite{yang2024mm} and mmWave radar~\cite{weng2024large}.
Although these efforts have resulted in numerous open-source datasets~\cite{misc_human_activity_recognition_using_smartphones_240,misc_mhealth_dataset_319,misc_opportunity_activity_recognition_226,vaizman2017recognizing} and powerful machine learning models~\cite{ma2019attnsense,xu2021limu,deldari2022cocoa,ouyang2022cosmo,ouyang2023harmony,xu2023practically}, they fail to handle queries in natural language, which are more creative and open-ended.
%the vast majority of the existing work has focused on human activity recognition. %, a \textit{passive} way of processing information where the user has no control. 
%How to further interpret the data beyond activity classification has remained less explored.
In this paper, we design \Method to facilitate long-term QA interactions based on time series sensor data.
The methodology of \Method can be further integrated with other sensing modalities and applications.

\section{Proposed Approach}\label{sc3}

This section first explains the motivation for learning noise-resistant representations, then introduces the S2C learning framework, and subsequently details the proposed technologies including contrastive change learning, grid sparsity loss, and the change mapping algorithms. Finally, the S2C methodologies are expanded to address unsupervised MMCD.

\subsection{Noise-resistant Semantic Embedding}\label{sc3.A}

DL-based CD essentially learns to project multi-temporal RS images ${I_1, I_2}$ into a binary change map $\mathbf{y}_c$. Let $f_\theta$ denote an encoding function parameterized by $\theta$, and $g$ being a projection function. This process can be represented as:
\begin{equation}\label{eq.cd}
    \mathbf{y_1} = f_\theta(I_1), \mathbf{y_2} = f_\theta(I_2), \mathbf{y_c} = g(\mathbf{y_1}, \mathbf{y_2})
\end{equation}
where $\mathbf{y_1}, \mathbf{y_2} \in \mathbb{R}^{c \times h \times w}$ are the learned semantic latent, $\mathbf{y_c} \in \mathbb{R}^{h \times w}$ is a change probabilistic map, $s$ and $h, w$ are the channel and spatial dimensions, respectively.

Let us denote the imaging process as $\Phi$, ground semantics as $s$, and semantic changes as $\delta$. In an ideal case where there is no temporal noise, this process can be re-written as:
\begin{equation}\label{eq.imaging}
    \mathbf{y}_s = f_\theta[\Phi_1(s)], \mathbf{y}_{s+\delta} = f_\theta[\Phi_2(s+\delta)], \mathbf{y_c} = g(\mathbf{y}_s, \mathbf{y}_{s+\delta})
\end{equation}
Since $\mathbf{y}_s$ and $\mathbf{y}_{s+\delta}$ share a common semantic space, $g$ can be implemented using a simple linear transformation with normalization. However, when considering practical cases that involve temporal noise (refer Fig.\ref{fig.challenge}), there exist insignificant changes (denoted $\epsilon$), spatial variance (denoted $\Omega$), and sensor differences ($\Phi_1 \neq \Phi_2$). Consequently, $I_2$ is projected into a different space:
\begin{equation}\label{eq.noise}
    f_\theta(I_2)=f_\theta\{\Omega[\Phi_2(s+\delta+\epsilon)]\}=\mathbf{y^{\prime}}_{s+\delta+\epsilon}
\end{equation}

Thus, the distance between $\mathbf{y}_s$ and $\mathbf{y^{\prime}}_{s+\delta+\epsilon}$ is nonlinear, and optimizing subsequent change embedding with $g$ typically requires supervised learning with task-specific labels.

To achieve unsupervised change learning, we utilize spatial and spectral augmentations to simulate various types of temporal noise, thus training a noise-resistant $f_\theta$ and formulating a training-free $g$. First, we apply augmentations functions $\phi$ on $I_1$ simulating the noise in Eq.(\ref{eq.noise}):
\begin{equation}\label{eq.phi_I1}
    \phi(I_1)= \hat{\Omega}\{\hat{\Phi}_{1 \rightarrow 2}[\Phi_1(s)]\} = \hat{\Omega}[\hat{\Phi}_{1,2}(s)],
\end{equation}
where $\hat{\Omega}$ is a set of spatial augmentations, $\hat{\Phi}_{1 \rightarrow 2}$ is a set of spectral augmentations that adapt the spectral distribution of $I_1$ approaching $I_2$. The augmentations are performed with stochastic parameters to simulate the imagining process with random noise, generating diverse training pairs. Further utilizing CL techniques elaborated in Sec.\ref{sc3.CL}, we can train $f_\theta$ towards projecting $I_1$ and $\phi(I_1)$ into similar representations, i.e., learning noise-invariant semantic latent:
\begin{equation}
    f_\theta(I_1) = \mathbf{\hat{y}}_s, 
    f_\theta[\phi(I_1)] = f_\theta\{ \hat{\Omega}[\hat{\Phi}_{1,2}(s)] \} \approx \mathbf{\hat{y}}_{s}
\end{equation}
These operations are also symmetrically performed on $I_2$ as:
\begin{equation}\label{eq.phi_I2}
\begin{aligned}
    \phi(I_2) = \hat{\Omega}\{\hat{\Phi}_{2 \rightarrow 1}[\Phi_2(s+\delta+\epsilon)]\} \approx \hat{\Omega}[\hat{\Phi}_{1,2}(s+\delta)],\\
    f_\theta(I_2) = \mathbf{\hat{y}}_{s+\delta}, f_\theta[\phi(I_2)] \approx f_\theta\{ \hat{\Omega}[\hat{\Phi}_{1,2}(s)] \} \approx \mathbf{\hat{y}}_{s+\delta}
\end{aligned}
\end{equation}
where $\epsilon$ denoting the subtle changes can be diminished since $\hat{\Omega}$ includes spatial filtering operations.
This enables modeling the semantic differences between $\mathbf{\hat{y}}_s, \mathbf{\hat{y}}_{s+\delta}$ in a common latent space, enabling a training-free formulation of $g$:
\begin{equation}
    \mathbf{y_c} = g(\mathbf{\hat{y}}_s, \mathbf{\hat{y}}_{s+\delta})
\end{equation}

\subsection{Overview of the S2C framework}
The theoretical analysis in Sec.\ref{sc3.A} offers a simplified understanding of the optimizing goals. In the following, we elaborate on the S2C architecture that trains $f_\theta$ to exploit noise-invariant semantic representations.

As depicted in Fig.\ref{fig.flowchar}, S2C is a UCD framework consisting of distinct stages of training and inference. During the training phase, a set of augmentation operations are first performed on the input images, corresponding to $\phi(\cdot)$ in Eq.(\ref{eq.phi_I1})(\ref{eq.phi_I2}). The augmentations comprise a sequential combination of spectral and spatial operations executed randomly at each training iteration, including \textit{RGBshift}, \textit{PCA adaptation}, \textit{down-sampling} and \textit{random shifting}. \textit{PCA adaptation} blends the spectral distribution of $I_1$ and $I_2$. The spatial operations are performed to simulate spatial misalignment and imaging degradation/distortion, while the spectral operations replicate imaging and seasonal variations. Collectively, their combination simulate stochastic injection of diverse temporal noise.

Subsequently, utilizing the constructed pre- and after-augmentation image sequences, two self-supervised CL strategies are jointly employed to learn task-specific semantic features. Although this training process can be performed from scratch using common DNNs, employing VFM as feature encoders empirically leads to better accuracy. In addition, considering the limitations of VFM in processing RS images \cite{ji2024segment}, we employ a VFM augmented with additional parameters $w$, denoted as $f_{\theta + w}$. $w$ is a trainable parameter implementing LoRA \cite{hu2021lora} (with $rank=4$), a parameter-efficient technique extensively used to adapt VFMs to a particular domain of interest. $\theta$ is frozen to retain pre-trained visual knowledge, whereas $w$ are the LoRA weights trained with CL paradigms to exploit temporal semantic features. The VFM can be any off-the-shelf models, as its inner structure is not modified in our S2C architecture.

The training process is conducted within two CL paradigms to learn the CD-relevant semantic representations. The two CL paradigms are introduced to embed difference and consistency representations, respectively, which are elaborated in Sec.\ref{sc3.CL}. The associated loss functions are $\mathcal{L}_{tri}$ and $\mathcal{L}_{info}$, respectively. In addition, we further introduce a sparsity regularization objective to learn sparse and compact change representations, noted as $\mathcal{L}_{spa}$. The joint training objective is:
\begin{equation}\label{eq.losses}
    \mathcal{L} = \mathcal{L}_{tri} + \alpha \mathcal{L}_{info} + \beta \mathcal{L}_{spa}
\end{equation}
where $\alpha$ and $\beta$ are two weighting parameters.

In the inference phase, the semantic latent $\mathbf{y}_1$ and $\mathbf{y}_2$ are first mapped to a coarse change map, then refined using the VFM decoder and an IoU matching function. The details are elaborated in \ref{sc3.chg_map}.

\subsection{Contrastive Change Learning} \label{sc3.CL}

\begin{figure*}[t]
\centering
    \includegraphics[width=0.95\linewidth]{illu_pic/CL_pipelines.png}
    \caption{Comparison of CL paradigms in CD. (a) \textbf{Consistency regularization}: $f_\theta$ extracts stable representations across weak/strong perturbations; (b) \textbf{Spatial contrast}: $f_\theta$ distinguishes same/different regions; (c) Proposed \textbf{Consistency-regularized Temporal Contrast} (CTC): $f_\theta$ identifies temporal differences independent of spectral or seasonal variations, and (d) Proposed \textbf{Consistency-regularized Spatial Contrast} (CSC): $f_\theta$ distinguishes same/different regions despite perturbations.}
\label{fig.CL_paradigms}
\end{figure*}

Before introducing the proposed CL paradigms, let us first review the two typical CL paradigms in CD, and analyze their usage and limitations.

\textit{1) Consistency Regularization (CR).} As depicted in Fig.\ref{fig.CL_paradigms}(a), a DNN $f_{\theta}$ learns to improve the robustness and generalization of feature embeddings. An image $I$ is first augmented with weak and strong transformations, resulting in two copies $\Tilde{I}$ and $\Bar{I}$. Then a distance loss is calculated between the two copies to ensure consistency across perturbations.

Since this learning paradigm does not explicitly model differences/similarities, it is often adopted in semi-supervised \cite{bandara2022revisiting} or weakly-supervised \cite{zhao2024pixellevel} learning settings to extend the CD insights learned with limited samples.

\textit{2) Spatial Contrast (SC).} As illustrated in Fig.\ref{fig.CL_paradigms}(b), $f_{\theta}$ learns to differentiate between bitemporal image pairs $[I_1^i, I_2^i]$ of the same region $i$ and $[I_1^i, I_2^j]$ of different regions $i$ and $j$. This drives $f_{\theta}$ to learn consistent embeddings against temporal variations. Areas with high similarity are identified as $unchanged$, whereas their opposites are detected as $changed$ \cite{chen2022selfsupervised}.

However, we identify that there are several limitations in this paradigm: i) changes are identified through negative embedding of the similarities rather than through explicit modeling. This often causes sensitivity to noise. ii) $f_{\theta}$ focuses on the discriminative elements within a region, such as certain edges or corners, rather than effectively exploiting the local semantic context. %iii) the supervision signal is given at the image level (same/different pairs), thus only coarse similarity representations are obtained.

Considering these limitations, we introduce two novel CL paradigms specifically tailored to the context of CD.

\textbf{Consistency-regularized Temporal Contrast} (CTC). An RS image $I_1$ is first augmented with a transform function $\phi(\cdot)$, producing a copy $\Bar{I}_1$. Subsequently, $I_1$ is employed as an anchor for comparison with both a positive sample $\Bar{I}_1$ and a negative sample $I_2$. $\phi(\cdot)$ simulates spectral and spatial noise between multi-temporal observations, as illustrated in Fig.\ref{fig.flowchar}. Consequently, $f_\theta$ learns to exploit noise-invariant difference representations, i.e., the semantic changes. With greater details, Fig.\ref{fig.flowchar} illustrates the CTC paradigm with bi-directional comparisons within $[I_1, \Bar{I}_1, I_2]$ and $[I_2, \Bar{I}_2, I_1]$. 

A triplet training objective $\mathcal{L}_{tri}$ using cosine distance is utilized for comparisons within the triplets. This is to align with the cosine difference embedding during the inference stage. The calculations are as follows:
\begin{equation}
\begin{aligned}
    %\mathbf{y_1}, \mathbf{y_2} = f_{\theta + w}(I_1), f_{\theta + w}(I_2),\\
    %\mathbf{\Bar{y}_1}, \mathbf{\Bar{y}_2} = f_{\theta+w}[\phi(I_1)], f_{\theta+w}[\phi(I_2)],\\
    \mathcal{L}_{tri} = max[cos(\mathbf{y_1}, \mathbf{y_2})-cos(\mathbf{y_1}, \mathbf{\Bar{y}_1})+m, 0]\\
    + max[cos(\mathbf{y_2}, \mathbf{y_1})-cos(\mathbf{y_2}, \mathbf{\Bar{y}_2})+m, 0]
\end{aligned}
\end{equation}
where $m=1$ is a margin parameter to promote separation between the anchor and positive.

\textbf{Consistency-regularized Spatial Contrast} (CSC). This contrastive learning paradigm integrates CR into typical SC learning, thereby enhancing the embedding of spatial consistency against perturbations. CSC alleviates the vulnerability to noise inherent in the SC paradigm by incorporating transformation $\phi(\cdot)$. The transformations, particularly the spatial transformations, reduce dependence on high-frequency spatial details, thereby prompting the exploitation of local semantic patterns such as color and texture.

We have introduced an additional variation in CSC, i.e., the calculation of consistency at each spatial position. Given a batch consisting of $N$ paired RS images $\{[I_1^i, I_2^i], [I_1^j, I_2^j], ..., [I_1^k, I_2^k]\}$, we first apply $\phi(\cdot)$ on each of the temporal images, thus getting two sets of augmented images. These images are further encoded with $f_{\theta+w}$, resulting in 4 sets of features: $[\mathbf{y_1^i, y_1^j, ..., y_1^k}]$, $[\mathbf{y_2^i, y_2^j, ..., y_2^k}]$, $[\mathbf{\Bar{y}_1^i, \Bar{y}_1^j, ..., \Bar{y}_1^k}]$ and $[\mathbf{\Bar{y}_2^i, \Bar{y}_2^j, ..., \Bar{y}_2^k}]$. We then calculate the co-occurrences between them, resulting in two matrices each with $N \times N$ dimensions, as illustrated in Fig.\ref{fig.CL_paradigms}(d). We utilize an infoNCE loss function to effectively train $f_{\theta+w}$ for the differentiation of genuine image pairs. It is calculated across both temporal phases, represented as:
\begin{equation} \label{eq.info_loss}
\begin{aligned}
    \mathcal{L}_{info} = -\frac{1}{N} \sum_{u=1}^{N} \log \left[ \frac{\exp \left( \mathbf{y}_1^u \odot \mathbf{\Bar{y}}_2^u \right)}{\sum_{v=1}^{N} \exp \left( \mathbf{y}_1^u \odot \mathbf{\Bar{y}}_2^v \right)} \right]\\
    -\frac{1}{N} \sum_{u=1}^{N} \log \left[ \frac{\exp \left( \mathbf{y}_2^u \odot \mathbf{\Bar{y}}_1^u \right)}{\sum_{v=1}^{N} \exp \left( \mathbf{y}_2^u \odot \mathbf{\Bar{y}}_1^v \right)} \right]
\end{aligned}
\end{equation}
where $\odot$ denotes a spatial similarity function that we introduce in this study. Instead of pooling the spatial features into single vectors for similarity calculation \cite{chen2022selfsupervised}, we compute the similarity at each spatial patch $p$, denoted as:
\begin{equation} \label{eq.sim_calc}
    \mathbf{y} \odot \mathbf{\Bar{y}} = \frac{1}{w \times h} \sum_{p} \left( \frac{\mathbf{y}^p \cdot \mathbf{\Bar{y}}^p}{|\mathbf{y}^p||\mathbf{\Bar{y}}^p|} \right)
\end{equation}
Both $\mathcal{L}_{tri}$ and $\mathcal{L}_{info}$ are calculated based on cosine similarity. While $\mathcal{L}_{tri}$ embeds appearance-invariant temporal differences, $\mathcal{L}_{info}$ embeds noise-resilient temporal consistencies. Therefore, when a certain temporal consistency pattern is captured in CSC, it suppresses the difference representations of the same area in CTC.

\subsection{Grid Sparsity loss} \label{sc3.loss_sparse}

Changed objects are commonly sparsely distributed in RS images and each present as compact regions. In contrast, edges and points are often associated with noise. Although training objectives that promote sparse representations have been explored in the literature, they typically calculate and penalize the average value of $\mathbf{y_c}$ \cite{bandara2023deep}. However, this approach does not guarantee sparsity, as there exists a trivial solution of learning an additional bias term on $\mathbf{y_c}$.

Differently, we propose a novel grid sparsity loss where sparsity is assessed at the level of each local grid rather than at each pixel. Considering the frequency of changes along with spatial resolution in an RS image, first, we predefine a sparsity threshold $T$ as well as a grid size $d$. Subsequently, the average density of each grid $\mathbf{g}$ is calculated and ranked, while a {\footnotesize$1-T$} ratio of grids with the lowest density are selected for loss calculation as follows:
\begin{equation} \label{eq.sparse_loss}
\begin{aligned}
    \mathcal{L}_{spa} = max\{ \frac{1}{n} \sum^{n} [sort\uparrow(y^\mathbf{g})], 0\},\\
    y^\mathbf{g} = \frac{1}{d*d} \sum_{p \in \mathbf{g}} \mathbf{y}_c^p, 
    n = wh*(1-T)/d^2
\end{aligned}
\end{equation}

We empirically set $d=16$ for HR RS images. This regularization objective ensures that a proportion of less than {\footnotesize$1-T$} potential changes exhibit high values, whereas the insignificant change representations in other areas are minimized.

\subsection{Change mapping} \label{sc3.chg_map}

\begin{algorithm}[t]
    \caption{Algorithm for IoU Matching and Refinement}
    \label{AlgorithmD}  \label{Algorithm.IoU_matching}
    \begin{algorithmic}[1]
        \renewcommand{\algorithmicrequire}{\textbf{Input:}}
        \renewcommand{\algorithmicensure}{\textbf{Output:}}
        \REQUIRE Change probability map $\mathbf{y}_c$,\\
                 VFM-extracted bi-temporal object masks $M_1$, $M_2$,\\
                 Parameter: IoU threshold $T_{IoU}$;
        \ENSURE Refined Change Map $M_c$;
        \FORALL{$m_i \in M_1, m_j \in M_2$}
        \IF {$(m_i \cap m_j)/(m_i \cup m_j)>T_{IoU}$}
            \STATE $M_{1} \leftarrow M_{1} \setminus \{m_i\}$
            \STATE $M_{2} \leftarrow M_{2} \setminus \{m_j\}$
            %\STATE $M_{12} \leftarrow M_{12} \cup \{m_i, m_j\}$
        \ENDIF
        \ENDFOR
        \STATE $M_{12} \leftarrow M_{1} \cup M_{2}$
        \FORALL{$m \in M_{12}$}
        \IF {$m \odot \mathbf{y}_c / \sum_{p} m_p >T_{IoU}$}
            \STATE $M_c \leftarrow M_{c} \cup \{m\}$
        \ENDIF
        \ENDFOR
 \RETURN $M_c$ 
    \end{algorithmic}
\end{algorithm}

The use of VFM and CL methodologies aims to enhance the effective exploitation of semantic contexts across multi-temporal image domains. In the inference phase, the major challenge lies in accurately mapping fine-grained changes. We employ a coarse-to-fine refinement strategy. First, a coarse change probability map $\mathbf{y}_c$ is derived by projecting the negative cosine embedding of the bi-temporal semantic embeddings:
\begin{equation}
    \mathbf{y}_c = \sigma[-cos(\mathbf{y}_1, \mathbf{y}_2)*\eta]
\end{equation}
where $\sigma$ is a $sigmoid$ function and $\eta=ln(1/0.07)$ is a scaling factor defined following literature practice \cite{deuser2023sample4geo}.

Then, we employ a pretrained VFM decoder $g_{\gamma}$ to segment two groups of bi-temporal masks $M_1=\{m_1^1, m_1^2, ..., m_1^k\}$ and $\{M_2=m_2^1, m_2^2, ..., m_2^k\}$ using the spatial prompts generated on high-response regions in $\mathbf{y}_c$. Given the logic implication of \textit{change}, high-overlap objects in $M_1$ and $M_2$ can be inferred as false alarms. Therefore, we use an XOR-alike matching algorithm (denoted $\ominus$) to merge $M_1$ and $M_2$ while eliminating the objects with large overlaps:
\begin{equation}
    M_{12} = M_1 \ominus M_2
\end{equation}
We further implement an Intersection-over-Union (IoU) analysis between $\mathbf{y}_c$ and $M_{12}$ to match the VFM-generated masks with the high-confidence regions in $\mathbf{y}_c$. The matched objects replace their counterparts in $\mathbf{y}_c$ as the changed items. For more details, a pseudo-code of this IoU analysis and matching algorithm is provided in Algorithm \ref{Algorithm.IoU_matching}.

Using this change mapping algorithm, the coarse predictions derived from the DNN are refined into detailed CD results, aligning with the spatial details present in the HR imagery.

\subsection{S2C for Unsupervised MMCD}

\begin{figure}[t]
\centering
    \includegraphics[width=1\linewidth]{illu_pic/S2C_flowchart_het.png}
    \caption{Illustration of the application of the proposed S2C for UCD in multimodal RS images. This learning framework applies to not only optical and SAR data, but also other image modalities.}
\label{fig.Het_flowchart}
\end{figure}

RS images observed by specialized sensors such as Synthetic Aperture Radar (SAR) and Infrared (IR) scanners demonstrate a notable modality difference compared to standard optical images. This significant domain gap precludes VFMs from extracting meaningful semantic representations \cite{ji2024segment}. Leveraging the joint modeling of consistency and differences, the S2C framework is capable of modeling modality-invariant change representations over multi-temporal observations.

Fig.\ref{fig.Het_flowchart} depicts the adapted S2C training pipeline for multimodal RS images, where the UCD process is exemplified through a pair of optical and SAR images. The semantic-to-change mapping is achieved through alignment of latent semantics, making it invariant to specific imaging modalities. Let us denote $I_{rgb}^i$ and $I_{sar}^i$ as a pair of multimodal images pertaining to region $i$. Two independent encoders with parameters $\theta$ and $\zeta$ are trained to extract features from optical and SAR images, respectively. It is worth noting that the VFMs are typically not applicable to SAR data and, therefore are not utilized in $f_\zeta$. However, with the continuous development of RS VFMs \cite{guo2024skysense, hong2024spectralgpt}, certain models may demonstrate the efficacy of semantic embedding within a specific domain. While we did not identify an optimal VFM for common SAR data, readers are encouraged to explore emerging VFMs within the S2C framework.

After feature embedding, the CTC and CSC paradigms are utilized to learn domain-invariant differences and consistencies. An adjustment in implementing the CTC is that it is asymmetrically applied, unlike its use with dual optical images. Given that optical data possess more comprehensive spatial information, the calculation of $\mathcal{L}_{tri}$ is centric to the optical data, i.e.:

\begin{equation} \label{eq.het_triplet_loss}
    \mathcal{L}_{tri} = max[cos(\mathbf{y}_{rgb}, \mathbf{y}_{sar})-cos(\mathbf{y}_{rgb}, \mathbf{\Bar{y}}_{rgb})+m, 0]
\end{equation}
where $\mathbf{y}_{rgb}$, $\mathbf{y}_{sar}$ and $\mathbf{\Bar{y}}_{rgb}$ are the semantic features encoded from $I_{rgb}$, $I_{sar}$ and $\phi(I_{rgb})$, respectively. The augmentation operations outlined in $\phi$ correspond to those detailed in Sec.\ref{sc3.CL}.

The presence of a domain gap poses more challenges to learning temporal consistency. Consequently, CSC learning is conducted directly on the original image sets $[\mathbf{y}_{rgb}^i, \mathbf{y}_{rgb}^j, ..., \mathbf{y}_{rgb}^k]$ and $[\mathbf{y}_{sar}^i, \mathbf{y}_{sar}^j, ..., \mathbf{y}_{sar}^k]$, rather than using their augmented counterparts. The loss function is computed as:
\begin{equation} \label{eq.het_info_loss}
\begin{aligned}
    \mathcal{L}_{info} = -\frac{1}{N} \sum_{u=1}^{N} \log \left[ \frac{\exp \left( \mathbf{y}_{rgb}^u \odot \mathbf{y}_{sar}^u \right)}{\sum_{v=1}^{N} \exp \left( \mathbf{y}_{rgb}^u \odot \mathbf{y}_{sar}^v \right)} \right]\\
    -\frac{1}{N} \sum_{u=1}^{N} \log \left[ \frac{\exp \left( \mathbf{y}_{sar}^u \odot \mathbf{y}_{rgb}^u \right)}{\sum_{v=1}^{N} \exp \left( \mathbf{y}_{sar}^u \odot \mathbf{y}_{rgb}^v \right)} \right]
\end{aligned}
\end{equation}

The comprehensive training objective is consistent with that in Eq.\ref{eq.losses}. The simultaneous application of CTC and CSC learning drives $f_\theta$ and $f_\zeta$ to align semantic representations across different domains, thus mapping domain-invariant changes. In the inference stage, since typical VFM decoders are unable to segment SAR objects, the refining algorithm is omitted. The change probability maps $\textbf{y}_c$ is directly binarized to map the multimodal changes.

\section{Experiments}\label{sc4}

\subsection{Datasets and Evaluation Metrics}
To test the effectiveness of the proposed method, experiments are conducted on three RS datasets with varied data distributions and semantic annotations. The experimental datasets include CLCD \cite{liu2022cnntransformer}, SECOND (binary) \cite{yang2022asymmetric} and Levir \cite{chen2020spatialtemporal}. In addition, MMCD experiments are conducted on the Wuhan dataset \cite{zhang2022domain}. Table \ref{Table.Datasets} presents an overview of each dataset. Since the experimental methods are unsupervised, we do not use any labels within the training set. Notably, the Levir dataset focuses solely on building changes, while CLCD, SECOND and Wuhan encompass various change types. Its larger spatial size and sparse change instances make it more challenging in the context of UCD. 

\begin{table}[ht]
    \centering
    \caption{Summary of the main characteristics of the experimental datasets.} \label{Table.Datasets}
    \resizebox{1\linewidth}{!}{%
    \begin{tabular}{c|c|c|c|c|c}
    \toprule
        \multirow{2}*{Datasets} & \multirow{2}*{Platform} & \multirow{2}*{Resolution} & Image & Dataset & Change \\
        & & & Size & Size & Type \\
        \hline
        CLCD & satellite & 0.5-2m & 512×512 & 600 & agricultural \\
        SECOND & aerial & 0.5-3m & 512×512 & 4,662 & land cover \\
        Levir & satellite & 0.5m & 1024×1024 & 637 & building \\
        \hline
        \multirow{2}*{Wuhan} & satellite & vis: 10m & \multirow{2}*{256×256} & \multirow{2}*{600} & \multirow{2}*{land cover} \\
        & (RGB\&SAR) & sar: 3m & & & \\
    \bottomrule
    \end{tabular}}
\end{table}

We adopt the most commonly used accuracy metrics in CD \cite{ding2024samcd, chen2023exchange} and binary segmentation tasks \cite{ding2021adversarial}, including Overall accuracy (\textit{OA}), Precision (\textit{Pre}), Recall (\textit{Rec}), and $F_1$ score. In these metrics, \textit{Pre} indicates the proportion of true positives among classified positives, while \textit{Rec} is the measure of identifying true positives. $F_1$ is the harmonic mean of \textit{Pre} and \textit{Rec}, therefore is more comprehensive in assessing the accuracy.

\begin{comment}
i) CLCD dataset. This is a data set for monitoring agricultural land change. It consists of 600 pairs of satellite images, with 320 pairs used for training, 120 pairs for validation, and 120 pairs for testing. The bi-temporal images were collected by the Gaofen-2 satellite in 2017 and 2019, with a spatial resolution of 0.5 to 2 meters. Each set of samples comprises two 512×512 pixels and a corresponding binary change label. The main types of annotated change objects include buildings, roads, lakes, and bare land.

ii) SECOND dataset. This is a semantic CD dataset that comprises 4,662 pairs of aerial images collected from multiple platforms and sensors. The image pairs are distributed across various cities such as Hangzhou, Chengdu, and Shanghai. Each image has 512 x 512 pixels, with a spatial resolution of 0.5 to 2 meters. 6 primary land cover changes are annotated, including bare land, trees, low vegetation, water, buildings, and playgrounds.     
\end{comment}

\begin{table}[t]
\centering
    \caption{Quantitative results of ablation study (tested on CLCD).}
    \resizebox{1\linewidth}{!}{%
        \begin{tabular}{l|c|cccc}
        \toprule
            \multirow{2}*{Methods} & \multirow{2}*{Backbone} & \multicolumn{4}{c}{Accuracy (\%)}\\
            \cline{3-6}
            & & $OA$ & $Pre$ & $Rec$ & $F_1$ \\
            \hline
            effi.SAM + CVA & effi.SAM (vit-t) & 61.24 & 13.90 & 81.01 & 23.73 \\
            effi.SAM + SC & effi.SAM (vit-t) & 92.23 & 45.76 & 24.07 & 31.55 \\
            \hline
            S2C (CSC only) & effi.SAM (vit-t) & 90.93 & 36.55 & 29.81 & 32.84 \\
            S2C (CTC only) & effi.SAM (vit-t) & 85.98 & 28.25 & 57.40 & 37.86 \\
            S2C (CSC + $\mathcal{L}_{spa}$) & effi.SAM (vit-t) & 89.71 & 33.35 & 38.33 & 35.67\\ 
            S2C (CTC + $\mathcal{L}_{spa}$) & effi.SAM (vit-t) & 91.06 & 39.52 & 38.03 & 38.76 \\
            S2C (CSC + CTC) & effi.SAM (vit-t) & 87.35 & 31.01 & 57.12 & 40.19 \\
            S2C & effi.SAM (vit-t) & 90.57 & 39.04 & 47.65 & 42.92 \\
            S2C & effi.SAM (vit-s) & 89.58 & 37.51 & \textbf{60.21} & 46.22  \\
            S2C + \textit{IoU refine} & effi.SAM (vit-s) & \textbf{91.47} & \textbf{43.85} & 52.28 & \textbf{47.69} \\
            \hline
            S2C (w/o. VFM) & ResNet18 & 87.75 & 26.80 & 37.34 & 31.21 \\
            S2C (w/o. VFM) & ResNet34 & 86.28 & 28.41 & 55.56 & 37.60 \\
            S2C & fastSAM & 86.65 & 24.71 & 38.81 & 17.78 \\
            S2C & SAM (vit-b) & 85.41 & 22.23 & 38.44 & 28.17  \\
            S2C & effi.SAM (vit-t) & 90.57 & 39.04 & 47.65 & 42.92  \\
            S2C & effi.SAM (vit-s) & 89.58 & 37.51 & 60.21 & 46.22  \\
            S2C & Dino-v2 (vit-b) & 93.95 & 59.04 & \textbf{61.24} & 60.12 \\
            S2C + \textit{IoU refine} & Dino-v2 (vit-b) & \textbf{94.46} & \textbf{63.82} & 59.12 & \textbf{61.38} \\
        \bottomrule
        \end{tabular} \label{Table.Ablation} }
\end{table}

\subsection{Implementation Details}\label{sc4.implement}
The training of S2C is performed using cropped images of $512 \times 512$ pixels over 20 epochs. The trained weights with the highest accuracy on the validation set are saved for subsequent evaluation on the test set. The training batch size depends on the backbone to fit in GPU memory, usually exceeding 12 in our implementation. The learning rate $lr$ is initially set to 0.01, and is updated at each iteration as: $0.01*(1-iterations/total\_iterations)^{1.5}$. The optimization algorithm is the Stochastic Gradient Descent with Nesterov momentum. The weighting parameters in eq.\ref{eq.losses} are set to $\alpha=0.2, \beta=1$, while $\alpha$ can be adjusted across datasets to balance $\mathcal{L}_{tri}$ and $\mathcal{L}_{info}$ in training dynamics. An sensitivity analysis on these paramenters are provided in Sec.\ref{sc.ablation}. , The sparsity threshold $T$ in Eq.(\ref{eq.sparse_loss}) is set according to the change sparsity in different datasets: $T=0.2$ for the CLCD and Wuhan datasets, $T=0.2$ for the Levir dataset with very sparse changes, and $T=0.4$ for the SECOND dataset with more change instances.

Apart from \textit{strong augmentations} as detailed in Sec.\ref{sc3.CL}, random flipping operations are also introduced as \textit{weak augmentations}. These operations are executed on each of the two temporal images to enhance sample diversity, rather than induce spatial or spectral perturbations.

%For more details, readers are encouraged to visit our codes released at: \url{https://github.com/ggsDing/SCanNet}.

\subsection{Ablation Study}\label{sc.ablation}

\textbf{Quantitative Evaluation.} An ablation study is conducted through cumulative integration of the proposed methodologies, including the CTC, CSC, grid sparse loss ($\mathcal{L}_{spa}$), and IoU matching and refinement (\textit{IoU refine}). Given that the proposed method employs both VFM and CL, an intuitive strategy is to combine these two techniques as a baseline approach. However, VFM alone is not capable of UCD, and there is no existing literature approach (to the best of our knowledge) that integrates CL with VFM for CD. Therefore, we implement these two approaches as the baseline: i) applying CVA and clustering on the VFM-encoded semantic features for CD following the practice in \cite{zheng2024segment}, and ii) conducting CS-based CL using the VFM features. Due to computational constraints, we employ \textit{efficient-SAM}(vit-t) \cite{xiong2024efficientsam} as a frozen encoder ($\theta$) in the initial baseline, which is an efficient variant of SAM \cite{Kirillov2023Segment}.

The quantitative results are presented in Table \ref{Table.Ablation}. As indicated by the results of \textit{eff.SAM + CVA}, direct change analysis on VFM features leads to suboptimal accuracy. By contrast, the foundational method that integrates VFM with SC-based CL demonstrates significant accuracy. The proposed CTC and CSC further surpass the conventional SC-based CL paradigm. It is worth noting that the CTC alone outperforms both SC and CSC by a large margin, establishing it as the leading single CL paradigm for CD. Meanwhile, the proposed grid sparsity regularization also exhibits notable effectiveness. Its addition to each CL paradigm results in an average enhancement of $~2\%$ in $F_1$. Adding CSC improves the robustness of CTC against temporal noise, leading to an increase of over $3\%$ in $F_1$. The refining algorithm substantially enhances the precision of the results, providing an increase of $1.47\%$ in $F_1$ and approximately $2\%$ in $OA$.

\begin{figure}[t]
\centering
    \includegraphics[width=0.8\linewidth]{ablation/param_acc.png}
    \caption{$F_1$ (\%) obtained by S2C with different weighting parameters.}
\label{fig.param_acc}
\end{figure}

\textbf{S2C with different backbones.} While S2C is introduced as a methodology that integrates CL and VFM, the core technique employed is a UCD framework, which is adaptable to other types of DNNs. Table \ref{Table.Ablation} also presents an evaluation of S2C utilizing various different backbones, including a vanilla ResNet \cite{he2016resnet} and several other VFMs. Surprisingly, the implementation of S2C with a simple ResNet34 backbone still yields considerably high accuracy. This further confirms its efficacy as a general framework for UCD.

Compared to using conventional DNNs, employing VFMs as backbone greatly improves the $Pre$ of S2C. This can be attributed to the rich semantic contexts inherent to VFMs, which significantly facilitate the discrimination of semantic changes. The implemented VFMs include SAM \cite{Kirillov2023Segment}, fastSAM \cite{zhao2023fast}, efficient SAM \cite{xiong2024efficientsam} and Dino-v2\cite{oquab2024dinov2}. Among SAM and its variants, efficient SAM obtains the highest accuracy. Its parameter size is considerably reduced compared to the original SAM, thereby improving its convergence in the context of UCD. Employing Dino-v2 results in the highest accuracy, with an advantage of approximately $12\%$ compared to the other backbones. Therefore, Dino-v2 is selected as the backbone in the subsequent experiments.

\begin{figure*}[t]
\centering
    \setlength{\tabcolsep}{1pt}
    \begin{tabular}{>{\centering\arraybackslash}m{0.4cm}>{\centering\arraybackslash}m{1.8cm}>{\centering\arraybackslash}m{1.8cm}>{\centering\arraybackslash}m{1.8cm}>{\centering\arraybackslash}m{1.8cm}>{\centering\arraybackslash}m{1.8cm}>{\centering\arraybackslash}m{1.8cm}>{\centering\arraybackslash}m{1.8cm}>{\centering\arraybackslash}m{1.8cm}>{\centering\arraybackslash}m{1.8cm}}
        (a)&
        \includegraphics[width=1.8cm]{ablation/00576_imgA.png} &
        \includegraphics[width=1.8cm]{ablation/00576_imgB.png} &
        \includegraphics[width=1.8cm]{ablation/00576_label.png} &
        \includegraphics[width=1.8cm]{ablation/00576_SC.png} &
        \includegraphics[width=1.8cm]{ablation/00576_infoNCE_only.png} &
        \includegraphics[width=1.8cm]{ablation/00576_triplet_only.png} &
        \includegraphics[width=1.8cm]{ablation/00576_noSparse.png}&
        \includegraphics[width=1.8cm]{ablation/00576_S2C.png}&
        \includegraphics[width=1.8cm]{ablation/00576_S2C_refined.png}\\
        (b)&
        \includegraphics[width=1.8cm]{ablation/00569_imgA.png} &
        \includegraphics[width=1.8cm]{ablation/00569_imgB.png} &
        \includegraphics[width=1.8cm]{ablation/00569_label.png} &
        \includegraphics[width=1.8cm]{ablation/00569_SC.png} &
        \includegraphics[width=1.8cm]{ablation/00569_infoNCEonly.png} &
        \includegraphics[width=1.8cm]{ablation/00569_triplet_only.png} &
        \includegraphics[width=1.8cm]{ablation/00569_noSparse.png}&
        \includegraphics[width=1.8cm]{ablation/00569_S2C.png}&
        \includegraphics[width=1.8cm]{ablation/00569_S2C_refined.png}\\
        (c)&
        \includegraphics[width=1.8cm]{ablation/09555_imgA.png} &
        \includegraphics[width=1.8cm]{ablation/09555_imgB.png} &
        \includegraphics[width=1.8cm]{ablation/09555_GT.png} &
        \includegraphics[width=1.8cm]{ablation/09555_SC.png} &
        \includegraphics[width=1.8cm]{ablation/09555_infoNCE_only.png} &
        \includegraphics[width=1.8cm]{ablation/09555_triplet_only.png} &
        \includegraphics[width=1.8cm]{ablation/09555_no_sparse.png}&
        \includegraphics[width=1.8cm]{ablation/09555_S2C.png}&
        \includegraphics[width=1.8cm]{ablation/09555_refined.png}\\
        (d)&
        \includegraphics[width=1.8cm]{ablation/Levir_74_A.png} &
        \includegraphics[width=1.8cm]{ablation/Levir_74_B.png} &
        \includegraphics[width=1.8cm]{ablation/levir_74_label.png} &
        \includegraphics[width=1.8cm]{ablation/Levir_74_CLSC.png} &
        \includegraphics[width=1.8cm]{ablation/Levir_74_infoNCE_only.png} &
        \includegraphics[width=1.8cm]{ablation/Levir_74_triplet_only.png} &
        \includegraphics[width=1.8cm]{ablation/Levir_74_no_sparse.png}&
        \includegraphics[width=1.8cm]{ablation/Levir_74_S2C.png}&
        \includegraphics[width=1.8cm]{ablation/Levir_74_S2C_refine.png}\\
        & $T_1$ image & $T_2$ image & GT & SC (baseline) & S2C {\small(CSC only)} & S2C {\small(CTC only)} & S2C {\small(CSC+CTC)} & S2C & S2C {\small(Refined)}\\
    \end{tabular}
    \caption{Example of results from different methods in the ablation study. CTC highlights differences while introducing much noise, CSC enhances change representations and reduces certain false alarms, while $\mathcal{L}_{spa}$ further suppresses insignificant changes.}
    \label{Fig.ablation_vis}
\end{figure*}

\begin{comment}
    \begin{table}[t]
\centering
    \caption{Results obtained with different weighting parameters.}
    \resizebox{1\linewidth}{!}{%
    \begin{tabular}{c | cccc | ccc }
    \toprule
    \multirow{2}*{} & \multicolumn{4}{c|}{$\beta=1$} & \multicolumn{3}{c}{$\alpha=0.2$}  \\
    \cline{2-8}
    & $\alpha=0.1$ & $\alpha=0.2$ & $\alpha=0.5$ & $\alpha=1.0$ & $\beta=1$ & $\beta=1$ & $\beta=3$ \\
    \hline
    $F_1 (\%)$ &   \\
    \bottomrule
    \end{tabular}}
    \label{Table.Params}
\end{table}
\end{comment}

\textbf{Parameter Analysis.} In Eq.\ref{eq.losses} two weighting parameters $\alpha$ and $\beta$ are introduced to balance the different training objectives. The optimal weighting parameters are determined by a sensitivity analysis. Considering that $\mathcal{L}_{info}$ poses considerable optimization challenges and commonly exhibits high values, $\alpha$ is assigned small values within the range of $[0, 1]$. In contrast, $\mathcal{L}_{spa}$ is straightforward to optimize and has relatively low values, thus $\beta$ is assigned with values over 1. Fig.\ref{fig.param_acc} reports the accuracy in $F_1$ obtained with different values of $\alpha$ and $\beta$. To mitigate random variability, the reported accuracy is the mean value across three trials. The results show a strong correlation between $\alpha$ and accuracy, whereas the effect of $\beta$ appears inconsistent. This finding aligns with the underlying learning mechanism within S2C, where $\alpha$ balances between $\mathcal{L}_{tri}$ and $\mathcal{L}_{info}$, affecting the focus on mapping differences or similarities. The allocation of $\alpha =0.2, \beta =2$ yields the highest observed accuracy.

\begin{figure}[t]
\centering
    \includegraphics[width=1\linewidth]{ablation/acc_curve.png}
    \caption{Average $F_1$ (\%) of S2C versus sample volume.}
\label{fig.acc_curve}
\end{figure}

\begin{table}[t]
    \centering
    \caption{Average $F_1$ (\%) of S2C obtained with varying amount of unlabeled samples.}
    \resizebox{1\linewidth}{!}{%
        \begin{tabular}{r|cccccc}
        \toprule
            \multirow{2}*{Dataset} & \multicolumn{6}{c}{Number of samples}\\
            \cline{2-7}
            & 5 & 10 & 50 & 100 & 200 & 300 \\
            \hline
            CLCD & 53.39\textcolor{gray}{$\pm$4.9} & 53.16\textcolor{gray}{$\pm$3.5} & 57.01\textcolor{gray}{$\pm$1.2} & 57.42\textcolor{gray}{$\pm$1.4} & 57.64\textcolor{gray}{$\pm$2.2} & 58.36\textcolor{gray}{$\pm$1.1} \\
            SECOND & 47.15\textcolor{gray}{$\pm$3.5} & 45.46\textcolor{gray}{$\pm$0.9} & 46.94\textcolor{gray}{$\pm$1.0} & 47.15\textcolor{gray}{$\pm$0.4} & 48.62\textcolor{gray}{$\pm$1.2} & 48.49\textcolor{gray}{$\pm$1.5} \\
            Levir & 35.98\textcolor{gray}{$\pm$3.1} & 35.74\textcolor{gray}{$\pm$0.7} & 38.79\textcolor{gray}{$\pm$0.7} & 40.41\textcolor{gray}{$\pm$2.1} & 40.63\textcolor{gray}{$\pm$0.1} & 41.17\textcolor{gray}{$\pm$1.5} \\
        \bottomrule
        \end{tabular}
        }\label{Table.FewSample}
\end{table}


\textbf{Qualitative Assessment.} Fig.\ref{Fig.ablation_vis} presents some examples of the CD results obtained using the different techniques. These results are selected in the 3 datasets covering different scenes, including (a) cropland, (b) countryside (c) factories, and (d) residential blocks. One can observe that using either the CSC or CTC alone leads to much noise, while their collaborative employment greatly reduces the false alarms. The grid sparsity generalization further removes much insignificant change representation and leads to more smoothed CD results. After the post-processing of \textit{IoU refinement}, binary CD results matching the object boundaries are produced.

\textbf{Training S2C with Few Samples.} An additional advantage of incorporating consistency regularization in contrastive learning is the enhancement of generalization ~\cite{yang2023revisiting, bandara2022revisiting}. Throughout the experimental process, we observe that the accuracy of the S2C framework is not sensitive to the number of training samples (unlabeled). Consequently, additional experiments are conducted to investigate the requisite quantity of unlabeled image samples for effective training of the S2C framework.

We train the S2C model using varying amounts of unlabeled data, ranging from 300 to as few as 5 and subsequently evaluate the resulting accuracy on the entire test set. To increase the diversity of samples, additional random cropping and enlarging augmentations are added in the preprocessing operations. This is added into the \textit{weak augmentation} process as elaborated in Sec.\ref{sc4.implement}. This operation is crucial for deploying the S2C with very few samples: when there are insufficient samples to enable spatial comparisons within the CSC paradigm, cropped sub-regions can be alternatives. To mitigate variations due to sample selection, the experiments are conducted over three trials, each randomly selecting the specified sample volume.

The average $F_1 (\%)$ results of the experimental trials are reported in Table \ref{Table.FewSample}. Compared with the results obtained using 300 training samples, utilizing only 100 samples yields an almost negligible average $F_1$ decrease of 1\% across the three datasets. Employing a minimal set of merely 5 samples results in an average $F_1$ decrease of about 3.8\%. This low requirement on training samples suggests that the S2C framework can be easily deployed in real-world applications. To intuitively illustrate the results, a graph of accuracy versus sample volume is presented in Fig.\ref{fig.acc_curve}. By analyzing the figure, one can infer that on the considered datasets, employing a minimum of 5 samples is sufficient for the effective training of the S2C, while utilizing 100 samples is adequate for attaining near-optimal accuracy.


\subsection{Comparative Experiments}

\begin{table*}[t]
    \centering
    \caption{Quantitative accuracy (\%) evaluation of the proposed S2C and SOTA UCD methods on various benchmark datasets.}
    \resizebox{1\linewidth}{!}{%
        \begin{tabular}{r|c|cccc|cccc|cccc}
        \toprule
            \multirow{2}*{Methods} & \multirow{2}*{Publication} & \multicolumn{4}{c|}{CLCD} & \multicolumn{4}{c|}{SECOND} & \multicolumn{4}{c}{Levir} \\
            \cline{3-14}
            & & $OA$ & $Pre$ & $Rec$ & $F_1$ & $OA$ & $Pre$ & $Rec$ & $F_1$ & $OA$ & $Pre$ & $Rec$ & $F_1$ \\
            \hline
            \color{gray} SAM-CD &\color{gray} \textit{TGRS 2024} \cite{ding2024samcd} & \color{gray}96.26 & \color{gray}73.01 & \color{gray}78.84 & \color{gray}75.81 & \color{gray}88.56 & \color{gray}73.32 & \color{gray}66.00 & \color{gray}69.47 & \color{gray}99.17 & \color{gray}92.62 & \color{gray}91.04 & \color{gray}91.82\\
            \hline
            CVA & \textit{TGRS 2000} \cite{Bruzzone2000diff} & 71.01 & 8.49 & 29.62 & 13.20 & 59.17 & 20.55 & 37.34 & 26.51 & 66.50 & 5.80 & 36.59 & 10.02 \\
            %IRMAD \cite{Nielsen2007IRMAD} & 74.64 & 8.85 & 25.89 & 13.19 & 60.80 & 32.43 & 19.82 & 24.60 & 70.13 & 6.02 & 33.27 & 10.19\\
            ISFA & \textit{TGRS 2014} \cite{Wu2014Slow} & 74.37 & 8.60 & 25.39 & 12.85 & 60.23 & 20.13 & 34.24 & 25.35 & 69.32 & 6.03 & 34.45 & 10.27\\
            DCVA & \textit{TGRS 2019} \cite{saha2019unsupervised} & 53.91 & 11.35 & \textbf{76.26} & 19.76 & 55.51 & 25.63 & \underline{66.07} & 36.93 & 48.28 & 7.20 & 76.94 & 13.16\\
            DSFA & \textit{TGRS 2019} \cite{Du2019Unsupervised} & 52.09 & 8.53 & 55.94 & 14.80 & 48.10 & 19.06 & 50.28 & 27.65 & 60.29 & 5.99 & 46.22 & 10.60\\
            KPCA & \textit{TCYB 2022} \cite{wu2021unsupervised} & 53.47 & 13.84 & 44.52 & 21.11 & 54.69 & 20.44 &  44.87 & 28.09 & 54.97 & 5.61 & 49.52 & 10.08\\
            CDRL & \textit{CVPR 2022} \cite{noh2022unsupervised} & 64.74 & 7.75 & 34.27 & 12.64 & 62.65 & 37.78 & 22.90 & 28.51 & 62.59 & 5.32 & 37.74 & 9.32\\
            SiROC & \textit{TGRS 2022} \cite{Kondmann2022Spatial} & 82.38 & 18.93 & 41.65 & 26.03 & 69.80 & 27.92 & 33.59 & 30.49 & 64.82 & 7.38 & 51.14 & 12.90 \\
            I3PE & \textit{ISPRS 2023} \cite{chen2023exchange} & 91.12 & 32.42 & 17.79 & 22.97 & 74.09 & 34.53 & 35.03 & 34.78 & \underline{91.03} & 17.39 & 20.31 & 18.74\\
            FCD-GAN & \textit{TPAMI 2023} \cite{wu2023fully} & 83.93 & 22.06 & 45.79 & 29.77 & 68.36 & 29.47 & 43.41 & 35.11 & 83.42 & 8.87 & 24.29 & 12.99 \\
            AnyChange & \textit{NeurIPS 2024} \cite{zheng2024segment} & - & - & - & - & - & 30.5 & \textbf{83.2} & 44.6 & - & 13.3 & \textbf{85.0} & 23.0 \\
            \hline    
            \multicolumn{2}{r|}{S2C-Dinov2 (proposed)} & \underline{93.95} & \underline{59.04} & \underline{61.24} & \underline{60.12} & \textbf{84.55} & \textbf{67.15} & 42.41 & \underline{51.99} & 89.28 & \underline{29.42} & \underline{78.88} & \underline{42.86} \\ 
            \multicolumn{2}{r|}{S2C + \textit{IoU Refine} (proposed)} & \textbf{94.46} & \textbf{63.82} & 59.12 & \textbf{61.38} & \underline{84.45} & \underline{64.91} & 46.02 & \textbf{53.86} & \textbf{92.84} & \textbf{34.85} & 70.69 & \textbf{46.69} \\ 
        \bottomrule
        \end{tabular}
        }\label{Table.CompareSOTA}
\end{table*}

\begin{figure*}[t]
\centering
    \setlength{\tabcolsep}{1pt}
    \begin{tabular}{>{\centering\arraybackslash}m{0.4cm}>{\centering\arraybackslash}m{1.5cm}>{\centering\arraybackslash}m{1.5cm}>{\centering\arraybackslash}m{1.5cm}>{\centering\arraybackslash}m{1.5cm}>{\centering\arraybackslash}m{1.5cm}>{\centering\arraybackslash}m{1.5cm}>{\centering\arraybackslash}m{1.5cm}>{\centering\arraybackslash}m{1.5cm}>{\centering\arraybackslash}m{1.5cm}>{\centering\arraybackslash}m{1.5cm}>{\centering\arraybackslash}m{1.5cm}}
        & & & & \multicolumn{7}{c}{\includegraphics[width=10cm]{Vis_SOTA/ColorBar.png}}\\
        (a)&
        \includegraphics[width=1.5cm]{Vis_SOTA/00498_im1.png} &
        \includegraphics[width=1.5cm]{Vis_SOTA/00498_im2.png} &
        \includegraphics[width=1.5cm]{Vis_SOTA/00498_GT.png} &
        \includegraphics[width=1.5cm]{Vis_SOTA/00498_DCVA.png} &
        \includegraphics[width=1.5cm]{Vis_SOTA/00498_DSFA.png} &
        \includegraphics[width=1.5cm]{Vis_SOTA/00498_KPCA.png} &
        \includegraphics[width=1.5cm]{Vis_SOTA/00498_CDRL.png} &
        \includegraphics[width=1.5cm]{Vis_SOTA/00498_SiROC.png}&
        \includegraphics[width=1.5cm]{Vis_SOTA/00498_FCDGAN.png}&
        \includegraphics[width=1.5cm]{Vis_SOTA/00498_I3PE.png}&
        \includegraphics[width=1.5cm]{Vis_SOTA/00498_S2C_refined.png}\\
        (b)&
        \includegraphics[width=1.5cm]{Vis_SOTA/00572_im1.png} &
        \includegraphics[width=1.5cm]{Vis_SOTA/00572_im2.png} &
        \includegraphics[width=1.5cm]{Vis_SOTA/00572_GT.png} &
        \includegraphics[width=1.5cm]{Vis_SOTA/00572_DCVA.png} &
        \includegraphics[width=1.5cm]{Vis_SOTA/00572_DSFA.png} &
        \includegraphics[width=1.5cm]{Vis_SOTA/00572_KPCA.png} &
        \includegraphics[width=1.5cm]{Vis_SOTA/00572_CDRL.png} &
        \includegraphics[width=1.5cm]{Vis_SOTA/00572_FCDGAN.png}&
        \includegraphics[width=1.5cm]{Vis_SOTA/00572_SiROC.png}&
        \includegraphics[width=1.5cm]{Vis_SOTA/00572_I3PE.png}&
        \includegraphics[width=1.5cm]{Vis_SOTA/00572_S2C_refined.png}\\
        (c)&
        \includegraphics[width=1.5cm]{Vis_SOTA/00581_im1.png} &
        \includegraphics[width=1.5cm]{Vis_SOTA/00581_im2.png} &
        \includegraphics[width=1.5cm]{Vis_SOTA/00581_GT.png} &
        \includegraphics[width=1.5cm]{Vis_SOTA/00581_DCVA.png} &
        \includegraphics[width=1.5cm]{Vis_SOTA/00581_DSFA.png} &
        \includegraphics[width=1.5cm]{Vis_SOTA/00581_KPCA.png} &
        \includegraphics[width=1.5cm]{Vis_SOTA/00581_CDRL.png} &
        \includegraphics[width=1.5cm]{Vis_SOTA/00581_SiROC.png}&
        \includegraphics[width=1.5cm]{Vis_SOTA/00581_FCDGAN.png}&
        \includegraphics[width=1.5cm]{Vis_SOTA/00581_I3PE.png}&
        \includegraphics[width=1.5cm]{Vis_SOTA/00581_S2C_refined.png}\\
        (d)&
        \includegraphics[width=1.5cm]{Vis_SOTA/04036_im1.png} &
        \includegraphics[width=1.5cm]{Vis_SOTA/04036_im2.png} &
        \includegraphics[width=1.5cm]{Vis_SOTA/04036_GT.png} &
        \includegraphics[width=1.5cm]{Vis_SOTA/04036_DCVA.png} &
        \includegraphics[width=1.5cm]{Vis_SOTA/04036_DSFA.png} &
        \includegraphics[width=1.5cm]{Vis_SOTA/04036_KPCA.png} &
        \includegraphics[width=1.5cm]{Vis_SOTA/04036_CDRL.png} &
        \includegraphics[width=1.5cm]{Vis_SOTA/04036_SiROC.png}&
        \includegraphics[width=1.5cm]{Vis_SOTA/04036_FCDGAN.png}&
        \includegraphics[width=1.5cm]{Vis_SOTA/04036_I3PE.png}&
        \includegraphics[width=1.5cm]{Vis_SOTA/04036_S2C_refined.png}\\
        (e)&
        \includegraphics[width=1.5cm]{Vis_SOTA/02803_im1.png} &
        \includegraphics[width=1.5cm]{Vis_SOTA/02803_im2.png} &
        \includegraphics[width=1.5cm]{Vis_SOTA/02803_GT.png} &
        \includegraphics[width=1.5cm]{Vis_SOTA/02803_DCVA.png} &
        \includegraphics[width=1.5cm]{Vis_SOTA/02803_DSFA.png} &
        \includegraphics[width=1.5cm]{Vis_SOTA/02803_KPCA.png} &
        \includegraphics[width=1.5cm]{Vis_SOTA/02803_CDRL.png} &
        \includegraphics[width=1.5cm]{Vis_SOTA/02803_SiROC.png}&
        \includegraphics[width=1.5cm]{Vis_SOTA/02803_FCDGAN.png}&
        \includegraphics[width=1.5cm]{Vis_SOTA/02803_I3PE.png}&
        \includegraphics[width=1.5cm]{Vis_SOTA/02803_S2C_refined.png}\\
        (f)&
        \includegraphics[width=1.5cm]{Vis_SOTA/05829_im1.png} &
        \includegraphics[width=1.5cm]{Vis_SOTA/05829_im2.png} &
        \includegraphics[width=1.5cm]{Vis_SOTA/05829_GT.png} &
        \includegraphics[width=1.5cm]{Vis_SOTA/05829_DCVA.png} &
        \includegraphics[width=1.5cm]{Vis_SOTA/05829_DSFA.png} &
        \includegraphics[width=1.5cm]{Vis_SOTA/05829_KPCA.png} &
        \includegraphics[width=1.5cm]{Vis_SOTA/05829_CDRL.png} &
        \includegraphics[width=1.5cm]{Vis_SOTA/05829_SiROC.png}&
        \includegraphics[width=1.5cm]{Vis_SOTA/05829_FCDGAN.png}&
        \includegraphics[width=1.5cm]{Vis_SOTA/05829_I3PE.png}&
        \includegraphics[width=1.5cm]{Vis_SOTA/05829_S2C_refined.png}\\
        (g)&
        \includegraphics[width=1.5cm]{Vis_SOTA/test_14_im1.png} &
        \includegraphics[width=1.5cm]{Vis_SOTA/test_14_im2.png} &
        \includegraphics[width=1.5cm]{Vis_SOTA/test_14_GT.png} &
        \includegraphics[width=1.5cm]{Vis_SOTA/test_14_DCVA.png} &
        \includegraphics[width=1.5cm]{Vis_SOTA/test_14_DSFA.png} &
        \includegraphics[width=1.5cm]{Vis_SOTA/test_14_KPCA.png} &
        \includegraphics[width=1.5cm]{Vis_SOTA/test_14_CDRL.png} &
        \includegraphics[width=1.5cm]{Vis_SOTA/test_14_SiROC.png}&
        \includegraphics[width=1.5cm]{Vis_SOTA/test_14_FCDGAN.png}&
        \includegraphics[width=1.5cm]{Vis_SOTA/test_14_I3PE.png}&
        \includegraphics[width=1.5cm]{Vis_SOTA/test_14_S2C_refined.png}\\
        (h)&
        \includegraphics[width=1.5cm]{Vis_SOTA/test_84_im1.png} &
        \includegraphics[width=1.5cm]{Vis_SOTA/test_84_im2.png} &
        \includegraphics[width=1.5cm]{Vis_SOTA/test_84_GT.png} &
        \includegraphics[width=1.5cm]{Vis_SOTA/test_84_DCVA.png} &
        \includegraphics[width=1.5cm]{Vis_SOTA/test_84_DSFA.png} &
        \includegraphics[width=1.5cm]{Vis_SOTA/test_84_KPCA.png} &
        \includegraphics[width=1.5cm]{Vis_SOTA/test_84_CDRL.png} &
        \includegraphics[width=1.5cm]{Vis_SOTA/test_84_SiROC.png}&
        \includegraphics[width=1.5cm]{Vis_SOTA/test_84_FCDGAN.png}&
        \includegraphics[width=1.5cm]{Vis_SOTA/test_84_I3PE.png}&
        \includegraphics[width=1.5cm]{Vis_SOTA/test_84_S2C_refined.png}\\
        (i)&
        \includegraphics[width=1.5cm]{Vis_SOTA/test_34_im1.png} &
        \includegraphics[width=1.5cm]{Vis_SOTA/test_34_im2.png} &
        \includegraphics[width=1.5cm]{Vis_SOTA/test_34_GT.png} &
        \includegraphics[width=1.5cm]{Vis_SOTA/test_34_DCVA.png} &
        \includegraphics[width=1.5cm]{Vis_SOTA/test_34_DSFA.png} &
        \includegraphics[width=1.5cm]{Vis_SOTA/test_34_KPCA.png} &
        \includegraphics[width=1.5cm]{Vis_SOTA/test_34_CDRL.png} &
        \includegraphics[width=1.5cm]{Vis_SOTA/test_34_SiROC.png}&
        \includegraphics[width=1.5cm]{Vis_SOTA/test_34_FCDGAN.png}&
        \includegraphics[width=1.5cm]{Vis_SOTA/test_34_I3PE.png}&
        \includegraphics[width=1.5cm]{Vis_SOTA/test_34_S2C_refined.png}\\
        & $T_1$ image & $T_2$ image & GT & DCVA & DSFA & KPCA & CDRL & SiROC & FCD-GAN & I3PE & S2C {\small(Refined)}\\
    \end{tabular}
    \caption{Qualitative comparison between the proposed S2C framework and SOTA methods for UCD. The samples are selected from (a)-(c) CLCD, (d)-(f) SECOND, and (g)-(i) Levir datasets.}
    \label{Fig.vis_SOTA}
\end{figure*}

We conduct comparative experiments with various SOTA methods for UCD in RS. The compared methods include several non-parameter methods based on difference analysis, including the CVA-based methods \cite{Bruzzone2000diff}, ISFA \cite{Wu2014Slow}, DCVA \cite{saha2019unsupervised}, DSFA \cite{Du2019Unsupervised}, KPCA \cite{wu2021unsupervised} and SiROC \cite{Kondmann2022Spatial}. In addition, we also compare generative methods CDRL \cite{noh2022unsupervised} and FCD-GAN\cite{wu2023fully}, an augmentation-based method I3PE \cite{chen2023exchange} and a very recent VFM-based method AnyChange.\cite{zheng2024segment}. To facilitate a comprehensive assessment of accuracy, SAM-CD\cite{ding2024samcd} is also included in the comparison, representing the SOTA accuracy of supervised CD.

Table.\ref{Table.CompareSOTA} presents the quantitative results of this comparative study. Among the difference analysis-based methods, DCVA \cite{saha2019unsupervised} exhibits a notable advantage in $Rec$. It obtains the highest $Rec$ on CLCD and the second one on SECOND and Levir datasets. Among the methods presented in recent literature, I3PE achieves high $Pre$ across the three datasets. FCD-GAN \cite{wu2023fully} obtains superior $F_1$ on the CLCD dataset, and also achieves high $F_1$ on the SECOND dataset. AnyChange \cite{zheng2024segment}, a training-free approach leveraging VFM for CD, demonstrates satisfactory accuracy by achieving the highest $Rec$ on two datasets. However, it obtains a relatively low $Pre$, suggesting a considerable percentage of false alarms present in the results.

The proposed S2C yields substantial and consistent improvements in accuracy compared to the SOTA methods. The coarse prediction of S2C results in a sharp improvement of more than $30\%$ on the CLCD dataset, and approaching $20\%$ on the Levir dataset. Subsequent post-processing utilizing the IoU refinement further enhances its balance between $Pre$ and $Rec$. Notably, this refinement algorithm brings the greatest enhancement on the Levir dataset, where the coarse prediction is relatively blurred and has low $Pre$. The final results demonstrate an improvement of nearly $4\%$ in $F_1$ on the Levir dataset, which is recognized as a challenging benchmark for UCD algorithms given its building-focused annotation and the small number of changes. The improvements of the proposed S2C over the SOTA methods, after refinement, are quantified as $31\%$, $9\%$, and $23\%$ in $F_1$ for the respective datasets. The achieved level of accuracy significantly reduces the gap with fully-supervised methods, with observed reductions of $14\%$ and $16\%$ in $F_1$ on the CLCD and SECOND datasets, respectively. However, the Levir dataset still exhibits a large gap in $F_1$ between supervised and unsupervised methods, owing to its sparsity of change instances and building-focused annotations.

Fig.\ref{Fig.vis_SOTA} presents a qualitative analysis of the state-of-the-art (SOTA) methods. One can observe that most of the literature methods generate numerous false alarms due to the impact of temporal noise, such as spectral variations in Fig.\ref{Fig.vis_SOTA}(b)(c)(e)(g)(h), spatial misalignment in Fig.\ref{Fig.vis_SOTA}(d)(f), and radiometric insignificant changes in Fig.\ref{Fig.vis_SOTA}(a)(i). The results of FCD-GAN and I3PE have fewer false alarms but exhibit substantial limitations in comprehensively segmenting the major changes. In contrast, the proposed S2C demonstrates a marked reduction in errors. Its results effectively cover the majority of significant changes, although it still contains few false alarms.

\subsection{Unsupervised MMCD with S2C}

\begin{table}[t]
\centering
    \caption{Accuracy obtained by the proposed S2C with different encoders (Wuhan dataset).}
    \resizebox{1\linewidth}{!}{%
        \begin{tabular}{c|cc|c}
        \toprule
            Methods & $RGB$ Backbone & $SAR$ Backbone & $F_1$ (\%)\\
            \hline
            effi.SAM+SC & effi.SAM (vit-t) & effi.SAM (vit-t) & 9.76 \\
            effi.SAM+SC & Dino-v2 (vit-b) & Dino-v2 (vit-b) & 15.91 \\
            \hline
            S2C & effi.SAM (vit-t) & effi.SAM (vit-t) & 23.23 \\
            (dual VFMs) & Dino-v2 (vit-b) & Dino-v2 (vit-b) & 27.45 \\
            \hline
            \multirow{4}*{} & effi.SAM (vit-t) & ResNet18 & 32.04 \\
            S2C & effi.SAM (vit-s) & ResNet18 & 34.75 \\
            (single VFM)  & Dino-v2 (vit-b) & ResNet18 & 35.86 \\
             & fastSAM & ResNet34 & 38.26 \\
            \hline
            S2C & ResNet34 & ResNet34 & 41.28 \\
            (w/o. VFMs) & ResNet18 & ResNet18 & \textbf{41.96} \\
        \bottomrule
        \end{tabular} \label{Table.Het_Ablation} }
\end{table}

\begin{table}[t]
    \centering
    \caption{Quantitative evaluation of accuracy (\%) provided by SOTA unsupervised MMCD methods (Wuhan dataset).}
    \resizebox{1\linewidth}{!}{%
        \begin{tabular}{c|c|c|cccc}
        \toprule
            \multicolumn{2}{c|}{\multirow{2}*{Methods}} & \multirow{2}*{Reference} & \multicolumn{4}{c}{Accuracy}  \\
            \cline{4-7}
            \multicolumn{2}{c|}{} & &  $OA$ & $Pre$ & $Rec$ & $F_1$ \\  
            \hline
            \multirow{9}*{\rotatebox{90}{Homogeneous UCD}} & CVA & \textit{TGRS 2000} \cite{Bruzzone2000diff} & 59.81 & 13.91 & 36.12 & 20.09 \\
            %IRMAD \cite{Nielsen2007IRMAD} & 64.00 & 14.40 & 31.85 & 19.83\\
            & ISFA & \textit{TGRS 2014} \cite{Wu2014Slow} & 62.93 & 13.97 & 32.01 & 19.46\\
            & DCVA & \textit{TGRS 2019} \cite{saha2019unsupervised} & 47.10 & 14.29 & \underline{55.66} & 22.74\\
            & DSFA & \textit{TGRS 2019} \cite{Du2019Unsupervised} & 51.27 & 14.16 & 49.08 & 21.98\\
            & KPCA & \textit{TCYB 2021} \cite{wu2021unsupervised} & 53.47 & 13.84 & 44.52 & 21.11\\
            & CDRL & \textit{CVPR 2022} \cite{noh2022unsupervised} & 60.12 & 13.45 & 34.07 & 19.29\\
            & SiROC & \textit{TGRS 2022} \cite{Kondmann2022Spatial} & 70.09 & 15.05 & 24.52 & 18.65 \\
            & I3PE & \textit{ISPRS 2023} \cite{chen2023exchange} & 70.45 & 16.06 & 26.34 & 19.95\\
            & FCD-GAN & \textit{TPAMI 2023} \cite{wu2023fully} & \textbf{85.33} & 15.06 & 1.05 & 1.97\\
            \hline
            \multirow{4}*{\rotatebox{90}{MMCD}} & SR-GCAE & \textit{TGRS 2022} \cite{Chen2022Unsupervised} & 18.54 & 13.94 & \textbf{90.64} & 24.16\\
            & AGSCC & \textit{TNNLS 2022} \cite{Sun2022AGSCC}  & 75.48 & 10.31 & 9.79 & 10.04 \\
            & CAAE & \textit{TNNLS 2024} \cite{Luppino2024CAAE} & 75.43 & 13.71 & 14.30 & 14.00\\
            & LPEM & \textit{TNNLS 2024} \cite{sun2024LPEM} & 68.19 & \underline{19.70} & 41.43 & \underline{26.70} \\
            \hline    
            \multicolumn{2}{c|}{ S2C (w/o. VFMs) }& proposed & \underline{83.85} & \textbf{42.18} & 41.74 & \textbf{41.96} \\ 
        \bottomrule
        \end{tabular}
        }\label{Table.Compare_Het}
\end{table}

In this section, we present the experimental performance of the S2C for unsupervised MMCD. First, we conduct an ablation study to examine the efficacy of different feature encoders applied to the visible and SAR images. The results are presented in Table\ref{Table.Het_Ablation}. Due to substantial heterogeneity of SAR images relative to their training domains, employing two VFMs as feature extractors results in low accuracy. Consequently, we proceed to utilize a basic ResNet as $f_\zeta$, i.e., feature extractor for the SAR branch. Under this experimental setting, fastSAM outperforms Dino-v2 and effi.SAM as a more effective feature extractor for the optical branch. This may due to the low spatial resolution of the optical images in this dataset, where CNN-based backbones can better retain the spatial details. We proceed to examine the application of S2C without using any VFM. Surprisingly, training simple ResNet18 backbones as $f_\theta$ and $f_\zeta$ leads to the highest accuracy. This indicates that VFMs are also less effective in visible images, as this dataset has lower spatial resolution compared to the optical datasets in Table \ref{Table.Datasets}. Theoretically, utilizing RS FMs with effective generalization to both SAR and optical data could potentially enhance the accuracy of S2C. However, this hypothesis remains unexplored due to the absence of such effective FMs, particularly in the case of this specific dataset. Nonetheless, these experimental results affirm the efficacy of the S2C framework for UCD, despite the absence of the VFMs.

We further compare the accuracy of the S2C (with dual ResNet18 encoders) with the SOTA methods in Table \ref{Table.Het_Ablation}. The compared methods include not only those homogeneous UCD methods in Table \ref{Table.CompareSOTA}, but also several SOTA unsupervised MMCD methods with available implementations. The compared MMCD methods include two methods based on graph analysis: AGSCC \cite{Sun2022AGSCC} and LPEM \cite{sun2024LPEM}, a graph convolution-based method: SR-GCAE \cite{Chen2022Unsupervised} and a generative transcoding method: CAAE \cite{Luppino2024CAAE}. 

The results indicate that homogeneous UCD methods, when applied to multimodal data, typically demonstrate reduced efficacy, with accuracy levels falling below 23\% in $F_1$. Most of the compared MMCD methods exhibit sensitivity to hyper-parameters, and fail to generalize effectively on the Wuhan heterogeneous dataset when with a substantial number of testing samples. Among these methods, the LPEM achieves the highest accuracy as indicated in $Pre$ and $F_1$. In contrast, the proposed S2C effectively identifies multimodal changes, surpassing literature methods by a margin exceeding 15\% in $F_1$. Fig.\ref{Fig.vis_SOTA_Het} present several samples of the MMCD results. One can observe that the SR-GCAE and LPEM produce considerable number of false alarms. These observations align with their evaluation metrics in Table \ref{Table.Het_Ablation} as indicated by high $Rec$ over $Pre$. In contrast, the CAAE identifies only limited areas of the changes. The proposed S2C method, although still exhibits ambiguity in delineating the object boundaries, effectively detects most of the multimodal changes.


\begin{figure}[t]
\centering
    \setlength{\tabcolsep}{1pt}
    \begin{tabular}{>{\centering\arraybackslash}m{1.2cm}>{\centering\arraybackslash}m{1.2cm}>{\centering\arraybackslash}m{1.2cm}>{\centering\arraybackslash}m{1.2cm}>{\centering\arraybackslash}m{1.2cm}>{\centering\arraybackslash}m{1.2cm}>{\centering\arraybackslash}m{1.2cm}}
        \multicolumn{7}{c}{\includegraphics[width=8cm]{Vis_SOTA/ColorBar.png}}\\
        \includegraphics[width=1.2cm]{Vis_SOTA/Het_1118_rgb.png} &
        \includegraphics[width=1.2cm]{Vis_SOTA/Het_1118_sar.png} &
        \includegraphics[width=1.2cm]{Vis_SOTA/Het_1118_GT.png} &
        \includegraphics[width=1.2cm]{Vis_SOTA/Het_1118_SRGCAE.png} &
        \includegraphics[width=1.2cm]{Vis_SOTA/Het_1118_CAAE.png} &
        \includegraphics[width=1.2cm]{Vis_SOTA/Het_1118_LPEM.png} &
        \includegraphics[width=1.2cm]{Vis_SOTA/Het_1118_S2C.png}\\
        \includegraphics[width=1.2cm]{Vis_SOTA/Het_1915_rgb.png} &
        \includegraphics[width=1.2cm]{Vis_SOTA/Het_1915_sar.png} &
        \includegraphics[width=1.2cm]{Vis_SOTA/Het_1915_GT.png} &
        \includegraphics[width=1.2cm]{Vis_SOTA/Het_1915_SRGCAE.png} &
        \includegraphics[width=1.2cm]{Vis_SOTA/Het_1915_CAAE.png} &
        \includegraphics[width=1.2cm]{Vis_SOTA/Het_1915_LPEM.png} &
        \includegraphics[width=1.2cm]{Vis_SOTA/Het_1915_S2C.png}\\
        \includegraphics[width=1.2cm]{Vis_SOTA/Het_2094_rgb.png} &
        \includegraphics[width=1.2cm]{Vis_SOTA/Het_2094_sar.png} &
        \includegraphics[width=1.2cm]{Vis_SOTA/Het_2094_GT.png} &
        \includegraphics[width=1.2cm]{Vis_SOTA/Het_2094_SRGCAE.png} &
        \includegraphics[width=1.2cm]{Vis_SOTA/Het_2094_CAAE.png} &
        \includegraphics[width=1.2cm]{Vis_SOTA/Het_2094_LPEM.png} &
        \includegraphics[width=1.2cm]{Vis_SOTA/Het_2094_S2C.png}\\
        \includegraphics[width=1.2cm]{Vis_SOTA/Het_1976_rgb.png} &
        \includegraphics[width=1.2cm]{Vis_SOTA/Het_1976_sar.png} &
        \includegraphics[width=1.2cm]{Vis_SOTA/Het_1976_GT.png} &
        \includegraphics[width=1.2cm]{Vis_SOTA/Het_1976_SRGCAE.png} &
        \includegraphics[width=1.2cm]{Vis_SOTA/Het_1976_CAAE.png} &
        \includegraphics[width=1.2cm]{Vis_SOTA/Het_1976_LPEM.png} &
        \includegraphics[width=1.2cm]{Vis_SOTA/Het_1976_S2C.png}\\
        $T_1$ & $T_2$ & \multirow{2}*{GT} & SR- & \multirow{2}*{CAAE} & \multirow{2}*{LPEM} & S2C \\
        image & image & & GCAE & & & (proposed)
    \end{tabular}
    \caption{Qualitative comparison between the proposed S2C framework and SOTA methods for unsupervised MMCD.}
    \label{Fig.vis_SOTA_Het}
\end{figure}
\section{Conclusions}\label{sc5}

This study explores the formulation of a CL framework to explicitly model the unsupervised learning of semantic changes in multimodal RS images. To address the challenges posed by spectral variations, spatial misalignment, insignificant changes and multimodal heterogeneity, an S2C learning framework is developed for CD of HR RS images.
It consists of two novel CL paradigms, i.e., the CSC and CTC, which are both trained with consistency regularization to enhance robustness against different types of temporal noise. Notably, within the CTC paradigm, we present an innovative multi-temporal triplet learning strategy that addresses the existing gap in explicit difference learning. In addition, a set of novel techniques are developed to translate the VFM semantics into CD results, including grid sparsity regularization, negative cosine embedding of change probability, and an IoU matching-based refinement algorithm. Furthermore, with minimal adjustments, the S2C framework can be applied on unsupervised MMCD. Joint learning of semantic differences and consistencies enables S2C to align semantic representations across different image modalities, thus mapping domain-invariant changes. 

Experimental results reveal that the proposed S2C obtains significant accuracy improvements over the current SOTA, with advantages in $F_1$ exceeding 31\%, 9\%, 23\%, and 15\% across the four benchmark CD datasets, respectively. Additionally, S2C is efficient in training, requiring only a minimal quantity of 5 unlabeled sample pairs on the considered data set while inducing only a marginal reduction in accuracy. This highlights its potential for efficient deployment in practical applications. S2C can also be integrated into standard supervised CD methods to improve their accuracy, as it is essentially a collection of CD-specific training paradigms and post-processing algorithms that are independent of any particular DNN architectures.

A remaining limitation in S2C, as well as other UCD methods in the literature, is their lack of awareness of the particular applicational contexts. For example, S2C generates a substantial number of false alarms when applied to the Levir dataset, which focuses solely on building changes. To address this limitation, future studies are encouraged to explore language-driven UCD with the injection of user intents. Additionally, the S2C framework possesses the potential to be extended to more intricate UCD tasks, such as multi-class CD and time-series CD, which is left for further research investigations.
%%%%%%%%%%%%%%%%%%%%%%%%%%%%%%%%%%%%%%%%%%

\bibliographystyle{IEEEtran}
\bibliography{refs}

\begin{comment}
\begin{IEEEbiography}[{\includegraphics[width=1in,height=1.25in,clip,keepaspectratio]{Photo/LeiDing1.jpg}}]{Lei Ding}
received the MS’s degree in Photogrammetry and Remote Sensing from the Information Engineering University (Zhengzhou, China), and the PhD \textit{(cum laude)} in Communication and Information Technologies from the University of Trento (Trento, Italy). He is now a lecturer at the Strategic Force Information Engineering University. Since 2024, he has also been a postdoctoral researcher at the Aerospace Information Research Institute, Chinese Academy of Sciences. His research interests are related to the recognition and localization of Remote Sensing Data. He has published more than 30 research articles on international journals including \textit{IEEE TIP}, \textit{IEEE TGRS} and \textit{IEEE GRSL}.
\end{IEEEbiography}

\begin{IEEEbiography}[{\includegraphics[width=1in,height=1.25in,clip,keepaspectratio]{Photo/Zuoxibing.jpg}}]{Xibing Zuo}
received the M.S. degree in Surveying and mapping engineering from the Information Engineering University, Zhengzhou, China, in 2022, where he is currently pursuing the Ph.D. degree in surveying and mapping science and technology.\\
His research interests include machine learning and remote sensing image processing.
\end{IEEEbiography}

\begin{IEEEbiography}[{\includegraphics[width=1in,height=1.25in,clip,keepaspectratio]{Photo/Danfeng_Hong.png}}]{Danfeng Hong} (Senior Member, IEEE) received the Dr-Ing degree \textit{(summa cum laude)} in signal processing in earth observation (SiPEO) from the Technical University of Munich (TUM), Munich, Germany, in 2019. Since 2022, he has been a full professor with the Aerospace Information Research Institute, Chinese Academy of Sciences. Before joining CAS, he was a research scientist and led a Spectral Vision Working Group with the Remote Sensing Technology Institute (IMF), German Aerospace Center (DLR), Oberpfaffenhofen, Germany. He was also an adjunct scientist with GIPSA-lab, Grenoble INP, CNRS, Univ. Grenoble Alpes, Grenoble, France. His research interests include artificial intelligence, multimodal remote sensing, large foundation models, hyperspectral imaging, and large-scale Earth observation. He is an associate editor for \textit{IEEE Transactions on Geoscience and Remote Sensing (TGRS)} and the editorial board member of \textit{Information Fusion} and \textit{ISPRS Journal of Photogrammetry and Remote Sensing}. He was a recipient of the Best Reviewer Award of the IEEE TGRS in 2021 and 2022, the Best Reviewer Award of the IEEE JSTARS in 2022, the Jose Bioucas Dias Award for recognizing the outstanding paper with WHISPERS in 2021, the Remote Sensing Young Investigator Award in 2022, the IEEE GRSS Early Career Award in 2022, and a Highly Cited Researcher (Clarivate Analytics) in 2022 and 2023.
\end{IEEEbiography}

\begin{IEEEbiography}[{\includegraphics[width=1in,height=1.25in,clip,keepaspectratio]{Photo/HaitaoGuo}}]{Haitao Guo}
received his M.S. degree and Ph.D. degree from Information Engineering University, China, in 2002 and 2008, respectively.
He is currently an associate professor of photogrammetry and remote sensing at Information Engineering University, Zhengzhou, China, where he teaches digital photogrammetry and geopositioning for remote sensing imagery. His current research interests are in the areas of deep learning for image interpretation and change detection, geopositioning without ground control points for satellite imagery. 
\end{IEEEbiography}


\begin{IEEEbiography}[{\includegraphics[width=1in,height=1.25in,clip,keepaspectratio]{Photo/LuJun.jpg}}]{Jun Lu} received the M.S. and Ph.D. degree from Information Engineering University, Zhengzhou, China, in 2008 and 2015, respectively. He is currently an Associate Professor at the Information Engineering University. His research interests are related to deep learning for image interpretation and cross-view geolocation in the domains of Photogrammetry and Remote Sensing.
\end{IEEEbiography}

\begin{IEEEbiography}[{\includegraphics[width=1in,height=1.25in,clip,keepaspectratio]{Photo/GongZhihui.jpg}}]{Zhihui Gong} obtained the master's degree in Photogrammetry and Remote Sensing at the Information Engineering University, Zhengzhou, China, in 2000. He is now a professor at the Information Engineering University and has been Engaged in teaching and research work in remote sensing science and technology for over 30 years. His research interests include remote sensing image localization, remote sensing image intelligence, and intelligent unmanned systems.
\end{IEEEbiography}


\begin{IEEEbiography}[{\includegraphics[width=1in,height=1.25in,clip,keepaspectratio]{Photo/Lorenzo}}]{Lorenzo Bruzzone}
(S'95-M'98-SM'03-F'10) received the Laurea (M.S.) degree in electronic engineering (\emph{summa cum laude}) and the Ph.D. degree in telecommunications from the University of Genoa, Italy, in 1993 and 1998, respectively. \\
He is currently a Full Professor of telecommunications at the University of Trento, Italy, where he teaches remote sensing, radar, and digital communications. Dr. Bruzzone is the founder and the director of the Remote Sensing Laboratory in the Department of Information Engineering and Computer Science, University of Trento. His current research interests are in the areas of remote sensing, radar and SAR, signal processing, machine learning and pattern recognition. He promotes and supervises research on these topics within the frameworks of many national and international projects. He is the Principal Investigator of many research projects. Among the others, he is the Principal Investigator of the \emph{Radar for icy Moon exploration} (RIME) instrument in the framework of the \emph{JUpiter ICy moons Explorer} (JUICE) mission of the European Space Agency. He is the author (or coauthor) of 215 scientific publications in referred international journals (154 in IEEE journals), more than 290 papers in conference proceedings, and 21 book chapters. He is editor/co-editor of 18 books/conference proceedings and 1 scientific book. He was invited as keynote speaker in more than 30 international conferences and workshops. Since 2009 he is a member of the Administrative Committee of the IEEE Geoscience and Remote Sensing Society (GRSS). 

Dr. Bruzzone was a Guest Co-Editor of many Special Issues of international journals. He is the co-founder of the IEEE International Workshop on the Analysis of Multi-Temporal Remote-Sensing Images (MultiTemp) series and is currently a member of the Permanent Steering Committee of this series of workshops. Since 2003 he has been the Chair of the SPIE Conference on Image and Signal Processing for Remote Sensing. He has been the founder of the IEEE Geoscience and Remote Sensing Magazine for which he has been Editor-in-Chief between 2013-2017. Currently he is an Associate Editor for the IEEE Transactions on Geoscience and Remote Sensing. He has been Distinguished Speaker of the IEEE Geoscience and Remote Sensing Society between 2012-2016. His papers are highly cited, as proven form the total number of citations (more than 27000) and the value of the h-index (78) (source: Google Scholar).
\end{IEEEbiography}
\end{comment}

\end{document}
