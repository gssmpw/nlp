\documentclass[lettersize,journal]{IEEEtran}
\usepackage{graphicx}
\usepackage{subcaption}
\usepackage{multirow}
\usepackage{booktabs}
\usepackage[table]{xcolor}
\usepackage{color}
\usepackage{colortbl}
\usepackage{tabularx}
\usepackage{bm}
\usepackage{url}
\usepackage{stackengine}
\usepackage{float}
\usepackage{verbatim}
\usepackage{array}
\usepackage{cite}
\usepackage{algorithm}
\usepackage{algorithmic}
\usepackage[colorlinks,linkcolor=blue]{hyperref}
\usepackage{amsmath,amssymb} % define this before the line numbering.
\usepackage{amsfonts}
\usepackage{comment}
\usepackage{svg}
\usepackage{orcidlink}
\usepackage{textcomp}
\usepackage{stfloats}
\hyphenation{op-tical net-works semi-conduc-tor IEEE-Xplore}

\begin{document}
%\title{S2C: A Contrastive Difference Learning Framework for Unsupervised Change Detection in VHR Remote Sensing Images}

%\title{S2C: A Noise-Resistant Difference Learning Framework for Unsupervised Change Detection in Optical and Multimodal Remote Sensing Images}

\title{S2C: Learning Noise-Resistant Differences for Unsupervised Change Detection in Multimodal Remote Sensing Images}

\author{Lei~Ding\orcidlink{0000-0003-0653-8373}, Xibing~Zuo, Danfeng~Hong,~\IEEEmembership{Senior Member,~IEEE,}, Haitao~Guo, Jun~Lu, Zhihui~Gong and Lorenzo~Bruzzone\orcidlink{0000-0002-6036-459X},~\IEEEmembership{Fellow,~IEEE}~

\thanks{L. Ding is with the Information Engineering University, Zhengzhou, China, and also with the Aerospace Information Research Institute, Chinese Academy of Sciences, Beijing, China (E-mail: dinglei14@outlook.com).}

\thanks{Xibing~Zuo, Haitao~Guo, Jun~Lu, and Zhihui~Gong are with the Information Engineering University, Zhengzhou, China.}

\thanks{D. Hong is with the Aerospace Information Research Institute, Chinese Academy of Sciences, Beijing, 100094, China, and also with the School of Electronic, Electrical and Communication Engineering, University of Chinese Academy of Sciences, 100049 Beijing, China. (e-mail: hongdf@aircas.ac.cn)}

\thanks{L. Bruzzone is with the Department of Information Engineering and Computer Science, University of Trento, 38123 Trento, Italy (E-mail: lorenzo.bruzzone@unitn.it).}

\thanks{This document is funded by the Natural Science Foundation of China under Grant 42201443. It is also funded by the Henan Provincial Key Technologies R \& D Program under Grant 242102211047. (Corresponding author: Lei Ding.)}}

\markboth{Manuscript under review}%
{Shell \MakeLowercase{\textit{et al.}}: Bare Demo of IEEEtran.cls for IEEE Journals}

\maketitle

\begin{abstract}
Unsupervised Change Detection (UCD) in multimodal Remote Sensing (RS) images remains a difficult challenge due to the inherent spatio-temporal complexity within data, and the heterogeneity arising from different imaging sensors. Inspired by recent advancements in Visual Foundation Models (VFMs) and Contrastive Learning (CL) methodologies, this research aims to develop CL methodologies to translate implicit knowledge in VFM into change representations, thus eliminating the need for explicit supervision. To this end, we introduce a Semantic-to-Change (S2C) learning framework for UCD in both homogeneous and multimodal RS images. Differently from existing CL methodologies that typically focus on learning multi-temporal similarities, we introduce a novel triplet learning strategy that explicitly models temporal differences, which are crucial to the CD task. Furthermore, random spatial and spectral perturbations are introduced during the training to enhance robustness to temporal noise. In addition, a grid sparsity regularization is defined to suppress insignificant changes, and an IoU-matching algorithm is developed to refine the CD results. Experiments on four benchmark CD datasets demonstrate that the proposed S2C learning framework achieves significant improvements in accuracy, surpassing current state-of-the-art by over 31\%, 9\%, 23\%, and 15\%, respectively. It also demonstrates robustness and sample efficiency, suitable for training and adaptation of various Visual Foundation Models (VFMs) or backbone neural networks. The relevant code will be available at: \href{github.com/DingLei14/S2C}{github.com/DingLei14/S2C}.
\end{abstract}

\begin{IEEEkeywords}
Unsupervised Change Detection, Visual Foundation Model, Contrastive Learning, Remote Sensing
\end{IEEEkeywords}

%%%%%%%%%%%%%%%%%%%%%%%%%%%%%%%%%%%%%%%%%%
\section{Introduction}

In recent years, with advancements in generative models and the expansion of training datasets, text-to-speech (TTS) models \cite{valle, voicebox, ns3} have made breakthrough progress in naturalness and quality, gradually approaching the level of real recordings. However, low-latency and efficient dual-stream TTS, which involves processing streaming text inputs while simultaneously generating speech in real time, remains a challenging problem \cite{livespeech2}. These models are ideal for integration with upstream tasks, such as large language models (LLMs) \cite{gpt4} and streaming translation models \cite{seamless}, which can generate text in a streaming manner. Addressing these challenges can improve live human-computer interaction, paving the way for various applications, such as speech-to-speech translation and personal voice assistants.

Recently, inspired by advances in image generation, denoising diffusion \cite{diffusion, score}, flow matching \cite{fm}, and masked generative models \cite{maskgit} have been introduced into non-autoregressive (NAR) TTS \cite{seedtts, F5tts, pflow, maskgct}, demonstrating impressive performance in offline inference.  During this process, these offline TTS models first add noise or apply masking guided by the predicted duration. Subsequently, context from the entire sentence is leveraged to perform temporally-unordered denoising or mask prediction for speech generation. However, this temporally-unordered process hinders their application to streaming speech generation\footnote{
Here, “temporally” refers to the physical time of audio samples, not the iteration step $t \in [0, 1]$ of the above NAR TTS models.}.


When it comes to streaming speech generation, autoregressive (AR) TTS models \cite{valle, ellav} hold a distinct advantage because of their ability to deliver outputs in a temporally-ordered manner. However, compared to recently proposed NAR TTS models,  AR TTS models have a distinct disadvantage in terms of generation efficiency \cite{MEDUSA}. Specifically, the autoregressive steps are tied to the frame rate of speech tokens, resulting in slower inference speeds.  
While advancements like VALL-E 2 \cite{valle2} have boosted generation efficiency through group code modeling, the challenge remains that the manually set group size is typically small, suggesting room for further improvements. In addition,  most current AR TTS models \cite{dualsteam1} cannot handle stream text input and they only begin streaming speech generation after receiving the complete text,  ignoring the latency caused by the streaming text input. The most closely related works to SyncSpeech are CosyVoice2 \cite{cosyvoice2.0} and IST-LM \cite{yang2024interleaved}, both of which employ interleaved speech-text modeling to accommodate dual-stream scenarios. However, their autoregressive process generates only one speech token per step, leading to low efficiency.



To seamlessly integrate with  upstream LLMs and facilitate dual-stream speech synthesis, this paper introduces \textbf{SyncSpeech}, designed to keep the generation of streaming speech in synchronization with the incoming streaming text. SyncSpeech has the following advantages: 1) \textbf{low latency}, which means it begins generating speech in a streaming manner as soon as the second text token is received,
and
2) \textbf{high efficiency}, 
which means for each arriving text token, only one decoding step is required to generate all the corresponding speech tokens.

SyncSpeech is based on the proposed \textbf{T}emporal \textbf{M}asked generative \textbf{T}ransformer (TMT).
During inference, SyncSpeech adopts the Byte Pair Encoding (BPE) token-level duration prediction, which can access the previously generated speech tokens and performs top-k sampling. 
Subsequently, mask padding and greedy sampling are carried out based on  the duration prediction from the previous step. 

Moreover, sequence input is meticulously constructed to incorporate duration prediction and mask prediction into a single decoding step.
During the training process, we adopt a two-stage training strategy to improve training efficiency and model performance. First, high-efficiency masked pretraining is employed to establish a rough alignment between text and speech tokens within the sequence, followed by fine-tuning the pre-trained model to align with the inference process.

Our experimental results demonstrate that, in terms of generation efficiency, SyncSpeech operates at 6.4 times the speed of the current dual-stream TTS model for English and at 8.5 times the speed for Mandarin. When integrated with LLMs, SyncSpeech achieves latency reductions of 3.2 and 3.8 times, respectively, compared to the current dual-stream TTS model for both languages.
Moreover, with the same scale of training data, SyncSpeech performs comparably to traditional AR models in terms of the quality of generated English speech. For Mandarin, SyncSpeech demonstrates superior quality and robustness compared to current dual-stream TTS models. This showcases the potential of  SyncSpeech as a foundational model to integrate with upstream LLMs.


\section{Related Work}\label{sec:relatedwork}

\par
Different GPU simulators have been developed to explore and propose architectural changes to these architectures. Some of the most popular ones are single-thread simulators, such as Multi2Sim \cite{multi2sim} or GPGPU-Sim \cite{gpgpusimOriginal}. The former models the AMD Evergreen \cite{amdevergreen} architecture, while the latter models the NVIDIA Tesla \cite{teslaHotchips}. Recently, GPGPU-Sim was updated and renamed as the Accel-sim framework \cite{accelsim} to include some major features introduced in the NVIDIA Volta \cite{voltaPaper} architecture.

\par
Some previous works have developed parallel GPU simulators. The first one is Barra \cite{barra}, a GPU functional simulator focused on the NVIDIA Tesla architecture, which achieves a speed-up of 3.53x with 4 threads. However, this simulator models an old architecture and does not provide a timing model. Another work that models the NVIDIA Tesla architecture is GpuTejas \cite{gputejas}, which includes a timing model and achieves a mean speed-up of 17.33x with 64 threads. Unfortunately, executing GpuTejas in parallel has an indeterministic behavior, leading to accuracy simulation errors of up to 7.7\% compared to the single-threaded execution. One of the most successful parallel simulators is MGPUSim \cite{mgpusim}, an event-driven simulator that includes functional and timing simulation targeting the AMD GCN3 \cite{amdgcn3}. MGPUSim follows a conservative parallel simulation approach for parallelizing the different concurrent events during the simulation, preventing any deviation error from executing the simulator in parallel. It achieves a mean speed-up of 2.5x when executed with 4 threads.

\par
Several works have parallelized the GPGPU-Sim simulator. MAFIA \cite{mafia} can run different kernels concurrently in multiple threads but cannot parallelize single-kernel simulations. Lee et al. \cite{parallelGPUSim1} \cite{parallelGPUSim2} have proposed a simulator framework built on top of GPGPU-Sim. Their proposal needs at least three threads in order to run. Two threads are always dedicated to executing the Interconnect-Memory Subsystem and the Work Distribution and Control components. The rest of the threads are devoted to parallelizing the execution of the multiple SMs of the GPU. Lee et al. approach has an average 3\% simulation error compared to the original sequential simulation, achieving an average speed-up of 5x and up to 8.9x in some benchmarks.

\par
Some simulators, such as NVAS \cite{nvas}, address the highly time-consuming problem of simulations by reducing the detail of some components. For example, modeling the GPU on-chip interconnects in low detail in NVAS reports a 2.13x speed-up and less than 1\% benefit in mean absolute error compared to a high-fidelity model. Avalos et al. \cite{pcaKernelGpuSampling} rely on sampling techniques to simulate huge workloads.

\par
In contrast to previous works, we follow a simple approach to parallelize the Accel-sim framework simulator, the most modern academic GPU simulator used for research and capable of executing modern NVIDIA GPU architectures and workloads. Our proposal employs OpenMP \cite{openmp} to implement a scalable implementation that allows parallelizing the simulator with a user-defined number of threads. Moreover, our approach does not compromise the simulation accuracy and determinism when the simulator runs in parallel and provides the same results as the sequential version. Thus, it eases developing and debugging tasks. This makes our work more robust than the implementations of Lee et al. \cite{parallelGPUSim1} \cite{parallelGPUSim2}, and GpuTejas \cite{gputejas}, where the parallel version results differ from the single-threaded one. Moreover, our work is orthogonal to approaches such as the ones followed by NVAS \cite{nvas} and Avalos et al. \cite{pcaKernelGpuSampling}, which reduce the detail of some components and use sampling to speed up simulations even more.


\section{Proposed Approach}\label{sc3}

This section first explains the motivation for learning noise-resistant representations, then introduces the S2C learning framework, and subsequently details the proposed technologies including contrastive change learning, grid sparsity loss, and the change mapping algorithms. Finally, the S2C methodologies are expanded to address unsupervised MMCD.

\subsection{Noise-resistant Semantic Embedding}\label{sc3.A}

DL-based CD essentially learns to project multi-temporal RS images ${I_1, I_2}$ into a binary change map $\mathbf{y}_c$. Let $f_\theta$ denote an encoding function parameterized by $\theta$, and $g$ being a projection function. This process can be represented as:
\begin{equation}\label{eq.cd}
    \mathbf{y_1} = f_\theta(I_1), \mathbf{y_2} = f_\theta(I_2), \mathbf{y_c} = g(\mathbf{y_1}, \mathbf{y_2})
\end{equation}
where $\mathbf{y_1}, \mathbf{y_2} \in \mathbb{R}^{c \times h \times w}$ are the learned semantic latent, $\mathbf{y_c} \in \mathbb{R}^{h \times w}$ is a change probabilistic map, $s$ and $h, w$ are the channel and spatial dimensions, respectively.

Let us denote the imaging process as $\Phi$, ground semantics as $s$, and semantic changes as $\delta$. In an ideal case where there is no temporal noise, this process can be re-written as:
\begin{equation}\label{eq.imaging}
    \mathbf{y}_s = f_\theta[\Phi_1(s)], \mathbf{y}_{s+\delta} = f_\theta[\Phi_2(s+\delta)], \mathbf{y_c} = g(\mathbf{y}_s, \mathbf{y}_{s+\delta})
\end{equation}
Since $\mathbf{y}_s$ and $\mathbf{y}_{s+\delta}$ share a common semantic space, $g$ can be implemented using a simple linear transformation with normalization. However, when considering practical cases that involve temporal noise (refer Fig.\ref{fig.challenge}), there exist insignificant changes (denoted $\epsilon$), spatial variance (denoted $\Omega$), and sensor differences ($\Phi_1 \neq \Phi_2$). Consequently, $I_2$ is projected into a different space:
\begin{equation}\label{eq.noise}
    f_\theta(I_2)=f_\theta\{\Omega[\Phi_2(s+\delta+\epsilon)]\}=\mathbf{y^{\prime}}_{s+\delta+\epsilon}
\end{equation}

Thus, the distance between $\mathbf{y}_s$ and $\mathbf{y^{\prime}}_{s+\delta+\epsilon}$ is nonlinear, and optimizing subsequent change embedding with $g$ typically requires supervised learning with task-specific labels.

To achieve unsupervised change learning, we utilize spatial and spectral augmentations to simulate various types of temporal noise, thus training a noise-resistant $f_\theta$ and formulating a training-free $g$. First, we apply augmentations functions $\phi$ on $I_1$ simulating the noise in Eq.(\ref{eq.noise}):
\begin{equation}\label{eq.phi_I1}
    \phi(I_1)= \hat{\Omega}\{\hat{\Phi}_{1 \rightarrow 2}[\Phi_1(s)]\} = \hat{\Omega}[\hat{\Phi}_{1,2}(s)],
\end{equation}
where $\hat{\Omega}$ is a set of spatial augmentations, $\hat{\Phi}_{1 \rightarrow 2}$ is a set of spectral augmentations that adapt the spectral distribution of $I_1$ approaching $I_2$. The augmentations are performed with stochastic parameters to simulate the imagining process with random noise, generating diverse training pairs. Further utilizing CL techniques elaborated in Sec.\ref{sc3.CL}, we can train $f_\theta$ towards projecting $I_1$ and $\phi(I_1)$ into similar representations, i.e., learning noise-invariant semantic latent:
\begin{equation}
    f_\theta(I_1) = \mathbf{\hat{y}}_s, 
    f_\theta[\phi(I_1)] = f_\theta\{ \hat{\Omega}[\hat{\Phi}_{1,2}(s)] \} \approx \mathbf{\hat{y}}_{s}
\end{equation}
These operations are also symmetrically performed on $I_2$ as:
\begin{equation}\label{eq.phi_I2}
\begin{aligned}
    \phi(I_2) = \hat{\Omega}\{\hat{\Phi}_{2 \rightarrow 1}[\Phi_2(s+\delta+\epsilon)]\} \approx \hat{\Omega}[\hat{\Phi}_{1,2}(s+\delta)],\\
    f_\theta(I_2) = \mathbf{\hat{y}}_{s+\delta}, f_\theta[\phi(I_2)] \approx f_\theta\{ \hat{\Omega}[\hat{\Phi}_{1,2}(s)] \} \approx \mathbf{\hat{y}}_{s+\delta}
\end{aligned}
\end{equation}
where $\epsilon$ denoting the subtle changes can be diminished since $\hat{\Omega}$ includes spatial filtering operations.
This enables modeling the semantic differences between $\mathbf{\hat{y}}_s, \mathbf{\hat{y}}_{s+\delta}$ in a common latent space, enabling a training-free formulation of $g$:
\begin{equation}
    \mathbf{y_c} = g(\mathbf{\hat{y}}_s, \mathbf{\hat{y}}_{s+\delta})
\end{equation}

\subsection{Overview of the S2C framework}
The theoretical analysis in Sec.\ref{sc3.A} offers a simplified understanding of the optimizing goals. In the following, we elaborate on the S2C architecture that trains $f_\theta$ to exploit noise-invariant semantic representations.

As depicted in Fig.\ref{fig.flowchar}, S2C is a UCD framework consisting of distinct stages of training and inference. During the training phase, a set of augmentation operations are first performed on the input images, corresponding to $\phi(\cdot)$ in Eq.(\ref{eq.phi_I1})(\ref{eq.phi_I2}). The augmentations comprise a sequential combination of spectral and spatial operations executed randomly at each training iteration, including \textit{RGBshift}, \textit{PCA adaptation}, \textit{down-sampling} and \textit{random shifting}. \textit{PCA adaptation} blends the spectral distribution of $I_1$ and $I_2$. The spatial operations are performed to simulate spatial misalignment and imaging degradation/distortion, while the spectral operations replicate imaging and seasonal variations. Collectively, their combination simulate stochastic injection of diverse temporal noise.

Subsequently, utilizing the constructed pre- and after-augmentation image sequences, two self-supervised CL strategies are jointly employed to learn task-specific semantic features. Although this training process can be performed from scratch using common DNNs, employing VFM as feature encoders empirically leads to better accuracy. In addition, considering the limitations of VFM in processing RS images \cite{ji2024segment}, we employ a VFM augmented with additional parameters $w$, denoted as $f_{\theta + w}$. $w$ is a trainable parameter implementing LoRA \cite{hu2021lora} (with $rank=4$), a parameter-efficient technique extensively used to adapt VFMs to a particular domain of interest. $\theta$ is frozen to retain pre-trained visual knowledge, whereas $w$ are the LoRA weights trained with CL paradigms to exploit temporal semantic features. The VFM can be any off-the-shelf models, as its inner structure is not modified in our S2C architecture.

The training process is conducted within two CL paradigms to learn the CD-relevant semantic representations. The two CL paradigms are introduced to embed difference and consistency representations, respectively, which are elaborated in Sec.\ref{sc3.CL}. The associated loss functions are $\mathcal{L}_{tri}$ and $\mathcal{L}_{info}$, respectively. In addition, we further introduce a sparsity regularization objective to learn sparse and compact change representations, noted as $\mathcal{L}_{spa}$. The joint training objective is:
\begin{equation}\label{eq.losses}
    \mathcal{L} = \mathcal{L}_{tri} + \alpha \mathcal{L}_{info} + \beta \mathcal{L}_{spa}
\end{equation}
where $\alpha$ and $\beta$ are two weighting parameters.

In the inference phase, the semantic latent $\mathbf{y}_1$ and $\mathbf{y}_2$ are first mapped to a coarse change map, then refined using the VFM decoder and an IoU matching function. The details are elaborated in \ref{sc3.chg_map}.

\subsection{Contrastive Change Learning} \label{sc3.CL}

\begin{figure*}[t]
\centering
    \includegraphics[width=0.95\linewidth]{illu_pic/CL_pipelines.png}
    \caption{Comparison of CL paradigms in CD. (a) \textbf{Consistency regularization}: $f_\theta$ extracts stable representations across weak/strong perturbations; (b) \textbf{Spatial contrast}: $f_\theta$ distinguishes same/different regions; (c) Proposed \textbf{Consistency-regularized Temporal Contrast} (CTC): $f_\theta$ identifies temporal differences independent of spectral or seasonal variations, and (d) Proposed \textbf{Consistency-regularized Spatial Contrast} (CSC): $f_\theta$ distinguishes same/different regions despite perturbations.}
\label{fig.CL_paradigms}
\end{figure*}

Before introducing the proposed CL paradigms, let us first review the two typical CL paradigms in CD, and analyze their usage and limitations.

\textit{1) Consistency Regularization (CR).} As depicted in Fig.\ref{fig.CL_paradigms}(a), a DNN $f_{\theta}$ learns to improve the robustness and generalization of feature embeddings. An image $I$ is first augmented with weak and strong transformations, resulting in two copies $\Tilde{I}$ and $\Bar{I}$. Then a distance loss is calculated between the two copies to ensure consistency across perturbations.

Since this learning paradigm does not explicitly model differences/similarities, it is often adopted in semi-supervised \cite{bandara2022revisiting} or weakly-supervised \cite{zhao2024pixellevel} learning settings to extend the CD insights learned with limited samples.

\textit{2) Spatial Contrast (SC).} As illustrated in Fig.\ref{fig.CL_paradigms}(b), $f_{\theta}$ learns to differentiate between bitemporal image pairs $[I_1^i, I_2^i]$ of the same region $i$ and $[I_1^i, I_2^j]$ of different regions $i$ and $j$. This drives $f_{\theta}$ to learn consistent embeddings against temporal variations. Areas with high similarity are identified as $unchanged$, whereas their opposites are detected as $changed$ \cite{chen2022selfsupervised}.

However, we identify that there are several limitations in this paradigm: i) changes are identified through negative embedding of the similarities rather than through explicit modeling. This often causes sensitivity to noise. ii) $f_{\theta}$ focuses on the discriminative elements within a region, such as certain edges or corners, rather than effectively exploiting the local semantic context. %iii) the supervision signal is given at the image level (same/different pairs), thus only coarse similarity representations are obtained.

Considering these limitations, we introduce two novel CL paradigms specifically tailored to the context of CD.

\textbf{Consistency-regularized Temporal Contrast} (CTC). An RS image $I_1$ is first augmented with a transform function $\phi(\cdot)$, producing a copy $\Bar{I}_1$. Subsequently, $I_1$ is employed as an anchor for comparison with both a positive sample $\Bar{I}_1$ and a negative sample $I_2$. $\phi(\cdot)$ simulates spectral and spatial noise between multi-temporal observations, as illustrated in Fig.\ref{fig.flowchar}. Consequently, $f_\theta$ learns to exploit noise-invariant difference representations, i.e., the semantic changes. With greater details, Fig.\ref{fig.flowchar} illustrates the CTC paradigm with bi-directional comparisons within $[I_1, \Bar{I}_1, I_2]$ and $[I_2, \Bar{I}_2, I_1]$. 

A triplet training objective $\mathcal{L}_{tri}$ using cosine distance is utilized for comparisons within the triplets. This is to align with the cosine difference embedding during the inference stage. The calculations are as follows:
\begin{equation}
\begin{aligned}
    %\mathbf{y_1}, \mathbf{y_2} = f_{\theta + w}(I_1), f_{\theta + w}(I_2),\\
    %\mathbf{\Bar{y}_1}, \mathbf{\Bar{y}_2} = f_{\theta+w}[\phi(I_1)], f_{\theta+w}[\phi(I_2)],\\
    \mathcal{L}_{tri} = max[cos(\mathbf{y_1}, \mathbf{y_2})-cos(\mathbf{y_1}, \mathbf{\Bar{y}_1})+m, 0]\\
    + max[cos(\mathbf{y_2}, \mathbf{y_1})-cos(\mathbf{y_2}, \mathbf{\Bar{y}_2})+m, 0]
\end{aligned}
\end{equation}
where $m=1$ is a margin parameter to promote separation between the anchor and positive.

\textbf{Consistency-regularized Spatial Contrast} (CSC). This contrastive learning paradigm integrates CR into typical SC learning, thereby enhancing the embedding of spatial consistency against perturbations. CSC alleviates the vulnerability to noise inherent in the SC paradigm by incorporating transformation $\phi(\cdot)$. The transformations, particularly the spatial transformations, reduce dependence on high-frequency spatial details, thereby prompting the exploitation of local semantic patterns such as color and texture.

We have introduced an additional variation in CSC, i.e., the calculation of consistency at each spatial position. Given a batch consisting of $N$ paired RS images $\{[I_1^i, I_2^i], [I_1^j, I_2^j], ..., [I_1^k, I_2^k]\}$, we first apply $\phi(\cdot)$ on each of the temporal images, thus getting two sets of augmented images. These images are further encoded with $f_{\theta+w}$, resulting in 4 sets of features: $[\mathbf{y_1^i, y_1^j, ..., y_1^k}]$, $[\mathbf{y_2^i, y_2^j, ..., y_2^k}]$, $[\mathbf{\Bar{y}_1^i, \Bar{y}_1^j, ..., \Bar{y}_1^k}]$ and $[\mathbf{\Bar{y}_2^i, \Bar{y}_2^j, ..., \Bar{y}_2^k}]$. We then calculate the co-occurrences between them, resulting in two matrices each with $N \times N$ dimensions, as illustrated in Fig.\ref{fig.CL_paradigms}(d). We utilize an infoNCE loss function to effectively train $f_{\theta+w}$ for the differentiation of genuine image pairs. It is calculated across both temporal phases, represented as:
\begin{equation} \label{eq.info_loss}
\begin{aligned}
    \mathcal{L}_{info} = -\frac{1}{N} \sum_{u=1}^{N} \log \left[ \frac{\exp \left( \mathbf{y}_1^u \odot \mathbf{\Bar{y}}_2^u \right)}{\sum_{v=1}^{N} \exp \left( \mathbf{y}_1^u \odot \mathbf{\Bar{y}}_2^v \right)} \right]\\
    -\frac{1}{N} \sum_{u=1}^{N} \log \left[ \frac{\exp \left( \mathbf{y}_2^u \odot \mathbf{\Bar{y}}_1^u \right)}{\sum_{v=1}^{N} \exp \left( \mathbf{y}_2^u \odot \mathbf{\Bar{y}}_1^v \right)} \right]
\end{aligned}
\end{equation}
where $\odot$ denotes a spatial similarity function that we introduce in this study. Instead of pooling the spatial features into single vectors for similarity calculation \cite{chen2022selfsupervised}, we compute the similarity at each spatial patch $p$, denoted as:
\begin{equation} \label{eq.sim_calc}
    \mathbf{y} \odot \mathbf{\Bar{y}} = \frac{1}{w \times h} \sum_{p} \left( \frac{\mathbf{y}^p \cdot \mathbf{\Bar{y}}^p}{|\mathbf{y}^p||\mathbf{\Bar{y}}^p|} \right)
\end{equation}
Both $\mathcal{L}_{tri}$ and $\mathcal{L}_{info}$ are calculated based on cosine similarity. While $\mathcal{L}_{tri}$ embeds appearance-invariant temporal differences, $\mathcal{L}_{info}$ embeds noise-resilient temporal consistencies. Therefore, when a certain temporal consistency pattern is captured in CSC, it suppresses the difference representations of the same area in CTC.

\subsection{Grid Sparsity loss} \label{sc3.loss_sparse}

Changed objects are commonly sparsely distributed in RS images and each present as compact regions. In contrast, edges and points are often associated with noise. Although training objectives that promote sparse representations have been explored in the literature, they typically calculate and penalize the average value of $\mathbf{y_c}$ \cite{bandara2023deep}. However, this approach does not guarantee sparsity, as there exists a trivial solution of learning an additional bias term on $\mathbf{y_c}$.

Differently, we propose a novel grid sparsity loss where sparsity is assessed at the level of each local grid rather than at each pixel. Considering the frequency of changes along with spatial resolution in an RS image, first, we predefine a sparsity threshold $T$ as well as a grid size $d$. Subsequently, the average density of each grid $\mathbf{g}$ is calculated and ranked, while a {\footnotesize$1-T$} ratio of grids with the lowest density are selected for loss calculation as follows:
\begin{equation} \label{eq.sparse_loss}
\begin{aligned}
    \mathcal{L}_{spa} = max\{ \frac{1}{n} \sum^{n} [sort\uparrow(y^\mathbf{g})], 0\},\\
    y^\mathbf{g} = \frac{1}{d*d} \sum_{p \in \mathbf{g}} \mathbf{y}_c^p, 
    n = wh*(1-T)/d^2
\end{aligned}
\end{equation}

We empirically set $d=16$ for HR RS images. This regularization objective ensures that a proportion of less than {\footnotesize$1-T$} potential changes exhibit high values, whereas the insignificant change representations in other areas are minimized.

\subsection{Change mapping} \label{sc3.chg_map}

\begin{algorithm}[t]
    \caption{Algorithm for IoU Matching and Refinement}
    \label{AlgorithmD}  \label{Algorithm.IoU_matching}
    \begin{algorithmic}[1]
        \renewcommand{\algorithmicrequire}{\textbf{Input:}}
        \renewcommand{\algorithmicensure}{\textbf{Output:}}
        \REQUIRE Change probability map $\mathbf{y}_c$,\\
                 VFM-extracted bi-temporal object masks $M_1$, $M_2$,\\
                 Parameter: IoU threshold $T_{IoU}$;
        \ENSURE Refined Change Map $M_c$;
        \FORALL{$m_i \in M_1, m_j \in M_2$}
        \IF {$(m_i \cap m_j)/(m_i \cup m_j)>T_{IoU}$}
            \STATE $M_{1} \leftarrow M_{1} \setminus \{m_i\}$
            \STATE $M_{2} \leftarrow M_{2} \setminus \{m_j\}$
            %\STATE $M_{12} \leftarrow M_{12} \cup \{m_i, m_j\}$
        \ENDIF
        \ENDFOR
        \STATE $M_{12} \leftarrow M_{1} \cup M_{2}$
        \FORALL{$m \in M_{12}$}
        \IF {$m \odot \mathbf{y}_c / \sum_{p} m_p >T_{IoU}$}
            \STATE $M_c \leftarrow M_{c} \cup \{m\}$
        \ENDIF
        \ENDFOR
 \RETURN $M_c$ 
    \end{algorithmic}
\end{algorithm}

The use of VFM and CL methodologies aims to enhance the effective exploitation of semantic contexts across multi-temporal image domains. In the inference phase, the major challenge lies in accurately mapping fine-grained changes. We employ a coarse-to-fine refinement strategy. First, a coarse change probability map $\mathbf{y}_c$ is derived by projecting the negative cosine embedding of the bi-temporal semantic embeddings:
\begin{equation}
    \mathbf{y}_c = \sigma[-cos(\mathbf{y}_1, \mathbf{y}_2)*\eta]
\end{equation}
where $\sigma$ is a $sigmoid$ function and $\eta=ln(1/0.07)$ is a scaling factor defined following literature practice \cite{deuser2023sample4geo}.

Then, we employ a pretrained VFM decoder $g_{\gamma}$ to segment two groups of bi-temporal masks $M_1=\{m_1^1, m_1^2, ..., m_1^k\}$ and $\{M_2=m_2^1, m_2^2, ..., m_2^k\}$ using the spatial prompts generated on high-response regions in $\mathbf{y}_c$. Given the logic implication of \textit{change}, high-overlap objects in $M_1$ and $M_2$ can be inferred as false alarms. Therefore, we use an XOR-alike matching algorithm (denoted $\ominus$) to merge $M_1$ and $M_2$ while eliminating the objects with large overlaps:
\begin{equation}
    M_{12} = M_1 \ominus M_2
\end{equation}
We further implement an Intersection-over-Union (IoU) analysis between $\mathbf{y}_c$ and $M_{12}$ to match the VFM-generated masks with the high-confidence regions in $\mathbf{y}_c$. The matched objects replace their counterparts in $\mathbf{y}_c$ as the changed items. For more details, a pseudo-code of this IoU analysis and matching algorithm is provided in Algorithm \ref{Algorithm.IoU_matching}.

Using this change mapping algorithm, the coarse predictions derived from the DNN are refined into detailed CD results, aligning with the spatial details present in the HR imagery.

\subsection{S2C for Unsupervised MMCD}

\begin{figure}[t]
\centering
    \includegraphics[width=1\linewidth]{illu_pic/S2C_flowchart_het.png}
    \caption{Illustration of the application of the proposed S2C for UCD in multimodal RS images. This learning framework applies to not only optical and SAR data, but also other image modalities.}
\label{fig.Het_flowchart}
\end{figure}

RS images observed by specialized sensors such as Synthetic Aperture Radar (SAR) and Infrared (IR) scanners demonstrate a notable modality difference compared to standard optical images. This significant domain gap precludes VFMs from extracting meaningful semantic representations \cite{ji2024segment}. Leveraging the joint modeling of consistency and differences, the S2C framework is capable of modeling modality-invariant change representations over multi-temporal observations.

Fig.\ref{fig.Het_flowchart} depicts the adapted S2C training pipeline for multimodal RS images, where the UCD process is exemplified through a pair of optical and SAR images. The semantic-to-change mapping is achieved through alignment of latent semantics, making it invariant to specific imaging modalities. Let us denote $I_{rgb}^i$ and $I_{sar}^i$ as a pair of multimodal images pertaining to region $i$. Two independent encoders with parameters $\theta$ and $\zeta$ are trained to extract features from optical and SAR images, respectively. It is worth noting that the VFMs are typically not applicable to SAR data and, therefore are not utilized in $f_\zeta$. However, with the continuous development of RS VFMs \cite{guo2024skysense, hong2024spectralgpt}, certain models may demonstrate the efficacy of semantic embedding within a specific domain. While we did not identify an optimal VFM for common SAR data, readers are encouraged to explore emerging VFMs within the S2C framework.

After feature embedding, the CTC and CSC paradigms are utilized to learn domain-invariant differences and consistencies. An adjustment in implementing the CTC is that it is asymmetrically applied, unlike its use with dual optical images. Given that optical data possess more comprehensive spatial information, the calculation of $\mathcal{L}_{tri}$ is centric to the optical data, i.e.:

\begin{equation} \label{eq.het_triplet_loss}
    \mathcal{L}_{tri} = max[cos(\mathbf{y}_{rgb}, \mathbf{y}_{sar})-cos(\mathbf{y}_{rgb}, \mathbf{\Bar{y}}_{rgb})+m, 0]
\end{equation}
where $\mathbf{y}_{rgb}$, $\mathbf{y}_{sar}$ and $\mathbf{\Bar{y}}_{rgb}$ are the semantic features encoded from $I_{rgb}$, $I_{sar}$ and $\phi(I_{rgb})$, respectively. The augmentation operations outlined in $\phi$ correspond to those detailed in Sec.\ref{sc3.CL}.

The presence of a domain gap poses more challenges to learning temporal consistency. Consequently, CSC learning is conducted directly on the original image sets $[\mathbf{y}_{rgb}^i, \mathbf{y}_{rgb}^j, ..., \mathbf{y}_{rgb}^k]$ and $[\mathbf{y}_{sar}^i, \mathbf{y}_{sar}^j, ..., \mathbf{y}_{sar}^k]$, rather than using their augmented counterparts. The loss function is computed as:
\begin{equation} \label{eq.het_info_loss}
\begin{aligned}
    \mathcal{L}_{info} = -\frac{1}{N} \sum_{u=1}^{N} \log \left[ \frac{\exp \left( \mathbf{y}_{rgb}^u \odot \mathbf{y}_{sar}^u \right)}{\sum_{v=1}^{N} \exp \left( \mathbf{y}_{rgb}^u \odot \mathbf{y}_{sar}^v \right)} \right]\\
    -\frac{1}{N} \sum_{u=1}^{N} \log \left[ \frac{\exp \left( \mathbf{y}_{sar}^u \odot \mathbf{y}_{rgb}^u \right)}{\sum_{v=1}^{N} \exp \left( \mathbf{y}_{sar}^u \odot \mathbf{y}_{rgb}^v \right)} \right]
\end{aligned}
\end{equation}

The comprehensive training objective is consistent with that in Eq.\ref{eq.losses}. The simultaneous application of CTC and CSC learning drives $f_\theta$ and $f_\zeta$ to align semantic representations across different domains, thus mapping domain-invariant changes. In the inference stage, since typical VFM decoders are unable to segment SAR objects, the refining algorithm is omitted. The change probability maps $\textbf{y}_c$ is directly binarized to map the multimodal changes.

\section{Experiments}

We use NVIDIA RTX 3090 or 4090 GPUs for experiments.


\subsection{Datasets}
\label{sec:datasets}


\begin{table}[h!]
\centering
\begin{adjustbox}{width=\columnwidth, center}
\renewcommand{\arraystretch}{0.9}
\begin{tabular}{ccccc}
\Xhline{3\arrayrulewidth}
\multirow{2}{*}{\textbf{Lang-pair}} & \multirow{2}{*}{\textbf{Dataset}} & \multicolumn{3}{c}{\textbf{\# Sentences}} \\ \cline{3-5} 
 & & \textbf{Train} & \textbf{Dev} & \textbf{Test} \\ \hline\hline 
\multirow{2}{*}{KO $\leftrightarrow$ IT} & TED 2020 v1& \multirow{2}{*}{357,733} & \multirow{2}{*}{2,000} & \multirow{2}{*}{2,000} \\
 & \cite{ted2020} & & & \\ \hline  
\multirow{2}{*}{AR $\leftrightarrow$ PT} & WikiMatrix v1& \multirow{2}{*}{153,441} & \multirow{2}{*}{2,000} & \multirow{2}{*}{2,000} \\
 & \cite{schwenk2019wikimatrix}& & & \\
\Xhline{3\arrayrulewidth}
\end{tabular}
\end{adjustbox}
\caption{Datasets statistics.}
\label{tab:dataset statistics}
\end{table}


We conduct experiments on the linguistically distant languages within pairs: not in the same language family and using different scripts.
We select 2 language pairs, resulting in 4 translation directions in total, Korean (\textit{Koreanic})$\leftrightarrow$Italian (\textit{Romance}) and Arabic (\textit{Arabic})$\leftrightarrow$Portuguese (\textit{Romance}).
The language family grouping is defined by the criteria presented in \citet{m2m100}.


We validate our approach across various domains.
For Korean$\leftrightarrow$Italian pair, we run experiments on TED2020~\cite{ted2020}.
For Arabic$\leftrightarrow$Portuguese pair, we use WikiMatrix~\cite{schwenk2019wikimatrix}.
All the datasets are obtained from the OPUS\footnote{\url{https://opus.nlpl.eu}}~\cite{opus} project.
The statistics for the datasets are listed in Table~\ref{tab:dataset statistics}.




\subsection{Evaluation Metrics}
We assess the translation quality using BLEU~\cite{papineni-etal-2002-bleu}, chrF++~\cite{popovic-2017-chrf}, and reference-based COMET (\textit{wmt22-COMET-da}) \cite{comet22}.
For reporting BLEU, \textit{SacreBLEU}~\cite{post-2018-call} is used with \texttt{ko-mecab} tokenizer for Korean and \texttt{13a} tokenizer for the others.


\subsection{Baselines}
\label{sec:baselines}

As an encoder-decoder NMT model, we use NLLB-200-distilled-600M~\cite{nllb}.
When training NLLB, we use the Transformers library from HuggingFace~\cite{wolf2020huggingfaces}.
AdamW optimizer~\cite{adamw} is used with a learning rate of $2e$$-$$5$, batch size of 2, and dropout with a probability of 0.1.
When validation BLEU was not improved for 3 checkpoints, with 30k steps between them, we stopped training.


For open-source LLMs, we use Vicuna 13B~\cite{Vicuna}, Baize 13B~\cite{baize}, and Llama-3-8B-Instruct~\cite{llama3modelcard} as the baseline. 
We fine-tuned these LLMs with QLoRA~\cite{dettmers2024qlora}; $r$=16, $\alpha$=64, dropout=0.1 for all linear layers.
For black-box LLMs, we use GPT-4~\cite{gpt4} and GPT-4o~\cite{gpt4o}.
The version of \texttt{gpt-4-1106-preview} and \texttt{gpt-4o-2024-08-06} are employed for GPT-4 and GPT-4o, respectively.
For GPT models, \textit{temperature} is set to 0.0 for stable responses~\cite{peng2023making} and \textit{top\_ p} is set to 0.1 to ensure reproducibility. 
For LLMs, we use the prompt template of~\citet{howgood}, as presented in Appendix~\ref{sec:apdx_prompt_templates}.


As state-of-the-art ensemble baselines, we employ \blender~\cite{llm-blender}, EVA~\cite{eva}, and MBR~\cite{mbr}.
For \blender and EVA, we fine-tuned the same open-source LLMs used in each study utilizing the parallel corpus described in Table~\ref{tab:dataset statistics}.
The list of the LLMs is in Appendix~\ref{sec:apdx_llms}.
\textit{temperature} is set to 0.1 to mitigate hallucination for low-resource pairs~\cite{guerreiro2023hallucinations}.
For MBR, we generate a set of 5 hypotheses using GPT-4.
When generating hypotheses, \textit{temperature} was set to 0.0 for its optimal performance, based on the results of our pilot experiments in Appendix~\ref{sec:MBR} and prior study~\cite{peng2023making}.
Other configurations are the same as in the original work~\cite{mbr}.

\subsection{Implementation Details}

In the candidate generation step of \ours, we employ \nllb.
For each source-target language pair, we use an NLLB fine-tuned for the language pair in Table~\ref{tab:dataset statistics} to generate the directly translated candidates.
For the merging module, we use Llama-3, GPT-4, and GPT-4o.
For all models used in \ours, including NLLB, Llama-3, GPT-4, and GPT-4o, we apply the same settings in \S\ref{sec:baselines}.


As detailed in \S\ref{sec:candidate aggregation}, we explore two approaches in the ensemble process: one dynamically selects the top-$\textit{k}$ ($\textit{k}$=3) candidates and another uses candidates obtained from fixed paths.
To select the top-$\textit{k}$ candidates for each source sentence, we use the reference-free COMETkiwi as described in \S\ref{sec:candidate aggregation}.
When selecting candidates from fixed paths, we used directly translated candidates and English-pivoted candidates, which were the top-performing paths on the FLORES-200 benchmark.


\subsection{Main Results}


\begin{table*}[t]
\centering
\renewcommand{\arraystretch}{0.9}
\begin{adjustbox}{width=0.95\textwidth, center}
\begin{tabular}{lcccccccccccc}
\Xhline{3\arrayrulewidth}

\multirow{2}{*}{\textbf{Model}}  & \multicolumn{3}{c}{\textbf{Korean$\rightarrow$Italian}} & \multicolumn{3}{c}{\textbf{Italian$\rightarrow$Korean}} & \multicolumn{3}{c}{\textbf{Arabic$\rightarrow$Portuguese}} & \multicolumn{3}{c}{\textbf{Portuguese$\rightarrow$Arabic}}\\ \cline{2-13}
 & BLEU & chrF++ & COMET & BLEU & chrF++ & COMET  & BLEU & chrF++ & COMET & BLEU & chrF++ & COMET \\ \hline\hline 

\textit{\textbf{Standalone NMT System}}\\ \hdashline[3pt/3pt]
NLLB~\cite{nllb} & 16.27 & 41.14 & 84.60 &            17.40 &           23.39 &           87.33 &           27.25 &           50.35 &           84.21 &            13.50 &            40.90 &           84.24 \\
Vicuna~\cite{Vicuna} &10.11  & 31.15 & 70.29 & 10.60 & 16.51 & 72.29 & 17.64 & 38.44 & 76.01 & 8.40 &  27.38& 79.18 \\
Baize~\cite{baize} & 10.62 & 31.87 & 73.62 & 10.38 & 16.44 & 76.63 & 16.56 & 36.67 & 76.87 & 8.50 &27.28 & 79.18   \\
Llama-3~\cite{llama3modelcard} & 11.79 & 34.82 & 77.37 & 13.82 & 18.95 & 85.80 & 18.78 & 40.20 & 78.73 & 12.25 & 35.16 & 82.79 \\
GPT-4~\cite{gpt4}         &          14.07 &           42.22 &            86.80 &           17.23 &           22.96 &           86.94 &           25.82 &           51.89 &           85.46 &           15.11 &           41.39 &           83.99 \\ 
GPT-4o~\cite{gpt4o} & 15.11 & 42.59 & 85.93 & 17.20 &  22.82 & 85.31 & 27.28 & 52.57 & 85.90 & 16.28 & 42.40 & 83.82 \\
\hline




\textit{\textbf{Prior Ensemble Method}}\\ \hdashline[3pt/3pt]
LLM-Blender~\cite{llm-blender} & 8.77 & 28.74 & 82.80 & 0.03 & 0.85 & 42.77 & 11.80 & 29.85 & 67.95 & 0.94 & 2.69 & 46.49 \\
EVA~\cite{eva} & 2.53 & 15.26 & 39.00 &1.51 & 3.57 &  37.17 & 9.77 &  28.40 &  68.75 & 7.99 & 27.00 & 73.15 \\
MBR~\cite{mbr} & 14.10 & 42.24 & 86.70 & 17.14 & 23.00 & 87.53 & 25.45 & 51.78 & 85.55 & 14.66 & 41.11 & 83.93 \\ \hline


\textit{\textbf{Proposed Method}}\\ \hdashline[3pt/3pt]
\ours (Llama-3; top3) & 15.60 & 39.86 & 84.10 & 14.56 & 19.92 & 87.34 & 23.41 & 45.95 & 81.66 & 14.27 & 38.25 & 81.80 \\
\ours (Llama-3; \textit{D, E}) & 13.85 & 37.36 & 69.96 & 14.97 & 20.21 & 85.42 & 21.35 & 43.75 & 79.71 & 12.37 & 36.51 & 82.09 \\


\ours (GPT-4; top3)   &          16.66 &           42.85 &  \textbf{86.82} &           17.95 &           23.84 &            87.50 &           27.22 &           51.73 &  85.65 &           16.53 &           42.41 &           84.46 \\

\ours (GPT-4; \textit{D, E})     &  17.10 &  43.29 &           85.92 &  18.18 &  24.05 &  \textbf{88.74} &  27.98 &  52.41 &           85.27 &  17.02 &  43.02 &  \textbf{84.82} \\

\ours (GPT-4o; top3) &
17.77 & 43.38 & 85.46 &
18.08 & 23.98 & 88.15 &
28.62 & 52.53 & 85.87 &
16.92 & 42.93 & 84.52 \\
 
\ours (GPT-4o; \textit{D, E}) &
\textbf{18.02} & \textbf{43.46} & 86.19 &
\textbf{18.31} & \textbf{24.32} & 88.33 &
\textbf{29.50} & \textbf{53.16} & \textbf{86.03} &
\textbf{17.66} & \textbf{43.73} & 84.27 \\


\Xhline{3\arrayrulewidth}
\end{tabular}
\end{adjustbox}
\caption{Main results. The best scores in each pair are marked \textbf{bold}. Within parentheses in the proposed method, the parts separated by semicolons denote the merging module and the candidates used. \textit{D} and \textit{E} represent candidates obtained from direct translation and English pivot, respectively.} 
\label{tab: main results}
\end{table*}


\begin{table}[t]
\centering
\scriptsize
\renewcommand{\arraystretch}{0.97}
\begin{adjustbox}{width=\columnwidth, center}
\begin{tabular}{lccc}
\Xhline{3\arrayrulewidth}

\multirow{2}{*}{\textbf{Model}}  & \multicolumn{3}{c}{\textbf{Korean$\rightarrow$Italian}}\\ \cline{2-4}
 & BLEU & chrF++ & COMET \\ \hline\hline 
                  
\textit{\textbf{Candidate}}\\\hdashline[3pt/3pt]
 \nllb (direct)   & 16.27 & 41.14 & 84.60 \\
 \nllb (Portuguese pivot) & 13.13 & 37.57 & 83.21  \\  
 \nllb (Spanish pivot) & 13.87 & 38.47 & 83.71 \\
 \nllb (English pivot) & 14.77 & 39.39 & 81.48 \\

\Xhline{3\arrayrulewidth}
\end{tabular}
\end{adjustbox}
\caption{Quality of candidates used for the ensemble.}
\label{tab:candidates}
\end{table}



Table~\ref{tab: main results} reports the overall performance of \ours and other methods.
The results demonstrate that \ours consistently outperforms baselines across all language pairs.
While standalone NMT systems rely solely on their pre-trained knowledge, \ours explicitly leverages candidates during the ensemble.
Even when training an open-source LLM, Llama-3, we can enhance translation capability by utilizing candidates obtained via pivoting.
Compared to using LLMs for translation, we can improve performance with only the minimal cost of utilizing a small 0.6B model.
Table~\ref{tab:candidates} presents the quality of candidates utilized in the ensemble.
We will further elaborate with a case study, showing that \ours achieves better translations by leveraging candidates to capture subtle nuances of the source sentence.
We report experiments in a setting that does not use training data in Appendix~\ref{apdx: Evaluation without using Training Data} and experiments with other GPT models in Appendix~\ref{apdx: Experiments with Additional Models}.
The analysis of the proportion of top-$\textit{k}$ candidates and performance variation with $\textit{k}$ are in Appendix~\ref{apdx:selecting top-k}.

\minisection{Comparison with multi-model ensemble}
We compare \ours with \blender~\cite{llm-blender} and EVA~\cite{eva}, state-of-the-art ensemble methods utilizing multiple models.
\blender employs $N$ ($N$=11) LLMs for candidate generation, picks top-3 candidates with \textsc{PairRanker}, and fuses them with \textsc{GenFuser}.
EVA is a token-level ensemble method that leverages vocabulary alignment across multiple models.





Results in Table~\ref{tab: main results} show that \ours outperforms multi-model ensemble baselines by a considerable margin.
\blender was unable to improve outputs compared to its candidate LLMs in non-English translation tasks.
Additionally, LLMs used for generating candidates in \blender, such as Vicuna and Baize, exhibit subpar performance on given tasks.
These results align with recent work~\cite{alma}; open-source LLMs often struggle when not translating into English.

EVA is not only ineffective on the given tasks but also has several limitations inherent to its design as a token-level ensemble.
First, EVA is unable to use black-box models such as GPT-4.
Second, it is memory-intensive, as it requires loading multiple models into memory simultaneously.
While multi-model ensemble methods generate candidates using up to 11 LLMs (with sizes up to 13B), \ours generates candidates with a significantly smaller single model (0.6B), thereby greatly reducing computational overhead.





\begin{table*}[t]
\centering
\large
% \renewcommand{\arraystretch}{1.04}
\renewcommand{\arraystretch}{1.1}
\begin{adjustbox}{width=\textwidth, center}
\begin{tabular}{llc}
% \begin{tabular}{|p{1cm}|p{2cm}|p{5cm}|}
\Xhline{3\arrayrulewidth}
\textbf{\#} & \textbf{Type}& \textbf{Example}\\ \hline\hline


\multirow{8}{*}{\textbf{1}} & Source Sentence & 그래서 그동안 자문해왔습니다. 왜 우리는 질병들과 싸우기에 더 현명하고, 정확하며 더욱 적합한 ... \\ 
& & \textit{(English Translation: So we've been asking ourselves, why should we limit this smarter, more precise, more appropriate ...)} \\ 
& Target Reference & Quindi \hlc[lightblue]{mi sono chiesta}: perché dovremmo limitare questo modo intelligente, preciso, migliore ... \\ \cline{2-3} 
& Top-1 Candidate & Quindi \hlc[lightblue]{ci siamo chiesti}, perché dovremmo limitare questo modo più intelligente, più preciso e più appropriato ... \\
& Top-2 Candidate & Quindi \hlc[lightblue]{ci siamo chiesti}: perché dovremmo limitare questo metodo più intelligente, più preciso e più adatto ... \\ 
& Top-3 Candidate & Quindi nel corso di questo tempo, \hlc[pink]{abbiamo chiesto}: perché dovremmo limitare questo metodo più intelligente, più preciso e più adeguato ... \\ \cline{2-3}
& GPT-4 & Quindi, \hlc[pink]{abbiamo cercato consigli} fino ad ora. Perché dobbiamo limitare questo metodo, che è più intelligente, preciso e più adatto ... \\ 
& \ours  & Quindi \hlc[lightblue]{ci siamo chiesti}: perché dovremmo limitare questo metodo più intelligente, più preciso e più adatto ... \\ \hline\hline 


\multirow{8}{*}{\textbf{2}} & Source Sentence & 많은 사람들이 헤드폰을 사용하는데 이것의 문제점은 3가지 큰 건강 이슈를 가져온다는 것입니다. \\
& & \textit{(English Translation: The trouble with widespread headphone use is it brings three really big health issues.)}\\
& Target Reference & Il problema dell'utilizzo diffuso degli auricolari è che scatenano tre grandi \hlc[lightblue]{problemi} di salute. \\ \cline{2-3}
& Top-1 Candidate & Il problema è che molte persone usano le cuffie, e questo Porta a tre grandi \hlc[lightblue]{problemi} di salute. \\
& Top-2 Candidate & Il problema è che molte persone usano le cuffie, e questo Porta a tre grandi \hlc[lightblue]{problemi} di salute. \\ 
& Top-3 Candidate & Il problema è che molte persone usano cuffie, e questo è ciò che causa tre \hlc[lightblue]{problemi} di salute principali. \\ \cline{2-3}
& GPT-4 & Molte persone utilizzano le cuffie, ma il problema è che ciò comporta tre importanti \hlc[pink]{questioni} di salute. \\
& \ours  &  Il problema è che molte persone usano le cuffie, e questo porta a tre grandi \hlc[lightblue]{problemi} di salute. \\


\Xhline{3\arrayrulewidth} 
\end{tabular}
\end{adjustbox}
\caption{Case study. Parts with the same meanings as the source and mistranslated parts are highlighted in \hlc[lightblue]{blue} and \hlc[pink]{red}, respectively. English translation of the source sentence is obtained from another pair within the same dataset.}
\label{tab:case study}
\end{table*}



\begin{table}[t!]
\centering
\small
\renewcommand{\arraystretch}{0.97}
\begin{adjustbox}{width=\columnwidth, center}
\begin{tabular}{lcccccc}
\Xhline{3\arrayrulewidth}

\textbf{Model} & BLEU & chrF++ & COMET & BLEU & chrF++ & COMET \\ \hline\hline 

& \multicolumn{6}{c}{\textbf{Distant Language Pairs}} \\ \cline{2-7}

& \multicolumn{3}{c}{Portuguese$\rightarrow$Russian} & \multicolumn{3}{c}{Russian$\rightarrow$Portuguese} \\
NLLB&	25.17&	51.77&	90.12&	29.69&	55.81&	86.01 \\
GPT-4&	26.50&	52.76&	91.11&	25.51&	54.05&	86.69 \\
\ours&	\textbf{27.48}&	\textbf{53.49}&	\textbf{91.74}&	\textbf{30.82}&	\textbf{56.73}&	\textbf{88.37} \\ \hline

& \multicolumn{3}{c}{Dutch$\rightarrow$Russian} & \multicolumn{3}{c}{Russian$\rightarrow$Dutch} \\
NLLB&	22.95&	50.21&	89.92&	25.56&	53.60&	88.18 \\
GPT-4&	24.37&	51.32&	91.28&	24.46&	53.85&	88.58 \\
\ours&	\textbf{25.45}&	\textbf{52.16}&	\textbf{91.47}&	\textbf{28.05}&	\textbf{55.80}&	\textbf{89.35} \\ \hline

& \multicolumn{3}{c}{French$\rightarrow$Ukrainian} & \multicolumn{3}{c}{Ukrainian$\rightarrow$French} \\
NLLB&	14.58&	37.11&	82.99&	20.69&	44.04&	80.61 \\
GPT-4&	13.84&	39.03&	84.12&	23.30&	47.13&	83.43 \\
\ours&	\textbf{17.20}&	\textbf{39.82}&	\textbf{86.55}&	\textbf{24.35}&	\textbf{47.17}&\textbf{84.36} \\ \hline \hline

& \multicolumn{6}{c}{\textbf{Similar Language Pair (Romance)}} \\ \cline{2-7}

& \multicolumn{3}{c}{Spanish$\rightarrow$Portuguese} & \multicolumn{3}{c}{Portuguese$\rightarrow$Spanish} \\
NLLB&	32.38&	56.97&	86.88&	33.63&	57.61&	85.13 \\
GPT-4&	29.94&	55.26&	84.84&	34.70&	58.63&	86.75 \\
\ours&	\textbf{34.06}&	\textbf{58.11}&	\textbf{87.70}&	\textbf{36.03}&	\textbf{59.32}&	\textbf{86.92} \\ \hline

& \multicolumn{6}{c}{\textbf{Similar Language Pair (Slavic)}} \\ \cline{2-7}

& \multicolumn{3}{c}{Ukrainian$\rightarrow$Russian} & \multicolumn{3}{c}{Russian$\rightarrow$Ukrainian} \\
NLLB&	22.16&	45.41&	89.82&	19.67&	43.35&	89.87 \\
GPT-4&	24.41&	\textbf{47.59}&	89.43&	\textbf{22.42}&	\textbf{45.61}&	90.39 \\
\ours&	\textbf{24.64}&	47.51&	\textbf{90.78}&	22.09&	45.40&	\textbf{90.70} \\ 


\Xhline{3\arrayrulewidth}
\end{tabular}
\end{adjustbox}
\caption{Results on all language pairs.}
\label{tab:Results on all translation directions}
\end{table}




\minisection{Results on all language pairs}
To validate generalizability, we report the results for all language pairs we experimented with, including those within the same language family.
Distant pairs refer to languages that belong to different families and use different scripts, while similar pairs belong to the same family and share the same script.
The statistics for each language pair are in Appendix~\ref{sec:apdx_dataset statistics}.
Language pairs used in the experiments are as follows:

\begin{itemize}
    \item Distant language pairs: Portuguese$\leftrightarrow$Russian, Dutch$\leftrightarrow$Russian, and French$\leftrightarrow$Ukrainian

    \item Similar language pairs: Spanish$\leftrightarrow$Portuguese and Ukrainian$\leftrightarrow$Russian
\end{itemize}

 
Table~\ref{tab:Results on all translation directions} shows the results with the top-performing baselines, NLLB~\cite{nllb} and GPT-4~\cite{gpt4}.
\ours consistently exhibits superior performance compared to strong baselines on distant language pairs. 
Surprisingly, it also showed improvements in similar language pairs, such as Spanish$\leftrightarrow$Portuguese. 







\minisection{Case study}
We conduct a qualitative analysis to verify the impact of candidates on the final translation. 
We compare the output of GPT-4, used as the merging module, with \ours, which utilizes candidates for the ensemble process.
In Table~\ref{tab:case study}, we provide two examples along with the source and target sentences, as well as the top-3 candidates.


Through the first example, we can observe that \ours can appropriately translate homonyms within the context.
In Korean, ``자문'' has the meaning of both ``consultation'' and ``asking oneself''.
Considering the context, the expression should be translated to convey the meaning of ``asking ourselves'', as also shown in the English translation.
However, GPT-4 mistranslated the source sentence, converting the phrase ``자문해왔습니다'' to ``abbiamo cercato consigli'' (``seeking consultation from others'').
On the other hand, \ours accurately translates with the expression ``ci sono chiesti'' that means ``asking ourselves'', aligning well with the context by leveraging information from candidates.


In the second sample, GPT-4 translates the source sentence by translating the noun ``이슈'' into ``questioni''.
However, given the topic of discussing potential health risks, this translation does not fit well with the overall context.
By contrast, the ensemble result of \ours, generated using the identical model, improves translation quality by using a more accurate expression ``problemi'', despite having access to the same pre-trained knowledge.
Additionally, when more suitable expressions (e.g., ``ne vale la pena'') appear in candidates, \ours utilizes them to refine the final translation.


\subsection{Analysis}


\begin{table}[t]
\centering
\large
\begin{adjustbox}{width=\columnwidth, center}
\renewcommand{\arraystretch}{0.97}
\begin{tabular}{lcccc}
\Xhline{3\arrayrulewidth}
\multirow{2}{*}{\parbox{4cm}{\textbf{Candidate Generation}}} & \multirow{2}{*}{\parbox{1.45cm}{\textbf{\# Cand.}}}  &  \multicolumn{3}{c}{\textbf{Korean$\rightarrow$Italian}} \\ \cline{3-5} 
 & & BLEU & chrF++ & COMET \\ \hline\hline 

LLMs (of \blender) & 11 &  14.75 & 41.29 & 86.20\\ 

LLMs + \nllb (direct) & 12 & 16.08 & 42.38 & 86.22 \\ 
 
\nllb (pivot, ours) & 4 & \textbf{16.66} & \textbf{42.85} & \textbf{86.82}  \\

\Xhline{3\arrayrulewidth}
\end{tabular}
\end{adjustbox}
\caption{Comparison of candidate generation methods.}
\label{tab:comparison changing candidates}
\end{table}




\minisection{Candidate generation}
To validate the effectiveness of \ours, we conduct experiments only varying the candidate generation method, while using the same merging module, GPT-4.
We compare a candidate pool of size 4 obtained through pivot translation (\ours) with a candidate pool of size 11 obtained using 11 LLMs as employed in \blender~\cite{llm-blender}.

As shown in Table~\ref{tab:comparison changing candidates}, the proposed method of generating candidates through pivot translation achieves the highest performance, despite using the smallest candidate pool.
From the perspective of direct translation in NLLB, leveraging 3 candidates obtained through pivot translation yields higher scores than incorporating candidates generated by 11 LLMs.
These results demonstrate that using stable-quality candidates generated by a single model via pivot translation outperforms the use of multiple models with performance disparities.

\begin{table}[!t]
\centering
% \small
\renewcommand{\arraystretch}{0.95}
\begin{adjustbox}{width=\columnwidth, center}
\begin{tabular}{lccc}
\Xhline{3\arrayrulewidth}

\multirow{2}{*}{\textbf{Model}}  & \multicolumn{3}{c}{\textbf{Korean$\rightarrow$Italian}}\\ \cline{2-4}
 & BLEU & chrF++ & COMET \\ \hline\hline 
                  
\textit{\textbf{Standalone NMT System}}\\\hdashline[3pt/3pt]
\nllb~\cite{nllb}   & 16.27 & 41.14 & 84.60 \\ \hline

\textit{\textbf{Encoder-Decoder}}\\\hdashline[3pt/3pt]
{\ours} (FiD)      & 13.74 & 36.78 & 78.98 \\
{\ours} (TRICE)    & 15.89 & 41.98 & 84.06 \\\hline
\textit{\textbf{LLM-based}}\\\hdashline[3pt/3pt]
{\ours} (\textsc{GenFuser}) & 14.56 & 39.32 & 80.07\\
{\ours} (GPT-4)    & \textbf{16.66} & \textbf{42.85} & \textbf{86.82}\\

\Xhline{3\arrayrulewidth}
\end{tabular}
\end{adjustbox}
\caption{Evaluation of merging module variants.}
\label{tab:merging module results}
\end{table}
\minisection{Candidate aggregation}
We first investigate whether \ours shows improvement when utilizing other merging modules.
As detailed in \S\ref{sec:candidate aggregation}, we run experiments with three architectures: FiD~\cite{fid}, TRICE~\cite{trice}, and \textsc{GenFuser}~\cite{llm-blender}.
When implementing FiD, we replace the backbone of FiD to $\text{mT5}_{\texttt{BASE}}$~\cite{xue-etal-2021-mt5}.
TRICE is a method proposed for multi-source translation. 
Since TRICE was not originally intended for ensemble use, we repurposed it by training on the following two tasks:
The first task is the original translation which converts source sentences into target sentences.
The second task is refining candidates that are paired with target references.
In the case of \trice, only the highest quality candidates, which are the directly translated ones, are used due to its architecture.
\fid and \textsc{GenFuser} use top-3 candidates.


Table~\ref{tab:merging module results} shows that the ensemble methods using encoder-decoder architectures and \textsc{GenFuser} do not yield improved results.
These methods struggle to leverage additional information from the candidates and, consequently, do not enhance performance. 
In contrast, using GPT-4 as the merging module leads to better performance compared to the standalone NMT system.
We also compare ranking methods COMETkiwi and \textsc{PairRanker}~\cite{llm-blender}.
While the performance is comparable, considering the efficiency factor, we opt for COMETkiwi.
Detailed experiments about the ranking method are in Appendix~\ref{sec:apdx_cands selecting method}.



\minisection{Comparison with selection-based ensemble}
With a selection-based ensemble, we can choose one of the existing candidates as the final translation, rather than generating a new one.
In this experiment, we compare our approach with a selection-based ensemble by selecting the top-1 translation using \textsc{PairRanker}~\cite{llm-blender} and COMETkiwi~\cite{rei2022cometkiwi}.
Additionally, we report results with an ideal case: selecting top-1 by considering references as well, which are not available in practice.
The ideal top-1 is selected by reference-based COMET~\cite{comet22}.


As shown in Table~\ref{tab:selection-based}, \ours exhibits superior performance compared to the selection-based ensemble methods.
Even when we leverage reference-based COMET, which is impossible in real-world scenarios due to the necessity for references, \ours outperforms it in chrF++ and COMET.
These results indicate that performing a generation-based ensemble with pivoting can effectively produce final translations that surpass those selected from the existing candidate pool.




\begin{table}[t]
\centering
\small
\begin{adjustbox}{width=\columnwidth, center}
\renewcommand{\arraystretch}{0.97}
\begin{tabular}{llccc}
\Xhline{3\arrayrulewidth}
\multirow{2}{*}{\textbf{Category}} & \multirow{2}{*}{\textbf{Method}} &  \multicolumn{3}{c}{\textbf{Korean$\rightarrow$Italian}} \\ \cline{3-5} 
& & BLEU & chrF++ & COMET \\ \hline\hline 
\multirow{3}{*}{\makecell[l]{Selection-based \\ (top-1)}} & \textsc{PairRanker} & 15.61 & 40.62 & 84.46 \\
& COMETkiwi & 15.61 & 40.71 & 84.10 \\
& COMET* (ideal) & \textbf{17.77} & 42.81 & 84.83\\ \hline
Generation-based & \ours & \underline{16.66} & \textbf{\underline{42.85}} & \textbf{\underline{86.82}} \\
\Xhline{3\arrayrulewidth}
\end{tabular}
\end{adjustbox}
\caption{Comparison with selection-based ensemble. Note that COMET* is the ideal baseline, as it requires references. Best scores including COMET* are \textbf{bolded}, while best scores excluding it are \underline{underlined}.}
\label{tab:selection-based}
\end{table}


\section{Conclusions}\label{sc5}

This study explores the formulation of a CL framework to explicitly model the unsupervised learning of semantic changes in multimodal RS images. To address the challenges posed by spectral variations, spatial misalignment, insignificant changes and multimodal heterogeneity, an S2C learning framework is developed for CD of HR RS images.
It consists of two novel CL paradigms, i.e., the CSC and CTC, which are both trained with consistency regularization to enhance robustness against different types of temporal noise. Notably, within the CTC paradigm, we present an innovative multi-temporal triplet learning strategy that addresses the existing gap in explicit difference learning. In addition, a set of novel techniques are developed to translate the VFM semantics into CD results, including grid sparsity regularization, negative cosine embedding of change probability, and an IoU matching-based refinement algorithm. Furthermore, with minimal adjustments, the S2C framework can be applied on unsupervised MMCD. Joint learning of semantic differences and consistencies enables S2C to align semantic representations across different image modalities, thus mapping domain-invariant changes. 

Experimental results reveal that the proposed S2C obtains significant accuracy improvements over the current SOTA, with advantages in $F_1$ exceeding 31\%, 9\%, 23\%, and 15\% across the four benchmark CD datasets, respectively. Additionally, S2C is efficient in training, requiring only a minimal quantity of 5 unlabeled sample pairs on the considered data set while inducing only a marginal reduction in accuracy. This highlights its potential for efficient deployment in practical applications. S2C can also be integrated into standard supervised CD methods to improve their accuracy, as it is essentially a collection of CD-specific training paradigms and post-processing algorithms that are independent of any particular DNN architectures.

A remaining limitation in S2C, as well as other UCD methods in the literature, is their lack of awareness of the particular applicational contexts. For example, S2C generates a substantial number of false alarms when applied to the Levir dataset, which focuses solely on building changes. To address this limitation, future studies are encouraged to explore language-driven UCD with the injection of user intents. Additionally, the S2C framework possesses the potential to be extended to more intricate UCD tasks, such as multi-class CD and time-series CD, which is left for further research investigations.
%%%%%%%%%%%%%%%%%%%%%%%%%%%%%%%%%%%%%%%%%%

\bibliographystyle{IEEEtran}
\bibliography{refs}

\begin{comment}
\begin{IEEEbiography}[{\includegraphics[width=1in,height=1.25in,clip,keepaspectratio]{Photo/LeiDing1.jpg}}]{Lei Ding}
received the MS’s degree in Photogrammetry and Remote Sensing from the Information Engineering University (Zhengzhou, China), and the PhD \textit{(cum laude)} in Communication and Information Technologies from the University of Trento (Trento, Italy). He is now a lecturer at the Strategic Force Information Engineering University. Since 2024, he has also been a postdoctoral researcher at the Aerospace Information Research Institute, Chinese Academy of Sciences. His research interests are related to the recognition and localization of Remote Sensing Data. He has published more than 30 research articles on international journals including \textit{IEEE TIP}, \textit{IEEE TGRS} and \textit{IEEE GRSL}.
\end{IEEEbiography}

\begin{IEEEbiography}[{\includegraphics[width=1in,height=1.25in,clip,keepaspectratio]{Photo/Zuoxibing.jpg}}]{Xibing Zuo}
received the M.S. degree in Surveying and mapping engineering from the Information Engineering University, Zhengzhou, China, in 2022, where he is currently pursuing the Ph.D. degree in surveying and mapping science and technology.\\
His research interests include machine learning and remote sensing image processing.
\end{IEEEbiography}

\begin{IEEEbiography}[{\includegraphics[width=1in,height=1.25in,clip,keepaspectratio]{Photo/Danfeng_Hong.png}}]{Danfeng Hong} (Senior Member, IEEE) received the Dr-Ing degree \textit{(summa cum laude)} in signal processing in earth observation (SiPEO) from the Technical University of Munich (TUM), Munich, Germany, in 2019. Since 2022, he has been a full professor with the Aerospace Information Research Institute, Chinese Academy of Sciences. Before joining CAS, he was a research scientist and led a Spectral Vision Working Group with the Remote Sensing Technology Institute (IMF), German Aerospace Center (DLR), Oberpfaffenhofen, Germany. He was also an adjunct scientist with GIPSA-lab, Grenoble INP, CNRS, Univ. Grenoble Alpes, Grenoble, France. His research interests include artificial intelligence, multimodal remote sensing, large foundation models, hyperspectral imaging, and large-scale Earth observation. He is an associate editor for \textit{IEEE Transactions on Geoscience and Remote Sensing (TGRS)} and the editorial board member of \textit{Information Fusion} and \textit{ISPRS Journal of Photogrammetry and Remote Sensing}. He was a recipient of the Best Reviewer Award of the IEEE TGRS in 2021 and 2022, the Best Reviewer Award of the IEEE JSTARS in 2022, the Jose Bioucas Dias Award for recognizing the outstanding paper with WHISPERS in 2021, the Remote Sensing Young Investigator Award in 2022, the IEEE GRSS Early Career Award in 2022, and a Highly Cited Researcher (Clarivate Analytics) in 2022 and 2023.
\end{IEEEbiography}

\begin{IEEEbiography}[{\includegraphics[width=1in,height=1.25in,clip,keepaspectratio]{Photo/HaitaoGuo}}]{Haitao Guo}
received his M.S. degree and Ph.D. degree from Information Engineering University, China, in 2002 and 2008, respectively.
He is currently an associate professor of photogrammetry and remote sensing at Information Engineering University, Zhengzhou, China, where he teaches digital photogrammetry and geopositioning for remote sensing imagery. His current research interests are in the areas of deep learning for image interpretation and change detection, geopositioning without ground control points for satellite imagery. 
\end{IEEEbiography}


\begin{IEEEbiography}[{\includegraphics[width=1in,height=1.25in,clip,keepaspectratio]{Photo/LuJun.jpg}}]{Jun Lu} received the M.S. and Ph.D. degree from Information Engineering University, Zhengzhou, China, in 2008 and 2015, respectively. He is currently an Associate Professor at the Information Engineering University. His research interests are related to deep learning for image interpretation and cross-view geolocation in the domains of Photogrammetry and Remote Sensing.
\end{IEEEbiography}

\begin{IEEEbiography}[{\includegraphics[width=1in,height=1.25in,clip,keepaspectratio]{Photo/GongZhihui.jpg}}]{Zhihui Gong} obtained the master's degree in Photogrammetry and Remote Sensing at the Information Engineering University, Zhengzhou, China, in 2000. He is now a professor at the Information Engineering University and has been Engaged in teaching and research work in remote sensing science and technology for over 30 years. His research interests include remote sensing image localization, remote sensing image intelligence, and intelligent unmanned systems.
\end{IEEEbiography}


\begin{IEEEbiography}[{\includegraphics[width=1in,height=1.25in,clip,keepaspectratio]{Photo/Lorenzo}}]{Lorenzo Bruzzone}
(S'95-M'98-SM'03-F'10) received the Laurea (M.S.) degree in electronic engineering (\emph{summa cum laude}) and the Ph.D. degree in telecommunications from the University of Genoa, Italy, in 1993 and 1998, respectively. \\
He is currently a Full Professor of telecommunications at the University of Trento, Italy, where he teaches remote sensing, radar, and digital communications. Dr. Bruzzone is the founder and the director of the Remote Sensing Laboratory in the Department of Information Engineering and Computer Science, University of Trento. His current research interests are in the areas of remote sensing, radar and SAR, signal processing, machine learning and pattern recognition. He promotes and supervises research on these topics within the frameworks of many national and international projects. He is the Principal Investigator of many research projects. Among the others, he is the Principal Investigator of the \emph{Radar for icy Moon exploration} (RIME) instrument in the framework of the \emph{JUpiter ICy moons Explorer} (JUICE) mission of the European Space Agency. He is the author (or coauthor) of 215 scientific publications in referred international journals (154 in IEEE journals), more than 290 papers in conference proceedings, and 21 book chapters. He is editor/co-editor of 18 books/conference proceedings and 1 scientific book. He was invited as keynote speaker in more than 30 international conferences and workshops. Since 2009 he is a member of the Administrative Committee of the IEEE Geoscience and Remote Sensing Society (GRSS). 

Dr. Bruzzone was a Guest Co-Editor of many Special Issues of international journals. He is the co-founder of the IEEE International Workshop on the Analysis of Multi-Temporal Remote-Sensing Images (MultiTemp) series and is currently a member of the Permanent Steering Committee of this series of workshops. Since 2003 he has been the Chair of the SPIE Conference on Image and Signal Processing for Remote Sensing. He has been the founder of the IEEE Geoscience and Remote Sensing Magazine for which he has been Editor-in-Chief between 2013-2017. Currently he is an Associate Editor for the IEEE Transactions on Geoscience and Remote Sensing. He has been Distinguished Speaker of the IEEE Geoscience and Remote Sensing Society between 2012-2016. His papers are highly cited, as proven form the total number of citations (more than 27000) and the value of the h-index (78) (source: Google Scholar).
\end{IEEEbiography}
\end{comment}

\end{document}
