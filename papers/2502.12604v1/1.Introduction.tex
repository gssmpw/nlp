
\section{Introduction}\label{sc1}

Change detection (CD) represents one of the most earliest and widely utilized technologies in remote sensing (RS)~\cite{bruzzone2009domain, wu2023fully}. Different from CD in surveillance \cite{lanza2011statistical}, streetview\cite{taneja2015geometric}, industrial \cite{roth2022towards} and medical \cite{li2020siamese} applications that mainly address image changes taken by fixed or mounted cameras, CD in RS leverages over-head platforms to identify and segment change regions on the Earth surface at different times \cite{bruzzone2012novel}. It has wide applications in urban development management \cite{benedek2011building}, land cover monitoring \cite{robin2010contrario}, damage assessment \cite{brunner2010earthquake}, etc. Recently, great advances have been achieved in RS CD utilizing deep learning (DL) techniques. State-Of-The-Art (SOTA) methodologies have obtained accuracy levels surpassing 90\% in the $F_1$ metric across various benchmark datasets for CD \cite{chen2021remote, ding2024samcd}. However, typical DL-based CD methods require large amounts of high-quality labeled data for training, which are difficult to collect due to the scarcity of change samples. Consequently, deployment of CD algorithms in real-world applications still faces significant challenges.

\begin{figure}[!t]
\centering
    \includegraphics[width=1\linewidth]{illu_pic/challenges_4.png}
    \caption{The major types of temporal noise in CD on HR RSIs include: (a) spectral/seasonal variations, (b) spatial misalignment, (c) insignificant changes, and (d) multimodal heterogeneity.}
\label{fig.challenge}
\end{figure}

To reduce dependence on training data, an increasing number of studies have been conducted on unsupervised change detection (UCD) in recent years. However, most of these investigations focus on UCD in medium-resolution multitemporal images \cite{chen2021self}. UCD in HR RS images presents greater challenges due to the increased spatial, temporal, and spectral complexity \cite{bruzzone2012novel}. Fig.\ref{fig.challenge} illustrates the main instances of noise encountered in the CD of RS images, including \textit{i) Spectral variations.} This can be attributed to seasonal variations or differences in imaging sensors and illumination conditions \cite{hong2018augmented}. \textit{ii) Spatial misalignment}. This may arise from varying image acquisition angles, optical distortion, or errors in image registration. \textit{iii) Insignificant changes.} In CD applications, only specific semantic changes are of interest, such as building changes in urban management and cropland changes in agriculture monitoring. Certain temporary randomness changes, such as the parked vehicles depicted in Fig. \ref{fig.challenge}(c), are often deemed as irrelevant noise. iv) Multimodal heterogeneity. Multi-temporal observations captured by different sensors are likely to exhibit substantial differences in radiometric patterns. Consequently, deep neural networks (DNNs) face significant challenges in learning to differentiate between relevant semantic changes and temporal radiometric differences in an unsupervised manner.
 
In recent studies, contrastive learning (CL) \cite{sohn2020fixmatch} and Visual Foundation Models (VFMs) \cite{Kirillov2023Segment} have been identified as two effective methodologies to mitigate data dependence in CD. The former leverages the intrinsic consistency within data, while the latter incorporates external knowledge to learn common semantic representations. Significant advances have been made in a semisupervised CD on HR RSIs leveraging CL \cite{yang2023revisiting, chen2022self} and VFMs \cite{ding2024samcd, zheng2024segment}. However, UCD in HR RSIs using either CL or VFM remains challenging due to the inherent spatio-temporal complexity within this task. In CL-based CD, various studies follow a consistency regularization framework \cite{sohn2020fixmatch}. Although this approach significantly enhances the generalization and robustness of feature representations, it still requires a certain proportion of training data. VFM-based CDs often employ VFMs, such as the Segment Anything Model \cite{Kirillov2023Segment}, to exploit semantic features and decode change masks \cite{zheng2024segment}. Nonetheless, a significant domain gap exists between the training domains of VFM and RS images \cite{ji2024segment}, adversely affecting their recognition capabilities. Furthermore, translating semantic features into CD results still necessitates supervised learning.

This paper aims to investigate the integration of CL and VFM to accomplish UCD in HR RS images. It is observed that these two methodologies effectively complement each other: CL provides self-supervised training objectives essential for adapting VFMs to the RS domain and for mapping the changes. Conversely, VFMs embed pixel-level semantic representations, a capability typically absent within conventional CL frameworks. Furthermore, we extend the existing CL frameworks to incorporate the spatio-temporal correlations unique to CD. Instead of focusing on learning robust and consistent embeddings, our approach also models the consistency and differences across multi-temporal observations.

The major technical contributions in this study can be summarized as follows:

\begin{itemize}
    \item Formulating a UCD framework that explicitly models the unsupervised learning of semantic changes. The proposed S2C framework, to the best of our knowledge, is also the first work that incorporates VFM into CL in the context of CD. It integrates multiple innovative designs and technologies, including CL, VFM, and Low-Rank Adaptation (LoRA), as well as an IoU-based refinement algorithm.
    \item Introducing two novel CL paradigms for UCD: Consistency-regularized Temporal Contrast (CTC) and Consistency-regularized Spatial Contrast (CSC), to capture differences and consistency across multi-temporal observations. Notably, the CTC presents a novel triplet learning strategy to model semantic changes, complementing literature studies that predominantly focus on similarities.
    \item Proposing a grid sparsity regularization that aims at promoting sparse and compact change mapping. The sparse calculations are executed at grid scales to avoid training collapse and ensure efficiency. This training objective is also applicable to other segmentation tasks that require sparse or compact results.
    \item Further extending S2C to unsupervised Multimodal Change Detection (MMCD). The inherent semantic alignment mechanism makes the S2C independent of the specific imaging modalities. With modifications to the image encoders and learning process, cross-modality change modeling is established within the S2C framework.
\end{itemize}

The developed methodology demonstrates significant improvements over the state-of-the-art (SOTA) methods. Specifically, the accuracy improvements are approximately $31\%$, $9\%$, and $23\%$ in $F_1$ across three HR benchmark datasets for CD. Moreover, the application of the proposed S2C learning framework to the Wuhan MMCD benchmark demonstrated an increase of $F_1$ by $15\%$ over the SOTA methods. These experimental findings establish the proposed S2C framework as a robust, accurate, sample-efficient, and modality-independent framework for UCD in RS images.