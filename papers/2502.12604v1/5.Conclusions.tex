\section{Conclusions}\label{sc5}

This study explores the formulation of a CL framework to explicitly model the unsupervised learning of semantic changes in multimodal RS images. To address the challenges posed by spectral variations, spatial misalignment, insignificant changes and multimodal heterogeneity, an S2C learning framework is developed for CD of HR RS images.
It consists of two novel CL paradigms, i.e., the CSC and CTC, which are both trained with consistency regularization to enhance robustness against different types of temporal noise. Notably, within the CTC paradigm, we present an innovative multi-temporal triplet learning strategy that addresses the existing gap in explicit difference learning. In addition, a set of novel techniques are developed to translate the VFM semantics into CD results, including grid sparsity regularization, negative cosine embedding of change probability, and an IoU matching-based refinement algorithm. Furthermore, with minimal adjustments, the S2C framework can be applied on unsupervised MMCD. Joint learning of semantic differences and consistencies enables S2C to align semantic representations across different image modalities, thus mapping domain-invariant changes. 

Experimental results reveal that the proposed S2C obtains significant accuracy improvements over the current SOTA, with advantages in $F_1$ exceeding 31\%, 9\%, 23\%, and 15\% across the four benchmark CD datasets, respectively. Additionally, S2C is efficient in training, requiring only a minimal quantity of 5 unlabeled sample pairs on the considered data set while inducing only a marginal reduction in accuracy. This highlights its potential for efficient deployment in practical applications. S2C can also be integrated into standard supervised CD methods to improve their accuracy, as it is essentially a collection of CD-specific training paradigms and post-processing algorithms that are independent of any particular DNN architectures.

A remaining limitation in S2C, as well as other UCD methods in the literature, is their lack of awareness of the particular applicational contexts. For example, S2C generates a substantial number of false alarms when applied to the Levir dataset, which focuses solely on building changes. To address this limitation, future studies are encouraged to explore language-driven UCD with the injection of user intents. Additionally, the S2C framework possesses the potential to be extended to more intricate UCD tasks, such as multi-class CD and time-series CD, which is left for further research investigations.