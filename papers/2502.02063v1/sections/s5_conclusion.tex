\vspace{-5pt}
\section{Conclusion} \label{sec:conclusion}
\vspace{-3pt}

% We propose {\modulename}, a simple yet effective composite aware semantic injection method, that is agnostic to various genres of text-to-motion generation models and motion representations, spanning across autoregressive to denoising diffusion and from descritized motion token to noisy motion sequence.
We propose {\modulename}, a simple yet effective method for semantic injection that works with various text-to-motion models and representations, from autoregressive to denoising diffusion and discretized motion tokens to continuous raw motion sequences.
Our experiments suggest that {\modulename} consistently improves the motion quality and strengthens the text-motion alignment across several state-of-the-art models on HumanML3D and KIT benchmarks.
The method shows promise in enhancing long-term human motion generation. 
Notably, it enables more precise motion control through input text compared to fixed-length semantic injection approaches.


\textbf{Limitations and Future Work.} While our semantic injection method shows potential for processing very long text inputs for zero-shot long-term motion generation, it sill relies on motion blending techniques such as DoubleTake.
% it does not significantly improve the performance without additional motion blending methods such as DoubleTake.
This limitation largely arises from the training dataset itself, which lacks long text and motion samples. 
Future work would focus on curating datasets with extended text-motion pairs and developing techniques to effectively leverage such data.
Though {\modulename} preserves the composite nature for text injection, it shows limited improvements with methods like MLD~\cite{chen2023executing} that encode motion as a fixed-length latent vector.
This compression itself constrains the learning of fine-grained text-motion correspondence. Please refer to Supp. Mat. for more discussion.





