%%%%%%%% ICML 2025 EXAMPLE LATEX SUBMISSION FILE %%%%%%%%%%%%%%%%%

\documentclass{article}

% Recommended, but optional, packages for figures and better typesetting:
\usepackage{microtype}
\usepackage{graphicx}
\usepackage{subfigure}
\usepackage{booktabs} % for professional tables

% hyperref makes hyperlinks in the resulting PDF.
% If your build breaks (sometimes temporarily if a hyperlink spans a page)
% please comment out the following usepackage line and replace
% \usepackage{icml2025} with \usepackage[nohyperref]{icml2025} above.
\usepackage{hyperref}


% Attempt to make hyperref and algorithmic work together better:
\newcommand{\theHalgorithm}{\arabic{algorithm}}

% Use the following line for the initial blind version submitted for review:
% \usepackage{icml2025}

% If accepted, instead use the following line for the camera-ready submission:
% \usepackage[accepted]{icml2025}
\usepackage[arxiv]{icml2025}

% For theorems and such
\usepackage{amsmath}
\usepackage{amssymb}
\usepackage{mathtools}
\usepackage{amsthm}

% if you use cleveref..
\usepackage[capitalize,noabbrev]{cleveref}

%%%%%%%%%%%%%%%%%%%%%%%%%%%%%%%%
% THEOREMS
%%%%%%%%%%%%%%%%%%%%%%%%%%%%%%%%
\theoremstyle{plain}
\newtheorem{theorem}{Theorem}[section]
\newtheorem{proposition}[theorem]{Proposition}
\newtheorem{lemma}[theorem]{Lemma}
\newtheorem{corollary}[theorem]{Corollary}
\theoremstyle{definition}
\newtheorem{definition}[theorem]{Definition}
\newtheorem{assumption}[theorem]{Assumption}
\theoremstyle{remark}
\newtheorem{remark}[theorem]{Remark}

\DeclareMathOperator{\mhsa}{MHSA}
\DeclareMathOperator{\mhca}{MHCA}


% Todonotes is useful during development; simply uncomment the next line
%    and comment out the line below the next line to turn off comments
%\usepackage[disable,textsize=tiny]{todonotes}
\usepackage[textsize=tiny]{todonotes}

%
% --- inline annotations
%
\newcommand{\red}[1]{{\color{red}#1}}
\newcommand{\todo}[1]{{\color{red}#1}}
\newcommand{\TODO}[1]{\textbf{\color{red}[TODO: #1]}}
% --- disable by uncommenting  
% \renewcommand{\TODO}[1]{}
% \renewcommand{\todo}[1]{#1}



\newcommand{\VLM}{LVLM\xspace} 
\newcommand{\ours}{PeKit\xspace}
\newcommand{\yollava}{Yo’LLaVA\xspace}

\newcommand{\thisismy}{This-Is-My-Img\xspace}
\newcommand{\myparagraph}[1]{\noindent\textbf{#1}}
\newcommand{\vdoro}[1]{{\color[rgb]{0.4, 0.18, 0.78} {[V] #1}}}
% --- disable by uncommenting  
% \renewcommand{\TODO}[1]{}
% \renewcommand{\todo}[1]{#1}
\usepackage{slashbox}
% Vectors
\newcommand{\bB}{\mathcal{B}}
\newcommand{\bw}{\mathbf{w}}
\newcommand{\bs}{\mathbf{s}}
\newcommand{\bo}{\mathbf{o}}
\newcommand{\bn}{\mathbf{n}}
\newcommand{\bc}{\mathbf{c}}
\newcommand{\bp}{\mathbf{p}}
\newcommand{\bS}{\mathbf{S}}
\newcommand{\bk}{\mathbf{k}}
\newcommand{\bmu}{\boldsymbol{\mu}}
\newcommand{\bx}{\mathbf{x}}
\newcommand{\bg}{\mathbf{g}}
\newcommand{\be}{\mathbf{e}}
\newcommand{\bX}{\mathbf{X}}
\newcommand{\by}{\mathbf{y}}
\newcommand{\bv}{\mathbf{v}}
\newcommand{\bz}{\mathbf{z}}
\newcommand{\bq}{\mathbf{q}}
\newcommand{\bff}{\mathbf{f}}
\newcommand{\bu}{\mathbf{u}}
\newcommand{\bh}{\mathbf{h}}
\newcommand{\bb}{\mathbf{b}}

\newcommand{\rone}{\textcolor{green}{R1}}
\newcommand{\rtwo}{\textcolor{orange}{R2}}
\newcommand{\rthree}{\textcolor{red}{R3}}
\usepackage{amsmath}
%\usepackage{arydshln}
\DeclareMathOperator{\similarity}{sim}
\DeclareMathOperator{\AvgPool}{AvgPool}

\newcommand{\argmax}{\mathop{\mathrm{argmax}}}     




% The \icmltitle you define below is probably too long as a header.
% Therefore, a short form for the running title is supplied here:
% \icmltitlerunning{Rethinking the Semantic Injection for Text to Motion Generation}
\icmltitlerunning{{\modulename}: Composite Aware Semantic Injection for Text to Motion Generation}
% \icmltitlerunning{Rethinking the Injection of CLIP Semantics for Text to Motion Generation}
% \icmltitlerunning{MODELNAME: An Attention-based Semantic Injection for Text to Motion Generation}


\begin{document}

\twocolumn[
%%%% Some candidates for title, should tailor to the taste of ICML reviewers
% \icmltitle{{\modulename}: Rethinking the Semantic Injection for Text to Motion Generation}
\icmltitle{{\modulename}: Composite Aware Semantic Injection for Text to \\ Motion Generation}
% \icmltitle{Rethinking the Injection of CLIP Semantics for Text to Motion Generation}
% \icmltitle{MODELNAME: An Attention-based Semantic Injection for Text to Motion Generation}


% It is OKAY to include author information, even for blind
% submissions: the style file will automatically remove it for you
% unless you've provided the [accepted] option to the icml2025
% package.

% List of affiliations: The first argument should be a (short)
% identifier you will use later to specify author affiliations
% Academic affiliations should list Department, University, City, Region, Country
% Industry affiliations should list Company, City, Region, Country

% You can specify symbols, otherwise they are numbered in order.
% Ideally, you should not use this facility. Affiliations will be numbered
% in order of appearance and this is the preferred way.
\icmlsetsymbol{equal}{*}

\begin{icmlauthorlist}
\icmlauthor{Che-Jui Chang}{equal,ru,amzn}
\icmlauthor{Qingze (Tony) Liu}{equal,ru}
\icmlauthor{Honglu Zhou}{crm}
\icmlauthor{Vladimir Pavlovic}{ru}
\icmlauthor{Mubbasir Kapadia}{rblx}
\end{icmlauthorlist}

\icmlaffiliation{ru}{Rutgers University}
\icmlaffiliation{amzn}{Amazon}
\icmlaffiliation{crm}{Salesforce AI Research}
\icmlaffiliation{rblx}{Roblox}

\icmlcorrespondingauthor{Qingze (Tony) Liu}{tony.liu@rutgers.edu}
\icmlcorrespondingauthor{Che-Jui Chang}{chejui.chang@rutgers.edu}

% You may provide any keywords that you
% find helpful for describing your paper; these are used to populate
% the "keywords" metadata in the PDF but will not be shown in the document
\icmlkeywords{Human Motion Generation, Text to Motion Generation, Semantic Injection, Composite Aware Semantic Injection, T2M, {\modulename}}

\vskip 0.3in
]

% this must go after the closing bracket ] following \twocolumn[ ...

% This command actually creates the footnote in the first column
% listing the affiliations and the copyright notice.
% The command takes one argument, which is text to display at the start of the footnote.
% The \icmlEqualContribution command is standard text for equal contribution.
% Remove it (just {}) if you do not need this facility.

% \printAffiliationsAndNotice{}  % leave blank if no need to mention equal contribution
\printAffiliationsAndNotice{\icmlEqualContribution} % otherwise use the standard text.

\begin{abstract}
% \vspace{-2pt}
Recent advances in generative modeling and tokenization have driven significant progress in text-to-motion generation, leading to enhanced quality and realism in generated motions. However, effectively leveraging textual information for conditional motion generation remains an open challenge. We observe that current approaches, primarily relying on fixed-length text embeddings (e.g., CLIP) for global semantic injection, struggle to capture the composite nature of human motion, resulting in suboptimal motion quality and controllability. To address this limitation, we propose the Composite Aware Semantic Injection Mechanism ({\modulename}), comprising a composite aware text encoder and a text-motion aligner that learns the dynamic correspondence between text and motion tokens. 
Notably, {\modulename} is model and representation-agnostic, readily integrating with both autoregressive and diffusion-based methods. Experiments on HumanML3D and KIT benchmarks demonstrate that {\modulename} consistently improves motion quality, text-motion alignment, and retrieval scores across state-of-the-art methods. Qualitative analyses further highlight the superiority of our composite aware approach over fixed-length semantic injection, enabling precise motion control from text prompts and stronger generalization to unseen text inputs.
Our code is available at our
project page: \url{https://cjerry1243.github.io/casim_t2m}.



\end{abstract}


\section{Introduction}\label{sec:intro}
\begin{figure}[h]  
    \centering
    \includegraphics[width=1.0\textwidth]{figures/fig1_new.pdf}  
    \caption{
    Web AI agents exhibit a significantly higher jailbreak rate (46.6\%) compared to standalone LLMs (0\%), highlighting their increased vulnerability in real-world deployment.}
    \label{fig:experiment_email}
\end{figure}

Recent advancements in Large Language Models (LLMs) have demonstrated impressive reasoning capabilities and proficiency in solving complex problems. These capabilities are increasingly being extended to multi-step tasks, driving the evolution of LLM-based AI agent systems \citep{shen2024scribeagentspecializedwebagents,yang2024agentoccamsimplestrongbaseline,yang2024swe,putta2024agent,zhang2024webpilot}. 
One such system is the Web (browser) AI agent, which integrates an LLM with software tools and APIs to execute sequences of actions aimed at achieving specific goals within a web environment.
These agents leverage LLM capabilities for planning \citep{zheng2024naturalplanbenchmarkingllms}, reflection \citep{Pallagani_2024}, and effective tool utilization \citep{yao2024taubenchbenchmarktoolagentuserinteraction,shi2024learningusetoolscooperative}, enabling more autonomous and adaptive web-based interactions.


Many previous studies \citep{openhands,shen2024scribeagentspecializedwebagents,su2025learnbyinteractdatacentricframeworkselfadaptive} have highlighted significant advancements in autonomous web agents. 
However, despite their promising potential, their safety and security vulnerabilities have not yet been systematically studied. 
Given their direct integration with web browsers, these agents could be exploited to distribute malware or send phishing Emails to extract personal information, posing serious security risks (as shown in Fig. \ref{fig:experiment_email}). 


In this study, we highlight the heightened vulnerability of Web AI agent frameworks to jailbreaking compared to traditional LLMs. Through comprehensive experiments, we demonstrate that web agents, by design, exhibit a significant higher susceptibility to following malicious commands due to fundamental component-level differences from standalone LLMs. 
Notably, while a standalone LLM (such as a regular chatbot)
refuses malicious requests with a 0\% success rate, the Web AI agent follows them at a rate of 46.6\% (Fig. \ref{fig:experiment_email}).


Importantly, we identify three primary factors contributing to the increased vulnerabilities of Web AI agents:
\textbf{(1)} Directly embedding user input into the LLM system prompt, 
\textbf{(2)} Generating
actions in a multi-turn manner, and 
\textbf{(3)} Processing observations and action histories, which increases the likelihood of executing harmful instructions and weakens the system’s ability to assess risks. 
\textbf{Additionally,} we find that mock-up testing environments may inadvertently distort security evaluations by oversimplifying real-world interactions, potentially leading to misleading conclusions about an agent's robustness.



To better understand the heightened vulnerability of Web AI agents to jailbreaking and their increased susceptibility to executing malicious commands, we introduce a 5-level fine-grained ablative metric that goes beyond the conventional binary assessments of LLM vulnerabilities, offering a more nuanced evaluation of jailbreak signals. 
Ultimately, our study 
raises awareness of the security challenges posed by Web AI agents and advocates for proactive measures to design safer, more resilient agent frameworks. 















\textbf{Our contributions:}
\begin{itemize}[leftmargin=*]
\item \textbf{Empirical evidence of Web AI agents’ heightened vulnerability: }We systematically compare Web AI agents with standalone LLM chatbots, revealing that Web AI agents are significantly more susceptible to jailbreaking and executing malicious commands.
\item \textbf{Root cause analysis of Web AI agent vulnerabilities:} We investigate the design-level differences between Web AI agents and standalone LLMs, identifying key factors—such as system prompt manipulation, multi-turn action generation, and reliance on historical observations—that contribute to their increased vulnerability.
\item \textbf{A fine-grained evaluation protocol for jailbreak susceptibility:} We introduce a structured, five-level harmfulness evaluation framework that goes beyond binary assessments, enabling a more detailed and nuanced analysis of Web AI agent vulnerabilities.
\item \textbf{Actionable insights for targeted defense strategies:} Based on our findings, we provide recommendations for mitigating security risks in Web AI agents, focusing on improving system prompt handling, action generation mechanisms, and contextual awareness in agent architectures.

    
    
\end{itemize}






\begin{figure*}[!t]
    \centering
    \includegraphics[width=\textwidth]{figures/Fig2.pdf}
    \vspace{-15pt}
    \caption{{\modulename} consists of two major components: Composite Aware Text Encoder (Left) for extracting granular word-level embeddings and Text-Motion Aligner (Middle) for aligning motion embeddings with relevant textual embeddings inside a motion generator. The attention score distribution between different motion tokens and the text tokens is visualized on the upper left. 
    % Red box show case a example of attention allocation between a subset of word and motion tokens.
    The Text-Motion Aligner can be integrated with three genres of motion generation models (Right).}
    \vspace{-5pt}
\label{fig:CASIM}
\end{figure*}

\vspace{-5pt}
\section{Related Works} \label{sec:related_works}
\vspace{-3pt}

\subsection{Human Motion Generation}
\vspace{-3pt}
% What are various tasks of X to motion generation
Generating realistic human motions has been a longstanding challenge in computer graphics and computer vision.
The field has evolved to embrace various input modalities and conditions for motion synthesis.
Image and video-based approaches have focused primarily on human pose and shape estimation~\cite{zhao2019semantic} and 3D body tracking~\cite{stathopoulos2024score}, enabling motion reconstruction and prediction from visual inputs.
Audio-driven motion generation is another important direction, with music-to-dance synthesis~\cite{alexanderson2023listen} and speech-to-gesture generation~\cite{chang2022ivi, chang2023importance} 
showing promising results in creating natural human movements that align with acoustic signals.
Text-to-motion generation has gained significant attention, as it offers intuitive control over motion synthesis through free-form text prompts~\cite{Guo_2022_CVPR}.
Scene-aware motion generation considers environmental constraints and spatial relationships, enabling the synthesis of contextually appropriate movements within 3D environments~\cite{cen2024text_scene_motion}.
Generating a coordinated group of human motions and interactions~\cite{chang2024learning, chang2024equivalency} 
has recently emerged as a novel research directionh adds another layer of difficulty to single-person motion generation due to the complex human interactions.
Lastly, several works \cite{li2024unimotion, zhou2024avatargpt} have attempted to unify motion generation, planning, and understanding in a single framework, extending the capabilities of large language models to human motion domains.


% \vspace{-5pt}
\subsection{Text-to-Motion Generation Models}
% \vspace{-3pt}

Text-to-motion generation models can be broadly categorized into two approaches: diffusion-based and autoregressive-based methods. 
Diffusion-based models leverage an iterative denoising scheme \cite{dhariwal2021diffusion} to generate motions from textual conditions. 
Notable works include MDM \cite{tevet2023human}, MotionDiffuse \cite{zhang2022motiondiffuse}, MLD \cite{chen2023executing}, GMD \cite{karunratanakul2023gmd}, FineMoGen \cite{zhang2023finemogen}, and GraphMotion \cite{jin2023act}. For example, MDM employs a transformer encoder within each diffusion step, processing concatenated motion frames with text and timestamp embeddings. While MLD adopts a similar architecture, it operates in a learned latent space by compressing motion sequences into fixed-length representations.
Autoregressive-based approaches, including T2M \cite{zhang2023generating}, TM2T \cite{guo2022tm2t}, MotionGPT \cite{jiang2024motiongpt}, MotionLLM \cite{chen2024motionllm}, T2MGPT \cite{zhang2023generating}, and CoMo \cite{huang2024como}, generate motions sequentially and typically require effective motion tokenization strategies. 
For instance, T2MGPT utilizes VQVAE \cite{van2017neural} for motion tokenization and implements a decoder-only architecture for motion token generation. CoMo follows a similar generator architecture but distinguishes itself by adopting heuristics-based posecodes \cite{delmas2022posescript} as its discrete motion representation.




\vspace{-2mm}
\subsection{Semantic Injection for Motion Generation}
\vspace{-1mm}

% More deeper dive into semantic injection and alignment for specific models, 
% Start with clip-based fixed length embedding
% such as finemogen, augmented prompts from chatgpt, CoMo keywords, heuristic-based hierarchical GraphMotion semantic injection 
% and how we differs from theirs.
For text-to-motion generation, the input prompts are typically encoded into a latent space with a well-trained text encoder before being passed to motion generation models. 
Previous works, such as MDM, MLD, T2MGPT and CoMo, leverage the pretrained CLIP text embedding to represent the full text prompt.
% as one of the inputs to their transformer-based encoder or decoder.
CoMo \cite{huang2024como} and FGMDM \cite{shi2023generating} includes several fine-grained keywords and descriptions from GPT4 \cite{openai2024gpt4technicalreport} as augmented prompts for motion generation.
Finemogen \cite{zhang2023finemogen} targets at fine-grained motion control and editing, by specifying the spatial and temporal motion descriptions.
GraphMotion \cite{jin2023act} parses the sentence structure into a hierarchical semantic graph for any given input texts. 
It utilizes a graph reasoning network and a coarse-to-fine diffusion model for motion generation.
Our {\modulename} is conceptually similar to GraphMotion as both are aimed at strengthening the text-motion correspondence by design. 
However, GraphMotion uses heuristic knowledge to create a static semantic graph and is only tied to its coarse-to-fine model.
{\modulename} learns the dynamic alignment and hierarchical structure in a soft manner. 
It can also be flexibly integrated with the most existing models.



\vspace{-5pt}
\section{Composite Aware Semantic Injection} \label{sec:casim}
\vspace{-1pt}

{\modulename}, Composite Aware Semantic Injection Mechanism, is designed to capture fine-grained semantic relationships between text descriptions and motion sequences. It preserves the composite nature of human motions and their causal textual ordering and allows each motion frame to dynamically align with relevant textual components at different granularities.
{\modulename} consists of two principal components: a composite aware text encoder and a text-motion aligner. 
{\modulename} exhibits model-agnostic properties, as it is applicable to both autoregressive and diffusion-based motion generators, which represent the two predominant genres for state-of-the-art models. 
In Section \ref{sub:formulation}, we introduce the formulation of {\modulename}, detailing its two major components. 
Section \ref{sub:autoregressive} describes for autoregressive motion generators and how {\modulename} is integrated. 
Section \ref{sub:duffsion} discusses diffusion-based motion generators and explains how to adopt {\modulename} 
% is adapted
in this framework.


% Key cross attention module + clip-based token embedding as a semantic injection module
\vspace{-3pt}
\subsection{{\modulename} Formulation} % module name
\label{sub:formulation}
\vspace{-3pt}

\textbf{Composite Aware Text Encoder.} Unlike traditional approaches that compress text descriptions into fixed-length [CLS] token, our text encoder preserves composite aware semantics through individual token-level embeddings. 
As shown in Fig. \ref{fig:CASIM} (left), the encoder comprises a pretrained text encoder inside which are blocks of multihead self-attention layers that learn the latent features for the input text. 
% , crucial for text-driven motion generation. 
We leverage the pre-trained text encoder from CLIP~\cite{radford2021learning}, project the latent encoder output to another embedding space, and inject the resulting token embeddings to the motion generator.
% The resulting text embeddings serve as conditions for the text-motion aligner. 
% The features capture both sentence-level context and word-level details.
Compared to fixed-length CLIP embeddings, our injection method preserve the semantics as granular as the token-level, which are essential for composite aware motion generation.

\textbf{Text-Motion Aligner.} The text-motion aligner is the core design of {\modulename} and can be integrated inside various motion generation models.
Specifically, it establishes dynamic correspondence between motion frames and text tokens using multi-head attention (MHA). 
As illustrated in Fig.~\ref{fig:CASIM} (middle), each motion token is used as query to attend with the keys and values obtained from all the text tokens. 
The subsequent motion embeddings then are updated through the attention-weighted aggregation from all relevant text tokens. 
Depending on the motion generation approach, the aligner employs multihead self-attention (MHSA) for autoregressive generation or multihead cross-attention (MHCA) for diffusion-based generation. 
% We detail these architectural variants in subsequent sections.


% \subsection{{\modulename} for Autoregressive Motion Generation}
% \label{sub:autoregressive}
% % Specific adaptation to autoregressive methods, T2MGPT, COMO
% Autoregressive motion generation is inspired by principles from language modeling \cite{jiang2024motiongpt, chen2024motionllm}. 
% These models first generate motion tokens, and subsequently, decode motion tokens into a complete motion sequence.

% These models initially employ an auto-encoder to transform a motion sequence $X=\{x_1,\dots,x_T\}$ into a sequence of motion tokens $M=\{m_1,
% \dots ,m_{T/l}\}$, wherein each subsequence of $l$ frames $X_l=\{x_i,...,x_{i+l}\}$ is encoded into a single motion token $z_l = \mathcal{E}(X_l)$. Concurrently, another decoder is trained to reconstruct the original motion $\hat{X}=\mathcal{D}(M)$. Subsequently, the models are trained to generate sequences of motion tokens autoregressively, conditioned upon a textual prompt $C$. \begin{equation}
%     P(M|T) = P(m_1|C)\prod_{i=1}^{T/l}P(m_{i+1}|m_1,\dots,m_i,C)
% \end{equation}

% In the context of autoregressive motion generation models, we select the MHSA block to serve as the aligner between text and motion sequences for each generative step $P(m_i|m_j^{<i},C)=MHSA(M^{<i}\oplus C)$. Initially, we concatenate the text token sequence $C=\{c_1,\dots c_L\}$ with the motion token sequence $M^{<i}$, and subsequently input this concatenated sequence into an autoregressive transformer configured in an encoder-only mode with stacks of MHSA blocks. The autoregressive transformer executes self-attention mechanisms between the motion and text token sequences to effectively align and update the motion tokens, facilitating the generation of the subsequent token, as demonstrated in Figure \ref{fig:CASIM} right.
\vspace{-3pt}
\subsection{{\modulename} for Autoregressive Motion Generation}
\label{sub:autoregressive}
\vspace{-1pt}

Autoregressive motion generation approaches, inspired by language modeling principles \cite{jiang2024motiongpt, chen2024motionllm}, typically follow a two-stage process. First, they learn a discrete representation of motions through tokenization. Given a motion sequence $X=\{x_1,\dots,x_T\}$, an encoder-decoder network is trained to transform it into a sequence of motion tokens $M=\{m_1,\dots,m_{T/l}\}$, where each subsequence of $l$ frames $X_l=\{x_i,\dots,x_{i+l}\}$ is mapped to a discrete token $m_i = \mathcal{E}(X_l)$. The decoder $\mathcal{D}$ learns to reconstruct the original motion: $\hat{X}=\mathcal{D}(M)$.
In the second stage, these methods train an autoregressive model to predict motion tokens sequentially conditioned on a text prompt $C$. The generation process is as follows:
\begin{equation}
P(M|C) = P(m_1|C)\prod_{i=1}^{T/l}P(m_{i+1}|m_1,\dots,m_i,C)
\end{equation}
where each new motion token is predicted based on both the text condition and previously generated tokens. The final motion is obtained by passing the generated tokens through the pretrained decoder $\mathcal{D}$.

% For autoregressive generation, {\modulename} utilizes MHSA as the text-motion aligner at each generation step $P(m_i|m_j^{<i},C)=MHSA(M^{<i}\oplus C)$. As shown in Figure \ref{fig:CASIM} (right), we concatenate the text token sequence $C={c_1,\dots,c_L}$ with the previously generated motion tokens $M^{<i}$. This concatenated sequence is processed by an encoder-only transformer with stacked MHSA blocks, which performs self-attention between motion and text tokens to guide the generation of the next motion token.


For autoregressive generation, {\modulename} leverages a GPT-style transformer \cite{radford2018improving, zhang2023generating} with MHSA blocks for the text-motion aligner to predict the motion tokens.
At generation step $i$, we concatenate the text token sequence $C=\{c_1,\dots,c_N\}$ with previously generated motion tokens $M^{<i}$, denoted as $C\oplus M^{<i}$. 
As shown in Figure \ref{fig:CASIM} (right), this concatenated sequence is processed by an GPT-style transformer with stacked MHSA blocks that enable dynamic interaction between motion and text tokens. 
The probability of generating the next token is computed as:
\begin{equation}
P(m_i|m_j^{<i},C)=\sigma(\mhsa(C \oplus M^{<i}))),
\end{equation} where $\sigma$ represents the softmax function.

This generation step is performed iteratively until the end-of-sequence token is predicted.
The integration of {\modulename} with autoregressive models enables each motion token to be generated with awareness of both previously generated motions and the full text description.


\begin{table}[t!]
% \small
% \footnotesize
% \scriptsize
\fontsize{7.5pt}{7.5pt}\selectfont
  \aboverulesep=0ex
  \belowrulesep=0.5ex 
\setlength{\tabcolsep}{4.2pt}
\centering
\caption{Results on the HumanML3D dataset. The original and {\modulename}-integrated models are shaded.}
\vspace{-5pt}
\begin{tabular}{lcccccc}
\toprule
\multirow{2}{*}{Method}  & \multicolumn{3}{c}{R-Precision$\uparrow$} & \multirow{2}{*}{FID$\downarrow$} & \multirow{2}{*}{MM-Dist.$\downarrow$} & \multirow{2}{*}{Div.$\uparrow$} \\
 \cmidrule{2-4}
 & Top1 & Top2 & Top3 &  &  &  \\
\midrule 
T2M & 0.457 & 0.639 & 0.740 & 1.067 & 3.340 & 9.188 \\
TM2T & 0.424 & 0.618 & 0.729 & 1.501 & 3.467 & 8.589 \\
MotionDiffuse & 0.491 & 0.681 & 0.782 & 0.630 & 3.113 & 9.410 \\
MLD & 0.469 & 0.659 & 0.760 & 0.532 & 3.282 & 9.570 \\
\rowcolor{Gray} MDM & 0.455 & 0.645 & 0.749 & 0.489 & 3.330 & 9.920 \\
\rowcolor{Gray} T2MGPT & 0.491 & 0.680 & 0.775 & 0.116 & 3.118 & 9.761 \\
FineMoGen & 0.504 & 0.690 & 0.784 & 0.151 & 2.998 & 9.263 \\
MotionGPT & 0.492 & 0.681 & 0.778 & 0.232 & 3.096 & 9.528 \\
CoMo & 0.502 & 0.692 & 0.790 & 0.262 & 3.032 & \textbf{9.936} \\
GraphMotion & 0.504 & 0.699 & 0.785 & 0.116 & 3.070 & 9.692 \\
\midrule
\rowcolor{Gray}
\modulename-MDM & 0.502 & 0.694 & 0.793 & 0.165 & 3.020 & 9.394 \\
\rowcolor{Gray}
\modulename-T2MGPT & \textbf{0.539} & \textbf{0.730} & \textbf{0.823} & \textbf{0.105} & \textbf{2.838} & 9.785 \\
%  & CoMo  & $\checkmark$ & 0.545 & 0.741 & 0.835 & 0.200 & 2.747 & 9.887 \\
\bottomrule
\label{tab:quant_res1}
\end{tabular}
\vspace{-5pt}
\end{table}


\begin{table}[t!]
% \small
% \footnotesize
% \scriptsize
\fontsize{7.5pt}{7.5pt}\selectfont
  \aboverulesep=0ex
  \belowrulesep=0.5ex 
\setlength{\tabcolsep}{4.2pt}
\centering
% \caption{Evaluation of generated motion on HumanML3D and KIT-ML dataset. {\modulename} integrated models is shaded}
\caption{Results on the KIT-ML dataset.
The original and {\modulename}-integrated models are shaded.}
\vspace{-5pt}
\begin{tabular}{lcccccc}
\toprule
\multirow{2}{*}{Method}  & \multicolumn{3}{c}{R-Precision$\uparrow$} & \multirow{2}{*}{FID$\downarrow$} & \multirow{2}{*}{MM-Dist.$\downarrow$} & \multirow{2}{*}{Div.$\uparrow$} \\
 \cmidrule{2-4}
 & Top1 & Top2 & Top3 &  &  &  \\
\midrule 
% Real & 0.424 & 0.649 & 0.779 & 0.031 & 2.788 & 11.080 \\
T2M & 0.370 & 0.569 & 0.693 & 2.770 & 3.401 & 10.910 \\
TM2T & 0.280 & 0.463 & 0.587 & 3.599 & 4.591 & 9.473 \\
MotionDiffuse & 0.417 & 0.621 & 0.739 & 1.954 & 2.958 & 11.100 \\
MLD & 0.390 & 0.609 & 0.734 & 0.404 & 3.204 & 10.800 \\
\rowcolor{Gray} MDM & 0.164 & 0.291 & 0.396 & 0.497 & 9.191 & 10.847 \\
\rowcolor{Gray} T2MGPT & 0.402 & 0.619 & 0.737 & 0.717 & 3.053 & 10.862 \\
FineMoGen & 0.432 & 0.649 & 0.772 & \textbf{0.178} & 2.869 & 10.850 \\
MotionGPT & 0.366 & 0.558 & 0.680 & 0.510 & 3.527 & 10.350 \\
CoMo & 0.422 & 0.638 & 0.765 & 0.332 & 2.873 & 10.950 \\
GraphMotion & 0.429 & 0.648 & 0.769 & 0.313 & 3.076 & 11.120 \\
\midrule
\rowcolor{Gray}
\modulename-MDM & \textbf{0.448} & \textbf{0.665} & \textbf{0.784} & 0.354 & \textbf{2.684} & \textbf{11.179} \\
\rowcolor{Gray}
\modulename-T2MGPT & 0.412 & 0.627 & 0.751 & 0.577 & 2.986 & 10.829 \\
\bottomrule
\label{tab:quant_res2}
\end{tabular}
\vspace{-5pt}
\end{table}




\vspace{-3pt}
\subsection{{\modulename} for Motion Diffusion Generation}
\label{sub:duffsion}
\vspace{-1pt}

Unlike autoregressive approaches, motion diffusion models generate motions through iterative denoising of a Gaussian noise sequence $X^\tau$. Let $T$ denote the sequence length and $D$ denote the motion dimension, then $X^\tau \in \mathbb{R}^{T \times D}$ represents the noised motion at diffusion step $\tau$. Given a noise-free motion sequence $X^0$, the diffusion process at each step $t$ samples the denoised motion according to:
\begin{equation}
X^{\tau-1} \sim \mathcal{N}(\mu_{\theta}(X^\tau, \tau, C), \Sigma(\tau)),
\end{equation}
where $\Sigma(\tau)$ is a fixed variance schedule, $C \in \mathbb{R}^{L \times d}$ denotes the text tokens with sequence length $L$ and embedding dimension $d$. The mean term $\mu_{\theta}(X^\tau, \tau, C)$ can be parameterized as $\mu_{\theta}(X^\tau, \hat{X^0})$, where $\hat{X^0}=G_\theta(X^\tau, \tau, C)$ is the predicted noise-free motion by the denoiser $G_\theta$.

While there are no constraints on how the denoiser should be designed as long as the input and output shape matches, the transformer encoder and decoder are the two widely adopted options in the literature.
To integrate {\modulename} with these denoiser variants in diffusion models, we adapt our text-motion aligner to employ MHSA for the transformer encoder and MHCA for the transformer decoder.


\textbf{Denoising with Transformer Encoder.} In the encoder-based approach, we first augment the text embeddings $C$ with diffusion step embeddings $TE(\tau)$, where $TE(\cdot)$ denotes the positional encoding function. The augmented text embeddings are then concatenated with the current motion embeddings $X^t$, and processed through MHSA blocks:
\begin{equation}
\hat{X^0} = \mhsa( (C+TE(\tau))\oplus X^\tau),
\end{equation}
where $\oplus$ denotes sequence concatenation.

\textbf{Denoising with Transformer Decoder.} The decoder-based variant first processes motion features through self-attention layers, enabling each motion frame to attend to other frames. It then applies cross-attention (MHCA) between the processed motion embeddings $X^\tau$ and the timestep-augmented text embeddings:
\begin{equation}
\hat{X^0}  = \mhca(X^\tau, C+TE(\tau)),
\end{equation}

Both formulations enable dynamic text-motion alignment during the denoising process. 
We analyze the performance of all model genres and variants in Section \ref{sub:quantitative}.




\begin{table}[t!]
\fontsize{7.5pt}{7.5pt}\selectfont
  \aboverulesep=0ex
  \belowrulesep=0.5ex 
\setlength{\tabcolsep}{4.2pt}
\centering
\caption{Quantitative results for various text-to-motion methods with {\modulename} on HumanML3D dataset.}
\vspace{-5pt}
\begin{tabular}{lcccccc}
\toprule
\multirow{2}{*}{Method} & \multirow{2}{*}{\modulename} & \multicolumn{3}{c}{R-Precision$\uparrow$} & \multirow{2}{*}{FID$\downarrow$} & \multirow{2}{*}{MM-Dist.$\downarrow$} \\
 \cmidrule{3-5}
 & & Top1 & Top2 & Top3 &  &   \\
\midrule 
\multirow{2}{*}{MotionDiffuse}  & & 0.491 & 0.681 & 0.782 & 0.630 & 3.113 \\
 & $\checkmark$  & \textbf{0.506} & \textbf{0.700} & \textbf{0.799} & \textbf{0.565} & \textbf{2.981} \\
\midrule
\multirow{2}{*}{MDM}  & & 0.471 & 0.661 & 0.760 & 0.325 & 3.249 \\
  & $\checkmark$ & \textbf{0.502} & \textbf{0.694} & \textbf{0.793} & \textbf{0.165} & \textbf{3.020} \\
\midrule
\multirow{2}{*}{T2MGPT}  & & 0.484 & 0.672 & 0.770 & 0.117 & 3.153 \\
  & $\checkmark$ & \textbf{0.539} & \textbf{0.730} & \textbf{0.823} & \textbf{0.105} & \textbf{2.838} \\
\midrule
\multirow{2}{*}{CoMo}  & & 0.502 & 0.692 & 0.790 & 0.262 & 3.032 \\
  & $\checkmark$ & \textbf{0.545} & \textbf{0.741} & \textbf{0.835} & \textbf{0.200} & \textbf{2.747} \\
\midrule
 % & MoMask  & & 0.521 & 0.713 & 0.807 & 0.045 & 2.958 & 9.645 \\
\multirow{2}{*}{MoMask}  & & 0.510 & 0.703 & 0.801 & 0.064 & 2.997 \\
  & $\checkmark$ & \textbf{0.532} & \textbf{0.719} & \textbf{0.813} & \textbf{0.057} & \textbf{2.911} \\
% \midrule
\bottomrule
\vspace{-5pt}

\label{tab:quant_res3}
\end{tabular}
\end{table}


\begin{table}[t!]

\fontsize{7.5pt}{7.5pt}\selectfont
  \aboverulesep=0ex
  \belowrulesep=0.5ex 
\setlength{\tabcolsep}{5.5pt}
\centering
\caption{Quantitative results for various text-to-motion methods with {\modulename} on KIT-ML dataset.}
\vspace{-7.5pt}
\begin{tabular}{lcccccc}
\toprule
\multirow{2}{*}{Method} & \multirow{2}{*}{\modulename} & \multicolumn{3}{c}{R-Precision$\uparrow$} & \multirow{2}{*}{FID$\downarrow$} & \multirow{2}{*}{MM-Dist.$\downarrow$} \\
 \cmidrule{3-5}
 & & Top1 & Top2 & Top3 &  &   \\
\midrule 

\multirow{2}{*}{MDM} &  & 0.164 & 0.291 & 0.396 & 0.497 & 9.191 \\
 & $\checkmark$ & \textbf{0.448} & \textbf{0.665} & \textbf{0.784} & \textbf{0.354} & \textbf{2.684} \\
\midrule 
\multirow{2}{*}{T2MGPT} & & 0.402 & 0.619 & 0.737 & 0.717 & 3.053 \\
 & $\checkmark$ & \textbf{0.412} & \textbf{0.627} & \textbf{0.751} & \textbf{0.577} & \textbf{2.986} \\
\midrule
\multirow{2}{*}{CoMo} &  & 0.399 & - & - & \textbf{0.399} & 2.898 \\
 & $\checkmark$ & \textbf{0.422} & \textbf{0.641} & \textbf{0.762} & 0.408 & \textbf{2.852} \\
\bottomrule
\label{tab:quant_res4}
\vspace{-5pt}
\end{tabular}
\end{table}



\vspace{-5pt}
\section{Experiments} \label{sec:experiments}
\vspace{-3pt}

We evaluate {\modulename} on two standard datasets through extensive quantitative and qualitative analyses. Section \ref{sub:eval_setup} describes our experimental setup, followed by quantitative results in Section \ref{sub:quantitative}, qualitative analysis in Section \ref{sub:qualitative}, and extension to long-form generation in Section \ref{sub:long-term}.

\vspace{-3pt}
\subsection{Evaluation Setup}
\label{sub:eval_setup}
\vspace{-3pt}

\textbf{Datasets and Metrics.} 
We evaluate on HumanML3D \cite{Guo_2022_CVPR} and KIT Motion Language (KIT-ML) \cite{Plappert2016}.
HumanML3D is a large-scale motion language dataset, containing a total of 14,616 motions from AMASS \cite{mahmood2019amass} and HumanAct12 \cite{guo2020action2motion} datasets. 
Each motion is paired with 3 textual annotations, totaling 44,970
descriptions.
KIT-ML dataset consists of a total of 3,911 motions and 6,278 text annotations, providing a small-scale evaluation benchmark. 
The datasets are split into train-valid-test sets with a ratio of 0.8:0.05:0.15. 
Performance is measured using the following metrics: 
(a) \textit{Frechet Inception Distance (FID)}, which evaluates the overall motion quality by computing the distributional difference between the latent features of the generated motions and those of real motions from test set; 
(b) \textit{R-Precision}, which reports the retrieval accuracy between input text and generated motions from their latent space; 
(c) \textit{MM-Distance}, which reports the distance between the input text and generated motions at the latent space; and
(d) \textit{Diversity}, which assesses the diversity of all generated motions.
All metrics are computed using a separate text-motion matching network from \cite{Guo_2022_CVPR}.

\textbf{Baseline Models.} 
We conduct experiments for {\modulename} with five state-of-the-art (SOTA) models: 
T2MGPT \cite{zhang2023generating}, CoMo \cite{huang2024como}, MoMask \cite{guo2023momask},
MotionDiffuse \cite{zhang2022motiondiffuse}, 
and MDM \cite{tevet2023human}, covering various genres of motion generation models as well as both continuous and discrete motion representations.

T2MGPT and CoMo are \textbf{autoregressive} motion generation models, which first tokenize the motion sequence using a quantization-based encoder-decoder structure. 
% After the motions are processed into a sequence of tokens, they employ an autoregressive generator that takes the injected text semantics as the initial token.
We use the autoregressive form of {\modulename}, as detailed in Section \ref{sub:autoregressive}, in place of the fixed-length text injection for their motion token generation.
% Among them, T2MGPT and CoMo are two \textbf{autoregressive} motion generation models. Here, we adopt the autoregressive form of {\modulename} detailed in section \ref{sub:autoregressive} to replace the original autoregressive generator that takes the fix-length [CLS] token as text injection.
% MotionDiffuse and MDM are \textbf{diffusion-based} motion generation models, which undertake conditional denoising at each step of the diffusion process on the entire motion sequence. 
% The core architecture under each diffusion scheme is either a transformer encoder or decoder, that maps the noised motion into the denoised one of the same shape. 
% We apply the composite aware text injection for both the encoder and decoder variants of MDM, while for MotionDiffuse, we use the encoder-based semantic injection. 
MotionDiffuse and MDM are \textbf{diffusion-based} motion generation models that conditionally denoise full motion sequences at each diffusion step with the fixed-length semantic embedding. 
We apply the composite aware text injection for both the encoder and decoder variants of MDM, while for MotionDiffuse, we use the encoder-based semantic injection. 
MoMask employs a hierarchical motion quantization scheme and a multi-layer masked motion generation framework.  
We apply the encoder-based semantic injection for their M- and R- Transformers in our study.

All the other settings follow the baseline methods and hyperparameters remain unchanged.




\begin{table}[t!]
\fontsize{7.25pt}{7.25pt}\selectfont
  \aboverulesep=0ex
  \belowrulesep=0.5ex 
\setlength{\tabcolsep}{3.0pt}
\centering
\caption{Additional quantiative results on HumanML3D with varying architectures and configuration settings.}
\vspace{-5pt}
\begin{tabular}{llcccccc}
\toprule
\multirow{2}{*}{Method} & \multirow{2}{*}{Arch} & \multirow{2}{*}{\modulename} & \multicolumn{3}{c}{R-Precision$\uparrow$} & \multirow{2}{*}{FID$\downarrow$} & \multirow{2}{*}{MM-Dist.$\downarrow$} \\
 \cmidrule{4-6}
 & & &  Top1 & Top2 & Top3 &  &   \\
\midrule 
% MDM (Paper) & Enc & & 0.455 & 0.645 & 0.749 & 0.489 & 3.330 \\
\textcolor{gray}{MDM (Paper)}  & \textcolor{gray}{Enc} & & \textcolor{gray}{0.418} & \textcolor{gray}{0.604} & \textcolor{gray}{0.707} & \textcolor{gray}{0.489} & \textcolor{gray}{3.630} \\
\textcolor{gray}{MDM (Paper)} & \textcolor{gray}{Dec} & & \textcolor{gray}{--} & \textcolor{gray}{--} & \textcolor{gray}{0.608} & \textcolor{gray}{0.767} & \textcolor{gray}{5.507} \\
MDM 50steps  & Enc & & 0.455 & 0.645 & 0.749 & 0.489 & 3.330 \\
MDM & Enc & & 0.471 & 0.661 & 0.760 & 0.325 & 3.249 \\
MDM 50steps & Enc & $\checkmark$ & 0.489 & 0.685 & 0.787 & 0.355 & 3.100 \\
MDM & Enc & $\checkmark$& 0.463 & 0.658 & 0.705 & 0.265 & 3.266 \\
MDM 50steps & Dec & $\checkmark$ & \textbf{0.509} & \textbf{0.698} & \textbf{0.793} & 0.230 & 3.035 \\
MDM & Dec & $\checkmark$ & 0.502 & 0.694 & \textbf{0.793} & \textbf{0.165} & \textbf{3.020} \\
\midrule
% T2MGPT (Paper) & AR & & 0.491 & 0.680 & 0.775 & 0.116 & 3.118 \\
T2MGPT $\tau=0$  & AR & & 0.417 & 0.589 & 0.685 & 0.140 & 3.730 \\
T2MGPT $\tau=0.5$  & AR & & 0.491 & 0.680 & 0.775 & 0.116 & 3.118 \\
T2MGPT $\tau=[0, 1]$  & AR & & 0.492 & 0.679 & 0.775 & 0.141 & 3.121 \\
% T2MGPT & AR & & 0.484 & 0.672 & 0.770 & 0.117 & 3.153  \\
T2MGPT $\tau=0$ & AR & $\checkmark$ & 0.465 & 0.651 & 0.746 & 0.117 & 3.308  \\
T2MGPT $\tau=0.5$ & AR & $\checkmark$ & \textbf{0.539} & \textbf{0.730} & \textbf{0.823} & \textbf{0.105} & \textbf{2.838}  \\
T2MGPT $\tau=[0,1]$ & AR & $\checkmark$ & 0.538 & 0.730 & 0.821 & \textbf{0.105} & 2.841  \\
\midrule
CoMo  & AR & & 0.502 & 0.692 & 0.790 & 0.262 & 3.032  \\
CoMo no keywords  & AR & & 0.487 & - & - & 0.263 & 3.044  \\
CoMo  & AR & $\checkmark$ & \textbf{0.545} & \textbf{0.741} & \textbf{0.835} & \textbf{0.200} & \textbf{2.747}  \\
CoMo no keywords  & AR & $\checkmark$ & 0.539 & 0.738 & 0.831 & 0.226 & 2.777  \\
\bottomrule
\label{tab:quant_res5}
\vspace{-10pt}
\end{tabular}
\end{table}



\begin{figure*}[!t]
    \centering
    \includegraphics[width=\textwidth]{figures/Fig3.pdf}
    \vspace{-15pt}
    \captionof{figure}{Qualitative comparison between two baselines, their {\modulename}-enhanced models, and ground truth (GT) on HumanML3D test prompts. 
    Action verbs and their modifiers are highlighted in red, with motion sequences shown in color gradients (light to dark) and root trajectories in black.
    {\modulename}-MDM and {\modulename}-T2MGPT generate the motions that better match the descriptions, showing stronger text-motion correspondence and better controllability.}
    \vspace{-5pt}
\label{fig:qual_res}
\end{figure*}

\vspace{-3pt}
\subsection{Quantitative Analysis} 
\label{sub:quantitative}
\vspace{-3pt}

\textbf{Comparison with SOTA Methods.}
For quantitative evaluation, we report the results for each metric averaged over 20 repeated iterations on both datasets. 
Tab.~\ref{tab:quant_res1} and \ref{tab:quant_res2} present quantitative comparison of {\modulename}-MDM and {\modulename}-T2MGPT with the SOTA text-to-motion generation methods on both datasets.
Our method outperforms most SOTA methods in terms of R-Precision and MM-Dist, showing the effectiveness and robustness of {\modulename} in learning the text-motion correspondence.
In terms of motion quality, both {\modulename}-MDM and {\modulename}-T2MGPT achieve comparable FID score with some leading methods, such as GraphMotion, FineMoGen, and T2MGPT, on both benchmarks. 
Notably, the positive results are achieved through our semantic injection mechanism alone, without additional heuristic knowledge or textual semantics from external source like GraphMotion, CoMo, and FineMoGen.


\textbf{Leveling up SOTA model performances.}
To demonstrate {\modulename}'s effectiveness across different architectures, we integrate it with five representative models and evaluate their performance improvements. 
As shown in Tab.~\ref{tab:quant_res3}, {\modulename} consistently enhances all baseline models on HumanML3D dataset, with particularly notable gains in text-motion alignment metrics (R-Precision and MM-Dist).

For diffusion-based models, {\modulename} brings substantial improvements. MDM sees a significant boost in both motion quality (FID: 0.325→0.165) and multimodality alignment (Top1 R-Precision: 0.471→0.502, MM-Dist: 3.249→3.020). 
Similar improvements are observed for MotionDiffuse, suggesting that {\modulename}'s semantic injection mechanism effectively addresses the limitations of fixed-length text injection in diffusion models.
Autoregressive models, despite their stronger baseline performance, also benefit from {\modulename}. 
T2MGPT and CoMo show remarkable improvements in text-motion matching, and seems to benefit more with the composite aware semantics. 
The performance in Top1 R-Precision increases by 5.5\% and 4.3\% respectively and their MM-Distance drops to the lowest among all methods, with 2.838 and 2.747 respectively.
Even MoMask, which already achieves strong FID scores, sees consistent improvements across all metrics.
The benefits of {\modulename} remain evident on the more challenging KIT-ML dataset (Tab.~\ref{tab:quant_res4}). 
{\modulename}-MDM achieves significant improvements in text-motion alignment (R-Precision Top1: 0.164→0.448 and Top3: 0.396→0.784) and motion quality (FID: 0.497→0.354). 
{\modulename}-T2MGPT also sees consistent performance gain across the metrics.
For {\modulename}-CoMo,
% \footnote{CoMo's extra fine-grained keyword not open-sourced for KIT-ML dataset}
the motion quality remains on par with the baseline while R-Precision and MM-Distance improves.

The quantitative results suggest that {\modulename} is effective across different genres of motion generation models and across different datasets.
The consistent improvement in text-motion matching demonstrates its capability in learning dynamic semantic relationships without compromising motion quality.


\textbf{Architecture and configuration analysis.}
We conduct extensive experiments to validate {\modulename}'s effectiveness across different architectural choices and configuration settings, as shown in Tab.~\ref{tab:quant_res5}.
For diffusion-based MDM, we explore both encoder and decoder-based implementations of {\modulename}. Both variants demonstrate substantial improvements over their baseline (Encoder: Top1 R-Precision 0.471→0.489, FID 0.325→0.265; Decoder: Top3 R-Precision 0.608→0.793, FID 0.767→0.165), showcasing {\modulename}'s adaptability to different architectural choices. Particularly noteworthy is that these improvements hold even with significantly reduced computation--using only 50 diffusion steps instead of 1000, both variants maintain strong performance gains while achieving 20× faster inference.

For autoregressive models like T2MGPT, we examine {\modulename}'s behavior under different teacher forcing settings. During training, random masking ($\tau=[0,1]$ or $\tau=0.5$) \footnote{$\tau$ variable here represent coefficient for teacher forcing, which follows original notation. This is different from diffusion step index used earlier.} helps bridge the gap between training and inference, where the predicted tokens may differ from the ground truth. 
While the best performance occurs when $\tau=0.5$, {\modulename} can further improve its text-motion alignment (R-Precision: 0.491→0.539, MM-Dist: 3.118→2.838).
{\modulename} demonstrates robust performance across these hyperparameter configurations,
with all $\tau=0$, $\tau=0.5$ and $\tau = [0,1]$ 
achieving significant improvements, showing its resilience to different autoregressive model configurations.

The CoMo experiments reveal another interesting aspect of {\modulename}'s capabilities. 
The original CoMo relies on 11 additional keywords per description, augmented through GPT-4 to provide detailed motion characteristics across body parts and styles. 
Remarkably, {\modulename}-CoMo without any keywords outperforms the keyword-augmented baseline (R-Precision: 0.539 vs 0.487, FID: 0.226 vs 0.263), demonstrating {\modulename}'s ability to extract rich motion semantics directly from the input text without requiring external semantic augmentation.

Across all experiments, {\modulename} shows consistent performance improvements regardless of model architecture (diffusion or autoregressive), configuration choices (encoder/decoder, teacher forcing rates), or additional inputs (with/without keywords). This robust adaptability suggests that {\modulename}'s semantic injection mechanism provides fundamental improvements in learning text-motion correspondence that generalize across different modeling approaches.


\begin{figure}[!t]
    \centering
    \includegraphics[width=\linewidth]{figures/Figure_wordcloud.pdf}
    \vspace{-20pt}
    \captionof{figure}{Analysis of attention patterns in {\modulename}. 
    Left: Word cloud showing top-5 attended words across all test prompts, highlighting focus on action verbs, motion modifiers, and spatial references. 
    Right: Word cloud for prompts containing `\textit{walk}', revealing attention to motion-specific contextual attributes.}
    \vspace{-10pt}
\label{fig:wordcloud}
\end{figure}

\vspace{-3pt}
\subsection{Qualitative Analysis}
\label{sub:qualitative}
\vspace{-1pt}




\begin{table*}[t!]
\fontsize{8.25pt}{8.25pt}\selectfont
  \aboverulesep=0ex
  \belowrulesep=0.5ex 
\setlength{\tabcolsep}{6.0pt}
\centering
\caption{Results on the HumanML3D dataset for long-term motion generation.}
 \vspace{-5pt}
\begin{tabular}{lcccccccc|cc}
\toprule
\multirow{2}{*}{Method} & \multirow{2}{*}{\makecell{Handshake\\(\#frames)}} & \multirow{2}{*}{\modulename} & \multicolumn{6}{c}{\textbf{Motion}} & \multicolumn{2}{c}{\textbf{Transition}} \\
\cmidrule{4-11}
 & & & Top1$\uparrow$ & Top2$\uparrow$ & Top3$\uparrow$ & FID$\downarrow$ & MM-Dist$\downarrow$ & Div.$\rightarrow$& FID$\downarrow$ & Div.$\rightarrow$ \\
\midrule
GT & & & 0.511 & 0.703 & 0.797 & 0.002 & 2.974 & 9.503 & 0.050 & 9.570 \\
 \cmidrule{1-11}
\multirow{6}{*}{DoubleTake} & \multirow{2}{*}{20} &  & 0.309 & 0.477 & 0.589 & 0.953 & 5.713 & 9.624 & \textbf{1.540} & 8.750 \\
 &  & $\checkmark$ & \textbf{0.358} & \textbf{0.526} & \textbf{0.627} & \textbf{0.463} & \textbf{5.508} & \textbf{9.668} & 2.052 & \textbf{8.775} \\
 \cmidrule{2-11}
 & \multirow{2}{*}{30} & & -- & -- & 0.600 & 1.030 & 5.600 & 9.530 & 2.220 & 8.640 \\
 &  & $\checkmark$  & \textbf{0.345} & \textbf{0.516} & \textbf{0.612} & \textbf{0.707} & \textbf{5.532} & \textbf{9.864} & \textbf{1.896} & \textbf{8.805} \\
 \cmidrule{2-11}
 & \multirow{2}{*}{40} & & -- & -- & 0.580 & 1.160 & 5.670 & 9.610 & 2.410 & 8.610 \\
 &  & $\checkmark$ & \textbf{0.330} & \textbf{0.497} & \textbf{0.594} & \textbf{0.959} & \textbf{5.584} & \textbf{9.897} & \textbf{1.905} & \textbf{8.723} \\
\bottomrule
\end{tabular}%
\label{tab:dt_res}
 \vspace{-5pt}
\end{table*}




\textbf{Qualitative Comparisons.}
Fig.~\ref{fig:qual_res} demonstrates {\modulename}'s ability to improve motion generation quality through representative examples from the HumanML3D test set. Compared to baseline models, both {\modulename}-MDM and {\modulename}-T2MGPT show superior ability in following complex action sequences and maintaining temporal order. 
Their generated motions closely align with ground truth (GT), particularly in capturing the nuaunced semantics for spatial relationships and action transitions.

Specifically, for the input text ``\textit{a man runs to the right then runs to the left then back to the middle}", both {\modulename}-MDM and {\modulename}-T2MGPT accurately capture directional changes and chronological order, while baseline models struggle with spatial positioning and temporal progression. 
For another text, ``\textit{a person is holding his arms straight out to the sides then lowers them, claps, and steps forward to sit in a chair}", our models precisely follow each action component in sequence, whereas baseline models either miss critical components or fail to maintain the correct order. 
Given the third prompt, ``\textit{a person walks forward then turns completely around and does a cartwheel}", {\modulename}-MDM and {\modulename}-T2MGPT successfully reproduce the complete action sequence including the cartwheel motion, despite its infrequent appearance in the dataset.

The qualitative results demonstrate {\modulename}'s effectiveness in enabling precise text-based motion control while maintaining generalization to less common actions. 
The improved text-motion alignment across various examples suggests that our dynamic semantic injection successfully addresses the limitations of fixed-length text representations in existing methods.




\textbf{Interpretability Analysis.}
With the impressive quantitative and qualitative performance, it is interesting to understand how {\modulename} processes textual information. 
We first analyze its attention patterns through word clouds. 
Fig.~\ref{fig:wordcloud} (left) visualizes the most attended words in HumanML3D test set descriptions. The left word cloud shows the top-five attended words across all text prompts, revealing {\modulename}'s focus on motion-critical elements: action verbs (e.g., ``hand", ``step"), motion modifiers (e.g., ``slowly", ``quickly"), and spatial references (e.g., ``forward", ``circle"). For text containing the word `\textit{walk}' (Fig.~\ref{fig:wordcloud} right), the attention focuses on contextual motion attributes like direction (``forward", ``backward") and style (``slowly"), demonstrating {\modulename}'s capability in capturing motion-specific semantic relationships.

We also visualize the attention weights to analyze how {\modulename} dynamically aligns text with motion frames (Appendix \ref{sup:vis_attn_weights}). 
The visualization reveals clear temporal correspondence between text tokens and motion progression, validating our design of composite aware semantic injection in learning complex and dynamic text-motion relationships.


\vspace{-3pt}
% \subsection{Long-term Motion Generation}
\subsection{Long-term Motion Generation}
\label{sub:long-term}
\vspace{-1pt}

% While {\modulename} primarily targets single motion generation, we explore its effectiveness in long-term motion generation using DoubleTake \cite{shafir2024human}, a framework that employs a two-stage diffusion process to generate smooth transitions between MDM-generated motions.
% As shown in Tab. \ref{tab:dt_res}, {\modulename} enhances motion quality, text-motion matching, and diversity across different handshake sizes in the generated motion clips, demonstrating its benefits beyond single motion generation.
% However, when evaluated on transition periods between motion clips, the improvements are less consistent and vary with handshake size.
% This mixed performance on transitions, while not directly addressed by {\modulename}'s design, opens interesting questions about how semantic injection methods influence diffusion-based transition generation in long-term motion synthesis.

While {\modulename} primarily targets single motion generation, we explore its effectiveness in long-term motion generation using DoubleTake \cite{shafir2024human}, a framework that employs an additional two-stage diffusion process to generate smooth transitions between the motion clips generated by diffusion models like MDM.

As shown in Tab.~\ref{tab:dt_res}, {\modulename} consistently improves the motion clips across metrics and handshake sizes overall. 
For handshake size of 20 frames, {\modulename} significantly reduces the motion FID from 0.953 to 0.463 while improving text alignment (Top1: 0.309→0.358) and maintaining motion diversity (9.624→9.668). 
Similar improvements are observed with longer handshake periods of 30 and 40 frames, though the gains in FID gradually decrease as the transition period extends.
Interestingly, the transition quality shows mixed results. With 20-frame handshake, the transition FID slightly increases (1.540→2.052), while longer handshakes see improvements (30 frames: 2.220→1.896; 40 frames: 2.410→1.905). 
This suggests that {\modulename}'s semantic injection, while effective for individual motion generation, interacts differently for compositing several motion clips during the motion blending and transition generation process.
The observations raise interesting questions about how semantic injection methods influence diffusion-based transition generation in long-term motion synthesis, particularly regarding the balance between local motion quality and smooth transitions. 


%%% 1 or 2 figures with qualitative samples and comparison for different methods, in-domain, spatial control, long text.




\section{Conclusion}
\vspace{-10pt}
In this work, we demonstrated that fine-tuning large language models on structured public opinion survey data markedly improves their ability to predict human response distributions. 
We curate \OURDATA—a dataset 6.5× larger than previous collections to fine-tune and evaluate LLMs on survey response distribution prediction task.
By fine-tuning on \OURDATA, we showed that LLMs can capture the nuanced, group-specific variability in public opinions, while also generalizing to unseen subpopulations, survey waves and question topics, and different survey families. 
Fine-tuning achieves consistent improvements across subpopulations of varying sizes, and our experiments demonstrate that fine-tuned LLMs are indeed \textit{adapting} their responses to the subpopulation specified in the prompt, even for subpopulations unseen during fine-tuning. 
Finally, our experiments also reveal that as the fine-tuning dataset grows, model performance continues to scale favorably, underscoring the value of our larger dataset. 

Generalization is a critical capability for LLMs, if they are to be used to assist public opinion research, as researchers are most in need of accurate opinion predictions for questions or subpopulations whom they have not surveyed before.
Our work, by greatly improving LLMs' ability to accurately predict opinions with fine-tuning and demonstrating strong generalization to out-of-distribution data, moves us closer towards the goal of leveraging LLMs for opinion prediction.
However, many open questions remain: 
why is the model able to generalize well to unseen subpopulations and questions, and under what conditions might it fail to do so?
How do we ensure that LLMs faithfully capture opinions along other dimensions that are not explored in this work, such as intersections of demographic identities or changing opinions over time? 
How should LLMs be integrated into survey designs, to serve as tools that can complement large-scale surveys with human participants? 
Answering these questions will require interdisciplinary collaborations with domain experts and critical assessments of LLMs' and traditional survey methods' strengths and weaknesses, so that we can most effectively and responsibly combine them to better estimate public opinions and inform public policies.
\section*{Impact Statement}
\system advances cost-efficient AI by demonstrating how small on-device language models can collaborate with powerful cloud-hosted models to perform data-intensive reasoning. By reducing reliance on expensive remote inference, \system makes advanced AI more accessible and sustainable. This has broad societal implications, including lowering barriers to AI adoption and enhancing data privacy by keeping more computations local. However, careful consideration must be given to potential biases in small models and the security risks of local code execution. 
%%% Acknowledgement is not needed till camera ready version
\section*{Acknowledgment}
This work was supported by the National Natural Science Foundation of China (62441239,~U23A20319,~62172056,~62472394,~62192784, \\ U22B2038) as well as the 8th Young Elite Scientists Sponsorship Program by CAST (2022QNRC001).
 





% In the unusual situation where you want a paper to appear in the
% references without citing it in the main text, use \nocite
\nocite{langley00}

\bibliography{reference}
\bibliographystyle{icml2025}


%%% APPENDIX
\subsection{Lloyd-Max Algorithm}
\label{subsec:Lloyd-Max}
For a given quantization bitwidth $B$ and an operand $\bm{X}$, the Lloyd-Max algorithm finds $2^B$ quantization levels $\{\hat{x}_i\}_{i=1}^{2^B}$ such that quantizing $\bm{X}$ by rounding each scalar in $\bm{X}$ to the nearest quantization level minimizes the quantization MSE. 

The algorithm starts with an initial guess of quantization levels and then iteratively computes quantization thresholds $\{\tau_i\}_{i=1}^{2^B-1}$ and updates quantization levels $\{\hat{x}_i\}_{i=1}^{2^B}$. Specifically, at iteration $n$, thresholds are set to the midpoints of the previous iteration's levels:
\begin{align*}
    \tau_i^{(n)}=\frac{\hat{x}_i^{(n-1)}+\hat{x}_{i+1}^{(n-1)}}2 \text{ for } i=1\ldots 2^B-1
\end{align*}
Subsequently, the quantization levels are re-computed as conditional means of the data regions defined by the new thresholds:
\begin{align*}
    \hat{x}_i^{(n)}=\mathbb{E}\left[ \bm{X} \big| \bm{X}\in [\tau_{i-1}^{(n)},\tau_i^{(n)}] \right] \text{ for } i=1\ldots 2^B
\end{align*}
where to satisfy boundary conditions we have $\tau_0=-\infty$ and $\tau_{2^B}=\infty$. The algorithm iterates the above steps until convergence.

Figure \ref{fig:lm_quant} compares the quantization levels of a $7$-bit floating point (E3M3) quantizer (left) to a $7$-bit Lloyd-Max quantizer (right) when quantizing a layer of weights from the GPT3-126M model at a per-tensor granularity. As shown, the Lloyd-Max quantizer achieves substantially lower quantization MSE. Further, Table \ref{tab:FP7_vs_LM7} shows the superior perplexity achieved by Lloyd-Max quantizers for bitwidths of $7$, $6$ and $5$. The difference between the quantizers is clear at 5 bits, where per-tensor FP quantization incurs a drastic and unacceptable increase in perplexity, while Lloyd-Max quantization incurs a much smaller increase. Nevertheless, we note that even the optimal Lloyd-Max quantizer incurs a notable ($\sim 1.5$) increase in perplexity due to the coarse granularity of quantization. 

\begin{figure}[h]
  \centering
  \includegraphics[width=0.7\linewidth]{sections/figures/LM7_FP7.pdf}
  \caption{\small Quantization levels and the corresponding quantization MSE of Floating Point (left) vs Lloyd-Max (right) Quantizers for a layer of weights in the GPT3-126M model.}
  \label{fig:lm_quant}
\end{figure}

\begin{table}[h]\scriptsize
\begin{center}
\caption{\label{tab:FP7_vs_LM7} \small Comparing perplexity (lower is better) achieved by floating point quantizers and Lloyd-Max quantizers on a GPT3-126M model for the Wikitext-103 dataset.}
\begin{tabular}{c|cc|c}
\hline
 \multirow{2}{*}{\textbf{Bitwidth}} & \multicolumn{2}{|c|}{\textbf{Floating-Point Quantizer}} & \textbf{Lloyd-Max Quantizer} \\
 & Best Format & Wikitext-103 Perplexity & Wikitext-103 Perplexity \\
\hline
7 & E3M3 & 18.32 & 18.27 \\
6 & E3M2 & 19.07 & 18.51 \\
5 & E4M0 & 43.89 & 19.71 \\
\hline
\end{tabular}
\end{center}
\end{table}

\subsection{Proof of Local Optimality of LO-BCQ}
\label{subsec:lobcq_opt_proof}
For a given block $\bm{b}_j$, the quantization MSE during LO-BCQ can be empirically evaluated as $\frac{1}{L_b}\lVert \bm{b}_j- \bm{\hat{b}}_j\rVert^2_2$ where $\bm{\hat{b}}_j$ is computed from equation (\ref{eq:clustered_quantization_definition}) as $C_{f(\bm{b}_j)}(\bm{b}_j)$. Further, for a given block cluster $\mathcal{B}_i$, we compute the quantization MSE as $\frac{1}{|\mathcal{B}_{i}|}\sum_{\bm{b} \in \mathcal{B}_{i}} \frac{1}{L_b}\lVert \bm{b}- C_i^{(n)}(\bm{b})\rVert^2_2$. Therefore, at the end of iteration $n$, we evaluate the overall quantization MSE $J^{(n)}$ for a given operand $\bm{X}$ composed of $N_c$ block clusters as:
\begin{align*}
    \label{eq:mse_iter_n}
    J^{(n)} = \frac{1}{N_c} \sum_{i=1}^{N_c} \frac{1}{|\mathcal{B}_{i}^{(n)}|}\sum_{\bm{v} \in \mathcal{B}_{i}^{(n)}} \frac{1}{L_b}\lVert \bm{b}- B_i^{(n)}(\bm{b})\rVert^2_2
\end{align*}

At the end of iteration $n$, the codebooks are updated from $\mathcal{C}^{(n-1)}$ to $\mathcal{C}^{(n)}$. However, the mapping of a given vector $\bm{b}_j$ to quantizers $\mathcal{C}^{(n)}$ remains as  $f^{(n)}(\bm{b}_j)$. At the next iteration, during the vector clustering step, $f^{(n+1)}(\bm{b}_j)$ finds new mapping of $\bm{b}_j$ to updated codebooks $\mathcal{C}^{(n)}$ such that the quantization MSE over the candidate codebooks is minimized. Therefore, we obtain the following result for $\bm{b}_j$:
\begin{align*}
\frac{1}{L_b}\lVert \bm{b}_j - C_{f^{(n+1)}(\bm{b}_j)}^{(n)}(\bm{b}_j)\rVert^2_2 \le \frac{1}{L_b}\lVert \bm{b}_j - C_{f^{(n)}(\bm{b}_j)}^{(n)}(\bm{b}_j)\rVert^2_2
\end{align*}

That is, quantizing $\bm{b}_j$ at the end of the block clustering step of iteration $n+1$ results in lower quantization MSE compared to quantizing at the end of iteration $n$. Since this is true for all $\bm{b} \in \bm{X}$, we assert the following:
\begin{equation}
\begin{split}
\label{eq:mse_ineq_1}
    \tilde{J}^{(n+1)} &= \frac{1}{N_c} \sum_{i=1}^{N_c} \frac{1}{|\mathcal{B}_{i}^{(n+1)}|}\sum_{\bm{b} \in \mathcal{B}_{i}^{(n+1)}} \frac{1}{L_b}\lVert \bm{b} - C_i^{(n)}(b)\rVert^2_2 \le J^{(n)}
\end{split}
\end{equation}
where $\tilde{J}^{(n+1)}$ is the the quantization MSE after the vector clustering step at iteration $n+1$.

Next, during the codebook update step (\ref{eq:quantizers_update}) at iteration $n+1$, the per-cluster codebooks $\mathcal{C}^{(n)}$ are updated to $\mathcal{C}^{(n+1)}$ by invoking the Lloyd-Max algorithm \citep{Lloyd}. We know that for any given value distribution, the Lloyd-Max algorithm minimizes the quantization MSE. Therefore, for a given vector cluster $\mathcal{B}_i$ we obtain the following result:

\begin{equation}
    \frac{1}{|\mathcal{B}_{i}^{(n+1)}|}\sum_{\bm{b} \in \mathcal{B}_{i}^{(n+1)}} \frac{1}{L_b}\lVert \bm{b}- C_i^{(n+1)}(\bm{b})\rVert^2_2 \le \frac{1}{|\mathcal{B}_{i}^{(n+1)}|}\sum_{\bm{b} \in \mathcal{B}_{i}^{(n+1)}} \frac{1}{L_b}\lVert \bm{b}- C_i^{(n)}(\bm{b})\rVert^2_2
\end{equation}

The above equation states that quantizing the given block cluster $\mathcal{B}_i$ after updating the associated codebook from $C_i^{(n)}$ to $C_i^{(n+1)}$ results in lower quantization MSE. Since this is true for all the block clusters, we derive the following result: 
\begin{equation}
\begin{split}
\label{eq:mse_ineq_2}
     J^{(n+1)} &= \frac{1}{N_c} \sum_{i=1}^{N_c} \frac{1}{|\mathcal{B}_{i}^{(n+1)}|}\sum_{\bm{b} \in \mathcal{B}_{i}^{(n+1)}} \frac{1}{L_b}\lVert \bm{b}- C_i^{(n+1)}(\bm{b})\rVert^2_2  \le \tilde{J}^{(n+1)}   
\end{split}
\end{equation}

Following (\ref{eq:mse_ineq_1}) and (\ref{eq:mse_ineq_2}), we find that the quantization MSE is non-increasing for each iteration, that is, $J^{(1)} \ge J^{(2)} \ge J^{(3)} \ge \ldots \ge J^{(M)}$ where $M$ is the maximum number of iterations. 
%Therefore, we can say that if the algorithm converges, then it must be that it has converged to a local minimum. 
\hfill $\blacksquare$


\begin{figure}
    \begin{center}
    \includegraphics[width=0.5\textwidth]{sections//figures/mse_vs_iter.pdf}
    \end{center}
    \caption{\small NMSE vs iterations during LO-BCQ compared to other block quantization proposals}
    \label{fig:nmse_vs_iter}
\end{figure}

Figure \ref{fig:nmse_vs_iter} shows the empirical convergence of LO-BCQ across several block lengths and number of codebooks. Also, the MSE achieved by LO-BCQ is compared to baselines such as MXFP and VSQ. As shown, LO-BCQ converges to a lower MSE than the baselines. Further, we achieve better convergence for larger number of codebooks ($N_c$) and for a smaller block length ($L_b$), both of which increase the bitwidth of BCQ (see Eq \ref{eq:bitwidth_bcq}).


\subsection{Additional Accuracy Results}
%Table \ref{tab:lobcq_config} lists the various LOBCQ configurations and their corresponding bitwidths.
\begin{table}
\setlength{\tabcolsep}{4.75pt}
\begin{center}
\caption{\label{tab:lobcq_config} Various LO-BCQ configurations and their bitwidths.}
\begin{tabular}{|c||c|c|c|c||c|c||c|} 
\hline
 & \multicolumn{4}{|c||}{$L_b=8$} & \multicolumn{2}{|c||}{$L_b=4$} & $L_b=2$ \\
 \hline
 \backslashbox{$L_A$\kern-1em}{\kern-1em$N_c$} & 2 & 4 & 8 & 16 & 2 & 4 & 2 \\
 \hline
 64 & 4.25 & 4.375 & 4.5 & 4.625 & 4.375 & 4.625 & 4.625\\
 \hline
 32 & 4.375 & 4.5 & 4.625& 4.75 & 4.5 & 4.75 & 4.75 \\
 \hline
 16 & 4.625 & 4.75& 4.875 & 5 & 4.75 & 5 & 5 \\
 \hline
\end{tabular}
\end{center}
\end{table}

%\subsection{Perplexity achieved by various LO-BCQ configurations on Wikitext-103 dataset}

\begin{table} \centering
\begin{tabular}{|c||c|c|c|c||c|c||c|} 
\hline
 $L_b \rightarrow$& \multicolumn{4}{c||}{8} & \multicolumn{2}{c||}{4} & 2\\
 \hline
 \backslashbox{$L_A$\kern-1em}{\kern-1em$N_c$} & 2 & 4 & 8 & 16 & 2 & 4 & 2  \\
 %$N_c \rightarrow$ & 2 & 4 & 8 & 16 & 2 & 4 & 2 \\
 \hline
 \hline
 \multicolumn{8}{c}{GPT3-1.3B (FP32 PPL = 9.98)} \\ 
 \hline
 \hline
 64 & 10.40 & 10.23 & 10.17 & 10.15 &  10.28 & 10.18 & 10.19 \\
 \hline
 32 & 10.25 & 10.20 & 10.15 & 10.12 &  10.23 & 10.17 & 10.17 \\
 \hline
 16 & 10.22 & 10.16 & 10.10 & 10.09 &  10.21 & 10.14 & 10.16 \\
 \hline
  \hline
 \multicolumn{8}{c}{GPT3-8B (FP32 PPL = 7.38)} \\ 
 \hline
 \hline
 64 & 7.61 & 7.52 & 7.48 &  7.47 &  7.55 &  7.49 & 7.50 \\
 \hline
 32 & 7.52 & 7.50 & 7.46 &  7.45 &  7.52 &  7.48 & 7.48  \\
 \hline
 16 & 7.51 & 7.48 & 7.44 &  7.44 &  7.51 &  7.49 & 7.47  \\
 \hline
\end{tabular}
\caption{\label{tab:ppl_gpt3_abalation} Wikitext-103 perplexity across GPT3-1.3B and 8B models.}
\end{table}

\begin{table} \centering
\begin{tabular}{|c||c|c|c|c||} 
\hline
 $L_b \rightarrow$& \multicolumn{4}{c||}{8}\\
 \hline
 \backslashbox{$L_A$\kern-1em}{\kern-1em$N_c$} & 2 & 4 & 8 & 16 \\
 %$N_c \rightarrow$ & 2 & 4 & 8 & 16 & 2 & 4 & 2 \\
 \hline
 \hline
 \multicolumn{5}{|c|}{Llama2-7B (FP32 PPL = 5.06)} \\ 
 \hline
 \hline
 64 & 5.31 & 5.26 & 5.19 & 5.18  \\
 \hline
 32 & 5.23 & 5.25 & 5.18 & 5.15  \\
 \hline
 16 & 5.23 & 5.19 & 5.16 & 5.14  \\
 \hline
 \multicolumn{5}{|c|}{Nemotron4-15B (FP32 PPL = 5.87)} \\ 
 \hline
 \hline
 64  & 6.3 & 6.20 & 6.13 & 6.08  \\
 \hline
 32  & 6.24 & 6.12 & 6.07 & 6.03  \\
 \hline
 16  & 6.12 & 6.14 & 6.04 & 6.02  \\
 \hline
 \multicolumn{5}{|c|}{Nemotron4-340B (FP32 PPL = 3.48)} \\ 
 \hline
 \hline
 64 & 3.67 & 3.62 & 3.60 & 3.59 \\
 \hline
 32 & 3.63 & 3.61 & 3.59 & 3.56 \\
 \hline
 16 & 3.61 & 3.58 & 3.57 & 3.55 \\
 \hline
\end{tabular}
\caption{\label{tab:ppl_llama7B_nemo15B} Wikitext-103 perplexity compared to FP32 baseline in Llama2-7B and Nemotron4-15B, 340B models}
\end{table}

%\subsection{Perplexity achieved by various LO-BCQ configurations on MMLU dataset}


\begin{table} \centering
\begin{tabular}{|c||c|c|c|c||c|c|c|c|} 
\hline
 $L_b \rightarrow$& \multicolumn{4}{c||}{8} & \multicolumn{4}{c||}{8}\\
 \hline
 \backslashbox{$L_A$\kern-1em}{\kern-1em$N_c$} & 2 & 4 & 8 & 16 & 2 & 4 & 8 & 16  \\
 %$N_c \rightarrow$ & 2 & 4 & 8 & 16 & 2 & 4 & 2 \\
 \hline
 \hline
 \multicolumn{5}{|c|}{Llama2-7B (FP32 Accuracy = 45.8\%)} & \multicolumn{4}{|c|}{Llama2-70B (FP32 Accuracy = 69.12\%)} \\ 
 \hline
 \hline
 64 & 43.9 & 43.4 & 43.9 & 44.9 & 68.07 & 68.27 & 68.17 & 68.75 \\
 \hline
 32 & 44.5 & 43.8 & 44.9 & 44.5 & 68.37 & 68.51 & 68.35 & 68.27  \\
 \hline
 16 & 43.9 & 42.7 & 44.9 & 45 & 68.12 & 68.77 & 68.31 & 68.59  \\
 \hline
 \hline
 \multicolumn{5}{|c|}{GPT3-22B (FP32 Accuracy = 38.75\%)} & \multicolumn{4}{|c|}{Nemotron4-15B (FP32 Accuracy = 64.3\%)} \\ 
 \hline
 \hline
 64 & 36.71 & 38.85 & 38.13 & 38.92 & 63.17 & 62.36 & 63.72 & 64.09 \\
 \hline
 32 & 37.95 & 38.69 & 39.45 & 38.34 & 64.05 & 62.30 & 63.8 & 64.33  \\
 \hline
 16 & 38.88 & 38.80 & 38.31 & 38.92 & 63.22 & 63.51 & 63.93 & 64.43  \\
 \hline
\end{tabular}
\caption{\label{tab:mmlu_abalation} Accuracy on MMLU dataset across GPT3-22B, Llama2-7B, 70B and Nemotron4-15B models.}
\end{table}


%\subsection{Perplexity achieved by various LO-BCQ configurations on LM evaluation harness}

\begin{table} \centering
\begin{tabular}{|c||c|c|c|c||c|c|c|c|} 
\hline
 $L_b \rightarrow$& \multicolumn{4}{c||}{8} & \multicolumn{4}{c||}{8}\\
 \hline
 \backslashbox{$L_A$\kern-1em}{\kern-1em$N_c$} & 2 & 4 & 8 & 16 & 2 & 4 & 8 & 16  \\
 %$N_c \rightarrow$ & 2 & 4 & 8 & 16 & 2 & 4 & 2 \\
 \hline
 \hline
 \multicolumn{5}{|c|}{Race (FP32 Accuracy = 37.51\%)} & \multicolumn{4}{|c|}{Boolq (FP32 Accuracy = 64.62\%)} \\ 
 \hline
 \hline
 64 & 36.94 & 37.13 & 36.27 & 37.13 & 63.73 & 62.26 & 63.49 & 63.36 \\
 \hline
 32 & 37.03 & 36.36 & 36.08 & 37.03 & 62.54 & 63.51 & 63.49 & 63.55  \\
 \hline
 16 & 37.03 & 37.03 & 36.46 & 37.03 & 61.1 & 63.79 & 63.58 & 63.33  \\
 \hline
 \hline
 \multicolumn{5}{|c|}{Winogrande (FP32 Accuracy = 58.01\%)} & \multicolumn{4}{|c|}{Piqa (FP32 Accuracy = 74.21\%)} \\ 
 \hline
 \hline
 64 & 58.17 & 57.22 & 57.85 & 58.33 & 73.01 & 73.07 & 73.07 & 72.80 \\
 \hline
 32 & 59.12 & 58.09 & 57.85 & 58.41 & 73.01 & 73.94 & 72.74 & 73.18  \\
 \hline
 16 & 57.93 & 58.88 & 57.93 & 58.56 & 73.94 & 72.80 & 73.01 & 73.94  \\
 \hline
\end{tabular}
\caption{\label{tab:mmlu_abalation} Accuracy on LM evaluation harness tasks on GPT3-1.3B model.}
\end{table}

\begin{table} \centering
\begin{tabular}{|c||c|c|c|c||c|c|c|c|} 
\hline
 $L_b \rightarrow$& \multicolumn{4}{c||}{8} & \multicolumn{4}{c||}{8}\\
 \hline
 \backslashbox{$L_A$\kern-1em}{\kern-1em$N_c$} & 2 & 4 & 8 & 16 & 2 & 4 & 8 & 16  \\
 %$N_c \rightarrow$ & 2 & 4 & 8 & 16 & 2 & 4 & 2 \\
 \hline
 \hline
 \multicolumn{5}{|c|}{Race (FP32 Accuracy = 41.34\%)} & \multicolumn{4}{|c|}{Boolq (FP32 Accuracy = 68.32\%)} \\ 
 \hline
 \hline
 64 & 40.48 & 40.10 & 39.43 & 39.90 & 69.20 & 68.41 & 69.45 & 68.56 \\
 \hline
 32 & 39.52 & 39.52 & 40.77 & 39.62 & 68.32 & 67.43 & 68.17 & 69.30  \\
 \hline
 16 & 39.81 & 39.71 & 39.90 & 40.38 & 68.10 & 66.33 & 69.51 & 69.42  \\
 \hline
 \hline
 \multicolumn{5}{|c|}{Winogrande (FP32 Accuracy = 67.88\%)} & \multicolumn{4}{|c|}{Piqa (FP32 Accuracy = 78.78\%)} \\ 
 \hline
 \hline
 64 & 66.85 & 66.61 & 67.72 & 67.88 & 77.31 & 77.42 & 77.75 & 77.64 \\
 \hline
 32 & 67.25 & 67.72 & 67.72 & 67.00 & 77.31 & 77.04 & 77.80 & 77.37  \\
 \hline
 16 & 68.11 & 68.90 & 67.88 & 67.48 & 77.37 & 78.13 & 78.13 & 77.69  \\
 \hline
\end{tabular}
\caption{\label{tab:mmlu_abalation} Accuracy on LM evaluation harness tasks on GPT3-8B model.}
\end{table}

\begin{table} \centering
\begin{tabular}{|c||c|c|c|c||c|c|c|c|} 
\hline
 $L_b \rightarrow$& \multicolumn{4}{c||}{8} & \multicolumn{4}{c||}{8}\\
 \hline
 \backslashbox{$L_A$\kern-1em}{\kern-1em$N_c$} & 2 & 4 & 8 & 16 & 2 & 4 & 8 & 16  \\
 %$N_c \rightarrow$ & 2 & 4 & 8 & 16 & 2 & 4 & 2 \\
 \hline
 \hline
 \multicolumn{5}{|c|}{Race (FP32 Accuracy = 40.67\%)} & \multicolumn{4}{|c|}{Boolq (FP32 Accuracy = 76.54\%)} \\ 
 \hline
 \hline
 64 & 40.48 & 40.10 & 39.43 & 39.90 & 75.41 & 75.11 & 77.09 & 75.66 \\
 \hline
 32 & 39.52 & 39.52 & 40.77 & 39.62 & 76.02 & 76.02 & 75.96 & 75.35  \\
 \hline
 16 & 39.81 & 39.71 & 39.90 & 40.38 & 75.05 & 73.82 & 75.72 & 76.09  \\
 \hline
 \hline
 \multicolumn{5}{|c|}{Winogrande (FP32 Accuracy = 70.64\%)} & \multicolumn{4}{|c|}{Piqa (FP32 Accuracy = 79.16\%)} \\ 
 \hline
 \hline
 64 & 69.14 & 70.17 & 70.17 & 70.56 & 78.24 & 79.00 & 78.62 & 78.73 \\
 \hline
 32 & 70.96 & 69.69 & 71.27 & 69.30 & 78.56 & 79.49 & 79.16 & 78.89  \\
 \hline
 16 & 71.03 & 69.53 & 69.69 & 70.40 & 78.13 & 79.16 & 79.00 & 79.00  \\
 \hline
\end{tabular}
\caption{\label{tab:mmlu_abalation} Accuracy on LM evaluation harness tasks on GPT3-22B model.}
\end{table}

\begin{table} \centering
\begin{tabular}{|c||c|c|c|c||c|c|c|c|} 
\hline
 $L_b \rightarrow$& \multicolumn{4}{c||}{8} & \multicolumn{4}{c||}{8}\\
 \hline
 \backslashbox{$L_A$\kern-1em}{\kern-1em$N_c$} & 2 & 4 & 8 & 16 & 2 & 4 & 8 & 16  \\
 %$N_c \rightarrow$ & 2 & 4 & 8 & 16 & 2 & 4 & 2 \\
 \hline
 \hline
 \multicolumn{5}{|c|}{Race (FP32 Accuracy = 44.4\%)} & \multicolumn{4}{|c|}{Boolq (FP32 Accuracy = 79.29\%)} \\ 
 \hline
 \hline
 64 & 42.49 & 42.51 & 42.58 & 43.45 & 77.58 & 77.37 & 77.43 & 78.1 \\
 \hline
 32 & 43.35 & 42.49 & 43.64 & 43.73 & 77.86 & 75.32 & 77.28 & 77.86  \\
 \hline
 16 & 44.21 & 44.21 & 43.64 & 42.97 & 78.65 & 77 & 76.94 & 77.98  \\
 \hline
 \hline
 \multicolumn{5}{|c|}{Winogrande (FP32 Accuracy = 69.38\%)} & \multicolumn{4}{|c|}{Piqa (FP32 Accuracy = 78.07\%)} \\ 
 \hline
 \hline
 64 & 68.9 & 68.43 & 69.77 & 68.19 & 77.09 & 76.82 & 77.09 & 77.86 \\
 \hline
 32 & 69.38 & 68.51 & 68.82 & 68.90 & 78.07 & 76.71 & 78.07 & 77.86  \\
 \hline
 16 & 69.53 & 67.09 & 69.38 & 68.90 & 77.37 & 77.8 & 77.91 & 77.69  \\
 \hline
\end{tabular}
\caption{\label{tab:mmlu_abalation} Accuracy on LM evaluation harness tasks on Llama2-7B model.}
\end{table}

\begin{table} \centering
\begin{tabular}{|c||c|c|c|c||c|c|c|c|} 
\hline
 $L_b \rightarrow$& \multicolumn{4}{c||}{8} & \multicolumn{4}{c||}{8}\\
 \hline
 \backslashbox{$L_A$\kern-1em}{\kern-1em$N_c$} & 2 & 4 & 8 & 16 & 2 & 4 & 8 & 16  \\
 %$N_c \rightarrow$ & 2 & 4 & 8 & 16 & 2 & 4 & 2 \\
 \hline
 \hline
 \multicolumn{5}{|c|}{Race (FP32 Accuracy = 48.8\%)} & \multicolumn{4}{|c|}{Boolq (FP32 Accuracy = 85.23\%)} \\ 
 \hline
 \hline
 64 & 49.00 & 49.00 & 49.28 & 48.71 & 82.82 & 84.28 & 84.03 & 84.25 \\
 \hline
 32 & 49.57 & 48.52 & 48.33 & 49.28 & 83.85 & 84.46 & 84.31 & 84.93  \\
 \hline
 16 & 49.85 & 49.09 & 49.28 & 48.99 & 85.11 & 84.46 & 84.61 & 83.94  \\
 \hline
 \hline
 \multicolumn{5}{|c|}{Winogrande (FP32 Accuracy = 79.95\%)} & \multicolumn{4}{|c|}{Piqa (FP32 Accuracy = 81.56\%)} \\ 
 \hline
 \hline
 64 & 78.77 & 78.45 & 78.37 & 79.16 & 81.45 & 80.69 & 81.45 & 81.5 \\
 \hline
 32 & 78.45 & 79.01 & 78.69 & 80.66 & 81.56 & 80.58 & 81.18 & 81.34  \\
 \hline
 16 & 79.95 & 79.56 & 79.79 & 79.72 & 81.28 & 81.66 & 81.28 & 80.96  \\
 \hline
\end{tabular}
\caption{\label{tab:mmlu_abalation} Accuracy on LM evaluation harness tasks on Llama2-70B model.}
\end{table}

%\section{MSE Studies}
%\textcolor{red}{TODO}


\subsection{Number Formats and Quantization Method}
\label{subsec:numFormats_quantMethod}
\subsubsection{Integer Format}
An $n$-bit signed integer (INT) is typically represented with a 2s-complement format \citep{yao2022zeroquant,xiao2023smoothquant,dai2021vsq}, where the most significant bit denotes the sign.

\subsubsection{Floating Point Format}
An $n$-bit signed floating point (FP) number $x$ comprises of a 1-bit sign ($x_{\mathrm{sign}}$), $B_m$-bit mantissa ($x_{\mathrm{mant}}$) and $B_e$-bit exponent ($x_{\mathrm{exp}}$) such that $B_m+B_e=n-1$. The associated constant exponent bias ($E_{\mathrm{bias}}$) is computed as $(2^{{B_e}-1}-1)$. We denote this format as $E_{B_e}M_{B_m}$.  

\subsubsection{Quantization Scheme}
\label{subsec:quant_method}
A quantization scheme dictates how a given unquantized tensor is converted to its quantized representation. We consider FP formats for the purpose of illustration. Given an unquantized tensor $\bm{X}$ and an FP format $E_{B_e}M_{B_m}$, we first, we compute the quantization scale factor $s_X$ that maps the maximum absolute value of $\bm{X}$ to the maximum quantization level of the $E_{B_e}M_{B_m}$ format as follows:
\begin{align}
\label{eq:sf}
    s_X = \frac{\mathrm{max}(|\bm{X}|)}{\mathrm{max}(E_{B_e}M_{B_m})}
\end{align}
In the above equation, $|\cdot|$ denotes the absolute value function.

Next, we scale $\bm{X}$ by $s_X$ and quantize it to $\hat{\bm{X}}$ by rounding it to the nearest quantization level of $E_{B_e}M_{B_m}$ as:

\begin{align}
\label{eq:tensor_quant}
    \hat{\bm{X}} = \text{round-to-nearest}\left(\frac{\bm{X}}{s_X}, E_{B_e}M_{B_m}\right)
\end{align}

We perform dynamic max-scaled quantization \citep{wu2020integer}, where the scale factor $s$ for activations is dynamically computed during runtime.

\subsection{Vector Scaled Quantization}
\begin{wrapfigure}{r}{0.35\linewidth}
  \centering
  \includegraphics[width=\linewidth]{sections/figures/vsquant.jpg}
  \caption{\small Vectorwise decomposition for per-vector scaled quantization (VSQ \citep{dai2021vsq}).}
  \label{fig:vsquant}
\end{wrapfigure}
During VSQ \citep{dai2021vsq}, the operand tensors are decomposed into 1D vectors in a hardware friendly manner as shown in Figure \ref{fig:vsquant}. Since the decomposed tensors are used as operands in matrix multiplications during inference, it is beneficial to perform this decomposition along the reduction dimension of the multiplication. The vectorwise quantization is performed similar to tensorwise quantization described in Equations \ref{eq:sf} and \ref{eq:tensor_quant}, where a scale factor $s_v$ is required for each vector $\bm{v}$ that maps the maximum absolute value of that vector to the maximum quantization level. While smaller vector lengths can lead to larger accuracy gains, the associated memory and computational overheads due to the per-vector scale factors increases. To alleviate these overheads, VSQ \citep{dai2021vsq} proposed a second level quantization of the per-vector scale factors to unsigned integers, while MX \citep{rouhani2023shared} quantizes them to integer powers of 2 (denoted as $2^{INT}$).

\subsubsection{MX Format}
The MX format proposed in \citep{rouhani2023microscaling} introduces the concept of sub-block shifting. For every two scalar elements of $b$-bits each, there is a shared exponent bit. The value of this exponent bit is determined through an empirical analysis that targets minimizing quantization MSE. We note that the FP format $E_{1}M_{b}$ is strictly better than MX from an accuracy perspective since it allocates a dedicated exponent bit to each scalar as opposed to sharing it across two scalars. Therefore, we conservatively bound the accuracy of a $b+2$-bit signed MX format with that of a $E_{1}M_{b}$ format in our comparisons. For instance, we use E1M2 format as a proxy for MX4.

\begin{figure}
    \centering
    \includegraphics[width=1\linewidth]{sections//figures/BlockFormats.pdf}
    \caption{\small Comparing LO-BCQ to MX format.}
    \label{fig:block_formats}
\end{figure}

Figure \ref{fig:block_formats} compares our $4$-bit LO-BCQ block format to MX \citep{rouhani2023microscaling}. As shown, both LO-BCQ and MX decompose a given operand tensor into block arrays and each block array into blocks. Similar to MX, we find that per-block quantization ($L_b < L_A$) leads to better accuracy due to increased flexibility. While MX achieves this through per-block $1$-bit micro-scales, we associate a dedicated codebook to each block through a per-block codebook selector. Further, MX quantizes the per-block array scale-factor to E8M0 format without per-tensor scaling. In contrast during LO-BCQ, we find that per-tensor scaling combined with quantization of per-block array scale-factor to E4M3 format results in superior inference accuracy across models. 



\end{document}
