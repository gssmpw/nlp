\section{Limitations} \label{sec:limitations}

While the above results demonstrate an important step toward flexible and robust humanoid locomotion, our proposed technique is not a panacea. 
%
Both HLIP and CI-MPC require parameter tuning, and their combination only increases the complexity of this process. While we used only one set of parameters for all the experiments, we did find that some parameters induced sharp tradeoffs. For example, a lower weight on base orientation tracking gave more natural-looking gaits, but reduced push recovery performance.
%


Our CI-MPC implementation uses significantly simplified collision geometries. This enables fast solve times, but precludes behaviors that involve contact away from the hands and the feet. As a result, the robot is not able to automatically recover from a fall. Furthermore, our CI-MPC solver's performance is reliant on smooth collision geometries, as sharp corners introduce problematic discontinuous gradients. 
%
Similarly, self-collisions present a major failure mode in the current implementation. Adding self-collision constraints either in the optimization problem \cite{grandia2021multi} or with a high order control barrier function \cite{khazoom2024tailoring, ames2019control, singletary2021safety} presents an obvious next step for improving reliability.

Finally, there are instances in which HLIP's suggested contact sequence guides the robot in an unhelpful direction. For example, if the robot is standing and pushed to the left, HLIP might suggest lifting the right leg, depending on the timing of the gait cycle. This could be mitigated with a richer reduced-order model, but illustrates a trade-off inherent to guiding whole-body behaviors with a reduced-order model.
