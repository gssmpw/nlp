\section{Introduction and Related Work}\label{sec:introduction}
Humanoid robots, due to their anthropomorphic structure, are well suited to perform tasks in environments built for humans. 
% Anthropomorphic humanoid robots are often pursued because their design aligns well with environments and tasks built for humans. 
To realize this potential, humanoid robots must transition from the lab and highly structured settings to real-world environments.  This necessitates whole-body control methods that can replan online in a contact rich fashion---leveraging the environment to facilitate tasks and achieve dynamic stability.  
% Humanoid robots are transitioning from controlled environments to real-world applications, creating demand for robust whole-body control methods. 
State-of-the-art whole-body control has advanced significantly in recent years \cite{kuindersma2016optimization, khazoom_humanoid_2022,khazoom2024tailoring}, but high degree-of-freedom (DoF), nonlinear, and hybrid dynamics make whole-body control of humanoid robots a formidable challenge---especially for contact aware real-time controllers.

\begin{figure}
    \centering
    \href{https://rom-cimpc.github.io/}
    {
    \includegraphics[width=\linewidth]{Figures/hero.png}
    }
    \vspace{-14pt} 
    \caption{A simulated Achilles humanoid walks over unmodeled terrain near a wall. The HLIP reduced-order model provides a nominal gait, while CI-MPC adjusts the contact sequence, bracing the arm against the wall.}
    \label{fig:hero}
    \vspace{-18pt} 
\end{figure}
%
Directly solving the whole-body humanoid optimal control problem is intractable in general. The problem involves not only high-dimensional nonlinear dynamics, but also a power set of potential contact modes. Whole body control becomes even more difficult when the robot is tasked with not only walking over flat ground, but also using its arms and other objects in the environment to move about, as shown in Fig.~\ref{fig:hero}.
%%% combining the paragraphs---the existing ones were too short
To circumvent these optimal control challenges, researchers have developed a rich literature on reduced-order models for locomotion. Point mass models like the Linear Inverted Pendulum (LIP) \cite{kajita20013d}, the Spring Loaded Inverted Pendulum (SLIP) \cite{blickhan1989spring, wensing_generation_2013}, and associated variants \cite{xiong_orbit_2019, dai2024multi,gong2020angular} capture the key characteristics of walking and running. Centroidal dynamics models \cite{orin2013centroidal, dai2014whole} offer a richer characterization while still avoiding the details of each limb's motion. While reduced-order models can achieve robust locomotion and enable fast real-time control, expressiveness at the whole-body level is lost due to abstraction of the full dynamics.

Methods such as \cite{zhao2017multi,hereid20163d,westervelt_hybrid_2003} directly solve the multi-domain trajectory generation problem for the whole body dynamics. To synthesize stable gaits, these methods involve solving difficult nonlinear optimization problems over several possible contact modes. Walking gaits produced with these methods often require an additional foot placement heuristic to mitigate model mismatch. Other methods like \cite{deits2014footstep, ding2020kinodynamic} guide offline trajectory generation with mixed-integer optimization. While the quality of trajectories for these methods are rich, computational requirements limit their usage for settings which require fast replanning.

Model Predictive Control (MPC) offers an alternative approach to the whole-body control problem. MPC addresses the intractability of whole-body control by optimizing over an individual trajectory in receding horizon fashion \cite{wensing_optimization-based_2024, tassa2012synthesis}, rather than considering the whole state-space. Solving the resulting non-convex optimization problem is difficult, and many MPC controllers require a pre-determined contact sequence. Recent work has shown that it is possible to optimize over both the contact sequence and whole-body dynamics in real time \cite{kurtz2023inverse, cleach_fast_2023, aydinoglu2023consensus, kim2023contact}. These Contact-Implicit MPC (CI-MPC) methods leverage specialized solvers to deal with the difficult numerics of contact, and have shown promise for a variety of locomotion and manipulation tasks. Nonetheless, they remain vulnerable to local minima and can require significant parameter tuning, especially for bipedal walking. In practice, direct application of CI-MPC for bipedal locomotion often leads to irregular gaits and foot dragging.

In this paper, we propose a framework that uses the Hybrid Linear Inverted Pendulum (HLIP) \cite{xiong_orbit_2019} to generate reference trajectories for a Contact-Implicit Model Predictive Controller (CI-MPC). This framework inherits the robust foot placement of HLIP while gaining full-body coordination from CI-MPC. HLIP suggests a reasonable contact sequence, guiding CI-MPC towards reasonable  behavior, but the robot is free to deviate from this suggestion or make contact with other objects in the environment.
%%% combining the paragraphs---the existing ones were too short
We demonstrate the effectiveness of this approach in simulation experiments with the 24-DoF Achilles humanoid. HLIP-guided CI-MPC enables versatile and robust locomotion, including walking over rough terrain, improved disturbance rejection, whole-body coordination of legs and arms, walking at different heights, and reaching out to brace against objects in the environment.

The rest of the paper is structured as follows. In Section \ref{sec:preliminary} we cover the background material  for our control framework, which includes a description of the system dynamics, the HLIP model, and CI-MPC. Next, in Section \ref{sec:control_approach} we present our control approach, where we describe how user inputs are mapped to HLIP references, and how these references are passed to our CI-MPC solver to generate stable, expressive behavior. We demonstrate the effectiveness of our approach in Section \ref{sec:results} through several simulation experiments on the 24-DoF humanoid Achilles. In Section \ref{sec:limitations} we discuss limitations of the approach. Finally, in Section \ref{sec:conclusion} we provide some concluding remarks summarizing our findings.

% Something, something, something...
% %
% It is well known that solving Nonlinear MPC problems for high-order systems is challenging and time-consuming, and the outcome can heavily depend on the quality of the nominal reference trajectory. Reference trajectories are often generated by simply forward propagating some nominal pose of the robot, or by using other simplified heuristics. In this paper we investigate use of the Hybrid Linear Inverted Pendulum (HLIP) model \cite{xiong20223} to rapidly generate high-quality reference trajectories for the CIMPC solver. The HLIP controller is a simple and robust controller that has proven itself to be effective in stabilizing highly under-actuated bipedal robots \cite{xiong20223} \cite{ghansah2024dynamic}. Furthermore, because of its linear step-to-step dynamics, it can quickly be forward-propagated to generate powerful reference trajectories. 
