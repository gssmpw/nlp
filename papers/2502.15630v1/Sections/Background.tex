\begin{figure*}[t]
    \vspace{10pt} 
    \centering
    \includegraphics[width=0.95\linewidth]{Figures/control_diagram.pdf}
    \caption{Control architecture that combines the HLIP with CI-MPC. User commands are given to the HLIP. Then, via inverse kinematics and finite difference, we obtain a state trajectory where after combination produces a trajectory for the legs. CI-MPC tracks this trajectory that is interpolated and passed to low-level control of the robot.}
    \label{fig:control_architecture}
    \vspace{-20pt} 
\end{figure*}

\section{Preliminary}\label{sec:preliminary}
%%%%%%%%%%%%%%%%%%%%%%%%%%%%%%%%%%%%%%%%%%%%%%%%%%%%%%%%%%%%%%%%%%%%%%%%%%%%%%%%%%%%%%%%%%%%%%%%%%
\subsection{System Dynamics}
We consider the full 3D dynamics of a humanoid robot. Let $n$ be the number of generalized coordinates of the system. We represent the configuration as:
%
$
    \bm q = 
    [
    \bm q_b^\T , \bm q_j^\T
    ]^\T
    \in  \Qcal
    \triangleq  SE(3) \times \Qcal_j,
$
%
and the velocities as:
$
    \bm v = 
    [
    \bm v_b^\T , \bm v_j^\T
    ]^\T
    \in  \Vcal 
    \triangleq \mathfrak{se}(3) \times \Vcal_j,
$
%
where the subscript $b$ denotes the floating base coordinates and $j$ denotes the joint coordinates that include the arms and legs.
%
The robot whole body state is then represented as $\bm x~=~[\bm q^\T, \bm v^\T]^{\T} \in \Xcal \triangleq \Qcal \times \Vcal$.

The dynamics of systems that make and break contact with the environment are often modeled by Euler Lagrange equations and holonomic constraints. In this work, we model contact forces via a differentiable function of the robot's state, $\bs \lambda(\bm q)$, as detailed in Section \ref{sec:cimpc}. Thus, our whole body dynamics can be expressed as:
%
\begin{align} 
    \bm D(\bm q) \dot{\bm v} + \bm{H}(\bm q, \bm v) &= \bm B \bm u + \bm J_c(\bm q)^\T \bs \lambda (\bm q)\label{eq:euler-lagrange} 
\end{align}
% 
where $\bm D: \Qcal \rightarrow \Spp^n$ is the mass-inertia matrix, $\bm H : \Qcal \times \Vcal \rightarrow \R^n$ contains centrifugal, Coriolis, and gravitational terms, $\bm B \in \R^{n \times m}$ is the input selection matrix, and the input vector, $\bm u \in \Ucal \subset \R^m$, includes the joint torques. The contact Jacobian, $\bm J_c: \Qcal \rightarrow \R^{3 n_c\times n}$ maps reaction forces, $\bs \lambda \in \R^{3 n_c}$, to the whole body dynamics, where $n_c$ is the number of contact points with the environment.

%%%%%%%%%%%%%%%%%%%%%%%%%%%%%%%%%%%%%%%%%%%%%%%%%%%%%%%%%%%%%%%%%%%%%%%%%%%%%%%%%%%%%%%%%%%%%%%%%%

\subsection{Hybrid Linear Inverted Pendulum Model}
The HLIP model considers a planar linear inverted pendulum point-mass model with fixed height, $z_0 \in \mathbb{R}$, where the state comprises of the horizontal position, $p \in \mathbb{R}$, and velocity, $v \in \mathbb{R}$, of the point mass relative to the stance foot, $\bm x^{\rm H} = [p, v]^\T$. In this model, the pendulum is assumed to be completely unactuated during the continuous single support phase (SSP). By assuming that the double support period (DSP) is instantaneous, the HLIP dynamics can be expressed as the following single domain hybrid control system,
\begin{equation} \label{eq:hlip_hc}
    \mathcal{HC} 
    =
    \begin{cases} 
        \dot{\bm{x}}_{\rm{ssp}}^{\rm H} = \bm{A}_{\rm{ssp}}\bm{x}_{\rm{ssp}}^{\rm H} & \; \text{if} \quad \bm{x}_{\rm{ssp}}^{\rm H} \in \mathcal{D} \backslash \mathcal{S} \\
        \bm{x}^+_{\rm{ssp}} = \Delta(\bm{x}^-_{\rm{ssp}}) & \; \text{if} \quad  \bm{x}^-_{\rm{ssp}} \in \mathcal{S} 
    \end{cases},
\end{equation}
where $\Dcal$ represents the set of continuous state dynamics and $\Delta: \Scal \rightarrow \Dcal$ is the reset map that maps pre-impact states to post-impact states at the switching surface $\mathcal{S}$. 

The switching of this system corresponds to completing an SSP, i.e., the pendulum falls over but places the next foot at a different location to stabilize $\bm x^\rm{H}$. This introduces the notion of step-to-step dynamics (S2S) which are dynamics considered at the discrete switches. At this level, we can stabilize the HLIP with a foot placement controller to stabilize the S2S dynamics \cite{xiong_3-d_2022}. Achieving walking with this model entails simply tracking the HLIP state. Furthermore, extending the HLIP controller for 3D simply requires an orthogonal composition of two planar HLIPs. We denote the hybrid solution of the 3D HLIP \eqref{eq:hlip_hc}, as:
%
\begin{equation} \label{eq:HLIP_solution}
    \chi(\bm x_0^{\rm {H}}) = (\Lambda, I, C)
\end{equation}
%
where $\bm x_0^{\rm {H}}$ is the initial condition, $\Lambda$ is an indexing set for each continuous SSP and $I$ is the set of time intervals for each set of continuous solutions of the SSP, $C$. 

%%%%%%%%%%%%%%%%%%%%%%%%%%%%%%%%%%%%%%%%%%%%%%%%%%%%%%%%%%%%%%%%%%%%%%%%%%%%%%%%%%%%%%%%%%%%%%%%%%

\subsection{Contact-Implicit Model Predictive Control} \label{sec:cimpc}
We aim to achieve full-body locomotion, where the arms and legs coordinate to actively stabilize the robot and interact with the environment. CI-MPC provides a means of generating such motions, including making and breaking contact.

%Thus, the objective is to design control inputs, $\bm u(t)$, such that the robot achieves robust and efficient locomotion given a reference trajectory, $\bm x_{\rm{ref}}(t)$. These control inputs will coordinate the movement of the legs and arms, making and breaking contact as needed.

Specifically, given an initial condition $\bm x_0$ and a reference $\bm x_{\rm{ref}}(t)$, CI-MPC solves the trajectory optimization
\begin{subequations} \label{eq:continuous-time-RHC}
\begin{align}
    \underset{\bm u(t)}{\rm {min}} \; \; & \int_{0}^{T} l(\bm x(t), {\bm u(t)}, t) dt + l_f(\bm x(T)), \\
    \rm{s.t.} \; \; 
    & \bm D(\bm q) \dot{\bm v} + \bm{H}(\bm q, \bm v) 
    = \bm B \bm u + \bm J_c(\bm q)^\T \bs \lambda (\bm q, \bm v), \\
    & \bm x(0) = \bm x_0,
\end{align}
\end{subequations}
in receding horizon fashion, where $l(\bm x, \bm u, t) = \| \bm x - \bm x_{\rm{ref}}(t) \|_{\bm Q}^2 + \|\bm u \|_{\bm R}^2$ is a quadratic running cost, $l_f(\bm x) = \| \bm x - \bm x_{\rm{ref}}(T) \|_{\bm V}^2$ is a terminal cost, and $\bm Q, \bm R,\bm V \succeq \bm 0$ are appropriately sized weighting matrices.

While there are various methods to solve this problem \cite{cleach_fast_2023, aydinoglu2023consensus, kim2023contact, jin2024complementarity}, we adopt the Inverse Dynamics Trajectory Optimization (IDTO) approach of \cite{kurtz2023inverse}. IDTO is relatively simple and does not support arbitrary constraints, but its simplicity enables speed and reliability. 

In IDTO, we discretize the trajectory into $N$ steps of size $\Delta t$, and use the generalized positions 
\begin{equation}
    \bm q = [\bm q_0, \bm q_1, \dots, \bm q_N]
\end{equation}
as the only decision variables. 

From generalized positions, we compute velocities and accelerations through forward and backward differences:
\begin{alignat}{2} 
    {\bm v}_k (\bm q) &= \bm N^\dagger(\bm q_k)
    \frac{\bm q_k - \bm q_{k-1}}{\Delta t}, &&\quad\forall k = 1,\dots, N, \label{eq:velocity_q}\\
    \dot {\bm v}_k (\bm q) &= 
    \frac{{\bm v}_{k+1}(\bm q) - {\bm v}_{k}(\bm q)}{\Delta t}, \; \;&&\quad\forall k = 0, \dots, N-1 \label{eq:acceleration_q}
\end{alignat}
where $\bm N^\dagger$ is the kinematic map from generalized position derivatives\footnote{$\bm N(\bm q)^\dagger$ is the left pseudo-inverse of $\bm N(\bm q)$, which maps generalized velocities $\bm v$ to the time derivative of generalized positions $\dot{\bm q}$ \cite{underactuated}.} to generalized velocities, $\bm v = \bm N^\dagger \dot{\bm q}$.
%
With a compliant contact model \cite{kurtz2023inverse}, the leg and arm contact forces $\bs \lambda$ are also a function of $\bm q$:
%
\begin{equation} \label{eq:GRF_q}
    \bs \lambda_k (\bm q) = \rm{func} (\bm q_{k+1}, {\bm v}_{k+1}(\bm q)).
\end{equation}
\begin{remark}
    This compliant contact model is used for CI-MPC but not for simulation, where we instead use the physics-based rigid contact model implemented in Drake \cite{drake}.
\end{remark}

Finally, the inverse dynamics \eqref{eq:euler-lagrange} allow us to write generalized forces as a function of $\bm q$ as well:
\begin{multline} \label{eq:inverse_dynamics}
    \bm B \bm u(\bm q) = \bm D(\bm q_{k+1}) \dot {\bm v}_k + 
    \bm H(\bm q_{k+1}, \bm v_{k+1}) \\
    - \bm J_c^\T(\bm q_{k+1}) \bs \lambda_k(\bm q_{k+1}),
\end{multline}
resulting in a compact nonlinear least squares problem where $\bm q$ are the only decision variable. Underactuation (e.g., low-rank $\bm B$) can be handled via constraint or penalty methods \cite{kurtz2023inverse}. 
% We refer the interested reader to \cite{kurtz2023inverse} for further details on solving this problem.

Importantly, since the contact forces are a smooth function of $\bm q$, we solve for the contact schedules and reaction forces \textit{implicitly}. We use this notion to our advantage to automatically consider different contact modes.

%We assume that an accurate state estimate is available, and that the environment geometry and any objects in it are known and folded into the dynamics.

% \subsection{Problem Statement}

% \begin{itemize}
%     \item Given some dynamic, how do you track a reference vel. 
%     \item Apply inputs u to track a base velocity
% \end{itemize}
