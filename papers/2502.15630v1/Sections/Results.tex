\section{Results}\label{sec:results}

\begin{figure*}
    \vspace{10pt} 
    \centering
    \begin{subfigure}{0.24\linewidth}
        \includegraphics[width=\linewidth]{Figures/terrain.png}
        \caption{Traversing unmodeled terrain}
        \label{fig:scenarios:terrain}
    \end{subfigure}
    \begin{subfigure}{0.24\linewidth}
        \includegraphics[width=\linewidth]{Figures/wall.png}
        \caption{Bracing against a wall}
        \label{fig:scenarios:wall}
    \end{subfigure}
    \begin{subfigure}{0.24\linewidth}
        \includegraphics[width=\linewidth]{Figures/disturbance.png}
        \caption{Recovering from disturbances}
        \label{fig:scenarios:wall}
    \end{subfigure}
    \begin{subfigure}{0.24\linewidth}
        \includegraphics[width=0.49\linewidth]{Figures/low_com.png}
        \includegraphics[width=0.49\linewidth]{Figures/high_com.png}
        \caption{Walking at various heights}
        \label{fig:scenarios:com}
    \end{subfigure}
    \caption{The Achilles humanoid in various simulation test scenarios. HLIP guides the robot toward a reasonable gait, while CI-MPC provides the flexibility to make and break contact on the fly.}
    \label{fig:scenarios}
\end{figure*}

Our proposed HLIP-guided CI-MPC approach enables a diverse array of behaviors, as illustrated in Fig.~\ref{fig:scenarios}. Each of these scenarios are also shown in the supplemental material. 
%
\subsection{Achilles 3D Humanoid}
We validated our approach on the novel 24-DoF Achilles humanoid platform. Notably, the lack of roll actuation at the feet introduces significant challenges in controlling the underactuated frontal plane dynamics \cite{westervelt2018feedback, buss2014preliminary}. Additionally, by considering the heels, toes, and hands, the robot presents 64 possible hybrid domains. We approximate the robot's contact geometries as spheres as depicted in Fig. \ref{fig:hero}. The proposed framework effectively handles these underactuated dynamics and the large space of contact modes.
%
\subsection{Simulation Environment}
We test the proposed method in a Drake simulation environment \cite{drake}. Note that while CI-MPC uses a simplified differentiable contact model \eqref{eq:GRF_q}, the simulator uses Drake's state-of-the-art model for rigid contact. We perform all experiments on an Ubuntu 22.04 PC with an AMD Ryzen 9 7950x @ 4.5 GHz, 128 GB RAM. 

\subsection{Parameters}

Key parameters for HLIP and CI-MPC planners are shown in Table~\ref{tab:HLIP_MPC_params}, while parameters for reference blending are listed in Table~\ref{tab:combo_parameters}. These parameters were used in all experiments.

\begin{table}
\centering
%\setlength{\tabcolsep}{5pt} % Adjust this value for tighter columns
\begin{tabular}{c c |c c c c }
    % Removed the top \hline here
    % First row: Simple content, no B2
    \multicolumn{2}{c}{\textbf{HLIP}} & \multicolumn{4}{c}{\textbf{CI-MPC}} \\
    \hline
    $T_{\rm{ssp}}$ [$s$] & $z_0$ [$m$] & $\Delta t$ [$s$]& $N$ & Freq. [$Hz$] & Iters.\\ 
    \hline
    0.35 & 0.62 & 0.05 & 25 & 50 &  3 \\
    \hline
\end{tabular}
\caption{HLIP and CI-MPC parameters used for experiments.}
\label{tab:HLIP_MPC_params}
\end{table}

\begin{table}
\centering
\setlength{\tabcolsep}{2.5pt} % Adjust this value for tighter columns
\begin{tabular}{c c | c c c c c c}
     \multicolumn{2}{c}{\textbf{Combination}} & \multicolumn{6}{c}{\textbf{Command Thresholds}}  \\
     \hline
        $\rho_1$ & $\rho_2$ &
        $v_x [\frac{m}{s}]$ & $v_y [\frac{m}{s}]$&
        $\omega_z [\frac{\rm{deg}}{s}]$& $z_0 [m]$&
        $v_{x,b} [\frac{m}{s}]$ & $v_{y,b} [\frac{m}{s}]$ \\
    \hline
    5.0 & 0.5 & 0.7 & 0.3 & 35.0 & 0.45 & 0.5 & 0.4 \\
    \hline
\end{tabular}
 
\caption{Parameters for $\alpha(\cdot)$ and maximum commands and base velocities considered in the diagonals of matrix $\bm P$.}
\label{tab:combo_parameters}
\end{table}

\begin{table}
    \centering
    \begin{tabular}{c|c c c}
         \textbf{Joint} & \textbf{Position} & \textbf{Velocity} & \textbf{Torque}  \\
         \hline
         Base orientation & 4.0 & 0.4 & 100.0 \\
         Base position & 1.0 & 0.1 & 100.0 \\
         Hip & 0.4 & 0.03 & 0.001 \\
         Knee & 0.3 & 0.03 & 0.001 \\
         Ankle & 0.8 & 0.08 & 0.0005 \\
         Shoulder & 0.1 & 0.01 & 0.005 \\
         Arm yaw & 0.01 & 0.001 & 0.005 \\
         Elbow & 0.01 & 0.003 & 0.005 \\
    \end{tabular}
    \caption{CI-MPC cost weights}
    \label{tab:mpc_weights}
\end{table}


Table~\ref{tab:mpc_weights} reports values of the diagonal cost weights matrices $\bm Q$ and $\bm R$ for CI-MPC. We use terminal cost weights $\bm V = 10 \bm Q$. Note that our CI-MPC solver, due to its inverse dynamics formulation, requires a penalty on torques on underactuated degrees of freedom like the base, see \cite{kurtz2023inverse} for details.

\subsection{Walking Results}
We begin with flat-ground walking. Applying CI-MPC alone results in a gait with significant foot dragging. Our proposed HLIP guidance results in the more natural gait shown in Fig.~\ref{fig:banner}. A heel-toe contact sequence emerges automatically during forward walking, while backward walking produces a toe-heel sequence. Furthermore, arm swinging emerges at high velocities and low walking heights.

\begin{figure} 
    \centering
    \includegraphics[width=0.8\linewidth]{Figures/velocity_tracking.pdf}
    \caption{Velocity tracking with HLIP only, CI-MPC only, and our proposed approach that combines the two. CI-MPC only and our proposed approach both take some time before moving forward, as the lowest-cost behavior at small velocity commands is to remain standing in place. Additionally, at low velocities, the the standing configuration reference dominates in \eqref{ref_combo}.}
    \label{fig:velocity_tracking}
    \vspace{-20pt} 
\end{figure}

Velocity tracking performance over flat ground is shown in Fig.~\ref{fig:velocity_tracking}. Our proposed combination of HLIP and CI-MPC provides tighter velocity tracking with less oscillation than either HLIP or CI-MPC alone. Note that HLIP alone continues to move back and forth after stopping, while the reference blending described in Sec.~\ref{sec:control_approach} allows the robot to come to a smooth and complete stop.

\subsection{Disturbances}
Next, we consider the robustness of our proposed approach to external disturbances. To measure this, we apply randomly generated forces in the horizontal plane to the base of the robot and record whether the robot remains standing for 4 seconds. Fig.~\ref{fig:disturbance_stats} shows the results of this experiment. HLIP remained standing in only 30\% of trials, while our proposed approach achieved a 62\% success rate.

\begin{figure}
    \vspace{10pt}
    \centering
    \includegraphics[width=1.0 \linewidth]{Figures/pizza.pdf}
    \caption{Push recovery comparison between HLIP only (left) and our proposed approach (right). A disturbance force was applied to the base for 0.1 seconds.}
    \label{fig:disturbance_stats}
    \vspace{-10pt} 
\end{figure}


We randomized the timing of the push disturbance uniformly across the gait cycle. This blurs the boundary between falls and successes: the same disturbance might be more or less difficult depending when it is applied. 

\begin{figure} 
    \centering
    \includegraphics[width=\linewidth]{Figures/disturbance_schedule.pdf}
    \caption{CI-MPC modulates the HLIP nominal contact sequence after being pushed. The dashed line indicates the time at which the disturbance is applied.}
    \label{fig:disturbance_contact}
    \vspace{-14pt} 
\end{figure}

Fig.~\ref{fig:disturbance_contact} shows the contact schedule that arises after a push disturbance. While the robot is tracking a forward velocity, a $0.2$ second disturbance of $\boldsymbol{f} = [50, \; -50, \; 0]^\T$ $\rm N$ is applied at the base frame of the robot. The proposed framework allows the robot to deviate from a nominal orbit and come back to it after recovery.

\begin{table}
\centering
    \begin{tabular}{c c c | c c}
        \begin{tabular}{@{}c@{}}Estimation\\Noise $\sigma_x$\end{tabular} & 
        \begin{tabular}{@{}c@{}}Link Mass\\Error $\sigma_m$\end{tabular} & 
        \begin{tabular}{@{}c@{}}Unmodeled\\Rough Terrain\end{tabular} & 
        HLIP only & 
        Proposed \\ 
        \hline
        0 & 0 & Yes & 1 / 10 & \textbf{7 / 10} \\
        0 & 0.2 & No & \textbf{10 / 10} & 9 / 10 \\
        0.05 & 0 & No & 0 / 10 & \textbf{8 / 10} \\
        0.05 & 0.2 & No & 0 / 10 & \textbf{3 / 10} \\
        0.05 & 0.2 & Yes & 0 / 10 & \textbf{1 / 10} \\
        0.01 & 0.1 & Yes & 0 / 10 & \textbf{7 / 10} \\
        \hline
    \end{tabular}
    \caption{Success rates, walking with model and estimation error.}
    \label{tab:uncertainty}
\end{table}

\subsection{Model and State Uncertainty on Rough Terrain}
Finally, while we have not yet validated HLIP-guided CI-MPC on hardware, the preliminary tests shown in Tab.~\ref{tab:uncertainty} indicate some level of robustness to state estimation and modeling error. In these experiments, we add Gaussian noise to the
state estimate,
\begin{equation}
    \hat{\bm x} \gets \hat{\bm x} + \bs \epsilon_x, \quad \bs \epsilon_x \sim \mathcal{N}(\bm 0, \sigma_x^2 \bm I),
\end{equation}
randomly alter the mass of each link,
\begin{equation}
    m_i \gets \max(0, m_i \cdot \epsilon_{m,i}), \quad \epsilon_{m,i} \sim \mathcal{N}(1, \sigma_m^2),
\end{equation}
and add randomly generated spheres (as in Fig.~\ref{fig:scenarios:terrain}) as unmodeled rough terrain. The robot is commanded to walk forward at 0.5 m/s, and a trial is counted as successful if the robot remains standing after 10 seconds. The HLIP-only controller, which has been validated on hardware \cite{ghansah2024dynamic}, is considerably less robust in all but one of these trials.
