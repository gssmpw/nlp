% CVPR 2025 Paper Template; see https://github.com/cvpr-org/author-kit

\documentclass[10pt,twocolumn,letterpaper]{article}

%%%%%%%%% PAPER TYPE  - PLEASE UPDATE FOR FINAL VERSION
% \usepackage{cvpr}              % To produce the CAMERA-READY version
% \usepackage[review]{cvpr}      % To produce the REVIEW version
\usepackage[pagenumbers]{cvpr} % To force page numbers, e.g. for an arXiv version

% Import additional packages in the preamble file, before hyperref
%
% --- inline annotations
%
\newcommand{\red}[1]{{\color{red}#1}}
\newcommand{\todo}[1]{{\color{red}#1}}
\newcommand{\TODO}[1]{\textbf{\color{red}[TODO: #1]}}
% --- disable by uncommenting  
% \renewcommand{\TODO}[1]{}
% \renewcommand{\todo}[1]{#1}



\newcommand{\VLM}{LVLM\xspace} 
\newcommand{\ours}{PeKit\xspace}
\newcommand{\yollava}{Yo’LLaVA\xspace}

\newcommand{\thisismy}{This-Is-My-Img\xspace}
\newcommand{\myparagraph}[1]{\noindent\textbf{#1}}
\newcommand{\vdoro}[1]{{\color[rgb]{0.4, 0.18, 0.78} {[V] #1}}}
% --- disable by uncommenting  
% \renewcommand{\TODO}[1]{}
% \renewcommand{\todo}[1]{#1}
\usepackage{slashbox}
% Vectors
\newcommand{\bB}{\mathcal{B}}
\newcommand{\bw}{\mathbf{w}}
\newcommand{\bs}{\mathbf{s}}
\newcommand{\bo}{\mathbf{o}}
\newcommand{\bn}{\mathbf{n}}
\newcommand{\bc}{\mathbf{c}}
\newcommand{\bp}{\mathbf{p}}
\newcommand{\bS}{\mathbf{S}}
\newcommand{\bk}{\mathbf{k}}
\newcommand{\bmu}{\boldsymbol{\mu}}
\newcommand{\bx}{\mathbf{x}}
\newcommand{\bg}{\mathbf{g}}
\newcommand{\be}{\mathbf{e}}
\newcommand{\bX}{\mathbf{X}}
\newcommand{\by}{\mathbf{y}}
\newcommand{\bv}{\mathbf{v}}
\newcommand{\bz}{\mathbf{z}}
\newcommand{\bq}{\mathbf{q}}
\newcommand{\bff}{\mathbf{f}}
\newcommand{\bu}{\mathbf{u}}
\newcommand{\bh}{\mathbf{h}}
\newcommand{\bb}{\mathbf{b}}

\newcommand{\rone}{\textcolor{green}{R1}}
\newcommand{\rtwo}{\textcolor{orange}{R2}}
\newcommand{\rthree}{\textcolor{red}{R3}}
\usepackage{amsmath}
%\usepackage{arydshln}
\DeclareMathOperator{\similarity}{sim}
\DeclareMathOperator{\AvgPool}{AvgPool}

\newcommand{\argmax}{\mathop{\mathrm{argmax}}}     


\usepackage{indentfirst}
\usepackage{multirow}
\usepackage{makecell}
\usepackage{float}
\usepackage{caption}
\usepackage{graphicx}
%\usepackage{floatrow}
\usepackage{subcaption}
\usepackage[accsupp]{axessibility}
\PassOptionsToPackage{bookmarks=false}{hyperref}




% It is strongly recommended to use hyperref, especially for the review version.
% hyperref with option pagebackref eases the reviewers' job.
% Please disable hyperref *only* if you encounter grave issues, 
% e.g. with the file validation for the camera-ready version.
%
% If you comment hyperref and then uncomment it, you should delete *.aux before re-running LaTeX.
% (Or just hit 'q' on the first LaTeX run, let it finish, and you should be clear).
\definecolor{cvprblue}{rgb}{0.21,0.49,0.74}
\usepackage[pagebackref,breaklinks,colorlinks,allcolors=cvprblue]{hyperref}

%%%%%%%%% PAPER ID  - PLEASE UPDATE
\def\paperID{9285} % *** Enter the Paper ID here
\def\confName{CVPR}
\def\confYear{2025}

%%%%%%%%% TITLE - PLEASE UPDATE
\title{PI-HMR: Towards Robust In-bed Temporal Human Shape Reconstruction with Contact Pressure Sensing}

%%%%%%%%% AUTHORS - PLEASE UPDATE
\author{Ziyu Wu\footnotemark[1], Yufan Xiong\footnotemark[1], Mengting Niu, Fangting Xie, Quan Wan, Qijun Ying, Boyan Liu, Xiaohui Cai\footnotemark[2]\\
University of Science and Technology of China\\
% {\tt\small firstauthor@i1.org}
% For a paper whose authors are all at the same institution,
% omit the following lines up until the closing ``}''.
% Additional authors and addresses can be added with ``\and'',
% just like the second author.
% To save space, use either the email address or home page, not both
% \and
% Second Author\\
% Institution2\\
% First line of institution2 address\\
% {\tt\small secondauthor@i2.org}
}

\begin{document}

\crefname{section}{Sec.}{Secs.}
\Crefname{section}{Section}{Sections}
\Crefname{table}{Table}{Tables}
\crefname{table}{Tab.}{Tabs.}
\Crefname{figure}{Figure}{Figures}
\crefname{figure}{Fig.}{Figs.}

% Shrink space around figures. This beats manually adding negative \vspace commands everywhere.
\setlength{\floatsep}{6pt plus 1pt minus 2pt}
\setlength{\textfloatsep}{4pt plus 1pt minus 2pt}
\setlength{\dbltextfloatsep}{4pt plus 1pt minus 2pt}
\setlength{\dblfloatsep}{4pt plus 1pt minus 2pt}
\setlength{\intextsep}{0pt}
\setlength{\abovecaptionskip}{3pt}
\setlength{\belowcaptionskip}{1pt}
\setlength{\parskip}{0pt}
% % around equations
\setlength{\abovedisplayskip}{0pt}
\setlength{\belowdisplayskip}{0pt}
\setlength\abovedisplayshortskip{0pt}
\setlength\belowdisplayshortskip{0pt}


\twocolumn[{
\renewcommand\twocolumn[1][]{#1}
% \vspace{-2em}
\maketitle
\begin{center}
    \centering
    \vspace{-2.5em}
  \includegraphics[width=1.\linewidth]{images/cover_pic.pdf}
  \captionof{figure}{We present a general framework for in-bed HPS tasks, containing a monocular optimization strategy to generate high-quality SMPL annotations in in-bed scenarios, SMPLify-IB; and a HPS network to predict in-bed motions from pressure sequence, PI-HMR.}
  \vspace{-0.5em}
  \label{fig: dataset_vis}
\end{center}
}]


% \maketitle
\renewcommand{\thefootnote}{\fnsymbol{footnote}} %将脚注符号设置为fnsymbol类型,即特殊符号表示
\footnotetext[1]{These authors contributed equally to this work.} %对应脚注[1]
\footnotetext[2]{Corresponding authors.} %对应脚注[2]

\begin{abstract}


The choice of representation for geographic location significantly impacts the accuracy of models for a broad range of geospatial tasks, including fine-grained species classification, population density estimation, and biome classification. Recent works like SatCLIP and GeoCLIP learn such representations by contrastively aligning geolocation with co-located images. While these methods work exceptionally well, in this paper, we posit that the current training strategies fail to fully capture the important visual features. We provide an information theoretic perspective on why the resulting embeddings from these methods discard crucial visual information that is important for many downstream tasks. To solve this problem, we propose a novel retrieval-augmented strategy called RANGE. We build our method on the intuition that the visual features of a location can be estimated by combining the visual features from multiple similar-looking locations. We evaluate our method across a wide variety of tasks. Our results show that RANGE outperforms the existing state-of-the-art models with significant margins in most tasks. We show gains of up to 13.1\% on classification tasks and 0.145 $R^2$ on regression tasks. All our code and models will be made available at: \href{https://github.com/mvrl/RANGE}{https://github.com/mvrl/RANGE}.

\end{abstract}

    
\section{Introduction}
Backdoor attacks pose a concealed yet profound security risk to machine learning (ML) models, for which the adversaries can inject a stealth backdoor into the model during training, enabling them to illicitly control the model's output upon encountering predefined inputs. These attacks can even occur without the knowledge of developers or end-users, thereby undermining the trust in ML systems. As ML becomes more deeply embedded in critical sectors like finance, healthcare, and autonomous driving \citep{he2016deep, liu2020computing, tournier2019mrtrix3, adjabi2020past}, the potential damage from backdoor attacks grows, underscoring the emergency for developing robust defense mechanisms against backdoor attacks.

To address the threat of backdoor attacks, researchers have developed a variety of strategies \cite{liu2018fine,wu2021adversarial,wang2019neural,zeng2022adversarial,zhu2023neural,Zhu_2023_ICCV, wei2024shared,wei2024d3}, aimed at purifying backdoors within victim models. These methods are designed to integrate with current deployment workflows seamlessly and have demonstrated significant success in mitigating the effects of backdoor triggers \cite{wubackdoorbench, wu2023defenses, wu2024backdoorbench,dunnett2024countering}.  However, most state-of-the-art (SOTA) backdoor purification methods operate under the assumption that a small clean dataset, often referred to as \textbf{auxiliary dataset}, is available for purification. Such an assumption poses practical challenges, especially in scenarios where data is scarce. To tackle this challenge, efforts have been made to reduce the size of the required auxiliary dataset~\cite{chai2022oneshot,li2023reconstructive, Zhu_2023_ICCV} and even explore dataset-free purification techniques~\cite{zheng2022data,hong2023revisiting,lin2024fusing}. Although these approaches offer some improvements, recent evaluations \cite{dunnett2024countering, wu2024backdoorbench} continue to highlight the importance of sufficient auxiliary data for achieving robust defenses against backdoor attacks.

While significant progress has been made in reducing the size of auxiliary datasets, an equally critical yet underexplored question remains: \emph{how does the nature of the auxiliary dataset affect purification effectiveness?} In  real-world  applications, auxiliary datasets can vary widely, encompassing in-distribution data, synthetic data, or external data from different sources. Understanding how each type of auxiliary dataset influences the purification effectiveness is vital for selecting or constructing the most suitable auxiliary dataset and the corresponding technique. For instance, when multiple datasets are available, understanding how different datasets contribute to purification can guide defenders in selecting or crafting the most appropriate dataset. Conversely, when only limited auxiliary data is accessible, knowing which purification technique works best under those constraints is critical. Therefore, there is an urgent need for a thorough investigation into the impact of auxiliary datasets on purification effectiveness to guide defenders in  enhancing the security of ML systems. 

In this paper, we systematically investigate the critical role of auxiliary datasets in backdoor purification, aiming to bridge the gap between idealized and practical purification scenarios.  Specifically, we first construct a diverse set of auxiliary datasets to emulate real-world conditions, as summarized in Table~\ref{overall}. These datasets include in-distribution data, synthetic data, and external data from other sources. Through an evaluation of SOTA backdoor purification methods across these datasets, we uncover several critical insights: \textbf{1)} In-distribution datasets, particularly those carefully filtered from the original training data of the victim model, effectively preserve the model’s utility for its intended tasks but may fall short in eliminating backdoors. \textbf{2)} Incorporating OOD datasets can help the model forget backdoors but also bring the risk of forgetting critical learned knowledge, significantly degrading its overall performance. Building on these findings, we propose Guided Input Calibration (GIC), a novel technique that enhances backdoor purification by adaptively transforming auxiliary data to better align with the victim model’s learned representations. By leveraging the victim model itself to guide this transformation, GIC optimizes the purification process, striking a balance between preserving model utility and mitigating backdoor threats. Extensive experiments demonstrate that GIC significantly improves the effectiveness of backdoor purification across diverse auxiliary datasets, providing a practical and robust defense solution.

Our main contributions are threefold:
\textbf{1) Impact analysis of auxiliary datasets:} We take the \textbf{first step}  in systematically investigating how different types of auxiliary datasets influence backdoor purification effectiveness. Our findings provide novel insights and serve as a foundation for future research on optimizing dataset selection and construction for enhanced backdoor defense.
%
\textbf{2) Compilation and evaluation of diverse auxiliary datasets:}  We have compiled and rigorously evaluated a diverse set of auxiliary datasets using SOTA purification methods, making our datasets and code publicly available to facilitate and support future research on practical backdoor defense strategies.
%
\textbf{3) Introduction of GIC:} We introduce GIC, the \textbf{first} dedicated solution designed to align auxiliary datasets with the model’s learned representations, significantly enhancing backdoor mitigation across various dataset types. Our approach sets a new benchmark for practical and effective backdoor defense.



\section{Related Work}

\subsection{Large 3D Reconstruction Models}
Recently, generalized feed-forward models for 3D reconstruction from sparse input views have garnered considerable attention due to their applicability in heavily under-constrained scenarios. The Large Reconstruction Model (LRM)~\cite{hong2023lrm} uses a transformer-based encoder-decoder pipeline to infer a NeRF reconstruction from just a single image. Newer iterations have shifted the focus towards generating 3D Gaussian representations from four input images~\cite{tang2025lgm, xu2024grm, zhang2025gslrm, charatan2024pixelsplat, chen2025mvsplat, liu2025mvsgaussian}, showing remarkable novel view synthesis results. The paradigm of transformer-based sparse 3D reconstruction has also successfully been applied to lifting monocular videos to 4D~\cite{ren2024l4gm}. \\
Yet, none of the existing works in the domain have studied the use-case of inferring \textit{animatable} 3D representations from sparse input images, which is the focus of our work. To this end, we build on top of the Large Gaussian Reconstruction Model (GRM)~\cite{xu2024grm}.

\subsection{3D-aware Portrait Animation}
A different line of work focuses on animating portraits in a 3D-aware manner.
MegaPortraits~\cite{drobyshev2022megaportraits} builds a 3D Volume given a source and driving image, and renders the animated source actor via orthographic projection with subsequent 2D neural rendering.
3D morphable models (3DMMs)~\cite{blanz19993dmm} are extensively used to obtain more interpretable control over the portrait animation. For example, StyleRig~\cite{tewari2020stylerig} demonstrates how a 3DMM can be used to control the data generated from a pre-trained StyleGAN~\cite{karras2019stylegan} network. ROME~\cite{khakhulin2022rome} predicts vertex offsets and texture of a FLAME~\cite{li2017flame} mesh from the input image.
A TriPlane representation is inferred and animated via FLAME~\cite{li2017flame} in multiple methods like Portrait4D~\cite{deng2024portrait4d}, Portrait4D-v2~\cite{deng2024portrait4dv2}, and GPAvatar~\cite{chu2024gpavatar}.
Others, such as VOODOO 3D~\cite{tran2024voodoo3d} and VOODOO XP~\cite{tran2024voodooxp}, learn their own expression encoder to drive the source person in a more detailed manner. \\
All of the aforementioned methods require nothing more than a single image of a person to animate it. This allows them to train on large monocular video datasets to infer a very generic motion prior that even translates to paintings or cartoon characters. However, due to their task formulation, these methods mostly focus on image synthesis from a frontal camera, often trading 3D consistency for better image quality by using 2D screen-space neural renderers. In contrast, our work aims to produce a truthful and complete 3D avatar representation from the input images that can be viewed from any angle.  

\subsection{Photo-realistic 3D Face Models}
The increasing availability of large-scale multi-view face datasets~\cite{kirschstein2023nersemble, ava256, pan2024renderme360, yang2020facescape} has enabled building photo-realistic 3D face models that learn a detailed prior over both geometry and appearance of human faces. HeadNeRF~\cite{hong2022headnerf} conditions a Neural Radiance Field (NeRF)~\cite{mildenhall2021nerf} on identity, expression, albedo, and illumination codes. VRMM~\cite{yang2024vrmm} builds a high-quality and relightable 3D face model using volumetric primitives~\cite{lombardi2021mvp}. One2Avatar~\cite{yu2024one2avatar} extends a 3DMM by anchoring a radiance field to its surface. More recently, GPHM~\cite{xu2025gphm} and HeadGAP~\cite{zheng2024headgap} have adopted 3D Gaussians to build a photo-realistic 3D face model. \\
Photo-realistic 3D face models learn a powerful prior over human facial appearance and geometry, which can be fitted to a single or multiple images of a person, effectively inferring a 3D head avatar. However, the fitting procedure itself is non-trivial and often requires expensive test-time optimization, impeding casual use-cases on consumer-grade devices. While this limitation may be circumvented by learning a generalized encoder that maps images into the 3D face model's latent space, another fundamental limitation remains. Even with more multi-view face datasets being published, the number of available training subjects rarely exceeds the thousands, making it hard to truly learn the full distibution of human facial appearance. Instead, our approach avoids generalizing over the identity axis by conditioning on some images of a person, and only generalizes over the expression axis for which plenty of data is available. 

A similar motivation has inspired recent work on codec avatars where a generalized network infers an animatable 3D representation given a registered mesh of a person~\cite{cao2022authentic, li2024uravatar}.
The resulting avatars exhibit excellent quality at the cost of several minutes of video capture per subject and expensive test-time optimization.
For example, URAvatar~\cite{li2024uravatar} finetunes their network on the given video recording for 3 hours on 8 A100 GPUs, making inference on consumer-grade devices impossible. In contrast, our approach directly regresses the final 3D head avatar from just four input images without the need for expensive test-time fine-tuning.


% 
\section{Brief Review of 3D Gaussian Splatting}
\label{sec:prelim}
For the sake of clarity, we first briefly review 3D Gaussian Splatting (3DGS)~\cite{kerbl202333dgs}, an explicit representation of a 3D scene for providing effective image rendering. 
% We also provide brief reviews of two powerful extensions of 3DGS, Gaussian Grouping~\cite{ye2023gaussiangrouping} and Relightable Gaussian~\cite{gao2023relightable}, which equip 3DGS with segmentation and relighting abilities and are utilized together with 3DGS as the backbone representation in our work. 


Given $K$ multi-view images $I_{1:K} = \{I_1, I_2, ..., I_K\}$ with corresponding camera poses $\xi_{1:K} = \{\xi_1, \xi_2, ..., \xi_K\}$ of a 3D scene, a scene-specific 3DGS is applied to model the scene with $N$ learnable 3D Gaussian ellipsoids (i.e., $G_{1:N} = \{G_1, G_2, ..., G_N \}$). Each Gaussian $G_i$ is parameterized with its 3-dimensional centroid $\mathbf{p}_i \in \mathbb{R}^{3}$, a 3-dimensional standard deviation $\mathbf{s}_i \in \mathbb{R}^{3}$, a 4-dimensional rotational quaternion $\mathbf{q}_i \in \mathbb{R}^{4}$, an opacity ${\alpha}_i \in [0,1]$, and color coefficients $\mathbf{c}_i$ for spherical harmonics in degree of 3. Hence, $G_i$ is represented with a set of the above parameters (i.e., $G_i = \{\mathbf{p}_i, \mathbf{s}_i, \mathbf{q}_i, {\alpha}_i, \mathbf{c}_i\}$). To model the scene with $G_{1:N}$, 2D images $\hat{I_{1:K}} = \{\hat{I_1}, \hat{I_2}, ..., \hat{I_K}\}$ are sequentially rendered from $G_{1:N}$ using $\xi_{1:K} = \{\xi_1, \xi_2, ..., \xi_K\}$ (please refer to~\cite{kerbl202333dgs} for a detailed rendering process), and supervised with $I_{1:K}$ by the rendering loss:
\begin{equation} \label{Limage}
    \mathcal{L}_{image} = \sum_{k\in {1...K}}\lambda\| I_k - {\hat{I_k}}\|_1 + \mathcal{L}_{SSIM}(I_k, \hat{I_k}),
\end{equation}
where $\mathcal{L}_{SSIM}(\cdot)$ represents a SSIM loss and $\lambda$ is a hyper-parameter (set to $0.2$ as mentioned in~\cite{kerbl202333dgs}).



% \subsection{Gaussian Grouping}
% To overcome the lack of fine-grained scene understanding in 3DGS, Gaussian Grouping~\cite{ye2023gaussiangrouping} extends 3DGS by incorporating segmentation capabilities. Along with $I_{1:K}$, Gaussian Grouping additionally takes the Segment Anything Model (SAM) to produce 2D semantic segmentation masks $S_{1:K} = \{S_1, S_2, ..., S_K\}$ from multiple views as inputs, and an additional 16-dimensional parameter $\mathbf{e}_i \in \mathbb{R}^{16}$ is introduced to represent a 3D Identity Encoding for each Gaussian $G_i$. Therefore, each Gaussian $G_i$ is extended as $G_i = \{\mathbf{p}_i, \mathbf{s}_i, \mathbf{q}_i, {\alpha}_i, \mathbf{c}_i, \mathbf{e}_i\}$. To make sure $G_{1:K}$ learns to segment each object represented by $S_{1:K}$ in the scene, a 2D identity loss $\mathcal{L}_{id}$ is applied by calculating cross-entropy between $\hat{S}_{1:K}$ and $S_{1:K}$, where $\hat{S}_{1:K} = \{\hat{S}_1, \hat{S}_2, ... , S_K\}$ denotes the rendered segmentation maps from $G_{1:K}$. Additionally, to further ensure that the Gaussians having the same identities are grouped together, a 3D regularization loss $\mathcal{L}_{3D}$ is applied to enforce each $G_i$'s k-nearest 3D spatial neighbors to be close in their feature distance of Identity Encodings. Please refer to the original paper~\cite{ye2023gaussiangrouping} for detailed formulations of segmentation map rendering and $\mathcal{L}_{3D}$. The design of Gaussian Grouping ensures that the segmentation results are coherent across multiple views, enabling the automatic generation of binary masks for any queried object in the scene.

% \subsection{Relightable Gaussians}
% Different from Gaussian Grouping, Relightable Gaussians~\cite{gao2023relightable} extends the capabilities of Gaussian Splatting by incorporating Disney-BRDF~\cite{burley2012brdf} decomposition and ray tracing to achieve realistic point cloud relighting. 
% % Unlike traditional Gaussian Splatting, which primarily focuses on appearance and geometry modeling, Relightable Gaussians also aim to model the physical interaction of light with different surfaces in the scene.
% Specifically, for each Gaussian $G_i$, the original color coefficients $\mathbf{c}_i$ is decomposed into a 3-dimensional base color $\mathbf{b}_i \in [0,1]^3$, a 1-dimensional roughness $r \in [0,1]$, and incident light coefficients $\mathbf{l}_i$ for spherical harmonics in degree of 3. Subsequently, the Physical-Based Rendering (PBR) process and a point-based ray tracing are applied to obtain the colored 2D images $\hat{I}^{PBR}_{1:K}$ and supervised by $I_{1:K}$ using the aforementioned $\mathcal{L}_{image}$ in Eqn.~\ref{Limage}. Besides the above extensions on PBR for relighting, Relightable Gaussians also introduces a 3-dimensional normal $\mathbf{n}_i$ for $G_i$ and leverages several techniques, including an unsupervised estimation of a depth map $D_i$ from each input view $\xi_i$, to enhance the geometry accuracy and smoothness. Please refer to the original paper of Relightable Gaussians~\cite{gao2023relightable} for detailed explanations.  

\pj{Our pipeline for 3D Inpainting is built on top of the 3DGS model. Additionally, we incorporate the design of Gaussian Grouping~\cite{ye2023gaussiangrouping} to introduce a 16-dimensional semantic feature $\mathbf{e}_i \in \mathbb{R}^{16}$ for each Gaussian $G_i$, so that the 2D segmentation maps of the Gaussians $G_{1:K}$ is rendered and the object mask for the object to be removed can be directly produced, as mentioned in Sect.~\ref{subsec:3Dinpaint}.
By combining these methods as our backbone, we are able to perform an automatic inpainting mask generation and a reliable depth estimation for depth-guided 3D inpainting. Please refer to our Supplementary material for a more detailed explanation of our backbones.}

% the backbone representation by parameterizing each Gaussian $G_i$ as $G_i = \{\mathbf{p}_i, \mathbf{s}_i, \mathbf{q}_i, {\alpha}_i, \mathbf{c}_i, \mathbf{e}_i, \mathbf{b}_i, r,  \mathbf{l}_i, \mathbf{n}_i\}$. By combining these methods, we are able to perform an automatic inpainting mask generation and a reliable depth estimation for depth-guided 3D inpainting.
% \section{Overview}
\section{Data}

\subsection{Pre-training Data}
We constructed a large-scale video dataset comprising 2B video-text pairs and 3.8B image-text pairs. Leveraging a comprehensive data pipeline, we transformed raw videos into high-quality video-text pairs suitable for model pre-training. As illustrated in Figure~\ref{fig:data_pipeline}, our pipeline consists of several key stages: Video Segmentation, Video Quality Assessment, Video Motion Assessment, Video Captioning, Video Concept Balancing and Video-Text Alignment. Each stage plays a crucial role in constructing the dataset, and we describe them in detail below.

\begin{figure*}[t]
    \centering
    \includegraphics[width=1.3\textwidth, center, trim=0 0 0 0, clip]{figure/data/data_pipeline.png}
    \caption{The pipeline of Step-Video-T2V data process.}
    \label{fig:data_pipeline}
    %\vspace{-6mm}
\end{figure*}


\paragraph{Video Segmentation} 
We began by processing raw videos using the \textit{AdaptiveDetector} function in the \texttt{PySceneDetect}~\citep{pyscenedetect} toolkit to dentify scene changes and use FFmpeg~\citep{ffmpeg} to split them into single-shot clips. 
We adjusted the splitting process for high-resolution videos not encoded with \texttt{libx264} to include necessary reference frames—specifically by properly setting the crop start time in \texttt{FFmpeg}; this prevented visual artifacts or glitches in the output video.
We also removed the first three frames and the last three frames of each clip, following practices similar to Panda70M \cite{chen2024panda} and Movie Gen Video \cite{polyak2024moviegencastmedia}. Excluding these frames eliminates unstable camera movements or transition effects often present at the beginnings and endings of videos.


\paragraph{Video Quality Assessment}

To construct a refined dataset optimized for model training on high-quality, we systematically evaluated and filtered video clips by assigning multiple Quality Assessment tags based on specific criteria. We uniformly sampled eight frames from each clip to compute these tags, providing a consistent and comprehensive assessment of each video.

\begin{itemize}[left=0cm]
\item \textbf{Aesthetic Score}: We used the public LAION CLIP-based aesthetic predictor~\cite{schuhmann2022laion} to predict the aesthetic scores of eight frames from each clip and calculated their average.

\item \textbf{NSFW Score}: We employed the public LAION CLIP-based NSFW detector~\cite{laion2021nsfw}, a lightweight two-class classifier using CLIP ViT-L/14 embeddings, to identify content inappropriate for safe work environments.

\item \textbf{Watermark Detection}: Employing an EfficientNet image classification model~\cite{tan2019efficientnet}, we detected the presence of watermarks within the videos.

\item \textbf{Subtitle Detection}: Utilizing PaddleOCR~\cite{paddleocr}, we recognized and localized text within video frames, identifying clips with excessive on-screen text or captions.

\item \textbf{Saturation Score}: 
We assessed color saturation by converting video frames from BGR to HSV color space and extracting the saturation channel, using OpenCV \cite{opencv_library}. We computed statistical measures—including mean, maximum, and minimum saturation values—across the frames. 


\item \textbf{Blur Score}: 
We detect blurriness by applying the variance of the Laplacian method~\cite{pech2000diatom} to measure the sharpness of each frame. Low variance values indicate blurriness caused by camera shake or lack of clarity.

\item \textbf{Black Border Detection}: We use \texttt{FFmpeg} to detect black borders in frames and record their dimensions to facilitate cropping, ensuring that the model trains on content free of distracting edges.
\end{itemize}



\paragraph{Video Motion Assessment}

Recognizing that motion content is crucial for representing dynamic scenes and ensuring effective model training, we calculate the motion score by averaging the mean magnitudes of the optical flow \cite{opencv_library} between pairs of resized grayscale frames, using the Farneback algorithm. We introduced three evaluative tags centered around motion scores:


\begin{itemize}[left=0cm]
    \item \textbf{Motion\_Mean}: The average motion magnitude across all frames in the clip, indicating the general level of motion. This score helps us identify clips with appropriate motion; clips with extremely low \texttt{Motion\_Mean} values suggest static or slow motion scenes that may not effectively contribute to training models focused on dynamic content.

    \item \textbf{Motion\_Max}: The maximum motion magnitude observed in the clip, highlighting instances of extreme motion or motion distortion. High \texttt{Motion\_Max} values may indicate the presence of frames with excessive or jittery motion.

    \item \textbf{Motion\_Min}: The minimum motion magnitude in the clip, identifying clips with minimal motion. Clips with very low \texttt{Motion\_Min} may contain idle frames or abrupt pauses, which could be undesirable for training purposes.
\end{itemize}





\paragraph{Video Captioning} 
Recent studies~\citep{openaisora, betker2023improving} have highlighted that both precision and richness of captions are crucial in enhancing the prompt-following ability and output quality of generative models. 
 
Motivated by this, we introduced three types of caption labeling into our video captioning process by employing an in-house Vision Language Model (VLM) designed to generate both short and dense captions for video clips.
\begin{itemize}[left=0cm]
    \item \textbf{Short Caption}: The short caption provides a concise description, focusing solely on the main subject and action, closely mirroring real user prompts.

    \item \textbf{Dense Caption}: The dense caption integrates key elements, emphasizing the main subject, events, environmental and visual aspects, video type and style, as well as camera shots and movements. To refine camera movements, we manually collected annotated data and performed SFT on our in-house VLM, incorporating common camera movements and shooting angles.
    
    \item \textbf{Original Title}: We also included a variety of caption styles by incorporating a portion of the original titles from the raw videos, adding diversity to the captions.
    
\end{itemize}



\paragraph{Video Concept Balancing}
To address category imbalances and facilitate deduplication in our dataset, we computed embeddings for all video clips using an internal VideoCLIP model and applied K-means clustering \cite{macqueen1967some} to group them into over 120,000 clusters, each representing a specific concept or category. By leveraging the cluster size and the distance to centroid tags, we balanced the dataset by filtering out clips that were outliers within their respective clusters. As part of this process, we added two new tags to each clip:

\begin{itemize}[left=0cm] 
    \item \textbf{Cluster\_Cnt}: The total number of clips in the cluster to which the clip belongs.

    \item \textbf{Center\_Sim}: The cosine distance between the clip's embedding and the cluster center.
\end{itemize}



\paragraph{Video-Text Alignment}

Recognizing that accurate alignment between video content and textual descriptions is essential to generate high-quality output and effective data filtering, we compute a \textbf{CLIP Score} to measure video-text alignment. This score assesses how well the captions align with the visual content of the video clips.

\begin{itemize}[left=0cm] 
\item \textbf{CLIP Score}: We begin by uniformly sampling eight frames from the given video clip. Using the CLIP model~\cite{yang2022chineseclip}, we then extract image embeddings for these frames and a text embedding for the video caption. The \texttt{CLIP Score} is computed by averaging the cosine similarities between each frame embedding and the caption embedding.



\end{itemize}


\subsection{Post-training Data}


For SFT in post-training, we curate a high-quality video dataset that captures good motion, realism, aesthetics, a broad range of concepts, and accurate captions. Inspired by \cite{dai2023emu, polyak2024moviegencastmedia, kong2024hunyuanvideo}, we utilize both automated and manual filtering techniques:

\begin{itemize}[left=0cm]


\item \textbf{Filtering by Video Assessment Scores}: Using video assessment scores and heuristic rules, we filter the entire dataset to a subset of 30M videos, significantly improving its overall quality.


\item \textbf{Filtering by Video Categories}: For videos within the same cluster, we use the "Distance to Centroid" values to remove those whose distance from the centroid exceeds a predefined threshold. This ensures that the resulting video subset contains a sufficient number of videos for each cluster while maintaining diversity within the subset.



\item \textbf{Labeling by Human Annotators}: In the final stage, human evaluators assess each video for clarity, aesthetics, appropriate motion, smooth scene transitions, and the absence of watermarks or subtitles. Captions are also manually refined to ensure accuracy and include essential details such as camera movements, subjects, actions, backgrounds, and lighting.

\end{itemize}



\section{Method}
\subsection{PI-HMR}
Our motivation is to utilize pressure data nature. So our efforts fall into three stages: alleviating the dataset bottleneck and learning cross-dataset human and motion priors in the pre-training stage; pressure-based PI-HMR's design; and learning user's habits to overcome information ambiguity in the TTO. Thus, the data flow includes: (1) pre-train: KD-based pre-training with the training set; (2) train: train the PI-HMR and VQ-VAE with the training set; (3) test: test with PI-HMR on the test set and improve the estimates with the TTO strategy. \cref{fig: pihmr_architecture} shows the framework of PI-HMR. The details of each module will be elaborated as follows:

\subsubsection{Overall Pipeline of PI-HMR}
Given an input pressure image sequence $V=\{I_i \in \mathbb{R}^{H\times W}\}^T_{t=1}$ with $T$ frames, PI-HMR outputs the SMPL predictions of the mid-frame by a three-stage feature extraction and fusion modules. Following~\cite{kocabas2020vibe, choi2021beyond, wei2022capturing}, we first use ResNet50 to extract the static feature of each frame to form a static representation sequence $X = \{x_t \in \mathbb{R}^{2048 \times H_1 \times W_1}\}^T_{t=1}$. The extracted $X$ is then fed into our Multi-scale Feature Fusion module~(MFF) to generate the fusion feature sequences $G = \{g_t\}^T_{t=1}$, with two-layer Transformer blocks behind to learn their long-term temporal dependencies and yield the temporal feature sequence $Z = \{z_t\}^T_{t=1}$. Finally, We use the mean feature of $Z$ as the integrated feature representation of the mid-frame and produce final estimations with an IEF SMPL regressor~\cite{kanazawa2018end}.   

\subsubsection{Multi-Scale Feature Fusion Module}
To exploit the characteristics of pressure images, our core insight lies in that both large-pressure regions and human joint projections are essential for model learning: large-pressure regions represent the primary contact areas between humans and environments, directly reflecting user's posture and movement tendencies; 2D joint positions, always accompanied by inherent information ambiguity, serve to assist the model in learning the local pressure distribution pattern between small and large pressure zones. Following the insight, we present the Multi-scale Feature Fusion module~(MFF), shown in~\cref{fig: sec7_pimesh_structure}. MFF extracts multi-scale features from the static feature $x_i$ with the supervision of high-pressure masks and human joints, and generates the fusion feature $g_i$ for the next-stage temporal encoder. Before delving into MFF, we first introduce our positional encoding and high-pressure sampling strategy.

\textbf{Spatial Position Embedding.} We introduce a novel position embedding approach to fuse spatial priors into model learning. Compared with visual pixels, we could acquire the position of each sensing unit and their spatial relationships, given that the sensors remain fixed during data collection. Specifically, for a sensing unit located in pixel $(i, j)$ of a pressure image, we could get its position representation $[i, j, i \cdot d_h, j \cdot d_w]$, with $d_h$, $d_w$ being the sensor intervals along x-axis and y-axis~($d_h=0.0311m$ and $d_w=0.0195m$ in TIP). The first two values mean its position within image, while the latter ones denote the position in the world coordinate system (with its origin at the top-left pixel position of the pressure image). The representation is then transformed into spatial tokens $P \in \mathbb{R}^{256}$ using a linear layer. During the training, we could generate the spatial position map for the whole pressure image, noted as $P_i \in \mathbb{R}^{256 \times H \times W}$.

\textbf{TopK-Mask and Learnable Mask.} We employ a Top-K selection algorithm to generate  high-pressure 0-1 masks for each pressure image~(elements larger than K-largest value is set as 1). The mask, noted as $H^K$, will be fed into MFF as contour priors. Besides, we incorporate a learnable mask $H^{LK}$ into our model, utilizing the initial pressure input $I_i$ and the TopK-Mask matrix $H_i^K$ to learn an attention distribution that evaluates the contribution of features in the feature map. The learnable mask is computed as:
\begin{equation} 
    H^{LK}_i = \text{Softmax}(\text{Conv}([I_i \odot H_i^K, H^K_i]))
\end{equation}
where $\odot$ is the Hadamard product. The product result will be stacked with the TopK-mask and fed into a 1-layer convolution layer and Softmax layer to generate the attention matrix $H^{LK}_i \in \mathbb{R}^{H \times W}$. We aim to explicitly integrate these pressure distributions to enhance learnable masks' quality. The K is set as 128 in PI-HMR, and we also conduct ablations to discuss the selection of K in~\cref{tab: ablations for PI-HMR}.

\textbf{Auxiliary Joint Regressor.} We use an auxiliary joint regressor to provide 2D joints for the multi-scale feature extraction~(shown in~\cref{fig: sec7_pimesh_structure}). The regressor takes the static feature $x_i$ as input and returns the 2D positions of 12 joints in the pressure image, noted as $J_i^{2D}$. The 2D regressor will be trained in conjunction with the entire model.

\textbf{Multi-Scale Feature Fusion.}
\begin{figure}[tbp]
  \centering
  \includegraphics[width=\linewidth]{images/pihmr_mff.pdf}
  \caption{\textbf{Framework of our multi-scale feature fusion module.}} 
  % \vspace{-0.5cm}
  \label{fig: sec7_pimesh_structure}
\end{figure}
We extract the global feature $g_i^g$, local feature $g_i^l$, and sampling feature $g_i^s$ from the static feature $x_i$, without replying on the temporal consistency. Firstly for global feature, we apply average pooling and downsampling to the static features $x_i \in \mathbb{R}^{2048 \times H_1 \times W_1}$ to generate global representation $g_i^g \in \mathbb{R}^{512}$. 

Subsequently, we perform dimension-upsampling on $x_i$ to obtain upsampled feature $x_i^{up} \in \mathbb{R}^{256 \times H \times W}$ that aligned with the initial pressure input scale, facilitating us to apply spatial position embedding and feature sampling. For local features, we add $x_i^{up}$ to the spatial position map $P_i$ we have learned, multiply it point-wise with the Learnable Mask $H^{LK}_i$, and then subject it to AttentionPooling to derive the local features $g_i^l \in \mathbb{R}^{256}$. 

As for the sampling features, we employ a feature sampling process on $x_i^{up}$ based on the pre-obtained TopK-Masks and 12 2D keypoint positions obtained from a auxiliary 2D keypoint regressor and get a medium feature $g_i^{mid} \in \mathbb{R}^{(K + 12) \times 256}$. After the same spatial position embedding, the medium feature will be input into a 1-layer Transformer layer to learn its spatial semantics, with the mean of the results serving as the sampling feature $g_i^s \in \mathbb{R}^{256}$.

Finally we get the fusion feature $ g_i \in \mathbb{R}^{1024}$ by concatenating aforesaid global, local, and sampling features.  

\subsubsection{Training Strategy}

The overall loss function can be expressed as follows:
\begin{equation} \label{eq: overall_func}
\small
   \mathcal{L}_{pi} = \lambda_{\text{SMPL}} \mathcal{L}_{\text{SMPL}} + \lambda_{3D} \mathcal{L}_{3D} + \lambda_{2D} \mathcal{L}_{2D} 
\end{equation}
where $\mathcal{L}_{\text{SMPL}}$ and $\mathcal{L}_{3D}$ presents the deviations between the estimated SMPL parameters and 3d joints with GTs,  and $\mathcal{L}_{2D}$ minimize errors in 2D joints for the auxiliary regressor.

\subsection{Encoder pre-train by cross-modal KD}
We employ a cross-modal KD framework to pretrain our PI-HMR's feature encoder, aiming at learning motion and shape priors from vision-based methods on paired pressure-RGB images. Specifically, we implement a HMR~\cite{kanazawa2018end} architecture as the student network $\mathcal{F}_S$~(with a ResNet50 as encoder and a IEF~\cite{kanazawa2018end} SMPL regressor), and choose CLIFF~(ResNet50)~\cite{li2022cliff} as the teacher model $\mathcal{F}_T$~(a HMR-based network). During pre-training, we apply extra feature-based and response-based KD~\cite{gou2021knowledge} to realize fine-grained knowledge transfer. Given input pressure-RGB-label groups~$(I_P, I_R, y)$, and 4 pairs of hidden feature maps from $\mathcal{F}_T$ and $\mathcal{F}_S$~(ResNet50 has 4 residual blocks, so we extract the feature maps after each residual block), i.e., $M_T$ from $\mathcal{F}_T$ and $M_S$ from $\mathcal{F}_S$, the loss function is:
\begin{equation}
\small
\begin{split}
    L_{KD} = \lambda_{kd}^y L_{pi}(\mathcal{F}_S(I_P), y) + \lambda_{kd}^T L_{pi}(\mathcal{F}_S(I_P), \mathcal{F}_T(I_R)) \\
    + \lambda_{kd}^F \sum_{i=1}^4||M_S^i - M_T^i|| 
\end{split}
\end{equation}
where $L_{pi}$ is the same as~\cref{eq: overall_func}, and $\lambda$ is the hyperparamter. After training and convergence, the ResNet50 encoder from $\mathcal{F}_S$ will be adopted as PI-HMR's pre-trained static encoder and finetuned in the following training process.

\setlength{\aboverulesep}{0pt}
\setlength{\belowrulesep}{0pt}
\begin{table*}[t] 
\small
\centering
\begin{tabular}{l|c|c|cccc}
\toprule[2pt]
Method       & Input  & Modalities      & MPJPE  & PA-MPJPE & MPVE  & ACC-ERR \\
\hline
% CLIFF        &  single                     & \multirow{3}{*}{RGB} &       &         &          &      &       &           \\ \cline{1-2} \cline{4-9}
% TCMR         & \multirow{2}{*}{sequence}     &                    &       &         &          &      &       &           \\
% MPS-NET      &                        &                    &       &         &          &      &       &           \\
% \hline
HMR~\cite{kanazawa2018end} & \multirow{3}{*}{single} & \multirow{9}{*}{Pressure} & 75.06 & 57.97 &  89.11 &   31.52  \\
HMR-KD       &            &         &   66.30    &          52.41      &  83.01        &     24.41      \\
% PressureNet  &                        &                    &       &                   &      &       &           \\
% BodyMap      &                        &                    &       &         &          &      &       &           \\
BodyMap-WS~\cite{tandon2024bodymap}   &           &        &   71.48    &     \textbf{40.91}      &   80.08       &   27.98     \\ \cline{1-2} \cline{4-7}
TCMR~\cite{choi2021beyond}         & \multirow{7}{*}{sequence}     &     & 64.37 &   46.76      &     74.66         &       20.12    \\
MPS-NET~\cite{wei2022capturing}      &               &         &         160.59    &    112.12     &     187.13        &      28.73         \\
PI-Mesh~\cite{wu2024seeing}      &        &         &    76.47   &     54.65     &   90.54       &    21.86  \\
 \cline{1-1} \cline{4-7}
PI-HMR (ours)       &         &      & 59.46         &        44.53      &      69.92    &       \textbf{9.12}     \\
PI-HMR + KD (ours)       &        &     &   \underline{57.13}  & 42.98         &  \underline{67.22}    &        9.84     \\
PI-HMR + TTO (ours)       &       &       &    57.76        &     43.31    &   67.76   &         \underline{9.83}    \\
PI-HMR + KD + TTO (ours) &        &       &      \textbf{55.50}     &    \underline{41.81}     &   \textbf{65.15}       &   9.96        \\
\bottomrule[2pt]
\end{tabular}
\caption{\textbf{Overall results of PI-HMR with SOTA methods}} \label{tab: overall_results}
\end{table*}
\begin{figure*}[t] \vspace{-0.1cm}
  \centering
  \includegraphics[width=\linewidth]{images/quali_results.pdf}
  \caption{\textbf{Qualitative visualization for PI-HMR.} PI-HMR and PI-Mesh's results are generated by pressure images, while CLIFF's outputs are generated by RGB images for cross-modal comparison. Predictions are rendered on RGB images for comparison convenience}
  \label{fig: overall_vis}
\end{figure*} 

\subsection{Test-Time Optimization}
We also explore a TTO routine to further enhance prediction quality of PI-HMR. Considering that there hasn't been a general 2D keypoint regressor for pressure images, we are inclined toward seeking an unsupervised, prior-based optimization strategy. We notice that humans exhibit similar movement patterns across various postural states~(\eg, timing, which hand to support, and leg movements). This inspires us to pre-learn such a motion habit as motion prior, playing as supplement cues to refine PI-HMR's prediction. 

We apply a VQ-VAE as the motion prior learner. The selection is rooted in our assumption that the distribution of bed-bound movements is rather constrained. In that case, for a noised motion prediction, VQ-VAE could match it to the closest pattern, thereby re-generating habit-based results. The VQ-VAE is based on Transformer blocks and show similar architecture with~\cite{feng2024stratified}. During training,  we only auto-reconstruct the pose sequences~($\theta$ in SMPL). More details are provided in Supplementary Materials.

The VQ-VAE will act as the only motion prior and supervision in our TTO routine. For terminological convenience, given a VQ-VAE $\mathbb{M}$ and PI-HMR initial predictions~$\Theta^0 = \{\theta^0_1, ... \theta^0_T\}$, the $i_{th}$ iteration objectives follows:
\begin{equation} 
\small
    \mathcal{L}_{TTO}^i = \mathcal{L}_{m}(\Theta^i, \Theta^0) + \mathcal{L}_{m}(\Theta^i, \mathbb{M}(\Theta^i)) + \mathcal{L}_{sm}(\Theta^i)
\end{equation}
$\mathcal{L}_{m}$ is the SMPL and joint error term, and $\mathcal{L}_{sm}$ is the smooth loss. The result of $i_{th}$ iteration will be input into $\mathbb{M}$ and optimized in the ${i + 1}_{th}$ iteration. The TTO will help maintain a balance between initial PI-HMR outputs and the reconstruction by VQ-VAE, thus learning robust motion priors.

% Vision-based approaches usually employ a robust 2D keypoint regressor~(\eg OpenPose~\cite{cao2017realtime}, ViTPose~\cite{xu2022vitpose}) as weak supervision to encourage the predictions fit to image cues.

% as a result of information ambiguity, the prediction noise is primarily concentrated during turning-over movements with less contact between humans and beds. Meanwhile,




\section{Experiments}

% \begin{minipage}{\textwidth}
% \begin{minipage}[c]{0.5\textwidth}
% \centering 
%     \begin{tabular}{c|cc}
%     \toprule[2pt]
%     Sampling Method      & MPJPE  & PA-MPJPE \\
%     \midrule[1.4pt]
%     Top 8       &    58.62   &   44.42   \\
%     Top 32      &    57.66   &   43.48   \\
%     Top 128     &    \textbf{57.13}   &   \textbf{42.98}   \\
%     Top 256     &   58.64    &  44.65    \\
%     \bottomrule[2pt]
% \end{tabular}
% \captionof{table}{\textbf{Ablations for the K selection in TopK algorithm.}} \label{tab: ablations for topk}
% \end{minipage}
% \begin{minipage}[c]{0.5\textwidth}
% \centering
%     \begin{tabular}{l|cc}
%     \toprule[2pt]
%     Method      & MPJPE  & PA-MPJPE \\
%     \midrule[1.5pt]
%     w/o. Learnable Masks            &   60.95    &    46.27  \\
%     w/o. Spatial Position Embedding           &   60.65    &   46.28   \\
%     w/o. AttentionPooling            &    59.21   &  45.08    \\
%     \hline
%     All               &    \textbf{57.13}   &  \textbf{42.98}    \\
%     \hline
%     \bottomrule[2pt]
%     \end{tabular}
%     \captionof{table}{\textbf{Ablations for other components in MFF.}} \label{tab: ablations for PI-HMR}
% \end{minipage}
% \end{minipage}

% \begin{table*}[t]
%     % \label{Tab}
%     \begin{subtable}{.33\linewidth}
%     \vspace{0pt}
%     \small
%     \centering
%         \resizebox{!}{1cm}{
%         \begin{tabular}{c|cc}
%         \toprule[2pt]
%         Sampling Mode      & MPJPE  & PA-MPJPE \\
%         \midrule[1.4pt]
%         Top 8       &    58.62   &   44.42   \\
%         Top 32      &    57.66   &   43.48   \\
%         Top 128     &    \textbf{57.13}   &   \textbf{42.98}   \\
%         Top 256     &   58.64    &  44.65    \\
%         \bottomrule[2pt]
%     \end{tabular}
%     }
%     \caption{\textbf{Ablations for the K selection in TopK algorithm.}} \label{tab: ablations for topk}
%     \end{subtable}%
%     \begin{subtable}{.33\linewidth}
%     \vspace{0pt}
%     \small
%     \centering
%       \resizebox{!}{1cm}{
%             \begin{tabular}{l|cc}
%             \toprule[2pt]
%             Method      & MPJPE  & PA-MPJPE \\
%             \midrule[1.5pt]
%             w/o. Learnable Masks            &   60.95    &    46.27  \\
%             w/o. Spatial Position Embedding           &   60.65    &   46.28   \\
%             w/o. AttentionPooling            &    59.21   &  45.08    \\
%             \hline
%             All               &    \textbf{57.13}   &  \textbf{42.98}    \\
%             \hline
%             \bottomrule[2pt]
%             \end{tabular}
%             }
%             \caption{\textbf{Ablations for other components in MFF.}} \label{tab: ablations for PI-HMR}
%     \end{subtable} 
%     \begin{subtable}{.33\linewidth}
%     \vspace{0pt}
%     \centering
%       \small
%       \resizebox{!}{1cm}{
%         \begin{tabular}{ccc|cc} 
%         \toprule[2pt]
%         GT         & Output-KD   & Feat.-KD       & MPJPE  & PA-MPJPE\\
%         \midrule[1.4pt]
%         \checkmark &            &            &  75.06     & 57.97 \\
%         \checkmark & \checkmark &            &   77.86  & 59.41\\
%         \checkmark &            & \checkmark &  67.34     &  52.16 \\
%         \checkmark & \checkmark & \checkmark &  \textbf{66.3}    &  \textbf{52.41}\\
%         \bottomrule[2pt]
%         \end{tabular} 
%         }
%         \caption{\textbf{Ablations for cross-modal KD}. GT, Output-KD, and Feat-KD represent supervision with GTs, CLIFF's outputs, and CLIFF's hidden feature maps, respectively.} \label{tab: ablations_kd} 
%     \end{subtable} 
% \end{table*}


We evaluate PI-HMR on the TIP dataset. Following~\cite{wu2024seeing},  we choose the second-to-last group of each subject as the val. set, the last group of each subject as the test set, and the remains as the training set. For evaluation, We use standard evaluation metrics including MPJPE~(without pelvis alignment), PA-MPJPE, MPVE for shape errors, and Acceleration errors~(ACC-ERR) to evaluate smoothness. The first three metrics are measured in millimeters~($mm$), and the rest are measured in $mm/s^2$.

We compare our model with previous SOTAs and vison-based classic structures, including: HMR~\cite{kanazawa2018end} and HMR-KD (HMR structure with and without cross-modal KD), BodyMap-WS~\cite{tandon2024bodymap}, TCMR~\cite{choi2021beyond}, MPS-NET~\cite{wei2022capturing}, and PI-Mesh~\cite{wu2024seeing}. All methods are re-trained on TIP with our re-generated SMPL p-GTs, and follow the same training setups with PI-HMR. We provide detailed implementation details of these approaches and PI-HMR in Sup. Mat.  

\subsection{Overall Results for PI-HMR} \label{sec:exp_overall_results}
We present quantitative evaluations in~\cref{tab: overall_results}. Our methods outperform all image or sequence-based methods, presenting about 17.01mm MPJPE decrease compared to PI-Mesh and also outperforms SOTA vision-based architecture HMR, TCMR with 15.6mm, 4.91mm MPJPE improvement, while maintaining comparable ACC-ERR compared with SOTA approaches. Moreover, our introduced cross-modal KD and TTO strategy further improve the robustness of PIHMR, bringing 2.33mm and 1.7mm MPJPE improvements compared with basic structure. In particular, the TTO strategy, as an unsupervised, entirely prior-based optimization strategy, demonstrates the effectiveness of learning and refinement based on user habits. We provide visual comparisons between CLIFF, PI-Mesh and PI-HMR in~\cref{fig: overall_vis}. 

\subsection{Ablations for PI-HMR}

In this section, we present various ablation studies to fully explore the best setup of PI-HMR. We select PI-HMR as shorthand to mean PI-HMR + KD, without the TTO routine, as the basic model for evaluation. All models are trained and tested with the same data as PI-HMR. 

\begin{table}[]
\footnotesize
\centering 
\begin{tabular}{cccc|cc}
\toprule[2pt]
GF & LF & SF-P & SF-K & MPJPE & PA-MPJPE\\
\midrule[1.4pt]
% \checkmark   &    &      &      &   57.61  &  \\
%    &  \checkmark  &      &      &      & \\
%    &    &   \checkmark   &      &   80.48  &  \\
%    &    &      &   \checkmark   &   100.73  &  \\
%    &    &   \checkmark   &  \checkmark    & 80.11   &   \\
%  \cline{1-4}
 \checkmark  &  \checkmark  &      &      &   57.84  &  43.18 \\
 \checkmark  &    &  \checkmark    &      &  59.26   &  45.27\\
 \checkmark &    &      &    \checkmark  &   58.31  &  43.92\\
 \checkmark  &    &    \checkmark  &  \checkmark    &  59.03  &  44.45  \\
 \cline{1-4}
 \checkmark  &  \checkmark  &   \checkmark   &      &   62.23  &  44.91 \\
 \checkmark  &  \checkmark  &      &   \checkmark   &   58.48  &  44.27\\
 \checkmark  &   \checkmark &    \checkmark  &  \checkmark    &  \textbf{57.13}  &  \textbf{42.98} \\
\bottomrule[2pt]
\end{tabular}
\caption{\textbf{Ablations for model structures}. GF, LF, SF-P, SF-K are the global features, local features, sampling features from high-pressure areas and joints, respectively.} \label{tab: res_comp_eff}
\end{table}

\textbf{Model Structures.} In~\cref{tab: res_comp_eff}, we summarize the results with different feature combinations in the MFF module. The method that integrates all branches surpasses other setups. Notably, we observe accuracy drops when sampling features are solely sampled from high-pressure areas, without joints. This could be attributed to the model's tendency to focus more on high pressure, neglecting the local distribution in boardline areas and low-pressure regions related with joints, thereby failing due to information ambiguity.

\textbf{Top-K Sampling.} We explore the rational selection K for the high-pressure masks in~\cref{tab: ablations for topk}. With an increase number of sampling points, the model's performance initially improves and then declines when K is 256. This implies that the model seeks a balance in multi-feature fusion: more sampling points entail more abundant contact and contour information and a broader field of perception, but bringing in redundancy and noises.

\textbf{Other Components in MFF.} We also conducted experiments to evaluate three essential modules including AttentionPooling for local features, learnable masks and spatial position embedding in MFF, as shown in~\cref{tab: ablations for PI-HMR}. Our results suggest that these components provide strong priors for supervision and significantly improve the prediction accuracy.

\begin{table}[]
    \small
    \centering
    \begin{tabular}{c|cc}
    \toprule[2pt]
    Sampling Method      & MPJPE  & PA-MPJPE \\
    \midrule[1.4pt]
    Top 8       &    58.62   &   44.42   \\
    Top 32      &    57.66   &   43.48   \\
    Top 128     &    \textbf{57.13}   &   \textbf{42.98}   \\
    Top 256     &   58.64    &  44.65    \\
    %                                            & Random 128  &      &      \\
    % \hline
    % \multirow{3}{*}{Temporal Encoder}           & RNN         &    60.26   &      \\
    %                                             & GRU         &    62.37   &      \\
    %                                             & Transformer &    \textbf{57.13}   &     \\
    \bottomrule[2pt]
\end{tabular}
\caption{\textbf{Ablations for the K selection in TopK algorithm.}} \label{tab: ablations for topk}
\end{table}

% Please add the following required packages to your document preamble:
% \usepackage{multirow}
\begin{table}[] \vspace{-0.08cm}
    \footnotesize
    \centering
    \begin{tabular}{l|cc}
    \toprule[2pt]
    Method      & MPJPE  & PA-MPJPE \\
    \midrule[1.5pt]
    w/o. Learnable Masks            &   60.95    &    46.27  \\
    w/o. Spatial Position Embedding           &   60.65    &   46.28   \\
    w/o. AttentionPooling            &    59.21   &  45.08    \\
    \hline
    All               &    \textbf{57.13}   &  \textbf{42.98}    \\
    \hline
    \bottomrule[2pt]
    \end{tabular}
    \caption{\textbf{Ablations for other components in MFF.}} \label{tab: ablations for PI-HMR}
    \vspace{-0.1cm}
\end{table}





% \begin{figure}[htbp]
%   \centering
%   \includegraphics[width=\linewidth]{images/exp_selection_principle.pdf}
%   \caption{A glimpse of our TIP dataset~(From left to right: RGB, RGB with 2D joints, depth, pressure images, pressure images with 2D joints, 3D shape~(rendering), 3D shape in side view).}
%   \label{fig: exp_selection_principle}
% \end{figure}



% model structure for PI-HMR

% model designs for PI-HMR

    % temporal encoder
    % w/o Atten Pooling
    % W/o mask
    % sampling numbers (random, 8, 32, 128, 256)
    % KD

\textbf{Ablations for KD.} We conduct experiments to evaluate cross-modal KD. \cref{tab: ablations_kd} shows that feature-based transfer plays a pivotal role in enhancing the performance, while CLIFF's results might, to some extent, misguide the learning of HMR, due to domain gaps~(CLIFF's encoder is pre-trained on ImageNet). When both supervisions coexist, HMR could learn the complete cognitive thought-chain of CLIFF, leading to refinement in predictions.

%  表
\begin{table}[] \vspace{-0.08cm}
    \footnotesize
    \centering
    \begin{tabular}{ccc|cc} 
    \toprule[2pt]
    GT         & Output-KD   & Feat.-KD       & MPJPE  & PA-MPJPE\\
    \midrule[1.4pt]
    \checkmark &            &            &  75.06     & 57.97 \\
    \checkmark & \checkmark &            &   77.86  & 59.41\\
    \checkmark &            & \checkmark &  67.34     &  52.16 \\
    \checkmark & \checkmark & \checkmark &  \textbf{66.3}    &  \textbf{52.41}\\
    \bottomrule[2pt]
    \end{tabular} 
    \caption{\textbf{Ablations for cross-modal KD}. GT, Output-KD, and Feat-KD represent supervision with GTs, CLIFF's outputs, and CLIFF's hidden feature maps, respectively.} \label{tab: ablations_kd} \vspace{-0.08cm}
\end{table} 

\subsection{Results for SMPLify-IB} \label{subsec:exp_re_SIB} 
% \small
\begin{table}[t]
    \footnotesize
    \centering
    \begin{tabular}{l||cc}
    \toprule[2pt]
        ~ & 2D MPJPE & Limb height \\ \midrule[1.4pt]
        CLIFF & 25.20 & - \\ \hline
        TIP & 14.02 & 142.84  \\ \hline
        SMPLify-IB & \textbf{9.65} & \textbf{66.68}  \\ 
    \bottomrule[2pt]
    \end{tabular}
    \caption{\textbf{Qualitative results for SMPLify-IB, compared with the p-GTs in TIP, and CLIFF's outputs}. We calculate the 2D projection errors~(in pixels), and the average height of limbs marked as stationary relative to the bed.} \label{tab:results_dataset}
\end{table}
\cref{tab:results_dataset} provides the evaluation of p-GTs generated by SMPLify-IB. Besides the 2D projection errors and acceleration metrics, we introduce the static limb height as an objective assessment of our refinement in implausible limb lifts. Given the prevalence of limbs placed on other body parts within TIP, this metric can only serve as a rough estimate under limited self-penetration premise. We provide visual results in the Sup. Mat. to present our enhancements.

\begin{table}[t] \vspace{-0.08cm}
    \footnotesize
    \centering
    \begin{tabular}{l||cccc}
    \toprule[2pt]
        ~ & recall & precision & accuracy & time  \\ \midrule[1.4pt]
        SMPLify-XMC & 100\% & 100\% & 100\% & 22.62s  \\ \hline
        Ours & 70.93\% & 80.64\% & 98.32\% & 0.42s  \\ \hline
        Ours (ds 1/3) & 65.66\% & 73.59\% & 98.03\% & 0.036s  \\ 
    \bottomrule[2pt]
    \end{tabular}
    \caption{\textbf{Comparisons between our penetration detection algorithm with SMPLify-XMC.} Time means time consumption in an iteration when deploying detection algorithms in our optimization. 'ds 1/3' means downsample SMPL vertices to their 1/3 scales. } \label{tab: res_detection}
\end{table}

We use SMPLify-XMC's detection results as the GTs and conduct comparison experiments to evaluate our light-weight self-penetration detection algorithm in~\cref{tab: res_detection}. The experiment run on the first group of the TIP dataset. For each batch with 128 images, we integrate both detection algorithms in our optimization routine, record the runtime for each iteration~(1000 iterations for a batch) and calculate the accuracy, precision, and recall of the detection. Compared with SMPLify-XMC, our detection module achieves 53.9 times faster while maintaining a detection accuracy of 98.32\%. We also implement a more lightweight version by downsampling the SMPL vertices into their 1/3 scale. The downsampled version further yields a more than tenfold increase in speed, accompanied by limited precision decrease. 

% \begin{figure}[t]
%   \centering
%   \includegraphics[width=\linewidth]{images/ib_on_slp_1.pdf}
%   \caption{\textbf{Visualizations of SMPLify-IB on SLP.}} 
%   \label{fig: ib_slp}
% \end{figure}

% We also implemented SMPLify-IB on the SLP dataset. Results show the 2D MPJPE drops from 37.6 to 6.9 pixels compared to CLIFF's outputs. \cref{fig: ib_slp} shows our pros in alleviating depth ambiguity.
% 

% 附录

% TTO VQ-VAE隐向量重建结果
\section{Conclusion}

In this work, we present a general framework for in-bed human shape estimation with pressure images, bridging from pseudo-label generation to algorithm design. For label generation, we present SMPLify-IB, a low-cost monocular optimization approach to generate SMPL p-GTs for in-bed scenes. By introducing gravity constraints and a lightweight but efficient self-penetration detection module, we regenerate higher-quality SMPL labels for a public dataset TIP. For model design, we introduce PI-HMR, a pressure-based HPS network to predict in-bed motions from pressure sequences. By fusing pressure distribution and spatial priors, accompanied with KD and TTO exploration, PI-HMR outperforms previous methods. Results verify the feasibility of enhancing model's performance by exploiting pressure's nature.


% This work would provide a whole tool-chain and baseline to support the development of in-bed HPS with pressure images and other modalities.


% In this paper, we propose PIMesh, a temporal human shape estimation network to directly generate human meshes from input pressure image sequences that are collected by a pressure-sensing bedsheet. Moreover, to overcome the dataset bottleneck for the human shape estimation task in in-bed scenarios, we present TIP, the first-of-its-kind multi-modal temporal human in-bed pose dataset with multiple human representation labels including posture, 2D joints, and 3D meshes. TIP contains 156K synchronized temporal images from 9 volunteers in three modalities~(RGB, depth, and pressure images). To generate reliable 3D mesh ground truths, we leverage state-of-the-art RGB-based human shape estimators and propose a 3D mesh label generation pipeline for in-bed scenarios. By deploying a SMPLify-based optimizer with strong human-scene penetration penalty terms, our optimized 3D shape ground truths present 25mm joint prediction errors compared with the manually annotated 2D joint ground truths. Finally, PIMesh achieves 79.17mm of MPJPE, 91.01mm of MPVE, and a minimum of 5.8mm/s$^2$ acceleration errors on the TIP dataset. This work provides a privacy-preserving approach to estimating in-bed human meshes in non-line-of-sight environments and demonstrates more potential application scenarios of pressure-sensing bedsheets

% {
%     \small
%     \bibliographystyle{ieeenat_fullname}
%     \bibliography{main}
% }

% WARNING: do not forget to delete the supplementary pages from your submission 
\setcounter{section}{0}
\renewcommand{\thesection}{\Alph{section}}
\clearpage
\setcounter{page}{1}
\maketitlesupplementary

\begin{figure*}[h]
  \centering
  \includegraphics[width=\linewidth]{images/overall_pipeline.pdf}
  \caption{Our data flow includes three stages: (1) pre-train: knowledge distillation-based cross-modal pre-training; (2) train \& evaluation: train the PI-HMR network with pressure sequences; (3) post-process: improve the estimates with the Test-Time Optimization strategy.} 
  % \vspace{-0.5cm}
  \label{fig: sup_overall_pipeline}
\end{figure*}

\section{Introduction}
In this material, we provide additional details regarding the network and implementation of our methods, as well as compared SOTAs. We further present more qualitative results to show the performance of PI-HMR and our re-generated p-GTs for TIP~\cite{wu2024seeing} and to explore their failure scenarios. The details include:
\begin{itemize}
    \item Implementation details for SMPLify-IB, PI-HMR, cross-modal knowledge distillation, VQ-VAE, test-time optimization, and SOTA methods compared to PI-HMR.
    \item More quantitative and qualitative results about SMPLify-IB, PI-HMR, and failure cases.
    \item Limitations and future works.
\end{itemize}

The overall pipeline of our pressure-to-motion flow is shown in~\cref{fig: sup_overall_pipeline}, and detailed architecture and implementation details will be elaborated below.

\section{Preliminary}

\textbf{Body Model.} The SMPL~\cite{loper2023smpl} model provides a differentiable function $V = \mathcal{M}(\theta, \beta, t)$ that outputs a posed 3D mesh with $N=6890$ vertices. The pose parameter $\theta \in \mathbb{R}^{24 \times 3}$ includes a $\mathbb{R}^3$ global body rotation and the relative rotation of 23 joints with respect to their parents. The shape parameter $\beta \in \mathbb{R}^{10}$ represents the physique of the body shape. And $t \in \mathbb{R}^3$ means the root translation w.r.t the world coordinate.

\section{Network and Implementation Details}

\subsection{Implementation details for SMPLify-IB}

\label{sec:appendix_section}
\subsubsection{The first stage}
In the first stage of our optimization algorithm, we jointly optimize body shape $\beta$, pose parameters $\theta$, and translation $t$ using a sliding-window~(set as 128) approach, with overlap~(set as 64) between adjacent windows. We minimize the following objective function:
\begin{align}
L_{s1}(\theta, \beta, t)&=\lambda_{J}\mathcal{L}_{J}+
\lambda_{p}\mathcal{L}_{p}+\lambda_{sm}\mathcal{L}_{sm}+\lambda_{cons}\mathcal{L}_{cons}\nonumber\\
&\quad+\lambda_{bc}\mathcal{L}_{bc}+\lambda_{g}\mathcal{L}_{g}+\lambda_{sc}\mathcal{L}_{\sc}
\end{align}

1. \textbf{Reprojection constraint term $\mathcal{L}_{J}$:} This term penalizes the weighted robust distance between the projections of the estimated 3D joints and the annotated 2D joint ground truths. Instead of the widely used weak-perspective projection in~\cite{bogo2016keep} with presumed focal length, we apply the perspective projection with calibrated focal length and camera-bed distance provided by TIP.

2. \textbf{Prior constraint term $\mathcal{L}_{p}$:} This term impedes the unrealistic poses while allowing possible ones. $\mathcal{L}_{pose}$, $\mathcal{L}_{shape}$ penalizes the out-of-distribution estimated postures and shapes, which is similar to terms in SMPLify, and $\mathcal{L}_{torso}$ ensures correct in-bed torso poses, where the height of hips should be less than shoulders and the height of waist is below the mean height of shoulders and hips.
\begin{align}
\mathcal{L}_{p}&= \mathcal{L}_{pos} + \mathcal{L}_{sha} + \mathcal{L}_{tor} \\
\mathcal{L}_{pos} &= \sum^{T}_{i}(\lambda^{pos}_{1} (\mathcal{G}(\theta(i)) + \sum_j \lambda^{pos}_{2, j} \cdot e^{\gamma_j \cdot \theta(i)_j})) \nonumber \\
\mathcal{L}_{sha} &= \lambda^{sha}_{} \sum^{T}_{i}||\beta(i)||^2 \nonumber \\
\mathcal{L}_{tor} &= \sum^T_i (\lambda^{tor}_{1} \cdot e^{\omega_{hip}d_{hip}(i)}+ \lambda^{tor}_{2} \cdot e^{\omega_{wai}d_{wai}(i)}) \nonumber \\
d_{hip}(i) &= z_{hip}(i) - z_{sho}(i) \nonumber \\
d_{wai}(i) &= z_{wai}(i) - mean(z_{hip}(i), z_{sho}(i)) \nonumber
\end{align}
where $\mathcal{G}$ is the Gaussian Mixture Model pre-trained in SMPLify, and the second term in $\mathcal{L}_{pos}$ penalizes impossible bending of limbs, neck and torso, such as shoulder twist.  $z_{hip}$, $z_{sho}$, $z_{wai}$ are the height of hip joints, shoulder joints, and waist joint, and $\omega_{hip}$, $\omega_{wai}$ are both set to 100.

3. \textbf{Smooth constraint term $\mathcal{L}_{sm}$:} This term reduces the jitters by minimizing the 3D joints velocity, acceleration and SMPL parameter differences.
\begin{align}
\mathcal{L}_{smo}&=\mathcal{L}_{par}+ \mathcal{L}_{vel}+\mathcal{L}_{acc}  \\
\mathcal{L}_{par}&=\sum^{T-1}_{i=1} (\lambda^{par}_{1} ||\beta (i+1)-\beta (i)||^2 \nonumber \\
&\quad+ \lambda^{par}_{2} ||\theta (i+1)-\theta (i)||^2+\lambda^{par}_{2} ||t (i+1)-t(i)||^2) \nonumber \\
\mathcal{L}_{vel}&=\sum^{T-1}_{i=1}(\lambda^{vel}_{1} ||J(i+1)_{3D}-J(i)_{3D}||^2  \nonumber\\
&\quad+ \lambda^{vel}_{2} ||V(i+1)-V(i)||^2)  \nonumber \\
\mathcal{L}_{acc}&=\sum^{T-1}_{i=2} ||2J(i)_{3D}-J(i-1)_{3D}-J(i+1)_{3D}||^2 \nonumber
\end{align}
where $V(i)$ and $J(i)$ are the coordinates of SMPL vertex set $V$ and 3D joints $J$ in the frame $i$. 

4. \textbf{Consistency constraint term $\mathcal{L}_{cons}$:} This term enhances the consistency between the overlapping parts of the current window and the previously optimized window.
\begin{align}
\mathcal{L}_{cons}&=\sum_{\substack{i\in overlap \\ frames}} (\lambda^{cons}_{1}||\theta(i, b_{1})-\theta(i, b_{2})||^2  \nonumber\\
&\quad+\lambda^{cons}_{2} ||t(i, b_{1})-t(i, b_{2})||^2  \nonumber \\
&\quad+\lambda^{cons}_{3} ||V(i, b_{1}) - V(i, b_{2})||^2 \nonumber \\
&\quad+\lambda^{cons}_{4} ||J(i, b_{1})_{3D} - J(i, b_{2})_{3D}||^2) 
\end{align}
where $t(i, b)$, $\theta(i, b)$ is the translation parameters and pose parameters in frame $i$ of window $b$, and  $V(i, b)$, $J(i, b)_{3D}$ is the coordinates of vertex set $V$, 3D joints $J$ in frame $i$ of window $b$. $b_1$, $b_2$ means the previous window and the present window, respectively.

5. \textbf{Bed contact constraint term $\mathcal{L}_{bc}$:} This term improves the plausibility of human-scene contact. We consider vertices that are close to the bed to be in contact with bed and encourage those vertices to contact with the bed plane while penalizing human-bed penetration.
\begin{align}
\mathcal{L}_{bc} &= \sum^{T}_{i} (\lambda^{in\_bed}_{}  \sum_{0<z(i)_{v}<thre_{bed}} \text{tanh}^2(\omega_{in\_bed} z(i)_v) \nonumber\\ 
&\quad+ \lambda^{out\_bed}_{} \sum_{z(i)_{v}<0} \text{tanh}^2(-\omega_{out\_bed} z(i)_{v})) 
\end{align}
where $z(i)_v$ is the signed distance to the bed plane of vertex $v$ in frame $i$, and $thre_{bed}$ is the contact threshold and set to 0.02m.

6. \textbf{Gravity constraint term $\mathcal{L}_{g}$:} This term penalizes abnormal limb-lifting and reduces depth ambiguity.
\begin{align}
\mathcal{L}_{g}&=\sum^{T}_{i}\sum_{\substack{j\in G_{J}\\z(i)_j>0}}\mathbb{I}(vel(i)_{j}< thre_{vel})e^{\omega(i)_{j}(z(i)_j)} 
\end{align}

where $thre_{vel}$ is set to $\sqrt{110}$, and $vel(i)_j$ denotes the velocity of joint $j$ in frame $i$, which is calculated from 2D annotations. $\omega(i)_j$ is a dynamic weight depends on the state of annotated 2D joint ground truths. Specifically, in addition to the velocity-based criterion, we have more complicated settings for potential corner cases. For example, when a person is seated on the bed, supporting the bed surface with both hands, the shoulders will be incorrectly judged as implausible lifts by sole velocity-based criterion~(This scenario is rarely encountered in TIP, yet it still exists). In that case, we alleviate the impact of gravity constraints on this scenario by dynamically adjusting $\omega(i)_j$. In practice, when the 2D projection lengths of limbs are less than 60\% of the projection lengths in the rest pose, according to geometry, we consider the corresponding limb to be normally lifted even if the corresponding joint speed is below $thre_{v}$, and thus $\omega(i)_j$ takes a smaller value. Besides, $\omega(i)_j$ takes a smaller value for hand joints whose 2D projections are inside the torso to avoid severe hand-torso intersection.

7. \textbf{Self-contact constraint term $\mathcal{L}_{sc}$:} This term is proposed to obtain plausible self-contact and abbreviate self-penetration. In the first stage, we only deal with the intersection between the hand and the torso. The self-contact between other body parts is optimized in the second stage.
\begin{align}
\mathcal{L}_{sc}&=\lambda^{p\_con}_{} \mathcal{L}_{p\_con}+ \lambda^{p\_isect}_{} \mathcal{L}_{p\_isect}+ \lambda^{pull}_{} \mathcal{L}_{pull}  \nonumber \\
&\quad+ \lambda^{push}_{} \mathcal{L}_{push}  \\
\mathcal{L}_{p\_con}&=\sum_{\substack{0<sdf_v<thre_{dist}}} \text{tanh}^2(\omega_{p\_con}sdf_v) \nonumber \\
\mathcal{L}_{p\_isect}&=\sum_{sdf_v<0} \text{tanh}^2(\omega_{p\_isect}|sdf_v|) \nonumber
\end{align}
where $sdf_v$ is the value of the signed distance field(SDF) at vertex $v$, which is calculated by our self-penetration detection algorithm. The details of $\mathcal{L}_{pull}$, $\mathcal{L}_{push}$ are given in the main body of the manuscript.

\subsubsection{The second stage}
We treat the results of the first stage as initialization for the second stage. Specifically, we use the mean $\beta$ of each subject and fix the shape parameters in the second stage. We optimize $\theta$ and $t$ to obtain more plausible human meshes. The objective function $L_{s2}$ is as follows:
\begin{align}
L_{s2}(\theta, \beta, t)&=\lambda_{J}\mathcal{L}_{J}+
\lambda_{p}\mathcal{L}_{p}+\lambda_{sm}\mathcal{L}_{sm}+\lambda_{cons}\mathcal{L}_{cons}\nonumber\\
&\quad+\lambda_{bc}\mathcal{L}_{bc}+\lambda_{g}\mathcal{L}_{g}+\lambda_{sc}\mathcal{L}_{sc}
\end{align}
$\mathcal{L}_J$, $\mathcal{L}_p$, $\mathcal{L}_{sm}$, $\mathcal{L}_{cons}$, $\mathcal{L}_{g}$, $\mathcal{L}_{bc}$ are the same as the first stage, while $L_{sc}$ penalizes self-intersection in all body segments rather than only hands and torso.

\subsubsection{Implementation details}
We use Adam as the optimizer with a learning rate of 0.01, and each stage involves 500 iterations. The length of the sliding window is 128, with 50\% overlapping to prevent abrupt changes between windows. The joints and virtual joints we use for the segmentation of SMPL mesh is listed in~\cref{tab:opt_SEG}.
\setlength{\aboverulesep}{0pt}
\setlength{\belowrulesep}{0pt}
\begin{table}[]
\centering
\begin{tabular}{m{7em}|m{15em}}
\toprule[2pt]
body parts & joints \& virtual joints \\
\midrule[1.2pt]
head & left ear, right ear, nose \\
\hline
torso-upper arm & left shoulder, right shoulder, spine2 \\
\hline
\par left arm & left elbow, left hand, \par mid of left elbow and hand, \par $\frac{2}{5}$ point from left elbow to shoulder \\
\hline
\par right arm & right elbow, right hand, \par mid of right elbow and hand, \par $\frac{2}{5}$ point from right elbow to shoulder \\
\hline
torso-thigh & left hip, right hip \\
\hline
\par left thigh & left knee, left ankle, \par mid of left knee and ankle, \par $\frac{2}{5}$ point from left ankle to hip \\
\hline
\par right thigh & right knee, right ankle, \par mid of right knee and ankle, \par$\frac{2}{5}$ point from right ankle to hip \\
\bottomrule[2pt]
\end{tabular}
\caption{\textbf{Positions of our selected segment centers}.}
\label{tab:opt_SEG}
\end{table}

\subsection{Implementation details for PI-HMR}

Before the aforesaid modules in the main body of our manuscript, PI-HMR also contains three different Transformer blocks for AttentionPooling, cross-attention for sampling features in MFF, and temporal consistency extraction. We will provide detailed designs of these Transformer layers.  (1) For AttentionPooling, we use the same structure in CLIP~\cite{radford2021learning}. (2) For the cross-attention module in MFF, we apply a one-layer Transformer block as the attention module with one attention head and  Dropout set as 0. (3) For the temporal encoder, we apply a two-layer Transformer block to extract the temporal consistency from the fusion feature sequence. In detail, each transformer layer contains a multi-head attention module with $N=8$ heads. These learned features are then fed into the feed-forward network with 512 hidden neurons. Dropout~($p=0.1$) and DropPath~($p_d=0.2$) are applied to avoid overfitting.

The loss of PI-HMR is defined as:
\begin{equation} \label{eq: sup_overall_func}
\small
   \mathcal{L}_{pi} = \lambda_{\text{SMPL}} \mathcal{L}_{\text{SMPL}} + \lambda_{3D} \mathcal{L}_{3D} + \lambda_{2D} \mathcal{L}_{2D} 
\end{equation}
where $\mathcal{L}_{\text{SMPL}}$, $\mathcal{L}_{3D}$, $\mathcal{L}_{2D}$ are calculated as:

\begin{equation}
\mathcal{L}_{\text{SMPL}} = \omega_{s}^{\text{SMPL}} ||\beta - \hat{\beta}||^2 + \omega_{p}^{\text{SMPL}} ||\theta - \hat{\theta}||^2  + \omega_{t}^{\text{SMPL}} ||t - \hat{t}||^2 \nonumber 
\end{equation}

\begin{equation}
\mathcal{L}_{3D} = || J_{3D} - \hat{J}_{3D} ||^2 \nonumber 
\end{equation}

\begin{equation}
\mathcal{L}_{2D} = || J_{2D} - \hat{J}_{2D} ||^2 \nonumber 
\end{equation}
where $\hat{x}$ represents the ground truth for the corresponding estimated variable $x$, and $\lambda$ and $\omega$ are hyper-parameters. We set $\lambda_{\text{SMPL}}=1$, $\lambda_{3D}=300$, $\lambda_{2D}=0.5$, $\omega_{t}^{\text{SMPL}}=\omega_{p}^{\text{SMPL}}=60$, and $\omega_{s}^{\text{SMPL}}=1$ for PI-HMR's training.

Before training, we first pad pressure images to $64 \times 64$ and set $T=15$ as the sequence length. No data augmentation strategy is applied during training. During the training process, we train PI-HMR for 100 epochs with a batchsize of 16, using the AdamW optimizer with a learning rate of 3e-4 and weight decay of 5e-3. We adopt a warm-up strategy in the initial 5 epochs and schedule periodically in a cosine-like function as~\cite{wu2024seeing}. The weight decay is set to 5e-3 to abbreviate overfitting. All implementation codes are implemented in the PyTorch 2.0.1 framework and run on an RTX4090 GPU. 

\subsection{Implementation details for cross-modal KD}\label{sec:sup_imp_kd}
\begin{figure*}[htbp]
  \centering
  \includegraphics[width=\linewidth]{images/sup_KD.pdf}
  \caption{\textbf{An overview of our KD-based network.}} 
  % \vspace{-0.5cm}
  \label{fig: sup_kd_structure}
\end{figure*}
We conduct a HMR-based network~(with a ResNet50 as encoder and an IEF~\cite{kanazawa2018end} SMPL regressor) to pre-train the ResNet50 encoder with SOTA vision-based method Cliff~\cite{li2022cliff}. The detailed structure is presented in~\cref{fig: sup_kd_structure} where we concurrently introduce label supervision, as well as distillation from Cliff's latent feature maps and prediction outcomes, to realize cross-modal knowledge transfer.

To train the KD-based network, like PI-HMR, we first pad pressure images to $64 \times 64$. No data augmentation strategy is applied during training. The training process is performed for 100 epochs with an AdamW optimizer in a minibatch of $256$ on the same training and validation dataset of PI-HMR. We adopt a warm-up strategy in the initial 5 epochs and schedule periodically in a cosine-like function. The weight decay is set to 5e-3 to abbreviate overfitting. All implementation codes are implemented in the PyTorch 2.0.1 framework and run on an NVIDIA. RTX4090 GPU. 

\subsection{Implementation details for VQ-VAE}
\begin{figure}[htbp]
  \centering
  \includegraphics[width=\linewidth]{images/sup_vqvae.pdf}
  \caption{\textbf{An overview of our VQ-VAE network.}} 
  % \vspace{-0.5cm}
  \label{fig: sup_kd_structure}
\end{figure}
The VQ-VAE follows the architecture in~\cite{feng2024stratified}, which incorporates two 4-layer Transformer blocks as the encoder and decoder, respectively, and a $\mathbb{R}^{512 \times 384}$ codebook with 512 entries and $\mathbb{R}^{384}$ for the discrete latent of each entry. Each Transformer layer consists of a 4-head self-attention module and a feedforward layer with 256 hidden units.

To train the VQ-VAE network, we only input the pose parameter sequence $\Theta=\{\theta_1, ..., \theta_T\}$, without the translation and shape parameters, to push the model learning the motion continuity of the turn-over process. The pose sequences will firstly be encoded to motion features $H$ in the Transformer encoder, quantized into discrete latent sequence $Z$ by finding its closest element in the codebook, and reconstruct the input motion sequence in the follow-up Transformer decoder. We follow the loss setting in~\cite{feng2024stratified} and minimize the following loss function in~\cref{eq: sup_loss_vqvae}.

\begin{align} \label{eq: sup_loss_vqvae}
\mathcal{L}_{vq} = \lambda^{vq}_{\theta}&\text{Smooth}_{L1}(\Theta, \hat{\Theta}) \\
&\quad+ \lambda^{vq}_{J}\text{Smooth}_{L1}(J_{3D}(\Theta), J_{3D}(\hat{\Theta})) \nonumber\\
&\quad+ \lambda^{vq}_{d}(||sg[Z] - H||_2 + \omega^{vq}_{b} ||Z - sg[H]||_2) \nonumber
\end{align}
where $\hat{x}$ represents the ground truth for the corresponding estimated variable $x$, $\mathcal{J}(\Theta)$ means 3D joint locations of given SMPL pose parameter sequences $\Theta$~($\beta$ and $t$ are default all-0 tensors), $sg$ denotes the stop gradient operator, and $\lambda$ and $\omega$ are hyper-parameters. We set $\lambda^{vq}_{\theta}=1$, $\lambda^{vq}_{J}=5$, $\lambda^{vq}_{d}=0.25$, and $\omega^{vq}_{b}=0.5$. 

The VQ-VAE is trained with a batchsize of 64 and a sequence length of 64 frames for 100 epochs on the same training and validation dataset of PI-HMR. Adam optimizer is adapted for training, with a fixed learning rate of 1e-4, and [0.9, 0.999] for $\beta$ of the optimizer. All implementation codes are implemented in the PyTorch 2.0.1 framework and run on an NVIDIA. RTX4090 GPU. 

\subsection{Implementation details for Test-Time Optimization}

We use the VQ-VAE to act as the only motion prior and supervision in our TTO routine. For terminological convenience, given a VQ-VAE $\mathbb{M}$ and PI-HMR initial predictions~$\Theta^0 = \{\theta^0_1, ... \theta^0_T\}$. For the $i_{th}$ iteration, we calculate the loss by~\cref{eq: sup_tto} and update the $\Theta$ by stochastic gradient descent. The result of $i_{th}$ iteration will be input into $\mathbb{M}$ and optimized in the ${i + 1}_{th}$ iteration:

\begin{equation} 
    \mathcal{L}_{TTO}^i = \alpha \mathcal{L}_{m}(\Theta^i, \Theta^0) + (1 - \alpha)\mathcal{L}_{m}(\Theta^i, \mathbb{M}(\Theta^i)) + \mathcal{L}_{sm}(\Theta^i)
\label{eq: sup_tto}
\end{equation}
where each term is calculated as:

\begin{align} 
 \mathcal{L}_{m}(\Theta_1, \Theta_2) =  \lambda_{\text{smpl}}^{TTO}  ||\Theta_1, \Theta_2||^2 \\
&+ \lambda_{3D}^{TTO} ||\mathcal{J}(\Theta_1) - \mathcal{J}(\Theta_2)||^2 \nonumber
\end{align}

\begin{equation}
\begin{split}
    \mathcal{L}_{sm}(\Theta) = \lambda_{sm}^{TTO} \frac{1}{T-1} \sum_{t=2}^{T} (|\Theta(t) - \Theta(t-1)| \\
    + |\mathcal{J}(\Theta(t)) - \mathcal{J}(\Theta(t-1)| \nonumber
\end{split}
\end{equation}
where $\mathcal{J}(\Theta)$ means 3D joint locations of given SMPL pose parameters $\Theta$~($\beta$ and $t$ are the initial predictions and won't be updated during the optimization), $\alpha$ is a balance weight to balance initial PI-HMR predictions and reconstructions of VQ-VAE, and $\lambda$s are hyperparameters. We set $\alpha=0.5$, $\lambda_{\text{smpl}}^{TTO}=0.5$, $\lambda_{3D}^{TTO}=0.1$, and $\lambda_{sm}^{TTO}=0.1$ for the test-time optimization.

During the optimization, we freeze the shape parameters $\beta$ and translation parameters $t$ of the initial PI-HMR's outputs, and only optimize pose parameters $\theta$. We employ a sliding window of size 64 to capture the initial PI-HMR predictions and  update them in 30 iterations with a learning rate of 0.01 and Adam as the optimizer. All optimization codes are implemented in the PyTorch 1.11.0 framework and run on an NVIDIA. RTX3090 GPU.

\subsection{Implementation details for SOTA methods}

In this section, we will provide implementation details of compared SOTA networks.

\textbf{HMR~\cite{kanazawa2018end} and HMR + KD}: The implementation details of HMR series are introduced in~\cref{sec:sup_imp_kd}. The distinction between the two lies in whether knowledge distillation supervision is employed during the training process.

\textbf{TCMR~\cite{choi2021beyond} and MPS-NET~\cite{wei2022capturing}}: We choose TCMR and MPS-NET as the compared vision-based architecture because they follow the same paradigm of VIBE~\cite{kocabas2020vibe}, which incorporates a static encoder for texture feature extraction, a temporal encoder for temporal consistency digestion, and a regressor for final SMPL predictions. We use the same architecture and loss weights of the default setting, except converting the initial ResNet50 input to a single channel and adjusting the first convolution layer's kernel size to $5\times 5$ to fit the single-channel pressure images.

\textbf{PI-Mesh~\cite{wu2024seeing}}: PI-Mesh is the first-of-its-kind temporal network to predict in-bed human motions from pressure image sequences. We follow the codes and implementation details provided in~\cite{wu2024seeing} with a ResNet50 as the static encoder and a two-layer Transformer block as the temporal encoder.  

\textbf{BodyMAP-WS}: BodyMap~\cite{tandon2024bodymap} is a SOTA dual-modal method to predict in-bed human meshes and 3D contact pressure maps from both pressure images and depth images. We realize a substitute version provided in their paper, named BodyMap-WS, because we don't have 3D pressure map labels. It is worth mentioning that we notice the TIP dataset fails to converge on the algorithm provided in their GitHub repository. So we remove part of the codes including rotation data augmentation and post-processing of the network outputs~(Line 139-150 and Line 231-242 in the \textit{PMM/MeshEstimator.py} of the GitHub repository) to ensure convergence. 

All methods are trained on the same training-validation dataset of PI-HMR. For TCMR, MPS-NET, and PI-Mesh, we adopt the same training routine as PI-HMR. To be specific, we first pad pressure images to $64 \times 64$ and set $T=15$ as the sequence length. No data augmentation strategy is applied during training. During the training process, we train these approaches for 100 epochs with a batchsize of 16, using the AdamW optimizer with the learning rate of 3e-4 and weight decay of 5e-3~(we firstly conduct a simple grid-search for the best learning rate selection on these methods), and adopt a warm-up strategy in the initial 5 epochs and scheduled periodically in a cosine-like function. For BodyMap-WS, we follow the training routine provided in~\cite{tandon2024bodymap}, resize the pressure images to $224\times224$, and apply RandomAffine, RandomCutOut, and PixelDropout as data augmentation strategies. The training process is performed for 100 epochs with an Adam optimizer in a minibatch of $32$, a learning rate of 1e-4 and weight decay of 5e-4. All codes are implemented in the PyTorch 2.0.1 framework and run on an NVIDIA. RTX4090 GPU.


\section{More ablations}

\subsection{Discussion on TopK sampling} The sampling functions as a low-value filter, freeing the model's attention from redundant, noisy backgrounds and focusing more on high-value regions. We provide a visualization in~\cref{fig: topk_sampling}, where, with 128 points, the pressure image can retain the human's outline while highlighting the core contact areas.

\begin{figure}[t]
  \centering
  \includegraphics[width=0.8\linewidth]{images/exp_selection_principle.pdf} 
  \caption{\textbf{Visualization of TopK Sampling.}} 
  % \vspace{-0.5cm}
  \label{fig: topk_sampling}
\end{figure}

\subsection{Comparisons with single-input models} For vision methods, single-image models usually exhibit lower MPJPE compared to temporal models~(\eg CLIFF vs PMCE). However, for pressure data, temporal models show superiority, likely due to their ability to leverage temporal context, mitigating information ambiguity. This implies the strength of temporal models in pressure data processing compared to single ones. For fair comparisons, we implemented a single-input-based PI-HMR, achieving a 62.01mm MPJPE~(71.48mm for BodyMAP-WS), showing the efficacy of our architecture framework.


\subsection{Results on the original TIP dataset}
The results are shown in~\cref{tab: results_on_o_tip}, which demonstrate a comparable magnitude of MPJPE reduction, proving the efficacy of PI-HMR.

\begin{table}[t] 
\small
\centering
\begin{tabular}{l|c|c|c}
\hline
Method      & TCMR  & PI-Mesh & PI-HMR\\
\hline
MPJPE/ACC-ERR & 67.9/14.6 & 79.2/18.2 & \textbf{68.38/5.24}\\
\hline
\end{tabular} 
\caption{\textbf{Quantitative results on the original TIP dataset.}} \label{tab: results_on_o_tip} 
\end{table}

\subsection{Ablations of TTO.} 
We conducted ablations involving the selection of the balance weight $\alpha$ in~\cref{tab: ablations_alpha} and the number of iterations in~\cref{tab: ablations_iters}. We also explored integrating the pre-trained VQ-VAE into PI-Mesh during training~(as it regresses the sequence rather than the mediate frame, making it suitable for VQ-VAE) and calculating the reconstruction loss. However, MPJPE drops limitedly (0.06mm). We will explore more potential methods~(\eg SPIN-like) in the future work.

\begin{table}[] 
\centering
\begin{tabular}{l|ccccc}
\hline
$\alpha$      & 0.1  & 0.3 & 0.5 & 0.7 & 0.9\\
\hline
MPJPE      &   56.94  &   55.93   &   55.50   & \textbf{ 55.43}   &   55.67  \\
\hline
\end{tabular}
\caption{\textbf{Ablations on balance weight $\alpha$.}} 
\label{tab: ablations_alpha}
\end{table}

\begin{table}[] 
\centering
\begin{tabular}{l|ccccc}
\hline
iters      & 10  & 30 & 50 & 70 & 90\\
\hline
MPJPE      &  56.14   &  55.50    &   55.25   &  55.15   &   \textbf{55.10 } \\
\hline
\end{tabular}
\caption{\textbf{Ablations on the number of iterations.}} 
\label{tab: ablations_iters}
\end{table}

\subsection{Generalization of SMPLify-IB on the SLP dataset.}  
We implemented SMPLify-IB on the SLP dataset. Results show the 2D MPJPE drops from 37.6 to 6.9 pixels compared to Cliff's outputs. \cref{fig: ib_slp} shows our pros in alleviating depth ambiguity. Meanwhile, we observed limb distortions in the optimization results, which may stem from erroneous initial estimations (CLIFF exhibits notable domain adaptation issues in an in-bed scene). In the absence of temporal context, these mis-predictions could exacerbate the likelihood of unreasonable limb angles, underscoring the significance of temporal information in in-bed human shape annotations.

\begin{figure}[t]
  \centering
  \includegraphics[width=\linewidth]{images/ib_on_slp_1.pdf}
  \caption{\textbf{Visualizations of SMPLify-IB on SLP.}} 
  \label{fig: ib_slp}
\end{figure}



\section{Visualization results}

In this section, we present additional visualization results to verify the efficiency of our general framework for the in-bed HPS task.
\subsection{Visualizations for Time Consumption of self-penetration algorithms}
\begin{figure}[htbp]
  \centering
  \includegraphics[width=\linewidth]{images/sup_smplify_time.pdf}
  \caption{\textbf{Time consumption when deploying the two self-penetration detection and computation algorithms in our optimization routine.} We count the time taken in an optimization stage with 500 iterations on a single batch (128 frames) and document the proportion of time spent by the self-penetration modules in the overall duration~(in deep blue).} 
  % \vspace{-0.5cm}
  \label{fig: sup_smplify_time}
\end{figure}
\cref{fig: sup_smplify_time} provides quantitative comparisons on time consumption of our optimization routines with SOTA self-penetration algorithm~(Self-Contact in SMPLify-XMC~\cite{muller2021self}) and our proposed light-weight approach~(downsample 1/3 version). The experiment is conducted on a NVIDIA. 3090 GPU, with each optimization performing with 500 iterations on a single batch (128 frames). While the Self-Contact algorithm yields high detection accuracy, it comes at a significant time and computational expense~(\ie, nearly 100s per frame on a RTX3090 GPU). Our detection module brings nearly 450 times speed while archiving comparable self-penetration refinement.

\subsection{Visualizations for gravity-based constraints.}
~\cref{fig: sup_smplify_gravity} provides more visual evidence on the efficiency of our gravity constraints in SMPLify-IB. Traditional single-view regression-based method~(yellow meshes by Cliff) and optimization-based method~(red meshes by a SMPLify-like approach adopted in TIP) face serious depth ambiguity in the in-bed scene, especially when limbs overlap from the camera perspective, thus leading to implausible limb lifts~(\eg, hand lifts in the first and third rows in~\cref{fig: sup_smplify_gravity}, and leg lifts when legs contact and overlap in the third row). Our proposed gravity constraints, accompanied by a strong self-penetration detection and penalty term, effectively alleviate the depth ambiguity issue while maintaining reasonable contact. This validates the feasibility of alleviating depth ambiguity issues with physical constraints in specific scenarios.
\begin{figure}[t]
  \centering
  \includegraphics[width=\linewidth]{images/sup_smplify_gravity.pdf}
  \caption{\textbf{Qualitative comparisons on the p-GTs generated by Cliff~(predicted on images), TIP and our generations by SMPLify-IB.} We highlight the implausible limb lifts by single-view depth ambiguity in red ellipses and our refinement with yellow ellipses.} 
  % \vspace{-0.5cm}
  \label{fig: sup_smplify_gravity}
\end{figure}

\subsection{Failure cases for SMPLify-IB}

About 1.6\% samples of our optimization results might fail due to severely false initialization by CLIFF, wrong judgment in gravity constraints, and trade-offs in the multiple-term optimization, as presented in~\cref{fig: sup_smplify_failure_case}. Thus we manually inspected all generated results and carried out another round of optimization to address these errors, aiming at generating reliable p-GTs for the TIP dataset. The refinement is highlighted with yellow ellipses in~\cref{fig: sup_smplify_failure_case}. 
\begin{figure}[t]
  \centering
  \includegraphics[width=\linewidth]{images/sup_smplify_failure_case.pdf}
  \caption{\textbf{Typical failure cases of SMPLify-IB.} We highlight the wrong generations with red markers and our refinement in the yellow ellipses.} 
  % \vspace{-0.5cm}
  \label{fig: sup_smplify_failure_case}
\end{figure}

\subsection{Failure cases for PI-HMR}
In~\cref{fig: sup_pihmr_failure_case}, we show a few examples where PI-HMR fails to reconstruct reasonable human bodies. The reason mainly falls in the information ambiguities, ranging from (a) PI-HMR mistakenly identifies the contact pressure between the foot and the bed as originating from the leg~(shown in the red ellipse), (b) hand lifts and (c) leg lifts.
\begin{figure}[htbp]
  \centering
  \includegraphics[width=\linewidth]{images/sup_pihmr_failure_case.pdf}
  \caption{\textbf{Typical failure cases of PI-HMR.} We highlight the mispredictions and corresponding pressure regions with red markers.} 
  % \vspace{-0.5cm}
  \label{fig: sup_pihmr_failure_case}
\end{figure}

\subsection{More Qualitative Visualizations}

We present more qualitative visualizations on the performance of our proposed optimization strategy SMPLify-IB in~\cref{fig: sup_simplify_quali} and PI-HMR in~\cref{fig: sup_pihmr_quali} and~\cref{fig: sup_pihmr_quali_2}.

\section{Limitations and Future works}

we conclude our limitations and future works in three main aspects:

(1) \textbf{Hand and foot parametric representations:} More diverse and flexible tactile interactions exist in the in-bed scenarios. For instance, the poses of the hands and feet vary with different human postures, thereby influencing the patterns of localized pressure. However, the SMPL model fails to accurately depict the poses of hands and feet, thereby calling for more fine-grained parametric body representations~\cite{pavlakos2019expressive, osman2022supr} to precisely delineate the contact patterns between human bodies and the environment.

(2) \textbf{Explicit constraints from contact cues:} In this work, we propose an end-to-end learning approach to predict human motions directly from pressure data. The learning-based pipeline can rapidly sense the pressure distribution patterns and generate high-quality predictions from pressure sequences, yet it may lead to underutilization of contact priors from pressure sensors and cause misalignment between limb position and contact regions~(\eg, torso and limbs lift). In future works, we aim to explicitly incorporate contact priors through learning or optimization methods~\cite{shimada2023decaf} to further enhance the authenticity of the model's predictions.

(3) \textbf{Efforts for information ambiguity:} In this work, we aspire to mitigate the information ambiguity issue through pressure-based feature sampling and habit-based Test-Time Optimization strategies, yielding accuracy improvement; however, challenges persist. Building upon the observation that users perform movements in certain habitual patterns, we expect to develop a larger-scale motion generation model reliant on VQ-VAE~\cite{van2017neural} or diffusion~\cite{ho2020denoising} techniques, to address the deficiencies in single-pressure modality based on users' motion patterns.

\newpage
\begin{figure*}[htbp]
  \centering
  \includegraphics[width=\linewidth]{images/sup_simplify_quali.pdf}
  \caption{\textbf{Qualitative results of our generated p-GTs on the TIP dataset.} We compare our results with SOTA vision-based methods Cliff and TCMR~(predicted on RGB images) and p-GTs provided in TIP.} 
  % \vspace{-0.5cm}
  \label{fig: sup_simplify_quali}
\end{figure*}

\newpage
\begin{figure*}[htbp]
  \centering
  \includegraphics[width=\linewidth]{images/sup_pihmr_quali.pdf}
  \caption{\textbf{Qualitative results of PI-HMR's performance on the TIP dataset.} We compare our results with SOTA vision-based methods Cliff~(predicted on RGB images) and pressure-based method PI-Mesh.} 
  % \vspace{-0.5cm}
  \label{fig: sup_pihmr_quali}
\end{figure*}

\newpage
\begin{figure*}[htbp]
  \centering
  \includegraphics[width=\linewidth]{images/sup_pihmr_quali_2.pdf}
  \caption{\textbf{More qualitative results of PI-HMR's performance on the TIP dataset.}} 
  % \vspace{-0.5cm}
  \label{fig: sup_pihmr_quali_2}
\end{figure*}







% \newpage
\clearpage
{
\small
\noindent
\textbf{Acknowledgements:} We thank the anonymous reviewers for their suggestions. This work is supported by the National Natural Science Foundation of China under Grant No. 62072420.
}

{
    \small
    \bibliographystyle{ieeenat_fullname}
    \bibliography{main}
}

\end{document}
