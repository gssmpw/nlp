\begin{abstract}
Long-term in-bed monitoring benefits automatic and real-time health management within healthcare, and the advancement of human shape reconstruction technologies further enhances the representation and visualization of users' activity patterns. However, existing technologies are primarily based on visual cues, facing serious challenges in non-light-of-sight and privacy-sensitive in-bed scenes. Pressure-sensing bedsheets offer a promising solution for real-time motion reconstruction. Yet, limited exploration in model designs and data have hindered its further development. To tackle these issues, we propose a general framework that bridges gaps in data annotation and model design. Firstly, we introduce SMPLify-IB, an optimization method that overcomes the depth ambiguity issue in top-view scenarios through gravity constraints, enabling generating high-quality 3D human shape annotations for in-bed datasets. Then we present PI-HMR, a temporal-based human shape estimator to regress meshes from pressure sequences. By integrating multi-scale feature fusion with high-pressure distribution and spatial position priors, PI-HMR outperforms SOTA methods with 17.01mm Mean-Per-Joint-Error decrease. This work provides a whole tool-chain to support the development of in-bed monitoring with pressure contact sensing.

% . This approach reduces prediction errors by 20mm compared to state-of-the-art models.

    % 人體姿態估計和可視化技術將輔助這一任務
    % Humans spend about one-third of their lives resting. Reconstructing human dynamics in in-bed scenarios is of considerable significance in sleep studies, bedsore monitoring, and biomedical factor extractions. However, the mainstream human pose and shape estimation methods mainly focus on visual cues, facing serious issues in non-line-of-sight environments. Since in-bed scenarios contain complicated human-environment contact, pressure-sensing bedsheets provide a non-invasive and privacy-preserving approach to capture the pressure distribution on the contact surface, and have shown prospects in many downstream tasks. However, few studies focus on in-bed human mesh recovery. To explore the potential of reconstructing human meshes from the sensed pressure distribution, we first build a high-quality temporal human in-bed pose dataset, TIP, with 156K multi-modality synchronized images. We then propose a label generation pipeline for in-bed scenarios to generate reliable 3D mesh labels with a SMPLify-based optimizer. Finally, we present PIMesh, a simple yet effective temporal human shape estimator to directly generate human meshes from pressure image sequences. We conduct various experiments to evaluate PIMesh's performance, showing that PIMesh archives 79.17mm joint position errors on our TIP dataset. The results demonstrate that the pressure-sensing bedsheet could be a promising alternative for long-term in-bed human shape estimation.
\end{abstract}