\begin{figure}[t]
  \centering
  \includegraphics[width=\linewidth]{images/dataset_pen_demo.pdf}
  \caption{A glimpse of TIP dataset, with p-GTs from TIP and our SMPLify-IB. we highlight its drawbacks with red ellipses and our refinements in yellow ones.}
  \label{fig: dataset_vis} 
\end{figure} 
\section{Related Work}
\textbf{Regression for HPS.} Recent years have witnessed tremendous advances in vision-based human shape reconstruction approaches from images~\cite{kanazawa2018end, kolotouros2019learning, kocabas2021spec, zhang2021pymaf, kocabas2021pare, li2022cliff, wang2023refit, shimada2023decaf, goel2023humans, dwivedi2024tokenhmr, shin2024wham, song2024posturehmr} based on the parametric human body model~(\ie, SMPL~\cite{loper2023smpl}). Meanwhile, several works take video clips as input to exploit the temporal cues~\cite{kanazawa2019learning, kocabas2020vibe, choi2021beyond, wei2022capturing, shen2023global, you2023co}, utilizing the temporal context to improve the smoothness. 

% A representative paradigm is VIBE~\cite{kocabas2020vibe}, which incorporates a static encoder for texture feature extraction, a temporal encoder for temporal consistency digestion, and a regressor for final SMPL predictions. In this work, PI-HMR also follows such a structure for its conciseness yet efficiency.

%\citet{kanazawa2018end} proposed HMR, an end-to-end framework coupled with an adversarial prior to reconstruct plausible human meshes from images. They further introduced HMR2.0~\cite{goel2023humans}, incorporating a stronger Vision-Transformer~\cite{dosovitskiy2020image} encoder instead of ResNet50~\cite{he2016deep} to improve the feature quality and generalization, and proved SOTA performance on benchmark datasets, especially for unusual poses. %\citet{kocabas2021spec} and \citet{li2022cliff} focused on refining the camera models to improve the robustness of 2D projections in model training.

We mainly focus on HPS from contact pressure sensing. Unlike visual information, the representation pattern of contact pressure data is influenced by its perceptual medium, thus necessitating a corresponding alteration in algorithm design. Typical sensing devices, combined with HPS algorithms,~(\eg, carpets~\cite{luo2021intelligent, chen2024cavatar}, clothes~\cite{zhou2023mocapose, zhang2024learn}, bedsheets or mattress~\cite{liu2022simultaneously, wu2024seeing, clever2020bodies, tandon2024bodymap}, and shoes~\cite{zhang2024mmvp, van2024diffusionposer}), are applied as a major modality or supplements to help generate robust body predictions in pre-defined scenes or tasks. Nevertheless, the process strategy of pressure data leans on vision pipelines, lacking a thorough contemplation of its inherent nature. 

% Hence, we present PI-HMR, as a preliminary attempt to tackle aforesaid bottlenecks with space-prior injection and multi-scale feature fusions. Our findings show that our streamline outperforms prior methods.

\textbf{Optimization for HPS.} Optimization-based methods typically fit the SMPL parameters to image cues~\cite{bogo2016keep, pavlakos2019expressive}~(\eg detected 2D joints~\cite{cao2017realtime, xu2022vitpose}), combined with data and prior terms. Follow-up studies further introduced supplement supervisions, including, but not limited to temporal consistency~\cite{arnab2019exploiting}, environment~\cite{kaufmann2023emdb}, human-human/scene contact~\cite{hassan2019resolving, muller2024generative, huang2024intercap}, self-contact~\cite{muller2021self} and large language models~(LLMs)~\cite{subramanian2024pose} to regularize motions in specific context. Besides, in recent years, efforts have emerged to integrate both optimization and regression methods as a cheap but effective annotation technique to produce pseudo-labels for visual datasets~\cite{wu2024seeing, zhang2024mmvp, huang2024intercap}, especially for monocular data from online images and videos~\cite{joo2021exemplar, muller2021self, lin2023one, yi2023generating}. 

%Compared with regression-based methods, optimizations could, to some extent, disregard the domain gaps between training and testing datasets to generate meshes that align with the 2D image evidence. 
% However, due to the inherent depth ambiguities in 2D projection, traditional monocular optimization methods struggle in sub-optimal solutions, especially in some specific scenes and views~(\eg images in top-down views from in-bed datasets like SLP~\cite{liu2022simultaneously} and TIP~\cite{wu2024seeing}, shown in Fig.~\ref{fig: dataset_vis}). This would significantly undermine the authenticity of the data labels and final predictions. 

% To generate high-quality and reliable pseudo-GTs for monocular in-bed images, we introduce SMPLify-IB, a gravity-constrained optimization approach coupled with a lightweight contact-penetration penalty module, aiming at alleviating depth ambiguity issues based on environmental and physical priors. 

% We present detailed explanations and result demonstrations in the following sections to validate its effectiveness.

\textbf{In-bed human pose and shape estimation. } 
Compared with other human-related tasks, in-bed HPS faces more serious challenges from data quality and privacy issues. Thus, efforts are devoted to pursuing environmental sensors for in such a non-light-of-sight~(NLOS) scenes, such as infrared camera~\cite{liu2019seeing, liu2019bed}, depth camera~\cite{grimm2012markerless, achilles2016patient, clever2020bodies}, pressure-sensing mattresses~\cite{clever20183d, clever2020bodies, davoodnia2023human, wu2024seeing}. Specifically for pressure-based approaches, \citet{clever2020bodies} conducted pioneering studies by involving pressure estimation to reconstruct in-bed shapes from a single pressure image~\cite{clever2020bodies}. \citet{wu2024seeing} collected a three-modality in-bed dataset TIP, and employed a VIBE-based network to predict in-bed motions from pressure sequences. 
% Considering the suitable scenarios and limitations of different sensors, some researchers employ multi-sensor information to fuse the strengths of each sensing unit~\cite{yin2022multimodal, tandon2024bodymap}. 
\citet{yin2022multimodal} proposed a pyramid scheme to infer in-bed shapes from aligned depth, LWIR, RGB, and pressure images, and \citet{tandon2024bodymap} improves accuracy on SLP~\cite{liu2022simultaneously} with depth and pressure modalities by integrating a pressure prediction module as auxiliary supervision. 

% \textbf{Pose Priors for optimization. } 


% tokenhmr, proxemics, pose priors, HMP

% \textbf{Cross-modal knowledge distillation. }
% Introduced by \citet{hinton2015distilling}, knowledge distillation~(KD) firstly aimed at transferring knowledge from a teacher network to a student model by soft logit distribution alignment, and evolves into a general framework for model compression~\cite{romero2014fitnets, zagoruyko2016paying, tung2019similarity}, continual learning~\cite{zhai2019lifelong}, cross-domain adaptation~\cite{liu2021source} and cross-modal fusion~\cite{gupta2016cross, aytar2016soundnet, zhao2018through, chen2021distilling, yun2023dense}. 

% \citet{gupta2016cross} proposed \textit{supervision transfer} to transfer knowledge from learned representation of paired RGB images to new unlabeled modalities. \citet{zhao2018through} conducted a radio-based pose estimation system with the guidance of a pre-trained RGB-based approach and paired image-radio data. 
