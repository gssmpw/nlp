%%
%% This is file `sample-acmlarge.tex',
%% generated with the docstrip utility.
%%
%% The original source files were:
%%
%% samples.dtx  (with options: `all,journal,bibtex,acmlarge')
%% 
%% IMPORTANT NOTICE:
%% 
%% For the copyright see the source file.
%% 
%% Any modified versions of this file must be renamed
%% with new filenames distinct from sample-acmlarge.tex.
%% 
%% For distribution of the original source see the terms
%% for copying and modification in the file samples.dtx.
%% 
%% This generated file may be distributed as long as the
%% original source files, as listed above, are part of the
%% same distribution. (The sources need not necessarily be
%% in the same archive or directory.)
%%
%%
%% Commands for TeXCount
%TC:macro \cite [option:text,text]
%TC:macro \citep [option:text,text]
%TC:macro \citet [option:text,text]
%TC:envir table 0 1
%TC:envir table* 0 1
%TC:envir tabular [ignore] word
%TC:envir displaymath 0 word
%TC:envir math 0 word
%TC:envir comment 0 0
%%
%% The first command in your LaTeX source must be the \documentclass
%% command.
%%
%% For submission and review of your manuscript please change the
%% command to \documentclass[manuscript, screen, review]{acmart}.
%%
%% When submitting camera ready or to TAPS, please change the command
%% to \documentclass[sigconf]{acmart} or whichever template is required
%% for your publication.
%%
%%

\PassOptionsToPackage{table}{xcolor}
\documentclass[manuscript]{acmart}

\settopmatter{printacmref=false} % Removes citation information below abstract
\renewcommand\footnotetextcopyrightpermission[1]{} % removes footnote with conference information in first column
\pagestyle{plain} % removes running headers
\usepackage{caption}
%\usepackage[table,xcdraw]{xcolor}

\usepackage{graphicx}
\usepackage{float} 
\usepackage{subfig}
\usepackage{subcaption}
\usepackage{overpic}
\usepackage{lipsum}
%%
%% \BibTeX command to typeset BibTeX logo in the docs
\AtBeginDocument{%
  \providecommand\BibTeX{{%
    Bib\TeX}}}

%% Rights management information.  This information is sent to you
%% when you complete the rights form.  These commands have SAMPLE
%% values in them; it is your responsibility as an author to replace
%% the commands and values with those provided to you when you
%% complete the rights form.
\setcopyright{acmlicensed}
\copyrightyear{2018}
\acmYear{2018}
\acmDOI{XXXXXXX.XXXXXXX}

%%
%% These commands are for a JOURNAL article.
\acmJournal{POMACS}
\acmVolume{37}
\acmNumber{4}
\acmArticle{111}
\acmMonth{8}

%%
%% Submission ID.
%% Use this when submitting an article to a sponsored event. You'll
%% receive a unique submission ID from the organizers
%% of the event, and this ID should be used as the parameter to this command.
%%\acmSubmissionID{123-A56-BU3}

%%
%% For managing citations, it is recommended to use bibliography
%% files in BibTeX format.
%%
%% You can then either use BibTeX with the ACM-Reference-Format style,
%% or BibLaTeX with the acmnumeric or acmauthoryear sytles, that include
%% support for advanced citation of software artefact from the
%% biblatex-software package, also separately available on CTAN.
%%
%% Look at the sample-*-biblatex.tex files for templates showcasing
%% the biblatex styles.
%%

%%
%% The majority of ACM publications use numbered citations and
%% references.  The command \citestyle{authoryear} switches to the
%% "author year" style.
%%
%% If you are preparing content for an event
%% sponsored by ACM SIGGRAPH, you must use the "author year" style of
%% citations and references.
%% Uncommenting
%% the next command will enable that style.
%%\citestyle{acmauthoryear}


%%
%% end of the preamble, start of the body of the document source.
\begin{document}

%%
%% The "title" command has an optional parameter,
%% allowing the author to define a "short title" to be used in page headers.
\title{Exploring the Effects of Traditional Chinese Medicine Scents on Mitigating Driving Fatigue}

%%
%% The "author" command and its associated commands are used to define
%% the authors and their affiliations.
%% Of note is the shared affiliation of the first two authors, and the
%% "authornote" and "authornotemark" commands
%% used to denote shared contribution to the research.
\author{Nengyue Su}
 \affiliation{
  \department{School of Computer Science and Engineering}
  \institution{University of Electronic Science and Technology of China}
  \city{Chengdu}
  \country{China}
 }
\author{Liang Luo}
 \affiliation{
  \department{School of Computer Science and Engineering}
  \institution{University of Electronic Science and Technology of China}
  \city{Chengdu}
  \country{China}
 }
\author{Yu Gu}
 \affiliation{
  \department{School of Computer Science and Engineering}
  \institution{University of Electronic Science and Technology of China}
  \city{Chengdu}
  \country{China}
 }
\author{Fuji Ren}
 \affiliation{
  \department{School of Computer Science and Engineering}
  \institution{University of Electronic Science and Technology of China}
  \city{Chengdu}
  \country{China}
 }
\renewcommand{\shortauthors}{Nengyue Su, et al.}


%%
%% By default, the full list of authors will be used in the page
%% headers. Often, this list is too long, and will overlap
%% other information printed in the page headers. This command allows
%% the author to define a more concise list
%% of authors' names for this purpose.


%%
%% The abstract is a short summary of the work to be presented in the
%% article.
\begin{abstract}
The rise of autonomous driving technology has led to concerns about inactivity-induced fatigue. This paper explores Traditional Chinese Medicine (TCM) scents for mitigating. Two human-involved studies have been conducted in a high-fidelity driving simulator. Study 1 maps six prevalent TCM scents onto the arousal/valence circumplex to select proper candidates, i.e., argy wormwood (with the highest arousal) and tangerine peel (with the highest valence). Study 2 tests both scents in an auto-driving course. Statistics show both scents can improve driver alertness and reaction-time, but should be used in different ways: argy wormwood is suitable for short-term use due to its higher intensity but poor acceptance, while tangerine peel is ideal for long-term use due to its higher likeness. These findings provide insights for in-car fatigue mitigation to enhance driver safety and well-being. However, issues such as scent longevity as for aromatherapy and automatic fatigue prediction remain unresolved.

\end{abstract}

%%
%% The code below is generated by the tool at http://dl.acm.org/ccs.cfm.
%% Please copy and paste the code instead of the example below.
%%

\begin{CCSXML}
<ccs2012>
   <concept>
       <concept_id>10003120.10003121.10011748</concept_id>
       <concept_desc>Human-centered computing~Empirical studies in HCI</concept_desc>
       <concept_significance>500</concept_significance>
       </concept>
 </ccs2012>
\end{CCSXML}

\ccsdesc[500]{Human-centered computing~Empirical studies in HCI}

%%
%% Keywords. The author(s) should pick words that accurately describe
%% the work being presented. Separate the keywords with commas.
\keywords{Perception, Smell, Odor Stimulation, Driving fatigue, In-Car systems}

% \received{20 February 2007}
% \received[revised]{12 March 2009}
% \received[accepted]{5 June 2009}

%%
%% This command processes the author and affiliation and title
%% information and builds the first part of the formatted document.
\maketitle

\section{Introduction}
Currently, the emergence of autonomous driving technology has rendered driving more monotonous, thereby elevating the likelihood of passive fatigue (as in Figure~\ref{fig:1}) \cite{jamson2013behavioural}. This mental condition, marked by low valence and arousal, prolongs drivers' reaction-times \cite{eyben2010emotion} and, if left unaddressed, can lead to accidents \cite{li2024review}, posing substantial risks to road safety \cite{zhang2016traffic}.

Addressing driving fatigue often entails sensory stimuli, which can be categorized into visual \cite{li2021visual,qin2021characteristics,hassib2019detecting}, auditory \cite{orsini2024music,trumbo2017name}, and olfactory \cite{dahlman2024vehicle,jiang2023study} techniques. Considering the pivotal roles of vision and hearing in driving, olfactory modulation presents distinct advantages in maintaining driver functionality. Previous scent-related research has predominantly centered on Western essential oils, which may not resonate widely with Chinese populations. However, insights from Chinese aromatherapy traditions \cite{nan2013aromatherapy,giannenas2020history} hint that scents derived from Traditional Chinese Medicine (TCM) could be more culturally acceptable. Nevertheless, the influence of TCM scents on emotions and driving fatigue remains unexplored. Hence, this study endeavors to delve into this domain.

More specifically, two studies involving Chinese participants were conducted in a high-fidelity driving simulator. In Study 1, six TCM scents (see Fig. 2 for their appearances), i.e., acorus tatarinowii, argy wormwood, aromatic turmeric root-tuber, Chinese angelica, spine date seed, and tangerine peel, are mapped onto the arousal/valence circumplex(as in \cite{dmitrenko2020caroma}). Among them, argy wormwood (with the highest arousal) and tangerine peel (with the highest valence) were chosen for further examination . Study 2 aimed to explore the specific effects of these two scents on driving fatigue in a predesigned auto-driving course (prolonged driving on a straight highway with no traffic on a cloudy day). The results indicate that both scents can enhance driver alertness and reaction-time. However, their usage differs: argy wormwood is suitable for short-term applications due to its high intensity but lower acceptance, whereas tangerine peel is preferred for long-term use due to its higher popularity. 

The studies have examined the effect of Traditional Chinese Medicine (TCM) scents on mitigating driving fatigue, yet leaving unresolved questions such as the short duration of scent effectiveness in aromatherapy and the comparison between TCM and conventional flower and fruit scents. Furthermore, how to promptly detect driving fatigue? Additionally, could we develop a scent-based fatigue mitigation system that can predict the onset of driving fatigue by analyzing driving contexts (like the setting we used in the simulator) via Artificial Intelligence (AI)?

%Our study aims to solve driving fatigue by Traditional Chinese Medicine(TCM) scents. As shown in the left picture\protect\footnotemark, autonomous driving can easily lead to driving fatigue. If not regulated, it may cause a car accident. Our research has found that regulation  with tangerine peel can make the driver comfortable (valence) and alert (arousal), while regulation with argy wormwood can make the driver less comfortable but more alert. Both benefits driving safety.%
\begin{figure}
    \centering
    \includegraphics[width=1\linewidth]{fig/1.1.pdf}
    \caption{Our study aims to mitigate driving fatigue with TCM scents. As the left picture\protect\footnotemark[1] shows, autonomous driving can cause driving fatigue easily. Unregulated, it may lead to a car accident. Our research found that tangerine peel regulation can make drivers comfortable and alert, while argy wormwood regulation makes drivers less comfortable but more alert. Both benefit driving safety. }
    \label{fig:1}
\end{figure}

\footnotetext[1]{Generated by Doubao with prompt "Please generate a picture showing a side view of the inside of a car. In the picture, the driver is yawning with one hand covering their mouth and the other hand on their leg. There are autonomous driving icons like Wi-Fi and navigation on the steering wheel, and the car can automatically detect vehicles outside the window."}

\section{Related Work}
We structured the related work around two research areas:


\subsection{How to mitigate driving fatigue?}
Based on the classification of sensory stimuli, methods aimed at mitigating driving fatigue can be categorized into visual, auditory, and olfactory approaches. With regard to visual methods, adjustments to ambient lighting have been employed to enhance driver alertness \cite{li2021visual,qin2021characteristics,hassib2019detecting}. For auditory methods, music\cite{guo2024could,orsini2024music} or name music game\cite{trumbo2017name} have proven effective . In driving scenarios, vision and hearing serve as the primary channels for information reception. While modulation through these two sensory modalities may lead to driver distraction, olfaction is less utilized during driving and exhibits a closer relationship with emotional states\cite{alaoui1997basic}. Consequently, this study employs olfactory modulation as a means to address driving fatigue.

Contemporary research has substantiated that scents such as mint, grapefruit, lemon, and lavender can markedly ameliorate driving fatigue \cite{jiang2024scented,dahlman2024vehicle,yoshida2011study}. However, previous studies have predominantly focused on common flower and fruit scents that are widely utilized in Western countries. Given the disparities in culture and physiological traits, these scents are not frequently utilized among the Chinese populace \cite{yunjun2013perfume}. Consequently, there is a pressing need to explore scents that are more familiar and acceptable to Chinese individuals.

\subsection{Why we choose TCM?}
The Chinese populace has maintained a long-standing tradition of utilizing Traditional Chinese Medicine (TCM) to regulate emotions since ancient times \cite{farrar2020clinical}. In the framework of TCM, a bi-directional causal relationship exists between the harmonious flow of \textit{qi}\footnote[1]{Chinese identify \textit{qi} as the natural energy intrinsic to all things that exist in the universe, as the fundamental life energy responsible for health and vitality\cite{mccaffrey2003qigong}.} and both a healthy mental state and emotional equilibrium \cite{zhou2021conceptualization}. Specific TCM herbs, including Chinese angelica, acorus tatarinowii, spine date seed, and tangerine peel, have demonstrated efficacy in treating depression \cite{li2020traditional,zhang2019challenge,sun2022dissecting}.
However, TCM is predominantly administered for internal consumption. Relatively few TCM formulations have been processed into high-purity essential oils for aromatherapy purposes, and the precise effects of TCM scents on emotions remain unclear. Therefore, this study selects TCM essential oils with potential relevance to emotion regulation for investigation.


\section{Experiments}

\subsection{Overview of Experiments}
To explore the effects of TCM scents on mitigating driving fatigue, we designed two studies.
\begin{itemize}
    \item \textbf{Study 1:} Mapping TCM scents onto the arousal/valence circumplex. 
    \item \textbf{Study 2\protect\footnotemark[2]:} Exploring the effects of TCM scents on driving fatigue.
\end{itemize}
Both studies have passed the ethical review of the University of Electronic Science and Technology of China. All participants were carefully screened to make sure they had no breathing problems and no smell allergies.

\footnotetext[2]{To explore the duration of the driving fatigue induction experiment and he scent concentration of the scent release experiment in Study 2, a Pre-study was designed, as shown in the appendix. }

\subsection{Study 1: Mapping TCM scents onto the arousal/valence circumplex.}


\begin{figure}
    \begin{minipage}[]{0.48\textwidth}
        \vspace{0.0in}
        \centering
        \resizebox{1\linewidth}{!}{
            \begin{tabular}{|l|l|l|}
            \hline
            TCM                                                                                                                    & Ingredients                                                             & Efficacy                                                                                                                              \\ \hline
            \cellcolor[HTML]{F2F2F2}{\color[HTML]{08090C} Argy Wormwood}                                                           & \begin{tabular}[c]{@{}l@{}}Eucalyptol, \\ Caryophyllene\end{tabular}    & \begin{tabular}[c]{@{}l@{}}Warm the meridians to stop bleeding, \\ dispel cold and relieve pain, \\ calm anxiety\end{tabular}         \\ \hline
            \cellcolor[HTML]{F2F2F2}{\color[HTML]{08090C} Acorus Tatarinowii}                                                      & \begin{tabular}[c]{@{}l@{}}p - Cymene, \\ Nerol acetate\end{tabular}    & \begin{tabular}[c]{@{}l@{}}Open orifices and resolve phlegm, \\ remove dampness \\ and promote appetite\end{tabular}                  \\ \hline
            \cellcolor[HTML]{F2F2F2}{\color[HTML]{08090C} Chinese Angelica}                                                        & Ligustilide                                                             & \begin{tabular}[c]{@{}l@{}}Nourish and activate blood, \\ regulate mood\end{tabular}                                                  \\ \hline
            \cellcolor[HTML]{F2F2F2}{\color[HTML]{08090C} \begin{tabular}[c]{@{}l@{}}Aromatic Turmeric \\ Root-tuber\end{tabular}} & \begin{tabular}[c]{@{}l@{}}Camphene, Camphor,\\  Curcumene\end{tabular} & \begin{tabular}[c]{@{}l@{}}Promote blood circulation \\ to relieve pain, promote \textit{qi} circulation\\ and relieve depression\end{tabular} \\ \hline
            \cellcolor[HTML]{F2F2F2}{\color[HTML]{08090C} Spine Date Seed}                                                         & Suanzaoren fatty acids                                                  & \begin{tabular}[c]{@{}l@{}}Nourish \textit{qi} and \\ tonify the liver, calm the heart \\ and soothe the nerves\end{tabular}                   \\ \hline
            \cellcolor[HTML]{F2F2F2}{\color[HTML]{08090C} Tangerine Peel}                                                          & Hesperidin                                                              & \begin{tabular}[c]{@{}l@{}}Regulate \textit{qi} and \\ nourish the spleen, \\ dry dampness and resolve phlegm\end{tabular}                     \\ \hline
            \end{tabular}
        }
        \vspace{0.1in}
        \footnotetext{(a) Selected TCM}
    \end{minipage}
    \begin{minipage}[]{0.5\textwidth}
        \centering
        \includegraphics[width=1\linewidth]{fig/2.1.pdf}
        \footnotetext{(b) Arousal/valence circumplex}
    \end{minipage}
    \caption{Material and results of Study 1:(a) shows the ingredients and efficacy of the six TCM scents selected in Study 1. (b) presents the results of Study 1, mapping the TCM scents onto the arousal/valence circumplex.}
    \label{Fig.2}
\end{figure}

In this study, we conducted a screening of existing TCM essential oils and selected six that are specifically associated with emotional regulation for further research. Fig.\ref{Fig.2}(a) depicts the six TCM scents, along with their constituent ingredients and efficacy, as outlined in the Pharmacopoeia of the People's Republic of China \cite{2020Pharmacopoeia}.

\subsubsection{Design}
We conducted a within-participants study asking the participants to rate six scents (argy wormwood, acorus tatarinowii, Chinese angelica, aromatic turmeric root-tuber, spine date seed, and tangerine peel).

\subsubsection{Setup}
Participants sniffed the scents from identical bottles located on a table. Each bottle contained 10ml of essential oil. They sniffed each bottle for 2s, with intervals of 20s(as in \cite{wilson2017multi}).

\subsubsection{Procedure}
After sniffing each scent, participants were asked to rate the valence and arousal of each scent using Self-Assessment Manikins (SAM)\cite{bradley1994measuring} on a 5-point Likert scale (1 = "low", 5 = "high", see Fig.~\ref{Fig.2}(b)).

\subsubsection{Results}
18 participants, 23-26 years old (M = 24.5, SD = 1.04, 8 females), volunteered for this study. 
The normality test showed that the scent ratings were normally distributed. We conducted a repeated measures ANOVA test to analyze the data and found that scents had a main effect on both arousal (F = 2.509, p < 0.05) and valence (F = 9.823, p < 0.001). The results of this study (see Fig. ~\ref{Fig.2}(b)) confirmed that among the six TCM scents, argy wormwood (Arousal: M = 3.5, SD = 1.04; Valence: M = 2.89, SD = 0.76) and acorus tatarinowii (Arousal: M = 3.28, SD = 1.45; Valence: M = 1.83, SD = 0.92) had low valence and high arousal. Chinese angelica (Arousal: M = 2.89, SD = 1.18; Valence: M = 2.17, SD = 0.71), aromatic turmeric root-tuber (Arousal: M = 2.61, SD = 1.29; Valence: M = 2.84, SD = 0.99), and spine date seed (Arousal: M = 2.44, SD = 1.2; Valence: M = 2.06, SD = 0.99) had low valence and low arousal. tangerine peel (Arousal: M = 2.39, SD = 1.09; Valence: M = 3.61, SD = 1.04) had low arousal and high valence. Since arousal is an important indicator of driving fatigue\cite{jiang2024scented}, we chose argy wormwood which has the highest arousal values for Study 2. Considering its valence value is low, we chose tangerine peel which has a relatively high valence for comparison.

\begin{figure}[!t]
\centering
\subfloat[The location of the experimental equipment]{
		\includegraphics[scale=0.35]{fig/3.1a.pdf}}
\subfloat[The scenes of the driving simulator]{
		\includegraphics[scale=0.35]{fig/3.3b.pdf}}
\caption{The on-site equipment for Study 2: Fig. (a) shows the position of the steering wheel, the buttons of the reaction-time recording program, and the scent release device. Fig. (b) shows the driving simulation scene used throughout Study 2. }
\label{fig:3}
\end{figure}


\subsection{Study 2:Exploring the effects of argy wormwood and tangerine peel on driving fatigue}

\subsubsection{Design}
This study followed a within-subjects experimental design of 1 (state: fatigued) × 3 (scents: tangerine peel, argy wormwood, water). It mainly measures the impact of TCM scents on three indicators: alertness, driving performance, and acceptance.

\footnotetext[1]{Generally, KSS is a 9-level scale. However, due to the limited number of participants in this experiment, for the convenience of data analysis, levels 2, 4, 6, and 8 were removed, and only levels 1, 3, 5, 7, and 9 were retained to form a 5-level scale. }

\begin{itemize}
    \item \textbf{Alertness:} \textit{Karolinska Sleepiness Scale (KSS)\cite{lee2020making}:} It is a subjective evaluation scale to measure the degree of sleepiness. 1 - 5 represents awake - sleepy\footnotemark[1].
    \item \textbf{Driving performance:}\textit{reaction-time:} Participants quickly press a button to record the reaction-time when they hear a beep.
    \item \textbf{Acceptance:} \textit{Questionnaire:} The questionnaire examines three indicators of scent acceptance: liking, comfort, and intensity\cite{dmitrenko2020caroma}. 1 - 5 represents low - high degree. \textit{Interview:} The interview examines whether participants can clearly smell the scent, the duration of the scent effectiveness and whether they are willing to choose this scent as a regulator for driving fatigue and why. 
\end{itemize}
    



\subsubsection{Setup}
\begin{itemize}
    \item Driving Simulator: Participants sit in the seat of the driving simulator equipped with a Logitech G29 Driving Force steering wheel and pedals(see Fig.~\ref{fig:3}(a)). The front main screen (27-inch, 60Hz refresh rate) shows the view outside the vehicle. The driving simulation scenario is built with Carla\cite{dosovitskiy2017carla}. The simulated scenario is a straight highway with no traffic on a cloudy day(see Fig.~\ref{fig:3}(b))\cite{huang2024chatbot}. A Python program is used to record the reaction-time.
    \item Scent Delivery Equipment: A Nimin aroma diffuser, which can be connected to the Mi Home APP to adjust the scent-emitting concentration and interval time. Three diffusers contain diluted essential oils (essential oil: alcohol: water = 1:6:3) (as in \cite{jiang2023study})of tangerine peel, argy wormwood, and water. The scent delivery equipment is located on the left side of the steering wheel.
\end{itemize}

\subsubsection{Procedure}

As in Fig. ~\ref{fig:4}, participants read, signed the consent form, and filled out a pre-questionnaire on basic info like gender and age. Then, they put on headphones, tested the volume and the reaction-time recording program, and drove autonomously for 5min in a simulated highway scene, hands on the wheel and feet on the brake to get familiar with the program and driving simulator requirements. Next, they used the KSS questionnaire to rate fatigue. Then, participants engaged in driving fatigue induction experiment for 10min. After that, they filled out the KSS again to confirm the effect. Subsequently, three rounds of scent release experiments were administered. Scents were released at 0, 3, and 6min for 30s in each round. Participants pressed a button after a beep at 6 min, with reaction-time recorded and KSS filled out. The round order was randomized by Latin-square design to mitigate
sequence effects\cite{fisher1970statistical}. Finally, participants filled out a scent-evaluation questionnaire, had an interview, and got 50RMB for participation. The whole experiment took about 70min. 

\begin{figure}
    \centering
    \includegraphics[width=1\linewidth]{fig/4.1.pdf}
    \caption{The timeline of the Study 2 procedure. Participants filled out a pre-questionnaire. Then, they drove autonomously for 5min to get used to the simulator. Next, they drove autonomously for 10min to induce fatigue. Then, 3 rounds of scent release experiments were conducted. The round order was randomized. Finally, participants filled out a scent-evaluation questionnaire and had an interview.}
    \label{fig:4}
\end{figure}

\subsubsection{Results}
\paragraph{Participants}
20 participants, 22-28 years of age (M = 24.7, SD = 1.45, 8 females).They were asked to be well-rested and avoid caffeine or alcohol before the study. Based on interview results, data from 3 participants who were unsuccessfully induced into driving fatigue or were unfocused during the experiment were excluded. 

\begin{figure}[!t]
\centering
%\includegraphics[width=3in]{fig5}
\subfloat[KSS before and after driving fatigue induction experiment]{
		\includegraphics[scale=0.45]{fig/5.2a.pdf}}
\subfloat[KSS after scent release experiments]{
		\includegraphics[scale=0.45]{fig/5.2b.pdf}}
\\
\subfloat[Reaction-time after scent release experiments]{
		\includegraphics[scale=0.45]{fig/5.1c.pdf}}
\subfloat[Acceptance of TCM scents]{
		\includegraphics[scale=0.45]{fig/5.1d.pdf}}
\caption{Results of Study 2. (* p < 0.05, ** p < 0.01, *** p < 0.001, all figures follow this annotation)}
\label{fig:5}
\end{figure}

\paragraph{Alertness before and after the driving fatigue induction experiment}
The Shapiro-Wilk test showed no variables were normally distributed. We used the Wilcoxon-Signed-Ranks test to compare the means between two variables. We compared KSS values before and after the driving fatigue induction experiment (see Fig. ~\ref{fig:5}(a)). The test showed the  KSS value (M = 2.8, SD = 0.93) before the experiment was significantly lower than the one (M = 3.6, SD = 0.94, Z = 2.762, p < 0.01) after the experiment. This means that the participants were awake at the start of the study and reported feeling fatigued after the driving fatigue induction experiment. 



\paragraph{Alertness after scent release experiments}
A repeated-measures ANOVA on KSS scores showed scent types had a main effect on KSS ratings (F = 3.917, p < 0.05)(see Fig. ~\ref{fig:5}(b)), indicating scent types can affect drivers' subjective alertness scores. The Friedman test showed drivers' subjective alertness under argy wormwood and tangerine peel was higher than the baseline (water) (p < 0.05), yet there was no significant difference between argy wormwood and tangerine peel (p = 0.29). This means both TCM scents can improve drivers' subjective alertness. Argy wormwood's refreshing effect(M = 3.3, SD = 0.86) is slightly stronger than tangerine peel's(M = 3.1, SD = 1.02), but not significantly so. 

\paragraph{Driving performance after scent release experiments}
A repeated-measures ANOVA on reaction-time showed scent types had a main effect on driving performance (F = 3.95, p < 0.05)(see Fig. ~\ref{fig:5} (c)), indicating scent types can affect drivers' driving performance. The Friedman test showed drivers' driving performance under argy wormwood and tangerine peel was better than the baseline (water) (p < 0.05) , yet there was no significant difference between argy wormwood and tangerine peel (p = 0.85). This means both TCM scents can improve driving performance, with no significant difference between them.








\paragraph{Acceptance of TCM scents}
A repeated-measures ANOVA on questionnaire data showed scent types had a main effect on liking (F = 8.99, p < 0.01), comfort (F = 11.60, p < 0.01), and intensity (F = 5.71, p < 0.05)(see Fig. ~\ref{fig:6}), indicating scent types can affect drivers' acceptance. Paired-sample tests showed argy wormwood's liking(p < 0.01)and comfort(p < 0.01) were significantly lower than tangerine peel's. But argy wormwood's intensity(p < 0.05) was significantly higher than tangerine peel's(see Fig. ~\ref{fig:5}(d)). Interview data  showed 75\% of participants would choose tangerine peel as a regulator for driving fatigue because it can mitigate driving fatigue and it is comfortable. Only 15\% would choose argy wormwood due to its discomfort. 


\section{Discussion and Future Work}

\subsection{Discussion}
\begin{figure}
    \centering
    \includegraphics[width=0.95\linewidth]{fig/6.3.pdf}
    \caption{Interview results in Study 2: Most people are willing to use tangerine peel for driving fatigue regulation because it's comfortable and can help them stay alert. Most people are not willing to use argy wormwood because it's uncomfortable.}
    \label{fig:6}
\end{figure}
\begin{figure}
    \centering
    \includegraphics[width=0.95\linewidth]{fig/7.2.pdf}
    \caption{Feasibility of TCM scents as regulator for driving fatigue: During the experiment, only a few participants could smell the TCM scents all the time, indicating the scents' short-lasting time. }
    \label{fig:7}
\end{figure}


\subsubsection{Effect of TCM scents on driving fatigue} 
The two chosen TCM scents both have clear effects on mitigating driving fatigue. They can effectively boost drivers' alertness, reduce the fatigue induced by autonomous driving, and thereby enhance driving safety. Between the two, argy wormwood exhibits a slightly stronger effect than tangerine peel, although the difference is not substantial. This might be due to the narrow gap in arousal levels between argy wormwood and tangerine peel. Moreover, tangerine peel has a higher valence, which can provide a more enjoyable driving experience.

\subsubsection{Effect of TCM scents on driving performance}
 Upon exposure to the TCM scents, the participants exhibited a significant reduction in reaction-time, thereby indicating an improvement in driving performance. This outcome is advantageous for enhancing driving safety. Consistent with the findings related to alertness, no statistically significant difference was observed between the effects of argy wormwood and tangerine peel.

\subsubsection{Acceptance of TCM scents}
The acceptance of the two TCM scents differs significantly. Tangerine peel has a higher level of liking and comfort but a lower intensity, while argy wormwood has a lower level of liking and comfort but a higher intensity. Based on the interview results, tangerine peel is perceived as a more favorable regulator for driving fatigue, compared to argy wormwood. Therefore, tangerine peel may be more suitable for long-term application in driving fragrance due to its comforting and refreshing attributes. On the contrary, argy wormwood may be more appropriate as an alertness-enhancing scent during periods of fatigue. It can be activated either actively or passively when drivers experience severe drowsiness, as it possesses a notable refreshing effect, even though it is not as comforting.

\subsection{Future Work}
\subsubsection{Feasibility of TCM scents as regulator for driving fatigue}
 The duration of the scent effectiveness for TCM scents is relatively short, as illustrated in Fig.~\ref{fig:7}. Maybe it's because TCM essential oils are mainly used for massage or medicinal baths, not as specialized fragrance essential oils. Unlike regular fragrances, TCM essential oils lack top, middle, and base notes. Mixing multiple essential oils to make compound essential oils might prolong the duration.

\subsubsection{Auto-regulation system for mitigating driving fatigue by combining AI and TCM scents}
 AI can determine whether a driver is fatigued and the extent of their fatigue by analyzing both in-car and out-car information. It can use data from the dashboard camera to assess whether the driver has been engaged in prolonged monotonous driving, and it can utilize in-car cameras to recognize facial expressions\cite{zhu2022research} or upper body posture variations\cite{ansari2022automatic} to judge if the driver is experiencing fatigue. Based on the driver's condition and the characteristics of TCM scents, AI can generate appropriate adjustment strategies to mitigate driving fatigue.
\section{Conclusion}
The results of this study prove that the TCM scents can mitigate driving fatigue induced by autonomous driving, and enhance the drivers' alertness and driving performance. Among them, argy wormwood has a relatively high intensity and a better refreshing effect, but its liking and comfort level are relatively low. So it is more suitable as a short-term reminder scent. Tangerine peel has a relatively low intensity, but a higher liking and comfort level, making it more suitable as a in-car fragrance for long-term use. This study pioneers the exploration of the application of TCM scents in mitigating driving fatigue, offering invaluable insights for the future development of in-car systems. 

\bibliographystyle{ACM-Reference-Format}
\bibliography{ref}


\appendix
\section{Pre-study}
In order to explore the proper duration of the driving fatigue induction experiment and the scent concentration of the scent release experiment in Study 2, we designed a Pre-study to conduct the exploration. 
\subsection{Design}
This study adopts a within-subjects design, and it mainly consists of three steps: 
\begin{enumerate}
    \item Get familiar with the driving simulator.
    \item Record the time when participants reach a state of general fatigue (Level 7 on the KSS, described as: Sleepy, but no effort to keep awake) and extreme fatigue(Level 9 on the KSS, described as: Very sleepy, great effort to keep awake, fighting sleep).
    \item Conduct three rounds of scent release experiments. In each round, argy wormwood at different concentrations will be used. Finally, participants select the round with the most comfortable concentration. The comfort criterion is that the scent can be clearly smelled and not too pungent. 
\end{enumerate}

\subsection{Setup}
The experimental equipment and the on-site settings are kept consistent with those in Study 2.

\subsection{Procedure}
As in Fig. ~\ref{fig:app-1}, the participants read, signed the consent form and completed a pre-questionnaire on basic information such as gender and age. Then, they drove autonomously for 5min in a simulated highway scene, hands on the wheel and feet on the brake to get familiar with the driving simulator requirements. Next, they engaged in a driving fatigue induction experiment and recorded the time when they reached a state of general drowsiness and extreme drowsiness. The simulator scenario was the same as that in Study 2. The maximum duration of this experiment is 25min. If the participants still have not reached a state of extreme fatigue, the experiment will be stopped, and the data will be marked as "failure to induce fatigue".  After that, three rounds of scent release experiments were administered. The concentration of scent in each round was controlled by regulating the scent release time each time. The release times for the three rounds were 8s every three minutes, 16s every three minutes, and 30s every three minutes respectively. The round order was randomized by Latin-square design to mitigate sequence effects\cite{fisher1970statistical}. Finally, the participants were required to select the round with the most comfortable scent concentration.
\begin{figure}
    \centering
    \includegraphics[width=1\linewidth]{fig/appendix-1.pdf}
    \caption{The timeline of the pre-study procedure. Participants filled out a pre-questionnaire. Then, they drove autonomously for 5min to get used to the simulator. Next, they drove autonomously and recorded the time when they were induced to driving fatigue. Then, 3 rounds of scent release experiments were conducted. The round order was randomized. Finally, participants filled out a questionnaire to choose the most comfortable round.}
    \label{fig:app-1}
\end{figure}

\subsection{Results}
\paragraph{Participants}7 participants, 23-25 years old (M = 24.1, SD = 1.21, 3 females), volunteered for this study.

\paragraph{The proper duration of the driving fatigue induction experiment}Since the data of one participant was marked as "failure to induce fatigue", we only calculated the mean and variance of six groups of data. The calculation shows that the average time for participants to reach a state of general fatigue is 3.5min (SD = 1.87), and the average time to reach a state of extreme fatigue is 7.5min (SD = 1.38). In order to ensure that the participants are induced to experience driving fatigue, we selected 10min as the duration of the driving fatigue induction experiment in Study 2.

\paragraph{The proper concentration of the scent release experiment}Among the seven experiment participants, two chose the lowest concentration because they were allergic to the smell of traditional Chinese medicine, and a high concentration might irritate them and cause sneezing. The remaining five participants chose the highest concentration because of the short duration of scent effectiveness. Therefore, in the formal experiment, we screened out the participants who were allergic to the TCM scents and selected the highest concentration as the experimental concentration for Study 2. Due to the limitations of the equipment, we were unable to explore the impact of higher concentrations on the participants. This may be further studied in future experiments. 

\end{document}
\endinput
%%
%% End of file `sample-acmlarge.tex'.
