\section{Related Work}
We structured the related work around two research areas:


\subsection{How to mitigate driving fatigue?}
Based on the classification of sensory stimuli, methods aimed at mitigating driving fatigue can be categorized into visual, auditory, and olfactory approaches. With regard to visual methods, adjustments to ambient lighting have been employed to enhance driver alertness ____. For auditory methods, music____ or name music game____ have proven effective . In driving scenarios, vision and hearing serve as the primary channels for information reception. While modulation through these two sensory modalities may lead to driver distraction, olfaction is less utilized during driving and exhibits a closer relationship with emotional states____. Consequently, this study employs olfactory modulation as a means to address driving fatigue.

Contemporary research has substantiated that scents such as mint, grapefruit, lemon, and lavender can markedly ameliorate driving fatigue ____. However, previous studies have predominantly focused on common flower and fruit scents that are widely utilized in Western countries. Given the disparities in culture and physiological traits, these scents are not frequently utilized among the Chinese populace ____. Consequently, there is a pressing need to explore scents that are more familiar and acceptable to Chinese individuals.

\subsection{Why we choose TCM?}
The Chinese populace has maintained a long-standing tradition of utilizing Traditional Chinese Medicine (TCM) to regulate emotions since ancient times ____. In the framework of TCM, a bi-directional causal relationship exists between the harmonious flow of \textit{qi}\footnote[1]{Chinese identify \textit{qi} as the natural energy intrinsic to all things that exist in the universe, as the fundamental life energy responsible for health and vitality____.} and both a healthy mental state and emotional equilibrium ____. Specific TCM herbs, including Chinese angelica, acorus tatarinowii, spine date seed, and tangerine peel, have demonstrated efficacy in treating depression ____.
However, TCM is predominantly administered for internal consumption. Relatively few TCM formulations have been processed into high-purity essential oils for aromatherapy purposes, and the precise effects of TCM scents on emotions remain unclear. Therefore, this study selects TCM essential oils with potential relevance to emotion regulation for investigation.