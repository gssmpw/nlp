\section{Related Work}
\paragraph{Discrete Diffusion Models}
Discrete diffusion directly models the Markov chain on discrete data space. The one-hot data distribution is gradually corrupted to a stationary distribution with specific transition matrices, where the noising process corresponds to the stochastic jumps between states in the Markov chain.
D3PM~\citep{austin2021d3pm} introduces discrete-time Markov forward processes with uniform and absorbing state transition matrices and has been generalized to continuous-time Markov chain framework~\citep{campbell2022ctmc}.
SEDD~\citep{lou2024sedd} proposes learning the score entropy of the discrete states instead of the mean prediction. 
Recent works~\citep{shi2024md4,sahoo2024simple} introduce continuous-time masked diffusion models with a simpler form of likelihood bounds.

% While directly modeling the discrete states shows promising results comparable to autoregressive language models, existing techniques for improving continuous diffusion such as self-conditioning~\citep{chen2023self} are not directly applicable.
% In particular, guidance methods, such as classifier-based and classifier-free guidance, cannot be applied to discrete diffusion models.



\paragraph{Continuous Diffusion Models for Discrete Data}
Early works approached by fully relaxing the discrete data into continuous space~\citep{han2022ssd} or embedding the tokens in a latent space~\citep{li2022diffusion,dieleman2022continuous}, without any constraint.
However, continuous relaxation without constraint fails to accurately model the discreteness of the categorical distribution.
Recent works utilize the logit space~\citep{hoogeboom2021multinomial,graves2023bayesian} or the probability simplex~\citep{avdeyev2023dirichlet,stark2024dirichlet} based on the Dirichlet distribution, which require strong assumptions on the diffusion noising processes.
Flow matching has been applied to the probability simplex by using the statistical manifold on categorical distribution~\citep{cheng2024categorical,davis2024fisherflow} but has limited performance lagging behind discrete diffusion models.


% In our work, we present a continuous diffusion model that accounts for the intrinsic geometry structure of the categorical distributions, which outperforms existing discrete diffusion models in diverse tasks.