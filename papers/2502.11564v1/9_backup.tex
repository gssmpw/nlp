\onecolumn

\begin{center}{\bf {\LARGE Backup}}\end{center}

\paragraph{Mixture Path of Masked and Uniform Bridges}
For $\lambda\in[0,1]$, we can construct a family of bridge processes with endpoint $\bm{e}_k$ starting from
\begin{align}
    \pi\left( \sum^{d}_{i=1} \frac{\lambda}{d}\bm{e}_i + (1-\lambda) \bm{e}_m \right)
\end{align}




\section{Uniform}


\begin{tcolorbox}
[colback=white,colframe=blue!30!white]
\begin{lemma}
\label{lem:uniform_expectation}
Let $p^{sym}$ be an arbitrary distribution on $\mathbb{S}^{d-1}_{+}$ that is symmetric w.r.t. the coordinates, for example, a uniform distribution on $\mathbb{S}^{d-1}_{+}$, i.e., $U(\mathbb{S}^{d-1}_{+})$, or a diffeomorphic mapping of a uniform distribution on $\Delta^{d-1}$, i.e., $\pi(U(\Delta^{d-1}))$.
Then for a random variable $\bm{v}\sim p^{sym}$, the following holds:
\begin{align}
    \mathbb{E}\langle \bm{e}_i,\bm{v}\rangle^2 = \frac{1}{d}, \; i\in\{1,\cdots,d\}
\end{align}
\end{lemma}
\end{tcolorbox}

\begin{proof}
Due to the symmetry of $p^{sym}$, 
\begin{align}
    \mathbb{E}\langle\bm{e}_1,v\rangle^2 = \cdots = \mathbb{E}\langle\bm{e}_d,v\rangle^2.
\end{align}
Using the fact that $\bm{v}\in\mathbb{S}^{d-1}$, we get the desired result:
\begin{align}
    \mathbb{E}\langle\bm{e}_i,v\rangle^2 
    = \frac{1}{d}\sum^{d}_{n=1} \mathbb{E}\langle\bm{e}_n,v\rangle^2 
    = \frac{1}{d}\mathbb{E}\left[\sum^{d}_{n=1}\langle\bm{e}_n,v\rangle^2\right]
    = \frac{1}{d}.
\end{align}
\end{proof}



\begin{tcolorbox}
[colback=white,colframe=blue!30!white]
\begin{proposition}
\label{prop:uniform_flow}
For any differentiable noise schedule $\alpha_t\!\in\![0,1]$ satisfying $\alpha_0\!\approx\!1$ and $\alpha_1\!\approx\!0$, there exists a scheduler $\kappa_t$ such that a flow $\bm{Y}_t$ defined on $\mathbb{S}^{d-1}_{+}$ in the time horizon $[0,T]$:
\begin{align}
    \frac{\mathrm{d}\bm{Y}_t}{\mathrm{d}t} 
    = -\frac{\mathrm{d}\log\kappa_t}{\mathrm{d}t} \exp^{-1}_{\bm{Y}_t}(\bm{u}),  \;\; \bm{Y}_0=\bm{e}_k, \;\; \bm{u}\sim p^{sym},
\label{eq:uniform_flow}
\end{align}
yields a random variable $\bm{Z}_t\coloneqq\bm{Y}_t\odot\bm{Y}_t\in\mathbb{R}^{d}$ that satisfy the following:
\begin{align}
    \mathbb{E}\bm{Z}_t = \sum_{j\neq k}\frac{1-\alpha_t}{d}\bm{e}_j + \frac{1 + (d-1)\alpha_t}{d}\bm{e}_k .
\label{eq:uniform_simplex}
\end{align}
\end{proposition}
\end{tcolorbox}

\begin{proof}
Let $\theta_0=\cos^{-1}\langle \bm{e}_k, \bm{u} \rangle$ and define a function $F:[0,1]\rightarrow\mathbb{R}$ as follows:
\begin{align}
    F(x) = \mathbb{E}_{\bm{u}\sim p^{sym}}\left[ \cos^2\big( (1-x)\theta_0 \big) \right].
\end{align}
Since $0\leq\theta_0=\cos^{-1}(\langle \bm{e}_k, \bm{u} \rangle)\leq \pi/2$ for all $\bm{u}\sim p^{sym}$, we can see that $F$ is an increasing function:
\begin{align}
    \frac{\mathrm{d}F(x)}{\mathrm{d}x} = \mathbb{E}_{\bm{u}}\left[ \theta_0 \sin\big( 2(1-x)\theta_0 \big) \right] \geq 0 ,
\end{align}
where the equality holds if and only if $x=1$.
Thus $F$ is a strictly increasing differentiable function with $1/d \leq F(x) \leq 1$, since $F(0)=1/d$ from the results of Lemma~\eqref{lem:uniform_expectation} and $F(1)=1$.
Therefore, there exists an differentiable inverse $F^{-1}:[1/d,1]\rightarrow[0,1]$ . 

Now define a scheduler $\kappa_t\coloneqq F^{-1}((1 + (d-1)\alpha_t)/d)$, which is well-defined due to $\alpha_t\in[0,1]$.
Using Lemma~\ref{prop:flow_solution}, we have the following representation of $\bm{Y}_t$ given $\bm{Y}_1=\bm{u}$:
\begin{align}
    \bm{Y}_t 
    =\frac{\sin(\theta_0-\theta_t)}{\sin\theta_0}\bm{u} + \frac{\sin\theta_t}{\sin\theta_0}\bm{e}_k, \;\; \theta_t = \kappa_t \cos^{-1}\langle \bm{e}_k, \bm{u} \rangle,
\label{eq:uniform_flow_as_theta}
\end{align}
which yields the following result:
\begin{align}
    \mathbb{E}\bm{Z}_t 
    &= \mathbb{E}_{\bm{u}} \left[ \bm{Y}_t\odot\bm{Y}_t | \bm{Y}_1=\bm{u}\right] \\
    &= \sum_{j\neq k}\mathbb{E}_{\bm{u}}\left[ \frac{\sin^2(\theta_0 - \theta_t)}{\sin^2\theta_0}\langle \bm{u},\bm{e}_j \rangle^2 \right] \bm{e}_j 
    + \mathbb{E}_{\bm{u}}\left[\frac{(\sin\theta_t + \sin(\theta_0-\theta_t)\cos\theta_0)^2}{\sin\theta_0}\right]\bm{e}_k \\
    &= \sum_{j\neq k}\mathbb{E}_{\bm{u}}\left[ \frac{\sin^2\big((1-\kappa_t)\theta_0\big)}{\sin^2\theta_0}\langle \bm{u},\bm{e}_j \rangle^2 \right] \bm{e}_j 
    + \mathbb{E}_{\bm{u}}\Big[ \cos^2\big((1-\kappa_t)\theta_0 \big) \Big] \bm{e}_k
\end{align}
where the last equality is due to the identity:
\begin{align}
    \sin x + \sin y\cos(x+y) = \sin(x+y)\cos y.
\end{align}
From the definition of scheduler $\kappa_t$, 
\begin{align}
    \mathbb{E}_{\bm{u}}\Big[ \cos^2\big((1-\kappa_t)\theta_0 \big) \Big] 
    = \frac{1+(d-1)\alpha_t}{d}.
\end{align}
Furthermore using the fact that $\mathbb{E}\bm{Z}_t$ lives on the probability simplex $\Delta^{d-1}$ and the symmetry of $\langle \bm{u}, \bm{e}_j\rangle$ for all $j\neq k$ gives the desired result:
\begin{align}
    \mathbb{E}_{\bm{u}}\left[ \frac{\sin^2\big((1-\kappa_t)\theta_0\big)}{\sin^2\theta_0}\langle \bm{u},\bm{e}_j \rangle^2 \right] 
    = \frac{1}{d-1} \left( 1 - \mathbb{E}_{\bm{u}}\Big[ \cos^2\big((1-\kappa_t)\theta_0 \big) \Big] \right)
    = \frac{1-\alpha_t}{d} .
\end{align}
\end{proof}



\section{Coordinate Processes}

\paragraph{Transformation to Coordinate Processes}
For the Logarithm bridge process $\mathbb{Q}^{\bm{u},\bm{v}}$ on $\mathbb{S}^d$, we can derive the 1-dimensional process $c^{\bm{e}}_t = \langle\bm{X}_t, \bm{e}\rangle\in\mathbb{R}$ using It\^{o}'s formula:
\begin{align}
    \mathrm{d}c^{\bm{e}}_t = \left[ 
        \frac{\sigma_t^2}{\tau_T - \tau_t} \frac{\cos^{-1}c^{\bm{v}}_t}{\sqrt{1 - (c^{\bm{v}}_t)^2}} \bigg( \langle\bm{e}, \bm{v}\rangle - c^{\bm{e}}_t c^{\bm{v}}_t \bigg) 
        -\frac{d\sigma_t^2}{2}c^{\bm{e}}_t 
    \right]\mathrm{d}t + \sigma_t\sqrt{1 - (c^{\bm{e}}_t)^2}\mathrm{d}W_t.
\label{eq:coord_process}
\end{align}
Note that $-\frac{d\sigma_t^2}{2}c^{\bm{e}}_t$ term in the drift corresponds to the Laplacian of the inner product.

Therefore, we can transform the Logarithm bridge process $\mathbb{Q}^{\bm{e}_m,\bm{e}_k}$ into a system of coordinate processes as follows:
\begin{align}
\begin{cases}
    \mathrm{d}c^l_t = \left[ \frac{\sigma_t^2}{\tau_T - \tau_t}
        \frac{\cos^{-1}c^{k}_t}{\sqrt{1 - (c^{k}_t)^2}} \bigg( \delta_{l,k} - c^{l}_t c^{k}_t \bigg) -\frac{d\sigma_t^2}{2}c^{l}_t 
    \right]\mathrm{d}t + \sigma_t\sqrt{1 - (c^{l}_t)^2}\mathrm{d}W^l_t.
\end{cases}
\end{align}
We can rewrite the system of SDEs as follows:
\begin{align}
    \mathrm{d}\bm{C}_t = \Bigg[ 
        \frac{\sigma_t^2}{\tau_T - \tau_t} \underbrace{\frac{\cos^{-1}c^k_t}{\sqrt{1 - (c^k_t)^2}} \Big( \bm{e}_k - c^k_t\bm{C}_t \Big)}_{\eta^{k}(\bm{C}_t,t)} - \frac{d\sigma_t^2}{2}\bm{C}_t
    \Bigg]\mathrm{d}t + \sigma_t\sqrt{1 - \bm{C}_t^2}\mathrm{d}\bm{W}_t,
\end{align}
where $c^k_t\coloneqq \langle \bm{C}_t,\bm{e}_k \rangle$ and $\bm{W}_t$ denotes $d$-dimensional Euclidean Brownian motion (i.e., standard Weiner process).

\paragraph{Mixture of Coordinate Processes}
The Logarithm bridge mixture $\bm{X}_t$ is defined by the following SDE:
\begin{align}
    \mathrm{d}\bm{X}_t = 
   \frac{\sigma_t^2}{\tau_T-\tau_t} \Bigg[ 
    \sum^d_{k=1} p_k(\bm{X}_t) \exp^{-1}_{\!\bm{X}_t}\!(\bm{e}_k)
    \Bigg]
    \mathrm{d}t + \sigma_t\mathrm{d}\mathbf{B}^{\mathcal{M}}_t, \;\; \bm{X}_0=\bm{e}_{d+1} ,
\end{align}
where $p_k(\bm{X}_t)\coloneqq p(\bm{X}_1=\bm{e}_k|\bm{X}_t)$ and $\mathbf{B}^{\mathcal{M}}_t$ denotes the Brownian motion defined on $\mathbb{S}^d$.
Then the coordinate processes for the mixture process $c^k_t\coloneqq \langle\bm{X}_t, \bm{e}_k \rangle$ for $k=1,\cdots,d$ can be derived as follows:
\begin{align}
    \mathrm{d}c^{k}_t = \Bigg[ 
        \frac{\sigma_t^2}{\tau_T - \tau_t} 
        \underbrace{\sum_l p_l(c^{1:d}_t) \frac{\cos^{-1}c^{l}_t}{\sqrt{1 - (c^{l}_t)^2}} \bigg( \delta_{l,k} - c^{l}_t c^{k}_t \bigg)}_{\eta(\bm{C}_t, t)}
        -\frac{d\sigma_t^2}{2}c^{k}_t
    \Bigg] \mathrm{d}t 
    + \sigma_t\sqrt{1 - (c^{k}_t)^2}\mathrm{d}W^k_t,
\end{align}
with initial conditions $c^k_0=0$, where $p_l(c^{1:d}_t) \coloneqq p(c^{1:d}_1=\bm{e}^{d}_l|c^{1:d}_t)$ and $W^k_t$ is a 1-dimensional standard Wiener process.

From Eq.~\eqref{eq:coord_process}, we can derive an objective as follows:
\begin{align}
    \mathcal{L}^{coo}(\theta) 
    = \mathbb{E}_{\substack{\bm{e}_k\sim p_{data} \\ t\sim [0,1]}}\mathbb{E}_{\bm{C}_t\sim\mathbb{Q}^{coo}}\left\| \sigma_t^{-1}\sqrt{1-\bm{C}^2_t}
    \frac{\sigma_t^2}{\tau_T-\tau_t}\bigg( 
        \eta(\bm{C}_t,t) - \eta^k(\bm{C}_t,t)
    \bigg) \right\|^2
\end{align}
which is equivalent to the objective in Eq.~\eqref{eq:mixture_objective}.


Since modeling $\bm{C}_t\coloneqq c^{1:d}_t$ is equivalent to modeling $\bm{X}_t$, i.e., $\bm{X}_t = \left(c^1_t,\cdots,c^d_t,\sqrt{1 - \sum (c^i_t)^2}\right)$, the objective of learning the transition probability is the same.
\begin{align}
    p(\bm{X}_1=\bm{e}_k|\bm{X}_t) = p(\bm{C}_1=\bm{e}^d_k|\bm{C}_t)
\end{align}


\section{Simulation-Free Training}
Before the training stage, we first simulate $M$ number of bridge processes $\mathbb{Q}^{\bm{u},\bm{v}}$ in parallel, using Euler-Maruyama solver with fixed time steps $t_1,\cdots,t_N$. During the simulation, we save the real numbers $\alpha_{t_k}$ and $\rho_{t_k}$ for $i=1,\cdots,N$ that satisfies the following:
\begin{align}
    \bm{\mu}_{t_k} &= \alpha_{t_k}\bm{v} + \sqrt{1 - \alpha_{t_k}^2}\bm{u}, \;\; \bm{\mu}_t \coloneqq \frac{\mathbb{E}\bm{X}_{t}}{\|\mathbb{E}\bm{X}_{t}\|}, \\
    \rho_{t_k}^2 &= \frac{1}{M-1} \sum^{M}_{i=1} \left[ \exp^{-1}_{\bm{\mu}_{t_k}}(\bm{X}^{(i)}_{t_k}) -  \mathbb{E}_j \exp^{-1}_{\bm{\mu}_{t_k}}(\bm{X}^{(j)}_{t_k})\right]^2 \notag
\end{align}
where $\bm{\mu}_{t_k}$ denotes the normalized centroid of the simulated $M$ samples, and $\rho_{t_k}^2$ denotes the variance of the lifts of samples to the tangent space at $\bm{\mu}_{t_k}$.

With $\{\alpha_{t_i}\}$ and $\{\rho_{t_i}\}$, we can approximate the transition distribution $p(\bm{X}_t|\bm{X}_0,\bm{X}_1)$ of the bridge process as follows:
\begin{align}
    &\tilde{\bm{X}}_t = \exp_{\tilde{\bm{\mu}}_t}(\tilde{\rho}_t \bm{z}), \;\;
    \tilde{\bm{\mu}}_t \coloneqq \tilde{\alpha}_t \bm{v} + \sqrt{1 - \tilde{\alpha}_t^2} \bm{u}, \;\;
    \bm{z}\sim \mathcal{N}_{T_{\tilde{\bm{\mu}}_t}\mathbb{S}^d}(\mathbf{0}, \mathbf{I}) \\ 
    &\lambda \coloneqq \frac{t - t_i}{t_{i+1} - t}, \;\;
    \tilde{\alpha}_t \coloneqq (1-\lambda) \alpha_{t_i} + \lambda \alpha_{t_{i+1}}, \;\;
    \tilde{\rho}_t = (1-\lambda) \rho_{t_i} + \lambda \rho_{t_{i+1}},
\end{align}