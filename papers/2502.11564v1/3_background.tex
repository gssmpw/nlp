\section{Background}


\subsection{Discrete diffusion models}
Discrete diffusion models~\citep{austin2021d3pm,lou2024sedd,sahoo2024simple,shi2024md4} define the diffusion process directly on discrete structures using the Markov chains. The forward process describes the transition from the current state to other states, which is formalized by multiplying
the transition matrix $Q_t$:
\begin{align}
    q(x_t|x_{t-1}) = \text{Cat}(x_t; {Q}_t x_{t-1}),
\end{align}
where $x_t$ is the random variable for the discrete states and $\text{Cat}(\cdot)$ denotes the categorical distribution. This induces the marginal distribution that corresponds to repeatedly multiplying the transition matrices over time steps:
\begin{align}
    q(x_t|x)  = \text{Cat}(x_t; \bar{Q}_t x) = \text{Cat}(x_t;Q_t\cdots Q_1 x).
\label{eq:discrete_transition}
\end{align}
\citet{austin2021d3pm} introduced several designs of the transition matrices, including the masked (absorbing state) diffusion and the uniform diffusion, and the continuous-time Markov chains (CTMC)~\citep{austin2021d3pm,campbell2022ctmc} extends the framework to continuous-time.

% $\bm{e}_k\in\mathbb{R}^{n}$ denote a unit vector with $k$-th entry 1 and others 0


\subsection{Statistical Manifold of Categorical Distribution}
Let $\mathcal{X}=\{1,\cdots,d\}$ denote the discrete data space and $\Delta^{d-1}=\{(p_1,\cdots, p_d)\in\mathbb{R}^d|\sum_i p_i=1, p_i\geq0\}$ denote the $(d-1)$-dimensional probability simplex.
A $d$-class categorical distribution over $\mathcal{X}$ can be parameterized by the parameters $p_1,\cdots,p_d$ such that $\{\sum_i p_i=1, p_i\geq0\}$. 
Then the statistical manifold $\mathcal{P}(\mathcal{X})$ of the categorical distribution corresponds to $\Delta^{d-1}$ equipped with the Fisher-Rao metric~\citep{rao1992information,amari2016information} (see Appendix~\ref{app:derivation:prelim}).
Moreover, there exists a diffeomorphism from $\mathcal{P}(\mathcal{X})$ to the positive orthant of a $(d-1)$-dimensional sphere $\mathbb{S}^{d-1}_{+}$:
\begin{align}
\begin{split}
    \pi: \mathcal{P}(\mathcal{X}) \rightarrow \mathbb{S}^{d-1}_{+} ; p_i\mapsto u_i=\sqrt{p_i},
    % &\pi^{-1}: \mathbb{S}^{d-1}_{+} \rightarrow \mathcal{P} ; x_i\mapsto p_i=x_i^2,
\end{split}
\label{eq:diffeomorphism}
\end{align}
which induces the following geodesic distance on $\mathbb{S}^{d-1}_{+}$:
\begin{align}
    d_g(\bm{u},\bm{v}) = \cos^{-1}\langle\bm{u}, \bm{v}\rangle,
\end{align}
where $\langle\cdot\rangle$ denotes the Euclidean inner product.
We provide further explanation in Appendix~\ref{app:derivation:prelim}.


\subsection{Riemannian Diffusion Mixture}
Riemannian diffusion mixture framework~\citep{jo2024riemannian} provides a simple approach to generative modeling on general manifolds.
The construction of the generative model starts with defining a bridge process $\mathbb{Q}^{\bm{z}}$ on the manifold $\mathcal{M}$ with endpoint $\bm{z}$: $\mathrm{d}\bm{X}^{\bm{z}}_t = \eta^{\bm{z}}(\bm{X}^{\bm{z}}_t,t)\mathrm{d}t + \sigma_t\mathrm{d}B^{\mathcal{M}}_t$ where $B^{\mathcal{M}}_t$ is the Brownian motion defined on $\mathcal{M}$.
The diffusion process transporting an initial distribution to the data distribution is modeled as a mixture of bridge processes: 
\begin{align}
\hspace{-1mm}
    \mathrm{d}\bm{X}_t \!=\! \left[ 
        \int \eta^{\bm{z}}\!(\bm{X}_t,t) \frac{p^{\bm{z}}_t(\bm{X}_t)}{p_t(\bm{X}_t)} p^{\ast}\!(\mathrm{d}\text{vol}_{\bm{z}}) 
    \right] \!\mathrm{d}t + \sigma_t \mathrm{d}B^{\mathcal{M}}_t
\end{align}
where $p^{\ast}$ denotes the data distribution, $p^{\bm{z}}_t$ is the marginal distribution of the bridge $\mathbb{Q}^{\bm{z}}$, and $p_t(\cdot)\coloneqq \int p^{\bm{z}}_t(\cdot) p^{\ast}\!(\mathrm{d}\text{vol}_{\bm{z}})$.
The drift of this process is regressed by a neural network $\eta^{\theta}$ with the bridge matching objective:
\begin{align}
    \mathbb{E}_{\substack{\bm{z}\sim  p^{\ast} \\ \bm{X}\sim\mathbb{Q}^{\bm{z}}}}\! \left[
        \int^T_0 \Big\|
            \sigma^{-1}_t \Big(\eta^{\theta}(\bm{X}_t,t) - \eta^{\bm{z}} (\bm{X}_t,t) \Big)
        \Big\|^2_{\mathcal{M}}\mathrm{d}t
    \right]
\end{align}
We provide further details in Appendix~\ref{app:derivation:mixture}.