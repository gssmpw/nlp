\section{What Makes Object/Subject Garden-Path Sentences Hard?}

Object/subject garden-path sentences, like (\ref{num:ex_gp}), include an embedded verb (``wash'') followed by a noun phrase (``the dog'') and the main verb (``barked''). These sentences cause processing difficulties, leading to slower reading—especially at the main verb—and reduced comprehension. In sentence (\ref{num:ex_gp}), \emph{``the dog''} will often be mistakenly interpreted as the object of \emph{``wash''}, prompting incorrect answers like \emph{``Yes''} to \emph{``Did the child wash the dog?''}. This happens even though in the final structure of the sentence, ``the dog'' is not an object of ``washed''. 

%Object/subject garden-path sentences, like example (\ref{num:ex_gp}), feature an initial clause with an optionally transitive verb (i.e., a verb that can appear with or without an object), followed by a noun phrase, and the verb of the main sentence. Such sentences lead to processing difficulties that result in slower reading, particularly on the main verb, and impaired comprehension. For example, in sentence (\ref{num:ex_gp}), \emph{``the dog''} will often be misinterpreted as the object of \emph{``wash''} even after the entire sentence has been read, leading people to answer \emph{``Yes''} to the question \emph{``Did the child wash the dog?''}.

\paragraph{}
Several (non-mutually exclusive) hypotheses can explain the misinterpretation described above, as summarized in Table~\ref{tab:changes}.
To describe the hypotheses, we consider the following sentence-question pair:
\begin{enumerate}[nosep,leftmargin=*]
    \setcounter{enumi}{2}
    \item \label{num:ex_gp_3} While the man hunted the deer ran into the woods.
\end{enumerate}
\begin{itemize}[nosep,leftmargin=*]
    \item[] \textbf{Question}: Did the man hunt the deer?   Y/N
\end{itemize}

Note that the accurate answer to the question above is ``Not necessarily''. It is a possible interpretation of the sentence, and may be inferred from it, but it is not entailed from the sentence. In our experiments, as in previous experiments, we consider "yes" to be a wrong answer here, whereas "no" is considered the right answer.  

\textbf{Hypothesis 1: The GP syntax is hard.}
    This hypothesis suggests that misinterpretation occurs because during incremental processing, the post-verbal noun phrase \emph{(``the deer'')} is first attached as the object of the verb, requiring reanalysis when the second verb is encountered. Often, the reanalysis is not complete, and the initial interpretation lingers. According to this, reordering the clauses (see Table~\ref{tab:changes}) should improve accuracy by preventing initial misattachment. 


 \textbf{Hypothesis 2: Readers attach the noun to the first verb when it is a plausible object for it. } 
    According to this hypothesis, readers interpret a noun as an object of a verb in the sentence whenever this is semantically plausible, regardless of sentence position. If the noun is an implausible direct object, it will not be interpreted as such, improving accuracy (see Table~\ref{tab:changes}).\textbf{
    Hypothesis 3: Readers search maximal interpretation of verb arguments.} 
    According to this hypothesis, optionally transitive verbs need objects for full interpretation, so available nouns are taken to fulfill this role. In contrast, alternating reflexive (``wash'') and unaccusative verbs (``drop'') allow complete interpretation intransitively, without an object (e.g., \emph{The boy washed}, \textit{The ball dropped}), Thus, such verb will allow more accurate interpretation than optionally transitive verbs (see Table~\ref{tab:changes}).
    Note that for sentences such as "While the boy washed the dog barked", the correct answer for ``Did the boy wash the dog?'' is ``No''. 
    As for the optionally transitive verbs, it can also be hypothesized that the tendency to interpret them as taking an object depends on the verb's transitivity bias (the probability that the verb appears with a direct object). According to this, verbs with a lower bias (e.g., ``walk'') should lead to better accuracy compared to those with a higher bias (e.g., ``explore''), as the noun is less likely to be considered as their object.


Next, we describe how we test the above hypotheses, starting with a human experiment.