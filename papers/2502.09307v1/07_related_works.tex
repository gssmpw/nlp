\section{Related Work}

%\jb{Maybe this can have structure (a) past work on garden paths in humans (b) past work on humans vs. llms in general (c) past work on humans vs. LLMs on gardent paths?}

%Research comparing humans and LLMs language processing mechanism has been increasingly interesting with the recent progress in NLP.

%There has been two main strands of research in this area. The first investigates the similarity between the processes underlying human cognition and LLM processes by comparing the brain's to the LLM's internal activation. Some recent research has shown that the way \emph{LLMs} process sentences correlates with the way the brain processes these sentences. For example, \cite{fedorenko_brain_corr} showed that the most recent neural models' internal computations while processing a sentence can effectively predict brain activations (collected using fMRI and EEG) while reading the same sentence. Similarly, \cite{cacheteux-middle-layer} found that most LLM layers show some correlation with brain activation, with the middle layer showing the largest correlation. \citet{Ren2024DoLL} studied 23 different SOTA LLMs and found they all show some similarity with human brain activations, with the similarity to human activations being positively correlated with the LLM size, the pretraining data size and the alignment to humans.

%Other works compared LLMs with human behavior, rather than brain activity. For example, \citet{Rego2024LanguageMO} showed that LLMs outperform cloze predictability in modeling human sentence processing metrics like reading time and eye-gaze. \citet{Sun2024ComputationalSM} developed sentence-level metrics with multilingual LLMs and found they can predict reading time across a variety of languages. \citet{Li2024IncrementalCO} took 24 sentences from \cite{christianson2001} and measured reading comprehension on these sentences, showing that LLMs as well make the same misinterpretation as humans.

%\jb{use the paragraph environment to give this more structure.}

\paragraph{Garden path sentences} Garden path sentences hold a significant place in psycholinguistics, as they sparked the initial interest in processing difficulty. Early psycholinguistic research demonstrated that these sentences are read more slowly, specifically at the word or region where reanalysis is required \cite{gp3, gp2, gp1}. \citet{christianson2001} found that  comprehension is poorer for GP sentences of the type exemplified in (\ref{num:ex_gp}) compared to non-GP sentences as in (\ref{num:ex_ngp}). Specifically, the initial interpretation whereby the post-verbal NP is analyzed as the object of the first verb lingers, even while that same NP is also analyzed as the subject of the following verb (see also \citet{christianson2006}. Additionally, \citet{patson2009lingering} revealed that when participants were asked to paraphrase GP sentences, they often failed to perform full reanalysis and paraphrased the sentences inaccurately, e.g. ``The boy washed the dog and the dog barked'' for (\ref{num:ex_gp}). 

\paragraph{Comparing LLMs and humans} Research comparing LLM and human language processing mechanisms can be divided into two primary categories. The first category analyzes the correlation between brain and LLM activations while processing the same sentences, demonstrating that LLM activations significantly correlate with brain activities \cite{cacheteux-middle-layer, fedorenko_brain_corr, Ren2024DoLL}. The second category examines parallels in behavior, either by employing LLMs to predict human psycholinguistic metrics such as reading times \cite{Rego2024LanguageMO, Sun2024ComputationalSM} or learnability \cite{island_learnability}, or by highlighting that certain syntactic structures are challenging for both LLMs and humans \cite{neural-lms-subjects, wilcox-gpt2-abilities}.

\paragraph{LLMs and Garden Path sentences} Studies comparing LLMs and humans on garden path sentences fall into the second category described above. \citet{gp_reading_time} attempted to predict human reading times using LLM metrics and showed that LLM-based metrics vastly underestimate human reading times in various GP sentences. \citet{bert_gp} demonstrated that BERT representations are influenced by the GP effect. Finally, \citet{Li2024IncrementalCO} utilized object/subject GP sentences from \cite{christianson2001} to assess the sentence comprehension abilities of four different LLMs, revealing that these LLMs face similar difficulties with GP sentences to humans.