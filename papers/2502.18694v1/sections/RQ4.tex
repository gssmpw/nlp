\subsection{RQ4: Evaluation Methods of REDAST Studies}
REDAST studies aim to generate reliable and usable test artifacts, from requirements to test artifacts. However, how to verify the efficacy of the generated test artifacts still remains in the mist. In this section, we plan to illustrate the related results of the selected studies to systematically discuss the evaluation methodologies for test artifacts. The results in this section consist of three parts: (1) types of evaluation methods, (2) target software platforms in the case studies, and (3) the usability of the studies.

\subsubsection{Types of Evaluation Methods}
\begin{table}[]
\small
\caption{Benchmark Type Results of Selected Studies (RQ4)}
\label{table:benchmark}
\begin{tabularx}{\textwidth}{lXc}
\hline
\textbf{Benchmark Types} & \textbf{Paper IDs} & \textbf{Num.} \\ \hline

Conceptual Case Demonstration & 
\citeP{P1, P2, P4, P6, P7, P8, P10, P11, P12, P13, P14, P16, P17, P21, P22, P26, P27, P35, P43, P46, P48, P49, P51, P53, P62, P64, P66, P67, P68, P69, P71, P74, P75, P77, P82, P84, P86, P89, P90, P92, P98, P99, P100, P105, P106, P111, P112, P114, P115, P119, P123, P124, P128, P132, P133, P134, P136, P141, P144, P145, P149, P150, P151, P152, P153, P155, P156, P159, P161} 
& \cellcolor{gray!65}69 \\

Real Case Demonstration & 
\citeP{P3, P5, P9, P12, P15, P18, P20, P23, P24, P25, P28, P29, P30, P32, P33, P34, P36, P37, P38, P42, P44, P45, P49, P50, P54, P55, P58, P59, P60, P61, P72, P78, P79, P80, P83, P88, P91, P94, P95, P101, P102, P107, P108, P109, P110, P113, P117, P118, P121, P123, P125, P126, P127, P129, P130, P131, P132, P134, P135, P137, P138, P140, P142, P147, P154, P160} 
& \cellcolor{gray!60}66 \\

NA & 
\citeP{P19, P31, P39, P40, P47, P52, P57, P63, P65, P70, P73, P76, P93, P96, P103, P104, P116, P122, P139, P143, P146, P148, P158} 
& \cellcolor{gray!30}23 \\

Dataset Evaluation & 
\citeP{P18, P41, P56, P157} 
& \cellcolor{gray!15}4 \\

\hline
\end{tabularx}
\end{table}
In the selected studies, we identified two primary evaluation methods: (1) case studies and (2) evaluations using given datasets. However, some studies rely on conceptually designed case studies rather than industrial demonstrations based on real-world cases. To provide a clearer classification, we restructured the categories for evaluation methods into three types: conceptual case study, industrial case study, and evaluation using given datasets. The summarized results are presented in Table \ref{table:benchmark}.
\begin{itemize}
    \item \textbf{Conceptual Case Demonstration}. These studies typically begin by designing a conceptual case and then applying REDAST methods to the designed case to demonstrate the test generation process. Commonly used conceptual cases include online systems (e.g., P11~\citeP{P11}, P21~\citeP{P21}, P119~\citeP{P124}), ATMs (e.g., P12~\citeP{P12}, P13~\citeP{P13}, P35~\citeP{P35}), and library systems (e.g., P6~\citeP{P6}), among others. While these demonstrations effectively illustrate the methodology procedure, they often lack compulsion and persuasiveness in demonstrating the efficacy of the results. However, the use of conceptual cases, which are based on widely familiar scenarios, enhances understandability. This familiarity benefits readers by making the methodologies easier to comprehend and follow.
    \item \textbf{Industrial Case Demonstration}. Studies in this category demonstrate their methods by incorporating industrial cases into their experiments. By re-organizing and utilizing data extracted from industrial scenarios, REDAST methods are validated under real-world conditions, offering stronger evidence and greater persuasiveness compared to conceptual case studies. Additionally, we observed that some studies intersect across categories, combining (i) both conceptual and industrial case studies, and (ii) industrial case studies with dataset evaluations (e.g., P28~\citeP{P28} on several industrial cases from public paper, P155~\citeP{P160} on postal systems). These overlaps occur because industrial cases not only serve as a basis for case studies but can also be used to formulate evaluation datasets, further enhancing their utility in validating methodologies.
    \item \textbf{Evaluation on Given Datasets}. Evaluations conducted on public or industrial datasets provide a compelling approach to demonstrating the efficacy and usability of a method. However, transitioning from discrete cases to datasets requires significant additional effort, including data cleaning, reorganization, and formulation. This challenge arises due to two primary factors: (1) the high usability requirements of REDAST studies, which are often difficult to demonstrate effectively using certain datasets, and (2) the absence of a standardized benchmark dataset for evaluating test artifacts. As a result, most studies rely on case demonstrations rather than systematic evaluations with well-formulated datasets. Only a small number of studies (e.g., P18~\citeP{P18}, P39~\citeP{P39}, P41~\citeP{P41}) employ dataset evaluations within the selected works.
\end{itemize}
The results indicate that conceptual case studies are the most commonly adopted method in the selected studies, with 69 instances, followed by industrial case studies with 66 instances, and dataset evaluations with 4 instances. Additionally, the ``NA'' category includes studies that do not provide any solid demonstration methods.
 
\subsubsection{Software Platforms in Case Demonstration}
\begin{table}[]
\small
\caption{Software Platforms in Case Study of Selected Studies (RQ4)}
\label{table:case_study}
\begin{tabularx}{\textwidth}{lXc}
\hline
\textbf{Software Platforms} & \textbf{Paper IDs} & \textbf{Num.} \\ \hline

NA & 
\citeP{P19, P31, P39, P40, P47, P52, P57, P63, P65, P70, P73, P76, P93, P96, P103, P104, P116, P122, P139, P143, P146, P148, P158} 
& \cellcolor{gray!65}23 \\

Automotive System & 
\citeP{P5, P23, P24, P32, P45, P58, P59, P61, P79, P80, P98, P102, P107, P108, P115, P127, P128, P129, P154, P155} 
& \cellcolor{gray!60}20 \\

Control System & 
\citeP{P27, P29, P30, P42, P44, P60, P64, P78, P94, P111, P113, P121, P125, P132, P140, P142, P144, P161} 
& \cellcolor{gray!60}18 \\


Safety System & 
\citeP{P49, P50, P54, P71, P72, P78, P88, P89, P102, P108, P117, P126, P131, P132, P135, P147, P154} 
& \cellcolor{gray!55}17 \\

Order System & 
\citeP{P4, P10, P12, P16, P17, P48, P53, P55, P66, P67, P68, P114, P132, P133, P134, P152} 
& \cellcolor{gray!55}16 \\



Business System & 
\citeP{P18, P20, P23, P36, P37, P68, P77, P90, P91, P92, P124, P160} 
& \cellcolor{gray!40}12 \\

Resource Management System & 
\citeP{P5, P25, P26, P33, P34, P68, P82, P118, P121, P126, P136, P157} 
& \cellcolor{gray!40}12 \\

Workforce System & 
\citeP{P8, P22, P23, P28, P41, P46, P74, P75, P86, P95, P99, P145} 
& \cellcolor{gray!40}12 \\

Healthcare System & 
\citeP{P15, P17, P34, P38, P41, P51, P58, P83, P110, P119, P159} 
& \cellcolor{gray!35}11 \\


ATM System & 
\citeP{P12, P13, P17, P22, P35, P105, P109, P151, P153, P155} 
& \cellcolor{gray!35}10 \\

Aerospace System & 
\citeP{P28, P35, P69, P80, P100, P101, P130, P149, P156} 
& \cellcolor{gray!30}9 \\

Banking System & 
\citeP{P5, P21, P43, P68, P121} 
& \cellcolor{gray!15}5 \\

Authentication System & 
\citeP{P1, P84, P106, P112} 
& \cellcolor{gray!10}4 \\


Mobile Application & 
\citeP{P2, P34, P62, P68} 
& \cellcolor{gray!10}4 \\

Library System & 
\citeP{P6, P7, P9} 
& \cellcolor{gray!10}3 \\


Booking System & 
\citeP{P55, P56, P150} 
& \cellcolor{gray!5}3 \\

Education System & 
\citeP{P11, P123, P159} 
& \cellcolor{gray!5}3 \\

Other & 
\citeP{P137, P138, P141} 
& \cellcolor{gray!5}3 \\

Database System & 
\citeP{P3, P117} 
& \cellcolor{gray!5}2 \\

Examination System & 
\citeP{P14} 
& \cellcolor{gray!5}1 \\

\hline
\end{tabularx}
\end{table}
Recognizing that case studies are the primary method of demonstration in REDAST studies, we further analyzed the details of these case studies by categorizing the software platforms used in their demonstrations (Table~\ref{table:case_study}). The adopted software platforms are grouped into 17 categories, including shopping systems, resource management systems, financial systems, and more. These categories encompass the most common usage scenarios and are intended to guide the design of case studies in future research.

The results reveal that, alongside domain-specific systems such as control and automotive systems, order processing systems and ATMs are frequently selected to demonstrate test generation methodologies. These systems are often chosen because they are widely recognized and understood in the public domain, making them effective tools for illustrating methodologies in a comprehensible manner.

\subsubsection{Examples of Evaluation or Experiments}
\begin{table}[]
\small
\caption{Usability Results of Selected Studies (RQ4)}
\label{table:usability}
\begin{tabularx}{\textwidth}{lXc}
\hline
\textbf{Usability} & \textbf{Paper IDs} & \textbf{Num.} \\ \hline

The methodology explanation & 
Almost all of the papers enable clear methodology explanation, except \citeP{P35, P36, P38, P138, P140, P156, P158} 
& \cellcolor{gray!65}154 \\

Case example & 
\citeP{P1, P2, P3, P4, P5, P7, P8, P9, P11, P12, P13, P14, P15, P16, P17, P18, P21, P22, P23, P24, P26, P27, P28, P29, P32, P33, P34, P35, P36, P37, P38, P39, P40, P41, P42, P43, P44, P46, P48, P49, P50, P55, P58, P59, P60, P61, P62, P64, P66, P67, P68, P69, P71, P78, P79, P83, P84, P86, P88, P89, P90, P91, P92, P93, P94, P95, P96, P98, P99, P101, P102, P107, P108, P109, P110, P111, P112, P113, P114, P115, P116, P117, P118, P119, P121, P123, P124, P125, P126, P127, P129, P130, P131, P132, P133, P134, P135, P136, P137, P138, P140, P141, P143, P144, P145, P146, P147, P148, P149, P150, P151, P152, P154, P155, P156, P158, P159, P160, P161} 
& \cellcolor{gray!50}120 \\

Discussion & 
\citeP{P1, P3, P4, P5, P6, P7, P8, P9, P10, P11, P12, P13, P15, P17, P18, P19, P20, P21, P22, P23, P24, P25, P26, P27, P28, P29, P31, P32, P33, P35, P36, P37, P38, P40, P41, P42, P43, P44, P45, P46, P47, P48, P49, P50, P51, P52, P54, P55, P56, P58, P59, P60, P61, P62, P63, P64, P65, P66, P67, P68, P69, P70, P71, P72, P73, P75, P76, P77, P78, P79, P80, P82, P83, P84, P86, P88, P92, P94, P96, P99, P104, P113, P117, P121, P127, P128, P129, P130, P131, P132, P134, P135, P137, P141, P144, P147, P152, P154, P155, P156, P157, P158, P159, P160, P161} 
& \cellcolor{gray!40}105 \\

Experiment & 
\citeP{P2, P3, P5, P7, P9, P12, P13, P14, P15, P17, P18, P22, P23, P24, P26, P27, P28, P30, P32, P34, P35, P37, P39, P40, P41, P42, P43, P44, P45, P52, P53, P56, P58, P59, P61, P70, P71, P72, P76, P79, P80, P82, P83, P86, P89, P91, P95, P99, P101, P102, P107, P108, P110, P113, P117, P118, P121, P125, P127, P128, P129, P130, P131, P132, P133, P134, P135, P137, P138, P140, P141, P142, P154, P157, P159, P160, P161} 
& \cellcolor{gray!25}77 \\

\hline
\end{tabularx}
\end{table}
A critical aspect of any REDAST approach is the level of detail provided by the papers on the methodology and the evaluation of the approach. In this subsection, we classified the selected papers by separately analyzing the clarity and explanation provided in four key sections: (1) methodology explanation, (2) case examples, (3) experiments, and (4) discussions. Table~\ref{table:usability} shows an overview of different evaluation and methodology sections covered in the selected studies. For example, the second row (Case example) lists papers that cover the description of case examples in some level of detail, which are also exemplified below for papers P159 and P24. 

\paragraph{Case 1: P155 - Generating Test Scenarios from NL Requirements using Retrieval-Augmented LLMs: An Industrial Study} P155~\citeP{P160} introduces an LLM-driven test scenario generation method. While LLMs offer flexible and powerful natural language generation capabilities, their limited controllability poses challenges for broader applications in REDAST studies. To address this issue and validate the methodology’s reliability, P155 incorporates a case study based on an industrial usage scenario. The experiment evaluates performance using both quantitative metrics and qualitative human assessments. Quantitative metrics include standard machine translation evaluation measures such as BLEU, ROUGE, and METEOR. However, the core of the evaluation is a human interview with software engineers, offering deeper insights into the practical utility of the methodology. While metrics provide a partial perspective from an NLP standpoint, the human interviews emphasize the method’s effectiveness and applicability in real-world scenarios.

\paragraph{Case 2: P24 - Automatic Generation of Acceptance Test Cases from Use Case Specifications: an NLP-based Approach} P24~\citeP{P24} proposes an NLP-based framework for generating executable test cases. The industrial case-based experiments address three research questions, including evaluations of correctness and manual comparisons. Notably, in the manual comparison, instead of conducting direct human interviews for the generated test artifacts, the study employs a manual comparison with test cases created by domain experts. The results of these comparisons provide a robust demonstration of the generated test cases, as their alignment with established “golden truth” test cases serves as compelling evidence of the methodology’s effectiveness.

\subsubsection{Findings: Cross-Analysis of Demonstration Types and Target Software Systems}
\label{sec:findings_demo_target}
\begin{figure}
    \centering
    \includegraphics[width=1\linewidth]{fig//rq4/RQ4_demo_target.pdf}
    \caption{Cross-Distribution of Demonstration Types and Target Software Systems}
    \label{fig:target_demo}
\end{figure}
While researchers demonstrate the efficiency of their methods through evaluations or practical demonstrations, the target software system ultimately reflects the primary objective of REDAST studies. Our goal is to establish a connection between these demonstrations and the target software system to provide deeper insights into the preference for demonstration methods across different usage scenarios. From our findings in Fig.~\ref{fig:target_demo}, we observed that conceptual case studies remain the most commonly adopted demonstration method. In contrast, real (industrial) case studies are generally more challenging to conduct due to the difficulty of obtaining real-world data. However, in domains such as web services, safety-critical systems, and objective-oriented systems, real case studies are predominantly chosen. These critical systems often impose stricter pass-rate criteria for test outcomes, necessitating more rigorous validation. We suggest that future researchers pay particular attention to the demonstration aspect when dealing with such systems.


\begin{tcolorbox}[mybox, breakable, title=RQ4 Key Takeaways]
$\bullet$ Conceptual case studies are the most common evaluation method in REDAST studies. While these (example) case studies enhance understandability, they often lack strong empirical validation compared to real-world industrial cases. Industry case study evaluations are also prominent in $\approx$40\% studies, demonstrating the real-world applicability and need for REDAST approaches.

$\bullet$ There are very few studies conducted on any public datasets. This showcases a gap in the standardized evaluation of REDAST approaches due to the lack of standard benchmarks for requirements-driven test case generation.

\end{tcolorbox}