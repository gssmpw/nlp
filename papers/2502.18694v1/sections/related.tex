\section{Background and Related Work}~\label{sec:RelatedWork}

% input part of automated test generation
\subsection{Requirements Engineering}
Requirement engineering (RE) is the initial phase in software development, guiding all subsequent stages~\cite{requirements_survey_02, requirements_chetan}. The RE process requires gathering user needs and implementing the non-structured requirements into modeling language or other formed statements~\cite{requirements_terms,requirements_book_pohl}. It encompasses various activities tailored to the specific demands of software systems, with requirements elicitation, analysis, specification, and validation being the most necessary stages~\cite{requirements_survey_03,requirements_book_kotonya}. Requirements can be broadly categorized into functional and non-functional requirements~\cite{nonfunctional}. Requirements can be specified in different formats, e.g., using natural language (NL), modeling languages, such as UML and SysML, templates, such as use cases, or using formal notations. Thus, rather than using a single categorization for requirements in REDAST, we adopted multiple-level analysis in RQ1.

% output test artifacts
\subsection{Automated Software Testing and Requirements Engineering}
% \begin{figure}
%     \centering
%     \includegraphics[width=0.75\linewidth]{fig/related_work/v-model.drawio.pdf}
%     \caption{System Development Cycle in V-Model Illustration}
%     \label{fig:v-model}
% \end{figure}

Software testing aims to provide objective, independent information about the quality of software and the risk of its failure to users or sponsors~\cite{testing_define_01,testing_define_02}. Automated software testing is using automation techniques to use specialized tools and scripts to execute test cases on a software application without manual intervention, which can improve time efficiency and human resource efficiency~\cite{rafi2012benefits, deming2021software}. While the satisfaction of stakeholders is one of the priorities in software testing, the relationship between software requirements and testing becomes a critical focus in SDLC~\cite{vmodel_sdlc_01,vmodel_sdlc_02}. The alignment between different stages of verification and validation, e.g., system analysis and system testing is key for effective software quality assurance. Here, we primarily focus on the requirements specification and testing, while testing verifies that the software meets its specified requirements. This relationship is fundamental to ensuring the final product aligns with stakeholder expectations and functions correctly.

\subsection{REDAST Secondary Studies}

REDAST studies have been long investigated in past research. However, only limited studies systematically discussed RE-driven automated software testing. Atoum et al.~\cite{atoum2021challenges} conducted a systematic study that examines the requirements of quality assurance and validation, where they reported a test-oriented approach. Unterkalmsteiner et al.~\cite{unterkalmsteiner2015assessing} built a taxonomy for aligning requirements engineering and software testing to enhance coordination between these activities. They pointed out the importance of integrating requirements into the testing process, which contains some REDAST studies. Mustafa et al.~\cite{mustafa2021automated}'s literature review is the most related paper. They investigated 30 selected papers by 2018 and limitedly analyzed the requirements-driven testing process from requirements input, techniques, and output perspectives. Their review provides a basic view of these parts but did not comprehensively discuss the details from various dimensions and levels due to the depth of understanding.
