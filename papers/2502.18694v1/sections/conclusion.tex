\section{Conclusion}~\label{sec:Conclusion}
Requirements Engineering-Driven Automated Software Testing (REDAST) presents a significant yet challenging task in contemporary software engineering research. Automating the generation of test artifacts from requirements has the potential to greatly enhance the efficiency and effectiveness of the testing workflow. However, the absence of systematic guidelines and comprehensive literature reviews on REDAST methodologies complicates research efforts and impedes progress in the field. This article presents a systematic literature review (SLR) on the technical approaches and solutions proposed for REDAST across various software systems.

Our review identified 156 relevant studies from an initial pool of 27,333 papers through a rigorous multi-stage filtering, selection, and processing methodology. These studies were analyzed from five key perspectives: requirements input, transformation techniques, test outcomes, evaluation methods, and limitations. Our results show that (1) Functional requirements, model-based specifications, and natural language (NL) requirements are the most frequently used types, formats, and notations, respectively; (2) Rule-based techniques dominate in REDAST studies, while machine learning (ML)-based techniques are relatively underexplored; (3) Most frameworks are sequential, employing a single intermediate representation; (4) Studies frequently focus on concrete test artifacts; (5) Test cases, structured textual formats, and requirements coverage are the most commonly discussed types, formats, and coverage approaches, respectively; (6) While most studies conduct conceptual demonstrations, relatively few utilize dataset-based evaluations; (7) Although most studies provide robust methodological explanations, only half report promising experimental outcomes; (8) Only 35 studies achieve full automation, with most requiring unnecessary manual intervention; (9) Framework design remains the most frequently cited limitation, particularly the inability to handle complex configurations; (10) Many studies propose extending coverage criteria or addressing other requirement types.

Building on these findings, we propose several recommendations to advance REDAST research. The remarkable advancements in large language models (LLMs) highlight the potential of AI-based techniques for transformation tasks. Emphasis should also be placed on enhancing automation and reproducibility to realize the full efficiency gains promised by REDAST methodologies.

This study focuses exclusively on requirements-driven testing due to the vast volume of related literature. However, other stages of verification and validation are equally critical for comprehensive exploration. We aim to expand future research to cover broader alignments within the software development lifecycle (SDLC), bridging gaps across the entire verification and validation spectrum.