\subsection{RQ1: Requirements Specification Formulation}
\label{sec:RQ1}
Based on our taxonomy schema in Section~\ref{sec:Taxonomy}, we explore the techniques for requirements type, format, and notation in the selected studies. Moreover, by comparing the characteristics of the studies, we analyze the features in different adoptions of specification techniques.

\subsubsection{Requirements Types}

% Please add the following required packages to your document preamble:
% \usepackage{graphicx}

\begin{table}[]
\small
\caption{Requirements Types in Selected Studies (RQ1)}
\label{table:req_types}
\begin{tabularx}{\textwidth}{lXc}
\hline
\textbf{Requirements Types} & \textbf{Paper IDs} & \textbf{Num.} \\ \hline
Functional Requirements & Almost all papers support functional requirements, except \citeP{P18, P74, P79, P155, P156}. & \cellcolor{gray!65}152 \\

Non-functional Requirements & \citeP{P2, P33, P41, P48, P50, P51, P56, P57, P61, P68, P69, P71, P73, P76, P79, P83, P88, P90, P96, P116, P123, P124, P146, P155, P156, P157, P161} & \cellcolor{gray!55}27 \\

User Requirements & \citeP{P2, P43, P74, P84, P88, P92, P119, P121, P139, P142} & \cellcolor{gray!40}10 \\

Constraint Requirements & \citeP{P86, P91, P94, P100, P101, P114, P161} & \cellcolor{gray!35}7 \\

Business Requirements & \citeP{P18, P51, P74, P89, P94, P95, P137} & \cellcolor{gray!25}7 \\

Implementation Requirements & \citeP{P95} & \cellcolor{gray!15}1 \\

\hline
\end{tabularx}
\end{table}

% \begin{figure}
%     \centering
%     \includegraphics[width=0.9\linewidth]{fig//rq1/Req_Type.pdf}
%     \caption{Number of Adopted Requirements Types}
%     \label{fig:req_type_bar}
% \end{figure}

In this section, we mainly discuss the results of the requirements type of the selected studies, where the requirements could be classified into six categories, functional, non-functional, user, constraint, business, and implementation requirements The total number of the adopted requirements type is not exactly the same as the number of publications, where some studies cover multiple requirements types as the input of the methodologies.

Based on the results in Table~\ref{table:req_types}, almost all of the REDAST studies cover functional requirements (152/156), followed by non-functional requirements (27/156), user requirements (10/156), constraint requirements (7/156), business requirements (7/156), and implementation requirements (1/156). This trend can be attributed to functional requirements being inherently \emph{testable}, as they explicitly define the system's expected behavior. Unlike non-functional requirements, which often involve abstract or qualitative criteria, functional requirements provide concrete, measurable specifications that align well with the design of test artifacts. Additionally, we noticed that some domain-specific requirements have been adopted in several studies, such as the security requirements in safety-critical systems (e.g., P118~\citeP{P123}), timing requirements in reactive systems (e.g., P45~\citeP{P45}), and so on (e.g., P45~\citeP{P45}, P28~\citeP{P28}, P119~\citeP{P124}). These requirements could be considered in the other requirements categories.

\subsubsection{Requirements Specification Format}

% \begin{figure}
%     \centering
%     \includegraphics[width=0.75\linewidth]{fig//rq1/Req_Format.pdf}
%     \caption{Distribution of Adopted Requirements Format}
%     \label{fig:req_format}
% \end{figure}

\begin{table}[]
\small
\caption{Requirements Format in Selected Studies (RQ1)}
\label{tab:req_form}
%\resizebox{\columnwidth}{!}{%
\begin{tabularx}{\textwidth}{lXc}
\hline
\textbf{Requirements Formats} & \textbf{Paper IDs} & \textbf{Num.} \\ \hline
Textual & \citeP{P2,P5,P6,P7,P8,P9,P10,P11,P12,P13,P14,P15,P16,P17,P18,P19,P22,P23,P24,P25,P26,P28,P31,P32,P34,P36,P37,P38,P39,P41,P42,P43,P44,P45,P46,P47,P48,P51,P52,P54,P55,P57,P58,P60,P61,P62,P63,P64,P65,P67,P68,P70,P71,P75,P76,P77,P80,P82,P83,P84,P86,P88,P92,P93,P96,P98,P99,P102,P103,P106,P108,P109,P110,P112,P114,P116,P117,P119,P121,P122,P123,P125,P126,P127,P128,P129,P130,P132,P133,P134,P135,P136,P137,P138,P139,P140,P145,P147,P149,P151,P152,P153,P156,P157,P160} & \cellcolor{gray!70}105 \\

Model-Based & \citeP{P1,P4,P7,P8,P10,P11,P14,P15,P16,P17,P20,P21,P25,P27,P28,P30,P33,P35,P46,P49,P50,P53,P56,P57,P65,P66,P69,P73,P74,P75,P78,P86,P89,P90,P95,P96,P99,P104,P109,P110,P115,P119,P123,P131,P137,P143,P144,P145,P148,P150,P152,P153,P154,P161} & \cellcolor{gray!46}54 \\

Formal & \citeP{P3,P21,P45,P58,P59,P72,P79,P105,P110,P111,P113,P124,P155} & \cellcolor{gray!25}13 \\

Constraint-Based & \citeP{P3,P27,P33,P59,P72,P78,P91,P100,P101,P105,P154} & \cellcolor{gray!15}11 \\

Tabular (Matrix-Based) & \citeP{P29,P33,P78,P92,P107,P154,P159} & \cellcolor{gray!10}7 \\

Other & \citeP{P40,P94,P118,P142,P141,P146,P158} & \cellcolor{gray!10}7 \\
\hline
\end{tabularx}%
%}
\end{table}


% Please add the following required packages to your document preamble:
% \usepackage{graphicx}
% \usepackage[table,xcdraw]{xcolor}
\begin{table}[]
\small
\caption{Requirements Format Results of Selected Studies with Unique Formulations (RQ1)}
\label{table:req_form_unique}
%\resizebox{0.5\textwidth}{!}{%
\begin{tabularx}{\textwidth}{lXc}
\hline
\textbf{Requirements Formats}          & \textbf{Requirements Formulations}                                     &                 
\textbf{Num.} 
\\ \hline
Model Specification            &  Behavior Tree, Graph-based, Finite State Machine, Specification and Description Language (SDL), Use Case Map, Activity Diagram, Communication Diagram, Misuse Case, Conditioned Requirements Specification, Use Case, Scenario conceptual model, UML, Models, Sequence Diagram, State Machine Diagram, Scenario, NL requirements, pseudo-natural language, Behavior Model, Extended Use Case Pattern, Linear temporal logic, Communication Event Diagram (CED), SysML, Formal Use Case, Textual Normal Form, Object Diagram & \cellcolor{gray!50}30 \\
Textual Specification            & 
Graph-based, Scenario specification, User Story, NL Requirements, Use Case Map, Misuse Case, Use Case, Scenario Model, Textual, Scenario conceptual model, DSL, Use Case/Scenario, NL requirements (Behavior), Business Process Modeling Language, Signal Temporal Logic (STL), Restricted Signals First-Order Logic (RFOL), Formal Requirements Specification, Formal Use Case
& \cellcolor{gray!40}26 \\
Constraint Specification            & Graph-based, OCL, DSL, Finite State Machine, Formal Requirements Specification, SysReq-CNL, Scenario, UML & \cellcolor{gray!20}7 \\
Formal Specification            & Signal Temporal Logic (STL), Restricted Signals First-Order Logic (RFOL), Formal Requirements Specification, Linear Temporal Logic, NL requirements, Use Case & \cellcolor{gray!20}7 \\
Other Specification            & Requirements Dependency, Requirements Priorities, Safety Requirements Specification, Test requirements & \cellcolor{gray!10}4 \\
Tabular Specification            & Scenario, Tabular Requirements Specification, Finite State Machine & \cellcolor{gray!10}3 \\
\hline
\end{tabularx}% 
%}
\end{table}


Requirements specification is categorized into seven different types: textual, model-based, constraint-based, formal (mathematical), tabular (matrix-based), and other specifications. The specification formats adopted in the selected studies are presented in Table~\ref{tab:req_form}. Note that we found some studies that use the transformation method further to convert raw requirement input into the other requirement formats. Here, we only consider the first raw requirement input in this section. For example, P76~\citeP{P76} adopts model-based specification as the raw input for requirements. However, the requirements are further converted to model-based requirements. To clarify our objective, we only consider scenario-based requirements in RQ1. The intermediate representations are discussed in RQ2.

\begin{itemize}
    \item \textit{Textual Specification.} 105 studies adopted textual specification methods. Our analysis shows that textual specification is the most commonly used format in REDAST studies. Textual specification, written in natural language (NL), is the predominant choice. Besides the textual specification, NL is widely integrated into other specification formats, including formal (mathematical) and tabular specifications. NL-based requirements are generally favored in RADAST studies due to their accessibility and ease of understanding, which supports both requirements description and parsing NL is commonly used in these studies due to its advanced explainability and flexibility. We separately discuss this category by distinguishing textual specifications from others and whether the requirements follow natural language logic~\cite{loucopoulos1995system,sommerville2011software}. Other formats especially involve specific specification rules or templates compared to textual specifications.
    
    \item \textit{Model-based Specification.} 54 studies specify requirements using model-based specifications. Model-based approaches construct semi-formal or formal meta-models to represent and analyze requirements. Compared to textual specifications, meta-models have better abstraction capabilities for illustrating the behaviors (e.g., P1~\citeP{P1}, P27~\citeP{P27}, P35~\citeP{P35}, etc.), activities (e.g., P25~\citeP{P25}, P139~\citeP{P144}, P146~\citeP{P151}, etc.), etc., of a software system~\cite{loniewski2010systematic}. 
    
    \item \textit{Constraint-based Specification.} We identified 11 studies in the selected papers that utilize constraint-based requirements specification. Constraint-based specification is also a welcomed requirements format, where Domain-Specific Language are adopted in the selected studies, e.g., P3~\citeP{P3}, P89~\citeP{P92}, P96~\citeP{P100}, etc. Constraint-based specification involves defining system properties as limits, conditions, or relationships that must hold within the system. Constraint-based specification can provide a concise and precise description of system behavior, especially in the context of complex software systems~\cite{definition_constraint_specification_02,definition_constraint_specification_01}. DSL is a typical constraint-based specification, e.g., P88~\citeP{P91}, P95~\citeP{P99}, and P96~\citeP{P100}.
    
    \item \textit{Formal (Mathematical) Specification.} In the selected studies, we identified 13 papers that used formal (mathematical) specifications in requirements elicitation. The formal (mathematical) specification can translate natural language requirements into a precise and unambiguous specification that can be used to guide the development of software systems~\cite{definition_formal_specification_01}, where the typical formal requirements specification are assertions (e.g., P109~\citeP{P113}), controlled language (e.g., P79~\citeP{P79}, P108~\citeP{P112}), and so on. Unlike textual specification, the logical expression can describe software requirements unambiguously~\cite{dulac2002use}.

    \item \textit{Tabular (Matrix-based) Specification.} We identified 7 studies that adopted tabular specifications. Tabular specification methods can formalize requirements in a structured and organized manner, where each row represents a requirement, and each column represents a specific attribute or aspect of the requirement~\cite{definition_formal_specification_01}. Tabular specification can greatly improve traceability and make it more friendly for verification and validation.
\end{itemize}


%For example, in P102, we identify the requirements in this study using NL. Based on their example of requirements specification, `` If the video sensor is not working, the predictive safety system shall not act'', we classify this requirements format into textual specification because this expression smoothly fits the logic of natural language, even if they use ``shall'' to formulate the expression.

% \paragraph{Scenario-based Specification.}
% We identified 34 studies ($21.12\%$ in selected studies) in the selected papers that utilize this approach. Scenario-based specification is also a welcomed requirements format, where use cases and requirements scenarios are adopted in the selected studies, e.g., P75~\citeP{P75}, P76~\citeP{P76}, P88~\citeP{P88}, etc. Scenario-based specifications use narratives (scenarios) to specify software requirements in a story-like format. These requirement scenarios can greatly improve understandability for both technical and non-technical stakeholders~\cite{pacheco2018requirements}. Actually, we found some studies use combined specification methods for their input, such as P16~\citeP{P16} (Use Case Diagram), P86~\citeP{P86} (Use Case Diagram), and so on. Here, we didn't consider them as scenario-based specifications because their main specification format is diagramized models.


\subsubsection{Requirements Specification Notation}


% \begin{figure}
%     \centering
%     \includegraphics[width=0.9\linewidth]{fig//rq1/Req_Notation.pdf}
%     \caption{Number of Adopted Requirements Notation}
%     \label{fig:req_nota}
% \end{figure}

\begin{table}[]
\small
\caption{Requirements Notation Results of Selected Studies (RQ1)}
\label{tab:req_notation}
\begin{tabularx}{\textwidth}{lXc}
\hline
\textbf{Requirements Notations} & \textbf{Paper IDs} & \textbf{Num.} \\ \hline
NL requirements & \citeP{P23} (PURE dataset), \citeP{P26,P34} (Textual User Story), \citeP{P37} (Usage Scenario), \citeP{P38} (NL Requirements), \citeP{P39} (Scenario Specification), \citeP{P40} (Test requirements), \citeP{P41} (Textual Use Case), \citeP{P42,P44}, \citeP{P52} (Textual Use Case), \citeP{P54,P55}, \citeP{P60} (Textual Use Case), \citeP{P63,P67,P71} (Template-based), \citeP{P77} (Scenario), \citeP{P82,P84,P87}, \citeP{P88} (Textual Use Case), \citeP{P93,P98,P102,P104,P106,P113,P116,P117}, \citeP{P121} (Claret Format), \citeP{P122,P126,P127,P128,P129,P130,P135,P136,P138,P139,P140}, \citeP{P147} (Technical Requirements Specification), \citeP{P149,P151}, \citeP{P157} (Positive and negative pair), \citeP{P160} & \cellcolor{gray!65}47 \\

UML & \citeP{P4,P7,P8,P10,P11,P15,P16,P17}, \citeP{P25} (Activity Diagram), \citeP{P28} (Sequence Diagram), \citeP{P35,P46,P49,P53,P56}, \citeP{P57,P66,P73,P74,P75,P89,P90}, \citeP{P96} (UML MAP), \citeP{P99}, \citeP{P109} (Sequence Diagram), \citeP{P110,P115,P119,P123}, \citeP{P145} (Activity Diagram), \citeP{P152} (Activity Diagram), \citeP{P153} (Sequence Diagram), \citeP{P161} (Modeling and Analysis of Real Time and Embedded Systems) & \cellcolor{gray!45}33 \\

Other & \citeP{P18} (Semi-Structured NL), \citeP{P21} (OWL-S Model), \citeP{P29} (State-Transition Table), \citeP{P58} (Formal NL Specification), \citeP{P64} (Structured Requirements Specification), \citeP{P65} (Class Diagram, Restricted-form of NL), \citeP{P68} (Semi-Formal Requirements Description), \citeP{P69} (Requirements Specification Modeling Language), \citeP{P94} (Safety Requirements Specification), \citeP{P107} (Expressive Decision Table), \citeP{P125} (Textual Use Case), \citeP{P131} (Functional Diagram), \citeP{P137} (Requirement Description Modeling Language), \citeP{P142} (Requirements Dependency Mapping), \citeP{P143} (Specification and Description Language), \citeP{P144} (Behavior Tree), \citeP{P148} (Domain-Specific Modeling Language), \citeP{P150} (Statechart Diagram), \citeP{P158} (Risk Factor), \citeP{P159} (Requirement Traceability Matrix) & \cellcolor{gray!30}20 \\

Custom & \citeP{P1} (Custom Metamodel), \citeP{P3} (Constraint-based Requirements Specification), \citeP{P9} (Custom CNL), \citeP{P12} (Semi-Structured NL Extended Lexicon), \citeP{P14} (Interaction Overview Diagram), \citeP{P27} (SCADE Specification), \citeP{P30} (Extended SysML), \citeP{P32} (RUCM with PL extension), \citeP{P47} (Aspect-Oriented PetriNet), \citeP{P50} (Safety SysML State Machine), \citeP{P59} (OCL-Combined AD), \citeP{P62} (State-based Use Case), \citeP{P86} (Contract Language for Functional PF Requirements (UML)), \citeP{P92} (Textual Scenario based on tabular expression), \citeP{P103} (NL requirements (Language Extended Lexicon)), \citeP{P111} (Specification language for Embedded Network Systems), \citeP{P114} (Requirements Specification Modeling Language) & \cellcolor{gray!25}17 \\

CNL & \citeP{P2}, \citeP{P5} (RUCM), \citeP{P6} (Use Case Specification Language (USL)), \citeP{P22} (RUCM), \citeP{P24} (RUCM), \citeP{P45}, \citeP{P61} (RUCM), \citeP{P80}, \citeP{P83} (Restricted Misuse Case Modeling), \citeP{P108}, \citeP{P112,P132,P133,P134,P156} & \cellcolor{gray!15}15 \\

Use Case Description Model & \citeP{P13} (Use Case Description Model), \citeP{P19} (Use Case Description Model), \citeP{P36} (Use Case Description Model), \citeP{P43} (Use Case Description Model), \citeP{P48} (Use Case Description Model), \citeP{P70} (Use Case Description Model), \citeP{P76} (Use Case Description Model) & \cellcolor{gray!10}7 \\

Requirements Priorities & \citeP{P118} (Customer-assigned priorities, Developer-assigned priorities), \citeP{P141} (Customer Assigned Priority), \citeP{P146} (Stakeholder Priority) & \cellcolor{gray!5}3 \\

SCR & \citeP{P33,P78,P154} & \cellcolor{gray!5}3 \\

DSL & \citeP{P91,P100,P101} & \cellcolor{gray!5}3 \\

Formal Equation & \citeP{P79,P124,P155} & \cellcolor{gray!5}3 \\

Cause-Effect-Graph & \citeP{P20,P95} & \cellcolor{gray!5}2 \\

OCL & \citeP{P72,P105} & \cellcolor{gray!5}2 \\

RSL & \citeP{P31,P51} & \cellcolor{gray!5}2 \\ \hline
\end{tabularx}
\end{table}

The requirements notation is an extended detail of the requirements format. We identified over 60 requirement notations across the selected studies. Specifically, for some variations of standard requirement notations, we categorized the similar notations into their original forms or grouped uncommon notations under the ``other'' category. The summary of these notations is presented in Table \ref{tab:req_notation}.

Based on the results, we identified that Natural language (NL) requirements specification is the most frequently adopted notation in the selected studies (47 studies), followed by UML notation (33 studies), other (20 studies), and custom requirements (17 studies). Overall, this result aligns with the trend observed in the requirements format results, where natural language is widely adopted in REDAST methods.
\begin{itemize}
    \item \textit{NL Requirements Specification.} We found that 47 studies introduced natural language (NL) requirements specifications in their methods. NL requirements specifications are used not only in REDAST studies but also in requirements elicitation and specification domains. For example, “shall” requirements (formally known as IEEE-830 style “shall” requirements~\citep{ieee1998ieee}) are widely used for requirements specification, enabling less ambiguity and more flexibility. NL requirements specifications are applicable for various processing methods, such as condition detection (e.g., P23~\citeP{P23}) and semantic analysis (e.g., P63~\citeP{P63}).

    \item \textit{Unified Modeling Language} Unified Modeling Language (UML) is a commonly used notation in model-based specification. We identified 33 studies that utilized UML in the selected papers. UML is versatile and can be combined with other notations to describe scenarios, behaviors, or events, effectively capturing functional requirements~\cite{henderson2005uml}. For instance, in the selected studies, P56~\citeP{P56} introduced a tabular-based UML for requirements traceability, while P17~\citeP{P17} employed UML use case diagrams specifically to depict requirement scenarios.

    \item \textit{Controlled Natural Language.} In the selected studies, 15 papers opted for controlled natural language (CNL) as a requirement notation. CNL is partly based on natural language but is structured using the Rimay pattern~\cite{veizaga2021systematically}, deviating from conventional expression syntax.

    \item \textit{Use Case, User Story, and Their Variations.} Use cases, user stories, and their variations are distinct requirement notations in scenario-based specifications, sharing similar characteristics. These notations generally consist of a cohesive set of possible dialogues that describe how an individual actor interacts with a system or use textual descriptions to depict the operational processes of the system. In this way, the system behavior is vividly explained.

    \item \textit{Other Specifications.} Other specification notations are not frequently adopted methods, where the ``Other'' category contains the notations that appear one time. Most of them are variations of common notations.
\end{itemize}

\subsubsection{Findings: Cross-Analysis of Requirements Input and Target Software}
\label{sec:findings_req_target}
% Please add the following required packages to your document preamble:
% \usepackage{graphicx}
\begin{table}[]
\small
\caption{Cross Distribution of Requirements Format and Target Software}
\label{tab:req_target}
\begin{tabularx}{\textwidth}{Xcccccc}
\hline
\textbf{Target Software}    & \textbf{Model-based} & \textbf{Textual} & \textbf{Constraint-based} & \textbf{Formal} & \textbf{Other} & \textbf{Tabular (Matrix-based)} \\ \hline
\textbf{General Software}   & 75             & 34               & 4                   & 4               & 3              & 6                \\
\textbf{Embedded System}    & 7              & 2                & 3                   & 1               & 2              & 1                \\
\textbf{Web Services}       & 3              & 5                & 1                   & 0               & 0              & 0                \\
\textbf{Safety-Critical}    & 1              & 4                & 1                   & 3               & 0              & 0                \\
\textbf{Timed Data-flow}    & 4              & 0                & 0                   & 0               & 0              & 0                \\
\textbf{Reactive}           & 2              & 0                & 1                   & 0               & 1              & 0                \\
\textbf{Real-time Embedded} & 2              & 1                & 0                   & 0               & 0              & 0                \\
\textbf{Product Line}       & 2              & 0                & 0                   & 0               & 0              & 0                \\
\textbf{Object-Oriented}    & 1              & 2                & 0                   & 0               & 0              & 0                \\
\textbf{Telecom}            & 1              & 1                & 0                   & 0               & 0              & 0                \\
\textbf{Automotive}         & 1              & 0                & 1                   & 1               & 0              & 0                \\ \hline
\end{tabularx}%
\end{table}

As the first step in the REDAST process, the selection of requirements formulations predominately decides the usage scenario of the framework. More specifically, the end goal of the framework forces the researchers to select appropriate requirements formats and notations to describe the different system behaviors, events, or activities. Here, besides the results in requirements format, we cross-discuss the requirements format and target software (in Section~\ref{sec:Trend}) to illustrate the requirements preference in REDAST in the context of usage scenarios.

\textit{Textual requirements dominate across all categories}, where general software (60\%), Embedded Systems (50\%), Real-Time Systems (67\%), and other domains primarily select textual requirements as default. This trend suggests that, due to the flexibility and simplicity of textual requirements, textual requirements can handle most usage scenarios in REDAST.

\textit{Model-based requirements are preferred for structured systems}. Detailly, the selection of model-based requirements, Web Serviced (P1~\citeP{P1}, P4~\citeP{P4}, P11~\citeP{P11}, P21~\citeP{P21}, P56~\citeP{P56}), Safety-Critical Systems (P27~\citeP{P27}, P50~\citeP{P50}), Object-Oriented Systems (P28~\citeP{P28}, P66~\citeP{P66}), Product Line Systems (P32~\citeP{P32}), and SOA-based Systems (P87~\citeP{P90}), indicates the preference of model-based requirements for service-oriented architectures, correctness and traceability assurance.

\textit{Formal and constraint-based requirements are crucial for high-reliability domains}, wherein formal requirements and constraint-based requirements can additionally satisfy the needs of strict verification and validations, e.g., (1) for formal requirements, Safety-Critical Systems (P72~\citeP{P72}), Automotive Systems (P59~\citeP{P59}), Cyber-Physical Systems (P58~\citeP{P58}), Embedded Systems (P107~\citeP{P111}, P109~\citeP{P113}), (2) for constraint-based requirements, Safety-Critical Systems (P27~\citeP{P27}, P72~\citeP{P72}, P97~\citeP{P101}), Automotive Systems (P59~\citeP{P59}), Event-driven Systems (P149~\citeP{P154}), and Complex Dynamic Systems (P88~\citeP{P91}).

\subsubsection{Findings: Trend of Requirements Input Over the Years}

\begin{figure}
    \centering
    \includegraphics[width=0.9\linewidth]{fig//rq1/RQ1_requirement_year.pdf}
    \caption{Trend of Requirements Input by Years}
    \label{fig:requirements_year}
\end{figure}

With the advancement of requirements engineering research, an increasing number of requirements specification methods have emerged over the past decade~\cite{software_requirements_specification_01,software_requirements_specification_02}. In this section, we analyze the trend of requirements format preferences over time, as illustrated in Fig.~\ref{fig:requirements_year}. Specifically, before 2008, although textual requirements were already widely employed in the REDAST methodology, their proportion did not significantly dominate among the six requirements formats. This observation aligns with the study distribution discussed in Section~\ref{sec:Trend}. After 2008, textual requirements gradually became the preferred choice for requirements specification. Furthermore, as indicated by the increasing trend in the “other” requirements format, we observed a growing adoption of diverse requirements formats in recent years. This trend suggests an increasing diversification in requirements selection over time.

\begin{tcolorbox}[mybox, title=RQ1 Key Takeaways]
$\bullet$ Textual specifications are the most prevalent format for REDAST studies and dominate the test artifact generation process in general-purpose software systems. Structured NL (e.g., CNL, RUCM) are preferred over unstructured NL, as they offer a balance between readability and automation in test artifact generation.

$\bullet$ Model-based, formal or constraint specifications are more structured and preferred in embedded, safety-critical, and real-time domains.

$\bullet$ Most studies focus on functional requirements, with only a few addressing non-functional requirements (e.g., performance, security, safety).
\end{tcolorbox}