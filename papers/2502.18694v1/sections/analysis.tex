\section{Discussion and Roadmap}~\label{sec:Discussion}
Based on the results of our predefined RQs, we present several guidelines, insights, and recommendations for future research in REDAST.
\subsection{Data Preprocessing for REDAST}
Data is the basis of the REDAST methodology. REDAST methodology is a data-oriented process, while the acquired data primarily determine the framework design in specific usage scenarios. We found that most of the selected papers specified their data usage strategy. We identified many papers that adopted industrial data for their development and demonstration, e.g., P85~\citeP{P88}, P109~\citeP{P113}, P155~\citeP{P160}, etc. These data, however, originated from raw industrial cases and require further pre-processing for development. Thus, we recommend researchers customize the pre-processing to match their framework design. Besides, we also identified that some studies require additional data for the development, e.g., the training data for ML-based methods (P18~\citeP{P18}, P52~\citeP{P52}, P92~\citeP{P95}, etc). Our other recommendation is to align data for development with the methodology framework and the experimental demonstration, which maintains performance consistency throughout the process, from design to development to evaluation and implementation.

\subsection{Requirements Input for REDAST}
The results of our research questions (RQs) highlight the diverse preferences for requirements input. Findings in Section~\ref{sec:findings_req_target} suggest that the adoption of requirements specification is closely related to the intended usage scenario. Therefore, rather than recommending a specific combination of format and notation, we first suggest that researchers select requirements specifications based on their actual application scenarios.

Under general end goals, textual requirements offer the greatest flexibility and broad applicability across various target software systems, allowing for diverse notational choices to accommodate different tasks. For specialized or critical systems, structured formats such as model-based and tabular requirements are typically preferred, while formal and constraint-based requirements are more commonly adopted in high-reliability domains. However, our findings indicate that the distinction between different types of requirements input is not always significant, as textual requirements are frequently used even in specialized systems.

Furthermore, the choice of requirements specification is not only influenced by the usage scenario but also affects subsequent implementation and scalability. The results of RQ1 demonstrate that a variety of requirement types and notations have been employed in previous REDAST studies. While only a few studies have successfully managed multiple types of requirements input, it is unrealistic to expect a single framework to accommodate all requirement formats. Thus, appropriately adapting the requirement type within the REDAST methodology can significantly expand its application scope and enhance its scalability.

\subsection{Transformation Techniques for REDAST}

The transformation techniques used in REDAST correspond to RQ2, where we categorize them into machine learning (ML)-based, NLP-pipeline-based, rule-based, metamodel-based, and search/graph-based approaches. Given that test artifacts are generally structured data, rule-based and metamodel-based approaches—being the most commonly adopted techniques (appearing in 122 and 102 out of 156 papers, respectively)—facilitate structural transformations from requirements to test outcomes. Findings from RQ2 and RQ3 indicate that recent studies increasingly adopt diverse transformation techniques, regardless of the types of requirements inputs or test artifacts used. Based on these observations, we recommend employing a combination of transformation techniques rather than relying solely on conventional methods. While NLP-pipeline- and ML-based methods were previously considered “uncontrolled,” the era of large language models (LLMs) has introduced advanced flexibility and generalization capabilities, which have driven significant advancements in various domains, including REDAST studies. By integrating these emerging techniques with traditional rule-based approaches, the risks associated with uncontrolled behavior in cutting-edge methods can be mitigated, ensuring a balanced and effective transformation process.

\subsection{Test Artifacts Output for REDAST}
We identified a lack of details for the specifications of test artifacts. Even though we categorized the test outcomes in RQ3 based on their technologies, they are not formally reported in the papers. We formulated the test outcomes on the abstraction level, format level, and notation level. Another factor that should be considered for test artifacts is the executability. The need for executability varies with respect to different testing stages.

In general, we recommend that, in future studies, (1) the implementation details, such as abstraction, format, notation, and so on, are encouraged to be specified in the technical descriptions; (2) the executability should be seriously specified under the consideration of test stages or phases.

\subsection{Evaluation Solutions for REDAST Studies}
In RQ4, we identified evaluation and demonstration methods in the selected studies, categorizing them into case studies and dataset evaluations. However, we found that dataset evaluation is rarely adopted due to the limited availability of data resources for pairing requirements with test artifacts, appearing in only 5 out of 156 papers. Regarding case studies, researchers typically choose between real-world and conceptual cases. Our findings indicate that both conceptual and real case studies can provide strong persuasive value. However, as highlighted in Section~\ref{sec:findings_demo_target}, real case studies are generally preferred in certain specialized domains, such as web services, safety-critical systems, and objective-oriented systems. Despite their advantages, both real and conceptual case studies frequently report demonstration limitations.

To enhance the demonstration of methodological efficacy, we suggest that future studies incorporate both conceptual and real case studies within a single study. Conceptual case studies can serve to illustrate the methodological framework, while real case studies can be introduced in the final evaluation and demonstration phase to strengthen empirical validation especially.

Additionally, we observed that publicly available datasets for REDAST—and even in broader requirements engineering (RE) and software testing domains—are extremely limited. We urge the research community to focus on developing and maintaining public datasets for REDAST, as this would significantly improve the research environment and facilitate further advancements in the field.

\subsection{Other Suggestion for REDAST Studies}
Besides the above suggestions from technical or demonstration perspectives, other points should be taken into account.

\textbf{Automation} is a significant factor for REDAST studies. We identified that even though some studies report ``automation'' in titles or methodologies, human operations are still necessary within the generation process, or there is a lack of description of the implementation details of the automation. We strongly suggest that future researchers keep the automation details transparent.

\textbf{Reproducibility}. While REDAST methods are designed for industrial applications, the implementation developers and end users expect the methods to be directly applied. The reproducibility of the methodologies determines if the method can be successfully spread among industrial workflows. Thus, we recommend that researchers provide sufficient implementation details for reproducing or directly attaching code links to the paper.