\section{Threats to Validity}~\label{sec:Threats}
\subsection{Internal Validity}
In this study, the first author designed the SLR protocol, which was reviewed and refined collaboratively with the second, third, and fourth authors before formal implementation. The detailed topics and search strings were iteratively adjusted and executed across multiple databases to optimize the retrieval of relevant results. To accommodate the varying search policies of these databases, the search strings were customized accordingly. The selection of studies followed a multi-stage filtering process to minimize selection bias. The first round of filtering was based on titles and abstracts. The second round involved brief reading and keyword matching, while the third round consisted of a comprehensive reading of the papers. The final selection was validated by all authors to ensure robustness. Following study selection, a data extraction process was designed using Google Forms. All authors participated in a pilot test to refine the data extraction procedure and ensure consistency in capturing the necessary information.

\subsection{Construct Validity}
To mitigate threats to construct validity, we conducted the search process across six widely used scientific databases, employing a combination of automated and manual search strategies. Extensive discussions among all authors were held to refine the inclusion and exclusion criteria, ensuring they effectively supported the selection of the most relevant studies for this SLR. Some of the selected studies included vague descriptions of their methodologies, posing potential threats to the validity of the study. These cases were carefully reviewed and deliberated upon by the first and second authors to reach a consensus on their inclusion.

\subsection{Conclusion Validity}
The threat to conclusion validity was minimized through a carefully planned and validated search and data extraction process. To ensure the extracted data aligned with our study requirements, we designed the data extraction form based on the predefined research questions (RQs). The first author initially extracted data from a subset of selected papers using this form, after which the extracted data was reviewed and verified by the other authors. Once validated, the first author used the refined form to extract data from the remaining studies. During data analysis and synthesis, multiple discussions were conducted to determine the most effective categorization and representation of the data, ensuring robust and meaningful conclusions.

\subsection{External Validity}
To address the threat to external validity, we employed a combination of automated and manual search strategies, adhering to widely accepted guidelines~\cite{kitchenham2009systematic, wohlin2014guidelines}. Our methodology section provides a detailed explanation of the inclusion and exclusion criteria. Specifically, we focused on peer-reviewed academic studies published in English, excluding grey literature, book chapters, opinion pieces, vision papers, and comparison studies. While these criteria may exclude some potentially relevant works, they were implemented to minimize bias in the selection process. We adopted a broad inclusion approach, considering studies regardless of their publication quality. Furthermore, our search encompassed publications from 1992 to the present, ensuring comprehensive coverage of advancements in the field of REDAST.