\subsection{Trend: General Results of the Publications}
\label{sec:Trend}
\begin{figure}
\centering
\begin{subfigure}{.5\textwidth}
  \centering
  \includegraphics[width=0.9\textwidth]{fig/rq0/Pub_Year.pdf}
  \caption{Study Distribution by Publication Year}
  \label{fig:year}
\end{subfigure}%
\begin{subfigure}{.5\textwidth}
  \centering
  \includegraphics[width=0.9\textwidth]{fig/rq0/Pub_Type.pdf}
  \caption{Study Distribution by Publication Type}
  \label{fig:dis_type}
\end{subfigure}
\caption{Study Distribution (Trend)}
\end{figure}

Before addressing the research questions, we present the publication trends of the 156 primary studies on REDAST, including study distribution, venue names, and features of the selected studies.

The study distribution is analyzed in two parts: (a) study distribution by the publication year and (b) study distribution by publication type, which are illustrated in Figures \ref{fig:year} and \ref{fig:dis_type}, respectively.

In analyzing the distribution of REDAST studies by year, we observed that the first study was published in 1993, followed by a steady annual increase. A notable surge occurred around 2008, prompting further investigation into the technical differences between studies conducted before and after this period. Two key conclusions emerged: (1) An increased adoption of NLP-pipeline-based methods from 2008 onward. Prior to 2008, only 20\% of studies employed these methods, whereas from 2008 to 2013, the proportion rose to 31.9\%. (2) A decline in the use of graph-based methods after 2008. The prevalence of graph-based approaches decreased from 35\% before 2008 to 19.14\% in the subsequent period from 2008 to 2013. By comparing with the landmark studies published between 2006 and 2008, including works by Hinton et al., \cite{hinton2006fast} and Van der Maaten and Hinton\cite{van2008visualizing}, we believe that deep learning technologies, including LLMs and Convolutional Neural Networks (CNNs), can promote the adoption of NLP-pipeline-based methods in SE domain.
\begin{table}[]
\small
\caption{Publication Venues with Two or More Studies in Selected Papers (Trend)}
\label{table:venue}
\begin{tabularx}{\textwidth}{Xlc}
\hline
\textbf{Venue Names} & \textbf{Type} & \textbf{Num.} \\ \hline
IEEE International Requirements Engineering Conference (RE) & Conference & \cellcolor{gray!70}6 \\
IEEE International Conference on Software Quality, Reliability, and Security (QRS) & Conference & \cellcolor{gray!70}6 \\
IEEE International Conference on Software Testing, Verification, and Validation (ICST) Workshops & Workshop & \cellcolor{gray!60}5 \\
IEEE Transactions on Software Engineering (TSE) & Journal & \cellcolor{gray!50}4 \\
Software Quality Journal (SQJ) & Journal & \cellcolor{gray!40}3 \\
Science of Computer Programming & Journal & \cellcolor{gray!40}3 \\
IEEE International Conference on Software Testing, Verification and Validation (ICST) & Conference & \cellcolor{gray!40}3 \\
International Conference on Quality Software (QSIC) & Conference & \cellcolor{gray!40}3 \\
International Conference on Evaluation of Novel Approaches to Software Engineering (ENASE) & Conference & \cellcolor{gray!40}3 \\
Innovations in Systems and Software Engineering & Journal & \cellcolor{gray!40}3 \\
IEEE International Workshop on Requirements Engineering and Testing (RET) & Workshop & \cellcolor{gray!40}3 \\
ACM SIGSOFT Software Engineering Notes & Journal & \cellcolor{gray!40}3 \\
Journal of Systems and Software (JSS) & Journal & \cellcolor{gray!30}2 \\
IEEE International Symposium on Software Reliability Engineering (ISSRE) & Conference & \cellcolor{gray!30}2 \\
IEEE International Requirements Engineering Conference Workshops (REW) & Workshop & \cellcolor{gray!30}2 \\
International Journal of System Assurance Engineering and Management & Journal & \cellcolor{gray!30}2 \\
International Conference on Enterprise Information Systems (ICEIS) & Conference & \cellcolor{gray!30}2 \\
International Conference on Emerging Trends in Engineering and Technology (ICETET) & Conference & \cellcolor{gray!30}2 \\
Electronic Notes in Theoretical Computer Science & Journal & \cellcolor{gray!30}2 \\
Australian Software Engineering Conference (ASWEC) & Conference & \cellcolor{gray!30}2 \\
International Journal of Advanced Computer Science and Applications & Journal & \cellcolor{gray!30}2 \\
Electronics & Journal & \cellcolor{gray!30}2 \\
\hline
\end{tabularx}
\end{table}
Regarding the publication type distribution, most studies were published in conferences (56.41\%) and journals (32.69\%). Our selected studies were published across 115 different venues. We present the venues with two or more published studies in Table \ref{table:venue}. The IEEE International Requirements Engineering Conference, the IEEE International Conference on Software Quality, Reliability, and Security, and the IEEE Transactions on Software Engineering have the highest percentages in their respective categories, which is unsurprising given their top-level reputation in the software engineering field.
\begin{table}[]
\small
\caption{Overview of Target Software in Selected Papers (Trend)}
\label{tab:software}
\begin{tabularx}{\textwidth}{lXc}
\hline
\textbf{Target Software Systems} & \textbf{Paper ID} & \textbf{Num.} \\ \hline
General Software & \citeP{P2, P3, P5, P6, P7, P8, P9, P10, P13, P14, P15, P16, P17, P18, P19, P23, P24, P25, P30, P31, P34, P35, P36, P37, P39, P40, P41, P42, P43, P46, P51, P52, P53, P54, P55, P57, P62, P63, P65, P68, P69, P70, P71, P73, P74, P75, P76, P78, P79, P82, P83, P84, P86, P88, P89, P92, P93, P96, P99, P100, P103, P104, P105, P106, P109, P114, P116, P117, P118, P119, P121, P122, P124, P125, P126, P127, P128, P129, P130, P132, P135, P136, P137, P138, P139, P141, P142, P143, P144, P145, P146, P148, P150, P151, P152, P153, P157, P158, P159, P160} & \cellcolor{gray!65}100 \\

Embedded System & \citeP{P29, P33, P38, P60, P61, P94, P98, P111, P113, P140, P147, P149, P155, P161} & \cellcolor{gray!60}14 \\

Web Services System & \citeP{P1, P4, P11, P21, P56, P67, P112} & \cellcolor{gray!55}7 \\

Safety-Critical System & \citeP{P27, P50, P72, P101, P123, P131} & \cellcolor{gray!50}6 \\

Timed Data-flow Reactive System & \citeP{P80, P108, P133, P134} & \cellcolor{gray!45}4 \\

Reactive System & \citeP{P45, P107, P156} & \cellcolor{gray!40}3 \\

Real-time Embedded System & \citeP{P26, P64, P115} & \cellcolor{gray!40}3 \\

Product Line System & \citeP{P32, P48} & \cellcolor{gray!30}2 \\

Object-Oriented System & \citeP{P28, P66} & \cellcolor{gray!30}2 \\

Telecommunication Application & \citeP{P22, P95} & \cellcolor{gray!30}2 \\

Automotive System & \citeP{P59, P102} & \cellcolor{gray!30}2 \\

Space Application & \citeP{P44} & \cellcolor{gray!15}1 \\

SOA-based System & \citeP{P90} & \cellcolor{gray!15}1 \\

Labeled Transition System & \citeP{P49} & \cellcolor{gray!15}1 \\

Healthcare Application & \citeP{P110} & \cellcolor{gray!15}1 \\

Event-driven System & \citeP{P154} & \cellcolor{gray!15}1 \\

Cyber-Physical System & \citeP{P58} & \cellcolor{gray!15}1 \\

Core Business System & \citeP{P20} & \cellcolor{gray!15}1 \\

Concurrent System & \citeP{P12} & \cellcolor{gray!15}1 \\

Complex Dynamic System & \citeP{P91} & \cellcolor{gray!15}1 \\

Aspect-Oriented Software & \citeP{P47} & \cellcolor{gray!15}1 \\

Agricultural Software & \citeP{P77} & \cellcolor{gray!15}1 \\

\hline
\end{tabularx}
\end{table}
Besides the trending information, we also investigated the target software in the selected studies. The results of the target software are exhibited in Table~\ref{tab:software}. Most of the selected studies are designed for general software (100/156), followed by embedded systems (14/156) web services systems (7/156), safety-critical systems (6/156), and so on.