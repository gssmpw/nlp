\subsection{RQ5: Limitation, Challenging and Future of REDAST Studies}

\subsubsection{Limitations in Selected Studies}
\begin{table}[]
\small
\caption{Limitation Results of Selected Studies (RQ5)}
\label{table:limitation}
\begin{tabularx}{\textwidth}{XXc}
\hline
\textbf{Limitations} & \textbf{Paper IDs} & \textbf{Num.} \\ \hline

Limitations of Framework Design & 
\citeP{P4, P8, P11, P12, P21, P22, P23, P28, P31, P32, P33, P35, P41, P44, P45, P46, P60, P62, P64, P67, P68, P86, P89, P91, P92, P102, P106, P109, P110, P117, P118, P121, P126, P131, P145, P146, P148, P150, P151, P156, P160} 
& \cellcolor{gray!70}41 \\

NA
& 
\citeP{P14, P15, P16, P18, P26, P29, P83, P90, P93, P95, P98, P99, P101, P114, P119, P123, P124, P125, P133, P134, P137, P138, P139, P140, P141, P142, P158, P161} 
& \cellcolor{gray!65}28 \\

Limitation of Evaluation or Demonstration & 
\citeP{P7, P19, P39, P40, P48, P74, P75, P96, P100, P102, P105, P108, P110, P122, P129, P132, P143, P146, P153, P160} 
& \cellcolor{gray!60}20 \\

Scalability to Large Systems & 
\citeP{P10, P30, P32, P35, P38, P47, P50, P55, P56, P66, P70, P75, P78, P80, P84, P107, P111, P126, P128, P159} 
& \cellcolor{gray!55}20 \\

Over Relying on Input Quality & 
\citeP{P2, P17, P34, P36, P37, P43, P44, P51, P53, P55, P65, P71, P107, P127, P144, P157} 
& \cellcolor{gray!50}16 \\

Complexity of Methodology & 
\citeP{P3, P54, P62, P69, P72, P73, P76, P77, P79, P82, P94, P116, P147, P149, P154} 
& \cellcolor{gray!45}15 \\

Automation (Methodology) & 
\citeP{P6, P8, P24, P27, P38, P57, P58, P59, P107, P112, P115, P116} 
& \cellcolor{gray!40}12 \\

Incomplete Requirements Coverage & 
\citeP{P13, P61, P88, P105, P108, P113, P131, P136, P155, P156} 
& \cellcolor{gray!40}10 \\

Automation (Test Implementation) & 
\citeP{P2, P5, P25, P73, P104, P122, P130, P135} 
& \cellcolor{gray!35}8 \\

Automation (Requirements Specification) & 
\citeP{P1, P9, P68, P103, P117, P151, P152} 
& \cellcolor{gray!30}7 \\

Requirements Ambiguities & 
\citeP{P42, P49, P52} 
& \cellcolor{gray!25}3 \\

Limitation of Implementation & 
\citeP{P20, P63} 
& \cellcolor{gray!20}2 \\

Time-Costing & 
\citeP{P20, P39} 
& \cellcolor{gray!20}2 \\

Additional Cost of Requirements Specification & 
\citeP{P7} 
& \cellcolor{gray!15}1 \\

\hline
\end{tabularx}
\end{table}
In the selected papers, we observed that some studies explicitly discuss their limitations. This section presents the identified limitations, categorized into 14 concise types, as summarized in Table \ref{table:limitation}. Notably, 28 studies do not explicitly mention any limitations in their content. Consequently, the following discussion primarily focuses on the remaining papers that explicitly acknowledge their limitations.

\begin{table}[]
\small
\caption{Automation Level of Selected Studies (RQ5)}
\label{table:automation}
\begin{tabularx}{\textwidth}{lXc}
\hline
\textbf{Automation Levels} & \textbf{Paper IDs} & \textbf{Num.} \\ \hline

Highly Automated & 
\citeP{P2, P3, P4, P5, P6, P7, P8, P9, P10, P12, P16, P17, P18, P20, P22, P23, P24, P28, P29, P32, P41, P42, P43, P44, P47, P48, P49, P51, P53, P54, P56, P57, P59, P60, P69, P70, P71, P74, P75, P77, P78, P82, P83, P89, P91, P92, P93, P98, P100, P101, P102, P103, P105, P106, P107, P108, P109, P111, P112, P117, P124, P125, P126, P127, P132, P133, P134, P136, P137, P140, P145, P146, P149, P151, P153, P154, P160} 
& \cellcolor{gray!65}77 \\

Semi-Automated & 
\citeP{P19, P26, P27, P38, P46, P58, P63, P64, P68, P84, P86, P90, P94, P95, P99, P104, P110, P113, P114, P115, P116, P118, P119, P121, P122, P129, P130, P131, P135, P138, P139, P141, P142, P148, P150, P152, P155, P156, P161} 
& \cellcolor{gray!40}39 \\

Fully Automated & 
\citeP{P1, P11, P13, P14, P15, P21, P25, P30, P31, P33, P34, P35, P36, P37, P39, P40, P45, P50, P52, P55, P61, P62, P65, P66, P67, P72, P73, P76, P79, P80, P128, P143, P144, P147, P157} 
& \cellcolor{gray!30}35 \\

Low Automation & 
\citeP{P88, P96, P123, P158, P159} 
& \cellcolor{gray!15}5 \\

\hline
\end{tabularx}
\end{table}
Automation is a frequently mentioned limitation in the selected studies. We identified that automation limitations vary across three key areas: requirements specification, methodology conduction, and test implementation. To provide a more detailed analysis, we classified these limitations into three categories: Automation (Requirements Specification), Automation (Methodology), and Automation (Test Implementation). Given that the automation level is a critical aspect of REDAST, we present the estimated automation levels for all the selected papers in Table \ref{table:automation}, as four levels, Fully automated - End-to-End Automation, highly automated - Automation-Dominant, semi-automated - Automation Supported, and low automated - Minimal Automation. The detailed definitions of automation are depicted in Section~\ref{sec:rq5_taxonomy}.

Scalability, framework design, incomplete requirements coverage, and requirements ambiguities are commonly identified limitations in the reviewed studies. Specifically, 41 studies reported incompatibilities with handling certain scenarios, 20 studies highlighted challenges in scaling to complex or larger systems, 10 studies acknowledged an inability to cover all requirements during test generation, and 3 studies discussed potential ambiguities in requirements. These limitations often stem from framework structure issues, where the methods fail to comprehensively account for diverse usage scenarios, leading to problems with generalization and applicability. For example, P38~\citeP{P38} and P48~\citeP{P48} discuss challenges with generalization due to difficulties in handling complex systems, while P124~\citeP{P129} highlights performance gaps when dealing with systems of varying sizes. Framework design limitations also constrain methods in specific contexts, as evidenced by P146’s~\citeP{P151} discussion of incompatibilities with non-functional requirements and P141’s~\citeP{P146} primary focus on extra-functional properties.

Additional limitations arise from the complexity of some framework components, including over-reliance on input quality, methodological complexity, and time inefficiencies. Over-reliance occurs when predefined rules or input formats disproportionately influence the performance of test generation, making the process vulnerable to input quality issues. For instance, P44~\citeP{P44} notes that dependency relations can affect test generation accuracy, while P103~\citeP{P107} highlights challenges arising from the conjunctive statement format, which complicates stable test generation. Methodological complexity and time-cost issues stem from the algorithms employed in these frameworks, which significantly increase the difficulty and runtime of the processes. For example, P79~\citeP{P79} mentions that the complexity of the test generation and analysis processes impacts performance and scalability, and P20~\citeP{P20} critiques the Specmate technique for its excessive complexity, which undermines runtime efficiency.

Furthermore, 20 studies identified limitations in evaluation or demonstration, emphasizing that their experiments were insufficient to validate the efficacy of methodologies in other scenarios, particularly from an industrial perspective (e.g., P7~\citeP{P7}, P19~\citeP{P19}, P74~\citeP{P74}).

\subsubsection{Insight Future View from Selected Studies}
\begin{table}[]
\small
\caption{Future Direction Results of Selected Studies (RQ5)}
\label{table:future}
\begin{tabularx}{\textwidth}{lXc}
\hline
\textbf{Future Directions} & \textbf{Paper IDs} & \textbf{Num.} \\ \hline

Extension to Other Coverage Criteria or Requirements & 
\citeP{P5, P11, P13, P14, P15, P16, P17, P18, P21, P22, P23, P28, P30, P32, P34, P35, P36, P42, P44, P45, P46, P47, P48, P49, P50, P51, P53, P54, P55, P57, P58, P59, P60, P61, P62, P65, P66, P67, P68, P73, P75, P78, P86, P92, P95, P98, P100, P101, P102, P103, P104, P105, P106, P107, P108, P110, P129, P130, P132, P133, P134, P137, P145, P146, P147, P148, P150, P151, P152, P158, P159, P160} 
& \cellcolor{gray!65}72 \\

Further Validation & 
\citeP{P5, P6, P9, P14, P15, P19, P20, P21, P26, P38, P48, P54, P59, P72, P74, P75, P86, P89, P93, P96, P101, P108, P109, P110, P113, P115, P118, P119, P121, P125, P126, P127, P138, P140, P142, P143, P144, P145, P146, P156, P159, P160} 
& \cellcolor{gray!55}42 \\

Completeness Improvement & 
\citeP{P24, P35, P36, P40, P45, P51, P52, P69, P71, P79, P80, P89, P92, P93, P95, P96, P112, P117, P119, P125, P126, P128, P129, P131, P134, P138, P140, P142, P149, P154, P155, P157} 
& \cellcolor{gray!45}32 \\

Automation Improvement & 
\citeP{P4, P5, P6, P10, P11, P12, P46, P53, P56, P58, P67, P72, P73, P76, P88, P95, P104, P106, P115, P139, P144, P151, P152, P154} 
& \cellcolor{gray!30}24 \\

Extension of Other Techniques & 
\citeP{P9, P19, P20, P22, P27, P30, P31, P33, P34, P37, P43, P44, P63, P70, P71, P76, P77, P88, P94, P127, P128, P147, P153, P157} 
& \cellcolor{gray!30}24 \\

Extension to Other Domains or Systems & 
\citeP{P8, P10, P24, P26, P28, P29, P33, P41, P42, P47, P56, P61, P63, P64, P66, P69, P73, P77, P79, P80, P100, P102, P107, P135} 
& \cellcolor{gray!25}24 \\

Extension to Other Phases or Test Patterns & 
\citeP{P1, P2, P7, P12, P16, P21, P25, P26, P37, P51, P52, P57, P62, P65, P74, P98, P105, P106, P111, P112, P114, P137, P153, P156} 
& \cellcolor{gray!25}24 \\

Performance Improvement & 
\citeP{P4, P27, P42, P52, P94, P106, P107, P130, P132, P133, P138, P143, P149, P155} 
& \cellcolor{gray!20}14 \\

Real Tool Development & 
\citeP{P1, P7, P13, P54, P56, P64, P78, P110, P112, P113, P114, P116, P117, P140} 
& \cellcolor{gray!20}14 \\

Benchmark Construction & 
\citeP{P3, P39, P116, P139} 
& \cellcolor{gray!10}4 \\

Traceability Improvement & 
\citeP{P31, P39, P121, P137} 
& \cellcolor{gray!10}4 \\

Robustness Improvement & 
\citeP{P18, P118, P141} 
& \cellcolor{gray!10}3 \\

NA & 
\citeP{P82, P83, P84, P90, P91, P99, P122, P123, P124, P136, P161} 
& \cellcolor{gray!10}11 \\

\hline
\end{tabularx}
\end{table}
This section aims to discuss the key challenges and directions for future research identified in the selected studies. Notably, these studies share common themes regarding challenges and proposed future work, including improving or extending existing methodologies, conducting further evaluations, and addressing unresolved issues. To provide a structured analysis, the challenges and future work are categorized into several different areas. We illustrate the results in Table~\ref{table:future}.

Extensions to other coverage criteria or requirements, test phases or patterns, and domains or systems are the three most commonly identified future directions and challenges in the reviewed studies. These directions correspond to limitations in framework design, as most frameworks are unable to address all usage scenarios, such as requirements coverage, test phases, or diverse software systems. Consequently, many studies propose expanding their scope to cover additional scenarios. For instance, P67~\citeP{P67} plans to emphasize non-functional requirements in their test generation process, rather than focusing solely on functional requirements. P65~\citeP{P65}, having designed a method for generating test scenarios, intends to broaden their approach to encompass other phases of software development or testing. Similarly, P25~\citeP{P25}, which focuses on acceptance test case generation for embedded systems, plans to extend their application beyond embedded systems to enhance generalization.

Although limitations in automation were discussed in the previous section, several papers propose future plans to reduce human intervention in the test generation process. We identified 24 papers outlining plans to improve automation levels (e.g., P67~\citeP{P67}, P72~\citeP{P72}, among others).

Robustness, completeness, and traceability are widely regarded as critical factors in automated software testing research. While it is challenging to qualitatively assess these factors for a REDAST method, certain steps can be identified that may negatively impact framework robustness, completeness, or traceability. In the selected papers, some studies explicitly discuss their future directions for improving these aspects. For example, P114~\citeP{P118} plans to enhance fault-handling diversity to strengthen framework robustness. P93~\citeP{P96} aims to add a user input monitoring function as part of future work to improve completeness. Meanwhile, P116~\citeP{P121} intends to further investigate the impact of requirement changes within the REDAST process, enabling better traceability between requirements and test artifacts.

Benchmark construction, real tool development, and further validation reflect researchers’ efforts to broaden the impact of their studies. By enhancing the post-generation environment, the potential capabilities of these studies can be more thoroughly explored. For instance, P3~\citeP{P3} plans to build benchmark test suites that are independent of specific model-based testing languages, which could significantly advance future research in related fields. P56~\citeP{P56} aims to investigate opportunities for integrating their approach into larger industrial applications to enhance work efficiency. Additionally, recognizing that their current case study is insufficient to fully demonstrate the methodology’s efficacy, P84~\citeP{P86} intends to conduct more comprehensive case studies to further evaluate the performance of their method.

Some studies discuss the extension of existing techniques or performance improvements in their work, often focusing on introducing new techniques or frameworks to enhance test generation performance. For example, P19~\citeP{P19} plans to expand NLP capabilities and explore more advanced automation techniques as part of their future work. Similarly, P91~\citeP{P94} intends to reduce the state space of the studied model to improve the efficiency and effectiveness of their test case generation methods.

In the result, we identified that most of the studies pose an extension to other coverage criteria or other requirements (72 papers), followed by further validation (42 papers), improvement of the completeness in their future work (32 papers), and so on, which matches the results in the limitation results. 

\subsubsection{Findings: Cross-Analysis of Demonstration Types and Validation Challenges}

% Please add the following required packages to your document preamble:
% \usepackage{graphicx}
\begin{table}[]
\small
\caption{Cross Distribution of Demonstration Types and Validation Related Future Direction or Limitation}
\label{tab:demo_limi}
\begin{tabularx}{\textwidth}{Xcccc}
\hline
                                          & \textbf{Conceptual Case Study} & \textbf{Real Case Study} & \textbf{NA} & \textbf{Dataset Evaluation} \\ \hline
\textbf{Further Validation}                        & 15                    & 22              & 5  & 0                  \\
\textbf{Limitation of Demonstration or Evaluation} & 14                    & 13              & 3  & 1                  \\ \hline
\end{tabularx}%
\end{table}

In analyzing the limitations and future directions of existing studies, we found that many studies identify evaluation or demonstration constraints and emphasize the need for further validation. This finding highlights the distribution of different demonstration types among studies that acknowledge validation-related limitations and future research directions. From our results in Table~\ref{tab:demo_limi}, 15 out of 69 papers that adopted conceptual case studies reported validation-related issues in their limitations or future directions, while 14 out of 69 identified similar concerns. Similarly, 22 out of 66 papers that employed real case studies reported validation-related issues in their limitations, with 13 out of 66 mentioning them in their future research directions. These findings suggest that validation challenges persist regardless of the chosen demonstration method. Additionally, we observed that papers employing real case studies more frequently reported validation-related issues. We hypothesize that this may be because conceptual case studies, being designed specifically for demonstration purposes, can more effectively represent the REDAST process in a controlled manner.

\subsubsection{Findings: Trend of Automation Level Over the Years}
\begin{figure}
    \centering
    \includegraphics[width=1\linewidth]{fig//rq5/RQ5_auto_year.pdf}
    \caption{Automation Level by Year}
    \label{fig:auto_year}
\end{figure}

As a practical and impact-driven research area, REDAST places significant emphasis on automation, particularly in the context of industrial applications. Given that technological advancements contribute to automation in REDAST, we analyze the trend of automation levels over time, as illustrated in Fig.~\ref{fig:auto_year}. For clarity, we classify automation levels into two categories: fully and highly automated systems, which represent a high level of automation, and semi- and low-automated systems, which indicate a lower level of automation. Our findings reveal that while the proportion of low and semi-automated approaches has gradually declined, fully and highly automated methods have become increasingly dominant over the years. This trend suggests that technological advancements have progressively improved the automation level in REDAST studies. However, despite this overall improvement, the proportion of fully automated systems has not increased significantly. We attribute this to a key limiting factor: while technological advancements enhance automation capabilities, achieving full automation still requires an appropriately designed framework. This aspect is largely independent of technological progress and instead relies on methodological and architectural considerations in REDAST research.


\begin{tcolorbox}[mybox, breakable, title=RQ5 Key Takeaways]

$\bullet$ The framework design improvement and extension is the most common limitation and future direction reported in REDAST studies. The selected papers frequently mention better coverage of the usage scenario and model configurations.

$\bullet$ Most REDAST studies are not fully automated, with some human invention still necessary in some key steps.

$\bullet$ Extension of evaluation and demonstration is always considered in the future directions. REDAST studies are strongly related to real applications, and hence, the evaluation and demonstration have a key future direction and need for evaluation in real practical settings.

\end{tcolorbox}