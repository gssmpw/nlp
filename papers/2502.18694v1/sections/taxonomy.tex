\section{Taxonomy}~\label{sec:Taxonomy}

\begin{figure}[!t]
    \centering
    \includegraphics[width=1\linewidth]{fig/taxonomy/Total_Taxonomy.pdf}
    \caption{Overview of Taxonomy Schema in REDAST Studies}
    \label{fig:taxonomy}
\end{figure}

%\fw{I create similar taxonomy figures for RQ2, 3 and 4.}
In literature review studies, the taxonomy schema plays a crucial role in shaping the quality of statistical analysis and addressing research questions. Aware that the complexity of the test generation process will lead to confusion in our results, we define a four-stage schema for our REDAST process based on our literature analysis and informed from Fig.~\ref{fig:taxonomy}. In each of the schemas, recognizing that the entire SE life cycle is a practical process, we incorporated multiple categorizations to enhance the structure and clarity of the schemas.

\begin{figure}[!t]
    \centering
    \includegraphics[width=0.75\linewidth]{fig/taxonomy/RQ1_Taxonomy.pdf}
    \caption{RQ1 Taxonomy - Requirements Input in REDAST Studies}
    \label{fig:RQ1_taxonomy}
\end{figure}

\subsection{Designing of Taxonomy Schema}

Several studies have discussed the taxonomy categories in the RE and software testing domains in previous surveys. However, these existing schemas do not provide a comprehensive view of the REDAST process, especially for the detailed transforming process from requirements to test artifacts. In order to address the gap, we propose a hierarchical structure that encompasses (1) requirements input, (2) technical methodology, (3) test outcomes, and (4) results evaluation by referring to some related studies~\cite{redast_survey,align_req_test_01}. The overview of the taxonomy schema in our literature review is illustrated in Fig. \ref{fig:taxonomy}, where we not only exhibit the taxonomy schema in this figure but also indicate the REDAST procedure.



\subsection{Requirements Input Category}
Requirements are the necessary input of the REDAST process. The formulation of requirements input will decisively affect the choices of the following steps, including processing technology, framework designing, and so on. In the Requirements input section (taxonomy in Fig.~\ref{fig:RQ1_taxonomy}), we focus on the formulation of requirements based on the schema outlined in~\cite{Pohl:11, wagner2019status}, which corresponds to RQ1 about requirements. However, as we mentioned, there is no universal requirements categorization. Thus, we adopted multiple categorizations in RQ1. The requirements type consists of (1)
functional requirements, (2) non-functional requirements, (3) business requirements, (4) user requirements, (5)
constraint requirements, and (6) implementation requirements, which describe the covering scale of requirements~\cite{requirements_terms,requirements_type}. Requirements format is the second categorization in requirements, which includes constraint-based, executable, formal (mathematical), goal-oriented, model-based, scenario-based, tabular (matrix-based), and textual specifications~\cite{scenario_requirements, tabular_requirements, goal_requirements, model_requirements, nl_requirements}. The format categorization of requirements specifications represents how requirements can be structured, documented, and communicated based on their intended purpose and context. The last categorization in RQ1 concerns the specific notation of requirements specifications, such as use case, NL specification, and SysML, which aims to illustrate the adoption trend in requirements notation. These schemas are expected to cover a broad range of requirements and provide a comprehensive view of the requirements input. Note that our study focuses specifically on requirements-related aspects, deliberately excluding papers focused on design-level information. For instance, Yang et al.~\cite{generating2021yang} proposed an automated test scenario generation method using SysML for modeling system behavior in the system design phase. Although SysML is a commonly used specification format for requirements and system architecture, this paper was excluded from our literature review due to its emphasis on system design rather than system requirements.

\subsection{Transformation Techniques Category}

\begin{figure}[!t]
    \centering
    \includegraphics[width=0.75\linewidth]{fig/taxonomy/RQ2_Taxonomy.pdf}
    \caption{RQ2 Taxonomy - Transformation Techniques in REDAST Studies}
    \label{fig:RQ2_taxonomy}
\end{figure}

The transformation techniques support the generation process from requirements to test artifacts, wherein we attempt to analyze the details of the methodologies employed. Fig.~\ref{fig:RQ2_taxonomy} shows the taxonomy schema of RQ2. We identified the following aspects,
\begin{itemize}
    \item \emph{Transformation Techniques}. From requirements to generated test artifacts, the transformation techniques are expected to transfer requirements to readable, understandable, and generation-friendly artifacts for test generation. However, varying from different usage scenarios, various types of techniques are adopted in the transformation framework. Even though there are only limited studies explicitly discussing the transformation techniques for REDAST, we referred to the survey in related fields~\cite{transformation_tech_01, transformation_tech_02, transformation_tech_03, transformation_tech_04}, such as automated software testing and software generation, to finalize our schema for transformation techniques. The categorization for transformation techniques can be formulated as five categories: 
    (1) \textbf{rule-based} techniques rely on predefined templates or rules to formulate requirements to test artefacts;
    (2) \textbf{meta-model-based} techniques employ the meta-models to define the behavior, structure, relationships, or constraints to enable enhanced expression ability; 
    (3) \textbf{graph-based} techniques mainly use the graph as representation (e.g., state-transition graphs, dependency graphs) with traversing or analyzing on paths, nodes, or conditions;
    (4) \textbf{natural language processing pipeline-based} techniques focus on leveraging open-source NLP tools for REDAST, like text segmentation and syntax analysis; 
    (5) \textbf{machine learning-based} techniques leverage ML (including deep learning) in the REDAST process, which always involves the patterns or feature learning process using training data. 
    
    \item \emph{Intermediate Representations} are related to the optional steps in the generation framework. Some papers employ a stepwise transformation approach instead of directly transforming requirements into test artifacts. This approach generates intermediate artifacts that enhance the traceability and explainability of the methodology. For example, the unstructured NL requirements could be transformed into an intermediate more structured representation that facilitates the generation of test artifacts. While intermediate representations are derived from requirements, we employed a categorization method for representation types similar to that used in requirements schemas. %Additionally, we introduced a categorization for the decomposition granularity of the intermediate representations. The granularity consists of three categories: (1) atomic decomposition indicates that the intermediate representations can offer a single and self-contained description that does not necessarily depend on other requirements or testing specifications (e.g., Shall Requirements: The login page shall allow users to enter a username and password.); (2) hierarchical decomposition enables a high-level definition for the intermediate representation, where each representation can be decomposed into more specific sub-components (e.g., The e-commerce platform shall support secure user - 1. The platform shall allow users to reset password; 2. The platform shall allow users to modify username.), (3) composite decompositions consist of multiple actions or behaviors in a sequence of related specifications, which is contradicted with atomic decomposition (e.g., User Story: The system shall allow users to 1. login, 2. view their profile, and 3. edit their account details). \ca{This needs to be very clearly defined - very difficult for any reader to gauge what this means.} \fw{I use a more detailed definition for this category.}
    \item \emph{Additional Inputs}. In addition to simply using requirements as input, some frameworks accommodate additional input types, such as supporting documents, user preferences, and more. To analyze these frameworks from the perspective of input composition, we introduce additional inputs that categorize and examine the variety of inputs utilized. 
    \item \emph{Framework Structure} refers to the underlying architectural approach used to transform requirements into test artifacts. It determines how different stages of transformation interact. Within the REDAST framework, transformation methodologies are categorized into four distinct structures (\cite{framework_structure_01,framework_structure_02}): (1) \textbf{Sequential} – Follows a strict, ordered sequence of transformation steps, maintaining logical continuity without deviations. Each step builds upon the previous one;
    (2) \textbf{Conditional} – Introduces decision points that enable alternative transformation paths based on specific conditions, increasing adaptability to varying requirements;
    (3) \textbf{Parallel} – Allows simultaneous processing of different representations across multiple transformation units, significantly improving efficiency;
    (4) \textbf{Loop} – Incorporates iterative cycles for continuous refinement, ensuring enhanced quality through repeated validation and adjustment.
\end{itemize}
Thus, based on the above aspects, we plan to introduce two aspects in RQ2: transformation techniques and framework design, where the input portion, framework structure, and intermediate representation are included in the framework design. In our review, we also explored the advantages and disadvantages of various techniques, with a particular emphasis on recent advancements in LLMs. This category is related to RQ2.

\subsection{Test Artifacts Category}
\begin{figure}[!t]
    \centering
    \includegraphics[width=0.75\linewidth]{fig/taxonomy/RQ3_Taxonomy.pdf}
    \caption{RQ3 Taxonomy - Test Artifacts in REDAST Studies}
    \label{fig:RQ3_taxonomy}
\end{figure}
In the \textit{Test Artifacts} section, we aim to focus on the formulation of generated testing artifacts. The illustration of RQ3 is shown in Fig.~\ref{fig:RQ3_taxonomy}. The branches within the test artifacts category, such as test format, notation, and coverage, have been explored in previous surveys~\cite{test_taxonomy_01, test_taxonomy_02, test_taxonomy_03}. Additionally, we introduce a new categorization based on the abstraction level of test artifacts, specifically designed for the REDAST process. While the generated test artifacts are typically applied to system testing and acceptance testing~\cite{test_standard}, they commonly include code, test descriptions, or test reports. The abstraction level categorization includes three categories: abstract, concrete, and report, which refers to the test artifacts in the system and acceptance testing. (1) Abstract test artifacts cannot be directly executed but provide enhanced traceability and coverage for requirements. (2) Executable test artifacts are executable, with multiple concrete artifacts often corresponding to a single requirement. (3) Report-level artifacts represent the outcomes of executing the test artifacts. The results of the test artifact-related categories are presented in response to RQ3.

\subsection{Results Demonstration Category}
\begin{figure}[!t]
    \centering
    \includegraphics[width=0.75\linewidth]{fig/taxonomy/RQ4_Taxonomy.pdf}
    \caption{RQ4 Taxonomy - Demonstration Methods in REDAST Studies}
    \label{fig:RQ4_taxonomy}
\end{figure}
REDAST always introduces case demonstration or dataset evaluation to assess the quality of the generated test artifacts. The taxonomy of RQ4 is exhibited in Fig.~\ref{fig:RQ4_taxonomy} Here, we include evaluation as a separate category in the taxonomy schema to obtain some results about the quality assessment criteria of test artifacts. Specifically, we planned to (1) conclude the evaluation methods used in relevant studies, (2) categorize the software platforms in the demonstration, and (2) report typical examples and analyze their efficacy based on their usage scenarios, pros, and cons, where \cite{evaluation_taxonomy} was opted as the guideline for designing the schema of this category. We introduce demonstration types and software platforms to illustrate the details of the demonstration method. As for the software platform, this categorization is introduced to identify the software platform adopted in the case demonstrations. Besides, we introduced a categorization for usability in evaluation, where we will manually evaluate the selected studies and illustrate their results for different parts, including methodology explanation, discussion, case example, and experiment.

\subsection{Future and Limitation Category}
\label{sec:rq5_taxonomy}
\begin{figure}[!t]
    \centering
    \includegraphics[width=0.75\linewidth]{fig/taxonomy/RQ5_Taxonomy.pdf}
    \caption{RQ5 Taxonomy - Future and Limitations in REDAST Studies}
    \label{fig:RQ5_taxonomy}
\end{figure}

RQ5 primarily examines the limitations and future directions of REDAST studies, which are illustrated in Fig.~\ref{fig:RQ5_taxonomy}. The categorizations for these aspects were determined post hoc, based on our analysis of the results; therefore, detailed categorization methods will not be presented. Additionally, considering the importance of automation in REDAST studies, we provide an analysis of the automation levels observed in the selected studies. Four levels of automation are defined as follows:
\begin{itemize}
    \item \textit{Fully Automated (End-to-End Automation)}: Studies in this category require no human intervention or operation.
    \item \textit{Highly Automated (Automation-Dominant)}: These studies demonstrate a high degree of automation, with human intervention incorporated into the methodology but not essential.
    \item \textit{Semi-Automated (Automation-Supported)}: This level involves significant manual operations at specific stages of the process.
    \item \textit{Low Automated (Minimal Automation)}: Studies at this level exhibit only basic automation, relying primarily on manual operations across all steps.
\end{itemize}