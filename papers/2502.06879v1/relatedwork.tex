\section{Related Work}
\label{subsec:related_work}
There has been a significant amount of research on graph clustering; we refer the reader to~\cite{FORTUNATO201075,HamannMapEq,watteau2024advancedgraphclusteringmethods} for thorough reviews of contributions in this field. Here, we survey some graph clustering algorithms relevant to our contribution. 

Several successful algorithms have been developed for prominent applications of graph clustering. The \textsc{Louvain} method, introduced by Blondel et al.~\cite{louvain}, is a multi-level clustering algorithm that optimizes modularity as its objective function. The \textsc{Leiden} algorithm~\cite{Traag2019} provides adaptations to the \textsc{Louvain} method. \textsc{VieClus} also optimizes modularity as its objective function, but uses the heuristics of evolutionary algorithms to tackle the graph clustering problem~\cite{vieclus}.     
Spectral clustering~\cite{vonLuxburg2007} partitions graph data into clusters using the eigenvalues and eigenvectors of a similarity matrix and, along with $k$-means/$k$-median/$k$-center methods~\cite{JIANG2023691,wanclust2017}, is particularly effective when the desired number of clusters is predefined. Stochastic Block Models (SBM)~\cite{JIANG2023691,Lee2019} are probabilistic models that analyze graphs with underlying (latent) structures, where nodes are grouped into blocks, and edge density between blocks is determined by probability distribution. Markov Clustering (MCL)~\cite{dongen2008} efficiently clusters graphs by leveraging random walks and Markov chain properties to identify densely connected portions of a graph. \textsc{DBSCAN}~\cite{apachegiraph} and correlation clustering~\cite{Bansal2004} excel at detecting anomalously dense clusters. 
Additionally, several distributed graph clustering algorithms have been developed, including frameworks like \textsc{Pregel}/\textsc{Giraph}~\cite{pregel,apachegiraph} and \textsc{MapReduce}~\cite{mapreduce}. \textsc{TeraHAC}~\cite{terahac} is a distributed algorithm for hierarchical clustering that addresses the runtime bottlenecks of other hierarchical clustering algorithms. Some recent approaches to graph clustering draw on deep learning and graph neural networks~\cite{ijcai2020p693,liu2023surveydeepgraphclustering,su2024surveydeeplearningcommunity,wangdeepgraphnodeclust2024}.

While there are several successful approaches to the graph clustering problem, most state-of-the-art clustering algorithms are in-memory algorithms, with limited research undertaken on streaming graph clustering. Hollocou et al.~\cite{hollocou} introduce a streaming graph clustering algorithm that reads edge streams and assigns nodes to clusters on-the-fly. When streaming an edge $e=(u,v)$, they either (a) assign $u$ to the cluster of $v$, (b) assign $v$ to the cluster of $u$ or (c) leave both in their respective cluster, attempting to optimize for modularity. Although this algorithm requires very few computational resources compared to state-of-the-art in-memory algorithms, its solution quality is lower due to the absence of global knowledge of the graph and further impacted by sub-optimal cluster assignment decisions. Of related interest, Assadi et al.~\cite{assadi2022hierarchical} present a theoretical proof for the runtime and memory complexity of a streaming algorithm for the distinct hierarchical clustering problem.