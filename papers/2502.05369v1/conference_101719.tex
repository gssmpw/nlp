\documentclass[conference]{IEEEtran}
\IEEEoverridecommandlockouts
% The preceding line is only needed to identify funding in the first footnote. If that is unneeded, please comment it out.
\usepackage{cite}
\usepackage{amsmath,amssymb,amsfonts}
\usepackage{textcomp}
\usepackage[ruled,vlined]{algorithm2e}
\usepackage{algpseudocode}
\usepackage{MnSymbol}
\usepackage{wasysym}
\usepackage{multirow}
\usepackage{subcaption}
\usepackage{enumitem}
\usepackage{graphicx} 
\usepackage{tikz}
\usepackage{pdfx}
\newcommand*\circled[1]{\tikz[baseline=(char.base)]{
            \node[shape=circle,draw,inner sep=0.5pt] (char) {#1};}}


\def\BibTeX{{\rm B\kern-.05em{\sc i\kern-.025em b}\kern-.08em
    T\kern-.1667em\lower.7ex\hbox{E}\kern-.125emX}}

\usepackage{cite}

\usepackage[utf8]{inputenc}
\usepackage[T1]{fontenc}
\usepackage{microtype}
\usepackage{graphicx}
\usepackage{balance}  %
\newcommand\norm[1]{\left\lVert#1\right\rVert}
\usepackage{multirow}
\usepackage{color}
\usepackage{listings}
\usepackage{float}
\usepackage{outlines}
\usepackage[normalem]{ulem}
\usepackage{cleveref}
\usepackage{enumitem}
\usepackage{soul}
\usepackage{cuted, lipsum}
\usepackage[listings,skins,breakable]{tcolorbox}
\usepackage{booktabs}
\PassOptionsToPackage{rgb,hyperref,table}{xcolor}
\usepackage{xcolor}
\usepackage{colortbl}
\usepackage{marginnote}
\usepackage{lineno}
\usepackage{dirtytalk}

\begin{document}
\SetAlgoNlRelativeSize{-1}
\SetNlSty{textbf}{}{}
\LinesNumbered
\title{\texttt{DobLIX}: A Dual-Objective Learned Index for Log-Structured Merge Trees}


\author{\IEEEauthorblockN{1\textsuperscript{st} Alireza Heidari}
\IEEEauthorblockA{\textit{Huawei} \\
alireza.heidarikhazaei@huawei.com}
\and
\IEEEauthorblockN{2\textsuperscript{nd} Amirhossein Ahmadi}
\IEEEauthorblockA{\textit{Huawei} \\
amirhossein.ahmadi@huawei.com}
\and
\IEEEauthorblockN{3\textsuperscript{rd} Wei Zhang
}
\IEEEauthorblockA{\textit{Huawei} \\
wei.zhang6@huawei.com}
}

\maketitle

\begin{abstract}
In this paper, we introduce \texttt{DobLIX}, a dual-objective learned index specifically designed for Log-Structured Merge (LSM) tree-based key-value stores. Although traditional learned indexes focus exclusively on optimizing index lookups, they often overlook the impact of data access from storage, resulting in performance bottlenecks. DobLIX addresses this by incorporating a second objective, data access optimization, into the learned index training process. This dual-objective approach ensures that both index lookup efficiency and data access costs are minimized, leading to significant improvements in read performance while maintaining write efficiency in real-world LSM-tree systems. Additionally, \texttt{DobLIX} features a reinforcement learning agent that dynamically tunes the system parameters, allowing it to adapt to varying workloads in real-time. Experimental results using real-world datasets demonstrate that \texttt{DobLIX} reduces indexing overhead and improves throughput by $1.19\times$ to $2.21\times$ compared to state-of-the-art methods within RocksDB, a widely used LSM-tree-based storage engine.
\end{abstract}

\documentclass[../main.tex]{subfiles}
\graphicspath{{../images/}}
\makeatletter
\def\input@path{{../images/}}
\makeatother
\begin{document}
\section{Introduction}
\begin{figure}
\centering
\begin{tikzpicture}
\node[inner sep=0pt] (ws) at (0, 0) {
\includegraphics[height=.4\textwidth, trim={10cm 0 10cm 0},clip]{world_space.png}};
\node[inner sep=0pt] (cs) at (6,0) {\includegraphics[height=.4\textwidth, trim={10cm 1cm 10cm 4cm},clip]{conf_space.png}};
\end{tikzpicture}
\vspace{-5pt}
\label{fig:pbrm_intro}
\caption{\textbf{Left}: Shows world space obstacles as grey spheres. Robots start and goal configuration is colored red and green, respectively. Configurations along the computed path are colored transparent blue. \textbf{Right:} Mapped world space scenario to configuration space. Obstacle region is the grey mesh. Red spheres are collision-free regions computed by the neural SCDF. The optimized shortest path in the convex corridor is the blue curve.}
\vspace{-25pt}
\end{figure}
Motion planning is the problem of finding a collision-free trajectory that connects a given start and goal configuration. The planning takes place in the configuration space of the robot. For single body robots, like mobile robots or drones, the configuration space and the world space are usually the same. This simplifies the planning, since explicit obstacle representations are available which enables geometrical tools like separating hyperplanes, smallest distance to obstacles etc., to be used when designing motion planning algorithms. For multi-body robots like manipulators, the situation is completely different. The world space obstacles are usually mapped to non-convex regions, and to make the problem even harder, the mapping is usually not known. Forming explicit representations of the obstacle region in the configuration space is usually too expensive or intractable. Despite all of this, sampling based planners are used with great success, which mainly is due to their use of implicit representations of the obstacle region. The basic idea is to construct a graph in the configuration space that covers and connects the collision-free region. From this graph, a path can be extracted that connects a given start and goal configuration. The approach is computationally expensive, since the graph is constructed with the smallest geometrical building block available, points, which represents a collision-check. Furthermore, the extracted paths from the graph are non-smooth and jagged due to the stochastic nature of the approach. This adds an additional post-processing step to the process, where the paths are shortcutted and smoothened, before the path can be used for tracking. Clearly a lot of time is invested to form this graph and produce smooth paths. Thus, if the obstacles start to move, then all of this work is done in no use, since all points that make up this graph need to be re-verified, which is simply too time consuming to be done in real time.
\\\\
In this work, we want to address the existing drawbacks of the sampling based planners. Our main contribution is an improved motion planner where each vertex in the graph covers a collision-free region in the form of a sphere instead of a point and where the edges are formed with neighboring intersecting spheres. This representation has the advantage of instead of returning piecewise linear paths, returning a sequence of overlapping spheres, i.e. a convex corridor, that connects a given start and goal configuration, illustrated in Figure \ref{fig:pbrm_intro}. This convex corridor allows us to use convex optimization to produce smooth trajectories, instead of computationally expensive post-processing methods. The representation further allows us to estimate the coverage of the collision-free space, which gives us awareness and feedback in the offline roadmap construction phase. Finally, our representation is simple to adapt to moving obstacles, simply requery for the new radii and recheck for intersections. 
\\\\
The spherical collision-free regions are formed using a signed distance function (SDF), which is a function that returns the smallest distance from an arbitrary point to the boundary of an obstacle. As the name implies, the distance is signed, thus if the point is inside the obstacle it is negative otherwise positive. If the distance is positive, a sphere with radius equal to the distance is guaranteed to cover a collision-free region. Using an SDF in motion planning is not new, but what is novel about our approach is that we express the distance in the configuration space instead of the world space and by doing so allows us to form these convex collision-free regions. We refer to the resulting SDF as a signed configuration distance function (SCDF). Computing an SCDF analytically is non-trivial, our approach is therefore to parameterize the SCDF with a deep neural network and learn the mapping by supervised learning. Our resulting neural SCDF can compute distances for different parameter values of obstacle shapes and we also show how multiple distances can be combined, thus making our approach flexible.
\section{Related work}
Motion planning algorithms can roughly be divided into three families, grid-based, sampling based and optimization based methods. Grid-based methods (GBM) discretize the planning space from which a graph is then compiled. A standard search method is A$^\star$ \citep{a_star}, which is classified as an \textit{informed} search method, since it employs a heuristic function to speed up the search. A$^\star$ guarantees to return an optimal path at the level of discretization used. GBMs usually discretize the planning space by a regular lattice and this limits the GBMs to problems with low dimensionality due to the curse of dimensionality. Thus, GBMs are usually limited to single-body robots where the degrees of freedom (DOF) are low. To overcome the inherent scaling problem with the GBMs, stochastic methods are usually used for multi-body robots. These methods are termed as sampling-based methods (SBM) and core members within this family are the rapidly-exploring random trees (RRT) \citep{rrt} and the probabilistic roadmap (PRM) \citep{prm}. RRT grows a tree from the start configuration and explores the collision-free region in a rapid way until it is able to connect to the goal region. RRT is usually improved by bi-directional planning \citep{rrt_connect}, i.e. an additional tree is grown from the goal configuration and the trees are tested for connection after any tree has been expanded. RRT is a single-query method, thus it searches for a path from scratch each time it is queried. Contrary to this, PRM is a multi-query method, which solves for multiple queries without starting from scratch. PRM does this by creating a roadmap (graph) that covers the collision-free space as an offline step. The graph is then used to solve for multiple queries. PRMs are used in cases where the environment does not change since the extra offline step is too computationally costly and needs to be re-done if the environment is changed. In our work, we address this inherent issue by using a different roadmap representation. Our vertices in the graph cover a collision-free region in the form of spheres and we form the edges by checking for intersecting spheres. If something in the environment changes, we recompute the spheres radii and recheck the intersections, without relying on collision detection. We use a trained neural network to compute the sphere radius, therefore querying for the radius can be done fast, hence our representation enables the PRM for dynamic environments.
\\\\
In the recent decades, optimization based methods (OBM) \citep{chomp, schulman, itomp, stomp} have been introduced as an alternative to SBM for multi-body robots. Like the SBM, the OBMs scale well to higher dimensional problems and produce smoother motion. It is common to use a SDF in the optimization since it is a smooth function, thus enabling gradient-based methods. However, the standard way of expressing the SDF is in world space. The distance therefore needs to be mapped to the configuration space by the forward kinematics. This mapping makes the optimization problem a non-linear program (NLP), which is computationally expensive to solve. Recently, a different approach has been proposed. In \cite{mp_gcs} motion planning is formulated as a convex optimization problem by using the graph of convex sets framework \citep{gcs}. The underlying idea is to decompose the collision-free space into intersecting convex sets from which a convex optimization problem is formulated. In cases where an explicit representation of the obstacles in the configuration space exists, like for single-body robots, creating collision-free convex regions can be done fast \citep{iris}. For multi-body robots, this is non-trivial. Existing work does this successfully \citep{iris_nlp, iris_c} by an optimization based approach, but the methods are still too time consuming to be used in the presence of moving obstacles. Our approach is instead to use deep learning to learn an SDF expressed in the configuration space. With this, we can query for shortest distances to the collision boundary, which allows us to expand spherical regions which are collision-free. Our approach is fast and therefore enables our suggested roadmap planner to be used in dynamic environments.
\\\\
Recent research has focused on learning collision detection \citep{fk_kernel_distance, diffco, graphdistnet} by predicting the signed distance between the robot links and the surrounding obstacles in the world space. The learned SDF is used in trajectory optimization but since the distance is expressed in the world space, the problem becomes an NLP and therefore takes a long time to solve. We take a novel approach and suggest to instead express the signed distance in the configuration space. This allows us to improve the PRM at the same time as it enables convex optimization for trajectory optimization, which runs faster and is more reliable than NLP solvers. In \cite{cspf} a learned signed distance function in the configuration space is proposed similar to our approach. However, their approach is restricted to point cloud representations, while we propose to represent the obstacles as parameterized geometric shapes, e.g. spheres. Furthermore, we also show how to use our learned SCDF to improve an existing roadmap planner.
\section{Problem formulation}
A robot is located in the world space, $\W \subset \R^3 $. The unique location of the robot is given by its configuration $\q \in \C$, where $\C$ is the configuration space. The set of points covered by the robots bodies at a certain configuration is expressed as $\B(\q) \subset \W$. The robot is surrounded by $\NrObst$ obstacles $\O = \bigcup_{i=1}^{\NrObst} \O_i$, where  $\O_i \subset \W$. The representation of the obstacle in the configuration space is the set $\C\O_i = \{\q \in \C \: |\: \B(\q) \cap \O_i \neq \emptyset \}$. The obstacle space is formed as $\Co = \bigcup_{i=1}^{\NrObst} \C \O_i$. The complement is referred to as the free space, $\Cf = \C \setminus \Co$. The path planning problem is a tuple, ($\Cf$, $\qStart$, $\qGoal$), where we want to connect a query pair, consisting of a start, $\qStart$, and goal configuration, $\qGoal$, with a geometric path, $\q(s): [0, 1] \mapsto \Cf$, such that $\q(0)=\qStart$ and $\q(1)=\qGoal$, or report correctly when such a path does not exist.
\end{document}

\section{Basic Background: Supervised Learning and the PAC Model}
\label{sec:background}

At this point almost everyone has heard of machine learning (ML). Anyone likely to stumble upon this article will have also heard of its most influential special case, supervised learning, and those theoretically inclined will also be familiar with the PAC model. Nonetheless, I will set the stage by  recapping the basics.

\subsection{Basics of Supervised Learning}%Let's set the stage in any case

\emph{Supervised Learning} is the task of ``coming up'' with a function $f: \X \to \Y$ to ``explain'' or ``fit'' a sequence of input/output examples   $(x_1,y_1), \ldots, (x_n,y_n)$, with $x_i \in \X$ and $y_i \in \Y$.  Here $\X$ is a \emph{data domain} consisting of \emph{datapoints} $x \in \X$, $\Y$ is a \emph{label set} consisting of \emph{labels} $y \in \Y$, and the sequence $(x_1,y_1),\ldots,(x_n,y_n)$ is the \emph{training data} consisting of \emph{labeled examples (a.k.a. samples)}~$(x_i,y_i)$.  I~will refer to the chosen function $f$ as a \emph{predictor}, and to $n$ as the \emph{sample size}. A \emph{learning algorithm} takes as input training data, and outputs (some representation of) a predictor $f \in \Y^\X$.\footnote{Note that this describes the usual \emph{batch}, a.k.a.~\emph{offline}, setting of supervised learning. I do not discuss other paradigms such as online or active learning in this article.} 



Success in supervised learning is defined as \emph{generalization} to  future examples: For a typical \emph{test example}  $(x_{\tst},y_{\tst})$, the predicted label $y'_{\tst}=f(x_{\tst})$ should ``equal'' $y_{\tst}$, perhaps approximately. We usually assume the test example is drawn from the same  ``source'' as the training data  --- commonly, i.i.d.~from the same distribution. The quality of the prediction is quantified by $\ell(y'_{\tst},y_{\tst})$, where $\ell:~\Y~\times~\Y \to \RR_{\geq 0}$ is a \emph{loss function} chosen as part of the problem definition. Common loss functions include the 0-1 loss $\ell_{0-1}(y',y) = [y' \neq y]$ for \emph{classification} problems,\footnote{The notation $[P]$ denotes $1$ when predicate $P$ is true, and denotes $0$ when $P$ is false.} as well as the absolute loss $|y'-y|$ or squared loss $(y'-y)^2$ for \emph{regression problems} featuring $\Y  \sse \RR$.

Nontrivial generalization properties are typically only possible if one assumes something about the data.\footnote{The need for such an assumption is formalized by the  \emph{no free lunch theorems} of supervised learning \cite{wolpert_connection_1992,wolpert_lack_1996,schaffer_conservation_1994}.} The Bayesian approach to  machine learning, common in many applications, assumes some parametric form for the distribution generating the data, and postulates a prior on the parameters. This is not the approach I will take in this article. Instead, I will focus on the frequentist --- and some would say ``worst-case'' or ``adversarial'' ---  approach that is common in the computational learning theory community, embodied by the PAC model. Here we assume that the (training and test) data can be explained, perhaps approximately, by a function in some ``simple enough to learn'' class of functions $\H \sse \Y^\X$, often called the \emph{hypotheses}. Equivalently, we  seek a predictor which explains the unseen data roughly  as well as the best hypothesis $h^* \in \H$, whether or not we assume that $h^*$ itself provides a perfect explanation.



 \paragraph{Common Algorithmic Templates.} Perhaps the best known general-purpose supervised learning algorithm is \emph{empirical risk minimization (ERM)}, which chooses as its predictor a hypothesis $f \in \H$ minimizing $\frac{1}{n} \sum_{i=1}^n \ell(f(x_i),y_i)$ --- a quantity called the \emph{training error}, \emph{empirical error}, or \emph{empirical risk} of $f$. %\footnote{When multiple hypotheses minimize the empirical risk, we assume ERM breaks ties arbitrarily.}
A common template for generalizing ERM involves adding a \emph{regularization term} $\psi(f)$ to the  objective function, typically chosen to measure some notion of ``hypothesis complexity.'' An algorithm instantiating this template is known as a \emph{structural risk minimizer (SRM)}, and chooses as its predictor the hypothesis $f \in \H$ minimizing the \emph{structural risk} $\frac{1}{n} \sum_{i=1}^n \ell(f(x_i),y_i) + \psi(f)$. Other well-known algorithms, such as gradient descent and its variations,  can frequently be interpreted as approximate implementations of ERM or SRM.


\paragraph{Proper vs Improper Learning.} A learning algorithm is said to be \emph{proper} if its predictor $f$ is always chosen from the hypothesis class, i.e., $f \in \H$, otherwise it is said to be \emph{improper}. ERM  is an example of a proper learning algorithm, as are SRM algorithms of the form described above.  In the \emph{proper regime} of learning, algorithms are required to be proper. This article will be concerned with the more flexible \emph{improper regime} (a.k.a \emph{representation-independent learning}), where no such constraint is placed on the learner. In other words, all we care about is predictive power at test time, rather than any insights derived from the functional form or representation of the predictor~itself.


\subsection{The PAC Model}
A standard mathematical setup for evaluation of supervised learning algorithms, at least in the theoretical computer science community, is Valiant's \emph{Probably Approximately Correct (PAC) model} of learning (see e.g.~\cite{kearns_introduction_1994,mohri_foundations_2018}). Here, we assume there is an unknown distribution $\D$ on $\X \times \Y$ from which training and test data are  drawn.  Specifically, the labeled datapoints of the training set  $(x_1,y_1), \ldots, (x_n,y_n)$, as well as the test data  $(x_\tst,y_\tst)$, are i.i.d.~from $\D$. Often it is assumed that $\D$ lies in some class of distributions of interest. The \emph{true expected loss}, or simply \emph{loss}, of a predictor $f: \X \to \Y$ is the expected loss it incurs on draws from $\D$, written $L_\D(f) = \Ex_{(x,y) \sim \D} \ell(f(x),y)$.


There are two main ``settings'' in PAC learning. The  \emph{realizable setting} only requires that the data be perfectly explained by some hypothesis in $\H$. More generally, the \emph{agnostic setting} makes no assumption relating the data to the hypotheses, but shifts the goalposts as necessary to allow nontrivial guarantees: the expected loss at test time is evaluated only ``relative'' to that of the best hypothesis $h^* \in \H$. There are other settings which make more nuanced assumptions, such as $\D$ being of a particular parametric form or its support living in some (unknown) lower-dimensional space, etc. I will mostly discuss the realizable and agnostic settings in this article, those being the simplest and most studied from a theoretical perspective. %TODO:We will briefly discuss other settings in Section ??

The PAC model demands high probability guarantees of learners, in the worst case over distributions of interest. Consider first the realizable setting, where $\D$ is such that $\min_{h \in \H} L_{\D}(h) = 0$. A PAC learner has \emph{error} $\epsilon=\epsilon(n)$ and \emph{confidence} $\delta=\delta(n)$ if, when training data consists of $n$ i.i.d~samples from a realizable distribution $\D$, it produces a predictor $f$  satisfying $L_\D(f) \leq \epsilon$ with probability at least $1-\delta$. In the agnostic setting, where $\D$ can be arbitrary, we require $L_\D(f) - \min_{h \in \H} L_\D(h) \leq \epsilon$ with probability $1-\delta$.

In both the realizable and agnostic settings, we look for PAC learners with small $\epsilon$ and $\delta$ as a function of the sample size $n$. An equivalent perspective looks at the sample complexity $m(\epsilon,\delta)$, which is the minimum sample size which guarantees error  at most $\epsilon$ with probability at least $1-\delta$. We say a problem is \emph{PAC learnable} if its PAC sample complexity is finite whenever $\epsilon,\delta > 0$.

For most PAC learning problems, learnability and sample complexity are characterized in terms of a  ``dimension'' of the hypothesis class. Most prominently this is the \emph{VC dimension} for binary classification, the \emph{fat shattering dimension} for agnostic regression, and the \emph{DS dimension} for multiclass classification (see \cite{anthony_neural_1999,daniely_optimal_2014,brukhim_characterization_2022}). Treatment of these is beyond the scope of this article. The unfamiliar reader need not worry, however,  as dimensions will feature only tangentially in our~discussion.




%\paragraph{Learning settings: Realizable, Agnostic, etc.} In learning theory, evaluating a supervised learning algorithm requires specifying a data model and an objective. We will leave the details of the data model flexible for now, to allow for both the PAC model and the adversarial transductive model. Nonetheless we will describe two variations, which we call ``settings'', which cut across different models. The  \emph{realizable setting}  requires only that the data be perfectly explained by some hypothesis $h \in \H$ --- i.e., there exists a hypothesis which is guaranteed to suffer a loss of $0$ on training and test data. The performance of the learning algorithm is its expected loss at test time for some ``worst case'' realizable instance. More generally, the \emph{agnostic setting} makes no assumption relating the data to the hypotheses, but shifts the goalposts as necessary to allow nontrivial guarantees: the expected loss at test time is evaluated only ``relative'' to that of the best hypothesis $h^* \in \H$, again for some ``worst case'' instance. There are other settings which make more nuanced assumptions about the data, such as it is drawn from a distribution of a particular parametric form, or that it lives in some (unknown) lower-dimensional space, etc. We will mostly discuss the realizable and agnostic settings, those being the simplest and most studied from a theoretical perspective.




%%% Local Variables:
%%% mode: latex
%%% TeX-master: "learning_matching"
%%% End:


\section{\texttt{DobLIX} Design}
\label{sec:doblix-design}

In this section, we describe how \texttt{DobLIX} is designed to speed up lookup queries. We first dive into the general architecture and core concepts of \texttt{DobLIX} (\S~\ref{sec:design:arch} and \S~\ref{sec:design:overview}). To align index modeling with dual objectives, \texttt{DobLIX} utilizes two LI approaches: it modifies the existing Piecewise Linear Approximation (PLA) method~\cite{kipf2020radixspline} by adjusting the spline operations to incorporate new objective functions, and introduces a novel indexing strategy called Piecewise Regression Approximation (PRA), which improves performance by effectively managing modeling errors (\S~\ref{sec:design:li}). \texttt{DobLIX} incorporates a string-compatible LI solution capable of handling variable-size KVs. To optimize the last-mile search process, it transfers model knowledge to narrow the search range and simplifies key comparisons by focusing solely on a limited part of the key bits decodable as an integer (\S~\ref{sec:design:lastmile}). Furthermore, it utilizes a reinforcement learning (RL) tuning agent to dynamically determine optimal parameters, such as the maximum approximation error of the allowed model and block size, and to select between the PLA and PRA algorithms (\S~\ref{sec:design:agent}).


\begin{figure}[t]
  \centering
  \makebox{\includegraphics[width=\columnwidth]{figs/overview.pdf}}
  \vspace{-1.7em}
  \caption{\small{\texttt{DobLIX} Architecture.}}
  \label{fig:DobLIX-arch}
  \vspace{-1.5em}
\end{figure}

\subsection{Overall Architecture}
\label{sec:design:arch}
Fig.~\ref{fig:DobLIX-arch} outlines the general architecture of \texttt{DobLIX}. This solution concentrates on learning the index at the SSTable level in detail. SSTables are preferred for LIs due to their unchanging nature, which removes the need for updates during their lifespan. Upon the formation of each SSTable, DobLIX trains an LI model based on its KVs. This model is crafted to accurately pinpoint the target block for all KVs within a specified error margin while ensuring that block sizes remain within the designated maximum limit. Consequently, \texttt{DobLIX} has an LI model and block partitioning, as illustrated in Fig.~\hyperref[fig:prev-designs]{\ref*{fig:prev-designs}d}. Subsequently, \texttt{DobLIX} serializes the LI model and deposits it in the index block within the SSTable metadata.


\vspace{1pt}
\noindent
{\small\textbf{Lookup Process.}}
In Fig.~\ref{fig:DobLIX-arch}, the stages involved in a \texttt{DobLIX} lookup query are outlined. \circled{1} Initially, it inspects the current MemTables; if the desired key is absent there, it looks through the unalterable MemTables. \circled{2} It then scrutinizes various levels of LSM-trees and \circled{3} loads the SSTable that may cover the target key within its range. \circled{4} \texttt{DobLIX} loads the LI model from the SSTable metadata into memory. \circled{5} It performs a search within its Trie tree to locate the node that houses the ultimate LI (\S~\ref{sec:li-on-strings}) and \circled{6} uses the trained CDF model within that node to \circled{7} determine the \textbf{exact block number} that contains the key and \textbf{narrows down the search scope} for the last-mile search in the block using the LI model. \circled{8} Following this, \texttt{DobLIX} loads the block from storage in memory and \circled{9} executes the final search within the specified range in $KVs~Adrr$ stored in the blocks' metadata to find the exact offset of the target KV pair in the $KVs~Data$ (\S~\ref{sec:design:lastmile}), and \circled{10} employs the retrieved address on the $KVs~Data$ to locate the actual KV. \circled{11} Finally, \texttt{DobLIX} measures the \textit{latency} of the current lookup query alongside the \textit{index size}, incorporating these measurements as feedback to refine the tuning agent (\S~\ref{sec:design:agent}).

\subsection{Concept Overview}
\label{sec:design:overview}
The management of LSM-tree data involves data partitioning and indexing phases (\S~\ref{sec:rocksdb}), and as we established earlier, any optimization strategy, especially those involving LIs, should improve overall performance. As depicted in Fig.~\ref{fig:prev-designs}, block partitioning is intertwined with block indexing ($Par_{Block} \not\perp I_{IndexBlock}$). Therefore, optimizing $I_{IndexBlock}$ requires the consideration of $Par_{Block}$. In contrast, previous designs illustrated in Fig.~\hyperref[fig:prev-designs]{\ref*{fig:prev-designs}\{a,b,c\}} from earlier research have significant data access expenses due to the independence between the LI model and the data partitioning component.

\noindent As described in \S~\ref{sec:li-storage}, within the traditional framework of LI modeling, $I(.)$ represents the result of approximating keys indexes drawn from an unknown distribution $\mathcal{D}_{keys}$ through practical optimization. Therefore, the classical design of the LI does not consider data access and the result coordinated by all data; however, in LSM-trees only a limited number of SSTable blocks reside in memory. In addition, the primary optimization objective is to minimize the error within the hypothesis spaces chosen, regardless of any secondary objectives. 

\texttt{DobLIX} aims to redefine LI models by integrating efficient block-based data access as a key objective. Enhances indexing performance by ensuring that the trained model accurately maps queries to the correct block, enabling the retrieval of only a single block while adhering to the optimal block size. A critical aspect of \texttt{DobLIX} is the relationship between the index approximation ($I_{IndexBlock}$) and block partitioning ($Par_{Block}$), where a one-to-one correspondence is established between the index approximation and the segments within $Par_{Block}$. This allows \texttt{DobLIX} to apply LI models that partition the key domain into segments, ensuring each segment corresponds to a specific block. The system performs a binary search on an array of offsets ($I_{IndexBlock}$) to find the start of each segment (i.e. block). Within each segment, it uses an index approximation ($I_{KV}$) to efficiently locate the keys. Since the trained index is based on the entire data in SSTable, the index model used for each block requires adjustment~(\S~\ref{sec:design:lastmile}). This approach optimizes both block access and key retrieval, providing efficient indexing.

To achieve this, we introduce a dual-objective optimization approach for two distinct LI methodologies. The first method is based on the piecewise linear approximation (PLA) modeling~\cite{kipf2020radixspline}, while the second method employs the piecewise regression approximation (PRA), based on the recursive model index~\cite{kraska2018case}. We delve into these methods in \S~\ref{sec:design:pla} and \S~\ref{sec:design:pra}.

\subsection{LI Approximation Methods}
\label{sec:design:li}
In this section, we present our LI algorithms that focus on dual-objective optimization to train the index model. Considering the importance of I/O performance in indexing, these algorithms are designed to partition the KV space based on their sizes and progressively systematically construct the index. Typically, we approximate the index for each segment using linear models. When it comes to searching for specific points during lookup processes, we employ a binary search on the points derived from the piecewise approximation to precisely pinpoint the required location, which is referred to as last-mile search. In the following, we elaborate on these algorithms in detail.


\begin{figure}[t]
  \centering
  \makebox{\includegraphics[width=0.9\columnwidth]{figs/methods.pdf}}
  % \vspace{-1em}
  \caption{\small{LI Models. $B_i$s represent the actual blocks added to SSTables.}}
  \label{fig:methods}
  \vspace{-1em}
\end{figure}

\subsubsection{\textbf{Dual-Objective Optimization}}
\label{sec:dual-objective-optimization}
To address both the data access and index approximation goals, we used lexicographic optimization (\S~\ref{sec:lexi_opt}). As demonstrated by the motivational experiment in Fig.~\hyperref[fig:rocksdb-lookup]{\ref*{fig:rocksdb-lookup}a}, data access significantly influences latency performance. Although various methods exist to tackle multi-objective optimization problems, we aim to prioritize the data access parameter more heavily than index lookup. Thus, we always finalize a block when its size exceeds the maximum block size $B_{max}$ by incorporating an additional pair of key values. This ensures that the block sizes remain below $B_{max}$, even if the approximation error $E$ has not yet been achieved. Note that both the configuration values $B_{max}$ and $E$ are given by the \textit{Tuning Agent} (\S~\ref{sec:design:agent}). 

\subsubsection{\textbf{Piecewise Linear Approximation (PLA)}}
\label{sec:design:pla}

In this method, the key space is divided into blocks $B_i$ using a linear approximation (spline), each block containing keys that share a common prefix. For each block $B_i$, a spline estimate of the positions is made, defined as $M_i(x) = a_i + b_i(x - x_i)$, where $a_i$ and $b_i$ are coefficients derived from control points and $x_i$ is the initial key in block $B_i$. The binary search is then used around $M_i(k)$ to identify the precise index $I(k)$. For each block $B_i$, the next key is added to a new block ($B_{i+1}$) of the SSTable if (1) the approximation error $M_i(x)$ reaches the maximum threshold $E$, or (2) the condition $|B_i|\ge B_{max}$ indicates that including the new pair of KV would exceed the maximum block size allowed. This approximation process is illustrated in Fig.~\hyperref[fig:methods]{\ref*{fig:methods}a}. As described in \S~\ref{sec:dual-objective-optimization} when the size of an added point exceeds $B_{max}$, a new block is created to maintain the optimal I/O (data access) performance of the previous block, as indicated by the cross in Fig.~\hyperref[fig:methods]{\ref*{fig:methods}a}.
This mechanism results in a new set of spline points, introducing new blocks when the secondary optimization criterion is met, in addition to the standard blocks formed by reaching the maximum approximation error. Algorithm~\ref{alg:pla} showcases this approach, where $APE(line,set)$ calculates the maximum distance the points in $set$ can have from the given line $line$. In this context, $a_i$ is defined as $\mathcal{R}[i][0][1]$, $x_i$ corresponds to $\mathcal{R}[i][0][0]$, and $b_i$ is calculated as $\frac{\mathcal{R}[i+1][0][1] - \mathcal{R}[i][0][1]}{\mathcal{R}[i+1][0][0] - \mathcal{R}[i][0][0]}$, while the term $\text{offset}_i$ refers to $\mathcal{R}[i][1]$.


\begin{algorithm}[t]
\SetKwInput{KwResult}{Output}
\SetKwInput{KwIn}{Input}
\DontPrintSemicolon
\LinesNumbered
\SetAlgoNlRelativeSize{-1}
\caption{Dual-objective PLA}\label{alg:pla}
\small
\KwIn{Set of KVs $D$}
\KwResult{Radix Points $\mathcal{R}$}
$\mathcal{R}\gets [~], \quad index\gets 0, \quad \text{offset}\gets 0$ \;

$E,B_{max} \gets TuningAgent()$ \;

$B_{curr} \gets [(k_0,v_0)]$\;

\While{$(k,v) \in D$}{
    \If{$\vert B_{curr}\vert > B_{max}$}{
      $\mathcal{R} \gets \mathcal{R} + [B_{curr}.last]$\;
      
      $B_{curr} \gets [(k,index)]$\;
    }
  \eIf{$\vert B_{curr}\vert>1~\wedge~
  APE(\text{Line}(B_{curr}.\text{first},(k,index)), B_{curr})\ge~E$}{
      $\mathcal{R} \gets \mathcal{R}+[(B_{curr}.\text{last},\text{offset})]$ \;
      
      $\text{offset} \gets \text{offset} + |B_{curr}|$\;
      
      $B_{curr} \gets [(k,index)]$\;
  }{
  $B_{curr} \gets B_{curr} + [(k,index)]$\;
  }
  $index \gets index + 1$\;
}
$\mathcal{R} \gets \mathcal{R} + [(B_{curr}.\text{last},\text{offset})]$\;
\end{algorithm}


\subsubsection{\textbf{Piecewise Regression Approximation (PRA)}}
\label{sec:design:pra}

This approach involves initially dividing the key domain into segments of size $B_{max}$ and then approximating each segment linearly. The maximum approximation error for each segment, $E'$, is noted and utilized during the last-mile search phase (refer to Fig.~\hyperref[fig:methods]{\ref*{fig:methods}b}). Start scanning the KVs from the beginning; if incorporating the new KV into the existing segment exceeds $B_{max}$, start a new segment. Algorithm~\ref{alg:par} performs this task in one pass, maintaining the integrity of each KV pair.

After partitioning the data using the optimal block size $B_{max}$, as outlined in Algorithm~\ref{alg:pra}, a new model can be constructed for each partition through a linear approximation. The boundary points are retained for search tasks to determine the appropriate model for lookup queries, and $E'$ is stored in $\mathcal{E}$, respectively.

\begin{algorithm}[t]
\SetKwInput{KwResult}{Output}
\SetKwInput{KwIn}{Input}
\DontPrintSemicolon
\LinesNumbered
\SetAlgoNlRelativeSize{-1}
\caption{Partition~$Par_B(.)$}\label{alg:par}
\small
\KwIn{Set of KVs $D$,  Maximum Block Size $B$}
\KwResult{Partition $\mathcal{P}$}

$t,\mathcal{P} \gets []~~\&~~ \text{offset}\gets 0$ \;

\While{$KV \in D$}{
    \If{$\vert t\vert + \vert KV\vert > B$}{
        $\mathcal{P} \gets \mathcal{P} + [(t,\text{offset})]$\;
        
        $\text{offset}\gets\text{offset} + |t| ~\&~ t \gets []$ \;
    }
    $t \gets t + [KV]$\;
}
$\mathcal{P} \gets \mathcal{P} + [(t,\text{offset})]$\;
\end{algorithm}

\begin{algorithm}
\SetKwInput{KwResult}{Output}
\SetKwInput{KwIn}{Input}
\DontPrintSemicolon
\LinesNumbered
\SetAlgoNlRelativeSize{-1}
\caption{Dual-objective PRA}\label{alg:pra}
\small
\KwIn{Set of KVs $D$}
\KwResult{Radix Points $\mathcal{R}$\newline Model Sets $\mathcal{M}$\newline Maximum Segment Errors $\mathcal{E}$}
$\mathcal{R},\mathcal{M}, \mathcal{E} \gets []$ \;

$B_{max} \gets TuningAgent()$ \;

$\mathcal{P} \gets Par_{B_{max}}(D)$ \Comment{from Algorithm~\ref{alg:par}}\;

\While{$Partition~par \in \mathcal{P}$}{
\If{$\vert par\vert>1$ }{
    $M\gets LinearRegression(par)$\;
    
    $\mathcal{M} \gets \mathcal{M} + [M]$\;

    $\mathcal{E} \gets \mathcal{E} + [APE(M, par)]$\;
}
    $\mathcal{R} \gets \mathcal{R} + [par.frist]$\;
}
$\mathcal{R} \gets \mathcal{R} + [par.last]$\;
\end{algorithm}

\begin{figure}[t]
  \centering
  \makebox{\includegraphics[width=0.9\columnwidth]{figs/compare-methods.pdf}}
  \vspace{-0.5em}
  \caption{\small{Comparison of PLA and PRA under different data distributions. (a) $E' < E$, indicating PRA performs better. (b) $E < E'$, indicating PLA performs better.}}
  \label{fig:compare-methods}
  \vspace{-1.6em}
\end{figure}

\noindent\subsubsection{\textbf{Comparing PRA and PLA}} In the context of building block approximations, both the Piecewise Linear Approximation (PLA) and the Piecewise Regression Approximation (PRA) rely on the scanning of data, resulting in a linear time complexity of $O(N)$, where $N$ is the number of data points. Each method utilizes closed-form formulas with a time complexity of $O(N)$ for computations within blocks: PLA determines maximum distances, $APE(.,.)$, to construct piecewise linear segments, while PRA computes two-dimensional regression, $LinearRegression(.)$, formulas within blocks.

With respect to space complexity, PLA requires one point per spline segment to be stored since the endpoint of one line serves as the beginning of the next. The distribution of data points influences PLA memory needs; for instance, with a uniform distribution, all data might fit within a single spline (provided its size is smaller than $B_{max}$), thus minimizing memory usage. In contrast, PRA must store both a point and a slope for each regression line, roughly doubling the memory requirement compared to PLA. The complexity of the number of regressions is $O(\frac{N}{B_{max}})$. Consequently, the memory comparison between PLA and PRA depends on the characteristics of the data: PLA may involve fewer blocks and needs just one point per block (linear segment), whereas PRA has to retain two parameters per block.

When comparing the behavior of PRA and PLA during lookups (\circled{7} in Fig.~\ref{fig:DobLIX-arch}), determining which method is superior is challenging, often leading to the interchangeable use of algorithms; in particular, we consider two scenarios that contrast PLA and PRA, as illustrated in Fig.~\ref{fig:compare-methods}, noting that both scenario~(a) and scenario~(b) can occur depending on the sizes of the KV pairs and the distribution of the keys. In the PRA model, the block is written in persistent storage once its size reaches the maximum block size $B_{max}$, whereas the PLA operates under two conditions for flushing: when the approximation error exceeds the error limit $E$, or when the block size reaches $B_{max}$ (the same maximum block size used in PRA). Consequently, depending on the arrangement and distribution of KV, the spline achievable with PLA can potentially lead to a maximum error $E$ that may be higher or lower than the maximum error $E'$ in PRA. As shown in Fig.~\hyperref[fig:compare-methods]{\ref*{fig:compare-methods}a}, PRA can result in $E > E'$, indicating a more accurate approximation than PLA. This directly affects the scope of the last-mile search and the overall efficiency of each algorithm.

This implies that the behavior of KVs beyond the block boundary determined by the maximum error $E$ plays a crucial role. If data points outside the block constrained by $E$ align with the regression trend of the points within the block, the approximation remains accurate. However, there might be situations where data points outside the block behave significantly differently from those within the block. In such cases, as clearly shown in Fig.~\hyperref[fig:compare-methods]{\ref*{fig:compare-methods}b}, this could lead to a less accurate linear regression in PRA, resulting in $E < E'$. This means that PRA has a lower performance than PLA in such scenarios. In some rare cases, all elements in $\mathcal{E}$~(Algorithm~\ref{alg:pra}), denoted as $E'$ for each segment, are equivalent to $E$. In such scenarios, both algorithms exhibit identical last-mile search performance.

\subsection{Serialize and Deserialize Models}
\label{sec:design:serialization}
After training, the LI models are serialized and stored in a metadata block called the Meta Index Block (see Fig.~\ref{fig:sst-format}). This process involves traversing the model tree structure via depth-first search (DFS), serializing each node byte-by-byte. During deserialization, the model reconstructs the tree by determining the type of each node (leaf or internal) and populating the corresponding data structure.

The learned indices generally outperform the binary search in terms of time complexity. Therefore, the size of the set of stored nodes for model approximation is smaller than $O(\log N)$ when reading, while writing the entire dataset takes $O(N)$. As a result, the serialized model size is asymptotically negligible (see \S~\ref{sec:eval:storage}). In practice, RocksDB writes occur in the background flush and compaction process, and since the model is deserialized only once, overall read performance is significantly improved (see \S\ref{sec:eval:throughput}).

\subsection{Last-mile Search Optimization}
\label{sec:design:lastmile}

\begin{figure}[t]
  \centering
\makebox{\includegraphics[width=\columnwidth]{figs/last-mile-optimization.pdf}}
\vspace{-1.6em}
  \caption{\small{Last-mile Search Optimization Flow}}
  \label{fig:last-mile-optimization}
  \vspace{-1.5em}
\end{figure}

The final phase of the search process involves the last-mile search of the recovered block from storage (step \circled{9} in Fig.~\ref{fig:DobLIX-arch}). As illustrated in Fig.~\hyperref[fig:rocksdb-lookup]{\ref*{fig:rocksdb-lookup}c}, this step accounts for more than $40\%$ of the indexing latency in RocksDB. Fig.~\ref{fig:last-mile-optimization} shows how \texttt{DobLIX} optimizes its performance by restoring data from the LI model computation trajectory (step \circled{5} in Fig.~\ref{fig:DobLIX-arch}) to improve the last-mile search process. Initially, the target key is searched within the string LI structure, as explained in \S~\ref{sec:li-storage}. The LI model in \texttt{DobLIX} subsequently provides: \textit{(1)} $Block~b_{P_L^j}$: The block containing the target KV pair.
\textit{(2)} $M(.)$: The LI estimation of the target KV pair index in $KVs Addr$ (Fig.~\ref{fig:DobLIX-arch}).
\textit{(3)} $Level~L$: The level at which the target key was found in the LI model.
\textit{(4)} $Error~E_{P_L^j}$: The maximum range required to search for the target KV pair.

\vspace{3pt}
\noindent
{\small\textbf{Optimizing Search Range.}}
As described in \S~\ref{sec:design:overview}, \texttt{DobLIX} necessitates a coordinate transformation to adapt the model output ($M(.)$) for indexing on the retrieved block ($b_{P_L^j}$). This is achieved by deducting the count of keys in the previous blocks (maintained as the parameter \say{offset} in the metadata of the block introduced in Algs.~\ref{alg:pla}\&~\ref{alg:par}):

\vspace{-0.5em}
{\small
\[
    I_{KV}(.)=M_{adj}(.) = M(.) - b_{P_L^j}.\text{offset}
    \vspace{-0.5em}
\]
}

\noindent
To better illustrate the adjustment, we consider $SST_j$ in Fig.~\ref{fig:prev-designs}, in which the model is trained on the whole SSTable indexes. However, using the above adjustment, the coordination of model output for our solution (i.e., Fig.~\hyperref[fig:prev-designs]{\ref*{fig:prev-designs}d}) is transformed to the retrieved block $B_2$.

\noindent Subsequently, \texttt{DobLIX} performs a binary search within the specified model error range $E_{P_L^j}$ on $M_{adj}(.)$. This error range can be less than the maximum error $E$ if a spline in the PLA method reaches the maximum block size $B_{max}$. In such cases, \texttt{DobLIX} calculates the spline error and incorporates it into the model, transmitting this information to the last-mile search process to streamline the number of key comparisons.

\vspace{1pt}
\noindent
{\small\textbf{Optimizing String Comparison.}}
\texttt{DobLIX} further optimizes the comparisons by avoiding full key comparisons. \texttt{DobLIX} provides the node level where the key is found in its string-LI structure. This level in the tree indicates the common prefix string, $P_L^j$, in the key (Fig.~\ref{fig:last-mile-optimization}). Then \texttt{DobLIX} can ignore this common prefix, as all keys within the retrieved block share the same prefix. Additionally, \texttt{DobLIX} only compares string keys up to their $K$ bytes, ensuring that the key can be identified by comparing only the $K$ byte following the prefix within the error range. Given that $K$ generally consists of $8$ or $16$ bytes as explained in \S~\ref{sec:li-on-strings} for shifts from demanding string comparisons to numerical comparisons.

\subsection{Tuning Agent}
\label{sec:design:agent}
\texttt{DobLIX} employs Q-learning~\cite{watkins1992q}, a lightweight reinforcement learning (RL) algorithm~\cite{mo2023learning}, to dynamically fine-tune the LI and data access parameters. \texttt{DobLIX} determines three key parameters in the creation of SSTables and the training model: \textit{(1)} The choice between using the PLA vs. the PRA. \textit{(2)} The maximum error of the LI in the PLA method ($E$), and \textit{(3)} The maximum size of a block ($B_{max}$).

The state space for the Q-learning agent comprises the index model (PLA with error values ranging from $32$ to $256$, doubling at each step, and PRA, resulting in $5$ distinct values), and the block size ($4KB$ to $32KB$, doubling at each step, yielding $4$ distinct values, as recommended by RocksDB). This creates a total of $20$ possible states. The action space consists of five actions: incrementing, decrementing, or maintaining the current value of the model error and block size.

The RL agent policy for exploring the feasible space is also guided by our optimization method (\S~\ref{sec:dual-objective-optimization}). It prioritizes block size over the approximation error. The policy evaluates the relationship between data block size and its load latency, identifying the first block size that alters this relationship (e.g. $O(log N)$ to $O(N)$) as the upper bound. This effectively reduces the feasible space for exploration by limiting larger block sizes.

The reward function is defined below to consider both system latency and index storage:
\noindent
\scalebox{0.8}{%
\centering{
$R(s, a) = -\nu.Norm(AVG(latency)) - (1 - \nu).Norm(AVG(index~size)) $
}
}

\noindent This reward function is calculated as the negative sum of the normalized average latency and the normalized average of SSTables index size. We use sigmoid normalization to ensure that both latency and index size contribute to the reward on a comparable scale. The weighting parameter $\nu$ controls the relative importance of latency and index size in determining the overall reward (evaluated in \S~\ref{sec:eval:param}). In our experiments, we set $\nu$ as 1 to achieve the lowest latency; however, in some scenarios, storage may be more critical than can be set by lower values of $\nu$.

\begin{algorithm}[t]
\small  % Set font size
\DontPrintSemicolon  % Disable automatic semicolon

\caption{System Tuning Agent}\label{alg:tune}

$Q \gets \text{initializeQTable()}$ \;
$\alpha, \gamma, \epsilon \gets \text{initializeVariables()}$ \;
\While{}{
    Fetch state $s_{t-1}$ \;
    $L_{t}, I_{t} \gets \text{fetchAverageLatencyAndSSTableIndexSize()}$ \;
    $R_{t} \gets \text{calculateReward}(L_{t}, I_{t})$ \;
    Observe state $s_{t}$ \;
    $a' \gets \arg\max_{a \in \text{Actions}} Q(s_{t},a)$ \;
    $Q(s_{t-1}, a_{t-1}) \gets (1-\alpha)Q(s_{t-1}, a_{t-1}) + \alpha\big(R_{t} + \gamma Q(s_{t},a')\big)$ \;
    $A_t \gets \text{getAvailableActions}(s_t)$ \;
    \eIf{$\text{generateRandNumber()} < \epsilon$}{
        $a_t \gets \text{getRandomAction}(A_t)$ \;
    }{
        $a_t \gets \arg\max_{a \in A_t} Q(s_t,a)$ \;
    }
    tuneSystem($a_t$) \;
    $s_t \gets s_{t+1}$ \;
    $\epsilon \gets \text{updateEpsilon}(\epsilon)$ \;
}

\end{algorithm}

Algorithm~\ref{alg:tune} outlines the agent tuning and execution process. The learning procedure begins by configuring the RL hyperparameters: $\alpha$ denotes the learning rate, while $\gamma$ signifies the discount factor that balances immediate and future rewards. An epsilon decay strategy is used by initially setting a high exploration rate ($\epsilon$) to investigate various actions, gradually lowering this value in successive iterations to exploit the optimal actions.
At every $20$ SSTables creation, the following steps occur: First, the agent obtains the average latency of the reads that have reached this level, as well as the index size of the previously created SSTable. Then, it measures the reward and updates it for the chosen action in the previous state. The agent then gets the available actions at that state. Then, depending on the value of $\epsilon$, the agent explores a new action or chooses the best action from the Q-table. In addition, we implement a reset mechanism for the RL agent that reverts $\epsilon$ to its initial value if the distribution undergoes a significant change. This ensures optimal exploration of the space under the new conditions.


\section{Evaluation}


\begin{table}[t]
    \centering
    % \vspace{-0.1in}
    \scalebox{0.78}{
    % \begin{small}
        \begin{tabular}{lccc}
            \toprule
            \multirow{2}*{\textbf{MoE Models}} & \textbf{Parameters} & \textbf{Experts Per Layer} & \textbf{Num. of} \\
            & \textbf{(active / total)} & \textbf{(active / total)} & \textbf{Layers} \\
            \otoprule 
            \mixtral~\cite{jiang2024mixtral} & 12.9B / 46.7B & 2 / 8 & 32 \\
            % \hline
            \qwen~\cite{yang2024qwen2} & 2.7B / 14.3B & 4 / 60 & 24 \\
            \phimoe~\cite{abdin2024phi} & 6.6B / 42B & 2 / 16 & 32 \\
            \bottomrule 
        \end{tabular}
    % \end{small}
    }
    \caption{Characteristics of three \MoE models in evaluation.}
    \vspace{-0.2in}
    \label{table:eval-moe-models}
\end{table}








\subsection{Experimental Setup}
\label{subsec:eval-setup}


% \begin{figure*}[t]
%     \centering
%     \begin{subfigure}[t]{0.48\textwidth}
%         \centering
%         \includegraphics[width=.9\linewidth]{figs/eval-overall-lmsys.pdf}
%         \caption{Serving three \MoE models with LMSYS-Chat-1M dataset.}
%     \end{subfigure}
%     \begin{subfigure}[t]{0.48\textwidth}
%         \centering
%         \includegraphics[width=.9\linewidth]{figs/eval-overall-sharegpt.pdf}
%         \caption{Serving three \MoE models with ShareGPT dataset.}
%     \end{subfigure}
%     \caption{Overall performance of prefill and decode stages for \sys and other four baselines.}
%     \label{fig:eval-overall.pdf}
% \end{figure*}


\noindent \textbf{Testbed.}
We conduct all experiments on a six-GPU testbed, where each GPU is an NVIDIA GeForce RTX 3090 with 24 GB GPU memory. 
%
All GPUs are inter-connected using pairwise NVLinks and connected to the CPU memory using PCIe 4.0 with 32GB/s bandwidth. 
%
Additionally, the testbed has a total of 32 AMD Ryzen Threadripper PRO 3955WX CPU cores and 480 GB CPU memory.


\noindent \textbf{Models.}
We employ three popular \MoE-based \LLMs in our evaluation: \mixtral~\cite{jiang2024mixtral}, \qwen~\cite{yang2024qwen2}, and \phimoe~\cite{abdin2024phi}.
Table~\ref{table:eval-moe-models} describes the parameters, number of \MoE layers, and number of experts per layer for the three models.
Following the evaluation of existing works~\cite{song2024promoe}, we profile the models to set the optimal prefetch distance $d$ to three before evaluation.
% We set $d$ of \mixtral, \qwen, and \phimoe to \todo{$xxx$}, \todo{$xxx$}, and \todo{$xxx$}, respectively.


\noindent \textbf{Datasets and traces.}
We employ two real-world prompt datasets commonly used for \LLM evaluation: LMSYS-Chat-1M~\cite{zheng2023lmsys} and ShareGPT~\cite{sharegpt}.
%
For most experiments, we split the sampled datasets in a standard 7:3 ratio, where 70\% of the prompts' context data (\ie, semantic embeddings and expert maps) are stored in \sys's Expert Map Store, and 30\% of the prompts are used for testing. 
%
For online serving experiments, we empty the Expert Map Store and use real-world \LLM inference traces~\cite{patel2024splitwise,stojkovic2025dynamollm} released by Microsoft Azure to set input and generation lengths and drive invocations.

\noindent \textbf{Baselines.}
We compare \sys against four \SOTA \MoE serving baselines:
1) \textbf{MoE-Infinity}~\cite{xue2024moe} uses coarse-grained request-level expert activation patterns and synchronous expert prediction and prefetching for \MoE serving. 
We prepare the expert activation matrix collection for MoE-Infinity before evaluation for a fair comparison.
%
% However, the open-sourced MoE-Infinity codebase~\cite{moe-infinity-code} lacks some features described in its original paper, we had to modify
%y 
2) \textbf{ProMoE}~\cite{song2024promoe} employs a stride-based speculative expert prefetching approach for \MoE serving. Since the codebase of ProMoE is not open-sourced and requires training predictors for each \MoE model, we reproduced a prototype of ProMoE on top of MoE-Infinity in our best effort.
%
3) \textbf{Mixtral-Offloading}~\cite{eliseev2023fast} combines a layer-wise speculative expert prefetching and a \LRU-based expert cache. 
%
4) \textbf{DeepSpeend-Inference} employs an expert-agnostic layer-wise parameter offloading approach, which uses pure on-demand loading and does not support prefetching. 
%
We implement the offloading logic of DeepSpeed-Inference in the MoE-Infinity codebase and add an expert cache for a fair comparison.
We enable all baselines to serve \MoE models from HuggingFace Transformer~\cite{wolf2020huggingface}. 


\noindent \textbf{Metrics.}
Following the standard evaluation methodology of existing works~\cite{song2024promoe,xue2024moe,zhong2024distserve,agrawal2024taming} on \LLM serving, we report the performance of the prefill and decode stages separately. 
We measure Time-to-First-Token (TTFT) for the prefill stage and Time-Per-Output-Token (TPOT) for the decode stage.
Additionally, we also report other system metrics, such as expert hit rate and overheads, for detailed evaluation.


% \noindent \textbf{\sys's setting.}
% The hyperparameters of \sys containing the prefetch distance $d$ for each \MoE model, Expert Map Store capacity $C$, and Expert Cache memory limit $M$.
% For most experiments, we profile the \MoE models and set the prefetch distance $d$ to their optimal values. The Expert Map Store capacity $C$ is set to \todo{$xxx$} expert maps. We configure the Expert Cache memory limit to \todo{$xxx$} GB.
% The hyperparameter sensitivity is analyzed in \S\ref{subsec:eval-sensitivity}.


\begin{figure}[t]
  \centering
  \includegraphics[width=.95\linewidth]{figs/eval-overall-arxiv.pdf}
  \vspace{-0.15in}
  \caption{Overall performance of prefill and decode stages for \sys and other four baselines.}
  \vspace{-0.2in}
  \label{fig:eval-overall}
\end{figure}


\subsection{Overall Performance}
\label{subsec:eval-overall}



We first evaluate the performance of prefill and decode stages when running \sys and other baselines with the three \MoE models, where we measure Time-To-First-Token (TTFT) and Time-Per-Output-Token (TPOT) for each stage.
Note that the inference latency with expert offloading tends to be higher than no offloading due to two reasons: 
1) During inference, an excessive amount of parameters in \MoE models are loaded and offloaded, which prolongs the inference latency.
2) All baselines and \sys are implemented on top of the MoE-Infinity codebase~\cite{moe-infinity-code}, whose inference latency is inherently impacted by MoE-Infinity's implementation.
Nevertheless, comparing \sys and baselines is fair with the same experimental setup.

Figure~\ref{fig:eval-overall} shows the \TTFT, \TPOT, and expert hit rate of \sys and other four baselines when serving three \MoE models with LMSYS-Chat-1M and ShareGPT datasets, respectively.
DeepSpeed has both the worst \TTFT and \TPOT due to expert-agnostic offloading and lacking expert prefetching.
While Mixtral-Offloading, ProMoE, and MoE-Infinity perform better than DeepSpeed-Inference, they are underperformed by \sys because of coarse-grained offloading designs.
Compared to DeepSpeed-Inference, Mixtral-Offloading, ProMoE, and MoE-Infinity, our \sys reduces the average \TTFT by 44\%, 35\%, 33\%, 30\%, and reduces the average \TPOT by 70\%, 61\%, 55\%, 48\%, across three \MoE models.
%
% Figure~\ref{fig:eval-overall} also reports the expert hit rate of \sys and each baseline. 
For expert hit rate, Mixtral-Offloading achieves a higher hit rate than the other three baselines because of its synchronous speculative prefetching with a prefetch distance of 1. However, due to synchronous prefetching, its \TTFT and \TPOT are worse than others except DeepSpeed-Inference.
\sys improves the average expert hit rate by 147\%, 11\%, 34\%, and 63\% over DeepSpeed-Inference, Mixtral-Offloading, ProMoE, and MoE-Infinity, respectively.

% \begin{figure}[t]
%   \centering
%   \includegraphics[width=.9\linewidth]{figs/eval-overall-sharegpt.pdf}
%   % \vspace{-0.15in}
%   \caption{}
%   % \vspace{-0.25in}
%   \label{fig:eval-overall-sharegpt.pdf}
% \end{figure}




\subsection{Online Serving Performance}
\label{subsec:eval-online}


Except for the offline evaluation (\ie, Expert Map Store in full capacity before serving), we also evaluate \sys against other baselines in online serving settings.
We empty the Expert Map Store of \sys and the expert activation matrix collection of MoE-Infinity for the online serving experiment.
%
The request traces are derived from Azure \LLM inference traces~\cite{patel2024splitwise,stojkovic2025dynamollm}, with 64 requests randomly sampled to drive LMSYS-Chat-1M prompts for each \MoE model serving. 
To ensure consistency, \sys and all baselines input and generate the exact number of tokens specified in the traces.
%
Figure~\ref{fig:eval-online-serve} illustrates the CDF of end-to-end request latency across three \MoE models. The results demonstrate that \sys significantly reduces overall request latency compared to other baselines in online serving scenarios.


\begin{figure}[t]
  \centering
  \includegraphics[width=.95\linewidth]{figs/eval-online-serve-arxiv.pdf}
  \vspace{-0.15in}
  \caption{CDF of request latency for \MoE online serving.}
  \vspace{-0.2in}
  \label{fig:eval-online-serve}
\end{figure}



\subsection{Impact of Expert Cache Limits}



We measure the \TPOT of \sys and other baselines by limiting the expert cache memory budget to investigate their performance in the latency-memory trade-off (\S\ref{subsec:bg-latency-memory-tradeoff}).
We mainly focus on \TPOT to show the end-to-end performance impacted by varying cache limits.
Figure~\ref{fig:eval-cache-limit.pdf} shows the \TPOT of \sys and other four baselines when serving three \MoE models under different expert cache limits.
We gradually increase the GPU memory allocated for caching experts from 6 GB to 96 GB while employing the same experimental setting in \S\ref{subsec:eval-overall}.
Similarly, DeepSpeed-Inference has the worst \TPOT due to being expert-agnostic.
\sys consistently outperforms Mixtral-Offloading, ProMoE, and MoE-Infinity under varying expert cache limits.
Especially for limited GPU memory sizes (\eg, 6GB), \sys reduces the \TPOT by 32\%, 24\%, 18\%, and 18\%, compared to DeepSpeed-Inference, Mixtral-Offloading, ProMoE, and MoE-Infinity, across three \MoE models, respectively.
With fine-grained expert offloading, \sys significantly reduces the expert on-demand loading latency while maintaining a lower GPU memory footprint, therefore achieving a better spot in the latency-memory trade-off of \MoE serving.

% \subsection{Impact of Inference Batch Size}

\subsection{Ablation Study}
\label{subsec:eval-ablation}


% \begin{figure}[t]
%   \centering
%   \includegraphics[width=.95\linewidth]{figs/eval-expert-tracking.pdf}
%   % \vspace{-0.15in}
%   \caption{Expert hit rate of different expert pattern tracking approaches.}
%   % \vspace{-0.25in}
%   \label{fig:eval-expert-tracking}
% \end{figure}



We present the ablation study of \sys's design.


\textbf{Effectiveness of expert map search.}
One of \sys's key designs is the expert map, which tracks expert selection preferences in fine granularity.
We evaluate the effectiveness of the expert map against five expert pattern-tracking approaches as follows.
%
1) \textbf{Speculate}: speculative prediction used by Mixtral-Offloading~\cite{eliseev2023fast} and ProMoE~\cite{song2024promoe}, 
%
2) \textbf{Hit count}: request-level expert hit count used by MoE-Infinity~\cite{xue2024moe}, 
%
3) \textbf{Map (T)}: expert map with only trajectory similarity search,
4) \textbf{Map (T+S)}: expert map with both trajectory and semantic similarity search,
%
and
5) \textbf{Map (T+S+$\delta$)}: expert map with full features enabled, including trajectory and semantic similarity search (\S\ref{subsec:design-similarity-match}) and dynamic expert selection (\S\ref{subsec:design-expert-prefetch}).
%
We implement the above methods in \sys's Expert Map Matcher for a fair comparison.
Figure~\ref{fig:eval-expert-tracking} shows the expert hit rate of the above expert pattern tracking methods.
%
Speculative prediction is effective due to the widespread presence of residual connections in Transformer blocks. However, its effectiveness decreases drastically as prefetch distance increases~\cite{song2024promoe}.
%
The request-level expert activation count has the worst performance due to coarse granularity.
%
As features are incrementally restored to \sys's expert map, the expert hit rate gradually increases, demonstrating its effectiveness.

% \textbf{Effectiveness of asynchronous map matching.}




\begin{figure}[t]
  \centering
  \includegraphics[width=.9\linewidth]{figs/eval-cache-limit-arxiv.pdf}
  \vspace{-0.15in}
  \caption{Performance of \sys and other four baselines under varying expert cache limits.}
  \vspace{-0.1in}
  \label{fig:eval-cache-limit.pdf}
\end{figure}

\begin{figure}[!t]
    \centering
    \begin{subfigure}[t]{0.585\linewidth}
        \centering
        \includegraphics[width=\linewidth]{figs/eval-expert-tracking.pdf}
        \caption{Expert pattern tracking approaches.}
        \label{fig:eval-expert-tracking}
    \end{subfigure}
    % \hspace{0.02in}
    \begin{subfigure}[t]{0.385\linewidth}
        \centering
        \includegraphics[width=\linewidth]{figs/eval-prefetch-and-cache-arxiv.pdf}
        \caption{Prefetch and caching.}
        \label{fig:eval-prefetch-and-cache}
    \end{subfigure}
    \vspace{-0.1in}
    \caption{Ablation study of \sys.}
    \label{fig:eval-ablation}
    \vspace{-0.2in}
\end{figure}

\textbf{Effectiveness of expert prefetching and caching.}
We evaluate \sys's expert prefetching and caching against two caching algorithms:
1) \textbf{\LRU} used by Mixtral-Offloading~\cite{eliseev2023fast}
and 
2) \textbf{\LFU} used by MoE-Infinity~\cite{xue2024moe}.
%
Figure~\ref{fig:eval-prefetch-and-cache} depicts the expert hit rate of \sys and two baselines.
The results show that \LRU performs poorly in expert offloading scenarios. Though \LFU achieves a higher hit rate than \LRU, \sys surpasses both, achieving the highest expert hit rate.

\subsection{Sensitivity Analysis}
\label{subsec:eval-sensitivity}


\begin{figure}[t]
  \centering
  \includegraphics[width=.9\linewidth]{figs/eval-prefetch-distance.pdf}
  \vspace{-0.15in}
  \caption{Performance of \sys serving \MoE models with different prefetch distances.}
  \vspace{-0.1in}
  \label{fig:eval-prefetch-distance}
\end{figure}

% \begin{figure}[t]
%   \centering
%   \includegraphics[width=.9\linewidth]{figs/eval-store-capacity.pdf}
%   % \vspace{-0.15in}
%   \caption{Semantic and trajectory similarity lower bounds in \sys's serving with different Expert Map Store capacity.}
%   % \vspace{-0.25in}
%   \label{fig:eval-store-capacity}
% \end{figure}

\begin{figure}[t]
    \centering
    \begin{subfigure}[t]{0.55\linewidth}
        \centering
        \includegraphics[width=\linewidth]{figs/eval-store-capacity.pdf}
        \caption{Expert Map Store capacity.}
        \label{fig:eval-store-capacity}
    \end{subfigure}
    % \hspace{0.02in}
    \begin{subfigure}[t]{0.435\linewidth}
        \centering
        \includegraphics[width=\linewidth]{figs/eval-batch-size-arxiv.pdf}
        \caption{Inference batch size.}
        \label{fig:eval-batch-size}
    \end{subfigure}
    \vspace{-0.1in}
    \caption{Sensitivity analysis of \sys.}
    \vspace{-0.2in}
    \label{fig:eval-sensitivity}
\end{figure}


We analyze the sensitivity of three hyperparameters: prefetch distance of \MoE models, the capacity of Expert Map Store, and inference batch size.


\textbf{Prefetch distance of \MoE models.}
Figure~\ref{fig:eval-prefetch-distance} shows the \TTFT and \TPOT of \sys when serving three \MoE models with different prefetch distances.
%
We have demonstrated that the expert hit rate decreases when gradually increasing the prefetch distance (Figure~\ref{fig:bg-hit-distance}).
%
When the prefetch distance is small ($<3$), \sys cannot perfectly hide its system delay from the inference process, such as the map matching and expert prefetching, leading to the increase of inference latency.
%
With larger prefetch distances ($>3$), \sys has worse expert hit rates that also degrade the performance. 
Therefore, we set the prefetch distance $d$ to 3 for evaluating \sys.


\textbf{Capacity of Expert Map Store.}
We measure the mean semantic and trajectory similarity scores searched in \sys's expert map matching for \MoE model serving.
%
Figure~\ref{fig:eval-store-capacity} presents the mean semantic and trajectory similarity scores of \sys with different Expert Map Store capacity sizes.
%
Both semantic and trajectory similarity scores improve as the store capacity increases.
%
While the similarity scores exhibit a significant increase with capacities below 1K, further capacity expansion yields diminishing similarity gains. 
To minimize \sys's memory overhead, we set \sys's Expert Map Store capacity to 1K in evaluation.


\textbf{Inference batch size.}
We investigate the impact of inference batch size on \sys and three baselines using \mixtral with LMSYS-Chat-1M.
%
Figure~\ref{fig:eval-batch-size} presents the performance of \sys, Mixtral-Offloading, ProMoE, and MoE-Infinity as the batch size increases from one to four. \sys achieves the lowest \TTFT and \TPOT in most cases.


% \textbf{Inference batch size.}


% \subsection{Scalability}
% \label{subsec:eval-scalability}
% From one to six GPUs


\begin{figure}[t]
  \centering
  \includegraphics[width=.92\linewidth]{figs/eval-overhead-latency.pdf}
  \vspace{-0.15in}
  \caption{Latency breakdown of \sys's one inference iteration with three \MoE models.}
  \vspace{-0.1in}
  \label{fig:eval-overhead-latency.pdf}
\end{figure}





\subsection{System Overheads}
\label{subsec:eval-overhead}


\noindent \textbf{Latency overheads of \sys's operations.}
Figure~\ref{fig:eval-overhead-latency.pdf} shows the latency breakdown of one inference iteration in \sys when serving the three \MoE models.
We report any operations of \sys in \S\ref{subsec:eval-overall} that may incur a significant latency delay, including context collection, map matching, expert on-demand loading, expert prefetching, and map update after the iteration completes.
\qwen has lower end-to-end iteration latency than \mixtral and \phimoe because of significantly fewer parameters.
Note that expert prefetching, map matching, and map update tasks are executed asynchronously, aside from the inference process. Hence, they do not contribute to the end-to-end iteration latency.
Excluding three asynchronous tasks, the total delay incurred by other operations is consistently less than 30ms (5\% of the iteration) across three \MoE models, which is negligible compared to the inference latency.


\noindent \textbf{Memory overheads of \sys's Expert Map Store.}
Figure~\ref{fig:eval-overhead-memory.pdf} shows the CPU memory footprint of \sys's Expert Map Store when varying the store capacity from 1K to 32K maps.
The memory needed to store expert maps for \qwen is more than \mixtral and \phimoe because it has more experts per layer over the other two models, which increases the map shape.
Even for the largest capacity (32K), the Expert Map Store requires less than 200MB of memory to store the maps, which is trivial since modern GPU servers usually have abundant CPU memory (\eg, p4d.24xlarge on AWS EC2~\cite{aws-ec2} has over 1100 GB of CPU memory).
In the evaluation, \sys's map store capacity with 1K maps is sufficient for maintaining performance (\S\ref{subsec:eval-sensitivity}), resulting in minimal memory overhead.



\begin{figure}[t]
  \centering
  \includegraphics[width=.85\linewidth]{figs/eval-overhead-memory.pdf}
  % \vspace{-0.1in}
  \caption{CPU memory footprint of \sys's Expert Map Store with different capacity.}
  \vspace{-0.1in}
  \label{fig:eval-overhead-memory.pdf}
\end{figure}

\section{Related Work}
\label{sec:related-work}
%We now contextualize our work with related literature so that our contributions are highlighted. We cover FMTS, perturbations in time-series, 
% robustness testing of FMs, 
%and rating of AI systems. 

\noindent \textbf{Foundation Models Supporting Time Series} 
The use of FMs for time series forecasting has advanced significantly. 
% \cite{lu2022frozen} first demonstrated that transformers pre-trained on text data (LLMs) can effectively solve sequence modeling tasks in other modalities, paving the way for leveraging language pre-trained transformers for time series analysis. Recent studies have focused on reprogramming LLMs for time series tasks through parameter-efficient fine-tuning and suitable tokenization strategies \cite{zhou2023one, gruver2024large, jin2023time, cao2023tempo, ekambaram2024tiny}. These methods have successfully adapted transformers to the unique challenges of time series forecasting. \cite{zhou2023one} and \cite{jin2023time} further illustrate the versatility and robustness of fine-tuned language pre-trained transformers for diverse time series tasks.
\cite{lu2022frozen} showed that transformers pre-trained on text data can solve sequence modeling tasks in other modalities, enabling their application to time series analysis. Recent studies have reprogrammed LLMs for time series tasks through parameter-efficient fine-tuning and tokenization strategies \cite{zhou2023one, gruver2024large, jin2023time, cao2023tempo, ekambaram2024tiny}. 
% These methods have successfully adapted transformers to the unique challenges of time series forecasting. 
\cite{zhou2023one} and \cite{jin2023time} further illustrate the versatility and robustness of fine-tuned language pre-trained transformers for diverse time series tasks.
% Several models have contributed to the advancement of time series forecasting. \cite{ansari2024chronos} and \cite{woo2024unified} have improved forecasting accuracy and model generalization.  
% % \cite{ansari2024chronos} and \cite{woo2024unified} have pushed the boundaries of forecasting accuracy and model generalization. 
% \cite{rasul2023lag} and \cite{das2023decoder} have explored new tokenization strategies and fine-tuning methods to improve model performance. Additionally, \cite{garza2023timegpt} and \cite{ekambaram2024tiny} have focused on creating lightweight and efficient models for real-time applications. \cite{talukder2024totem} stands out with its unique approach to integrating multiple temporal patterns, enhancing forecasting precision.
% FMs trained from scratch have achieved SOTA on time series tasks. Zero-shot forecasting, exemplified by \cite{gruver2024large}, showcases the ability of these models to make accurate predictions without domain-specific training. \cite{cao2023tempo} and \cite{goswami2024moment} have introduced approaches to enhance the performance and efficiency of time series models, leveraging transformer architectures to capture temporal dependencies more effectively. In our experiments, we select Gemini-V and Phi-3 as the GP models and Chronos and MOMENT as TS models due to their SOTA performance in their respective categories.
Several models have advanced time series forecasting. \cite{ansari2024chronos} and \cite{woo2024unified} have improved forecasting accuracy and model generalization, while
% \cite{ansari2024chronos} and \cite{woo2024unified} have pushed the boundaries of forecasting accuracy and model generalization. 
\cite{rasul2023lag} and \cite{das2023decoder} have explored new tokenization strategies and fine-tuning methods. \cite{garza2023timegpt} and \cite{ekambaram2024tiny} developed lightweight models for real-time applications, and \cite{talukder2024totem} integrated multiple temporal patterns to improve precision. FMs trained from scratch, like \cite{gruver2024large}, achieved SOTA in zero-shot forecasting, with \cite{cao2023tempo} and \cite{goswami2024moment} further improving model performance. 
%In our experiments, we select Gemini-V and Phi-3 as the GP models and Chronos and MOMENT as TS models due to their SOTA performance in their respective categories.
Please see Section~\ref{sec:exp_app} for the FMTS we selected due to their SOTA performance in their respective categories.

%The use of FMs for time series forecasting has seen significant advancements in recent years. \cite{lu2022frozen} first demonstrated that transformers pre-trained on text data (LLMs) can effectively solve sequence modeling tasks in other modalities. This work opened the door to leveraging language pre-trained transformers for time series analysis. Recent studies have built on this foundation, focusing on reprogramming LLMs for time series tasks through parameter-efficient fine-tuning and suitable tokenization strategies \cite{zhou2023one, gruver2024large, jin2023time, cao2023tempo, ekambaram2024tiny}. These methods have proven successful in adapting the powerful capabilities of transformers to the unique challenges of time series forecasting. OneFitsAll \cite{zhou2023one} and Time-LLM \cite{jin2023time} further illustrate how language pre-trained transformers can be fine-tuned for diverse time series tasks, demonstrating their versatility and robustness. 
% \zhen{reason why we didn't include these models in our study, weights not available? or other justification, to prevent that naturally raised question from readers.}\kl{Good point. We need to discuss. I added 2 sentences at the bottom but they are probably not very convincing.}
%Several other models have contributed to the advancement of time series forecasting. Chronos \cite{ansari2024chronos} and Moirai \cite{woo2024unified} have pushed the boundaries of forecasting accuracy and model generalization. Lag-llama \cite{rasul2023lag} and TimesFM \cite{das2023decoder} have explored new tokenization strategies and fine-tuning methods to improve model performance. Additionally, Time-GPT1 \cite{garza2023timegpt} and Tiny-Time Mixers \cite{ekambaram2024tiny} have focused on creating lightweight and efficient models suitable for real-time applications. TOTEM \cite{talukder2024totem} stands out with its unique approach to integrating multiple temporal patterns, further enhancing forecasting precision.
%Aside from reprogramming LLMs for time series, FMs trained from scratch have achieved SOTA on times series tasks. 
%Zero-shot forecasting, exemplified by \cite{gruver2024large}, showcases the ability of these models to make accurate predictions without domain-specific training.  TEMPO \cite{cao2023tempo} and MOMENT \cite{goswami2024moment} have introduced approaches to enhance the performance and efficiency of time series models, leveraging transformer architectures to capture temporal dependencies more effectively.
% \zhen{and these are on various time series tasks including time series forecasting?}
% \zhen{These are models that are specifically trained for time series forecasting, I'd suggest mentioning them first after the LLM reprogramming, and then expanding to the models that are trained across time series tasks instead. The flow of this subsection feels a bit odd as of now.} \kl{Done.}
%In our experiments, we select Gemini-V and Phi-3 as the GP models and Chronos and MOMENT as TS models due to their SOTA performance in their respective categories. 

%\vspace{-0.3em}
\noindent \textbf{Perturbations in Time Series Data} TS data is commonly stored in spreadsheets and databases, which are prone to changes due to acts of omission (e.g., negligence, data-entry errors) or commission (e.g., adversarial attacks, sabotage). Omission errors are most common \cite{spreadsheets-errors-risks-survey}. Tools like Microsoft Excel and Google Sheets are widely used for data collection and analysis, allowing end-user programming \cite{spreadsheets-future-workshop}. However, over 90\% of spreadsheets contain errors due to issues like incorrect formulae, leading to multi-billion dollar losses \cite{spreadsheet-qa-survey}.
%\cite{spreadsheet-qa-survey,spreadsheets-errors-risks-survey}.
Adversarial attacks are also increasing in data stores and AI models for tasks like forecasting.
% \cite{papernot2016transferability} introduced a black-box attack method using a substitute model to generate adversarial examples, demonstrating transferability across tasks. \cite{baluja2017adversarial} focused on white-box attacks using gradient information. 
\cite{karim2019adversarial} adapted these concepts to time series, exploring both black-box and white-box attacks. \cite{oregi2018adversarial} revealed the vulnerability of distance-based classifiers. \cite{rathore2020untargeted} examined various adversarial attacks on time series classifiers. TSFool \cite{li2022tsfool} introduced a multi-objective black-box attack to craft imperceptible adversarial time series to fool RNN classifiers.
%Time series (TS) data is widely stored and manipulated in spreadsheets and databases. These are also the tools which see considerable changes or perturbations due to acts of omission that are unintended (e.g., negligence, data-entry errors) or commission which are deliberate (e.g., adversarial attacks, sabotage). 
%Among these, changes due to omission are most common \cite{spreadsheets-errors-risks-survey}.
%For example, a spreadsheet, implemented in tools like Microsoft Excel and Google Sheets, is a common data collection and analysis environment that also allows end-user programming \cite{spreadsheets-future-workshop}. They are used widely at the workplace and are often a door opener to more advanced scientific tools. But gaining expertise in them needs practice since a large proportion of spreadsheets ($\succ$ 90\%) are known to have errors due to issues like incorrect formulae caused by improper understanding of behavior during routine operations like copy-paste and end-user programming, which have caused losses of multi-billion dollars \cite{spreadsheet-qa-survey,spreadsheets-errors-risks-survey}.
% \zhen{do we need to relate our perturbations to these attacks? otherwise, we must manage the readers' expectations on what types of perturbations we focus on other than adversarial attacks, and motivate it properly}
%\zhen{Play down this a bit, and emphasize and justify why we focus on the type of perturbations we consider in the paper, to mimic operational errors in practice apart from adversarial attacks, citing the 2024 and 1996 papers Biplav added.} 
%Furthermore, adversarial attacks are also increasing both in data stores and in AI models created to solve tasks like forecasting.
%Foundational work by ~\cite{papernot2016transferability} introduced a black-box attack method that involved training a substitute model to generate adversarial examples capable of misleading the target model, demonstrating the transferability property across similar tasks. In contrast, research by ~\cite{baluja2017adversarial} focused on white-box attacks, using gradient information and probabilistic outputs to craft adversarial examples. Researchers~\cite{karim2019adversarial} have adapted these concepts to the time series domain, exploring both black-box and white-box attacks on time series classification models. In addition, ~\cite{oregi2018adversarial} revealed the susceptibility of distance-based time-series classifiers to adversarial examples. ~\cite{rathore2020untargeted} examined untargeted, targeted, and universal adversarial attacks on time series classifiers, demonstrating the effectiveness of these attacks across various datasets. TSFool~\cite{li2022tsfool} introduced a multi-objective black-box attack to craft highly imperceptible adversarial time series to fool RNN classifiers.
%Adversarial attacks on time-series data are initially focused on time-series classification tasks, leveraging concepts adapted from adversarial attacks in other domains.
%explored adversarial sample crafting for time series classification using elastic similarity measures,  %These works collectively underscore the ongoing efforts to understand and mitigate the risks posed by adversarial attacks on time series classification models.
% More recently, research into adversarial attacks on time series forecasting models has revealed distinct challenges and novel attack strategies. One primary challenge is targeted attacks. While targeted adversarial attacks on time series classification aim to misclassify specific instances, achieving similar precision in time series forecasting is more complex due to the sequential nature of the data. Perturbations must be designed to influence specific aspects of the forecast (e.g., directional shifts or amplitude changes) without disrupting the overall temporal dependencies, making precise control more challenging~\cite{govindarajulu2023targeted}. Another challenge is attacks on multivariate forecasting. Adversarial attacks could exploit the inter-dependences between variables. ~\cite{liu2022robust} introduced sparse and indirect cross-time-series attacks in multivariate settings, which are more effective and realistic than direct attacks in univariate cases.
% \zhen{Biplav, could we make a quick comment here as well that we focus more on data error side in practice, other than attacks? and cite the paper that you mentioned on data errors? Otherwise this section of adversarial attacks feel a bit standalone to other sections}
%These challenges underscore the need for ongoing research to develop effective adversarial attack strategies and robust defense mechanisms tailored to the unique characteristics of time series forecasting models.
% -----


\noindent \textbf{Rating AI Systems} Several works have assessed and rated AI systems for trustworthiness from a third-party perspective without access to training data. \cite{srivastava2020rating} proposed a method to rate AI systems for bias, specifically targeting gender bias in machine translators \cite{srivastava2018towards}, and used visualizations to communicate these ratings \cite{bernagozzi2021vega}. They conducted user studies on trust perception through visualizations \cite{vega-userstudy-translatorbias}, but these lacked causal interpretation. \cite{kausik2024rating} introduced a causal analysis approach to rate bias in sentiment analysis systems, extending it to assess their impact when used with translators \cite{kausik2023the}. We extend their method to rate MM-TSFM for robustness against perturbations. Causal analysis offers advantages over statistical analysis by determining accountability, aligning with humanistic values, and quantifying the direct influence of various attributes on forecasting accuracy.




{\footnotesize \bibliographystyle{acm}
\bibliography{sample}}

\end{document}
