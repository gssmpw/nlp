\documentclass[conference]{IEEEtran}
\IEEEoverridecommandlockouts
% The preceding line is only needed to identify funding in the first footnote. If that is unneeded, please comment it out.
\usepackage{cite}
\usepackage{amsmath,amssymb,amsfonts}
\usepackage{textcomp}
\usepackage[ruled,vlined]{algorithm2e}
\usepackage{algpseudocode}
\usepackage{MnSymbol}
\usepackage{wasysym}
\usepackage{multirow}
\usepackage{subcaption}
\usepackage{enumitem}
\usepackage{graphicx} 
\usepackage{tikz}
\usepackage{pdfx}
\newcommand*\circled[1]{\tikz[baseline=(char.base)]{
            \node[shape=circle,draw,inner sep=0.5pt] (char) {#1};}}


\def\BibTeX{{\rm B\kern-.05em{\sc i\kern-.025em b}\kern-.08em
    T\kern-.1667em\lower.7ex\hbox{E}\kern-.125emX}}

\usepackage{cite}

\usepackage[utf8]{inputenc}
\usepackage[T1]{fontenc}
\usepackage{microtype}
\usepackage{graphicx}
\usepackage{balance}  %
\newcommand\norm[1]{\left\lVert#1\right\rVert}
\usepackage{multirow}
\usepackage{color}
\usepackage{listings}
\usepackage{float}
\usepackage{outlines}
\usepackage[normalem]{ulem}
\usepackage{cleveref}
\usepackage{enumitem}
\usepackage{soul}
\usepackage{cuted, lipsum}
\usepackage[listings,skins,breakable]{tcolorbox}
\usepackage{booktabs}
\PassOptionsToPackage{rgb,hyperref,table}{xcolor}
\usepackage{xcolor}
\usepackage{colortbl}
\usepackage{marginnote}
\usepackage{lineno}
\usepackage{dirtytalk}

\begin{document}
\SetAlgoNlRelativeSize{-1}
\SetNlSty{textbf}{}{}
\LinesNumbered
\title{\texttt{DobLIX}: A Dual-Objective Learned Index for Log-Structured Merge Trees}


\author{\IEEEauthorblockN{1\textsuperscript{st} Alireza Heidari}
\IEEEauthorblockA{\textit{Huawei} \\
alireza.heidarikhazaei@huawei.com}
\and
\IEEEauthorblockN{2\textsuperscript{nd} Amirhossein Ahmadi}
\IEEEauthorblockA{\textit{Huawei} \\
amirhossein.ahmadi@huawei.com}
\and
\IEEEauthorblockN{3\textsuperscript{rd} Wei Zhang
}
\IEEEauthorblockA{\textit{Huawei} \\
wei.zhang6@huawei.com}
}

\maketitle

\begin{abstract}
In this paper, we introduce \texttt{DobLIX}, a dual-objective learned index specifically designed for Log-Structured Merge (LSM) tree-based key-value stores. Although traditional learned indexes focus exclusively on optimizing index lookups, they often overlook the impact of data access from storage, resulting in performance bottlenecks. DobLIX addresses this by incorporating a second objective, data access optimization, into the learned index training process. This dual-objective approach ensures that both index lookup efficiency and data access costs are minimized, leading to significant improvements in read performance while maintaining write efficiency in real-world LSM-tree systems. Additionally, \texttt{DobLIX} features a reinforcement learning agent that dynamically tunes the system parameters, allowing it to adapt to varying workloads in real-time. Experimental results using real-world datasets demonstrate that \texttt{DobLIX} reduces indexing overhead and improves throughput by $1.19\times$ to $2.21\times$ compared to state-of-the-art methods within RocksDB, a widely used LSM-tree-based storage engine.
\end{abstract}

\section{Introduction}
\label{sec:introduction}
The business processes of organizations are experiencing ever-increasing complexity due to the large amount of data, high number of users, and high-tech devices involved \cite{martin2021pmopportunitieschallenges, beerepoot2023biggestbpmproblems}. This complexity may cause business processes to deviate from normal control flow due to unforeseen and disruptive anomalies \cite{adams2023proceddsriftdetection}. These control-flow anomalies manifest as unknown, skipped, and wrongly-ordered activities in the traces of event logs monitored from the execution of business processes \cite{ko2023adsystematicreview}. For the sake of clarity, let us consider an illustrative example of such anomalies. Figure \ref{FP_ANOMALIES} shows a so-called event log footprint, which captures the control flow relations of four activities of a hypothetical event log. In particular, this footprint captures the control-flow relations between activities \texttt{a}, \texttt{b}, \texttt{c} and \texttt{d}. These are the causal ($\rightarrow$) relation, concurrent ($\parallel$) relation, and other ($\#$) relations such as exclusivity or non-local dependency \cite{aalst2022pmhandbook}. In addition, on the right are six traces, of which five exhibit skipped, wrongly-ordered and unknown control-flow anomalies. For example, $\langle$\texttt{a b d}$\rangle$ has a skipped activity, which is \texttt{c}. Because of this skipped activity, the control-flow relation \texttt{b}$\,\#\,$\texttt{d} is violated, since \texttt{d} directly follows \texttt{b} in the anomalous trace.
\begin{figure}[!t]
\centering
\includegraphics[width=0.9\columnwidth]{images/FP_ANOMALIES.png}
\caption{An example event log footprint with six traces, of which five exhibit control-flow anomalies.}
\label{FP_ANOMALIES}
\end{figure}

\subsection{Control-flow anomaly detection}
Control-flow anomaly detection techniques aim to characterize the normal control flow from event logs and verify whether these deviations occur in new event logs \cite{ko2023adsystematicreview}. To develop control-flow anomaly detection techniques, \revision{process mining} has seen widespread adoption owing to process discovery and \revision{conformance checking}. On the one hand, process discovery is a set of algorithms that encode control-flow relations as a set of model elements and constraints according to a given modeling formalism \cite{aalst2022pmhandbook}; hereafter, we refer to the Petri net, a widespread modeling formalism. On the other hand, \revision{conformance checking} is an explainable set of algorithms that allows linking any deviations with the reference Petri net and providing the fitness measure, namely a measure of how much the Petri net fits the new event log \cite{aalst2022pmhandbook}. Many control-flow anomaly detection techniques based on \revision{conformance checking} (hereafter, \revision{conformance checking}-based techniques) use the fitness measure to determine whether an event log is anomalous \cite{bezerra2009pmad, bezerra2013adlogspais, myers2018icsadpm, pecchia2020applicationfailuresanalysispm}. 

The scientific literature also includes many \revision{conformance checking}-independent techniques for control-flow anomaly detection that combine specific types of trace encodings with machine/deep learning \cite{ko2023adsystematicreview, tavares2023pmtraceencoding}. Whereas these techniques are very effective, their explainability is challenging due to both the type of trace encoding employed and the machine/deep learning model used \cite{rawal2022trustworthyaiadvances,li2023explainablead}. Hence, in the following, we focus on the shortcomings of \revision{conformance checking}-based techniques to investigate whether it is possible to support the development of competitive control-flow anomaly detection techniques while maintaining the explainable nature of \revision{conformance checking}.
\begin{figure}[!t]
\centering
\includegraphics[width=\columnwidth]{images/HIGH_LEVEL_VIEW.png}
\caption{A high-level view of the proposed framework for combining \revision{process mining}-based feature extraction with dimensionality reduction for control-flow anomaly detection.}
\label{HIGH_LEVEL_VIEW}
\end{figure}

\subsection{Shortcomings of \revision{conformance checking}-based techniques}
Unfortunately, the detection effectiveness of \revision{conformance checking}-based techniques is affected by noisy data and low-quality Petri nets, which may be due to human errors in the modeling process or representational bias of process discovery algorithms \cite{bezerra2013adlogspais, pecchia2020applicationfailuresanalysispm, aalst2016pm}. Specifically, on the one hand, noisy data may introduce infrequent and deceptive control-flow relations that may result in inconsistent fitness measures, whereas, on the other hand, checking event logs against a low-quality Petri net could lead to an unreliable distribution of fitness measures. Nonetheless, such Petri nets can still be used as references to obtain insightful information for \revision{process mining}-based feature extraction, supporting the development of competitive and explainable \revision{conformance checking}-based techniques for control-flow anomaly detection despite the problems above. For example, a few works outline that token-based \revision{conformance checking} can be used for \revision{process mining}-based feature extraction to build tabular data and develop effective \revision{conformance checking}-based techniques for control-flow anomaly detection \cite{singh2022lapmsh, debenedictis2023dtadiiot}. However, to the best of our knowledge, the scientific literature lacks a structured proposal for \revision{process mining}-based feature extraction using the state-of-the-art \revision{conformance checking} variant, namely alignment-based \revision{conformance checking}.

\subsection{Contributions}
We propose a novel \revision{process mining}-based feature extraction approach with alignment-based \revision{conformance checking}. This variant aligns the deviating control flow with a reference Petri net; the resulting alignment can be inspected to extract additional statistics such as the number of times a given activity caused mismatches \cite{aalst2022pmhandbook}. We integrate this approach into a flexible and explainable framework for developing techniques for control-flow anomaly detection. The framework combines \revision{process mining}-based feature extraction and dimensionality reduction to handle high-dimensional feature sets, achieve detection effectiveness, and support explainability. Notably, in addition to our proposed \revision{process mining}-based feature extraction approach, the framework allows employing other approaches, enabling a fair comparison of multiple \revision{conformance checking}-based and \revision{conformance checking}-independent techniques for control-flow anomaly detection. Figure \ref{HIGH_LEVEL_VIEW} shows a high-level view of the framework. Business processes are monitored, and event logs obtained from the database of information systems. Subsequently, \revision{process mining}-based feature extraction is applied to these event logs and tabular data input to dimensionality reduction to identify control-flow anomalies. We apply several \revision{conformance checking}-based and \revision{conformance checking}-independent framework techniques to publicly available datasets, simulated data of a case study from railways, and real-world data of a case study from healthcare. We show that the framework techniques implementing our approach outperform the baseline \revision{conformance checking}-based techniques while maintaining the explainable nature of \revision{conformance checking}.

In summary, the contributions of this paper are as follows.
\begin{itemize}
    \item{
        A novel \revision{process mining}-based feature extraction approach to support the development of competitive and explainable \revision{conformance checking}-based techniques for control-flow anomaly detection.
    }
    \item{
        A flexible and explainable framework for developing techniques for control-flow anomaly detection using \revision{process mining}-based feature extraction and dimensionality reduction.
    }
    \item{
        Application to synthetic and real-world datasets of several \revision{conformance checking}-based and \revision{conformance checking}-independent framework techniques, evaluating their detection effectiveness and explainability.
    }
\end{itemize}

The rest of the paper is organized as follows.
\begin{itemize}
    \item Section \ref{sec:related_work} reviews the existing techniques for control-flow anomaly detection, categorizing them into \revision{conformance checking}-based and \revision{conformance checking}-independent techniques.
    \item Section \ref{sec:abccfe} provides the preliminaries of \revision{process mining} to establish the notation used throughout the paper, and delves into the details of the proposed \revision{process mining}-based feature extraction approach with alignment-based \revision{conformance checking}.
    \item Section \ref{sec:framework} describes the framework for developing \revision{conformance checking}-based and \revision{conformance checking}-independent techniques for control-flow anomaly detection that combine \revision{process mining}-based feature extraction and dimensionality reduction.
    \item Section \ref{sec:evaluation} presents the experiments conducted with multiple framework and baseline techniques using data from publicly available datasets and case studies.
    \item Section \ref{sec:conclusions} draws the conclusions and presents future work.
\end{itemize}
\section{Background}\label{sec:backgrnd}

\subsection{Cold Start Latency and Mitigation Techniques}

Traditional FaaS platforms mitigate cold starts through snapshotting, lightweight virtualization, and warm-state management. Snapshot-based methods like \textbf{REAP} and \textbf{Catalyzer} reduce initialization time by preloading or restoring container states but require significant memory and I/O resources, limiting scalability~\cite{dong_catalyzer_2020, ustiugov_benchmarking_2021}. Lightweight virtualization solutions, such as \textbf{Firecracker} microVMs, achieve fast startup times with strong isolation but depend on robust infrastructure, making them less adaptable to fluctuating workloads~\cite{agache_firecracker_2020}. Warm-state management techniques like \textbf{Faa\$T}~\cite{romero_faa_2021} and \textbf{Kraken}~\cite{vivek_kraken_2021} keep frequently invoked containers ready, balancing readiness and cost efficiency under predictable workloads but incurring overhead when demand is erratic~\cite{romero_faa_2021, vivek_kraken_2021}. While these methods perform well in resource-rich cloud environments, their resource intensity challenges applicability in edge settings.

\subsubsection{Edge FaaS Perspective}

In edge environments, cold start mitigation emphasizes lightweight designs, resource sharing, and hybrid task distribution. Lightweight execution environments like unikernels~\cite{edward_sock_2018} and \textbf{Firecracker}~\cite{agache_firecracker_2020}, as used by \textbf{TinyFaaS}~\cite{pfandzelter_tinyfaas_2020}, minimize resource usage and initialization delays but require careful orchestration to avoid resource contention. Function co-location, demonstrated by \textbf{Photons}~\cite{v_dukic_photons_2020}, reduces redundant initializations by sharing runtime resources among related functions, though this complicates isolation in multi-tenant setups~\cite{v_dukic_photons_2020}. Hybrid offloading frameworks like \textbf{GeoFaaS}~\cite{malekabbasi_geofaas_2024} balance edge-cloud workloads by offloading latency-tolerant tasks to the cloud and reserving edge resources for real-time operations, requiring reliable connectivity and efficient task management. These edge-specific strategies address cold starts effectively but introduce challenges in scalability and orchestration.

\subsection{Predictive Scaling and Caching Techniques}

Efficient resource allocation is vital for maintaining low latency and high availability in serverless platforms. Predictive scaling and caching techniques dynamically provision resources and reduce cold start latency by leveraging workload prediction and state retention.
Traditional FaaS platforms use predictive scaling and caching to optimize resources, employing techniques (OFC, FaasCache) to reduce cold starts. However, these methods rely on centralized orchestration and workload predictability, limiting their effectiveness in dynamic, resource-constrained edge environments.



\subsubsection{Edge FaaS Perspective}

Edge FaaS platforms adapt predictive scaling and caching techniques to constrain resources and heterogeneous environments. \textbf{EDGE-Cache}~\cite{kim_delay-aware_2022} uses traffic profiling to selectively retain high-priority functions, reducing memory overhead while maintaining readiness for frequent requests. Hybrid frameworks like \textbf{GeoFaaS}~\cite{malekabbasi_geofaas_2024} implement distributed caching to balance resources between edge and cloud nodes, enabling low-latency processing for critical tasks while offloading less critical workloads. Machine learning methods, such as clustering-based workload predictors~\cite{gao_machine_2020} and GRU-based models~\cite{guo_applying_2018}, enhance resource provisioning in edge systems by efficiently forecasting workload spikes. These innovations effectively address cold start challenges in edge environments, though their dependency on accurate predictions and robust orchestration poses scalability challenges.

\subsection{Decentralized Orchestration, Function Placement, and Scheduling}

Efficient orchestration in serverless platforms involves workload distribution, resource optimization, and performance assurance. While traditional FaaS platforms rely on centralized control, edge environments require decentralized and adaptive strategies to address unique challenges such as resource constraints and heterogeneous hardware.



\subsubsection{Edge FaaS Perspective}

Edge FaaS platforms adopt decentralized and adaptive orchestration frameworks to meet the demands of resource-constrained environments. Systems like \textbf{Wukong} distribute scheduling across edge nodes, enhancing data locality and scalability while reducing network latency. Lightweight frameworks such as \textbf{OpenWhisk Lite}~\cite{kravchenko_kpavelopenwhisk-light_2024} optimize resource allocation by decentralizing scheduling policies, minimizing cold starts and latency in edge setups~\cite{benjamin_wukong_2020}. Hybrid solutions like \textbf{OpenFaaS}~\cite{noauthor_openfaasfaas_2024} and \textbf{EdgeMatrix}~\cite{shen_edgematrix_2023} combine edge-cloud orchestration to balance resource utilization, retaining latency-sensitive functions at the edge while offloading non-critical workloads to the cloud. While these approaches improve flexibility, they face challenges in maintaining coordination and ensuring consistent performance across distributed nodes.



\section{\texttt{DobLIX} Design}
\label{sec:doblix-design}

In this section, we describe how \texttt{DobLIX} is designed to speed up lookup queries. We first dive into the general architecture and core concepts of \texttt{DobLIX} (\S~\ref{sec:design:arch} and \S~\ref{sec:design:overview}). To align index modeling with dual objectives, \texttt{DobLIX} utilizes two LI approaches: it modifies the existing Piecewise Linear Approximation (PLA) method~\cite{kipf2020radixspline} by adjusting the spline operations to incorporate new objective functions, and introduces a novel indexing strategy called Piecewise Regression Approximation (PRA), which improves performance by effectively managing modeling errors (\S~\ref{sec:design:li}). \texttt{DobLIX} incorporates a string-compatible LI solution capable of handling variable-size KVs. To optimize the last-mile search process, it transfers model knowledge to narrow the search range and simplifies key comparisons by focusing solely on a limited part of the key bits decodable as an integer (\S~\ref{sec:design:lastmile}). Furthermore, it utilizes a reinforcement learning (RL) tuning agent to dynamically determine optimal parameters, such as the maximum approximation error of the allowed model and block size, and to select between the PLA and PRA algorithms (\S~\ref{sec:design:agent}).


\begin{figure}[t]
  \centering
  \makebox{\includegraphics[width=\columnwidth]{figs/overview.pdf}}
  \vspace{-1.7em}
  \caption{\small{\texttt{DobLIX} Architecture.}}
  \label{fig:DobLIX-arch}
  \vspace{-1.5em}
\end{figure}

\subsection{Overall Architecture}
\label{sec:design:arch}
Fig.~\ref{fig:DobLIX-arch} outlines the general architecture of \texttt{DobLIX}. This solution concentrates on learning the index at the SSTable level in detail. SSTables are preferred for LIs due to their unchanging nature, which removes the need for updates during their lifespan. Upon the formation of each SSTable, DobLIX trains an LI model based on its KVs. This model is crafted to accurately pinpoint the target block for all KVs within a specified error margin while ensuring that block sizes remain within the designated maximum limit. Consequently, \texttt{DobLIX} has an LI model and block partitioning, as illustrated in Fig.~\hyperref[fig:prev-designs]{\ref*{fig:prev-designs}d}. Subsequently, \texttt{DobLIX} serializes the LI model and deposits it in the index block within the SSTable metadata.


\vspace{1pt}
\noindent
{\small\textbf{Lookup Process.}}
In Fig.~\ref{fig:DobLIX-arch}, the stages involved in a \texttt{DobLIX} lookup query are outlined. \circled{1} Initially, it inspects the current MemTables; if the desired key is absent there, it looks through the unalterable MemTables. \circled{2} It then scrutinizes various levels of LSM-trees and \circled{3} loads the SSTable that may cover the target key within its range. \circled{4} \texttt{DobLIX} loads the LI model from the SSTable metadata into memory. \circled{5} It performs a search within its Trie tree to locate the node that houses the ultimate LI (\S~\ref{sec:li-on-strings}) and \circled{6} uses the trained CDF model within that node to \circled{7} determine the \textbf{exact block number} that contains the key and \textbf{narrows down the search scope} for the last-mile search in the block using the LI model. \circled{8} Following this, \texttt{DobLIX} loads the block from storage in memory and \circled{9} executes the final search within the specified range in $KVs~Adrr$ stored in the blocks' metadata to find the exact offset of the target KV pair in the $KVs~Data$ (\S~\ref{sec:design:lastmile}), and \circled{10} employs the retrieved address on the $KVs~Data$ to locate the actual KV. \circled{11} Finally, \texttt{DobLIX} measures the \textit{latency} of the current lookup query alongside the \textit{index size}, incorporating these measurements as feedback to refine the tuning agent (\S~\ref{sec:design:agent}).

\subsection{Concept Overview}
\label{sec:design:overview}
The management of LSM-tree data involves data partitioning and indexing phases (\S~\ref{sec:rocksdb}), and as we established earlier, any optimization strategy, especially those involving LIs, should improve overall performance. As depicted in Fig.~\ref{fig:prev-designs}, block partitioning is intertwined with block indexing ($Par_{Block} \not\perp I_{IndexBlock}$). Therefore, optimizing $I_{IndexBlock}$ requires the consideration of $Par_{Block}$. In contrast, previous designs illustrated in Fig.~\hyperref[fig:prev-designs]{\ref*{fig:prev-designs}\{a,b,c\}} from earlier research have significant data access expenses due to the independence between the LI model and the data partitioning component.

\noindent As described in \S~\ref{sec:li-storage}, within the traditional framework of LI modeling, $I(.)$ represents the result of approximating keys indexes drawn from an unknown distribution $\mathcal{D}_{keys}$ through practical optimization. Therefore, the classical design of the LI does not consider data access and the result coordinated by all data; however, in LSM-trees only a limited number of SSTable blocks reside in memory. In addition, the primary optimization objective is to minimize the error within the hypothesis spaces chosen, regardless of any secondary objectives. 

\texttt{DobLIX} aims to redefine LI models by integrating efficient block-based data access as a key objective. Enhances indexing performance by ensuring that the trained model accurately maps queries to the correct block, enabling the retrieval of only a single block while adhering to the optimal block size. A critical aspect of \texttt{DobLIX} is the relationship between the index approximation ($I_{IndexBlock}$) and block partitioning ($Par_{Block}$), where a one-to-one correspondence is established between the index approximation and the segments within $Par_{Block}$. This allows \texttt{DobLIX} to apply LI models that partition the key domain into segments, ensuring each segment corresponds to a specific block. The system performs a binary search on an array of offsets ($I_{IndexBlock}$) to find the start of each segment (i.e. block). Within each segment, it uses an index approximation ($I_{KV}$) to efficiently locate the keys. Since the trained index is based on the entire data in SSTable, the index model used for each block requires adjustment~(\S~\ref{sec:design:lastmile}). This approach optimizes both block access and key retrieval, providing efficient indexing.

To achieve this, we introduce a dual-objective optimization approach for two distinct LI methodologies. The first method is based on the piecewise linear approximation (PLA) modeling~\cite{kipf2020radixspline}, while the second method employs the piecewise regression approximation (PRA), based on the recursive model index~\cite{kraska2018case}. We delve into these methods in \S~\ref{sec:design:pla} and \S~\ref{sec:design:pra}.

\subsection{LI Approximation Methods}
\label{sec:design:li}
In this section, we present our LI algorithms that focus on dual-objective optimization to train the index model. Considering the importance of I/O performance in indexing, these algorithms are designed to partition the KV space based on their sizes and progressively systematically construct the index. Typically, we approximate the index for each segment using linear models. When it comes to searching for specific points during lookup processes, we employ a binary search on the points derived from the piecewise approximation to precisely pinpoint the required location, which is referred to as last-mile search. In the following, we elaborate on these algorithms in detail.


\begin{figure}[t]
  \centering
  \makebox{\includegraphics[width=0.9\columnwidth]{figs/methods.pdf}}
  % \vspace{-1em}
  \caption{\small{LI Models. $B_i$s represent the actual blocks added to SSTables.}}
  \label{fig:methods}
  \vspace{-1em}
\end{figure}

\subsubsection{\textbf{Dual-Objective Optimization}}
\label{sec:dual-objective-optimization}
To address both the data access and index approximation goals, we used lexicographic optimization (\S~\ref{sec:lexi_opt}). As demonstrated by the motivational experiment in Fig.~\hyperref[fig:rocksdb-lookup]{\ref*{fig:rocksdb-lookup}a}, data access significantly influences latency performance. Although various methods exist to tackle multi-objective optimization problems, we aim to prioritize the data access parameter more heavily than index lookup. Thus, we always finalize a block when its size exceeds the maximum block size $B_{max}$ by incorporating an additional pair of key values. This ensures that the block sizes remain below $B_{max}$, even if the approximation error $E$ has not yet been achieved. Note that both the configuration values $B_{max}$ and $E$ are given by the \textit{Tuning Agent} (\S~\ref{sec:design:agent}). 

\subsubsection{\textbf{Piecewise Linear Approximation (PLA)}}
\label{sec:design:pla}

In this method, the key space is divided into blocks $B_i$ using a linear approximation (spline), each block containing keys that share a common prefix. For each block $B_i$, a spline estimate of the positions is made, defined as $M_i(x) = a_i + b_i(x - x_i)$, where $a_i$ and $b_i$ are coefficients derived from control points and $x_i$ is the initial key in block $B_i$. The binary search is then used around $M_i(k)$ to identify the precise index $I(k)$. For each block $B_i$, the next key is added to a new block ($B_{i+1}$) of the SSTable if (1) the approximation error $M_i(x)$ reaches the maximum threshold $E$, or (2) the condition $|B_i|\ge B_{max}$ indicates that including the new pair of KV would exceed the maximum block size allowed. This approximation process is illustrated in Fig.~\hyperref[fig:methods]{\ref*{fig:methods}a}. As described in \S~\ref{sec:dual-objective-optimization} when the size of an added point exceeds $B_{max}$, a new block is created to maintain the optimal I/O (data access) performance of the previous block, as indicated by the cross in Fig.~\hyperref[fig:methods]{\ref*{fig:methods}a}.
This mechanism results in a new set of spline points, introducing new blocks when the secondary optimization criterion is met, in addition to the standard blocks formed by reaching the maximum approximation error. Algorithm~\ref{alg:pla} showcases this approach, where $APE(line,set)$ calculates the maximum distance the points in $set$ can have from the given line $line$. In this context, $a_i$ is defined as $\mathcal{R}[i][0][1]$, $x_i$ corresponds to $\mathcal{R}[i][0][0]$, and $b_i$ is calculated as $\frac{\mathcal{R}[i+1][0][1] - \mathcal{R}[i][0][1]}{\mathcal{R}[i+1][0][0] - \mathcal{R}[i][0][0]}$, while the term $\text{offset}_i$ refers to $\mathcal{R}[i][1]$.


\begin{algorithm}[t]
\SetKwInput{KwResult}{Output}
\SetKwInput{KwIn}{Input}
\DontPrintSemicolon
\LinesNumbered
\SetAlgoNlRelativeSize{-1}
\caption{Dual-objective PLA}\label{alg:pla}
\small
\KwIn{Set of KVs $D$}
\KwResult{Radix Points $\mathcal{R}$}
$\mathcal{R}\gets [~], \quad index\gets 0, \quad \text{offset}\gets 0$ \;

$E,B_{max} \gets TuningAgent()$ \;

$B_{curr} \gets [(k_0,v_0)]$\;

\While{$(k,v) \in D$}{
    \If{$\vert B_{curr}\vert > B_{max}$}{
      $\mathcal{R} \gets \mathcal{R} + [B_{curr}.last]$\;
      
      $B_{curr} \gets [(k,index)]$\;
    }
  \eIf{$\vert B_{curr}\vert>1~\wedge~
  APE(\text{Line}(B_{curr}.\text{first},(k,index)), B_{curr})\ge~E$}{
      $\mathcal{R} \gets \mathcal{R}+[(B_{curr}.\text{last},\text{offset})]$ \;
      
      $\text{offset} \gets \text{offset} + |B_{curr}|$\;
      
      $B_{curr} \gets [(k,index)]$\;
  }{
  $B_{curr} \gets B_{curr} + [(k,index)]$\;
  }
  $index \gets index + 1$\;
}
$\mathcal{R} \gets \mathcal{R} + [(B_{curr}.\text{last},\text{offset})]$\;
\end{algorithm}


\subsubsection{\textbf{Piecewise Regression Approximation (PRA)}}
\label{sec:design:pra}

This approach involves initially dividing the key domain into segments of size $B_{max}$ and then approximating each segment linearly. The maximum approximation error for each segment, $E'$, is noted and utilized during the last-mile search phase (refer to Fig.~\hyperref[fig:methods]{\ref*{fig:methods}b}). Start scanning the KVs from the beginning; if incorporating the new KV into the existing segment exceeds $B_{max}$, start a new segment. Algorithm~\ref{alg:par} performs this task in one pass, maintaining the integrity of each KV pair.

After partitioning the data using the optimal block size $B_{max}$, as outlined in Algorithm~\ref{alg:pra}, a new model can be constructed for each partition through a linear approximation. The boundary points are retained for search tasks to determine the appropriate model for lookup queries, and $E'$ is stored in $\mathcal{E}$, respectively.

\begin{algorithm}[t]
\SetKwInput{KwResult}{Output}
\SetKwInput{KwIn}{Input}
\DontPrintSemicolon
\LinesNumbered
\SetAlgoNlRelativeSize{-1}
\caption{Partition~$Par_B(.)$}\label{alg:par}
\small
\KwIn{Set of KVs $D$,  Maximum Block Size $B$}
\KwResult{Partition $\mathcal{P}$}

$t,\mathcal{P} \gets []~~\&~~ \text{offset}\gets 0$ \;

\While{$KV \in D$}{
    \If{$\vert t\vert + \vert KV\vert > B$}{
        $\mathcal{P} \gets \mathcal{P} + [(t,\text{offset})]$\;
        
        $\text{offset}\gets\text{offset} + |t| ~\&~ t \gets []$ \;
    }
    $t \gets t + [KV]$\;
}
$\mathcal{P} \gets \mathcal{P} + [(t,\text{offset})]$\;
\end{algorithm}

\begin{algorithm}
\SetKwInput{KwResult}{Output}
\SetKwInput{KwIn}{Input}
\DontPrintSemicolon
\LinesNumbered
\SetAlgoNlRelativeSize{-1}
\caption{Dual-objective PRA}\label{alg:pra}
\small
\KwIn{Set of KVs $D$}
\KwResult{Radix Points $\mathcal{R}$\newline Model Sets $\mathcal{M}$\newline Maximum Segment Errors $\mathcal{E}$}
$\mathcal{R},\mathcal{M}, \mathcal{E} \gets []$ \;

$B_{max} \gets TuningAgent()$ \;

$\mathcal{P} \gets Par_{B_{max}}(D)$ \Comment{from Algorithm~\ref{alg:par}}\;

\While{$Partition~par \in \mathcal{P}$}{
\If{$\vert par\vert>1$ }{
    $M\gets LinearRegression(par)$\;
    
    $\mathcal{M} \gets \mathcal{M} + [M]$\;

    $\mathcal{E} \gets \mathcal{E} + [APE(M, par)]$\;
}
    $\mathcal{R} \gets \mathcal{R} + [par.frist]$\;
}
$\mathcal{R} \gets \mathcal{R} + [par.last]$\;
\end{algorithm}

\begin{figure}[t]
  \centering
  \makebox{\includegraphics[width=0.9\columnwidth]{figs/compare-methods.pdf}}
  \vspace{-0.5em}
  \caption{\small{Comparison of PLA and PRA under different data distributions. (a) $E' < E$, indicating PRA performs better. (b) $E < E'$, indicating PLA performs better.}}
  \label{fig:compare-methods}
  \vspace{-1.6em}
\end{figure}

\noindent\subsubsection{\textbf{Comparing PRA and PLA}} In the context of building block approximations, both the Piecewise Linear Approximation (PLA) and the Piecewise Regression Approximation (PRA) rely on the scanning of data, resulting in a linear time complexity of $O(N)$, where $N$ is the number of data points. Each method utilizes closed-form formulas with a time complexity of $O(N)$ for computations within blocks: PLA determines maximum distances, $APE(.,.)$, to construct piecewise linear segments, while PRA computes two-dimensional regression, $LinearRegression(.)$, formulas within blocks.

With respect to space complexity, PLA requires one point per spline segment to be stored since the endpoint of one line serves as the beginning of the next. The distribution of data points influences PLA memory needs; for instance, with a uniform distribution, all data might fit within a single spline (provided its size is smaller than $B_{max}$), thus minimizing memory usage. In contrast, PRA must store both a point and a slope for each regression line, roughly doubling the memory requirement compared to PLA. The complexity of the number of regressions is $O(\frac{N}{B_{max}})$. Consequently, the memory comparison between PLA and PRA depends on the characteristics of the data: PLA may involve fewer blocks and needs just one point per block (linear segment), whereas PRA has to retain two parameters per block.

When comparing the behavior of PRA and PLA during lookups (\circled{7} in Fig.~\ref{fig:DobLIX-arch}), determining which method is superior is challenging, often leading to the interchangeable use of algorithms; in particular, we consider two scenarios that contrast PLA and PRA, as illustrated in Fig.~\ref{fig:compare-methods}, noting that both scenario~(a) and scenario~(b) can occur depending on the sizes of the KV pairs and the distribution of the keys. In the PRA model, the block is written in persistent storage once its size reaches the maximum block size $B_{max}$, whereas the PLA operates under two conditions for flushing: when the approximation error exceeds the error limit $E$, or when the block size reaches $B_{max}$ (the same maximum block size used in PRA). Consequently, depending on the arrangement and distribution of KV, the spline achievable with PLA can potentially lead to a maximum error $E$ that may be higher or lower than the maximum error $E'$ in PRA. As shown in Fig.~\hyperref[fig:compare-methods]{\ref*{fig:compare-methods}a}, PRA can result in $E > E'$, indicating a more accurate approximation than PLA. This directly affects the scope of the last-mile search and the overall efficiency of each algorithm.

This implies that the behavior of KVs beyond the block boundary determined by the maximum error $E$ plays a crucial role. If data points outside the block constrained by $E$ align with the regression trend of the points within the block, the approximation remains accurate. However, there might be situations where data points outside the block behave significantly differently from those within the block. In such cases, as clearly shown in Fig.~\hyperref[fig:compare-methods]{\ref*{fig:compare-methods}b}, this could lead to a less accurate linear regression in PRA, resulting in $E < E'$. This means that PRA has a lower performance than PLA in such scenarios. In some rare cases, all elements in $\mathcal{E}$~(Algorithm~\ref{alg:pra}), denoted as $E'$ for each segment, are equivalent to $E$. In such scenarios, both algorithms exhibit identical last-mile search performance.

\subsection{Serialize and Deserialize Models}
\label{sec:design:serialization}
After training, the LI models are serialized and stored in a metadata block called the Meta Index Block (see Fig.~\ref{fig:sst-format}). This process involves traversing the model tree structure via depth-first search (DFS), serializing each node byte-by-byte. During deserialization, the model reconstructs the tree by determining the type of each node (leaf or internal) and populating the corresponding data structure.

The learned indices generally outperform the binary search in terms of time complexity. Therefore, the size of the set of stored nodes for model approximation is smaller than $O(\log N)$ when reading, while writing the entire dataset takes $O(N)$. As a result, the serialized model size is asymptotically negligible (see \S~\ref{sec:eval:storage}). In practice, RocksDB writes occur in the background flush and compaction process, and since the model is deserialized only once, overall read performance is significantly improved (see \S\ref{sec:eval:throughput}).

\subsection{Last-mile Search Optimization}
\label{sec:design:lastmile}

\begin{figure}[t]
  \centering
\makebox{\includegraphics[width=\columnwidth]{figs/last-mile-optimization.pdf}}
\vspace{-1.6em}
  \caption{\small{Last-mile Search Optimization Flow}}
  \label{fig:last-mile-optimization}
  \vspace{-1.5em}
\end{figure}

The final phase of the search process involves the last-mile search of the recovered block from storage (step \circled{9} in Fig.~\ref{fig:DobLIX-arch}). As illustrated in Fig.~\hyperref[fig:rocksdb-lookup]{\ref*{fig:rocksdb-lookup}c}, this step accounts for more than $40\%$ of the indexing latency in RocksDB. Fig.~\ref{fig:last-mile-optimization} shows how \texttt{DobLIX} optimizes its performance by restoring data from the LI model computation trajectory (step \circled{5} in Fig.~\ref{fig:DobLIX-arch}) to improve the last-mile search process. Initially, the target key is searched within the string LI structure, as explained in \S~\ref{sec:li-storage}. The LI model in \texttt{DobLIX} subsequently provides: \textit{(1)} $Block~b_{P_L^j}$: The block containing the target KV pair.
\textit{(2)} $M(.)$: The LI estimation of the target KV pair index in $KVs Addr$ (Fig.~\ref{fig:DobLIX-arch}).
\textit{(3)} $Level~L$: The level at which the target key was found in the LI model.
\textit{(4)} $Error~E_{P_L^j}$: The maximum range required to search for the target KV pair.

\vspace{3pt}
\noindent
{\small\textbf{Optimizing Search Range.}}
As described in \S~\ref{sec:design:overview}, \texttt{DobLIX} necessitates a coordinate transformation to adapt the model output ($M(.)$) for indexing on the retrieved block ($b_{P_L^j}$). This is achieved by deducting the count of keys in the previous blocks (maintained as the parameter \say{offset} in the metadata of the block introduced in Algs.~\ref{alg:pla}\&~\ref{alg:par}):

\vspace{-0.5em}
{\small
\[
    I_{KV}(.)=M_{adj}(.) = M(.) - b_{P_L^j}.\text{offset}
    \vspace{-0.5em}
\]
}

\noindent
To better illustrate the adjustment, we consider $SST_j$ in Fig.~\ref{fig:prev-designs}, in which the model is trained on the whole SSTable indexes. However, using the above adjustment, the coordination of model output for our solution (i.e., Fig.~\hyperref[fig:prev-designs]{\ref*{fig:prev-designs}d}) is transformed to the retrieved block $B_2$.

\noindent Subsequently, \texttt{DobLIX} performs a binary search within the specified model error range $E_{P_L^j}$ on $M_{adj}(.)$. This error range can be less than the maximum error $E$ if a spline in the PLA method reaches the maximum block size $B_{max}$. In such cases, \texttt{DobLIX} calculates the spline error and incorporates it into the model, transmitting this information to the last-mile search process to streamline the number of key comparisons.

\vspace{1pt}
\noindent
{\small\textbf{Optimizing String Comparison.}}
\texttt{DobLIX} further optimizes the comparisons by avoiding full key comparisons. \texttt{DobLIX} provides the node level where the key is found in its string-LI structure. This level in the tree indicates the common prefix string, $P_L^j$, in the key (Fig.~\ref{fig:last-mile-optimization}). Then \texttt{DobLIX} can ignore this common prefix, as all keys within the retrieved block share the same prefix. Additionally, \texttt{DobLIX} only compares string keys up to their $K$ bytes, ensuring that the key can be identified by comparing only the $K$ byte following the prefix within the error range. Given that $K$ generally consists of $8$ or $16$ bytes as explained in \S~\ref{sec:li-on-strings} for shifts from demanding string comparisons to numerical comparisons.

\subsection{Tuning Agent}
\label{sec:design:agent}
\texttt{DobLIX} employs Q-learning~\cite{watkins1992q}, a lightweight reinforcement learning (RL) algorithm~\cite{mo2023learning}, to dynamically fine-tune the LI and data access parameters. \texttt{DobLIX} determines three key parameters in the creation of SSTables and the training model: \textit{(1)} The choice between using the PLA vs. the PRA. \textit{(2)} The maximum error of the LI in the PLA method ($E$), and \textit{(3)} The maximum size of a block ($B_{max}$).

The state space for the Q-learning agent comprises the index model (PLA with error values ranging from $32$ to $256$, doubling at each step, and PRA, resulting in $5$ distinct values), and the block size ($4KB$ to $32KB$, doubling at each step, yielding $4$ distinct values, as recommended by RocksDB). This creates a total of $20$ possible states. The action space consists of five actions: incrementing, decrementing, or maintaining the current value of the model error and block size.

The RL agent policy for exploring the feasible space is also guided by our optimization method (\S~\ref{sec:dual-objective-optimization}). It prioritizes block size over the approximation error. The policy evaluates the relationship between data block size and its load latency, identifying the first block size that alters this relationship (e.g. $O(log N)$ to $O(N)$) as the upper bound. This effectively reduces the feasible space for exploration by limiting larger block sizes.

The reward function is defined below to consider both system latency and index storage:
\noindent
\scalebox{0.8}{%
\centering{
$R(s, a) = -\nu.Norm(AVG(latency)) - (1 - \nu).Norm(AVG(index~size)) $
}
}

\noindent This reward function is calculated as the negative sum of the normalized average latency and the normalized average of SSTables index size. We use sigmoid normalization to ensure that both latency and index size contribute to the reward on a comparable scale. The weighting parameter $\nu$ controls the relative importance of latency and index size in determining the overall reward (evaluated in \S~\ref{sec:eval:param}). In our experiments, we set $\nu$ as 1 to achieve the lowest latency; however, in some scenarios, storage may be more critical than can be set by lower values of $\nu$.

\begin{algorithm}[t]
\small  % Set font size
\DontPrintSemicolon  % Disable automatic semicolon

\caption{System Tuning Agent}\label{alg:tune}

$Q \gets \text{initializeQTable()}$ \;
$\alpha, \gamma, \epsilon \gets \text{initializeVariables()}$ \;
\While{}{
    Fetch state $s_{t-1}$ \;
    $L_{t}, I_{t} \gets \text{fetchAverageLatencyAndSSTableIndexSize()}$ \;
    $R_{t} \gets \text{calculateReward}(L_{t}, I_{t})$ \;
    Observe state $s_{t}$ \;
    $a' \gets \arg\max_{a \in \text{Actions}} Q(s_{t},a)$ \;
    $Q(s_{t-1}, a_{t-1}) \gets (1-\alpha)Q(s_{t-1}, a_{t-1}) + \alpha\big(R_{t} + \gamma Q(s_{t},a')\big)$ \;
    $A_t \gets \text{getAvailableActions}(s_t)$ \;
    \eIf{$\text{generateRandNumber()} < \epsilon$}{
        $a_t \gets \text{getRandomAction}(A_t)$ \;
    }{
        $a_t \gets \arg\max_{a \in A_t} Q(s_t,a)$ \;
    }
    tuneSystem($a_t$) \;
    $s_t \gets s_{t+1}$ \;
    $\epsilon \gets \text{updateEpsilon}(\epsilon)$ \;
}

\end{algorithm}

Algorithm~\ref{alg:tune} outlines the agent tuning and execution process. The learning procedure begins by configuring the RL hyperparameters: $\alpha$ denotes the learning rate, while $\gamma$ signifies the discount factor that balances immediate and future rewards. An epsilon decay strategy is used by initially setting a high exploration rate ($\epsilon$) to investigate various actions, gradually lowering this value in successive iterations to exploit the optimal actions.
At every $20$ SSTables creation, the following steps occur: First, the agent obtains the average latency of the reads that have reached this level, as well as the index size of the previously created SSTable. Then, it measures the reward and updates it for the chosen action in the previous state. The agent then gets the available actions at that state. Then, depending on the value of $\epsilon$, the agent explores a new action or chooses the best action from the Q-table. In addition, we implement a reset mechanism for the RL agent that reverts $\epsilon$ to its initial value if the distribution undergoes a significant change. This ensures optimal exploration of the space under the new conditions.




\section{Implementation and Evaluation}
\label{sec:evaluation}

We prototype our proposal into a tool \toolName, using approximately 5K lines of OCaml (for the program analysis) and 5K lines of Python code (for the repair). 
In particular, we employ Z3~\cite{DBLP:conf/tacas/MouraB08} as the SMT solver, clingo~\cite{DBLP:books/sp/Lifschitz19} as the ASP solver, and Souffle~\cite{scholz2016fast} as the Datalog engine. %, respectively.
To show the effectiveness, 
we design the experimental evaluation to answer the 
following research questions (RQ):
(Experiments ran on a server with an Intel® Xeon® Platinum 8468V, 504GB RAM, and 192 cores. All the dataset are publicly available from \cite{zenodo_benchmark})

\begin{itemize}[align=left, leftmargin=*,labelindent=0pt]
\item \textbf{RQ1:} How effective is \toolName in verifying CTL properties for relatively small but complex programs, compared to the state-of-the-art tool  \function~\cite{DBLP:conf/sas/UrbanU018}?


\item \textbf{RQ2:} What is the effectiveness of \toolName in detecting real-world bugs, which can be encoded using both CTL and linear temporal logic (LTL), such as non-termination gathered from GitHub \cite{DBLP:conf/sigsoft/ShiXLZCL22} and unresponsive behaviours in protocols  \cite{DBLP:conf/icse/MengDLBR22}, compared with \ultimate~\cite{DBLP:conf/cav/DietschHLP15}?

\item \textbf{RQ3:} How effective is \toolName in repairing CTL violations identified in RQ1 and RQ2? which has not been achieved by any existing tools. 


 

\end{itemize}



% \begin{itemize}[align=left, leftmargin=*,labelindent=0pt]
% \item \textbf{RQ1:} What is the effectiveness of \toolName in verifying CTL properties in a set of relatively small yet challenging programs, compared to the state-of-the-art tools, T2~\cite{DBLP:conf/fmcad/CookKP14},  \function~\cite{DBLP:conf/sas/UrbanU018}, and \ultimate~\cite{DBLP:conf/cav/DietschHLP15}?


% \item \textbf{RQ2:} What is the effectiveness of \toolName in finding  real-world bugs, which can be encoded using CTL properties, such as non-termination 
% gathered from GitHub \cite{DBLP:conf/sigsoft/ShiXLZCL22} and unresponsive behaviours in protocol implementations \cite{DBLP:conf/icse/MengDLBR22}?

% \item \textbf{RQ3:} What is the effectiveness of \toolName in repairing CTL bugs from RQ1--2?

% \end{itemize}

%The benchmark programs are from various sources. More specifically, termination bugs from real-world projects \cite{DBLP:conf/sigsoft/ShiXLZCL22} and CTL analysis \cite{DBLP:conf/fmcad/CookKP14} \cite{DBLP:conf/sas/UrbanU018}, and temporal bugs in real-world protocol implementations \cite{DBLP:conf/icse/MengDLBR22}. 



% \ly{are termination bugs ok? Do we need to add new CTL bugs?}
\subsection{RQ1: Verifying CTL Properties}

% Please add the following required packages to your document preamble:
%  \Xhline{1.5\arrayrulewidth}

\hide{\begin{figure}[!h]
\vspace{-8mm}
\begin{lstlisting}[xleftmargin=0.2em,numbersep=6pt,basicstyle=\footnotesize\ttfamily]
(*@\textcolor{mGray}{//$EF(\m{resp}{\geq}5)$}@*)
int c = *; int resp = 0;
int curr_serv = 5; 
while (curr_serv > 0){ 
 if (*) {  
   c--; 
   curr_serv--;
   resp++;} 
 else if (c<curr_serv){
   curr_serv--; }}
\end{lstlisting} 
\vspace{-2mm}
\caption{A possibly terminating loop} 
\label{fig:terminating_loop}
\vspace{-2mm}
\end{figure}}


%loses precision due to a \emph{dual widening} \cite{DBLP:conf/tacas/CourantU17}, and 

The programs listed in \tabref{tab:comparewithFuntionT2} were obtained from the evaluation benchmark of \function, which includes a total of 83 test cases across over 2,000 lines of code. We categorize these test cases into six groups, labeled according to the types of CTL properties. 
These programs are short but challenging, as they often involve complex loops or require a more precise analysis of the target properties. The \function tends to be conservative, often leading it to return ``unknown" results, resulting in an accuracy rate of 27.7\%. In contrast, \toolName demonstrates advantages with improved accuracy, particularly in \ourToolSmallBenchmark. 
%achieved by the novel loop summaries. 
The failure cases faced by \toolName highlight our limitations when loop guards are not explicitly defined or when LRFs are inadequate to prove termination. 
Although both \function and \toolName struggle to obtain meaningful invariances for infinite loops, the benefits of our loop summaries become more apparent when proving properties related to termination, such as reachability and responsiveness.  




\begin{table}[!t]
\vspace{1.5mm}
\caption{Detecting real-world CTL bugs.}
\normalsize
\label{tab:comparewithCook}
\renewcommand{\arraystretch}{0.95}
\setlength{\tabcolsep}{4pt}  
\begin{tabular}{c|l|c|cc|cc}
\Xhline{1.5\arrayrulewidth}
\multicolumn{1}{l|}{\multirow{2}{*}{\textbf{}}} & \multirow{2}{*}{\textbf{Program}}        & \multirow{2}{*}{\textbf{LoC}} & \multicolumn{2}{c|}{\textbf{\ultimateshort}}   & \multicolumn{2}{c}{\textbf{\toolName}}             \\ \cline{4-7} 
  \multicolumn{1}{l|}{}                           &                                          &                               & \multicolumn{1}{c|}{\textbf{Res.}} & \textbf{Time} & \multicolumn{1}{c|}{\textbf{Res.}} & \textbf{Time} \\ \hline
  1 \xmark                                      & \multirow{2}{*}{\makecell[l]{libvncserver\\(c311535)}}   & 25                            & \multicolumn{1}{c|}{\xmark}      & 2.845         & \multicolumn{1}{c|}{\xmark}      & 0.855         \\  
  1 \cmark                                      &                                          & 27                            & \multicolumn{1}{c|}{\cmark}      & 3.743         & \multicolumn{1}{c|}{\cmark}      & 0.476         \\ \hline
  2 \xmark                                      & \multirow{2}{*}{\makecell[l]{Ffmpeg\\(a6cba06)}}         & 40                            & \multicolumn{1}{c|}{\xmark}      & 15.254        & \multicolumn{1}{c|}{\xmark}      & 0.606         \\  
  2 \cmark                                      &                                          & 44                            & \multicolumn{1}{c|}{\cmark}      & 40.176        & \multicolumn{1}{c|}{\cmark}      & 0.397         \\ \hline
  3 \xmark                                      & \multirow{2}{*}{\makecell[l]{cmus\\(d5396e4)}}           & 87                            & \multicolumn{1}{c|}{\xmark}      & 6.904         & \multicolumn{1}{c|}{\xmark}      & 0.579         \\  
  3 \cmark                                      &                                          & 86                            & \multicolumn{1}{c|}{\cmark}      & 33.572        & \multicolumn{1}{c|}{\cmark}      & 0.986         \\ \hline
  4 \xmark                                      & \multirow{2}{*}{\makecell[l]{e2fsprogs\\(caa6003)}}      & 58                            & \multicolumn{1}{c|}{\xmark}      & 5.952         & \multicolumn{1}{c|}{\xmark}      & 0.923         \\  
  4 \cmark                                      &                                          & 63                            & \multicolumn{1}{c|}{\cmark}      & 4.533         & \multicolumn{1}{c|}{\cmark}      & 0.842         \\ \hline
  5 \xmark                                      & \multirow{2}{*}{\makecell[l]{csound-an...\\(7a611ab)}} & 43                            & \multicolumn{1}{c|}{\xmark}      & 3.654         & \multicolumn{1}{c|}{\xmark}      & 0.782         \\  
  5 \cmark                                      &                                          & 45                            & \multicolumn{1}{c|}{TO}          & -             & \multicolumn{1}{c|}{\cmark}      & 0.648         \\ \hline
  6 \xmark                                      & \multirow{2}{*}{\makecell[l]{fontconfig\\(fa741cd)}}     & 25                            & \multicolumn{1}{c|}{\xmark}      & 3.856         & \multicolumn{1}{c|}{\xmark}      & 0.769         \\  
  6 \cmark                                      &                                          & 25                            & \multicolumn{1}{c|}{Error}       & -             & \multicolumn{1}{c|}{\cmark}      & 0.651         \\ \hline
  7 \xmark                                      & \multirow{2}{*}{\makecell[l]{asterisk\\(3322180)}}       & 22                            & \multicolumn{1}{c|}{\unk}        & 12.687        & \multicolumn{1}{c|}{\unk}        & 0.196         \\  
  7 \cmark                                      &                                          & 25                            & \multicolumn{1}{c|}{\unk}        & 11.325        & \multicolumn{1}{c|}{\unk}        & 0.34          \\ \hline
  8 \xmark                                      & \multirow{2}{*}{\makecell[l]{dpdk\\(cd64eeac)}}          & 45                            & \multicolumn{1}{c|}{\xmark}      & 3.712         & \multicolumn{1}{c|}{\xmark}      & 0.447         \\  
  8 \cmark                                      &                                          & 45                            & \multicolumn{1}{c|}{\cmark}      & 2.97          & \multicolumn{1}{c|}{\unk}        & 0.481         \\ \hline
  9 \xmark                                      & \multirow{2}{*}{\makecell[l]{xorg-server\\(930b9a06)}}   & 19                            & \multicolumn{1}{c|}{\xmark}      & 3.111         & \multicolumn{1}{c|}{\xmark}      & 0.581         \\  
  9 \cmark                                      &                                          & 20                            & \multicolumn{1}{c|}{\cmark}      & 3.101         & \multicolumn{1}{c|}{\cmark}      & 0.409         \\ \hline
  10 \xmark                                      & \multirow{2}{*}{\makecell[l]{pure-ftpd\\(37ad222)}}      & 42                            & \multicolumn{1}{c|}{\cmark}      & 2.555         & \multicolumn{1}{c|}{\xmark}      & 0.933         \\  
  10 \cmark                                      &                                          & 49                            & \multicolumn{1}{c|}{\cmark}        & 2.286         & \multicolumn{1}{c|}{\cmark}      & 0.383         \\ \hline
  11 \xmark  & \multirow{2}{*}{\makecell[l]{live555$_a$\\(181126)}} & 34  & \multicolumn{1}{c|}{\cmark} &  2.715         & \multicolumn{1}{c|}{\xmark}    & 0.513   \\  
  11 \cmark  &     &   37    & \multicolumn{1}{c|}{\cmark} &  2.837         & \multicolumn{1}{c|}{\cmark}      & 0.341 \\ \hline
  12 \xmark  & \multirow{2}{*}{\makecell[l]{openssl\\(b8d2439)}} & 88  & \multicolumn{1}{c|}{\xmark} &  4.15          & \multicolumn{1}{c|}{\xmark}    & 0.78   \\
  12 \cmark  &     &  88     & \multicolumn{1}{c|}{\cmark} &  3.809         & \multicolumn{1}{c|}{\cmark}      & 0.99 \\ \hline
  13 \xmark  & \multirow{2}{*}{\makecell[l]{live555$_b$\\(131205)}} & 83  & \multicolumn{1}{c|}{\xmark} & 2.838         & \multicolumn{1}{c|}{\xmark}    & 0.602     \\  
  13 \cmark  &    &   84     & \multicolumn{1}{c|}{\cmark} &  2.393         & \multicolumn{1}{c|}{\cmark}      & 0.565 \\ \Xhline{1.5\arrayrulewidth}
                                                   & {\bf{Total}}                                  & 1249  & \multicolumn{1}{c|}{\bestBaseLineReal}          & $>$180       & \multicolumn{1}{c|}{\ourToolRealBenchmark}              & 16.01        \\ \Xhline{1.5\arrayrulewidth}
  \end{tabular}
  \end{table}

\subsection{RQ2: CTL Analysis on  Real-world Projects}




Programs in \tabref{tab:comparewithCook} are from real-world repositories, each associated with a Git commit number where developers identify and fix the bug manually. 
In particular, the property used for programs 1-9 (drawn from \cite{DBLP:conf/sigsoft/ShiXLZCL22}) is  \code{AF(Exit())}, capturing non-termination bugs. The properties used for programs 10-13 (drawn from \cite{DBLP:conf/icse/MengDLBR22}) are of the form \code{AG(\phi_1{\rightarrow}AF(\phi_2))}, capturing unresponsive behaviours from the protocol implementation. 
We extracted the main segments of these real-world bugs into smaller programs (under 100 LoC each), preserving features like data structures and pointer arithmetic. Our evaluation includes both buggy (\eg 1\,\xmark) and developer-fixed (\eg 1\,\cmark) versions.
After converting the CTL properties to LTL formulas, we compared our tool with the latest release of UltimateLTL (v0.2.4), a regular participant in SV-COMP \cite{svcomp} with competitive performance. 
Both tools demonstrate high accuracy in bug detection, while \ultimateshort often requires longer processing time. 
This experiment indicates that LRFs can effectively handle commonly seen real-world loops, and \toolName performs a more lightweight summary computation without compromising accuracy. 



%Following the convention in \cite{DBLP:conf/sigsoft/ShiXLZCL22}, t
%Prior works \cite{DBLP:conf/sigsoft/ShiXLZCL22} gathered such examples by extracting 
%\toolName successfully identifies the majority of buggy and correct programs, with the exception of programs 7 and 8. 







{
\begin{table*}[!h]
  \centering
\caption{\label{tab:repair_benchmark}
{Experimental results for repairing CTL bugs. Time spent by the ASP solver is separately recorded. 
}
}
\small
\renewcommand{\arraystretch}{0.95}
  \setlength{\tabcolsep}{9pt}
\begin{tabular}{l|c|c|c|c|c|c|c|c}
  \Xhline{1.5\arrayrulewidth}
  \multicolumn{1}{c|}{\multirow{2}{*}{\textbf{Program}}} & \multicolumn{1}{c|}{\multirow{2}{*}{\shortstack{\textbf{LoC}\\\textbf{(Datalog)}}}} & \multicolumn{3}{c|}{\textbf{Configuration}}                                 & \multicolumn{1}{c|}{\multirow{2}{*}{\textbf{Fixed}}} & \multicolumn{1}{c|}{\multirow{2}{*}{\textbf{\#Patch}}} & \multicolumn{1}{c|}{\multirow{2}{*}{\textbf{ASP(s)}}} & \multirow{2}{*}{\textbf{Total(s)}} \\ \cline{3-5}

  \multicolumn{1}{c|}{}                                  & \multicolumn{1}{c|}{}                              & \multicolumn{1}{c|}{\textbf{Symbols}} & \multicolumn{1}{c|}{\textbf{Facts}} & \multicolumn{1}{c|}{\textbf{Template}} & \multicolumn{1}{c|}{} & \multicolumn{1}{c|}{} & \multicolumn{1}{c|}{}  &                                      \\ \hline

AF\_yEQ5 (\figref{fig:first_Example})                                           & 115                           & 3+0                   & 0+1                & Add                & \cmark     & 1                   & 0.979                              & 1.593                                \\
test\_until.c                                         & 101                            & 0+3                   & 1+0                & Delete                & \cmark     & 1                   & 0.023                              & 0.498                                \\
next.c                                                & 87                            & 0+4                   & 1+0                & Delete                & \cmark     & 1                   & 0.023                              & 0.472                                \\
libvncserver                                          & 118                            & 0+6                   & 1+0                & Delete                & \cmark     & 3                   & 0.049                              & 1.081                                \\
Ffmpeg                                                & 227                           & 0+12                  & 1+0                & Delete                & \cmark     & 4                   & 13.113                              & 13.335                                \\
cmus                                                  & 145                           & 0+12                  & 1+0                & Delete                & \cmark     & 4                   & 0.098                              & 2.052                                \\
e2fsprogs                                             & 109                           & 0+8                   & 1+0                & Delete                & \cmark     & 2                   & 0.075                              & 1.515                                \\
csound-android                                        & 183                           & 0+8                   & 1+0                & Delete                & \cmark     & 4                   & 0.076                              & 1.613                                \\
fontconfig                                            & 190                           & 0+11                  & 1+0                & Delete                & \cmark     & 6                   & 0.098                              & 2.507                                \\
dpdk                                                  & 196                           & 0+12                  & 1+0                & Delete                & \cmark     & 1                   & 0.091                              & 2.006                                \\
xorg-server                                           & 118                            & 0+2                   & 1+0                & Delete                & \cmark     & 2                   & 0.026                              & 0.605                                \\
pure-ftpd                                             & 258                           & 0+21                  & 1+0                & Delete                & \cmark     & 2                   & 0.069                              & 3.590                               \\
live$_a$                                              & 112                            & 3+4                   & 1+1                & Update                & \cmark     & 1                   & 0.552                              & 0.816                                \\
openssl                                               & 315                           & 1+0                   & 0+1                & Add.                & \cmark     & 1                   & 1.188                              & 2.277                                \\
live$_b$                                              & 217                           & 1+0                   & 0+1                & Add                & \cmark     & 1                   & 0.977                              & 1.494                                 \\
  \Xhline{1.5\arrayrulewidth}
\textbf{Total}                                                 & 2491                          &                       &                    &                   &           &                     & 17.437                              & 35.454                               \\ 
  \Xhline{1.5\arrayrulewidth}           
\end{tabular}

\vspace{-2mm}
\end{table*}
}


\subsection{RQ3: Repairing CTL Property Violations} 


\tabref{tab:repair_benchmark} gathers all the program instances (from \tabref{tab:comparewithFuntionT2} and \tabref{tab:comparewithCook}) that violate their specified CTL properties and are sent to \toolName for repair.   
The \textbf{Symbols} column records the number of symbolic constants + symbolic signs, while the \textbf{Facts} column records the number of facts allowed to be removed + added. 
We gradually increase the number of symbols and the maximum number of facts that can be added or deleted. 
The \textbf{Configuration} column shows the first successful configuration that led to finding patches, and we record the total searching time till reaching such configurations. 
We configure \toolName to apply three atomic templates in a breadth-first manner with a depth limit of 1, \ie, \tabref{tab:repair_benchmark} records the patch result after one iteration of the repair. 
The templates are applied sequentially in the order: delete, update, and add. The repair process stops when a correct patch is found or when all three templates have been attempted. 
%without success. 
% Because of this configuration, \toolName only finds one patch for Program 1 (AF\_yEQ5). 
% The patch inserting \plaincode{if (i>10||x==y) \{y=5; return;\}} mentioned in \figref{fig:Patched-program} cannot be found in current configuration, as it requires deleting facts then adding new facts on the updated program.
% The `Configuration' column in \tabref{tab:repair_benchmark} shows the number of symbolic constants and signs, the number of facts allowed to be removed and added, and the template used when a patch is found.

Due to the current configuration, \toolName only finds patch (b) for Program 1 (AF\_yEQ5), while the patch (a) shown in \figref{fig:Patched-program} can be obtained by allowing two iterations of the repair: the first iteration adds the conditional than a second iteration to add a new assignment on the updated program. 
Non-termination bugs are resolved within a single iteration by adding a conditional statement that provides an earlier exit. 
For instance, \figref{fig:term-Patched-program} illustrates the main logic of 1\,\xmark, which enters an infinite loop when \code{\m{linesToRead}{\leq}0}. 
\toolName successfully 
provides a fix that prevents \code{\m{linesToRead}{\leq}0} from occurring before entering the loop. Note that such patches are more desirable which fix the non-termination bug without dropping the loops completely. 
%much like the example shown in  \figref{fig:term-Patched-program}. At the same time, 
Unresponsive bugs involve adding more function calls or assignment modifications. 
%Most repairs were completed within seconds. 

On average, the time taken to solve ASP accounts for 49.2\% (17.437/35.454) of the total repair time. We also keep track of the number of patches that successfully eliminate the CTL violations. More than one patch is available for non-termination bugs, as some patches exit the entire program without entering the loop. 
While all the patches listed are valid, those that intend to cut off the main program logic can be excluded based on the minimum change criteria. 
After a manual inspection of each buggy program shown in \tabref{tab:repair_benchmark}, we confirmed that at least one generated patch is semantically equivalent to the fix provided by the developer. 
As the first tool to achieve automated repair of CTL violations, \toolName successfully resolves all reported bugs. 



\begin{figure}[!t]
\begin{lstlisting}[xleftmargin=6em,numbersep=6pt,basicstyle=\footnotesize\ttfamily]
void main(){ //AF(Exit())
  int lines ToRead = *;
  int h = *;
  (*@\repaircode{if ( linesToRead <= 0 )  return;}@*)
  while(h>0){
    if(linesToRead>h)  
        linesToRead=h; 
    h-=linesToRead;} 
  return;}
\end{lstlisting}
\caption{Fixing a Possible Hang Found in libvncserver \cite{LibVNCClient}}
\label{fig:term-Patched-program}
\end{figure}



%\section{Related Work}
%\label{sec:related-work}

%\subsection{Background}

%Defect detection is critical to ensure the yield of integrated circuit manufacturing lines and reduce faults. Previous research has primarily focused on wafer map data, which engineers produce by marking faulty chips with different colors based on test results. The specific spatial distribution of defects on a wafer can provide insights into the causes, thereby helping to determine which stage of the manufacturing process is responsible for the issues. Although such research is relatively mature, the continual miniaturization of integrated circuits and the increasing complexity and density of chip components have made chip-level detection more challenging, leading to potential risks\cite{ma2023review}. Consequently, there is a need to combine this approach with magnified imaging of the wafer surface using scanning electron microscopes (SEMs) to detect, classify, and analyze specific microscopic defects, thus helping to identify the particular process steps where defects originate.

%Previously, wafer surface defect classification and detection were primarily conducted by experienced engineers. However, this method relies heavily on the engineers' expertise and involves significant time expenditure and subjectivity, lacking uniform standards. With the ongoing development of artificial intelligence, deep learning methods using multi-layer neural networks to extract and learn target features have proven highly effective for this task\cite{gao2022review}.

%In the task of defect classification, it is typical to use a model structure that initially extracts features through convolutional and pooling layers, followed by classification via fully connected layers. Researchers have recently developed numerous classification model structures tailored to specific problems. These models primarily focus on how to extract defect features effectively. For instance, Chen et al. presented a defect recognition and classification algorithm rooted in PCA and classification SVM\cite{chen2008defect}. Chang et al. utilized SVM, drawing on features like smoothness and texture intricacy, for classifying high-intensity defect images\cite{chang2013hybrid}. The classification of defect images requires the formulation of numerous classifiers tailored for myriad inspection steps and an Abundance of accurately labeled data, making data acquisition challenging. Cheon et al. proposed a single CNN model adept at feature extraction\cite{cheon2019convolutional}. They achieved a granular classification of wafer surface defects by recognizing misclassified images and employing a k-nearest neighbors (k-NN) classifier algorithm to gauge the aggregate squared distance between each image feature vector and its k-neighbors within the same category. However, when applied to new or unseen defects, such models necessitate retraining, incurring computational overheads. Moreover, with escalating CNN complexity, the computational demands surge.

%Segmentation of defects is necessary to locate defect positions and gather information such as the size of defects. Unlike classification networks, segmentation networks often use classic encoder-decoder structures such as UNet\cite{ronneberger2015u} and SegNet\cite{badrinarayanan2017segnet}, which focus on effectively leveraging both local and global feature information. Han Hui et al. proposed integrating a Region Proposal Network (RPN) with a UNet architecture to suggest defect areas before conducting defect segmentation \cite{han2020polycrystalline}. This approach enables the segmentation of various defects in wafers with only a limited set of roughly labeled images, enhancing the efficiency of training and application in environments where detailed annotations are scarce. Subhrajit Nag et al. introduced a new network structure, WaferSegClassNet, which extracts multi-scale local features in the encoder and performs classification and segmentation tasks in the decoder \cite{nag2022wafersegclassnet}. This model represents the first detection system capable of simultaneously classifying and segmenting surface defects on wafers. However, it relies on extensive data training and annotation for high accuracy and reliability. 

%Recently, Vic De Ridder et al. introduced a novel approach for defect segmentation using diffusion models\cite{de2023semi}. This approach treats the instance segmentation task as a denoising process from noise to a filter, utilizing diffusion models to predict and reconstruct instance masks for semiconductor defects. This method achieves high precision and improved defect classification and segmentation detection performance. However, the complex network structure and the computational process of the diffusion model require substantial computational resources. Moreover, the performance of this model heavily relies on high-quality and large amounts of training data. These issues make it less suitable for industrial applications. Additionally, the model has only been applied to detecting and segmenting a single type of defect(bridges) following a specific manufacturing process step, limiting its practical utility in diverse industrial scenarios.

%\subsection{Few-shot Anomaly Detection}
%Traditional anomaly detection techniques typically rely on extensive training data to train models for identifying and locating anomalies. However, these methods often face limitations in rapidly changing production environments and diverse anomaly types. Recent research has started exploring effective anomaly detection using few or zero samples to address these challenges.

%Huang et al. developed the anomaly detection method RegAD, based on image registration technology. This method pre-trains an object-agnostic registration network with various images to establish the normality of unseen objects. It identifies anomalies by aligning image features and has achieved promising results. Despite these advancements, implementing few-shot settings in anomaly detection remains an area ripe for further exploration. Recent studies show that pre-trained vision-language models such as CLIP and MiniGPT can significantly enhance performance in anomaly detection tasks.

%Dong et al. introduced the MaskCLIP framework, which employs masked self-distillation to enhance contrastive language-image pretraining\cite{zhou2022maskclip}. This approach strengthens the visual encoder's learning of local image patches and uses indirect language supervision to enhance semantic understanding. It significantly improves transferability and pretraining outcomes across various visual tasks, although it requires substantial computational resources.
%Jeong et al. crafted the WinCLIP framework by integrating state words and prompt templates to characterize normal and anomalous states more accurately\cite{Jeong_2023_CVPR}. This framework introduces a novel window-based technique for extracting and aggregating multi-scale spatial features, significantly boosting the anomaly detection performance of the pre-trained CLIP model.
%Subsequently, Li et al. have further contributed to the field by creating a new expansive multimodal model named Myriad\cite{li2023myriad}. This model, which incorporates a pre-trained Industrial Anomaly Detection (IAD) model to act as a vision expert, embeds anomaly images as tokens interpretable by the language model, thus providing both detailed descriptions and accurate anomaly detection capabilities.
%Recently, Chen et al. introduced CLIP-AD\cite{chen2023clip}, and Li et al. proposed PromptAD\cite{li2024promptad}, both employing language-guided, tiered dual-path model structures and feature manipulation strategies. These approaches effectively address issues encountered when directly calculating anomaly maps using the CLIP model, such as reversed predictions and highlighting irrelevant areas. Specifically, CLIP-AD optimizes the utilization of multi-layer features, corrects feature misalignment, and enhances model performance through additional linear layer fine-tuning. PromptAD connects normal prompts with anomaly suffixes to form anomaly prompts, enabling contrastive learning in a single-class setting.

%These studies extend the boundaries of traditional anomaly detection techniques and demonstrate how to effectively address rapidly changing and sample-scarce production environments through the synergy of few-shot learning and deep learning models. Building on this foundation, our research further explores wafer surface defect detection based on the CLIP model, especially focusing on achieving efficient and accurate anomaly detection in the highly specialized and variable semiconductor manufacturing process using a minimal amount of labeled data.



{\footnotesize \bibliographystyle{acm}
\bibliography{sample}}

\end{document}
