\section{Related Work}
\label{sec:related-work}

%\noindent{\small\textbf{Read Optimization on LSM-tree-based KV Stores.}}
The read performance of LSM-tree is compromised due to its multilayer structure. Research has focused on optimizing it through three main strategies: filter optimization, cache optimization, and index optimization. \textit{(1) Filter Optimization.}  Several works utilize filters to skip unnecessary reading since the filters do not return a false negative~\cite{li2019elasticbf, wang2024grf}. However, the filters suffer from high false positive rates and excessive memory usage. \textit{(2) Cache Optimization.} LSbM-tree~\cite{teng2017lsbm} adds a buffer to minimize cache invalidations due to compactions. AC-Key~\cite{wu2020ac} introduces an adaptive caching algorithm to adjust the cache size based on the workload. Although caching accelerates the lookup of recently accessed data, it consumes significant storage as the cache size increases. \textit{(3) Index Optimization.} SLM-DB~\cite{kaiyrakhmet2019slm} implemented a persistent global B+tree index on NVM, and Kvell~\cite{lepers2019kvell} adopts various memory index structures. Several studies~\cite{Bourbon2020, leaderkv2024, TridentKV2022} also built an LI using static data stored in SSTables to speed up read queries on SSTables. However, these studies aim to improve indexing without considering their system adjustment on the data access part for data retrieval from persistent memory. 
%They either increase the block needed to read from the storage or read multiple storage blocks to reach a target KV. They also lack supporting variable-sized key-values, making them impractical in real applications. \texttt{DobLIX}, conversely, introduces a double-objective LI mechanism on LSM-tree that optimizes indexing while considering the storage data access part, ensuring read only one block from the storage and guaranteeing a maximum block size. \texttt{DobLIX} also supports variable-sized key-values of any type.

% \noindent{\small\textbf{LI.}}
% The structure of the LI proposed by Tim Kraska~\cite{kraska2018case} posits that traditional indexes can be replaced with machine learning models to enhance performance. Although this approach has shown promise, the initial implementation has limitations: it is read-only, single-threaded, operates only in memory, and does not support string keys. Research such as ~\cite{ding2020alex,tang2020xindex} provides techniques for improving writing performance and concurrency in LIs, and \cite{wang2020sindex} focuses on enabling support for string keys. However, these studies primarily address scenarios in which all data reside in memory. In contrast, recent investigations \cite{lidisk2024sigmod, wipe22024, apex2021, liDisk2024} indicate that their efficacy diminishes when data is stored in persistent memory. APEX~\cite{apex2021} and PLIN~\cite{zhang2022plin} are designed in persistent memory. Some works integrate LIs with hardware to improve hardware performance. ROLEX~\cite{li2023rolex} integrates the LIs into disaggregated memory systems to process data requests using one-sided Remote Direct Memory Access (RDMA) operations. Unlike previous studies, \texttt{DobLIX} introduces a double-objective LI method aimed at optimizing the read performance of LSM-tree-based KV stores considering indexing and access to storage data.



\section{Conclusion}
\label{sec:conclusion}
\texttt{DobLIX} presents a novel index architecture for LSM-trees, enhancing both indexing efficiency and data retrieval. We show that \texttt{DobLIX} markedly increases throughput in RocksDB, a widely used LSM-tree-based KV store. Redesigning index optimization introduces new possibilities for system engineering. This innovative method, which addresses two specific goals, provides a new perspective on designing systems with multiple seemingly unrelated objectives. Future research will explore multi-objective optimization techniques that consider various parameter combinations concurrently.