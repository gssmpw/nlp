\begin{abstract}
This study explores how to enhance the reasoning capabilities of large language models (LLMs) in knowledge base question answering (KBQA) by leveraging Monte Carlo Tree Search (MCTS).
Semantic parsing-based KBQA methods are particularly challenging as these approaches require locating elements from knowledge bases and generating logical forms, demanding not only extensive annotated data but also strong reasoning capabilities.
Although recent approaches leveraging LLMs as agents have demonstrated considerable potential, these studies are inherently constrained by their linear decision-making processes.
To address this limitation, we propose a MCTS-based framework that enhances LLMs' reasoning capabilities through tree search methodology.
We design a carefully designed step-wise reward mechanism that requires only direct prompting of open-source instruction LLMs without additional fine-tuning.
Experimental results demonstrate that our approach significantly outperforms linear decision-making methods, particularly in low-resource scenarios.
Additionally, we contribute new data resources to the KBQA community by annotating intermediate reasoning processes for existing question-SPARQL datasets using distant supervision.
Experimental results on the extended dataset demonstrate that our method achieves comparable performance to fully supervised models while using significantly less training data.
\end{abstract}
