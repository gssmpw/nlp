\section{Related Work}
\label{2}

In this section, we review related works on stylus-shaped interfaces for haptic feedback and discuss haptic perception via the stylus.
We also briefly overview the current technologies of soft actuators and mechanisms to contextualize our device design.
%このセクションでは、まず触覚フィードバック用のスタイラス型インターフェースに関する関連研究をレビューし、次にスタイラスによる触覚知覚について説明します。また、デバイス設計の位置づけを明確にするために、ソフトアクチュエータの現在の技術についても簡単に説明します。

\subsection{Stylus-Shaped Interface for Haptic Feedback}
\label{2_1}

Several haptic systems have been proposed to generate sensations of touching virtual objects~\cite{MultimodalHapticDisplay}, one of which involves the use of stylus-shaped interfaces to deliver haptic feedback. 
A notable example is the Touch X~\cite{Touchx} by 3D Systems, which provides force feedback via a stylus in an indirect manner when users interact with objects displayed on a PC screen. 
In contrast, Withana \etal\ introduced ImpAct,~\cite{ImpAct} an interface that can change its length when pushed directly against a surface to provide kinesthetic haptic sensation to the hand.
Haptylus~\cite{Haptylus} employs similar mechanisms with ImpAct that can alter the length of the stylus, but it incorporates vibrations via a voice coil motor to simulate the texture of virtual objects. 
FeelPen~\cite{FeelPen} offers a wide range of sensations such as viscosity, roughness, friction, and temperature, combining various modalities to create a more immersive experience.
%仮想空間で感覚を生成するために、いくつかの触覚提示方法が提案されている。顕著な例として、3D Systems社の ``Touch X''があり、ユーザーがPC画面上に表示されたオブジェクトとインタラクションする際に、スタイラスを通してフォースフィードバックを提供する。Withanaらは、触覚フィードバックを提供するために、表面に押し当てたときにスタイラス先端の長さを調整するデバイス、 ``ImpAct''を発表した。同様に、長坂らの「Haptylus」は、スタイラス先端の長さを変え、ボイスコイルモーターを介して振動を取り入れ、仮想物体の質感をシミュレートする。Benceらの ``FeelPen''は、粘性、ざらつき、摩擦、温度など幅広い感覚を提供し、さまざまなモダリティを組み合わせることで、より没入感のある体験を生み出す。
\del{
These interfaces use internal power units to deliver haptic feedback directly to a user’s hand, thereby reproducing a virtual sense of contact. 
}
%これらのデバイスは、内蔵のパワーユニットを使ってユーザーの手に直接触覚フィードバックを与え、バーチャルな接触感覚を再現します。
% \add{
% These interfaces rely on internal power units to directly transmit haptic feedback to the user’s hand, thereby effectively recreating the sensation of contact with virtual objects.
% }
%これらのデバイスは、内部の電源ユニットを使用して触覚フィードバックをユーザーの手に直接伝え、仮想オブジェクトとの接触感覚を効果的に再現します。
\add{
Lee \etal\ proposed a Haptic Pen~\cite{Hapticpen} that improves user interaction with touch screens by delivering localized vibrations and force cues. 
In a similar vein, Fellion \etal\ introduced FlexStylus~\cite{FlexStylus}, a stylus interface that incorporates bend input as an additional interaction modality, thereby broadening the potential applications of pen-based systems.
}
%Leeらは、局所的な振動と力の合図を提供することでタッチスクリーンとのユーザーインタラクションを改善するHaptic Pen [23]を提案しました。同様に、Fellionらは、追加のインタラクション様式として曲げ入力を組み込んだスタイラスインターフェースであるFlexStylus [9]を発表し、ペンベースシステムの潜在的な用途を広げました。
\add{
Through these developments, stylus-shaped interfaces have demonstrated their ability to enhance touch-based interactions with screens by offering diverse haptic feedback mechanisms and interaction capabilities that are tailored to various scenarios.
}
%これらの開発を通じて、スタイラス型インターフェースは、さまざまなシナリオに合わせて調整された多様な触覚フィードバックメカニズムとインタラクション機能を提供することで、画面とのタッチベースのインタラクションを強化する能力を実証しました。

\del{
In contrast, some interfaces create varied haptic experiences by altering the characteristics of the stylus itself. 
}
%しかし、これらの実装では、スタイラスの硬さは一定のままである。
\add{
In a related approach, some studies focused on altering the physical characteristics of the stylus to provide varied haptic experience. 
}
%関連するアプローチとして、いくつかの研究では、スタイラスの物理的特性を変更してさまざまな触覚体験を提供することに焦点を当てました。
Liu \etal\ proposed the FlexStroke~\cite{FlexStroke}, a pen interface that uses a jamming transition mechanism to replicate the haptic sensations of a Chinese brush, an oil brush, and crayon by controlling the air pressure at the tip of the interface, thereby providing variable stiffness. 
%Liuらは、「FlexStroke」というペンインターフェースを提案しました。このインターフェースは、デバイスの先端の空気圧を制御することによって硬さを変化させ、中国筆、油筆、クレヨンの触感を再現するジャミング遷移機構を使用しています。
\add{
Kara presented the VnStylus~\cite{VnStylus}, a haptic stylus that integrates variable tip compliance to enhance haptic rendering of virtual environments.
}
%Karaは、可変先端コンプライアンスを統合して仮想環境の触覚レンダリングを強化する触覚スタイラスである VnStylusを発表しました。
Moreover, we have been proposing a stylus-shaped haptic interface, Transtiff~\cite{ogura_uist, ogura_siggraph}, with a mechanism that incorporates an artificial muscle.
The interface features a joint that controls its stiffness, thus altering the force required for bending.
%また,小倉らは,人工筋肉機構を備えた棒状触覚デバイス ``Transtiff''を提案している.棒の中継部に硬軟が変化する関節を作成し,曲げるのに必要な力を変化させている.
Although these studies successfully control haptic feedback by varying the stylus stiffness, they did not investigate how varying stylus stiffness affects the perception of objects touched via the stylus.
%これらの研究ではスタイラス自体の硬さを変えることで触覚フィードバックを制御する方法が提案されている.しかし,主にスタイラス型インターフェースの技術的な実装に焦点を当てており、剛性の異なるスタイラスを介した物体の知覚を十分に調査していません。
Specifically, it remains unclear whether users perceive changes in stiffness as an attribute of the stylus or an object touched by the stylus.  

Therefore, we believe that clarifying how users perceive stiffness when interacting with objects via styluses of different stiffness levels will deepen our understanding of haptic presentation via stylus-shaped interfaces and unlock new possibilities for application in VR.
%したがって、異なる硬さのスタイラスを通してオブジェクトとインタラクションする際に、ユーザーがどのように硬さを知覚するかを明らかにすることは、スタイラス型デバイスを使用した触覚提示の理解を深め、VR環境でのアプリケーションの新たな可能性を引き出すと信じています。

\subsection{Haptic Perception via Stylus}
\label{2_2}

Regarding haptic perception via a stylus, LaMotte \etal\ demonstrated that stiffness discrimination is possible when touching a flexible object via a stylus, with discrimination accuracy particularly enhanced by making a tapping motion~\cite{Softness_Discrimination}. 
Similarly, Klatzky \etal\ investigated roughness perception when touching a rough surface through a rigid tool (\eg, a stylus-shaped probe) and found that roughness discrimination was achievable through a rigid body, albeit with less accuracy than when using a fingertip~\cite{klatzky1999tactile}. 
Chung \etal\ reported comparable results with a rubber-tipped stylus. 
Kato \etal\ further showed that their proposed chopstick-like device for virtual weight presentation resulted in a lower weight discrimination rate compared to the perception of real objects~\cite{Kato}.
%棒を介した触覚知覚についてLaMotteらは,スタイラスを介して柔軟な物体に触れる際の硬軟識別が可能であることを示し,「叩く」動作をすることで硬軟の識別率が特に増加することを明らかにしている \cite{Softness_Discrimination}.Klatzkyらは,剛体(棒状のプローブ等)を通して粗面に触れる際の粗さ知覚を調査しており,指腹より精度は低いが,剛体を通しても粗さの識別が可能なことを明らかにしている \cite{klatzky1999tactile}.Chungらは,先端がゴム製のスタイラスと弾性表面のディスプレイがタップ動作に最適であることを示している \cite{CHUNG201512}.Katoらは,提案する箸型デバイスの仮想重量感提示によって,実物体の重量知覚よりも重量の識別率が低下する錯覚が生じると報告している \cite{Kato}.

Numerous studies have examined haptic perception using force-sensing interfaces similar to the one such as Touch X~\cite{Touchx}.
Heather \etal\ reported that when recreating a surface in a virtual environment using a 3D haptic interface, the perceived intensity of slipperiness, stiffness, and roughness could be varied by presenting transient friction and contact stimuli along with surface textures~\cite{Importance}.
%\ref{2-1}節で前述した力覚デバイス \cite{Touchx}と同機能を持つ装置を用いて知覚調査をしている事例も多い.Heatherらは,3次元触覚デバイスを用いてバーチャル空間の表面を再現する際,摩擦・接触刺激の過渡現象・表面テクスチャを提示することで,ユーザが感じる滑りやすさ・硬さ・粗さの強度を変化させられると報告している \cite{Importance}.
\add{
These findings highlight the significance of combining multiple sensory modalities in creating realistic virtual environments.
}
%これらの研究結果は、リアルな仮想環境を作成するために複数の感覚様式を組み合わせることの重要性を強調しています。

\add{
While these studies focused on the mechanical properties of objects and tools, the interaction between visual and haptic cues also plays a key role in shaping users' perception of object properties even when physical feedback is constant. A notable example is pseudo-touch, in which visual cues affect haptic sensations.  
%これらの研究は物体や道具の機械的特性に焦点を当てていますが、視覚的手がかりと触覚的手がかりの相互作用も、物理的なフィードバックが一定である場合でも、物体の特性に対するユーザーの認識を形成する上で重要な役割を果たします。注目すべき例は、視覚的手がかりが触覚感覚に影響を与える疑似タッチです。
Although pseudo-touch has been extensively studied in fingertip interactions~\cite{Pseudo-Stiffness}, its role in stylus-based systems remains underexplored. Our findings bridge this gap by showing how compliant styluses leverage visual cues to enhance haptic interaction fidelity, particularly through stiffness control.  
%疑似タッチは指先インタラクションで広く研究されてきましたが~\cite{疑似剛性}、スタイラスベースのシステムでの役割はまだ十分に調査されていません。私たちの研究結果は、柔軟なスタイラスが視覚的手がかりを活用して、特に動的剛性制御を通じて触覚インタラクションの忠実度を高める方法を示すことで、このギャップを埋めています。
}

\add{
Prior studies have largely overlooked the influence of stylus stiffness on haptic perception. This study addresses this gap by examining how varying stylus stiffness affects perception, laying the groundwork for designing haptic interfaces that use stylus stiffness as a controllable parameter to enhance virtual interaction.
%これまでの研究では、スタイラスの剛性が触覚知覚に与える影響はほとんど見過ごされてきました。この研究では、スタイラスの硬さの変化が知覚にどのように影響するかを調べることでこのギャップに対処し、仮想インタラクションを強化するためにスタイラスの硬さを制御可能なパラメータとして使用する触覚インターフェースを設計するための基礎を築きます。
}

\del{
Although various studies have investigated haptic perception through styluses, to our knowledge, none have explored how the stiffness of the stylus itself influences haptic perception. 
In contrast, this study investigates the effects on haptic perception when an object is touched with styluses of different stiffness and applies these findings to the design of haptic interfaces.
}
%このように,棒を介した触覚知覚について複数の調査がされているものの,いずれも棒自体の硬軟が異なる場合の触覚知覚については言及されていない.対して,本研究では,硬軟の異なる棒を用いて物体に触れたときの触覚知覚に与える影響を調査して,触覚インタフェースに応用する.

\subsection{Soft Actuators \add{and Mechanisms}}\label{2_3}

Research on soft actuators \add{and mechanisms} for joints capable of flexibly altering their stiffness has progressed, contributing to advancements in soft robotics technology \add{and relevant fields}.
The commonly used actuation method is controlling the expansion, contraction, and flexion of actuators through pneumatic systems, which have been applied in robotic hands~\cite{abondance2020dexterous} and power-assist devices~\cite{yap2015soft}. 
Among these, pneumatic actuators, specifically ``McKibben-type artificial muscles ''~\cite{tondu2012modelling, zhang2017modeling, Schulte1961TheCO, meller2016improving}, are frequently used.
%ソフトロボティクス技術の発展に伴い,柔軟な剛性変化が可能な関節の研究が進んでいる.特に,空気圧によってアクチュエータの伸縮や屈曲を制御する手法はよく用いられており,ロボットハンドや \cite{abondance2020dexterous},パワーアシストデバイスなどに利用されている \cite{yap2015soft}.これらの空気圧アクチュエータは,2層のチューブ内で流体を加圧することにより軸方向へ収縮を起こす ``マッキベン型人工筋肉'' \cite{tondu2012modelling, zhang2017modeling, Schulte1961TheCO, meller2016improving}が提案されている. 
Devices utilizing McKibben-type artificial muscles have been applied in various applications, such as tele-surgery~\cite{ForcepsManipulator}, shape-changing interfaces~\cite{MckkibenInteraction}, and haptic gloves~\cite{Use_of_McKibben_Muscle}.

Others have proposed the combination of pneumatic soft actuators with jamming transitions. 
Wall \etal\ introduced a soft actuator using layer jamming, a technique that induces a jamming transition in overlapping sheets, to achieve significant stiffness changes~\cite{layerjaming}. 
%他にも、空気圧式ソフトアクチュエータとジャミング遷移の組み合わせを提案した研究者もいる。Wallらは、重なり合ったシートにジャミング遷移を誘発する技術であるレイヤージャミングを使用して、大幅な剛性の変化を実現するソフトアクチュエータを導入した。
This actuator allows for more efficient modulation of stiffness and softness than conventional approaches. 
However, it limits the actuators' bending direction to perpendicular to the sheet layers, making it difficult to bend parallel to them.
%このアクチュエータは、従来のアプローチよりも効率的に剛性と柔らかさを調整できます。ただし、アクチュエータの曲げ方向はシート層に対して垂直に制限されるため、シート層と平行に曲げることは困難です。
To address this, Yang \etal\ developed a system that can 3D print variable stiffness jamming interfaces in arbitrary shapes~\cite{MultiJam}. 
By configuring the arrangement of membranes and beads, this system can print haptic interfaces and shape-changing controllers that function like soft actuators.
However, a common challenge for jamming-based mechanisms is the need for a large external air compressor.
%これに対処するため、Yang らは、任意の形状の可変剛性ジャミング インターフェースを 3D プリントできるシステムを開発しました ~\cite{MultiJam}。このシステムは、膜とビーズの配置を構成することで、ソフト アクチュエータのように機能する触覚インターフェースと形状変更コントローラーをプリントできます。ただし、ジャミング ベースのメカニズムに共通する課題は、大型の外部エアコンプレッサーが必要になることです。

In response to the challenge of requiring external compressor, many soft actuators utilizing the characteristics of functional materials have been developed. 
For example, low-melting-point alloy (LMPA)~\cite{hao2018eutectic} and shape-memory polymers (SMP)~\cite{zhang2019fast} have been used as actuators, where temperature control to adjust the stiffness variation.
Magnetorheological (MR) fluids, which change the stiffness based on the strength of the applied magnetic field~\cite{de2011magnetorheological}, is also widely used.  
Ohba \etal\ proposed a linear motion joint that enables soft motion by adjusting the holding force of a spring through the magnetic field control of the MR fluid surrounding a compression spring~\cite{ohba2012elastic}.
Additionally, electrorheological (ER) fluids, which change liquid viscosity by an electric field, are used in soft robots, often in contrast to MR fluids~\cite{electrorheological}. 
However, these functional materials can be difficult to obtain and manage, limiting their ease of use.
%機能性材料の持つ特徴を利用するソフトアクチュエータも多く開発されている.Haoらは,低融点合金(LMPA)を埋め込み,ヒータで液体--固体の状態変化を起こすことで,従来よりも剛性の変化幅が大きいアクチュエータを提案している \cite{hao2018eutectic}.Zhangらは,形状記憶ポリマー(SMP)をアクチュエータに埋め込むことにより,温度制御でポリマーの剛性を変化させ,耐荷重を向上させている \cite{zhang2019fast}.また,磁界の強度によって剛性が変化する流体であるMR流体 \cite{de2011magnetorheological}もよく利用されている.大場らは,圧縮ばねの周りに満たしたMR流体にかける磁界を制御することで,ばねの保持力を変化させ,やわらかい動作を実現する直動型関節を提案している \cite{ohba2012elastic}.一方で,電場を制御することで液体の粘性が変化する電気粘性流体(ER流体)も,ソフトロボットにおける構成要素として用いられており,MR流体との使い分けが進んでいる \cite{electrorheological}.しかし,これらの機能性材料を用いた手法に関しては,入手や管理が難しく,容易に使用できないという難点がある.

\add{
As another stiffness-changing mechanism, Ryu \etal\ proposed ElaStick~\cite{ElaStick}, a handheld device that dynamically modulates the stiffness of four elastic tendons for 2-DoF haptic feedback, simulating flexible objects, but its size limits integration into stylus-shaped interfaces.
}
%剛性を変える別のメカニズムとして、Ryuらは、柔軟なオブジェクトをシミュレートする 2-DoF 触覚フィードバックのために 4 つの弾性腱の剛性を動的に調整するハンドヘルド デバイスである ElaStick を提案しましたが、そのサイズによりスタイラス形状のインターフェイスへの統合が制限されます。

While methods like layer jamming and functional materials have limitations, we believe that artificial muscles, controlled by simple fluid manipulation, offer a more suitable and straightforward solution for stiffness control in stylus-shaped interfaces.
Consequently, we employed McKibben-type artificial muscles to develop an interface capable of controlling stiffness.
\add{The details of the proposed interface are described in \cref{4}.}
%以上のように,柔軟な剛性変化関節を実現しようとする研究は多い.レイヤージャミングを用いた手法では,関節が曲がる方向が限定されてしまい,触覚デバイスとして制限が生じてしまう.また,機能性材料を用いた手法に関しては,入手や管理が難しいという難点がある.これらの点から,単純な流体操作のみで実現できる人工筋肉の構造が,触覚デバイスへの応用に適していると考えられる.