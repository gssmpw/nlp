\section{Conclusion}
\label{sec:conclusion}
In this paper, we proposed \MethodName{}, a novel radiance field deblurring approach that leverages DNN-based deblurring to overcome the limitations of the linear blur model of previous methods.
To integrate DNN-based deblurring with radiance field construction, we presented a novel RF-guided deblurring scheme and the framework that alternates between RF-guided deblurring and radiance field training.
We also presented \SynthDataName{}, the first large-scale dataset for training and evaluating radiance field deblurring frameworks.
Experimental results showed that our method outperforms others, handling both camera motion blur and defocus blur effectively with highly reduced computation time.

\paragraph{Limitations and future work}
Our method leverages deblurring networks to remove blur from training views, but restoring images with more severe blur than in the training data remains challenging. 
Our RF-guided deblurring uses only rendered images from each input view for guidance. 
Imposing additional priors on other views and incorporating additional information, e.g., depth maps rendered from radiance fields, may further improve the deblurring performance.

\paragraph{Acknowledgements}
This work was supported by Institute of Information \& communications Technology Planning \& Evaluation (IITP) grants funded by the Korea government(MSIT) (RS-2024-00457882, AI Research Hub Project, RS-2019-II191906, Artificial Intelligence Graduate School Program(POSTECH), RS-2024-00437866, ITRC(Information Technology Research Center)).