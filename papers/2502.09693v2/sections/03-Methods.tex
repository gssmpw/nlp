\subsection{Study Design}
\subsubsection{Study Overview}

The study aims to explore the impact of GenAI mediation on the argumentative behavior of individuals with opposing viewpoints in online forums. We chose a controversial topic among soccer fans "Messi vs. Ronaldo: Who is better?" as our case because this topic is a long-standing and heated debate among soccer fans. To provide the space for the online debate, we created a forum on the "Forumotion"~\footnote{Forumotion:~\url{https://www.forumotion.com/}} platform where we focused on the conflicts that arise when fans support different players to study how individuals with opposing viewpoints engage in debates with the help and support of ChatGPT.


Instead of constructing a large online community, we designed a small online forum that accommodates only three participants per session~\cite{govers_ai-driven_2024, kim_influence_2013}, as it allowed us to focus on the essence of the debate between the participants as well as each of their interactions with ChatGPT in assisting them with the debate . Hence, it promoted more detailed observation and analysis of individual interactions with both ChatGPT and other participants, aiming to achieve a nuanced understanding of interaction behaviors in AI-mediated online debates, rather than examining the broad impacts of the online community on individuals. Each study session in this forum was divided into two parts (\autoref{fig2}), which were conducted sequentially. In both parts of the study session, participants were encouraged to demonstrate their advocacy and form their arguments well. Three researchers were each responsible for guiding, monitoring, and interviewing one of the three participants in each study session. To control the time spent waiting for the responses of the individuals, we strongly recommend that participants write posts of fewer than 100 words (count = 256 posts, avg = 69.6 words). An overview of the number of posts in the study can be found in ~\autoref{C}.

At the beginning of each session, all participants were guided by the researchers via instruction slides, where the study instructions were detailed (\autoref{A}). All participants were then guided to log in to the forum after they had familiarized themselves with the study procedure and the assigned tasks. We required participants to use ChatGPT (GPT-4o) as the only external information provider to support writing posts during the online debate, while prohibiting any other use of external resources (e.g., search engines). However, we did not impose any restrictions on the frequency of ChatGPT usage. In addition, as we intentionally focused on text-based debates, we only allowed participants to use plain text (including emojis) to create posts on the forum, while their interactions with ChatGPT were not restricted in this regard.


The entire study including the interview was fully conducted online via Microsoft Teams~\footnote{Microsoft Teams:~\url{https://www.microsoft.com/teams/}}. Each meeting room was exclusively occupied by one researcher and one participant, with no other individuals present. We required all participants to share their monitor display screens throughout the session but did not mandate the use of cameras due to privacy considerations. To simulate an online debate in the real world, all participants were only allowed to communicate with the other participants through forum posts. Upon completing the study, participants were compensated for their time.

\begin{figure*}
    \centering
    \includegraphics[width=1\linewidth]{figs/figure2.pdf}
    \caption{The study consists of two parts: Part 1: One-on-One Turn-Based Debate and Part 2: Three-Person Free Debate. In Part 1, two participants, Alice and Bob, with opposing stances, engaged in a turn-based debate. In Part 2, a third participant, Carol, who shares the same stance as one of the original participants, joined the ongoing debate. Image credit: ESPN FC.}
    \label{fig2}
    \Description{The study consists of two parts: Part 1: One-on-One Turn-Based Debate and Part 2: Three-Person Free Debate. In Part 1, two participants, Alice and Bob, with opposing stances, engaged in a turn-based debate. In Part 2, a third participant, Carol, who shares the same stance as one of the original participants, joined the ongoing debate. Image credit: ESPN FC.}
\end{figure*}

\subsubsection{Part 1: One-on-One Turn-Based Debate (around 45 minutes)}

In Part 1, to trigger a simple debate, we purposely matched two participants, a supporter of Lionel Messi and a supporter of Cristiano Ronaldo, to debate in the forum in a turn-based manner. Participants had unlimited use of ChatGPT and were allowed to prompt ChatGPT while waiting for the response from the other participant. To determine the number of posting turns in Part 1, we conducted three pilot study sessions and analyzed the data collected before the formal study. To ensure participants had enough time to write their debate posts and that conversations could be responded to promptly, we used a flexible approach: 2 to 3 turns per participant based on the time spent, resulting in 4 to 6 posts (2 turns/4 posts: 7 sessions, 3 turns/6 posts: 6 sessions), after which Part 2 began.


\subsubsection{Part 2: Three-Person Free Debate (around 45 minutes)}

In Part 2, to initiate a more free-form debate that mirrors the spontaneous and often unstructured nature of debate commonly seen on social media, we introduced a new participant who supports either Lionel Messi or Cristiano Ronaldo to the existing debate. Unlike Part 1, all three participants could post freely without the turn-based restriction (\autoref{fig2}). The study ended after all participants had posted at least three times in Part 2.

\subsection{Participants and Recruitment}

Our prospective participants were recruited via university email and social media platforms and were pre-screened for eligibility based on the soccer player they supported. Only those over 18 years old who have a clear stance on the Lionel Messi versus Cristiano Ronaldo rivalry and possess knowledge about soccer tactics were invited. We recruited 39 participants, but one participant (P39) withdrew at the beginning of the study session. The demographics are shown in \autoref{table1}. The participants were assigned to 13 designated sessions based on their stances. They provided their consent to participate in the study by signing a consent form and agreeing to have their data collected anonymously. The study passed the university's ethics review, and the data collected was analyzed while maintaining the anonymity of the subjects' identities.

\begin{table*}
  \caption{An overview of participant demographics in our study. The 39 participants were divided into 13 study sessions, with each session comprising 3 participants.}
  \label{table1}
  \begin{tabular}{cccccccc}
    \toprule
    \textbf{Session} & \textbf{ID} & \textbf{Age} & \textbf{Gender} & \textbf{Stance} & \textbf{Region} & \textbf{Experience in GenAI} & \textbf{Education}\\
    \midrule
    I  & \textbf{P1} & 21 & F & Ronaldo & Mainland China & Moderate & Undergraduate\\
       & \textbf{P2} & 25 & M & Messi & Hong Kong & No & Postgraduate\\
       & \textbf{P3} & 27 & F & Ronaldo & Mainland China & Knowledgeable & Postgraduate\\
    \hline
    II & \textbf{P4} & 20 & M & Messi & Hong Kong & Limited & Undergraduate\\
       & \textbf{P5} & 24 & F & Ronaldo & Mainland China & Moderate & Postgraduate\\
       & \textbf{P6} & 24 & F & Messi & Mainland China & Moderate & Postgraduate\\
    \hline
    III & \textbf{P7} & 23 & M & Ronaldo & South Africa & Moderate & Undergraduate\\
        & \textbf{P8} & 22 & M & Messi & Hong Kong & Limited & Undergraduate\\
        & \textbf{P9} & 24 & F & Ronaldo & Mainland China & Knowledgeable & Postgraduate\\
    \hline
    IV & \textbf{P10} & 19 & M & Messi & Hong Kong & Knowledgeable & Undergraduate\\
       & \textbf{P11} & 19 & M & Ronaldo & Kazakhstan & Knowledgeable & Undergraduate\\
       & \textbf{P12} & 21 & M & Messi & Hong Kong & Moderate & Undergraduate\\
    \hline
    V  & \textbf{P13} & 26 & M & Ronaldo & Mainland China & Knowledgeable & Postgraduate\\
       & \textbf{P14} & 22 & F & Messi & Mainland China & Moderate & Postgraduate\\
       & \textbf{P15} & 19 & M & Ronaldo & Mainland China & Limited & Undergraduate\\
    \hline
    VI & \textbf{P16} & 21 & M & Ronaldo & Hong Kong & Moderate & Undergraduate \\
       & \textbf{P17} & 24 & M & Messi & Mainland China & Moderate & Postgraduate\\
       & \textbf{P18} & 23 & M & Ronaldo & Hong Kong & Moderate & Postgraduate\\
    \hline
    VII & \textbf{P19} & 22 & M & Messi & Australia & Knowledgeable & Postgraduate\\
       & \textbf{P20} & 23 & M & Ronaldo & Mainland China & Knowledgeable & Postgraduate\\
       & \textbf{P21} & 22 & M & Messi & Mainland China & Moderate & Undergraduate\\
    \hline
    VIII & \textbf{P22} & 22 & M & Messi & Mainland China & Limited & Undergraduate\\
       & \textbf{P23} & 21 & M & Ronaldo & Mainland China & Limited & Undergraduate\\
       & \textbf{P24} & 23 & M & Messi & Mainland China & Moderate & Undergraduate\\
    \hline
    IX & \textbf{P25} & 23 & F & Ronaldo & Mainland China & Expert & Postgraduate\\
       & \textbf{P26} & 21 & F & Messi & Mainland China & Limited & Undergraduate\\
       & \textbf{P27} & 32 & F & Ronaldo & Mainland China & Knowledgeable & Postgraduate\\
    \hline
    X  & \textbf{P28} & 27 & M & Ronaldo & Mainland China & Limited & Postgraduate\\
       & \textbf{P29} & 24 & M & Messi & Hong Kong & Knowledgeable & Postgraduate\\
       & \textbf{P30} & 20 & F & Messi & Mainland China & Moderate & Undergraduate\\
    \hline
    XI & \textbf{P31} & 19 & M & Messi & Hong Kong & Knowledgeable & Undergraduate\\
       & \textbf{P32} & 26 & M & Ronaldo & Mainland China & Limited & Undergraduate\\
       & \textbf{P33} & 24 & M & Messi & South Korea & Limited & Undergraduate\\
    \hline
    XII & \textbf{P34} & 24 & M & Ronaldo & Mainland China & Knowledgeable & Postgraduate\\
       & \textbf{P35} & 24 & M & Messi & Mainland China & Moderate & Postgraduate\\
       & \textbf{P36} & 23 & M & Ronaldo & Mainland China & Knowledgeable & Postgraduate\\
    \hline
    XIII & \textbf{P37} & 22 & M & Messi & Mainland China & Knowledgeable & Postgraduate\\
       & \textbf{P38} & 24 & F & Ronaldo & Singapore & Moderate & Postgraduate\\
       & \textbf{P39} & 38 & M & Messi & Hong Kong & Limited & Postgraduate\\
    \bottomrule
  \end{tabular}
\end{table*}

\subsection{Interview Protocol}

In addition to the participants' debate on the online forum, we conducted a semi-structured interview with each participant. The semi-structured interview lasted around 20 minutes. Before the interview, all participants were informed that the conversations would be audio-recorded and transcribed verbatim. During the interview, the researchers encouraged participants to recall and articulate their experiences of using ChatGPT to formulate their arguments and post on the online forum, as well as identify their specific strategies. For those participants who were involved in only Part 2 of the study, additional questions were asked regarding the experience of synthesizing existing information. Interviews were conducted on the Microsoft Teams Meet platform. For two of the 39 participants, interviews were conducted in Mandarin and translated by three native-speaking researchers. The remaining interviews were conducted in English. The interview data were audio-recorded and subsequently transcribed by the research team.

\subsection{Data Analysis}

\subsubsection{Content Analysis}
To identify patterns (RQ2) in the arguments created by participants in the online forum, the research team conducted a content analysis of the forum posts. All the coding processes were conducted manually using spreadsheets. Two researchers independently analyzed all forum posts, with each post as a unit of analysis. Next, two researchers consulted previous literature on argumentation and reasoning to ensure that their understanding was accurate and appropriate. The two researchers then held meetings to discuss the patterns emerging from the forum in several rounds. After resolving disagreements and reaching consensus between the two researchers, a codebook was developed, including descriptions of each pattern and examples of sentences from the forum (\autoref{B}). To compile statistics on pattern occurrence, the researchers treated each post as a single unit. The researchers counted how many posts contained these patterns, without considering the number of patterns or the frequency of a single pattern within a post (\autoref{fig7}). For formal analysis, two researchers independently coded the forum posts of the first study session, resulting in a Cohen's kappa coefficient of 0.83. Because of the high agreement between the two researchers in their coding, each researcher coded the forum posts of the remaining study sessions separately.

\subsubsection{Thematic Analysis}
An open coding method was adopted to analyze three types of data collected in the study, i.e., the forum posts, ChatGPT records, and interview transcripts~\cite{corbin_basics_2008}. Three researchers independently coded the data and grouped the codes into emergent themes. After completing individual coding, three researchers held a discussion session, collectively analyzing their codes until a consensus was reached on them. Notably, the researchers would cross-reference among the three types of data when necessary (e.g., referring to ChatGPT records while coding interview transcripts) to better understand the participants' intentions. To minimize the potential interference of participants from different sessions with each other, we cleared previous ChatGPT records to prevent their effects on new participants. We then exported each participant's ChatGPT records into separate Google Docs ~\footnote{Google Docs:~\url{https://docs.google.com/}} to preserve the data, making it available only to the research team. Each of the three researchers then downloaded a local copy of the documents for individual analysis. For the analysis of interview transcripts, as two participants were interviewed in Mandarin, the three researchers, who are also native Mandarin speakers, first gained consensus on their interview transcript codes and then translated them into English for further analysis. The research team used affinity diagramming~\cite{beyer_contextual_1997} as a modified version of grounded theory analysis~\cite{corbin_basics_2008}. All codes were transcribed on sticky notes with random arrangements. After several iterations, the research team then arranged the sticky notes into a hierarchy of themes and reached a consensus on the content themes.