\subsection{Making Arguments in Online Forums}
Online communities are inherently heterogeneous and multi-faceted, with goals that include entertaining, information exchange, social support, and prestige~\cite{kairam_how_2024, moore_redditors_2017}. Given the diverse range of discussion topics, there are online communities focused on politics~\cite{papakyriakopoulos_upvotes_2023, hua_characterizing_2020, lyu_exploring_2023}, fan fiction~\cite{campbell_thousands_2016}, sports~\cite{zhang_this_2018,kim_social_2015, zhang_intergroup_2019, wang_making_2023}, career mentoring~\cite{tomprou_career_2019}, Korean popular music (K-pop) groups~\cite{park_armed_2021}, virtual communities~\cite{fu_i_2023}, live streaming~\cite{lu_you_2018,lu_i_2019,lu_more_2021}, and so on. As online communities can vary greatly in purpose, scope, and topic~\cite{hwang_why_2021}, our research focuses on argument-based and forum-style communities. These types of forums are identified as essential places for people to voice their opinions and engage in debates with each other~\cite{qiu_modeling_2015}.

Even though the motivations for establishing communities are always to benefit their members and form a bond among them~\cite{kairam_how_2024, matthews_goals_2014}, dissonance may arise in forum discussions as part of the community activities. Specifically, political forums may be inherently more prone to incivility than forums about other topics~\cite{efstratiou_non-polar_2022}, as research has suggested that "interactions between ideologically opposed users were significantly more negative than like-minded ones"~\cite{marchal_be_2022}. Nevertheless, another study challenged this popular belief by suggesting that intra-group members holding the same sides of the political spectrum can have an even higher amount of polarizing and aggressive comments compared to inter-group members~\cite{efstratiou_non-polar_2022}.

Similar debates can also occur in sports communities. Sports fans who support different players may treat each other as enemies, and their attitudes can vary according to team performances~\cite{zhang_this_2018}. In these circumstances, expressing emotions can easily turn into aggressive posts, trolling behaviors, and even a vicious circle by down-voting and spreading negative feelings~\cite{wang_making_2023}. Another work revealed that members with higher inter-group contact levels tended to use more negative words, swear words, and produce more hate speech comments in their affiliated group discussions compared to those who only had single-group identity~\cite{zhang_intergroup_2019}.

In light of this, online debates constitute a pivotal component of online interactions among forum members. Unlike traditional face-to-face debates, the persuasiveness and effectiveness of online debates predominantly rely on a form of designing for persuasive influence~\cite{lc_designing_2021, lc_designing_2022}, thereby highlighting the importance of persuasive writing.


\subsection{Persuasive Writing}
It is common for people holding different views to try to persuade others when discussing online~\cite{xia_persua_2022,tan_winning_2016}. Historically, rhetoric and argumentation can be traced back to Aristotle's modes of persuasion~\cite{wang_argulens_2020}. Contemporary rhetoric studies also focus on argumentation, the audience, and the conditions for rational debates~\cite{herrick_history_2020}. Toulmin's model~\cite{toulmin_uses_2003}, one of the most influential argumentation models~\cite{wang_argulens_2020}, proposed six fundamental argumentative components including claim, ground, warrant, qualifier, rebuttal, and backing~\cite{wang_argulens_2020,bentahar_taxonomy_2010,toulmin_uses_2003}. Previous research has widely adopted Toulmin's model as a foundation to improve the persuasiveness of usability feedback~\cite{norgaard_evaluating_2008}, unveil community opinions on usability~\cite{wang_argulens_2020}, and support system building to enhance argumentation~\cite{zhang_using_2016,wambsganss_modeling_2022}. Compared to other models, Toulmin's model and extensions have distinct advantages in specifying various components of the argument structure, their interconnections, and the inference rules for constructing textual arguments~\cite{bentahar_taxonomy_2010}.

More persuasion models have been developed to explain how people respond to persuasive attempts in marketing and advertising. For example, the Heuristic-Systematic Model (HSM) of persuasion describes how people process persuasive messages through heuristic and systematic processing~\cite{reimer_use_2004}. The Persuasive Knowledge Model (PKM) addresses how people recognize, evaluate, and respond to persuasive content~\cite{friestad_persuasion_1994}.

Building on Toulmin's model~\cite{toulmin_uses_2003}, researchers have established a framework that includes claims, evidence (the information or data that support the claim), and reasoning (a justification that shows why the data count as evidence to support the claim)~\cite{berland_making_2009}. Claims can be further classified into different types, including definitive and descriptive ones~\cite{van_der_wall_statement_2012}. In addition to claims, evidence also comes in various categories such as numerical data~\cite{berland_making_2009}, observations~\cite{berland_making_2009}, facts~\cite{berland_making_2009}, examples~\cite{southerland_examples_2017}, and counterexamples~\cite{johnson-laird_how_2008}. In terms of reasoning, besides typical techniques such as rebuttal~\cite{toulmin_uses_2003} and analogy~\cite{winebrenner_argumentation_1991}, some fallacies can lead to misunderstanding and even deceive readers. Fallacies in reasoning can take many forms, such as hasty generalization~\cite{van_eemeren_argumentation_2016,kord_grey_2021}, ad hominem attacks~\cite{van_eemeren_argumentation_2016, kord_grey_2021}, straw man arguments~\cite{van_eemeren_argumentation_2016}, misplacing the burden of proof~\cite{kord_grey_2021}, and irrelevant conclusion~\cite{kord_grey_2021}.

\subsection{Co-Writing with AI Assistants}

Unlike writing alone, collaborative writing, with either human or AI assistance, is common and has been applied in various aspects of our daily life~\cite{storch_collaborative_2005, li_computer-mediated_2018, barile_computer-mediated_2002}. With the support of AI writing assistants such as Grammarly~\footnote{Grammarly:~\url{https://www.grammarly.com/}}, the writing quality can be significantly improved~\cite{fitria_grammarly_2021}. In 2022, the release of ChatGPT by OpenAI represented a pivotal advancement in the field of human-AI collaborative writing, drawing substantial attention from various research communities, such as Human-Computer Interaction (HCI), Natural Language Processing (NLP), and Computational Social Science (CSS)  ~\cite{lee_design_2024}. Beyond general writing purposes, human-AI co-writing is widely adopted in specific use cases such as fiction writing~\cite{zhong_fiction-writing_2023,yang_ai_2022}, poetry writing~\cite{lc_imitations_2022}, theater script writing~\cite{mirowski_co-writing_2023}, science and scientific writing~\cite{gero_sparks_2022, kim_metaphorian_2023, shen_convxai_2023}, etc. Prior research has also highlighted the promising future of human-AI co-writing across various application scenarios~\cite{luther_teaming_2024}.

In the HCI community, people have designed various human-AI co-writing tools to explore new writing paradigms. For example, Dramatron, derived from a large language model, enables participants to collaborate with AI systems to create theater scripts and screenplays, proving especially useful for hierarchical text generation~\cite{mirowski_co-writing_2023}. Similarly, CoPoet is tailored to assist human writers in crafting poems, enhancing the final outcomes~\cite{chakrabarty_help_2022}. Wordcraft, an interface designed for story writing, allows AI to serve various roles such as idea generator, scene interpolator, and copy editor~\cite{yuan_wordcraft_2022}. Audiences prefer specific modes with fine-grained control over generated text, often expressing satisfaction~\cite{zhong_fiction-writing_2023}. Wan et al.~\cite{wan_it_2024} investigated human-AI co-creativity in the prewriting scenario to shift the focus from convergent to divergent thinking.

Previous research shows that the AI mediator can enhance critical thinking, which helps in bursting filter bubbles and depolarizing online communities~\cite{govers_ai-driven_2024, tanprasert_debate_2024, lin_case_2024}. However, online debates are inherently adversarial, often thriving on polarization to stimulate engagement and argumentation. This contrast motivates the exploration of how the use of generative AI can be adapted to support such a polarized and competitive context effectively.