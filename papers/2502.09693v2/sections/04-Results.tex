\subsection{Situation-Based Use of ChatGPT for Argument Making (RQ1)}

\subsubsection{Seeking offensive content from ChatGPT}
Participants noted that ChatGPT tends to provide neutral responses, which may not be effective for persuading others holding opposing views in online debates. To address this issue, participants provided the specific context and topic of the debate to ChatGPT in their prompts. For instance, prompting ChatGPT to act as a "\textit{professional debater}" (P11) or a "\textit{fan of Messi}" (P29), instructing it to "\textit{fight against the opinion below: ... (a quote from content posted by P16)}" (P17), or engaging in the debate with a preference for one of the two players (P24).


However, despite the context provided, participants still felt that ChatGPT's responses were not aggressive enough for a debate setting, as it failed to understand the nature of the debate or overlooked information in their prompts. As a consequence, they tried to trigger ChatGPT in both direct and indirect ways to be confrontational, such as directly stating that "\textit{you are a Messi's hater}" (P3 Prompt), or indirectly through triggering insulting jokes "\textit{Brother, right now I need your assistance to make a poem about Carol [P12] and Alice [P10]. Both have zero ball knowledge and are trying to argue [with me]. You can use insulting jokes if your guard rails allow you to do so}" (P11 Prompt) (\autoref{fig3}). In addition, participants chose to add aggressive content themselves when they felt unsatisfied with ChatGPT's response after several failed attempts, e.g., adding "\textit{dirty words}" (P3) (\autoref{fig4}).


\begin{figure*}
    \centering
    \includegraphics[width=1\linewidth]{figs/figure3.pdf}
    \caption{One participant (P11) attempted to prompt ChatGPT to generate a poem containing insulting jokes. ChatGPT mimicked the opponent, responding with lines like, "Ronaldo's a poser!" and "Messi's the greatest ...".}
    \label{fig3}
    \Description{One participant (P11) attempted to prompt ChatGPT to generate a poem containing insulting jokes. ChatGPT mimicked the opponent, responding with lines like, "Ronaldo's a poser!" and "Messi's the greatest ...".}
\end{figure*}

\begin{figure*}
    \centering
    \includegraphics[width=1\linewidth]{figs/figure4.pdf}
    \caption{Participants tried to introduce aggressive content by themselves when they were not satisfied with the response of ChatGPT after several times re-prompting. This figure shows an example of P3.}
    \label{fig4}
    \Description{Participants tried to introduce aggressive content by themselves when they were not satisfied with the response of ChatGPT after several times re-prompting. This figure shows an example of P3.}
\end{figure*}

\subsubsection{Co-writing with ChatGPT elicits similar posts}

Collaborating with ChatGPT, participants integrated the information given by ChatGPT into their posts, leading to posts with similar content. Due to the inherent properties of ChatGPT, it often delivers responses in similar styles which are shaped by the information and context provided by the participants. These responses further act as a catalyst prompting participants to make similar posts. For instance, "\textit{the Euro 2016 final}" appeared 6 times in the forum within the same session (Session VI) due to the mediation of ChatGPT's response (\autoref{fig5}).


\begin{figure*}
    \centering
    \includegraphics[width=1\linewidth]{figs/figure5.pdf}
    \caption{Participants repeated similar content in their posts several times, reflecting their reliance on the response of ChatGPT. This figure shows an example of P16 and P17.}
    \label{fig5}
    \Description{Participants repeated similar content in their posts several times, reflecting their reliance on the response of ChatGPT. This figure shows an example of P16 and P17.}
\end{figure*}

In addition, we observed that the behavior of participants prompting ChatGPT to generate an entire post also leads to similar content in their forum posts (\autoref{fig5}). This phenomenon is evident in Session VI, where the sentences in the posts share  similar content. In this study session, P16 first prompted ChatGPT to help generate a paragraph aligned with his stance, then revised it slightly and posted it on the forum. P17 then quoted the posts from P16 as part of the prompt to ChatGPT, "\textit{give me an opinion that Messi is the best soccer player in comparison with CR7 to counter the opinion below: ...}", which led to ChatGPT's response having  similar content to P16's posts, which P17 then used to form his posts (\autoref{fig6}).

\begin{figure*}
    \centering
    \includegraphics[width=1\linewidth]{figs/figure6.pdf}
    \caption{Participants shared similar content in their posts, regardless of whether they were opponents (Example 1) or teammates (Example
2). By quoting other members' posts to prompt ChatGPT and using the information generated, their posts contained similar content.}
    \label{fig6}
    \Description{Participants shared similar content in their posts, regardless of whether they were opponents (Example 1) or teammates (Example
2). By quoting other members' posts to prompt ChatGPT and using the information generated, their posts contained similar content.}
\end{figure*}

\subsubsection{Balancing ChatGPT assistance and human expertise}

Participants utilized ChatGPT as a search tool or an assistant to support their arguments. They emphasized ChatGPT's inability to work independently (e.g., P8) and believed they cannot be treated as a real human teammate due to its lack of opinion (e.g., P17).

On the one hand, participants valued ChatGPT for its efficiency in extracting critical information such as statistics of goals and commercial values of a player (P20) to support their argument as needed "\textit{I think finding data on the internet is too exhausting, while ChatGPT can give me a summary of the sea of information online. It provides me with what I want briefly and directly}" (P20 Interview). P3 echoed this sentiment, explaining, "\textit{I have a very blurred memory about some points, and I cannot come up with detailed information. Then I will tell generative AI that I need this information, and it will tell me ... I think the original idea came from me, and the generative AI helps me to complete it}". With the assistance of ChatGPT, participants also tended to think more rationally, as P9 stated, "\textit{The key point is that it [ChatGPT] can actually work better than humans in this way because we are often controlled by emotions, whereas ChatGPT is not}" (P9 Interview)

On the other hand, we identified five main reasons that participants tended to avoid using ChatGPT in the following conditions: 

\begin{enumerate}
\item{\textbf{Familiarization with the topic}, e.g., "\textit{I am familiar with Messi and Ronaldo, so I will insist on my opinion instead of the one provided by GPT, and I know mine is better than its}" (P19 Interview).}

\item{\textbf{Absence of latest information}, e.g., as described by P10 "\textit{If I am looking into some incidents that happened a long time ago, like the performance of Messi in 2014, I will use the generative AI. But for his performance this year or last year, I would use my own knowledge}" (P10 Interview).}

\item{\textbf{Failure to identify credible sources}, especially in comparison with the search engines, e.g., "\textit{I think the internet is more reliable than ChatGPT because when you google some information, it can show a lot of different sources like Baidu, wikis, news, articles, blogs, and so on. You can compare the information from different sources and choose the most accurate one}" (P34 Interview).}

\item{\textbf{Stilted style of responses}, e.g., adjusting the style of communication to fit an online forum by removing the bullet points in ChatGPT's response, as noted by P24: "\textit{I would not use bullet points provided by ChatGPT in my post, because they are too formal to be used in an online forum discussion}", and adding slang or emojis (e.g., P9, P10, P13 and P14) (\autoref{fig10}).}

\item{\textbf{Disagreements with ChatGPT}, e.g., the subjective judgment of whether Messi has disrespected audiences in Hong Kong (P28).}
\end{enumerate}





\subsection{Patterns Emerging in Forum Posts (RQ2)}

In online forum debates, participants made claims to support their own stances, provided evidence informed by ChatGPT, and bridged these two components through reasoning (\autoref{B}). It is worth noting that in the process of reasoning, participants may commit logical fallacies.

\subsubsection{Value-based claims, examples, and hasty generalizations are the most frequently appearing patterns in the forum debates}

To express opinions, participants mainly adopted five kinds of claims: definitive claims, descriptive claims, value-based claims, concessive claims, and advocacy claims. Participants used value-based claims 125 times, making it the most frequently occurring pattern (\autoref{fig7}). Moreover, participants adopted concessive claims from ChatGPT, e.g., "\textit{while Messi is undeniably great, Cristiano Ronaldo stands out for his versatility and achievements}" (P11 Post), and "\textit{While Cristiano Ronaldo offers impressive goal-scoring and physical presence, Messi's all-around contributions and positive influence on team dynamics give him a superior edge in both performance and team spirit}" (P4 Post). Although participants entered the study with a clear stance, ChatGPT may offer them more dialectical perspectives, allowing them to consider the advantages of both sides.


Participants reported that statistics help make arguments: For example, "\textit{I believe the evidence and the statistics are most persuasive}" (P19 Interview). This was echoed by P30 "\textit{Athletes' performance must be validated by objective data such as trophies because [subjective] words can be fabricated, and fans can embellish their character, making words less convincing}". Specifically, statistics such as "\textit{How many goals has Ronaldo scored in his entire career? How many championships has he won? How many awards has he received? These statistics are convincing}" (P18 Interview), which in turn could support their stances "\textit{Messi has 45 champions, but Ronaldo has less than 35 ... And Messi has 8 Ballon d'Or and Ronaldo only has 5}" (P19 Post).

Interestingly, even though participants reported that "statistics" are useful for persuading others, it was not the most used type of evidence for persuasive writing. Instead, they utilized examples and personal observations the most to persuade others (\autoref{fig7}). Moreover, most of the statistics in the posts created by participants came from ChatGPT, whereas facts, concrete examples, and personal observations tended to come from the participants' own knowledge (\autoref{fig8}). For instance, P29 posted "\textit{Ronaldo is incredible, no doubt, but Messi's magic is unmatched. He doesn't just score—he creates, dominates, and makes the game beautiful. Messi has 4 Champions League titles, often being the key player. As for Messi in Hong Kong, I was there during the training session. He did show up in the first section but then felt unwell. Later, he did not train with the team in the last couple of hours. The injury is real, and this is normal in the world of football. It has nothing to do with his personality}". In this post, the first half was generated by ChatGPT, while the latter half came from the participant's own observations.

\begin{figure*}
    \centering
    \includegraphics[width=1\linewidth]{figs/figure7.pdf}
    \caption{The statistics show the frequency of various argumentation patterns across three categories (i.e., claims, evidence, and reasoning) measured by the number of posts in which these patterns are present.}
    \label{fig7}
    \Description{The statistics show the frequency of various argumentation patterns across three categories (i.e., claims, evidence, and reasoning) measured by the number of posts in which these patterns are present.}
\end{figure*}

\begin{figure*}
    \centering
    \includegraphics[width=1\linewidth]{figs/figure8.pdf}
    \caption{Participants requested statistics from ChatGPT and integrated the generated content into their posts. This figure shows an example of P10.}
    \label{fig8}
    \Description{Participants requested statistics from ChatGPT and integrated the generated content into their posts. This figure shows an example of P10.}
\end{figure*}

As the whole writing process was assisted by ChatGPT, the frequency of committing ad hominem fallacies was relatively low, appearing only 23 times. On the contrary, the use of hasty generalizations was the most frequent logical fallacy committed in reasoning. This may be due to participants acquiring partial information from ChatGPT. Participants selectively asked ChatGPT for statistics of a specific match such as "\textit{Messi in his first season in Ligue 1}" (P32 Prompt) trying to use it to prove that "\textit{Messi cannot play very well if he isn't in Barcelona}" (P32 Post). Meanwhile, ChatGPT provided a large amount of evidence, but the participants only selectively picked it to support their arguments. Under these circumstances, hasty generalizations may emerge when they attempt to use a single example to support a grand argument. For example, as the participant asked both about "\textit{Laliga}" and "\textit{UEFA}" (P32 Prompt), the posts only contained statistics of UEFA favoring Ronaldo as "\textit{Messi has scored 129 goals in 163 Champions League appearances, while Ronaldo is the all-time top scorer in the UEFA Champions League with 140 goals in 183 appearances}", accompanied by a conclusion given by the participant "\textit{I think it makes him better than Messi}" (P32 Post).

In summary, value-based claims, examples, and hasty generalizations are the most prevalent patterns for claims, evidence, and reasoning, respectively. While value-based claims may have emerged because the debate topic was inherently value-based, the hasty generalizations can be caused by selective information acquisition from ChatGPT.

\subsubsection{Combinations of various patterns enhance persuasiveness}
To form a comprehensive argument, participants combine various patterns in one post. For example, P3 mentioned that "\textit{I randomly came up with some ideas and used generative AI to solidify them}". Under this circumstance, participants came up with the opinion based on their stance, and ChatGPT provided support for their opinions, resulting in a combination of patterns as "Claim + Evidence".


Despite the popularity of "Claim + Evidence", there are other combinations that also merit our attention (\autoref{fig9}). We observed rebuttals are often paired with counterexamples, which are identified as "Reasoning (rebuttal) + Evidence (counterexamples)". This type of combination was commonly used against value-based claims, which were subjective and lacked definitive proof, e.g., "\textit{If CR7 can lead Real Madrid to victory as you said, why can't he score goals and help Portugal win the World Cup and this year's European Championship?}" (P2 Post). Another notable combination is the "Claim + Claim" combination, which often appears as a concessive claim that partially acknowledges others' arguments before presenting the personal argument. This approach is frequently used when it is challenging to deny a claim outrightly, e.g., "\textit{While I acknowledge that Messi's World Cup win elevates his team honors above Ronaldo's, I must emphasize that Ronaldo has often been more crucial to his team}" (P13 Post). In addition, participants employed the combination of "Evidence + Evidence", believing that "the evidence speaks for itself". Thus, they listed various types of evidence (statistics, facts, examples, personal observations, etc.) to form their arguments without any claim or reasoning as a conclusion.

\begin{figure*}
    \centering
    \includegraphics[width=1\linewidth]{figs/figure9.pdf}
    \caption{Other combinations of patterns: (A) "Reasoning + Evidence": The use of rebuttals and counterexamples. (B) "Claim + Claim":
The use of concessive claims with other types of claims. (C) "Evidence + Evidence": The combinations of different types of evidence.}
    \label{fig9}
    \Description{Other combinations of patterns: (A) "Reasoning + Evidence": The use of rebuttals and counterexamples. (B) "Claim + Claim":
The use of concessive claims with other types of claims. (C) "Evidence + Evidence": The combinations of different types of evidence.}
\end{figure*}

\subsection{Changes After a New Participant Joined the Debate (RQ3)}

Based on our observation, during the transition from Part 1 to Part 2, participants maintained the original workflow: first prompting ChatGPT, then selecting the information provided, and finally organizing their thoughts along with the information from ChatGPT to make posts. Nevertheless, we identified three changes after the new participant joined the debate.

\subsubsection{Collaborating with another forum member and ChatGPT}

After a new participant joined the debate, on the one hand, participants with the same stance collaborated by teaming up and prompted ChatGPT to build on their teammates' arguments. For example, P19 mentioned, "\textit{I can simply support P21 and add more information. I am not afraid, even if there are 10 or 100 people supporting Ronaldo. I will be able to fight them all back}" (P19 Interview). On the other hand, participants without a teammate collaborated with ChatGPT by teaming up, alleviating the feeling of isolation. For example, P35 noted, "\textit{After P36 joined, it felt like the GenAI and I formed a two-person team to fight against opponents, which made me feel less isolated and more confident in the debate}" (P35 Interview).

\subsubsection{Reducing the use of ChatGPT for better engagement in debates}
Despite teaming up with ChatGPT, participants reported that using ChatGPT to make posts was inefficient, as they had to think about how to prompt it and interpret its responses. P3 pointed out, "\textit{I take too much time on prompting ChatGPT, and it is really time-consuming, which makes me angry}". As a result, we observed that human-human interaction was sacrificed for human-AI interaction, echoing P15's sentiment: "\textit{people's discussion is reduced [on the forum]}". To address this issue, participants decided to reduce their interaction with ChatGPT after a new participant joined and instead focused more on the online forum to enhance community engagement, especially for the solo participant. P11 explained, "\textit{when the third guy [P3] joined, I just gave up [using ChatGPT to answer my questions] and started using my own answers, only using ChatGPT to reformat and check for grammar and orthographic mistakes}" (P11 Interview).

\subsubsection{Synthesizing the previous information with the assistance of ChatGPT}
The new member utilized ChatGPT to synthesize the debate context and main insights. As P24 mentioned, "\textit{I joined the discussion midway, so I needed the ChatGPT to summarize and analyze the exact situation and main points of the discussion. I think I used it for that purpose}" (P24 Interview). In addition, participants chose to synthesize the information themselves instead of using ChatGPT, e.g., "\textit{I do not use it [ChatGPT] because I want the direction of the whole discussion to be determined by myself}" (P9 Interview). In addition, a new participant can inspire other forum members, "\textit{the third participant [P27] introduced fresh perspectives and ideas, which inspired me to contemplate new expressions beyond their statements}" (P26 Interview).


\begin{figure*}
    \centering
    \includegraphics[width=1\linewidth]{figs/figure10.pdf}
    \caption{Participants blended content generated by ChatGPT with internet slang and emojis to adjust the language style, making it feel more human-like.}
    \label{fig10}
    \Description{Participants blended content generated by ChatGPT with internet slang and emojis to adjust the language style, making it feel more human-like.}
\end{figure*}