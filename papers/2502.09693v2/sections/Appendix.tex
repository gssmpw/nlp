\newpage
\onecolumn
\section{Instructions for the Study}
\label{A}
\subsection{General Guidelines for Using ChatGPT}
\begin{enumerate}
    \item{Exclusive Use: Please use only GPT-4o and the designated forum throughout the study. Some usage examples are as follows: 
    \begin{itemize}
        \item{Use ChatGPT to help you formulate the argument that you want to express}
        \item{Directly use ChatGPT's arguments}
        \item{Summarize ChatGPT's arguments}
        \item{Take inspiration from ChatGPT and write your own post}
    \end{itemize}}
    \item{Topic Relevance: Please ensure that your posts are related to the debate topic "Messi vs. Ronaldo: Who is better?".}
    \item{Post Format: Please limit your posts to plain text and emojis. Please do not include other formats (e.g. photos, links).}
\end{enumerate}

\subsection{Part 1: One-on-One Turn-Based Debate}
\begin{enumerate}
    \item{You should post at least two messages (Each post is strongly recommended to be limited to 100 words).}
    \item{You should wait for the response before you post your next message.}
\end{enumerate}

\subsection{Part 2: Three-Person Free Debate}
\begin{enumerate}
    \item{The debate will end until all have posted at least 3 messages (Each post is strongly recommended to be limited to 100 words).}
    \item{No need to be turn-based. If you want to post something, you can post it immediately.}
\end{enumerate}

\clearpage
\section{Codebook}
\label{B}

\begin{table*}[h!]
  \caption{Emerging "claim" patterns from the forum.}
  \begin{tabular}{>{\raggedright\arraybackslash}p{2.5cm}>{\raggedright\arraybackslash}p{4cm}>{\raggedright\arraybackslash}p{9.5cm}}
    \toprule
    \textbf{Patterns} & \textbf{Description} & \textbf{Examples}\\
    \midrule
    Use of definition-based claims & Making arguments to clarify the essence or core identities & 
    Messi's resilience is not just about physical endurance but also mental strength. \\
    \hline
    Use of descriptive claims & Making arguments to portray the peripheral characteristics & (Cristiano Ronaldo contributed a brilliant reverse stick goal in the semi-final against Juventus.) \textbf{He also contributed superb goals in the knockout stages.}\\
    \hline
    Use of value-based claims & Making judgment to prove that some action, belief, or condition is right or wrong, good or bad, beautiful or ugly, worthwhile or undesirable ~\cite{rottenberg_structure_2014} & Lionel Messi is the greatest footballer in history due to his unparalleled skill, creativity, and consistent brilliance. \\
    \hline
    Use of concessive claims & Continuing to make one's own arguments while agreeing with part of the other person's point ~\cite{tanskanen_concessive_2008} & I agree that any man who commits sexual assault is trash. But let's talk about Messi's personal life. \\
    \hline
    Use of advocacy claims & Arguing that certain conditions should exist ~\cite{rottenberg_structure_2014} & First of all, comparing team honors \textbf{should} not be done in terms of numbers alone, but rather in terms of the gold content of the championships as well as the degree of contribution to the tournament as a whole. \\
    \bottomrule
  \end{tabular}
\end{table*}

\begin{table*}[h!]
  \caption{Emerging "evidence" patterns from the forum.}
  \begin{tabular}{>{\raggedright\arraybackslash}p{2.5cm}>{\raggedright\arraybackslash}p{4cm}>{\raggedright\arraybackslash}p{9.5cm}}
    \toprule
    \textbf{Patterns} & \textbf{Description} & \textbf{Examples}\\
    \midrule
    Use of statistics (numerical data) in the evidence & Using numerical summaries ~\cite{rieke_argumentation_2012} & Moreover, Messi has provides more assists \textbf{(318 vs. 229)} than Ronaldo. And his goal-to-game ratio is superior, averaging \textbf{0.87} goals per game compared to Ronaldo's \textbf{0.77}. \\
    \hline
    Use of background information in the evidence & Providing contexts that are essential for understanding the arguments ~\cite{lunsford_everythings_2018} & \textbf{Even when injured in the 2016 Euro final}, Ronaldo led and inspired his teammates from the sidelines, playing a key role in Portugal's triumph. \\
    \hline
    Use of personal observations in the evidence & Justifying claims based on what has been directly seen or experienced ~\cite{kuhn_students_2005} & I would like to bring up the best argument for Messi being the GOAT, the recent world cup win. \textbf{I will admit I thank my lucky stars that I was able to witness a football match like that.} \\
    \hline
    Use of facts in the evidence & Providing evidence that the audience will accept as as being objectively verifiable ~\cite{rottenberg_structure_2014} & Ronaldo has faced serious allegations of sexual assault. \\
    \hline
    Use of examples in the evidence & Using instances to provide the empirical grounding for the claims~\cite{rieke_argumentation_2012} & The most impressive examples are his hat-trick against Spain in the 2018 World Cup and his hat-tricks against Wolfsburg and Atletico Madrid in the Champions League. \\
    \hline
    Use of counterexamples in the evidence & Providing a possibility that is consistent with the premises but inconsistent with the conclusion ~\cite{johnson-laird_how_2008} & \textbf{Alice}: ... In my opinion, the GOAT should excel under different coaches and with different teammates. Messi found a system that worked for him at Barcelona and stayed, but it's hard to argue he'd be just as successful at other clubs. The ability to thrive in various environments is crucial, and that's where Ronaldo has proven himself superior. \textbf{Bob}: Playing for more clubs and countries should not be considered as better. \textbf{On the other hand Ronaldo did not show dominance after his first Man Utd and Real years.} \\
    \bottomrule
  \end{tabular}
\end{table*}

\begin{table*}[h!]
  \caption{Emerging "reasoning" patterns from the forum.}
  \begin{tabular}{>{\raggedright\arraybackslash}p{2.5cm}>{\raggedright\arraybackslash}p{4cm}>{\raggedright\arraybackslash}p{9.5cm}}
    \toprule
    \textbf{Patterns} & \textbf{Description} & \textbf{Examples}\\
    \midrule
    Use of rebuttal in reasoning & Countering the validity of previous arguments by setting aside the general authority of warrant ~\cite{toulmin_uses_2003} & "Messi is not the most important player in the final" means that you did not watch the game. Football is not only about scoring. \\
    \hline
    Use of analogy in reasoning & Making a comparison between two similar cases and inferring that what is true in one case is true in the other ~\cite{freeley_argumentation_2008} & Football is not only about scoring. \textbf{If so, Anthony is probably better than Modric.}\\
    \hline
    Use of irrelevant conclusion as fallacies in reasoning & Ending with a conclusion that is not related in any necessary way to the premises ~\cite{kord_grey_2021} & Market value means nothing lol, Messi is 2 years younger than Ronaldo and that's why.\\
    \hline
    Use of hasty generalization as fallacies in reasoning & Trying to support a general claim by offering a story, which is just a single incident ~\cite{kord_grey_2021, van_eemeren_argumentation_2016} & Despite Messi and Ronaldo they have both won the Ballon d'Or and their individual abilities are outstanding, Messi is the one who led the Argentine national team to win the World Cup and Copa America, which has fully demonstrated his leadership ability. Since 2022 was a tough time for Argentina, Messi's achievement of leading Argentina to the World Cup somehow gave the Argentine people a great encouragement. That's why I consider Messi is more outstanding than Ronaldo, cause he knows how to be a better leader and how to cheered his team up. \\
    \hline
    Use of ad hominem attack as fallacies in reasoning & Rejecting or dismissing another person's statement by attacking the person rather than the statement itself ~\cite{kord_grey_2021, van_eemeren_argumentation_2016}& Chris [Carol] you have 0 ball knowledge. \\
    \hline
    Use of misplacing the burden of proof as fallacies in reasoning & Arguing that something is true simply because no one has proved it false, or that something is false simply because no one has proved it true ~\cite{kord_grey_2021} & I recognize that Ronaldo has not won a World Cup. Soccer is a team sport and Portugal is a little less strong overall. How many of Argentina's goals to win the World Cup came from Messi? How many of the goals were sporting goals? And how many goals came from penalties? \\
    \hline
    Use of straw man argument as fallacies in reasoning & Distorting or misrepresenting the opponent's argument, thus making it easier to knock it down or refute it ~\cite{van_eemeren_argumentation_2016} & Bob: In terms of team honor, whatever in National team or club team, Messi had more champion cup than CR7, with a total number of 45, and also has a biggest honor-- the FIFA world cup. In terms of personal honor, Messi has eight Ballon d' Or awards and more European Golden Shoe and some other personal honor. All in all, whether it is individual honor or team honor, Messi is better than CR7, CR7 outstanding place only more Champions League. \textbf{Alice: First of all, comparing team honors should not be done in terms of numbers alone, but rather in terms of the gold content of the championships as well as the degree of contribution to the tournament as a whole.} \\
    \bottomrule
  \end{tabular}
\end{table*}

\clearpage
\section{Number of Posts}
\label{C}
\begin{table*}[h!]
  \caption{An overview of post counts in the study.}
  \begin{tabular}{c|c|c|cc|c|ccc}
    \toprule
    \textbf{Session} & \textbf{Total} & \textbf{Part 1} &\textbf{Alice} &\textbf{Bob} &\textbf{Part 2} &\textbf{Alice} &\textbf{Bob} &\textbf{Carol}\\
    \midrule
    \textbf{I} & \textbf{18} & \textbf{6} & 3 & 3 & \textbf{12} & 3 & 5 & 4\\
    \textbf{II} & \textbf{15} & \textbf{6} & 3 & 3 & \textbf{9} & 3 & 3 & 3\\
    \textbf{III} & \textbf{20} & \textbf{4} & 2 & 2 & \textbf{16} & 7 & 6 & 3\\
    \textbf{IV} & \textbf{31} & \textbf{6} & 3 & 3 & \textbf{25} & 11 & 9 & 5\\
    \textbf{V} & \textbf{18} & \textbf{6} & 3 & 3 & \textbf{12} & 6 & 3 & 3\\
    \textbf{VI} & \textbf{18} & \textbf{6} & 3 & 3 & \textbf{12} & 6 & 3 & 3\\
    \textbf{VII} & \textbf{17} & \textbf{6} & 3 & 3 & \textbf{11} & 5 & 3 & 3\\
    \textbf{VIII} & \textbf{14} & \textbf{4} & 2 & 2 & \textbf{10} & 3 & 4 & 3\\
    \textbf{IX} & \textbf{20} & \textbf{4} & 2 & 2 & \textbf{16} & 10 & 3 & 3\\
    \textbf{X} & \textbf{13} & \textbf{4} & 2 & 2 & \textbf{9} & 3 & 3 & 3\\
    \textbf{XI} & \textbf{15} & \textbf{4} & 2 & 2 & \textbf{11} & 5 & 3 & 3\\
    \textbf{XII} & \textbf{13} & \textbf{4} & 2 & 2 & \textbf{9} & 3 & 3 & 3\\
    \textbf{XIII} & \textbf{44} & \textbf{6} & 3 & 3 & \textbf{38} & 35 & 3 & 0\\
    \bottomrule
  \end{tabular}
\end{table*}