Table 1: "An overview of participant demographics in our study. The 39 participants were divided into 13 study sessions, with each session comprising 3 participants."

Table 2: "Emerging "claim" patterns from the forum."

Table 3: "Emerging "evidence" patterns from the forum."

Table 4: "Emerging "reasoning" patterns from the forum."

Table 5: "An overview of post counts in the study."

Figure 1: "Three participants, Alice, Bob, and Carol, are engaging in an online forum debate on the topic "Messi vs. Ronaldo: Who is better?" with the assistance of ChatGPT. Image credits: Football España (left), BBC Sport (right)."

Figure 2: "The study consists of two parts: Part 1: One-on-One Turn-Based Debate and Part 2: Three-Person Free Debate. In Part 1, two participants, Alice and Bob, with opposing stances, engaged in a turn-based debate. In Part 2, a third participant, Carol, who shares the same stance as one of the original participants, joined the ongoing debate. Image credit: ESPN FC."

Figure 3: "One participant (P11) attempted to prompt ChatGPT to generate a poem containing insulting jokes. ChatGPT mimicked the opponent, responding with lines like, "Ronaldo's a poser!" and "Messi's the greatest ..."."

Figure 4: "Participants tried to introduce aggressive content by themselves when they were not satisfied with the response of ChatGPT after several times re-prompting. This figure shows an example of P3."

Figure 5: "Participants repeated similar content in their posts several times, reflecting their reliance on the response of ChatGPT. This figure shows an example of P16 and P17."

Figure 6: "Participants shared similar content in their posts, regardless of whether they were opponents (Example 1) or teammates (Example
2). By quoting other members' posts to prompt ChatGPT and using the information generated, their posts contained similar content."

Figure 7: "The statistics show the frequency of various argumentation patterns across three categories (i.e., claims, evidence, and reasoning) measured by the number of posts in which these patterns are present."

Figure 8: "Participants requested statistics from ChatGPT and integrated the generated content into their posts. This figure shows an example of P10."

Figure 9: "Other combinations of patterns: (A) "Reasoning + Evidence": The use of rebuttals and counterexamples. (B) "Claim + Claim":
The use of concessive claims with other types of claims. (C) "Evidence + Evidence": The combinations of different types of evidence."

Figure 10: "Participants blended content generated by ChatGPT with internet slang and emojis to adjust the language style, making it feel more human-like"