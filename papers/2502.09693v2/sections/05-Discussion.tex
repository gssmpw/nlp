\subsection{Theoretical Implications}

Our study explored the progress of online argument-making with the assistance of GenAI. To answer the first research question, our findings suggest that participants prompted GenAI by providing the specific context of the debate, trying to provoke aggressive responses. In this process, they also tried to balance their original stances and opinions with the content provided by GenAI. Various patterns emerged from the online forum posts, and participants combined different patterns for argumentation. They also committed logical fallacies in collaboration with GenAI. After a new person joined the debate, participants tended to maintain the original workflow of interacting with GenAI, while some reduced the usage of GenAI. In the free debate, two participants in the one-on-one debate formed teams with either a new member or GenAI, depending on their stances.

\subsubsection{Balancing the role of GenAI in the debate (RQ1)}

Previous research has primarily focused on the outcomes of co-writing with GenAI, evaluating the benefits and challenges. However, few studies have delved into the detailed process of argument-making. In our study, we observed that participants adapted their strategies to tailor GenAI to fit the debate scenario in online forums better. Some strategies are adjusting prompts from general to specific, providing detailed context, or assigning a particular role for GenAI, such as "a football fan" or "a professional debater". These findings extend the understanding of previous work as prompting can be challenging for participants in teamwork~\cite{han_when_2024}, and can also be challenging for non-experts to prompt GenAI~\cite{zamfirescu-pereira_why_2023}. In our context, where GenAI was used simultaneously by online forum members, the prompting process was straightforward for participants.

With support from GenAI, participants gained the confidence to express their opinions. Previous research has also shown that GenAI-powered assistance is beneficial for lifting people's confidence in writing~\cite{li_value_2024}. GenAI tools such as ChatGPT efficiently extract information from the Internet, allowing participants to create more straightforward outlines in academic writing and draw direct inspiration from it~\cite{tu_augmenting_2024}. However, in our study, participants noted that the content provided by ChatGPT was too formal and unnatural for forum posts. As a result, they adjusted the posting style to better fit the online forum's tone. The adoption of ChatGPT produced posts with similar content. Although GenAI has been utilized as a tool for enhancing critical thinking skills~\cite{tanprasert_debate_2024}, our findings revealed its potential harmfulness in inhibiting participants from developing a dialectical perspective and depth of thought.

In our research, participants did not tend to embrace opinions from ChatGPT or build up reciprocal relationships with it. In other words, they did not tend to adapt their opinions or stances to fulfill ChatGPT's expectations. Instead, they tended to maintain control over the entire debate. This aligns with previous literature implying that GenAI has limited normative influence on the co-writing process~\cite{jakesch_co-writing_2023}. Our research also suggests the situational use of GenAI, as participants chose to ignore ChatGPT's responses when there were disagreements of opinions among them. Participants wanted to integrate GenAI's content with their thoughts or utilize it to support their ideas. This notion corroborates with previous research on GenAI's roles when doing creative design tasks, showing that there was a latent hierarchy placing human thoughts above GenAI's content. Specifically, participants viewed GenAI as a validator when disagreements arose, whereas they treated GenAI as a supporter when agreements were reached~\cite{han_when_2024}. This observation also aligns with previous findings about GenAI's limitations in changing people's stances~\cite{tanprasert_debate_2024}, as participants reported that when disagreements arose, they chose to insist on their own opinions rather than follow the guidance of GenAI. In conclusion, participants strategically prompted ChatGPT to acquire information and support their opinions, and they even gave up using GenAI when facing disagreements, resulting in the situational use of ChatGPT. These findings, to some extent, challenged previous studies which suggest that  ChatGPT could decrease users' sense of ownership for argumentative writing~\cite{lee_design_2024, li_value_2024}. We infer that when polarized fans have a clear stance in online forums, they have a sense of accountability to take control of the debate.

\subsubsection{Creating similar posts and logical fallacies (RQ2)}

Our research indicates that participants brainstormed debate strategies with GenAI, acquired vital information, such as statistics and examples from GenAI, and incorporated them into their arguments. This finding echoes prior research which indicates that  GenAI could shift participants' opinions by exerting informational influence, emphasizing its capability of providing new information and persuasive arguments~\cite{jakesch_co-writing_2023}, which may escalate into ethical concerns on the manipulation of people's opinions~\cite{hancock_ai-mediated_2020}.

Participants in an online debate produced posts with similar content when collaborating with ChatGPT. For example, P4 made  arguments based on the same angle of "vision and creativity" three times. Within the context of argumentative essay writing, previous studies have also reported that utilizing GenAI could largely reduce the diversity of people's writing~\cite{li_value_2024}. In addition, homogenization of content may further undermine people's critical thinking skills~\cite{razi_not_2024}.


In addition to similar content, participants also committed logical fallacies in their posts. Previous research has found that deficiencies of GenAI caused by the internally synthesized algorithm of language models~\cite{fischer_generative_2023, razi_not_2024}, which include biased information~\cite{razi_not_2024} and misinformation~\cite{fischer_generative_2023, zhou_understanding_2024}. In contrast, we focused on the behaviors being manifested in collaboration with GenAI. We explored logical fallacies users commit, such as hasty generalizations, ad hominem attacks, and straw man arguments.

Although it is widely recognized that the sports community was overwhelmed with inter-group conflicts and hostile comments~\cite{wang_making_2023, zhang_intergroup_2019}, in our study, ad hominem attacks in the posts were relatively low compared to other kinds of fallacies (\autoref{fig7}). In light of this, future research may explore GenAI's latent persuasive abilities and its potential for alleviating hostile online debates~\cite{jakesch_co-writing_2023}.

\subsubsection{Maintaining the original workflow while reducing the usage of GenAI after a new member joined (RQ3)}

Our research also revealed the impact of GenAI on human behaviors. Previous work found that GenAI may disrupt the argument-making process and force participants to evaluate GenAI's suggestions~\cite{jakesch_co-writing_2023}.  However, prior research did not explore the detailed workflow of this process. In contrast, our research revealed that participants derived a behavioral route of prompting, obtaining information, and organizing thoughts in their interactions with GenAI and tended to maintain this behavior throughout the process. 

After the third participant came into the forum, participants' perceptions toward GenAI changed. We observed that participants teamed up with GenAI during the debate, especially those without a human teammate in Part 2 whose feelings of isolation urged them to do so. This finding extends prior literature on the relationship between humans and GenAI~\cite{han_when_2024}. However, after teaming up with ChatGPT and spending more time interacting with it, the participants without a teammate may give up using GenAI for a more timely response. This finding contradicts previous quantitative measurements showing that GenAI-powered assistance benefits people's productivity~\cite{li_value_2024}. Even though the time for writing may decrease for argumentative essay writing~\cite{li_value_2024}, participants can spend more time interacting with GenAI. This disparity might be caused by the differences between the formal setting of essay writing and the informal setting in online forums. Furthermore, there might also be discrepancies between participants' thoughts and actions, and thus, even though they may improve their productivity with the assistance of GenAI, they could still perceive this process as time-consuming.

While previous research has suggested that GenAI can help students become more engaged with asynchronous online discussions~\cite{lin_case_2024}, our study within a debate setting contradicts this to some extent. Participants found communication with GenAI to be distracting, which hindered their engagement in the debate. This may be explained by GenAI's strengths in providing information, coupled with its limitations in reasoning.

\subsection{Practical Implications}

\subsubsection{Visualizing logical constructs by GenAI}
Participants committed logical fallacies in their posts, highlighting issues in logical construction during the GenAI-mediated online argument-making process. With the continuous evolution of GenAI, it is becoming increasingly flexible in supporting various multimodal input/output (I/O) combinations. Practitioners may consider leveraging various techniques to visualize content structure and logical flow when writing opinion-based pieces. For example, the system could explicitly highlight the logical relationships among different pieces of content. This practice could help enhance users' awareness of the structural and logical aspects of their arguments, promoting iterative rethinking and critical evaluation of logic during argument formation. By doing so, users might create more logically coherent content, thereby enhancing efficient and constructive argument-making on online platforms.

\subsubsection{Developing intent-based argument-writing AI assistants}
We observed that participants adopted diverse methods to interact with ChatGPT, negotiating and balancing their own thoughts with the content provided by ChatGPT when drafting posts. This practice is often time-consuming and sometimes fails to meet participants' personalized needs when arguing with others online. In light of this, practitioners may consider adapting the characteristics of AI agents to better fit users' argument-writing needs based on their previous argument-making styles and human-AI interaction records. This may involve analyzing the patterns they commonly use when arguing with others and the types of information they retrieve from AI agents. This approach could create a more personalized argument-writing companion, reducing the direct prompt engineering effort required and promoting intent-AI interaction~\cite{ding_towards_2024}. Consequently, this may be helpful in improving users' experience, attitudes, and continued intention to use GenAI.


\subsection{Limitations and Future Work}

\subsubsection{Generalizability of participant characteristics}
Although we selected a topic that is relatively well-known globally and tried to include participants with diverse demographic characteristics, the majority of our recruited participants were non-native English speakers from Asia. As a result, the debate in the study may reflect culture-specific perspectives and vary across different ethnic backgrounds. In addition, all participants had an educational background as undergraduate students or even received postgraduate education. Thus, we probably ignored some marginalized groups on online forums. Therefore, other research may consider further diversifying the pool of participants to improve the generalizability of the study and pay much more attention to the marginalized groups, who might be vulnerable to hostile opinions and have less training in critical thinking skills.

\subsubsection{Modalities of content in online forum posts}
One limitation is that participants were required to post text-based content and emojis to the online forum. This meant that content with other modalities (e.g., images, audio, video, etc.) was excluded from this study. However, online forums in the real world usually support posting content in various formats, each of which can help forum members express their opinions and feelings. In light of this, future research may consider including richer modalities in online posts such as images co-created with GenAI in diverse contexts~\cite{fu_being_2024, lc_speculative_2024, lc_together_2023, lc_time_2024}, and exploring the patterns that emerge from these posts.

\subsubsection{Number of online forum members}
Real online discussion often involves multiple members, some joining early and others joining later. In our study, the first two participants were introduced in Part 1, and the third participant was introduced in Part 2, representing those who joined subsequently. The number of participants was limited to three to prevent potential chaos during data collection and presentation. However, the limited number of forum members may not fully capture the dynamics of real online forum discussions. A larger scale of the forum discussion might lead to more intricate discussions and interactions between participants and ChatGPT, potentially influencing the depth and complexity of the discourse. Therefore, further investigation on this topic may consider involving more forum members to understand people in real-world scenarios better.

\subsubsection{User interface and interaction design}
Participants were required to share their screens throughout the entire study process, during which we observed a degree of incoherence when they accessed ChatGPT to construct arguments on the forum. Participants needed to interact with ChatGPT while communicating with other forum members in separate panels. Frequently switching between ChatGPT and the forum may have reduced participants' willingness to use ChatGPT and distracted them from the online discussion. Future work may consider seamlessly integrating GenAI into the online forum interface to promote both human-AI interaction and human-human communication.

\subsubsection{Evaluation methods of online arguments' persuasiveness}
We primarily employed qualitative methods to interpret data from forum posts, ChatGPT records, and interviews. While qualitative methods are effective for probing participants' perceptions, behaviors, and experiences, we did not measure the persuasiveness of their arguments. Therefore, future research may consider adopting quantitative methods to assess the persuasiveness of writing outcomes in collaboration with GenAI. This approach may provide direct evidence to evaluate the effectiveness of GenAI in co-creating arguments with humans.


\subsubsection{Lack of representation of actual online posting environments}
To better observe the argument-making process, we designed both a turn-based debate and a free debate, aiming to gain a nuanced understanding of argument-making behavior in online forums and participants' usage of ChatGPT. However, this artificial setup cannot perfectly replicate natural online debate in a forum where members might hold a variety of stances rather than being extremely polarized as we assumed, either supporting Messi or Ronaldo. If the research setting were based on real online forums instead of the one we designed, it might better represent actual online communication environments and reduce the Hawthorne effect caused by the research.

\subsubsection{Constrained use of ChatGPT and other tools}
To better understand how people use ChatGPT, participants were not allowed to use third-party search engines such as Google during the study. However, in reality, forum members are not forced to use ChatGPT or other specific tools in a constrained way. Additionally, as we used only one GenAI tool, ChatGPT (GPT-4o), as our study apparatus, it also constrained how people obtained the data. Consequently, it may be worth exploring the interplay between GenAI and other types of tools complementing each other to see how GenAI can integrate with participants' information acquisition more naturally.