% \section{Experimental Details}\label{sec:appendix}




\section{Training Dataset}\label{apdx-train_data}
We utilize OpenHermes~\cite{OpenHermes2-5} and FLAN~\cite{longpre2023flan} as sources for text instruction queries. Open-Hermes comprises a diverse collection of text instructions from various sources and FLAN contains a substantial set of task-specific instructions. We anticipate these instruction sets will enhance the model's language capability in both general scenarios and complex reasoning tasks. Following query acquisition, we employ Qwen2-72B-instruct to distill specific instruction responses. This approach is adopted because the distilled responses demonstrate higher quality compared to the original responses. Additionally, since Qwen2-72B-instruct shares the same training data as our base LLM (though with different parameter scales), we hypothesize this alignment would better preserve the original language capabilities.

As for vision data, we collect extensive caption datasets encompassing both general and domain-specific vision domains. Additionally, we incorporate a few simple VQA datasets to enhance the model's learning of vision knowledge. The details of the vision data are presented in \autoref{tab:table-caption-statistic} and \autoref{tab:table-vqa-statistic}.

\section{Evaluation Datasets}\label{evaluation_dataset}
We evaluate ViFT on four downstream benchmarks, the details of the benchmarks are as followed:

\begin{itemize}
    \item \textit{MathVista:}~\citep{lu2023mathvista} it evaluates the LVLM's mathematical reasoning capabilities in multiple vision domains. It contains 6141 evaluation data samples, collected from 28 existing datasets and 3 newly created datasets. 
    \item \textit{MathVerse:}~\citep{zhang2025mathverse} it is an in-depth benchmark for evaluating LVLM's reasoning capability. It consists of 2612 math problems, and each problem is transformed into 6 distinct problem versions. We report the full performance~(ALL) and the performance on vision-mini~(V-mini) subset in our experiments. 
    \item \textit{MathVision:}~\citep{wang2024measuring} it develops a comprehensive and challenging benchmark for 
    evaluating the LVLM's advanced reasoning skill. It comprises 3040 high-quality mathematical problems derived from authentic mathematics competitions. These problems encompass several distinct mathematical disciplines and are categorized across five difficulty levels.
    
    \item \textit{LLaVABench:}~\citep{liu2024visual} t evaluates the model's instruction-following capabilities across diverse visual scenarios, assessing its conversation, detailed description, and complex reasoning skills. Through carefully designed prompts, it measures the model's ability to generate accurate responses based on complex instructions in visual contexts.

\end{itemize}

Following previous studies~\cite{zhang2024internlm,li2024llava}, we conduct evaluations on the {\emph{testmini}} set of MathVista and MathVision. For our evaluation framework, we employ GPT-3.5-turbo to evaluate the performance on visual reasoning, and GPT-4 to evaluate the performance on visual instruction following.


\begin{table}[t]
  \centering
  % \renewcommand\arraystretch{1.1}
  \scalebox{0.9}{
  \begin{tabular}{
  >{\centering\arraybackslash}p{.28\linewidth}
  *{2}{>{\centering\arraybackslash}p{.24\linewidth}
  }}
    \toprule
    Image Source &  Domains   &  Numbers  \\
    \midrule
    ALLaVA  & General  &  564976      \\
    LLaVAR  &  General   &  250000 \\
    ShareGPT    &  General   &  200000 \\
    COCO  &  General   &  82783  \\
    VG &  General  &  158557  \\ 
    Synthdog & OCR  &  29765  \\
    GeoQA  &  Math   &   6027   \\
    CLEVR  &  Math   &  20000 \\
    FigureQA  &  Figure  &  20000 \\
    DocVQA  &  Document  &  10194  \\
    TabMWP   &   Table   &  20000 \\
    ChartQA  &  Chart   &  18317  \\
    DVQA  &  Diagram   &  30000 \\
    MMarxiv  &  Academic  &  54399 \\
    IconQA  &  Icons  &  18946  \\
    AI2D   &  Science  &  4903  \\
    ScienceQA   &  Science  &  6757  \\

  \bottomrule
  \end{tabular}}
    \caption{Details of caption training data.
    \label{tab:table-caption-statistic}
  }
\end{table}

% \begin{figure}[t]
%     \centering
%     \includegraphics[width=0.95\linewidth]{images/prompts.pdf}
%     \caption{Prompts for caption generation.}
%     \label{fig:prompts}
% \end{figure}


\begin{table}[t]
  \centering
  % \renewcommand\arraystretch{1.1}
  \scalebox{0.9}{
  \begin{tabular}{
  >{\centering\arraybackslash}p{.29\linewidth}
  *{2}{>{\centering\arraybackslash}p{.24\linewidth}
  }}
    \toprule
    Image Source &  Domains   &  Numbers  \\
    \midrule
    VQAv2  &  General   &  20000  \\
    GQA &  General  &  20000  \\ 
    AI2D   &  Science  &  12413  \\
    CLEVR  &  Math   &  20000 \\
    CLEVR-Math  &  Math   &  20000 \\
    DocVQA  &  Document  &  10194  \\
    FigureQA  &  Figure  &  20000 \\
    Geometry3K  &  Math  &  2101 \\
    GeoQA  &  Math  & 20000  \\
    IconQA  &  Icons  &  20000  \\
    PlotQA  &  Figure   &  20000 \\ 
    DocVQA  &  Document  &  10194  \\
    ChartQA  &  Chart   &  18317  \\
    DVQA  &  Diagram   &  20000 \\
  \bottomrule
  \end{tabular}}
    \caption{Details of VQA training data~(optional).
    \label{tab:table-vqa-statistic}
  }
\end{table}



\section{Baselines}\label{apdx-baselines}


We compare ViFT with a number of existing open-source LVLMs. 
% These models include early LVLMs~(MiniGPT4~\cite{zhu2023minigpt}, LLaVA-1.0~\cite{liu2024visual}, and LLaVA-1.5~\cite{liu2024improved}), advanced LVLM trained with better-quality visual instructions~(LLaVA-Next~\cite{liu2024llavanext}, LLaMA-3.2-Vision~\cite{meta2024llama}, and MiniCPM-V-2.5~\cite{yao2024minicpm}), and current state-of-the-art LVLMs~(InternLM-XComposer~(IXL-2.5)~\cite{zhang2024internlm}, Qwen2-VL~\cite{wang2024qwen2} and LLaVA-OneVision~\cite{li2024llava}). 
Notably, Although IXL-2.5 and Qwen2-VL achieve impressive performance, they are trained on a extensive multimodal datasets (exceeding 80M samples). LLaVA-OneVision, on the other hand, utilizes a relatively smaller but still substantial dataset of 9.5M samples. Compared to these models, our ViFT demonstrates superior data efficiency by requiring only 2.7M~(2.9M for ViFT-A) training samples. We present more detailed information of baseline LVLMs in \autoref{apdx-baselines}.
Apart from baseline LVLMs introduced above, we also include LLaVA-Adapter~\cite{zhang2024llama} and Mini-Gemini~\cite{li2024mini} in \autoref{fig:intro_graph}. We report the training data size of these models in \autoref{tab:table-statistic}. For models without exact number of training data size in papers, we estimate the lower bound of the data size. For IXL-2.5, we report the incomplete training data size in their paper. For Qwen2-VL, we estimate the training data size by dividing the total training tokens with the max token length for each sample.

\begin{table}[t]
  \centering
  % \renewcommand\arraystretch{1.1}
  \scalebox{0.9}{
  \begin{tabular}{>{\centering\arraybackslash}p{.33\linewidth}
  |>{\centering\arraybackslash}p{.22\linewidth}
  >{\centering\arraybackslash}p{.18\linewidth}
  >{\centering\arraybackslash}p{.17\linewidth}
   }
    \toprule
    Model  & \# Captions  &  \# Others &  \# Total  \\
    \midrule
    LLaVA-1.5  &  560K  &  665K  &  1.2M  \\
    LLaVA-Next  &  560K  &  760K  &  1.3M  \\
    MiniCPM-V-2.5  &  570M  &  8.3M  & 578M   \\
    IXL-2.5  &  > 400M  &  > 2M  &  > 402M   \\
    Qwen2-VL  &  > 87.5M  &  -  &  > 87.5M  \\
    LLaVA-OV   &  5.5M  &  4.0M  &  9.5M  \\
    ViFT  &  2.7M  &  0     &  2.7M  \\
    ViFT-A  &  2.7M  &  200K   &  2.9M  \\
    
  \bottomrule
  \end{tabular}}
    \caption{The statistics of training data for ViFT and other baseline LVLMs. Given that Qwen2-VL and IXL do not provide their specific training data volumes, we estimate the minimum data size based on the training data descriptions provided in their  papers.}
    \label{tab:table-statistic}
\end{table}


% \section{Implementation Detail}
% Ideally, our approach could be completed in a single training stage. However, due to the significant length disparity between low-quality web captions and high-quality captions, aligning their lengths during training would introduce additional computational overhead. Therefore, we adopted a two-phase training strategy: In the first stage, we train on web captions. In the second stage, we train on a mixture of high-quality captions and text instructions. This training approach effectively reduces time complexity without compromising performance. During fine-tuning, the learning rate is set to 1e-5 for the LLM and 2e-6 for the projector. The batch size is configured as 8 for each GPU, All models are trained for one epoch.

% During inference, we omitt the entanglement process in the first 25\% of layers. This decision is supported by previous studies~\cite{chen2025image} demonstrating that visual information aggregates with instruction tokens in the early layers of the model. We aim to preserve this visual information aggregation process to maintain robust vision capability. We set different $\beta$ for different tasks to achieve optimal performance. Concretely, we set $\beta=0.05$ for visual reasoning tasks and $\beta=0.1$ for visual instruction following tasks.


\section{Additional Experiments}\label{appendix-exp}
\subsection{One-stage training vs two-stage}
During training, we adopt a two-stage training strategy to reduce cost. We investigate whether a two-stage training strategy produces significant different results compared to a one-stage approach. The results are presented in \autoref{tab:table-train}. As we can observe, single-stage training exhibits almost the same performance as two-stage training, yet is more efficient.

% \subsection{The Effect of Ability-fused Inference}

% We investigate the effectiveness of our proposed ability-fused inference. Specifically, we set both $\alpha$ and $\beta$ to 0 and evaluate the model's performance. The results are shown in \autoref{tab:table-ablation-afi}. We observe that without ability fusion, the model exhibits significant performance degradation across all benchmarks. Specifically, during standard inference, the model merely generates image captions rather than following instructions, as shown in \autoref{tab:case1} and \autoref{tab:case2}. This indicates that the individual abilities acquired through fine-tuning cannot be effectively combined through standard inference to address practical visual tasks. However, our proposed ability-fused inference successfully addresses this limitation.

\begin{table}[t]
  \centering
  % \renewcommand\arraystretch{1.1}
  \scalebox{0.9}{
  \begin{tabular}{
  >{\centering\arraybackslash}p{.22\linewidth}
   >{\centering\arraybackslash}p{.15\linewidth}
  >{\centering\arraybackslash}p{.18\linewidth}
  >{\centering\arraybackslash}p{.12\linewidth}
  >{\centering\arraybackslash}p{.12\linewidth}
   }
    \toprule
    Strategy   & Time & MathVista  &  LBench  &  MathV  \\
    \midrule
    Two-Stage  &  $\approx$ 36h  &  49.2  &  24.0 &  82.2    \\
    One-Stage  &  $\approx$ 47h &   49.8  &  23.2  &  81.9   \\

  \bottomrule
  \end{tabular}}
    \caption{Comparison of ViFT with different training strategy.
  }
  \label{tab:table-train}
\end{table}


% \begin{table}[t]
%   \caption{The ablation of ability-fused inference. $\neg$ AF indicates inference without ability fusion.}
%   \label{tab:table-ablation-afi}
%   \centering
%   % \renewcommand\arraystretch{1.1}
%   \scalebox{0.9}{
%   \begin{tabular}{>{\centering\arraybackslash}p{.18\linewidth}
%   |*{2}{>{\centering\arraybackslash}p{.21\linewidth}}
%   >{\centering\arraybackslash}p{.24\linewidth}
%    }
%     \toprule
%     Model  &  MathVista  &  MathVision  &  LLaVABench  \\
%     \midrule
%     ViFT  &  49.2  &  24.0 &  82.2   \\
%     $\neg$ AF  &  46.1 &  13.5  &  59.1   \\
%   \bottomrule
%   \end{tabular}}
% \end{table}

\section{Case Studies}\label{apdx-cases} 
This section presents ViFT's generating behaviour under various experimental settings, and a comparison of ViFT with other baseline LVLMs. The detailed results are presented below.

\paragraph{Comparison of ViFT with Existing LVLMs.}  We present the results of ViFT and other SoTA LVLMs in \autoref{tab:case5}, \autoref{tab:case10}, and \autoref{tab:case11}. As we can observe, many state-of-the-art baseline LVLMs fail to  generate a high-quality reasoning path even when chain-of-thought prompt is leveraged, such as Qwen2-VL and LLaVA-OneVision. LLaMA-3.2-Vision-11B is the only baseline LVLM that is capable of generating relatively coherent chains of thought, but it may occasionally produce minor error during the reasoning process~(\eg misinterprets the sine function), which results in an erroneous final result. InternLM-XComposer-2.5 sometimes generates promising reasoning path, but at other times produces shorter responses. For Qwen2-VL-7B and LLaVA-OneVision-7B, their reasoning is extremely short, leading to incorrect results. As this is actually a simple geometric problem, which should not pose a challenge for LLMs, the observed erroneous behavior is likely due to a ability degradation resulting from conflicts between knowledge learned from visual instruction tuning and the intrinsic abilities of LLM itself. Compared to these models, our ViFT is the only one that can almost fully inherit the original reasoning capabilities from its backbone LLM, generate high-quality reasoning paths, and ensure the accuracy and consistency of intermediate steps, ultimately achieving more correct results.

\paragraph{ViFT's Behaviour With or Without Ability-fused inference.}

We present the results of ViFT with or without ability-fused inference in \autoref{tab:case0}, \autoref{tab:case1}, and \autoref{tab:case2}. First, we discover that the fine-tuned model will elicit diverse individual ability with different modality inputs, instead of a combined capability via standard inference. Specifically, it mostly produces a image description for multimodal inputs, regardless of the actual instruction. However, when we leverage the same instruction part as text-only input, the model follows the instruction and exhibits expected behavior, though it fails to provide a correct answer due to the missing image. Second, after implementing our ability-fused inference, ViFT maintains its visual perception capability while successfully leveraging the fused task-solving ability to generate correct responses, validating the effectiveness of our method.

% In \autoref{tab:case0}, we present the results of ViFT without ability-fused inference. We discover that after ability-specific fine-tuning, the model can not elicit normal multimodal task-solving ability during standard inference and will elicit diverse behavior with different modality inputs. Specifically, it mostly produces a description for the given image, regardless of the actual instruction. However, when we leverage the same instruction without the image as input, the model successfully follows the given instruction and exhibits expected behavior, though it fails to provide a correct answer due to the missing image.

% In \autoref{tab:case1} and \autoref{tab:case2}, we present the results of ViFT with or without ability-fused inference. We discover that ViFT tends to generate captions without ability fusion. In \autoref{tab:case1}, ViFT generates a caption first, then directly produce a incorrect answer without actually reasoning. In \autoref{tab:case2}, ViFT merely generates a caption and doesn't answer the question at all. The generated captions are accurate, demonstrating ViFT's effective visual perception ability. However, the lack of task-solving ability results in incorrect behavior. Nevertheless, after implementing our ability-fused inference approach, ViFT maintains its visual perception capability while successfully leveraging the fused task-solving ability to generate correct responses, validating the effectiveness of our method.


\paragraph{ViFT's Behaviour With Different $\alpha$ Setting.} We present the results of ViFT's behaviour with different $\alpha$ in \autoref{tab:case4}. We observe that when $\alpha$ is set to a low value~(0.8 in our case), ViFT rejects to answer the question for lack of visual information. This may be attributed to diminished visual perception ability caused by low value of $\alpha$. As $\alpha$ gradually increases, the model begins to exhibit normal reasoning behaviour and generates accurate results. Within a certain range of $\alpha$, the model's behavior remains notably consistent, showing no significant variations despite the changes of $\alpha$ values. We also notice that when $\alpha$ reaches $1.1$, the model starts to first generate a image description, then answer the question. While the reasoning behavior remains largely consistent, this indicates that a larger $\alpha$ may enhance the model's focus on visual information.

\paragraph{ViFT's Behaviour With Different $\beta$ Setting.}

The results of ViFT's behaviour with different $\beta$ are presented in \autoref{tab:case3}. We discover that when $\beta=0$, the model exhibits a caption-only behaviour, similar to ViFT without ability fusion. As $\beta$ increases, the model progressively demonstrates consistent reasoning behavior, aligning with the behavioral patterns observed during $\alpha$ variation. However, when $\beta$ reaches $0.4$, ViFT will output random strings, which indicates that a large $\beta$ can result in unstable behaviour in the model. This also corresponds to the previous findings where the model's performance experiences a sudden and drastic decline once the beta value reaches a certain threshold.


\begin{table*}[htbp]
    \centering
    \renewcommand{\arraystretch}{1.2} 
    \setlength{\tabcolsep}{8pt} 
    \begin{tabular}{p{14cm}}
        \centering
        \includegraphics[width=0.6\textwidth]{cases/cases_5.jpg}
        
        \begin{tcolorbox}[colback=gray!5, colframe=gray!40, sharp corners, width=\linewidth, boxrule=0.5mm]
        \small
        \textbf{Problem: }
        Please first conduct reasoning, and then answer the question and provide the correct option letter, e.g., A, B, C, D, E, at the end.
        
        As shown in the figure, then the value of sinB is equal to ()
        Choices:
        
        A:4/3
        
        B:3/4
        
        C:4/5
        
        D:3/5

        \textbf{Ground Truth:} C
        \end{tcolorbox}  % Reduce the vertical space between input and output
        \begin{tcolorbox}[colback=gray!5, colframe=gray!40, sharp corners, width=\linewidth, boxrule=0.5mm]
            \small
            \textbf{LLaMA-3.2-Vision-11B:}
            \\
**Step 1: Understand the problem**

The problem asks for the value of sinB in the given figure. To find this, we need to use trigonometric concepts, specifically the definition of sine in a right-angled triangle.

**Step 2: Identify the triangle and its sides**

From the figure, we can see that triangle ABC is a right-angled triangle with sides labeled as follows:

*   Side AB = 3
*   Side BC = 4
*   Hypotenuse AC = 5

**Step 3: Recall the definition of sine**

The sine of an angle in a right-angled triangle is defined as the ratio of the length of the side opposite the angle to the length of the hypotenuse.

**Step 4: Apply the definition of sine to find sinB**

In this case, angle B is opposite side AB, and the hypotenuse is AC. Therefore, sinB = AB / AC.

**Step 5: Calculate sinB**

Using the values from step 2, we can calculate sinB as follows:

sinB = AB / AC = 3 / 5

Answer: D


        \end{tcolorbox}
        

        \begin{tcolorbox}[colback=gray!5, colframe=gray!40, sharp corners, width=\linewidth, boxrule=0.5mm]
        \small
        \textbf{InternLM-XComposer-2.5-7B:}
        \\
To find the value of sinB, we need to use the definition of sine in a right-angled triangle. The sine of an angle in a right-angled triangle is the ratio of the length of the side opposite the angle to the length of the hypotenuse. In this case, side AC is opposite angle B, and AB is the hypotenuse.

Given that AC = 3 and AB = 4, we can calculate sinB as follows:

sinB = opposite/hypotenuse
sinB = 3/4

Therefore, the correct answer is B: \(\frac{3}{4}\).

\end{tcolorbox}

        \begin{tcolorbox}[colback=gray!5, colframe=gray!40, sharp corners, width=\linewidth, boxrule=0.5mm]
        \small
        \textbf{Qwen2-VL-7B:}
        \\
Solution: In right triangle triangle ABC, angle C=90°, AC=3, BC=4.
According to the Pythagorean theorem, we have AB=5.
Therefore, sinB=AC/AB=3/5.
So, the answer is D.


\end{tcolorbox}

\end{tabular}
\caption{A comparison of ViFT with other baseline LVLMs~(part1).}
\label{tab:case5}
\end{table*}



\begin{table*}[htbp]
    \ContinuedFloat
    \begin{tabular}{p{14cm}}
        \centering

                \begin{tcolorbox}[colback=gray!5, colframe=gray!40, sharp corners, width=\linewidth, boxrule=0.5mm]
        \small
        \textbf{LLaVA-OneVision-7B:}
        \\
Solution: Since angle C=90°, we have sinB=AC/AB=3/5.
Therefore, the answer is D.


\end{tcolorbox}
        \begin{tcolorbox}[colback=gray!5, colframe=gray!40, sharp corners, width=\linewidth, boxrule=0.5mm]
        \small
        \textbf{ViFT-7B:}
        \\
To determine the value of sinB, we need to identify the sides of the triangle relative to angle B. The sine of an angle in a right-angled triangle is defined as the ratio of the length of the side opposite the angle to the length of the hypotenuse.

From the diagram:
- Side AC (opposite to angle B) has a length of 4 units.
- Side BC (adjacent to angle B) has a length of 3 units.
- Side AB (the hypotenuse) is not explicitly given but can be calculated using the Pythagorean theorem if needed.

Using the Pythagorean theorem for triangle ABC:
\[ AB^2 = AC^2 + BC^2 \]
\[ AB^2 = 4^2 + 3^2 \]
\[ AB^2 = 16 + 9 \]
\[ AB^2 = 25 \]
\[ AB = 5 \]

Now, we can calculate sinB:
\[ \sin B = \frac{\text{opposite}}{\text{hypotenuse}} = \frac{AC}{AB} = \frac{4}{5} \]

Therefore, the correct answer is C: \(\frac{4}{5}\).

\end{tcolorbox}
\end{tabular}
\caption{A comparison of ViFT with other baseline LVLMs~(part 1).}
\end{table*}

% #######


\begin{table*}[htbp]
    \centering
    \renewcommand{\arraystretch}{1.2} 
    \setlength{\tabcolsep}{8pt} 
    \begin{tabular}{p{14cm}}
        \centering
        \includegraphics[width=0.4\textwidth]{cases/cases_10.jpg}
        
        \begin{tcolorbox}[colback=gray!5, colframe=gray!40, sharp corners, width=\linewidth, boxrule=0.5mm]
        \small
        \textbf{Problem: }
        Please first conduct reasoning, and then answer the question and provide the correct answer.
        
        Find the surface area of the given cylinder. All measurements in the diagram are in mm.

        Round your answer to two decimal places.

        \textbf{Ground Truth:} 109603.88
        \end{tcolorbox}  % Reduce the vertical space between input and output
        \begin{tcolorbox}[colback=gray!5, colframe=gray!40, sharp corners, width=\linewidth, boxrule=0.5mm]
            \small
            \textbf{LLaMA-3.2-Vision-11B:}
            \\
**Step 1: Understand the problem**

The problem asks for the surface area of a cylinder with a height of 80 mm and a radius of 98 mm.

**Step 2: Identify the formula for the surface area of a cylinder**

The formula for the surface area of a cylinder is:
\[ \text{Surface Area} = 2\pi r^2 + 2\pi rh \]
where \(r\) is the radius and \(h\) is the height.

**Step 3: Plug in the values given in the problem**
Given that the radius \(r = 98\) mm and the height \(h = 80\) mm, we can substitute these values into the formula.

**Step 4: Calculate the surface area**
\[ \text{Surface Area} = 2\pi (98)^2 + 2\pi (98)(80) \]
\[ \text{Surface Area} = 19208\pi + 15680\pi = 34888\pi \approx 34888 \times 3.14159 \approx 109541.31 \, \text{mm}^2\]

**Step 5: Round the answer to two decimal places**
Rounded to two decimal places, the surface area is approximately 109541.31 mm\(^2\).

        \end{tcolorbox}
        

        \begin{tcolorbox}[colback=gray!5, colframe=gray!40, sharp corners, width=\linewidth, boxrule=0.5mm]
        \small
        \textbf{InternLM-XComposer-2.5-7B:}
        \\
To find the surface area of a cylinder, we need to calculate the area of the two circular bases and the lateral surface area. The formula for the area of a circle is \( A = \pi r^2 \), where \( A \) is the area and \( r \) is the radius.
First, let's find the area of one base:
\[ A_{\text{base}} = \pi r^2 = \pi (98)^2 = 9624.6 \, \text{mm}^2 \]
Since there are two bases, we double the area of one base to get the total area of both bases:
\[ A_{\text{total bases}} = 2 \times A_{\text{base}} = 2 \times 9624.6 = 19249.2 \, \text{mm}^2 \]
Next, we need to find the lateral surface area. The lateral surface area of a cylinder is given by the formula \( A = 2\pi rh \), where \( h \) is the height of the cylinder.
\[ A_{\text{lateral}} = 2\pi rh = 2\pi (98) (80) = 19200 \, \text{mm}^2 \]
Finally, we add the areas of the bases and the lateral surface area to get the total surface area of the cylinder:
\[ A_{\text{total}} = A_{\text{total bases}} + A_{\text{lateral}} = 19249.2 + 19200 = 38449.2 \, \text{mm}^2 \]

\end{tcolorbox}

\end{tabular}
\caption{A comparison of ViFT with other baseline LVLMs~(part 2).}
\label{tab:case10}
\end{table*}



\begin{table*}[htbp]
    \ContinuedFloat
    \begin{tabular}{p{14cm}}
        \centering
        
        \begin{tcolorbox}[colback=gray!5, colframe=gray!40, sharp corners, width=\linewidth, boxrule=0.5mm]
        \small
        \textbf{Qwen2-VL-7B:}
        \\
The surface area of a cylinder is given by the formula \(2\pi rh + 2\pi r^2\), where \(r\) is the radius and \(h\) is the height. Substituting the given values, we get \(2\pi \times 98 \times 80 + 2\pi \times 98^2\). Calculating this, we find the surface area to be approximately 23,567.44 mm². Therefore, the answer is 23567.44.



\end{tcolorbox}
                \begin{tcolorbox}[colback=gray!5, colframe=gray!40, sharp corners, width=\linewidth, boxrule=0.5mm]
        \small
        \textbf{LLaVA-OneVision-7B:}
        \\
The surface area of a cylinder is given by the formula: 2$\pi r(h + r)$, where r is the radius and h is the height.
Substituting the given values, we get: $2 * 3.1416 * 98 * (98 + 80) = 2 * 3.1416 * 98 * 178 = 12210.4 mm^2$
Therefore, the surface area of the cylinder is $12210.4 mm^2$.


\end{tcolorbox}
        \begin{tcolorbox}[colback=gray!5, colframe=gray!40, sharp corners, width=\linewidth, boxrule=0.5mm]
        \small
        \textbf{ViFT-7B:}
        \\
To solve this problem, we need to analyze the given geometric configuration and use properties of circles and angles.

Given:

- There is a circle with center \( A \).

- Points \( B \), \( C \), \( D \), and \( E \) lie on the circumference of the circle.

- Angle \( BAC \) and angle \( EAD \) sum up to 180 degrees.

\#\#\# Analysis:

1. **Circle Properties**: Since \( A \) is the center of the circle, any line segment from \( A \) to a point on the circumference (e.g., \( B \), \( C \), \( D \), or \( E \)) is a radius of the circle.

2. **Angle Sum Property**: The sum of opposite angles in a cyclic quadrilateral is 180 degrees. This means that if \( BAC \) and \( EAD \) are supplementary, they form a straight line when extended through the circle.

\#\#\# Steps to Solve:

To find the surface area of a cylinder, we use the formula:

\[ A = 2\pi r(h + r) \]

where:
- \( A \) is the surface area,
- \( r \) is the radius of the base,
- \( h \) is the height of the cylinder.

Given:
- Radius (\( r \)) = 98 mm,
- Height (\( h \)) = 80 mm.

Substitute these values into the formula:

\[ A = 2\pi (98)(80 + 98) \]
\[ A = 2\pi (98)(178) \]
\[ A = 2\pi \times 17444 \]
\[ A = 34888\pi \]

Now, calculate the numerical value using \(\pi \approx 3.14159\):

\[ A \approx 34888 \times 3.14159 \]
\[ A \approx 109678.472 \]

Rounding to two decimal places:

\[ A \approx 109678.47 \]

Therefore, the surface area of the cylinder is approximately 109678.47 square millimeters.

\end{tcolorbox}
\end{tabular}
\caption{A comparison of ViFT with other baseline LVLMs~(part 2).}
\end{table*}


% ########

\begin{table*}[htbp]
    \centering
    \renewcommand{\arraystretch}{1.2} 
    \setlength{\tabcolsep}{8pt} 
    \begin{tabular}{p{14cm}}
        \centering
        \includegraphics[width=0.4\textwidth]{cases/cases_11.jpg}
        
        \begin{tcolorbox}[colback=gray!5, colframe=gray!40, sharp corners, width=\linewidth, boxrule=0.5mm]
        \small
        \textbf{Problem: }
        Please first conduct reasoning, and then answer the question and provide the correct answer.
        
        Three circles with centres $A, B, C$ touch each other in pairs from the outside (see diagram). Their radii are 3,2 and 1. How big is the area of the triangle $A B C$?

        \textbf{Ground Truth:} 6
        \end{tcolorbox}  % Reduce the vertical space between input and output
        \begin{tcolorbox}[colback=gray!5, colframe=gray!40, sharp corners, width=\linewidth, boxrule=0.5mm]
            \small
            \textbf{LLaMA-3.2-Vision-11B:}
            \\
To find the area of triangle $ABC$, we can use the formula for the area of a triangle given its base and height. The base of the triangle is the distance between points $A$ and $B$, and the height is the distance from point $C$ to line $AB$.

The distance between points $A$ and $B$ is equal to the sum of their radii, which is $3 + 2 = 5$. This is because the circles touch each other in pairs from the outside, so the distance between their centers is equal to the sum of their radii.

The distance from point $C$ to line $AB$ is equal to the radius of circle $C$, which is $1$. This is because point $C$ is on the circumference of circle $C$, and the radius of a circle is the distance from its center to any point on its circumference.

Now, we can use the formula for the area of a triangle:

Area = (base × height) / 2

Substituting the values we found earlier, we get:

Area = (5 × 1) / 2
= 5/2
= 2.5

Therefore, the area of triangle $ABC$ is 2.5 square units.

*Answer*: 2.5

        \end{tcolorbox}
        

        \begin{tcolorbox}[colback=gray!5, colframe=gray!40, sharp corners, width=\linewidth, boxrule=0.5mm]
        \small
        \textbf{InternLM-XComposer-2.5-7B:}
        \\
To find the area of triangle \(ABC\), we can use Heron's formula, which states that for a triangle with sides of length \(a\), \(b\), and \(c\), the area (\(A\)) is given by:

\[ A = \sqrt{s(s-a)(s-b)(s-c)} \]

where \(s\) is the semi-perimeter of the triangle, defined as:

\[ s = \frac{a + b + c}{2} \]

In this case, the sides of the triangle are the distances between the centers of the circles. The distance between the centers of circles with radii 3 and 2 is the sum of their radii, so it is \(3 + 2 = 5\). The distance between the centers of circles with radii 2 and 1 is \(2 + 1 = 3\), and the distance between the centers of circles with radii 3 and 1 is \(3 + 1 = 4\).

The semi-perimeter \(s\) of triangle \(ABC\) is therefore:

\[ s = \frac{5 + 3 + 4}{2} = \frac{12}{2} = 6 \]

Using Heron's formula, the area \(A\) of triangle \(ABC\) is:

\[ A = \sqrt{6(6-5)(6-3)(6-4)} = \sqrt{6 \cdot 1 \cdot 3 \cdot 2} = \sqrt{36} = 6 \]

Therefore, the area of triangle \(ABC\) is \(6\) square units.

\end{tcolorbox}

\end{tabular}
\caption{A comparison of ViFT with other baseline LVLMs~(part 3).}
\label{tab:case11}
\end{table*}



\begin{table*}[htbp]
    \ContinuedFloat
    \begin{tabular}{p{14cm}}
        \centering
        
        \begin{tcolorbox}[colback=gray!5, colframe=gray!40, sharp corners, width=\linewidth, boxrule=0.5mm]
        \small
        \textbf{Qwen2-VL-7B:}
        \\
The lengths of the sides of triangle ABC are equal to the sums of the radii of the circles that touch at those points. Therefore, the lengths of the sides of triangle ABC are 3+2=5, 2+1=3, and 3+1=4. The semi-perimeter of triangle ABC is (5+3+4)/2=6. Therefore, the area of triangle ABC is $\sqrt{(6×(6-5)×(6-3)×(6-4))}=3\sqrt{5}$. Therefore, the answer is $3\sqrt{5}$.



\end{tcolorbox}
                \begin{tcolorbox}[colback=gray!5, colframe=gray!40, sharp corners, width=\linewidth, boxrule=0.5mm]
        \small
        \textbf{LLaVA-OneVision-7B:}
        \\
The answer is 4.


\end{tcolorbox}
        \begin{tcolorbox}[colback=gray!5, colframe=gray!40, sharp corners, width=\linewidth, boxrule=0.5mm]
        \small
        \textbf{ViFT-7B:}
        \\
To find the area of triangle \(ABC\), we need to determine the lengths of its sides and then use Heron's formula or another suitable method.

Given:
- Circle \(A\) has a radius of 3 units.
- Circle \(B\) has a radius of 2 units.
- Circle \(C\) has a radius of 1 unit.

Since the circles touch each other externally, the distance between the centers of any two circles is equal to the sum of their radii.

1. **Distance between centers \(A\) and \(B\)**:
   \[
   AB = r_A + r_B = 3 + 2 = 5 \text{ units}
   \]

2. **Distance between centers \(A\) and \(C\)**:
   \[
   AC = r_A + r_C = 3 + 1 = 4 \text{ units}
   \]

3. **Distance between centers \(B\) and \(C\)**:
   \[
   BC = r_B + r_C = 2 + 1 = 3 \text{ units}
   \]

Now, we have the side lengths of triangle \(ABC\):
- \(AB = 5\) units
- \(AC = 4\) units
- \(BC = 3\) units

We can use Heron's formula to calculate the area of triangle \(ABC\). Heron's formula states that the area \(K\) of a triangle with sides of lengths \(a\), \(b\), and \(c\) is given by:
\[
K = \sqrt{s(s-a)(s-b)(s-c)}
\]
where \(s\) is the semi-perimeter of the triangle:
\[
s = \frac{a + b + c}{2}
\]

For our triangle:
\[
s = \frac{5 + 4 + 3}{2} = \frac{12}{2} = 6 \text{ units}
\]

Substituting the values into Heron's formula:
\[
K = \sqrt{6(6-5)(6-4)(6-3)} = \sqrt{6 \cdot 1 \cdot 2 \cdot 3} = \sqrt{36} = 6 \text{ square units}
\]

Therefore, the area of triangle \(ABC\) is:
\[
\boxed{6}
\]

\end{tcolorbox}
\end{tabular}
\caption{A comparison of ViFT with other baseline LVLMs~(part 3).}
\end{table*}



\begin{table*}[htbp]
    \centering
    \renewcommand{\arraystretch}{1.2} 
    \setlength{\tabcolsep}{8pt} 
    \begin{tabular}{p{14cm}}
        \centering
        \includegraphics[width=0.4\textwidth]{cases/Cases_0.jpg}
        
        \begin{tcolorbox}[colback=gray!5, colframe=gray!40, sharp corners, width=\linewidth, boxrule=0.5mm]
        \small
        \textbf{Problem: }
        Subtract all small cyan cubes. Subtract all large yellow rubber cubes. How many objects are left?

        \textbf{Ground Truth:} 8
        \end{tcolorbox}  % Reduce the vertical space between input and output
        \begin{tcolorbox}[colback=gray!5, colframe=gray!40, sharp corners, width=\linewidth, boxrule=0.5mm]
            \small
            \textbf{ViFT With Vision-Text Inputs}
            \\
The image depicts a collection of various geometric shapes, including cylinders and cubes, arranged on a flat surface. The objects vary in size, color, and material properties such as metallic or matte finishes.

Here is a detailed description of the objects present:

1. **Cylinders**:
   - There are five cylindrical objects in total.
   
   - Two large metallic cylinders: one gray and one red.
   
   - One medium-sized metallic cylinder: gold-colored.
   
   - One small metallic cylinder: blue.
   
   - One small matte cylinder: gray.

2. **Cubes**:
   - There are four cube-shaped objects.
   
   - One large metallic cube: red.
   
   - One large matte cube: gray.
   
   - One small metallic cube: cyan.
   
   - One small matte cube: cyan.

3. **Arrangement**:
   - The objects are scattered across the surface without any specific pattern.
   
   - The large metallic cylinders (gray and red) are positioned towards the back left side.
   
   - The medium-sized metallic cylinder (gold) is located near the center-left.
   
   - The small metallic cylinder (blue) is placed towards the right side.
   
   - The small matte cylinder (gray) is situated close to the small metallic cylinder (blue).
   
   - The large metallic cube (red) is positioned towards the front right.
   
   - The large matte cube (gray) is located behind the large metallic cube (red).
   
   - The two small metallic cubes (cyan) are placed in the foreground, with one slightly to the left and the other more centered.
   
   - The two small matte cubes (cyan) are also in the foreground, with one slightly to the left and the other more centered.


        \end{tcolorbox}



        \begin{tcolorbox}[colback=gray!5, colframe=gray!40, sharp corners, width=\linewidth, boxrule=0.5mm]
        \small
        \textbf{ViFT With Text-only Inputs:}
        \\
To solve this problem, we need to know the initial number of small cyan cubes and large yellow rubber cubes before any subtraction takes place. Since these values are not provided in the question, I will create a hypothetical scenario to illustrate how you would solve this problem.

Let's assume:
- There are initially 10 small cyan cubes.
- There are initially 5 large yellow rubber cubes.
- No other types of objects are mentioned.

Step 1: Subtract all small cyan cubes.
- Initial count: 10 small cyan cubes.
- After subtraction: 10 - 10 = 0 small cyan cubes.

Step 2: Subtract all large yellow rubber cubes.
- Initial count: 5 large yellow rubber cubes.
- After subtraction: 5 - 5 = 0 large yellow rubber cubes.

Step 3: Calculate the total number of objects left.
- Small cyan cubes left: 0
- Large yellow rubber cubes left: 0
- Total objects left: 0 + 0 = 0

Therefore, the final value is 0.


\end{tcolorbox}
\end{tabular}
\caption{An example of ViFT with or without ability-fused inference (part 1).}
\label{tab:case0}
\end{table*}




\begin{table*}[htbp]
    \centering
    \renewcommand{\arraystretch}{1.2} 
    \setlength{\tabcolsep}{8pt} 
    \begin{tabular}{p{14cm}}
        \centering
        \includegraphics[width=0.4\textwidth]{cases/Cases_2.jpg}
        
        \begin{tcolorbox}[colback=gray!5, colframe=gray!40, sharp corners, width=\linewidth, boxrule=0.5mm]
        \small
        \textbf{Problem: }
        Is the number of big shiny cars in front of the red airliner greater than the number of big purple road bikes? Think first, then answer.

        \textbf{Ground Truth:} Yes
        \end{tcolorbox}  % Reduce the vertical space between input and output
        \begin{tcolorbox}[colback=gray!5, colframe=gray!40, sharp corners, width=\linewidth, boxrule=0.5mm]
            \small
            \textbf{ViFT Without Ability-Fused Inference:}
            \\
In the image, there is one big shiny car located to the right side of the red airliner. There are no big purple road bikes present in the scene. Therefore, the number of big shiny cars (1) is not greater than the number of big purple road bikes (0).



Answer: (B) No


        \end{tcolorbox}



        \begin{tcolorbox}[colback=gray!5, colframe=gray!40, sharp corners, width=\linewidth, boxrule=0.5mm]
        \small
        \textbf{ViFT With Ability-Fused Inference:}
        \\
To solve this problem, let's break it down step by step:

1. **Identify the objects**: We need to count the number of "big shiny cars" and "big purple road bikes" in relation to the red airliner.

2. **Counting the big shiny cars**:
   - There is one big shiny car (a purple SUV) in front of the red airliner.

3. **Counting the big purple road bikes**:
   - There are no big purple road bikes visible in the image.

4. **Comparison**:
   - The number of big shiny cars in front of the red airliner is 1.
   - The number of big purple road bikes is 0.

5. **Conclusion**:
   - Since 1 is greater than 0, the statement "Is the number of big shiny cars in front of the red airliner greater than the number of big purple road bikes?" is true.

Therefore, the correct answer is (A) Yes.


\end{tcolorbox}
\end{tabular}
\caption{An example of ViFT with or without ability-fused inference (part 2).}
\label{tab:case1}
\end{table*}






\begin{table*}[htbp]
    \centering
    \renewcommand{\arraystretch}{1.2} 
    \setlength{\tabcolsep}{8pt} 
    \begin{tabular}{p{14cm}}
        \centering
        \includegraphics[width=0.3\textwidth]{cases/Cases_1.jpg}
        
        \begin{tcolorbox}[colback=gray!5, colframe=gray!40, sharp corners, width=\linewidth, boxrule=0.5mm]
        \small
        \textbf{Problem: }
        What material appears to make up the creature? Think first, then answer.

        \end{tcolorbox}  % Reduce the vertical space between input and output
        \begin{tcolorbox}[colback=gray!5, colframe=gray!40, sharp corners, width=\linewidth, boxrule=0.5mm]
            \small
            \textbf{ViFT Without Ability-Fused Inference:}
            \\ 
The image presents a 3D rendering of a creature that is a fusion of a camel and a flame. The creature stands on four legs, each ending in black hooves. Its body is predominantly red, with yellow flames dancing across its surface, giving it an appearance reminiscent of a living fire. 

The creature's head is adorned with a pair of glasses, adding a touch of whimsy to its fiery form. The background is a simple gray gradient, which serves to highlight the vibrant colors of the creature. The creature is positioned centrally in the image, drawing the viewer's attention immediately to its unique blend of elements. 

There are no discernible texts or other objects in the image. The relative position of the creature to the background suggests it is standing still, with no indication of movement or action. The image does not provide any additional context or narrative, leaving the interpretation open to the viewer.


        \end{tcolorbox}
        
        \begin{tcolorbox}[colback=gray!5, colframe=gray!40, sharp corners, width=\linewidth, boxrule=0.5mm]
        \small
        \textbf{ViFT With Ability-Fused Inference:}
        \\
To determine what material appears to make up the creature in the image, let's analyze its characteristics:

1. **Color and Texture**: The creature has a vibrant red body with yellow and orange flames emanating from it. This suggests that the material might be something that glows or emits light, possibly indicating heat or energy.

2. **Shape and Structure**: The creature resembles an animal, specifically a camel, but with exaggerated features like large eyes and a flame-like appearance. Its legs are sturdy and end in black hooves.

3. **Material Properties**:
   - **Red Body**: The red color could indicate a material like clay, plastic, or even a painted surface.
   - **Flames**: The presence of flames suggests that the material could be designed to mimic fire or glow-in-the-dark properties.
   - **Sturdy Legs**: The legs appear solid and robust, which might suggest a material like metal or hard plastic.

4. **Contextual Clues**: Given the fantastical nature of the creature (a flaming camel), it is likely a representation rather than a real-life object. This implies that the material could be anything used for artistic purposes, such as clay, plastic, or even digital rendering.

5. **Lighting and Shadows**: The way the light interacts with the creature suggests a smooth, possibly glossy surface, which could be indicative of materials like plastic or polished stone.

Considering these points, the most plausible material would be a combination of clay or plastic for the body, with additional elements like paint or lighting effects to create the flame effect. The sturdy legs could be made of a different material, perhaps metal or hard plastic, to provide stability.

In conclusion, while the exact material cannot be definitively determined without further information, the creature seems to be crafted from a combination of clay/plastic for the body and possibly metal or hard plastic for the legs, with added visual effects to simulate flames.


\end{tcolorbox}
\end{tabular}
\caption{An example of ViFT with or without ability-fused inference (part 3).}
\label{tab:case2}
\end{table*}



\begin{table*}[htbp]
    \centering
    \renewcommand{\arraystretch}{1.2} 
    \setlength{\tabcolsep}{8pt} 
    \begin{tabular}{p{14cm}}
        \centering
        \includegraphics[width=0.5\textwidth]{cases/cases_4.jpg}
        
        \begin{tcolorbox}[colback=gray!5, colframe=gray!40, sharp corners, width=\linewidth, boxrule=0.5mm]
        \small
        \textbf{Problem: }
        What is this function most likely be? Think first, then answer.

        \textbf{Ground Truth:} B. a trigonometric function
        \end{tcolorbox}  % Reduce the vertical space between input and output
        \begin{tcolorbox}[colback=gray!5, colframe=gray!40, sharp corners, width=\linewidth, boxrule=0.5mm]
            \small
            \textbf{$\alpha=0.8$:}
            \\
However, without the actual graph or additional context, I cannot provide a specific identification of the function. The options given (polynomial, trigonometric, exponential, or logarithmic) suggest different types of functions that could fit the description. If you have more details about the graph or the function itself, please provide them so I can give a more accurate identification.


        \end{tcolorbox}
        

        \begin{tcolorbox}[colback=gray!5, colframe=gray!40, sharp corners, width=\linewidth, boxrule=0.5mm]
        \small
        \textbf{$\alpha=0.9$:}
        \\
To determine which type of function the given graph represents, let's analyze its characteristics:

1. **Shape**: The graph shows a smooth, periodic wave-like pattern with peaks and troughs that repeat at regular intervals.

2. **Amplitude**: The peaks and troughs appear to reach up to 1 unit above and below the x-axis.

3. **Periodicity**: The pattern repeats every 5 units along the x-axis.

Given these observations, we can infer the following:

- **Polynomial Functions**: These typically have a more complex shape with multiple turning points (peaks and troughs), but they do not exhibit the repetitive nature seen in the graph.

- **Trigonometric Functions**: These functions, such as sine or cosine, are known for their periodic behavior, which matches the observed pattern in the graph.

- **Exponential Functions**: These functions grow or decay exponentially and do not form a repeating pattern like the one shown.

- **Logarithmic Functions**: These functions also do not exhibit the periodicity seen in the graph.

Based on the analysis, the most likely function represented by the graph is a trigonometric function.

Therefore, the correct answer is (B) a trigonometric function.

\end{tcolorbox}

        \begin{tcolorbox}[colback=gray!5, colframe=gray!40, sharp corners, width=\linewidth, boxrule=0.5mm]
        \small
        \textbf{$\alpha=1.0$:}
        \\
To determine what type of function the given graph represents, let's analyze its characteristics:

1. **Shape**: The graph shows a smooth, periodic wave-like pattern with peaks and troughs that repeat at regular intervals.

2. **Amplitude**: The peaks and troughs appear to reach approximately ±1 on the y-axis.

3. **Periodicity**: The pattern repeats every 5 units along the x-axis, indicating a period of 5.

Given these observations, we can infer that the function is likely a sine or cosine function, which are fundamental trigonometric functions known for their wave-like behavior.

- **Polynomial Functions**: These typically have a smooth curve but do not exhibit the repetitive nature seen in the graph.

- **Exponential Functions**: These usually show rapid growth or decay and do not form a repeating pattern.

- **Logarithmic Functions**: These also do not exhibit the periodicity observed in the graph.

Therefore, based on the shape and periodic nature of the graph, the most likely function is a trigonometric function.

Answer: (B) a trigonometric function


\end{tcolorbox}

\end{tabular}
\caption{An example of ViFT with different $\alpha$ for ability fusion.}
\label{tab:case4}
\end{table*}

\begin{table*}[htbp]
    \ContinuedFloat
    \begin{tabular}{p{14cm}}
        \centering

                \begin{tcolorbox}[colback=gray!5, colframe=gray!40, sharp corners, width=\linewidth, boxrule=0.5mm]
        \small
        \textbf{$\alpha=1.1$:}
        \\
The given graph depicts a periodic oscillation with a sinusoidal pattern. The function appears to be a sine or cosine wave, which are fundamental trigonometric functions. These functions are characterized by their smooth, repetitive curves that rise and fall in a regular manner.

To determine the specific type of trigonometric function, we can analyze the shape and properties of the curve:

1. **Periodicity**: The graph shows a repeating pattern every 5 units along the x-axis. This indicates that the period of the function is 5 units.

2. **Amplitude**: The maximum value of the function is 1, and it reaches this peak twice within each period (once at the top and once at the bottom).

3. **Shape**: The curve starts at zero, rises to its maximum positive value, then decreases back to zero, continues to its minimum negative value, and finally returns to zero before starting another cycle.

Given these characteristics, the function is best described as a trigonometric function, specifically a sine or cosine function. Since the graph starts at the origin and reaches its first peak at \( x = 5 \), it suggests that the function could be a sine function shifted horizontally.

Therefore, the most likely function represented in the graph is a trigonometric function.

Answer: (B)


\end{tcolorbox}
        \begin{tcolorbox}[colback=gray!5, colframe=gray!40, sharp corners, width=\linewidth, boxrule=0.5mm]
        \small
        \textbf{$\alpha=1.2$:}
        \\
The given graph depicts a periodic oscillation with a sinusoidal pattern. The function appears to be a sine or cosine wave, which are fundamental trigonometric functions. These functions are characterized by their smooth, repetitive curves that rise and fall in a regular manner.

To determine the specific type of trigonometric function, we can analyze the shape and properties of the curve:

1. **Periodicity**: The graph shows a repeating pattern every 5 units along the x-axis. This indicates that the period of the function is 5 units.

2. **Amplitude**: The maximum value of the function is 1, and it reaches this peak twice within each period (once at the top and once at the bottom).

3. **Shape**: The curve starts at zero, rises to its maximum positive value, then decreases back to zero, continues to its minimum negative value, and finally returns to zero before starting another cycle.

Given these characteristics, the function is best described as a trigonometric function, specifically a sine or cosine function. Since the graph starts at the origin and reaches its first peak at \( x = 5 \), it suggests that the function could be a sine function shifted horizontally.

Therefore, the most likely function represented in the graph is a trigonometric function.

Answer: (B)


\end{tcolorbox}
\end{tabular}
\caption{An example of ViFT with different $\alpha$ for ability fusion.}
\end{table*}


\begin{table*}[htbp]
    \centering
    \renewcommand{\arraystretch}{1.2} 
    \setlength{\tabcolsep}{8pt} 
    \begin{tabular}{p{14cm}}
        \centering
        \includegraphics[width=0.6\textwidth]{cases/cases_3.jpg}
        
        \begin{tcolorbox}[colback=gray!5, colframe=gray!40, sharp corners, width=\linewidth, boxrule=0.5mm]
        \small
        \textbf{Problem: }
        Move the ruler to measure the length of the line to the nearest centimeter. How long is the line? Think first, then answer.

        \textbf{Ground Truth:} 7
        \end{tcolorbox}  % Reduce the vertical space between input and output
        \begin{tcolorbox}[colback=gray!5, colframe=gray!40, sharp corners, width=\linewidth, boxrule=0.5mm]
            \small
            \textbf{$\beta=0$:}
            \\
The image shows a ruler with measurements in centimeters, ranging from 0 to 10 cm. There is a black line above the ruler that spans from approximately 2 cm to just past the 6 cm mark. Therefore, the length of the line is about 5 centimeters.


        \end{tcolorbox}
        

        \begin{tcolorbox}[colback=gray!5, colframe=gray!40, sharp corners, width=\linewidth, boxrule=0.5mm]
        \small
        \textbf{$\beta=0.1$:}
        \\
To determine the length of the line using the ruler provided, we need to follow these steps:

1. **Identify the Scale**: The ruler has markings from 0 cm to 10 cm.

2. **Align the Line with the Ruler**: Place one end of the line at the 0 cm mark on the ruler.

3. **Measure the Length**: Extend the line along the ruler until it reaches the other end.

From the image, the line starts at the 0 cm mark and extends to just past the 7 cm mark but not reaching the 8 cm mark. Therefore, the line measures approximately 7 centimeters.

**Answer:** The line is about 7 centimeters long.

\end{tcolorbox}

        \begin{tcolorbox}[colback=gray!5, colframe=gray!40, sharp corners, width=\linewidth, boxrule=0.5mm]
        \small
        \textbf{$\beta=0.2$:}
        \\
To determine the length of the line using the ruler provided, we need to follow these steps:

1. **Identify the Scale**: The ruler has markings from 0 cm to 10 cm.

2. **Align the Line**: Place the ruler such that one end of the line aligns with the 0 cm mark on the ruler.

3. **Measure the Length**: Extend the ruler along the line until it reaches the other end of the line.

Given the hint to provide an integer answer, let's assume the line extends to just past the 7 cm mark but not reaching 8 cm. Therefore, the length of the line would be approximately 7 centimeters.

**Answer:** 7


\end{tcolorbox}

        \begin{tcolorbox}[colback=gray!5, colframe=gray!40, sharp corners, width=\linewidth, boxrule=0.5mm]
        \small
        \textbf{$\beta=0.3$:}
        \\
To determine the length of the line using the ruler provided, we need to follow these steps:

1. **Identify the Scale**: The ruler has markings from 0 cm to 10 cm.

2. **Align the Line**: Place the ruler such that one end of the line aligns with the 0 cm mark on the ruler.

3. **Measure the Length**: Extend the ruler along the line until it reaches the other end of the line.

Given the hint to provide an integer answer, let's assume the line extends to just past the 7 cm mark but not reaching 8 cm. Therefore, the length of the line would be approximately 7 centimeters.

**Answer:** 7


\end{tcolorbox}

        \begin{tcolorbox}[colback=gray!5, colframe=gray!40, sharp corners, width=\linewidth, boxrule=0.5mm]
        \small
        \textbf{$\beta=0.4$:}
        \\
!!!!!!!!!!!!!!!!!!!!!!!!!!!!!!!!!!!!!!!!!!!!!!!!!!!!!!!!!!!!!!!!!!!!!!!!!!!!!!!!!!!!!!!!!!!!!!!!!!!!!!!!!!!!!!!!!!!!!!!!


\end{tcolorbox}
\end{tabular}
\caption{An example of ViFT with different $\beta$ for ability fusion.}
\label{tab:case3}
\end{table*}











