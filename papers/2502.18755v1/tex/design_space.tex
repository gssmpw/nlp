\section{\proj Quantization Framework}
\label{sec:dse}


This section introduces the \proj quantization framework, focusing on the quantization methods for weight, activation, and KV cache.
Initially, we discuss the weight quantization with \mbox{\proj} in \mbox{\Sec{sec:weight_quant}}.
Subsequently, we illustrate why activation is quantized with \texttt{INT8} in \Sec{sec:dse_act}.
% Finally, we explore the challenges of group-wise quantization in the KV cache and present our solutions in \Sec{sec:dse_kv}.
Finally, we introduce a real-time \proj quantization mechanism to tackle the challenge brought by the dynamic KV cache and customize quantization strategies for K and V cache, respectively, in \Sec{sec:dse_kv}.

% We first detail the quantization process for key LLM components, including activation, weight, and KV cache.

\subsection{Weight Quantization}
\label{sec:weight_quant}

Weight is encoded offline to select the most suitable coefficient $a$ for each group.
Within the \proj framework, a calibration dataset~\cite{gao2020pile} is used to identify the coefficient $a$ that minimizes the mean square error (MSE) of the output.
The following equation describes the optimization objective:
\begin{equation}	
    a = \mathop{argmin}\limits_{a} ||X \hat{W}_{a} - X W||^2_{2}
    \label{eqn:min_mse}
\end{equation}
$\hat{W}_{a}$ represents the weight that is first quantized and subsequently dequantized using a specific coefficient $a$.
% We represent the original activations and weights as $X$ and $W$, respectively.
Broadening the range of data types in the search space typically enhances accuracy.
Nevertheless, slight modifications to $a$, increasing or decreasing it by one, only slightly alters the data distribution.
% Extending the variety of data types available during the search phase can increase accuracy, but the improvements are marginal when a sufficient variety is already in use.
Consequently, we selected 16 data types for weight quantization, including the set $\{0, 5, 10, 17, 20, 30, 40, 50, 60, 70, 80, 90, 100, 110, 120\}$ and an additional \texttt{INT} option.


\subsection{Activation Quantization}
\label{sec:dse_act}

For the activation quantization, we follow the common practice that quantizes them using the group-wise \texttt{INT8} for near-lossless accuracy and efficient hardware implementation~\cite{frantar2023gptq,lin2023awq, kim2023squeezellm}.
Meanwhile, unlike weight and KV cache, activations are temporary variables.
They are dynamically generated, quantized, and consumed, so they cannot be reused in the latter inference iteration.
Finally, the weights and KV caches already consume most of memory in the LLM inference so that activations occupy only a minor fraction of the memory ($<$5\%~\cite{yuan2024llm}).
As such, further reducing the activation bit width only leads to marginal improvement.

 

This choice of using \texttt{INT8} to activation quantization also allows the computation between the weight that quantized using our \proj{} and the activation that quantized using \texttt{INT8} to exploit the low-bit computation units, as detailed in \mbox{\Sec{sec:encode_decode}}.
\proj further leverages a streaming comparator unit to determine the maximum value of activations, facilitating the activation quantization.

Real-time quantizing activation requires two operations: derive the maximum value of a group to calculate scaling factor and map the \texttt{FP16} value to \texttt{INT8}.
\proj hides the latency of searching the maximum value with hardware design fusing this operation into the streaming of systolic array output, as detailed in \Sec{sec:real_time_engine}.



\subsection{KV Cache Quantization}
\label{sec:dse_kv}


\paragraph{The Quantization Dimension.}
We quantize K and V along the accumulation dimension, as we discuss in \Fig{fig:kv_process}.
Some prior algorithmic works~\cite{liu2024kivi,hooper2024kvquant} do use a different quantization direction for V cache, which we believe is orthogonal to our work. 
First, MANT can also work with the channel direction of V cache quantization, which may sacrifice the computation efficiency of all quantization methods, including ANT. 
Second, those prior works are based on channel-level quantization or extremely low bit, while the impact of directions for the 4-bit group-level quantization is much smaller, as we show later. 
Finally, emerging incoherent processing algorithms~\cite{ashkboos2024quarot,tseng2024quipbetterllmquantization,xiao2023smoothquant} (where SmoothQuant~\cite{xiao2023smoothquant} is a special case) are very promising to further mitigate this gap.

\paragraph{Select \proj through Variance.}
In KV cache quantization, real-time data type selection is required.
Although the searching method based on MSE achieves less accuracy loss, it requires performing quantization to each data type for MSE searching, which is intolerable in a real-time scenario.
Thus, we develop a mapping mechanism based on the data characteristics like variance, which can be derived in a streaming way.

Since variance can reflect the distribution of a data group and \proj with different $a$ has a different variance, \proj determines the coefficient $a$ using a variance.
% For a group of data, we first normalize them to the range [-1, 1] based on their maximum value.
Then, calculate the variance:
\begin{equation}
    \sigma^2 = \frac{1}{n} \sum_{i=1}^{n} x_i^2 - \left( \frac{1}{n} \sum_{i=1}^{n} x_i \right)^2
    \label{eqn:variance}
\end{equation}
% Variance can be calculated using $\sigma^2 = \frac{1}{n} \sum_{i=1}^{n} x_i^2 - \left( \frac{1}{n} \sum_{i=1}^{n} x_i \right)^2$,  where $\sigma^2$ is the variance and $x_i$ is a parameter in the group.

In \proj, a larger $a$ corresponds to higher variance, with each value of $a$ associated with a specific variance range.
We first sample the K and V tensors through a calibration dataset~\cite{gao2020pile}, and select $a$ for each group to minimize quantization error.
Next, we calculate the variance of the groups with different $a$ to decide the appropriate range.
The elements within each group are first normalized, ensuring that the absolute maximum value in the group is scaled to 1. 
Following this normalization step, the variance of the normalized elements is then computed.
For example, when $a=35$, the variance is 0.104; when $a=45$, the variance is 0.118.
We define the range for $a=40$ as [0.104, 0.118].
If the variance of normalized data falls within a specific range of $a$, it will be quantized with that $a$ with \proj.
We fuse the computation flow of variance with matrix multiplication to hide latency, which we will detail in \Sec{sec:real_time_engine}.


\paragraph{Prefill Stage.}
During the prefill stage, the input comprises a sequence, so the K and V are both matrices, where the sequence length typically exceeds the group size.
Thus, both the K cache and V cache can obtain the data needed to calculate variance.
% The K cache and V cache in a group will be normalized first, and their variance is calculated.
By selecting the appropriate $a$ based on the variance, the K cache and V cache can be quantized to 4-bit \proj.

\paragraph{Decode Stage.}
In the decode stage, since inputs are vectors, generated K and V are vectors.
Thus, the K cache can obtain all data of the group in a single iteration to execute real-time quantization, similar to activation in decode stage.
The difference is that each group needs to calculate the partial sum of $x_i$ and $x_i^2$ as well as the maximum value since K cache needs to be quantized to 4-bit \proj.
However, when it comes to V cache, a new challenge arises as each iteration only generates one element of a group.

To address this, we propose a two-phase quantization scheme for the V cache, as shown in \Fig{fig:v_update}.
We define every $G$ iterations in the decode phase as a process window for V cache, where $G$ is the group size. 
In the first phase, newly generated V vector is quantized to \texttt{INT8} with channel-wise scaling factors derived from prefill stage, denoted as `scales' in \Fig{fig:v_update}. 
Meanwhile, we update the maximum value and partial sum of $v_i$ and $v_i^2$, denoting the parameter in a group as $v_i$.
This operation persists until the process window is full. 


The second phase involves quantizing the 8-bit V cache to 4-bit \proj.
When the process window is full, we calculate the variance with partial sum of $v_i$ and $v_i^2$ by ~\eqref{eqn:variance}.
Then, the coefficient $a$ is decided similarly to the prefill phase, and the stacked \texttt{INT8} V cache is quantized to 4-bit \proj.

This two-phase quantization scheme effectively quantizes all but the latest V vectors in the processing window to 4 bits.
The process window is similar to the residual group used in KIVI~\cite{liu2024kivi}.
The overhead of \texttt{INT8} operation of V cache in processing window is marginal and tolerable.
Besides, it helps improve the quality of newly generated tokens, since some studies have shown that the latest tokens are more important~\cite{duanmu2024skvqslidingwindowkeyvalue}.


\begin{figure}[t] 
    \centering 
    \includegraphics[width=0.9\linewidth]{fig/dse_v_update.pdf}  
    \caption{The update process of V cache with a group size of 64. Each iteration generates a new $V$ vector, quantizes it to INT8, and accumulates the parameters for \proj quantization. In the 64th iteration, the V cache is quantized to 4-bit \proj.}
    % \congsay{The V group is wrong, should be vertical. Also labeled in it.64 }} 
    \label{fig:v_update}
    % \vspace*{-0.4cm}
\end{figure}