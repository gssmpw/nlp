
%\vspace{-0.2cm}

\section{Conclusion}
\label{sec:conclusion}
%\vspace{-0.1cm}

In our paper we extended the theoretical framework of GFlowNets to encompass non-acyclic discrete environments, revisiting and simplifying the previous constructions by \cite{brunswic2024theory}. In addition, we provided a number of theoretical insights regarding backward policies and the nature of flows in non-acyclic GFlowNets, generalized known connections between GFlowNets training and entropy-regularized RL to this setting, and experimentally re-examined the importance of the concept of loss stability proposed in \cite{brunswic2024theory}.

%We believe that exploring different algorithms to approach the constraint optimization problem~\eqref{eq:opt_cyclic_gflow} for the purpose of improving the practical efficiency of non-acyclic GFlowNet training is a crucial direction for future research. 

Future work could explore applying other losses from acyclic GFlowNet literature~\cite{madan2023learning, da2024divergence, hu2024beyond} to the non-acyclic setting. Based on Theorem~\ref{th:rl_reduction}, another promising direction is to apply RL techniques and algorithms to GFlowNets in the non-acyclic case, following their success for acyclic GFlowNets~\cite{tiapkin2024generative, mohammadpour2024maximum, lau2024qgfn, morozov2024improving}. Finally, environments where all states are terminal, i.e., have a transition into $s_f$, naturally arise in the non-acyclic case. Then, special modifications can be introduced to improve the propagation of a reward signal during training~\cite{deleu2022bayesian, pan2023better}.

