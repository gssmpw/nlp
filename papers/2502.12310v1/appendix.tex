\appendix

\section{Notation}
The Euclidean norm of a vector $x$ is denoted $\norm{x}$. For a matrix $A$, the spectral norm is denoted $\norm{A}$, and the Frobenius norm is denoted $\norm{A}_F$. 
A symmetric, positive semi-definite matrix $A = A^\top$ is denoted $A \succeq 0$.  $A \succeq B$ denotes that $A-B$ is positive semi-definite. Similarly, a symmetric, positive definite matrix $A$ is denoted $A \succ 0$. 
The minimum eigenvalue of a symmetric, positive semi-definite matrix $A$ is denoted $\lambda_{\min}(A)$. For a positive definite matrix $A$, we define the $A$-norm as $\norm{x}_A^2 = x^\top A x$. 
The gradient of a scalar valued function $f: \R^n \to \R$ is denoted $\nabla f$, and the Hessian is denoted $\nabla^2 f$. 
The Jacobian of a vector-valued function $g: \R^n \to \R^m$ is denoted $D g$, and follows the convention for any $x\in\R^n$, the rows of $D g(x)$ are the transposed gradients of $g_i(x)$.
The $p^{th}$ order derivative of $g$  is  denoted by $D^{p} g$. Note that for $p \geq 2$, $D^{p} g(x)$ is a tensor for any $x\in\R^{n}$. 
The operator norm of such a tensor is denoted by $\norm{D^p g(x)}_{\mathsf{op}}$. 
For a function $f: \mathsf{X} \to \R^{\dy}$, we define $\norm{f}_{\infty} \triangleq \sup_{x \in \mathsf{X}} \norm{f(x)}$. 
A Euclidean norm ball of radius $r$ centered at $x$ is denoted $\calB(x,r)$. The state covariance matrix of the system under controller K is denoted as $\Sigma^K(\theta) \triangleq \dlyap((A(\theta)+B(\theta)K)^T, I )$. 
The solution to the discrete algebraic Ricatti equation satisfies $P \succeq I$ as long as $Q\succeq I$. 
The Hessian of the objective function is calculated as 
\begin{align*}
 H(\theta) &\triangleq \nabla^2_\theta C(K(\theta), \theta) = \mathsf{D}_{\theta}\VEC K(\theta)^T(\Psi(\theta)\kron\Sigma_X^{K(\theta)}(\theta))\mathsf{D}_{\theta}\VEC K(\theta)
\end{align*}
where $\Psi(\theta) \triangleq B(\theta)^TP(\theta)B(\theta) + R$.


\section{Perturbation Analysis}

% \Tesshu{Check $\Sigma_w$}




Here we present a number of perturbation results that we re-use throughout our analysis.

% Expectation with respect to all the randomness of the underlying probability space is denoted by $\E$.


\begin{lemma}[Performance Difference Lemma, Lemma 12 of \citet{fazel2018global}]
    \label{lem: performance difference}
    Let $\theta$ denote the parameter for a dynamical system, and $K$ be an arbitrary gain that stabilizes this system. Then it holds that
    \begin{align*}
        C(K, \theta) - C(K(\theta), \theta) = \trace\paren{(K-K(\theta) \Sigma^K(\theta) (K-K(\theta))^\top \Psi(\theta)}, 
    \end{align*}
    where we recall that $\Sigma^K(\theta)$ is the state covariance of the system under controller $K$. 
\end{lemma}

\begin{lemma}[Lyapunov Perturbation]
    \label{lem: lyap perturbation}
    Let $A_1, A_2 \in \R^{d \times d}$ satisfy  $\rho(A_1) < 1$ and $\rho(A_2) < 1$. Let $Q$ be a $d$ dimensional positive definite matrix. Define $P_1 = \dlyap(A_1, Q)$, and $P_2 = \dlyap(A_2, Q)$. Then it holds that 
    \begin{align*}
        \norm{P_1 - P_2} \leq \frac{1}{\lambda_{\min}(Q)}\norm{P_1} \norm{P_2}\paren{2\norm{A_2} \norm{A_1-A_2}+\norm{A_1 - A_2}^2}. 
    \end{align*}
\end{lemma}
\begin{proof}
    By definition of $P_1$ and $P_2$, it holds that 
    \begin{align*}
        P_1 - P_2 &= A_1^\top P_1 A_1 - A_2^\top P_2 A_2 \\
        &= (A_2 + (A_1-A_2))^\top P_1 (A_2 + (A_1-A_2)) - A_2^\top P_2 A_2 \\
        &= \dlyap(A_2, A_2^\top P_1 (A_1-A_2) +(A_1-A_2)^\top P_1 A_2 +  (A_1-A_2)^\top P_1(A_1-A_2)) \\
        &\preceq \dlyap(A_2, I) \norm{A_2^\top P_1 (A_1-A_2) +(A_1-A_2)^\top P_1 A_2 +  (A_1-A_2)^\top P_1(A_1-A_2)}.
    \end{align*}
    Note that $\dlyap(A_2, I)\preceq \frac{1}{\lambda_{\min}(Q)} P_2$. The result then follows by the triangle inequality and submultiplicativity.
\end{proof}

\begin{lemma}
    \label{lem: cov lower bound}
    Fix a system $\theta$ and two stabilizing controllers $K_1$ and $K_2$. As long as $\norm{B(\theta)(K_1 - K_2)} \leq \frac{1}{12 \norm{\Sigma^{K_1}(\theta)}^{5/2}}$, it holds that 
    \begin{align*}
        \Sigma^{K_2}(\theta) \succeq \frac{1}{2} \Sigma^{K_1}(\theta).
    \end{align*}
\end{lemma}
\begin{proof}
    The result follows by applying the reverse triangle inequality along with \Cref{lem: lyap perturbation}.
\end{proof}

\begin{lemma}[Riccati Perturbation, Proposition 4 and 6 of \citet{simchowitz2020naive}]
    \label{lem: Riccati perturbation}
    Let $\theta_1$ denote the parameter for a stabilizable system, and $\theta_2$ denote the parameter for an alternate system. Suppose that $\norm{\theta_1-\theta_2} \leq \frac{1}{16}\norm{P(\theta_1)}^{-2}$. Then the system described by $\theta_2$ is stabilizable, and the following inequalities hold.
    \begin{itemize}
        \item $\norm{P(\theta_2)}\leq \sqrt{2} \norm{P(\theta_1)}$
        \item $\max\curly{\norm{K(\theta_2) - K(\theta_1)}, \norm{B(\theta_1) (K(\theta_2) - K(\theta_1))}
        }\leq  32\norm{P(\theta_1)}^{7/2} \norm{\theta_1-\theta_2}.$
        \item $\norm{P(\theta_2) - P(\theta_1)} \leq 8\sqrt{2}\norm{P(\theta_1)}^3\norm{\theta_1-\theta_2}$
    \end{itemize}
\end{lemma}

\begin{lemma}[Simplifying inequalities]
    \label{lem: simplifying inequalities}
    Let $\theta$ be a parameter describing a stabilizable system instance. 
    Define $\tau_{B(\theta)} = \max\curly{1, \norm{B(\theta)}}$.
    The following inequalities hold 
    \begin{itemize}
        \item $\norm{\Sigma^{K(\theta)}(\theta)} \leq \norm{P(\theta)}$
        \item $\norm{A(\theta) + B(\theta) K(\theta)} \leq \norm{P(\theta)}^{1/2}$. 
        \item $\norm{K(\theta)}\leq\norm{P(\theta)}^{1/2}$
        \item $\norm{\Psi(\theta)} \leq 2 \tau_{B(\theta)}^2 \norm{P(\theta)}$.
        \item $\norm{\Psi(\theta)\kron\Sigma_X^{K(\theta)}(\theta)} \leq 2\tau^2_{B(\theta)}\norm{P(\theta)}^2$. 
    \end{itemize}
\end{lemma}
\begin{proof}
    The first inequality follows by observing that 
    \begin{align*}
        \norm{\Sigma^{K(\theta)}(\theta)} &= \norm{\dlyap((A+BK(\theta))^\top, I)} 
        \leq \norm{\dlyap((A+BK(\theta)), Q + K(\theta)^\top R K(\theta))},
    \end{align*}
    by the fact that $Q \succeq I$. The second inequality follows by noting that 
    \begin{align*}
        \norm{(A(\theta) + B(\theta) K(\theta))^\top (A(\theta) + B(\theta) K(\theta))}^{1/2} \leq \norm{\dlyap(A(\theta)+B(\theta)K(\theta), I)}^{1/2}.    
    \end{align*}
    The third inequality follows from
    \begin{align*}
        \norm{K(\theta)} \leq \norm{Q+K(\theta)^TRK(\theta)}^{1/2} \leq \norm{\dlyap(A(\theta)+B(\theta)K(\theta), Q+K(\theta)^TRK(\theta))}^{1/2}.
    \end{align*}
    The fourth and fifth inequality follow from
    \begin{align*}
        \norm{\Psi(\theta)\kron\Sigma_X^{K(\theta)}(\theta)} &\leq \norm{\Psi(\theta)}\norm{\Sigma_X^{K(\theta)}(\theta)} \leq (\norm{B(\theta)}^2+1)\|P(\theta)\|^2\leq 2\tau^2_{B(\theta)}\norm{P(\theta)}^2. 
    \end{align*}
    where we used $\norm{X\kron Y}\leq\norm{X}\norm{Y}$.
\end{proof}

\begin{lemma}[Certainty Equivalent Stabilization]
    \label{lem: CE stabilization}
    Let $\theta_1$ denote the parameter for a stabilizable system, and $\theta_2$ denote the parameter for an alternate system. Suppose that $\norm{\theta_1-\theta_2} \leq \frac{1}{256} \norm{P(\theta_1)}^{-5}$. Then the system described by $\theta_2$ is stabilizable
    \begin{align*}
        \norm{\Sigma^{K(\theta_2)}(\theta_1)} \leq 2 \norm{P(\theta_1)}. 
    \end{align*}
\end{lemma}
\begin{proof}
    To verify this fact, first apply
 \Cref{lem: lyap perturbation} and \Cref{lem: simplifying inequalities} to find
 \begin{align*}
    & \norm{\Sigma^{K(\theta_2)}(\theta_1)} \leq  \norm{\Sigma^{K(\theta_1)}(\theta_1)} + \norm{\Sigma^{K(\theta_1)}(\theta_1)}\norm{\Sigma^{K(\theta_2)}(\theta_1)} \\
    & \paren{2\norm{A(\theta_1) + B(\theta_1) K(\theta_1)} \norm{B(\theta_1) (K(\theta_2)-K(\theta_1)} + \norm{B(\theta_1) (K(\theta_2)-K(\theta_1)}^2  } \\
    &\leq \norm{P(\theta_1)} + \norm{P(\theta_1)} \norm{\Sigma^{K(\theta_2)}(\theta_1)} \paren{2 \norm{P(\theta_1)}^{1/2} \norm{B(\theta_1)(K(\theta_2)  - K(\theta_1))} +\norm{B(\theta_1)(K(\theta_2)  - K(\theta_1))}^2}.
 \end{align*}
 By \Cref{lem: Riccati perturbation} it holds that $\norm{B(\theta_1) (K(\theta_2) - K(\theta_1))} \leq  32 \norm{P(\theta_1)}^{7/2} \norm{\theta_2 - \theta_1}$. Leveraging that $\norm{\theta_2 - \theta_1}\leq\frac{1}{256} \norm{P(\theta_1)}^{-5}$, we conclude 
 \begin{align*}
     \norm{\Sigma^{K(\theta_2)}(\theta_1)} \leq \norm{P(\theta_1)} + \frac{1}{2}\norm{\Sigma^{K(\theta_2)}(\theta_1)}.
 \end{align*}
 Rearranging provides the desired inequality. 
 \end{proof}



\begin{lemma}[Bound on the first and second derivative of K]
    \label{lem: Bound on K' and K''}
    Let $\theta_1, \theta_2$ be the parameters describing two stabilizable systems satisfying $\norm{\theta_1-\theta_2}\leq \frac{1}{16} \norm{P(\theta_1)}^{-2}$. Then for any $t\in[0,1]$, the first and second derivative is bounded as 
    \begin{align*}
        \|\mathsf{D}\VEC K(\tilde\theta)\|_{\mathsf{op}}\leq 24\|P\paren{\theta_1}\|^{7/2}, ~ 
        \|\mathsf{D}^2\VEC K(\tilde\theta)\|_{\mathsf{op}}\leq 4000\|P\paren{\theta_1}\|^{15/2},
    \end{align*}
    where $\tilde \theta = t\theta_1 + (1-t)\theta_2$.
\end{lemma}
\begin{proof}
    It suffices to consider the quantity $K(t)$ defined as
    \begin{align*}
        &K(t) \triangleq K(t\theta_1 + (1-t)\theta_2). 
    \end{align*}
    Additionally define
    \begin{align*}
        P(t) \triangleq P(t\theta_1 + (1-t)\theta_2).
    \end{align*}
    By the proof of Lemma 3.2, C.5 of \citet{simchowitz2020naive} and keeping track of the degree of the term $\norm{P(\theta_1)}$, we get
    \begin{align*}
        \|P'(t)\|&\leq 4\|P(t)\|^3 , 
        ~ \|P''(t)\|\leq178\|P(t)\|^7 \\
        \|K'(t)\|&\leq7\|P(t)\|^{7/2} , ~\|K''(t)\|\leq290\|P(t)\|^{15/2}.
    \end{align*}
    From \Cref{lem: Riccati perturbation}, it holds that $\norm{P(t)} \leq \sqrt{2}\norm{P(\theta_1)}$. Therefore, by noting that the derivatives of $K(t)$ are directional derivatives of $K(\theta)$, it follows that
    \begin{align*}
        &\|\mathsf{D}\VEC K(\tilde\theta)\|_{op}\leq \sup_{t\in[
        0,1]}7\|P(t)\|^{7/2} < 24\|P\paren{\theta_1}\|^{7/2} \\
        &\|\mathsf{D}^2\VEC K(\tilde\theta)\|_{op}\leq \sup_{t\in[
        0,1]}290\|P(t)\|^{15/2} <
        4000\|P\paren{\theta_1}\|^{15/2}.
    \end{align*}
\end{proof}

Leveraging the above result along with a Taylor expansion leads to the following lemma. 
\begin{lemma} [LQR Taylor Expansion]
    \label{lem: LQR Taylor expansion}
    Let $\theta_1, \theta_2$ be the parameters describing two stabilizable systems satisfying $\norm{\theta_1-\theta_2}\leq \frac{1}{16} \norm{P(\theta_1)}^{-2}$. It holds that
    \begin{align*}
        \VEC (K(\theta_2) - K(\theta_1)) &= \mathsf{D}_{\theta}\VEC K(\theta_1)[\theta_2 - \theta_1] + R,
    \end{align*}
    where $R$ is the remainder term that satisfies
    \begin{align*}
        \norm{R} &\leq \frac{1}{2}\sup_{\tilde\theta\in[\theta_1, \theta_2]}\|\mathsf{D}^2\VEC K(\tilde\theta)\|_{op}\|\theta_2-\theta_1\|^2 \leq 2000\|P_{\theta_1}\|^{15/2}\|\theta_2-\theta_1\|^2.
    \end{align*}
\end{lemma}

\begin{lemma}[Suboptimality Gap Bound]
    \label{lem: excess cost decomposition}
    Let $K$ be any controller that stabilizes $\theta$. Then it holds that
    \begin{align*}
        &C(K,\theta) - C(K(\theta), \theta) \\
        &\leq \trace ((K-K(\theta))\Sigma_X^{K(\theta)}(\theta)(K-K(\theta))^T\Psi(\theta))\\
        &+ 2 \norm{P(\theta)}^2 \|\Sigma_X^{K}(\theta)\|\tau_{B(\theta)}^3\|(K-K(\theta))\|^3(\|B(\theta)(K-K(\theta))\| + 2 \norm{P(\theta)}^{1/2}).
    \end{align*}
\end{lemma}
\begin{proof}
    By \Cref{lem: performance difference}, 
    \begin{align*}
        &C(K,\theta) - C(K(\theta), \theta) \\
        &\leq \trace ((K-K(\theta))\Sigma_X^{K(\theta)}(\theta)(K-K(\theta))^T\Psi(\theta) + (K-K(\theta))(\Sigma_X^K(\theta) - \Sigma_X^{K(\theta)}(\theta))(K-K(\theta))^T\Psi(\theta))
\end{align*}
where $\Sigma_X^K(\theta) = \dlyap((A(\theta)+B(\theta) K)^T, I)$. 
By applying \Cref{lem: lyap perturbation} to the second term, we get
\begin{align*}
    &\trace(K-K(\theta))(\Sigma_X^K(\theta) - \Sigma_X^{K(\theta)}(\theta))(K-K(\theta))^T\Psi(\theta))\\
    &\leq \|\Sigma_X^K(\theta)\|\|\Sigma_X^{K(\theta)}(\theta)\|\|K-K(\theta)\|^3\|\Psi(\theta)\|\|B(\theta)\|(\|B(\theta)(K-K(\theta))\| + 2\|A(\theta)+B(\theta) K(\theta)\|).
\end{align*}
To conclude, we apply the inequalities of \Cref{lem: simplifying inequalities} to $\norm{A(\theta) + B(\theta) K(\theta)}$, $\norm{\Sigma^{K(\theta)}(\theta)}$, and $\norm{\Psi(\theta)}$. 
\end{proof}

\begin{lemma}[Taylor Expansion Substitution for Suboptimality Gap]
    \label{lem: cost gap taylor substitution}
    Let $K(\theta_1), K(\theta_2)$ be any certainty equivalence controller that stabilize $\theta_1, \theta_2$, respectively, where $\theta_1, \theta_2$ satisfy $\norm{\theta_1-\theta_2}\leq \frac{1}{16} \norm{P(\theta_1)}^{-2}$. 
    Then it holds that
    \begin{align*}
        &\trace ((K(\theta_2)-K(\theta_1))\Sigma_X^{K(\theta_1)}(\theta_1)(K(\theta_2)-K(\theta_1))^T\Psi(\theta_1)) \\
        &\leq \|\theta_2-\theta_1\|_{H(\theta_1)}^2 + 2e5\tau_{B(\theta_1)}^2\|P(\theta_1)\|^{13}\|\theta_2-\theta_1\|^3 + 8e6\tau_{B(\theta_1)}^2\|P(\theta_1)\|^{17}\|\|\theta_2-\theta_1\|^4.
    \end{align*}
\end{lemma}
\begin{proof}
    By \Cref{lem: simplifying inequalities}, \Cref{lem: Bound on K' and K''} and \Cref{lem: LQR Taylor expansion}, it follows that
    \begin{align*}
        &\trace ((K(\theta_2)-K(\theta_1))\Sigma_X^{K(\theta_1)}(\theta_1)(K(\theta_2)-K(\theta_1))^T\Psi(\theta_1)) \\
        &= \VEC (K(\theta_2) - K(\theta_1))^T(\Psi(\theta_1)\kron\Sigma_X^{K(\theta_1)}(\theta_1))\VEC(K(\theta_2)-K(\theta_1)) \\
        &\leq [\theta_2 - \theta_1]^T\mathsf{D}_{\theta}\VEC K(\theta_1)^T(\Psi(\theta_1)\kron\Sigma_X^{K(\theta_1)}(\theta_1))\mathsf{D}_{\theta}\VEC K(\theta_1)[\theta_2 -\theta_1] \\
        &\hspace{5mm} + \sym([\theta_2 - \theta_1]^T\mathsf{D}_{\theta}\VEC K(\theta_1)^T(\Psi(\theta_1)\kron\Sigma_X^{K(\theta_1)}(\theta_1))R) + R^T(\Psi(\theta_1)\kron\Sigma_X^{K(\theta_1)}(\theta_1))R \\
        &\leq \|\theta_2-\theta_1\|_{H(\theta_1)}^2 + 2e5\tau_{B(\theta_1)}^2\|P(\theta_1)\|^{13}\|\theta_2-\theta_1\|^3 + 8e6\tau_{B(\theta_1)}^2\|P(\theta_1)\|^{17}\|\|\theta_2-\theta_1\|^4.
    \end{align*} 
    where it follows from $\trace(A^T, B) = \VEC (A)^T\VEC B$ and $\VEC (ABC) = (C^T\kron A)\VEC (B)$ in the first equality, and we let the operator $\sym$ denote $\sym(A) = A+A^T$. 
\end{proof}

\begin{lemma}[Perturbation on $B(\theta), H(\theta)$]
    \label{lem: helper lemma for RC}
    Let $\theta_1, \theta_2$ be the parameters describing two stabilizable systems satisfying $\norm{\theta_1 - \theta_2}\leq\frac{1}{16}\norm{P(\theta_1)}^{-2}$. Then it holds that
    \begin{itemize}
        \item $\norm{B(\theta_2)}\leq \norm{B(\theta_1)} + \norm{\theta_1 - \theta_2}$
        \item $\norm{\Psi(\theta_2) - \Psi(\theta_1)} \leq 15\tau^2_{B(\theta_1)}\norm{P(\theta_1)}^3\norm{\theta_1-\theta_2}$
        \item $\norm{H(\theta_2)-H(\theta_1)}\leq 5e6\tau_{B(\theta_1)}^2\norm{P(\theta_1)}^{17}\norm{\theta_1-\theta_2}$.
        \item $\norm{H(\theta_2)} \leq 8e3\tau^2_{B(\theta_1)}\norm{P(\theta_1)}^9$
    \end{itemize}
\end{lemma}
\begin{proof}
    By the triangle inequality,
    \begin{align*}
        \norm{B(\theta_2)} = \norm{B(\theta_1) + (B(\theta_2) - B(\theta_1))} \leq \norm{B(\theta_1)} + \norm{\theta_1 - \theta_2}.
    \end{align*}
    It follows that
    \begin{align*}
        \tau^2_{B(\theta_2)} &\leq \max\{\norm{B_2}^2, 1\} \leq \max\{\norm{B(\theta_1)}^2+2\norm{B(\theta_1)}\norm{\theta_1-\theta_2} + \norm{\theta_1-\theta_2}, 1\} \leq 2\tau_{B(\theta_1)}^2 \\
        \tau^3_{B(\theta_2)} &\leq \max\{\norm{B_2}^3, 1\} \leq \max\{\norm{B(\theta_1)}^3+3\norm{B(\theta_1)}^2\norm{\theta_1-\theta_2} + 3\norm{B(\theta_1)}\norm{\theta_1-\theta_2}^2
        \norm{\theta_1-\theta_2}^3, 1\} \\
        &\leq 2\tau_{B(\theta_1)}^3
    \end{align*}
    where we applied $\norm{\theta_1-\theta_2} \leq \frac{1}{16}\norm{P(\theta_1)}^{-2}$ in the last inequality.
    For the second inequality, it holds from \Cref{lem: Riccati perturbation} that
    \begin{align*}
        &\norm{\Psi(\theta_2) - \Psi(\theta_1)} \\
        &\leq \norm{\{B(\theta_1) + B(\theta_2)-B(\theta_1)\}^TP(\theta_2)\{B(\theta_1) + B(\theta_2)-B(\theta_1)\} - B(\theta_1)^TP(\theta_1)B(\theta_1)} \\
        &\leq \norm{B(\theta_1)^T\paren{P(\theta_2)-P(\theta_1)}B(\theta_1)} + \norm{\sym\paren{B(\theta_1)^TP(\theta_2)\paren{B(\theta_2)-B(\theta_1)}}} \\
        &\quad+ \norm{\paren{B(\theta_2)-B(\theta_1)}^TP(\theta_2)\paren{B(\theta_2)-B(\theta_1)}} \\
        &\leq 8\sqrt{2}\norm{B(\theta_1)}^2\norm{P(\theta_1)}^3\norm{\theta_1-\theta_2} + 
        2\sqrt{2}\norm{B(\theta_1)}\norm{B(\theta_2)-B(\theta_1)}\norm{P(\theta_1)}  \\
        &\quad + \sqrt{2}\norm{B(\theta_2)-B(\theta_1)}^2\norm{P(\theta_1)} \\
        &\leq 15\tau^2_{B(\theta_1)}\norm{P(\theta_1)}^3\norm{\theta_1-\theta_2}.
    \end{align*}
    where we applied $\norm{\theta_1-\theta_2} \leq \frac{1}{16}\norm{P(\theta_1)}^{-2}$ in the last inequality. 
    Next, let $K'(\theta) \triangleq \mathsf{D}_{\theta}\VEC K(\theta)$. Then from \Cref{lem: Bound on K' and K''}, 
    \begin{align*}
        \norm{K'(\theta_2)-K'(\theta_1)} 
        &= \sup_{\tilde\theta\in[\theta_1, \theta_2]}\norm{K''(\tilde\theta)}\norm{\theta_1-\theta_2}
        \leq 4000\norm{P(\theta_1)}^{15/2}\norm{\theta_1-\theta_2}.
    \end{align*}
    From \Cref{lem: Riccati perturbation} and \Cref{lem: simplifying inequalities}
    \begin{align*}
        &\norm{A(\theta_2)+B(\theta_2)K(\theta_2) - A(\theta_1)+B(\theta_1)K(\theta_1)} \\
        &\leq \norm{A(\theta_2)-A(\theta_1)} + \norm{(B(\theta_2)-B(\theta_1))K(\theta_2)} + \norm{B(\theta_1)(K(\theta_2)-K(\theta_1))} \\
        &\leq \norm{\theta_1-\theta_2} + \norm{\theta_1-\theta_2}\norm{P(\theta_2)}^{1/2} + 32\norm{P(\theta_1)}^{7/2}\norm{\theta_1-\theta_2} \\
        &\leq 35\norm{P(\theta_1)}^{7/2}\norm{\theta_1-\theta_2}. 
    \end{align*}
    For $\Sigma_X^{K(\theta)}(\theta)$, from \Cref{lem: lyap perturbation} and above calculation, 
    \begin{align*}
        &\norm{\Sigma_X^{K(\theta_2)}(\theta_2) - \Sigma_X^{K(\theta_1)}(\theta_1)} \\
        &\leq \norm{\Sigma_X^{K(\theta_1)}(\theta_1)}\norm{\Sigma_X^{K(\theta_2)}(\theta_2)}\paren{70\norm{P(\theta_1)}^{4}\norm{\theta_1-\theta_2} + 35^2\norm{P(\theta_1)}^{7}\norm{\theta_1-\theta_2}^2} \\
        &\leq \sqrt{2}\norm{P(\theta_1)}^2\paren{70\norm{P(\theta_1)}^{4}\norm{\theta_1-\theta_2} + 35^2\norm{P(\theta_1)}^{7}\norm{\theta_1-\theta_2}^2} \\
        &\leq 225\norm{P(\theta_1)}^{7}\norm{\theta_1-\theta_2},
    \end{align*}
    where we applied $\norm{\theta_1-\theta_2} \leq \frac{1}{16}\norm{P(\theta_1)}^{-2}$ in the last inequality.
    Now let $M(\theta)$ denote $M(\theta) \triangleq \Psi(\theta)\kron\Sigma_X^{K(\theta)}(\theta)$. 
    Then we get
    \begin{align*}
        &\norm{M(\theta_2) - M(\theta_1)} \\
        &= \norm{\paren{\Psi(\theta_1) + \Psi(\theta_2) - \Psi(\theta_1)}\kron\paren{\Sigma_X^{K(\theta_1)}(\theta_1) + \Sigma_X^{K(\theta_2)}(\theta_2) - \Sigma_X^{K(\theta_1)}(\theta_1)} - \Psi(\theta_1)\kron\Sigma_X^{K(\theta_1)}(\theta_1)} \\
        &\leq \norm{\Psi(\theta_2) - \Psi(\theta_1)}\norm{\Sigma_X^{K(\theta_1)}(\theta_1)} + \norm{\Psi(\theta_1)}\norm{\Sigma_X^{K(\theta_2)}(\theta_2) - \Sigma_X^{K(\theta_1})(\theta_1)} \\
        &\hspace{5mm}+ \norm{\Psi(\theta_2) - \Psi(\theta_1)}\norm{\Sigma_X^{K(\theta_2)}(\theta_2) - \Sigma_X^{K(\theta_1)}(\theta_1)} \\
        &\leq 4e3\tau_{B(\theta_1)}^2\norm{P(\theta_1)}^{10}\norm{\theta_1-\theta_2}
    \end{align*}
    where we used $\norm{X\kron Y}\leq\norm{X}\norm{Y}$.
    Thus from \Cref{lem: Bound on K' and K''}
    \begin{align*}
        &\norm{H(\theta_2) - H(\theta_1)} \\
        &= \norm{K'(\theta_2)^TM(\theta_2)K'(\theta_2) - K'(\theta_1)^TM(\theta_1)K'(\theta_1)} \\
        &= \norm{\{K'(\theta_1) + K'(\theta_2) - K'(\theta_1)\}^TM(\theta_2)\{K'(\theta_1) + K'(\theta_2) - K'(\theta_1)\} -K'(\theta_1)^TM(\theta_1)K'(\theta_1)} \\
        &= \norm{K'(\theta_1)^T\{M(\theta_2) - M(\theta_1)\}K'(\theta_1)} + \norm{\sym\{K'(\theta_1)^TM(\theta_2)(K'(\theta_2) - K'(\theta_1))\}} \\
        &\hspace{5mm} + \norm{\{K'(\theta_2) - K'(\theta_1)\}^TM(\theta_2)\{K'(\theta_2) - K'(\theta_1)\}} \\
        &\leq \norm{K'(\theta_1)}^2\norm{M(\theta_2)-M(\theta_1)} + 2\norm{K'(\theta_1)}\norm{K'(\theta_2) - K'(\theta_1)}\norm{M(\theta_2)} \\
        &\quad + \norm{K'(\theta_2) - K'(\theta_1)}^2\norm{M(\theta_2)} \\
        &\leq 5e6\tau_{B(\theta_1)}^2\norm{P(\theta_1)}^{17}\norm{\theta_1-\theta_2}
    \end{align*}
    For the last fact, from \Cref{lem: simplifying inequalities} and \Cref{lem: Bound on K' and K''}, 
    \begin{align*}
        \norm{H(\theta_2)} &\leq \norm{K'(\theta_2)}^2\norm{M(\theta_2)} 
        \leq 2e3\norm{P(\theta_1)}^7\cdot4\tau^2_{B(\theta_1)}\norm{P(\theta_1)}^2 = 8e3\tau^2_{B(\theta_1)}\norm{P(\theta_1)}^9
    \end{align*}
\end{proof}



\section{Least Squares Analysis}
\label{s: id bound proof}

In this section, we provide a characterization of the least squares error for the procedure discussed in \Cref{s: methods}. The following result characterizes the weighted parameter identification error of the least squares. %The weighting is es captures how the quality of estimation for particular parameters impacts the control cost achieved by using the estimated parameters for synthesis.

\begin{lemma}[Least Squares Identification Bound]
    \label{thm: identification bound}
    Let $H$ be a positive definite matrix belonging to $\mathbb{R}^{d_{\theta}\times d_{\theta}}$. Suppose the dataset $\curly{(X_t^n, U_t^n, X_{t+1}^n)}_{t=1, n=1}^{T,N}$ is collected from system \eqref{eq: linear system} using random noise $U_t \sim \calN(0, \Sigma_u)$. Let $\hat \theta$ be the least squares estimate \eqref{eq: least squares}, and $\hat {\mathsf{FI}}$ be the Fisher Information estimate \eqref{eq: fisher estimate}. Let $\delta \in (0,1)$. It holds that with probability at least $1-\delta$ that 
    \begin{align}
        \norm{\hat \theta - \theta^\star}_H^2 \leq 4 \frac{\trace\paren{H \mathsf{FI}(\theta^\star)^{-1}}}{N} + 8 \frac{\norm{H \mathsf{FI}(\theta^\star)^{-1}}}{N} \log\frac{2}{\delta},
    \end{align}
    and
    % \begin{align*}
        $0.5 \mathsf{FI}(\theta^\star) \preceq \hat{\mathsf{FI}} \preceq 2 \mathsf{FI}(\theta^\star)$
    % \end{align*}
    as long as the number of experiments satisfies 
    % \begin{align*}
        $N \geq N_{\mathsf{ID}},$
    % \end{align*}
    for $N_{\mathsf{ID}}$ in \eqref{eq: id burn-in}.  
    %\begin{align*}
    %    N_{\mathsf{ID}} = \mathsf{poly}(\log(1/\delta), \dx, \du,, \frac{1}{\lambda_{\min}(\mathsf{FI}(\theta^\star)^{-1})}, \norm{\Sigma_w}, \norm{\Sigma_u}, \lambda_{\min}(\Sigma_w), \lambda_{\min}(\Sigma_u), \calJ(\theta^\star)),
    %\end{align*}
    %and $\calJ(\theta^\star) = \sum_{t=0}^{T-1} \norm{A(\theta^\star)^t \bmat{I & B(\theta^\star)}}$. 
\end{lemma}
The result follows standard arguments applying concentration inequalities to linear systems \cite{ziemann2023tutorial, tu2024learning}. The specific form with the weighting matrix $H$ is an instantiation of Theorem 3.1 in \cite{lee2024active}. For completeness, a proof is provided below. 

To prove \Cref{thm: identification bound}, we first state two supporting lemmas. The first is a standard result on covariance concentration from \citet{jedra2020finite}. We define the matrix $\Gamma_x$ as 
\begin{align*}
    \Gamma_x = \bmat{0 \\ \bmat{I & B} \\ A \bmat{I & B} &\bmat{I & B} \\ \vdots \\ A^{T-1} \bmat{I & B} & \dots \bmat{I & B}}
\end{align*}
such that for any experiment $n$,
\begin{align*}
    \bmat{X_1^n \\ \vdots \\ X_T^n} = \Gamma_x \bmat{W_1^n \\ U_1^n \\ \vdots \\ W_{T-1}^n \\ U_{T-1}^n}. 
\end{align*}

\begin{lemma}[Covariance Concentration]
    \label{lem: covariance concentration}
    Consider collecting $N$ trajectories of length $T$ from \eqref{eq: linear system}. 
    Let $\beta \in \paren{0, \frac{1}{4}\lambda_{\min}\paren{\mathbf{E} \sum_{t=1}^T \bmat{X_t \\ U_t} \bmat{X_t\\ U_t}^\top}}$. Define the event 
    \begin{align}
        \label{eq: covariance concentration event}
        \calE \triangleq \norm{ \frac{1}{N} \sum_{n=1}^N \sum_{t=1}^T \bmat{X_t^n \\ U_t^n} \bmat{X_t^n\\ U_t^n}^\top - \mathbf{E} \sum_{t=1}^T \bmat{X_t \\ U_t} \bmat{X_t\\ U_t}^\top} \leq \beta.
    \end{align}
    There exists a universal positive constant $c$ such that if 
    \begin{align*}
        N \geq c \frac{\norm{\Gamma_x}^2 \norm{\Sigma_u} \norm{\Sigma_w} \norm{\mathbf{E} \sum_{t=1}^T \bmat{X_t \\U_t}\bmat{X_t \\U_t}^\top}^2}{\lambda_{\min}\paren{\bfE  \sum_{t=1}^T \bmat{X_t \\U_t}\bmat{X_t \\U_t}^\top}\beta^2} \paren{\log\frac{1}{\delta} + \dx + \du},
    \end{align*}
    then $\calE$ holds with probability at least $1-\delta$.
\end{lemma}
\begin{proof}
Let  $\eta \sim \calN(0, I_{\dx \du T})$ and $\Gamma$ be the matrix mapping from 
\begin{align*}
    \bmat{W_1^n \\ U_1^n \\ \vdots \\ W_{T-1}^n \\ U_{T-1}} \textrm{ to } \bmat{X_1^n \\ U_1^n \\X_2^n \\ U_2^n \\ \vdots \\ X_T^n \\ U_T^n}. \mbox{ Then  } \bmat{X_1^n \\ U_1^n \\X_2^n \\ U_2^n \\ \vdots \\ X_T^n \\ U_T^n} \overset{d}{=} \tilde \Gamma \eta, \mbox{ where } \tilde \Gamma = \Gamma \paren{I_T \otimes \bmat{\Sigma_w^{1/2} \\ & \Sigma_u^{1/2}}}.
\end{align*}
Let $M = \left(N \mathbf{E} \sum_{t=1}^T \bmat{X_t \\ U_t} \bmat{X_t\\ U_t}^\top\right)^{-1/2}.$
It holds that
\begin{align*}
    &\norm{M \sum_{n=1}^N \sum_{t=1}^T \bmat{X_t^n \\ U_t^n} \bmat{X_t^n \\ U_t^n}^\top M - I}\\
    &= \sup_{v \in \calS^{d-1}} v^\top \paren{M \sum_{n=1}^N \sum_{t=1}^T \bmat{X_t^n \\ U_t^n} \bmat{X_t^n \\ U_t^n}^\top M - I}v \\
    &= \sup_{v \in \calS^{d-1}} v^\top \paren{M \sum_{n=1}^N \sum_{t=1}^T \bmat{X_t^n \\ U_t^n} \bmat{X_t^n \\ U_t^n}^\top M - \mathbf{E}\brac{\paren{M \sum_{n=1}^N \sum_{t=1}^T \bmat{X_t^n \\ U_t^n} \bmat{X_t^n \\ U_t^n}^\top} M}}v \\
    &= \sup_{v \in \calS^{d-1}} \norm{\sigma_{Mv}^\top \tilde \Gamma \eta}^2 - \mathbf{E} \norm{\sigma_{Mv}^\top \tilde \Gamma \eta}^2,
\end{align*}
where $\sigma_{Mv} \triangleq I_{NT} \otimes (Mv).$ By the Hanson-Wright inequality applied to Gaussian quadratic forms (presented for subGaussian forms in Theorem 6.3.2 of \citep{vershynin2020high}, and specialized to Gaussians by \citet{laurent2000adaptive}) along with a covering argument, it holds that with probability at least $1-\delta$, 
\begin{align*}
    \sup_{v \in \calS^{d-1}} \norm{\sigma_{Mv}^\top \tilde \Gamma \eta}^2 - \mathbf{E} \norm{\sigma_{Mv}^\top \tilde \Gamma \eta}^2 \leq \frac{\beta}{\norm{\mathbf{E} \sum_{t=1}^T \bmat{X_t \\ U_t} \bmat{X_t\\ U_t}^\top}}
\end{align*}
with probability at least $1-\exp\paren{-c_1 N \beta^2 \frac{\lambda_{\min}\paren{\mathbf{E} \sum_{t=1}^T \bmat{X_t \\ U_t} \bmat{X_t\\ U_t}^\top}}{\norm{\tilde \Gamma}^2 \norm{\mathbf{E} \sum_{t=1}^T \bmat{X_t \\ U_t} \bmat{X_t\\ U_t}^\top}^2} + c_2 d}$. Inverting this and applying submultiplicativity to bound $\norm{\tilde \Gamma}$ concludes the statement.



\end{proof}
% \begin{align*}
%     \Gamma = \bmat{0 \\
%     \bmat{I & B} \\ 
%     A \bmat{I & B} & \bmat{I & B}  \\ 
%     \vdots \\
%     A^{T-1} \bmat{I & B} & \dots & \dots  & \bmat{I & B}   
%     } \textrm{ such that for any $n\in[N]$} \bmat{X_{1}^n \\ X_2^n \\ \vdots X_T^n} = \Gamma \bmat{W_1^n \\ U_1^n \\ \vdots \\ W_{T-1}^n \\ U_{T-1}}
% \end{align*}

The second is a self-normalized martingale bound, adapted in Lemma A.8 of \citet{lee2024active} from the standard self-normalized martingale bound in Theorem 14.7 of \citet{pena2009self}.The specific form that we use is from Lemma A.8 of \citet{lee2024active}. 
\begin{lemma}[Lemma A.8 of \citet{lee2024active}]
\label{lem: sn martingale bnd}
    Let $\curly{W_k}_{k=1}^K$ be a sequence of standard normal Gaussian random variables.  Let $\curly{Z_{k}}_{k=1}^K$ be a sequence of Gaussian random vectors assuming values in $\R^{d_{\theta}}$ such that $Z_k$ is independent from $W_\tau$ for $\tau \geq k$.  Let $\Sigma_Z \triangleq \mathbf{E} \frac{1}{K} \sum_{k=1}^K Z_k Z_k^\top$. Suppose $H \in \R^{d_{\theta} \times d_{\theta}}$ is positive definite and $\beta \in \R$ satisfies 
    \[
        0 < \beta \leq \frac{\lambda_{\min}\paren{\Sigma_Z}}{2}.
    \]
    As long as the event $\norm{\frac{1}{K} \sum_{k=1}^K Z_k Z_k^\top - \frac{1}{K} \mathbf{E} \sum_{k=1}^K Z_k Z_k^\top} \leq \beta$, then the following holds with probability at least $1-\delta$, 
    \begin{align*}
        &\norm{\left(\sum_{k=1}^K Z_k Z_k^\top\right)^{-1} \sum_{k=1}^K Z_k W_k}_H^2 \leq 2 \paren{1 + \frac{4\beta}{\lambda_{\min}(\Sigma_Z)}}\paren{ \trace\paren{(K \Sigma_Z)^{-1} H} + 2  \norm{(K \Sigma_Z)^{-1} H} \log \frac{1}{\delta}}. 
    \end{align*}
\end{lemma}
Leveraging the above two results, we can prove \Cref{thm: identification bound}. We write the least squares identification error as
\begin{align*}
    \bmat{ A(\hat \theta) & B(\hat \theta)} - \bmat{ A(\theta^\star) & B(\theta^\star)} = \sum_{t=1,n=1}^{T,N} W_{t}^n \bmat{X_t^n \\ U_t^n}^\top \left(\sum_{t=1,n=1}^{T,N} \bmat{X_t^n \\ U_t^n} \bmat{X_t^n \\ U_t^n}^\top\right).
\end{align*}
The noise covariance of $W_t^n$ can be pulled out such that for $\xi_{t,n} =\Sigma_w^{-1/2} W_{t}^n$, 
\begin{align*}
    \bmat{ A(\hat \theta) & B(\hat \theta)} - \bmat{ A(\theta^\star) & B(\theta^\star)} = \sum_{t=1,n=1}^{T,N} \Sigma_w^{1/2} \xi_{t}^n \bmat{X_t^n \\ U_t^n}^\top \left(\sum_{t=1,n=1}^{T,N} \bmat{X_t^n \\ U_t^n} \bmat{X_t^n \\ U_t^n}^\top\right).
\end{align*}
Applying the vectorization identity $\VEC(XYZ) = (Z^\top \otimes X) \VEC Y$, we find that
\begin{align*}
    &\VEC \paren{\bmat{ A(\hat \theta) & B(\hat \theta)} - \bmat{ A(\theta^\star) & B(\theta^\star)}} \\
    &= \left(\sum_{t=1,n=1}^{T,N}    \paren{\bmat{X_t^n \\ U_t^n} \otimes \Sigma_w^{-1/2}}\paren{ \bmat{X_t^n \\ U_t^n}^\top  \otimes \Sigma_w^{-1/2}}\right)^{-1}\sum_{t=1,n=1}^{T,N} \paren{\bmat{X_t^n \\ U_t^n} \otimes \Sigma_w^{-1/2}} \eta_t^n.
\end{align*}
This can be further decomposed by letting $Z_{t}^n[i]$ be the $i^{\mathsf{th}}$ column of $\bmat{X_t^n \\ U_t^n} \otimes \Sigma_w^{-1/2}$, and $\eta_t^n[i]$ be the $i^{\mathsf{th}}$ entry of $\eta_t^n$. Then 
\begin{align*}
    &\VEC \paren{\bmat{ A(\hat \theta) & B(\hat \theta)} - \bmat{ A(\theta^\star) & B(\theta^\star)}} = \left(\sum_{t=1, n=1, i=1}^{T,N,\dx} Z_t^n[i] Z_t^n[i]^\top\right)^{-1}\sum_{t=1,n=1,i=1}^{T,N, \dx} Z_t^n[i] \eta_t^n[i].
\end{align*}
Invoking \Cref{lem: covariance concentration}, the event
\begin{align*}
    \calE = \curly{\norm{\frac{1}{N} \sum_{t=1, n=1, i=1}^{T,N,\dx} Z_t^n[i] Z_t^n[i]^\top - \Sigma_Z} \leq \frac{1}{4} \lambda_{\min}(\Sigma_Z)},
\end{align*}
with $\Sigma_Z = \mathbf{E} \sum_{t=1, i=1}^{T,\dx} Z_t[i] Z_t[i]^\top$
holds with probability at least $1-\delta/2$ as long as 
\begin{align*}
   N \geq c \frac{\norm{\Gamma_x}^2 \norm{\Sigma_u} \norm{\Sigma_w}^3 \norm{\mathbf{E} \sum_{t=1}^T \bmat{X_t \\U_t}\bmat{X_t \\U_t}^\top}^2}{\lambda_{\min}\paren{\bfE  \sum_{t=1}^T \bmat{X_t \\U_t}\bmat{X_t \\U_t}^\top}^3 \lambda_{\min}(\Sigma_w)^2} \paren{\log\frac{1}{\delta} + \dx + \du},
\end{align*}
for a universal positive constant $c$.  Under this event, \Cref{lem: sn martingale bnd} implies that with probability at least $1-\delta/2$,
\begin{align*}
    & \norm{\VEC \paren{\bmat{ A(\hat \theta) & B(\hat \theta)} - \bmat{ A(\theta^\star) & B(\theta^\star)}}}_H^2 \leq  4\paren{ \trace\paren{(N \Sigma_Z)^{-1} H} + 2  \norm{(N \Sigma_Z)^{-1} H} \log \frac{2}{\delta}}.
\end{align*}
To conclude the proof, note that $\Sigma_Z = \mathsf{FI}(\theta^\star)$, and union bound over the success events. Additionally, note that $\norm{\mathbf{E} \sum_{t=1}^T \bmat{X_t \\U_t}\bmat{X_t \\U_t}^\top}$ and $\norm{\Gamma_x}$ may be bounded in terms of $\calJ(\theta^\star) = \sum_{t=0}^{T-1} \norm{A(\theta^\star)^t \bmat{I & B(\theta^\star)}}$, $\norm{\Sigma_w}$ and $\norm{\Sigma_w}$ \citep{jedra2020finite}. 

\section{Fundamental Limits}
\label{s: lower bound proof}

While good algorithm design can decrease the suboptimality gap of the learned controller, there are fundamental limits on the achievable performance. These limits are characterized by the signal-to-noise ratio of the experiment procedure, as well as the sensitivity of the LQR problem to error in the parameter estimates. To capture the signal-to-noise ratio for the experimental procedure, we define the Fisher Information matrix:
\begin{align}
    \mathsf{FI}(\theta) \triangleq \mathbf{E}_{\theta}\brac{\sum_{t=1}^T \bmat{X_t \\ U_t} \bmat{X_t \\ U_t}^\top } \otimes \Sigma_W^{-1}. \label{eq:FI}
    %D_{\theta} \VEC \bmat{A(\theta) & B(\theta)}. \paren{D_{\theta} \VEC \bmat{A(\theta) & B(\theta)} }^\top  
\end{align}
To characterize the sensitivity of the LQR problem, we define the matrix $H(\theta)$ as 
\begin{align}
    \label{eq: model task Hessian}
    H(\theta) \triangleq D_{\theta} \VEC K(\theta)^\top  (\Sigma_X(\theta) \otimes (B(\theta)^\top P(\theta) B(\theta) + R)) D_{\theta} \VEC K(\theta).
\end{align}
One can verify that this matrix is the Hessian of the cost $C(K(\tilde \theta), \theta)$ with repsect to $\tilde \theta$, and evaluated at $\tilde \theta = \theta$ \citep{wagenmaker2021task}. The below theorem presents a lower bound on the  $\varepsilon$-local minimax excess cost gap in terms of these quantities. 
\begin{theorem}
    \label{thm: lower bound}
    Let $\rho(\theta', T, N)$ denote the distribution of the experiment data induced by running the aforementioned experiment procedure on the system $X_{t+1} = A(\theta') X_t + B(\theta') U_t + W_t$ for $N$ episodes of length $T$. Assume that $\rho(A(\theta^\star)) < 1$.   Consider applying any learning algorithm $\calA$ that maps a dataset \eqref{eq: dataset} to a controller $K$. Let $\varepsilon \in \R$ satisfy $0 \leq \varepsilon \leq {\varepsilon_{UB}}$, where $\varepsilon_{\mathsf{UB}} = \frac{1}{\mathsf{poly}(\norm{P(\theta^\star}, \tau_{B(\theta^\star)}, \norm{\mathsf{FI}(\theta^\star)}, \frac{1}{\lambda_{\min}(\mathsf{FI}(\theta^\star))})}$. Additionally suppose that $N \geq N_{LB}$ where $N_{LB} = \frac{1}{\varepsilon^2} \frac{1}{\lambda_{\min}(\mathsf{FI}(\theta^\star))} \mathsf{poly}(d_{\theta}, \norm{P(\theta)})$. It holds that 
    \begin{align*}
        &\sup_{\theta' \in \calB(\theta^\star, \varepsilon)} \mathbf{E}_{\mathsf{Data} \sim p(\theta', T, N)}\brac{C\paren{\calA\paren{\mathsf{Data}}, \theta'} - C(K(\theta'), \theta')} \geq \frac{1}{8} \trace\paren{H(\theta^\star) (N\mathsf{FI}(\theta^\star))^{-1}}.  
    \end{align*}
\end{theorem}
The above result follows from Theorem 2.2 of \citet{lee2023fundamental}; however, it is rewritten to demonstrate tight dependence on the system-theoretic quantities. A proof of this result is provided in below.
%  In light of the above fundamental limit, a learning algorithm is classified as efficient if we can find an upper bound that matches this lower bound up to universal constants for large $N$.

\begin{proof}
    

Denote the data by $Z$. Let $\lambda$ be a prior density over $\theta$ satisfying $\lambda(\theta) \propto (1 - \frac{1}{\varepsilon^2} \norm{\theta - \theta^\star}^2)^2$ for $\theta \in \calB(\theta^\star, \varepsilon)$. We may lower bound the minimax quantity by an expectation over the prior:
\begin{align*}
    &\sup_{\theta'\in\calB(\theta^\star, \varepsilon)} \bfE_{\mathsf{Z} \sim p(\theta', T, N)} \brac{C(\calA(Z), \theta') -C(K(\theta'), \theta') } \geq \bfE_{\Theta \sim \lambda} \bfE_{Z \sim p(\Theta, T, N)} \brac{C(\calA(Z), \Theta) -C(K(\Theta), \Theta) }.
    \end{align*}
Next, we will apply the performance difference lemma to lower bound the expected excess cost in terms of the gap between the controller output by our algorithm and the optimal controller. To do so, we condition on the event that our algorithm outputs a controller close enough to the certainty equivalent controllers within the ball. In particular, we define 
\begin{align*}
    \calE &= \curly{\sup_{\theta\in\calB(\theta^\star, \varepsilon)} \norm{\calA(Z) - K(\theta)} \leq \alpha }, \mbox{ where }
    \alpha = \inf_{\theta\in\calB(\theta^\star, \varepsilon)} \frac{1}{12 \norm{\Sigma^{K(\theta)}}^{5/2}}.%\inf_{\theta\in\calB(\theta^\star, \varepsilon)} \min\curly{\frac{\norm{A_{cl}(\theta)}}{\norm{B(\theta)}}, \frac{\lambda_{\min}(\Sigma^{K(\theta)}(\theta))/24}{\norm{A_{cl}(\theta)} \norm{B(\theta)} \sum_{t=0}^\infty \norm{A_{cl}(\theta)^t} \norm{\Sigma^{K(\theta)}(\theta)}}}. 
\end{align*}
Additionally define $\tilde \Psi$ and $\tilde \Sigma$ as the largest matrices in semidefinite order such that $\tilde \Sigma \preceq \Sigma^{K(\theta)}(\theta)$ and $\tilde \Psi \preceq \Psi(\theta)$ for all $\theta \in B(\theta^\star, \varepsilon)$. 
This allows us to apply \Cref{lem: performance difference} to achieve the following lower bound:
\begin{align*}
    &\sup_{\theta'\in\calB(\theta^\star, \varepsilon)} \bfE_{\mathsf{Z} \sim p(\theta', T, N)} \brac{C(\calA(Z), \theta') -C(K(\theta'), \theta') } \\
    &\geq  \bfE_{\Theta \sim \lambda} \bfE_{Z\sim p(\Theta, T, N)} \brac{ \trace\paren{(\calA(Z)  - K(\Theta)) \Sigma^{\calA(Z)}(\Theta) (\calA(Z) - K(\Theta))^\top \Psi(\Theta)} \mathbf{1}_{\calE}}, \\
    &\geq \frac{1}{2} \bfE_{\Theta \sim \lambda} \bfE_{Z \sim p(\Theta, T, N)} \brac{ \trace\paren{(\calA(Z) - K(\Theta)) \tilde \Sigma (\calA(Z) - K(\Theta))^\top \tilde \Psi} \mathbf{1}_{\calE}},
\end{align*}
where the final inequality follows from the fact that if $\calE$ holds, then \Cref{lem: lyap perturbation} ensures $\Sigma^{\calA(\mathsf{Data}}(\Theta) \succeq \frac{1}{2}\Sigma^{K(\Theta)}(\Theta),$ (\Cref{lem: cov lower bound}) and by substituting the lower bounds $\tilde \Sigma \preceq \Sigma^{K(\Theta)}(\Theta)$, $\tilde \Psi \preceq \Psi(\Theta)$. 

By application of the Van Trees inequality, as in Theorem 2.1 of \citet{lee2023fundamental}, it holds that 
\begin{align*}
    &\sup_{\theta'\in\calB(\theta^\star, \varepsilon)} \bfE_{Z \sim p(\theta', T, N)} \brac{C(\calA(Z), \theta') -C(K(\theta'), \theta') } \\
    &\geq \frac{1}{2} \trace\paren{ \paren{\tilde \Sigma \otimes \tilde \Psi}  \bfE \brac{D_{\theta} \VEC K(\Theta) \mathsf{1}_{\calE}} \paren{N\bfE\brac{\mathsf{FI}(\Theta)} + J(\lambda)}^{-1} \bfE \brac{D_{\theta} \VEC K(\Theta) \mathsf{1}_{\calE}} } \\
    &\geq \frac{1}{2} \inf_{\tilde \theta_1, \tilde \theta_2, \tilde\theta_3 \in \calB(\theta^\star, \varepsilon)} \trace\paren{ \paren{\tilde \Sigma \otimes \tilde \Psi}  D_{\theta} \VEC K(\tilde \theta_1)  \paren{N\mathsf{FI}(\tilde \theta_3) + J(\lambda)}^{-1} D_{\theta} \VEC K(\tilde \theta_2)}  \mathbf{P}(\calE)^2,
\end{align*}
where $J(\lambda) = \int \nabla \log \lambda(\theta) (\nabla \log \lambda(\theta))^\top \lambda(\theta) d\theta$ satisfies $\norm{J(\lambda)} \leq \frac{1}{\varepsilon^2} \frac{32}{d_\theta+2} \frac{\Gamma((d_\theta+5)/2)}{\Gamma(d_{\theta}/2)^2}$ (by the triangle inequality and direct calculation). Furthermore, first order Taylor expansions of $\Sigma^{K(\theta)}(\theta)\kron \Psi(\theta)$, $D_\theta \mathsf{vec} K(\theta)$, and $\mathsf{FI}(\theta)$ about $\theta^\star$, combined with the bounds of \Cref{lem: LQR Taylor expansion}, \Cref{lem: helper lemma for RC}, and \Cref{lem: lyap perturbation} we can express the above quantity as
\begin{align}\label{eq: lb perturbation}
    \frac{1}{2}  \trace\paren{ \paren{H(\theta^\star) + M_1}  (D_{\theta} \VEC K( \theta^\star) + M_2)  \paren{N\mathsf{FI}(\theta^\star) +  M_3+ J(\lambda)}^{-1} (D_{\theta} \VEC K(\theta^\star) + M_4)}  \mathbf{P}(\calE)^2,
\end{align} 
\sloppy where $\norm{M_1} \leq 1e7 \tau_{B(\theta^\star)}^2 \norm{P(\theta^\star)}^{15} \varepsilon$, $\norm{M_2} \leq 2000 \norm{P(\theta^\star)}^{15/2} \varepsilon$, $\norm{M_3} \leq  2N \norm{\mathsf{FI}(\theta^\star)} \varepsilon$ and $\norm{M_4} \leq 2000 \norm{P(\theta^\star)}^{15/2} \varepsilon$.

We will show that the above quantity is at least 
\begin{align*}
    \frac{1}{8}  \trace\paren{ \paren{H(\theta^\star)}  (D_{\theta} \VEC K( \theta^\star))  \paren{N\mathsf{FI}(\theta^\star)}^{-1} (D_{\theta} \VEC K(\theta^\star))}.
\end{align*}
To do so, assume to the contrary that the inequality does not hold. 
Under this assumption, we will show that $\bfP(\calE) \geq \frac{1}{\sqrt{2}}$. This follows by observing that $\sup_{\theta\in\calB(\theta^\star, \varepsilon)} \norm{\calA(Z) - K(\theta)}\leq \inf_{\theta\in B(\theta^\star, \varepsilon)} \norm{\calA(Z) - K(\theta)} + \sup_{\theta_1,\theta_2\in\calB(\theta^\star, \varepsilon))} \norm{K(\theta_1) - K(\theta_2)} \leq\inf_{\theta\in B(\theta^\star, \varepsilon)} \norm{\calA(Z) - K(\theta)}  + 64 \norm{P(\theta^\star)}^{7/2} \varepsilon$. By a sufficiently small choice of $\varepsilon$, we may show the desired condition holds if  $\inf_{\theta\in B(\theta^\star, \varepsilon)} \norm{\calA(Z) - K(\theta)} \leq \alpha/2$ with high probability. In particular, we may bound $\inf_{\theta\in B(\theta^\star, \varepsilon)} \norm{\calA(Z) - K(\theta)} \leq \sqrt{{\norm{\calA(Z) - K(\Theta)}^2}} \leq \sqrt{\trace((\calA(Z) - K(\Theta)) \Sigma^{\calA(Z)}(\Theta)(\calA(Z) - K(\Theta))^\top \Psi(\Theta)}$. This quantity is precisely the excess cost of the algorithm applied to dataset $Z$ on system $\Theta$. Markov's inequality then implies that this quantity exceeds $\alpha^2/4$ with probabiliy at most $\frac{4\bfE\brac{C(\calA(Z), \Theta) - C(K(\Theta), \Theta)}}{\alpha^2}$. Under our assumption, this probability can be bounded as
\begin{align*}
    P(\calE^c) \leq \frac{1}{2 \alpha^2}  \trace\paren{ \paren{H(\theta^\star)}  (D_{\theta} \VEC K( \theta^\star))  \paren{N\mathsf{FI}(\theta^\star)}^{-1} (D_{\theta} \VEC K(\theta^\star))}.
\end{align*}
For $N$ sufficiently large, as given in the statement, this implies that $P(\calE) \geq \frac{1}{\sqrt{2}}$. Then, by a sufficiently small choice of $\varepsilon$ as given in the theorem statement, \eqref{eq: lb perturbation} implies that the minimax excess cost exceeds $\frac{1}{8}  \trace\paren{ \paren{H(\theta^\star)}  (D_{\theta} \VEC K( \theta^\star))  \paren{N\mathsf{FI}(\theta^\star)}^{-1} (D_{\theta} \VEC K(\theta^\star))}$, contradicting the assumption that the excess cost falls below this quantity. 

\end{proof}


% \section{Upper Bound for the CE Controller Proofs}
The Taylor's expansion of $C(K(\hat\theta), \theta^\star)$ evaluated at $\theta^\star$ gives us
\begin{align*}
    &C(K(\hat{\theta}), \theta^\star)) - C(K(\theta^\star), \theta^\star) \\ 
    &= \frac{1}{2}
    [\hat\theta - \theta^\star]^T\left(\left.\frac{\partial K(\theta)}{\partial \theta}\right|_{\theta = \theta^\star}\right)^T
    \left.\frac{\partial^2 C(K(\theta), \theta^\star)}{\partial K(\theta)^2}\right|_{K(\theta) = K(\theta^\star)}
    \left.\frac{\partial K(\theta)}{\partial \theta}\right|_{\theta = \theta^\star}
    [\hat{\theta} - \theta^\star] \\
    &+ \frac{1}{6}
    \left.\frac{\partial^3 C(K(\theta), \theta^\star)}{\partial K(\theta)^3}\right|_{K(\theta) = K(\theta')}
    \left[\left.\frac{\partial K(\theta)}{\partial \theta}\right|_{\theta = \theta'}
    (\hat{\theta} - \theta^\star), 
    \left.\frac{\partial K(\theta)}{\partial \theta}\right|_{\theta = \theta'}(\hat{\theta} - \theta^\star), 
    \left.\frac{\partial K(\theta)}{\partial \theta}\right|_{\theta = \theta'}(\hat{\theta} - \theta^\star)\right] \\
    &+ \frac{1}{2}
    \left.\frac{\partial^2 C(K(\theta), \theta^\star)}{\partial K(\theta)^2}\right|_{K(\theta) = K(\theta')}
    \left[\left.\frac{\partial^2 K(\theta)}{\partial \theta^2}\right|_{\theta = \theta'}
    [\hat{\theta} - \theta^\star,\hat{\theta} - \theta^\star] ,
    \left.\frac{\partial K(\theta)}{\partial \theta,}\right|_{\theta = \theta'}(\hat{\theta} - \theta^\star)\right] 
\end{align*}
where $\theta' = \theta^\star + t(\hat\theta - \theta^\star)$ for $t\in[0,1]$, and we used $\frac{\partial C(K(\theta^\star), \theta^\star)}{\partial K(\theta)}= 0$ since $K_{CE}(\theta^\star) = \argmin_KC(K, \theta^\star)$.

From (\ref{eq:d2C/dK2 bound}), (\ref{eq:d3C/dK3 bound}), (\ref{eq:PK bound}),(\ref{eq:dk/dtheta bound}), (\ref{eq:dk/dtheta bound}), (\ref{eq:d2k/dtheta2 bound}), (\ref{eq:P(A',B') bound}), the upper bound on the CE controller is given as
\begin{align*}
    &C(K(\hat{\theta}), \theta^\star)) - C(K(\theta^\star), \theta^\star) \\ 
    &\leq \frac{1}{2}
    [\hat\theta - \theta^\star]^T\left(\left.\frac{\partial K(\theta)}{\partial \theta}\right|_{\theta = \theta^\star}\right)^T
    \left.\frac{\partial^2 C(K(\theta), \theta^\star)}{\partial K(\theta)^2}\right|_{K(\theta) = K(\theta^\star)}
    \left.\frac{\partial K(\theta)}{\partial \theta}\right|_{\theta = \theta^\star}
    [\hat{\theta} - \theta^\star] \\
    &+\frac{7^3}{6}\left(\frac{21}{20}\right)^{9/2}\left(\frac{11}{10}\right)^{21/2}(68\|B\|_F^3 + 28\|B\|_F^2 + 6\|B\|_F)\|\Sigma_w\|_F\|P_\star\|^{15}_{op}\|\hat\theta - \theta^\star\|_2^3 \\
    &+1015\left(\frac{21}{20}\right)^3\left(\frac{11}{10}\right)^{11}(10\|B\|_F^2 + 4\|B\|_F + 1)\|\Sigma_w\|_F\|P_\star\|^{14}_{op}\|\hat\theta - \theta^\star\|_2^3
\end{align*}
if $\epsilon_{op}\leq\frac{1}{54}\|P_\star\|_{op}^{-5}$.

From \Cref{thm: identification bound}, we get
\begin{align*}
    &C(K(\hat{\theta}), \theta^\star)) - C(K(\theta^\star), \theta^\star) \\ 
    &\leq \frac{1}{2}\|\hat{\theta} - \theta^\star\|_H + 
    poly(\|B\|_F, \|\Sigma_w\|_F)\|P_\star\|^{15}_{op}\|\hat\theta - \theta^\star\|^3_2
    \\
    &\leq 2 \frac{\trace\paren{H \mathsf{FI}(\theta^\star)^{-1}}}{N} + 4 \frac{\norm{H \mathsf{FI}(\theta^\star)^{-1}}}{N} \log\frac{2}{\delta} 
    + \frac{poly(\|B\|_F, \|\Sigma_W\|_F, \mathsf{FI}(\theta^\star), \delta)\|P_\star\|^{15}_{op}}{N^{3/2}}
\end{align*}
as long as $N \geq N_{\mathsf{ID}}$. 

\subsection{Norm bound on $\frac{\partial C}{\partial K}, \frac{\partial^2 C}{\partial K^2}, \frac{\partial^3 C}{\partial K^3}$}
Consider the perturbation of any $K$ along any matrix $\Delta$:
\begin{align}
    K(t) = K + t\Delta, ~ t \in \R \label{eq:K(t)}, ~ \|\Delta\|_F\leq1.
\end{align}
Let us adopt shorthand $(\cdot) \triangleq \frac{d}{dt}(\cdot)$. Then the norm bound on $\frac{\partial C}{\partial K}$ would be
\begin{align*}
    \left\|\frac{\partial C}{\partial K}\right\|_{op} &= \sup_{\|\Delta\|_F\leq1}\left|\frac{d}{dt}C(K+t\Delta, \theta^\star)|_{t=0}\right| = \sup_{\|\Delta\|_F\leq1} Tr(P'_K(0)\Sigma_w) \leq \sup_{\|\Delta\|_F\leq1}\|P_K'(0)\|_F\|\Sigma_w\|_F
\end{align*}
where $P_K(t)$ is the solution to the following Lyapunov equation:
\begin{align*}
    P_K(t) &= (A + BK(t))^TP_K(t)(A + BK(t)) + Q + K(t)^TRK(t) \\
    &= \dlyap(A+BK(t), Q + K(t)^TRK(t)).
\end{align*}
By taking the derivative, $P_K'(t)$ is given as 
\begin{align*}
    P_K'(t) &= \dlyap(A+BK(t), Q_1(t)) \\
    Q_1(t) &= \sym((A+BK(t))^TP_K(t)B\Delta) + \sym(K(t)^TR\Delta)
\end{align*}
where we used the notation $\sym(X) = X + X^T$. 
Then the norm bound on $P'_K(t)$ is given as
\begin{align}
    \|P_K'(t)\|_o &= \|\dlyap(A + BK(t), Q_1)\|_o \nonumber\\
    &\leq \|\dlyap(A+BK(t),I)\|_{op}\|Q_1\|_{o} \nonumber\\
    &\leq \|P_K(t)\|_{op}\|Q_1\|_o \nonumber\\
    &\leq 2\|P_K(t)\|^{5/2}_{op}\|B\|_o + \|P_K(t)\|^{3/2}_{op} \nonumber \\
    &\leq (2\|B\|_o+1)\|P_K(t)\|^{5/2}_{op}    \label{eq:Pk' bound} 
\end{align}
where the norm bound on $Q_1$ is given as 
\begin{align*}
    \|Q_1\|_{o} &\leq 2\|A+BK(t)\|_{o}\|P_K(t)\|_{op}\|B\Delta\|_{o} + 2\|K(t)\|_{op}\|R\Delta\|_o \\
    &\leq 2\|P_K(t)\|^{3/2}_{op}\|B\|_o + \|P_K(t)\|^{1/2}_{op} \hspace{5mm} \because R = I, \|\Delta\|_o\leq1
\end{align*}
Thus 
\begin{align}
    \left\|\left.\frac{\partial C(K(\theta), \theta^\star)}{\partial K}\right|_{K(\theta) = K(\theta')}\right\|_{op} 
    &\leq (2\|B\|_F+1)\|P_{K(\theta')}(0)\|^{5/2}_{op} \|\Sigma_w\|_F \label{eq:dC/dK bound}
\end{align}
Also the second derivative is given as
\begin{align*}
    P''_K(t) &= \dlyap(A+BK(t), Q_2) \\
    Q_2 &= 2\{sym((A+BK(t))^TP_K'(t)B\Delta) + (B\Delta)^TP_K(t)B\Delta + \Delta^TR\Delta\}
\end{align*}
Then the norm bound would be
\begin{align*}
    \|P_K''(t)\|_o &= \|\dlyap((A+BK(t)), Q_2)\|_o \\
    &\leq \|P_K(t)\|_{op}\|Q_2\|_o \\
    &\leq (10\|B\|_o^2 + 4\|B\| + 1)\|P_K(t)\|^{3}_{op}
\end{align*}
where, from (\ref{eq:Pk' bound}), 
\begin{align*}
    \|Q_2\|_o &\leq 4\|A+BK(t)\|_o\|P_K'(t)\|_{op}\|B\Delta\|_o + 2\|B\Delta\|_o^2\|P_K(t)\|_{op} + \|R^{1/2}\Delta\|^2_o \\
    &\leq 4\|B\|_o(2\|B\|_o+1)\|P_K(t)\|^{3}_{op} + 2\|B\|_o^2\|P_K(t)\|_{op} + 1 \\
    &\leq (10\|B\|_o^2 + 4\|B\| + 1)\|P_K(t)\|^{3}_{op}
\end{align*}
Therefore
\begin{align}
    &\left\|\left.\frac{\partial^2 C(K(\theta),\theta^\star)}{\partial K^2}\right|_{K(\theta) = K(\theta')}\right\|_{op} 
    \leq (10\|B\|_F^2 + 4\|B\|_F + 1)\|P_{K(\theta')}(0)\|^{3}_{op}\|\Sigma_w\|_F \label{eq:d2C/dK2 bound} 
\end{align}

Finally the third derivative is 
\begin{align*}
    P'''_K(t) &= \dlyap(A+BK, Q_3) \\
    Q_3 &= 3sym((A+BK(t))^TP_K''(t)B\Delta) + 4(B\Delta)^TP_K'(t)B\Delta 
\end{align*}
Then the norm bound would be
\begin{align*}
    \|P'''_K(t)\|_o &\leq \|Q_3\|_o\|P_K(t)\|_{op} \\
    &\leq (68\|B\|_o^3 + 28\|B\|_o^2 + 6\|B\|_o)\|P_K(t)\|^{9/2}_{op}
\end{align*}
where 
\begin{align*}
    \|Q_3\|_o &\leq 6\|A+BK(t)\|_o\|P_K''(t)\|_{op}\|B\Delta\|_o + 4\|P_K'(t)\|_{op} \|B\Delta\|_o^2 \\
    &\leq 6\|B\|_o(10\|B\|_o^2 + 4\|B\|_o + 1)\|P_K(t)\|^{7/2}_{op} + 4\|B\|_o^2(2\|B\|_o+1)\|P_K(t)\|^{5/2}_{op} \\
    &\leq (68\|B\|_o^3 + 28\|B\|_o^2 + 6\|B\|_o)\|P_K(t)\|^{7/2}_{op}
\end{align*}
thus 
\begin{align}
    \left\|\left.\frac{\partial^3 C(K(\theta), \theta^\star)}{\partial K^3}\right|_{K(\theta) = K(\theta')}\right\|_{op}\leq (68\|B\|_F^3 + 28\|B\|_F^2 + 6\|B\|_F)\|P_{K(\theta')}(0)\|^{9/2}_{op}\|\Sigma_w\|_F 
    \label{eq:d3C/dK3 bound}
\end{align}

From Theorem 5 in \citep{simchowitz2020naive}, the bound on $\|P_{K(\theta')}(0)\|_{op}$ is given as follows: if $\epsilon_{op}\leq\frac{1}{54}\|P_\star\|_{op}^{-5}$, we get
\begin{align}
    P_{K(\theta')}(0) \preceq \frac{21}{20}P_\star  \label{eq:PK bound}
\end{align}
% Note that this satisfies the necessary condition $\|B(K(\theta') - K(\theta^\star))\|_2 \leq \frac{1}{5}\|P_\star\|_{op}^{-3/2}$.



\subsection{Norm bound on $\frac{\partial K}{\partial \theta}, \frac{\partial^2 K}{\partial \theta^2}$}
Consider the perturbation of A and B, i.e.
\begin{align*}
    (A(t), B(t)) = (A_* + t\Delta_1, B_* + t\Delta_2)
\end{align*}
where $\|\Delta_1\|_F, \|\Delta_2\|_F\leq1, t\in\R$.
As long as $(A(t), B(t))$ is stabilizable, the discrete algebraic ricatti equation has a unique solution and associated optimal cost matrices, controllers, and closed-loop dynamics matrices are defined as
\begin{align*}
    P(t) \triangleq P_{\infty}(A(t), B(t)), ~ K(t) \triangleq K_{\infty}(A(t), B(t)), ~ A_{cl}(t) \triangleq A(t) + B(t)K(t)
\end{align*}
where $P_{\infty}$ is the solution of
\begin{align*}
    \dare([A, B], P) = A^TPA - P - A^TPB(R + B^TPB)^{-1}B^TPA + Q 
\end{align*}
and $K_{\infty} \triangleq argmin_{K}C(K, \theta^*)$ is given by
\begin{align}
    K(t) = -(R + B^TPB)^{-1}B^TPA \label{eq:K_inf}
\end{align}

By taking the derivative of $P(t)$, we get
\begin{align*}
    P'(t) &= \dlyap(A_{cl}(t), Q_4(t)) \\
    Q_4(t) &= \sym(A_{cl}^TP(t)\Delta_{A_{cl}}), ~ \Delta_{A_{cl}} = \Delta_1 + \Delta_2K(t)
\end{align*}
Thus the norm bound on $P'(t)$ is given by
\begin{align*}
    \|P'(t)\|_o &= \|dlyap(A_{cl}(t), Q_4(t)\|_o 
    \\
    &\leq \|dlyap(A_{cl}(t), I)\|_{op}\|Q_4(t)\|_o &\because  B.5
    \\
    &\leq \|P(t)\|_{op}\cdot2(\|A_{cl}(t)\|_{op}\|P(t)\|_{op}\|\Delta_{A_{cl}}\|_o) &\because B.5\\
    &\leq 2\|P(t)\|^2_{op}\|P(t)\|^{1/2}_{op}\cdot2\|P(t)\|_{op}^{1/2}&\because  B.8, C.3\\
    &= 4\|P(t)\|^3_{op}
\end{align*}
where $\|\Delta_{A_{cl}}\|_o \leq \|\Delta_1\|_o + \|\Delta_2\|_o\|K\|_{op}\leq
1 + \|K\|_{op} \leq 2\|P(t)\|_{op}^{1/2}$

Also the derivative of $K(t)$ is
\begin{align*}
    K'(t) = -(R + B^TPB)^{-1}\left(\Delta_2^TPA_{cl} + B^TP\Delta_{A_{cl}} + B^TP'A_{cl}\right)
\end{align*}
Let $R_0 = R + B^TPB$. 
Then the norm bound is given by
\begin{align*}
    \|K'(t)\|_o &\leq \|R_0^{-1}\|_{op}\|\Delta_2\|_{o}\|P\|_{op}\|A_{cl}\|_{op} + \|P\|^{1/2}_{op}\|\Delta_{A_{cl}}\|_{o} + \|P^{-1/2}\|_{op}\|P'\|_{op}\|A_{cl}\|_{op} \\
    &\leq 7\|P(t)\|^{7/2}_{op}
\end{align*}
where we used $\|R_0^{-1}B^TP^{1/2}\|_{op}\leq1 (\because$ C.3).

Then the norm bound of the first derivative is given by
\begin{align}
    \left\|\left.\frac{\partial K(\theta)}{\partial\theta}\right|_{\theta = \theta'}\right\|_{op} \leq 7\|P_{\infty}(A', B')\|_{op}^{7/2} \label{eq:dk/dtheta bound}
\end{align}

Also, $P''(t)$ is calculated as follows
\begin{align*}
    P''(t) &= \dlyap(A_{cl}, Q_5) \\
    Q_5 &= \sym(A_{cl}'P'A_{cl}) + \sym(A_{cl}'P\Delta_{A_{cl}}) + \sym(A_{cl}^TP\Delta_{A_{cl}}') + \sym(A_{cl}^TP'\Delta_{A_{cl}}')
\end{align*}
Therefore we get
\begin{align*}
    \|P''(t)\|_o &\leq \|P\|_{op}\|Q_5\|_o \\
    &\leq 2\|P\|_{op}(\|A_{cl}'\|\|P'\|\|A_{cl}\| + \|A_{cl}'\|\|P\|\|\Delta_{A_{cl}}\| + \|A_{cl}\|\|P\|\|\Delta_{A_{cl}}'\| + \|A_{cl}\|\|P'\|\|\Delta_{A_{cl}}'\|) \\
    &\leq 2\|P\|_{op}(36\|P\|^7 + 18\|P\|^5 + 7\|P\|^5 + 28\|P\|^5 &\because C.4 \\
    &\leq 178\|P(t)\|^7
\end{align*}

For K'', let $Q_6 = \Delta_B^TPA_{cl} + B^TP\Delta_{A_{cl}} + B^TP'A_{cl}$. Then 
\begin{align*}
    K'' = -R_0^{-1}Q_6' - R_0^{-1}R_0'K'
\end{align*}
Since $Q_6'$ is given as
\begin{align*}
    Q_6' &= \Delta_B^TP'A_{cl} + \Delta_B^TPA_{cl}' + \Delta_B^TP\Delta_{A_{cl}} + B^TP'\Delta_{A_{cl}} +B^TP\Delta_{A_{cl}}' + \Delta_B^TP'A_{cl} + B^TP''A_{cl} + B^TP'A_{cl}' \\
    &=\Delta_B^T(2P'A_{cl} + PA_{cl}' + P\Delta_{A_{cl}}) + B^T(P\Delta_{A_{cl}}' + P'A_{cl}' + P'\Delta_{A_{cl}} + P''A_{cl})
\end{align*}
we get the following norm bound 
\begin{align*}
    \|R_0^{-1}Q_6'\|_o &\leq \|R_0^{-1}\|(2\|P'\|\|P\|^{1/2} + 9\|P\|\|P\|^{7/2} + 2\|P\|\|P\|^{1/2}) \\
    &\hspace{5mm} + \|R_0^{-1}B\|(7\|P\|\|P\|^{7/2}+ 9\|P'\|\|P\|^{7/2} + 2\|P'\|\|P\|^{1/2} + \|P''\|\|P\|^{1/2}) \\
    &\leq 8\|P\|^{7/2} + 9\|P\|^4 + 2\|P\|^{3/2} + 7\|P\|^8 + 36\|P\|^{13/2} + 8\|P\|^{7/2} + 178\|P\|^{15/2} \\
    &\leq 248\|P\|^{15/2}
\end{align*}
For the second term, we get
\begin{align*}
    \|R_0^{-1}R_0'K'\|_o &\leq \|R_0^{-1}(B'^TPB + B^TP'B + B^TPB')K'\| \\
    &\leq (\|R_0^{-1}\|\|P\| + \|R_0^{-1}B\|\|P'\|)\|BK'\| + \|R_0^{-1}B\|\|P\|\|K'\| \\
    &\leq (\|P\| + 4\|P\|)\|BK'\| + \|P\|\|P\|^{7/2}\\
    &\leq 35\|P\|^{9/2} + 7\|P\|^{9/2} \\
    &\leq 42\|P\|^{9/2}
\end{align*}
Thus the norm bound on K'' is
\begin{align*}
    \|K''(t)\|_o &\leq 290\|P(t)\|^{15/2}_{op}
\end{align*}
Thus
\begin{align}
    \left\|\left.\frac{\partial^2 K(\theta)}{\partial\theta^2}\right|_{\theta=\theta'}\right\|_{op}
    &\leq 290\|P_{\infty}(A', B')\|^{15/2}_{op}\label{eq:d2k/dtheta2 bound}
\end{align}

Again from Theorem 5 in \citep{simchowitz2020naive}, the norm bound on $P_{\infty}(A', B')$ is given as follows: if $\epsilon_{op}\leq\frac{1}{54}\|P_\star\|_{op}^{-5}$, we get
\begin{align}
    \|P_{\infty}(A', B')\|_{op} \leq \frac{11}{10}\|P_\star\|_{op} \label{eq:P(A',B') bound}
\end{align}
\section{Certainty Equivalence Upper Bound}
\label{s: certainty equivalence bound}
Prior work \citep{wagenmaker2021task} has demonstrated that certainty equivalence is efficient by providing upper bounds on the excess cost that match the lower bound of \Cref{thm: lower bound}. In particular, the following bound holds.
\begin{theorem}
    \label{thm: certainty equivalence bound}
    Suppose the dataset $\curly{(X_t^n, U_t^n, X_{t+1}^n)}_{t=1, n=1}^{T,N}$ is collected from N trajectories of the system \eqref{eq: linear system} via a random control input $U_t \sim \calN(0, \Sigma_u)$. Let $\hat\theta$ be the least square estimate computed by \eqref{eq: least squares}. Let also $\delta\in(0,1)$. Then it holds with probability at least $1-\delta$ that 
    \begin{align}
        &C(K_{CE}(\hat\theta), \theta^\star) - C(K(\theta^\star), \theta^\star) \nonumber\\
        &\leq 4\frac{\trace\paren{H(\theta^\star)\mathsf{FI}(\theta^\star)^{-1}}}{N} + 8\frac{\norm{H(\theta^\star)\mathsf{FI}(\theta^\star)^{-1}}}{N}\log\frac{2}{\delta} + L_{\mathsf{CE}}(\theta^\star)\frac{\norm{\mathsf{FI}(\theta^\star)^{-1}}^{3/2}}{N^{3/2}}, \label{eq:CE Upper bound}
    \end{align}
    where $L_{\mathsf{CE}}(\theta^\star) = 2e7\tau_{B(\theta^\star)}^3\norm{P(\theta^\star)}^{14}\paren{d_\theta+\log\frac{2}{\delta}}^{3/2}$, as long as the number of trajectories $N$ satisfies
    %\begin{align}
     $   N \geq \max\curly{N_{\mathsf{ID}}, ~ 6e5(d_\theta+\log\frac{2}{\delta})\norm{\mathsf{FI}(\theta^\star)^{-1}}\norm{P(\theta^\star)}^{10}}.$% \label{eq:CE burn-in time}
    %\end{align}
\end{theorem}
The above result is sharpened from Theorem 2.1 of \citet{wagenmaker2021task} to avoid a logarthmic factor of the state dimension. A proof is provided in \Cref{subsec: Proof of CE upper bound}. By inverting the above high probability tail bound, one can show that for $N$ sufficiently large, the following inequality holds:
\begin{align*}
    \mathbf{E}_{\mathsf{data}} \brac{C(K_{\mathsf{CE}}(\hat\theta), \theta^\star) - C(K(\theta^\star), \theta^\star)} \leq c \trace\paren{H(\theta^\star) \mathsf{FI}(\theta^\star)^{-1}},
\end{align*}
where $c$ is a universal positive constant. This matches the lower bound of \Cref{thm: lower bound} up to a universal constant. 
%We present a bound on the performance achieved by certainty equivalence, sharpening the result of \citet{wagenmaker2021task} by removing a logarithmic factor in the state dimension.


%To prove this result, 
We first present a helping lemma that bounds the suboptimality gap in terms of a quadratic function of the parameter estimation error. 

\begin{lemma}
    \label{lem: CE upper bound}
    Suppose $\hat\theta$ is some parameter satisfying $\norm{\hat\theta-\theta^\star} \leq \frac{1}{256}\norm{P(\theta^\star)}^{-5}$. Then the excess cost of $K_{CE}(\hat\theta)$ would be
    \begin{align*}
        C(K_{CE}(\hat\theta), \theta^\star) - C(K(\theta^\star), \theta^\star) \leq \|\hat\theta - \theta^\star\|_{H(\theta^\star)}^2 +   L_{\mathsf{RC}}(\theta^\star)\norm{\hat\theta-\theta^\star}^3, 
    \end{align*}
    where 
    \begin{align*}
        L_{\mathsf{RC}}(\theta^\star) = 6e5\tau_{B(\theta^\star)}^3\norm{P(\theta^\star)}^{14}
    \end{align*}
\end{lemma}
\begin{proof}
    When $\|\hat\theta - \theta^\star\|\leq\frac{1}{16}\norm{P(\theta^\star)}^{-2}\leq \frac{1}{256} \norm{P(\theta^\star)}^{-5}$, 
    from \Cref{lem: Riccati perturbation}, \Cref{lem: CE stabilization}, \Cref{lem: excess cost decomposition} and \Cref{lem: cost gap taylor substitution}, we have
    \begin{align*}
        &C(K(\hat\theta), \theta^\star) - C(K(\theta^\star), \theta^\star)\\ 
        &\leq \|\hat\theta - \theta^\star\|_{H(\theta^\star)}^2 + 2 \norm{P(\theta^\star)}^2 \|\Sigma_X^{K(\hat\theta)}(\theta^\star)\|\tau_{B(\theta^\star)}^3\|(K(\hat{\theta})-K(\theta^\star))\|^3(\|B(\theta^\star)(K(\hat\theta-K(\theta^\star))\| + 2 \norm{P(\theta^\star)}^{1/2}) \\
        &\hspace{5mm}+ 2e5\tau_{B(\theta^\star)}^2\|P(\theta^\star)\|^{13}\|\hat\theta-\theta^\star\|^3 
        + 8e6\tau_{B(\theta^\star)}^2\|P(\theta^\star
        )\|^{17}\|\|\hat\theta-\theta^\star\|^4 \quad \mbox{(\Cref{lem: excess cost decomposition} and \Cref{lem: cost gap taylor substitution})} \\
        &\leq \|\hat\theta - \theta^\star\|_{H(\theta^\star)}^2 + 2^{17} \norm{P(\theta^\star)}^3\tau_{B(\theta^\star)}^3
        \norm{P(\theta^\star)}^{21/2} \norm{\hat\theta-\theta^\star}^3
        (32\norm{P(\theta^\star)}^{7/2} \norm{\hat\theta-\theta^\star} + 2 \norm{P(\theta^\star)}^{1/2}) \\
        &\hspace{5mm}+ 2e5\tau_{B(\theta^\star)}^2\|P(\theta^\star)\|^{13}\|\hat\theta-\theta^\star\|^3 
        + 8e6\tau_{B(\theta^\star)}^2\|P(\theta^\star
        )\|^{17}\|\|\hat\theta-\theta^\star\|^4 \quad \mbox{(\Cref{lem: CE stabilization} and \Cref{lem: Riccati perturbation})}\\
        &\leq \|\hat\theta - \theta^\star\|_{H(\theta^\star)}^2 + 
        \left(
        2^{18}\tau_{B(\theta^\star)}^3 + 2e5\tau_{B(\theta^\star)}^2
        \right)\norm{P(\theta^\star)}^{14}\norm{\hat\theta-\theta^\star}^3 \\
        &\hspace{5mm}+ \left(
        2^{22}\tau_{B(\theta^\star)}^3 + 8e6\tau_{B(\theta^\star)}^2
        \right)\norm{P(\theta^\star)}^{17}\norm{\hat\theta-\theta^\star}^4\\
        &\leq  \|\hat\theta - \theta^\star\|_{H(\theta^\star)}^2 +     6e5\tau_{B(\theta^\star)}^3\norm{P(\theta^\star)}^{14}\norm{\hat\theta-\theta^\star}^3, 
    \end{align*}
    where the final inequality follows from the fact that $\norm{\hat\theta-\theta^\star}\leq \frac{1}{256}\norm{P(\theta^\star)}^{-5}$.
\end{proof}

\subsection{Proof of \texorpdfstring{\Cref{thm: certainty equivalence bound}}{}}
\label{subsec: Proof of CE upper bound}
\begin{proof}
    From \Cref{lem: CE upper bound} and \Cref{thm: identification bound}, we get \eqref{eq:CE Upper bound} where we used $\trace\paren{\mathsf{FI}(\theta^\star)^{-1}}\leq\norm{\mathsf{FI}(\theta^\star)^{-1}}d_\theta$. Furthermore, from the closeness condition in \Cref{lem: CE upper bound}, i.e.  $\norm{\hat\theta-\theta^\star}\leq\frac{1}{256}\norm{P(\theta^\star)}^{-5}$, $N$ needs to satisfy the following condition:
    \begin{align*}
        &\norm{\hat\theta-\theta^\star}^2 \leq \frac{8(d_\theta+\log\frac{2}{\delta})\norm{\mathsf{FI}(\theta^\star)^{-1}}}{N}\leq\frac{1}{256^2}\norm{P(\theta^\star)}^{-10} \\
        &\iff N \geq 6e5(d_\theta+\log\frac{2}{\delta})\norm{\mathsf{FI}(\theta^\star)^{-1}}\norm{P(\theta^\star)}^{10}.
    \end{align*}
\end{proof}


% \section{Proof of Robust Control Upper Bound (Unused)}
\begin{enumerate}
\item Let $G = \{\theta : (\theta - \hat\theta)^T\mathsf{\hat{FI}}(\hat\theta)(\theta-\hat\theta)\leq8(d_x+\log\frac{1}{\delta})\}$ be the confidence region around $\hat\theta$. From \cref{thm: identification bound}, $\theta^\star\in G$ with probability at least $1-\delta$. 
\item Let the robust controller be 
\begin{align*}
    {K}_{RC}(G) = \argmin_{K}\sup_{\theta\in G}(C(K, {\theta}) - C(K(\theta), {\theta})).
\end{align*}
\item When $\theta^\star\in G$, the suboptimality gap is given by
\begin{align*}
    C(K_{RC}(G), \theta^\star) - C(K(\theta^\star), \theta^\star) 
    &\leq \sup_GC(K_{RC}(G), \theta) - C(K(\theta), \theta) \\
    &= \inf_K\sup_GC(K, \theta) - C(K(\theta), \theta)
\end{align*}
Now, assuming $\Sigma_w = I$, we get 
\begin{align*}
    &C(K,\theta) - C(K(\theta), \theta) \\
    &= \trace(P_K) - \trace (P_{K(\theta)}) \\
    &= \trace (\dlyap (A+BK, (K-K(\theta)^T\Psi(\theta)(K-K(\theta)))) \\
    &= \trace ((K-K(\theta))\Sigma_X^K(\theta)(K-K(\theta))^T\Psi(\theta)) \\
    &= \trace ((K-K(\theta))\Sigma_X^{K(\theta)}(\theta)(K-K(\theta))^T\Psi(\theta) + (K-K(\theta))(\Sigma_X^K(\theta) - \Sigma_X^{K(\theta)}(\theta))(K-K(\theta))^T\Psi(\theta))
\end{align*}
where $\Psi(\theta) = R + B^TP(\theta)B, \Sigma_X^K(\theta) = \dlyap((A+BK)^T, I)$ and used
\begin{align*}
    \trace(\dlyap(X,Y)) &= \trace\left(\sum_{t=0}^{\infty}(X^t)^TYX^t\right) \\
    &= \trace \left(Y\sum_{t=0}^{\infty}(X^t)^TX^t\right)\\
    &= \trace (Y\dlyap(X^T, I))
\end{align*}

\item For any $K_1, K_2$, 
\begin{align*}
    \Sigma_X^{K_1} - \Sigma_X^{K_2} &= \dlyap((A+BK_1)^T, I) - \dlyap((A+BK_2)^T, I) \\
    &= (A+BK_1)\Sigma_X^{K_1}(A+BK_1)^T - (A+BK_2)\Sigma_X^{K_2}(A+BK_2)^T \\
    &= (A+BK_2+B(K_1-K_2))\Sigma_X^{K_1}(A+BK_2+B(K_1-K_2))^T \\
    & \hspace{5mm}- (A+BK_2)\Sigma_X^{K_2}(A+BK_2)^T \\
    &= (A+BK_2)(\Sigma_X^{K_1}-\Sigma_X^{K_2})(A+BK_2)^T + B(K_1-K_2)\Sigma_X^{K_1}(K_1-K_2)^TB^T\\
    &\hspace{5mm}+ (A+BK_2)\Sigma_X^{K_1}(K_1-K_2)^TB^T + B(K_1-K_2)\Sigma_X^{K_1}(A+BK_2)^T \\
    &= \dlyap(A+BK_2, B(K_1-K_2)\Sigma_X^{K_1}(K_1-K_2)^TB^T \\
    &\hspace{5mm}+ \sym((A+BK_2)\Sigma_X^{K_1}(K_1-K_2)^TB^T))
\end{align*}
Thus for $K, K(\theta)$, we get
\begin{align*}
    \|\Sigma_X^{K} - \Sigma_X^{K_\theta}\| 
    &\leq \|\dlyap(A+BK(\theta), I)\|(\|\Sigma_X^{K}\|\|B\|^2\|K-K(\theta)\|^2 \\
    &\hspace{5mm}+ 2\|A+BK(\theta)\|\|\Sigma_X^{K}\|\|K-K(\theta)\|\|B\|) \\
    &\leq \|\|\Sigma_X^{K(\theta)}\|\|\|\Sigma_X^{K}\|\|B\|\|K-K(\theta)\|(\|B\|\|K-K(\theta)\| + 2\|A+BK(\theta)\|)
\end{align*}

\item The suboptimality gap would be
\begin{align*}
    &C(K_{RC}(G), \theta^\star) - C(K(\theta^\star), \theta^\star) \\
    &\leq \inf_K\sup_G \trace ((K-K(\theta))\Sigma_X^{K(\theta)}(\theta)(K-K(\theta))^T\Psi(\theta)) \\
    &\hspace{5mm}+ \|\Sigma_X^{K(\theta)}(\theta)\|\|\Sigma_X^{K}(\theta)\|\|B\|\|K-K(\theta)\|^3\|\Psi(\theta)\|(\|B\|\|K-K(\theta)\| + 2\|A+BK(\theta)\|)\Tesshu{\times du?} \\
    &\triangleq UB(K, \Sigma_X^K(\theta), \theta)
\end{align*}

\item Restricting the class over $K$ to CE policies yields
\begin{align*}
    C(K_{RC}(G), \theta^\star) - C(K(\theta^\star), \theta^\star) &\leq \inf_{\tilde\theta}\sup_{\theta\in G} UB(K(\tilde\theta), \Sigma_X^{K(\tilde\theta)}(\theta), \theta) \\
    &\leq \inf_{\tilde\theta_1}\sup_{\tilde\theta_2, \theta\in G}UB(K(\tilde\theta_2), \Sigma_X^{K(\tilde\theta_1)}(\theta), \theta)
\end{align*}

Also, let us introduce the stabilizablity condition
\begin{definition}
    Let $\tau, r\in\R_{++}$. An instance $(A, B, \theta^\star, Q, R)$ is $(\tau,r)$-robustly LQR stabilizable if for any $G\subseteq\mathcal{B}(\theta^{\star}, r)$, there exists $\tilde\theta\in G$ such that, for any $\theta\in G$, 
    \begin{align*}
        \|\Sigma_X^{K(\tilde\theta)}\|\leq\tau
    \end{align*}
\end{definition}

\item Fix any $r$, and let $\tau$ be the smallest number such that $(A, B, \theta^\star, Q, R)$ is $(\tau, r)$-robustly stabilizable. Also let $N$ be sufficiently large that $G\subseteq\mathcal{B}(\theta^{\star}, r)$ with high probability. Then it holds by definition that $ \|\Sigma_X^{K(\tilde\theta)}\|\leq\tau$. Thus we get
\begin{align*}
    &C(K_{RC}(G), \theta^\star) - C(K(\theta^\star), \theta^\star) \\
    &\leq \inf_{\tilde\theta_1}\sup_{\tilde\theta_2, \theta\in G} \trace ((K(\tilde\theta_2)-K(\theta))\Sigma_X^{K(\theta)}(\theta)(K(\tilde\theta_2)-K(\theta))^T\Psi(\theta)) \\
    &\hspace{5mm}+ \tau\|\Sigma_X^{K(\theta)}(\theta)\|^{1/2}\|B\|\|K(\tilde\theta_2)-K(\theta)\|^3\|\Psi(\theta)\|(\|B\|\|K(\tilde\theta_2)-K(\theta)\| + 2\|A+BK(\theta)\|)
\end{align*}

For the first term, we get
\begin{align*}
    &\trace ((K(\tilde\theta_2)-K(\theta))\Sigma_X^{K(\theta)}(\theta)(K(\tilde\theta_2)-K(\theta))^T\Psi(\theta)) \\
    &= \VEC(K(\tilde\theta_2) - K(\theta))(\Psi(\theta)\kron\Sigma_X^{K(\theta)})\VEC(K(\tilde\theta_2)-K(\theta)) \\
    &= \|\tilde\theta_2-\theta\|_{K'(\theta)^T(\Psi(\theta)\kron\Sigma_X^{K(\theta)})K'(\theta)} + h.o.t ~\Tesshu{\text{need to explicitly calculate?}}
\end{align*}
where we used the following decomposition from the mean value theorem:
\begin{align*}
    \VEC(K(\tilde\theta_2)-K(\theta)) &= K'(\theta')[\tilde\theta_2 - \theta] \\
    &= K'(\theta)[\tilde\theta_2 - \theta] + \left(K'(\theta') - K'(\theta)\right)[\tilde\theta_2 - \theta]
\end{align*}
where $\theta' = \theta + t(\tilde\theta_2 - \theta)$ and $K'(\theta') \triangleq \left.\frac{\partial K}{\partial \theta}\right|_{\theta = \theta'}$. 
Taking the supremum of $\tilde\theta_2, \theta$ over $G$ yields
\begin{align*}
    \sup_{\tilde\theta_2, \theta\in G}\|\tilde\theta_2 - \theta\|_{H(\theta)} \leq \sup_{\theta\in G}16(d_x + log1/\delta)\|\mathsf{\hat{FI}}(\hat\theta)^{-1}H(\theta)\|
\end{align*}
where $H(\theta) = K'(\theta)^T(\Psi(\theta)\kron\Sigma_X^{K(\theta)})K'(\theta)$. 
\Tesshu{Bound $H(\theta)$}
Let $M(\theta) = (\Psi(\theta)\kron\Sigma_X^{K(\theta)})$. Then
\begin{align*}
    &H(\theta) - H(\theta^\star) \\
    &= K'(\theta)^TM(\theta)K'(\theta) - K'(\theta^\star)^TM(\theta^\star)K'(\theta^\star) \\
    &= K'(\theta^\star)^TM(\theta)K'(\theta) + (K'(\theta) - K'(\theta^\star)^TM(\theta)(K'(\theta) - K'(\theta^\star)) + 2K'(\theta^\star)^TM(\theta)(K'(\theta) - K'(\theta^\star)) \\
    &\hspace{5mm}- K'(\theta^\star)^TM(\theta^\star)K'(\theta^\star) \\
    &\leq K'(\theta^\star)^TM(\theta)K'(\theta) - K'(\theta^\star)^TM(\theta^\star)K'(\theta^\star)
    + \|K''(\theta')\|^2\|\theta-\theta^\star\|^2\|M(\theta)\| \\
    &\hspace{5mm}+ 2\|K'(\theta^\star)\|\|M(\theta)\|\|\|K''(\theta)\|\|\theta-\theta^\star\|\\
    &=  K'(\theta^\star)^T(M(\theta) - M(\theta^\star))K'(\theta^\star) + \|K''(\theta')\|^2\|\theta-\theta^\star\|^2\|M(\theta)\| \\
    &\hspace{5mm}+ 2\|K'(\theta^\star)\|\|M(\theta)\|\|\|K''(\theta)\|\|\theta-\theta^\star\| \\
    &\leq \|K'(\theta^\star)\|^2\|M(\theta) - M(\theta^\star)\| + \|K''(\theta')\|^2\|\theta-\theta^\star\|^2\|M(\theta)\|\\
    &\hspace{5mm}+ 2\|K'(\theta^\star)\|\|M(\theta)\|\|\|K''(\theta)\|\|\theta-\theta^\star\|
\end{align*}


For the second term, from Proposition 4 in \citep{simchowitz2020naive}, if $u\triangleq8\|P_\star\|_{op}^2\epsilon_{op}<1$, $\hat\theta$ is stabilizable and the following bound holds:
\begin{align*}
    \|K(\hat\theta) - K_\star\|_o \leq 7(1-u)^{-7/4}\|P_\star\|_{op}^{7/2}\epsilon_o
\end{align*}
Thus if $G$ is small enough that $\theta^\star\in G$, then
\begin{align*}
    \sup_{\tilde\theta_2,\theta\in G}\|K(\tilde\theta_2) - K(\theta)\|
    &\leq 2\sup_{\theta\in G}\|K(\theta) - K(\theta^\star)\| \Tesshu{\text{does it hold?}} \\
    &\leq 14(1-u)^{-7/4}\|P_\star\|_{op}^{7/2}\epsilon_o
\end{align*}
as long as $\|\theta-\theta^\star\|_{op}\leq\frac{1}{16}\|P_\star\|_{op}^2$

\end{enumerate}

\section{Proof of Robust Control Upper Bound}
\label{s: robust control proof}

\begin{lemma}
    \label{lem: Robust Control Upper Bound}
    Let $G$ denote the ellipsoid:
    \begin{align*}
        G \triangleq \{\theta ~:~ \theta=\hat \theta + \Gamma w, ~ w\in\mathcal{B}(0, 1)\}.
    \end{align*}
    Suppose $G$ is $R$-robustly stabilizable, the diameter of $G$ satisfies $\mathsf{diam}(G)\leq\frac{1}{16}\inf_{\theta\in G}\norm{P(\theta)}^{-2}$, and $\theta^\star\in G$. Then the excess cost of $K_{RC}(G)$ would be
    \begin{align*}
        C(K(G), \theta^\star) - C(K(\theta^\star), \theta^\star) \leq \sup_{\theta_1, \theta_2\in G}\paren{\norm{\theta_1-\theta_2}_{H(\theta^\star)} + L_{RC}(\theta^\star)\norm{\theta_1-\theta_2}^3}, 
    \end{align*}
    where
    \begin{align*}
        L_{\mathsf{RC}}(\theta^\star) = 6e6M\tau^3_{B(\theta^\star)}\norm{P(\theta^\star)}^{17}.
    \end{align*}
\end{lemma}

\begin{proof}
    By the definition of the robust controller and \Cref{lem: excess cost decomposition},
    \begin{align*}
        &C(K(G), \theta^\star) - C(K(\theta^\star), \theta^\star) \\
        &\leq \sup_{\theta\in G}C(K(G), \theta) - C(K(\theta), \theta) = \inf_K\sup_{\theta\in G}C(K, \theta) - C(K(\theta), \theta) \\
        &\leq \inf_K\sup_{\theta\in G} \trace ((K-K(\theta))\Sigma_X^{K(\theta)}(\theta)(K-K(\theta))^T\Psi(\theta))\\
        &\hspace{5mm}+ 2 \norm{P(\theta)}^2 \norm{\Sigma_X^{K}(\theta)}\tau_{B(\theta)}^3\norm{K-K(\theta)}^3(\norm{B(\theta)(K-K(\theta))} + 2 \norm{P(\theta)}^{1/2}) \\
        &\triangleq \inf_K\sup_{\theta\in G}\mathsf{UB}(K, \Sigma^K_{\theta}, \theta),
    \end{align*}
    where we defined $\mathsf{UB}(K, \Sigma^K_{\theta}, \theta)$ as in the last equality.  
    Now restricting our policy $K$ to CE policies $K(\tilde\theta)$ yields
    \begin{align*}
        C(K(G), \theta^\star) - C(K(\theta^\star, \theta^\star) 
        &\leq \inf_{\tilde\theta\in G}\sup_{\theta\in G}\mathsf{UB}(K(\tilde\theta), \Sigma^{K(\tilde\theta)}(\theta), \theta) \\
        &\leq \inf_{\tilde\theta_1\in G}\sup_{\tilde\theta_2, \theta\in G}\mathsf{UB}(K(\tilde\theta_2), \Sigma^{K(\tilde\theta_1)}(\theta), \theta), 
    \end{align*}
    where we took infimum only over $K(\tilde\theta_1)$ in the last inequality. 
    Then since $\mathsf{diam}(G)\leq\frac{1}{16}\inf_{\theta\in G}\norm{P(\theta)}^{-2}$, $\norm{\tilde\theta_2-\theta}\leq\frac{1}{16}\norm{P(\theta)}^{-2}$ for any $\tilde\theta_2, \theta\in G$. 
    Thus we can apply \Cref{lem: cost gap taylor substitution} and the first term of $\mathsf{UB}(K(\tilde\theta_2), \Sigma^{K(\tilde\theta_1)}(\theta), \theta)$ results in 
    \begin{align*}
        &\sup_{\tilde\theta_2, \theta\in G}\trace ((K(\tilde\theta_2)-K(\theta))\Sigma_X^{K(\theta)}(\theta)(K(\tilde\theta_2)-K(\theta))^T\Psi(\theta)) \\
        &\leq \sup_{\tilde\theta_2, \theta\in G}\norm{\tilde\theta_2-\theta}_{H(\theta)}^2 + 2e5\tau_{B(\theta)}^2\norm{P(\theta)}^{13}\norm{\tilde\theta_2-\theta}^3 + 8e6\tau_{B(\theta)}^2\norm{P(\theta)}^{17}\|\norm{\tilde\theta_2-\theta}^4 .
    \end{align*} 
    From \Cref{lem: helper lemma for RC}, 
    \begin{align*}
        \sup_{\tilde\theta_2, \theta\in G}\norm{\tilde\theta_2-\theta}_{H(\theta)}^2
        &= \sup_{\tilde\theta_2, \theta\in G}\paren{\tilde\theta_2-\theta}^TH(\theta)\paren{\tilde\theta_2-\theta} \\
        &\leq  \sup_{\tilde\theta_2, \theta\in G}\norm{\tilde\theta_2-\theta}_{H(\theta^\star)} + 5e6\tau_{B(\theta^\star)}^2\norm{P(\theta^\star)}^{17}\norm{\tilde\theta_2-\theta}^3.
    \end{align*}
    
    For the second term, from \Cref{lem: Riccati perturbation} and robust stabilizability of $G$, we get
    \begin{align*}
        &\inf_{\tilde\theta_1\in G}\sup_{\tilde\theta_2, \theta\in G}2 \norm{P(\theta)}^2 \norm{\Sigma_X^{K(\tilde\theta_1)}(\theta)}\tau_{B(\theta)}^3\norm{K(\tilde\theta_2)-K(\theta)}^3\left(\norm{B(\theta)(K(\tilde\theta_2)-K(\theta))} + 2 \norm{P(\theta)}^{1/2}\right) \\
        &\leq \sup_{\tilde\theta_2, \theta\in G}2\cdot32^3M\tau^3_{B(\theta)}\norm{P(\theta)}^{21/2}\norm{\tilde\theta_2-\theta}^3\left(32\norm{P(\theta)}^{7/2}\norm{\tilde\theta_2-\theta} + 2\norm{P(\theta)}^{1/2}\right) \\
        &\leq 2^{17}M\tau^3_{B(\theta)}\norm{P(\theta)}^{11}\norm{\tilde\theta_2-\theta}^3 + 2^{21}M\tau^3_{B(\theta)}\norm{P(\theta)}^{14}\norm{\tilde\theta_2-\theta}^4.
    \end{align*}
    By applying $\norm{\tilde\theta_2-\theta}\leq\frac{1}{16}\norm{P(\theta)}^{-2}$ and grouping the term, we get
    \begin{align*}
        &C(K(G), \theta^\star) - C(K(\theta^\star, \theta^\star) \leq \sup_{\tilde\theta_2, \theta\in G}\norm{\tilde\theta_2-\theta}_{H(\theta^\star)} + 6e6M\tau^3_{B(\theta^\star)}\norm{P(\theta^\star)}^{17}\norm{\tilde\theta_2-\theta}^3.
    \end{align*}
\end{proof}
    


\subsection{Proof of \texorpdfstring{\Cref{thm: Robust Control upper bound}}{}}
\label{subsec: proof of RC upper bound}
\begin{proof}
    From \Cref{thm: identification bound}, 
    \begin{align*}
        (\hat\theta-\theta^\star)^\top \hat{\mathsf{FI}}N (\hat\theta - \theta^\star) &\leq 8\trace\paren{\hat{\mathsf{FI}} \times \mathsf{FI}(\theta^\star)^{-1}} + 8\norm{\hat{\mathsf{FI}}\times\mathsf{FI}(\theta^\star)^{-1}}\log\frac{2}{\delta} \\
        &\leq 16\paren{d_\theta+\log\frac{2}{\delta}},
    \end{align*}
    where we used $\hat{\mathsf{FI}}\preceq2\mathsf{FI}(\theta^\star)$. Thus $\theta^\star\in G$ with probability at least $1-\delta$. Also, from the robust stabilizability condition and the diameter condition in \Cref{lem: Robust Control Upper Bound}, i.e., $\mathsf{diam}(G)\leq\frac{1}{16}\inf_{\theta\in G}\norm{P(\theta)}^{-2}$, $N$ must satisfy
    \begin{align*}
        &\sup_{\theta_1, \theta_2\in G}\norm{\theta_1-\theta_2}^2
        \leq \frac{32\paren{d_\theta+\log\frac{2}{\delta}}}{N\lambda_{\min}\paren{\hat{\mathsf{FI}}}} 
        \leq \min\curly{\frac{1}{256}\inf_{\theta\in G}\norm{P(\theta)}^{-4}, r^2} \\
        &\iff N \geq \max\curly{\frac{2e4\norm{P(\theta^\star)}^4\paren{d_\theta+\log\frac{2}{\delta}}}{\lambda_{\min}\paren{\mathsf{FI}(\theta^\star)}}, \frac{64 \paren{d_\theta+\log\frac{2}{\delta}}}{\mathsf{r}^2 \lambda_{\min}\paren{\mathsf{FI}(\theta^\star)}}}.
    \end{align*}
    where we applied $0.5\mathsf{FI}(\theta^\star)\preceq\hat{\mathsf{FI}}(\theta^\star)$ and \Cref{lem: Riccati perturbation} to get the last inequality. 
    Furthermore,it follows from \eqref{eq: confidence ellipsoid} that
    \begin{align*}
        \sup_{\theta_1, \theta_2\in G}\norm{\theta_1-\theta_2}_{H\paren{\theta^\star}} 
        &\leq \frac{64\paren{d_\theta+\log\frac{2}{\delta}}\norm{H(\theta^\star)\mathsf{FI}(\theta^\star)^{-1}}}{N}.
    \end{align*}
    which, combined with \Cref{lem: Robust Control Upper Bound}, result in the upper bound in \Cref{thm: Robust Control upper bound}. 
\end{proof}






\section{Proof of Domain Randomization Upper Bound}
\label{s: domain randomization proofs}


We begin our proof with an intermediate results that characterizes the average excess cost incurred by the domain randomized controller relative to the optimal costs of all systems in the randomization domain.  
 
\begin{lemma}
    \label{lem: DR objective upper bound}
    Let $G$ denote the ellipsoid:
    \begin{align*}
        G \triangleq \{\theta ~:~ \theta=\hat \theta + \Gamma w, ~ w\in\mathcal{B}(0, 1)\},
    \end{align*} 
    Suppose the diameter of $G$ satisfies $\mathsf{diam(G)}\leq\frac{1}{256}\inf_{\theta\in G}\norm{P(\theta)}^{-5}$ and $\theta^\star\in G$. 
    Let $\calD$ be a distribution of system parameters over $G$. Then it holds that
    \begin{align*}
        \mathbf{E}_{\theta \sim \calD} 
        \brac{C(K_{\mathsf{DR}}(\calD), \theta) - C(K(\theta), \theta)} \leq \trace\paren{\mathbf{V}(\calD) H(\theta^\star)} + L_{\mathsf{DR,1}}(\theta^\star)  \norm{\Gamma}^3,
    \end{align*}
    where 
    \begin{align*}
        L_{\mathsf{DR,1}}(\theta^\star) = 1.7e8\tau_{B(\theta^\star)}^3\norm{P(\theta^\star)}^{17}.
    \end{align*} 
\end{lemma}
\begin{proof}
    By the definition of the domain randomized controller
    \begin{align*}
        \mathbf{E}_{\theta \sim \calD} 
        \brac{C(K_{\mathsf{DR}}(\calD), \theta) - C(K(\theta), \theta)} &= \min_{K} \mathbf{E}_{\theta \sim \calD} 
        \brac{C(K , \theta) - C(K(\theta), \theta)} \\
        &= \min_{K} \mathbf{E}_{\theta \sim \calD} \trace\paren{(K-K(\theta))\Sigma^K(\theta) (K-K(\theta))^\top \Psi(\theta)}
    \end{align*}
    where the second equality holds from \Cref{lem: performance difference}. 
    Then we can plug in the certainty equivalence controller defined by $K(\hat \theta)$ to achieve an upper bound:
    \begin{align*}
        \mathbf{E}_{\theta \sim \calD} 
        \brac{C(K_{\mathsf{DR}}(\calD), \theta) - C(K(\theta), \theta)} &\leq \mathbf{E}_{\theta \sim \calD} \trace\paren{(K(\hat\theta)-K(\theta))\Sigma^{K(\hat\theta)}(\theta) (K(\hat\theta)-K(\theta))^\top \Psi(\theta)},
    \end{align*}
    where we observe that due to the restricted diameter of $G$, \Cref{lem: CE stabilization} ensures $K(\hat\theta)$ stabilizes all system instances in the support of the distribution. In particular, by substituting the bound of \Cref{lem: CE upper bound} into the above inequality, it holds that
    \begin{align}
        \mathbf{E}_{\theta \sim \calD} 
        \brac{C(K_{\mathsf{DR}}(\calD), \theta) - C(K(\theta), \theta)} \leq \mathbf{E}_{\theta \sim \calD}  \brac{\|\hat\theta - \theta\|_{H(\theta)}^2}  +   \mathbf{E}_{\theta \sim \calD}  \brac{6e5\tau_{B(\theta)}^3\norm{P(\theta)}^{14}\norm{\hat\theta-\theta}^3}. \label{eq: DR objective upper bound with theta}
    \end{align}
    The second term may be bounded by leveraging the radius of the support for $\calD$. In particular, it holds that for all $\theta\in G$, $\norm{\hat \theta - \theta}^3 \leq \norm{\Gamma}^3$. 
    From \Cref{lem: helper lemma for RC}, the first term can be bounded by
    \begin{align*}
        \norm{\hat\theta-\theta}_{H(\theta)} \leq \norm{\hat\theta-\theta}_{H(\theta^\star)} + 5e6\tau_{B(\theta^\star)}^2\norm{P(\theta^\star)}^{17}\norm{\Gamma}^3.
    \end{align*}
    Consequently the expectation evaluates to the trace of the variance of the distribution $\calD$, weighted by $H(\theta^\star)$. Combining these bounds leads to the inequality in the statement. 
\end{proof}

We now leverage the statement about the the domain randomization objective to characterize several system theoretic quantities for the system $\theta^\star$ under the the domain randomized controller. 
\begin{lemma}
    \label{lem: DR helper lemmas}
    Let $G$ denote the ellipsoid:
    \begin{align*}
        G \triangleq \{\theta ~:~ \theta=\hat \theta + \Gamma w, ~ w\in\mathcal{B}(0, 1)\}.
    \end{align*} 
    Suppose the diameter of $G$ satisfies $\mathsf{diam(G)}\leq\frac{1}{256}\inf_{\theta\in G}\norm{P(\theta)}^{-5}$ and $\theta^\star\in G$. 
    Let $\calD$ be a distribution of system parameters over $G$. Let 
    \begin{align*}
        \textnormal{DR Objective } \triangleq  \bfE_{\theta\sim\calD} \brac{\trace\paren{(K_{\mathsf{DR}}(\calD) - K(\theta)) \Sigma^{K_{\mathsf{DR}}(\calD)}(\theta)  (K_{\mathsf{DR}}(\calD) - K(\theta)) \Psi(\theta)}}.
    \end{align*}
    It holds that 
    \begin{itemize}
        \item $\bfE_{\theta\sim\calD} \norm{K_{\mathsf{DR}}(\calD)- K(\theta)}^2 \leq \textnormal{DR Objective}$
        \item $\norm{K_{\mathsf{DR}}(\calD) - K(\theta^\star)} \leq \textnormal{DR Objective} + 32 \norm{P(\theta^\star)}^{7/2} \norm{\Gamma}$
        \item $\norm{K_{\mathsf{DR}}(\calD)}\leq \textnormal{DR Objective} + 2\norm{P(\theta^\star)}^{1/2}$ 
        \item $\textnormal{DR Objective} \leq 2.6e3\tau^3_{B(\theta^\star)}\norm{P(\theta^\star)}^4 \norm{\Gamma}$
        \item $\norm{\Sigma^{K_{\mathsf{DR}}(\calD)}(\theta)}\leq2\norm{\Sigma^{K_{\mathsf{DR}}(\calD)}(\theta^\star)}$ if $\mathsf{diam}(G)\leq\frac{1}{8\norm{\Sigma^{K_{\mathsf{DR}}}(\calD)(\theta^\star)}^{3/2}\paren{1+\norm{K_{\mathsf{DR}}}}^2}$
        \item $\norm{\Sigma^{K_{\mathsf{DR}}(\calD)}(\theta^\star)} 
        \leq 2\norm{\Sigma^{K(\theta^\star)}(\theta^\star)}$ if $\mathsf{diam}(G)\leq\frac{1}{6.4e4\tau^8_{B(\theta^\star)}\norm{P(\theta^\star)}^4\norm{\Sigma^{K(\theta^\star)}(\theta^\star)}^{3/2}}$. 
    \end{itemize}
\end{lemma}
\begin{proof}
    The first fact follows immediately from \Cref{lem: performance difference} along with the fact that $\Sigma^{K_{\mathsf{DR}}(\calD)}(\theta) \succeq I$ and $\Psi(\theta) \succeq I$. The second fact is derived from the first fact and \Cref{lem: Riccati perturbation}, which also yields the third fact using the diameter condition of $G$. 
    DR Objective is bounded as
    \begin{align*}
        \textnormal{DR Objective} &\leq \sup_{\theta\in G}\norm{\hat\theta-\theta}_{H(\theta)}^2 + 1.6e8\tau_{B(\theta^\star)}^3\norm{P(\theta^\star)}^{14}\norm{\hat\theta-\theta}^3 \\
        &\leq \paren{32\tau^2_{B(\theta^\star)}+ 2.5e3\tau^3_{B(\theta^\star)}}\norm{P(\theta^\star)}^4 \norm{\Gamma} 
    \end{align*}
    from \Cref{lem: helper lemma for RC} and the diameter condition. 
    It follows from \Cref{lem: lyap perturbation} that
    \begin{align*}
        &\bfE_{\theta\sim\calD}\norm{\Sigma^{K_{\mathsf{DR}}(\calD)}(\theta^\star) - \Sigma^{K_{\mathsf{DR}}(\calD)}(\theta)} \\
        &\leq \sup_{\theta\in G}\norm{\Sigma^{K_{\mathsf{DR}}(\calD)}(\theta^\star)}^{3/2}\norm{\Sigma^{K_{\mathsf{DR}}(\calD)}(\theta)}\norm{\theta-\theta^\star}\paren{1+\norm{K_{\mathsf{DR}}(\theta)}}\paren{2+\norm{\theta-\theta^\star}\paren{1+\norm{K_{\mathsf{DR}}(\theta)}}}.
    \end{align*}
    Let $\zeta(\theta) = \norm{\theta-\theta^\star}\paren{1+\norm{K_{\mathsf{DR}}(\theta)}}$, then we get the fifth fact: 
    \begin{align*}
        &\norm{\Sigma^{K_{\mathsf{DR}}(\calD)(\theta)}} 
        \leq \sup_{\theta\in G}\norm{\Sigma^{K_{\mathsf{DR}}(\calD)(\theta^\star)}} + \norm{\Sigma^{K_{\mathsf{DR}}(\calD)}(\theta^\star)}^{3/2}\norm{\Sigma^{K_{\mathsf{DR}}(\calD)}(\theta)}\zeta(\theta)(2+\zeta(\theta)) \\
        &\iff \norm{\Sigma^{K_{\mathsf{DR}}(\calD)}(\theta)} 
        \leq \sup_{\theta\in G}\frac{\norm{\Sigma^{K_{\mathsf{DR}}(\calD)}(\theta^\star)}}{1-\norm{\Sigma^{K_{\mathsf{DR}}}(\calD)(\theta^\star)}^{3/2}\zeta(\theta)(2+\zeta(\theta))} \leq 2\norm{\Sigma^{K_{\mathsf{DR}}(\calD)}(\theta^\star)}, 
    \end{align*}
    where the last inequality holds as long as $\norm{\Gamma}\leq\frac{1}{8\norm{\Sigma^{K_{\mathsf{DR}}}(\calD)(\theta^\star)}^{3/2}\paren{1+\norm{K_{\mathsf{DR}}}}^2}$. 

    Similarly, from \Cref{lem: lyap perturbation}, it holds that
    \begin{align*}
        &\norm{\Sigma^{K_{\mathsf{DR}}(\calD)}(\theta^\star) - \Sigma^{K(\theta^\star)}(\theta^\star)} \\
        &\leq \norm{\Sigma^{K(\theta^\star)}(\theta^\star)}^{3/2}\norm{\Sigma^{K_{\mathsf{DR}}(\calD)}(\theta^\star)}\norm{B(\theta^\star)\paren{K_{\mathsf{DR}}(\calD) - K(\theta^\star)}}\paren{2 + \norm{B(\theta^\star)\paren{K_{\mathsf{DR}}(\calD) - K(\theta^\star)}}} 
    \end{align*}  
    Let $\eta = \norm{B(\theta^\star)\paren{K_{\mathsf{DR}}(\calD) - K(\theta^\star)}}$, Then
    \begin{align*}
        & \norm{\Sigma^{K_{\mathsf{DR}}(\calD)}(\theta^\star)}\leq  \norm{\Sigma^{K(\theta^\star)}(\theta^\star)}+ \norm{\Sigma^{K(\theta^\star)}(\theta^\star)}^{3/2}\norm{\Sigma^{K_{\mathsf{DR}}(\calD)}(\theta^\star)}\eta(2+\eta) \\
        &\iff \norm{\Sigma^{K_{\mathsf{DR}}(\calD)}(\theta^\star)} \leq \frac{\norm{\Sigma^{K(\theta^\star)}(\theta^\star)}}{1-\norm{\Sigma^{K(\theta^\star)}(\theta^\star)}^{3/2}\eta(2+\eta)} \leq 2\norm{\Sigma^{K_{\mathsf{DR}}(\calD)}(\theta^\star)} 
    \end{align*}
    where the last inequality holds when
    \begin{align*}
        \eta(2+\eta) \leq \frac{1}{2\norm{\Sigma^{K(\theta^\star)}(\theta^\star)}^{3/2}}
    \end{align*}
    Now from the second and fourth fact, \eqref{eq: DR objective upper bound with theta} and \Cref{lem: helper lemma for RC}, $\eta$ is bounded as
    \begin{align*}
        &\eta = \norm{B(\theta^\star)\paren{K_{\mathsf{DR}}(\calD) - K(\theta^\star)}} \\
        &\leq \sup_{\theta\in G}\tau_{B(\theta^\star)}\paren{\textnormal{DR Objective} + 32 \norm{P(\theta^\star)}^{7/2} \norm{\theta-\theta^\star}} \\
        &\leq 2.5e3\tau^3_{B(\theta^\star)}\norm{P(\theta^\star)}^4 \norm{\Gamma} + 32\tau_{B(\theta^\star)}\norm{P(\theta^\star)}^{7/2}\norm{\theta-\theta^\star} = 2.6e3\tau^4_{B(\theta^\star)}\norm{P(\theta^\star)}^4\norm{\Gamma} \\
        &\iff \eta^2+2\eta \leq 3.2e4\tau^8_{B(\theta^\star)}\norm{P(\theta^\star)}^4\norm{\Gamma}
    \end{align*}
    where we applied $\mathsf{diam(G)}\leq\frac{1}{256}\inf_{\theta\in G}\norm{P(\theta)}^{-5}$ multiple times.
    Therefore the sixth inequality holds as long as
    \begin{align*}
        \norm{\Gamma} \leq \frac{1}{6.4e4\tau^8_{B(\theta^\star)}\norm{P(\theta^\star)}^4\norm{\Sigma^{K(\theta^\star)}(\theta^\star)}^{3/2}}
    \end{align*}
\end{proof}

\begin{lemma}
    \label{lem: DR suboptimality upper bound}
    Let $G$ denote the ellipsoid:
    \begin{align*}
        G \triangleq \{\theta ~:~ \theta=\hat \theta + \Gamma w, ~ w\in\mathcal{B}(0, 1)\},
    \end{align*} 
    Suppose the diameter of $G$ satisfies $\mathsf{diam}(G)\leq\frac{1}{6.4e4}\inf_{\theta\in G}\norm{P(\theta)}^{-5.5} \tau_{B(\theta^\star)}^{-8}$ and $\theta^\star\in G$. 
    Let $\calD$ be a distribution of system parameters over $G$.  Then it holds that
     \begin{align*}
        &C(K_{\mathsf{DR}}(\calD), \theta^\star) - C(K(\theta^\star), \theta^\star) \leq 2\norm{\hat\theta-\theta^\star}^2_{H(\theta^\star)} + 4\trace\paren{\mathbf{V}(\calD) H(\theta^\star)}+ L_{\mathsf{DR}}(\theta^\star)\norm{\Gamma}^3
    \end{align*}
    where
    \begin{align*}
        L_{\mathsf{DR}}(\theta^\star) = 5e8\du\tau_{B(\theta^\star)}^8\norm{P(\theta^\star)}^{17}
    \end{align*}
\end{lemma}

\begin{proof}
    By the fact that $K_{\mathsf{DR}}(\calD)$ stabilizes the system $\theta^\star$, we may write by \Cref{lem: performance difference}
    \begin{equation}
    \begin{aligned}
        \label{eq: DR excess cost decomposiotion}
        &C(K_{\mathsf{DR}}(\calD), \theta^\star) - C(K(\theta^\star), \theta^\star) = \trace\paren{(K_{\mathsf{DR}}(\calD) - K(\theta^\star)) \Sigma^{K_{\mathsf{DR}}(\calD)}(\theta^\star)  (K_{\mathsf{DR}}(\calD) - K(\theta^\star)) \Psi(\theta^\star)} \\
        &\leq 2 \bfE_{\theta\sim\calD} \brac{\trace\paren{(K_{\mathsf{DR}}(\calD) - K(\theta)) \Sigma^{K_{\mathsf{DR}}(\calD)}(\theta^\star)  (K_{\mathsf{DR}}(\calD) - K(\theta)) \Psi(\theta^\star)}} \\
        &+ 2 \bfE_{\theta\sim\calD} \brac{\trace\paren{(K(\theta) - K(\theta^\star)) \Sigma^{K_{\mathsf{DR}}(\calD)}(\theta^\star)  (K(\theta) - K(\theta^\star)) \Psi(\theta^\star)}},
    \end{aligned}
    \end{equation}
    where the inequality follows from the Cauchy-Schwarz. Next, we massage the first term to have the form of the domain randomization objective, and the second term to have the form of the certainty equivalence objective. To this end, first note that by \Cref{lem: helper lemma for RC} and $\Psi(\theta)\succeq I$, we get
    \begin{align*}
        \Psi(\theta^\star) \preceq \Psi(\theta)\paren{1 + 15 \tau_{B(\theta^\star)}^2 \norm{P(\theta^\star)}^3 \norm{\theta - \theta^\star}}.
    \end{align*}
    Additionally, write expand 
    \begin{align*}
       \Sigma^{K_{\mathsf{DR}}(\calD)}(\theta^\star) &\preceq \Sigma^{K_{\mathsf{DR}}(\calD)}(\theta) + I \times \norm{\Sigma^{K_{\mathsf{DR}}(\calD)}(\theta^\star) - \Sigma^{K_{\mathsf{DR}}(\calD)}(\theta)}  \mbox{ and }\\
       \Sigma^{K_{\mathsf{DR}}(\calD)}(\theta^\star) &= \Sigma^{K(\theta^\star)}(\theta^\star) + I \times \norm{\Sigma^{K_{\mathsf{DR}}(\calD)}(\theta^\star) - \Sigma^{K(\theta^\star)}(\theta^\star)}.
    \end{align*}
    Substituting the above set of inequalities into \eqref{eq: DR excess cost decomposiotion} we achieve the bound 
    \begin{align}
        &C(K_{\mathsf{DR}}(\calD), \theta^\star) - C(K(\theta^\star), \theta^\star) \nonumber \\
        &\leq 2 \times  \mbox{DR objective}\times \paren{1 + 15 \tau_{B(\theta^\star)}^2 \norm{P(\theta^\star)}^3 \norm{\theta - \theta^\star}} \nonumber\\
        &+ 2 \times \mbox{Expected CE cost gap} \nonumber\\
        & + 2 \du \norm{\Psi(\theta^\star)} \bfE_{\theta\sim\calD}\brac{\norm{\Sigma^{K_{\mathsf{DR}}(\calD)}(\theta^\star) - \Sigma^{K_{\mathsf{DR}}(\calD)}(\theta)} \norm{K_{\mathsf{DR}}(\calD) - K(\theta)}^2} \nonumber\\
        &+ 2 \du \norm{\Psi(\theta^\star)} \norm{\Sigma^{K_{\mathsf{DR}}(\calD)}(\theta^\star) - \Sigma^{K(\theta^\star)}(\theta^\star)}\bfE_{\theta\sim\calD}\brac{ \norm{K(\theta^\star) - K(\theta)}^2}, \label{eq:DR suboptimality gap}
    \end{align}
    where 
    \begin{align*}
        &\mbox{Expected CE cost gap}  = \bfE_{\theta\sim\calD} \brac{\trace\paren{(K(\theta) - K(\theta^\star)) \Sigma^{K(\theta^\star)}(\theta^\star)  (K(\theta) - K(\theta^\star)) \Psi(\theta^\star)}}, \\
         &\mbox{DR Objective }=  \bfE_{\theta\sim\calD} \brac{\trace\paren{(K_{\mathsf{DR}}(\calD) - K(\theta)) \Sigma^{K_{\mathsf{DR}}(\calD)}(\theta)  (K_{\mathsf{DR}}(\calD) - K(\theta)) \Psi(\theta)}}.
    \end{align*}
    We let $D(\theta)$ denote the DR objective.
    From \Cref{lem: DR objective upper bound}, the first term of \eqref{eq:DR suboptimality gap} becomes
    \begin{align}
        &2\times D(\theta)\times \paren{1 + 15 \tau_{B(\theta^\star)}^2 \norm{P(\theta^\star)}^3 \norm{\theta - \theta^\star}}\nonumber\\
        &\leq 2 \times \paren{\trace\paren{\mathbf{V}(\calD) H(\theta^\star)} + L_{\mathsf{DR,1}}(\theta^\star)  \norm{\Gamma}^3} \times \paren{1 + 15 \tau_{B(\theta^\star)}^2 \norm{P(\theta^\star)}^3\norm{\theta - \theta^\star}} \nonumber\\
        &\leq 2\trace\paren{\mathbf{V}(\calD) H(\theta^\star)} + 3.5e8\tau_{B(\theta^\star)}^5\norm{P(\theta^\star)}^{17}\norm{\Gamma}^3 \label{eq: first term of DR suboptimality gap}
    \end{align}
    where we used \Cref{lem: helper lemma for RC} and applied the closeness condition to $\norm{\theta-\theta^\star}$ in the last inequality. 
    Next the second term of \eqref{eq:DR suboptimality gap} follows from \Cref{lem: CE upper bound} that
    \begin{align}
        2 \times \mbox{Expected CE cost gap} \leq \bfE_{\theta\sim\calD}\brac{2\norm{\theta-\theta^\star}^2_{H(\theta^\star)}} + 1.2e6\tau_{B(\theta^\star)}^3\norm{P(\theta^\star)}^{14}\norm{\Gamma}^3 \label{eq:second term of DR suboptimality gap}
    \end{align}
    Then from the fifth fact of \Cref{lem: DR helper lemmas}, 
    \begin{align*}
        &\bfE_{\theta\sim\calD}\norm{\Sigma^{K_{\mathsf{DR}}(\calD)}(\theta^\star) - \Sigma^{K_{\mathsf{DR}}(\calD)}(\theta)} \\
        &\leq\sup_{\theta\in G}3\norm{\Sigma^{K(\theta^\star)}(\theta^\star)}^{5/2}\paren{1+\norm{K_{\mathsf{DR}}(\calD)}}\norm{\theta-\theta^\star}\paren{2 + \frac{1}{8\norm{\Sigma^{K_{\mathsf{DR}}}(\calD)(\theta^\star)}^{3/2}\paren{1+\norm{K_{\mathsf{DR}}(\calD)}}}}
        \\
        &\leq \sup_{\theta\in G}5\norm{\Sigma^{K(\theta^\star)}(\theta^\star)}^{5/2}\paren{1+\norm{K_{\mathsf{DR}}(\calD)}}\norm{\theta-\theta^\star}
    \end{align*}
    since $\norm{\Sigma^{K(\theta^\star)}(\theta^\star)} \geq 1$. Also from the sixth fact of \Cref{lem: DR helper lemmas}
    \begin{align*}
        &\norm{\Sigma^{K_{\mathsf{DR}}(\calD)}(\theta^\star) - \Sigma^{K(\theta^\star)}(\theta^\star)} \leq \sup_{\theta\in G}6.4e4\norm{P(\theta^\star)}^{13/2}\norm{\theta-\theta^\star}
    \end{align*}
    By combining with \Cref{lem: Riccati perturbation}, the third term of \eqref{eq:DR suboptimality gap} becomes
    \begin{align}
        &2 \du \norm{\Psi(\theta^\star)} \bfE_{\theta\sim\calD}\brac{\norm{\Sigma^{K_{\mathsf{DR}}(\calD)}(\theta^\star) - \Sigma^{K_{\mathsf{DR}}(\calD)}(\theta)} \norm{K_{\mathsf{DR}}(\calD) - K(\theta)}^2} \nonumber \\
        &\leq \du \norm{\Psi(\theta^\star)}  \paren{8e4\tau_{B(\theta^\star)}^2\paren{1+\norm{K_{\mathsf{DR}}(\theta)}}\norm{P(\theta^\star)}^{23/2} + 3.4e8\tau_{B(\theta^\star)}^3\norm{P(\theta^\star)}^{39/2}\norm{\Gamma}}\norm{\Gamma}^3 \nonumber  \\
        &\leq 5e6\du\tau^8_{B(\theta)}\norm{P(\theta^\star)}^{33/2}\norm{\Gamma}^3 \label{eq:third term of DR suboptimality gap}
    \end{align}
    where the last inequality follows from \Cref{lem: simplifying inequalities} and noting that $1+\norm{K_{\mathsf{DR}}(\theta)}\leq12\tau_{B(\theta^\star)}^4\norm{P(\theta^\star)}^{1/2}$ from \Cref{lem: DR helper lemmas}.
    Furthermore, the fourth term of \eqref{eq:DR suboptimality gap} is given by
    \begin{align}
        &2 \du \norm{\Psi(\theta^\star)} \norm{\Sigma^{K_{\mathsf{DR}}(\calD)}(\theta^\star) - \Sigma^{K(\theta^\star)}(\theta^\star)}\bfE_{\theta\sim\calD}\brac{ \norm{K(\theta^\star) - K(\theta)}^2} \leq 1.4e8 \du \tau^2_{B(\theta)}\norm{P(\theta^\star)}^{31/2}\norm{\Gamma}^3 \label{eq:fourth term of DR suboptimality gap}
    \end{align}
    where the last inequatliy follows from \Cref{lem: Riccati perturbation} and \Cref{lem: simplifying inequalities}. 
    Finally, from \eqref{eq: first term of DR suboptimality gap}, \eqref{eq:second term of DR suboptimality gap}, \eqref{eq:third term of DR suboptimality gap} and \eqref{eq:fourth term of DR suboptimality gap}, we get
    \begin{align*}
         &C(K_{\mathsf{DR}}(\calD), \theta^\star) - C(K(\theta^\star), \theta^\star) \\
         &\leq 2\bfE_{\theta\sim\calD}{\norm{\theta-\theta^\star}^2_{H(\theta^\star)}} + 2\trace\paren{\mathbf{V}(\calD) H(\theta^\star)} + L_{\mathsf{DR}}(\theta^\star)\norm{\Gamma}^3
    \end{align*}
    where
    \begin{align*}
        L_{\mathsf{DR}}(\theta^\star) &= \paren{3.5e8\tau_{B(\theta^\star)}^5 + 1.2e6\tau_{B(\theta^\star)}^3 + 5e6\du\tau^8_{B(\theta)} +  1.4e8\du \tau^2_{B(\theta)}}\norm{P(\theta^\star)}^{17}\\
        &= 5e8\du\tau_{B(\theta^\star)}^8\norm{P(\theta^\star)}^{17}
    \end{align*}
    We conclude the proof by noting that
    \begin{align*}
        2\bfE_{\theta\sim\calD}\norm{\theta-\theta^\star}^2_{H(\theta^\star)} = 
        2\norm{\hat\theta - \theta^\star}_{H(\theta^\star)}^2 + 
        2\trace\paren{\mathbf{V}(\calD) H(\theta^\star)}.
    \end{align*}
\end{proof}

\subsection{Proof of \texorpdfstring{\Cref{lem: domain randomization general}}{}}
\label{subsec: Proof of DR upper bound}
\begin{proof}
    As proved in \Cref{subsec: proof of RC upper bound}, $\theta^\star\in G$ with probability at least $1-\delta$. 
    From the diameter condition in \Cref{lem: DR suboptimality upper bound}, i.e., $\mathsf{diam(G)}\leq\frac{1}{6.4e4}\inf_{\theta\in G}\norm{P(\theta)}^{-5.5} \tau_{B(\theta^\star)}^{-8}$, $N$ must satisfy
    \begin{align*}
        & \sup_{\theta_1, \theta_2\in G}\norm{\theta_1-\theta_2}^2\leq \frac{32\paren{d_\theta+\log\frac{2}{\delta}}}{N\lambda_{\min}\paren{\hat{\mathsf{FI}}}} \leq \frac{1}{(6.4e4)^2}\inf_{\theta\in G}\norm{P(\theta)}^{-11} \tau_{B(\theta^\star)}^{-16}.  \\
        &\iff N \geq \frac{5e6\norm{P(\theta^\star)}^{11} \tau_{B(\theta^\star)}^{16}\paren{d_\theta+\log\frac{2}{\delta}}}{\lambda_{\min}\paren{\mathsf{FI}(\theta^\star)}}
    \end{align*} 
    where we applied $0.5\mathsf{FI}(\theta^\star)\preceq\hat{\mathsf{FI}}$ and \Cref{lem: Riccati perturbation} to get the last inequality.
    As long as this holds, we get \eqref{eq:DR upper bound} from \Cref{thm: identification bound} and \Cref{lem: DR suboptimality upper bound}. 
\end{proof}

\subsection{Proof of \texorpdfstring{\Cref{thm: Domain Randomization Upper Bound}}{}}
\label{subsec: Proof of DR upper bound w/ uniform distribution}
\begin{proof}
    Since the variance of uniform distribution over the ellipsoid $\curly{w \colon w^T\Gamma w\leq1, w\in\R^d}$ is given as $\frac{1}{d+2}\Gamma^{-1}$, 
    \begin{align*}
        \trace(V(\calD)H(\theta^\star)) &\leq \frac{16\paren{d_\theta+\log\frac{2}{\delta}}}{Nd_\theta}\trace\paren{\hat{\mathsf{FI}}^{-1}H(\theta^\star)} \leq \frac{32\paren{1+\log\frac{2}{\delta}/d_\theta}}{N}\trace\paren{{\mathsf{FI}}(\theta^\star)^{-1}H(\theta^\star})
    \end{align*}
    where we used $0.5\hat{\mathsf{FI}}\preceq {\mathsf{FI}}(\theta^\star)$ in the last inequality. Then from \Cref{lem: domain randomization general}, we get \eqref{eq:DR upper bound w/ unif dist}. 
\end{proof}


\section{Implementation Details}
\label{s: implementation details}
\subsection{Linear System}
We explain the details about the linear system experiments in \Cref{s: numerical}. The goal is to compare the three controller synthesis methods considered in this work (\Cref{s: methods}) to study show how the suboptimality gap of the LQR cost changes. 
All the code is implemented in JAX \citep{jax2018github}. 
First we identify the least square estimate $\hat\theta$ by solving \eqref{eq: least squares} given collected data by running a variable number of experiments with $U_t\sim\mathcal{N}(0, I)$ and $W_t\sim\mathcal{N}(0, I)$, and followed by the controller design. 
\begin{itemize}
\item 
For the certainty equivalence controller synthesis, we simply synthesize a controller by solving the \textit{Algebraic Ricatti Equation}. 
\item 
For the domain randomization controller synthesis, we implement \Cref{alg: dr lqr} via stochastic gradient descent. 
We choose $\mathcal{D}$ as a uniform distribution over the confidence ellipsoid $G$. We set $M = 10000$ and $\eta=0.0005$ in \Cref{alg: dr lqr}. We use the closed-form expression of the policy gradient as described in Lemma $1$ of \cite{fazel2018global} to implement gradient descent. 
\item 
For the robust controller synthesis, we adopt the scenario-approach \citep{calafiore2006scenario}, where we sample 30 points from the ellipsoid $G$ \eqref{eq: confidence ellipsoid}, by formulating as the semidefinite programming problem with linear matrix inequalities \citep{caverly2019lmi} and solve the convex optimization problem using CVXPY \citep{diamond2016cvxpy} using the SCS solver \citep{ocpb:16}. 
\end{itemize}

Since these randomization procedures result in high variance, we perform $500$ trials and report the median and quartiles over those trials in \Cref{fig:result dr lqr}. 

\subsection{Pendulum}
The pendulum is goverend by the following dynamics
\begin{align*}
    X_{t+1} = f(X_t, U_t+W_t; \theta^\star),
\end{align*}
where the state space $X_t\in\R^2$ consists of angle $\psi_t$(rad) and angular velocity $\dot\psi_t$(rad/s), and the action $U_t\in\R$ is the actuation torque $\tau_t$(N m) applied directly to the free end of the pendulum, which is corrupted by i.i.d Gaussian noise $W_t \sim \mathcal{N}(0, 1)$. The unknown parameter $\theta\in\R^3$ accounts for the mass $m$ and the length $l$ of the pole, and the gravitiy $g$, which take values $1.0$, $1.0$, and $9.81$, respectively.  

As in the linear experiments, first we identify the estimate $\hat\theta$ by solving the following least squares problem:
\begin{align*}
    \hat\theta \triangleq \argmin_\theta \sum_{n=1}^N\sum_{t=1}^T\norm{X_{t+1}^n - f(X_t, U_t; \theta)}^2.
\end{align*}
The data consists of varying numbers of length $10$ trajectories collected from the pendulum starting from the downward position ($\theta = \pi, \dot \theta =0$) with $U_t\sim\mathcal{N}(0, 1)$ and $W_t\sim\mathcal{N}(0, 1)$ 
This estimation procedure is followed by the sampling-based Model Predictive Controller synthesis based on the cross-entropy method \citep{botev2013cross} with the different design methods.
\begin{itemize}
    \item For the certainty equivalence control, actions are selected to  minimize the cost of the rollout in the receding horizon which is simulated with $\hat\theta$ as if it were the ground truth.
    \item For the domain randomization control, we first design a sampling distribution $\calD$ as a uniform distribution over a sphere centered at the estimate $\hat \theta$, with radius given by a hyperparameter $2.0$ divided by the number of trajectories used to identify $\theta$. We then sample $K=15$ instances $\theta_1, \dots, \theta_{K} \in\mathcal{D}$. The controller then plans sequences of actions which minimize the average cost of all the rollouts simulated with $\theta_1, \dots, \theta_{K}$. 
\end{itemize}
We set the stage cost $C(X_t, U_t)$ as
\begin{align*}
    C(X_t, U_t) = \left\{\begin{array}{ll}
        \psi_t^2 + 0.1\dot\psi_t^2 + 2.0\tau_t^2 & -\pi/4 \leq \psi \leq \pi/4 \\
        50.0 & \text{otherwise}
    \end{array}\right.
\end{align*}
This cost design promotes the controller to keep the pendulum upright using the minimal control effort. CE with an inaccurate parameter estimate might underestimate control effort required to avoid letting the pendulum fall.
On the other hand, DR plans such that it keeps the pendulum upright for all sampled systems $\theta_1, \dots, \theta_K$, therefore is able to keep upright position even when the estimates are poor.

For other hyperparameters and detailed implementation, refer to the code \footnote{Codes can be found at \url{https://github.com/Tesshuuuu/domain-randomization-l4dc2025}}.


