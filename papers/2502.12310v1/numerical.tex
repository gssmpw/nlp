\section{Numerical Experiments}
\label{s: numerical}


%In this example, $\mathcal{D}$ is chosen as a uniform distribution over the ellipsoid with a shape determined by $\mathsf{FI}(\hat \theta)$ and the scale determined by our desired confidence that the ellipsoid contains $\theta^\star$.  
% \Tesshu{Note that in this algorithm, we chose $\mathcal{D}$ as a uniform distribution that partially covers a normal distribution $\mathcal{N}(\hat\theta, \mathsf{FI}(\hat\theta))$, which is estimated by the least square identification using the dataset.}

\subsection{Linear System}

\begin{wrapfigure}[9]{r}{0.5\textwidth}
    \centering
    \vspace{-25pt} % Adjust spacing if needed
    \includegraphics[width=\linewidth]{figures/EC_comparison.pdf}
    \vspace{-21pt} % Adjust spacing if needed
    \caption{Excess cost of controllers found via three methods using models fit with various amounts of data.}
    \label{fig:result dr lqr}
\end{wrapfigure}
% \vspace{30pt}
\renewcommand{\arraystretch}{1.0}
We validate the trends predicted in \Cref{tab:sample_efficiency} through a case study on the linear system 
\vspace{-3pt}
\begin{align}
    &A = \bmat{
        1.01 & 0.01 & 0 \\
        0.01 & 1.01 & 0.01 \\
        0 & 0.01 & 1.01}, \nonumber\\
    &B = I, Q = 10^{-3}I, R = I. \label{eq:numerical experiments params}
\end{align}
We first estimate $A$ and $B$ using least squares identification, then synthesize CE, DR, and RC controllers based on the identified models. Further details are provided in \Cref{s: implementation details}.


In \Cref{fig:result dr lqr}, we plot the median and shade 25 \% to 75 \% quantile over 500 random seeds. 
This result reveals two key observations. First, both DR and RC stabilize the system with fewer experiments than CE. While the improved performance of DR in the low-data regime lacks concrete theoretical justification, \Cref{thm: Robust Control upper bound} supports this trend for RC. Second, after an initial period, DR converges faster than RC, eventually matching the convergence rate of CE. This aligns with the conclusion of \Cref{thm: Domain Randomization Upper Bound} and is consistent with the conceptual illustration in \Cref{fig:conceptual figure}.


\subsection{Pendulum}
\begin{wrapfigure}{r}{0.5\textwidth}
    \centering
    \vspace{-90pt}
    \includegraphics[width=\linewidth]{figures/pendulum.pdf}
    \vspace{-90pt}
    \caption{Cost of CE and DR controllers based on models fit with various amounts of data.}
    \label{fig:pendulum}    
\end{wrapfigure}

We extend the approach to nonlinear systems of pendulum by applying receding horizon control. For CE, planning occurs by minimizing a finite horizon cost for a trajectory planned using the nominal system estimate. For domain randomization, planning occurs by minimizing the average finite horizon cost for trajectories planned with a collection of systems sampled from a distribution around the system estimate. 

The result is shown in \Cref{fig:pendulum}, which plots the mean and standard error over 100 random seeds. As observed in the linear system, the trends that (i) DR outperforms CE in the low-data regime and (ii) the convergence rate of CE and DR matches hold even for the nonlinear system. 
Implementation details are deferred to the \Cref{s: implementation details}.
