% \vspace{-0pt}
\begin{abstract} 
   We study the sample efficiency of domain randomization and robust control for the benchmark problem of learning the linear quadratic regulator (LQR). Domain randomization, which synthesizes controllers by minimizing average performance over a distribution of model parameters, has achieved empirical success in robotics, but its theoretical properties remain poorly understood. We establish that with an appropriately chosen sampling distribution, domain randomization achieves the optimal asymptotic rate of decay in the excess cost, matching certainty equivalence. We further demonstrate that robust control, while potentially overly conservative, exhibits superior performance in the low-data regime due to its ability to stabilize uncertain systems with coarse parameter estimates. We propose a gradient-based algorithm for domain randomization that performs well in numerical experiments, which enables us to validate the trends predicted by our analysis. These results provide insights into the use of domain randomization in learning-enabled control, and highlight several open questions about its application to broader classes of systems. 
\end{abstract}