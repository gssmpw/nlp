\section{Proof of Domain Randomization Upper Bound}
\label{s: domain randomization proofs}


We begin our proof with an intermediate results that characterizes the average excess cost incurred by the domain randomized controller relative to the optimal costs of all systems in the randomization domain.  
 
\begin{lemma}
    \label{lem: DR objective upper bound}
    Let $G$ denote the ellipsoid:
    \begin{align*}
        G \triangleq \{\theta ~:~ \theta=\hat \theta + \Gamma w, ~ w\in\mathcal{B}(0, 1)\},
    \end{align*} 
    Suppose the diameter of $G$ satisfies $\mathsf{diam(G)}\leq\frac{1}{256}\inf_{\theta\in G}\norm{P(\theta)}^{-5}$ and $\theta^\star\in G$. 
    Let $\calD$ be a distribution of system parameters over $G$. Then it holds that
    \begin{align*}
        \mathbf{E}_{\theta \sim \calD} 
        \brac{C(K_{\mathsf{DR}}(\calD), \theta) - C(K(\theta), \theta)} \leq \trace\paren{\mathbf{V}(\calD) H(\theta^\star)} + L_{\mathsf{DR,1}}(\theta^\star)  \norm{\Gamma}^3,
    \end{align*}
    where 
    \begin{align*}
        L_{\mathsf{DR,1}}(\theta^\star) = 1.7e8\tau_{B(\theta^\star)}^3\norm{P(\theta^\star)}^{17}.
    \end{align*} 
\end{lemma}
\begin{proof}
    By the definition of the domain randomized controller
    \begin{align*}
        \mathbf{E}_{\theta \sim \calD} 
        \brac{C(K_{\mathsf{DR}}(\calD), \theta) - C(K(\theta), \theta)} &= \min_{K} \mathbf{E}_{\theta \sim \calD} 
        \brac{C(K , \theta) - C(K(\theta), \theta)} \\
        &= \min_{K} \mathbf{E}_{\theta \sim \calD} \trace\paren{(K-K(\theta))\Sigma^K(\theta) (K-K(\theta))^\top \Psi(\theta)}
    \end{align*}
    where the second equality holds from \Cref{lem: performance difference}. 
    Then we can plug in the certainty equivalence controller defined by $K(\hat \theta)$ to achieve an upper bound:
    \begin{align*}
        \mathbf{E}_{\theta \sim \calD} 
        \brac{C(K_{\mathsf{DR}}(\calD), \theta) - C(K(\theta), \theta)} &\leq \mathbf{E}_{\theta \sim \calD} \trace\paren{(K(\hat\theta)-K(\theta))\Sigma^{K(\hat\theta)}(\theta) (K(\hat\theta)-K(\theta))^\top \Psi(\theta)},
    \end{align*}
    where we observe that due to the restricted diameter of $G$, \Cref{lem: CE stabilization} ensures $K(\hat\theta)$ stabilizes all system instances in the support of the distribution. In particular, by substituting the bound of \Cref{lem: CE upper bound} into the above inequality, it holds that
    \begin{align}
        \mathbf{E}_{\theta \sim \calD} 
        \brac{C(K_{\mathsf{DR}}(\calD), \theta) - C(K(\theta), \theta)} \leq \mathbf{E}_{\theta \sim \calD}  \brac{\|\hat\theta - \theta\|_{H(\theta)}^2}  +   \mathbf{E}_{\theta \sim \calD}  \brac{6e5\tau_{B(\theta)}^3\norm{P(\theta)}^{14}\norm{\hat\theta-\theta}^3}. \label{eq: DR objective upper bound with theta}
    \end{align}
    The second term may be bounded by leveraging the radius of the support for $\calD$. In particular, it holds that for all $\theta\in G$, $\norm{\hat \theta - \theta}^3 \leq \norm{\Gamma}^3$. 
    From \Cref{lem: helper lemma for RC}, the first term can be bounded by
    \begin{align*}
        \norm{\hat\theta-\theta}_{H(\theta)} \leq \norm{\hat\theta-\theta}_{H(\theta^\star)} + 5e6\tau_{B(\theta^\star)}^2\norm{P(\theta^\star)}^{17}\norm{\Gamma}^3.
    \end{align*}
    Consequently the expectation evaluates to the trace of the variance of the distribution $\calD$, weighted by $H(\theta^\star)$. Combining these bounds leads to the inequality in the statement. 
\end{proof}

We now leverage the statement about the the domain randomization objective to characterize several system theoretic quantities for the system $\theta^\star$ under the the domain randomized controller. 
\begin{lemma}
    \label{lem: DR helper lemmas}
    Let $G$ denote the ellipsoid:
    \begin{align*}
        G \triangleq \{\theta ~:~ \theta=\hat \theta + \Gamma w, ~ w\in\mathcal{B}(0, 1)\}.
    \end{align*} 
    Suppose the diameter of $G$ satisfies $\mathsf{diam(G)}\leq\frac{1}{256}\inf_{\theta\in G}\norm{P(\theta)}^{-5}$ and $\theta^\star\in G$. 
    Let $\calD$ be a distribution of system parameters over $G$. Let 
    \begin{align*}
        \textnormal{DR Objective } \triangleq  \bfE_{\theta\sim\calD} \brac{\trace\paren{(K_{\mathsf{DR}}(\calD) - K(\theta)) \Sigma^{K_{\mathsf{DR}}(\calD)}(\theta)  (K_{\mathsf{DR}}(\calD) - K(\theta)) \Psi(\theta)}}.
    \end{align*}
    It holds that 
    \begin{itemize}
        \item $\bfE_{\theta\sim\calD} \norm{K_{\mathsf{DR}}(\calD)- K(\theta)}^2 \leq \textnormal{DR Objective}$
        \item $\norm{K_{\mathsf{DR}}(\calD) - K(\theta^\star)} \leq \textnormal{DR Objective} + 32 \norm{P(\theta^\star)}^{7/2} \norm{\Gamma}$
        \item $\norm{K_{\mathsf{DR}}(\calD)}\leq \textnormal{DR Objective} + 2\norm{P(\theta^\star)}^{1/2}$ 
        \item $\textnormal{DR Objective} \leq 2.6e3\tau^3_{B(\theta^\star)}\norm{P(\theta^\star)}^4 \norm{\Gamma}$
        \item $\norm{\Sigma^{K_{\mathsf{DR}}(\calD)}(\theta)}\leq2\norm{\Sigma^{K_{\mathsf{DR}}(\calD)}(\theta^\star)}$ if $\mathsf{diam}(G)\leq\frac{1}{8\norm{\Sigma^{K_{\mathsf{DR}}}(\calD)(\theta^\star)}^{3/2}\paren{1+\norm{K_{\mathsf{DR}}}}^2}$
        \item $\norm{\Sigma^{K_{\mathsf{DR}}(\calD)}(\theta^\star)} 
        \leq 2\norm{\Sigma^{K(\theta^\star)}(\theta^\star)}$ if $\mathsf{diam}(G)\leq\frac{1}{6.4e4\tau^8_{B(\theta^\star)}\norm{P(\theta^\star)}^4\norm{\Sigma^{K(\theta^\star)}(\theta^\star)}^{3/2}}$. 
    \end{itemize}
\end{lemma}
\begin{proof}
    The first fact follows immediately from \Cref{lem: performance difference} along with the fact that $\Sigma^{K_{\mathsf{DR}}(\calD)}(\theta) \succeq I$ and $\Psi(\theta) \succeq I$. The second fact is derived from the first fact and \Cref{lem: Riccati perturbation}, which also yields the third fact using the diameter condition of $G$. 
    DR Objective is bounded as
    \begin{align*}
        \textnormal{DR Objective} &\leq \sup_{\theta\in G}\norm{\hat\theta-\theta}_{H(\theta)}^2 + 1.6e8\tau_{B(\theta^\star)}^3\norm{P(\theta^\star)}^{14}\norm{\hat\theta-\theta}^3 \\
        &\leq \paren{32\tau^2_{B(\theta^\star)}+ 2.5e3\tau^3_{B(\theta^\star)}}\norm{P(\theta^\star)}^4 \norm{\Gamma} 
    \end{align*}
    from \Cref{lem: helper lemma for RC} and the diameter condition. 
    It follows from \Cref{lem: lyap perturbation} that
    \begin{align*}
        &\bfE_{\theta\sim\calD}\norm{\Sigma^{K_{\mathsf{DR}}(\calD)}(\theta^\star) - \Sigma^{K_{\mathsf{DR}}(\calD)}(\theta)} \\
        &\leq \sup_{\theta\in G}\norm{\Sigma^{K_{\mathsf{DR}}(\calD)}(\theta^\star)}^{3/2}\norm{\Sigma^{K_{\mathsf{DR}}(\calD)}(\theta)}\norm{\theta-\theta^\star}\paren{1+\norm{K_{\mathsf{DR}}(\theta)}}\paren{2+\norm{\theta-\theta^\star}\paren{1+\norm{K_{\mathsf{DR}}(\theta)}}}.
    \end{align*}
    Let $\zeta(\theta) = \norm{\theta-\theta^\star}\paren{1+\norm{K_{\mathsf{DR}}(\theta)}}$, then we get the fifth fact: 
    \begin{align*}
        &\norm{\Sigma^{K_{\mathsf{DR}}(\calD)(\theta)}} 
        \leq \sup_{\theta\in G}\norm{\Sigma^{K_{\mathsf{DR}}(\calD)(\theta^\star)}} + \norm{\Sigma^{K_{\mathsf{DR}}(\calD)}(\theta^\star)}^{3/2}\norm{\Sigma^{K_{\mathsf{DR}}(\calD)}(\theta)}\zeta(\theta)(2+\zeta(\theta)) \\
        &\iff \norm{\Sigma^{K_{\mathsf{DR}}(\calD)}(\theta)} 
        \leq \sup_{\theta\in G}\frac{\norm{\Sigma^{K_{\mathsf{DR}}(\calD)}(\theta^\star)}}{1-\norm{\Sigma^{K_{\mathsf{DR}}}(\calD)(\theta^\star)}^{3/2}\zeta(\theta)(2+\zeta(\theta))} \leq 2\norm{\Sigma^{K_{\mathsf{DR}}(\calD)}(\theta^\star)}, 
    \end{align*}
    where the last inequality holds as long as $\norm{\Gamma}\leq\frac{1}{8\norm{\Sigma^{K_{\mathsf{DR}}}(\calD)(\theta^\star)}^{3/2}\paren{1+\norm{K_{\mathsf{DR}}}}^2}$. 

    Similarly, from \Cref{lem: lyap perturbation}, it holds that
    \begin{align*}
        &\norm{\Sigma^{K_{\mathsf{DR}}(\calD)}(\theta^\star) - \Sigma^{K(\theta^\star)}(\theta^\star)} \\
        &\leq \norm{\Sigma^{K(\theta^\star)}(\theta^\star)}^{3/2}\norm{\Sigma^{K_{\mathsf{DR}}(\calD)}(\theta^\star)}\norm{B(\theta^\star)\paren{K_{\mathsf{DR}}(\calD) - K(\theta^\star)}}\paren{2 + \norm{B(\theta^\star)\paren{K_{\mathsf{DR}}(\calD) - K(\theta^\star)}}} 
    \end{align*}  
    Let $\eta = \norm{B(\theta^\star)\paren{K_{\mathsf{DR}}(\calD) - K(\theta^\star)}}$, Then
    \begin{align*}
        & \norm{\Sigma^{K_{\mathsf{DR}}(\calD)}(\theta^\star)}\leq  \norm{\Sigma^{K(\theta^\star)}(\theta^\star)}+ \norm{\Sigma^{K(\theta^\star)}(\theta^\star)}^{3/2}\norm{\Sigma^{K_{\mathsf{DR}}(\calD)}(\theta^\star)}\eta(2+\eta) \\
        &\iff \norm{\Sigma^{K_{\mathsf{DR}}(\calD)}(\theta^\star)} \leq \frac{\norm{\Sigma^{K(\theta^\star)}(\theta^\star)}}{1-\norm{\Sigma^{K(\theta^\star)}(\theta^\star)}^{3/2}\eta(2+\eta)} \leq 2\norm{\Sigma^{K_{\mathsf{DR}}(\calD)}(\theta^\star)} 
    \end{align*}
    where the last inequality holds when
    \begin{align*}
        \eta(2+\eta) \leq \frac{1}{2\norm{\Sigma^{K(\theta^\star)}(\theta^\star)}^{3/2}}
    \end{align*}
    Now from the second and fourth fact, \eqref{eq: DR objective upper bound with theta} and \Cref{lem: helper lemma for RC}, $\eta$ is bounded as
    \begin{align*}
        &\eta = \norm{B(\theta^\star)\paren{K_{\mathsf{DR}}(\calD) - K(\theta^\star)}} \\
        &\leq \sup_{\theta\in G}\tau_{B(\theta^\star)}\paren{\textnormal{DR Objective} + 32 \norm{P(\theta^\star)}^{7/2} \norm{\theta-\theta^\star}} \\
        &\leq 2.5e3\tau^3_{B(\theta^\star)}\norm{P(\theta^\star)}^4 \norm{\Gamma} + 32\tau_{B(\theta^\star)}\norm{P(\theta^\star)}^{7/2}\norm{\theta-\theta^\star} = 2.6e3\tau^4_{B(\theta^\star)}\norm{P(\theta^\star)}^4\norm{\Gamma} \\
        &\iff \eta^2+2\eta \leq 3.2e4\tau^8_{B(\theta^\star)}\norm{P(\theta^\star)}^4\norm{\Gamma}
    \end{align*}
    where we applied $\mathsf{diam(G)}\leq\frac{1}{256}\inf_{\theta\in G}\norm{P(\theta)}^{-5}$ multiple times.
    Therefore the sixth inequality holds as long as
    \begin{align*}
        \norm{\Gamma} \leq \frac{1}{6.4e4\tau^8_{B(\theta^\star)}\norm{P(\theta^\star)}^4\norm{\Sigma^{K(\theta^\star)}(\theta^\star)}^{3/2}}
    \end{align*}
\end{proof}

\begin{lemma}
    \label{lem: DR suboptimality upper bound}
    Let $G$ denote the ellipsoid:
    \begin{align*}
        G \triangleq \{\theta ~:~ \theta=\hat \theta + \Gamma w, ~ w\in\mathcal{B}(0, 1)\},
    \end{align*} 
    Suppose the diameter of $G$ satisfies $\mathsf{diam}(G)\leq\frac{1}{6.4e4}\inf_{\theta\in G}\norm{P(\theta)}^{-5.5} \tau_{B(\theta^\star)}^{-8}$ and $\theta^\star\in G$. 
    Let $\calD$ be a distribution of system parameters over $G$.  Then it holds that
     \begin{align*}
        &C(K_{\mathsf{DR}}(\calD), \theta^\star) - C(K(\theta^\star), \theta^\star) \leq 2\norm{\hat\theta-\theta^\star}^2_{H(\theta^\star)} + 4\trace\paren{\mathbf{V}(\calD) H(\theta^\star)}+ L_{\mathsf{DR}}(\theta^\star)\norm{\Gamma}^3
    \end{align*}
    where
    \begin{align*}
        L_{\mathsf{DR}}(\theta^\star) = 5e8\du\tau_{B(\theta^\star)}^8\norm{P(\theta^\star)}^{17}
    \end{align*}
\end{lemma}

\begin{proof}
    By the fact that $K_{\mathsf{DR}}(\calD)$ stabilizes the system $\theta^\star$, we may write by \Cref{lem: performance difference}
    \begin{equation}
    \begin{aligned}
        \label{eq: DR excess cost decomposiotion}
        &C(K_{\mathsf{DR}}(\calD), \theta^\star) - C(K(\theta^\star), \theta^\star) = \trace\paren{(K_{\mathsf{DR}}(\calD) - K(\theta^\star)) \Sigma^{K_{\mathsf{DR}}(\calD)}(\theta^\star)  (K_{\mathsf{DR}}(\calD) - K(\theta^\star)) \Psi(\theta^\star)} \\
        &\leq 2 \bfE_{\theta\sim\calD} \brac{\trace\paren{(K_{\mathsf{DR}}(\calD) - K(\theta)) \Sigma^{K_{\mathsf{DR}}(\calD)}(\theta^\star)  (K_{\mathsf{DR}}(\calD) - K(\theta)) \Psi(\theta^\star)}} \\
        &+ 2 \bfE_{\theta\sim\calD} \brac{\trace\paren{(K(\theta) - K(\theta^\star)) \Sigma^{K_{\mathsf{DR}}(\calD)}(\theta^\star)  (K(\theta) - K(\theta^\star)) \Psi(\theta^\star)}},
    \end{aligned}
    \end{equation}
    where the inequality follows from the Cauchy-Schwarz. Next, we massage the first term to have the form of the domain randomization objective, and the second term to have the form of the certainty equivalence objective. To this end, first note that by \Cref{lem: helper lemma for RC} and $\Psi(\theta)\succeq I$, we get
    \begin{align*}
        \Psi(\theta^\star) \preceq \Psi(\theta)\paren{1 + 15 \tau_{B(\theta^\star)}^2 \norm{P(\theta^\star)}^3 \norm{\theta - \theta^\star}}.
    \end{align*}
    Additionally, write expand 
    \begin{align*}
       \Sigma^{K_{\mathsf{DR}}(\calD)}(\theta^\star) &\preceq \Sigma^{K_{\mathsf{DR}}(\calD)}(\theta) + I \times \norm{\Sigma^{K_{\mathsf{DR}}(\calD)}(\theta^\star) - \Sigma^{K_{\mathsf{DR}}(\calD)}(\theta)}  \mbox{ and }\\
       \Sigma^{K_{\mathsf{DR}}(\calD)}(\theta^\star) &= \Sigma^{K(\theta^\star)}(\theta^\star) + I \times \norm{\Sigma^{K_{\mathsf{DR}}(\calD)}(\theta^\star) - \Sigma^{K(\theta^\star)}(\theta^\star)}.
    \end{align*}
    Substituting the above set of inequalities into \eqref{eq: DR excess cost decomposiotion} we achieve the bound 
    \begin{align}
        &C(K_{\mathsf{DR}}(\calD), \theta^\star) - C(K(\theta^\star), \theta^\star) \nonumber \\
        &\leq 2 \times  \mbox{DR objective}\times \paren{1 + 15 \tau_{B(\theta^\star)}^2 \norm{P(\theta^\star)}^3 \norm{\theta - \theta^\star}} \nonumber\\
        &+ 2 \times \mbox{Expected CE cost gap} \nonumber\\
        & + 2 \du \norm{\Psi(\theta^\star)} \bfE_{\theta\sim\calD}\brac{\norm{\Sigma^{K_{\mathsf{DR}}(\calD)}(\theta^\star) - \Sigma^{K_{\mathsf{DR}}(\calD)}(\theta)} \norm{K_{\mathsf{DR}}(\calD) - K(\theta)}^2} \nonumber\\
        &+ 2 \du \norm{\Psi(\theta^\star)} \norm{\Sigma^{K_{\mathsf{DR}}(\calD)}(\theta^\star) - \Sigma^{K(\theta^\star)}(\theta^\star)}\bfE_{\theta\sim\calD}\brac{ \norm{K(\theta^\star) - K(\theta)}^2}, \label{eq:DR suboptimality gap}
    \end{align}
    where 
    \begin{align*}
        &\mbox{Expected CE cost gap}  = \bfE_{\theta\sim\calD} \brac{\trace\paren{(K(\theta) - K(\theta^\star)) \Sigma^{K(\theta^\star)}(\theta^\star)  (K(\theta) - K(\theta^\star)) \Psi(\theta^\star)}}, \\
         &\mbox{DR Objective }=  \bfE_{\theta\sim\calD} \brac{\trace\paren{(K_{\mathsf{DR}}(\calD) - K(\theta)) \Sigma^{K_{\mathsf{DR}}(\calD)}(\theta)  (K_{\mathsf{DR}}(\calD) - K(\theta)) \Psi(\theta)}}.
    \end{align*}
    We let $D(\theta)$ denote the DR objective.
    From \Cref{lem: DR objective upper bound}, the first term of \eqref{eq:DR suboptimality gap} becomes
    \begin{align}
        &2\times D(\theta)\times \paren{1 + 15 \tau_{B(\theta^\star)}^2 \norm{P(\theta^\star)}^3 \norm{\theta - \theta^\star}}\nonumber\\
        &\leq 2 \times \paren{\trace\paren{\mathbf{V}(\calD) H(\theta^\star)} + L_{\mathsf{DR,1}}(\theta^\star)  \norm{\Gamma}^3} \times \paren{1 + 15 \tau_{B(\theta^\star)}^2 \norm{P(\theta^\star)}^3\norm{\theta - \theta^\star}} \nonumber\\
        &\leq 2\trace\paren{\mathbf{V}(\calD) H(\theta^\star)} + 3.5e8\tau_{B(\theta^\star)}^5\norm{P(\theta^\star)}^{17}\norm{\Gamma}^3 \label{eq: first term of DR suboptimality gap}
    \end{align}
    where we used \Cref{lem: helper lemma for RC} and applied the closeness condition to $\norm{\theta-\theta^\star}$ in the last inequality. 
    Next the second term of \eqref{eq:DR suboptimality gap} follows from \Cref{lem: CE upper bound} that
    \begin{align}
        2 \times \mbox{Expected CE cost gap} \leq \bfE_{\theta\sim\calD}\brac{2\norm{\theta-\theta^\star}^2_{H(\theta^\star)}} + 1.2e6\tau_{B(\theta^\star)}^3\norm{P(\theta^\star)}^{14}\norm{\Gamma}^3 \label{eq:second term of DR suboptimality gap}
    \end{align}
    Then from the fifth fact of \Cref{lem: DR helper lemmas}, 
    \begin{align*}
        &\bfE_{\theta\sim\calD}\norm{\Sigma^{K_{\mathsf{DR}}(\calD)}(\theta^\star) - \Sigma^{K_{\mathsf{DR}}(\calD)}(\theta)} \\
        &\leq\sup_{\theta\in G}3\norm{\Sigma^{K(\theta^\star)}(\theta^\star)}^{5/2}\paren{1+\norm{K_{\mathsf{DR}}(\calD)}}\norm{\theta-\theta^\star}\paren{2 + \frac{1}{8\norm{\Sigma^{K_{\mathsf{DR}}}(\calD)(\theta^\star)}^{3/2}\paren{1+\norm{K_{\mathsf{DR}}(\calD)}}}}
        \\
        &\leq \sup_{\theta\in G}5\norm{\Sigma^{K(\theta^\star)}(\theta^\star)}^{5/2}\paren{1+\norm{K_{\mathsf{DR}}(\calD)}}\norm{\theta-\theta^\star}
    \end{align*}
    since $\norm{\Sigma^{K(\theta^\star)}(\theta^\star)} \geq 1$. Also from the sixth fact of \Cref{lem: DR helper lemmas}
    \begin{align*}
        &\norm{\Sigma^{K_{\mathsf{DR}}(\calD)}(\theta^\star) - \Sigma^{K(\theta^\star)}(\theta^\star)} \leq \sup_{\theta\in G}6.4e4\norm{P(\theta^\star)}^{13/2}\norm{\theta-\theta^\star}
    \end{align*}
    By combining with \Cref{lem: Riccati perturbation}, the third term of \eqref{eq:DR suboptimality gap} becomes
    \begin{align}
        &2 \du \norm{\Psi(\theta^\star)} \bfE_{\theta\sim\calD}\brac{\norm{\Sigma^{K_{\mathsf{DR}}(\calD)}(\theta^\star) - \Sigma^{K_{\mathsf{DR}}(\calD)}(\theta)} \norm{K_{\mathsf{DR}}(\calD) - K(\theta)}^2} \nonumber \\
        &\leq \du \norm{\Psi(\theta^\star)}  \paren{8e4\tau_{B(\theta^\star)}^2\paren{1+\norm{K_{\mathsf{DR}}(\theta)}}\norm{P(\theta^\star)}^{23/2} + 3.4e8\tau_{B(\theta^\star)}^3\norm{P(\theta^\star)}^{39/2}\norm{\Gamma}}\norm{\Gamma}^3 \nonumber  \\
        &\leq 5e6\du\tau^8_{B(\theta)}\norm{P(\theta^\star)}^{33/2}\norm{\Gamma}^3 \label{eq:third term of DR suboptimality gap}
    \end{align}
    where the last inequality follows from \Cref{lem: simplifying inequalities} and noting that $1+\norm{K_{\mathsf{DR}}(\theta)}\leq12\tau_{B(\theta^\star)}^4\norm{P(\theta^\star)}^{1/2}$ from \Cref{lem: DR helper lemmas}.
    Furthermore, the fourth term of \eqref{eq:DR suboptimality gap} is given by
    \begin{align}
        &2 \du \norm{\Psi(\theta^\star)} \norm{\Sigma^{K_{\mathsf{DR}}(\calD)}(\theta^\star) - \Sigma^{K(\theta^\star)}(\theta^\star)}\bfE_{\theta\sim\calD}\brac{ \norm{K(\theta^\star) - K(\theta)}^2} \leq 1.4e8 \du \tau^2_{B(\theta)}\norm{P(\theta^\star)}^{31/2}\norm{\Gamma}^3 \label{eq:fourth term of DR suboptimality gap}
    \end{align}
    where the last inequatliy follows from \Cref{lem: Riccati perturbation} and \Cref{lem: simplifying inequalities}. 
    Finally, from \eqref{eq: first term of DR suboptimality gap}, \eqref{eq:second term of DR suboptimality gap}, \eqref{eq:third term of DR suboptimality gap} and \eqref{eq:fourth term of DR suboptimality gap}, we get
    \begin{align*}
         &C(K_{\mathsf{DR}}(\calD), \theta^\star) - C(K(\theta^\star), \theta^\star) \\
         &\leq 2\bfE_{\theta\sim\calD}{\norm{\theta-\theta^\star}^2_{H(\theta^\star)}} + 2\trace\paren{\mathbf{V}(\calD) H(\theta^\star)} + L_{\mathsf{DR}}(\theta^\star)\norm{\Gamma}^3
    \end{align*}
    where
    \begin{align*}
        L_{\mathsf{DR}}(\theta^\star) &= \paren{3.5e8\tau_{B(\theta^\star)}^5 + 1.2e6\tau_{B(\theta^\star)}^3 + 5e6\du\tau^8_{B(\theta)} +  1.4e8\du \tau^2_{B(\theta)}}\norm{P(\theta^\star)}^{17}\\
        &= 5e8\du\tau_{B(\theta^\star)}^8\norm{P(\theta^\star)}^{17}
    \end{align*}
    We conclude the proof by noting that
    \begin{align*}
        2\bfE_{\theta\sim\calD}\norm{\theta-\theta^\star}^2_{H(\theta^\star)} = 
        2\norm{\hat\theta - \theta^\star}_{H(\theta^\star)}^2 + 
        2\trace\paren{\mathbf{V}(\calD) H(\theta^\star)}.
    \end{align*}
\end{proof}

\subsection{Proof of \texorpdfstring{\Cref{lem: domain randomization general}}{}}
\label{subsec: Proof of DR upper bound}
\begin{proof}
    As proved in \Cref{subsec: proof of RC upper bound}, $\theta^\star\in G$ with probability at least $1-\delta$. 
    From the diameter condition in \Cref{lem: DR suboptimality upper bound}, i.e., $\mathsf{diam(G)}\leq\frac{1}{6.4e4}\inf_{\theta\in G}\norm{P(\theta)}^{-5.5} \tau_{B(\theta^\star)}^{-8}$, $N$ must satisfy
    \begin{align*}
        & \sup_{\theta_1, \theta_2\in G}\norm{\theta_1-\theta_2}^2\leq \frac{32\paren{d_\theta+\log\frac{2}{\delta}}}{N\lambda_{\min}\paren{\hat{\mathsf{FI}}}} \leq \frac{1}{(6.4e4)^2}\inf_{\theta\in G}\norm{P(\theta)}^{-11} \tau_{B(\theta^\star)}^{-16}.  \\
        &\iff N \geq \frac{5e6\norm{P(\theta^\star)}^{11} \tau_{B(\theta^\star)}^{16}\paren{d_\theta+\log\frac{2}{\delta}}}{\lambda_{\min}\paren{\mathsf{FI}(\theta^\star)}}
    \end{align*} 
    where we applied $0.5\mathsf{FI}(\theta^\star)\preceq\hat{\mathsf{FI}}$ and \Cref{lem: Riccati perturbation} to get the last inequality.
    As long as this holds, we get \eqref{eq:DR upper bound} from \Cref{thm: identification bound} and \Cref{lem: DR suboptimality upper bound}. 
\end{proof}

\subsection{Proof of \texorpdfstring{\Cref{thm: Domain Randomization Upper Bound}}{}}
\label{subsec: Proof of DR upper bound w/ uniform distribution}
\begin{proof}
    Since the variance of uniform distribution over the ellipsoid $\curly{w \colon w^T\Gamma w\leq1, w\in\R^d}$ is given as $\frac{1}{d+2}\Gamma^{-1}$, 
    \begin{align*}
        \trace(V(\calD)H(\theta^\star)) &\leq \frac{16\paren{d_\theta+\log\frac{2}{\delta}}}{Nd_\theta}\trace\paren{\hat{\mathsf{FI}}^{-1}H(\theta^\star)} \leq \frac{32\paren{1+\log\frac{2}{\delta}/d_\theta}}{N}\trace\paren{{\mathsf{FI}}(\theta^\star)^{-1}H(\theta^\star})
    \end{align*}
    where we used $0.5\hat{\mathsf{FI}}\preceq {\mathsf{FI}}(\theta^\star)$ in the last inequality. Then from \Cref{lem: domain randomization general}, we get \eqref{eq:DR upper bound w/ unif dist}. 
\end{proof}