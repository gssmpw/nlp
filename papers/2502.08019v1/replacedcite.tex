\section{Related Work}
In this paper, we apply percolation theory to a sphere for the first time to discuss the ability to generate large-scale continuous coverage paths of LEO satellites. Therefore, we divide the related works into: i) LEO satellite communications based on stochastic geometry and ii) percolation theory applications on wireless networks.\\%i) LEO satellite constellations for next-generation mobile communications and ii) percolation theory applications on wireless networks.\\
% \indent \textit{LEO satellite constellations for next-generation mobile communications}: In ____, authors proposed a distributed lightweight stateless satellite core network architecture, to avoid frequent transmission and service interruption. Authors in ____ developed a tractable framework to evaluate the performance of downlink hybrid terrestrial/satellite networks in rural areas. To accommodate time-sensitive service required on industrial IoT devices, a network-layer-based latency scheduling architecture was proposed in ____ \textcolor{red}{(2023)}. Authors in ____ established an optimization algorithm to maximize reliability and minimize latency, and obtained the ideal upper bound for these performances \textcolor{red}{(2024)}. For a heterogeneous satellite network, authors in ____ proved that the open access scenario can obtain a higher coverage probability than the closed access \textcolor{red}{(2024)}. Authors in ____ proposed a distributed low-complexity satellite-terrestrial cooperative routing approach to overcome the long end-to-end delay caused by multi-hop routing and routing table construction \textcolor{red}{(2024)}. To break the barriers between computing and networking, a concept named integrated computing and networking for LEO satellite mega-constellations (ICN-LSMC) was proposed in ____ \textcolor{red}{(2024)}. Authors in ____ designed a dynamic satellite-ground integrated mobility management strategy (DSG-MMS) to minimize handover and migration delays \textcolor{red}{(2024)}. Authors in ____ proposed a theoretical framework for an LEO satellite-aided shore-to-ship communication network (LEO-SSCN) and derived analytical expressions of end-to-end transmission success probability and average transmission rate capacity \textcolor{red}{(2024)}. \\
\indent \textit{LEO satellite communications based on stochastic geometry}: In recent years, LEO satellite communication has been a focus of next-generation communication technology. Stochastic geometry is widely used to evaluate the performance of communication systems with LEO satellites. In ____, authors studied the coverage performance of LEO satellite communication system, where satellite gateways on the ground act as relays between users and satellites. Authors in ____ and ____ evaluated the average downlink success probability for dense satellite networks and optimal satellite constellation altitude. Authors in ____ extended the work and investigated the optimal beamwidth and altitude for maximal uplink coverage of satellite networks. Authors in ____ and ____ evaluated the average data rate and coverage probability using BPP and nonhomogeneous PPP, respectively. In ____, they derived and verified the coverage probability of a multi-altitude LEO network. In ____, authors derived the tight lower bound of coverage probability and found out the relationship between optimal average number of satellites and the altitude of satellites. A tractable framework was developed in ____ to evaluate the performance of downlink hybrid terrestrial/satellite networks in rural areas. Authors in ____ derived the joint distance distribution of cooperative LEO satellites to the typical user, and obtained the optimal satellite density and satellite altitude to maximize the coverage probability. For Space-air-ground integrated networks (SAGIN), authors in ____ proposed a simulated annealing algorithm-based optimization algorithm to optimize THz and RF channel allocation. Authors in ____ studied the influence of gateway density and the setting of satellites constellations on latency and coverage probability. They established an optimization algorithm to maximize reliability and minimize latency, and obtained the ideal upper bound for these performances in ____. For a heterogeneous satellite network, authors in ____ proved that the open access scenario can obtain a higher coverage probability than the closed access. Authors in ____ investigated the key performance indices of a delay-tolerant data harvesting architecture, including the CDF of delay and harvest capacity. For different communication scenarios, authors in ____ analyzed the uplink performance of IoT over LEO satellite communication with reliable coverage. An adaptive coverage enhancement (ACE) method was proposed in ____ for random access parameter configurations under diverse applications. Authors in ____ investigated the impact of onboard energy limitation, minimum elevation angle on downlink steady-state probability and availability. Authors in ____ proposed a throughput optimization algorithm for LEO satellite-based IoT networks and derived the closed-form expression of the throughput when LEO satellites are equipped with capture effect (CE) receiver and successive interference cancellation (SIC) receiver, respectively.

\indent \textit{Percolation theory applications on wireless networks}: Percolation theory and graph theory are widely used to evaluate the connectivity of large-scale networks, including multi-hops links, detective paths, continuous coverage, security, to name a few ____. Authors in ____ modeled the homogeneous wireless balloon network (WBN) as a Gilbert disk model (GDM) and modeled the heterogeneous WBN as a Random Gilbert disk model (RGDM).
They derived the bounds of the critical node density of such WBNs. In ____, they also derived the critical density of unmanned aerial vehicles (UAVs) to ensure the network coverage of UAV networks (UN). Using percolation theory, authors in ____ derived the critical density of camera sensors in clustered 3D wireless camera sensor networks (WCSN). Authors in ____ characterized the critical density of spatial firewalls to prevent malware epidemics in large-scale wireless networks (LSWN). In ____, authors established a model for the coexistence of random primary and secondary cognitive networks and proved the feasibility of the simultaneous connectivity. Based on dynamic bound percolation, authors in ____ characterized the reliable topology evolution and proved that the dynamic topology evolution (DTE) model can improve the overall network performance.
In ____, authors investigated the connectivity of large-scale reconfigurable intelligent surface (RIS) assisted integrated access and backhaul (IAB) networks.
Considering the directional antenna, authors in ____ analyzed the connectivity of networks assisted by transmissive RIS.