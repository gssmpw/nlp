% \documentclass[journal,draftclsnofoot,onecolumn,12pt]{IEEEtran}
\documentclass[final]{IEEEtran}
\usepackage{amsthm,amssymb,graphicx,multirow,amsmath,color,amsfonts,physics}%,ulem}
\usepackage[update,prepend]{epstopdf}
\usepackage[noadjust]{cite}
% \usepackage[latin1]{inputenc}
\usepackage{tikz}
\usepackage{bbm} % for \nbb1 
\usepackage{pdfpages}
\usepackage{balance}
%\usepackage{flushend}
%\usepackage{tabulary}
\usepackage{multirow}
\usepackage{comment}
\usepackage{subfigure}
\usepackage{yhmath}
%\usepackage[justification=centering]{caption}
% Colors
\def\chr#1{{\color{red} #1}}
\def\chb#1{{\color{blue} #1}}
\newtheorem{corollary}{Corollary}
\allowdisplaybreaks % Allows breaking of eqnarray over multiple pages (avoids unnecessary blanks in the document before eqnarray)
\allowdisplaybreaks % Allows breaking of eqnarray over multiple pages (avoids unnecessary blanks in the document before eqnarray)
% \usepackage{setspace}    % Remove in double-column version. Also, search for \setstretch in the body of the paper and comment these commands for double column

%  %Reduces space around equations and figures 
\setlength\abovedisplayskip{3pt plus 2pt minus 2pt}     % Reduce space before equation
\setlength\belowdisplayskip{3pt plus 2pt minus 2pt}    % Reduce space after equation
\setlength\textfloatsep{10pt plus 2pt minus 2pt}    
\usepackage{float}

% \usepackage{gensymb}
\begin{document}
\newcommand{\ours}{$\text{Q}$LASS}
\pagenumbering{gobble}
\graphicspath{{./Figures/}}
\title{
Connectivity of LEO Satellite Mega Constellations: An Application of Percolation Theory on a Sphere}
\author{
 Hao Lin,  Mustafa A. Kishk and Mohamed-Slim Alouini
\thanks{Hao Lin is with the Electrical and Computer Engineering Program, CEMSE Division, King Abdullah University of Science and Technology (KAUST),
Thuwal 23955-6900, Saudi Arabia (e-mail: hao.lin.std@gmail.com).\\
\indent Mustafa A. Kishk is with the Department of Electronic Engineering,
Maynooth University, Maynooth, W23 F2H6 Ireland (e-mail:
mustafa.kishk@mu.ie).\\
\indent Mohamed-Slim Alouini is with the CEMSE Division, King Abdullah
University of Science and Technology (KAUST), Thuwal 23955-6900,
Saudi Arabia (e-mail: slim.alouini@kaust.edu.sa).}
}

\maketitle
\vspace{-2cm}
%\thispagestyle{empty}
%\pagestyle{empty}
\begin{abstract}
With the advent of the 6G era, global connectivity has become a common goal in the evolution of communications, aiming to bring Internet services to more unconnected regions. Additionally, the rise of applications such as the Internet of Everything and remote education also requires global connectivity. Non-terrestrial networks (NTN), particularly low earth orbit (LEO) satellites, play a crucial role in this future vision. Although some literature already analyze the coverage performance using stochastic geometry, the ability of generating large-scale continuous service area is still expected to analyze.  Therefore, in this paper, we mainly investigate the necessary conditions of LEO satellite deployment for large-scale continuous service coverage on the earth. Firstly, we apply percolation theory to a closed spherical surface and define the percolation on a sphere for the first time. We introduce the sub-critical and super-critical cases to prove the existence of the phase transition of percolation probability. Then, through stereographic projection, we introduce the tight bounds and closed-form expression of the critical number of LEO satellites on the same constellation. In addition, we also investigate how the altitude and maximum slant range of LEO satellites affect percolation probability, and derive the critical values of them. Based on our findings, we provide useful recommendations for companies planning to deploy LEO satellite networks to enhance connectivity.
\end{abstract}
% \vspace{-0.3cm}
\begin{IEEEkeywords}
% \vspace{-0.3cm}
Percolation theory, LEO satellites, non-terrestrial networks, stereographic projection.
\end{IEEEkeywords}

\section{Introduction} \label{sec:Intro}

As a key technology of next-generation communications, non-terrestrial networks (NTN) have been proposed to enhance high-capacity global connectivity \cite{giordani2020non}. 3GPP Release 17 defined the New Radio (NR) to support NTN broadband Internet services, especially for rural and remote areas \cite{3GPPRelease17}. Narrowband Internet of Things (NB-IoT) over NTN has been preliminarily standardised and commercial deployments are ongoing, where low earth orbit (LEO) satellite constellations can be a solution to provide low latency service in a low cost \cite{3GPPRelease18,liberg2020narrowband}. Furthermore, in 3GPP Release 18 and 19, NTNs including LEO satellites aim to support regenerative payloads, and coverage and mobility enhancements, for more requirements from handheld terminals, NB-IoT and enhanced machine-type communication (eMTC) \cite{10500741}. They can help not only support 5G NR but also pave the way towards 6G technologies. LEO-satellite access networks have been deployed to provide seamless massive access and connect the unconnected areas on the earth \cite{xiao2022leo}. Examples of currently deployed or planned LEO satellite constellations include Starlink, OneWeb and Kuiper \cite{osoro2021techno,voicu2024handover}. Therefore, it is necessary to capture the ability of LEO satellites to enhance global connectivity.  \\
\indent Stochastic geometry is an important tool to evaluate the performance of large-scale wireless networks without losing accuracy \cite{haenggi2012stochastic}. It is widely used to evaluate the coverage performance of 2D or 3D wireless networks. For global coverage, it can also provide a basic geometric framework, especially using the binomial point process (BPP) and Poisson point process (PPP). Global connectivity is a vital performance indicator of large-scale wireless networks, where graph theory and percolation theory can help capture such performance metric \cite{haenggi2009stochastic,haenggi2012stochastic}. Percolation probability represents the probability of generating large-scale connected components in a network, where the nodes can be connected through multi-hops. Such an indicator can be used to capture a network's connectivity, robustness, cyber-security and path exposure \cite{elsawy2023tutorial}.\\
\indent Therefore, it is necessary to evaluate the connectivity of LEO satellites' coverage, that is, the ability to generate large-scale continuous paths on the earth that are covered by LEO satellites, through percolation theory. However, conventional percolation on wireless networks relies on a 2D plane or a 3D system, which is different from the realistic deployment of user equipment (UEs) or IoT devices on the earth and the coverage regions of LEO satellites. So that, the percolation analysis on the 3D sphere is still an unsolved problem.


\subsection{Related Work}
 In this paper, we apply percolation theory to a sphere for the first time to discuss the ability to generate large-scale continuous coverage paths of LEO satellites. Therefore, we divide the related works into: i) LEO satellite communications based on stochastic geometry and ii) percolation theory applications on wireless networks.\\%i) LEO satellite constellations for next-generation mobile communications and ii) percolation theory applications on wireless networks.\\
% \indent \textit{LEO satellite constellations for next-generation mobile communications}: In \cite{liu2023stateless}, authors proposed a distributed lightweight stateless satellite core network architecture, to avoid frequent transmission and service interruption. Authors in \cite{salem2023exploiting} developed a tractable framework to evaluate the performance of downlink hybrid terrestrial/satellite networks in rural areas. To accommodate time-sensitive service required on industrial IoT devices, a network-layer-based latency scheduling architecture was proposed in \cite{wang2023time} \textcolor{red}{(2023)}. Authors in \cite{wang2024ultra} established an optimization algorithm to maximize reliability and minimize latency, and obtained the ideal upper bound for these performances \textcolor{red}{(2024)}. For a heterogeneous satellite network, authors in \cite{choi2024modeling} proved that the open access scenario can obtain a higher coverage probability than the closed access \textcolor{red}{(2024)}. Authors in \cite{sun2024distributed} proposed a distributed low-complexity satellite-terrestrial cooperative routing approach to overcome the long end-to-end delay caused by multi-hop routing and routing table construction \textcolor{red}{(2024)}. To break the barriers between computing and networking, a concept named integrated computing and networking for LEO satellite mega-constellations (ICN-LSMC) was proposed in \cite{huang2024integrated} \textcolor{red}{(2024)}. Authors in \cite{ji2024dynamic} designed a dynamic satellite-ground integrated mobility management strategy (DSG-MMS) to minimize handover and migration delays \textcolor{red}{(2024)}. Authors in \cite{hu2024performance} proposed a theoretical framework for an LEO satellite-aided shore-to-ship communication network (LEO-SSCN) and derived analytical expressions of end-to-end transmission success probability and average transmission rate capacity \textcolor{red}{(2024)}. \\
\indent \textit{LEO satellite communications based on stochastic geometry}: In recent years, LEO satellite communication has been a focus of next-generation communication technology. Stochastic geometry is widely used to evaluate the performance of communication systems with LEO satellites. In \cite{talgat2020stochastic}, authors studied the coverage performance of LEO satellite communication system, where satellite gateways on the ground act as relays between users and satellites. Authors in \cite{al2021analytic} and \cite{al2021optimal} evaluated the average downlink success probability for dense satellite networks and optimal satellite constellation altitude. Authors in \cite{al2022optimal} extended the work and investigated the optimal beamwidth and altitude for maximal uplink coverage of satellite networks. Authors in \cite{okati2020downlink} and \cite{okati2022nonhomogeneous} evaluated the average data rate and coverage probability using BPP and nonhomogeneous PPP, respectively. In \cite{okati2023stochastic}, they derived and verified the coverage probability of a multi-altitude LEO network. In \cite{park2022tractable}, authors derived the tight lower bound of coverage probability and found out the relationship between optimal average number of satellites and the altitude of satellites. A tractable framework was developed in \cite{salem2023exploiting} to evaluate the performance of downlink hybrid terrestrial/satellite networks in rural areas. Authors in \cite{shang2023coverage} derived the joint distance distribution of cooperative LEO satellites to the typical user, and obtained the optimal satellite density and satellite altitude to maximize the coverage probability. For Space-air-ground integrated networks (SAGIN), authors in \cite{yuan2023joint} proposed a simulated annealing algorithm-based optimization algorithm to optimize THz and RF channel allocation. Authors in \cite{wang2022ultra} studied the influence of gateway density and the setting of satellites constellations on latency and coverage probability. They established an optimization algorithm to maximize reliability and minimize latency, and obtained the ideal upper bound for these performances in \cite{wang2024ultra}. For a heterogeneous satellite network, authors in \cite{choi2024modeling} proved that the open access scenario can obtain a higher coverage probability than the closed access. Authors in \cite{choi2024analysis} investigated the key performance indices of a delay-tolerant data harvesting architecture, including the CDF of delay and harvest capacity. For different communication scenarios, authors in \cite{talgat2024stochastic} analyzed the uplink performance of IoT over LEO satellite communication with reliable coverage. An adaptive coverage enhancement (ACE) method was proposed in \cite{hong2024narrowband} for random access parameter configurations under diverse applications. Authors in \cite{bliss2024orchestrating} investigated the impact of onboard energy limitation, minimum elevation angle on downlink steady-state probability and availability. Authors in \cite{taojoint} proposed a throughput optimization algorithm for LEO satellite-based IoT networks and derived the closed-form expression of the throughput when LEO satellites are equipped with capture effect (CE) receiver and successive interference cancellation (SIC) receiver, respectively.

\indent \textit{Percolation theory applications on wireless networks}: Percolation theory and graph theory are widely used to evaluate the connectivity of large-scale networks, including multi-hops links, detective paths, continuous coverage, security, to name a few \cite{haenggi2009stochastic,haenggi2012stochastic,elsawy2023tutorial}. Authors in \cite{anjum2019percolation} modeled the homogeneous wireless balloon network (WBN) as a Gilbert disk model (GDM) and modeled the heterogeneous WBN as a Random Gilbert disk model (RGDM).
They derived the bounds of the critical node density of such WBNs. In \cite{anjum2020coverage}, they also derived the critical density of unmanned aerial vehicles (UAVs) to ensure the network coverage of UAV networks (UN). Using percolation theory, authors in \cite{wang2019cooperative} derived the critical density of camera sensors in clustered 3D wireless camera sensor networks (WCSN). Authors in \cite{zhaikhan2020safeguarding} characterized the critical density of spatial firewalls to prevent malware epidemics in large-scale wireless networks (LSWN). In \cite{yemini2019simultaneous}, authors established a model for the coexistence of random primary and secondary cognitive networks and proved the feasibility of the simultaneous connectivity. Based on dynamic bound percolation, authors in \cite{han2024dynamic} characterized the reliable topology evolution and proved that the dynamic topology evolution (DTE) model can improve the overall network performance.
In \cite{wu2023connectivity}, authors investigated the connectivity of large-scale reconfigurable intelligent surface (RIS) assisted integrated access and backhaul (IAB) networks.
Considering the directional antenna, authors in \cite{zhu2023connectivity} analyzed the connectivity of networks assisted by transmissive RIS.

\subsection{Contributions}
The contributions of this paper can be summarized as follows:\\
\indent \textit{A new perspective of connectivity of LEO satellite coverage.} In this paper, we evaluate the connectivity of LEO satellites' coverage using percolation theory. We adopt the percolation probability to show the ability to form giant continuous LEO satellite service areas or footprints for users on the earth. Such performance metric describes whether the devices that are moving on the earth can achieve continuous service and whether the Internet of Everything on a large scale can be supported by LEO satellites. \\
\indent \textit{Percolation theory on the Sphere.} In this paper, we first define the face percolation on the sphere using its stereographic projection onto a plane. We introduce the sub-critical and super-critical cases to investigate the lower and upper bounds of critical number of LEO satellites for percolation. Through stereographic projection, we derive the tight bounds for the critical threshold of the number of LEO satellites. By discussing the relationship between continuous percolation and discrete percolation, we obtain the closed-form expression of the critical number of LEO satellites. In addition, we also derive the closed-form critical expressions of altitude and maximum slant range of LEO satellites.\\
% \indent \textit{Suggestions for LEO satellite operators.} Through the percolation probability and coverage probability, we give different suggestions for different LEO satellite operators of different sizes. 

\section{System Model} \label{sec:SysMod}
% \begin{figure*}[htbp]
%     \centering
% \includegraphics[width=0.75\linewidth]{Figures/}
%     \caption{...}
%     \label{fig:...}
% \end{figure*}

For ease of tractability, the earth's surface is commonly considered a standard sphere with a radius $r_e=6371\, \rm km$, where the center of the earth is defined as $\oe$. The North Pole is located at $\textbf{w}_N$ and the South Pole is located at $\textbf{w}_S$. We assume that LEO satellites are uniformly distributed on a sphere at an altitude of $h$ from the ground, that is, the sphere centered at $\oe$ with a radius $r_s=r_e+h$. The locations of LEO satellites follow a BPP $\Phi=\{\yi\}$ with the number of satellites $N$, where $\yi$ represents the location of any satellite  \cite{talgat2020stochastic,wang2023coverage,talgat2024stochastic}. Notice that current LEO satellite constellations adopt different orbital design schemes, where Antarctic and Arctic region have different satellite densities from other regions. However, such a BPP assumption describes the future LEO satellite mega constellations with a massive number of satellites around the earth. Through antenna array and beamforming technique, each satellite can serve the users within the transmission angle $\eta$, \ie the nadir angle. The LEO satellite located at $\yi$ can provide the communication service for its spherical coverage area $\mathcal{A}_i=\mathcal{A}(\yi,\eta)$, which is defined as:
\begin{equation}
    \mathcal{A}(\yi,\eta)=\{\wik:\|\wik-\oe\|=r_e,\,\angle \oe\yi\wik\leq\eta\}.
\end{equation}
We can also define the center of the coverage area as $\xi$ so that such a coverage area can be defined using another symbol $\mathcal{O}_i=\mathcal{O}(\xi,\gamma)$, where:
\begin{equation}
    \mathcal{O}(\xi,\gamma)=\{\wik:\|\wik-\oe\|=r_e,\,\angle \xi\oe\wik\leq\gamma\},
\end{equation}
where $\gamma$ is the coverage angle of LEO satellites on the earth. Notice that $\mathcal{O}_i$ and $\mathcal{A}_i$ both represent the coverage area of the same satellite, they must satisfy
\begin{equation}
    \mathcal{O}(\xi,\gamma)=\mathcal{A}(\yi,\eta).
\end{equation}
Considering a user located at the boundary of LEO satellite's coverage, the distance to the satellite is called the maximum slant range $d_m$. The geometric relationships between $\gamma$, $\eta$, $h$ and $d_m$ are shown in the Fig. \ref{fig:etagamma} and Lemma \ref{lem:etagamma} in the following section.\\
\indent In this paper, we aim to analyze the connectivity of coverage areas of LEO satellites. Percolation theory, has its unique advantage of analyzing the connectivity, especially in a 2D plane. However, the definition of percolation on a sphere is less common in literature. Therefore, based on graph theory, we first define the connectivity of satellites' coverage areas as a 3D random graph $G_x(V_x,E_x)$, where $V_x=\{\xi\}$ is the set of locations of coverage centers and $E_x$ is edge set that shows whether the coverage areas of the considered satellites are connected. The edge set $E_x$ can be expressed as:
\begin{equation}
    E_x=\left\{\overline{\xi\xj}:    
    \left\{\begin{matrix}
 \angle \xi\oe\xj\leq 2\gamma \,\\
{\rm or}\\
\wideparen{\xi\xj}\leq 2r_e \gamma \, \\
{\rm or}\\
\|\xi\xj\|\leq 2r_e\sin\gamma

\end{matrix}\right\}\right\},
\end{equation}
where $\wideparen{\xi\xj}$ and $\|\xi\xj\|$ represent the spherical distance and Euclidean distance, respectively, between $\xi$ and $\xj$.\\
\indent In this paper, we propose to project the earth's surface onto a 2D plane which is tangent to the earth on the South Pole $\textbf{w}_S$. The random graph on the projection plane, which corresponds to $G_x(V_x,E_x)$, is defined as $G_z(V_z,E_z)$, where $V_z$ is the vertex set and $E_z$ is the edge set. We let $K_z\subset G_z(V_z,E_z)$ denote a connected component inside $G_z$ and let $K_z(0)$ denote the connected component covering the origin $\textbf{o}_z$ on the projection plane. On a 2D plane, percolation probability is commonly defined as the probability of generating a giant component whose set cardinality is infinite, \ie $\mathbb{P}\{|K_z|=\infty\}$. In our system, we define the percolation probability of LEO satellite coverage areas as a function of the number of satellites $N$ and the coverage angle $\gamma$, that is
\begin{equation}
    \theta(N,\gamma)=\mathbb{P}\{|K_z(0)|=\infty\},
\label{defineper}
\end{equation}
where the 2D connected component $K_z$ on a plane is projected from the 3D connected component $K_x$. Therefore, considering a fixed coverage angle $\gamma$ of each LEO satellite, the design objective of the considered system is:
\begin{equation}
    \begin{array}{ll}
       \text{ minimize}  & N  \\
       \text{ subject to}  & \theta(N,\gamma)>0.  \\
    \end{array}
    \label{design1}
\end{equation}
Similarly, considering a fixed number of LEO satellites deployed on the same altitude, we can also formulate the design objective as:
\begin{equation}
    \begin{array}{ll}
       \text{ minimize}  & \gamma  \\
       \text{ subject to}  & \theta(N,\gamma)>0. \\
    \end{array}
    \label{design2}
\end{equation}
It is worth noting that, the coverage angle $\gamma$ depends on the altitude $h$, maximum slant range $d_m$ or nadir angle $\eta$.\\
\indent In conclusion, using the tools of percolation theory, we mainly study the necessary conditions to form large-scale connected coverage areas on the earth. For ease of reading, we summarize most of the symbols in Table \ref{tab:TableOfNotations}.

\begin{table*}[htbp]
\caption{Table of Notations}
\centering
\begin{center}
\resizebox{\textwidth}{!}{
\renewcommand{\arraystretch}{1}%1.4
    \begin{tabular}{ {c} | {l} }
    \hline
        \hline
    \textbf{Notation} & \textbf{Description} \\ \hline
    $\textbf{o}_e$; $r_e$; $\textbf{w}_N$; $\textbf{w}_S$ & The center of earth; the radius of earth; the North Pole of earth; the South Pole of earth \\ \hline
    $h$; $r_s$; $N$ & The altitude of LEO satellites; the radius of LEO satellites' orbit; the number of LEO satellites\\ \hline
    $\Phi$; $\textbf{y}_i$; $\textbf{x}_i$ & The set of LEO satellites' locations; the location of the $i$th LEO satellite; the projection of the $i$th LEO satellites on the earth\\ \hline
    $\eta$; $\gamma$; $\epsilon$ & The nadir angle of LEO satellites; the coverage angle on the earth; the minimum elevation angle of users\\ \hline
    $p_{\rm cov}$; $p_{\rm ncov}$& The probability of each point on the earth being covered; the probability of each point on the earth being not covered \\ \hline
     $V_x$; $E_x$& The set of vertices (\ie coverage centers); the edge set which represents the connection between coverage areas\\ \hline
    $G_x(V_x,E_x)=\{V_x,E_x\}$& The random graph containing the vertex set $V_x$ and edge set $E_x$ on the earth (sphere)\\ \hline
    $G_z(V_z,E_z)=\{V_z,E_z\}$& The corresponding random graph of $G_x$ on the projection plane\\ \hline
    $K_x$; $K_x(0)$& The connected component on the sphere; the connected component on the sphere containing the South Pole $w_S$\\ \hline
    $K_z$; $K_z(0)$& The projected connected component on the considered projection plane; the $K_z$ containing the origin on the plane $\textbf{o}_z$ \\ \hline
    $\mathcal{F}$; $\mathcal{F}^{-1}$& The stereographic projection; the inverse stereographic projection\\ \hline
    $\theta(N,\gamma)$ & The percolation probability related to $N$ and $\gamma$\\ \hline
     \hline
    \end{tabular}
}
\end{center}
\label{tab:TableOfNotations}
%\vspace{-8mm}
\end{table*}

\section{Coverage Analysis} \label{sec:coverage}
\indent In this section, we first investigate the coverage analysis of LEO satellites. Then, we introduce how to project LEO satellite coverage areas on the sphere onto a tangent plane. Based on the stereographic projection, we define the percolation on the sphere and introduce the sub-critical and super-critical cases.

\subsection{Coverage Analysis of LEO Satellites}
\indent In Sec. \ref{sec:SysMod}, we introduce two different expressions of satellite coverage areas $\mathcal{A}_i=\mathcal{A}(\yi,\eta)$ and $\mathcal{O}_i=\mathcal{O}(\xi,\gamma)$. Because they represent the coverage area of the same LEO satellite, $\mathcal{O}(\xi,\gamma)=\mathcal{A}(\yi,\eta)$ must be satisfied. As shown in Fig.\ref{fig:etagamma}, we can obtain the relationships between $\gamma$, $\eta$, $h$ and $d_m$ as described in the below lemma.
\begin{figure}
    \centering
    \includegraphics[width=0.6\linewidth]{Figures/etagamma.pdf}
    \caption{The geometric relationships between coverage angle $\gamma$, nadir angle $\eta$, satellite constellation altitude $h$ and maximum slant range $d_m$. }
    \label{fig:etagamma}
\end{figure}


\begin{lemma}\label{lem:etagamma}
   The relationships between the coverage angle $\gamma$, nadir angle $\eta$, constellation altitude $h$ and maximum slant range $d_m$ can be expressed as:
   \begin{equation}
       \gamma=\arcsin(\frac{d_m}{r_e}\sin\eta),
   \end{equation}
where
\begin{equation}
    d_m=-\sqrt{r_e^2-r_s^2\sin^2\eta}+r_s \cos\eta
\end{equation}
and
\begin{equation}
    0<\eta\leq \arcsin{\frac{r_e}{r_s}},\, r_s=r_e+h.
\end{equation}
\end{lemma}
\begin{IEEEproof}
    See Appendix~\ref{app:etagamma}.
\end{IEEEproof}
 Next, we introduce the coverage analysis of each point on the sphere in Theorem \ref{theo:covpro}.
\begin{theorem}\label{theo:covpro} Assume that the number of LEO satellites is $N$ and the coverage angle of each satellite is $
\gamma$. The probability of each point on the sphere being covered by at least one LEO satellites is:
    \begin{equation}
        p_{\rm cov}(N,\gamma)=1-\bigg(\frac{1+\cos\gamma}{2}\bigg)^{N}.
    \end{equation}
    \label{theo:pcov}
Correspondingly, the probability of each point on the sphere being not covered by any LEO satellite is:
    \begin{equation}
        p_{\rm ncov}(N,\gamma)=\bigg(\frac{1+\cos\gamma}{2}\bigg)^{N}.
    \end{equation}
    
\end{theorem}
\begin{IEEEproof}
    See Appendix~\ref{app:pcov}.
\end{IEEEproof}
It is worth noting that, for any point on the sphere, the sum of probabilities of being covered and being not covered equals 1, \ie $p_{\rm cov}+p_{\rm ncov}=1$. However, for any area $\mathcal{B}$ on the plane, we focus on the probability of it being completely covered or completely not covered to analyze the sub-critical case and super-critical case. Therefore, if we mention an event where $\mathcal{B}$ is `covered' or `not covered', they represent $\mathcal{B}$ is `completely covered' or `completely not covered'. These two probabilities must satisfy:  
\begin{equation}
    \P\{\mathcal{B} {\rm \; is\;covered}\}+\P\{\mathcal{B} {\rm \; is\;not\;covered}\}\leq 1
\end{equation}
because $\P\{\mathcal{B} {\rm \; is\;partially\;covered}\}\geq 0$.

\subsection{Percolation through Stereographic Projection}
\indent To analyze the percolation on the sphere, we need to separate the whole sphere using some special lattice. Classical percolation analysis is always based on triangular, square or hexagonal lattices. Unlike a plane, the sphere can not be divided using a homogeneous lattice. Mercator projection in Fig.\ref{fig:caparea} can map the sphere on a square area, which is commonly used in geography \cite{snyder1987map}, however, the spherical coverage areas of LEO satellites become irregular and not tractable. Therefore, we propose to use the stereographic projection to analyze the percolation on the sphere \cite{snyder1987map}, which is a specific example of Alexandroff extension mapping a sphere onto a plane \cite{willard2012general}. The stereographic projection is introduced in Theorem \ref{theo:stereo}.

\begin{theorem}
\label{theo:stereo}
    As shown in Fig.\ref{fig:stereographic}, $P(x,y,z)$ is a point on the sphere and $P'(x',y',z')$ is its stereographic projection on the projection plane. The stereographic projection $P'=\mathcal{F}(P)$ leads to:
\begin{equation}
\begin{array}{r@{}l}
(x',y',z')
=\displaystyle\bigg(\frac{2r_e}{2r_e-z}x,\frac{2r_e}{2r_e-z}y,0\bigg),
\end{array}
\end{equation}
and the inverse stereographic projection $P=\mathcal{F}^{-1}(P')$ leads to:
\begin{equation}
\begin{array}{l}
    (x,y,z)\\=\displaystyle\bigg(\frac{4r_e^2 x'}{4r_e^2+x'^2+y'^2},\frac{4r_e^2 x'}{4r_e^2+x'^2+y'^2},\frac{2r_e(x'^2+y'^2)}{4r_e^2+x'^2+y'^2}\bigg).
\end{array}
\end{equation}
\end{theorem}
\begin{figure}[ht]
    \centering
    \includegraphics[width=0.75\linewidth]{Figures/caparea.pdf}
    \caption{The area of spherical cap and the Mercator projection. }
    \label{fig:caparea}
\end{figure}
\begin{figure}
    \centering
    \includegraphics[width=0.65\linewidth]{Figures/stereographic.pdf}
    \caption{The stereographic projection. The projection plane is on the $xo_zy$ plane, which is tangent to the earth on the South Pole $\textbf{w}_S(\textbf{o}_z)$. The earth's center $\textbf{o}_e$ and North Pole $\textbf{w}_N$ are both on the z-axis. $P'$ is the stereographic projection of $P$. Any circle on the sphere corresponds to a circle on the projection plane. If the spherical cap excludes the North Pole $\textbf{w}_N$, the spherical cap is projected to a finite circular area. Inversely, any finite circular area corresponds to a spherical cap excluding $\textbf{w}_N$.}
    \label{fig:stereographic}
\end{figure}
\begin{IEEEproof}
    Notice that the South Pole of the sphere is the origin of the projection plane, \ie $\textbf{w}_S=\textbf{o}_{z}$. The coordinate relationship in the stereographic projection and inverse stereographic projection can be easily proved using $\overrightarrow{w_{N}P}=\frac{2r_e-z}{2r_e}\overrightarrow{w_{N}P'}$.  
\end{IEEEproof}
Except for the North Pole $w_N$, the stereographic projection is a bijection between a sphere and a plane. Therefore, we introduce a property of stereographic projection in Lemma \ref{lem:subset}.

\begin{lemma}\label{lem:subset}
    Define the mapping from the sphere to the plane through stereographic projection as a function $\mathcal{F}$. For two spherical areas on the sphere $\mathcal{B}$ and $\mathcal{C}$, their projections on the plane satisfy:
    \begin{equation}
        \mathcal{B}\subseteq \mathcal{C} \Leftrightarrow \mathcal{F}(\mathcal{B})\subseteq \mathcal{F}(\mathcal{C}).
        \label{mappingrelation}
    \end{equation}
    \label{lem:mappingrelation}
\end{lemma}

\begin{IEEEproof}
    As a kind of bijection, stereographic projection, except for the North Pole  $\textbf{w}_N$, has the same property as bijection.
\end{IEEEproof}
\begin{remark}
    Notice that a closed shape on the sphere is not always projected to a closed shape on the projection plane. Once the North Pole is included in the $\mathcal{B}$, the size of projection $\mathcal{F}(\mathcal{B})$ goes to infinite. However, the property (\ref{mappingrelation}) still holds.
\end{remark}

% Based on Lemma \ref{lem:subset}, we can show the below corollary:
% \begin{corollary}
%     If a spherical area $\mathcal{B}$ is covered by a LEO satellite's coverage area $\mathcal{A}_i$, \ie $\mathcal{B}\subseteq\mathcal{A}_i$, their projections on the plane also satisfy $\mathcal{F}(\mathcal{B})\subseteq\mathcal{F}(\mathcal{A}_i)$, vice versa, that is:
%     \begin{equation}
%         \mathcal{B}\subseteq \mathcal{A}_i \Leftrightarrow \mathcal{F}(\mathcal{B})\subseteq \mathcal{F}(\mathcal{A}_i).
%         \label{corsubset2}
%     \end{equation}
% \label{cor:mapspherical}
% \end{corollary}
\begin{figure}
    \centering
    \includegraphics[width=0.8\linewidth]{Figures/Hexagons.pdf}
    \caption{Hexagonal lattice on the projected plane. The side length of hexagons is $a$ and $\mathcal{H}_0$ is the hexagon which is centered at the origin.}
    \label{fig:hexagon}
\end{figure}
\indent To make the percolation on the sphere a tractable problem, we propose to discuss the percolation on the stereographic projection plane. As shown in Fig.\ref{fig:hexagon}, we define the homogeneous hexagons on the plane as $\mathcal{H}_l$'s with the side length $a$. Through inverse stereographic projection, we can also find the original area of $H_l$ on the sphere, \ie $\mathcal{F}^{-1}(\mathcal{H}_l)$. For percolation on the sphere, we focus on whether $\mathcal{F}^{-1}(\mathcal{H}_l)$ can be covered by the LEO satellites' coverage areas, \ie $\mathcal{A}_i$'s. Using the property (\ref{mappingrelation}), the problem is equivalent to whether the hexagon $\mathcal{H}_l$ on the plane can be covered by the projections of $\mathcal{A}_i$'s, \ie $\mathcal{F}(\mathcal{A}_i)$'s.
% \begin{corollary}
%     If a hexagon on the plane $\mathcal{H}_l$ is covered by the projection of a LEO satellite's coverage area $\mathcal{F}(\mathcal{A}_i)$, \ie $\mathcal{H}_l\subseteq \mathcal{F}(\mathcal{A}_i)$, their original areas on the sphere also satisfy $\mathcal{F}^{-1}(\mathcal{H}_l)\subseteq\mathcal{A}_i$, vice versa, that is:
%     \begin{equation}
%         \mathcal{H}_l\subseteq \mathcal{F}(\mathcal{A}_i) \Leftrightarrow \mathcal{F}^{-1}(\mathcal{H}_l)\subseteq\mathcal{A}_i.
%         \label{corsubset}
%     \end{equation}
% \label{cor:maphexagon}
% \end{corollary}
 On a plane, the face percolation of hexagons means there exist giant components whose cardinality is infinite. As shown in (\ref{defineper}), percolation probability is always defined as the probability of generating a giant component that crosses the origin. Through inverse stereographic projection, such a giant 
 component is projected from a continuous coverage area from the South Pole to the North Pole. This also represents the `farthest coverage on the earth'. Therefore, we propose to define the percolation probability on the sphere as below.
\begin{definition}
    On a sphere, percolation probability is defined as the probability of generating continuous coverage areas that contain two symmetry points about the sphere's center. Especially, we can also define it using the probability of connecting the North Pole and the South Pole of earth, \ie
\begin{equation}
    \theta(N,\gamma)=\P\{\textbf{w}_N\in K_x(\textbf{w}_S)\}.
\end{equation} 
where $K_x$ denotes the giant component on the sphere, $\textbf{w}_S$ and $\textbf{w}_N$ represent the South Pole and the North Pole, respectively. $K_x(\textbf{w}_S)$ represents the connected component starting from the South Pole.
\label{def:concomsphere}
\end{definition}

\begin{remark}
    On the earth, the spherical distance between any two points is less than or equal to $\pi r_e$, that is, the two points that are symmetric about the earth's center have the maximum spherical distance. Unlike the analysis on an infinite plane, the cardinality of the connected component will not reach infinity. The farthest spherical distance it can reach is determined, which can be used as a judgement of percolation. In addition, the cardinality of the connected component has its upper bound, that is, the entire sphere.
\end{remark}
\indent As a basic knowledge of hexagonal face percolation, the sufficient and necessary condition for face percolation of hexagons is that the probability of each hexagon being covered should be larger than $\frac{1}{2}$, \ie
\begin{equation}
    \theta(N,\gamma)>0,\,{\rm if}\,\P\{\mathcal{H}_l\,{\rm is\, covered}\}>\frac{1}{2}.
\end{equation}
\indent It is difficult to calculate $\P\{\mathcal{H}_l\,{\rm is\, covered}\}$ directly because the shape of $\mathcal{F}^{-1}(\mathcal{H}_l)$ is irregular. However, we can use some circular areas to help find the tight upper bound and lower bound of it. At the same time, the coverage areas of LEO satellites are typically modeled as circular areas. Therefore, we introduce the below lemma.
\begin{lemma}
    For a spherical cap on the sphere, its projection on the plane is a circular area, unless the border of the spherical cap crosses the top of the sphere.
    Inversely, for each circular area on the projected plane, its inverse projection on the sphere is a circular area. 
    \label{lem:captocircle}
\end{lemma}
\begin{IEEEproof}
    See the proof in \cite[88.1]{pedoe2013geometry}. %See Appendix~\ref{app:captocircle}.
\end{IEEEproof}
\begin{remark}
    For stereographic projection, the North Pole $\textbf{w}_N$ is considered the top of the earth. There exist three cases: \textbf{i) the spherical cap excludes the top}, where the projection is a closed circular area on the plane, \textbf{ii) the border of the spherical cap crosses the top}, where the projection is a region divided by a straight line that does not include the origin $\textbf{o}_z$, \textbf{iii) the spherical cap includes the top}, where the projection is open and its inner envelope is a circular area on the plane. On the other hand, for inverse stereographic projection, closed circular areas on the plane can be always projected to spherical caps on the sphere, which does not contain the North Pole $\textbf{w}_N$.
\end{remark}
% According to Lemma \ref{lem:captocircle}, we introduce the below corollary:
% \begin{corollary}
%     For a LEO satellite's coverage area $\mathcal{O}(\xi,\gamma)=\mathcal{A}(\yi,\eta)$, its projection on the plane is a circular area unless the angle $\angle \textbf{w}_{N}\textbf{o}_e\yi=\gamma$ where $\textbf{w}_{N}$ is the top of the sphere.
% \end{corollary}
% \begin{IEEEproof}
%     Let the coverage area $\mathcal{O}(\xi,\gamma)$ be the spherical cap in Lemma \ref{lem:captocircle}, its coverage is a circular area on the plane. When $\angle \textbf{w}_{N}\textbf{o}_e\yi=\gamma$, the considered plane crosses the top of the sphere, and the projection of the LEO satellite's coverage area is a straight line on the plane.
% \end{IEEEproof}
% \indent Therefore, almost all LEO satellites' coverage areas on the sphere are projected to circular areas on the plane. We can obtain the radius of these coverage areas. Next, we introduce Lemma \ref{lem:circularradius}:
\indent In this paper, we need to first analyze the property of circular areas on the projection plane. As introduced in Lemma \ref{lem:captocircle}, their inverse projections on the sphere can be always modeled as circular areas that does not contain the North Pole $\textbf{w}_N$. So that, we introduce the relationship between central angle of a spherical cap and the radius of its projected circular area.


\begin{lemma}\label{lem:circularradius}
     For a spherical cap that does not contain the North Pole $\textbf{w}_N$ with a central angle $\gamma_0$, the radius of its projected circular area on the projection plane can be expressed as:
\begin{equation}
    r(\psi)=r_e\bigg|\tan(\frac{\psi+\gamma_0}{2})-\tan(\frac{\psi-\gamma_0}{2})\bigg|        
\label{radiusprojection}
\end{equation}
where $\psi=\angle \textbf{w}_S \oe \textbf{x}<\pi-\gamma_0$ and $\textbf{x}$ is the center of the spherical cap. The range of $r(\psi)$ is:
\begin{equation}
    2r_e\tan(\frac{\gamma}{2})\leq r(\psi)<+\infty.
\end{equation}
Conversely, for a circular area on the projection plane with a radius $r$, the central angle of its original spherical cap is upper and lower bounded as follows:
\begin{equation}
    0<\gamma_0\leq 2\arctan(\frac{r}{2r_e}).
\end{equation}
\end{lemma}
\begin{IEEEproof}
    See Appendix~\ref{app:circularradius}.
\end{IEEEproof}
\indent Lemma \ref{lem:captocircle} and Lemma \ref{lem:circularradius} already exhibit how to project the spherical caps on the sphere to its tangent projection plane, which are used to do the critical analysis in the next section.
% Therefore, another corollary is introduced:
% \begin{corollary}
%     Define $\tilde{\mathcal{O}}(\textbf{z}_l,r_l)$ as a circular area centered at $\textbf{z}_l$ with radius $r_l$ on the considered plane, its original area is $\mathcal{F}^{-1}(\tilde{\mathcal{O}}(\textbf{z}_l,r_l))$ is a spherical cap on the sphere. When the center of $\tilde{\mathcal{O}}(\textbf{z}_l,r_l)$  is infinite far from the origin $\textbf{o}_z$, the area of $\mathcal{F}^{-1}(\tilde{\mathcal{O}}(\textbf{z}_l,r_l))$ and its center angle $\gamma_l$ are both 0. When the center of $\tilde{\mathcal{O}}(\textbf{z}_l,r_l)$ is the origin $\textbf{o}_z$, the area of $\mathcal{F}^{-1}(\tilde{\mathcal{O}}(\textbf{z}_l,r_l))$ and its center angle both achieve their maximum.
% \label{cor:areaofcircularplane}
% \end{corollary}
% \begin{IEEEproof}
%     Define the distance between $\textbf{z}_l$ and $\textbf{o}_z$ is $|\textbf{o}_z\textbf{z}_l|$. The same as (\ref{radiusprojection}), the radius of circular area on the plane is an increasing function of $\psi_l$ and $\gamma_l$ at the same time, \ie 
%     \begin{equation}
%     r_{l}=r_e\bigg|\tan(\frac{\psi_l+\gamma_l}{2})-\tan(\frac{\psi_l-\gamma_l}{2})\bigg|.
%     \end{equation}
%     If $r_l$ is fixed, the center angle of the original spherical cap $\gamma_l$ can be considered a decreasing function of $\psi_l$, where the increase in $\psi_l$ leads to the increase in $|\textbf{o}_z\textbf{z}_l|$ as well. Therefore, for a circular area on the considered plane with a fixed radius, the center angle $\gamma_l$ and the original area on the sphere achieve their maximum when $|\textbf{o}_z\textbf{z}_i|=0$. Similarly, $\gamma_0$ and the area of the original spherical cap both achieve their minimum 0 when $|\textbf{o}_z\textbf{z}_i|=\infty$. 
% \end{IEEEproof}


\section{Critical Analysis}\label{sec:critical}
In this section, we first prove that percolation probability is a non-decreasing function of the number of satellites. Next, We introduce the sub-critical and super-critical cases where the percolation probability is zero and non-zero, respectively. Based on these, we prove the existence of the critical number of LEO satellites to realize the phase transition of percolation probability on the sphere. After that, we discuss the critical case and derive a closed-form expression of the critical satellite number $N_c$.  
\subsection{Phase Transition}\label{subsec:phasetransition}
  
 \indent In order to prove the existence of phase transition of percolation probability and derive the critical number of satellites, we first introduce the relationship between the percolation probability $\theta$ and the number of satellites $N$ in the below lemma.
\begin{lemma}\label{lem:nondecreasing}
    When the LEO satellites are deployed at the same altitude following a BPP with a coverage angle $\gamma$, the percolation probability and the number of LEO satellites satisfy:
    \begin{equation}
        \theta(N_1,\gamma)\leq\theta(N_2,\gamma), \;{\rm for}\; 0<N_1<N_2,
    \end{equation}
when the value of $\gamma$ is fixed.
\end{lemma}
\begin{IEEEproof}
    See Appendix~\ref{app:nondecreasing}.
\end{IEEEproof}
\indent Therefore, the percolation probability $\theta(N,\gamma)$ does not decrease as $N$ increases. Next, we introduce a sub-critical case to obtain a lower bound $N_L$ of the critical number of LEO satellites, where percolation probability is zero when $N\leq N_L$.\\

\noindent \textbf{Sub-critical case:} We choose a certain meridian (e.g. the prime meridian). The LEO satellites are deployed along the longitude and the borders of two adjacent coverage areas are tangent. If the longest spherical distance inside the covered areas is less than $\pi r_e$, the probability of percolation is 0. Therefore, we first introduce the sufficient condition for zero percolation probability in the below lemma. 
\begin{lemma}\label{lem:lowerbound}
    When the number of LEO satellites is less than $N_L$, the percolation probability $\theta(N,\gamma)$ is zero, \ie
\begin{equation}
    \theta(N,\gamma)=0\;{\rm if}\;N\leq N_L.
\end{equation}    
    The expression of $N_L$ is
\begin{equation}
    N_L = \left \lfloor \frac{\pi}{2\gamma} \right \rfloor 
\label{lowerbound}
\end{equation}
where $\left \lfloor x \right \rfloor$ denotes the largest integer less than x, and $\gamma$ is the coverage angle of each LEO satellite.
\end{lemma}
\begin{IEEEproof}
    As shown in Fig.\ref{fig:lowerbound}, when $N\leq\left \lfloor \frac{\pi}{2\gamma} \right \rfloor$, even though the satellites are deployed on the same orbit, any continuous coverage path containing $\textbf{w}_N$ and $\textbf{w}_S$ can not be generated.
\end{IEEEproof}
\begin{figure}
    \centering
    \includegraphics[width=0.4\linewidth]{Figures/Lowerbound.pdf}
    \caption{Sub-critical case: All satellites are deployed on the same meridian, where neighbour coverage areas are tangent to each other. However, the longest spherical distance inside the coverage areas does not exceed $\pi r_e$ when the number of satellites is not large enough.}
    \label{fig:lowerbound}
\end{figure}

\begin{corollary}
    If the critical number of satellites $N_c$ for the phase transition of percolation probability exists, $N_L$ is the lower bound of $N_c$, \ie $N_c\geq N_L$.
    \label{cor:subcritical}
\end{corollary}

Next, in order to obtain an upper bound of the critical number of LEO satellites, where percolation probability is non-zero when $N\geq N_U$, we need to ensure that the percolation probability has a computable and non-zero lower bound when $N=N_U$. Therefore, we introduce a super-critical case as shown below.\\ 

\noindent \textbf{Super-critical case:} This super-critical case is designed in six steps: \textit{i)} Along the meridians, we divide the whole sphere into $2m$ `slices', where each slice spans $\frac{\pi}{m}$ of longitude. \textit{ii)} Each two slices symmetric about the earth's center can be contained by a `belt'. Therefore, $m$ of belts can cover the whole sphere. \textit{iii)} Rotate a belt and make it symmetric about the equatorial plane, it can be considered a belt spanning $\frac{\pi}{m}$ of latitude. \textit{iv)} By dividing such a belt into $n$ uniform `pieces', we can use in total $m\times n$ pieces to cover the whole sphere. Each piece spans $\frac{\pi}{m}$ of longitude and $\frac{2\pi}{n}$ of latitude. \textit{v)} Each piece can be contained by a spherical cap which is smaller than the coverage area of a satellite. \textit{vi)} Use the $m\times n$ of satellites to cover the target spherical caps one by one. The steps from i) to v) are shown in Fig.\ref{fig:Upperbound}, which explains how to represent the whole sphere using the union of spherical caps. 
\\
\indent In the super-critical case, we need to design $m$ and $n$ large enough to make such a full coverage deployment feasible and obtain a computable non-zero lower bound of percolation probability. Therefore, we introduce the sufficient condition for non-zero percolation probability in the below lemma.
\begin{lemma}\label{lem:upperbound1}
    When the number of LEO satellites is larger than $N_U$, the percolation probability $\theta(N,\gamma)$ is non-zero, \ie
\begin{equation}
    \theta(N,\gamma)>0\;{\rm if}\;N\geq N_U. 
\end{equation}    
    The expression of $N_U$ is 
\begin{equation}
    N_U = m\times n 
\label{upperbound}
\end{equation}
with
\begin{equation}
    m = \left \lceil \frac{\pi}{\gamma} \right \rceil,\;n= \left \lceil \frac{\pi}{\arccos{\frac{\cos\gamma}{\cos\frac{\pi}{2m}}}} \right \rceil +1
\label{upperboundmn}
\end{equation}
where $\left \lceil x \right \rceil$ denotes the smallest integer greater than x and $\gamma$ is the coverage angle of each LEO satellite.
\end{lemma}
\begin{IEEEproof}
    See Appendix~\ref{app:upperbound}. 
\end{IEEEproof}
\begin{corollary}
    If the critical number of satellites $N_c$ exists, $N_U$ is the upper bound of $N_c$, \ie $N_c\leq N_U$.
    \label{cor:supercritical}
\end{corollary}
\begin{figure}
    \centering
    \includegraphics[width=1\linewidth]{Figures/Upperbound.pdf}
    \caption{Super-critical case: a full coverage scheme. Above (from left to right): \textit{a)} The whole sphere is firstly divided into $2m$ slices. \textit{b)} Because each belt can contain two slices that are symmetric, the whole sphere can be considered as the union of $m$ belts. \textit{c)} Rotate the belt. Below (from left to right): \textit{d)} Each belt can be divided into $n$ pieces. \textit{e)} The whole sphere can be consider the union of $m\times n$ pieces. \textit{f)} Because each piece can be contained by a spherical cap, the whole sphere can be considered the union of $m\times n$ spherical caps.}
    \label{fig:Upperbound}
\end{figure}
% \begin{corollary}
%     The upper bound in Lemma \ref{lem:upperbound1} shows that: if the number of LEO satellites is greater than $N_U$, \ie $N>N_L$ the percolation probability $\theta(N,\gamma)>0$.
%     \label{cor:supercritical}
% \end{corollary}
% \begin{remark}
%     When the number of satellites tends to $\infty$, the coverage probability of each point on the sphere $p_{cov}$ is almost surely 1. In this case, the percolation probability is almost surely 1, \ie $\theta(\infty,\gamma)=1$, almost surely. This extreme case can be used to further prove that the existence of upper bound of the critical number of satellites.
%     \label{rem:upperboundinfty}
% \end{remark}

 Based on the sub-critical and super-critical cases, we can prove the existence of the critical value of $N$, \ie $N_c$, in the following lemma.

\begin{lemma}\label{lem:phasetransition}
    When the LEO satellites are deployed at the same altitude following a BPP with a fixed value of $\gamma$, there exists a critical value $N_c$ satisfying:
    \begin{equation}
    \begin{array}{c}
        \theta(N,\gamma)=0,\, {\rm for}\; N<  N_c,\\
        \theta(N,\gamma)>0,\, {\rm for}\; N> N_c. 
    \end{array}
    \end{equation}
where $N_L\leq \left\lfloor N_c\right\rfloor$ and $\left\lceil N_c\right\rceil\leq N_U$.
\end{lemma}
\begin{IEEEproof}
    See Appendix~\ref{app:phasetransition}.
\end{IEEEproof}
Therefore, we prove the existence of the critical value of $N$, \ie $N_c$, which exhibits the phase transition of percolation probability.
\subsection{Tight bounds and critical analysis}
 In Lemma \ref{lem:phasetransition}, we have proved that the critical number of LEO satellites for phase transition of percolation probability exists. In this part, we propose to use the stereographic projection to find a tight lower bound and a tight upper bound for $N_c$, and introduce the closed-form expression of $N_c$.\\
 
 \noindent\textbf{Hexagonal face percolation on the projection plane:} As shown in Definition \ref{def:concomsphere}, the percolation on the sphere containing the South Pole $\textbf{w}_S$ represents the percolation on the projection plane containing the origin $\textbf{o}_z$. We first consider the hexagons with side length $a$. In percolation theory, if all hexagonal faces have the same probabilities $\P\{\mathcal{H}_{l} {\rm \; is\;open}\}$ and $\P\{\mathcal{H}_{l} {\rm \; is\;closed}\}$, we have: \textit{i) $\theta=0$ when $\P\{\mathcal{H}_{l} {\rm \; is\;closed}\}>1/2$ }and\textit{ ii) $\theta>0$ when $\P\{\mathcal{H}_{l} {\rm \; is\;open}\}>1/2$}. To find the tight bounds, we first introduce the below theorem.

 \begin{figure}
    \centering
    \includegraphics[width=0.6\linewidth]{Figures/Gamma.pdf}
    \caption{The minimum circumscribed circle of the hexagon on the projected plane, which is centered at the origin of the projection plane. Its central angle of the corresponding original spherical cap is the maximum one, that is $\gamma_m$. }
    \label{fig:gamma}
\end{figure}

\begin{theorem}
    Consider the hexagonal lattice on the plane where the side length of each hexagon is $a$. If the coverage probabilities of different hexagons are different,
    the sufficient condition for non-zero face percolation probability is:
\begin{equation}
    \P\{\mathcal{H}_{l} {\rm \; is\;open}\}>1/2,
\label{sufficientcover}
\end{equation}
and the sufficient condition for zero face percolation probability is:
\begin{equation}
    \P\{\mathcal{H}_{l} {\rm \; is\;closed}\}>1/2.
\label{sufficientnotcover}
\end{equation}
\label{theo:inhomohexagon}
\end{theorem}
\begin{IEEEproof}
    See Appendix~\ref{app:inhomohexagon}.
\end{IEEEproof}
\indent Next, we introduce the lower bounds $\P\{\mathcal{H}_{l} {\rm \; is\;open}\}$ and $\P\{\mathcal{H}_{l} {\rm \; is\;closed}\}$ in the below lemma.

\begin{lemma}\label{lem:boundsforhexagons}
Let $\P\{\mathcal{H}_{l} {\rm \; is\;open}\}$ denote the probability of a hexagon $\mathcal{H}_{l}$ on the projection plane being covered by LEO satellites. 
The lower bound of $\P\{\mathcal{H}_{l} {\rm \; is\;open}\}$ is shown as:
\begin{equation}
\begin{array}{c}
    \P\{\mathcal{H}_{l} {\rm \; is\;open}\}\geq\displaystyle 1-\bigg(\frac{1+\cos(\gamma-\gamma_m)}{2}\bigg)^{N}, 
\end{array}
\label{mincoverageprob}
\end{equation}
where
\begin{equation}
    \gamma_m=2\arctan \frac{a}{2 r_e}.
\end{equation}
Let $\P\{\mathcal{H}_{l} {\rm \; is\;closed}\}$ denote the probability of the hexagon $\mathcal{H}_{l}$ on the projection plane being not covered by LEO satellites. The lower bound of $\P\{\mathcal{H}_{l} {\rm \; is\;closed}\}$ is shown as:
\begin{equation}
\begin{array}{c}
    \P\{\mathcal{H}_{l} {\rm \; is\;closed}\}\geq\displaystyle\bigg(\frac{1+\cos(\gamma+\gamma_m)}{2}\bigg)^{N}.
\end{array}
\label{minnotcoverageprob}
\end{equation}
% where
% \begin{equation}
%   \gamma_m=2\arctan \frac{a}{r_e}.
% \end{equation}
\label{lem:minprobs}
\end{lemma}
\begin{IEEEproof}
    See Appendix~\ref{app:boundsforhexagons}.
\end{IEEEproof}

Substitute (\ref{mincoverageprob}) and (\ref{minnotcoverageprob}) in Lemma \ref{lem:minprobs} into the sufficient conditions for non-zero or zero percolation probability (\ref{sufficientcover}) and (\ref{sufficientnotcover}) in Theorem \ref{theo:inhomohexagon}, we can obtain the sufficient conditions of the number of LEO satellites for non-zero or zero percolation probability that are shown in the below theorem.
\begin{theorem}
Given that the coverage angle of each satellite is $\gamma$, $r_e$ is the radius of the earth and $a$ is the side length of hexagons on the projection plane. The sufficient condition of the number of LEO satellites for non-zero percolation probability is:
\begin{equation}
    N>N_c^U
\end{equation}
where 
\begin{equation}
    N_c^U=\frac{\ln 2}{\ln 2-\ln(1+\cos(\gamma-2\arctan \frac{a}{2r_e}))}
\end{equation}

\noindent is the upper bound of critical number of LEO satellites for phase transition of percolation probability, \ie $N_c\leq N_c^U$.\\
\indent The sufficient condition of the number of LEO satellites for zero percolation probability is:
\begin{equation}
    N<N_c^L
\end{equation}
where 
\begin{equation}
    N_c^L=\frac{\ln 2}{\ln 2-\ln(1+\cos(\gamma+2\arctan \frac{a}{2r_e}))}
\end{equation}
is the lower bound of critical number of LEO satellites for phase transition of percolation probability, \ie $N_c\geq N_c^L$.
\label{theo:sufficientconditions}
\end{theorem}
\begin{IEEEproof}
The upper bound $N_c^{U}$ and lower bound $N_c^{L}$ are obtained by substituting (\ref{mincoverageprob}) and (\ref{minnotcoverageprob}) in Lemma \ref{lem:minprobs} into the sufficient conditions for non-zero or zero percolation probability (\ref{sufficientcover}) and (\ref{sufficientnotcover}) in Theorem \ref{theo:inhomohexagon}, respectively.
\end{IEEEproof}
\indent To analyze the continuous percolation on the sphere, we also consider the continuous percolation on the plane. Therefore, the side length of considered hexagons on the plane is assumed to approach 0. We obtain the explicit expression for the critical number of LEO satellites in the below lemma.
\begin{lemma}
The critical number of LEO satellites for phase transition of percolation probability is:
\begin{equation}
N_c=\frac{\ln 2}{\ln 2-\ln(1+\cos\gamma)}.
\label{criticalNc}
\end{equation}
\label{lem:criticalanalysis}
\end{lemma}
\begin{IEEEproof}
See Appendix~\ref{app:criticalanalysis}. 
\end{IEEEproof}
% \begin{remark}
% We obtain the lower bound $N_L$ and upper bound $N_U$ through coverage analysis on the sphere. We also obtain the tight bounds $N_c^L$ and $N_c^U$ through percolation on the considered projection plane, and further obtain the critical number of LEO satellites for phase transition of percolation probability, \ie $N_c$. It is necessary to verify the relationship between the $N_L$, $N_U$ and $N_c$, which is also shown in the proof of Lemma \ref{lem:criticalanalysis}.
% \end{remark}
$N_c$ is the only explicit value which is always located between bounds $N_c^{L}$ and $N_c^{U}$. At the same time, the upper and lower bounds $N_c^{L}$ and $N_c^{U}$ are both tighter than $N_L$ and $N_U$ when $a$ approaches 0. Theoretically, for $N\geq\left \lceil N_c \right \rceil$, $\theta(N,\gamma)>0$ and for $N\leq\left \lfloor N_c \right \rfloor$, $\theta(N,\gamma)=0$.\\
\indent It is worth noting that, the critical number $N_c$ is the same as the solution of $p_{\rm cov}(N,\gamma)=1/2$ or $p_{\rm ncov}(N,\gamma)=1/2$. When $N>N_c$, $ p_{\rm cov}(N,\gamma)>1/2$. When $N<N_c$, $p_{\rm ncov}(N,\gamma)>1/2$ and $p_{\rm cov}(N,\gamma)<1/2$. Therefore, we obtain the critical condition for phase transition of percolation probability in the below theorem.
\begin{theorem}
    Assume that all points on the sphere has the same probability of being covered, that is $p_{\rm cov}$. 
    The phase transition of continuous percolation on the sphere is expressed as:
\begin{equation}
\begin{array}{r@{}l}
    \theta(p_{\rm cov})=0,\;& for\;p_{\rm cov}<\frac{1}{2}, \\\theta(p_{\rm cov})>0,\;& for\;p_{\rm cov}>\frac{1}{2}. \\
    
\end{array}
\end{equation}
where $p_{\rm cov}$ represents the homogenerous coverage probability on the sphere and $\theta(p_{\rm cov})$ is the percolation probability based on this coverage probability.
\label{theo:perpro_covpro}
\end{theorem}
In this paper, the coverage probability and the percolation probability both depend on the number of LEO satellites $N$ and its coverage angle $\gamma$. We have used the expression $\theta(N,\gamma)$ for percolation probability. Similar to Lemma \ref{lem:criticalanalysis}, for a fixed number of LEO satellites, the relationship between the critical constellation altitude $h^c$ and maximum slant range $d_m$ is shown in the following lemma.
\begin{lemma}
    When the number of LEO satellites is fixed, the critical constellation altitude $h$ can be expressed using the maximum slant range $d_m$:
    \begin{equation}
        h^c = \sqrt{d_m^2-r_e^2+t^2(N) r_e^2}+t(N)r_e-r_e.
        \label{criticalaltitude}
    \end{equation}
    Correspondingly, the critical maximum slant range $d_m^c$ can be also expressed using the constellation altitude $h$:
    \begin{equation}
        d_m^c=\sqrt{r_e^2+(r_e+h)^2-2t(N)r_e(r_e+h)},
    \end{equation}
    where $t(N)=2\times(\frac{1}{2})^{\frac{1}{N}}-1$.
\end{lemma}
\begin{IEEEproof}
    From Lemma \ref{lem:criticalanalysis} and Theorem \ref{theo:perpro_covpro}, the critical relationship between $N$ and $\gamma$, (\ref{criticalNc}), can be rewritten as $\cos{\gamma^c}=2\times(\frac{1}{2})^{\frac{1}{N}}-1$, that is, $\gamma^c=\arccos{\big(2\times(\frac{1}{2})^{\frac{1}{N}}-1\big)}$. Using the law of cosines, $\cos{\gamma^c}=\frac{r_s^2+r_e^2-d_m^2}{2r_e r_s}$ where $r_s=r_e+h$, we can obtain the critical relationship between the altitude $h$ and the maximum slant range $d_m$ above.
\end{IEEEproof}
The above lemma is an extension of the optimization problem (\ref{design2}), where the optimal value of $\gamma$ is related to $d_m$ and $h$, which are both important parameters of LEO satellite constellations.
\section{Simulation results and discussion}
In this paper, we aim to prove the relationship between coverage probability of users on the sphere and the percolation probability as we defined. The parameters of three existing LEO satellite constellation: Starlink, Oneweb and Kuiper, that we adopt, are shown in Table.\ref{tab:TableOfParam} \cite{osoro2021techno,cakaj2021parameters}.

\begin{table}[ht]\caption{Parameters of LEO Systems}
\centering
    \begin{tabular}{ {l} | {l} | {l} | {l} }
    \hline
        \hline
    \textbf{Systems} & \textbf{Starlink} & \textbf{Oneweb} & \textbf{Kuiper} \\ \hline
    % \textbf{Number} & 4519 & 648 & 720 \\ \hline
    \textbf{Altitude ($\rm km$)} & 550 & 1200 & 610 \\ \hline
    \textbf{Elevation Angle $\epsilon$ ($^{\circ}$)} & 40.0 & 55.0 & 35.2 \\ \hline
    \textbf{Coverage Angle $\gamma$ ($^{\circ}$)} & 5.20 & 6.14 & 6.58 \\ \hline
    \textbf{Nadir Angle $\eta$ ($^{\circ}$)} & 44.80 & 28.86 & 48.22 \\ \hline
    \textbf{Max Slant Range $d_{m}$ ($\rm km$)} & 809.5 & 1411.9 & 978.5 \\ \hline
    \textbf{Coverage Areas ($\times 10^6\; {\rm km}^2$)} & $1.05$ & $1.46$ & $1.68$ \\ \hline
     \hline
    \end{tabular}
\label{tab:TableOfParam}
%\vspace{-8mm}
\end{table}
 \indent As shown in Fig.\ref{fig:gamma52}, we firstly adopt the Starlink's coverage angle $\gamma=5.2$°. When the number of LEO satellites equals the lower bound (\ref{lowerbound}) in Lemma \ref{lem:lowerbound}, the percolation probability $\theta(N,\gamma)=0$. When the number of LEO satellites equals the upper bound (\ref{upperbound}) in Lemma \ref{lem:upperbound1}, the percolation probability is non-zero. The phase transition of percolation probability is between the lower bound and upper bound. The critical threshold (\ref{criticalNc}) derived in Lemma \ref{lem:criticalanalysis} is slightly higher than the simulated result, but its corresponding percolation probability is extremely low. At this threshold, the coverage probability exceeds 0.5, and percolation probability increases rapidly from a low level (close to 0). This result supports the concept in Theorem \ref{theo:perpro_covpro}.\\
 \begin{figure}
    \centering
    \includegraphics[width=0.8\linewidth]{Figures/Pro_Num_gamma52.pdf}
    \caption{Percolation probability, coverage probability of LEO satellite coverage when $\gamma=5.2$°, with the lower bound $N_L$, upper bound $N_U$, simulated critical value and theoretical value of critical number of LEO satellites $N_c$.}
    \label{fig:gamma52}
\end{figure}
\indent Next, we aim to show the effect of parameters of LEO satellites on the percolation probability. In Fig.\ref{fig:companies}, we adopt the coverage angles $\gamma$ of Starlink, Oneweb and Kuiper. When the number of LEO satellites increases, percolation probability also increases and the derived critical value $N_c$ is the necessary condition for phase transition of percolation probability. For example, Starlink need to provide at least 340 LEO satellites to meet the needs of random large-scale continuous service path, while Oneweb needs 240 LEO satellites and Kuiper needs 200. The critical threshold works well for different values of coverage angle $\gamma$. For realistic applications, the number of LEO satellites also depends on the capacity we need, and our derived critical value is only a necessary condition.\\
\begin{figure}
    \centering
    \includegraphics[width=0.8\linewidth]{Figures/Pro_Num_Companies.pdf}
    \caption{Percolation probability $\theta(N,\gamma)$ versus the number of LEO satellites $N$. Each curve has its corresponding critical number of satellites, $N_c$.}
    \label{fig:companies}
\end{figure}
\indent Considering 500 LEO satellites, we can observe how the altitude $h$ and maximum slant range $d_m$ of LEO satellites affect the percolation probability together. In Fig.\ref{fig:Pro_SLR_Altitude}, we adopt the maximum slant range $d_m$ of Starlink, Oneweb and Kuiper. When the $h$ increases, the percolation probability decreases because the coverage angle $\gamma$ becomes lower. The critical altitude of LEO satellites for phase transition of percolation probability from non-zero to zero is shown in (\ref{criticalaltitude}). We can notice that these three companies already deploy their LEO satellites at suitable altitudes lower than the critical threshold, where 500 LEO satellites can successfully provide large-scale continuous service for any applications.\\
\begin{figure}
    \centering
    \includegraphics[width=0.8\linewidth]{Figures/Pro_SLR_Altitude.pdf}
    \caption{Percolation probability versus the altitude of satellites when $N=500$. Each curve has its corresponding critical constellation altitude $h_c$.}
    \label{fig:Pro_SLR_Altitude}
\end{figure}
\indent Similarly, as shown in Fig.\ref{fig:Pro_Alt_SlantRange}, we consider 500 LEO satellites and the altitudes of Starlink, Oneweb and Kuiper constellations. When the $d_m$ increases, percolation probability increases from zero to non-zero due to the increase in nadir $\eta$ and coverage angle $\gamma$. The maximum slant ranges of these three constellations can already support large-scale continuous service. 
\begin{figure}
    \centering
    \includegraphics[width=0.8\linewidth]{Figures/Pro_Alt_SlantRange.pdf}
    \caption{Percolation probability versus the maximum slant range of LEO satellites when $N=500$. Each curve has its corresponding critical maximum slant range $d_m^c$.}
    \label{fig:Pro_Alt_SlantRange}
\end{figure}

In this paper, we verify that the proposed closed-form expression reflects the phase transition behavior of percolation probability when the number of LEO satellite increases. It is worth noting that, the design of constellation is a complex question, where we also need to consider the service strategy and expense. For example, if a considered LEO constellation can only use 100 LEO satellites to provide continuous service for international flights, the required maximum slant range should be higher than the result when $N=500$. We also need to consider the capacity and dynamic selection 
strategy of LEO satellites. In the future, more realistic simulations through orbital propagation tool are expected to be conducted, and multi-layer structure of LEO satellites and the effect from massive users on the traffic should be considered. For example, the kinds of service requirements from IoT devices and mobile users will also lead to different coverage areas and traffic congestion problem of LEO satellite system, which are expected to be solve through routing algorithm and capacity enhancement.
\section{Conclusion}
This paper is the first attempt to show and prove the concept of percolation on the sphere, especially considering the connections between spherical coverage areas. Using the stereographic projection, we defined the percolation on the sphere using the percolation on the projection plane. We first introduced sub-critical and super-critical cases, where the percolation probability $\theta$ is zero and non-zero respectively. We considered two special deployments and derived the lower bound $N_L$ and upper bound $N_U$ of the critical number of LEO satellites $N$. We proved the existence of the critical condition for phase transition of percolation probability from zero to non-zero. Using the hexagonal face percolation on the projection plane, we derived the tight lower bound $N_c^L$ and upper bound $N_c^U$ for percolation, and obtained the closed-form expression of the critical number of satellites $N_c$. We also obtained the expression of critical condition of altitude $h$ and maximum slant range $d_m$. We conducted the simulations to show how these parameters affect the percolation probability and our derived critical expressions can work well to show the phase transition. We emphasized that the critical expressions we derived are the necessary conditions. However, for realistic applications, it is necessary to consider the dynamic selection strategy, capacity and cost.
\appendices
\section{Proof of Lemma~\ref{lem:etagamma}}\label{app:etagamma}
\indent In the triangle $\triangle \yy\oe\textbf{w}_{0,1}$, the maximum slant range $d_m=\|\yy-\textbf{w}_{0,1}\|$. Using the Law of Cosines, we have:
\begin{equation}
    d_m^2+r_s^2-r_e^2=2d_mr_s\cos\eta.
\end{equation}
Therefore, 
\begin{equation}
    d_m=-\sqrt{r_e^2-r_s^2\sin^2\eta}+r_s\cos\eta.
    \label{cosresult}
\end{equation}
Using the Law of Sines, we have:
\begin{equation}
    \frac{d_m}{\sin\gamma}=\frac{r_e}{\sin\eta}.
    \label{sin}
\end{equation}
Substitute (\ref{cosresult}) into (\ref{sin}), we obtain the relationship between $\gamma$ and $\eta$:
   \begin{equation}
       \gamma=\arcsin(\frac{\sin\eta}{r_e}\bigg(-\sqrt{r_e^2-r_s^2\sin^2\eta}+r_s \cos\eta\bigg)).
   \end{equation}

\section{Proof of Theorem \ref{theo:pcov}}\label{app:pcov}
To achieve the coverage probability of each point on the earth, we need to first calculate the area of the spherical cap. As shown in Fig.\ref{fig:caparea},  we can obtain the area of the spherical cap using the integration in polar coordinates:
\begin{equation}
\begin{array}{r@{}l}
    
\mathcal{S}(\gamma)&=\int_{0}^{\gamma}2\pi r(\theta)\cdot r_e\,\dd \theta=\int_{0}^{\gamma}2\pi r_e^2\sin\theta\,\dd \theta\\
&=\displaystyle2\pi r_e^2\cos\theta|_{\gamma}^{0}=\displaystyle2\pi r_e^2(1-\cos\gamma).
\end{array}
\end{equation}


\indent As shown in Fig.\ref{fig:caparea}, the surface area of the spherical cap is equal to the lateral surface area of a cylinder, whose radius is the same as the sphere $r_e$ and height is the same as the spherical cap $H$. This is a classic example of the Mercator projection.\\
\begin{figure}
    \centering
    \includegraphics[width=0.8\linewidth]{Figures/covpro.pdf}
    \caption{Examples for the event of a point on the sphere located at $\textbf{w}_k$ being covered (left) or not being covered (right) by LEO satellites located at $\yi'$s. }
    \label{fig:covpro}
\end{figure}
\indent As shown on the right of Fig.\ref{fig:covpro}, if the satellite located at $\textbf{y}_1$ is outside the angle range $\gamma$, its coverage area $\mathcal{O}_1=\mathcal{A}_1$ does not include the considered point $\textbf{w}_k$ on the earth. Correspondingly, its coverage center $\textbf{x}_1$ is outside the spherical cap centered at $\textbf{w}_k$, which can be called as the service request area $\mathcal{O}(\textbf{w}_k,\gamma)$. Therefore, the probability of the point located at $\textbf{w}_k$ not being covered by the LEO satellites located at $\yi'$s is:
\begin{equation}%\displaystyle
\begin{array}{r@{}l}
    p&_{\rm ncov}(N,\gamma)\triangleq \mathbb{P}\{\textbf{w}_k {\rm\,is\,not\,covered\,by\,}\yi'{\rm s}\}\\
    &\overset{(a)}{=} \prod_{i=1}^{N}\mathbb{P}\{\textbf{w}_k {\rm\,is\,not\,covered\,by\,}\yi\}\\
    &\overset{(b)}{=}\big(1-\frac{\mathcal{S(\gamma)}}{4\pi r_e^2}\big)^{N}=\big(1-\frac{2\pi r_e^2 (1-\cos\gamma)}{4\pi r_e^2}\big)^{N}\\
    &=\big(\frac{1+\cos\gamma}{2}\big)^{N},\\
\end{array}
\end{equation}
% \\
%     &
where steps (a) and (b) both come from the BPP assumption of LEO satellite deployment. As shown on the left of Fig.\ref{fig:covpro}, if there is at least one LEO satellite located inside $\mathcal{O}(\textbf{w}_k,\gamma)$ , the considered point $\textbf{w}_k$ on the earth is covered. Therefore, the coverage probability of each point on the earth can be defined as:
\begin{equation}
\begin{array}{r@{}l}
    p&_{\rm cov}(N,\gamma)\triangleq \mathbb{P}\{\textbf{w}_k {\rm\,is\,covered\,by\,at\,least\,one\,of\,}\yi'{\rm s}\}\\
    &=1-\mathbb{P}\{\textbf{w}_k {\rm\,is\,not\,covered\,by\,}\yi'{\rm s}\}\\
    &=1-\big(\frac{1+\cos\gamma}{2}\big)^{N}.
\end{array}
\end{equation}

% \section{Proof of Lemma~\ref{lem:subset}}\label{app:subset}

% \section{Proof of Lemma~\ref{lem:captocircle}}\label{app:captocircle}
% Assume that a point $P'(a,b,0)$ is located at the plane. Its corresponding point on the sphere is $P(x,y,z)$, \ie $P'=\mathcal{F}(P)$ and $P=\mathcal{F}^{-1}(P)$. Therefore, the coordinates of $P$ is $(\frac{4r_e^2 a}{4r_e^2 +a^2+b^2},\frac{4r_e^2 b}{4r_e^2 +a^2+b^2},\frac{2r_e(a^2+b^2)}{4r_e^2 +a^2+b^2})$. A circle on the earth can be viewed as the intersection of the sphere 
% \begin{equation}
%   x^2+y^2+z^2-2zr_e=0  
%   \label{sphereequation}
% \end{equation}
%  and a 3D plane 
% \begin{equation}
%      Ax+By+Cz+D=0.
%      \label{planeequation}
% \end{equation} 
% To ensure that there exists the intersection, the sufficient condition is:
% \begin{equation}
%     \frac{|Cr_e+D|}{\sqrt{A^2+B^2+C^2}}\leq r_e.
% \end{equation}
% Assume that $B=0$, the center of the circular area is located at the plane $xoz$. Since each point $P$ is located at the plane, we substitute its coordinates into (\ref{planeequation}):
% \begin{equation}
%    A\frac{4r_e^2 a}{4r_e^2 +a^2+b^2}+C\frac{2r_e(a^2+b^2)}{4r_e^2 +a^2+b^2}+D=0.
% \end{equation}
% % Because $r_e^2+a^2+b^2>0$, we obtain:
% % \begin{equation}
% % \begin{array}{r@{}}
% %     \displaystyle (2r_e C+D)a^2+(2r_e C+D)b^2+4r_e^2 Aa+4r_e^2 D=0
% % \end{array}
% % \end{equation}
% If $2r_e C+D\neq 0$, we have

% \begin{equation}
% \begin{array}{r@{}}
%     \displaystyle a^2+b^2+\frac{4r_e^2 A}{2r_e C+D}a+(\frac{2r_e^2 A}{2r_e C+D})^2\\
%    \displaystyle =(\frac{2r_e^2 A}{2r_e C+D})^2-\frac{4r_e^2 D}{2r_e C+D},
% \end{array}
% \end{equation}
% \ie
% \begin{equation}
%     \displaystyle (a-\frac{2r_e^2 A}{2r_e C+D})^2+b^2=\frac{4r_e^2(r_e^2 A^2 -2r_e CD-D^2)}{(2r_e C+D)^2}.
% \end{equation}
% Therefore, the projection of a circle on the sphere is also a circle on the projection plane. That means the spherical cap on the sphere can be mapped to a circular area on the projection plane. However, when the original circle cross the top of the sphere, \ie the point $(0,0,2r_e)$ is located at the plane $Ax+Cz+D=0$, $2r_e C+D=0$ and the projection on the plane is a straight line $Aa+D=0$. 

\section{Proof of Lemma~\ref{lem:circularradius}}\label{app:circularradius}
\indent Assume that $\psi=\angle\textbf{w}_S \oe \textbf{x}$ and $\gamma_0$ is the central angle of a spherical cap. The diameter of the projection and its corresponding arcs are both located at the plane $xoz$. First consider the case where $0\leq \psi< \pi-\gamma_0$. As shown in Fig.\ref{fig:OQOQ}, the points at both ends of the diameters satisfy the projection relationship
:
\begin{equation}
    Q'_{l}=\mathcal{F}(Q_{l}),\,Q'_{r}=\mathcal{F}(Q_{r}).
\end{equation}
\begin{figure}
    \centering
    \includegraphics[width=0.75\linewidth]{Figures/OQOQ.pdf}
    \caption{The stereographic projection $Q'_{l}=\mathcal{F}(Q_{l})$ and $Q'_{r}=\mathcal{F}(Q_{r})$. }
    \label{fig:OQOQ}
\end{figure}
Therefore, the radius of projected circular area satisfies:
\begin{equation}
    r(\psi)=\frac{1}{2}\bigg|\|\textbf{o}_z Q'_{r}\|-\|\textbf{o}_z Q'_{l}\|\bigg|.
\label{RQQ}
\end{equation}
\indent We have
\begin{equation}
\begin{array}{r@{}l}
    \|\textbf{o}_z& Q'_{r}\|\displaystyle=2r_e \tan(\angle \textbf{w}_S \textbf{w}_N Q'_{r})\\
    &\displaystyle=2r_e \tan(\frac{\angle \textbf{w}_S \textbf{o}_e Q_{r}}{2})\displaystyle=2r_e \tan(\frac{\psi+\gamma_0}{2}).
\end{array}
\label{0Qr}
\end{equation}
and 
\begin{equation}
\begin{array}{r@{}l}
    \|\textbf{o}_z& Q'_{l}\|\displaystyle=2r_e \tan(\angle \textbf{w}_S \textbf{w}_N Q'_{l})\\
    &\displaystyle=2r_e \tan(\frac{\angle \textbf{w}_S \textbf{o}_e Q_{l}}{2})\displaystyle=2r_e \tan(\frac{\psi-\gamma_0}{2}).
\end{array}
\label{0Ql}
\end{equation}
\indent Substitute (\ref{0Qr}) and (\ref{0Ql}) into (\ref{RQQ}), we obtain:
\begin{equation}
    r(\psi)=r_e\bigg|\tan(\frac{\psi+\gamma_0}{2})-\tan(\frac{\psi-\gamma_0}{2})\bigg|.    
\label{rpsi}
\end{equation}
% \indent It is worth noting that this equation is also suitable for the case where $\pi-\gamma<\phi_i\leq\pi$. When $\psi_i=\pi-\gamma$, the radius goes to $+\infty$, therefore, $r_{i}(\psi_i)<+\infty$. Next, we prove that there exists the minimum value of $r_{i}(\psi_i)$.\\ 
% \indent First consider the case where $0\leq\psi_i <\pi-\gamma$. Taking the derivative of $r_{i}(\psi_i)$ with respect to $\psi_i$, we have
% \begin{equation}
% \begin{array}{r@{}l}
%     &\displaystyle\frac{\partial r_{i}(\psi_i)}{\partial \psi_i}=\displaystyle r_e\bigg(\frac{1}{2\cos^2(\frac{\psi_i+\gamma}{2})}-\frac{1}{2\cos^2(\frac{\psi_i-\gamma}{2})}\bigg)\\
%     &=\displaystyle r_e\bigg(\frac{1}{1+\cos(\psi_i+\gamma)}-\frac{1}{1+\cos(\psi_i-\gamma)}\bigg)\geq 0.
% \end{array}
% \end{equation}
% Similarly, for the case where $\pi-\gamma<\psi_i\leq\pi$, 
% \begin{equation}
% \begin{array}{r@{}l}
%     &\displaystyle\frac{\partial r_{i}(\psi_i)}{\partial \psi_i}=\displaystyle r_e\bigg(\frac{1}{2\cos^2(\frac{\psi_i-\gamma}{2})}-\frac{1}{2\cos^2(\frac{\psi_i+\gamma}{2})}\bigg)\\
%     &=\displaystyle r_e\bigg(\frac{1}{1+\cos(\psi_i-\gamma)}-\frac{1}{1+\cos(\psi_i+\gamma)}\bigg)\leq 0.
% \end{array}
% \end{equation}
% Therefore, the minimum value of $r_{i}(\psi_i)$ is:
% \begin{equation}
% \begin{array}{r@{}l}
%     R_{z,min}&=\min\{r_{i}(0),r_{i}(\pi)\}\\
%     &=\min\{2r_e \tan (\frac{\gamma}{2}),2r_e \cot (\frac{\gamma}{2})\}
% \end{array}
% \end{equation}
%  Therefore, $R_{z,\min}=2r_e \tan(\frac{\gamma}{2})$ and the range of $r_{i}(\psi_i)$ is:
When the central angle $\gamma_0$ is less than $\frac{\pi}{2}$, (\ref{rpsi}) works for $\gamma_0-\pi<\psi<\pi-\gamma_0$. The range of the radius of the projected circular area is:
\begin{equation}
    2r_e \tan(\frac{\gamma_0}{2})\leq r(\psi)<+\infty.
\end{equation}
Conversely, if the radius of the projected circular area is $r$, the range of the central angle of the original spherical cap, $\gamma_0$, is:
\begin{equation}
    0<\gamma_0\leq 2\arctan(\frac{r}{2r_e}).
\label{gamma0range}
\end{equation}

% \section{Proof of Theorem~\ref{theo:concomsphere}}\label{app:concomsphere}
% In this appendix, we need to prove the concept of percolation on the sphere.

% \section{Proof of Lemma~\ref{lem:lowerbound}} \label{app:lowerbound}
% Here, we need to prove the lower bound.
% \section{Proof of Lemma~\ref{lem:upperbound1}} \label{app:upperbound1}
% Here, we need to prove the upper bound.
\section{Proof of Lemma~\ref{lem:nondecreasing}\label{app:nondecreasing}}
To prove that there exists a critical number of LEO satellites that causes the phase transition of percolation probability, we need to first prove that the percolation probability is a non-decreasing function of $N$ when the $\gamma$ is fixed. \\
\indent Firstly, we assume that the LEO satellites are deployed at the same altitude with the same nadir angle $\eta$, therefore, the coverage angle $\gamma$ is also fixed. The locations of satellites follow a BPP around the earth at a certain altitude. Consider two sets of satellites $\Phi_1$ and $\Phi_2$ with the number of vertices $N_1$ and $N_2$, respectively, where $N_1<N_2$. Since $\Phi_1$ and $\Phi_2$ are both BPPs, $\Phi_1$ can be constructed by removing any one vertice inside $\Phi_2$. Similarly, $\Phi_2$ can be constructed by adding any other vertice into $\Phi_1$. Because we discuss their coverage areas on the sphere, the set of spherical caps' centers $V_1$ and $V_2$ can be generated following BPPs with number $N_1$ and $N_2$ at the altitude $0\;\rm km$ (on the sphere), where $V_1\subseteq V_2$ when $\Phi_1\subseteq\Phi_2$.\\
\indent Removing vertice from $\Phi_2$ or adding vertice into $\Phi_1$ both lead to the change of the edge set, where $E_1\subseteq E_2$. The connected components in these two random graphs are defined as: $K_{x,1}\subseteq G_{x}(V_{x,1},V_{x,1})$ and $K_{x,2}\subseteq G_{x}(V_{x,2},V_{x,2})$, which satisfy $K_{x,1}\subseteq K_{x,2}$. Therefore, $0<N_1<N_2$ indicates $\theta(N_1,\gamma)\leq \theta(N_2,\gamma)$, \ie the percolation probability is a non-decreasing function of $N$.\\
\indent It is worth noting that, in this paper, we use two kinds of projection: \textit{i) mapping the satellite locations to the sphere}, which obtains the coverage centers, \textit{ii) mapping every point on the sphere to the projection plane}, which helps us define the percolation on the sphere. The mapping from satellites to coverage centers keep the BPP properties, and mapping the sphere to the projection plane keeps almost all the connections between coverage areas, which has been introduced in the Lemma \ref{lem:mappingrelation}. Therefore, the projections of connected components on the considered projection plane also satisfy $K_{z,1}\subseteq K_{z,2}$. On the plane, we can focus on the connected component containing the origin $\textbf{o}_z$, \ie $K_z(0)$. Therefore, the percolation probabilities for these two cases also satisfy $\P(|K_{z,1}(0)|=\infty)\leq \P(|K_{z,2}(0)|=\infty)$.
\section{Proof of Lemma \ref{lem:upperbound1}}
\label{app:upperbound}
Because the area of a sphere is finite, the graph percolates once the whole sphere is covered by LEO satellites. Therefore, the percolation probability can be lower bounded by the full coverage probability, \ie
\begin{equation}
    \theta(N,\gamma)\geq \P\{{\rm Full\;Coverage}|N,\gamma\}.
\label{full}
\end{equation}
\indent We have proposed our `full coverage scheme' in Sec.\ref{subsec:phasetransition}. Because it is one way to realize full coverage, the probability of a successful deployment is less than or equal to the full coverage probability, \ie
\begin{equation}
\begin{array}{r@{}l}
    \P&\{{\rm Full\;Coverage}|N,\gamma\}\\
    &\geq \P\{{\rm The\; Proposed\;Full\;Coverage\;Scheme}|N,\gamma\}.
\end{array}
\label{successfull}
\end{equation}
Therefore, if we can find a computable and non-zero value of the probability of our proposed full coverage scheme, the percolation probability is proved to be non-zero. Firstly, we need to prove that full coverage can be realized under such a scheme. \\
\indent In Fig.\ref{fig:Upperbound}, we propose to represent the whole sphere using the union of sphere caps and introduce the steps we need. Let $\Omega$ denote the whole sphere and $\mathcal{A}_i$ denote the coverage area of the \textit{i}th LEO satellite. Therefore, the full coverage defined as an event: $\Omega\subseteq \bigcup_{i=1}^{N}\mathcal{A}_i$.\\
\indent We first uniformly divide the whole sphere into $2m$ `slices' using $2m$ meridians, where the \textit{j}th slice is denoted by $\mathcal{S}_j$ and spans $\frac{\pi}{m}$ of longitude. The whole sphere can be expressed as the union of $\mathcal{S}_j$'s, that is, $\Omega=\bigcup_{j=1}^{2m}\mathcal{S}_j$.\\
\indent On the sphere, we assume that slices $\mathcal{S}_j$ and $\mathcal{S}_{m+j}$ are symmetric about the earth's center. They can be contained by a `belt' defined as $\mathcal{T}_j$, \ie $\mathcal{S}_j\bigcup\mathcal{S}_{m+j}\subseteq \mathcal{T}_j$. Therefore, the union of slices is the subset of the union of belts, \ie $\bigcup_{j=1}^{2m}\mathcal{S}_j\subseteq \bigcup_{j=1}^{m}\mathcal{T}_j$. Because $\Omega=\bigcup_{j=1}^{2m}\mathcal{S}_j$ and $\bigcup_{j=1}^{m}\mathcal{T}_j\subseteq \Omega$, the union of the belts is the same as the whole sphere, \ie $\Omega=\bigcup_{j=1}^{m}\mathcal{T}_j$.\\
\indent Next, we can rotate the belt $\mathcal{T}_1$ and make it symmetric about the equatorial plane. Belt $\mathcal{T}_1$ spans $\frac{\pi}{m}$ of latitude. It is able to uniformly divide $\mathcal{T}_1$ into $n$ pieces using $n$ meridians, \ie $\mathcal{T}_1=\bigcup_{k=1}^{n}\mathcal{D}_{1,k}$. where each piece $\mathcal{D}_{1,k}$ spans $\frac{\pi}{m}$ of latitude and $\frac{2\pi}{n}$ of longitude. Other belts can be divided in the same way, and all pieces have the same shape and size. Therefore, the whole sphere is the same as the union of such `pieces', \ie $\Omega=\bigcup_{j=1}^{m}\bigcup_{k=1}^{n}\mathcal{D}_{j,k}$.\\
\indent We consider the piece $\mathcal{D}_{1,1}$ which spans $\frac{\pi}{m}$ of latitude and $\frac{2\pi}{n}$ of longitude firstly. We can find the minimum spherical cap $\mathcal{E}_{1,1}$ containing $\mathcal{D}_{1,1}$, \ie $\mathcal{D}_{1,1}\subseteq \mathcal{E}_{1,1}$. The spherical cap has the same center of $\mathcal{D}_{1,1}$, and its central angle $\zeta$ satisfies:
\begin{equation}
    \zeta =\arccos{\bigg(\cos{\frac{\pi}{2m}}\cos{\frac{\pi}{n}}\bigg)}.
\end{equation}
For each piece $\mathcal{D}_{j,k}$, we can find its corresponding spherical cap $\mathcal{E}_{j,k}$ with the same central angle $\zeta$, where $\mathcal{D}_{j,k}\subseteq \mathcal{E}_{j,k}$. Therefore, the union of pieces is the subset of the union of these spherical caps, \ie $\bigcup_{j=1}^{m}\bigcup_{k=1}^{n}\mathcal{D}_{j,k}\subseteq \bigcup_{j=1}^{m}\bigcup_{k=1}^{n}\mathcal{E}_{j,k}$. Because $\bigcup_{j=1}^{m}\bigcup_{k=1}^{n}\mathcal{E}_{j,k}\subseteq \Omega$ and $\Omega=\bigcup_{j=1}^{m}\bigcup_{k=1}^{n}\mathcal{D}_{j,k}$, we obtain: $\Omega=\bigcup_{j=1}^{m}\bigcup_{k=1}^{n}\mathcal{E}_{j,k}$.\\
\indent To realize the full coverage, we aim to ensure that each spherical cap $\mathcal{E}_{j,k}$ is covered. We can follow three steps: i) make $m$ and $n$ large enough to make the spherical cap $\mathcal{E}_{j,k}$ can be covered by one satellite whose coverage angle is $\gamma$, \ie $\zeta< \gamma$, ii) use $m\times n$ LEO satellites and deploy them one by one, \ie $\mathcal{E}_{j,k}\subseteq \mathcal{A}_{n(j-1)+k}$. In this case, we can ensure that $\Omega=\bigcup_{j=1}^{m}\bigcup_{k=1}^{n}\mathcal{A}_{n(j-1)+k}$ because $\bigcup_{j=1}^{m}\bigcup_{k=1}^{n}\mathcal{A}_{n(j-1)+k}\subseteq \Omega$ and $\bigcup_{j=1}^{m}\bigcup_{k=1}^{n}\mathcal{E}_{j,k}\subseteq \bigcup_{j=1}^{m}\bigcup_{k=1}^{n}\mathcal{A}_{n(j-1)+k}$.\\
\indent To make $\zeta<\gamma$, we can design $m$ first and then $n$. The requirements are: i) $\frac{\pi}{2m}<\gamma$ and $\frac{\pi}{n}<\gamma$ and ii) $\zeta = \arccos{(\cos{\frac{\pi}{2m}}\cos{\frac{\pi}{n}})}<\gamma$, that is, $\cos{\frac{\pi}{2m}}\cos{\frac{\pi}{n}}>\cos{\gamma}$. We can first choose a feasible $m$, for example, $m=\left \lceil \frac{\pi}{\gamma} \right \rceil$ to let the inequality $\frac{\pi}{2m}<\gamma$ hold. Next, because $\cos{\frac{\pi}{n}}>\max\{\cos{\gamma},\frac{\cos{\gamma}}{\cos{\frac{\pi}{2m}}}\}$,  $\frac{\pi}{n}<\arccos{\frac{\cos{\gamma}}{\cos{\frac{\pi}{2m}}}}$, the choice of $n$ should satisfy $n>\left \lceil \frac{\pi}{\arccos{\frac{\cos{\gamma}}{\cos{\frac{\pi}{2m}}}}} \right \rceil$. Therefore, we can let $n=\left \lceil \frac{\pi}{\arccos{\frac{\cos{\gamma}}{\cos{\frac{\pi}{2m}}}}} \right \rceil+1$ and the total number of satellites is $N_U=m\times n$. Therefore, when $N=N_U$, we can realize the full coverage on the sphere.\\
\indent Based on such a special deployment, we can derive the lower bound of its probability. We assume that the center of the first satellite's coverage area $\mathcal{A}_{1,1}$ is $\textbf{x}_{1,1}$ and the center of the spherical cap $\mathcal{E}_{1,1}$ is $\textbf{w}_{1,1}$. When $\angle \textbf{x}_{1,1} \textbf{o}_e \textbf{w}_{1,1}<\gamma-\zeta$, $\mathcal{E}_{1,1}$ is totally covered by $\mathcal{A}_{1,1}$. The probability of such an event is $\frac{1-\cos{(\gamma-\zeta)}}{2}$. We can deploy other satellites in the same way to ensure all $\mathcal{E}_{j,k}$'s are covered. The probability of such a successful full deployment is:
\begin{equation}
\begin{array}{r@{}l}
    \P&\{{\rm The\; Proposed\;Full\;Coverage\;Scheme}|N=N_U,\gamma\}\\
    &=\big(\frac{1-\cos{(\gamma-\zeta)}}{2}\big)^N,
\end{array}
\end{equation}
which is computable and non-zero. Therefore, using the inequalities (\ref{full}) and (\ref{successfull}), we prove that the percolation probability is strictly larger than 0 when $N=N_U$. Similarly, when $N>N_U$, we can deploy the first $N_U$ satellites in the same way, and deploy the other satellites randomly.
Therefore, for $N\geq N_U$, percolation probability is always non-zero, \ie
\begin{equation}
    \theta(N,\gamma)>0\;{\rm for}\;N\geq N_U.
\end{equation}

\section{Proof of Lemma \ref{lem:phasetransition}}\label{app:phasetransition}
\indent According to Lemma \ref{lem:lowerbound} and \ref{lem:upperbound1}, we know that: i) $\theta(N,\gamma)=0$ for $N\leq N_L$ and ii) $\theta(N,\gamma)>0$ for $N\geq N_U$. Because the percolation probability $\theta(N,\gamma)$ is a non-decreasing function of $N$, there must exist a critical value of $N$, \ie $N_c$, which satisfies:
\begin{equation}
    \begin{array}{c}
        \theta(N,\gamma)=0,\, {\rm for}\; N< N_c,\\
        \theta(N,\gamma)>0,\, {\rm for}\; N> N_c. 
    \end{array}
\end{equation}
% It is worth noting that $N$ should be an integer but $N_c$ does not need to be defined as an integer. Therefore, we can define the $N_c$ using:
% \begin{equation}
%     \begin{array}{c}
%         \theta(N,\gamma)=0,\, {\rm for}\; N\leq \left\lfloor N_c\right\rfloor,\\
%         \theta(N,\gamma)>0,\, {\rm for}\; N\geq \left\lceil N_c\right\rceil. 
%     \end{array}
% \end{equation}
The upper and lower bounds should satisfy $N_L\leq \left\lfloor N_c\right\rfloor$ and $\left\lceil N_c\right\rceil\leq N_U$. However, these two inequalities do not hold equality at the same time because the percolation probability can not be zero and non-zero at the same time.
% \indent Next, we aim to prove that $N_c$ satisfies $N_L<N_c<N_U$, that is, the critical value does not approach 0 or $+\infty$. First, in the sub-critical case, we know that $\theta(N,\gamma)=0$ for $N<N_L$. Therefore, $N_c\geq N_L$. We still need to prove that $N_c<N_U$ where $N_L$ is obtained in the super-critical case.\\
% \indent We have proved that there exists a critical number of satellites $N_c$ which remarks the phase transition of percolation probability. Now we introduce the events below:
% \begin{equation}
% \begin{array}{l}
%     \mathcal{M}|N=\{{\rm There\;exist\;full\;coverage\;deployments}\\
%     \hspace{5.3cm} {\rm \;when}\;N\;{\rm is\;known}\} \\
%     \mathcal{N}|N=\{{\rm Removing\;any\;satellite\;will\;break\;the\;}\\
%     \hspace{2.2cm}{\rm existing\;percolation\;when}\;N\;{\rm is\;known  }\}
% \end{array}
% \end{equation}
% \indent If there exists full coverage deployment when $N$ is known, removing one satellite will not always avoid the percolation. That is because the remaining coverage is shaped as a complete sphere except for a spherical cap smaller than a hemisphere, and the percolation in these cases is kept. If $N=N_c$, removing any satellites will break the existing percolation almost surely because $\theta(N_{\rm c}-1,\gamma)=0$. It is worth noting that, there might exist some cases where the coverage areas still connect two symmetric points on the sphere when $N=N_c-1$. However, based on the BPP assumption, the probability of these events is 0 since $\theta(N_c-1,\gamma)=0$. Therefore, $\mathcal{M}$ leads to $\mathcal{N}$ almost surely, that is $\P\{\mathcal{N}\; {\rm leads\;to\;\overline{\mathcal{M}}}\}=\P\{\mathcal{M}\; {\rm leads\;to\;}\overline{\mathcal{N}}\}=1$. From the perspective of probability, $\P\{\mathcal{N}|N\}\leq \P\{\overline{\mathcal{M}}|N\}$ and $\P\{\mathcal{M}|N\}\leq \P\{\overline{\mathcal{N}}|N\}$. The probability $\P\{\mathcal{M}|N\}$ is a non-decreasing function of $N$ with possible values 0 or 1, where $\P\{\mathcal{M}|N_c\}\leq \P\{\overline{\mathcal{N}}|N_c\}=1-\P\{\mathcal{N}|N_c\}=0$, $\P\{\mathcal{N}|N_U\}\leq \P\{\overline{\mathcal{M}}|N_U\}=1-\P\{\mathcal{M}|N_U\}=0$. Therefore, $\P\{\mathcal{M}|N_c\}<\P\{\mathcal{M}|N_U\}$ because $\P\{\mathcal{M}|N_c\}=0$ and $\P\{\mathcal{M}|N_U\}=1$. Therefore, we can obtain $N_c< N_U$.
\section{Proof of Theorem \ref{theo:inhomohexagon}}\label{app:inhomohexagon}
\indent Assume that different hexagons have different probabilities of being open or closed and both of them have their minimum value $p_{\rm cov}^{\min}$ and $p_{\rm ncov}^{\min}$, \ie $\P\{\mathcal{H}_{l} {\rm \; is\;open}\} \geq p_{\rm cov}^{\min}$ and $\P\{\mathcal{H}_{l} {\rm \; is\;closed}\} \geq p_{\rm ncov}^{\min}$.\\ %, where
% \begin{equation}
%     \P\{\mathcal{H}_{l} {\rm \; is\;open}\} \geq p_{\rm cov}^{\min}
% \end{equation}
% and
% \begin{equation}
%     \P\{\mathcal{H}_{l} {\rm \; is\;closed}\} \geq p_{\rm ncov}^{\min}.
% \end{equation}
\indent We firstly discuss the case where the $\P\{\mathcal{H}_{l} {\rm \; is\;open}\}$ has its minimum value $p_{\rm cov}^{\min}$. Considering the hexagons with side length $a$, we can cover the hexagons following their coverage probability $\P\{\mathcal{H}_{l} {\rm \; is\;open}\}$. The random graph in this case is defined as $G_{z}^{\rm cov}$. The probabilities $p_{\rm cov}^{\min}$ and $\P\{\mathcal{H}_{l} {\rm \; is\;open}\}$ are assumed larger than 0. \\
\indent Define the random graph of `open hexagonal faces' generated by probability $p_{\rm cov}^{\min}$ as $G_{z,\min}^{\rm cov}$. We can generate the $G_{z,\min}^{\rm cov}$ through removing each open face $\mathcal{H}_{l}$ in $G_{z}^{\rm cov}$ by probability $p_{1}^l=1-p_{\rm cov}^{\min}/\P\{\mathcal{H}_{l} {\rm \; is\;open}\}$ where $p_{1}^l\in [0,1)$. Therefore, all open faces $G_{z,\min}^{\rm cov}$ are contained by $G_{z}^{\rm cov}$, \ie $G_{z,\min}^{\rm cov}\subseteq G_{z}^{\rm cov}$. When $\P\{\mathcal{H}_{l} {\rm \; is\;open}\}>1/2$, $p_{\rm cov}^{\min}>1/2$, the percolation probability $\P\{|G_{z,\min}^{\rm cov}|=\infty\}>0$, and the percolation probability of $G_z^{\rm cov}$ also satisfies $\P\{|G_{z}^{\rm cov}|=\infty\}>0$. In conclusion, the sufficient condition for non-zero percolation probability is 
\begin{equation}
    \P\{\mathcal{H}_{l} {\rm \; is\;open}\}>1/2.
\end{equation}
\indent Similarly, we define the random graph of `closed hexagonal faces' generated by probability $p_{\rm ncov}^{\min}$ as $G_{z,\min}^{\rm ncov}$. We can generate the $G_{z,\min}^{\rm ncov}$ through removing each closed face $\mathcal{H}_{l}$ in $G_{z}^{\rm ncov}$ by probability $p_{2}^l=1-p_{\rm ncov}^{\min}/\P\{\mathcal{H}_{l} {\rm \; is\;closed}\}$ where $p_{2}^l\in [0,1)$. Therefore, all closed faces $G_{z,\min}^{\rm ncov}$ are contained by $G_{z}^{\rm ncov}$, \ie $G_{z,\min}^{\rm ncov}\subseteq G_{z}^{\rm ncov}$. When $\P\{\mathcal{H}_{l} {\rm \; is\;closed}\}>1/2$, $p_{\rm ncov}^{\min}>1/2$, the percolation probability $\P\{|G_{z,\min}^{\rm ncov}|=\infty\}>0$, and the percolation probability of $G_z^{\rm ncov}$ also satisfies $\P\{|G_{z}^{\rm ncov}|=\infty\}>0$. In conclusion, the sufficient condition for zero percolation probability is 
\begin{equation}
    \P\{\mathcal{H}_{l} {\rm \; is\;closed}\}>1/2.
\end{equation}
\section{Proof of Lemma \ref{lem:boundsforhexagons}}\label{app:boundsforhexagons}
\indent We first consider the hexagon $\mathcal{H}_{l}$ with the side length 
$a$. The circular area $\tilde{\mathcal{O}}_{l}$ with radius $a$ has the same center as $\mathcal{H}_{l}$. The center of 
$\mathcal{F}^{-1}(\tilde{\mathcal{O}}_{l})$ is $\textbf{x}_{o,l}$ and the center of $\mathcal{A}_i$ is $\textbf{x}_i$. The probability each hexagonal face $\mathcal{H}_{l}$ being closed satisfy:
\begin{equation}%\displaystyle
\begin{array}{r@{}l}
    \P&\displaystyle\{\mathcal{H}_{l}\;{\rm is\;closed}\}\\
    &=\P\{\mathcal{H}_{l}\;{\rm is\;not\;covered\;by\;}\bigcup_{i=1}^{N}\mathcal{F}(\mathcal{A}_i)\}\\
    &\geq\P\{\tilde{\mathcal{O}}_{l}\;{\rm is\;not\;covered\;by\;}\bigcup_{i=1}^{N}\mathcal{F}(\mathcal{A}_i)\}\\
    % &=\P\displaystyle\{\mathcal{F}^{-1}(\tilde{\mathcal{O}}_{l})\;{\rm is\;not\;covered\;by\;}\bigcup_{i=1}^{N}\mathcal{A}_i\}\\
    &\geq\prod_{i=1}^{N}\P\{\mathcal{F}^{-1}(\tilde{\mathcal{O}}_{l})\;{\rm is\;not\;covered\;by\;}\mathcal{A}_i\}\\
    &\geq\prod_{i=1}^{N}\P\{\angle \textbf{x}_{o,l} \textbf{o}_e \textbf{x}_i>\gamma+\gamma_m\}\\
    &=\big(\frac{1+\cos(\gamma+\gamma_m)}{2}\big)^{N},\\
\end{array}
\end{equation}
where $\gamma_m$ is the maximum central angle of the original spherical cap of hexagon's minimum circumscribed circle. Because the radius of the minimum circumscribed circle is $a$, from (\ref{gamma0range}), we can obtain its expression:
\begin{equation}
    \gamma_m=2\arctan \frac{a}{2r_e}.
\end{equation}
% \indent Next, we consider another circular area $\tilde{\mathcal{O}}_{l}^{\rm inner}$ with radius $\frac{\sqrt{3}}{2}a$ has the same center as $\mathcal{H}_{l}$. The center of $\mathcal{F}^{-1}(\tilde{\mathcal{O}}_{l}^{\rm inner})$ is also $\textbf{x}_{o,l}$ and the center of $\mathcal{A}_i$ is $\textbf{x}_i$. 
Similarly, the probability of $\mathcal{H}_{l}$ being covered satisfy:
\begin{equation}
\begin{array}{r@{}l}
    \P&\displaystyle\{\mathcal{H}_{l}\;{\rm is\;open}\}\\
    &=\P\{\mathcal{H}_{l}\;{\rm is\;covered\;by\;}\bigcup_{i=1}^{N}\mathcal{F}(\mathcal{A}_i)\}\\
    &\geq\P\{\tilde{\mathcal{O}}_{l}\;{\rm is\;covered\;by\;}\bigcup_{i=1}^{N}\mathcal{F}(\mathcal{A}_i)\}\\
    % &=\P\displaystyle\{\mathcal{F}^{-1}(\tilde{\mathcal{O}}_{l})\;{\rm is\;covered\;by\;}\bigcup_{i=1}^{N}\mathcal{A}_i\}\\
    &\geq\P\{\mathcal{F}^{-1}(\tilde{\mathcal{O}}_{l})\;{\rm is\;covered\;by\;at\; least\;one\;of\;}\mathcal{A}_i\}\\
    & =1-\prod_{i=1}^{N}\P\{\mathcal{F}^{-1}(\tilde{\mathcal{O}}_{l})\;{\rm is\;not\;covered\;by\;}\mathcal{A}_i\}\\
    &\geq1-\prod_{i=1}^{N}\P\{\angle \textbf{x}_{o,l} \textbf{o}_e \textbf{x}_i>\gamma-\gamma_m\}\\
    &=1-\big(\frac{1+\cos(\gamma-\gamma_m)}{2}\big)^{N}.\\
\end{array}
\end{equation}
% where
% \begin{equation}
%     \gamma_m=2\arctan \frac{\sqrt{3}a}{2 r_e}.
% \end{equation}
% We assume that the side length $a$ is much smaller than the coverage radius of satellites on the plane, $\gamma_m$ and $\gamma_m$ are both assumed much less than $\gamma$, which is the coverage angle of each LEO satellite.
\indent We assume that the side length $a$ is much smaller than the coverage radius of satellites on the plane, $\gamma_m$ is assumed much less than the coverage angle $\gamma$ of each LEO satellite.
\section{Proof of Lemma \ref{lem:criticalanalysis}}\label{app:criticalanalysis}
\indent Notice that, the upper bound \begin{equation}
\displaystyle N_c^U=\displaystyle\frac{\ln 2}{\ln 2-\ln(1+\cos(\gamma-2\arctan \frac{a}{2r_e}))}
\end{equation}
can be considered as an increasing function of $a$ and the lower bound
\begin{equation}
    N_c^L=\frac{\ln 2}{\ln 2-\ln(1+\cos(\gamma+2\arctan \frac{a}{2r_e}))}
\end{equation}
can be considered as a decreasing function of $a$. When the side length $a$ approaches 0, the limit values of the upper bound $N_c^{U}$ and lower bound $N_c^{L}$ are both approach to the same value:
\begin{equation}
\begin{array}{r@{}l}
\lim\limits_{a\rightarrow 0^+} N_c^U&=\lim\limits_{a\rightarrow 0^+}\frac{\ln 2}{\ln 2-\ln(1+\cos(\gamma-2\arctan \frac{a}{2r_e}))}\\
&=\frac{\ln 2}{\ln 2-\ln(1+\cos\gamma)}
\end{array}
\end{equation}
and 
\begin{equation}
\begin{array}{r@{}l}
\lim\limits_{a\rightarrow 0^+} N_c^L&=\lim\limits_{a\rightarrow 0^+}\frac{\ln 2}{\ln 2-\ln(1+\cos(\gamma+2\arctan \frac{a}{2r_e}))}\\
&=\frac{\ln 2}{\ln 2-\ln(1+\cos\gamma)}
\end{array}
\end{equation}

Because $N_c^{L}\leq N_c\leq N_c^U$, the limit value of $N_c$ should be the same as them, \ie
\begin{equation}
    N_c=\displaystyle\frac{\ln 2}{\ln 2-\ln(1+\cos\gamma)}.
\end{equation}

This is also the closed-form expression of critical number of LEO satellites which is always located between these two bounds. %Because the limit value of $N_c^L$ and $N_c^U$ are the same, the upper and lower bounds obtained through this stereographic projection method are both tight. However, they can not be used directly because we focus on continuous percolation on the sphere rather than the hexagonal lattic design on the plane, therefore, we only choose $a=0$ and adopt the critical number $N_c$.\\
% \indent After proving the expression of $N_c$, we need to verify that it should be strictly located between the lower bound $N_L$ and upper bound $N_U$.\\
% \indent First, we aim to prove that $N_L\leq N_c$. We aim to prove:
% \begin{equation}
%     \frac{\pi}{\gamma}<\frac{\ln 2}{\ln 2-\ln(1+\cos\gamma)}
% \label{LC}
% \end{equation}
% That is
% \begin{equation}
%     \pi \ln 2-\pi \ln(1+\cos\gamma)<2\gamma\ln 2
% \end{equation}
% Define that
% \begin{equation}
%     f(\gamma)=\pi \ln(1+\cos\gamma)+(2\gamma-\pi)\ln 2
% \end{equation}
% Take the derivative of $f(\gamma)$ with respect to $\gamma$:
% \begin{equation}
%     f'(\gamma)=-\frac{\pi\sin\gamma}{1+\cos\gamma}+2\ln 2
% \end{equation}
% The second derivative is
% \begin{equation}
%     f''(\gamma)=-\frac{\pi}{1+\cos\gamma}<0
% \end{equation}
% where $\gamma\in(0,\frac{\pi}{2})$.
% Therefore, $f'(\gamma)$ decreases as $\gamma$ increases, where $f'(\gamma)\in(2\ln2-\pi,2\ln2)$. So that $f(\gamma)$ increases first and then decreases as $\gamma$ increases, where $f(\gamma)>\min\{f(0),f(\frac{\pi}{2})\}=0$. That is the inequality in (\ref{LC}). Because the lower bound $N_L=\left \lfloor \frac{\pi}{2\gamma} \right \rfloor< \frac{\pi}{2\gamma}$, $N_L\leq N_c$. When $\gamma=0$, they both goes to infinity. When $\gamma=\frac{\pi}{2}$, they are both 1.\\
% \indent Next, we introduce a necessary condition of number of LEO satellites for full coverage. If $N<N_f$ where $N_f$ is the ratio of the area of the whole Earth to the area of each coverage area. It is easy to obtain because if we can achieve the full coverage, the sum of coverage area must be larger than the whole Earth. Now we want to prove that $N_c\leq N_f$. We have 
% \begin{equation}
% N_f=\frac{4\pi r_e^2}{2\pi r_e^2 (1-\cos\gamma)}=\frac{2}{1-\cos\gamma}
% \end{equation}
% To prove that
% \begin{equation}
%     \displaystyle\frac{\ln 2}{\ln 2-\ln(1+\cos\gamma)}<\frac{2}{1-\cos\gamma}
% \end{equation}
% That is:
% \begin{equation}
%     2\ln (1+\cos\gamma)<\ln 2(1+\cos\gamma).
% \end{equation}
% Because $\gamma\in(0,\frac{\pi}{2})$, define $x=1+\cos\gamma\in(1,2)$. Define
% \begin{equation}
%     g(x)=\frac{\ln x}{x}
% \end{equation}
% the first derivative is:
% \begin{equation}
%     g'(x)=\frac{1-\ln x}{x^2}>0,\;for\;x\in(1,2)
% \end{equation}
% We have
% \begin{equation}
%     g(x)<g(2)=\frac{\ln 2}{2}
% \end{equation}
% That is
% \begin{equation}
%     \frac{\ln (1+\cos\gamma)}{1+\cos\gamma}<\frac{\ln 2}{2}.
% \end{equation}
% Therefore, \begin{equation}
%     \displaystyle\frac{\ln 2}{\ln 2-\ln(1+\cos\gamma)}<\frac{2}{1-\cos\gamma}
% \end{equation}
% when $\gamma=0$, $N_c$ and $N_f$ both goes to infinite.\\

% Now, we need to prove that $N_f<N_U$. We know that
% \begin{equation}
%     N_U = \left \lceil \frac{\pi}{\gamma} \right \rceil  \left \lceil \frac{\pi}{\arccos{\frac{\cos\gamma}{\cos\frac{\gamma}{2}}}} \right \rceil> \frac{\pi}{\gamma}  \frac{\pi}{\arccos{\frac{\cos\gamma}{\cos\frac{\gamma}{2}}}}
% \end{equation}

% Because for $\gamma\in(0,\frac{\pi}{2})$,
% \begin{equation}
%    \frac{\cos\gamma}{\cos \frac{\gamma}{2}}>\cos\gamma
% \end{equation}
% and
% \begin{equation}
%     \frac{\cos\gamma}{\cos \frac{\gamma}{2}}=\frac{2\cos^2 \frac{\gamma}{2}-1}{\cos \frac{\gamma}{2}}\in(0,1)
% \end{equation}
% We have $\arccos{\frac{\cos\gamma}{\cos\frac{\gamma}{2}}}<\gamma$ and

% \begin{equation}
% \frac{\pi}{\gamma}  \frac{\pi}{\arccos{\frac{\cos\gamma}{\cos\frac{\gamma}{2}}}}>\frac{\pi^2}{\gamma^2}.
% \end{equation}
% So that, $N_u>\frac{\pi^2}{\gamma^2}$. To prove that
% \begin{equation}
%     \frac{\pi^2}{\gamma^2}>\frac{2}{1-\cos\gamma}
% \end{equation}
% We define that
% \begin{equation}
%     h(\gamma)=\pi^2(1-\cos\gamma)-2\gamma^2
% \end{equation}
% Take the derivative of $h(\gamma)$ with respect to $\gamma$:
% \begin{equation}
%     h'(\gamma)=\pi\sin\gamma-4\gamma
% \end{equation}
% The second derivative is:
% \begin{equation}
%     h^{(2)}(\gamma)=\pi^2\cos\gamma-4
% \end{equation}
% And the third derivative is:
% \begin{equation}
%     h^{(3)}(\gamma)=-\pi^2\sin\gamma<0
% \end{equation}
% As $\gamma$ increases, the the second derivative decreases from -1 to $\pi^2-4$. Therefore, as $\gamma$ increases, the first derivative increases first and then decreases, and
% \begin{equation}
%     h'(\gamma)>\min\bigg\{f(0),f(\frac{\pi}{2})\bigg\}=\min\bigg\{0,\pi^2-2\pi\bigg\}=0
% \end{equation}
% Therefore, $h(\gamma)>h(0)=0$, that is
% \begin{equation}
%     \frac{\pi^2}{\gamma^2}>\frac{2}{1-\cos\gamma}.
% \end{equation}
% And then we obtain $N_f<\pi^2/\gamma^2<N_U$. 

% \indent In conclusion, we have:
% \begin{equation}
%     N_L<N_c<N_f<N_U.
% \end{equation}


\ifCLASSOPTIONcaptionsoff
  \newpage
\fi

\bibliographystyle{IEEEtran}
\documentclass{MITstyle}

%\usepackage[table]{xcolor}
\usepackage{chngcntr}
\usepackage{hyperref}
\usepackage{microtype}

\title{A Lightweight and Extensible Cell Segmentation and Classification Model for Whole Slide Images}

\author{Nikita Shvetsov~$^{1, }$\footnote{Correspondence e-mail: nikita.shvetsov@uit.no}, Thomas K. Kilvaer~$^{2, 3}$, Masoud Tafavvoghi~$^{4}$, Anders Sildnes~$^{1}$, \\ Kajsa Møllersen~$^{4}$, Lill-Tove Rasmussen Busund~$^{5, 6}$, Lars Ailo Bongo~$^{1}$ \\
%
\vspace{1em} % Space between authors and afilliations
%
\normalfont{\small $^{1}$Department of Computer Science, UiT The Arctic University of Norway}\\
\normalfont{\small $^{2}$Department of Oncology, University Hospital of North Norway}\\
\normalfont{\small $^{3}$Department of Clinical Medicine, UiT The Arctic University of Norway}\\
\normalfont{\small $^{4}$Department of Community Medicine, UiT The Arctic University of Norway}\\
\normalfont{\small $^{5}$Department of Medical Biology, UiT The Arctic University of Norway} \\
\normalfont{\small $^{6}$Department of Clinical Pathology, University Hospital of North Norway} %\vspace{2em}
}

\begin{document}
\maketitle

\section*{Abstract}

% \begin{abstract}
% Developing clinically useful cell-level analysis tools in digital pathology remains challenging due to limitations in dataset granularity, inconsistent annotations, computational demands of advanced models, and difficulties in integrating new technologies into clinical workflows. To address these challenges, we propose a multi-faceted solution that enhances data quality, model performance, and usability to create a lightweight and extensible cell segmentation and classification model.

% First, we update data labels by employing a cross-relabeling process that refines the labels of two existing datasets, PanNuke and MoNuSAC, to create a new unified dataset with enhanced granularity, encompassing seven distinct cell types. Second, we leverage the H-Optimus foundation model as a fixed encoder to improve feature representation for simultaneous cell segmentation and classification tasks. Third, to address the computational demands of foundation models, we employ knowledge distillation to reduce model size and complexity while maintaining comparable performance. Finally, to facilitate integration into clinical workflows, we integrate the distilled model into the QuPath software, a widely used open-source platform in digital pathology.

% Our results demonstrate improvements in cell segmentation and classification performance using the H‑Optimus-based model compared to a CNN-based model. Specifically, the average $R^2$ improved from 0.575 to 0.871, and the average $PQ$ score improved from 0.450 to 0.492, indicating better alignment with actual cell counts and enhanced segmentation and classification quality. Furthermore, the distilled student model maintains performance comparable to the larger foundation model while reducing the parameter count by a factor of 48.
% Overall, by reducing computational complexity and integrating it into existing workflows, the proposed approach may significantly impact diagnostic processes, reduce the workload of pathologists, and contribute to improved patient outcomes. Though our approach shows potential enhancements in efficiency and usability of cell segmentation and classification models in digital pathology, extensive validation is needed to deploy these models in clinical practice.
% \end{abstract}

%%% shortened abstract
\begin{abstract}
Developing clinically useful cell-level analysis tools in digital pathology remains challenging due to limitations in dataset granularity, inconsistent annotations, high computational demands, and difficulties integrating new technologies into workflows. To address these issues, we propose a solution that enhances data quality, model performance, and usability by creating a lightweight, extensible cell segmentation and classification model. 

First, we update data labels through cross-relabeling to refine annotations of PanNuke and MoNuSAC, producing a unified dataset with seven distinct cell types. Second, we leverage the H-Optimus foundation model as a fixed encoder to improve feature representation for simultaneous segmentation and classification tasks. Third, to address foundation models' computational demands, we distill knowledge to reduce model size and complexity while maintaining comparable performance. Finally, we integrate the distilled model into QuPath, a widely used open-source digital pathology platform. 

Results demonstrate improved segmentation and classification performance using the H-Optimus-based model compared to a CNN-based model. Specifically, average $R^2$ improved from 0.575 to 0.871, and average $PQ$ score improved from 0.450 to 0.492, indicating better alignment with actual cell counts and enhanced segmentation quality. The distilled model maintains comparable performance while reducing parameter count by a factor of 48. By reducing computational complexity and integrating into workflows, this approach may significantly impact diagnostics, reduce pathologist workload, and improve outcomes. Although the method shows promise, extensive validation is necessary prior to clinical deployment.
\end{abstract}
\clearpage

\section{Introduction}
In digital pathology, accurate segmentation and classification of cells are crucial for many diagnostic, prognostic, and predictive analyses \cite{Jaber_Beziaeva_etal._2019,Lin_Pan_etal._2022,Park_Ock_etal._2022,Shen_Choi_etal._2024}. Nowadays, developments in computational pathology offer multiple solutions \cite{H._Qu_P._Wu_etal._2020,Javed_Mahmood_etal._2020} to utilize cell-level datasets to train machine learning models that solve these problems. The quality and specificity of training datasets are critical for robust and accurate models. Adhering to the principle of "garbage in, garbage out", it is essential to ensure that these datasets are extensively and accurately labeled with distinct classes that reflect the diverse biological characteristics of different cell types. Unfortunately, the number of open-source datasets comprising such high-quality annotations is limited. Existing cell segmentation datasets \cite{Gamper_Koohbanani_etal._2019,Graham_Vu_etal._2019,Verma_Kumar_etal._2021} may offer extensive annotations for certain cell types while providing more general labels for others. For example, in PanNuke, which is one of the largest open-source datasets comprising labeled cells, various types of morphologically and functionally different inflammatory cells like macrophages and lymphocytes are clustered in a broad "inflammatory" class. Consequently, these classes are frequently omitted from analyses or aggregated into broader meta-classes \cite{Gamper_Koohbanani_etal._2020} and likely interfere with other cell classes included in the dataset. This and similar inconsistencies in annotation granularity limit the ability of machine learning models to learn the comprehensive and nuanced features necessary for accurate cell segmentation and classification. To address these challenges, methods for refining and standardizing dataset annotations are essential to enhance the quality of training data.

A complementary approach to mitigate the absence of high-quality training data is the use of foundation models. Foundation models as encoders are defined as large-scale, versatile networks pre-trained on vast, diverse datasets using self-supervised learning, contrasting with convolutional neural network (CNN) pre-trained encoders that rely on supervised learning with labeled data. In practice, foundation models leverage enormous amounts of weakly or unlabeled data from millions of whole slide images (WSIs) and employ self-attention mechanisms to capture long-range dependencies and global context \cite{Chen_Ding_etal._2024,Saillard_Jenatton_etal._2024,Vorontsov_Bozkurt_etal._2024,Xu_Usuyama_etal._2024}. As a consequence, foundation models are able to produce transferable feature representations across different cell types and tissue environments. The feature representations can be leveraged by decoder networks to produce segmentation masks and pixel-level classifications. Because foundation models have comprehensive feature representations, they can be effectively fine-tuned using much smaller amounts of cell-level data compared to the large datasets needed to train models from scratch. Furthermore, foundation models incorporate adversarial training elements or contrastive learning \cite{Chen_Ding_etal._2024,Xu_Usuyama_etal._2024}, enhancing their resilience and adaptability by exposing them to challenging and varied scenarios during training. This may result in more generalizable models, often making them well-suited for diverse and complex tasks in digital pathology.

Despite the inherent advantages of foundation models, their deployment for practical use faces its own obstacles. In particular, they require substantial computational power, financial investments and rigorous testing to ensure reliability and efficacy for a given task \cite{Akkus_Dangott_etal._2022,Dragomir_Cocuz_etal._2022,Go_2022,Jafri_Farooqui_etal._2024}. Moreover, while foundation models enhance feature representation and performance, they depend on the quality of available annotations for decoder fine-tuning and, like any other model, cannot resolve existing inconsistencies or ambiguities in data labels. Therefore, there remains a critical need for solutions that address both data quality and practical deployment considerations.
Further, integrating new technologies into existing clinical workflows often encounters resistance, as it necessitates adjustments to established diagnostic processes. So, there is a need to develop solutions that could be integrated into current practices, minimizing the burden on medical professionals to adopt new tools \cite{King_Williams_etal._2023}.

Existing solutions \cite{Goldsborough_Philps_etal._2024,Hörst_Rempe_etal._2024}, while addressing some aspects of these challenges, fall short in providing a comprehensive approach. To address the data quality and clinical deployment issues, we propose a multi-faceted solution that encompasses data refinement, model optimization, and integration with existing pathology tools (\hyperref[fig:fig1]{Figure 1}). The outcome is a lightweight cell segmentation and classification model that can be integrated into digital pathology workflows for practical clinical use.

\begin{figure}[h!]
    \centering
    \includegraphics[width=\textwidth, height=0.82\textheight, keepaspectratio]{images/Figure_1.pdf}
    \caption{Overview of the proposed solution, including 1) Data refinement using cross-relabeling, 2) Teacher model development and fine tuning, 3) Student model optimization with knowledge distillation and 4) Student model and QuPath integration}
    \label{fig:fig1}
\end{figure}
\clearpage

Our approach begins with preparing the data for the fine-tuning and training of the machine learning models. We create a refined dataset, acquired via cross-relabeling two cell-level datasets, enhancing annotation specificity and consistency of the labeled data. Subsequently, we create a cell segmentation and classification model based on the foundation model. We leverage the foundation model as a fixed encoder and fine-tune a decoder using the refined dataset to improve generalization across diverse tissue- and cell types.
To ensure that the model remains lightweight and deployable in a possibly resource-constrained environment, we employ knowledge distillation to approximate the functionality of the foundation model. Finally, to facilitate the practical application of our model in digital pathology workflows, we integrate it with the QuPath \cite{Bankhead_Loughrey_etal._2017} application. Each methodological component contributes to the overarching goal of enhancing model performance, generalizability, and usability in clinical settings.

The primary contributions of this paper are:
\begin{enumerate}
    \item \textit{Data labels refinement through cross-relabeling:}
    
    We propose a new method for refining labels of cell-level datasets through cross-relabeling. This method employs classification models to re-label broad and ambiguous instances, resulting in a more diverse dataset. Our evaluation demonstrates that these classification models achieve high accuracy on test subsets, indicating the reliability of the method for label refinement.

    \item \textit{Enhanced model performance via foundation models:}
    
    We employ a foundation model as a feature extractor for the cell segmentation and classification task. In comparison with training a CNN model from scratch, the foundation model backbone only needs fine-tuning, which significantly reduces training time, computational resources and data requirements. We show that using a foundation model encoder leads to better performance in cell segmentation and classification networks than using a CNN-based encoder. This improvement may enable the model to generalize more effectively across various tissue types and imaging methods.
    
    \item \textit{Model optimization through knowledge distillation:}
    
    We show that a smaller student model trained using knowledge distillation on the refined dataset obtained via our cross-relabeling approach from a foundation model achieves comparable performance in cell segmentation and quantification tasks. As a result, this model is more suitable for deployment in environments without high-performance computing resources.
    
    \item \textit{Integration with QuPath:}
    
    We integrate the distilled cell segmentation and classification model into QuPath, a widely used open-source digital pathology platform, to accelerate clinical adaptation by enabling pathologists to more easily incorporate advanced computational tools into their existing workflows.
\end{enumerate}

Through these methodological steps, we aim to bridge the gap between advanced machine learning techniques and practical clinical applications, making accurate and efficient digital pathology accessible in a broader range of healthcare settings.

\section{Refining Existing Datasets Using Cross-Relabeling}
To address the limitations of sparse and ambiguous labeling of cell-level datasets, we propose a generalizable cross-relabeling strategy that can be applied to any dataset containing broadly categorized or imprecisely labeled cell types. This approach involves training and subsequently leveraging classification models to refine broad categories into more specific or biologically relevant classes.
When applied to cell-level data, the methodology includes extracting individual cell images from the dataset patches, preprocessing these images to standardize the size and accommodate partial cells, and then training deep learning classifiers capable of distinguishing between the finer cell subtypes within the coarser categories. 
To illustrate our approach, we focus on the PanNuke \cite{Gamper_Koohbanani_etal._2020, Gamper_Koohbanani_etal._2019} and MoNuSAC \cite{Verma_Kumar_etal._2021} datasets that we have used to train models for cell quantification in our previous works \cite{Shvetsov_Grønnesby_etal._2022,Shvetsov_Sildnes_etal._2024}. We find that for better cell differentiation we have to introduce more granular labels. PanNuke includes a broad classification of "inflammatory" cells, encompassing lymphocytes, macrophages, and neutrophils. Each cell type differs significantly in structure, function, and clinical relevance. Conversely, MoNuSAC uses the label "epithelial" for a class that comprises both benign epithelial cells and malignant neoplastic cells. This practice makes it challenging to differentiate between benign and malignant epithelial cells in the dataset, which is a critical distinction when identifying tumor areas within tissue samples. To address these issues, we implement a cross-relabeling strategy as shown in \hyperref[fig:fig2]{Figure 2}. The key components are two classification models: one is trained on singular cell images from PanNuke data to classify the epithelial meta-class into epithelial and neoplastic classes. The other is trained on MoNuSAC to refine the inflammatory class into lymphocytes, neutrophils, and macrophages.

\begin{figure}[h!]
    \centering
    \includegraphics[width=\textwidth]{images/Figure_2.pdf}
    \caption{Refined dataset generation via cross relabeling}
    \label{fig:fig2}
\end{figure}

The refining approach consists of three consecutive steps. The first is the preprocessing step, in which we extract individual cells from both datasets (\hyperref[fig:fig3]{Figure 3}). The specifics of PanNuke and MoNuSAC patch preparation before cell preprocessing are provided in \hyperref[chap:S1]{Appendix S1}.

\begin{figure}[h!]
    \centering
    \includegraphics[width=\textwidth]{images/Figure_3.pdf}
    \caption{Cell instances preprocessing including (1) cell map extraction, (2) bounding box delineation, (3) adjusting cell boxes and (4) cropping and resizing of cell images}
    \label{fig:fig3}
\end{figure}

During preprocessing, we extract cell type maps from the ground truth label mask and calculate bounding boxes around each cell instance. To accommodate partial cells at patch borders, a common issue in cropped patch images, we employ mirror padding and extend the field of view of the cell label by 15 pixels to capture adjacent cells. We then crop and resize the identified regions to $64 \times 64$ pixels using bicubic interpolation.

The preprocessed PanNuke dataset comprises 68,031 neoplastic and 23,207 epithelial cell images, while MoNuSAC comprises  33,104 lymphocytes, 1,252 neutrophils, and 1,695 macrophages, which we subsequently use in training cell classification models and classifying the cell image data \hyperref[fig:S2]{Appendix Figure S2 (1)}. 

The next step is to train two distinct ResNet50-based classifiers tailored to address the specific labeling challenges inherent in each dataset. We use ResNet50 for classification models due to its proven effectiveness for image classification tasks in histopathology \cite{pan2022reviewmachinelearningapproaches}, and its compatibility with small images. For the PanNuke dataset, we design the classifier, trained on MoNuSAC data, to disaggregate the heterogeneous "inflammatory" cell category into distinct subtypes: lymphocytes, macrophages, and neutrophils. Similarly, for the MoNuSAC dataset, the classifier is trained on PanNuke data and distinguishes between benign and malignant epithelial cells within the overarching "epithelial" label. By applying these targeted classifiers to their respective datasets, we assign more specific labels to individual cell instances, thus enabling us to create a unified dataset.
To ensure a balanced representation of classes, we train both models on datasets that had been equalized to match the size of the least represented class. Thus, we obtain datasets comprising 23,207 samples per class for PanNuke and 1,252 samples per class for MoNuSAC data. Next, we partition both of them into training (70\%), validation (20\%), and testing (10\%) subsets. To mitigate the risk of overfitting, we use a single dropout layer with a rate of p=0.5 in both models and data augmentation using randomized color perturbations, rotation, and horizontal and vertical flipping. We employ AdamW optimizer and the cross-entropy loss function for the training criterion.

To evaluate the two trained models, we measure the classification accuracy on the respective test subsets. The accuracies on the test subset for both classifiers are presented in \hyperref[tab:1]{Table 1}. The PanNuke model achieves an average accuracy of 93.57\%, with higher accuracy for neoplastic cells (96.06\%) compared to epithelial cells (86.26\%). The confusion matrix in Figure A3.1 shows that the model predominantly distinguishes accurately between epithelial and neoplastic tissues, with a substantial number of correct classifications and relatively few misclassifications. The MoNuSAC model demonstrates an average accuracy of 98.92\%, excelling in classifying lymphocytes (99.67\%) and macrophages (94.12\%), with lower performance for neutrophils (85.71\%). The confusion matrix in Figure A3.2 shows that the model identifies lymphocytes and performs reasonably well with macrophages and neutrophils.

\begin{table}[h!]
\renewcommand{\arraystretch}{1.5}
  \centering
  \caption{Cell classification results for PanNuke and MoNuSAC trained models (CI 95\%).}
  \label{tab:1}
  \begin{tabular}{|l|c|c|}
   \hline
   %\rowcolor{gray!30}
    Accuracy               & PanNuke model              & MoNuSAC model              \\
    \hline
    Average      & 0.936 (0.931--0.941)         & 0.989 (0.986--0.993)        \\
    \hline
    Neoplastic   & 0.961 (0.956--0.965)         & -                          \\
    \hline
    Epithelial   & 0.863 (0.849--0.877)         & -                          \\
    \hline
    Lymphocytes  & -                          & 0.997 (0.995--0.999)        \\
    \hline
    Neutrophils  & -                          & 0.857 (0.796--0.918)        \\
    \hline
    Macrophages  & -                          & 0.941 (0.906--0.976)        \\
    \hline
  \end{tabular}
\end{table}

Finally, during the last step, we use the model trained on PanNuke data for epithelial cells in MoNuSAC and the model trained on MoNuSAC for the inflammatory cells class in PanNuke. Specifically, we use classifier models to relabel epithelial cells in MoNuSAC and inflammatory cells in PanNuke data. Then we combine cells with refined labels and the rest of the cells in both datasets to create a refined dataset (\hyperref[fig:S2]{Appendix Figure S2 (2)}). The process of relabeling cells and visualizing them on a patch is shown in \hyperref[fig:fig4]{Figure 4}. The cell counts in the refined dataset are provided in \hyperref[tab:S4]{Appendix Table S4}.

\begin{figure}[h!]
    \centering
    \includegraphics[width=\textwidth, height=0.42\textheight, keepaspectratio]{images/Figure_4.pdf}
    \caption{Cell relabeling procedure for epithelial and inflammatory cell classes}
    \label{fig:fig4}
\end{figure}

%\hfill

Relabeling and combining datasets have been explored in a prior study \cite{Parulekar_Kanwat_etal._2023}, where consecutive fine-tuning on multiple datasets was employed to account for hierarchical class label structures. While the method presented in \cite{Parulekar_Kanwat_etal._2023} is intuitive, it often lacks consistency and requires multiple fine-tuning runs, which can be cumbersome and time-consuming. 
In contrast, cross-relabeling simplifies this process by using specialized classification models tailored to each dataset's specific labeling challenges. This approach provides better transparency and produces a unified dataset encompassing seven distinct cell types across multiple tissue samples, enhancing data diversity for further model training or fine-tuning.

Despite these improvements, cross-relabeling does not entirely resolve issues related to poor labeling quality or the amount of labeled data. Specifically, our results show lower accuracies persist for underrepresented classes, such as macrophages, which may stem from a limited sample availability and intrinsic challenges in distinguishing these cells based solely on H\&E staining. Furthermore, while our method enhances label specificity, it relies on the initial quality of the broad labels; thus, any fundamental inaccuracies in the original annotations can propagate through the relabeling process. Addressing the overall problem of limited data labels may require integrating additional data sources or utilizing complementary immunohistochemical staining methods.
Although the reported performance metrics are obtained from evaluations on the native test sets of each dataset, it is important to note that the primary application of these classifiers is to perform cross-relabeling, where a model trained on one dataset (e.g., PanNuke) is applied to another (e.g., MoNuSAC) and vice versa. We acknowledge that a more systematic evaluation of cross-dataset generalization is needed and could be performed in future work.

Overall, the refined dataset produced by our approach can enhance the supervised training or fine-tuning of cell segmentation and classification models, especially those that utilize pre-trained foundation models to improve feature extraction robustness. In addition, these models can detect nuanced classes that enable researchers to conduct more detailed analyses of biological processes in computational pathology.

\section{Foundation models for robust cell segmentation and classification}

Accurate cell segmentation and classification in digital pathology are hindered by limited labeled data and the fact that conventional CNNs are unable to capture global contextual information due to their local receptive field constraints \cite{Gheflati_Rivaz_2022,Yang_Marcus_etal.}. Traditional approaches in cell quantification have predominantly relied on CNN encoders, such as ResNet50, given their proven effectiveness in semantic segmentation tasks \cite{Deshmane_2023,Graham_Vu_etal._2019,Mukasheva_Koishiyeva_etal._2024,Stringer_Wang_etal._2021}. However, approaches that include fine-tuning of pretrained CNNs, data augmentation, and stain normalization to partially increase data variability and address staining differences often fail to achieve the necessary generalization and robustness across diverse tissue types and staining conditions \cite{G._Wang_W._Li_etal._2018,Gao_Bagci_etal._2018,Karim_El_Khoury_Martin_Fockedey_etal._2021}.

To overcome these challenges, we leverage an encoder-decoder network that uses a foundation model as the encoder and a CNN upsampling decoder (\hyperref[fig:fig5]{Figure 5}) for simultaneous cell segmentation and classification in 2D patches extracted from WSIs. Foundation models with transformer-based architectures are viable alternatives to CNN-based encoders \cite{Shamshad_Khan_etal._2023,Sourget_2023}. They enable the creation of more advanced architectures that can decode or transform learned features more effectively \cite{Chen_Duan_etal._2023,Cheng_Misra_etal._2022,Xie_Wang_etal._2021}.

\begin{figure}[h!]
    \centering
    \includegraphics[width=\textwidth]{images/Figure_5.pdf}
    \caption{UNETR-like model with foundational model as backbone}
    \label{fig:fig5}
\end{figure}

By utilizing a transformer-based encoder, we incorporate global contextual information into the feature extraction process, which is a key advantage of such architectures \cite{Chen_Lu_etal._2021}. This foundation model integration facilitates accurate pixel-wise segmentation and classification without the need for extensive encoder training, thereby potentially improving generalization across varied cellular structures and tissue types.
In our implementation, we employ a modified UNETR \cite{Hatamizadeh_Tang_etal._2021} architecture that combines a vision transformer (ViT) \cite{Dosovitskiy_Beyer_etal._2021} encoder with a CNN-based decoder. The encoder utilizes the pretrained H-Optimus foundation model, which contains 1.1 billion parameters and is trained on over 500,000 H\&E stained WSIs \cite{Saillard_Jenatton_etal._2024}. We extract outputs from four evenly spaced transformer blocks $Z_i$, where $i \in [1, 14, 26, 38]$, to serve as residual connections for the CNN decoder. We select these blocks based on our observation that features from non-adjacent levels of the encoder lead to better overall performance on the test subset.

The CNN decoder upsamples the feature representations, acquired from the transformer blocks, to generate an intermediate vector that is handled by two task-specific layers that generate cell segmentation and classification masks. The first task-specific layer is the ‘Cellpose head’,  which is used to delineate cell instances. The layer generates horizontal and vertical gradient maps to form vector fields that are refined through gradient tracking in a post-processing step using the Cellpose algorithm \cite{Stringer_Wang_etal._2021}, known for its efficacy in cell segmentation tasks and generalizability across multiple domains \cite{Pachitariu_Stringer_2022,Stringer_Pachitariu_2024}. The second task-specific layer is the "Cell type head", which assigns labels to individual pixels. In the post-processing step, we determine the output classification label of each segmented cell instance by majority voting over the labeled pixels that comprise the cell in the segmentation map.

To evaluate model performance and measure the impact of adding a foundation model as backbone, we compare it to a ResNet50-based model. ResNet50 is a widely used solution for encoders in segmentation architectures in the medical domain \cite{Deshmane_2023,Graham_Vu_etal._2019,Mukasheva_Koishiyeva_etal._2024,Stringer_Wang_etal._2021}. For the H-Optimus-based model, we utilize frozen weights for the encoder and only fine-tune the decoder to take advantage of the extensive pre-training of the foundation model. For the ResNet50-based model we start with ImageNet \cite{Deng_Dong_etal.} weights and train both encoder and decoder parts. Hyperparameters for the training step are set to be identical, where possible, for comparable evaluation. 
For this evaluation, we deliberately use the PanNuke dataset to provide a standardized and controlled comparison between the H‑Optimus and ResNet50-based models (\hyperref[fig:S2]{Appendix Figure S2 (3)}). Specifically, we use two of the default PanNuke dataset splits (66\%) for training and validation, and reserve the third split (33\%) for testing.

To address the challenge of cell class imbalance in the PanNuke dataset, which is a common characteristic in most cell-level H\&E patch datasets, both models’ training processes employ a weighted loss function comprising cross-entropy and focal loss \cite{Lin_Goyal_etal._2018}. The focal loss component is adjusted with coefficients derived from each cell class' instance frequency, emphasizing learning from underrepresented classes and enhancing the model's sensitivity to rare but significant cellular patterns. The cross-entropy loss is augmented with spectral decoupling regularization \cite{Pezeshki_Kaba_etal._2021,Pohjonen_Stürenberg_etal._2022} and spatially varying label smoothing \cite{Islam_Glocker_2021}, which potentially stabilizes training and improves generalization in case of complex tissue morphologies. For optimization, we employ the \textit{AdamW} \cite{Loshchilov_Hutter_2019} to counter unbalanced class scenarios, with cosine annealing learning rate scheduler.

We utilize the scikit-learn library \cite{Van_der_Walt_Schönberger_etal._2014} and HoVer-Net \cite{Graham_Vu_etal._2019} implementations of $R^2$ (the coefficient of determination) and $PQ$ (panoptic quality) to evaluate our experiments. Complete mathematical formulations and detailed explanations of these metrics are provided in \hyperref[chap:S5]{Appendix S5}. To compute confidence intervals, we use nonparametric bootstrapping, where after calculating the metric on the full sample, we generated 1000 bootstrap replicates by resampling with replacement and then determined the 95\% confidence intervals as the 2.5th and 97.5th percentiles of the resulting empirical distribution.

%\hfill

The model comparisons are summarized in \hyperref[tab:2]{Table 2}. The H‑Optimus-based model achieves higher $R^2$ across all cell classes compared to the ResNet50-based model, which means that its predictions are more closely aligned with the PanNuke cell counts, indicating a stronger correlation with the observed data. Notably, the improvement of $R^2_{dead}$ may be an indicator of better global contextual representations provided by the foundation model backbone. In terms of segmentation and classification quality combined, measured by the PQ score, the H‑Optimus-based model demonstrates notable improvements across most cell classes. Overall, the average $R^2$ improved from 0.575 to 0.871, while the average $PQ$ score improved from 0.450 to 0.492, demonstrating better performance of the H-Optimus-based model.

\begin{table}[h!]
\renewcommand{\arraystretch}{1.5}
  \centering
  \caption{Cell quantification metrics for baseline and proposed models (CI 95\%).}
  \label{tab:2}
  \begin{tabular}{|l|c|c|}
    \hline
    %\rowcolor{gray!30}
    Metric             & Resnet50-based            & H-optimus-based              \\
    \hline
    $R^2_{neoplastic}$    & 0.681 (0.576--0.769)       & \textbf{0.941 (0.917--0.960)} \\
    \hline
    $R^2_{inflammatory}$  & 0.863 (0.778--0.903)       & \textbf{0.949 (0.918--0.966)} \\
    \hline
    $R^2_{connective}$    & 0.600 (0.488--0.698)       & 0.609 (0.436--0.772)          \\
    \hline
    $R^2_{dead}$          & 0.097 (-11.389--0.669)     & 0.925 (0.404--0.982)          \\
    \hline
    $R^2_{epithelial}$    & 0.635 (0.490--0.747)       & \textbf{0.930 (0.886--0.964)} \\
    \hline
    $PQ_{neoplastic}$       & 0.517 (0.499--0.535)       & \textbf{0.589 (0.575--0.604)} \\
    \hline
    $PQ_{inflammatory}$     & 0.455 (0.429--0.482)       & \textbf{0.528 (0.507--0.549)} \\
    \hline
    $PQ_{connective}$       & 0.416 (0.400--0.431)       & \textbf{0.451 (0.436--0.465)} \\
    \hline
    $PQ_{dead}$             & 0.374 (0.342--0.408)       & 0.292 (0.209--0.365)          \\
    \hline
    $PQ_{epithelial}$       & 0.488 (0.460--0.519)       & \textbf{0.599 (0.579--0.618)} \\
    \hline
  \end{tabular}
\end{table}

Our results  show that integrating the H‑Optimus foundation model within the UNETR architecture enhances the model's ability to segment and classify cells across diverse tissues from PanNuke data. The pretrained transformer encoder provides robust feature representations, resulting in higher average $R^2$ and $PQ$ scores compared to the CNN-based model. This leads to more reliable cell quantification and more accurate downstream analysis. Additionally, the streamlined fine-tuning process reduces computational overhead and training time, making the model more adaptable for new data.

Despite these advancements, the foundation model-based approach does not fully resolve all challenges related to cell segmentation and classification. We observe lower metric scores for underrepresented classes in the training data. Furthermore, foundation models typically encompass billions of parameters, resulting in substantial computational and memory requirements. It therefore poses challenges for deployment in resource-constrained environments, limiting their practical applicability in certain clinical settings.

\section{Model optimization via Knowledge Distillation}

To address the limitations posed by the extensive size of foundation models, we implement knowledge distillation — a model compression technique that leverages the teacher-student paradigm \cite{Hinton_Vinyals_etal._2015}. By training a smaller, more efficient student model to replicate the output of a larger, pre-trained teacher model, we retain performance while significantly reducing the model's complexity and resource requirements (\hyperref[fig:fig6]{Figure 6}).

\begin{figure}[h!]
    \centering
    \includegraphics[width=\textwidth, height=0.45\textheight, keepaspectratio]{images/Figure_6.pdf}
    \caption{Knowledge distillation framework for training a student model using a pre-trained teacher}
    \label{fig:fig6}
\end{figure}

We employ knowledge distillation to compress the H‑Optimus-based teacher model into a more efficient student model. The teacher model is the modified UNETR architecture with the H‑Optimus foundation model described in the previous chapter. The student model is based on a UNet architecture augmented with residual connections and incorporates a smaller ViT encoder with 9 million parameters \cite{Steiner_Kolesnikov_etal._2022,Wightman_2019}. 

First, we fine-tune the teacher model using the refined dataset from the cross-relabeling procedure (Section 2). Initially we train the decoder of the teacher model while keeping the encoder weights frozen. We split the refined dataset into train (70\%), validation (20\%) and test (10\%) subsets (\hyperref[fig:S2]{Appendix Figure S2 (4)}). During fine-tuning, we use the train and validation subsets, while leaving the test subset for model evaluation. We set the training procedure and model hyperparameters to be identical to those that were used to demonstrate the utility of foundation models for the simultaneous cell segmentation and classification task.

Next, we perform knowledge distillation from teacher to student using the refined dataset used to fine-tune the teacher model. The student model is trained to replicate the teacher model's outputs. We utilize a specialized loss function that aligns the student's predicted probability distribution with the teacher's, incorporating the teacher's class probability distribution derived from the output. Following the methodology of Hinton et al. \cite{Hinton_Vinyals_etal._2015}, we experiment with various hyperparameter settings for the temperature ($T$) and the balancing coefficients ($\alpha$ and $\beta$) in the loss function. We vary $T$ from 1 to 20 and adjust $\alpha$ and $\beta$ to balance the distillation and student losses. Through iterative tuning and evaluation, we identify that setting $T=14$, $\alpha=0.3$, and $\beta=0.7$ yields a configuration that converges and closely approximates the teacher model's performance during training.

Finally, we assess the performance of both models using the $R^2$ and $PQ$ (defined in \hyperref[chap:S5]{Appendix S5}) on the test set of the refined dataset (\hyperref[tab:3]{Table 3}). We observe that the 95\% confidence intervals overlap for most cell types, so we cannot claim statistically significant performance differences between the teacher and student models. One exception appears in the neoplastic class. The teacher model produces an $R^2$ of 0.919, while the student model shows an $R^2$ of 0.852. In addition, the student model achieves higher $PQ$ values for the neoplastic and connective classes, though the confidence intervals show overlap.

\begin{table}[h!]
\renewcommand{\arraystretch}{1.5}
  \centering
  \caption{Cell quantification metrics for teacher and distilled student models (CI 95\%).}
  \label{tab:3}
  \begin{tabular}{|l|c|c|}
    \hline
    %\rowcolor{gray!30}
    Metric & Teacher & Student \\
    \hline
    $R^2_{neoplastic}$    & \textbf{0.919} (0.898--0.939) & 0.852 (0.800--0.891) \\
    \hline
    $R^2_{lymphocyte}$    & 0.969 (0.956--0.977)         & 0.969 (0.956--0.978) \\
    \hline
    $R^2_{connective}$    & 0.694 (0.548--0.809)         & 0.618 (0.469--0.741) \\
    \hline
    $R^2_{dead}$          & 0.755 (0.400--0.908)         & 0.424 (0.100--0.731) \\
    \hline
    $R^2_{epithelial}$    & 0.922 (0.870--0.958)         & 0.843 (0.738--0.917) \\
    \hline
    $R^2_{macrophage}$    & 0.384 (-0.369--0.724)        & 0.704 (0.352--0.859) \\
    \hline
    $R^2_{neutrofil}$     & 0.854 (0.578--0.929)         & 0.833 (0.502--0.925) \\
    \hline
    $PQ_{neoplastic}$       & 0.581 (0.569--0.593)         & 0.601 (0.588--0.613) \\
    \hline
    $PQ_{lymphocyte}$       & 0.536 (0.520--0.553)         & 0.563 (0.544--0.579) \\
    \hline
    $PQ_{connective}$       & 0.436 (0.421--0.451)         & 0.457 (0.441--0.474) \\
    \hline
    $PQ_{dead}$             & 0.272 (0.235--0.315)         & 0.279 (0.201--0.369) \\
    \hline
    $PQ_{epithelial}$       & 0.522 (0.500--0.545)         & 0.530 (0.506--0.555) \\
    \hline
    $PQ_{macrophage}$       & 0.524 (0.459--0.588)         & 0.474 (0.405--0.543) \\
    \hline
    $PQ_{neutrofil}$        & 0.541 (0.490--0.592)         & 0.565 (0.522--0.607) \\
    \hline
  \end{tabular}
\end{table}


We further decompose the $PQ$ metric into its $SQ$ and $DQ$ components (\hyperref[tab:S6]{Appendix Table S6}). Both models produce nearly identical $SQ$ values, which indicates that they predict instance boundaries with similar precision. Although the student model shows some improvement in $DQ$ scores for certain classes, the confidence intervals overlap and do not confirm a statistically significant difference.

We observe that the student and teacher models yield comparable detection performance despite the student model using a much smaller and simpler architecture. A model with fewer parameters reduces the risk of overfitting when training data are scarce relative to the model’s complexity \cite{Farias_Ludermir_etal._2022}. The knowledge distillation process also encourages the student model to focus on the most generalizable detection features learned from the teacher. These factors enable the student model to achieve similar detection performance across different cell types.

Additionally, considering the model sizes reported in \hyperref[tab:4]{Table 4}, the distilled model achieves a significant reduction compared to the teacher model, with a 48-fold decrease in parameter count and a 5.5-fold reduction in on-disk size. In inference mode, the teacher model requires 16 GB of VRAM for a batch size of 32, while the distilled model only needs 3 GB of VRAM for the same batch size. These reductions make the distilled model significantly more practical for fine-tuning and deployment in resource-constrained environments.

\begin{table}[h!]
\renewcommand{\arraystretch}{1.5}
  \centering
  \caption{Parameter counts and size of teacher and distilled model}
  \label{tab:4}
  \adjustbox{max width=\textwidth}{%
  \begin{tabular}{|l|c|c|c|}
    \hline
    %\rowcolor{gray!30}
    Metric & H-optimus-based (Teacher) & mobileViT-based (Student) & Magnitude of difference \\
    \hline
    Parameters count       & 1,158,917,906   & \textbf{24,093,393}   & \textbf{48x}  \\
    \hline
    Estimated Total Size (MB) & 87,912       & \textbf{15,935}    & \textbf{5.5x} \\
    \hline
  \end{tabular}%
}
\end{table}

%\hfill

With recent advancements in complex network architectures and the use of pretrained encoders to achieve state-of-the-art performance \cite{Baumann_Dislich_etal._2024,Hörst_Rempe_etal._2024} in cell segmentation and classification tasks, model size, computational complexity, and processing times have increased. This limits the scalability and accessibility of these models. As we demonstrate, this may be mitigated using knowledge distillation. Studies in the field of natural language processing have demonstrated the efficacy of knowledge distillation in retaining the capabilities of the teacher model while achieving significant reductions in size and complexity \cite{Huangpu_Gao_2024,Sun_Yu_etal.}. 

We demonstrate the feasibility of knowledge distillation in digital pathology, specifically for cell segmentation and classification tasks. Moreover, we achieve this performance while also significantly reducing the parameter count. In addressing the challenge of knowledge transfer, we found that distillation from a transformer-based model to a smaller transformer is more straightforward than attempting to map transformer features to CNN blocks. In our experiments, using a CNN-based network as a student results in worse cell quantification performance due to the structural constraints of CNN feature space dimensions. 

Although our primary approach relies on a transformer-based student model that performs well, it can be further optimized to incorporate advantages from CNN architectures. For example, employing alternative techniques such as using ViT adapters \cite{Chen_Duan_etal._2023} or $1 \times 1$ convolutions to adjust feature map sizes may be beneficial for harnessing CNN advantages like enhanced local feature extraction. Moreover, if additional performance improvements are desired, the process can be further enhanced by applying supplementary knowledge distillation techniques, such as self-distillation \cite{Zhang_Song_etal._2019} or online distillation \cite{Houyon_Cioppa_etal._2023}.

Despite these promising results, further validation on independent datasets is necessary to fully understand the model's limitations. Underrepresented classes may pose challenges when addressing complex cases. Pathologists need to validate these models to adopt them in clinical settings. While the distilled models are smaller and more deployable, a technological gap persists because pathologists traditionally rely on established methods for inspecting WSIs and diagnosing diseases. Addressing the complexities involved in deploying models for inference and supporting pathologists in adopting new tools is essential for integrating these models into clinical workflows.

\section{Model integration with QuPath}
Digital pathology tools with graphical user interfaces are essential for visualizing and analyzing WSIs. To make our student model useful in clinical pathology workflows, it needs to be integrated into a tool that enables inspecting regions, creating annotations, and providing quantitative analyses of biomarkers. Therefore, we integrate the trained student model from the previous chapter into the QuPath open‑source platform \cite{Bankhead_Loughrey_etal._2017}. QuPath provides the required annotation, visualization, and analysis tools to interpret complex histological data, including workflows for cell segmentation, classification, and quantification (\hyperref[fig:fig7]{Figure 7}). 

\begin{figure}[h!]
    \centering
    \includegraphics[width=\textwidth]{images/Figure_7.pdf}
    \caption{Visualization of model-generated cell quantification annotations (left) and the corresponding unannotated slide (right) in QuPath}
    \label{fig:fig7}
\end{figure}

To identify the regions in a WSI critical for prognosticating tumor development, such as specific tumor areas or border regions without overlapping healthy tissue, the pathologist uses QuPath to outline these regions. Then, the pathologist initiates a cell segmentation and classification script through the QuPath interface for the selected regions. The resulting annotations and quantified cell information are then directly overlaid onto the WSI in the QuPath interface. Additional design and implementation details are in \hyperref[chap:S7]{Appendix S7}. 

Two common approaches for integrating deep learning models into QuPath are Java‑based native QuPath extensions \cite{Goldsborough_Philps_etal._2024} and the execution of RESTful API requests to a model server coupled with handling the response via an extension, as demonstrated in the application of cell segmentation models applied to immunofluorescence images \cite{Sugawara_2023}. While the community is actively working on these integration strategies, there is currently no universal solution that fully addresses all integration and performance requirements.

Extensions may offer better integration with QuPath, allowing slightly improved performance and more widespread usage of the built-in QuPath models, but they lack the flexibility to customize models and modify their behavior. For example, the newest version of QuPath includes models such as StarDist \cite{Weigert_Schmidt} and InstanSeg \cite{Goldsborough_Philps_etal._2024} that can perform cell segmentation. Both models pose limitations when applied to simultaneous cell segmentation and classification. StarDist performs well only on convex, round shapes by design, whereas some neoplastic, inflammatory, and connective cells exhibit complex and non-convex shapes. InstanSeg provides only semantic segmentation without assigning classes to the segmented cells.

%\hfill

In contrast, our approach offers an alternative integration strategy. It utilizes the paquo library to directly interact with QuPath’s internal application programming interface from within Python. This enables data exchange and processing without the need for intermediate conversion steps and provides greater control over model customization, retraining, and the incorporation of custom processing steps.

The integration of our custom model with QuPath underscores its potential to significantly enhance the diagnostic process by reducing the time burden on pathologists and enabling them to focus on more complex interpretative tasks using familiar software. Leveraging a tool that is already well-established among pathologists increases the likelihood of its adoption into daily clinical workflows. The quantitative data generated through the automated workflow is critical for both clinical decision-making and research, facilitating more accurate biomarker analysis, enabling robust statistical evaluations, and supporting hypothesis generation and testing. Additionally, by streamlining cell segmentation and classification, the tool enhances the scalability and reproducibility of pathological assessments, ultimately contributing to improved diagnostic accuracy and patient outcomes.

\section{Conclusion and future work}

In this study, we address critical challenges in digital pathology and tackle the usability and deployment issues of the developed models in standard computing environments without the need for high-performance computing systems. Our multi-faceted approach encompasses data refinement through cross-relabeling, leveraging foundation models for robust cell segmentation and classification, optimizing model performance via knowledge distillation, and integrating the optimized model into the QuPath software for practical application. This approach is used to construct a capable, versatile, and adjustable model for cell segmentation and classification, with enhanced performance and usability.

\begin{sloppypar}
While our approach shows potential in the field of computational pathology, certain limitations persist. 
For example, our implementation currently exhibits lower performance in detecting macrophages. 
This serves as an instance of the broader challenge of accurately identifying complex cell types. In order to address this issue, extending our approach to incorporate additional data sources, exploring alternative modeling approaches, and integrating other imaging modalities such as immunohistochemical staining may help improve detection accuracy. Moreover, although the distilled model reduces computational demands, integrating advanced deep learning models into clinical practice requires addressing technological gaps and potential resistance to adopting new tools within established diagnostic processes.
\end{sloppypar}

Future work could focus on several key areas to refine the proposed approach and facilitate its adoption in clinical environments. Enhancing the cell-relabeling process with additional datasets \cite{Graham_Jahanifar_etal._2021} could improve the representation of underrepresented cell types and enhance overall model performance. Also, incorporating additional data sources, such as multi-modal imaging or complementary staining methods, may address limitations related to cell type differentiation and class imbalance. Exploring other foundation models \cite{Vorontsov_Bozkurt_etal._2024,Zimmermann_Vorontsov_etal._2024} or introducing additional modalities \cite{Ding_Wagner_etal._2024,Vaidya_Zhang_etal._2025} may provide alternative architectures better suited to specific tasks or offer improved efficiency. Implementing more complex knowledge distillation techniques \cite{Houyon_Cioppa_etal._2023,Zhang_Song_etal._2019} could further optimize the model's performance and adaptability. Additionally, deeper integration with QuPath or other digital pathology software could provide pathologists more control over cell quantification analysis directly within the QuPath interface, thereby increasing accessibility and usability. Such enhancements would not only refine model performance but also ensure greater adaptability and scalability within various clinical environments. Finally, extensive validation of the model by pathologists and benchmarking against independent datasets are essential steps toward establishing the model's reliability and fostering confidence in its clinical utility.

\section*{Acknowledgments} 
This work was funded in part by the Research Council of Norway grant no. 309439 SFI Visual Intelligence, and the North Norwegian Health Authority grant no. HNF1521-20.

\bibliographystyle{IEEEtran}
\begin{sloppypar}
\begin{thebibliography}{99}

\bibitem{chaplot2020neural} Chaplot, Devendra Singh, et al. "Neural topological slam for visual navigation." Proceedings of the IEEE/CVF conference on computer vision and pattern recognition. 2020.

\bibitem{maksymets2021thda} Maksymets, Oleksandr, et al. "Thda: Treasure hunt data augmentation for semantic navigation." Proceedings of the IEEE/CVF International Conference on Computer Vision. 2021.

\bibitem{mezghan2022memory} Mezghan, Lina, et al. "Memory-augmented reinforcement learning for image-goal navigation." 2022 IEEE/RSJ International Conference on Intelligent Robots and Systems (IROS). IEEE, 2022.

\bibitem{al2022zero} Al-Halah, Ziad, Santhosh Kumar Ramakrishnan, and Kristen Grauman. "Zero experience required: Plug \& play modular transfer learning for semantic visual navigation." Proceedings of the IEEE/CVF Conference on Computer Vision and Pattern Recognition. 2022.

\bibitem{ye2021auxiliary} Ye, Joel, et al. "Auxiliary tasks and exploration enable objectgoal navigation." Proceedings of the IEEE/CVF international conference on computer vision. 2021.

\bibitem{chaplot2020object} Chaplot, Devendra Singh, et al. "Object goal navigation using goal-oriented semantic exploration." Advances in Neural Information Processing Systems 33 (2020)

\bibitem{ramakrishnan2022poni} Ramakrishnan, Santhosh Kumar, et al. "Poni: Potential functions for objectgoal navigation with interaction-free learning." Proceedings of the IEEE/CVF Conference on Computer Vision and Pattern Recognition. 2022.

\bibitem{ramrakhya2022habitat} Ramrakhya, Ram, et al. "Habitat-web: Learning embodied object-search strategies from human demonstrations at scale." Proceedings of the IEEE/CVF Conference on Computer Vision and Pattern Recognition. 2022.

\bibitem{mousavian2019visual} Mousavian, Arsalan, et al. "Visual representations for semantic target driven navigation." 2019 International Conference on Robotics and Automation (ICRA). IEEE, 2019.

\bibitem{dhariwal2021diffusion} Dhariwal, Prafulla, and Alexander Nichol. "Diffusion models beat gans on image synthesis." Advances in neural information processing systems 34 (2021)

\bibitem{ho2022classifier} Ho, Jonathan, and Tim Salimans. "Classifier-free diffusion guidance." arXiv preprint arXiv:2207.12598 (2022).

\bibitem{nichol2021glide} Nichol, Alex, et al. "Glide: Towards photorealistic image generation and editing with text-guided diffusion models." arXiv preprint arXiv:2112.10741 (2021)

\bibitem{brooks2023instructpix2pix} Brooks, Tim, Aleksander Holynski, and Alexei A. Efros. "Instructpix2pix: Learning to follow image editing instructions." Proceedings of the IEEE/CVF Conference on Computer Vision and Pattern Recognition. 2023.

\bibitem{fu2023guiding} Fu, Tsu-Jui, et al. "Guiding instruction-based image editing via multimodal large language models." arXiv preprint arXiv:2309.17102 (2023).

\bibitem{geng2024instructdiffusion} Geng, Zigang, et al. "Instructdiffusion: A generalist modeling interface for vision tasks." Proceedings of the IEEE/CVF Conference on Computer Vision and Pattern Recognition. 2024.

\bibitem{zhou2024minedreamer} Zhou, Enshen, et al. "Minedreamer: Learning to follow instructions via chain-of-imagination for simulated-world control." arXiv preprint arXiv:2403.12037 (2024).

\bibitem{zhou2023esc} Zhou, Kaiwen, et al. "Esc: Exploration with soft commonsense constraints for zero-shot object navigation." International Conference on Machine Learning. PMLR, 2023.

\bibitem{yu2023l3mvn} Yu, Bangguo, Hamidreza Kasaei, and Ming Cao. "L3mvn: Leveraging large language models for visual target navigation." 2023 IEEE/RSJ International Conference on Intelligent Robots and Systems (IROS). IEEE, 2023.

\bibitem{gadre2023cows} Gadre, Samir Yitzhak, et al. "Cows on pasture: Baselines and benchmarks for language-driven zero-shot object navigation." Proceedings of the IEEE/CVF Conference on Computer Vision and Pattern Recognition. 2023.

\bibitem{shah2023navigation} Shah, Dhruv, et al. "Navigation with large language models: Semantic guesswork as a heuristic for planning." Conference on Robot Learning. PMLR, 2023.

\bibitem{cai2024bridging} Cai, Wenzhe, et al. "Bridging zero-shot object navigation and foundation models through pixel-guided navigation skill." 2024 IEEE International Conference on Robotics and Automation (ICRA). IEEE, 2024.

\bibitem{yu2023co} Yu, Bangguo, Hamidreza Kasaei, and Ming Cao. "Co-NavGPT: Multi-robot cooperative visual semantic navigation using large language models." arXiv preprint arXiv:2310.07937 (2023).

\bibitem{wu2024voronav} Wu, Pengying, et al. "Voronav: Voronoi-based zero-shot object navigation with large language model." arXiv preprint arXiv:2401.02695 (2024).

\bibitem{qin2023mp5} Qin, Yiran, et al. "Mp5: A multi-modal open-ended embodied system in minecraft via active perception." arXiv preprint arXiv:2312.07472 (2023).

\bibitem{du2024learning} Du, Yilun, et al. "Learning universal policies via text-guided video generation." Advances in Neural Information Processing Systems 36 (2024).

\bibitem{ajay2024compositional} Ajay, Anurag, et al. "Compositional foundation models for hierarchical planning." Advances in Neural Information Processing Systems 36 (2024).

\bibitem{liang2024skilldiffuser} Liang, Zhixuan, et al. "Skilldiffuser: Interpretable hierarchical planning via skill abstractions in diffusion-based task execution." Proceedings of the IEEE/CVF Conference on Computer Vision and Pattern Recognition. 2024.

\bibitem{heusel2017gans} Heusel, Martin, et al. "Gans trained by a two time-scale update rule converge to a local nash equilibrium." Advances in neural information processing systems 30 (2017).

\bibitem{zhang2018unreasonable} Zhang, Richard, et al. "The unreasonable effectiveness of deep features as a perceptual metric." Proceedings of the IEEE conference on computer vision and pattern recognition. 2018.

\bibitem{brown2020language} Brown, Tom B. "Language models are few-shot learners." arXiv preprint arXiv:2005.14165 (2020).

\bibitem{podell2023sdxl} Podell, Dustin, et al. "Sdxl: Improving latent diffusion models for high-resolution image synthesis." arXiv preprint arXiv:2307.01952 (2023).

\bibitem{brohan2022rt} Brohan, Anthony, et al. "Rt-1: Robotics transformer for real-world control at scale." arXiv preprint arXiv:2212.06817 (2022).

\bibitem{brohan2023rt} Brohan, Anthony, et al. "Rt-2: Vision-language-action models transfer web knowledge to robotic control." arXiv preprint arXiv:2307.15818 (2023).

\bibitem{li2024manipllm} Li, Xiaoqi, et al. "Manipllm: Embodied multimodal large language model for object-centric robotic manipulation." Proceedings of the IEEE/CVF Conference on Computer Vision and Pattern Recognition. 2024.

\bibitem{shah2023vint} Shah, Dhruv, et al. "ViNT: A foundation model for visual navigation." arXiv preprint arXiv:2306.14846 (2023).

\bibitem{liu2024visual} Liu, Haotian, et al. "Visual instruction tuning." Advances in neural information processing systems 36 (2024).

\bibitem{hu2021lora} Hu, Edward J., et al. "Lora: Low-rank adaptation of large language models." arXiv preprint arXiv:2106.09685 (2021).

\bibitem{qin2023supfusion} Qin, Yiran, et al. "SupFusion: Supervised LiDAR-camera fusion for 3D object detection." Proceedings of the IEEE/CVF International Conference on Computer Vision. 2023.

\bibitem{qin2024worldsimbench} Qin, Yiran, et al. "Worldsimbench: Towards video generation models as world simulators." arXiv preprint arXiv:2410.18072 (2024).

\bibitem{yu2025gamefactory} Yu, Jiwen, et al. "GameFactory: Creating New Games with Generative Interactive Videos." arXiv preprint arXiv:2501.08325 (2025).

\bibitem{zhou2024code} Zhou, Enshen, et al. "Code-as-Monitor: Constraint-aware Visual Programming for Reactive and Proactive Robotic Failure Detection." arXiv preprint arXiv:2412.04455 (2024).

\bibitem{zhang2024ad} Zhang, Zaibin, et al. "AD-H: Autonomous Driving with Hierarchical Agents." arXiv preprint arXiv:2406.03474 (2024).

\bibitem{wang2024toward} Wang, Chaoqun, et al. "Toward Accurate Camera-based 3D Object Detection via Cascade Depth Estimation and Calibration." arXiv preprint arXiv:2402.04883 (2024).

\bibitem{huang2024story3d} Huang, Yuzhou, et al. "Story3d-agent: Exploring 3d storytelling visualization with large language models." arXiv preprint arXiv:2408.11801 (2024).

\bibitem{savinov2018semi} Savinov, Nikolay, Alexey Dosovitskiy, and Vladlen Koltun. "Semi-parametric topological memory for navigation." arXiv preprint arXiv:1803.00653 (2018).

\bibitem{majumdar2022zson} Majumdar, Arjun, et al. "Zson: Zero-shot object-goal navigation using multimodal goal embeddings." Advances in Neural Information Processing Systems 35 (2022): 32340-32352.

\bibitem{yadav2023offline} Yadav, Karmesh, et al. "Offline visual representation learning for embodied navigation." Workshop on Reincarnating Reinforcement Learning at ICLR 2023. 2023.

\bibitem{yadav2023ovrl} Yadav, Karmesh, et al. "Ovrl-v2: A simple state-of-art baseline for imagenav and objectnav." arXiv preprint arXiv:2303.07798 (2023).

\bibitem{sun2024fgprompt} Sun, Xinyu, et al. "FGPrompt: fine-grained goal prompting for image-goal navigation." Advances in Neural Information Processing Systems 36 (2024).

\bibitem{zhu2017target} Zhu, Yuke, et al. "Target-driven visual navigation in indoor scenes using deep reinforcement learning." 2017 IEEE international conference on robotics and automation (ICRA). IEEE, 2017.

\bibitem{koh2024generating} Koh, Jing Yu, Daniel Fried, and Russ R. Salakhutdinov. "Generating images with multimodal language models." Advances in Neural Information Processing Systems 36 (2024).

\bibitem{krantz2022instance} Krantz, Jacob, et al. "Instance-specific image goal navigation: Training embodied agents to find object instances." arXiv preprint arXiv:2211.15876 (2022).

\bibitem{schulman2017proximal} Schulman, John, et al. "Proximal policy optimization algorithms." arXiv preprint arXiv:1707.06347 (2017).

\bibitem{anderson2018evaluation} Anderson, Peter, et al. "On evaluation of embodied navigation agents." arXiv preprint arXiv:1807.06757 (2018).

\bibitem{lin2024navcot} Lin, Bingqian, et al. "NavCoT: Boosting LLM-Based Vision-and-Language Navigation via Learning Disentangled Reasoning." arXiv preprint arXiv:2403.07376 (2024).

\bibitem{NavGPT} Zhou, Gengze, Yicong Hong, and Qi Wu. "Navgpt: Explicit reasoning in vision-and-language navigation with large language models." Proceedings of the AAAI Conference on Artificial Intelligence.

\bibitem{hahn2021no} Hahn, Meera, et al. "No rl, no simulation: Learning to navigate without navigating." Advances in Neural Information Processing Systems 34 (2021): 26661-26673.

\bibitem{li2025t2isafety} Li, Lijun, et al. "T2ISafety: Benchmark for Assessing Fairness, Toxicity, and Privacy in Image Generation." arXiv preprint arXiv:2501.12612 (2025).

\bibitem{an2024agfsync} An, Jingkun, et al. "AGFSync: Leveraging AI-Generated Feedback for Preference Optimization in Text-to-Image Generation." arXiv preprint arXiv:2403.13352 (2024).


\end{thebibliography}
\end{sloppypar}

\clearpage
\beginsupplement
\section*{Appendix}
\renewcommand{\thesubsection}{S\arabic{subsection}}

\subsection{\label{chap:S1}PanNuke and MoNuSAC preprocessing}
The PanNuke dataset comprises a set of 7,901 RGB patches, each with dimensions of $256 \times 256$ pixels, which we set as the standard patch size for our analysis. In contrast, the MoNuSAC dataset encompasses 294 images of heterogeneous dimensions. To standardize the MoNuSAC images with our experiments, we implement a standardization protocol. Specifically, for images exceeding the dimensions of $256 \times 256$ pixels, we segment them into equal-sized patches and apply mirror padding to the remaining portions to avoid information loss at the peripherals. Patches with dimensions less than $128 \times 128$ pixels are excluded from the dataset due to the insufficient resolution to capture relevant cellular details. For patches where either dimension falls between 128 and 256 pixels, we employ upsampling to achieve the standard patch size. As a result, we obtain a total of 2,823 RGB patches derived from the MoNuSAC dataset for subsequent analysis. For additional details on the MoNuSAC data preparation process, refer to the source code \cite{Shvetsov_2025a}.
\clearpage

\subsection{\label{chap:S2}Data usage for the methodology}

\counterwithin{figure}{subsection}
\renewcommand{\thefigure}{S\arabic{subsection}}

\begin{figure}[h!]
    \centering
    \includegraphics[width=\textwidth, height=0.85\textheight, keepaspectratio]{images/A2.pdf}
    \caption{Overview of the methodology for cross-labeling, dataset refinement, and model comparison. (1) Cross-relabeling - training and testing cell classification models, (2) Cross-relabeling - using cell classification models to create refined dataset, (3) Fine-tuning and training models for comparison, (4) Student knowledge distillation with refined dataset}
    \label{fig:S2}
\end{figure}
\clearpage

\subsection{\label{chap:S3}Confusion matrices for classification models}
\counterwithin{figure}{subsection}
\renewcommand{\thefigure}{S\arabic{subsection}.\arabic{figure}}

\begin{figure}[h!]
    \centering
    \includegraphics[width=\textwidth, height=0.4\textheight, keepaspectratio]{images/A3_1.pdf}
    \caption{Confusion matrix for PanNuke trained model}
    \label{fig:S3.1}
\end{figure}

\begin{figure}[h!]
    \centering
    \includegraphics[width=\textwidth, height=0.4\textheight, keepaspectratio]{images/A3_2.pdf}
    \caption{Confusion matrix for MoNuSAC trained model}
    \label{fig:S3.2}
\end{figure}

\clearpage

\subsection{\label{chap:S4}Datasets cell counts}

\counterwithin{table}{subsection}
\renewcommand{\thetable}{S\arabic{subsection}}

\begin{table}[h!]
\renewcommand{\arraystretch}{2.0}
\centering
\caption{\label{tab:S4}Cell counts for PanNuke, MoNuSAC and refined datasets. Numbers in parentheses indicate preprocessed cell counts for cell classifier models training and testing.}
%\adjustbox{max width=\textwidth}{%
\begin{tabular}{|l|c|c|c|}
\hline
%\rowcolor{gray!30}
Cell type & PanNuke & MoNuSAC & Refined \\
\hline
Neoplastic & 77,403 (68,031) & - & 105,451 \\
\hline
Epithelial & 26,572 (23,207) & - & 29,926 \\
\hline
Epithelial (benign and malignant) & - & 31,402 & - \\
\hline
Inflammatory & 32,276 & - & - \\
\hline
Lymphocytes & - & 37,045 (33,104) & 65,275 \\
\hline
Neutrophils & - & 1,355 (1,252) & 3,833 \\
\hline
Macrophage & - & 1,842 (1,695) & 3,410 \\
\hline
Dead & 2,908 & - & 2,908 \\
\hline
Connective & 50,585 & - & 50,585 \\
\hline
\end{tabular}
%
%}
\end{table}



\clearpage

\subsection{\label{chap:S5}Definition of validation metrics}
\counterwithin{equation}{subsection}
\renewcommand{\theequation}{\arabic{equation}}

\subsubsection{\label{chap:S5.1}R\textsuperscript{2}}
The coefficient of determination, denoted as $R^2$, is a statistical measure that represents the proportion of variance in the dependent variable that is predictable from the independent variables. In the context of cell quantification in pathology, $R^2$ is used to assess how well the predicted quantities of different cell types in a patch align with the actual quantities observed in the ground truth data, with higher values representing more accurate quantification. $R^2$ is defined as
\begin{equation*}
R^2 = 1 - \frac{\sum_{i=1}^n (y_i - \hat{y}_i)^2}{\sum_{i=1}^n (y_i - \bar{y})^2},
\end{equation*}
where $y_i$ represents the actual number of cells of a specific type in the $i$-th image, $\hat{y}_i$ represents the predicted number of cells of that type in the $i$-th image, $\bar{y}$ is the mean of the actual numbers across all images, and $n$ is the total number of images in the dataset.

The $R^2$ metric has a range of $(-\infty, 1]$. An $R^2$ of 1 indicates perfect prediction, where all predicted values exactly match the actual values. An $R^2$ of 0 suggests that the model explains none of the variability of the response data around its mean. If $R^2$ is negative, it indicates that the model performs worse than a model that simply predicts the mean of the actual values for all observations.

\subsubsection{\label{chap:S5.2}PQ}
Panoptic Quality ($PQ$) is a comprehensive metric used to evaluate the performance of segmentation models in tasks that require both instance segmentation and classification. $PQ$ provides a single score that encapsulates both the detection accuracy (i.e., how many objects were correctly identified) and the segmentation quality (i.e., how accurately the objects' boundaries were delineated). This metric is particularly useful in multiclass scenarios where each pixel is classified into distinct categories, such as different cell types in pathology images.

$PQ$ is calculated as the product of two terms: Detection Quality ($DQ$) and Segmentation Quality ($SQ$). It can be expressed as
\begin{equation*}
PQ = DQ \cdot SQ,
\end{equation*}
where
\begin{equation*}
DQ = \frac{TP}{TP + 0.5\, FP + 0.5\, FN},
\end{equation*}
\begin{equation*}
SQ = \frac{\sum_{(p, g) \in \mathcal{M}} IoU(p, g)}{TP}.
\end{equation*}
In these formulas, $TP$ denotes the number of correctly matched instances between ground truth and prediction, $FP$ denotes the predicted instances that have no corresponding ground truth, $FN$ denotes the ground truth instances that were not detected, $IoU(p, g)$ is the Intersection over Union for a pair of matched instances $p$ (prediction) and $g$ (ground truth), and $\mathcal{M}$ is the set of matched pairs.

The $PQ$ metric is calculated for each class and is averaged across classes to provide a global performance measure.

The $PQ$ score has a range of $[0, 1.0]$, where a higher score indicates better performance in both detecting and segmenting the instances correctly. A $PQ$ of 1 signifies perfect identification and segmentation of all instances, whereas a $PQ$ of 0 indicates that no instances were correctly identified and segmented.

\clearpage

\subsection{\label{chap:S6}Segmentation and Detection quality metrics for teacher and student models}

\begin{table}[h!]
\renewcommand{\arraystretch}{2.0}
\centering
\caption{Segmentation and detection quality for student and teacher models (CI 95\%)}
\label{tab:S6}
%\adjustbox{max width=\textwidth}{%
\begin{tabular}{|l|c|c|}
\hline
%\rowcolor{gray!30}
Metric & Teacher & Student \\
\hline
$SQ_{neoplastic}$ & 0.819 (0.815--0.823) & 0.824 (0.819--0.828) \\
\hline
$SQ_{lymphocyte}$ & 0.795 (0.788--0.802) & 0.790 (0.783--0.796) \\
\hline
$SQ_{connective}$ & 0.770 (0.762--0.776) & 0.780 (0.772--0.786) \\
\hline
$SQ_{dead}$ & 0.659 (0.623--0.688) & 0.657 (0.624--0.695) \\
\hline
$SQ_{epithelial}$ & 0.780 (0.770--0.790) & 0.788 (0.779--0.797) \\
\hline
$SQ_{macrophage}$ & 0.788 (0.760--0.810) & 0.757 (0.730--0.783) \\
\hline
$SQ_{neutrofil}$ & 0.782 (0.761--0.801) & 0.775 (0.759--0.792) \\
\hline
$DQ_{neoplastic}$ & 0.706 (0.692--0.719) & 0.727 (0.712--0.741) \\
\hline
$DQ_{lymphocyte}$ & 0.675 (0.656--0.698) & 0.713 (0.691--0.734) \\
\hline
$DQ_{connective}$ & 0.566 (0.546--0.584) & 0.583 (0.565--0.602) \\
\hline
$DQ_{dead}$ & 0.410 (0.361--0.465) & 0.435 (0.306--0.561) \\
\hline
$DQ_{epithelial}$ & 0.668 (0.639--0.694) & 0.673 (0.644--0.702) \\
\hline
$DQ_{macrophage}$ & 0.657 (0.583--0.727) & 0.615 (0.531--0.703) \\
\hline
$DQ_{neutrofil}$ & 0.691 (0.625--0.753) & 0.729 (0.679--0.778) \\
\hline
\end{tabular}
%
%}
\end{table}

\clearpage

\subsection{\label{chap:S7}QuPath integration method}
We adopt an integration strategy leveraging the paquo \cite{Bayer_AG} library, a Python package that enables direct interaction with QuPath’s internal API, thereby facilitating seamless data exchange without intermediate conversion steps. The data processing pipeline (\hyperref[fig:S7]{Appendix Figure S7}) begins with the acquisition of WSIs and their associated annotations from QuPath, which are represented as Shapely \cite{Gillies_Wel_etal._2024} polygons. Utilizing paquo, we directly read, create, and modify these annotations and detections within a QuPath project in the Python environment. Images are then cropped using these polygons and processed by cell segmentation and classification models employing standard vision processing toolkits such as OpenCV, pyvips, and PyTorch. Additionally, QuPath employs Groovy scripts to initiate a Python process that starts the entire pipeline from QuPath graphical interface: fetching polygons, extracting images from them, and running deep learning model inference on the cropped images. 
The results are returned to QuPath, leveraging paquo's Python bindings to manipulate QuPath data while minimizing the computational overhead typically associated with cross-environment communication.

\counterwithin{figure}{subsection}
\renewcommand{\thefigure}{S\arabic{subsection}}

\begin{figure}[h!]
    \centering
    \includegraphics[width=\textwidth]{images/A7.pdf}
    \caption{QuPath integration workflow using Python environment}
    \label{fig:S7}
\end{figure}

Compared to traditional workflows that involve exporting annotations as GeoJSON, classifying them in Python, and reimporting them into QuPath, our approach offers several advantages. We eliminate the need to switch between programming languages, providing a cohesive and streamlined development process entirely within QuPath software and removing the necessity to use other tools. Meanwhile, we avoid storing annotations as intermediate JSON files unless required for external use or archiving. By conducting the entire inference and post-processing workflow within the Python environment, we leverage the power and flexibility of Python libraries for image processing and machine learning. This approach also enables adjustments to any set of labels and models, thereby improving its applicability.

%\hfill

The distilled model and QuPath integration code are packaged into a Docker container, enabling streamlined execution with the Docker engine. Detailed integration code and deployment instructions can be found in the GitHub repository \cite{Shvetsov_2025b}.

Despite these benefits, we acknowledge that the paquo library is a proof‑of‑concept project in its early development stage and has not been tested across all versions of QuPath.

\clearpage

\subsection{\label{chap:S8}Data and code availability statement}
All datasets, models, and code used in this study are publicly available and can be obtained from the repositories listed below. 
The PanNuke \cite{Gamper_Koohbanani_etal._2019} and MoNuSAC \cite{Verma_Kumar_etal._2021} datasets are publicly accessible, and download information along with detailed descriptions can be found in their respective articles. Preprocessing scripts for PanNuke and MoNuSAC data, as well as individual cell extraction scripts, are available on GitHub \cite{Shvetsov_2025a}. The H-Optimus foundation model used in our experiments can be downloaded from the HuggingFace repository \cite{hoptimus2024}, and model information is available on GitHub \cite{Saillard_Jenatton_etal._2024}. In addition, the integration code for QuPath and the distilled model packaged in a Docker container are provided in the repository \cite{Shvetsov_2025b}, and paquo Python library is available from the authors GitHub repository \cite{Bayer_AG}.
\clearpage

\end{document}

% \bibliography{ref}

% \begin{IEEEbiography}
% % [{\includegraphics[width=1in,height=1.25in,clip,keepaspectratio]{./AuthorPhotos/Linhao.jpg}}]{Lin Hao}
% A
% \end{IEEEbiography}
% \begin{IEEEbiography}
% % [{\includegraphics[width=1in,height=1.25in,clip,keepaspectratio]{./AuthorPhotos/MustafaKishk.jpg}}]{Mustafa A. Kishk} 
% (Member, IEEE) received the
% B.Sc. and M.Sc. degrees from Cairo University
% in 2013 and 2015, respectively, and the Ph.D. degree
% from Virginia Tech in 2018. He is currently a
% Post-Doctoral Research Fellow with the CTL at
% KAUST. His current research interests include stochastic geometry, energy harvesting wireless networks, UAV-enabled communication systems, and
% satellite communications.
% \end{IEEEbiography}



\end{document}

