% \documentclass[journal,draftclsnofoot,onecolumn,12pt]{IEEEtran}
\documentclass[final]{IEEEtran}
\usepackage{amsthm,amssymb,graphicx,multirow,amsmath,color,amsfonts,physics}%,ulem}
\usepackage[update,prepend]{epstopdf}
\usepackage[noadjust]{cite}
% \usepackage[latin1]{inputenc}
\usepackage{tikz}
\usepackage{bbm} % for \nbb1 
\usepackage{pdfpages}
\usepackage{balance}
%\usepackage{flushend}
%\usepackage{tabulary}
\usepackage{multirow}
\usepackage{comment}
\usepackage{subfigure}
\usepackage{yhmath}
%\usepackage[justification=centering]{caption}
% Colors
\def\chr#1{{\color{red} #1}}
\def\chb#1{{\color{blue} #1}}
\newtheorem{corollary}{Corollary}
\allowdisplaybreaks % Allows breaking of eqnarray over multiple pages (avoids unnecessary blanks in the document before eqnarray)
\allowdisplaybreaks % Allows breaking of eqnarray over multiple pages (avoids unnecessary blanks in the document before eqnarray)
% \usepackage{setspace}    % Remove in double-column version. Also, search for \setstretch in the body of the paper and comment these commands for double column

%  %Reduces space around equations and figures 
\setlength\abovedisplayskip{3pt plus 2pt minus 2pt}     % Reduce space before equation
\setlength\belowdisplayskip{3pt plus 2pt minus 2pt}    % Reduce space after equation
\setlength\textfloatsep{10pt plus 2pt minus 2pt}    
\usepackage{float}

% \usepackage{gensymb}
\begin{document}
% !TeX root = main.tex 


\newcommand{\lnote}{\textcolor[rgb]{1,0,0}{Lydia: }\textcolor[rgb]{0,0,1}}
\newcommand{\todo}{\textcolor[rgb]{1,0,0.5}{To do: }\textcolor[rgb]{0.5,0,1}}


\newcommand{\state}{S}
\newcommand{\meas}{M}
\newcommand{\out}{\mathrm{out}}
\newcommand{\piv}{\mathrm{piv}}
\newcommand{\pivotal}{\mathrm{pivotal}}
\newcommand{\isnot}{\mathrm{not}}
\newcommand{\pred}{^\mathrm{predict}}
\newcommand{\act}{^\mathrm{act}}
\newcommand{\pre}{^\mathrm{pre}}
\newcommand{\post}{^\mathrm{post}}
\newcommand{\calM}{\mathcal{M}}

\newcommand{\game}{\mathbf{V}}
\newcommand{\strategyspace}{S}
\newcommand{\payoff}[1]{V^{#1}}
\newcommand{\eff}[1]{E^{#1}}
\newcommand{\p}{\vect{p}}
\newcommand{\simplex}[1]{\Delta^{#1}}

\newcommand{\recdec}[1]{\bar{D}(\hat{Y}_{#1})}





\newcommand{\sphereone}{\calS^1}
\newcommand{\samplen}{S^n}
\newcommand{\wA}{w}%{w_{\mathfrak{a}}}
\newcommand{\Awa}{A_{\wA}}
\newcommand{\Ytil}{\widetilde{Y}}
\newcommand{\Xtil}{\widetilde{X}}
\newcommand{\wst}{w_*}
\newcommand{\wls}{\widehat{w}_{\mathrm{LS}}}
\newcommand{\dec}{^\mathrm{dec}}
\newcommand{\sub}{^\mathrm{sub}}

\newcommand{\calP}{\mathcal{P}}
\newcommand{\totspace}{\calZ}
\newcommand{\clspace}{\calX}
\newcommand{\attspace}{\calA}

\newcommand{\Ftil}{\widetilde{\calF}}

\newcommand{\totx}{Z}
\newcommand{\classx}{X}
\newcommand{\attx}{A}
\newcommand{\calL}{\mathcal{L}}



\newcommand{\defeq}{\mathrel{\mathop:}=}
\newcommand{\vect}[1]{\ensuremath{\mathbf{#1}}}
\newcommand{\mat}[1]{\ensuremath{\mathbf{#1}}}
\newcommand{\dd}{\mathrm{d}}
\newcommand{\grad}{\nabla}
\newcommand{\hess}{\nabla^2}
\newcommand{\argmin}{\mathop{\rm argmin}}
\newcommand{\argmax}{\mathop{\rm argmax}}
\newcommand{\Ind}[1]{\mathbf{1}\{#1\}}

\newcommand{\norm}[1]{\left\|{#1}\right\|}
\newcommand{\fnorm}[1]{\|{#1}\|_{\text{F}}}
\newcommand{\spnorm}[2]{\left\| {#1} \right\|_{\text{S}({#2})}}
\newcommand{\sigmin}{\sigma_{\min}}
\newcommand{\tr}{\text{tr}}
\renewcommand{\det}{\text{det}}
\newcommand{\rank}{\text{rank}}
\newcommand{\logdet}{\text{logdet}}
\newcommand{\trans}{^{\top}}
\newcommand{\poly}{\text{poly}}
\newcommand{\polylog}{\text{polylog}}
\newcommand{\st}{\text{s.t.~}}
\newcommand{\proj}{\mathcal{P}}
\newcommand{\projII}{\mathcal{P}_{\parallel}}
\newcommand{\projT}{\mathcal{P}_{\perp}}
\newcommand{\projX}{\mathcal{P}_{\mathcal{X}^\star}}
\newcommand{\inner}[1]{\langle #1 \rangle}

\renewcommand{\Pr}{\mathbb{P}}
\newcommand{\Z}{\mathbb{Z}}
\newcommand{\N}{\mathbb{N}}
\newcommand{\R}{\mathbb{R}}
\newcommand{\E}{\mathbb{E}}
\newcommand{\F}{\mathcal{F}}
\newcommand{\var}{\mathrm{var}}
\newcommand{\cov}{\mathrm{cov}}


\newcommand{\calN}{\mathcal{N}}

\newcommand{\jccomment}{\textcolor[rgb]{1,0,0}{C: }\textcolor[rgb]{1,0,1}}
\newcommand{\fracpar}[2]{\frac{\partial #1}{\partial  #2}}

\newcommand{\A}{\mathcal{A}}
\newcommand{\B}{\mat{B}}
%\newcommand{\C}{\mat{C}}

\newcommand{\I}{\mat{I}}
\newcommand{\M}{\mat{M}}
\newcommand{\D}{\mat{D}}
%\newcommand{\U}{\mat{U}}
\newcommand{\V}{\mat{V}}
\newcommand{\W}{\mat{W}}
\newcommand{\X}{\mat{X}}
\newcommand{\Y}{\mat{Y}}
\newcommand{\mSigma}{\mat{\Sigma}}
\newcommand{\mLambda}{\mat{\Lambda}}
\newcommand{\e}{\vect{e}}
\newcommand{\g}{\vect{g}}
\renewcommand{\u}{\vect{u}}
\newcommand{\w}{\vect{w}}
\newcommand{\x}{\vect{x}}
\newcommand{\y}{\vect{y}}
\newcommand{\z}{\vect{z}}
\newcommand{\fI}{\mathfrak{I}}
\newcommand{\fS}{\mathfrak{S}}
\newcommand{\fE}{\mathfrak{E}}
\newcommand{\fF}{\mathfrak{F}}

\newcommand{\Risk}{\mathcal{R}}

\renewcommand{\L}{\mathcal{L}}
\renewcommand{\H}{\mathcal{H}}

\newcommand{\cn}{\kappa}
\newcommand{\nn}{\nonumber}


\newcommand{\Hess}{\nabla^2}
\newcommand{\tlO}{\tilde{O}}
\newcommand{\tlOmega}{\tilde{\Omega}}

\newcommand{\calF}{\mathcal{F}}
\newcommand{\fhat}{\widehat{f}}
\newcommand{\calS}{\mathcal{S}}

\newcommand{\calX}{\mathcal{X}}
\newcommand{\calY}{\mathcal{Y}}
\newcommand{\calD}{\mathcal{D}}
\newcommand{\calZ}{\mathcal{Z}}
\newcommand{\calA}{\mathcal{A}}
\newcommand{\fbayes}{f^B}
\newcommand{\func}{f^U}


\newcommand{\bayscore}{\text{calibrated Bayes score}}
\newcommand{\bayrisk}{\text{calibrated Bayes risk}}

\newtheorem{example}{Example}[section]
\newtheorem{exc}{Exercise}[section]
%\newtheorem{rem}{Remark}[section]

\newtheorem{theorem}{Theorem}[section]
\newtheorem{definition}{Definition}
\newtheorem{proposition}[theorem]{Proposition}
\newtheorem{corollary}[theorem]{Corollary}

\newtheorem{remark}{Remark}[section]
\newtheorem{lemma}[theorem]{Lemma}
\newtheorem{claim}[theorem]{Claim}
\newtheorem{fact}[theorem]{Fact}
\newtheorem{assumption}{Assumption}

\newcommand{\iidsim}{\overset{\mathrm{i.i.d.}}{\sim}}
\newcommand{\unifsim}{\overset{\mathrm{unif}}{\sim}}
\newcommand{\sign}{\mathrm{sign}}
\newcommand{\wbar}{\overline{w}}
\newcommand{\what}{\widehat{w}}
\newcommand{\KL}{\mathrm{KL}}
\newcommand{\Bern}{\mathrm{Bernoulli}}
\newcommand{\ihat}{\widehat{i}}
\newcommand{\Dwst}{\calD^{w_*}}
\newcommand{\fls}{\widehat{f}_{n}}


\newcommand{\brpi}{\pi^{br}}
\newcommand{\brtheta}{\theta^{br}}

% \newcommand{\M}{\mat{M}}
% \newcommand\Mmh{\mat{M}^{-1/2}}
% \newcommand{\A}{\mat{A}}
% \newcommand{\B}{\mat{B}}
% \newcommand{\C}{\mat{C}}
% \newcommand{\Et}[1][t]{\mat{E_{#1}}}
% \newcommand{\Etp}{\Et[t+1]}
% \newcommand{\Errt}[1][t]{\mat{\bigtriangleup_{#1}}}
% \newcommand\cnM{\kappa}
% \newcommand{\cn}[1]{\kappa\left(#1\right)}
% \newcommand\X{\mat{X}}
% \newcommand\fstar{f_*}
% \newcommand\Xt[1][t]{\mat{X_{#1}}}
% \newcommand\ut[1][t]{{u_{#1}}}
% \newcommand\Xtinv{\inv{\Xt}}
% \newcommand\Xtp{\mat{X_{t+1}}}
% \newcommand\Xtpinv{\inv{\left(\mat{X_{t+1}}\right)}}
% \newcommand\U{\mat{U}}
% \newcommand\UTr{\trans{\mat{U}}}
% \newcommand{\Ut}[1][t]{\mat{U_{#1}}}
% \newcommand{\Utinv}{\inv{\Ut}}
% \newcommand{\UtTr}[1][t]{\trans{\mat{U_{#1}}}}
% \newcommand\Utp{\mat{U_{t+1}}}
% \newcommand\UtpTr{\trans{\mat{U}_{t+1}}}
% \newcommand\Utptild{\mat{\widetilde{U}_{t+1}}}
% \newcommand\Us{\mat{U^*}}
% \newcommand\UsTr{\trans{\mat{U^*}}}
% \newcommand{\Sigs}{\mat{\Sigma}}
% \newcommand{\Sigsmh}{\Sigs^{-1/2}}
% \newcommand{\eye}{\mat{I}}
% \newcommand{\twonormbound}{\left(4+\DPhi{\M}{\Xt[0]}\right)\twonorm{\M}}
% \newcommand{\lamj}{\lambda_j}

% \renewcommand\u{\vect{u}}
% \newcommand\uTr{\trans{\vect{u}}}
% \renewcommand\v{\vect{v}}
% \newcommand\vTr{\trans{\vect{v}}}
% \newcommand\w{\vect{w}}
% \newcommand\wTr{\trans{\vect{w}}}
% \newcommand\wperp{\vect{w}_{\perp}}
% \newcommand\wperpTr{\trans{\vect{w}_{\perp}}}
% \newcommand\wj{\vect{w_j}}
% \newcommand\vj{\vect{v_j}}
% \newcommand\wjTr{\trans{\vect{w_j}}}
% \newcommand\vjTr{\trans{\vect{v_j}}}

% \newcommand{\DPhi}[2]{\ensuremath{D_{\Phi}\left(#1,#2\right)}}
% \newcommand\matmult{{\omega}}

\pagenumbering{gobble}
\graphicspath{{./Figures/}}
\title{
Connectivity of LEO Satellite Mega Constellations: An Application of Percolation Theory on a Sphere}
\author{
 Hao Lin,  Mustafa A. Kishk and Mohamed-Slim Alouini
\thanks{Hao Lin is with the Electrical and Computer Engineering Program, CEMSE Division, King Abdullah University of Science and Technology (KAUST),
Thuwal 23955-6900, Saudi Arabia (e-mail: hao.lin.std@gmail.com).\\
\indent Mustafa A. Kishk is with the Department of Electronic Engineering,
Maynooth University, Maynooth, W23 F2H6 Ireland (e-mail:
mustafa.kishk@mu.ie).\\
\indent Mohamed-Slim Alouini is with the CEMSE Division, King Abdullah
University of Science and Technology (KAUST), Thuwal 23955-6900,
Saudi Arabia (e-mail: slim.alouini@kaust.edu.sa).}
}

\maketitle
\vspace{-2cm}
%\thispagestyle{empty}
%\pagestyle{empty}
\begin{abstract}
With the advent of the 6G era, global connectivity has become a common goal in the evolution of communications, aiming to bring Internet services to more unconnected regions. Additionally, the rise of applications such as the Internet of Everything and remote education also requires global connectivity. Non-terrestrial networks (NTN), particularly low earth orbit (LEO) satellites, play a crucial role in this future vision. Although some literature already analyze the coverage performance using stochastic geometry, the ability of generating large-scale continuous service area is still expected to analyze.  Therefore, in this paper, we mainly investigate the necessary conditions of LEO satellite deployment for large-scale continuous service coverage on the earth. Firstly, we apply percolation theory to a closed spherical surface and define the percolation on a sphere for the first time. We introduce the sub-critical and super-critical cases to prove the existence of the phase transition of percolation probability. Then, through stereographic projection, we introduce the tight bounds and closed-form expression of the critical number of LEO satellites on the same constellation. In addition, we also investigate how the altitude and maximum slant range of LEO satellites affect percolation probability, and derive the critical values of them. Based on our findings, we provide useful recommendations for companies planning to deploy LEO satellite networks to enhance connectivity.
\end{abstract}
% \vspace{-0.3cm}
\begin{IEEEkeywords}
% \vspace{-0.3cm}
Percolation theory, LEO satellites, non-terrestrial networks, stereographic projection.
\end{IEEEkeywords}

\section{Introduction} \label{sec:Intro}

As a key technology of next-generation communications, non-terrestrial networks (NTN) have been proposed to enhance high-capacity global connectivity \cite{giordani2020non}. 3GPP Release 17 defined the New Radio (NR) to support NTN broadband Internet services, especially for rural and remote areas \cite{3GPPRelease17}. Narrowband Internet of Things (NB-IoT) over NTN has been preliminarily standardised and commercial deployments are ongoing, where low earth orbit (LEO) satellite constellations can be a solution to provide low latency service in a low cost \cite{3GPPRelease18,liberg2020narrowband}. Furthermore, in 3GPP Release 18 and 19, NTNs including LEO satellites aim to support regenerative payloads, and coverage and mobility enhancements, for more requirements from handheld terminals, NB-IoT and enhanced machine-type communication (eMTC) \cite{10500741}. They can help not only support 5G NR but also pave the way towards 6G technologies. LEO-satellite access networks have been deployed to provide seamless massive access and connect the unconnected areas on the earth \cite{xiao2022leo}. Examples of currently deployed or planned LEO satellite constellations include Starlink, OneWeb and Kuiper \cite{osoro2021techno,voicu2024handover}. Therefore, it is necessary to capture the ability of LEO satellites to enhance global connectivity.  \\
\indent Stochastic geometry is an important tool to evaluate the performance of large-scale wireless networks without losing accuracy \cite{haenggi2012stochastic}. It is widely used to evaluate the coverage performance of 2D or 3D wireless networks. For global coverage, it can also provide a basic geometric framework, especially using the binomial point process (BPP) and Poisson point process (PPP). Global connectivity is a vital performance indicator of large-scale wireless networks, where graph theory and percolation theory can help capture such performance metric \cite{haenggi2009stochastic,haenggi2012stochastic}. Percolation probability represents the probability of generating large-scale connected components in a network, where the nodes can be connected through multi-hops. Such an indicator can be used to capture a network's connectivity, robustness, cyber-security and path exposure \cite{elsawy2023tutorial}.\\
\indent Therefore, it is necessary to evaluate the connectivity of LEO satellites' coverage, that is, the ability to generate large-scale continuous paths on the earth that are covered by LEO satellites, through percolation theory. However, conventional percolation on wireless networks relies on a 2D plane or a 3D system, which is different from the realistic deployment of user equipment (UEs) or IoT devices on the earth and the coverage regions of LEO satellites. So that, the percolation analysis on the 3D sphere is still an unsolved problem.


\subsection{Related Work}
 In this paper, we apply percolation theory to a sphere for the first time to discuss the ability to generate large-scale continuous coverage paths of LEO satellites. Therefore, we divide the related works into: i) LEO satellite communications based on stochastic geometry and ii) percolation theory applications on wireless networks.\\%i) LEO satellite constellations for next-generation mobile communications and ii) percolation theory applications on wireless networks.\\
% \indent \textit{LEO satellite constellations for next-generation mobile communications}: In \cite{liu2023stateless}, authors proposed a distributed lightweight stateless satellite core network architecture, to avoid frequent transmission and service interruption. Authors in \cite{salem2023exploiting} developed a tractable framework to evaluate the performance of downlink hybrid terrestrial/satellite networks in rural areas. To accommodate time-sensitive service required on industrial IoT devices, a network-layer-based latency scheduling architecture was proposed in \cite{wang2023time} \textcolor{red}{(2023)}. Authors in \cite{wang2024ultra} established an optimization algorithm to maximize reliability and minimize latency, and obtained the ideal upper bound for these performances \textcolor{red}{(2024)}. For a heterogeneous satellite network, authors in \cite{choi2024modeling} proved that the open access scenario can obtain a higher coverage probability than the closed access \textcolor{red}{(2024)}. Authors in \cite{sun2024distributed} proposed a distributed low-complexity satellite-terrestrial cooperative routing approach to overcome the long end-to-end delay caused by multi-hop routing and routing table construction \textcolor{red}{(2024)}. To break the barriers between computing and networking, a concept named integrated computing and networking for LEO satellite mega-constellations (ICN-LSMC) was proposed in \cite{huang2024integrated} \textcolor{red}{(2024)}. Authors in \cite{ji2024dynamic} designed a dynamic satellite-ground integrated mobility management strategy (DSG-MMS) to minimize handover and migration delays \textcolor{red}{(2024)}. Authors in \cite{hu2024performance} proposed a theoretical framework for an LEO satellite-aided shore-to-ship communication network (LEO-SSCN) and derived analytical expressions of end-to-end transmission success probability and average transmission rate capacity \textcolor{red}{(2024)}. \\
\indent \textit{LEO satellite communications based on stochastic geometry}: In recent years, LEO satellite communication has been a focus of next-generation communication technology. Stochastic geometry is widely used to evaluate the performance of communication systems with LEO satellites. In \cite{talgat2020stochastic}, authors studied the coverage performance of LEO satellite communication system, where satellite gateways on the ground act as relays between users and satellites. Authors in \cite{al2021analytic} and \cite{al2021optimal} evaluated the average downlink success probability for dense satellite networks and optimal satellite constellation altitude. Authors in \cite{al2022optimal} extended the work and investigated the optimal beamwidth and altitude for maximal uplink coverage of satellite networks. Authors in \cite{okati2020downlink} and \cite{okati2022nonhomogeneous} evaluated the average data rate and coverage probability using BPP and nonhomogeneous PPP, respectively. In \cite{okati2023stochastic}, they derived and verified the coverage probability of a multi-altitude LEO network. In \cite{park2022tractable}, authors derived the tight lower bound of coverage probability and found out the relationship between optimal average number of satellites and the altitude of satellites. A tractable framework was developed in \cite{salem2023exploiting} to evaluate the performance of downlink hybrid terrestrial/satellite networks in rural areas. Authors in \cite{shang2023coverage} derived the joint distance distribution of cooperative LEO satellites to the typical user, and obtained the optimal satellite density and satellite altitude to maximize the coverage probability. For Space-air-ground integrated networks (SAGIN), authors in \cite{yuan2023joint} proposed a simulated annealing algorithm-based optimization algorithm to optimize THz and RF channel allocation. Authors in \cite{wang2022ultra} studied the influence of gateway density and the setting of satellites constellations on latency and coverage probability. They established an optimization algorithm to maximize reliability and minimize latency, and obtained the ideal upper bound for these performances in \cite{wang2024ultra}. For a heterogeneous satellite network, authors in \cite{choi2024modeling} proved that the open access scenario can obtain a higher coverage probability than the closed access. Authors in \cite{choi2024analysis} investigated the key performance indices of a delay-tolerant data harvesting architecture, including the CDF of delay and harvest capacity. For different communication scenarios, authors in \cite{talgat2024stochastic} analyzed the uplink performance of IoT over LEO satellite communication with reliable coverage. An adaptive coverage enhancement (ACE) method was proposed in \cite{hong2024narrowband} for random access parameter configurations under diverse applications. Authors in \cite{bliss2024orchestrating} investigated the impact of onboard energy limitation, minimum elevation angle on downlink steady-state probability and availability. Authors in \cite{taojoint} proposed a throughput optimization algorithm for LEO satellite-based IoT networks and derived the closed-form expression of the throughput when LEO satellites are equipped with capture effect (CE) receiver and successive interference cancellation (SIC) receiver, respectively.

\indent \textit{Percolation theory applications on wireless networks}: Percolation theory and graph theory are widely used to evaluate the connectivity of large-scale networks, including multi-hops links, detective paths, continuous coverage, security, to name a few \cite{haenggi2009stochastic,haenggi2012stochastic,elsawy2023tutorial}. Authors in \cite{anjum2019percolation} modeled the homogeneous wireless balloon network (WBN) as a Gilbert disk model (GDM) and modeled the heterogeneous WBN as a Random Gilbert disk model (RGDM).
They derived the bounds of the critical node density of such WBNs. In \cite{anjum2020coverage}, they also derived the critical density of unmanned aerial vehicles (UAVs) to ensure the network coverage of UAV networks (UN). Using percolation theory, authors in \cite{wang2019cooperative} derived the critical density of camera sensors in clustered 3D wireless camera sensor networks (WCSN). Authors in \cite{zhaikhan2020safeguarding} characterized the critical density of spatial firewalls to prevent malware epidemics in large-scale wireless networks (LSWN). In \cite{yemini2019simultaneous}, authors established a model for the coexistence of random primary and secondary cognitive networks and proved the feasibility of the simultaneous connectivity. Based on dynamic bound percolation, authors in \cite{han2024dynamic} characterized the reliable topology evolution and proved that the dynamic topology evolution (DTE) model can improve the overall network performance.
In \cite{wu2023connectivity}, authors investigated the connectivity of large-scale reconfigurable intelligent surface (RIS) assisted integrated access and backhaul (IAB) networks.
Considering the directional antenna, authors in \cite{zhu2023connectivity} analyzed the connectivity of networks assisted by transmissive RIS.

\subsection{Contributions}
The contributions of this paper can be summarized as follows:\\
\indent \textit{A new perspective of connectivity of LEO satellite coverage.} In this paper, we evaluate the connectivity of LEO satellites' coverage using percolation theory. We adopt the percolation probability to show the ability to form giant continuous LEO satellite service areas or footprints for users on the earth. Such performance metric describes whether the devices that are moving on the earth can achieve continuous service and whether the Internet of Everything on a large scale can be supported by LEO satellites. \\
\indent \textit{Percolation theory on the Sphere.} In this paper, we first define the face percolation on the sphere using its stereographic projection onto a plane. We introduce the sub-critical and super-critical cases to investigate the lower and upper bounds of critical number of LEO satellites for percolation. Through stereographic projection, we derive the tight bounds for the critical threshold of the number of LEO satellites. By discussing the relationship between continuous percolation and discrete percolation, we obtain the closed-form expression of the critical number of LEO satellites. In addition, we also derive the closed-form critical expressions of altitude and maximum slant range of LEO satellites.\\
% \indent \textit{Suggestions for LEO satellite operators.} Through the percolation probability and coverage probability, we give different suggestions for different LEO satellite operators of different sizes. 

\section{System Model} \label{sec:SysMod}
% \begin{figure*}[htbp]
%     \centering
% \includegraphics[width=0.75\linewidth]{Figures/}
%     \caption{...}
%     \label{fig:...}
% \end{figure*}

For ease of tractability, the earth's surface is commonly considered a standard sphere with a radius $r_e=6371\, \rm km$, where the center of the earth is defined as $\oe$. The North Pole is located at $\textbf{w}_N$ and the South Pole is located at $\textbf{w}_S$. We assume that LEO satellites are uniformly distributed on a sphere at an altitude of $h$ from the ground, that is, the sphere centered at $\oe$ with a radius $r_s=r_e+h$. The locations of LEO satellites follow a BPP $\Phi=\{\yi\}$ with the number of satellites $N$, where $\yi$ represents the location of any satellite  \cite{talgat2020stochastic,wang2023coverage,talgat2024stochastic}. Notice that current LEO satellite constellations adopt different orbital design schemes, where Antarctic and Arctic region have different satellite densities from other regions. However, such a BPP assumption describes the future LEO satellite mega constellations with a massive number of satellites around the earth. Through antenna array and beamforming technique, each satellite can serve the users within the transmission angle $\eta$, \ie the nadir angle. The LEO satellite located at $\yi$ can provide the communication service for its spherical coverage area $\mathcal{A}_i=\mathcal{A}(\yi,\eta)$, which is defined as:
\begin{equation}
    \mathcal{A}(\yi,\eta)=\{\wik:\|\wik-\oe\|=r_e,\,\angle \oe\yi\wik\leq\eta\}.
\end{equation}
We can also define the center of the coverage area as $\xi$ so that such a coverage area can be defined using another symbol $\mathcal{O}_i=\mathcal{O}(\xi,\gamma)$, where:
\begin{equation}
    \mathcal{O}(\xi,\gamma)=\{\wik:\|\wik-\oe\|=r_e,\,\angle \xi\oe\wik\leq\gamma\},
\end{equation}
where $\gamma$ is the coverage angle of LEO satellites on the earth. Notice that $\mathcal{O}_i$ and $\mathcal{A}_i$ both represent the coverage area of the same satellite, they must satisfy
\begin{equation}
    \mathcal{O}(\xi,\gamma)=\mathcal{A}(\yi,\eta).
\end{equation}
Considering a user located at the boundary of LEO satellite's coverage, the distance to the satellite is called the maximum slant range $d_m$. The geometric relationships between $\gamma$, $\eta$, $h$ and $d_m$ are shown in the Fig. \ref{fig:etagamma} and Lemma \ref{lem:etagamma} in the following section.\\
\indent In this paper, we aim to analyze the connectivity of coverage areas of LEO satellites. Percolation theory, has its unique advantage of analyzing the connectivity, especially in a 2D plane. However, the definition of percolation on a sphere is less common in literature. Therefore, based on graph theory, we first define the connectivity of satellites' coverage areas as a 3D random graph $G_x(V_x,E_x)$, where $V_x=\{\xi\}$ is the set of locations of coverage centers and $E_x$ is edge set that shows whether the coverage areas of the considered satellites are connected. The edge set $E_x$ can be expressed as:
\begin{equation}
    E_x=\left\{\overline{\xi\xj}:    
    \left\{\begin{matrix}
 \angle \xi\oe\xj\leq 2\gamma \,\\
{\rm or}\\
\wideparen{\xi\xj}\leq 2r_e \gamma \, \\
{\rm or}\\
\|\xi\xj\|\leq 2r_e\sin\gamma

\end{matrix}\right\}\right\},
\end{equation}
where $\wideparen{\xi\xj}$ and $\|\xi\xj\|$ represent the spherical distance and Euclidean distance, respectively, between $\xi$ and $\xj$.\\
\indent In this paper, we propose to project the earth's surface onto a 2D plane which is tangent to the earth on the South Pole $\textbf{w}_S$. The random graph on the projection plane, which corresponds to $G_x(V_x,E_x)$, is defined as $G_z(V_z,E_z)$, where $V_z$ is the vertex set and $E_z$ is the edge set. We let $K_z\subset G_z(V_z,E_z)$ denote a connected component inside $G_z$ and let $K_z(0)$ denote the connected component covering the origin $\textbf{o}_z$ on the projection plane. On a 2D plane, percolation probability is commonly defined as the probability of generating a giant component whose set cardinality is infinite, \ie $\mathbb{P}\{|K_z|=\infty\}$. In our system, we define the percolation probability of LEO satellite coverage areas as a function of the number of satellites $N$ and the coverage angle $\gamma$, that is
\begin{equation}
    \theta(N,\gamma)=\mathbb{P}\{|K_z(0)|=\infty\},
\label{defineper}
\end{equation}
where the 2D connected component $K_z$ on a plane is projected from the 3D connected component $K_x$. Therefore, considering a fixed coverage angle $\gamma$ of each LEO satellite, the design objective of the considered system is:
\begin{equation}
    \begin{array}{ll}
       \text{ minimize}  & N  \\
       \text{ subject to}  & \theta(N,\gamma)>0.  \\
    \end{array}
    \label{design1}
\end{equation}
Similarly, considering a fixed number of LEO satellites deployed on the same altitude, we can also formulate the design objective as:
\begin{equation}
    \begin{array}{ll}
       \text{ minimize}  & \gamma  \\
       \text{ subject to}  & \theta(N,\gamma)>0. \\
    \end{array}
    \label{design2}
\end{equation}
It is worth noting that, the coverage angle $\gamma$ depends on the altitude $h$, maximum slant range $d_m$ or nadir angle $\eta$.\\
\indent In conclusion, using the tools of percolation theory, we mainly study the necessary conditions to form large-scale connected coverage areas on the earth. For ease of reading, we summarize most of the symbols in Table \ref{tab:TableOfNotations}.

\begin{table*}[htbp]
\caption{Table of Notations}
\centering
\begin{center}
\resizebox{\textwidth}{!}{
\renewcommand{\arraystretch}{1}%1.4
    \begin{tabular}{ {c} | {l} }
    \hline
        \hline
    \textbf{Notation} & \textbf{Description} \\ \hline
    $\textbf{o}_e$; $r_e$; $\textbf{w}_N$; $\textbf{w}_S$ & The center of earth; the radius of earth; the North Pole of earth; the South Pole of earth \\ \hline
    $h$; $r_s$; $N$ & The altitude of LEO satellites; the radius of LEO satellites' orbit; the number of LEO satellites\\ \hline
    $\Phi$; $\textbf{y}_i$; $\textbf{x}_i$ & The set of LEO satellites' locations; the location of the $i$th LEO satellite; the projection of the $i$th LEO satellites on the earth\\ \hline
    $\eta$; $\gamma$; $\epsilon$ & The nadir angle of LEO satellites; the coverage angle on the earth; the minimum elevation angle of users\\ \hline
    $p_{\rm cov}$; $p_{\rm ncov}$& The probability of each point on the earth being covered; the probability of each point on the earth being not covered \\ \hline
     $V_x$; $E_x$& The set of vertices (\ie coverage centers); the edge set which represents the connection between coverage areas\\ \hline
    $G_x(V_x,E_x)=\{V_x,E_x\}$& The random graph containing the vertex set $V_x$ and edge set $E_x$ on the earth (sphere)\\ \hline
    $G_z(V_z,E_z)=\{V_z,E_z\}$& The corresponding random graph of $G_x$ on the projection plane\\ \hline
    $K_x$; $K_x(0)$& The connected component on the sphere; the connected component on the sphere containing the South Pole $w_S$\\ \hline
    $K_z$; $K_z(0)$& The projected connected component on the considered projection plane; the $K_z$ containing the origin on the plane $\textbf{o}_z$ \\ \hline
    $\mathcal{F}$; $\mathcal{F}^{-1}$& The stereographic projection; the inverse stereographic projection\\ \hline
    $\theta(N,\gamma)$ & The percolation probability related to $N$ and $\gamma$\\ \hline
     \hline
    \end{tabular}
}
\end{center}
\label{tab:TableOfNotations}
%\vspace{-8mm}
\end{table*}

\section{Coverage Analysis} \label{sec:coverage}
\indent In this section, we first investigate the coverage analysis of LEO satellites. Then, we introduce how to project LEO satellite coverage areas on the sphere onto a tangent plane. Based on the stereographic projection, we define the percolation on the sphere and introduce the sub-critical and super-critical cases.

\subsection{Coverage Analysis of LEO Satellites}
\indent In Sec. \ref{sec:SysMod}, we introduce two different expressions of satellite coverage areas $\mathcal{A}_i=\mathcal{A}(\yi,\eta)$ and $\mathcal{O}_i=\mathcal{O}(\xi,\gamma)$. Because they represent the coverage area of the same LEO satellite, $\mathcal{O}(\xi,\gamma)=\mathcal{A}(\yi,\eta)$ must be satisfied. As shown in Fig.\ref{fig:etagamma}, we can obtain the relationships between $\gamma$, $\eta$, $h$ and $d_m$ as described in the below lemma.
\begin{figure}
    \centering
    \includegraphics[width=0.6\linewidth]{Figures/etagamma.pdf}
    \caption{The geometric relationships between coverage angle $\gamma$, nadir angle $\eta$, satellite constellation altitude $h$ and maximum slant range $d_m$. }
    \label{fig:etagamma}
\end{figure}


\begin{lemma}\label{lem:etagamma}
   The relationships between the coverage angle $\gamma$, nadir angle $\eta$, constellation altitude $h$ and maximum slant range $d_m$ can be expressed as:
   \begin{equation}
       \gamma=\arcsin(\frac{d_m}{r_e}\sin\eta),
   \end{equation}
where
\begin{equation}
    d_m=-\sqrt{r_e^2-r_s^2\sin^2\eta}+r_s \cos\eta
\end{equation}
and
\begin{equation}
    0<\eta\leq \arcsin{\frac{r_e}{r_s}},\, r_s=r_e+h.
\end{equation}
\end{lemma}
\begin{IEEEproof}
    See Appendix~\ref{app:etagamma}.
\end{IEEEproof}
 Next, we introduce the coverage analysis of each point on the sphere in Theorem \ref{theo:covpro}.
\begin{theorem}\label{theo:covpro} Assume that the number of LEO satellites is $N$ and the coverage angle of each satellite is $
\gamma$. The probability of each point on the sphere being covered by at least one LEO satellites is:
    \begin{equation}
        p_{\rm cov}(N,\gamma)=1-\bigg(\frac{1+\cos\gamma}{2}\bigg)^{N}.
    \end{equation}
    \label{theo:pcov}
Correspondingly, the probability of each point on the sphere being not covered by any LEO satellite is:
    \begin{equation}
        p_{\rm ncov}(N,\gamma)=\bigg(\frac{1+\cos\gamma}{2}\bigg)^{N}.
    \end{equation}
    
\end{theorem}
\begin{IEEEproof}
    See Appendix~\ref{app:pcov}.
\end{IEEEproof}
It is worth noting that, for any point on the sphere, the sum of probabilities of being covered and being not covered equals 1, \ie $p_{\rm cov}+p_{\rm ncov}=1$. However, for any area $\mathcal{B}$ on the plane, we focus on the probability of it being completely covered or completely not covered to analyze the sub-critical case and super-critical case. Therefore, if we mention an event where $\mathcal{B}$ is `covered' or `not covered', they represent $\mathcal{B}$ is `completely covered' or `completely not covered'. These two probabilities must satisfy:  
\begin{equation}
    \P\{\mathcal{B} {\rm \; is\;covered}\}+\P\{\mathcal{B} {\rm \; is\;not\;covered}\}\leq 1
\end{equation}
because $\P\{\mathcal{B} {\rm \; is\;partially\;covered}\}\geq 0$.

\subsection{Percolation through Stereographic Projection}
\indent To analyze the percolation on the sphere, we need to separate the whole sphere using some special lattice. Classical percolation analysis is always based on triangular, square or hexagonal lattices. Unlike a plane, the sphere can not be divided using a homogeneous lattice. Mercator projection in Fig.\ref{fig:caparea} can map the sphere on a square area, which is commonly used in geography \cite{snyder1987map}, however, the spherical coverage areas of LEO satellites become irregular and not tractable. Therefore, we propose to use the stereographic projection to analyze the percolation on the sphere \cite{snyder1987map}, which is a specific example of Alexandroff extension mapping a sphere onto a plane \cite{willard2012general}. The stereographic projection is introduced in Theorem \ref{theo:stereo}.

\begin{theorem}
\label{theo:stereo}
    As shown in Fig.\ref{fig:stereographic}, $P(x,y,z)$ is a point on the sphere and $P'(x',y',z')$ is its stereographic projection on the projection plane. The stereographic projection $P'=\mathcal{F}(P)$ leads to:
\begin{equation}
\begin{array}{r@{}l}
(x',y',z')
=\displaystyle\bigg(\frac{2r_e}{2r_e-z}x,\frac{2r_e}{2r_e-z}y,0\bigg),
\end{array}
\end{equation}
and the inverse stereographic projection $P=\mathcal{F}^{-1}(P')$ leads to:
\begin{equation}
\begin{array}{l}
    (x,y,z)\\=\displaystyle\bigg(\frac{4r_e^2 x'}{4r_e^2+x'^2+y'^2},\frac{4r_e^2 x'}{4r_e^2+x'^2+y'^2},\frac{2r_e(x'^2+y'^2)}{4r_e^2+x'^2+y'^2}\bigg).
\end{array}
\end{equation}
\end{theorem}
\begin{figure}[ht]
    \centering
    \includegraphics[width=0.75\linewidth]{Figures/caparea.pdf}
    \caption{The area of spherical cap and the Mercator projection. }
    \label{fig:caparea}
\end{figure}
\begin{figure}
    \centering
    \includegraphics[width=0.65\linewidth]{Figures/stereographic.pdf}
    \caption{The stereographic projection. The projection plane is on the $xo_zy$ plane, which is tangent to the earth on the South Pole $\textbf{w}_S(\textbf{o}_z)$. The earth's center $\textbf{o}_e$ and North Pole $\textbf{w}_N$ are both on the z-axis. $P'$ is the stereographic projection of $P$. Any circle on the sphere corresponds to a circle on the projection plane. If the spherical cap excludes the North Pole $\textbf{w}_N$, the spherical cap is projected to a finite circular area. Inversely, any finite circular area corresponds to a spherical cap excluding $\textbf{w}_N$.}
    \label{fig:stereographic}
\end{figure}
\begin{IEEEproof}
    Notice that the South Pole of the sphere is the origin of the projection plane, \ie $\textbf{w}_S=\textbf{o}_{z}$. The coordinate relationship in the stereographic projection and inverse stereographic projection can be easily proved using $\overrightarrow{w_{N}P}=\frac{2r_e-z}{2r_e}\overrightarrow{w_{N}P'}$.  
\end{IEEEproof}
Except for the North Pole $w_N$, the stereographic projection is a bijection between a sphere and a plane. Therefore, we introduce a property of stereographic projection in Lemma \ref{lem:subset}.

\begin{lemma}\label{lem:subset}
    Define the mapping from the sphere to the plane through stereographic projection as a function $\mathcal{F}$. For two spherical areas on the sphere $\mathcal{B}$ and $\mathcal{C}$, their projections on the plane satisfy:
    \begin{equation}
        \mathcal{B}\subseteq \mathcal{C} \Leftrightarrow \mathcal{F}(\mathcal{B})\subseteq \mathcal{F}(\mathcal{C}).
        \label{mappingrelation}
    \end{equation}
    \label{lem:mappingrelation}
\end{lemma}

\begin{IEEEproof}
    As a kind of bijection, stereographic projection, except for the North Pole  $\textbf{w}_N$, has the same property as bijection.
\end{IEEEproof}
\begin{remark}
    Notice that a closed shape on the sphere is not always projected to a closed shape on the projection plane. Once the North Pole is included in the $\mathcal{B}$, the size of projection $\mathcal{F}(\mathcal{B})$ goes to infinite. However, the property (\ref{mappingrelation}) still holds.
\end{remark}

% Based on Lemma \ref{lem:subset}, we can show the below corollary:
% \begin{corollary}
%     If a spherical area $\mathcal{B}$ is covered by a LEO satellite's coverage area $\mathcal{A}_i$, \ie $\mathcal{B}\subseteq\mathcal{A}_i$, their projections on the plane also satisfy $\mathcal{F}(\mathcal{B})\subseteq\mathcal{F}(\mathcal{A}_i)$, vice versa, that is:
%     \begin{equation}
%         \mathcal{B}\subseteq \mathcal{A}_i \Leftrightarrow \mathcal{F}(\mathcal{B})\subseteq \mathcal{F}(\mathcal{A}_i).
%         \label{corsubset2}
%     \end{equation}
% \label{cor:mapspherical}
% \end{corollary}
\begin{figure}
    \centering
    \includegraphics[width=0.8\linewidth]{Figures/Hexagons.pdf}
    \caption{Hexagonal lattice on the projected plane. The side length of hexagons is $a$ and $\mathcal{H}_0$ is the hexagon which is centered at the origin.}
    \label{fig:hexagon}
\end{figure}
\indent To make the percolation on the sphere a tractable problem, we propose to discuss the percolation on the stereographic projection plane. As shown in Fig.\ref{fig:hexagon}, we define the homogeneous hexagons on the plane as $\mathcal{H}_l$'s with the side length $a$. Through inverse stereographic projection, we can also find the original area of $H_l$ on the sphere, \ie $\mathcal{F}^{-1}(\mathcal{H}_l)$. For percolation on the sphere, we focus on whether $\mathcal{F}^{-1}(\mathcal{H}_l)$ can be covered by the LEO satellites' coverage areas, \ie $\mathcal{A}_i$'s. Using the property (\ref{mappingrelation}), the problem is equivalent to whether the hexagon $\mathcal{H}_l$ on the plane can be covered by the projections of $\mathcal{A}_i$'s, \ie $\mathcal{F}(\mathcal{A}_i)$'s.
% \begin{corollary}
%     If a hexagon on the plane $\mathcal{H}_l$ is covered by the projection of a LEO satellite's coverage area $\mathcal{F}(\mathcal{A}_i)$, \ie $\mathcal{H}_l\subseteq \mathcal{F}(\mathcal{A}_i)$, their original areas on the sphere also satisfy $\mathcal{F}^{-1}(\mathcal{H}_l)\subseteq\mathcal{A}_i$, vice versa, that is:
%     \begin{equation}
%         \mathcal{H}_l\subseteq \mathcal{F}(\mathcal{A}_i) \Leftrightarrow \mathcal{F}^{-1}(\mathcal{H}_l)\subseteq\mathcal{A}_i.
%         \label{corsubset}
%     \end{equation}
% \label{cor:maphexagon}
% \end{corollary}
 On a plane, the face percolation of hexagons means there exist giant components whose cardinality is infinite. As shown in (\ref{defineper}), percolation probability is always defined as the probability of generating a giant component that crosses the origin. Through inverse stereographic projection, such a giant 
 component is projected from a continuous coverage area from the South Pole to the North Pole. This also represents the `farthest coverage on the earth'. Therefore, we propose to define the percolation probability on the sphere as below.
\begin{definition}
    On a sphere, percolation probability is defined as the probability of generating continuous coverage areas that contain two symmetry points about the sphere's center. Especially, we can also define it using the probability of connecting the North Pole and the South Pole of earth, \ie
\begin{equation}
    \theta(N,\gamma)=\P\{\textbf{w}_N\in K_x(\textbf{w}_S)\}.
\end{equation} 
where $K_x$ denotes the giant component on the sphere, $\textbf{w}_S$ and $\textbf{w}_N$ represent the South Pole and the North Pole, respectively. $K_x(\textbf{w}_S)$ represents the connected component starting from the South Pole.
\label{def:concomsphere}
\end{definition}

\begin{remark}
    On the earth, the spherical distance between any two points is less than or equal to $\pi r_e$, that is, the two points that are symmetric about the earth's center have the maximum spherical distance. Unlike the analysis on an infinite plane, the cardinality of the connected component will not reach infinity. The farthest spherical distance it can reach is determined, which can be used as a judgement of percolation. In addition, the cardinality of the connected component has its upper bound, that is, the entire sphere.
\end{remark}
\indent As a basic knowledge of hexagonal face percolation, the sufficient and necessary condition for face percolation of hexagons is that the probability of each hexagon being covered should be larger than $\frac{1}{2}$, \ie
\begin{equation}
    \theta(N,\gamma)>0,\,{\rm if}\,\P\{\mathcal{H}_l\,{\rm is\, covered}\}>\frac{1}{2}.
\end{equation}
\indent It is difficult to calculate $\P\{\mathcal{H}_l\,{\rm is\, covered}\}$ directly because the shape of $\mathcal{F}^{-1}(\mathcal{H}_l)$ is irregular. However, we can use some circular areas to help find the tight upper bound and lower bound of it. At the same time, the coverage areas of LEO satellites are typically modeled as circular areas. Therefore, we introduce the below lemma.
\begin{lemma}
    For a spherical cap on the sphere, its projection on the plane is a circular area, unless the border of the spherical cap crosses the top of the sphere.
    Inversely, for each circular area on the projected plane, its inverse projection on the sphere is a circular area. 
    \label{lem:captocircle}
\end{lemma}
\begin{IEEEproof}
    See the proof in \cite[88.1]{pedoe2013geometry}. %See Appendix~\ref{app:captocircle}.
\end{IEEEproof}
\begin{remark}
    For stereographic projection, the North Pole $\textbf{w}_N$ is considered the top of the earth. There exist three cases: \textbf{i) the spherical cap excludes the top}, where the projection is a closed circular area on the plane, \textbf{ii) the border of the spherical cap crosses the top}, where the projection is a region divided by a straight line that does not include the origin $\textbf{o}_z$, \textbf{iii) the spherical cap includes the top}, where the projection is open and its inner envelope is a circular area on the plane. On the other hand, for inverse stereographic projection, closed circular areas on the plane can be always projected to spherical caps on the sphere, which does not contain the North Pole $\textbf{w}_N$.
\end{remark}
% According to Lemma \ref{lem:captocircle}, we introduce the below corollary:
% \begin{corollary}
%     For a LEO satellite's coverage area $\mathcal{O}(\xi,\gamma)=\mathcal{A}(\yi,\eta)$, its projection on the plane is a circular area unless the angle $\angle \textbf{w}_{N}\textbf{o}_e\yi=\gamma$ where $\textbf{w}_{N}$ is the top of the sphere.
% \end{corollary}
% \begin{IEEEproof}
%     Let the coverage area $\mathcal{O}(\xi,\gamma)$ be the spherical cap in Lemma \ref{lem:captocircle}, its coverage is a circular area on the plane. When $\angle \textbf{w}_{N}\textbf{o}_e\yi=\gamma$, the considered plane crosses the top of the sphere, and the projection of the LEO satellite's coverage area is a straight line on the plane.
% \end{IEEEproof}
% \indent Therefore, almost all LEO satellites' coverage areas on the sphere are projected to circular areas on the plane. We can obtain the radius of these coverage areas. Next, we introduce Lemma \ref{lem:circularradius}:
\indent In this paper, we need to first analyze the property of circular areas on the projection plane. As introduced in Lemma \ref{lem:captocircle}, their inverse projections on the sphere can be always modeled as circular areas that does not contain the North Pole $\textbf{w}_N$. So that, we introduce the relationship between central angle of a spherical cap and the radius of its projected circular area.


\begin{lemma}\label{lem:circularradius}
     For a spherical cap that does not contain the North Pole $\textbf{w}_N$ with a central angle $\gamma_0$, the radius of its projected circular area on the projection plane can be expressed as:
\begin{equation}
    r(\psi)=r_e\bigg|\tan(\frac{\psi+\gamma_0}{2})-\tan(\frac{\psi-\gamma_0}{2})\bigg|        
\label{radiusprojection}
\end{equation}
where $\psi=\angle \textbf{w}_S \oe \textbf{x}<\pi-\gamma_0$ and $\textbf{x}$ is the center of the spherical cap. The range of $r(\psi)$ is:
\begin{equation}
    2r_e\tan(\frac{\gamma}{2})\leq r(\psi)<+\infty.
\end{equation}
Conversely, for a circular area on the projection plane with a radius $r$, the central angle of its original spherical cap is upper and lower bounded as follows:
\begin{equation}
    0<\gamma_0\leq 2\arctan(\frac{r}{2r_e}).
\end{equation}
\end{lemma}
\begin{IEEEproof}
    See Appendix~\ref{app:circularradius}.
\end{IEEEproof}
\indent Lemma \ref{lem:captocircle} and Lemma \ref{lem:circularradius} already exhibit how to project the spherical caps on the sphere to its tangent projection plane, which are used to do the critical analysis in the next section.
% Therefore, another corollary is introduced:
% \begin{corollary}
%     Define $\tilde{\mathcal{O}}(\textbf{z}_l,r_l)$ as a circular area centered at $\textbf{z}_l$ with radius $r_l$ on the considered plane, its original area is $\mathcal{F}^{-1}(\tilde{\mathcal{O}}(\textbf{z}_l,r_l))$ is a spherical cap on the sphere. When the center of $\tilde{\mathcal{O}}(\textbf{z}_l,r_l)$  is infinite far from the origin $\textbf{o}_z$, the area of $\mathcal{F}^{-1}(\tilde{\mathcal{O}}(\textbf{z}_l,r_l))$ and its center angle $\gamma_l$ are both 0. When the center of $\tilde{\mathcal{O}}(\textbf{z}_l,r_l)$ is the origin $\textbf{o}_z$, the area of $\mathcal{F}^{-1}(\tilde{\mathcal{O}}(\textbf{z}_l,r_l))$ and its center angle both achieve their maximum.
% \label{cor:areaofcircularplane}
% \end{corollary}
% \begin{IEEEproof}
%     Define the distance between $\textbf{z}_l$ and $\textbf{o}_z$ is $|\textbf{o}_z\textbf{z}_l|$. The same as (\ref{radiusprojection}), the radius of circular area on the plane is an increasing function of $\psi_l$ and $\gamma_l$ at the same time, \ie 
%     \begin{equation}
%     r_{l}=r_e\bigg|\tan(\frac{\psi_l+\gamma_l}{2})-\tan(\frac{\psi_l-\gamma_l}{2})\bigg|.
%     \end{equation}
%     If $r_l$ is fixed, the center angle of the original spherical cap $\gamma_l$ can be considered a decreasing function of $\psi_l$, where the increase in $\psi_l$ leads to the increase in $|\textbf{o}_z\textbf{z}_l|$ as well. Therefore, for a circular area on the considered plane with a fixed radius, the center angle $\gamma_l$ and the original area on the sphere achieve their maximum when $|\textbf{o}_z\textbf{z}_i|=0$. Similarly, $\gamma_0$ and the area of the original spherical cap both achieve their minimum 0 when $|\textbf{o}_z\textbf{z}_i|=\infty$. 
% \end{IEEEproof}


\section{Critical Analysis}\label{sec:critical}
In this section, we first prove that percolation probability is a non-decreasing function of the number of satellites. Next, We introduce the sub-critical and super-critical cases where the percolation probability is zero and non-zero, respectively. Based on these, we prove the existence of the critical number of LEO satellites to realize the phase transition of percolation probability on the sphere. After that, we discuss the critical case and derive a closed-form expression of the critical satellite number $N_c$.  
\subsection{Phase Transition}\label{subsec:phasetransition}
  
 \indent In order to prove the existence of phase transition of percolation probability and derive the critical number of satellites, we first introduce the relationship between the percolation probability $\theta$ and the number of satellites $N$ in the below lemma.
\begin{lemma}\label{lem:nondecreasing}
    When the LEO satellites are deployed at the same altitude following a BPP with a coverage angle $\gamma$, the percolation probability and the number of LEO satellites satisfy:
    \begin{equation}
        \theta(N_1,\gamma)\leq\theta(N_2,\gamma), \;{\rm for}\; 0<N_1<N_2,
    \end{equation}
when the value of $\gamma$ is fixed.
\end{lemma}
\begin{IEEEproof}
    See Appendix~\ref{app:nondecreasing}.
\end{IEEEproof}
\indent Therefore, the percolation probability $\theta(N,\gamma)$ does not decrease as $N$ increases. Next, we introduce a sub-critical case to obtain a lower bound $N_L$ of the critical number of LEO satellites, where percolation probability is zero when $N\leq N_L$.\\

\noindent \textbf{Sub-critical case:} We choose a certain meridian (e.g. the prime meridian). The LEO satellites are deployed along the longitude and the borders of two adjacent coverage areas are tangent. If the longest spherical distance inside the covered areas is less than $\pi r_e$, the probability of percolation is 0. Therefore, we first introduce the sufficient condition for zero percolation probability in the below lemma. 
\begin{lemma}\label{lem:lowerbound}
    When the number of LEO satellites is less than $N_L$, the percolation probability $\theta(N,\gamma)$ is zero, \ie
\begin{equation}
    \theta(N,\gamma)=0\;{\rm if}\;N\leq N_L.
\end{equation}    
    The expression of $N_L$ is
\begin{equation}
    N_L = \left \lfloor \frac{\pi}{2\gamma} \right \rfloor 
\label{lowerbound}
\end{equation}
where $\left \lfloor x \right \rfloor$ denotes the largest integer less than x, and $\gamma$ is the coverage angle of each LEO satellite.
\end{lemma}
\begin{IEEEproof}
    As shown in Fig.\ref{fig:lowerbound}, when $N\leq\left \lfloor \frac{\pi}{2\gamma} \right \rfloor$, even though the satellites are deployed on the same orbit, any continuous coverage path containing $\textbf{w}_N$ and $\textbf{w}_S$ can not be generated.
\end{IEEEproof}
\begin{figure}
    \centering
    \includegraphics[width=0.4\linewidth]{Figures/Lowerbound.pdf}
    \caption{Sub-critical case: All satellites are deployed on the same meridian, where neighbour coverage areas are tangent to each other. However, the longest spherical distance inside the coverage areas does not exceed $\pi r_e$ when the number of satellites is not large enough.}
    \label{fig:lowerbound}
\end{figure}

\begin{corollary}
    If the critical number of satellites $N_c$ for the phase transition of percolation probability exists, $N_L$ is the lower bound of $N_c$, \ie $N_c\geq N_L$.
    \label{cor:subcritical}
\end{corollary}

Next, in order to obtain an upper bound of the critical number of LEO satellites, where percolation probability is non-zero when $N\geq N_U$, we need to ensure that the percolation probability has a computable and non-zero lower bound when $N=N_U$. Therefore, we introduce a super-critical case as shown below.\\ 

\noindent \textbf{Super-critical case:} This super-critical case is designed in six steps: \textit{i)} Along the meridians, we divide the whole sphere into $2m$ `slices', where each slice spans $\frac{\pi}{m}$ of longitude. \textit{ii)} Each two slices symmetric about the earth's center can be contained by a `belt'. Therefore, $m$ of belts can cover the whole sphere. \textit{iii)} Rotate a belt and make it symmetric about the equatorial plane, it can be considered a belt spanning $\frac{\pi}{m}$ of latitude. \textit{iv)} By dividing such a belt into $n$ uniform `pieces', we can use in total $m\times n$ pieces to cover the whole sphere. Each piece spans $\frac{\pi}{m}$ of longitude and $\frac{2\pi}{n}$ of latitude. \textit{v)} Each piece can be contained by a spherical cap which is smaller than the coverage area of a satellite. \textit{vi)} Use the $m\times n$ of satellites to cover the target spherical caps one by one. The steps from i) to v) are shown in Fig.\ref{fig:Upperbound}, which explains how to represent the whole sphere using the union of spherical caps. 
\\
\indent In the super-critical case, we need to design $m$ and $n$ large enough to make such a full coverage deployment feasible and obtain a computable non-zero lower bound of percolation probability. Therefore, we introduce the sufficient condition for non-zero percolation probability in the below lemma.
\begin{lemma}\label{lem:upperbound1}
    When the number of LEO satellites is larger than $N_U$, the percolation probability $\theta(N,\gamma)$ is non-zero, \ie
\begin{equation}
    \theta(N,\gamma)>0\;{\rm if}\;N\geq N_U. 
\end{equation}    
    The expression of $N_U$ is 
\begin{equation}
    N_U = m\times n 
\label{upperbound}
\end{equation}
with
\begin{equation}
    m = \left \lceil \frac{\pi}{\gamma} \right \rceil,\;n= \left \lceil \frac{\pi}{\arccos{\frac{\cos\gamma}{\cos\frac{\pi}{2m}}}} \right \rceil +1
\label{upperboundmn}
\end{equation}
where $\left \lceil x \right \rceil$ denotes the smallest integer greater than x and $\gamma$ is the coverage angle of each LEO satellite.
\end{lemma}
\begin{IEEEproof}
    See Appendix~\ref{app:upperbound}. 
\end{IEEEproof}
\begin{corollary}
    If the critical number of satellites $N_c$ exists, $N_U$ is the upper bound of $N_c$, \ie $N_c\leq N_U$.
    \label{cor:supercritical}
\end{corollary}
\begin{figure}
    \centering
    \includegraphics[width=1\linewidth]{Figures/Upperbound.pdf}
    \caption{Super-critical case: a full coverage scheme. Above (from left to right): \textit{a)} The whole sphere is firstly divided into $2m$ slices. \textit{b)} Because each belt can contain two slices that are symmetric, the whole sphere can be considered as the union of $m$ belts. \textit{c)} Rotate the belt. Below (from left to right): \textit{d)} Each belt can be divided into $n$ pieces. \textit{e)} The whole sphere can be consider the union of $m\times n$ pieces. \textit{f)} Because each piece can be contained by a spherical cap, the whole sphere can be considered the union of $m\times n$ spherical caps.}
    \label{fig:Upperbound}
\end{figure}
% \begin{corollary}
%     The upper bound in Lemma \ref{lem:upperbound1} shows that: if the number of LEO satellites is greater than $N_U$, \ie $N>N_L$ the percolation probability $\theta(N,\gamma)>0$.
%     \label{cor:supercritical}
% \end{corollary}
% \begin{remark}
%     When the number of satellites tends to $\infty$, the coverage probability of each point on the sphere $p_{cov}$ is almost surely 1. In this case, the percolation probability is almost surely 1, \ie $\theta(\infty,\gamma)=1$, almost surely. This extreme case can be used to further prove that the existence of upper bound of the critical number of satellites.
%     \label{rem:upperboundinfty}
% \end{remark}

 Based on the sub-critical and super-critical cases, we can prove the existence of the critical value of $N$, \ie $N_c$, in the following lemma.

\begin{lemma}\label{lem:phasetransition}
    When the LEO satellites are deployed at the same altitude following a BPP with a fixed value of $\gamma$, there exists a critical value $N_c$ satisfying:
    \begin{equation}
    \begin{array}{c}
        \theta(N,\gamma)=0,\, {\rm for}\; N<  N_c,\\
        \theta(N,\gamma)>0,\, {\rm for}\; N> N_c. 
    \end{array}
    \end{equation}
where $N_L\leq \left\lfloor N_c\right\rfloor$ and $\left\lceil N_c\right\rceil\leq N_U$.
\end{lemma}
\begin{IEEEproof}
    See Appendix~\ref{app:phasetransition}.
\end{IEEEproof}
Therefore, we prove the existence of the critical value of $N$, \ie $N_c$, which exhibits the phase transition of percolation probability.
\subsection{Tight bounds and critical analysis}
 In Lemma \ref{lem:phasetransition}, we have proved that the critical number of LEO satellites for phase transition of percolation probability exists. In this part, we propose to use the stereographic projection to find a tight lower bound and a tight upper bound for $N_c$, and introduce the closed-form expression of $N_c$.\\
 
 \noindent\textbf{Hexagonal face percolation on the projection plane:} As shown in Definition \ref{def:concomsphere}, the percolation on the sphere containing the South Pole $\textbf{w}_S$ represents the percolation on the projection plane containing the origin $\textbf{o}_z$. We first consider the hexagons with side length $a$. In percolation theory, if all hexagonal faces have the same probabilities $\P\{\mathcal{H}_{l} {\rm \; is\;open}\}$ and $\P\{\mathcal{H}_{l} {\rm \; is\;closed}\}$, we have: \textit{i) $\theta=0$ when $\P\{\mathcal{H}_{l} {\rm \; is\;closed}\}>1/2$ }and\textit{ ii) $\theta>0$ when $\P\{\mathcal{H}_{l} {\rm \; is\;open}\}>1/2$}. To find the tight bounds, we first introduce the below theorem.

 \begin{figure}
    \centering
    \includegraphics[width=0.6\linewidth]{Figures/Gamma.pdf}
    \caption{The minimum circumscribed circle of the hexagon on the projected plane, which is centered at the origin of the projection plane. Its central angle of the corresponding original spherical cap is the maximum one, that is $\gamma_m$. }
    \label{fig:gamma}
\end{figure}

\begin{theorem}
    Consider the hexagonal lattice on the plane where the side length of each hexagon is $a$. If the coverage probabilities of different hexagons are different,
    the sufficient condition for non-zero face percolation probability is:
\begin{equation}
    \P\{\mathcal{H}_{l} {\rm \; is\;open}\}>1/2,
\label{sufficientcover}
\end{equation}
and the sufficient condition for zero face percolation probability is:
\begin{equation}
    \P\{\mathcal{H}_{l} {\rm \; is\;closed}\}>1/2.
\label{sufficientnotcover}
\end{equation}
\label{theo:inhomohexagon}
\end{theorem}
\begin{IEEEproof}
    See Appendix~\ref{app:inhomohexagon}.
\end{IEEEproof}
\indent Next, we introduce the lower bounds $\P\{\mathcal{H}_{l} {\rm \; is\;open}\}$ and $\P\{\mathcal{H}_{l} {\rm \; is\;closed}\}$ in the below lemma.

\begin{lemma}\label{lem:boundsforhexagons}
Let $\P\{\mathcal{H}_{l} {\rm \; is\;open}\}$ denote the probability of a hexagon $\mathcal{H}_{l}$ on the projection plane being covered by LEO satellites. 
The lower bound of $\P\{\mathcal{H}_{l} {\rm \; is\;open}\}$ is shown as:
\begin{equation}
\begin{array}{c}
    \P\{\mathcal{H}_{l} {\rm \; is\;open}\}\geq\displaystyle 1-\bigg(\frac{1+\cos(\gamma-\gamma_m)}{2}\bigg)^{N}, 
\end{array}
\label{mincoverageprob}
\end{equation}
where
\begin{equation}
    \gamma_m=2\arctan \frac{a}{2 r_e}.
\end{equation}
Let $\P\{\mathcal{H}_{l} {\rm \; is\;closed}\}$ denote the probability of the hexagon $\mathcal{H}_{l}$ on the projection plane being not covered by LEO satellites. The lower bound of $\P\{\mathcal{H}_{l} {\rm \; is\;closed}\}$ is shown as:
\begin{equation}
\begin{array}{c}
    \P\{\mathcal{H}_{l} {\rm \; is\;closed}\}\geq\displaystyle\bigg(\frac{1+\cos(\gamma+\gamma_m)}{2}\bigg)^{N}.
\end{array}
\label{minnotcoverageprob}
\end{equation}
% where
% \begin{equation}
%   \gamma_m=2\arctan \frac{a}{r_e}.
% \end{equation}
\label{lem:minprobs}
\end{lemma}
\begin{IEEEproof}
    See Appendix~\ref{app:boundsforhexagons}.
\end{IEEEproof}

Substitute (\ref{mincoverageprob}) and (\ref{minnotcoverageprob}) in Lemma \ref{lem:minprobs} into the sufficient conditions for non-zero or zero percolation probability (\ref{sufficientcover}) and (\ref{sufficientnotcover}) in Theorem \ref{theo:inhomohexagon}, we can obtain the sufficient conditions of the number of LEO satellites for non-zero or zero percolation probability that are shown in the below theorem.
\begin{theorem}
Given that the coverage angle of each satellite is $\gamma$, $r_e$ is the radius of the earth and $a$ is the side length of hexagons on the projection plane. The sufficient condition of the number of LEO satellites for non-zero percolation probability is:
\begin{equation}
    N>N_c^U
\end{equation}
where 
\begin{equation}
    N_c^U=\frac{\ln 2}{\ln 2-\ln(1+\cos(\gamma-2\arctan \frac{a}{2r_e}))}
\end{equation}

\noindent is the upper bound of critical number of LEO satellites for phase transition of percolation probability, \ie $N_c\leq N_c^U$.\\
\indent The sufficient condition of the number of LEO satellites for zero percolation probability is:
\begin{equation}
    N<N_c^L
\end{equation}
where 
\begin{equation}
    N_c^L=\frac{\ln 2}{\ln 2-\ln(1+\cos(\gamma+2\arctan \frac{a}{2r_e}))}
\end{equation}
is the lower bound of critical number of LEO satellites for phase transition of percolation probability, \ie $N_c\geq N_c^L$.
\label{theo:sufficientconditions}
\end{theorem}
\begin{IEEEproof}
The upper bound $N_c^{U}$ and lower bound $N_c^{L}$ are obtained by substituting (\ref{mincoverageprob}) and (\ref{minnotcoverageprob}) in Lemma \ref{lem:minprobs} into the sufficient conditions for non-zero or zero percolation probability (\ref{sufficientcover}) and (\ref{sufficientnotcover}) in Theorem \ref{theo:inhomohexagon}, respectively.
\end{IEEEproof}
\indent To analyze the continuous percolation on the sphere, we also consider the continuous percolation on the plane. Therefore, the side length of considered hexagons on the plane is assumed to approach 0. We obtain the explicit expression for the critical number of LEO satellites in the below lemma.
\begin{lemma}
The critical number of LEO satellites for phase transition of percolation probability is:
\begin{equation}
N_c=\frac{\ln 2}{\ln 2-\ln(1+\cos\gamma)}.
\label{criticalNc}
\end{equation}
\label{lem:criticalanalysis}
\end{lemma}
\begin{IEEEproof}
See Appendix~\ref{app:criticalanalysis}. 
\end{IEEEproof}
% \begin{remark}
% We obtain the lower bound $N_L$ and upper bound $N_U$ through coverage analysis on the sphere. We also obtain the tight bounds $N_c^L$ and $N_c^U$ through percolation on the considered projection plane, and further obtain the critical number of LEO satellites for phase transition of percolation probability, \ie $N_c$. It is necessary to verify the relationship between the $N_L$, $N_U$ and $N_c$, which is also shown in the proof of Lemma \ref{lem:criticalanalysis}.
% \end{remark}
$N_c$ is the only explicit value which is always located between bounds $N_c^{L}$ and $N_c^{U}$. At the same time, the upper and lower bounds $N_c^{L}$ and $N_c^{U}$ are both tighter than $N_L$ and $N_U$ when $a$ approaches 0. Theoretically, for $N\geq\left \lceil N_c \right \rceil$, $\theta(N,\gamma)>0$ and for $N\leq\left \lfloor N_c \right \rfloor$, $\theta(N,\gamma)=0$.\\
\indent It is worth noting that, the critical number $N_c$ is the same as the solution of $p_{\rm cov}(N,\gamma)=1/2$ or $p_{\rm ncov}(N,\gamma)=1/2$. When $N>N_c$, $ p_{\rm cov}(N,\gamma)>1/2$. When $N<N_c$, $p_{\rm ncov}(N,\gamma)>1/2$ and $p_{\rm cov}(N,\gamma)<1/2$. Therefore, we obtain the critical condition for phase transition of percolation probability in the below theorem.
\begin{theorem}
    Assume that all points on the sphere has the same probability of being covered, that is $p_{\rm cov}$. 
    The phase transition of continuous percolation on the sphere is expressed as:
\begin{equation}
\begin{array}{r@{}l}
    \theta(p_{\rm cov})=0,\;& for\;p_{\rm cov}<\frac{1}{2}, \\\theta(p_{\rm cov})>0,\;& for\;p_{\rm cov}>\frac{1}{2}. \\
    
\end{array}
\end{equation}
where $p_{\rm cov}$ represents the homogenerous coverage probability on the sphere and $\theta(p_{\rm cov})$ is the percolation probability based on this coverage probability.
\label{theo:perpro_covpro}
\end{theorem}
In this paper, the coverage probability and the percolation probability both depend on the number of LEO satellites $N$ and its coverage angle $\gamma$. We have used the expression $\theta(N,\gamma)$ for percolation probability. Similar to Lemma \ref{lem:criticalanalysis}, for a fixed number of LEO satellites, the relationship between the critical constellation altitude $h^c$ and maximum slant range $d_m$ is shown in the following lemma.
\begin{lemma}
    When the number of LEO satellites is fixed, the critical constellation altitude $h$ can be expressed using the maximum slant range $d_m$:
    \begin{equation}
        h^c = \sqrt{d_m^2-r_e^2+t^2(N) r_e^2}+t(N)r_e-r_e.
        \label{criticalaltitude}
    \end{equation}
    Correspondingly, the critical maximum slant range $d_m^c$ can be also expressed using the constellation altitude $h$:
    \begin{equation}
        d_m^c=\sqrt{r_e^2+(r_e+h)^2-2t(N)r_e(r_e+h)},
    \end{equation}
    where $t(N)=2\times(\frac{1}{2})^{\frac{1}{N}}-1$.
\end{lemma}
\begin{IEEEproof}
    From Lemma \ref{lem:criticalanalysis} and Theorem \ref{theo:perpro_covpro}, the critical relationship between $N$ and $\gamma$, (\ref{criticalNc}), can be rewritten as $\cos{\gamma^c}=2\times(\frac{1}{2})^{\frac{1}{N}}-1$, that is, $\gamma^c=\arccos{\big(2\times(\frac{1}{2})^{\frac{1}{N}}-1\big)}$. Using the law of cosines, $\cos{\gamma^c}=\frac{r_s^2+r_e^2-d_m^2}{2r_e r_s}$ where $r_s=r_e+h$, we can obtain the critical relationship between the altitude $h$ and the maximum slant range $d_m$ above.
\end{IEEEproof}
The above lemma is an extension of the optimization problem (\ref{design2}), where the optimal value of $\gamma$ is related to $d_m$ and $h$, which are both important parameters of LEO satellite constellations.
\section{Simulation results and discussion}
In this paper, we aim to prove the relationship between coverage probability of users on the sphere and the percolation probability as we defined. The parameters of three existing LEO satellite constellation: Starlink, Oneweb and Kuiper, that we adopt, are shown in Table.\ref{tab:TableOfParam} \cite{osoro2021techno,cakaj2021parameters}.

\begin{table}[ht]\caption{Parameters of LEO Systems}
\centering
    \begin{tabular}{ {l} | {l} | {l} | {l} }
    \hline
        \hline
    \textbf{Systems} & \textbf{Starlink} & \textbf{Oneweb} & \textbf{Kuiper} \\ \hline
    % \textbf{Number} & 4519 & 648 & 720 \\ \hline
    \textbf{Altitude ($\rm km$)} & 550 & 1200 & 610 \\ \hline
    \textbf{Elevation Angle $\epsilon$ ($^{\circ}$)} & 40.0 & 55.0 & 35.2 \\ \hline
    \textbf{Coverage Angle $\gamma$ ($^{\circ}$)} & 5.20 & 6.14 & 6.58 \\ \hline
    \textbf{Nadir Angle $\eta$ ($^{\circ}$)} & 44.80 & 28.86 & 48.22 \\ \hline
    \textbf{Max Slant Range $d_{m}$ ($\rm km$)} & 809.5 & 1411.9 & 978.5 \\ \hline
    \textbf{Coverage Areas ($\times 10^6\; {\rm km}^2$)} & $1.05$ & $1.46$ & $1.68$ \\ \hline
     \hline
    \end{tabular}
\label{tab:TableOfParam}
%\vspace{-8mm}
\end{table}
 \indent As shown in Fig.\ref{fig:gamma52}, we firstly adopt the Starlink's coverage angle $\gamma=5.2$°. When the number of LEO satellites equals the lower bound (\ref{lowerbound}) in Lemma \ref{lem:lowerbound}, the percolation probability $\theta(N,\gamma)=0$. When the number of LEO satellites equals the upper bound (\ref{upperbound}) in Lemma \ref{lem:upperbound1}, the percolation probability is non-zero. The phase transition of percolation probability is between the lower bound and upper bound. The critical threshold (\ref{criticalNc}) derived in Lemma \ref{lem:criticalanalysis} is slightly higher than the simulated result, but its corresponding percolation probability is extremely low. At this threshold, the coverage probability exceeds 0.5, and percolation probability increases rapidly from a low level (close to 0). This result supports the concept in Theorem \ref{theo:perpro_covpro}.\\
 \begin{figure}
    \centering
    \includegraphics[width=0.8\linewidth]{Figures/Pro_Num_gamma52.pdf}
    \caption{Percolation probability, coverage probability of LEO satellite coverage when $\gamma=5.2$°, with the lower bound $N_L$, upper bound $N_U$, simulated critical value and theoretical value of critical number of LEO satellites $N_c$.}
    \label{fig:gamma52}
\end{figure}
\indent Next, we aim to show the effect of parameters of LEO satellites on the percolation probability. In Fig.\ref{fig:companies}, we adopt the coverage angles $\gamma$ of Starlink, Oneweb and Kuiper. When the number of LEO satellites increases, percolation probability also increases and the derived critical value $N_c$ is the necessary condition for phase transition of percolation probability. For example, Starlink need to provide at least 340 LEO satellites to meet the needs of random large-scale continuous service path, while Oneweb needs 240 LEO satellites and Kuiper needs 200. The critical threshold works well for different values of coverage angle $\gamma$. For realistic applications, the number of LEO satellites also depends on the capacity we need, and our derived critical value is only a necessary condition.\\
\begin{figure}
    \centering
    \includegraphics[width=0.8\linewidth]{Figures/Pro_Num_Companies.pdf}
    \caption{Percolation probability $\theta(N,\gamma)$ versus the number of LEO satellites $N$. Each curve has its corresponding critical number of satellites, $N_c$.}
    \label{fig:companies}
\end{figure}
\indent Considering 500 LEO satellites, we can observe how the altitude $h$ and maximum slant range $d_m$ of LEO satellites affect the percolation probability together. In Fig.\ref{fig:Pro_SLR_Altitude}, we adopt the maximum slant range $d_m$ of Starlink, Oneweb and Kuiper. When the $h$ increases, the percolation probability decreases because the coverage angle $\gamma$ becomes lower. The critical altitude of LEO satellites for phase transition of percolation probability from non-zero to zero is shown in (\ref{criticalaltitude}). We can notice that these three companies already deploy their LEO satellites at suitable altitudes lower than the critical threshold, where 500 LEO satellites can successfully provide large-scale continuous service for any applications.\\
\begin{figure}
    \centering
    \includegraphics[width=0.8\linewidth]{Figures/Pro_SLR_Altitude.pdf}
    \caption{Percolation probability versus the altitude of satellites when $N=500$. Each curve has its corresponding critical constellation altitude $h_c$.}
    \label{fig:Pro_SLR_Altitude}
\end{figure}
\indent Similarly, as shown in Fig.\ref{fig:Pro_Alt_SlantRange}, we consider 500 LEO satellites and the altitudes of Starlink, Oneweb and Kuiper constellations. When the $d_m$ increases, percolation probability increases from zero to non-zero due to the increase in nadir $\eta$ and coverage angle $\gamma$. The maximum slant ranges of these three constellations can already support large-scale continuous service. 
\begin{figure}
    \centering
    \includegraphics[width=0.8\linewidth]{Figures/Pro_Alt_SlantRange.pdf}
    \caption{Percolation probability versus the maximum slant range of LEO satellites when $N=500$. Each curve has its corresponding critical maximum slant range $d_m^c$.}
    \label{fig:Pro_Alt_SlantRange}
\end{figure}

In this paper, we verify that the proposed closed-form expression reflects the phase transition behavior of percolation probability when the number of LEO satellite increases. It is worth noting that, the design of constellation is a complex question, where we also need to consider the service strategy and expense. For example, if a considered LEO constellation can only use 100 LEO satellites to provide continuous service for international flights, the required maximum slant range should be higher than the result when $N=500$. We also need to consider the capacity and dynamic selection 
strategy of LEO satellites. In the future, more realistic simulations through orbital propagation tool are expected to be conducted, and multi-layer structure of LEO satellites and the effect from massive users on the traffic should be considered. For example, the kinds of service requirements from IoT devices and mobile users will also lead to different coverage areas and traffic congestion problem of LEO satellite system, which are expected to be solve through routing algorithm and capacity enhancement.
\section{Conclusion}
This paper is the first attempt to show and prove the concept of percolation on the sphere, especially considering the connections between spherical coverage areas. Using the stereographic projection, we defined the percolation on the sphere using the percolation on the projection plane. We first introduced sub-critical and super-critical cases, where the percolation probability $\theta$ is zero and non-zero respectively. We considered two special deployments and derived the lower bound $N_L$ and upper bound $N_U$ of the critical number of LEO satellites $N$. We proved the existence of the critical condition for phase transition of percolation probability from zero to non-zero. Using the hexagonal face percolation on the projection plane, we derived the tight lower bound $N_c^L$ and upper bound $N_c^U$ for percolation, and obtained the closed-form expression of the critical number of satellites $N_c$. We also obtained the expression of critical condition of altitude $h$ and maximum slant range $d_m$. We conducted the simulations to show how these parameters affect the percolation probability and our derived critical expressions can work well to show the phase transition. We emphasized that the critical expressions we derived are the necessary conditions. However, for realistic applications, it is necessary to consider the dynamic selection strategy, capacity and cost.
\appendices
\section{Proof of Lemma~\ref{lem:etagamma}}\label{app:etagamma}
\indent In the triangle $\triangle \yy\oe\textbf{w}_{0,1}$, the maximum slant range $d_m=\|\yy-\textbf{w}_{0,1}\|$. Using the Law of Cosines, we have:
\begin{equation}
    d_m^2+r_s^2-r_e^2=2d_mr_s\cos\eta.
\end{equation}
Therefore, 
\begin{equation}
    d_m=-\sqrt{r_e^2-r_s^2\sin^2\eta}+r_s\cos\eta.
    \label{cosresult}
\end{equation}
Using the Law of Sines, we have:
\begin{equation}
    \frac{d_m}{\sin\gamma}=\frac{r_e}{\sin\eta}.
    \label{sin}
\end{equation}
Substitute (\ref{cosresult}) into (\ref{sin}), we obtain the relationship between $\gamma$ and $\eta$:
   \begin{equation}
       \gamma=\arcsin(\frac{\sin\eta}{r_e}\bigg(-\sqrt{r_e^2-r_s^2\sin^2\eta}+r_s \cos\eta\bigg)).
   \end{equation}

\section{Proof of Theorem \ref{theo:pcov}}\label{app:pcov}
To achieve the coverage probability of each point on the earth, we need to first calculate the area of the spherical cap. As shown in Fig.\ref{fig:caparea},  we can obtain the area of the spherical cap using the integration in polar coordinates:
\begin{equation}
\begin{array}{r@{}l}
    
\mathcal{S}(\gamma)&=\int_{0}^{\gamma}2\pi r(\theta)\cdot r_e\,\dd \theta=\int_{0}^{\gamma}2\pi r_e^2\sin\theta\,\dd \theta\\
&=\displaystyle2\pi r_e^2\cos\theta|_{\gamma}^{0}=\displaystyle2\pi r_e^2(1-\cos\gamma).
\end{array}
\end{equation}


\indent As shown in Fig.\ref{fig:caparea}, the surface area of the spherical cap is equal to the lateral surface area of a cylinder, whose radius is the same as the sphere $r_e$ and height is the same as the spherical cap $H$. This is a classic example of the Mercator projection.\\
\begin{figure}
    \centering
    \includegraphics[width=0.8\linewidth]{Figures/covpro.pdf}
    \caption{Examples for the event of a point on the sphere located at $\textbf{w}_k$ being covered (left) or not being covered (right) by LEO satellites located at $\yi'$s. }
    \label{fig:covpro}
\end{figure}
\indent As shown on the right of Fig.\ref{fig:covpro}, if the satellite located at $\textbf{y}_1$ is outside the angle range $\gamma$, its coverage area $\mathcal{O}_1=\mathcal{A}_1$ does not include the considered point $\textbf{w}_k$ on the earth. Correspondingly, its coverage center $\textbf{x}_1$ is outside the spherical cap centered at $\textbf{w}_k$, which can be called as the service request area $\mathcal{O}(\textbf{w}_k,\gamma)$. Therefore, the probability of the point located at $\textbf{w}_k$ not being covered by the LEO satellites located at $\yi'$s is:
\begin{equation}%\displaystyle
\begin{array}{r@{}l}
    p&_{\rm ncov}(N,\gamma)\triangleq \mathbb{P}\{\textbf{w}_k {\rm\,is\,not\,covered\,by\,}\yi'{\rm s}\}\\
    &\overset{(a)}{=} \prod_{i=1}^{N}\mathbb{P}\{\textbf{w}_k {\rm\,is\,not\,covered\,by\,}\yi\}\\
    &\overset{(b)}{=}\big(1-\frac{\mathcal{S(\gamma)}}{4\pi r_e^2}\big)^{N}=\big(1-\frac{2\pi r_e^2 (1-\cos\gamma)}{4\pi r_e^2}\big)^{N}\\
    &=\big(\frac{1+\cos\gamma}{2}\big)^{N},\\
\end{array}
\end{equation}
% \\
%     &
where steps (a) and (b) both come from the BPP assumption of LEO satellite deployment. As shown on the left of Fig.\ref{fig:covpro}, if there is at least one LEO satellite located inside $\mathcal{O}(\textbf{w}_k,\gamma)$ , the considered point $\textbf{w}_k$ on the earth is covered. Therefore, the coverage probability of each point on the earth can be defined as:
\begin{equation}
\begin{array}{r@{}l}
    p&_{\rm cov}(N,\gamma)\triangleq \mathbb{P}\{\textbf{w}_k {\rm\,is\,covered\,by\,at\,least\,one\,of\,}\yi'{\rm s}\}\\
    &=1-\mathbb{P}\{\textbf{w}_k {\rm\,is\,not\,covered\,by\,}\yi'{\rm s}\}\\
    &=1-\big(\frac{1+\cos\gamma}{2}\big)^{N}.
\end{array}
\end{equation}

% \section{Proof of Lemma~\ref{lem:subset}}\label{app:subset}

% \section{Proof of Lemma~\ref{lem:captocircle}}\label{app:captocircle}
% Assume that a point $P'(a,b,0)$ is located at the plane. Its corresponding point on the sphere is $P(x,y,z)$, \ie $P'=\mathcal{F}(P)$ and $P=\mathcal{F}^{-1}(P)$. Therefore, the coordinates of $P$ is $(\frac{4r_e^2 a}{4r_e^2 +a^2+b^2},\frac{4r_e^2 b}{4r_e^2 +a^2+b^2},\frac{2r_e(a^2+b^2)}{4r_e^2 +a^2+b^2})$. A circle on the earth can be viewed as the intersection of the sphere 
% \begin{equation}
%   x^2+y^2+z^2-2zr_e=0  
%   \label{sphereequation}
% \end{equation}
%  and a 3D plane 
% \begin{equation}
%      Ax+By+Cz+D=0.
%      \label{planeequation}
% \end{equation} 
% To ensure that there exists the intersection, the sufficient condition is:
% \begin{equation}
%     \frac{|Cr_e+D|}{\sqrt{A^2+B^2+C^2}}\leq r_e.
% \end{equation}
% Assume that $B=0$, the center of the circular area is located at the plane $xoz$. Since each point $P$ is located at the plane, we substitute its coordinates into (\ref{planeequation}):
% \begin{equation}
%    A\frac{4r_e^2 a}{4r_e^2 +a^2+b^2}+C\frac{2r_e(a^2+b^2)}{4r_e^2 +a^2+b^2}+D=0.
% \end{equation}
% % Because $r_e^2+a^2+b^2>0$, we obtain:
% % \begin{equation}
% % \begin{array}{r@{}}
% %     \displaystyle (2r_e C+D)a^2+(2r_e C+D)b^2+4r_e^2 Aa+4r_e^2 D=0
% % \end{array}
% % \end{equation}
% If $2r_e C+D\neq 0$, we have

% \begin{equation}
% \begin{array}{r@{}}
%     \displaystyle a^2+b^2+\frac{4r_e^2 A}{2r_e C+D}a+(\frac{2r_e^2 A}{2r_e C+D})^2\\
%    \displaystyle =(\frac{2r_e^2 A}{2r_e C+D})^2-\frac{4r_e^2 D}{2r_e C+D},
% \end{array}
% \end{equation}
% \ie
% \begin{equation}
%     \displaystyle (a-\frac{2r_e^2 A}{2r_e C+D})^2+b^2=\frac{4r_e^2(r_e^2 A^2 -2r_e CD-D^2)}{(2r_e C+D)^2}.
% \end{equation}
% Therefore, the projection of a circle on the sphere is also a circle on the projection plane. That means the spherical cap on the sphere can be mapped to a circular area on the projection plane. However, when the original circle cross the top of the sphere, \ie the point $(0,0,2r_e)$ is located at the plane $Ax+Cz+D=0$, $2r_e C+D=0$ and the projection on the plane is a straight line $Aa+D=0$. 

\section{Proof of Lemma~\ref{lem:circularradius}}\label{app:circularradius}
\indent Assume that $\psi=\angle\textbf{w}_S \oe \textbf{x}$ and $\gamma_0$ is the central angle of a spherical cap. The diameter of the projection and its corresponding arcs are both located at the plane $xoz$. First consider the case where $0\leq \psi< \pi-\gamma_0$. As shown in Fig.\ref{fig:OQOQ}, the points at both ends of the diameters satisfy the projection relationship
:
\begin{equation}
    Q'_{l}=\mathcal{F}(Q_{l}),\,Q'_{r}=\mathcal{F}(Q_{r}).
\end{equation}
\begin{figure}
    \centering
    \includegraphics[width=0.75\linewidth]{Figures/OQOQ.pdf}
    \caption{The stereographic projection $Q'_{l}=\mathcal{F}(Q_{l})$ and $Q'_{r}=\mathcal{F}(Q_{r})$. }
    \label{fig:OQOQ}
\end{figure}
Therefore, the radius of projected circular area satisfies:
\begin{equation}
    r(\psi)=\frac{1}{2}\bigg|\|\textbf{o}_z Q'_{r}\|-\|\textbf{o}_z Q'_{l}\|\bigg|.
\label{RQQ}
\end{equation}
\indent We have
\begin{equation}
\begin{array}{r@{}l}
    \|\textbf{o}_z& Q'_{r}\|\displaystyle=2r_e \tan(\angle \textbf{w}_S \textbf{w}_N Q'_{r})\\
    &\displaystyle=2r_e \tan(\frac{\angle \textbf{w}_S \textbf{o}_e Q_{r}}{2})\displaystyle=2r_e \tan(\frac{\psi+\gamma_0}{2}).
\end{array}
\label{0Qr}
\end{equation}
and 
\begin{equation}
\begin{array}{r@{}l}
    \|\textbf{o}_z& Q'_{l}\|\displaystyle=2r_e \tan(\angle \textbf{w}_S \textbf{w}_N Q'_{l})\\
    &\displaystyle=2r_e \tan(\frac{\angle \textbf{w}_S \textbf{o}_e Q_{l}}{2})\displaystyle=2r_e \tan(\frac{\psi-\gamma_0}{2}).
\end{array}
\label{0Ql}
\end{equation}
\indent Substitute (\ref{0Qr}) and (\ref{0Ql}) into (\ref{RQQ}), we obtain:
\begin{equation}
    r(\psi)=r_e\bigg|\tan(\frac{\psi+\gamma_0}{2})-\tan(\frac{\psi-\gamma_0}{2})\bigg|.    
\label{rpsi}
\end{equation}
% \indent It is worth noting that this equation is also suitable for the case where $\pi-\gamma<\phi_i\leq\pi$. When $\psi_i=\pi-\gamma$, the radius goes to $+\infty$, therefore, $r_{i}(\psi_i)<+\infty$. Next, we prove that there exists the minimum value of $r_{i}(\psi_i)$.\\ 
% \indent First consider the case where $0\leq\psi_i <\pi-\gamma$. Taking the derivative of $r_{i}(\psi_i)$ with respect to $\psi_i$, we have
% \begin{equation}
% \begin{array}{r@{}l}
%     &\displaystyle\frac{\partial r_{i}(\psi_i)}{\partial \psi_i}=\displaystyle r_e\bigg(\frac{1}{2\cos^2(\frac{\psi_i+\gamma}{2})}-\frac{1}{2\cos^2(\frac{\psi_i-\gamma}{2})}\bigg)\\
%     &=\displaystyle r_e\bigg(\frac{1}{1+\cos(\psi_i+\gamma)}-\frac{1}{1+\cos(\psi_i-\gamma)}\bigg)\geq 0.
% \end{array}
% \end{equation}
% Similarly, for the case where $\pi-\gamma<\psi_i\leq\pi$, 
% \begin{equation}
% \begin{array}{r@{}l}
%     &\displaystyle\frac{\partial r_{i}(\psi_i)}{\partial \psi_i}=\displaystyle r_e\bigg(\frac{1}{2\cos^2(\frac{\psi_i-\gamma}{2})}-\frac{1}{2\cos^2(\frac{\psi_i+\gamma}{2})}\bigg)\\
%     &=\displaystyle r_e\bigg(\frac{1}{1+\cos(\psi_i-\gamma)}-\frac{1}{1+\cos(\psi_i+\gamma)}\bigg)\leq 0.
% \end{array}
% \end{equation}
% Therefore, the minimum value of $r_{i}(\psi_i)$ is:
% \begin{equation}
% \begin{array}{r@{}l}
%     R_{z,min}&=\min\{r_{i}(0),r_{i}(\pi)\}\\
%     &=\min\{2r_e \tan (\frac{\gamma}{2}),2r_e \cot (\frac{\gamma}{2})\}
% \end{array}
% \end{equation}
%  Therefore, $R_{z,\min}=2r_e \tan(\frac{\gamma}{2})$ and the range of $r_{i}(\psi_i)$ is:
When the central angle $\gamma_0$ is less than $\frac{\pi}{2}$, (\ref{rpsi}) works for $\gamma_0-\pi<\psi<\pi-\gamma_0$. The range of the radius of the projected circular area is:
\begin{equation}
    2r_e \tan(\frac{\gamma_0}{2})\leq r(\psi)<+\infty.
\end{equation}
Conversely, if the radius of the projected circular area is $r$, the range of the central angle of the original spherical cap, $\gamma_0$, is:
\begin{equation}
    0<\gamma_0\leq 2\arctan(\frac{r}{2r_e}).
\label{gamma0range}
\end{equation}

% \section{Proof of Theorem~\ref{theo:concomsphere}}\label{app:concomsphere}
% In this appendix, we need to prove the concept of percolation on the sphere.

% \section{Proof of Lemma~\ref{lem:lowerbound}} \label{app:lowerbound}
% Here, we need to prove the lower bound.
% \section{Proof of Lemma~\ref{lem:upperbound1}} \label{app:upperbound1}
% Here, we need to prove the upper bound.
\section{Proof of Lemma~\ref{lem:nondecreasing}\label{app:nondecreasing}}
To prove that there exists a critical number of LEO satellites that causes the phase transition of percolation probability, we need to first prove that the percolation probability is a non-decreasing function of $N$ when the $\gamma$ is fixed. \\
\indent Firstly, we assume that the LEO satellites are deployed at the same altitude with the same nadir angle $\eta$, therefore, the coverage angle $\gamma$ is also fixed. The locations of satellites follow a BPP around the earth at a certain altitude. Consider two sets of satellites $\Phi_1$ and $\Phi_2$ with the number of vertices $N_1$ and $N_2$, respectively, where $N_1<N_2$. Since $\Phi_1$ and $\Phi_2$ are both BPPs, $\Phi_1$ can be constructed by removing any one vertice inside $\Phi_2$. Similarly, $\Phi_2$ can be constructed by adding any other vertice into $\Phi_1$. Because we discuss their coverage areas on the sphere, the set of spherical caps' centers $V_1$ and $V_2$ can be generated following BPPs with number $N_1$ and $N_2$ at the altitude $0\;\rm km$ (on the sphere), where $V_1\subseteq V_2$ when $\Phi_1\subseteq\Phi_2$.\\
\indent Removing vertice from $\Phi_2$ or adding vertice into $\Phi_1$ both lead to the change of the edge set, where $E_1\subseteq E_2$. The connected components in these two random graphs are defined as: $K_{x,1}\subseteq G_{x}(V_{x,1},V_{x,1})$ and $K_{x,2}\subseteq G_{x}(V_{x,2},V_{x,2})$, which satisfy $K_{x,1}\subseteq K_{x,2}$. Therefore, $0<N_1<N_2$ indicates $\theta(N_1,\gamma)\leq \theta(N_2,\gamma)$, \ie the percolation probability is a non-decreasing function of $N$.\\
\indent It is worth noting that, in this paper, we use two kinds of projection: \textit{i) mapping the satellite locations to the sphere}, which obtains the coverage centers, \textit{ii) mapping every point on the sphere to the projection plane}, which helps us define the percolation on the sphere. The mapping from satellites to coverage centers keep the BPP properties, and mapping the sphere to the projection plane keeps almost all the connections between coverage areas, which has been introduced in the Lemma \ref{lem:mappingrelation}. Therefore, the projections of connected components on the considered projection plane also satisfy $K_{z,1}\subseteq K_{z,2}$. On the plane, we can focus on the connected component containing the origin $\textbf{o}_z$, \ie $K_z(0)$. Therefore, the percolation probabilities for these two cases also satisfy $\P(|K_{z,1}(0)|=\infty)\leq \P(|K_{z,2}(0)|=\infty)$.
\section{Proof of Lemma \ref{lem:upperbound1}}
\label{app:upperbound}
Because the area of a sphere is finite, the graph percolates once the whole sphere is covered by LEO satellites. Therefore, the percolation probability can be lower bounded by the full coverage probability, \ie
\begin{equation}
    \theta(N,\gamma)\geq \P\{{\rm Full\;Coverage}|N,\gamma\}.
\label{full}
\end{equation}
\indent We have proposed our `full coverage scheme' in Sec.\ref{subsec:phasetransition}. Because it is one way to realize full coverage, the probability of a successful deployment is less than or equal to the full coverage probability, \ie
\begin{equation}
\begin{array}{r@{}l}
    \P&\{{\rm Full\;Coverage}|N,\gamma\}\\
    &\geq \P\{{\rm The\; Proposed\;Full\;Coverage\;Scheme}|N,\gamma\}.
\end{array}
\label{successfull}
\end{equation}
Therefore, if we can find a computable and non-zero value of the probability of our proposed full coverage scheme, the percolation probability is proved to be non-zero. Firstly, we need to prove that full coverage can be realized under such a scheme. \\
\indent In Fig.\ref{fig:Upperbound}, we propose to represent the whole sphere using the union of sphere caps and introduce the steps we need. Let $\Omega$ denote the whole sphere and $\mathcal{A}_i$ denote the coverage area of the \textit{i}th LEO satellite. Therefore, the full coverage defined as an event: $\Omega\subseteq \bigcup_{i=1}^{N}\mathcal{A}_i$.\\
\indent We first uniformly divide the whole sphere into $2m$ `slices' using $2m$ meridians, where the \textit{j}th slice is denoted by $\mathcal{S}_j$ and spans $\frac{\pi}{m}$ of longitude. The whole sphere can be expressed as the union of $\mathcal{S}_j$'s, that is, $\Omega=\bigcup_{j=1}^{2m}\mathcal{S}_j$.\\
\indent On the sphere, we assume that slices $\mathcal{S}_j$ and $\mathcal{S}_{m+j}$ are symmetric about the earth's center. They can be contained by a `belt' defined as $\mathcal{T}_j$, \ie $\mathcal{S}_j\bigcup\mathcal{S}_{m+j}\subseteq \mathcal{T}_j$. Therefore, the union of slices is the subset of the union of belts, \ie $\bigcup_{j=1}^{2m}\mathcal{S}_j\subseteq \bigcup_{j=1}^{m}\mathcal{T}_j$. Because $\Omega=\bigcup_{j=1}^{2m}\mathcal{S}_j$ and $\bigcup_{j=1}^{m}\mathcal{T}_j\subseteq \Omega$, the union of the belts is the same as the whole sphere, \ie $\Omega=\bigcup_{j=1}^{m}\mathcal{T}_j$.\\
\indent Next, we can rotate the belt $\mathcal{T}_1$ and make it symmetric about the equatorial plane. Belt $\mathcal{T}_1$ spans $\frac{\pi}{m}$ of latitude. It is able to uniformly divide $\mathcal{T}_1$ into $n$ pieces using $n$ meridians, \ie $\mathcal{T}_1=\bigcup_{k=1}^{n}\mathcal{D}_{1,k}$. where each piece $\mathcal{D}_{1,k}$ spans $\frac{\pi}{m}$ of latitude and $\frac{2\pi}{n}$ of longitude. Other belts can be divided in the same way, and all pieces have the same shape and size. Therefore, the whole sphere is the same as the union of such `pieces', \ie $\Omega=\bigcup_{j=1}^{m}\bigcup_{k=1}^{n}\mathcal{D}_{j,k}$.\\
\indent We consider the piece $\mathcal{D}_{1,1}$ which spans $\frac{\pi}{m}$ of latitude and $\frac{2\pi}{n}$ of longitude firstly. We can find the minimum spherical cap $\mathcal{E}_{1,1}$ containing $\mathcal{D}_{1,1}$, \ie $\mathcal{D}_{1,1}\subseteq \mathcal{E}_{1,1}$. The spherical cap has the same center of $\mathcal{D}_{1,1}$, and its central angle $\zeta$ satisfies:
\begin{equation}
    \zeta =\arccos{\bigg(\cos{\frac{\pi}{2m}}\cos{\frac{\pi}{n}}\bigg)}.
\end{equation}
For each piece $\mathcal{D}_{j,k}$, we can find its corresponding spherical cap $\mathcal{E}_{j,k}$ with the same central angle $\zeta$, where $\mathcal{D}_{j,k}\subseteq \mathcal{E}_{j,k}$. Therefore, the union of pieces is the subset of the union of these spherical caps, \ie $\bigcup_{j=1}^{m}\bigcup_{k=1}^{n}\mathcal{D}_{j,k}\subseteq \bigcup_{j=1}^{m}\bigcup_{k=1}^{n}\mathcal{E}_{j,k}$. Because $\bigcup_{j=1}^{m}\bigcup_{k=1}^{n}\mathcal{E}_{j,k}\subseteq \Omega$ and $\Omega=\bigcup_{j=1}^{m}\bigcup_{k=1}^{n}\mathcal{D}_{j,k}$, we obtain: $\Omega=\bigcup_{j=1}^{m}\bigcup_{k=1}^{n}\mathcal{E}_{j,k}$.\\
\indent To realize the full coverage, we aim to ensure that each spherical cap $\mathcal{E}_{j,k}$ is covered. We can follow three steps: i) make $m$ and $n$ large enough to make the spherical cap $\mathcal{E}_{j,k}$ can be covered by one satellite whose coverage angle is $\gamma$, \ie $\zeta< \gamma$, ii) use $m\times n$ LEO satellites and deploy them one by one, \ie $\mathcal{E}_{j,k}\subseteq \mathcal{A}_{n(j-1)+k}$. In this case, we can ensure that $\Omega=\bigcup_{j=1}^{m}\bigcup_{k=1}^{n}\mathcal{A}_{n(j-1)+k}$ because $\bigcup_{j=1}^{m}\bigcup_{k=1}^{n}\mathcal{A}_{n(j-1)+k}\subseteq \Omega$ and $\bigcup_{j=1}^{m}\bigcup_{k=1}^{n}\mathcal{E}_{j,k}\subseteq \bigcup_{j=1}^{m}\bigcup_{k=1}^{n}\mathcal{A}_{n(j-1)+k}$.\\
\indent To make $\zeta<\gamma$, we can design $m$ first and then $n$. The requirements are: i) $\frac{\pi}{2m}<\gamma$ and $\frac{\pi}{n}<\gamma$ and ii) $\zeta = \arccos{(\cos{\frac{\pi}{2m}}\cos{\frac{\pi}{n}})}<\gamma$, that is, $\cos{\frac{\pi}{2m}}\cos{\frac{\pi}{n}}>\cos{\gamma}$. We can first choose a feasible $m$, for example, $m=\left \lceil \frac{\pi}{\gamma} \right \rceil$ to let the inequality $\frac{\pi}{2m}<\gamma$ hold. Next, because $\cos{\frac{\pi}{n}}>\max\{\cos{\gamma},\frac{\cos{\gamma}}{\cos{\frac{\pi}{2m}}}\}$,  $\frac{\pi}{n}<\arccos{\frac{\cos{\gamma}}{\cos{\frac{\pi}{2m}}}}$, the choice of $n$ should satisfy $n>\left \lceil \frac{\pi}{\arccos{\frac{\cos{\gamma}}{\cos{\frac{\pi}{2m}}}}} \right \rceil$. Therefore, we can let $n=\left \lceil \frac{\pi}{\arccos{\frac{\cos{\gamma}}{\cos{\frac{\pi}{2m}}}}} \right \rceil+1$ and the total number of satellites is $N_U=m\times n$. Therefore, when $N=N_U$, we can realize the full coverage on the sphere.\\
\indent Based on such a special deployment, we can derive the lower bound of its probability. We assume that the center of the first satellite's coverage area $\mathcal{A}_{1,1}$ is $\textbf{x}_{1,1}$ and the center of the spherical cap $\mathcal{E}_{1,1}$ is $\textbf{w}_{1,1}$. When $\angle \textbf{x}_{1,1} \textbf{o}_e \textbf{w}_{1,1}<\gamma-\zeta$, $\mathcal{E}_{1,1}$ is totally covered by $\mathcal{A}_{1,1}$. The probability of such an event is $\frac{1-\cos{(\gamma-\zeta)}}{2}$. We can deploy other satellites in the same way to ensure all $\mathcal{E}_{j,k}$'s are covered. The probability of such a successful full deployment is:
\begin{equation}
\begin{array}{r@{}l}
    \P&\{{\rm The\; Proposed\;Full\;Coverage\;Scheme}|N=N_U,\gamma\}\\
    &=\big(\frac{1-\cos{(\gamma-\zeta)}}{2}\big)^N,
\end{array}
\end{equation}
which is computable and non-zero. Therefore, using the inequalities (\ref{full}) and (\ref{successfull}), we prove that the percolation probability is strictly larger than 0 when $N=N_U$. Similarly, when $N>N_U$, we can deploy the first $N_U$ satellites in the same way, and deploy the other satellites randomly.
Therefore, for $N\geq N_U$, percolation probability is always non-zero, \ie
\begin{equation}
    \theta(N,\gamma)>0\;{\rm for}\;N\geq N_U.
\end{equation}

\section{Proof of Lemma \ref{lem:phasetransition}}\label{app:phasetransition}
\indent According to Lemma \ref{lem:lowerbound} and \ref{lem:upperbound1}, we know that: i) $\theta(N,\gamma)=0$ for $N\leq N_L$ and ii) $\theta(N,\gamma)>0$ for $N\geq N_U$. Because the percolation probability $\theta(N,\gamma)$ is a non-decreasing function of $N$, there must exist a critical value of $N$, \ie $N_c$, which satisfies:
\begin{equation}
    \begin{array}{c}
        \theta(N,\gamma)=0,\, {\rm for}\; N< N_c,\\
        \theta(N,\gamma)>0,\, {\rm for}\; N> N_c. 
    \end{array}
\end{equation}
% It is worth noting that $N$ should be an integer but $N_c$ does not need to be defined as an integer. Therefore, we can define the $N_c$ using:
% \begin{equation}
%     \begin{array}{c}
%         \theta(N,\gamma)=0,\, {\rm for}\; N\leq \left\lfloor N_c\right\rfloor,\\
%         \theta(N,\gamma)>0,\, {\rm for}\; N\geq \left\lceil N_c\right\rceil. 
%     \end{array}
% \end{equation}
The upper and lower bounds should satisfy $N_L\leq \left\lfloor N_c\right\rfloor$ and $\left\lceil N_c\right\rceil\leq N_U$. However, these two inequalities do not hold equality at the same time because the percolation probability can not be zero and non-zero at the same time.
% \indent Next, we aim to prove that $N_c$ satisfies $N_L<N_c<N_U$, that is, the critical value does not approach 0 or $+\infty$. First, in the sub-critical case, we know that $\theta(N,\gamma)=0$ for $N<N_L$. Therefore, $N_c\geq N_L$. We still need to prove that $N_c<N_U$ where $N_L$ is obtained in the super-critical case.\\
% \indent We have proved that there exists a critical number of satellites $N_c$ which remarks the phase transition of percolation probability. Now we introduce the events below:
% \begin{equation}
% \begin{array}{l}
%     \mathcal{M}|N=\{{\rm There\;exist\;full\;coverage\;deployments}\\
%     \hspace{5.3cm} {\rm \;when}\;N\;{\rm is\;known}\} \\
%     \mathcal{N}|N=\{{\rm Removing\;any\;satellite\;will\;break\;the\;}\\
%     \hspace{2.2cm}{\rm existing\;percolation\;when}\;N\;{\rm is\;known  }\}
% \end{array}
% \end{equation}
% \indent If there exists full coverage deployment when $N$ is known, removing one satellite will not always avoid the percolation. That is because the remaining coverage is shaped as a complete sphere except for a spherical cap smaller than a hemisphere, and the percolation in these cases is kept. If $N=N_c$, removing any satellites will break the existing percolation almost surely because $\theta(N_{\rm c}-1,\gamma)=0$. It is worth noting that, there might exist some cases where the coverage areas still connect two symmetric points on the sphere when $N=N_c-1$. However, based on the BPP assumption, the probability of these events is 0 since $\theta(N_c-1,\gamma)=0$. Therefore, $\mathcal{M}$ leads to $\mathcal{N}$ almost surely, that is $\P\{\mathcal{N}\; {\rm leads\;to\;\overline{\mathcal{M}}}\}=\P\{\mathcal{M}\; {\rm leads\;to\;}\overline{\mathcal{N}}\}=1$. From the perspective of probability, $\P\{\mathcal{N}|N\}\leq \P\{\overline{\mathcal{M}}|N\}$ and $\P\{\mathcal{M}|N\}\leq \P\{\overline{\mathcal{N}}|N\}$. The probability $\P\{\mathcal{M}|N\}$ is a non-decreasing function of $N$ with possible values 0 or 1, where $\P\{\mathcal{M}|N_c\}\leq \P\{\overline{\mathcal{N}}|N_c\}=1-\P\{\mathcal{N}|N_c\}=0$, $\P\{\mathcal{N}|N_U\}\leq \P\{\overline{\mathcal{M}}|N_U\}=1-\P\{\mathcal{M}|N_U\}=0$. Therefore, $\P\{\mathcal{M}|N_c\}<\P\{\mathcal{M}|N_U\}$ because $\P\{\mathcal{M}|N_c\}=0$ and $\P\{\mathcal{M}|N_U\}=1$. Therefore, we can obtain $N_c< N_U$.
\section{Proof of Theorem \ref{theo:inhomohexagon}}\label{app:inhomohexagon}
\indent Assume that different hexagons have different probabilities of being open or closed and both of them have their minimum value $p_{\rm cov}^{\min}$ and $p_{\rm ncov}^{\min}$, \ie $\P\{\mathcal{H}_{l} {\rm \; is\;open}\} \geq p_{\rm cov}^{\min}$ and $\P\{\mathcal{H}_{l} {\rm \; is\;closed}\} \geq p_{\rm ncov}^{\min}$.\\ %, where
% \begin{equation}
%     \P\{\mathcal{H}_{l} {\rm \; is\;open}\} \geq p_{\rm cov}^{\min}
% \end{equation}
% and
% \begin{equation}
%     \P\{\mathcal{H}_{l} {\rm \; is\;closed}\} \geq p_{\rm ncov}^{\min}.
% \end{equation}
\indent We firstly discuss the case where the $\P\{\mathcal{H}_{l} {\rm \; is\;open}\}$ has its minimum value $p_{\rm cov}^{\min}$. Considering the hexagons with side length $a$, we can cover the hexagons following their coverage probability $\P\{\mathcal{H}_{l} {\rm \; is\;open}\}$. The random graph in this case is defined as $G_{z}^{\rm cov}$. The probabilities $p_{\rm cov}^{\min}$ and $\P\{\mathcal{H}_{l} {\rm \; is\;open}\}$ are assumed larger than 0. \\
\indent Define the random graph of `open hexagonal faces' generated by probability $p_{\rm cov}^{\min}$ as $G_{z,\min}^{\rm cov}$. We can generate the $G_{z,\min}^{\rm cov}$ through removing each open face $\mathcal{H}_{l}$ in $G_{z}^{\rm cov}$ by probability $p_{1}^l=1-p_{\rm cov}^{\min}/\P\{\mathcal{H}_{l} {\rm \; is\;open}\}$ where $p_{1}^l\in [0,1)$. Therefore, all open faces $G_{z,\min}^{\rm cov}$ are contained by $G_{z}^{\rm cov}$, \ie $G_{z,\min}^{\rm cov}\subseteq G_{z}^{\rm cov}$. When $\P\{\mathcal{H}_{l} {\rm \; is\;open}\}>1/2$, $p_{\rm cov}^{\min}>1/2$, the percolation probability $\P\{|G_{z,\min}^{\rm cov}|=\infty\}>0$, and the percolation probability of $G_z^{\rm cov}$ also satisfies $\P\{|G_{z}^{\rm cov}|=\infty\}>0$. In conclusion, the sufficient condition for non-zero percolation probability is 
\begin{equation}
    \P\{\mathcal{H}_{l} {\rm \; is\;open}\}>1/2.
\end{equation}
\indent Similarly, we define the random graph of `closed hexagonal faces' generated by probability $p_{\rm ncov}^{\min}$ as $G_{z,\min}^{\rm ncov}$. We can generate the $G_{z,\min}^{\rm ncov}$ through removing each closed face $\mathcal{H}_{l}$ in $G_{z}^{\rm ncov}$ by probability $p_{2}^l=1-p_{\rm ncov}^{\min}/\P\{\mathcal{H}_{l} {\rm \; is\;closed}\}$ where $p_{2}^l\in [0,1)$. Therefore, all closed faces $G_{z,\min}^{\rm ncov}$ are contained by $G_{z}^{\rm ncov}$, \ie $G_{z,\min}^{\rm ncov}\subseteq G_{z}^{\rm ncov}$. When $\P\{\mathcal{H}_{l} {\rm \; is\;closed}\}>1/2$, $p_{\rm ncov}^{\min}>1/2$, the percolation probability $\P\{|G_{z,\min}^{\rm ncov}|=\infty\}>0$, and the percolation probability of $G_z^{\rm ncov}$ also satisfies $\P\{|G_{z}^{\rm ncov}|=\infty\}>0$. In conclusion, the sufficient condition for zero percolation probability is 
\begin{equation}
    \P\{\mathcal{H}_{l} {\rm \; is\;closed}\}>1/2.
\end{equation}
\section{Proof of Lemma \ref{lem:boundsforhexagons}}\label{app:boundsforhexagons}
\indent We first consider the hexagon $\mathcal{H}_{l}$ with the side length 
$a$. The circular area $\tilde{\mathcal{O}}_{l}$ with radius $a$ has the same center as $\mathcal{H}_{l}$. The center of 
$\mathcal{F}^{-1}(\tilde{\mathcal{O}}_{l})$ is $\textbf{x}_{o,l}$ and the center of $\mathcal{A}_i$ is $\textbf{x}_i$. The probability each hexagonal face $\mathcal{H}_{l}$ being closed satisfy:
\begin{equation}%\displaystyle
\begin{array}{r@{}l}
    \P&\displaystyle\{\mathcal{H}_{l}\;{\rm is\;closed}\}\\
    &=\P\{\mathcal{H}_{l}\;{\rm is\;not\;covered\;by\;}\bigcup_{i=1}^{N}\mathcal{F}(\mathcal{A}_i)\}\\
    &\geq\P\{\tilde{\mathcal{O}}_{l}\;{\rm is\;not\;covered\;by\;}\bigcup_{i=1}^{N}\mathcal{F}(\mathcal{A}_i)\}\\
    % &=\P\displaystyle\{\mathcal{F}^{-1}(\tilde{\mathcal{O}}_{l})\;{\rm is\;not\;covered\;by\;}\bigcup_{i=1}^{N}\mathcal{A}_i\}\\
    &\geq\prod_{i=1}^{N}\P\{\mathcal{F}^{-1}(\tilde{\mathcal{O}}_{l})\;{\rm is\;not\;covered\;by\;}\mathcal{A}_i\}\\
    &\geq\prod_{i=1}^{N}\P\{\angle \textbf{x}_{o,l} \textbf{o}_e \textbf{x}_i>\gamma+\gamma_m\}\\
    &=\big(\frac{1+\cos(\gamma+\gamma_m)}{2}\big)^{N},\\
\end{array}
\end{equation}
where $\gamma_m$ is the maximum central angle of the original spherical cap of hexagon's minimum circumscribed circle. Because the radius of the minimum circumscribed circle is $a$, from (\ref{gamma0range}), we can obtain its expression:
\begin{equation}
    \gamma_m=2\arctan \frac{a}{2r_e}.
\end{equation}
% \indent Next, we consider another circular area $\tilde{\mathcal{O}}_{l}^{\rm inner}$ with radius $\frac{\sqrt{3}}{2}a$ has the same center as $\mathcal{H}_{l}$. The center of $\mathcal{F}^{-1}(\tilde{\mathcal{O}}_{l}^{\rm inner})$ is also $\textbf{x}_{o,l}$ and the center of $\mathcal{A}_i$ is $\textbf{x}_i$. 
Similarly, the probability of $\mathcal{H}_{l}$ being covered satisfy:
\begin{equation}
\begin{array}{r@{}l}
    \P&\displaystyle\{\mathcal{H}_{l}\;{\rm is\;open}\}\\
    &=\P\{\mathcal{H}_{l}\;{\rm is\;covered\;by\;}\bigcup_{i=1}^{N}\mathcal{F}(\mathcal{A}_i)\}\\
    &\geq\P\{\tilde{\mathcal{O}}_{l}\;{\rm is\;covered\;by\;}\bigcup_{i=1}^{N}\mathcal{F}(\mathcal{A}_i)\}\\
    % &=\P\displaystyle\{\mathcal{F}^{-1}(\tilde{\mathcal{O}}_{l})\;{\rm is\;covered\;by\;}\bigcup_{i=1}^{N}\mathcal{A}_i\}\\
    &\geq\P\{\mathcal{F}^{-1}(\tilde{\mathcal{O}}_{l})\;{\rm is\;covered\;by\;at\; least\;one\;of\;}\mathcal{A}_i\}\\
    & =1-\prod_{i=1}^{N}\P\{\mathcal{F}^{-1}(\tilde{\mathcal{O}}_{l})\;{\rm is\;not\;covered\;by\;}\mathcal{A}_i\}\\
    &\geq1-\prod_{i=1}^{N}\P\{\angle \textbf{x}_{o,l} \textbf{o}_e \textbf{x}_i>\gamma-\gamma_m\}\\
    &=1-\big(\frac{1+\cos(\gamma-\gamma_m)}{2}\big)^{N}.\\
\end{array}
\end{equation}
% where
% \begin{equation}
%     \gamma_m=2\arctan \frac{\sqrt{3}a}{2 r_e}.
% \end{equation}
% We assume that the side length $a$ is much smaller than the coverage radius of satellites on the plane, $\gamma_m$ and $\gamma_m$ are both assumed much less than $\gamma$, which is the coverage angle of each LEO satellite.
\indent We assume that the side length $a$ is much smaller than the coverage radius of satellites on the plane, $\gamma_m$ is assumed much less than the coverage angle $\gamma$ of each LEO satellite.
\section{Proof of Lemma \ref{lem:criticalanalysis}}\label{app:criticalanalysis}
\indent Notice that, the upper bound \begin{equation}
\displaystyle N_c^U=\displaystyle\frac{\ln 2}{\ln 2-\ln(1+\cos(\gamma-2\arctan \frac{a}{2r_e}))}
\end{equation}
can be considered as an increasing function of $a$ and the lower bound
\begin{equation}
    N_c^L=\frac{\ln 2}{\ln 2-\ln(1+\cos(\gamma+2\arctan \frac{a}{2r_e}))}
\end{equation}
can be considered as a decreasing function of $a$. When the side length $a$ approaches 0, the limit values of the upper bound $N_c^{U}$ and lower bound $N_c^{L}$ are both approach to the same value:
\begin{equation}
\begin{array}{r@{}l}
\lim\limits_{a\rightarrow 0^+} N_c^U&=\lim\limits_{a\rightarrow 0^+}\frac{\ln 2}{\ln 2-\ln(1+\cos(\gamma-2\arctan \frac{a}{2r_e}))}\\
&=\frac{\ln 2}{\ln 2-\ln(1+\cos\gamma)}
\end{array}
\end{equation}
and 
\begin{equation}
\begin{array}{r@{}l}
\lim\limits_{a\rightarrow 0^+} N_c^L&=\lim\limits_{a\rightarrow 0^+}\frac{\ln 2}{\ln 2-\ln(1+\cos(\gamma+2\arctan \frac{a}{2r_e}))}\\
&=\frac{\ln 2}{\ln 2-\ln(1+\cos\gamma)}
\end{array}
\end{equation}

Because $N_c^{L}\leq N_c\leq N_c^U$, the limit value of $N_c$ should be the same as them, \ie
\begin{equation}
    N_c=\displaystyle\frac{\ln 2}{\ln 2-\ln(1+\cos\gamma)}.
\end{equation}

This is also the closed-form expression of critical number of LEO satellites which is always located between these two bounds. %Because the limit value of $N_c^L$ and $N_c^U$ are the same, the upper and lower bounds obtained through this stereographic projection method are both tight. However, they can not be used directly because we focus on continuous percolation on the sphere rather than the hexagonal lattic design on the plane, therefore, we only choose $a=0$ and adopt the critical number $N_c$.\\
% \indent After proving the expression of $N_c$, we need to verify that it should be strictly located between the lower bound $N_L$ and upper bound $N_U$.\\
% \indent First, we aim to prove that $N_L\leq N_c$. We aim to prove:
% \begin{equation}
%     \frac{\pi}{\gamma}<\frac{\ln 2}{\ln 2-\ln(1+\cos\gamma)}
% \label{LC}
% \end{equation}
% That is
% \begin{equation}
%     \pi \ln 2-\pi \ln(1+\cos\gamma)<2\gamma\ln 2
% \end{equation}
% Define that
% \begin{equation}
%     f(\gamma)=\pi \ln(1+\cos\gamma)+(2\gamma-\pi)\ln 2
% \end{equation}
% Take the derivative of $f(\gamma)$ with respect to $\gamma$:
% \begin{equation}
%     f'(\gamma)=-\frac{\pi\sin\gamma}{1+\cos\gamma}+2\ln 2
% \end{equation}
% The second derivative is
% \begin{equation}
%     f''(\gamma)=-\frac{\pi}{1+\cos\gamma}<0
% \end{equation}
% where $\gamma\in(0,\frac{\pi}{2})$.
% Therefore, $f'(\gamma)$ decreases as $\gamma$ increases, where $f'(\gamma)\in(2\ln2-\pi,2\ln2)$. So that $f(\gamma)$ increases first and then decreases as $\gamma$ increases, where $f(\gamma)>\min\{f(0),f(\frac{\pi}{2})\}=0$. That is the inequality in (\ref{LC}). Because the lower bound $N_L=\left \lfloor \frac{\pi}{2\gamma} \right \rfloor< \frac{\pi}{2\gamma}$, $N_L\leq N_c$. When $\gamma=0$, they both goes to infinity. When $\gamma=\frac{\pi}{2}$, they are both 1.\\
% \indent Next, we introduce a necessary condition of number of LEO satellites for full coverage. If $N<N_f$ where $N_f$ is the ratio of the area of the whole Earth to the area of each coverage area. It is easy to obtain because if we can achieve the full coverage, the sum of coverage area must be larger than the whole Earth. Now we want to prove that $N_c\leq N_f$. We have 
% \begin{equation}
% N_f=\frac{4\pi r_e^2}{2\pi r_e^2 (1-\cos\gamma)}=\frac{2}{1-\cos\gamma}
% \end{equation}
% To prove that
% \begin{equation}
%     \displaystyle\frac{\ln 2}{\ln 2-\ln(1+\cos\gamma)}<\frac{2}{1-\cos\gamma}
% \end{equation}
% That is:
% \begin{equation}
%     2\ln (1+\cos\gamma)<\ln 2(1+\cos\gamma).
% \end{equation}
% Because $\gamma\in(0,\frac{\pi}{2})$, define $x=1+\cos\gamma\in(1,2)$. Define
% \begin{equation}
%     g(x)=\frac{\ln x}{x}
% \end{equation}
% the first derivative is:
% \begin{equation}
%     g'(x)=\frac{1-\ln x}{x^2}>0,\;for\;x\in(1,2)
% \end{equation}
% We have
% \begin{equation}
%     g(x)<g(2)=\frac{\ln 2}{2}
% \end{equation}
% That is
% \begin{equation}
%     \frac{\ln (1+\cos\gamma)}{1+\cos\gamma}<\frac{\ln 2}{2}.
% \end{equation}
% Therefore, \begin{equation}
%     \displaystyle\frac{\ln 2}{\ln 2-\ln(1+\cos\gamma)}<\frac{2}{1-\cos\gamma}
% \end{equation}
% when $\gamma=0$, $N_c$ and $N_f$ both goes to infinite.\\

% Now, we need to prove that $N_f<N_U$. We know that
% \begin{equation}
%     N_U = \left \lceil \frac{\pi}{\gamma} \right \rceil  \left \lceil \frac{\pi}{\arccos{\frac{\cos\gamma}{\cos\frac{\gamma}{2}}}} \right \rceil> \frac{\pi}{\gamma}  \frac{\pi}{\arccos{\frac{\cos\gamma}{\cos\frac{\gamma}{2}}}}
% \end{equation}

% Because for $\gamma\in(0,\frac{\pi}{2})$,
% \begin{equation}
%    \frac{\cos\gamma}{\cos \frac{\gamma}{2}}>\cos\gamma
% \end{equation}
% and
% \begin{equation}
%     \frac{\cos\gamma}{\cos \frac{\gamma}{2}}=\frac{2\cos^2 \frac{\gamma}{2}-1}{\cos \frac{\gamma}{2}}\in(0,1)
% \end{equation}
% We have $\arccos{\frac{\cos\gamma}{\cos\frac{\gamma}{2}}}<\gamma$ and

% \begin{equation}
% \frac{\pi}{\gamma}  \frac{\pi}{\arccos{\frac{\cos\gamma}{\cos\frac{\gamma}{2}}}}>\frac{\pi^2}{\gamma^2}.
% \end{equation}
% So that, $N_u>\frac{\pi^2}{\gamma^2}$. To prove that
% \begin{equation}
%     \frac{\pi^2}{\gamma^2}>\frac{2}{1-\cos\gamma}
% \end{equation}
% We define that
% \begin{equation}
%     h(\gamma)=\pi^2(1-\cos\gamma)-2\gamma^2
% \end{equation}
% Take the derivative of $h(\gamma)$ with respect to $\gamma$:
% \begin{equation}
%     h'(\gamma)=\pi\sin\gamma-4\gamma
% \end{equation}
% The second derivative is:
% \begin{equation}
%     h^{(2)}(\gamma)=\pi^2\cos\gamma-4
% \end{equation}
% And the third derivative is:
% \begin{equation}
%     h^{(3)}(\gamma)=-\pi^2\sin\gamma<0
% \end{equation}
% As $\gamma$ increases, the the second derivative decreases from -1 to $\pi^2-4$. Therefore, as $\gamma$ increases, the first derivative increases first and then decreases, and
% \begin{equation}
%     h'(\gamma)>\min\bigg\{f(0),f(\frac{\pi}{2})\bigg\}=\min\bigg\{0,\pi^2-2\pi\bigg\}=0
% \end{equation}
% Therefore, $h(\gamma)>h(0)=0$, that is
% \begin{equation}
%     \frac{\pi^2}{\gamma^2}>\frac{2}{1-\cos\gamma}.
% \end{equation}
% And then we obtain $N_f<\pi^2/\gamma^2<N_U$. 

% \indent In conclusion, we have:
% \begin{equation}
%     N_L<N_c<N_f<N_U.
% \end{equation}


\ifCLASSOPTIONcaptionsoff
  \newpage
\fi

\bibliographystyle{IEEEtran}
% This must be in the first 5 lines to tell arXiv to use pdfLaTeX, which is strongly recommended.
\pdfoutput=1
% In particular, the hyperref package requires pdfLaTeX in order to break URLs across lines.

\documentclass[11pt]{article}

% Change "review" to "final" to generate the final (sometimes called camera-ready) version.
% Change to "preprint" to generate a non-anonymous version with page numbers.
\usepackage{acl}

% Standard package includes
\usepackage{times}
\usepackage{latexsym}

% Draw tables
\usepackage{booktabs}
\usepackage{multirow}
\usepackage{xcolor}
\usepackage{colortbl}
\usepackage{array} 
\usepackage{amsmath}

\newcolumntype{C}{>{\centering\arraybackslash}p{0.07\textwidth}}
% For proper rendering and hyphenation of words containing Latin characters (including in bib files)
\usepackage[T1]{fontenc}
% For Vietnamese characters
% \usepackage[T5]{fontenc}
% See https://www.latex-project.org/help/documentation/encguide.pdf for other character sets
% This assumes your files are encoded as UTF8
\usepackage[utf8]{inputenc}

% This is not strictly necessary, and may be commented out,
% but it will improve the layout of the manuscript,
% and will typically save some space.
\usepackage{microtype}
\DeclareMathOperator*{\argmax}{arg\,max}
% This is also not strictly necessary, and may be commented out.
% However, it will improve the aesthetics of text in
% the typewriter font.
\usepackage{inconsolata}

%Including images in your LaTeX document requires adding
%additional package(s)
\usepackage{graphicx}
% If the title and author information does not fit in the area allocated, uncomment the following
%
%\setlength\titlebox{<dim>}
%
% and set <dim> to something 5cm or larger.

\title{Wi-Chat: Large Language Model Powered Wi-Fi Sensing}

% Author information can be set in various styles:
% For several authors from the same institution:
% \author{Author 1 \and ... \and Author n \\
%         Address line \\ ... \\ Address line}
% if the names do not fit well on one line use
%         Author 1 \\ {\bf Author 2} \\ ... \\ {\bf Author n} \\
% For authors from different institutions:
% \author{Author 1 \\ Address line \\  ... \\ Address line
%         \And  ... \And
%         Author n \\ Address line \\ ... \\ Address line}
% To start a separate ``row'' of authors use \AND, as in
% \author{Author 1 \\ Address line \\  ... \\ Address line
%         \AND
%         Author 2 \\ Address line \\ ... \\ Address line \And
%         Author 3 \\ Address line \\ ... \\ Address line}

% \author{First Author \\
%   Affiliation / Address line 1 \\
%   Affiliation / Address line 2 \\
%   Affiliation / Address line 3 \\
%   \texttt{email@domain} \\\And
%   Second Author \\
%   Affiliation / Address line 1 \\
%   Affiliation / Address line 2 \\
%   Affiliation / Address line 3 \\
%   \texttt{email@domain} \\}
% \author{Haohan Yuan \qquad Haopeng Zhang\thanks{corresponding author} \\ 
%   ALOHA Lab, University of Hawaii at Manoa \\
%   % Affiliation / Address line 2 \\
%   % Affiliation / Address line 3 \\
%   \texttt{\{haohany,haopengz\}@hawaii.edu}}
  
\author{
{Haopeng Zhang$\dag$\thanks{These authors contributed equally to this work.}, Yili Ren$\ddagger$\footnotemark[1], Haohan Yuan$\dag$, Jingzhe Zhang$\ddagger$, Yitong Shen$\ddagger$} \\
ALOHA Lab, University of Hawaii at Manoa$\dag$, University of South Florida$\ddagger$ \\
\{haopengz, haohany\}@hawaii.edu\\
\{yiliren, jingzhe, shen202\}@usf.edu\\}



  
%\author{
%  \textbf{First Author\textsuperscript{1}},
%  \textbf{Second Author\textsuperscript{1,2}},
%  \textbf{Third T. Author\textsuperscript{1}},
%  \textbf{Fourth Author\textsuperscript{1}},
%\\
%  \textbf{Fifth Author\textsuperscript{1,2}},
%  \textbf{Sixth Author\textsuperscript{1}},
%  \textbf{Seventh Author\textsuperscript{1}},
%  \textbf{Eighth Author \textsuperscript{1,2,3,4}},
%\\
%  \textbf{Ninth Author\textsuperscript{1}},
%  \textbf{Tenth Author\textsuperscript{1}},
%  \textbf{Eleventh E. Author\textsuperscript{1,2,3,4,5}},
%  \textbf{Twelfth Author\textsuperscript{1}},
%\\
%  \textbf{Thirteenth Author\textsuperscript{3}},
%  \textbf{Fourteenth F. Author\textsuperscript{2,4}},
%  \textbf{Fifteenth Author\textsuperscript{1}},
%  \textbf{Sixteenth Author\textsuperscript{1}},
%\\
%  \textbf{Seventeenth S. Author\textsuperscript{4,5}},
%  \textbf{Eighteenth Author\textsuperscript{3,4}},
%  \textbf{Nineteenth N. Author\textsuperscript{2,5}},
%  \textbf{Twentieth Author\textsuperscript{1}}
%\\
%\\
%  \textsuperscript{1}Affiliation 1,
%  \textsuperscript{2}Affiliation 2,
%  \textsuperscript{3}Affiliation 3,
%  \textsuperscript{4}Affiliation 4,
%  \textsuperscript{5}Affiliation 5
%\\
%  \small{
%    \textbf{Correspondence:} \href{mailto:email@domain}{email@domain}
%  }
%}

\begin{document}
\maketitle
\begin{abstract}
Recent advancements in Large Language Models (LLMs) have demonstrated remarkable capabilities across diverse tasks. However, their potential to integrate physical model knowledge for real-world signal interpretation remains largely unexplored. In this work, we introduce Wi-Chat, the first LLM-powered Wi-Fi-based human activity recognition system. We demonstrate that LLMs can process raw Wi-Fi signals and infer human activities by incorporating Wi-Fi sensing principles into prompts. Our approach leverages physical model insights to guide LLMs in interpreting Channel State Information (CSI) data without traditional signal processing techniques. Through experiments on real-world Wi-Fi datasets, we show that LLMs exhibit strong reasoning capabilities, achieving zero-shot activity recognition. These findings highlight a new paradigm for Wi-Fi sensing, expanding LLM applications beyond conventional language tasks and enhancing the accessibility of wireless sensing for real-world deployments.
\end{abstract}

\section{Introduction}

In today’s rapidly evolving digital landscape, the transformative power of web technologies has redefined not only how services are delivered but also how complex tasks are approached. Web-based systems have become increasingly prevalent in risk control across various domains. This widespread adoption is due their accessibility, scalability, and ability to remotely connect various types of users. For example, these systems are used for process safety management in industry~\cite{kannan2016web}, safety risk early warning in urban construction~\cite{ding2013development}, and safe monitoring of infrastructural systems~\cite{repetto2018web}. Within these web-based risk management systems, the source search problem presents a huge challenge. Source search refers to the task of identifying the origin of a risky event, such as a gas leak and the emission point of toxic substances. This source search capability is crucial for effective risk management and decision-making.

Traditional approaches to implementing source search capabilities into the web systems often rely on solely algorithmic solutions~\cite{ristic2016study}. These methods, while relatively straightforward to implement, often struggle to achieve acceptable performances due to algorithmic local optima and complex unknown environments~\cite{zhao2020searching}. More recently, web crowdsourcing has emerged as a promising alternative for tackling the source search problem by incorporating human efforts in these web systems on-the-fly~\cite{zhao2024user}. This approach outsources the task of addressing issues encountered during the source search process to human workers, leveraging their capabilities to enhance system performance.

These solutions often employ a human-AI collaborative way~\cite{zhao2023leveraging} where algorithms handle exploration-exploitation and report the encountered problems while human workers resolve complex decision-making bottlenecks to help the algorithms getting rid of local deadlocks~\cite{zhao2022crowd}. Although effective, this paradigm suffers from two inherent limitations: increased operational costs from continuous human intervention, and slow response times of human workers due to sequential decision-making. These challenges motivate our investigation into developing autonomous systems that preserve human-like reasoning capabilities while reducing dependency on massive crowdsourced labor.

Furthermore, recent advancements in large language models (LLMs)~\cite{chang2024survey} and multi-modal LLMs (MLLMs)~\cite{huang2023chatgpt} have unveiled promising avenues for addressing these challenges. One clear opportunity involves the seamless integration of visual understanding and linguistic reasoning for robust decision-making in search tasks. However, whether large models-assisted source search is really effective and efficient for improving the current source search algorithms~\cite{ji2022source} remains unknown. \textit{To address the research gap, we are particularly interested in answering the following two research questions in this work:}

\textbf{\textit{RQ1: }}How can source search capabilities be integrated into web-based systems to support decision-making in time-sensitive risk management scenarios? 
% \sq{I mention ``time-sensitive'' here because I feel like we shall say something about the response time -- LLM has to be faster than humans}

\textbf{\textit{RQ2: }}How can MLLMs and LLMs enhance the effectiveness and efficiency of existing source search algorithms? 

% \textit{\textbf{RQ2:}} To what extent does the performance of large models-assisted search align with or approach the effectiveness of human-AI collaborative search? 

To answer the research questions, we propose a novel framework called Auto-\
S$^2$earch (\textbf{Auto}nomous \textbf{S}ource \textbf{Search}) and implement a prototype system that leverages advanced web technologies to simulate real-world conditions for zero-shot source search. Unlike traditional methods that rely on pre-defined heuristics or extensive human intervention, AutoS$^2$earch employs a carefully designed prompt that encapsulates human rationales, thereby guiding the MLLM to generate coherent and accurate scene descriptions from visual inputs about four directional choices. Based on these language-based descriptions, the LLM is enabled to determine the optimal directional choice through chain-of-thought (CoT) reasoning. Comprehensive empirical validation demonstrates that AutoS$^2$-\ 
earch achieves a success rate of 95–98\%, closely approaching the performance of human-AI collaborative search across 20 benchmark scenarios~\cite{zhao2023leveraging}. 

Our work indicates that the role of humans in future web crowdsourcing tasks may evolve from executors to validators or supervisors. Furthermore, incorporating explanations of LLM decisions into web-based system interfaces has the potential to help humans enhance task performance in risk control.






\section{Related Work}
\label{sec:relatedworks}

% \begin{table*}[t]
% \centering 
% \renewcommand\arraystretch{0.98}
% \fontsize{8}{10}\selectfont \setlength{\tabcolsep}{0.4em}
% \begin{tabular}{@{}lc|cc|cc|cc@{}}
% \toprule
% \textbf{Methods}           & \begin{tabular}[c]{@{}c@{}}\textbf{Training}\\ \textbf{Paradigm}\end{tabular} & \begin{tabular}[c]{@{}c@{}}\textbf{$\#$ PT Data}\\ \textbf{(Tokens)}\end{tabular} & \begin{tabular}[c]{@{}c@{}}\textbf{$\#$ IFT Data}\\ \textbf{(Samples)}\end{tabular} & \textbf{Code}  & \begin{tabular}[c]{@{}c@{}}\textbf{Natural}\\ \textbf{Language}\end{tabular} & \begin{tabular}[c]{@{}c@{}}\textbf{Action}\\ \textbf{Trajectories}\end{tabular} & \begin{tabular}[c]{@{}c@{}}\textbf{API}\\ \textbf{Documentation}\end{tabular}\\ \midrule 
% NexusRaven~\citep{srinivasan2023nexusraven} & IFT & - & - & \textcolor{green}{\CheckmarkBold} & \textcolor{green}{\CheckmarkBold} &\textcolor{red}{\XSolidBrush}&\textcolor{red}{\XSolidBrush}\\
% AgentInstruct~\citep{zeng2023agenttuning} & IFT & - & 2k & \textcolor{green}{\CheckmarkBold} & \textcolor{green}{\CheckmarkBold} &\textcolor{red}{\XSolidBrush}&\textcolor{red}{\XSolidBrush} \\
% AgentEvol~\citep{xi2024agentgym} & IFT & - & 14.5k & \textcolor{green}{\CheckmarkBold} & \textcolor{green}{\CheckmarkBold} &\textcolor{green}{\CheckmarkBold}&\textcolor{red}{\XSolidBrush} \\
% Gorilla~\citep{patil2023gorilla}& IFT & - & 16k & \textcolor{green}{\CheckmarkBold} & \textcolor{green}{\CheckmarkBold} &\textcolor{red}{\XSolidBrush}&\textcolor{green}{\CheckmarkBold}\\
% OpenFunctions-v2~\citep{patil2023gorilla} & IFT & - & 65k & \textcolor{green}{\CheckmarkBold} & \textcolor{green}{\CheckmarkBold} &\textcolor{red}{\XSolidBrush}&\textcolor{green}{\CheckmarkBold}\\
% LAM~\citep{zhang2024agentohana} & IFT & - & 42.6k & \textcolor{green}{\CheckmarkBold} & \textcolor{green}{\CheckmarkBold} &\textcolor{green}{\CheckmarkBold}&\textcolor{red}{\XSolidBrush} \\
% xLAM~\citep{liu2024apigen} & IFT & - & 60k & \textcolor{green}{\CheckmarkBold} & \textcolor{green}{\CheckmarkBold} &\textcolor{green}{\CheckmarkBold}&\textcolor{red}{\XSolidBrush} \\\midrule
% LEMUR~\citep{xu2024lemur} & PT & 90B & 300k & \textcolor{green}{\CheckmarkBold} & \textcolor{green}{\CheckmarkBold} &\textcolor{green}{\CheckmarkBold}&\textcolor{red}{\XSolidBrush}\\
% \rowcolor{teal!12} \method & PT & 103B & 95k & \textcolor{green}{\CheckmarkBold} & \textcolor{green}{\CheckmarkBold} & \textcolor{green}{\CheckmarkBold} & \textcolor{green}{\CheckmarkBold} \\
% \bottomrule
% \end{tabular}
% \caption{Summary of existing tuning- and pretraining-based LLM agents with their training sample sizes. "PT" and "IFT" denote "Pre-Training" and "Instruction Fine-Tuning", respectively. }
% \label{tab:related}
% \end{table*}

\begin{table*}[ht]
\begin{threeparttable}
\centering 
\renewcommand\arraystretch{0.98}
\fontsize{7}{9}\selectfont \setlength{\tabcolsep}{0.2em}
\begin{tabular}{@{}l|c|c|ccc|cc|cc|cccc@{}}
\toprule
\textbf{Methods} & \textbf{Datasets}           & \begin{tabular}[c]{@{}c@{}}\textbf{Training}\\ \textbf{Paradigm}\end{tabular} & \begin{tabular}[c]{@{}c@{}}\textbf{\# PT Data}\\ \textbf{(Tokens)}\end{tabular} & \begin{tabular}[c]{@{}c@{}}\textbf{\# IFT Data}\\ \textbf{(Samples)}\end{tabular} & \textbf{\# APIs} & \textbf{Code}  & \begin{tabular}[c]{@{}c@{}}\textbf{Nat.}\\ \textbf{Lang.}\end{tabular} & \begin{tabular}[c]{@{}c@{}}\textbf{Action}\\ \textbf{Traj.}\end{tabular} & \begin{tabular}[c]{@{}c@{}}\textbf{API}\\ \textbf{Doc.}\end{tabular} & \begin{tabular}[c]{@{}c@{}}\textbf{Func.}\\ \textbf{Call}\end{tabular} & \begin{tabular}[c]{@{}c@{}}\textbf{Multi.}\\ \textbf{Step}\end{tabular}  & \begin{tabular}[c]{@{}c@{}}\textbf{Plan}\\ \textbf{Refine}\end{tabular}  & \begin{tabular}[c]{@{}c@{}}\textbf{Multi.}\\ \textbf{Turn}\end{tabular}\\ \midrule 
\multicolumn{13}{l}{\emph{Instruction Finetuning-based LLM Agents for Intrinsic Reasoning}}  \\ \midrule
FireAct~\cite{chen2023fireact} & FireAct & IFT & - & 2.1K & 10 & \textcolor{red}{\XSolidBrush} &\textcolor{green}{\CheckmarkBold} &\textcolor{green}{\CheckmarkBold}  & \textcolor{red}{\XSolidBrush} &\textcolor{green}{\CheckmarkBold} & \textcolor{red}{\XSolidBrush} &\textcolor{green}{\CheckmarkBold} & \textcolor{red}{\XSolidBrush} \\
ToolAlpaca~\cite{tang2023toolalpaca} & ToolAlpaca & IFT & - & 4.0K & 400 & \textcolor{red}{\XSolidBrush} &\textcolor{green}{\CheckmarkBold} &\textcolor{green}{\CheckmarkBold} & \textcolor{red}{\XSolidBrush} &\textcolor{green}{\CheckmarkBold} & \textcolor{red}{\XSolidBrush}  &\textcolor{green}{\CheckmarkBold} & \textcolor{red}{\XSolidBrush}  \\
ToolLLaMA~\cite{qin2023toolllm} & ToolBench & IFT & - & 12.7K & 16,464 & \textcolor{red}{\XSolidBrush} &\textcolor{green}{\CheckmarkBold} &\textcolor{green}{\CheckmarkBold} &\textcolor{red}{\XSolidBrush} &\textcolor{green}{\CheckmarkBold}&\textcolor{green}{\CheckmarkBold}&\textcolor{green}{\CheckmarkBold} &\textcolor{green}{\CheckmarkBold}\\
AgentEvol~\citep{xi2024agentgym} & AgentTraj-L & IFT & - & 14.5K & 24 &\textcolor{red}{\XSolidBrush} & \textcolor{green}{\CheckmarkBold} &\textcolor{green}{\CheckmarkBold}&\textcolor{red}{\XSolidBrush} &\textcolor{green}{\CheckmarkBold}&\textcolor{red}{\XSolidBrush} &\textcolor{red}{\XSolidBrush} &\textcolor{green}{\CheckmarkBold}\\
Lumos~\cite{yin2024agent} & Lumos & IFT  & - & 20.0K & 16 &\textcolor{red}{\XSolidBrush} & \textcolor{green}{\CheckmarkBold} & \textcolor{green}{\CheckmarkBold} &\textcolor{red}{\XSolidBrush} & \textcolor{green}{\CheckmarkBold} & \textcolor{green}{\CheckmarkBold} &\textcolor{red}{\XSolidBrush} & \textcolor{green}{\CheckmarkBold}\\
Agent-FLAN~\cite{chen2024agent} & Agent-FLAN & IFT & - & 24.7K & 20 &\textcolor{red}{\XSolidBrush} & \textcolor{green}{\CheckmarkBold} & \textcolor{green}{\CheckmarkBold} &\textcolor{red}{\XSolidBrush} & \textcolor{green}{\CheckmarkBold}& \textcolor{green}{\CheckmarkBold}&\textcolor{red}{\XSolidBrush} & \textcolor{green}{\CheckmarkBold}\\
AgentTuning~\citep{zeng2023agenttuning} & AgentInstruct & IFT & - & 35.0K & - &\textcolor{red}{\XSolidBrush} & \textcolor{green}{\CheckmarkBold} & \textcolor{green}{\CheckmarkBold} &\textcolor{red}{\XSolidBrush} & \textcolor{green}{\CheckmarkBold} &\textcolor{red}{\XSolidBrush} &\textcolor{red}{\XSolidBrush} & \textcolor{green}{\CheckmarkBold}\\\midrule
\multicolumn{13}{l}{\emph{Instruction Finetuning-based LLM Agents for Function Calling}} \\\midrule
NexusRaven~\citep{srinivasan2023nexusraven} & NexusRaven & IFT & - & - & 116 & \textcolor{green}{\CheckmarkBold} & \textcolor{green}{\CheckmarkBold}  & \textcolor{green}{\CheckmarkBold} &\textcolor{red}{\XSolidBrush} & \textcolor{green}{\CheckmarkBold} &\textcolor{red}{\XSolidBrush} &\textcolor{red}{\XSolidBrush}&\textcolor{red}{\XSolidBrush}\\
Gorilla~\citep{patil2023gorilla} & Gorilla & IFT & - & 16.0K & 1,645 & \textcolor{green}{\CheckmarkBold} &\textcolor{red}{\XSolidBrush} &\textcolor{red}{\XSolidBrush}&\textcolor{green}{\CheckmarkBold} &\textcolor{green}{\CheckmarkBold} &\textcolor{red}{\XSolidBrush} &\textcolor{red}{\XSolidBrush} &\textcolor{red}{\XSolidBrush}\\
OpenFunctions-v2~\citep{patil2023gorilla} & OpenFunctions-v2 & IFT & - & 65.0K & - & \textcolor{green}{\CheckmarkBold} & \textcolor{green}{\CheckmarkBold} &\textcolor{red}{\XSolidBrush} &\textcolor{green}{\CheckmarkBold} &\textcolor{green}{\CheckmarkBold} &\textcolor{red}{\XSolidBrush} &\textcolor{red}{\XSolidBrush} &\textcolor{red}{\XSolidBrush}\\
API Pack~\cite{guo2024api} & API Pack & IFT & - & 1.1M & 11,213 &\textcolor{green}{\CheckmarkBold} &\textcolor{red}{\XSolidBrush} &\textcolor{green}{\CheckmarkBold} &\textcolor{red}{\XSolidBrush} &\textcolor{green}{\CheckmarkBold} &\textcolor{red}{\XSolidBrush}&\textcolor{red}{\XSolidBrush}&\textcolor{red}{\XSolidBrush}\\ 
LAM~\citep{zhang2024agentohana} & AgentOhana & IFT & - & 42.6K & - & \textcolor{green}{\CheckmarkBold} & \textcolor{green}{\CheckmarkBold} &\textcolor{green}{\CheckmarkBold}&\textcolor{red}{\XSolidBrush} &\textcolor{green}{\CheckmarkBold}&\textcolor{red}{\XSolidBrush}&\textcolor{green}{\CheckmarkBold}&\textcolor{green}{\CheckmarkBold}\\
xLAM~\citep{liu2024apigen} & APIGen & IFT & - & 60.0K & 3,673 & \textcolor{green}{\CheckmarkBold} & \textcolor{green}{\CheckmarkBold} &\textcolor{green}{\CheckmarkBold}&\textcolor{red}{\XSolidBrush} &\textcolor{green}{\CheckmarkBold}&\textcolor{red}{\XSolidBrush}&\textcolor{green}{\CheckmarkBold}&\textcolor{green}{\CheckmarkBold}\\\midrule
\multicolumn{13}{l}{\emph{Pretraining-based LLM Agents}}  \\\midrule
% LEMUR~\citep{xu2024lemur} & PT & 90B & 300.0K & - & \textcolor{green}{\CheckmarkBold} & \textcolor{green}{\CheckmarkBold} &\textcolor{green}{\CheckmarkBold}&\textcolor{red}{\XSolidBrush} & \textcolor{red}{\XSolidBrush} &\textcolor{green}{\CheckmarkBold} &\textcolor{red}{\XSolidBrush}&\textcolor{red}{\XSolidBrush}\\
\rowcolor{teal!12} \method & \dataset & PT & 103B & 95.0K  & 76,537  & \textcolor{green}{\CheckmarkBold} & \textcolor{green}{\CheckmarkBold} & \textcolor{green}{\CheckmarkBold} & \textcolor{green}{\CheckmarkBold} & \textcolor{green}{\CheckmarkBold} & \textcolor{green}{\CheckmarkBold} & \textcolor{green}{\CheckmarkBold} & \textcolor{green}{\CheckmarkBold}\\
\bottomrule
\end{tabular}
% \begin{tablenotes}
%     \item $^*$ In addition, the StarCoder-API can offer 4.77M more APIs.
% \end{tablenotes}
\caption{Summary of existing instruction finetuning-based LLM agents for intrinsic reasoning and function calling, along with their training resources and sample sizes. "PT" and "IFT" denote "Pre-Training" and "Instruction Fine-Tuning", respectively.}
\vspace{-2ex}
\label{tab:related}
\end{threeparttable}
\end{table*}

\noindent \textbf{Prompting-based LLM Agents.} Due to the lack of agent-specific pre-training corpus, existing LLM agents rely on either prompt engineering~\cite{hsieh2023tool,lu2024chameleon,yao2022react,wang2023voyager} or instruction fine-tuning~\cite{chen2023fireact,zeng2023agenttuning} to understand human instructions, decompose high-level tasks, generate grounded plans, and execute multi-step actions. 
However, prompting-based methods mainly depend on the capabilities of backbone LLMs (usually commercial LLMs), failing to introduce new knowledge and struggling to generalize to unseen tasks~\cite{sun2024adaplanner,zhuang2023toolchain}. 

\noindent \textbf{Instruction Finetuning-based LLM Agents.} Considering the extensive diversity of APIs and the complexity of multi-tool instructions, tool learning inherently presents greater challenges than natural language tasks, such as text generation~\cite{qin2023toolllm}.
Post-training techniques focus more on instruction following and aligning output with specific formats~\cite{patil2023gorilla,hao2024toolkengpt,qin2023toolllm,schick2024toolformer}, rather than fundamentally improving model knowledge or capabilities. 
Moreover, heavy fine-tuning can hinder generalization or even degrade performance in non-agent use cases, potentially suppressing the original base model capabilities~\cite{ghosh2024a}.

\noindent \textbf{Pretraining-based LLM Agents.} While pre-training serves as an essential alternative, prior works~\cite{nijkamp2023codegen,roziere2023code,xu2024lemur,patil2023gorilla} have primarily focused on improving task-specific capabilities (\eg, code generation) instead of general-domain LLM agents, due to single-source, uni-type, small-scale, and poor-quality pre-training data. 
Existing tool documentation data for agent training either lacks diverse real-world APIs~\cite{patil2023gorilla, tang2023toolalpaca} or is constrained to single-tool or single-round tool execution. 
Furthermore, trajectory data mostly imitate expert behavior or follow function-calling rules with inferior planning and reasoning, failing to fully elicit LLMs' capabilities and handle complex instructions~\cite{qin2023toolllm}. 
Given a wide range of candidate API functions, each comprising various function names and parameters available at every planning step, identifying globally optimal solutions and generalizing across tasks remains highly challenging.



\section{Preliminaries}
\label{Preliminaries}
\begin{figure*}[t]
    \centering
    \includegraphics[width=0.95\linewidth]{fig/HealthGPT_Framework.png}
    \caption{The \ourmethod{} architecture integrates hierarchical visual perception and H-LoRA, employing a task-specific hard router to select visual features and H-LoRA plugins, ultimately generating outputs with an autoregressive manner.}
    \label{fig:architecture}
\end{figure*}
\noindent\textbf{Large Vision-Language Models.} 
The input to a LVLM typically consists of an image $x^{\text{img}}$ and a discrete text sequence $x^{\text{txt}}$. The visual encoder $\mathcal{E}^{\text{img}}$ converts the input image $x^{\text{img}}$ into a sequence of visual tokens $\mathcal{V} = [v_i]_{i=1}^{N_v}$, while the text sequence $x^{\text{txt}}$ is mapped into a sequence of text tokens $\mathcal{T} = [t_i]_{i=1}^{N_t}$ using an embedding function $\mathcal{E}^{\text{txt}}$. The LLM $\mathcal{M_\text{LLM}}(\cdot|\theta)$ models the joint probability of the token sequence $\mathcal{U} = \{\mathcal{V},\mathcal{T}\}$, which is expressed as:
\begin{equation}
    P_\theta(R | \mathcal{U}) = \prod_{i=1}^{N_r} P_\theta(r_i | \{\mathcal{U}, r_{<i}\}),
\end{equation}
where $R = [r_i]_{i=1}^{N_r}$ is the text response sequence. The LVLM iteratively generates the next token $r_i$ based on $r_{<i}$. The optimization objective is to minimize the cross-entropy loss of the response $\mathcal{R}$.
% \begin{equation}
%     \mathcal{L}_{\text{VLM}} = \mathbb{E}_{R|\mathcal{U}}\left[-\log P_\theta(R | \mathcal{U})\right]
% \end{equation}
It is worth noting that most LVLMs adopt a design paradigm based on ViT, alignment adapters, and pre-trained LLMs\cite{liu2023llava,liu2024improved}, enabling quick adaptation to downstream tasks.


\noindent\textbf{VQGAN.}
VQGAN~\cite{esser2021taming} employs latent space compression and indexing mechanisms to effectively learn a complete discrete representation of images. VQGAN first maps the input image $x^{\text{img}}$ to a latent representation $z = \mathcal{E}(x)$ through a encoder $\mathcal{E}$. Then, the latent representation is quantized using a codebook $\mathcal{Z} = \{z_k\}_{k=1}^K$, generating a discrete index sequence $\mathcal{I} = [i_m]_{m=1}^N$, where $i_m \in \mathcal{Z}$ represents the quantized code index:
\begin{equation}
    \mathcal{I} = \text{Quantize}(z|\mathcal{Z}) = \arg\min_{z_k \in \mathcal{Z}} \| z - z_k \|_2.
\end{equation}
In our approach, the discrete index sequence $\mathcal{I}$ serves as a supervisory signal for the generation task, enabling the model to predict the index sequence $\hat{\mathcal{I}}$ from input conditions such as text or other modality signals.  
Finally, the predicted index sequence $\hat{\mathcal{I}}$ is upsampled by the VQGAN decoder $G$, generating the high-quality image $\hat{x}^\text{img} = G(\hat{\mathcal{I}})$.



\noindent\textbf{Low Rank Adaptation.} 
LoRA\cite{hu2021lora} effectively captures the characteristics of downstream tasks by introducing low-rank adapters. The core idea is to decompose the bypass weight matrix $\Delta W\in\mathbb{R}^{d^{\text{in}} \times d^{\text{out}}}$ into two low-rank matrices $ \{A \in \mathbb{R}^{d^{\text{in}} \times r}, B \in \mathbb{R}^{r \times d^{\text{out}}} \}$, where $ r \ll \min\{d^{\text{in}}, d^{\text{out}}\} $, significantly reducing learnable parameters. The output with the LoRA adapter for the input $x$ is then given by:
\begin{equation}
    h = x W_0 + \alpha x \Delta W/r = x W_0 + \alpha xAB/r,
\end{equation}
where matrix $ A $ is initialized with a Gaussian distribution, while the matrix $ B $ is initialized as a zero matrix. The scaling factor $ \alpha/r $ controls the impact of $ \Delta W $ on the model.

\section{HealthGPT}
\label{Method}


\subsection{Unified Autoregressive Generation.}  
% As shown in Figure~\ref{fig:architecture}, 
\ourmethod{} (Figure~\ref{fig:architecture}) utilizes a discrete token representation that covers both text and visual outputs, unifying visual comprehension and generation as an autoregressive task. 
For comprehension, $\mathcal{M}_\text{llm}$ receives the input joint sequence $\mathcal{U}$ and outputs a series of text token $\mathcal{R} = [r_1, r_2, \dots, r_{N_r}]$, where $r_i \in \mathcal{V}_{\text{txt}}$, and $\mathcal{V}_{\text{txt}}$ represents the LLM's vocabulary:
\begin{equation}
    P_\theta(\mathcal{R} \mid \mathcal{U}) = \prod_{i=1}^{N_r} P_\theta(r_i \mid \mathcal{U}, r_{<i}).
\end{equation}
For generation, $\mathcal{M}_\text{llm}$ first receives a special start token $\langle \text{START\_IMG} \rangle$, then generates a series of tokens corresponding to the VQGAN indices $\mathcal{I} = [i_1, i_2, \dots, i_{N_i}]$, where $i_j \in \mathcal{V}_{\text{vq}}$, and $\mathcal{V}_{\text{vq}}$ represents the index range of VQGAN. Upon completion of generation, the LLM outputs an end token $\langle \text{END\_IMG} \rangle$:
\begin{equation}
    P_\theta(\mathcal{I} \mid \mathcal{U}) = \prod_{j=1}^{N_i} P_\theta(i_j \mid \mathcal{U}, i_{<j}).
\end{equation}
Finally, the generated index sequence $\mathcal{I}$ is fed into the decoder $G$, which reconstructs the target image $\hat{x}^{\text{img}} = G(\mathcal{I})$.

\subsection{Hierarchical Visual Perception}  
Given the differences in visual perception between comprehension and generation tasks—where the former focuses on abstract semantics and the latter emphasizes complete semantics—we employ ViT to compress the image into discrete visual tokens at multiple hierarchical levels.
Specifically, the image is converted into a series of features $\{f_1, f_2, \dots, f_L\}$ as it passes through $L$ ViT blocks.

To address the needs of various tasks, the hidden states are divided into two types: (i) \textit{Concrete-grained features} $\mathcal{F}^{\text{Con}} = \{f_1, f_2, \dots, f_k\}, k < L$, derived from the shallower layers of ViT, containing sufficient global features, suitable for generation tasks; 
(ii) \textit{Abstract-grained features} $\mathcal{F}^{\text{Abs}} = \{f_{k+1}, f_{k+2}, \dots, f_L\}$, derived from the deeper layers of ViT, which contain abstract semantic information closer to the text space, suitable for comprehension tasks.

The task type $T$ (comprehension or generation) determines which set of features is selected as the input for the downstream large language model:
\begin{equation}
    \mathcal{F}^{\text{img}}_T =
    \begin{cases}
        \mathcal{F}^{\text{Con}}, & \text{if } T = \text{generation task} \\
        \mathcal{F}^{\text{Abs}}, & \text{if } T = \text{comprehension task}
    \end{cases}
\end{equation}
We integrate the image features $\mathcal{F}^{\text{img}}_T$ and text features $\mathcal{T}$ into a joint sequence through simple concatenation, which is then fed into the LLM $\mathcal{M}_{\text{llm}}$ for autoregressive generation.
% :
% \begin{equation}
%     \mathcal{R} = \mathcal{M}_{\text{llm}}(\mathcal{U}|\theta), \quad \mathcal{U} = [\mathcal{F}^{\text{img}}_T; \mathcal{T}]
% \end{equation}
\subsection{Heterogeneous Knowledge Adaptation}
We devise H-LoRA, which stores heterogeneous knowledge from comprehension and generation tasks in separate modules and dynamically routes to extract task-relevant knowledge from these modules. 
At the task level, for each task type $ T $, we dynamically assign a dedicated H-LoRA submodule $ \theta^T $, which is expressed as:
\begin{equation}
    \mathcal{R} = \mathcal{M}_\text{LLM}(\mathcal{U}|\theta, \theta^T), \quad \theta^T = \{A^T, B^T, \mathcal{R}^T_\text{outer}\}.
\end{equation}
At the feature level for a single task, H-LoRA integrates the idea of Mixture of Experts (MoE)~\cite{masoudnia2014mixture} and designs an efficient matrix merging and routing weight allocation mechanism, thus avoiding the significant computational delay introduced by matrix splitting in existing MoELoRA~\cite{luo2024moelora}. Specifically, we first merge the low-rank matrices (rank = r) of $ k $ LoRA experts into a unified matrix:
\begin{equation}
    \mathbf{A}^{\text{merged}}, \mathbf{B}^{\text{merged}} = \text{Concat}(\{A_i\}_1^k), \text{Concat}(\{B_i\}_1^k),
\end{equation}
where $ \mathbf{A}^{\text{merged}} \in \mathbb{R}^{d^\text{in} \times rk} $ and $ \mathbf{B}^{\text{merged}} \in \mathbb{R}^{rk \times d^\text{out}} $. The $k$-dimension routing layer generates expert weights $ \mathcal{W} \in \mathbb{R}^{\text{token\_num} \times k} $ based on the input hidden state $ x $, and these are expanded to $ \mathbb{R}^{\text{token\_num} \times rk} $ as follows:
\begin{equation}
    \mathcal{W}^\text{expanded} = \alpha k \mathcal{W} / r \otimes \mathbf{1}_r,
\end{equation}
where $ \otimes $ denotes the replication operation.
The overall output of H-LoRA is computed as:
\begin{equation}
    \mathcal{O}^\text{H-LoRA} = (x \mathbf{A}^{\text{merged}} \odot \mathcal{W}^\text{expanded}) \mathbf{B}^{\text{merged}},
\end{equation}
where $ \odot $ represents element-wise multiplication. Finally, the output of H-LoRA is added to the frozen pre-trained weights to produce the final output:
\begin{equation}
    \mathcal{O} = x W_0 + \mathcal{O}^\text{H-LoRA}.
\end{equation}
% In summary, H-LoRA is a task-based dynamic PEFT method that achieves high efficiency in single-task fine-tuning.

\subsection{Training Pipeline}

\begin{figure}[t]
    \centering
    \hspace{-4mm}
    \includegraphics[width=0.94\linewidth]{fig/data.pdf}
    \caption{Data statistics of \texttt{VL-Health}. }
    \label{fig:data}
\end{figure}
\noindent \textbf{1st Stage: Multi-modal Alignment.} 
In the first stage, we design separate visual adapters and H-LoRA submodules for medical unified tasks. For the medical comprehension task, we train abstract-grained visual adapters using high-quality image-text pairs to align visual embeddings with textual embeddings, thereby enabling the model to accurately describe medical visual content. During this process, the pre-trained LLM and its corresponding H-LoRA submodules remain frozen. In contrast, the medical generation task requires training concrete-grained adapters and H-LoRA submodules while keeping the LLM frozen. Meanwhile, we extend the textual vocabulary to include multimodal tokens, enabling the support of additional VQGAN vector quantization indices. The model trains on image-VQ pairs, endowing the pre-trained LLM with the capability for image reconstruction. This design ensures pixel-level consistency of pre- and post-LVLM. The processes establish the initial alignment between the LLM’s outputs and the visual inputs.

\noindent \textbf{2nd Stage: Heterogeneous H-LoRA Plugin Adaptation.}  
The submodules of H-LoRA share the word embedding layer and output head but may encounter issues such as bias and scale inconsistencies during training across different tasks. To ensure that the multiple H-LoRA plugins seamlessly interface with the LLMs and form a unified base, we fine-tune the word embedding layer and output head using a small amount of mixed data to maintain consistency in the model weights. Specifically, during this stage, all H-LoRA submodules for different tasks are kept frozen, with only the word embedding layer and output head being optimized. Through this stage, the model accumulates foundational knowledge for unified tasks by adapting H-LoRA plugins.

\begin{table*}[!t]
\centering
\caption{Comparison of \ourmethod{} with other LVLMs and unified multi-modal models on medical visual comprehension tasks. \textbf{Bold} and \underline{underlined} text indicates the best performance and second-best performance, respectively.}
\resizebox{\textwidth}{!}{
\begin{tabular}{c|lcc|cccccccc|c}
\toprule
\rowcolor[HTML]{E9F3FE} &  &  &  & \multicolumn{2}{c}{\textbf{VQA-RAD \textuparrow}} & \multicolumn{2}{c}{\textbf{SLAKE \textuparrow}} & \multicolumn{2}{c}{\textbf{PathVQA \textuparrow}} &  &  &  \\ 
\cline{5-10}
\rowcolor[HTML]{E9F3FE}\multirow{-2}{*}{\textbf{Type}} & \multirow{-2}{*}{\textbf{Model}} & \multirow{-2}{*}{\textbf{\# Params}} & \multirow{-2}{*}{\makecell{\textbf{Medical} \\ \textbf{LVLM}}} & \textbf{close} & \textbf{all} & \textbf{close} & \textbf{all} & \textbf{close} & \textbf{all} & \multirow{-2}{*}{\makecell{\textbf{MMMU} \\ \textbf{-Med}}\textuparrow} & \multirow{-2}{*}{\textbf{OMVQA}\textuparrow} & \multirow{-2}{*}{\textbf{Avg. \textuparrow}} \\ 
\midrule \midrule
\multirow{9}{*}{\textbf{Comp. Only}} 
& Med-Flamingo & 8.3B & \Large \ding{51} & 58.6 & 43.0 & 47.0 & 25.5 & 61.9 & 31.3 & 28.7 & 34.9 & 41.4 \\
& LLaVA-Med & 7B & \Large \ding{51} & 60.2 & 48.1 & 58.4 & 44.8 & 62.3 & 35.7 & 30.0 & 41.3 & 47.6 \\
& HuatuoGPT-Vision & 7B & \Large \ding{51} & 66.9 & 53.0 & 59.8 & 49.1 & 52.9 & 32.0 & 42.0 & 50.0 & 50.7 \\
& BLIP-2 & 6.7B & \Large \ding{55} & 43.4 & 36.8 & 41.6 & 35.3 & 48.5 & 28.8 & 27.3 & 26.9 & 36.1 \\
& LLaVA-v1.5 & 7B & \Large \ding{55} & 51.8 & 42.8 & 37.1 & 37.7 & 53.5 & 31.4 & 32.7 & 44.7 & 41.5 \\
& InstructBLIP & 7B & \Large \ding{55} & 61.0 & 44.8 & 66.8 & 43.3 & 56.0 & 32.3 & 25.3 & 29.0 & 44.8 \\
& Yi-VL & 6B & \Large \ding{55} & 52.6 & 42.1 & 52.4 & 38.4 & 54.9 & 30.9 & 38.0 & 50.2 & 44.9 \\
& InternVL2 & 8B & \Large \ding{55} & 64.9 & 49.0 & 66.6 & 50.1 & 60.0 & 31.9 & \underline{43.3} & 54.5 & 52.5\\
& Llama-3.2 & 11B & \Large \ding{55} & 68.9 & 45.5 & 72.4 & 52.1 & 62.8 & 33.6 & 39.3 & 63.2 & 54.7 \\
\midrule
\multirow{5}{*}{\textbf{Comp. \& Gen.}} 
& Show-o & 1.3B & \Large \ding{55} & 50.6 & 33.9 & 31.5 & 17.9 & 52.9 & 28.2 & 22.7 & 45.7 & 42.6 \\
& Unified-IO 2 & 7B & \Large \ding{55} & 46.2 & 32.6 & 35.9 & 21.9 & 52.5 & 27.0 & 25.3 & 33.0 & 33.8 \\
& Janus & 1.3B & \Large \ding{55} & 70.9 & 52.8 & 34.7 & 26.9 & 51.9 & 27.9 & 30.0 & 26.8 & 33.5 \\
& \cellcolor[HTML]{DAE0FB}HealthGPT-M3 & \cellcolor[HTML]{DAE0FB}3.8B & \cellcolor[HTML]{DAE0FB}\Large \ding{51} & \cellcolor[HTML]{DAE0FB}\underline{73.7} & \cellcolor[HTML]{DAE0FB}\underline{55.9} & \cellcolor[HTML]{DAE0FB}\underline{74.6} & \cellcolor[HTML]{DAE0FB}\underline{56.4} & \cellcolor[HTML]{DAE0FB}\underline{78.7} & \cellcolor[HTML]{DAE0FB}\underline{39.7} & \cellcolor[HTML]{DAE0FB}\underline{43.3} & \cellcolor[HTML]{DAE0FB}\underline{68.5} & \cellcolor[HTML]{DAE0FB}\underline{61.3} \\
& \cellcolor[HTML]{DAE0FB}HealthGPT-L14 & \cellcolor[HTML]{DAE0FB}14B & \cellcolor[HTML]{DAE0FB}\Large \ding{51} & \cellcolor[HTML]{DAE0FB}\textbf{77.7} & \cellcolor[HTML]{DAE0FB}\textbf{58.3} & \cellcolor[HTML]{DAE0FB}\textbf{76.4} & \cellcolor[HTML]{DAE0FB}\textbf{64.5} & \cellcolor[HTML]{DAE0FB}\textbf{85.9} & \cellcolor[HTML]{DAE0FB}\textbf{44.4} & \cellcolor[HTML]{DAE0FB}\textbf{49.2} & \cellcolor[HTML]{DAE0FB}\textbf{74.4} & \cellcolor[HTML]{DAE0FB}\textbf{66.4} \\
\bottomrule
\end{tabular}
}
\label{tab:results}
\end{table*}
\begin{table*}[ht]
    \centering
    \caption{The experimental results for the four modality conversion tasks.}
    \resizebox{\textwidth}{!}{
    \begin{tabular}{l|ccc|ccc|ccc|ccc}
        \toprule
        \rowcolor[HTML]{E9F3FE} & \multicolumn{3}{c}{\textbf{CT to MRI (Brain)}} & \multicolumn{3}{c}{\textbf{CT to MRI (Pelvis)}} & \multicolumn{3}{c}{\textbf{MRI to CT (Brain)}} & \multicolumn{3}{c}{\textbf{MRI to CT (Pelvis)}} \\
        \cline{2-13}
        \rowcolor[HTML]{E9F3FE}\multirow{-2}{*}{\textbf{Model}}& \textbf{SSIM $\uparrow$} & \textbf{PSNR $\uparrow$} & \textbf{MSE $\downarrow$} & \textbf{SSIM $\uparrow$} & \textbf{PSNR $\uparrow$} & \textbf{MSE $\downarrow$} & \textbf{SSIM $\uparrow$} & \textbf{PSNR $\uparrow$} & \textbf{MSE $\downarrow$} & \textbf{SSIM $\uparrow$} & \textbf{PSNR $\uparrow$} & \textbf{MSE $\downarrow$} \\
        \midrule \midrule
        pix2pix & 71.09 & 32.65 & 36.85 & 59.17 & 31.02 & 51.91 & 78.79 & 33.85 & 28.33 & 72.31 & 32.98 & 36.19 \\
        CycleGAN & 54.76 & 32.23 & 40.56 & 54.54 & 30.77 & 55.00 & 63.75 & 31.02 & 52.78 & 50.54 & 29.89 & 67.78 \\
        BBDM & {71.69} & {32.91} & {34.44} & 57.37 & 31.37 & 48.06 & \textbf{86.40} & 34.12 & 26.61 & {79.26} & 33.15 & 33.60 \\
        Vmanba & 69.54 & 32.67 & 36.42 & {63.01} & {31.47} & {46.99} & 79.63 & 34.12 & 26.49 & 77.45 & 33.53 & 31.85 \\
        DiffMa & 71.47 & 32.74 & 35.77 & 62.56 & 31.43 & 47.38 & 79.00 & {34.13} & {26.45} & 78.53 & {33.68} & {30.51} \\
        \rowcolor[HTML]{DAE0FB}HealthGPT-M3 & \underline{79.38} & \underline{33.03} & \underline{33.48} & \underline{71.81} & \underline{31.83} & \underline{43.45} & {85.06} & \textbf{34.40} & \textbf{25.49} & \underline{84.23} & \textbf{34.29} & \textbf{27.99} \\
        \rowcolor[HTML]{DAE0FB}HealthGPT-L14 & \textbf{79.73} & \textbf{33.10} & \textbf{32.96} & \textbf{71.92} & \textbf{31.87} & \textbf{43.09} & \underline{85.31} & \underline{34.29} & \underline{26.20} & \textbf{84.96} & \underline{34.14} & \underline{28.13} \\
        \bottomrule
    \end{tabular}
    }
    \label{tab:conversion}
\end{table*}

\noindent \textbf{3rd Stage: Visual Instruction Fine-Tuning.}  
In the third stage, we introduce additional task-specific data to further optimize the model and enhance its adaptability to downstream tasks such as medical visual comprehension (e.g., medical QA, medical dialogues, and report generation) or generation tasks (e.g., super-resolution, denoising, and modality conversion). Notably, by this stage, the word embedding layer and output head have been fine-tuned, only the H-LoRA modules and adapter modules need to be trained. This strategy significantly improves the model's adaptability and flexibility across different tasks.


\section{Experiment}
\label{s:experiment}

\subsection{Data Description}
We evaluate our method on FI~\cite{you2016building}, Twitter\_LDL~\cite{yang2017learning} and Artphoto~\cite{machajdik2010affective}.
FI is a public dataset built from Flickr and Instagram, with 23,308 images and eight emotion categories, namely \textit{amusement}, \textit{anger}, \textit{awe},  \textit{contentment}, \textit{disgust}, \textit{excitement},  \textit{fear}, and \textit{sadness}. 
% Since images in FI are all copyrighted by law, some images are corrupted now, so we remove these samples and retain 21,828 images.
% T4SA contains images from Twitter, which are classified into three categories: \textit{positive}, \textit{neutral}, and \textit{negative}. In this paper, we adopt the base version of B-T4SA, which contains 470,586 images and provides text descriptions of the corresponding tweets.
Twitter\_LDL contains 10,045 images from Twitter, with the same eight categories as the FI dataset.
% 。
For these two datasets, they are randomly split into 80\%
training and 20\% testing set.
Artphoto contains 806 artistic photos from the DeviantArt website, which we use to further evaluate the zero-shot capability of our model.
% on the small-scale dataset.
% We construct and publicly release the first image sentiment analysis dataset containing metadata.
% 。

% Based on these datasets, we are the first to construct and publicly release metadata-enhanced image sentiment analysis datasets. These datasets include scenes, tags, descriptions, and corresponding confidence scores, and are available at this link for future research purposes.


% 
\begin{table}[t]
\centering
% \begin{center}
\caption{Overall performance of different models on FI and Twitter\_LDL datasets.}
\label{tab:cap1}
% \resizebox{\linewidth}{!}
{
\begin{tabular}{l|c|c|c|c}
\hline
\multirow{2}{*}{\textbf{Model}} & \multicolumn{2}{c|}{\textbf{FI}}  & \multicolumn{2}{c}{\textbf{Twitter\_LDL}} \\ \cline{2-5} 
  & \textbf{Accuracy} & \textbf{F1} & \textbf{Accuracy} & \textbf{F1}  \\ \hline
% (\rownumber)~AlexNet~\cite{krizhevsky2017imagenet}  & 58.13\% & 56.35\%  & 56.24\%& 55.02\%  \\ 
% (\rownumber)~VGG16~\cite{simonyan2014very}  & 63.75\%& 63.08\%  & 59.34\%& 59.02\%  \\ 
(\rownumber)~ResNet101~\cite{he2016deep} & 66.16\%& 65.56\%  & 62.02\% & 61.34\%  \\ 
(\rownumber)~CDA~\cite{han2023boosting} & 66.71\%& 65.37\%  & 64.14\% & 62.85\%  \\ 
(\rownumber)~CECCN~\cite{ruan2024color} & 67.96\%& 66.74\%  & 64.59\%& 64.72\% \\ 
(\rownumber)~EmoVIT~\cite{xie2024emovit} & 68.09\%& 67.45\%  & 63.12\% & 61.97\%  \\ 
(\rownumber)~ComLDL~\cite{zhang2022compound} & 68.83\%& 67.28\%  & 65.29\% & 63.12\%  \\ 
(\rownumber)~WSDEN~\cite{li2023weakly} & 69.78\%& 69.61\%  & 67.04\% & 65.49\% \\ 
(\rownumber)~ECWA~\cite{deng2021emotion} & 70.87\%& 69.08\%  & 67.81\% & 66.87\%  \\ 
(\rownumber)~EECon~\cite{yang2023exploiting} & 71.13\%& 68.34\%  & 64.27\%& 63.16\%  \\ 
(\rownumber)~MAM~\cite{zhang2024affective} & 71.44\%  & 70.83\% & 67.18\%  & 65.01\%\\ 
(\rownumber)~TGCA-PVT~\cite{chen2024tgca}   & 73.05\%  & 71.46\% & 69.87\%  & 68.32\% \\ 
(\rownumber)~OEAN~\cite{zhang2024object}   & 73.40\%  & 72.63\% & 70.52\%  & 69.47\% \\ \hline
(\rownumber)~\shortname  & \textbf{79.48\%} & \textbf{79.22\%} & \textbf{74.12\%} & \textbf{73.09\%} \\ \hline
\end{tabular}
}
\vspace{-6mm}
% \end{center}
\end{table}
% 

\subsection{Experiment Setting}
% \subsubsection{Model Setting.}
% 
\textbf{Model Setting:}
For feature representation, we set $k=10$ to select object tags, and adopt clip-vit-base-patch32 as the pre-trained model for unified feature representation.
Moreover, we empirically set $(d_e, d_h, d_k, d_s) = (512, 128, 16, 64)$, and set the classification class $L$ to 8.

% 

\textbf{Training Setting:}
To initialize the model, we set all weights such as $\boldsymbol{W}$ following the truncated normal distribution, and use AdamW optimizer with the learning rate of $1 \times 10^{-4}$.
% warmup scheduler of cosine, warmup steps of 2000.
Furthermore, we set the batch size to 32 and the epoch of the training process to 200.
During the implementation, we utilize \textit{PyTorch} to build our entire model.
% , and our project codes are publicly available at https://github.com/zzmyrep/MESN.
% Our project codes as well as data are all publicly available on GitHub\footnote{https://github.com/zzmyrep/KBCEN}.
% Code is available at \href{https://github.com/zzmyrep/KBCEN}{https://github.com/zzmyrep/KBCEN}.

\textbf{Evaluation Metrics:}
Following~\cite{zhang2024affective, chen2024tgca, zhang2024object}, we adopt \textit{accuracy} and \textit{F1} as our evaluation metrics to measure the performance of different methods for image sentiment analysis. 



\subsection{Experiment Result}
% We compare our model against the following baselines: AlexNet~\cite{krizhevsky2017imagenet}, VGG16~\cite{simonyan2014very}, ResNet101~\cite{he2016deep}, CECCN~\cite{ruan2024color}, EmoVIT~\cite{xie2024emovit}, WSCNet~\cite{yang2018weakly}, ECWA~\cite{deng2021emotion}, EECon~\cite{yang2023exploiting}, MAM~\cite{zhang2024affective} and TGCA-PVT~\cite{chen2024tgca}, and the overall results are summarized in Table~\ref{tab:cap1}.
We compare our model against several baselines, and the overall results are summarized in Table~\ref{tab:cap1}.
We observe that our model achieves the best performance in both accuracy and F1 metrics, significantly outperforming the previous models. 
This superior performance is mainly attributed to our effective utilization of metadata to enhance image sentiment analysis, as well as the exceptional capability of the unified sentiment transformer framework we developed. These results strongly demonstrate that our proposed method can bring encouraging performance for image sentiment analysis.

\setcounter{magicrownumbers}{0} 
\begin{table}[t]
\begin{center}
\caption{Ablation study of~\shortname~on FI dataset.} 
% \vspace{1mm}
\label{tab:cap2}
\resizebox{.9\linewidth}{!}
{
\begin{tabular}{lcc}
  \hline
  \textbf{Model} & \textbf{Accuracy} & \textbf{F1} \\
  \hline
  (\rownumber)~Ours (w/o vision) & 65.72\% & 64.54\% \\
  (\rownumber)~Ours (w/o text description) & 74.05\% & 72.58\% \\
  (\rownumber)~Ours (w/o object tag) & 77.45\% & 76.84\% \\
  (\rownumber)~Ours (w/o scene tag) & 78.47\% & 78.21\% \\
  \hline
  (\rownumber)~Ours (w/o unified embedding) & 76.41\% & 76.23\% \\
  (\rownumber)~Ours (w/o adaptive learning) & 76.83\% & 76.56\% \\
  (\rownumber)~Ours (w/o cross-modal fusion) & 76.85\% & 76.49\% \\
  \hline
  (\rownumber)~Ours  & \textbf{79.48\%} & \textbf{79.22\%} \\
  \hline
\end{tabular}
}
\end{center}
\vspace{-5mm}
\end{table}


\begin{figure}[t]
\centering
% \vspace{-2mm}
\includegraphics[width=0.42\textwidth]{fig/2dvisual-linux4-paper2.pdf}
\caption{Visualization of feature distribution on eight categories before (left) and after (right) model processing.}
% 
\label{fig:visualization}
\vspace{-5mm}
\end{figure}

\subsection{Ablation Performance}
In this subsection, we conduct an ablation study to examine which component is really important for performance improvement. The results are reported in Table~\ref{tab:cap2}.

For information utilization, we observe a significant decline in model performance when visual features are removed. Additionally, the performance of \shortname~decreases when different metadata are removed separately, which means that text description, object tag, and scene tag are all critical for image sentiment analysis.
Recalling the model architecture, we separately remove transformer layers of the unified representation module, the adaptive learning module, and the cross-modal fusion module, replacing them with MLPs of the same parameter scale.
In this way, we can observe varying degrees of decline in model performance, indicating that these modules are indispensable for our model to achieve better performance.

\subsection{Visualization}
% 


% % 开始使用minipage进行左右排列
% \begin{minipage}[t]{0.45\textwidth}  % 子图1宽度为45%
%     \centering
%     \includegraphics[width=\textwidth]{2dvisual.pdf}  % 插入图片
%     \captionof{figure}{Visualization of feature distribution.}  % 使用captionof添加图片标题
%     \label{fig:visualization}
% \end{minipage}


% \begin{figure}[t]
% \centering
% \vspace{-2mm}
% \includegraphics[width=0.45\textwidth]{fig/2dvisual.pdf}
% \caption{Visualization of feature distribution.}
% \label{fig:visualization}
% % \vspace{-4mm}
% \end{figure}

% \begin{figure}[t]
% \centering
% \vspace{-2mm}
% \includegraphics[width=0.45\textwidth]{fig/2dvisual-linux3-paper.pdf}
% \caption{Visualization of feature distribution.}
% \label{fig:visualization}
% % \vspace{-4mm}
% \end{figure}



\begin{figure}[tbp]   
\vspace{-4mm}
  \centering            
  \subfloat[Depth of adaptive learning layers]   
  {
    \label{fig:subfig1}\includegraphics[width=0.22\textwidth]{fig/fig_sensitivity-a5}
  }
  \subfloat[Depth of fusion layers]
  {
    % \label{fig:subfig2}\includegraphics[width=0.22\textwidth]{fig/fig_sensitivity-b2}
    \label{fig:subfig2}\includegraphics[width=0.22\textwidth]{fig/fig_sensitivity-b2-num.pdf}
  }
  \caption{Sensitivity study of \shortname~on different depth. }   
  \label{fig:fig_sensitivity}  
\vspace{-2mm}
\end{figure}

% \begin{figure}[htbp]
% \centerline{\includegraphics{2dvisual.pdf}}
% \caption{Visualization of feature distribution.}
% \label{fig:visualization}
% \end{figure}

% In Fig.~\ref{fig:visualization}, we use t-SNE~\cite{van2008visualizing} to reduce the dimension of data features for visualization, Figure in left represents the metadata features before model processing, the features are obtained by embedding through the CLIP model, and figure in right shows the features of the data after model processing, it can be observed that after the model processing, the data with different label categories fall in different regions in the space, therefore, we can conclude that the Therefore, we can conclude that the model can effectively utilize the information contained in the metadata and use it to guide the model for classification.

In Fig.~\ref{fig:visualization}, we use t-SNE~\cite{van2008visualizing} to reduce the dimension of data features for visualization.
The left figure shows metadata features before being processed by our model (\textit{i.e.}, embedded by CLIP), while the right shows the distribution of features after being processed by our model.
We can observe that after the model processing, data with the same label are closer to each other, while others are farther away.
Therefore, it shows that the model can effectively utilize the information contained in the metadata and use it to guide the classification process.

\subsection{Sensitivity Analysis}
% 
In this subsection, we conduct a sensitivity analysis to figure out the effect of different depth settings of adaptive learning layers and fusion layers. 
% In this subsection, we conduct a sensitivity analysis to figure out the effect of different depth settings on the model. 
% Fig.~\ref{fig:fig_sensitivity} presents the effect of different depth settings of adaptive learning layers and fusion layers. 
Taking Fig.~\ref{fig:fig_sensitivity} (a) as an example, the model performance improves with increasing depth, reaching the best performance at a depth of 4.
% Taking Fig.~\ref{fig:fig_sensitivity} (a) as an example, the performance of \shortname~improves with the increase of depth at first, reaching the best performance at a depth of 4.
When the depth continues to increase, the accuracy decreases to varying degrees.
Similar results can be observed in Fig.~\ref{fig:fig_sensitivity} (b).
Therefore, we set their depths to 4 and 6 respectively to achieve the best results.

% Through our experiments, we can observe that the effect of modifying these hyperparameters on the results of the experiments is very weak, and the surface model is not sensitive to the hyperparameters.


\subsection{Zero-shot Capability}
% 

% (1)~GCH~\cite{2010Analyzing} & 21.78\% & (5)~RA-DLNet~\cite{2020A} & 34.01\% \\ \hline
% (2)~WSCNet~\cite{2019WSCNet}  & 30.25\% & (6)~CECCN~\cite{ruan2024color} & 43.83\% \\ \hline
% (3)~PCNN~\cite{2015Robust} & 31.68\%  & (7)~EmoVIT~\cite{xie2024emovit} & 44.90\% \\ \hline
% (4)~AR~\cite{2018Visual} & 32.67\% & (8)~Ours (Zero-shot) & 47.83\% \\ \hline


\begin{table}[t]
\centering
\caption{Zero-shot capability of \shortname.}
\label{tab:cap3}
\resizebox{1\linewidth}{!}
{
\begin{tabular}{lc|lc}
\hline
\textbf{Model} & \textbf{Accuracy} & \textbf{Model} & \textbf{Accuracy} \\ \hline
(1)~WSCNet~\cite{2019WSCNet}  & 30.25\% & (5)~MAM~\cite{zhang2024affective} & 39.56\%  \\ \hline
(2)~AR~\cite{2018Visual} & 32.67\% & (6)~CECCN~\cite{ruan2024color} & 43.83\% \\ \hline
(3)~RA-DLNet~\cite{2020A} & 34.01\%  & (7)~EmoVIT~\cite{xie2024emovit} & 44.90\% \\ \hline
(4)~CDA~\cite{han2023boosting} & 38.64\% & (8)~Ours (Zero-shot) & 47.83\% \\ \hline
\end{tabular}
}
\vspace{-5mm}
\end{table}

% We use the model trained on the FI dataset to test on the artphoto dataset to verify the model's generalization ability as well as robustness to other distributed datasets.
% We can observe that the MESN model shows strong competitiveness in terms of accuracy when compared to other trained models, which suggests that the model has a good generalization ability in the OOD task.

To validate the model's generalization ability and robustness to other distributed datasets, we directly test the model trained on the FI dataset, without training on Artphoto. 
% As observed in Table 3, compared to other models trained on Artphoto, we achieve highly competitive zero-shot performance, indicating that the model has good generalization ability in out-of-distribution tasks.
From Table~\ref{tab:cap3}, we can observe that compared with other models trained on Artphoto, we achieve competitive zero-shot performance, which shows that the model has good generalization ability in out-of-distribution tasks.


%%%%%%%%%%%%
%  E2E     %
%%%%%%%%%%%%


\section{Conclusion}
In this paper, we introduced Wi-Chat, the first LLM-powered Wi-Fi-based human activity recognition system that integrates the reasoning capabilities of large language models with the sensing potential of wireless signals. Our experimental results on a self-collected Wi-Fi CSI dataset demonstrate the promising potential of LLMs in enabling zero-shot Wi-Fi sensing. These findings suggest a new paradigm for human activity recognition that does not rely on extensive labeled data. We hope future research will build upon this direction, further exploring the applications of LLMs in signal processing domains such as IoT, mobile sensing, and radar-based systems.

\section*{Limitations}
While our work represents the first attempt to leverage LLMs for processing Wi-Fi signals, it is a preliminary study focused on a relatively simple task: Wi-Fi-based human activity recognition. This choice allows us to explore the feasibility of LLMs in wireless sensing but also comes with certain limitations.

Our approach primarily evaluates zero-shot performance, which, while promising, may still lag behind traditional supervised learning methods in highly complex or fine-grained recognition tasks. Besides, our study is limited to a controlled environment with a self-collected dataset, and the generalizability of LLMs to diverse real-world scenarios with varying Wi-Fi conditions, environmental interference, and device heterogeneity remains an open question.

Additionally, we have yet to explore the full potential of LLMs in more advanced Wi-Fi sensing applications, such as fine-grained gesture recognition, occupancy detection, and passive health monitoring. Future work should investigate the scalability of LLM-based approaches, their robustness to domain shifts, and their integration with multimodal sensing techniques in broader IoT applications.


% Bibliography entries for the entire Anthology, followed by custom entries
%\bibliography{anthology,custom}
% Custom bibliography entries only
\bibliography{main}
\newpage
\appendix

\section{Experiment prompts}
\label{sec:prompt}
The prompts used in the LLM experiments are shown in the following Table~\ref{tab:prompts}.

\definecolor{titlecolor}{rgb}{0.9, 0.5, 0.1}
\definecolor{anscolor}{rgb}{0.2, 0.5, 0.8}
\definecolor{labelcolor}{HTML}{48a07e}
\begin{table*}[h]
	\centering
	
 % \vspace{-0.2cm}
	
	\begin{center}
		\begin{tikzpicture}[
				chatbox_inner/.style={rectangle, rounded corners, opacity=0, text opacity=1, font=\sffamily\scriptsize, text width=5in, text height=9pt, inner xsep=6pt, inner ysep=6pt},
				chatbox_prompt_inner/.style={chatbox_inner, align=flush left, xshift=0pt, text height=11pt},
				chatbox_user_inner/.style={chatbox_inner, align=flush left, xshift=0pt},
				chatbox_gpt_inner/.style={chatbox_inner, align=flush left, xshift=0pt},
				chatbox/.style={chatbox_inner, draw=black!25, fill=gray!7, opacity=1, text opacity=0},
				chatbox_prompt/.style={chatbox, align=flush left, fill=gray!1.5, draw=black!30, text height=10pt},
				chatbox_user/.style={chatbox, align=flush left},
				chatbox_gpt/.style={chatbox, align=flush left},
				chatbox2/.style={chatbox_gpt, fill=green!25},
				chatbox3/.style={chatbox_gpt, fill=red!20, draw=black!20},
				chatbox4/.style={chatbox_gpt, fill=yellow!30},
				labelbox/.style={rectangle, rounded corners, draw=black!50, font=\sffamily\scriptsize\bfseries, fill=gray!5, inner sep=3pt},
			]
											
			\node[chatbox_user] (q1) {
				\textbf{System prompt}
				\newline
				\newline
				You are a helpful and precise assistant for segmenting and labeling sentences. We would like to request your help on curating a dataset for entity-level hallucination detection.
				\newline \newline
                We will give you a machine generated biography and a list of checked facts about the biography. Each fact consists of a sentence and a label (True/False). Please do the following process. First, breaking down the biography into words. Second, by referring to the provided list of facts, merging some broken down words in the previous step to form meaningful entities. For example, ``strategic thinking'' should be one entity instead of two. Third, according to the labels in the list of facts, labeling each entity as True or False. Specifically, for facts that share a similar sentence structure (\eg, \textit{``He was born on Mach 9, 1941.''} (\texttt{True}) and \textit{``He was born in Ramos Mejia.''} (\texttt{False})), please first assign labels to entities that differ across atomic facts. For example, first labeling ``Mach 9, 1941'' (\texttt{True}) and ``Ramos Mejia'' (\texttt{False}) in the above case. For those entities that are the same across atomic facts (\eg, ``was born'') or are neutral (\eg, ``he,'' ``in,'' and ``on''), please label them as \texttt{True}. For the cases that there is no atomic fact that shares the same sentence structure, please identify the most informative entities in the sentence and label them with the same label as the atomic fact while treating the rest of the entities as \texttt{True}. In the end, output the entities and labels in the following format:
                \begin{itemize}[nosep]
                    \item Entity 1 (Label 1)
                    \item Entity 2 (Label 2)
                    \item ...
                    \item Entity N (Label N)
                \end{itemize}
                % \newline \newline
                Here are two examples:
                \newline\newline
                \textbf{[Example 1]}
                \newline
                [The start of the biography]
                \newline
                \textcolor{titlecolor}{Marianne McAndrew is an American actress and singer, born on November 21, 1942, in Cleveland, Ohio. She began her acting career in the late 1960s, appearing in various television shows and films.}
                \newline
                [The end of the biography]
                \newline \newline
                [The start of the list of checked facts]
                \newline
                \textcolor{anscolor}{[Marianne McAndrew is an American. (False); Marianne McAndrew is an actress. (True); Marianne McAndrew is a singer. (False); Marianne McAndrew was born on November 21, 1942. (False); Marianne McAndrew was born in Cleveland, Ohio. (False); She began her acting career in the late 1960s. (True); She has appeared in various television shows. (True); She has appeared in various films. (True)]}
                \newline
                [The end of the list of checked facts]
                \newline \newline
                [The start of the ideal output]
                \newline
                \textcolor{labelcolor}{[Marianne McAndrew (True); is (True); an (True); American (False); actress (True); and (True); singer (False); , (True); born (True); on (True); November 21, 1942 (False); , (True); in (True); Cleveland, Ohio (False); . (True); She (True); began (True); her (True); acting career (True); in (True); the late 1960s (True); , (True); appearing (True); in (True); various (True); television shows (True); and (True); films (True); . (True)]}
                \newline
                [The end of the ideal output]
				\newline \newline
                \textbf{[Example 2]}
                \newline
                [The start of the biography]
                \newline
                \textcolor{titlecolor}{Doug Sheehan is an American actor who was born on April 27, 1949, in Santa Monica, California. He is best known for his roles in soap operas, including his portrayal of Joe Kelly on ``General Hospital'' and Ben Gibson on ``Knots Landing.''}
                \newline
                [The end of the biography]
                \newline \newline
                [The start of the list of checked facts]
                \newline
                \textcolor{anscolor}{[Doug Sheehan is an American. (True); Doug Sheehan is an actor. (True); Doug Sheehan was born on April 27, 1949. (True); Doug Sheehan was born in Santa Monica, California. (False); He is best known for his roles in soap operas. (True); He portrayed Joe Kelly. (True); Joe Kelly was in General Hospital. (True); General Hospital is a soap opera. (True); He portrayed Ben Gibson. (True); Ben Gibson was in Knots Landing. (True); Knots Landing is a soap opera. (True)]}
                \newline
                [The end of the list of checked facts]
                \newline \newline
                [The start of the ideal output]
                \newline
                \textcolor{labelcolor}{[Doug Sheehan (True); is (True); an (True); American (True); actor (True); who (True); was born (True); on (True); April 27, 1949 (True); in (True); Santa Monica, California (False); . (True); He (True); is (True); best known (True); for (True); his roles in soap operas (True); , (True); including (True); in (True); his portrayal (True); of (True); Joe Kelly (True); on (True); ``General Hospital'' (True); and (True); Ben Gibson (True); on (True); ``Knots Landing.'' (True)]}
                \newline
                [The end of the ideal output]
				\newline \newline
				\textbf{User prompt}
				\newline
				\newline
				[The start of the biography]
				\newline
				\textcolor{magenta}{\texttt{\{BIOGRAPHY\}}}
				\newline
				[The ebd of the biography]
				\newline \newline
				[The start of the list of checked facts]
				\newline
				\textcolor{magenta}{\texttt{\{LIST OF CHECKED FACTS\}}}
				\newline
				[The end of the list of checked facts]
			};
			\node[chatbox_user_inner] (q1_text) at (q1) {
				\textbf{System prompt}
				\newline
				\newline
				You are a helpful and precise assistant for segmenting and labeling sentences. We would like to request your help on curating a dataset for entity-level hallucination detection.
				\newline \newline
                We will give you a machine generated biography and a list of checked facts about the biography. Each fact consists of a sentence and a label (True/False). Please do the following process. First, breaking down the biography into words. Second, by referring to the provided list of facts, merging some broken down words in the previous step to form meaningful entities. For example, ``strategic thinking'' should be one entity instead of two. Third, according to the labels in the list of facts, labeling each entity as True or False. Specifically, for facts that share a similar sentence structure (\eg, \textit{``He was born on Mach 9, 1941.''} (\texttt{True}) and \textit{``He was born in Ramos Mejia.''} (\texttt{False})), please first assign labels to entities that differ across atomic facts. For example, first labeling ``Mach 9, 1941'' (\texttt{True}) and ``Ramos Mejia'' (\texttt{False}) in the above case. For those entities that are the same across atomic facts (\eg, ``was born'') or are neutral (\eg, ``he,'' ``in,'' and ``on''), please label them as \texttt{True}. For the cases that there is no atomic fact that shares the same sentence structure, please identify the most informative entities in the sentence and label them with the same label as the atomic fact while treating the rest of the entities as \texttt{True}. In the end, output the entities and labels in the following format:
                \begin{itemize}[nosep]
                    \item Entity 1 (Label 1)
                    \item Entity 2 (Label 2)
                    \item ...
                    \item Entity N (Label N)
                \end{itemize}
                % \newline \newline
                Here are two examples:
                \newline\newline
                \textbf{[Example 1]}
                \newline
                [The start of the biography]
                \newline
                \textcolor{titlecolor}{Marianne McAndrew is an American actress and singer, born on November 21, 1942, in Cleveland, Ohio. She began her acting career in the late 1960s, appearing in various television shows and films.}
                \newline
                [The end of the biography]
                \newline \newline
                [The start of the list of checked facts]
                \newline
                \textcolor{anscolor}{[Marianne McAndrew is an American. (False); Marianne McAndrew is an actress. (True); Marianne McAndrew is a singer. (False); Marianne McAndrew was born on November 21, 1942. (False); Marianne McAndrew was born in Cleveland, Ohio. (False); She began her acting career in the late 1960s. (True); She has appeared in various television shows. (True); She has appeared in various films. (True)]}
                \newline
                [The end of the list of checked facts]
                \newline \newline
                [The start of the ideal output]
                \newline
                \textcolor{labelcolor}{[Marianne McAndrew (True); is (True); an (True); American (False); actress (True); and (True); singer (False); , (True); born (True); on (True); November 21, 1942 (False); , (True); in (True); Cleveland, Ohio (False); . (True); She (True); began (True); her (True); acting career (True); in (True); the late 1960s (True); , (True); appearing (True); in (True); various (True); television shows (True); and (True); films (True); . (True)]}
                \newline
                [The end of the ideal output]
				\newline \newline
                \textbf{[Example 2]}
                \newline
                [The start of the biography]
                \newline
                \textcolor{titlecolor}{Doug Sheehan is an American actor who was born on April 27, 1949, in Santa Monica, California. He is best known for his roles in soap operas, including his portrayal of Joe Kelly on ``General Hospital'' and Ben Gibson on ``Knots Landing.''}
                \newline
                [The end of the biography]
                \newline \newline
                [The start of the list of checked facts]
                \newline
                \textcolor{anscolor}{[Doug Sheehan is an American. (True); Doug Sheehan is an actor. (True); Doug Sheehan was born on April 27, 1949. (True); Doug Sheehan was born in Santa Monica, California. (False); He is best known for his roles in soap operas. (True); He portrayed Joe Kelly. (True); Joe Kelly was in General Hospital. (True); General Hospital is a soap opera. (True); He portrayed Ben Gibson. (True); Ben Gibson was in Knots Landing. (True); Knots Landing is a soap opera. (True)]}
                \newline
                [The end of the list of checked facts]
                \newline \newline
                [The start of the ideal output]
                \newline
                \textcolor{labelcolor}{[Doug Sheehan (True); is (True); an (True); American (True); actor (True); who (True); was born (True); on (True); April 27, 1949 (True); in (True); Santa Monica, California (False); . (True); He (True); is (True); best known (True); for (True); his roles in soap operas (True); , (True); including (True); in (True); his portrayal (True); of (True); Joe Kelly (True); on (True); ``General Hospital'' (True); and (True); Ben Gibson (True); on (True); ``Knots Landing.'' (True)]}
                \newline
                [The end of the ideal output]
				\newline \newline
				\textbf{User prompt}
				\newline
				\newline
				[The start of the biography]
				\newline
				\textcolor{magenta}{\texttt{\{BIOGRAPHY\}}}
				\newline
				[The ebd of the biography]
				\newline \newline
				[The start of the list of checked facts]
				\newline
				\textcolor{magenta}{\texttt{\{LIST OF CHECKED FACTS\}}}
				\newline
				[The end of the list of checked facts]
			};
		\end{tikzpicture}
        \caption{GPT-4o prompt for labeling hallucinated entities.}\label{tb:gpt-4-prompt}
	\end{center}
\vspace{-0cm}
\end{table*}
% \section{Full Experiment Results}
% \begin{table*}[th]
    \centering
    \small
    \caption{Classification Results}
    \begin{tabular}{lcccc}
        \toprule
        \textbf{Method} & \textbf{Accuracy} & \textbf{Precision} & \textbf{Recall} & \textbf{F1-score} \\
        \midrule
        \multicolumn{5}{c}{\textbf{Zero Shot}} \\
                Zero-shot E-eyes & 0.26 & 0.26 & 0.27 & 0.26 \\
        Zero-shot CARM & 0.24 & 0.24 & 0.24 & 0.24 \\
                Zero-shot SVM & 0.27 & 0.28 & 0.28 & 0.27 \\
        Zero-shot CNN & 0.23 & 0.24 & 0.23 & 0.23 \\
        Zero-shot RNN & 0.26 & 0.26 & 0.26 & 0.26 \\
DeepSeek-0shot & 0.54 & 0.61 & 0.54 & 0.52 \\
DeepSeek-0shot-COT & 0.33 & 0.24 & 0.33 & 0.23 \\
DeepSeek-0shot-Knowledge & 0.45 & 0.46 & 0.45 & 0.44 \\
Gemma2-0shot & 0.35 & 0.22 & 0.38 & 0.27 \\
Gemma2-0shot-COT & 0.36 & 0.22 & 0.36 & 0.27 \\
Gemma2-0shot-Knowledge & 0.32 & 0.18 & 0.34 & 0.20 \\
GPT-4o-mini-0shot & 0.48 & 0.53 & 0.48 & 0.41 \\
GPT-4o-mini-0shot-COT & 0.33 & 0.50 & 0.33 & 0.38 \\
GPT-4o-mini-0shot-Knowledge & 0.49 & 0.31 & 0.49 & 0.36 \\
GPT-4o-0shot & 0.62 & 0.62 & 0.47 & 0.42 \\
GPT-4o-0shot-COT & 0.29 & 0.45 & 0.29 & 0.21 \\
GPT-4o-0shot-Knowledge & 0.44 & 0.52 & 0.44 & 0.39 \\
LLaMA-0shot & 0.32 & 0.25 & 0.32 & 0.24 \\
LLaMA-0shot-COT & 0.12 & 0.25 & 0.12 & 0.09 \\
LLaMA-0shot-Knowledge & 0.32 & 0.25 & 0.32 & 0.28 \\
Mistral-0shot & 0.19 & 0.23 & 0.19 & 0.10 \\
Mistral-0shot-Knowledge & 0.21 & 0.40 & 0.21 & 0.11 \\
        \midrule
        \multicolumn{5}{c}{\textbf{4 Shot}} \\
GPT-4o-mini-4shot & 0.58 & 0.59 & 0.58 & 0.53 \\
GPT-4o-mini-4shot-COT & 0.57 & 0.53 & 0.57 & 0.50 \\
GPT-4o-mini-4shot-Knowledge & 0.56 & 0.51 & 0.56 & 0.47 \\
GPT-4o-4shot & 0.77 & 0.84 & 0.77 & 0.73 \\
GPT-4o-4shot-COT & 0.63 & 0.76 & 0.63 & 0.53 \\
GPT-4o-4shot-Knowledge & 0.72 & 0.82 & 0.71 & 0.66 \\
LLaMA-4shot & 0.29 & 0.24 & 0.29 & 0.21 \\
LLaMA-4shot-COT & 0.20 & 0.30 & 0.20 & 0.13 \\
LLaMA-4shot-Knowledge & 0.15 & 0.23 & 0.13 & 0.13 \\
Mistral-4shot & 0.02 & 0.02 & 0.02 & 0.02 \\
Mistral-4shot-Knowledge & 0.21 & 0.27 & 0.21 & 0.20 \\
        \midrule
        
        \multicolumn{5}{c}{\textbf{Suprevised}} \\
        SVM & 0.94 & 0.92 & 0.91 & 0.91 \\
        CNN & 0.98 & 0.98 & 0.97 & 0.97 \\
        RNN & 0.99 & 0.99 & 0.99 & 0.99 \\
        % \midrule
        % \multicolumn{5}{c}{\textbf{Conventional Wi-Fi-based Human Activity Recognition Systems}} \\
        E-eyes & 1.00 & 1.00 & 1.00 & 1.00 \\
        CARM & 0.98 & 0.98 & 0.98 & 0.98 \\
\midrule
 \multicolumn{5}{c}{\textbf{Vision Models}} \\
           Zero-shot SVM & 0.26 & 0.25 & 0.25 & 0.25 \\
        Zero-shot CNN & 0.26 & 0.25 & 0.26 & 0.26 \\
        Zero-shot RNN & 0.28 & 0.28 & 0.29 & 0.28 \\
        SVM & 0.99 & 0.99 & 0.99 & 0.99 \\
        CNN & 0.98 & 0.99 & 0.98 & 0.98 \\
        RNN & 0.98 & 0.99 & 0.98 & 0.98 \\
GPT-4o-mini-Vision & 0.84 & 0.85 & 0.84 & 0.84 \\
GPT-4o-mini-Vision-COT & 0.90 & 0.91 & 0.90 & 0.90 \\
GPT-4o-Vision & 0.74 & 0.82 & 0.74 & 0.73 \\
GPT-4o-Vision-COT & 0.70 & 0.83 & 0.70 & 0.68 \\
LLaMA-Vision & 0.20 & 0.23 & 0.20 & 0.09 \\
LLaMA-Vision-Knowledge & 0.22 & 0.05 & 0.22 & 0.08 \\

        \bottomrule
    \end{tabular}
    \label{full}
\end{table*}




\end{document}

% \bibliography{ref}

% \begin{IEEEbiography}
% % [{\includegraphics[width=1in,height=1.25in,clip,keepaspectratio]{./AuthorPhotos/Linhao.jpg}}]{Lin Hao}
% A
% \end{IEEEbiography}
% \begin{IEEEbiography}
% % [{\includegraphics[width=1in,height=1.25in,clip,keepaspectratio]{./AuthorPhotos/MustafaKishk.jpg}}]{Mustafa A. Kishk} 
% (Member, IEEE) received the
% B.Sc. and M.Sc. degrees from Cairo University
% in 2013 and 2015, respectively, and the Ph.D. degree
% from Virginia Tech in 2018. He is currently a
% Post-Doctoral Research Fellow with the CTL at
% KAUST. His current research interests include stochastic geometry, energy harvesting wireless networks, UAV-enabled communication systems, and
% satellite communications.
% \end{IEEEbiography}



\end{document}

