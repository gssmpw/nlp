\section{Experiment}
\subsection{Dataset Preprocessing and Evaluation Metric}\label{sec4}
\noindent \textbf{Dataset:}
We evaluate the performance of \textbf{H3DE-Net} using the publicly available 3D skull landmark detection dataset from~\cite{3D-CNN1-anchor}. The dataset includes 658 CT scans (458 / training, 100 / validation, 100 / testing),  each carefully annotated with 14 anatomical landmarks for the mandibular molar landmarking (MML) task. This task aims to identify the anatomical positions of the crowns and roots of mandibular molars. The dataset is systematically categorized into two distinct subsets to facilitate more comprehensive evaluations: \textbf{Complete Cases} and \textbf{Incomplete Cases}. 

The Complete Cases subset contains CT scans in which all 14 landmarks are thoroughly annotated. This subset consists of 283 training volumes, 56 validation volumes, and 60 test volumes. In contrast, the Incomplete Cases subset introduces additional complexity, as some landmarks are missing due to various factors such as occlusion, reduced image quality, or imaging artifacts. This incomplete subset comprises 175 volumes for training, 44 for validation, and 40 for testing. This subset is particularly significant because it reflects the challenges encountered in real-world clinical scenarios, where datasets are often incomplete and affected by noise. Incomplete cases serve as a critical aspect of the evaluation process, testing the robustness and adaptability of the model to ambiguous or missing data.

\noindent \textbf{Preprocessing Steps:}
To prepare CT images for training and evaluation, we implement a series of pre-processing steps designed to standardize the input data, reduce computational complexity, and enhance the robustness of the model through data augmentation. These steps are described in detail below:

Firstly, the voxel intensity is normalized to all CT images. The raw intensity values, which can vary significantly between different scans due to variations in imaging protocols and acquisition devices, are normalized to $[0, 255]$. This normalization step ensures consistency in input data, allowing the model to focus on anatomical features rather than being influenced by imaging artifacts or variations in intensity scaling.

Secondly, we perform random cropping of the input volumes to accommodate GPU memory limitations and maintain computational efficiency. Each CT image is cropped to a fixed size of $128 \times 128 \times 64$ voxels. This strategy reduces the overall computational cost and ensures that the model can process input volumes efficiently during training and inference. The random cropping process introduces some variability in the spatial positioning of anatomical structures within the input data, which helps the model learn more generalized feature representations.

\noindent \textbf{Evaluation Metric:} Recent studies have used voxel Euclidean distance (VED), mean radial error (MRE), successful
detection rate (SDR), and root mean square error (RMSE) as criteria for evaluating models. MRE and SDR are widely used and considered standard metrics for 3D cephalometric landmark detection. Therefore, our evaluation metrics use MRE and SDR with standard deviation. The MRE represents the average Euclidean distance between the predicted and true values, while the SDR measures the accuracy of the detected landmarks at different distances, especially at 2 mm, 2.5 mm, 3 mm, and 4 mm. The formulas for these metrics are as follows:
% TODO: 简化这里的公式
\vspace{2mm}
\begin{equation}
\begin{aligned}
MRE &= \frac{1}{n} \sum_{i=1}^{n} \sqrt{\sum_{\tau \in \{x, y, z\}} \left( (g_{\tau}^i - f_{\tau}^i) \cdot (s_{\tau}^i)^2 \right)}, \\
\end{aligned}
\end{equation}
\vspace{2mm}

\begin{equation}
\begin{aligned}
SDR &= \frac{\text{number of accurate detections}}{\text{number of detections}} \times 100\%,
\end{aligned}
\end{equation}
\vspace{2mm}

In this landmark detection task using CT dental data, \( n \) represents the total number of data points. \( g_{\tau}^i \) denotes the ground truth value of the \( i \)-th landmark for the coordinate axis \( \tau \) (where \( \tau \in \{x, y, z\} \)), which corresponds to the true location of the dental landmarks in the CT data. \( f_{\tau}^i \) is the predicted value of the \( i \)-th landmark for the coordinate axis \( \tau \), which corresponds to the predicted location of the dental landmarks. \( s_{\tau}^i \) is the scaling factor for the \( i \)-th point in the coordinate axis \( \tau \), adjusting for the spacing or resolution of the CT data.



\subsection{Baseline Methods}
To rigorously validate the effectiveness and robustness of \textbf{H3DE-Net}, we conduct a comprehensive performance comparison against several baseline methods widely recognized in the field of medical image analysis. The methods included in the comparison are as follows:

\begin{itemize}
    \item \textbf{UNet3D~\cite{3dunet}:} UNet is a basic encoder-decoder method that skips connections is widely used in heat-map-based landmark detection. UNet3D is an improved 3D version of UNet for MRI and CT images.
    
    \item \textbf{VNet~\cite{3dvnet}:} VNet replaces pooling layers with convolutional layers for downsampling, uses transposed convolution to restore resolution, and adds horizontal residual connections to address the vanishing gradient problem. It excels in medical imaging and 3D object segmentation tasks.
    
    \item \textbf{ResUNet3D~\cite{3dresunet}:} ResUNet is one of the most important improved versions of U-Net, replacing U-Net’s original encoder with ResNet. It combines the advantages of ResNet and U-Net, effectively addressing the vanishing gradient problem and the loss of semantic information.
    
    \item \textbf{HRNet~\cite{HRnet}:} A widely recognized method in landmark detection, HRNet has achieved significant success. This paper evaluates two configurations, HRNet-32 and HRNet-48.
    \item \textbf{SRPose~\cite{SRpose}:} SRPose is a super-resolution head for human pose estimation, showing significant improvements in landmark detection accuracy.
    \item \textbf{MTL~\cite{Zhang2019ContextguidedFC}:} MTL (Multi-task Learning) integrates landmark detection with tasks like bone segmentation, leveraging shared spatial information. Zhang et al.~\cite{Zhang2019ContextguidedFC} propose a context-guided fully convolutional neural network using displacement maps to learn spatial information, enabling simultaneous completion of both tasks.
    \item \textbf{Two-Stage~\cite{HE202115}: } Recent works decompose landmark detection into successive sub-tasks for simplification. He et al.~\cite{HE202115} propose a two-phase framework: the first stage uses a global detection network (GDN) to generate candidate landmarks, and the second stage employs a local refinement network (LRN) to refine landmark locations by processing image blocks. This two-stage approach balances performance and computation.
    \item \textbf{SRLD-UNet \& SR-UNet~\cite{3D-CNN2-sr}:} Zhang et al. propose two super-resolution landmark detection networks, SRLD-Net and SR-UNet, to improve accuracy and reduce errors in medical imaging. 
    % By integrating pyramid pooling, fusion modules, and super-resolution techniques, these methods enhance feature representation learning and produce more refined predicted heatmaps, showing superior performance on craniofacial, nasal, and mandibular datasets.
    
\end{itemize}

% The selection of these baselines reflects a deliberate effort to encompass a broad spectrum of methodological approaches in the field of medical image analysis. By including classic segmentation architectures, multitask learning frameworks, and regression-based models, this evaluation ensures a comprehensive and balanced assessment of the proposed \textbf{H3DE-Net}. Each baseline offers distinct advantages and challenges, making them valuable for highlighting the strengths and limitations of the proposed method. Such a diverse set of comparisons provides critical insights into how \textbf{H3DE-Net} advances the state-of-the-art, particularly in handling complex landmark localization scenarios characterized by high anatomical variability and incomplete data.

% To further understand the design choices and investigate the flexibility of \textbf{H3DE-Net}, we evaluated two architectural variants of the model, each employing a different approach to landmark localization:

% \begin{enumerate}
%     \item \textbf{Anchor-based Design (Anchor):} This variant incorporates anchor-based offset regression, a common strategy in object detection tasks, for landmark localization. By using predefined anchor points as reference, the model predicts offsets to refine landmark positions. This design leverages spatial priors and is particularly advantageous when landmarks exhibit predictable spatial relationships.

%     \item \textbf{Non-Anchor Design (Non-Anchor):} In contrast, the non-anchor-based variant directly predicts landmark positions without relying on predefined anchors. This approach removes the dependency on anchor configurations, enabling the model to adapt to varying landmark distributions and exhibit greater flexibility in handling irregular or unpredictable spatial layouts. Such a design is especially beneficial in scenarios where anatomical landmarks are sparsely distributed or exhibit significant variability.
% \end{enumerate}

% By exploring both anchor-based and non-anchor-based strategies, this study not only evaluates the robustness and generalizability of \textbf{H3DE-Net} but also sheds light on the trade-offs associated with each design. These experiments highlight the versatility of the proposed framework, demonstrating its potential to adapt to diverse medical imaging tasks and its ability to push the boundaries of existing methods in landmark localization.

\subsection{Experimental Results}





\subsubsection{Quantitative Results}

To evaluate the performance of \textbf{H3DE-Net} and compare it to these baseline methods, experiments are conducted on three dataset configurations: 1)\textit{complete dataset}, which only includes complete cases where all landmarks are fully annotated; there are no missing landmarks in this dataset, so Anchor-free Architecture is not needed; 2)\textit{incomplete dataset}, which only includes incomplete cases where some landmarks are missing due to occlusion, imaging artifacts, or other factors. On such datasets, Anchor-Based Architecture is used because the landmarks are missing; 3)\textit{all cases dataset}, which combines complete and incomplete cases to evaluate the generalization ability of the model in different scenarios. In this kind of data set, because some of the landmarks are missing, Anchor-Based Architecture is used.

Table \ref{tab:tab1} summarizes the quantitative results of \textbf{H3DE-Net} compared with the baseline methods including \textbf{VNet}, \textbf{UNet3D} and \textbf{ResUNet3D}. On \textit{complete dataset}, \textbf{H3DE-Net} achieves an MRE of $1.68 \pm 0.45$mm and a SDR of 97.02\% at a threshold of 4mm. It significantly exceeds the strongest baseline \textbf{UNet3D}, which has an MRE of $1.90 \pm 0.65$mm and an SDR (4mm) of 93.54\%. Experimental results show that the proposed model has high accuracy when dealing with noiseless data.

\begin{table*}[htbp]
\centering
\small
\caption{The experimental results for VNet, UNet3D, ResUNet3D, and H3DE-Net. We train on the \texttt{All}, \texttt{Complete}, and \texttt{Incomplete} datasets, and test on the same sets. “$\checkmark$” indicates the use of an anchor, while “$\times$” indicates no anchor is used.
}
\renewcommand{\arraystretch}{1}
\setlength{\extrarowheight}{1.1pt} % 增加行和线之间的距离
\resizebox{1\linewidth}{!}
{
\begin{tabular}{lccccccc}
\toprule
\multirow{2}{*}{\textbf{Networks}} & \multirow{2}{*}{\textbf{Datasets}} & \multirow{2}{*}{\textbf{Anchor}} & \multirow{2}{*}{\textbf{MRE$\pm$Std (mm)} }& \textbf{SDR (\%)} \\ 
\cmidrule{5-8}
& & & & 2 mm & 2.5 mm & 3 mm & 4 mm \\ \midrule
\multirow{3}{*}{VNet \cite{3dvnet}} & \multirow{1}{*}{All} & \checkmark & 1.82 $\pm$ 0.73 & 71.72 & 83.50 & 90.07 & 94.95 \\ \cmidrule{2-8}
 % & & \xmark & 3.00 $\pm$ 1.42& 41.50 & 55.39 & 65.91 & 79.12  \\ \cmidrule{2-8}
 & \multirow{1}{*}{Complete}  & \xmark & 1.99 $\pm$ 0.71 & 67.86 & 78.31 & 85.20 & 91.20 \\ \cmidrule{2-8}
  % & & \xmark & 1.90 $\pm$ 0.65 & 65.94 & 77.81& 86.99 & 93.54 \\  \cmidrule{2-8}
 & \multirow{1}{*}{Incomplete} & \checkmark & 2.59 $\pm$ 1.82 & 49.19 & 64.05 & 75.68 & 90.54 \\  \midrule 
 % & & \xmark & 5.62 $\pm$ 1.80 & 56.12 & 70.03 & 80.23 & 90.43 \\\midrule
\multirow{3}{*}{UNet3D \cite{3dunet}} & \multirow{1}{*}{All} & \checkmark & 1.77 $\pm$ 0.87 & 73.97 & 85.21 & 91.98 & 96.45 \\  \cmidrule{2-8}
  % & & \xmark & 2.80 $\pm$ 2.10 & 44.11 & 57.74 & 70.62 & 85.77 \\ \cmidrule{2-8}
 & \multirow{1}{*}{Complete}  & \xmark & 1.90 $\pm$ 0.65 & 65.94 & 77.81 & 86.99 & 93.54 \\  \cmidrule{2-8}
 % & & \xmark & 2.22 $\pm$ 0.74 & 64.54 & 76.28 & 82.40 & 87.76 \\  \cmidrule{2-8}
 & \multirow{1}{*}{Incomplete}  & \checkmark & 2.26 $\pm$ 1.26 & 61.89 & 74.86 & 82.43& 91.89 \\  \midrule
 % & & \xmark & 2.30 $\pm$ 1.17 & 56.12 & 70.03 & 80.23 & 90.43 \\ \midrule
\multirow{3}{*}{ResUNet3D \cite{3dresunet}} & \multirow{1}{*}{All}& \checkmark & 1.70 $\pm$ 0.72 & \textbf{76.43}& 86.45 &90.91 & 95.20 \\  \cmidrule{2-8}
 % & & \xmark & 3.00 $\pm$ 1.81 & 37.04 & 51.09 & 64.81 & 82.41 \\ \cmidrule{2-8}
 & \multirow{1}{*}{Complete} & \xmark & 1.96 $\pm$ 0.82 & 70.03 & 79.97 & 86.10 & 92.73 \\  \cmidrule{2-8}
 % & & \xmark & 2.09 $\pm$ 0.67 & 58.67 & 71.81 & 82.65 & 93.11 \\  \cmidrule{2-8}
 & \multirow{1}{*}{Incomplete}  & \checkmark & 2.29 $\pm$ 1.44 & 58.65 & 72.79 & 81.08 & 89.46 \\  \midrule

\multirow{3}{*}{H3DE-Net(Ours)} & \multirow{1}{*}{All} & \checkmark & \textbf{1.67 $\pm$ 0.64 }& 76.03 & \textbf{86.50} & \textbf{92.31} &\textbf{ 96.69} \\ \cmidrule{2-8}

 & \multirow{1}{*}{Complete} & \xmark & \textbf{1.68 $\pm$ 0.45} & \textbf{71.19} & \textbf{85.24} & \textbf{91.67}& \textbf{97.02}\\ \cmidrule{2-8}

 & \multirow{1}{*}{Incomplete}  & \checkmark &\textbf{2.07 $\pm$ 0.96} &\textbf{64.55} & \textbf{76.61} &\textbf{85.29} & \textbf{93.72}\\  \bottomrule

\end{tabular}
}
\label{tab:tab1}
\end{table*}
On the All Cases dataset, \textbf{H3DE-Net} maintained good performance, with MRE reaching $1.67 \pm 0.64$mm. This is a significant improvement over the three baselines. In addition, at the 2mm and 4mm thresholds, SDR reaches 76.03\% and 96.69\%, respectively, while \textbf{ResUNet3D} reaches 76.43\% and 95.20\%, respectively. These results highlight the ability of the model to effectively generalize on heterogeneous datasets, capturing anatomical variations without sacrificing accuracy.


Even under the challenging \textit{incomplete dataset} configuration, where missing landmarks and imaging artifacts make the localization task very difficult, \textbf{H3DE-Net} shows robust performance. Its MRE is $2.07 \pm 0.96$mm, and SDR (4mm) is 93.72\%, which are better than \textbf{UNet3D}'s MRE of $2.26 \pm 1.26$mm and SDR (4mm) of 91.89\%. These findings highlight the robustness of H3DE-Net, especially when incomplete data introduces ambiguity.


In addition to the above three baseline methods, we also compare the performance of H3DE-Net with several competitive methods, including HRNet-32, HRNet-48, SRPose, Two-Stage, MTL, SRLD-UNet, and SR-UNet; we conduct experiments on the All training dataset. Table~\ref{tab:tab2} summarizes the results in terms of MRE and SDR at thresholds of 2 mm, 2.5 mm, 3 mm, and 4 mm.

\begin{table*}[htbp]
\centering
\small
\caption{The performance comparison between H3DE-Net and other SOTA methods. We extract the results from~\cite{3D-CNN2-sr}, where '-' indicates that the result is not presented.}
\renewcommand{\arraystretch}{1}
\setlength{\extrarowheight}{1.1pt} % 增加行和线之间的距离
\resizebox{0.8\linewidth}{!}
{
\begin{tabular}{lccccl}
\hline
\multirow{2}{*}{\textbf{Networks}}  &  \multirow{2}{*}{\textbf{MRE$\pm$Std (mm)} } & \textbf{SDR (\%)} \\ 
\cmidrule{3-6}
& & 2 mm & 2.5 mm & 3 mm & 4 mm \\ \hline
HRNet-32 \cite{HRnet}   & 2.64 $\pm$ 5.18  & 60.81& -- &82.27 & 90.22 \\  
HRNet-48 \cite{HRnet}   & 2.77 $\pm$ 5.83  & 58.89& -- &82.05 & 90.16 \\  
SRPose \cite{SRpose}   & 2.58 $\pm$ 5.73  & 70.09& -- &87.27 & 92.46 \\ 
Two-Stage \cite{HE202115}   & 1.78 $\pm$ 0.81  &68.93& 82.48 &88.68 &95.37 \\
MTL \cite{Zhang2019ContextguidedFC}   & 1.91 $\pm$ 0.75  & 69.70& 80.47 &87.29 & 93.86\\
SRLD-UNet \cite{3D-CNN2-sr}   & 2.40 $\pm$ 5.63  & 74.98& -- &90.39 & 94.45 \\ 
SR-UNet \cite{3D-CNN2-sr}   & 2.01 $\pm$ 4.33  & \textbf{76.14}& -- &92.02 & 95.84 \\  
H3DE-Net(Ours)   & \textbf{1.67 $\pm$ 0.64 }& 76.03&\textbf{ 86.50 }&\textbf{92.31} & \textbf{96.69} \\  
\bottomrule
\end{tabular}
}
\label{tab:tab2}
\end{table*}

Among all comparison methods, H3DE-Net demonstrates excellent performance across all metrics, showcasing its advantages in accuracy and robustness. Specifically, compared to HRNet-32 and HRNet-48, H3DE-Net reduced MRE by 36.74\% and 39.71\%, respectively, while improving SDR (4 mm) by 6.47\% and 6.53\%. Although the Two-Stage method performs relatively better in MRE (1.78 $\pm$ 0.81 mm), it lagged behind H3DE-Net in SDR (2 mm) and SDR (3 mm) by 7.10\% and 3.63\%, respectively, indicating H3DE-Net's stronger advantages in high-precision detection scenarios. 

Additionally, compared to the MTL method, H3DE-Net reduces MRE by 12.57\% and improves the detection success rate in SDR (4 mm) by 2.83\%, further validating its generalization ability across different task scenarios. Notably, compared to SRPose and SRLD-UNet, H3DE-Net significantly improves all SDR metrics, especially at the challenging 2 mm detection threshold, where it increases by 5.94\% and 1.59\%, respectively, demonstrating its superiority in detecting small-range errors.

These experimental results indicate that H3DE-Net's excellence stems from its innovative network architecture, which effectively captures multi-scale features and fuses global and local information, achieving higher accuracy and robustness in complex data. Compared to traditional networks, H3DE-Net has achieved SOTA performance in the current task through efficient modeling of deep features and adaptive handling of missing or noisy data.

Overall, the results demonstrate the consistent superiority of H3DE-Net across all dataset configurations. The proposed method not only outperforms baseline models under ideal conditions but also excels in real-world scenarios where incomplete or heterogeneous data is prevalent. These findings validate the robustness and versatility of the proposed framework, establishing it as a reliable tool for landmark localization tasks in medical imaging.



\subsubsection{Qualitative Results}
We visually analyze the experimental results as follows. The ground truth landmark positions are represented in green, and the predicted landmark positions are represented in red. To ensure clear visibility, we remove some voxels to prevent obstructing the landmarks.
%TODO: 下面那组图要补一个,记得check一下每个图和表的标题
\begin{figure*}[htbp]
\centering
\begin{minipage}{\textwidth}
  \centering
  \includegraphics[width=\linewidth,trim={0cm 0cm 0cm 0cm},clip]{p1.png}
  \caption{Landmark detection performance of H3DE-Net train and test on the all datasets. The prefix 'A-' represents the Anchor-Based method, and the way anchors are added is the same as shown in Fig.~\ref{fig: network2}.}
  \label{fig: p}
\end{minipage}
\end{figure*}


\begin{figure*}[htbp]
\centering
\begin{minipage}{\textwidth}
  \centering
  \includegraphics[width=\linewidth,trim={0cm 0cm 0cm 0cm},clip]{base1.png}
  \caption{Landmark detection performance of H3DE-Net train and test on the complete datasets. }
  \label{fig: base}
\end{minipage}
\end{figure*}

Figure~\ref{fig: p} shows the visualization results of three Anchor-Based methods and H3DE-Net using all training data. Figure~\ref{fig: base} shows the results of three Anchor-Based methods on the complete dataset.

It can be observed that H3DE-Net consistently achieves superior localization accuracy, especially in anatomically intricate regions such as the nasal cavity and the skull base. These regions are characterized by high structural variability and proximity to adjacent anatomical landmarks, making them particularly challenging for baseline models. In comparison, baseline methods often show significant deviations in these areas, leading to errors that are visually apparent in the overlay of predicted and ground truth landmarks. Specifically, H3DE-Net has the following advantages:

\begin{itemize}
    \item \textbf{Higher Local Accuracy:} In regions with complex anatomical structures, such as the skull base, H3DE-Net significantly reduces the deviation between predicted points and ground truth points. In contrast, models like VNet and UNet tend to overfit to incomplete or noisy data in these areas, leading to deviations in predicted locations. H3DE-Net accurately aligns predictions with true anatomical landmarks. As evident from visual overlays, the proposed method achieves more precise predictions around ground truth points.
    
    \item \textbf{Enhanced Robustness:} Compared to baseline models, H3DE-Net demonstrates superior robustness when handling incomplete cases caused by missing or occluded landmarks. Specifically, even in the absence of certain anatomical cues, H3DE-Net can accurately predict key point locations. This is attributed to its effective utilization of both global and local contextual information, ensuring high prediction stability in complex scenarios.
\end{itemize}





% \subsection{Contribution of Anchor-Based Design}

% The anchor-based design is a pivotal factor in the superior performance of \textbf{H3DE-Net}, enabling precise and robust landmark localization through the integration of spatial priors and localized regression. By explicitly modeling the probability and offset of landmarks within predefined anchor regions, this design introduces several key advantages:

% \begin{enumerate}
%     \item \textbf{Improved Localization Accuracy:}  
%     The anchor-based design narrows the search space for landmark localization, significantly reducing regression errors. By leveraging anchor regions as spatial priors, the model effectively aligns predictions with the true anatomical landmarks, resulting in marked improvements in Mean Radial Error (MRE) and Success Detection Rate (SDR). For example, the MRE improvement from $2.45 \pm 1.03$~mm (non-anchor) to $1.70 \pm 0.72$~mm (anchor-based) on the \textit{All Cases} dataset highlights the impact of this design choice.

%     \item \textbf{Enhanced Robustness:}  
%     By explicitly modeling the probability of landmark presence within anchor regions, the anchor-based design enables the model to handle missing landmarks effectively. This is particularly advantageous in challenging scenarios such as the \textit{Incomplete Dataset}, where the model achieved an SDR (4mm) of 94.40\%, outperforming the non-anchor variant (92.50\%) and baseline methods. This robustness ensures reliable performance even in cases with occluded or ambiguous landmarks.

%     \item \textbf{Stability Across Configurations:}  
%     The anchor-based design demonstrated consistently lower standard deviations in both MRE and SDR metrics across all dataset configurations. This stability underscores the model’s reliability and its ability to maintain consistent performance in diverse and challenging conditions. For instance, the standard deviation of the MRE on the \textit{Complete Dataset} was reduced by nearly 30\% compared to non-anchor designs, further emphasizing the model's robustness.
% \end{enumerate}

% \subsection{Statistical Significance Analysis}

% To rigorously evaluate the statistical significance of the performance improvements, we conducted paired \textit{t-tests} between \textbf{H3DE-Net (Anchor)} and the baseline methods for both MRE and SDR metrics across all dataset configurations. The results confirm that the observed improvements are statistically significant, with $p$-values consistently below 0.01 ($p < 0.01$). These findings provide strong evidence that the anchor-based design contributes meaningfully to the overall performance enhancements.

% \noindent Specifically, for the \textit{All Cases Dataset}, the paired \textit{t-test} revealed a statistically significant reduction in MRE from $2.01 \pm 4.33$~mm (best-performing baseline, \textbf{ABRM}) to $1.70 \pm 0.72$~mm (proposed method). Similarly, the improvements in SDR at both 2mm and 4mm thresholds were statistically validated, with the anchor-based design achieving a mean SDR (4mm) of 95.20\% compared to 90.73\% for the baseline (\textit{t-test}, $p < 0.01$).

% These statistical analyses substantiate the performance gains introduced by the anchor-based design, affirming its effectiveness in improving both accuracy and robustness in landmark localization tasks.

