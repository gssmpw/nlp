\section{Conclusion}
This paper proposes H3DE-Net, a novel hybrid framework for 3D landmark detection in medical images. By combining the strengths of CNNs and Transformers, the model effectively addresses the challenges of volumetric data, such as sparse landmark distribution, complex anatomical structures, and multi-scale dependencies. The CNN backbone ensures efficient local feature extraction and multi-scale representation, while the 3D BiFormer module leverages a bi-level routing attention mechanism to efficiently model global context with reduced computational overhead. Additionally, integrating the feature fusion module further enhances the model’s robustness and precision. Extensive experiments on a public dataset demonstrate that H3DE-Net significantly outperforms existing methods, achieving state-of-the-art accuracy and robustness. The proposed model excels in challenging scenarios, such as missing landmarks or complex anatomical variations, making it a promising approach for real-world clinical applications.

Future work will explore further optimization of computational efficiency and model scalability to address even larger datasets and more diverse imaging modalities. Moreover, the potential of extending the framework to multi-task learning scenarios, such as combining landmark detection with segmentation or registration, offers exciting opportunities for advancing medical image analysis.


% \textbf{Acknowledgments}
% This study was supported by:
% \begin{itemize}
%     \item National Social Science Foundation in 2024 (No. 24BMZ101)
%     \item 2023 Project of the 14th Five-Year Plan for Scientific Research of the State Language Commission (No. YB145-73)
% \end{itemize}


% \textbf{Conflict of Interest Statement}
% The authors have no relevant conflicts of interest to disclose.


% \textbf{Data Availability Statement}
% Data and code are already open-source.