% -----------------------------------------------
% Template for ISMIR Papers
% 2023 version, based on previous ISMIR templates

% Requirements :
% * 6+n page length maximum
% * 10MB maximum file size
% * Copyright note must appear in the bottom left corner of first page
% * Clearer statement about citing own work in anonymized submission
% (see conference website for additional details)
% -----------------------------------------------

\documentclass{article}
\usepackage[T1]{fontenc} % add special characters (e.g., umlaute)
\usepackage[utf8]{inputenc} % set utf-8 as default input encoding
\usepackage{ismir,amsmath,cite,url}
\usepackage{graphicx}
\usepackage{color}


\usepackage{lineno}
%\linenumbers

% Title. Please use IEEE-compliant title case when specifying the title here,
% as it has implications for the copyright notice
% ------
% \title{Prioritizing register over pitch values in popular music: the example of the Roland TR-808 bass drum}


\title{Harmonic and Transposition Constraints Arising from the Use of the Roland TR-808 Bass Drum}

% Note: Please do NOT use \thanks or a \footnote in any of the author markup

% Single address
% To use with only one author or several with the same address
% ---------------
\oneauthor
 {Emmanuel Deruty}
 {Sony Computer Science Laboratories, Paris, France; Aalborg University, Denmark \\ {\tt emmanuel.deruty@sony.com}}

% Two addresses
% --------------
%\twoauthors
%  {First author} {School \\ Department}
%  {Second author} {Company \\ Address}

% Three addresses
% --------------\input{ISMIR2021_paper.tex}

%\threeauthors
%  {First Author} {Affiliation1 \\ {\tt author1@ismir.edu}}
%  {Second Author} {\bf Retain these fake authors in\\\bf %submission to preserve the formatting}
%  {Third Author} {Affiliation3 \\ {\tt author3@ismir.edu}}

% Four or more addresses
% OR alternative format for large number of co-authors
% ------------
%\multauthor
%{First author$^1$ \hspace{1cm} Second author$^1$ \hspace{1cm} Third author$^2$} { \bfseries{Fourth author$^3$ \hspace{1cm} Fifth author$^2$ \hspace{1cm} Sixth author$^1$}\\
%  $^1$ Department of Computer Science, University , Country\\
%$^2$ International Laboratories, City, Country\\
%$^3$  Company, Address\\
%{\tt\small CorrespondenceAuthor@ismir.edu, PossibleOtherAuthor@ismir.edu}
%}

% Please enter author names for the author list in the Creative Common license. 
% Please abbreviate authors' first names and add 'and' between the second to last and last authors.
\def\authorname{E. Deruty}

% Optional: To use hyperref, uncomment the following.
%\usepackage[bookmarks=false,pdfauthor={\authorname},pdfsubject={\papersubject},hidelinks]{hyperref}
% Mind the bookmarks=false option; bookmarks are incompatible with ismir.sty.

\sloppy % please retain sloppy command for improved formatting

\begin{document}

%
\maketitle
%
%%%%%%%%%%%%%%%%%%%%%%%%%%%%%%%%%%%%%%%%%%%%%%%%%%%%%%%%%%%%%%%%%%%%%%%%%%% Abstract

\begin{abstract}

% recent

%In popular music production, a documented technique involves tuning the drums to match the song's key. This method explicitly targets the kick drum and the Roland TR-808 bass drum. 

The study investigates hip-hop music producer Scott Storch's approach to tonality, where the song's key is transposed to fit the Roland TR-808 bass drum instead of tuning the drums to the song’s key. This process, involving the adjustment of all tracks except the bass drum, suggests significant production motives. The primary constraint stems from the limited usable pitch range of the TR-808 bass drum if its characteristic sound is to be preserved. The research examines drum tuning practices, the role of the Roland TR-808 in music, and the sub-bass qualities of its bass drum. Analysis of TR-808 samples reveals their characteristics and their integration into modern genres like trap and hip-hop. The study also considers the impact of loudspeaker frequency response and human ear sensitivity on bass drum perception. The findings suggest that Storch’s method prioritizes the spectral properties of the bass drum over traditional pitch values to enhance the bass response. The need to maintain the unique sound of the TR-808 bass drum underscores the importance of spectral formants and register in contemporary popular music production.

% about 150-200 words, it is 136

\end{abstract}

%%%%%%%%%%%%%%%%%%%%%%%%%%%%%%%%%%%%%%%%%%%%%%%%%%%%%%%%%%%%%%%%%%%%%%%%%%% Abstract end



%%%%%%%%%%%%%%%%%%%%%%%%%%%%%%%%%%%%%%%%%%%%%%%%%%%%%%%%%%%%%%%%%%%%%%%%%%% Introduction
\section{Introduction}
\label{sec:intro}

% In popular music, a practice consists of tuning the drums to the song's key \cite{toulson2009perception}. However, in a 2007 interview \cite{zisook2007}, R\&B producer Scott Storch suggests that during the production of music involving a Roland TR-808 drum machine, it may be beneficial to do the opposite and transpose the song key to make it fit the 808 bass drum \cite{storch2007}. The process involves transposing all the tracks (but one) in a song to make them fit a single one (the 808 bass drum). The recourse to such a potentially time-consuming set of operations suggests that the producer has strong motives to implement it.


%In popular music, a practice consists of tuning the drums to the song's key \cite{toulson2009perception}. However, in a 2007 interview \cite{zisook2007}, R\&B producer Scott Storch suggests that during the production of music involving a Roland TR-808 drum machine, it may be beneficial to do the opposite and transpose the song key to make it fit the 808 bass drum \cite{storch2007}. The process involves transposing all the tracks (but one) in a song to make them fit a single one (the 808 bass drum). Storch's motive is to conserve the perceptual features of the bass drum -- a motive strong enough to resort to such a potentially time-consuming set of operations.

In popular music, a common practice is to tune the drums to the song's key \cite{toulson2009perception}. However, in a 2007 interview \cite{zisook2007}, R\&B producer Scott Storch suggests that during the production of music involving a Roland TR-808 drum machine, it may be beneficial to do the opposite and transpose the song's key to fit the 808 bass drum \cite{storch2007}. The process involves transposing all the tracks but the bass drum. Storch's motive for undertaking such a potentially time-consuming set of operations is to conserve the characteristic sound of the 808 bass drum.


 
% Storch's assertion implies that if he goes as far as transposing all the tracks (but one) in a song to make them fit a single one (the 808 bass drum), then he must find the 808 bass drum part \textcolor{blue}{and either its pitch value or its register to be of paramount importance.} %\david{...but because of the effect of the pitch on the match between the drum spectrum and the sensitivity curves of loudspeakers and auditory systems, not because of the absolute pitch itself, suitable?}
 
 The present study investigates aspects of the music production process that may explain Storch's position. In Section~\ref{sec:drumsandtuning}, we shortly address the issue of drum tuning in popular music. In Section~\ref{subsec:808andproduction}, we provide an overview of the importance and usage of the Roland TR-808 in popular music production, focusing on its bass drum voice. In Section~\ref{subsec:808analysis}, we analyze the content of TR-808 bass drum samples. In Section \ref{subsec:subbassand808}, we relate spectral features of TR-808 bass drum samples to a diachronic analysis of the power spectrum in popular music. In Section~\ref{sec:tuningto}, we can understand Storch's position by involving the frequency response of loudspeakers and the sensitivity of the human ear. Finally, in Section~\ref{sec:pitchandregister}, we discuss how the practice suggested by Storch may be a particular case of how properties of the spectrum might be considered more important than pitch values.




%%%%%%%%%%%%%%%%%%%%%%%%%%%%%%%%%%%%%%%%%%%%%%%%%%%%%%%%%%%%%%%%%%%%%%%%%%% Drums and tuning
\section{Drums and tuning}\label{sec:drumsandtuning}

The musical signal has been divided into two categories: ``percussion has a short temporal duration and is rich in noise, while harmonic elements have a long temporal duration with most of the signal energy concentrated in pitch spikes'' \cite{rump2010autoregressive}. ``The harmonic and percussive components of music signals have much different structures in the power spectrogram domain, the former is horizontal, while the latter is vertical'' \cite{ono2008real}. These observations are the basis for source separation methods distinguishing “drums” from “pitched instruments” \cite{fitzgerald2010harmonic,yoo2010nonnegative}.


Yet, drums can contain pitched content \cite{richardson2010acoustic}. In drum sounds, relations between eigenfrequencies are not necessarily harmonic \cite{antunes2017possible}. “The tonal elements in drums are usually not structured like partials in a harmonic series. Instead, their frequency relationship can range from inharmonic to chaotic” \cite{wu2018review}. From a music producer's perspective, ``drums make several different notes simultaneously'' \cite{Roberts}.

Recent source separation methods do not involve prior hypotheses. They're based on models trained on actual data. Listening to the audio output stemming from such technology indicates that drum stems extracted from popular music do contain pitch. Demonstrations of Steinberg's SpectraLayers \cite{Attack2022}, Native Instruments' iZotope RX 8 \cite{Attack2022}, iZotope RX 9 \cite{Loose2022}, and StemRoller \cite{Stemroller2023}, provide relevant examples. 

If drums contain pitched content, they can be tuned. In popular music, the ``[i]ntricate tuning of acoustic drums can have a significant impact on the quality and contextuality of the instrument'' \cite{toulson2009perception}. % when played live or in the recording studio 
There is no consensus on how to tune drums: ``[t]alk to ten different drummers and you’ll get ten different ways to tune drums [...] there’s actually no wrong or right way to tune a drum, or right or wrong pitches to tune it to'' \cite{drummagazine}. %An added difficulty to the tuning of drums lies in the fact that ``[t]he pitch changes when a drum is struck with minimal force [or] an increased force'' \cite[p. 32]{richardson2010acoustic}. Such a problem does not apply in electronic or programmed drums, in which the producer chooses the drum sample.






Scott Storch is an American record producer and songwriter. Storch has been referred to as a ``producer that changed the R\&B game'' \cite{Singleton2022}, a ``superproducer'' \cite{chapman2008ill}, \textit{i.e.} a wave of artists ``who have established a new degree of visibility for the rap producer, earning star billings virtually equal in prominence to the artists that they produce'' \cite{Levine2003}. For Scott Storch, drum tuning is an integral part of the music production process:

\begin{quote}
%``I mean, sometimes we think we’re hearing it right, and then sometimes you gotta go and fine-tune it [...]
``I know there’s a lot of producers [who will] put an 808 in the song, and there will be chords and stuff clashing with it, and [...] if [...] your ears are really in tune with that stuff, you realize it’s just like [``not so convincing'' kind of gesture]... Sometimes, it actually does something cool to the track, but [...] I like to [...] get into that and tune the kick to match [...] the bass line or whatever the chords are doing [...], I just try different stuff... then... [even when there is] not an incredible amount of tune that carries over regular kicks, like short kicks, and I find myself sometimes at least even trying to tune [...] a regular [...] kick drum sound, and get it close to where most of the chords are in the song...'' \cite[0:30]{zisook2007}
\end{quote}

In the above, Storch mentions the Roland TR-808 bass drum and testifies to tuning bass drums to match the music's key.


%%%%%%%%%%%%%%%%%%%%%%%%%%%%%%%%%%%%%%%%%%%%%%%%%%%%%%%%%%%%%%%%%%%%%%%%%%% Roland TR-808
    \section{The Roland TR-808}\label{subsec:808andproduction}

%\subsection{808 and music production}\label{subsec:808andproduction}

The Roland TR-808 Rhythm Composer is an analog drum machine manufactured between 1980 and 1983 \cite{hasnain2017tr}. It is ``one of the most influential and unique drum machines of its time'' \cite{meyers2003tr}. ``To this day, the 808 remains a benchmark against which all other analog drum machines are measured'' \cite{werner2014physically}. It can be found in many music genres. The TR-808’s distinctive presets are classic sounds in hip-hop, techno, electro, R\&B, and house music \cite{dayal2014tr}. The 808 ``play[ed] a central role in the development of acid house''\cite{werner2014physically}. Pop music star Phil Collins used it throughout his entire career \cite[1:21:28]{Dunn2015}. It is ``a fixture in hip-hop culture, not only as a tool for producers but as a defining sound of the genre'' \cite{hasnain2017tr}. According to Scott Storch, in modern trap music, producers ``live in an 808 world'' \cite{storch2022}. One reason for the success of the 808 resides in the fact that ``it sounded like nothing else [...] and this is what made it so distinctive'' \cite{carter1997tr}. %According to Phil Collins, its sounds were ``unlike any other'' \cite[1:21:28]{Dunn2015}.
Perhaps as a result, the 808 has been seen not only as a drum machine but as an ``instrument in its own right'' \cite[0:06:51]{Dunn2015}.


%``The 808 is the critical element designed for huge speakers and club systems'' \cite{hasnain2017tr}, 
 %Bomb Squad's producer Hans Schoklee testifies that at the end of the '80s, ``every record had to have an 808 in it to have any sort of success in the [New-York] dance floor'' \cite[0:17:00]{Dunn2015}.


One notable voice of the 808 is its ``long and velvet deep, almost subsonic'' bass drum \cite{carter1997tr}, which can be made into a ``multi-second-long decaying pseudo-sinusoid with a characteristic sighing pitch'' \cite{werner2014physically}. According to producer Pharrell Williams, the 808 bass drum ``filled a massive void in the sound spectrum that was not there [...] once the 808 started to occupy that space, it became like something that you missed if you did not have it'' \cite[1:20:52]{Dunn2015}.

Over time, the 808 bass drum became used as both kick drum and bass. According to producer Remi Kabaka Jr., ``the kick drum would play the bass at the same time [...] there was drums and there was bass, but now the two were sort of fused, so the fill was not just complex and rhythmical, but it was also tonal'' \cite[1:11:11]{Dunn2015}. Musician and writer Alex Lavoie notes that ``[i]n most contemporary music genres, especially in trap and hip-hop, the 808 often carries the bassline, providing both the low-end foundation and outlining the harmonic progression of the song'' \cite{lavoie2020}. Musician and producer Charles Burchell writes that the TR-808 ``brings a sound closer to a traditional bass line while retaining the power of a drum [...]  In many cases, producers will not use a kick drum sample. Instead, they program drum patterns with a tuned 808 as the kick drum'' \cite{burchell2022}.

As a tonal instrument, the 808 bass drum can be tuned: as Lavoie states, ``[a]n 808 kick, particularly when it has a long decay, effectively functions as a bass instrument. That’s why tuning your 808s is so crucial'' \cite{lavoie2020}. %\textcolor{blue}{Burchell observes that  ``[l]ike the timpani, you can tune the 808 bass drum to different pitches'' \cite{burchell2022}}.
Lavoie warns that ``[i]f the pitch of your 808 kick doesn’t match the key of your song, it can create a dissonant effect'' \cite{lavoie2020}.

% Music algorithm engineer Gavin Burke writes that ``producers were loading up their samplers and playing 808 kicks no longer as drum samples but as full-on booming bass lines'' \cite{burke2019}. 

%%%%%%%%%%%%%%%%%%%%%%%%%%%%%%%%%%%%%%%%%%%%%%%%%%%%%%%%%%%%%%%%%%%%%%%%%%% 
%% end introduction
%%%%%%%%%%%%%%%%%%%%%%%%%%%%%%%%%%%%%%%%%%%%%%%%%%%%%%%%%%%%%%%%%%%%%%%%%%% 

\section{The 808 bass drum}\label{sec:808bassdrum}

\subsection{Signal analysis of 808 bass drum samples}\label{subsec:808analysis}

Figure~\ref{fig:TR_Waveform} shows the waveform corresponding to the ``TR808 BD Bass Drum Long 01'' preset. All samples considered in this paper originate from the TR-808 Trisample library \cite{trdownload}. The waveform confirms that the sample is tonal. The tonal aspect derives from the TR-808 generation technique, during which an oscillator produces a sawtooth wave that is filtered to make it close to a sine wave \cite{reid2002bass}.

\begin{figure}[htbp]
  \centering
  \includegraphics[width=.9\columnwidth]{imgs/Waveform.png}
  \caption{\it ``TR808 BD Bass Drum Long 01'' sample, waveform.}
\label{fig:TR_Waveform}
\end{figure}

Figure~\ref{fig:TR_STFT} shows the STFT for the same sample. Harmonics are present near the start of the sample and then fade out. The sample’s pitch value is briefly higher near the beginning, then decreases to a stable value. A study of the 37 ``long'' samples from the Trisample library shows that the median range for the initial frequency sweep is close to one half-tone.

\begin{figure}[htbp]
  \centering
  \includegraphics[width=1\columnwidth]{imgs/TR808_BD_Bass_Drum_Long_01_stfft_upload.png}
  \caption{\it``TR808 BD Bass Drum Long 01'' sample, STFT. The horizontal lines follow the fundamental and harmonics. The blue line stops when the energy of the corresponding bin is lower than 0.7 times the peak energy of all bins. The red lines stop when the energy of the corresponding bin is lower than 0.5 times the peak energy of all bins. }
\label{fig:TR_STFT}
\end{figure}

 The Tristar library features ``driven'' samples (a reference to the slang term ``drive'' for ``overdrive'', \textit{i.e.} ``distortion''). Figure~\ref{fig:TR_driven_STFT} shows the STFT for one ``driven'' sample. The threshold conditioning the display of the partials as red lines is the same as in Figure~\ref{fig:TR_STFT}, which indicates that the distortion boosts the level of the overtones.

\begin{figure}[htbp]
  \centering
  \includegraphics[width=1\columnwidth]{imgs/TR808_BD_Bass_Drum_Driven_01_stfft_upload.png}
  \caption{\it ``TR808 BD Bass Drum Driven 01'' sample, STFT.}
\label{fig:TR_driven_STFT}
\end{figure}

Figure~\ref{fig:Mask_off} shows the STFT for an extract from the 2017 song ``Mask Off'', by the American rapper Future. The track has been described as an example of heavy 808 use~\cite{hasnain2017tr}. The initial frequency sweep on each bass drum occurrence is similar to the samples shown in Figures~\ref{fig:TR_STFT} and~\ref{fig:TR_driven_STFT}. The vertical distribution of high energy values at the beginning of each bass drum occurrence suggests that the 808 is superimposed with a noisier kick drum. The 808 samples are tuned to the song’s tonality (D minor). The pitch values (D1 and B$\flat$0) are very low: they stand one minor second and one perfect fourth above the piano's lowest note. The corresponding frequency range (ca. 40Hz) recalls the ``almost subsonic'' aspect of the 808 bass drum samples \cite{carter1997tr}.

%From Figure~\ref{fig:Mask_off}, we can draw several observations. 

%\begin{enumerate}
    
%    \item The initial frequency sweep on each bass drum occurrence is similar to the ``driven'' sample example. %It is homogeneous, with the bass drum part in ``Mask Off'' being indeed produced using an 808.
    
%    \item The vertical distribution of high energy values at the beginning of each bass drum occurrence suggests that the 808 is superimposed with a noisier kick drum.% As a result, the 808 bass drum assumes the role of a melodic bass, not that of a bass drum.
    
 %   \item The track's key is D minor. The 808 samples are tuned to the song’s tonality.
    
 %   \item The pitch values (B$\flat$0 and D1) are very low. They respectively stand one minor second and one perfect fourth above the piano's lowest note. The corresponding frequency range (ca. 40Hz) recalls the ``almost subsonic'' aspect of the 808 bass drum samples \cite{carter1997tr}.
    
%\end{enumerate}

\begin{figure}[htbp]
  \centering
  \includegraphics[width=1\columnwidth]{imgs/mask_on_stfft_annot.png}
  \caption{\it Future, ``Mask Off'', 8 beats from 0'25 to 0'30, STFT. The vertical lines denote the kick drum’s onsets. The horizontal lines follow the TR’s fundamental and harmonics. The corresponding pitch values are shown at the top.}
\label{fig:Mask_off}
\end{figure}


%%%%%% subSection sub bass freqs and 808 bass
\subsection{Sub-bass frequencies and the 808 bass drum}\label{subsec:subbassand808}

Producers recognize three distinct regions of sub-bass: the ``boom'' (ca. 30Hz), the ``thump'' (ca. 50Hz) and the ``punch'' (ca. 80Hz) \cite[pp. 88--118]{fink2018relentless}\cite[p. 282]{fink2020boom}.  Figure~\ref{fig:Dists} confirms that 50Hz (the ``thump'') is the ``frequency range occupied by the Roland TR-808 analog kick'' \cite{fink2020boom}.

Before the advent of digital audio, low frequencies were attenuated to protect amplifiers and speakers from the adverse effects of mechanical noise and harmonic distortion \cite[p. 282]{fink2020boom}\cite{read1952reproduction,millard2005america}. Musical information in this frequency range only became possible by using digital audio as a medium. Figure~\ref{fig:specs} shows the evolution of the power spectrum in popular music. The measures were derived from a dataset containing 30435 tracks released between 1961 and 2022. The choice of the tracks stems from the ``Best Ever Albums'' website, a review aggregator that proposes the best-rated albums for each year of production \cite{BEA}. For each year, we select the best-rated albums. The overall spectral profile is consistent with Pestana's results \cite{pestana2013spectral}. The increase of energy in the lower band, also testified by Hove et al. \cite{hove2019increased}, is concomitant to the advent of digital audio. %, which we now summarize.



%The Sony PCM-3324, the first widely accepted multi-track digital recorder, was released in 1982 \cite{Sony}. The same year, the first music album was released on CD \cite{CD}. A portable recordable digital tape format, the DAT, was introduced in 1987 \cite{DAT}. In 1994, the Fraunhofer Society released the first software MP3 encoder \cite{MP3}. ``Software environments for music production, or Digital Audio Workstations (DAWs), have since the early 2000s become central to the creation of commercially released music'' \cite{marrington2017composing}. %Amongst them, ProTools was introduced in 1991 \cite{ProTools}; Cubase VST, the first Cubase version to handle audio recording, was released in 1997 \cite{Cubase}; Ableton’s Live was introduced in 2001 \cite{Live}; and Reaper was first released in 2006 \cite{Reaper}.

%In 2001, Théberge \cite{theberge2001plugged} notes that ``[t]he micro-manipulation of digital audio has become more and more the primary focus of contemporary recording practice''.

\begin{figure}[h] %htbp
  \centering
  \includegraphics[width=1\columnwidth]{imgs/distribution_upload.png}
  \caption{\it Distribution of fundamental frequencies of TR-808 bass drum samples. The fundamental frequencies are evaluated on 0.2-second windows. The contribution of each window is weighted according to the energy at the fundamental frequency. In the non-``driven'' presets, the maximum of the distribution corresponds to $f_0 = 49.48$~Hz. The $f_0$ values for the ``driven'' presets are higher. ``Short'' presets involve a secondary local maximum ($f_0=51.05$~Hz) corresponding to the samples' earliest windows.}
\label{fig:Dists}
\end{figure}

\begin{figure}[h] % htbp
  \centering
  \includegraphics[width=1\columnwidth]{imgs/spectra_10bands_upload.png}
  \caption{\it Evolution of the power spectrum in popular music. Top, raw energy values. Bottom, values for each frequency band are normalized to the same mean.}
\label{fig:specs}
\end{figure}


The analysis results shown in Figure~\ref{fig:TR_STFT} indicate that after the initial 0.4s-long attack, ``long'' 808 samples are based on a single low-frequency sine wave. The sine wave's frequency is ca. 50~Hz according to  Figure~\ref{fig:Dists}. The results shown in Figures~\ref{fig:TR_driven_STFT} and~\ref{fig:Mask_off} suggest that this very low frequency remains an essential component of the 808 bass drum with added harmonics. Confronting these observations with the power spectrum evolution in popular music (Figure~\ref{fig:specs}), it follows that the sound of the 808 bass drum was not fully reproduced before the end of the '90s, even though the machine itself was sold between 1980 and 1983.

%Section \ref{subsec:808analysis}
%Section \ref{subsec:808analysis},

After 2009, ``the characteristic 808-kick drums [...] started entering mainstream music in general'', and trap music, a ``tradition of rap that developed during the 1990s'', an ``808 world'' according to Scott Storch (see Section~\ref{subsec:808andproduction}), ``began to reach strong presence on the mainstream Billboard music charts'' \cite{kaluvza2018reality}. So strong is the presence of trap in the charts that this formerly underground genre has been qualified as ``pop'', in the sense that ``[p]eople’s ears have adjusted'' to it \cite{Lee2017}.

%To summarise, 
The extended bandwidth provided by the emergence of digital audio made possible the faithful restitution of the entire spectrum of the 808 bass drum, which favored the birth and rise of a music genre that became mainstream and influenced popular music in general. %\david{Again, I don't see how the data in Figure 6 supports this general claim. Isn't it possible that what we observe in the lowest two bands in Figure 6 resulted from something other than the TR-808?}



%%%%%%%%%%%%%%%%%%%%%%%%%%%%%%%%%%%%%%%%%%%%%%%%%%%%%%%%%%%%%%%%%%%%%%%%%%% Register over pitch
\section{Tuning the song's key to the TR-808 bass drum}\label{sec:tuningto}


In Section~\ref{sec:drumsandtuning}, Scott Storch describes how he tries to tune the bass drum (808 in particular) to the music's key. Later in the same interview, Storch suggests that instead of tuning the 808 bass drum sample to the song's tonality, one can do the opposite and adjust the song's key to the 808 bass drum sample:

\begin{quote}
``[S]ometimes, producers will program a song in a certain key, and they’ll try to program an 808 under it, and it’s like the key of the song is almost too low to really let speakers do what they need to do with the bass so, I recommend [...] modulating the song up, transposing it up a couple of keys, and you’ll be surprised how much more level you can get out of the song. [Because] anything really below […] a low E [...], it’s like the speakers are gonna not, let you turn it up, you don’t feel the bass response.'' \cite[1:37]{zisook2007}
\end{quote}

Storch describes a situation in which a producer previously set the key for a song, tunes an 808 bass drum to make it fit the key, and, as a result, the 808 bass drum does not sound ``right''. 

\subsection{Transposition of the TR-808 bass drum: effect on the lowest partial}\label{subsec:fundamental}

Let us consider an example where the song's key is D, as in the extract from Figure~\ref{fig:Mask_off}. We focus on the fundamental, the only lasting component in samples from the ``long'' type (Figure~\ref{fig:TR_STFT}). As seen in Figure~\ref{fig:Dists}, the $f_0$ of an 808 bass drum is ca. 49.5Hz, corresponding to a G1. The producer, therefore, transposes the 808 bass drum one perfect fourth down (5 semitones) to a D1 -- one tone below the ``low E'' mentioned by Storch. Storch states that the loudspeakers may not reproduce the bass correctly in such a situation.

%Section \ref{subsec:808analysis}, 
%Section \ref{subsec:808analysis}, 
%Section~\ref{subsec:subbassand808},

Professional mixing engineers mainly use near-field monitors \cite[p. 3]{senior2011mixing}. With such monitors, they can produce ``masters which `travel' well to their use by the record buyers'' \cite{newell2001yamaha}. The use of near-field monitors extends to producers. %at least some
%For instance, 
Nigel Godrich testifies that during the production of Radiohead's ``OK Computer'', he always used near-field monitors, but never the studios' main monitors, which ``don't relate to anything'' and are ``fairly useless'' \cite{Robinson1997}. In the many videos documenting his work, Storch can be seen using near-field monitors. % such as KRKs. %\footnote{Scott Storch is seen using them in ``Masterclass: becoming a hitmaker with Scott Storch,'' Chapter 10, ``Live studio session'', and in ``Best Of Scott Storch In The Studio [Part 2]'', \url{https://youtu.be/L-dADlXbX7M?t=117}}. 
% such as the Yamaha NS10 and the Acoustic Energy AE1

Newell et al. \cite{newell2001yamaha} provide the frequency response for 36 near-field monitoring loudspeakers. Figure~\ref{fig:Responses} graphs the median frequency response for these loudspeakers against the median $f_0$ for the 808 bass drum samples (49.5Hz / G1) and the TR bass drum median frequency transposed down one perfect fourth (37Hz / D1). The downward transposition results in a gain loss of 6.3 dB. Following Storch's suggestion and transposing up the song key instead of transposing down the 808 sample would avoid the 6.3dB loss. In Storch's terms, transposing the song up may ``let speakers do what they need to do with the bass''.

\begin{figure}[htbp]
  \centering
  \includegraphics[width=1\columnwidth]{imgs/speaker_bottomfreqs_minus5_upload.png}
  \caption{\it near field loudspeaker responses as a function of frequency. The center horizontal line in each box represents the median, and the two surrounding horizontal lines represent the 25\textsuperscript{th} and 75\textsuperscript{th} percentiles. The blue line shows the smoothed median response. The red vertical line represents the median $f_0$ for the 808 bass drum as shown in Section~\ref{subsec:subbassand808}, Figure~\ref{fig:Dists}. The gray rectangle denotes a -5 semitone transposition of the median $f_0$. The textual representation displays the difference in the response that occurs.}
\label{fig:Responses}
\end{figure}

Loudspeakers are not the only frequency-dependent transducers involved in the listening process. The human ear is also sensitive to frequency. In particular, as the frequencies get closer to the lower limit of human hearing, a sine wave with the same sound pressure level but a lower frequency will be perceived as less loud.
The phenomenon is described by equal loudness contours, representing the sound pressure levels at different frequencies that are perceived as equally loud \cite{fletcher1933loudness}. Figure~\ref{fig:ISO} graphs the ISO226-2003 \cite{iso2262003} equal loudness contours against the median 808 bass drum $f_0$, and the same frequency transposed down one perfect fourth. If we choose a loudness of 60 phon, a +5.5dB gain would be required so that the transposed $f_0$ remains at the same loudness. Therefore, considering the human ear as one of the transducers in the signal path, the gain it applies to the signal when transposing down the original median $f_0$ is ca. -5.5dB. As a result, the overall gain loss following the downward transposition originating from both the loudspeakers and the ear can be estimated to be ca. 11.8 dB. %Conversely, transposing up the song instead of transposing down the 808 bass drum would result in an approximate 11.8 dB gain.

\begin{figure}[htbp]
  \centering
  \includegraphics[width=1\columnwidth]{imgs/TRiso226_minus5_upload.png}
  \caption{\it Equal-loudness contours according to \cite{iso2262003}. The numbers superimposed on each contour indicate the loudness value corresponding to the contour (in phon). The red vertical line represents the median $f_0$ for the 808 bass drum as shown in Section~\ref{subsec:subbassand808}, Figure~\ref{fig:Dists}. The gray rectangle indicates a -5 semitone transposition of the median $f_0$ (one perfect fourth down). The textual representation displays the gain that would be required so that the transposed $f_0$ remains at the same 60-phon loudness.}
\label{fig:ISO}
\end{figure}

If different 808 bass drum notes result in different gains, then a sequence of different 808 bass drum notes will result in gain changes within the sequence. Quoting Storch, ``for 808s […] I try to stay in the comfort zone of the speaker, so I don’t [...] have the volumes jumping out for different notes'' \cite{Storch2022b}. In other words, 808 bass drum parts' pitch should remain largely static to achieve a stable gain. In turn, largely static bass pitch values may result in a limited variety of chords. The phenomenon illustrates how loudness stability may take precedence over harmonic complexity.

%See \cite{JID2023} for an example of a recent 808-based track involving long passages based on one chord.



\subsection{Transposition of the TR-808 bass drum: involvement of the harmonics}

The 808 bass drum samples corresponding to Figures~\ref{fig:TR_driven_STFT} and~\ref{fig:Mask_off} involve lasting harmonics. Figure~\ref{fig:both} shows the combined response deriving from both the near field loudspeakers and the ear's sensitivity at 60 phon. The lower the frequency, the greater the influence of transposition on the overall gain. The gain loss diminishes with each harmonic. It is almost zero for the fifth harmonic.


%The ``flattening'' of the overall response curve for increasing frequencies suggests that the transposition of the elements in the song that are not the 808 bass drum (in Storch's words, ``transposing [the song] up a couple of keys''), will result in gain changes that are not as significant as in the case of the 808 bass drum's fundamental.

%To better understand the influence of the transposition on the TR bass drum sample,
We generate a 49.5Hz five-partial harmonic complex tone. The amplitudes of the partials are the same as in the ``TR808 BD Bass Drum Driven 01'' sample when the frequency values reach a static regime (see Figure~\ref{fig:TR_driven_STFT}). The overall power change following a 5-semitone downward transposition is -4.5dB. It is much less than the -11.8 dB gain brought by the downward transposition of the lone fundamental. The result suggests that the issues mentioned by Scott Storch (gain conservation and gain stability) mainly concern the fundamental or, at least, the lowest harmonics. In other words, Storch is specifically concerned with the audibility and stability %of the perceived loudness 
of the 808 bass drum's bottom partials. 

%Section \ref{subsec:808analysis}

\begin{figure}[htbp]
  \centering
  \includegraphics[width=1\columnwidth]{imgs/All_minus5_upload.png}
  \caption{\it The black line shows the combined response deriving from both the near field loudspeakers and the ear's sensitivity at 60 phon (Figures \ref{fig:Responses} and \ref{fig:ISO}). The red vertical lines represent the median values for the 808 bass drum's fundamental and harmonics. The textual representations display the difference in the response that occurs from a -5 semitone transposition.}
\label{fig:both}
\end{figure}


%To understand the influence of transposition on the fundamental and harmonics of the 808 bass drum sample, we evaluate the influence of the gains shown in Figure~\ref{fig:both}. For this purpose, 




%Section \ref{subsec:808analysis}, 

The phenomenon known as the "missing fundamental" \cite{licklider1951duplex} suggests that even if a negative gain is applied to lower harmonics, the perceived pitch remains unchanged due to the auditory system's temporal pitch processing. Only timbre is affected. In the case of the sub-bass register, \textit{i.e.} frequencies lower than 100Hz according to Fink \cite[p. 281]{fink2020boom}, another perceptual aspect may be mentioned. In relation to findings by Takahashi et al. \cite{takahashi2002relationship}, Fink et al. \cite[pp. 88-118]{fink2018relentless} suggest that one aspect of the perceptual effects of bass stems from small body surface displacements. According to the author, each sub-bass range can be associated with a body region in which the corresponding frequencies are imaginatively felt. The ``boom'' (ca. 30Hz) is ``the semi-audible vibration in the gut felt during the deepest drops in dancehall and dubstep''. The ``thump'' (ca. 50Hz) is felt in the stomach, and the ``punch'' (ca. 80Hz) in the chest. Even when listeners use headphones, bass frequencies may be associated with a ``tactile sensation'' \cite{hove2020feel}. Fink \cite{fink2020boom} and Hove et al. \cite{hove2020feel}'s views suggest that low frequencies may play a role beyond pitch and timbre, in this case, a haptic role. 

A downward transposition and the resulting negative gain applied to these frequencies may affect both the resulting timbre and bodily sensation. As a result, they may be prejudicial to at least some music genres, independently from the presence of upper harmonics.

%\david{There are both place and periodicity mechanisms that generate a sense of pitch - missing fundamental is only to do with periodicity, if I remember correctly. Could be that other pitches gain relative strength by moving the spectrum.} 

%\david{Indeed, makes us aware that we should consider the experience as haptic as well as auditory.} 

%“As you start working with these sounds, you start hearing what speakers are responding to the best, and what stuff jumps out”

%Insert ref of subbass
%MaxxBass plug-in \cite{white1988maxxbass}
%``If you mix specifically for compact speakers and filter out a lot of the existing bass, you'll end up with a mix that sounds more punchy on small speakers, but somewhat 'tubby' and lacking in deep bass if played back on big speakers. It's therefore probably best, assuming that most people have speakers that extend to around 60Hz, to drop the existing bass by 3dB or so, then add just enough harmonics to restore the subjective level of bass you had to begin with. This should give you a mix that works reasonably well on all speakers.''

%Insert ref from Luc, TR plug-in



%\begin{figure}[htbp]
%  \centering
%  \includegraphics[width=1\columnwidth]{imgs/speaker_bottomfreqs_upload.png}
%  \caption{Loudspeaker responses as a function of frequency. In each box, the center horizontal line represents the median and the two surrounding horizontal lines represent the upper and lower limits of the interquartile range. The blue line shows the smoothed median response. The red vertical line represents the median $f_0$ for the 808 bass drum as shown in Section~\ref{subsec:subbassand808}, Figure~\ref{fig:Dists}. The gray rectangle indicates a +/- 7 semitone transposition of the median $f_0$ (one perfect fifth). The textual representation displays the difference in the response that occurs.}
%\label{fig:Responses}
%\end{figure}




\section{Pitch and register}\label{sec:pitchandregister}


Scott Storch's advice according to which a song's key may be adjusted to the 808 bass drum sample is based on the following premise: the transposition of the elements of the music that are not the 808 bass drum is less problematic than the transposition of the 808 bass drum. %Transposition changes both the pitch values and the register.
The change in pitch values does not affect the musical intervals, but the shift in register affects the perceived spectral profile. The change in perceived spectral profile is more important in the case of the TR-808 bass drum due to its low-frequency content. %In light of Section~\ref{sec:tuningto}, we now understand why Storch may consider the distortion of the spectral profile to be a critical problem.

%In light of Section~\ref{sec:tuningto}, the negative consequence of transposition that Storch refers to can be expressed as appearances of frequency-dependent gains. 

%is a particular case of giving priority to register over pitch values. Transposition affects both pitch and register, and Storch suggests that at least in some cases, a change in pitch values is preferable to 

%The Section discusses the relationship between pitch and register in music and how transposition


%Section~\ref{sec:tuningto}
% based on rational interval proportions'' that ``
Following Frisius \cite[p. 81]{frisius2010search}, ``a [music] theory [that] posits a principle of neutral transposition, according to which groups of pitches essentially do not change their character if one transposes them'', does not take into account the transposition of the sounds themselves. Frisius remarks that such a theory may not be suited to music from the 20\textsuperscript{th} century. He mentions the composer Luigi Russolo, who found it difficult to ``transpos[e] sonic gestures into other registers without losing their identity''. According to Frisius, this difficulty is ``felt above all when pitch is not clearly definable''. One way to understand the phenomenon is that transposed melodies are only the ``same as'' each other because they are constructed using a set of pitches whose chromas repeat at the octave. The listener encodes them in terms of pitch sequences. If two sounds are ``transpositions'' of each other but are not perceived in terms of pitch, then they are just different sounds. %Frisius cites similar findings from Pierre Schaeffer. A difference between Schaeffer's and Frisius' points of view resides in the fact that the latter considers that the transposition of content with clear pitch does modify its character \cite[pp. 14, 34-35, 141]{schaeffer2012search}.

%\david{This makes sense. Transposed melodies are only the `same as' each other because they are melodies constructed using a set of pitches whose chromas repeat at the octave and we encode them in terms of pitch sequences. If two sounds are ``transpositions'' of each other but are not perceived in terms of pitch, then they are just different sounds.}

%These observations highlight the fact that transposition affects both pitch and register.
Pitch intervals may be robust to transposition, but the register will change, and so will timbre. The phenomenon has been described in orchestration treatises \cite{berlioz1844grand,gevaert1885nouveau,koechlin1954traite}. %in which the authors associate provide descriptions for tonalities.
According to Hector Berlioz, as far as the violin is concerned, C major may be ``\textit{grave, mais sourd et terne}'' (rich in low frequencies, but dull and muted), and F\# minor ``\textit{tragique, sonore, incisif}'' (tragic, resonant, incisive) \cite{berlioz1844grand}.
More recently, Reymore et al. \cite{reymore2023timbre} have studied the relation between pitch height and timbre in acoustic instruments. 

Personal interviews with music producers from the production company Hyper-Music (\url{https://www.hyper-music.com/}) suggest that in recent popular music, the simultaneous consideration for pitch values and register when considering transposition is paramount. According to one of Hyper-Music producers, Storch prioritizing the register of an instrument over particular pitch values is a ``basic rule'' of modern music production. Pitch is subservient to spectral formants. Priority is given to the absolute position of the formants in the spectrum. If pitch has to be changed so that the formants of the sound carrying the pitch reach the desired positions, it will be changed. Another producer from the same company claims to be always cautious with transposition, as it may affect timbre. In accordance with Frisius' point of view, if the pitch content of the part is not too strong, the same producer may simply forego transposition, even if the pitched content of the sample conflicts with other tonal elements. %Such a position is coherent with Frisius' point of view, 

In productions involving a TR-808 bass drum, Hyper-Music's producers often set the tonality to D or Eb to take full advantage of the bass drum's character. Such tonalities neighbor that of the example shown in Figure~\ref{fig:Mask_off}. %According to Figure~\ref{fig:Dists}, it corresponds to a downward transposition of the original TR-808 bass sample that approximates a perfect fourth.


%%%%%%%%%%%%%%%%%%%%%%%%%%%%%%%%%%%%%%%%%%%%%%%%%%%%%%%%%%%%%%%%%%%%%%%%%%% Conclusion
\section{Conclusion}

According to Section~\ref{sec:drumsandtuning}, some authors have previously divided the musical signal into two categories: percussion (rich in noise, short duration), and harmonic elements (long duration, most of the energy concentrated in spikes in the spectrum) \cite{ono2008real,rump2010autoregressive,yoo2010nonnegative}. However, other authors have studied the existence of pitch in percussion \cite{richardson2010acoustic,antunes2017possible,wu2018review}. In music production, drum tuning has been seen as essential \cite{toulson2009perception} but sometimes difficult \cite{richardson2010acoustic}. Scott Storch, a renowned music producer, has emphasized the importance of fine-tuning drums to match the music's key despite the inherent challenges in doing so.

In Section~\ref{subsec:808andproduction}, we showed that the Roland TR-808 Rhythm Composer has been deemed an influential analog drum machine \cite{meyers2003tr,werner2014physically,hasnain2017tr}, primarily known for its distinctive and deep bass drum sound \cite{carter1997tr}. Producers and musicians from various music genres have testified to its efficiency in providing low-end foundation. They use the 808 bass drum not only as a kick drum but also as a tonal instrument that plays basslines, thus emphasizing the importance of its tuning. 

Signal analyses of 808 bass drum samples reported in Section~\ref{sec:808bassdrum} show that its fundamental frequency can be found ca. 50Hz and may or may not have lasting harmonics. The measured evolution of the power spectrum in popular music suggests that digital audio technology enabled the faithful reproduction of the 808 bass drum's extended bandwidth, which played a crucial role in the rise of trap music's popularity and its subsequent influence on mainstream music \cite{kaluvza2018reality}. 

In Section~\ref{sec:tuningto}, we discussed tuning the song's key to the 808 bass drum. Producers often try to tune the bass drum to match the song's key. However, Scott Storch suggests an alternative approach: adjusting the song's key to fit the 808 bass drum sample. Storch explains that some songs might have a key that is too low for the bass to be correctly reproduced by speakers. Instead, he recommends transposing the music up to achieve a more balanced and powerful bass response. %The discussion delves into the technical aspects of this approach.
If, for instance, the bass drum is transposed down one perfect fourth to match the song's key, its fundamental frequency loses ca. 11.8 dB in overall gain, considering the response of near field loudspeakers of the type that producers customarily use \cite{newell2001yamaha,senior2011mixing} and the human ear's sensitivity to frequency \cite{iso2262003}. The loss may affect the instrument's timbre and invalidate the specific bodily sensations the sub-bass range may evoke \cite{fink2018relentless,fink2020boom,hove2020feel}. The analysis also suggests that the gain loss primarily affects the fundamental frequency and lower harmonics. % while higher harmonics are less affected.
The discussion emphasizes the importance of controlling the level of bass in music production. It suggests that adjusting the song's key to the 808 bass drum can indeed be a helpful technique to achieve this goal.

In Section~\ref{sec:pitchandregister}, we briefly discussed the relationship between pitch and register in music and how transposition may affect these elements. While classical Western music theory emphasizes the robustness of pitch intervals to transposition, other perspectives \cite{frisius2010search,reymore2023timbre} suggest that transposition has significant consequences on timbre. Orchestration treatises have long associated specific timbral characteristics with different keys, highlighting the importance of considering both pitch and register \cite{berlioz1844grand,gevaert1885nouveau,koechlin1954traite}. Recent interviews with popular music producers suggest the approach is significant in modern music production.

An intriguing research direction may stem from the assessment of one of the interviewees, according to which spectral formants have precedence over pitch values in modern popular music. Storch's handling of the 808 bass drum is an example of this principle. If such a claim proves to have merit, it may have consequences on music analysis and user interaction in generative systems applied to popular music. 




%% %%%%%%%%%%%%%%



%\section{Citations}

%All bibliographical references should be listed at the end of the submission, in a Section named ``REFERENCES,''
%numbered and in the order that they first appear in the text. Formatting in the REFERENCES Section must conform to the
%IEEE standard (\url{https://ieeeauthorcenter.ieee.org/wp-content/uploads/IEEE-Reference-Guide.pdf}). Approved
%IEEE abbreviations (Proceedings $\rightarrow$ Proc.) may be used to shorten reference listings. All references listed
%should be cited in the text. When referring to documents, place the numbers in square brackets (e.g., \cite{ISMIR17Author:01}
%for a single reference, or \cite{JNMR10Someone:01,Book20Person:01,Chapter09Person:01} for a range).

%\textcolor{red}{As submission is double blind, refer to your own published work in the third person. That is, use ``In the previous work of \cite{ISMIR17Author:01},'' not ``In our previous work \cite{ISMIR17Author:01}.'' If you cite your other papers that are not widely available (e.g., a journal paper under review), use anonymous author names in the citation, e.g., an author of the form ``A. Anonymous.''}

%\section{Acknowledgments}

%\textbf{Do not include in your submission, only in your camera ready version}. This Section can be used to refer to any
%individuals or organizations that should be acknowledged in this paper. This Section does \textit{not} count towards the 
%page limit for scientific content.


\section{Acknowledgments}

%\textcolor{blue}{[hidden]}

Many thanks to Yann Mac\'e and Luc Leroy from the music production company Hyper Music for their insights into Scott Storch's work and the subsequent discussions. Special thanks to David Meredith (Aalborg University) for his valuable comments.

%\newpage
\nocite{*}
\bibliographystyle{IEEEbib.bst}
%\bibliography{Main.bib} % requires file DAFx24_tmpl.bib
% For bibtex users:
\bibliography{ISMIRtemplate}

% For non bibtex users:
%\begin{thebibliography}{citations}
% \bibitem{Author:17}
% E.~Author and B.~Authour, ``The title of the conference paper,'' in {\em Proc.
% of the Int. Society for Music Information Retrieval Conf.}, (Suzhou, China),
% pp.~111--117, 2017.
%
% \bibitem{Someone:10}
% A.~Someone, B.~Someone, and C.~Someone, ``The title of the journal paper,''
%  {\em Journal of New Music Research}, vol.~A, pp.~111--222, September 2010.
%
% \bibitem{Person:20}
% O.~Person, {\em Title of the Book}.
% \newblock Montr\'{e}al, Canada: McGill-Queen's University Press, 2021.
%
% \bibitem{Person:09}
% F.~Person and S.~Person, ``Title of a chapter this book,'' in {\em A Book
% Containing Delightful Chapters} (A.~G. Editor, ed.), pp.~58--102, Tokyo,
% Japan: The Publisher, 2009.
%
%
%\end{thebibliography}

\end{document}

