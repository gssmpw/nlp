\section{Related Work}
\subsection{Cognitive Theory of Writing Process}

Research on human writing shifted from analyzing final written products to examining cognitive processes across various writing phases \cite{diederich1974measuring, Krapels1990SecondLW, macarthur2016writing}. 
Building on this process-oriented approach, \citet{f508427a-e4c0-3d6a-8abf-03a5d21ec6c4}’s cognitive writing theory outlines three key sub-processes: (1) \textit{planning}, (2) \textit{translating}, and (3) \textit{reviewing}. These interconnected and non-linear stages of human writing inform our work, where we expand this model to a more granular taxonomy of cognitive writing patterns in scholarly communication.

\citet{koo2023decoding} introduced a taxonomy of scholarly writing process based on keystroke data from short research plans written in 30-minute sessions. Expanding on this, we perform a larger-scale keystroke collection over months, whose final products culminated in research publications. Expert-reviewed and grounded in prior literature, our taxonomy encompasses end-to-end cognitive trajectories of scholarly writing.

% Research on human writing significantly shifted from focusing solely on the final written product to exploring the cognitive stages of various writing phases \cite{diederich1974measuring, perl1979composing, Krapels1990SecondLW, macarthur2016writing}. 

% Building on this process-oriented approach, \citet{f508427a-e4c0-3d6a-8abf-03a5d21ec6c4}'s cognitive writing theory outlines three key cognitive sub-processes grounded in empirical evidence: (1) \textit{planning}; (2) \textit{translating}; and (3) \textit{reviewing}. These processes are complex, interconnected, and non-linear, with writers moving fluidly between them. Our work expands this model to develop our own taxonomy of more granular cognitive patterns of human writing, specifically in scholarly communication.

% \citet{koo2023decoding} presented a taxonomy of the scholarly writing process based on keystroke collections of just a short research plan over a thirty-minute session. Inspired by this work, we extend their approach to a larger-scale collection of keystrokes over months, whose final written products turned into research publications. Reviewed by a linguistics expert, our comprehensive taxonomy is built on top of prior literature and encompasses end-to-end writing trajectories of the cognitive process in a scholarly domain. 

% Also, our collected keystrokes are created by more than 10 authors who work on a manuscript simultaneously.

% The cognitive process theory of writing \cite{f508427a-e4c0-3d6a-8abf-03a5d21ec6c4} presents three subprocesses of human writing: \textit{planning}, \textit{translation}, and \textit{reviewing}. Planning generates and organizes writing ideas and setting goals, translation implements their plans into tangible narratives, and reviewing evaluates and revises their texts adaptively. These phases are highly embedded within each other and not linear, meaning that planning and revision can call upon each other at the same time and writers can move back and forth to each other across the entire writing process. Thus, identifying the patterns of the entire cognitive process of human writing is a highly complex but essential process for developing writing assistants with optimal performance. 

% To meet such goal of understanding the human writing process, our work builds upon \citet{f508427a-e4c0-3d6a-8abf-03a5d21ec6c4} to identify certain schema of cognitive human writing process, specifically for scholarly communication in the scientific domain. \citet{koo2023decoding} presented a keystroke dataset of scholarly writing, but their studies engaged only four participants who wrote a short research plan individually as to the given prompts over a thirty-minute session. Our work extends to a larger-scale collection ($>100K$) of keystroke data over months of period, whose final products are now publicly available in a publication format. Also, those collected keystrokes are created by more than 10 authors who work on a manuscript simultaneously. Annotated and reviewed by a linguistics expert, our keystroke collection will better encompass the writing trajectory of scholarly scientific communication made through human collaboration. 

%-----------------------------------------------------

\subsection{Keystroke Loggers for Scholarly Writing}

Keystroke logging tools (e.g., Inputlog) allow researchers to observe digital writing without disrupting the writing process \cite{chan2017using, johansson2010looking, leijten2013keystroke, lindgren2019observing}. However, they are often restricted to a closed ecosystem, such as MS Word, and thus less accessible for scientific communities who use \LaTeX. Moreover, they are not well-suited for collecting data from extended writing sessions.
% Keystroke logging tools such as Inputlog allow researchers to observe digital writing without interfering with the writer's process \cite{chan2017using, johansson2010looking, leijten2013keystroke, lindgren2019observing}. However, they tend to operate in a closed ecosystem, i.e., only running in MS Word. This makes the current tools less accessible, especially for the scientific communities who use other typesetting such as \LaTeX. Moreover, it is not well-suited for collecting data from extended writing sessions. 

Current studies rarely examined scholarly communication that uses \LaTeX, particularly for scientific writing. To address this gap, we developed systems that securely collect real-time keystrokes over months from Overleaf, a widely-used online \LaTeX $\,$editor, while ensuring privacy. The system workflow enables span-level annotation of writing intentions, offering a natural, uninterrupted environment for studying cognitive writing activities over long periods and across multiple sections.
% Currently, such keystroke logging studies have rarely examined scholarly communication settings that use \LaTeX, especially for scientific writing purposes.
% To address this gap, our work implemented novel several interconnected systems that securely collect keystrokes in real-time over months from Overleaf, a widely-used online \LaTeX $\,$ editor, store them without privacy issues, and retrieve them for annotating writing intentions at span level. Our system workflow provides a more natural environment for observing cognitive writing activities by multiple researchers over long periods and across multiple sections without interrupting their writing. 




%-----------------------------------------------------

\subsection{Datasets for Scholarly Writing} 
% \minhwa{need papers about dataset for scientific writing - revision, feedback \& peer review, citation, etc. - that only focuses on one phase of writing - and \& comparison between paper versions - and \& small portion of entire writing (intro or RW)}

\begin{figure}[t!]
    \centering \hspace*{-0.3cm}
    \includegraphics[width=0.85\linewidth]{figures/fig2_dk.pdf}
    % \includegraphics[width=\linewidth]{latex/figures/fig2_a.pdf}
    % \includegraphics[width=\linewidth]{latex/figures/fig2_b.pdf}
    \caption{Previous studies primarily compared finalized edits between subsequent versions of revisions in open-source preprints (e.g., arXiv, OpenReview) \cite{du-etal-2022-understanding-iterative, jiang-etal-2022-arxivedits, kuznetsov2022revise, darcy-etal-2024-aries}. Our work, however, collects trajectories of keystrokes that comprise sentences to observe the cognitive process of end-to-end scholarly writing.\vspace{-4mm}}
    \label{fig:comparison-previous}
\end{figure}

Previous work primarily focused on constructing datasets to analyze scholarly writing processes, which vary in content and scale. Publicly available datasets tend to track linguistic style changes or grammatical edits during revision \cite{du-etal-2022-understanding-iterative, jiang-etal-2022-arxivedits, ito2019diamonds, mita2022towards}, while others capture edits based on feedback and peer review \cite{darcy-etal-2024-aries, jourdan2024casimir, kuznetsov2022revise} or citation generation \cite{kobayashi-etal-2022-dataset, narimatsu2021task}. Also, most focus on specific sections of papers (e.g., abstracts, introductions) \cite{du-etal-2022-understanding-iterative, mita2022towards}. In contrast, our dataset covers all cognitive phases of writing, from inception to final manuscript, over an extended period.
% Previous work primarily focused on constructing datasets for understanding the writing process of scholarly domains, which vary in their content and scale. Much of the publicly available datasets analyzed linguistic style changes or grammatical edits during the revision process \cite{du-etal-2022-understanding-iterative, jiang-etal-2022-arxivedits, ito2019diamonds, mita2022towards}. Also, the datasets from \citet{darcy-etal-2024-aries, jourdan2024casimir, kuznetsov2022revise} collected edits conditioned on feedback and peer review process, while \citet{kobayashi-etal-2022-dataset, narimatsu2021task} constructed datasets for citation generations. Existing work is also limited to only specific sections of scientific papers (e.g., abstract, introduction) \cite{du-etal-2022-understanding-iterative, mita2022towards}. In comparison, we construct a dataset that covers all cognitive phases of the writing process from its beginning to the final written product, through long periods of the data collection process. 

Furthermore, those existing corpora often compare multiple versions of final manuscripts from preprint databases \cite{jiang-etal-2022-arxivedits, du-etal-2022-understanding-iterative, darcy-etal-2024-aries, jourdan2024casimir}, missing how those manuscripts evolved \cite{jourdan2023text}. To address this, we build a keystroke-based corpus that captures the progression of publications in real-time (See Figure \ref{fig:comparison-previous}).
% Furthermore, those existing corpora for scientific writing research are created from the comparison between multiple versions of full-length, final products retrieved from preprint databases \cite{jiang-etal-2022-arxivedits, du-etal-2022-understanding-iterative, darcy-etal-2024-aries, jourdan2024casimir}. As pointed out by \citet{jourdan2023text}, the comparison of finalized outputs does not present how those manuscripts evolved. Our work addresses this problem by building a corpus of keystrokes each of which represents the progress of the publications (See Figure \ref{fig:comparison-previous}).




% Recent advancements in natural language generation (NLG) tasks have substantially benefited the development of intelligent writing assistants, capable of understanding and generating human-like texts for a wide range of writing tasks, including text revision \cite{kim-etal-2022-improving, du-etal-2022-understanding-iterative, raheja-etal-2023-coedit} to creative story generation \cite{chakrabarty-etal-2022-help, mirowski2023co, dang2023choice, ippolito2022creative, yuan2022wordcraft} to academic writing \cite{pinto2023large}. 

% Despite their emergent capabilities, AI-powered writing support for scholarly writing domains is still underexplored due to its rigorous standards of communication styles that prioritize information delivery over creative narratives. Also, current training approaches for those LLMs solely leveraged millions of text collections that are yet static and final byproducts of writing progress, not in a format of real-time trajectory. Our work builds the groundwork for developing AI writing assistants that provide suggestions based on cognitive writing intentions, by providing a large amount collection of keystrokes that present the reasoning trajectory of genuine human writing across all three stages of the cognitive writing process introduced in \citet{f508427a-e4c0-3d6a-8abf-03a5d21ec6c4}.