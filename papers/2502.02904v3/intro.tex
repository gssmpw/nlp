\section{Introduction}

Scholarly writing requires researchers to produce texts that concisely and precisely present novel findings and to follow a systematized structure and style by their targeted venue \cite{jourdan2023text, kallestinova2011write, bourekkache2022english}. To address this need, researchers have leveraged large language models (LLMs) to develop intelligent support systems in several writing tasks, such as revision process \cite{du-etal-2022-understanding-iterative, kim-etal-2022-improving} or feedback generation \cite{liang2024can, lu2024ai}. However, a critical question arises: \textit{Are LLMs able to truly understand and generate writing as human scientists do?}



% Recent advancements in large language models (LLMs) have benefited the development of intelligent support systems in several many writing tasks, including text revision \cite{du-etal-2022-understanding-iterative, kim-etal-2022-improving, raheja-etal-2023-coedit} or creative story generation \cite{yuan2022wordcraft, gomez2023confederacy, chakrabarty-etal-2022-help, mirowski2023co, ippolito2022creative}. However, a critical question arises: \textit{Are LLMs able to truly understand and generate writing as humans do?} 

\begin{figure}[t]
    \centering   
    \vspace{-4mm}
    \includegraphics[width=0.9\columnwidth]{figures/fig1_human_dk.pdf}
    \vspace{-6mm}
    \caption{An example of cognitive processes of human writing: it is iterative, non-linear, and switches frequently between multiple activities, tools, and writing intentions over a long range of time. \textsc{ScholaWrite} captures such complex characteristics of human scholarly writing.\vspace{-4mm}} 
    \label{fig:comparison-gpt4-human}
\end{figure}

LLMs are generally trained to progress autoregressively, generating text from left to right. In contrast, human writing typically involves multiple iterations of complex and non-linear cognitive actions to refine the main message \cite{f508427a-e4c0-3d6a-8abf-03a5d21ec6c4, koo2023decoding}, as illustrated in Figure \ref{fig:comparison-gpt4-human}. 
% For example, a researcher may introduce new findings as an idea (e.g., \texttt{\% put more findings}). However, they may revisit and revise previous sentences unlike LLMs do (e.g.\texttt{\textcolor{teal}{In}\textcolor{red}{\sout{According to}}}) and provide additional context to use more accurate terminology (e.g., \texttt{high \textcolor{teal}{Pearson's r} correlation}'') to earlier content. 
%Given the distinct patterns in the human writing process, it is imperative to understand the underlying cognitive processes of how people produce whole texts in a scientific setting.
% neccesiates the need of better writing assitant cognitively aligned with human writing process TODO
This fundamental cognitive gap between human writing and language models necessitates a deeper understanding of human writing processes and the development of cognitively-aligned writing assistance tools, moving beyond basic auto-completion.



% Human writing involves multiple iterations of complex and non-linear cognitive actions to clearly refine the main messages for its targeted audience \cite{f508427a-e4c0-3d6a-8abf-03a5d21ec6c4}. In particular, scholarly scientific writing is expected to meet high standards of scientific communication. These rigorous aspects of scholarly scientific writing have been less explored in the literature on intelligent writing assistants. Therefore, it is imperative to understand the underlying cognitive processes of how scientists produce texts that meet the expectations of academic communities.


% To aid this process, researchers have leveraged recent advancements in large language models (LLMs) for text generation tasks, developing intelligent writing assistants that provide editing suggestions.


Although much literature has examined the cognitive processes used in general writing tasks \cite{lindgren2003stimulated, hayes2014cognitive, zhu2023insights, Wengelin2023}, the specific cognitive patterns in scholarly writing remain understudied. Our work observes the end-to-end cognitive process of scholarly writing, inspired by the method of keystroke collections that have been a major methodology to observe individual writing processes \cite{doi:10.1177/0741088312451108, latif2008state, leijten2013keystroke} in cognitive science or a few recent works in NLP at a small scale \cite{koo2023decoding, velentzas2024logging}. 

% result overview 
We believe our work (\textsc{ScholaWrite}) is the first attempt to present a keystroke corpus of scholarly writing with annotations of \textit{cognitive writing intentions}, which were collected over \textit{multiple months} and produced by early-career researchers. Furthermore, we developed novel data collection and annotation systems, which improves data accessibility to the public and enables real-time collection of \LaTeX-based keystrokes from multiple researchers. We also present a comprehensive taxonomy of cognitive writing processes specific to the scholarly writing domain. Throughout several experiments with LLMs, \textsc{ScholaWrite} shows promising usability as a resource for enabling LLMs to understand the human cognitive process of scholarly writing and provide writing suggestions that are aware of writers' cognitive behaviors. 

\paragraph{Contributions} 
We present \textsc{ScholaWrite}, a curated dataset of nearly 62K \LaTeX-based keystrokes that were turned into publications in the computer science domain, annotated by experts in linguistics and computer science. We develop a taxonomy of cognitive process of scholarly writing, providing an overall understanding of how scholars tend to develop writing manuscripts. Experiment results showed that the Llama-8B model fine-tuned on the \textsc{ScholaWrite} dataset achieved the high linguistic quality of the final writing. 



%%%%%%%%%%%%%

% However, a critical question arises regarding these LLM-powered writing assistants: \textit{Are LLMs able to truly understand and generate writing as humans do?} As illustrated in Figure \ref{fig:comparison-gpt4-human}, LLMs generate the next sentence that logically follows from the previous one, progressing straightforwardly. In contrast, human writing is not strictly sequential; humans may revisit previous sentences, introduce new elements from memory (e.g., \textit{Starbucks}''), and provide additional context (e.g., \textit{one of his favorites}'') to earlier content. Additionally, humans can introduce new topics (e.g., ``\textit{a dog}'') in subsequent sentences that are not directly related to the preceding ones. Our study aims to explore these distinct patterns in the human writing process to enable LLMs to better mimic human reasoning during training.

% Moreover, current LLM-powered writing assistants often focus on specific phases of the writing process (e.g., revision) \cite{du-etal-2022-understanding-iterative, kim-etal-2022-improving, raheja-etal-2023-coedit} or on creative story writing tasks \cite{yuan2022wordcraft, gomez2023confederacy, chakrabarty-etal-2022-help, mirowski2023co, ippolito2022creative}. Also, these assistants do not adhere to practices expected in scholarly writing setups \cite{lu2024ai}.

% While much literature examined the patterns and complexities of cognitive processes in distinct writing actions \cite{lindgren2003stimulated, hayes2014cognitive, zhu2023insights, Wengelin2023}, less explored is the cognitive process for scholarly scientific writing. Our work observes the end-to-end writing trajectories of scholarly scientific writing, inspired by the method of keystroke collections that have been a major methodology to observe individual writing processes \cite{doi:10.1177/0741088312451108, latif2008state, leijten2013keystroke} in cognitive psychology fields. 

% % result overview 
% To the best of our knowledge, this is the first work to present a keystroke corpus of scholarly scientific writing with annotations regarding cognitive processes, which were collected over months and produced by early-career researchers. Furthermore, we developed novel systems that improve accessibility to the public and enable real-time collection of LaTex-based keystrokes from multiple researchers. We also present a comprehensive taxonomy of cognitive writing processes specific to the scientific writing domain, based on the annotated keystrokes. We found that .. \minhwa{findings - TBD after analysis}

% \paragraph{Contributions} 
% We present a curated dataset of XX LaTeX-based keystrokes that were turned into publications in the computer science domain, annotated by experts in linguistics and computer science. We develop a taxonomy of scholarly scientific writing intentions, providing an overall understanding of how scholars tend to produce their ideas.

% To examine various writing activities, previous studies analyzed the revision process by comparing multiple versions of final written outputs and extracting the spans of the edits \cite{jiang-etal-2022-arxivedits, kuznetsov2022revise, darcy-etal-2024-aries}. 
% (see Figure \ref{fig:comparison-previous}). 

% Previous works have examined the patterns and complexities of cognitive processes in distinct writing actions \cite{lindgren2003stimulated, hayes2014cognitive, zhu2023insights, Wengelin2023}, by leveraging keystroke data to observe individual writing processes logged in computer programs \cite{doi:10.1177/0741088312451108, van2012logging, latif2008state}. However, there are several limitations in previous literature: (1) a limited setup of short-timed essay writing by non-native English speakers; and (2) less accessibility for keystroke collection (e.g., only MS Word-specific programs). Most importantly, such keystroke logging studies have rarely examined scholarly communication settings, especially for scientific writing purposes.

% scholarly scientific writing is expected to meet rigorous standards of academic communication. These rigorous aspects of scholarly scientific writing have been less explored in the literature on intelligent writing assistants. Therefore, it is imperative to understand the underlying cognitive processes of how scientists produce texts that meet the expectations of academic communities.

% Previous works have examined the patterns and complexities of cognitive processes in distinct writing actions \cite{lindgren2003stimulated, hayes2014cognitive, zhu2023insights, Wengelin2023}, leveraging keystroke data to observe individual writing processes logged in computer programs \cite{doi:10.1177/0741088312451108, van2012logging, latif2008state}. However, these keystroke logs are often produced under limited conditions (e.g., short-timed essay writing) and are collected from specific word processing programs (e.g., Microsoft Word). Most importantly, such keystroke logging studies have rarely examined scholarly communication settings, especially for scientific writing purposes.

% \minhwa{our objectives}
% To the best of our knowledge, this is the first work to present a keystroke corpus of scholarly scientific writing with annotations regarding cognitive processes, which were collected over months and produced by early-career researchers. Furthermore, we developed novel systems that improve accessibility to the public and enable real-time collection of LaTex-based keystrokes from multiple researchers. We also present a comprehensive taxonomy of cognitive writing processes specific to the scientific writing domain, based on the annotated keystrokes. We found that .. \minhwa{findings - TBD after analysis}
% Our work is motivated by the idea that annotated keystrokes, along with a taxonomy of the scientific writing process, will help LLM-powered writing assistants understand the end-to-end scientific writing process in scholarly settings.

% This system can capture and record the dynamics of writing flow such as typing, revisions, and mouse movements, and it can reveal ``behind the scenes'' of writing during text production. However, those logs are limited to timed essay writing among L2 students (e.g., English as a second language). Also, the current programs can collect keystroke actions from a certain word processor program (e.g., Microsoft Word). Most importantly, those keystroke logging-based studies less examined scholarly communication settings, especially for scientific writing purposes. 
% \minhwa{what is missing in this previous study for keystroke logging - not many for scientific writing, focus on short, time-limited essay writing, L2 foreign people, and only MS-word based}. 


% % \minhwa{our contributions}
% \paragraph{Contributions} 
% We present a curated dataset of XX LaTeX-based keystrokes that were turned into publications in the computer science domain, annotated by experts in linguistics and computer science. We develop a taxonomy of scholarly scientific writing intentions, providing an overall understanding of how scholars tend to produce their ideas.
% % \minhwa{findings - TBD after analysis}





