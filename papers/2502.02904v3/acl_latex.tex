% This must be in the first 5 lines to tell arXiv to use pdfLaTeX, which is strongly recommended.
\pdfoutput=1

% In particular, the hyperref package requires pdfLaTeX in order to break URLs across lines.

\documentclass[11pt]{article}

% Change "review" to "final" to generate the final (sometimes called camera-ready) version.
% Change to "preprint" to generate a non-anonymous version with page numbers.
\usepackage[preprint]{acl}
% \usepackage[preprint]{acl}

% Standard package includes
\usepackage{times}
\usepackage{latexsym}


% \usepackage[dvipsnames]{xcolor}
% For proper rendering and hyphenation of words containing Latin characters (including in bib files)
\usepackage[T1]{fontenc}
% For Vietnamese characters
% \usepackage[T5]{fontenc}
% See https://www.latex-project.org/help/documentation/encguide.pdf for other character sets

% This assumes your files are encoded as UTF8
\usepackage[utf8]{inputenc}
% This is not strictly necessary, and may be commented out,
% but it will improve the layout of the manuscript,
% and will typically save some space.
\usepackage{microtype}

% This is also not strictly necessary, and may be commented out.
% However, it will improve the aesthetics of text in
% the typewriter font.
\usepackage{inconsolata}

%Including images in your LaTeX document requires adding
%additional package(s)
\usepackage{graphicx}

\usepackage{booktabs}
\usepackage{multirow}
\usepackage{multicol}

\usepackage{subcaption}

\usepackage[normalem]{ulem}

% \usepackage[table,xcdraw,dvipsnames]{xcolor} % Load xcolor package with table option

\definecolor{planningcolor}{HTML}{EF9D65}
\definecolor{implementationcolor}{HTML}{84BCD1}
\definecolor{revisioncolor}{HTML}{9584D1}

\usepackage{listings}
\usepackage{caption}
\usepackage{adjustbox}

\usepackage{hyperref}

\usepackage{array,multirow}
\usepackage{tikz}


% If the title and author information does not fit in the area allocated, uncomment the following
%
%\setlength\titlebox{<dim>}
%
% and set <dim> to something 5cm or larger.

\title{\textsc{ScholaWrite}: A Dataset of End-to-End Scholarly Writing Process
}

% Author information can be set in various styles:
% For several authors from the same institution:
% \author{Author 1 \and ... \and Author n \\
%         Address line \\ ... \\ Address line}
% if the names do not fit well on one line use
%         Author 1 \\ {\bf Author 2} \\ ... \\ {\bf Author n} \\
% For authors from different institutions:
% \author{Author 1 \\ Address line \\  ... \\ Address line
%         \And  ... \And
%         Author n \\ Address line \\ ... \\ Address line}
% To start a separate ``row'' of authors use \AND, as in
% \author{Author 1 \\ Address line \\  ... \\ Address line
%         \AND
%         Author 2 \\ Address line \\ ... \\ Address line \And
%         Author 3 \\ Address line \\ ... \\ Address line}

\author{Linghe Wang\thanks{Equal contribution.} \quad Minhwa Lee\textsuperscript{$\ast$} \quad \textbf{Ross Volkov} \quad \textbf{Luan Tuyen Chau} \quad \textbf{Dongyeop Kang}\\
University of Minnesota \\ 
\texttt{\{wang9257,lee03533,volko032,chau0139,dongyeop\}@umn.edu}
}

%\author{
%  \textbf{First Author\textsuperscript{1}},
%  \textbf{Second Author\textsuperscript{1,2}},
%  \textbf{Third T. Author\textsuperscript{1}},
%  \textbf{Fourth Author\textsuperscript{1}},
%\\
%  \textbf{Fifth Author\textsuperscript{1,2}},
%  \textbf{Sixth Author\textsuperscript{1}},
%  \textbf{Seventh Author\textsuperscript{1}},
%  \textbf{Eighth Author \textsuperscript{1,2,3,4}},
%\\
%  \textbf{Ninth Author\textsuperscript{1}},
%  \textbf{Tenth Author\textsuperscript{1}},
%  \textbf{Eleventh E. Author\textsuperscript{1,2,3,4,5}},
%  \textbf{Twelfth Author\textsuperscript{1}},
%\\
%  \textbf{Thirteenth Author\textsuperscript{3}},
%  \textbf{Fourteenth F. Author\textsuperscript{2,4}},
%  \textbf{Fifteenth Author\textsuperscript{1}},
%  \textbf{Sixteenth Author\textsuperscript{1}},
%\\
%  \textbf{Seventeenth S. Author\textsuperscript{4,5}},
%  \textbf{Eighteenth Author\textsuperscript{3,4}},
%  \textbf{Nineteenth N. Author\textsuperscript{2,5}},
%  \textbf{Twentieth Author\textsuperscript{1}}
%\\
%\\
%  \textsuperscript{1}Affiliation 1,
%  \textsuperscript{2}Affiliation 2,
%  \textsuperscript{3}Affiliation 3,
%  \textsuperscript{4}Affiliation 4,
%  \textsuperscript{5}Affiliation 5
%\\
%  \small{
%    \textbf{Correspondence:} \href{mailto:email@domain}{email@domain}
%  }
%}

\newcommand{\minhwa}[1]{\textcolor{magenta}{\bf\small [#1 --Minhwa]}}
\newcommand{\linghe}[1]{\textcolor{cyan}{\bf\small [#1 --Linghe]}}
\newcommand{\ross}[1]{\textcolor{brown}{\bf\small [#1 --Ross]}}
\newcommand{\luan}[1]{\textcolor{blue}{\bf\small [#1 --Luan]}}
\newcommand{\dk}[1]{\textcolor{teal}{\bf\small [#1 --DK]}}

\begin{document}
\maketitle
\begin{abstract}
% Scholarly writing involves complex, non-linear cognitive processes that require not only frequent transitions between various writing intentions but also high expectations of academic communication. While writing assistants powered by large language models (LLMs) have been applied to various writing tasks, their effectiveness in supporting end-to-end scholarly writing remains less explored. Previous work focuses on specific writing stages, overlooking the complexities of research writing, which necessitates the factual accuracy of scientific findings and persuasive narratives with rigorous logical reasoning. To address this gap, we introduce the first annotated dataset of keystroke trajectories from LaTeX-based scientific writing, collected over several months from early-career researchers. Our dataset, comprising over XX keystrokes collected through our thoroughly designed systems, is augmented by linguistics expert review and provides insights into the cognitive writing processes of scholarly scientific papers. We also propose a comprehensive taxonomy of scholarly writing processes in scientific domains, which can be useful resources for enhancing LLM-powered writing assistants for research writing purposes. This work provides a stepping stone for improved AI tools that can assist throughout the entire scholarly writing process, offering tailored suggestions for research writing.

% Recent advancements in large language models (LLMs) have facilitated the development of AI-powered intelligent writing assistants, including scientific manuscripts. 

Writing is a cognitively demanding task involving continuous decision-making, heavy use of working memory, and frequent switching between multiple activities.
Scholarly writing is particularly complex as it requires authors to coordinate many pieces of multiform knowledge.
To fully understand writers' cognitive thought process, one should fully decode the \textit{end-to-end writing data} (from individual ideas to final manuscript) and understand their complex cognitive mechanisms in scholarly writing.
We introduce \textsc{ScholaWrite} dataset, a first-of-its-kind keystroke corpus of an end-to-end scholarly writing process for complete manuscripts, with thorough annotations of cognitive writing intentions behind each keystroke. 
Our dataset includes \LaTeX-based keystroke data from five preprints with nearly 62K total text changes and annotations across 4 months of paper writing.
% Our novel system captures real-time LaTeX-based keystrokes over extended periods, enabling a deep analysis of the cognitive aspects of writing in scholarly communications. 
% Furthermore, by leveraging these data we propose a comprehensive taxonomy of cognitive writing processes specific to the scientific domain.
\textsc{ScholaWrite} shows promising usability and applications (e.g., iterative self-writing), demonstrating the importance of collection of end-to-end writing data, rather than the final manuscript, for the development of future writing assistants to support the cognitive thinking process of scientists.
Our de-identified data examples and code are available on our project page\footnote{\url{https://minnesotanlp.github.io/scholawrite/}}.
%\footnote{Our de-identified dataset and code repository will be released to the public upon acceptance.}
% https://anonymous.4open.science/w/scholawrite-anonymous/
% https://minnesotanlp.github.io/scholawrite/

% Writing is a cognitively demanding task involving continuous decision-making, heavy use of working memory, and frequent switching between multiple activities.
% Scholarly writing is particularly complex as it requires authors to coordinate many pieces of multiform knowledge.
% To fully understand writers' cognitive thought process, one should fully decode the \textit{end-to-end writing data} (from individual ideas to final manuscript) and understand their complex cognitive mechanisms in scholarly writing.
% We introduce \textsc{ScholaWrite}, a first-of-its-kind dataset of keystroke-intention pairs of end-to-end multi-author scholarly writing processes for 5 complete manuscripts. Each keystroke is annotated with cognitive writing intentions behind each keystroke. 
% Our dataset includes \LaTeX-based keystroke data from five preprints with nearly 62K total text changes and annotations across 4 months of paper writing.
% % Our novel system captures real-time LaTeX-based keystrokes over extended periods, enabling a deep analysis of the cognitive aspects of writing in scholarly communications. 
% % Furthermore, by leveraging these data we propose a comprehensive taxonomy of cognitive writing processes specific to the scientific domain.
% \textsc{ScholaWrite} shows promising usability and applications (e.g., iterative self-writing) for the future development of AI writing assistants for academic research, which necessitate complex methods beyond LLM prompting.
% Our experiments clearly demonstrated the importance of collection of end-to-end writing data, rather than just the final manuscript, for the development of future writing assistants to support the cognitive thinking process of scientists.
% Our de-identified dataset, demo, and code repository are available on our project page\footnote{\url{https://minnesotanlp.github.io/scholawrite/}}.
% %\footnote{Our de-identified dataset and code repository will be released to the public upon acceptance.}
\end{abstract}




\section{Introduction}


\begin{figure}[t]
\centering
\includegraphics[width=0.6\columnwidth]{figures/evaluation_desiderata_V5.pdf}
\vspace{-0.5cm}
\caption{\systemName is a platform for conducting realistic evaluations of code LLMs, collecting human preferences of coding models with real users, real tasks, and in realistic environments, aimed at addressing the limitations of existing evaluations.
}
\label{fig:motivation}
\end{figure}

\begin{figure*}[t]
\centering
\includegraphics[width=\textwidth]{figures/system_design_v2.png}
\caption{We introduce \systemName, a VSCode extension to collect human preferences of code directly in a developer's IDE. \systemName enables developers to use code completions from various models. The system comprises a) the interface in the user's IDE which presents paired completions to users (left), b) a sampling strategy that picks model pairs to reduce latency (right, top), and c) a prompting scheme that allows diverse LLMs to perform code completions with high fidelity.
Users can select between the top completion (green box) using \texttt{tab} or the bottom completion (blue box) using \texttt{shift+tab}.}
\label{fig:overview}
\end{figure*}

As model capabilities improve, large language models (LLMs) are increasingly integrated into user environments and workflows.
For example, software developers code with AI in integrated developer environments (IDEs)~\citep{peng2023impact}, doctors rely on notes generated through ambient listening~\citep{oberst2024science}, and lawyers consider case evidence identified by electronic discovery systems~\citep{yang2024beyond}.
Increasing deployment of models in productivity tools demands evaluation that more closely reflects real-world circumstances~\citep{hutchinson2022evaluation, saxon2024benchmarks, kapoor2024ai}.
While newer benchmarks and live platforms incorporate human feedback to capture real-world usage, they almost exclusively focus on evaluating LLMs in chat conversations~\citep{zheng2023judging,dubois2023alpacafarm,chiang2024chatbot, kirk2024the}.
Model evaluation must move beyond chat-based interactions and into specialized user environments.



 

In this work, we focus on evaluating LLM-based coding assistants. 
Despite the popularity of these tools---millions of developers use Github Copilot~\citep{Copilot}---existing
evaluations of the coding capabilities of new models exhibit multiple limitations (Figure~\ref{fig:motivation}, bottom).
Traditional ML benchmarks evaluate LLM capabilities by measuring how well a model can complete static, interview-style coding tasks~\citep{chen2021evaluating,austin2021program,jain2024livecodebench, white2024livebench} and lack \emph{real users}. 
User studies recruit real users to evaluate the effectiveness of LLMs as coding assistants, but are often limited to simple programming tasks as opposed to \emph{real tasks}~\citep{vaithilingam2022expectation,ross2023programmer, mozannar2024realhumaneval}.
Recent efforts to collect human feedback such as Chatbot Arena~\citep{chiang2024chatbot} are still removed from a \emph{realistic environment}, resulting in users and data that deviate from typical software development processes.
We introduce \systemName to address these limitations (Figure~\ref{fig:motivation}, top), and we describe our three main contributions below.


\textbf{We deploy \systemName in-the-wild to collect human preferences on code.} 
\systemName is a Visual Studio Code extension, collecting preferences directly in a developer's IDE within their actual workflow (Figure~\ref{fig:overview}).
\systemName provides developers with code completions, akin to the type of support provided by Github Copilot~\citep{Copilot}. 
Over the past 3 months, \systemName has served over~\completions suggestions from 10 state-of-the-art LLMs, 
gathering \sampleCount~votes from \userCount~users.
To collect user preferences,
\systemName presents a novel interface that shows users paired code completions from two different LLMs, which are determined based on a sampling strategy that aims to 
mitigate latency while preserving coverage across model comparisons.
Additionally, we devise a prompting scheme that allows a diverse set of models to perform code completions with high fidelity.
See Section~\ref{sec:system} and Section~\ref{sec:deployment} for details about system design and deployment respectively.



\textbf{We construct a leaderboard of user preferences and find notable differences from existing static benchmarks and human preference leaderboards.}
In general, we observe that smaller models seem to overperform in static benchmarks compared to our leaderboard, while performance among larger models is mixed (Section~\ref{sec:leaderboard_calculation}).
We attribute these differences to the fact that \systemName is exposed to users and tasks that differ drastically from code evaluations in the past. 
Our data spans 103 programming languages and 24 natural languages as well as a variety of real-world applications and code structures, while static benchmarks tend to focus on a specific programming and natural language and task (e.g. coding competition problems).
Additionally, while all of \systemName interactions contain code contexts and the majority involve infilling tasks, a much smaller fraction of Chatbot Arena's coding tasks contain code context, with infilling tasks appearing even more rarely. 
We analyze our data in depth in Section~\ref{subsec:comparison}.



\textbf{We derive new insights into user preferences of code by analyzing \systemName's diverse and distinct data distribution.}
We compare user preferences across different stratifications of input data (e.g., common versus rare languages) and observe which affect observed preferences most (Section~\ref{sec:analysis}).
For example, while user preferences stay relatively consistent across various programming languages, they differ drastically between different task categories (e.g. frontend/backend versus algorithm design).
We also observe variations in user preference due to different features related to code structure 
(e.g., context length and completion patterns).
We open-source \systemName and release a curated subset of code contexts.
Altogether, our results highlight the necessity of model evaluation in realistic and domain-specific settings.





\section{RELATED WORK}
\label{sec:relatedwork}
In this section, we describe the previous works related to our proposal, which are divided into two parts. In Section~\ref{sec:relatedwork_exoplanet}, we present a review of approaches based on machine learning techniques for the detection of planetary transit signals. Section~\ref{sec:relatedwork_attention} provides an account of the approaches based on attention mechanisms applied in Astronomy.\par

\subsection{Exoplanet detection}
\label{sec:relatedwork_exoplanet}
Machine learning methods have achieved great performance for the automatic selection of exoplanet transit signals. One of the earliest applications of machine learning is a model named Autovetter \citep{MCcauliff}, which is a random forest (RF) model based on characteristics derived from Kepler pipeline statistics to classify exoplanet and false positive signals. Then, other studies emerged that also used supervised learning. \cite{mislis2016sidra} also used a RF, but unlike the work by \citet{MCcauliff}, they used simulated light curves and a box least square \citep[BLS;][]{kovacs2002box}-based periodogram to search for transiting exoplanets. \citet{thompson2015machine} proposed a k-nearest neighbors model for Kepler data to determine if a given signal has similarity to known transits. Unsupervised learning techniques were also applied, such as self-organizing maps (SOM), proposed \citet{armstrong2016transit}; which implements an architecture to segment similar light curves. In the same way, \citet{armstrong2018automatic} developed a combination of supervised and unsupervised learning, including RF and SOM models. In general, these approaches require a previous phase of feature engineering for each light curve. \par

%DL is a modern data-driven technology that automatically extracts characteristics, and that has been successful in classification problems from a variety of application domains. The architecture relies on several layers of NNs of simple interconnected units and uses layers to build increasingly complex and useful features by means of linear and non-linear transformation. This family of models is capable of generating increasingly high-level representations \citep{lecun2015deep}.

The application of DL for exoplanetary signal detection has evolved rapidly in recent years and has become very popular in planetary science.  \citet{pearson2018} and \citet{zucker2018shallow} developed CNN-based algorithms that learn from synthetic data to search for exoplanets. Perhaps one of the most successful applications of the DL models in transit detection was that of \citet{Shallue_2018}; who, in collaboration with Google, proposed a CNN named AstroNet that recognizes exoplanet signals in real data from Kepler. AstroNet uses the training set of labelled TCEs from the Autovetter planet candidate catalog of Q1–Q17 data release 24 (DR24) of the Kepler mission \citep{catanzarite2015autovetter}. AstroNet analyses the data in two views: a ``global view'', and ``local view'' \citep{Shallue_2018}. \par


% The global view shows the characteristics of the light curve over an orbital period, and a local view shows the moment at occurring the transit in detail

%different = space-based

Based on AstroNet, researchers have modified the original AstroNet model to rank candidates from different surveys, specifically for Kepler and TESS missions. \citet{ansdell2018scientific} developed a CNN trained on Kepler data, and included for the first time the information on the centroids, showing that the model improves performance considerably. Then, \citet{osborn2020rapid} and \citet{yu2019identifying} also included the centroids information, but in addition, \citet{osborn2020rapid} included information of the stellar and transit parameters. Finally, \citet{rao2021nigraha} proposed a pipeline that includes a new ``half-phase'' view of the transit signal. This half-phase view represents a transit view with a different time and phase. The purpose of this view is to recover any possible secondary eclipse (the object hiding behind the disk of the primary star).


%last pipeline applies a procedure after the prediction of the model to obtain new candidates, this process is carried out through a series of steps that include the evaluation with Discovery and Validation of Exoplanets (DAVE) \citet{kostov2019discovery} that was adapted for the TESS telescope.\par
%



\subsection{Attention mechanisms in astronomy}
\label{sec:relatedwork_attention}
Despite the remarkable success of attention mechanisms in sequential data, few papers have exploited their advantages in astronomy. In particular, there are no models based on attention mechanisms for detecting planets. Below we present a summary of the main applications of this modeling approach to astronomy, based on two points of view; performance and interpretability of the model.\par
%Attention mechanisms have not yet been explored in all sub-areas of astronomy. However, recent works show a successful application of the mechanism.
%performance

The application of attention mechanisms has shown improvements in the performance of some regression and classification tasks compared to previous approaches. One of the first implementations of the attention mechanism was to find gravitational lenses proposed by \citet{thuruthipilly2021finding}. They designed 21 self-attention-based encoder models, where each model was trained separately with 18,000 simulated images, demonstrating that the model based on the Transformer has a better performance and uses fewer trainable parameters compared to CNN. A novel application was proposed by \citet{lin2021galaxy} for the morphological classification of galaxies, who used an architecture derived from the Transformer, named Vision Transformer (VIT) \citep{dosovitskiy2020image}. \citet{lin2021galaxy} demonstrated competitive results compared to CNNs. Another application with successful results was proposed by \citet{zerveas2021transformer}; which first proposed a transformer-based framework for learning unsupervised representations of multivariate time series. Their methodology takes advantage of unlabeled data to train an encoder and extract dense vector representations of time series. Subsequently, they evaluate the model for regression and classification tasks, demonstrating better performance than other state-of-the-art supervised methods, even with data sets with limited samples.

%interpretation
Regarding the interpretability of the model, a recent contribution that analyses the attention maps was presented by \citet{bowles20212}, which explored the use of group-equivariant self-attention for radio astronomy classification. Compared to other approaches, this model analysed the attention maps of the predictions and showed that the mechanism extracts the brightest spots and jets of the radio source more clearly. This indicates that attention maps for prediction interpretation could help experts see patterns that the human eye often misses. \par

In the field of variable stars, \citet{allam2021paying} employed the mechanism for classifying multivariate time series in variable stars. And additionally, \citet{allam2021paying} showed that the activation weights are accommodated according to the variation in brightness of the star, achieving a more interpretable model. And finally, related to the TESS telescope, \citet{morvan2022don} proposed a model that removes the noise from the light curves through the distribution of attention weights. \citet{morvan2022don} showed that the use of the attention mechanism is excellent for removing noise and outliers in time series datasets compared with other approaches. In addition, the use of attention maps allowed them to show the representations learned from the model. \par

Recent attention mechanism approaches in astronomy demonstrate comparable results with earlier approaches, such as CNNs. At the same time, they offer interpretability of their results, which allows a post-prediction analysis. \par


\section{OUR APPROACH: DQuaG}

\begin{figure*}[tb]
\centering
\includegraphics[width=0.7\textwidth]{Figures/structure.pdf}
%\vspace{-0.6\baselineskip}
\caption{Data Quality Validation Framework Using GNN. Top: Training on clean data. Bottom: Validating unseen data by reconstruction error comparison.}
%\vspace*{-0.3cm}
%\soror{highlight in the caption the process at the top and the one at the bottom of teh figure }
\label{fig:framework}
\end{figure*}

%In this section, we detail our novel approach to automated data quality validation using graph representation learning and a Variational Autoencoder (VAE) framework. Assuming we start with a clean dataset, our method addresses the limitations of traditional data quality verification techniques through a series of steps designed to capture intrinsic relationships within tabular data and assess data quality with minimal expert intervention.


In this section, we present DQuaG (Data Quality Graph), a novel approach for data quality validation. 
Figure~\ref{fig:framework} illustrates the framework of our approach, which includes two main phases: model training on a clean dataset and data quality validation and repair for new data. 
In Phase 1, we train a model using a clean dataset to learn the normal patterns and relationships between features. 
In Phase 2, we use the trained model to assess the quality of new data and provide repair suggestions for any detected errors. 
%Our method incorporates several key innovations, including the use of an improved Graph Neural Network (GNN) encoder combining Graph Attention Network (GAT) and Graph Isomorphism Network (GIN), and a multi-task learning framework with dual decoders for data quality validation and repair. 


\subsection{Phase 1: Training GNN on Clean Data}
%\soror{it's betetr to remove data preprocessing as a step as we don't see it in the figure! we can only keep its text}
%\subsubsection{\textbf{Data Preprocessing}}
We assume the availability of a high-quality, clean dataset, $\mathcal{D}_{\text{clean}}$, that has undergone rigorous quality control and is free from errors. This dataset serves as the foundation for training our model. 
For feature encoding and normalization, categorical features are converted to numerical form using label encoding, where the encoder is fitted on both clean data and any possible future data to ensure consistency. 
For numerical features, we apply min-max normalization to scale values to the range [0, 1], which helps improve training stability and ensures that all features are on a comparable scale.


\subsubsection{\textbf{Feature Graph Construction}}

We use ChatGPT-4~\cite{openai2024gpt4} to automate the feature graph construction. 
Given a clean dataset, we extract the feature names ($F$) and their descriptions ($D$) from the data source. We then randomly sample 100 data points from the dataset, denoted as ($S$). 
These feature names, descriptions, and sample points are provided to ChatGPT-4 in a structured format to infer potential relationships between features. 
The output from ChatGPT-4 is a JSON file capturing feature relationships, which we denote as  \(\text{Feature\_Relationships} = \{ (f_i, f_j) \mid f_i, f_j \in F \}\), indicating that there is a relationship between features \( f_i \) and \( f_j \).

%The prompt used for ChatGPT-4 is as follows:
\begin{tcolorbox}[
    sharp corners=south,
    colback=white!98!black,
    colframe=white!45!black,
    boxrule=0.5mm,
    width=0.48\textwidth,
    enlarge left by=0mm,
    enlarge right by=0mm,
    arc=5mm,
    outer arc=3mm,
    %drop shadow south east={shadow xshift=0.5ex, shadow yshift=-0.5ex, fill=black!20},
    fonttitle=\bfseries,
    title=Prompt for Feature Relationship Inference,
    before upper=\par\small,
    after upper=\par\small
]
\vspace*{-0.2cm}
\small
Given the following information, please infer the relationships between features. Provide your output in JSON format, capturing the type of relationships.\newline
\textbf{Feature Names:} {List of feature names ($F$)}\newline
\textbf{Feature Descriptions:} {List of descriptions ($D$) for each feature}\newline
\textbf{Sample Data Points:} {100 data samples ($S$) from the dataset}\newline
\textbf{Output:} Please return a JSON object in the format:
\begin{verbatim}
{"relationships": [{"feature1", "feature2"}, 
                   {"feature3", "feature4"}, ...]}
\end{verbatim}
\vspace*{-0.3cm}
\end{tcolorbox}




Using these relationships, we construct the knowledge-based feature graph \( G = (V, E) \), where \( V \) represents features and \( E \) represents edges indicating relationships between features.

% \subsubsection{\textbf{Feature Graph Construction}}

% The initial step in our approach involves constructing a feature graph from clean tabular data to capture intrinsic relationships and dependencies between data features.
% First, we address the challenge of diverse data types: categorical variables are transformed using label encoding, and timestamp data is broken into components (i.e., day, month, year). 
% This uniform input format is critical for graph-based processing.

% We use ChatGPT-4~\cite{openai2024gpt4} to automate the feature graph construction. 
% Given a clean dataset, we extract the feature names \( F \) and their descriptions \( D \) from the data source. We then randomly sample 100 data points from the dataset, denoted as \( S \). These feature names, descriptions, and sample data points are provided to the ChatGPT-4, structured as follows: \(\text{Input} = \{ F, D, S \}\), then ChatGPT-4 generates a JSON file capturing feature relationships.
% The output format is \(\text{Feature\_Relationships} = \{ (f_i, f_j) \mid f_i, f_j \in F \}\), indicating that there is a relationship between features \( f_i \) and \( f_j \).

% Using these relationships, we construct the knowledge-based feature graph \( G = (V, E) \), where \( V \) represents features and \( E \) represents edges indicating relationships between features.

% \subsubsection{\textbf{Feature Graph Construction}}

% The initial step in our methodology involves constructing a feature graph from clean tabular data, which is essential for capturing the intrinsic relationships and dependencies between different data features.

% To facilitate this process, our approach first addresses the challenge of handling diverse data types. 
% In the preprocessing stage, categorical variables are transformed using label encoding, which assigns each unique category a unique integer based on alphabetical ordering. 
% For timestamp data, we extract significant components such as day, month, and year. 
% This uniform input format is critical for the subsequent graph-based processing. 

% Following the preprocessing, we utilize a large language model, ChatGPT-4 \cite{openai2024gpt4}, to automate the construction of the feature graph. This integration allows for a more nuanced capture of feature relationships and dependencies, reducing reliance on expert knowledge and manual effort.

% Given a clean dataset, we extract the feature names \( F = \{f_1, f_2, \ldots, \\f_n\} \) and their descriptions \( D = \{d_1, d_2, \ldots, d_n\} \) from the data source. We then randomly sample 100 data points from the dataset, denoted as \( S = \{s_1, s_2, \ldots, s_{100}\} \). These feature names, descriptions, and sample data points are provided to the LLM, structured as follows: \(\text{Input} = \{ F, D, S \}\).

% The LLM analyzes the provided input and generates a structured JSON file capturing the relationships between features. The output format is \(\text{Feature\_Relationships} = \{ (f_i, f_j) \mid f_i, f_j \in F \}\), indicating that there is a relationship between features \( f_i \) and \( f_j \).

% Using the relationships provided by the LLM, we construct the feature graph \( G = (V, E) \) where \( V = F \) (nodes representing features) and \( E = \{(f_i, f_j) \mid (f_i, f_j) \in \text{Feature\_Relationships} \} \) (edges representing relationships). 

%----------------------------
%This graph-based representation allows us to model complex interdependencies within the data that are often overlooked by traditional methods, enhancing our ability to perform thorough data quality assessments.



% \subsubsection{\textbf{Training the Graph Neural Network (GNN) and Representing the Clean Dataset}}
\subsubsection{\textbf{GNN Model Architecture}}
Our model architecture combines the strengths of different graph neural network variants to effectively capture complex feature relationships. 
The architecture consists of three main components: an improved GNN encoder that fuses Graph Attention Network (GAT)~\cite{velivckovic2017graph} layers and Graph Isomorphism Network (GIN)~\cite{xu2018powerful} layers, and two specialized decoders for quality validation and repair suggestion generation.

\noindent{\textbf{GNN Encoder (GAT + GIN)}.
Our encoder consists of four layers: alternating \textit{Graph Attention Network (GAT)} and \textit{Graph Isomorphism Network (GIN)} layers, in the order of GAT-GIN-GAT-GIN. 
%We employ this combination to leverage the complementary strengths of both GAT and GIN. 
This design is inspired by recent findings in the field of graph representation learning that demonstrate how combining different types of graph layers can yield improved performance in feature extraction and relational representation tasks~\cite{zhang2019heterogeneous}. 
Our experimental results demonstrate the advantages of this structure.

The GAT layers compute attention weights between connected features, enabling the model to adaptively assign importance to significant relationships in the data. This allows the model to focus on critical connections and ignore irrelevant information, which enhances its ability to learn meaningful feature representations. Our approach uses GAT layers, which automatically learn edge weights through attention mechanisms during training. This eliminates the need to manually assign weights in the initial feature graph. 
%\soror{I think here we should well emphasize the concerns of R1W2: we are not using statistical correlations as suggested by the reviewer? how to prove that the rleationships generated with our approach improve the model’s representation of real-world datadependencies }

The GIN layers aggregate feature information from neighboring nodes to capture structural information more effectively. By using GIN, the encoder gains a strong ability to represent the underlying structure of the data, preserving key relationships crucial for data quality validation and repair tasks.

This alternating GAT and GIN structure enhances the model's ability to both prioritize important features and learn intricate structural relationships, thereby making it more effective at representing complex feature dependencies in the data.
Specifically, the GNN encoder processes the feature graph $G = (V, E)$ along with the input data matrix $\mathbf{X} \in \mathbb{R}^{n \times d}$, where $n$ is the number of nodes (features) and $d$ is the dimensionality of each feature vector. The output from the GNN encoder is a feature embedding matrix $\mathbf{Z} \in \mathbb{R}^{n \times h}$, where $h$ represents the size of the learned feature embeddings.


% \noindent{\textbf{Encoder Structure}.}
% Our encoder consists of two layers: a GAT layer followed by a GIN layer. We combine GAT and GIN because of their complementary strengths. 

% The GAT layer computes attention weights between connected features, allowing the model to prioritize significant relationships. 
% For each feature node $i$ and its neighbor $j$, the attention coefficient $alpha_{ij}$ is computed as:
% \begin{equation}
% \alpha_{ij} = \frac{\exp(\text{LeakyReLU}(\mathbf{a}^T[\mathbf{W}\mathbf{h}_i \Vert \mathbf{W}\mathbf{h}j]))}{\sum{k \in \mathcal{N}(i)} \exp(\text{LeakyReLU}(\mathbf{a}^T[\mathbf{W}\mathbf{h}_i \Vert \mathbf{W}\mathbf{h}_k]))}
% \end{equation}

% In this equation, $\mathbf{h}_i$ is the feature representation of node $i$, $\mathbf{W}$ is a learned weight matrix, $\mathbf{a}$ is a learned attention vector, and the symbol $\Vert$ denotes concatenation. The GAT layer enables the model to adaptively assign importance to different features based on the data context. 

% Following the GAT layer, the GIN layer enhances the model's ability to capture structural information by aggregating feature information from neighboring nodes. The GIN layer updates the representation of each node $i$ as follows:
% \begin{equation}
% \mathbf{h}_i^{(l+1)} = \text{MLP}^{(l)}\left((1 + \epsilon^{(l)})\mathbf{h}i^{(l)} + \sum{j \in \mathcal{N}(i)} \mathbf{h}_j^{(l)}\right)
% \end{equation}

% Here, $\epsilon^{(l)}$ is a learnable parameter at layer $l$ that controls the importance of the original node representation, and $\text{MLP}^{(l)}$ represents a multi-layer perceptron, which adds non-linearity to enhance the model's expressive power. 

% The combination of GAT and GIN layers ensures that our model learns both the local importance of features and their broader, structural context, leading to richer and more accurate feature representations.
% Specifically, the GNN encoder processes the feature graph $G = (V, E)$ along with the input data matrix $\mathbf{X} \in \mathbb{R}^{n \times d}$, where $n$ is the number of nodes (features) and $d$ is the dimensionality of each feature vector. The output from the GNN encoder is a feature embedding matrix $\mathbf{Z} \in \mathbb{R}^{n \times h}$, where $h$ represents the size of the learned feature embeddings.

\noindent{\textbf{Dual Decoder Structure}.}  
Our model employs two separate decoders to address the tasks of Data Quality Validation and Repair Suggestion, enabling focused optimization for each objective.

\textit{Data Quality Validation Decoder} is responsible for reconstructing the original feature space from the learned embeddings, denoted as \(\mathbf{Z}\). 
The primary objective of this decoder is to learn the correct patterns from clean data and reconstruct the features in a way that captures the underlying structure of the dataset. This allows us to identify abnormalities by measuring reconstruction errors. We have designed a unique loss function that ensures the model focuses on learning accurate representations of clean data while effectively distinguishing abnormal samples.

For normal data samples, the decoder should ideally have a low reconstruction error, as the learned embeddings should effectively capture the true relationships between the features, resulting in an accurate reconstruction. For abnormal samples, the reconstruction error will be higher, indicating that these samples do not conform to the learned patterns from the clean data.
The reconstruction loss is defined as 
\( L_{\text{validation}} = \frac{1}{N} \sum_{i=1}^{N} w_i \left\| \mathbf{X}_i - \hat{\mathbf{X}}_i \right\|_2^2 \),
% The reconstruction loss is defined as follows:
% \begin{equation}
% L_{\text{validation}} = \frac{1}{N} \sum_{i=1}^{N} w_i \left\| \mathbf{X}_i - \hat{\mathbf{X}}_i \right\|_2^2
% \end{equation}
where \(\mathbf{X}_i\) represents the original input features, and \(\hat{\mathbf{X}}_i\) is the reconstructed feature vector for the \(i\)-th sample. The weights \(w_i\) are assigned to each sample based on its reconstruction error.

We assign larger weights to normal data samples (\(w_i\) s higher for samples with smaller reconstruction errors), giving them a greater influence in minimizing their reconstruction loss. This encourages the model to accurately reconstruct the normal data and effectively learn the correct data distribution. 
For samples with potential quality issues, the weights \(w_i\) are reduced, meaning that their influence on the overall loss is diminished. 
This allows the model to focus on minimizing the reconstruction loss for normal data while maintaining high reconstruction errors for problematic samples during the backpropagation. 
By using this weighting mechanism, we ensure that the validation decoder can distinguish between normal data and data with potential issues based on reconstruction errors.


\textit{Data Repair Decoder}, on the other hand, takes the same learned embeddings \(\mathbf{Z}\) as input, but its goal is different: it aims to suggest repaired values for features identified as erroneous. 
Unlike the Data Quality Validation Decoder, which reconstructs data to highlight discrepancies, the Data Repair Decoder attempts to produce an output that aligns with the clean, underlying data distribution, effectively suggesting corrections for the detected errors. 
The objective of this decoder is defined through the following loss function:
\(
L_{\text{repair}} = \frac{1}{N} \sum_{i=1}^{N} \left\| {\mathbf{X}}_i - \tilde{\mathbf{X}}_i \right\|_2^2
\).
Here, \(\tilde{\mathbf{X}}_i\) represents the feature values repaired by the decoder, while \(\mathbf{X}_i\) stands for the corresponding clean feature values from the input dataset. Since the input is already clean, \(\mathbf{X}_i\) can directly serve as the target for the repair task.

The combination of these two decoders is essential for effectively handling data quality issues. 
The overall loss function is a weighted sum of the validation and repair losses:
\(
L_{\text{total}} = \alpha L_{\text{validation}} + \beta L_{\text{repair}}
\),
where \(\alpha\) and \(\beta\) are hyperparameters used to balance the contributions of reconstruction and repair, both of which are set to 1 in our experiments.

The two decoders serve different purposes: the \textit{Data Quality Validation Decoder} is optimized to detect data issues by maximizing reconstruction errors for problematic instances, while the \textit{Data Repair Decoder} aims to provide realistic corrections for identified issues. By separating these tasks, the model avoids conflicting optimization goals, ensuring it is both effective at identifying problems and providing reliable repairs.


\noindent{\textbf{Multi-Task Learning Framework}.}
%\paragraph{\textbf{Multi-Task Learning Framework}} 
% The encoder is shared between the quality validation and data repair tasks, while each task-specific decoder learns independently. 
% This multi-task framework enables the model to exploit shared information between these two tasks, allowing the model to learn a unified representation that is beneficial for both.
The encoder is shared between the quality validation and repair tasks, while each task-specific decoder learns independently. This multi-task framework enables the model to exploit shared information between these tasks, allowing the model to learn a unified representation beneficial for both.

\subsubsection{\textbf{Training Process.}}
We train the model on the clean dataset using an optimizer Adam to minimize $L_{\text{total}}$. %During training, the GNN encoder alternates between GAT and GIN layers to improve its representation capabilities, enabling it to better capture complex feature relationships. 
%Training is performed iteratively, with each iteration including forward propagation through both the encoder and decoders, loss calculation, and parameter updates.

\subsubsection{\textbf{Collecting the statistics of reconstruction errors.}}
During training, we record the reconstruction error for each instance.
The reconstruction error is essentially the loss for each instance.
Let $e_i$ denote the reconstruction error for instance $i$, and let $\mathcal{E}$ be the set of all reconstruction errors from the clean dataset. 
Given that even cleaned datasets may contain undetected errors, we do not set the maximum reconstruction error as the threshold for identifying problematic instances. 
Instead, we set the threshold at the 95th percentile of \(\mathcal{E}\), denoted as \( e_{threshold} \).
% :
% \begin{equation}
% e_{\text{threshold}} = \text{Quantile}(\mathcal{E}, 0.95)
% \end{equation}

Instances in the next phase with reconstruction errors above \( e_{threshold} \) are flagged as potentially problematic.


\subsection{Phase 2: Data Quality Validate and Repair}

\subsubsection{\textbf{Data Quality Validation Process}}
In this phase, we validate the quality of incoming data by comparing it to the patterns learned during model training.

%\noindent{\textbf{New Data Preprocessing}.}
The new unseen data is preprocessed in the same manner as the clean dataset to ensure consistency in feature encoding, normalization, and feature graph construction. These new unseen datasets must keep the same schema as the original clean dataset.

\noindent{\textbf{Detecting Data Quality Issues by Reconstruction Errors}.}
After preprocessing, the model uses the validation decoder to reconstruct the features of the new data. 
For each data instance, the reconstruction error $e_i$ is calculated. We then obtain a list of reconstruction errors, denoted as \(\mathcal{E}_{\text{new}}\).
Next, we compare each reconstruction error in \(\mathcal{E}_{\text{new}}\) with the threshold \( e_{threshold} \) from the clean dataset. 
We calculate the proportion of instances in the new dataset with reconstruction errors exceeding \( e_{threshold} \), denoted as \( R_{error} \). 
Since the threshold was set at the 95th percentile for the clean dataset, we expect around 5\% of clean data instances to exceed this value. 

To account for data variability, if \( R_{error} \) exceeds \( 5\% \times n \), we classify the new dataset as problematic. This means if more than \( 5n\% \) of instances in the new dataset have errors greater than \( e_{threshold} \), we will report the dataset has data quality issues. The parameter \( n \) can be adjusted based on observed reconstruction errors after deployment.
In our experiments, we set \( n = 1.2 \), which exhibited good performance.
Finally, we report the indices of all instances in the new dataset with reconstruction errors above \( e_{threshold} \), clearly identifying problematic samples.


\noindent{\textbf{Detecting Feature Errors}.}
Each instance's reconstruction error \( e \) is a list corresponding to each feature's loss. To identify specific problematic features, we detect outliers with significantly higher reconstruction errors.
For an instance \( \mathbf{x}_i \), let \( \mathbf{e}_i = [e_{i1}, e_{i2}, \ldots, e_{in}] \) be the reconstruction errors for the \( n \) features. We calculate the mean \( \mu_i \) and standard deviation \( \sigma_i \) of the errors. Features with errors greater than \( \mu_i + 5\sigma_i \) are flagged as problematic.

% By reporting these outlier features, we can pinpoint which specific parts of an instance contribute most to data quality issues. 
% This drill-down process helps identify exact feature-level problems within instances, facilitating targeted data cleaning.

\subsubsection{\textbf{Repair Suggestion Generation}}
In this phase, we provide repair suggestions for detected errors to improve the quality of the data for downstream use.
The repair decoder is used to generate a repaired feature vector, which includes suggested repaired values for all features. 
In the previous step, we flagged which specific instances and features were problematic. 
Then we selectively apply modifications only to the flagged problematic features. For categorical features, the repair decoder predicts the most likely corrected category, while for numerical features, it predicts a value that aligns with the learned data distribution. 
%This approach ensures that the repaired values are both accurate and contextually coherent, reducing the risk of introducing new inconsistencies.


% \noindent{\textbf{Confidence Scoring}.}
% A confidence score $c_{ij} \in [0, 1]$ is assigned to each repaired feature $\tilde{x}{ij}$ to quantify the reliability of the suggested repair. The confidence score is defined based on the distance between $\tilde{x}{ij}$ and the expected distribution of clean data, using the standard deviation $\sigma_j$ of feature $j$ in the clean dataset:

% \begin{equation} c_{ij} = \exp\left(-\frac{(\tilde{x}_{ij} - \mu_j)^2}{2\sigma_j^2}\right) \end{equation}

% where $\mu_j$ and $\sigma_j$ are the mean and standard deviation of feature $j$ from the clean dataset. A higher confidence score indicates that the repaired value $\tilde{x}_{ij}$ is more consistent with the expected normal distribution of clean data.

% By providing confidence scores, we enable data engineers to better assess the trustworthiness of the repair suggestions, facilitating informed decision-making in the data cleaning process.


% \noindent{\textbf{Confidence Scoring}.}  
% To quantify the reliability of each repaired feature \(\tilde{x}_{ij}\), we assign a confidence score \(c_{ij} \in [0, 1]\). This score reflects how closely the repaired value matches the expected distribution of clean data. It is computed using the mean \(\mu_j\) and standard deviation \(\sigma_j\) of feature \(j\) from the clean dataset, where higher confidence scores indicate greater consistency with the expected normal distribution.

% By providing confidence scores, we help data engineers evaluate the trustworthiness of repair suggestions, facilitating better decision-making in the data cleaning process.

%Our approach offers several key advantages over traditional data quality verification methods. By leveraging GNNs and VAEs, it automatically identifies data quality issues without predefined constraints and detects hidden relationships within the data. This reduces the need for continuous expert input, making the process more efficient and scalable. Additionally, it can pinpoint problematic samples and specific features, facilitating targeted data cleaning and correction.





% \section{Taxonomy}

% As illustrated by Fig. \ref{}, the typical process of vision models based time series analysis has five components: (1) normalization/scaling; (2) time series to image transformation; (3) image modeling; (4) image to time series recovery; and (5) task processing. In the rest of this paper, we will discuss the typical methods for each of these components. The detailed taxonomy of the methods are summarized in Table \ref{tab.taxonomy}.

%Typical step: normalization/scaling, transformation, vision modeling, task-specific head, inverse transformation (for tasks that output time series, e.g., forecasting, generation, imputation, anomaly detection). Normalization is to fit the arbitrary range of time series values to RGB representation.

\begin{figure*}[!t]
\centering
\includegraphics[width=1.0\textwidth]{fig/fig_3.pdf}
% \vspace{-1em}
\caption{An illustration of different methods for imaging time series with a sample (length=336) from the \textit{Electricity} benchmark dataset \protect\cite{nie2023time}. (a)(c)(d)(e)(f) %are univariate methods.
visualize the same variate. (b) visualizes all 321 variates. Filterbank is omitted due to its %high
similarity to STFT.}\label{fig.tsimage}
\vspace{-0.2cm}
\end{figure*}

\begin{table*}[t]
\centering
\scriptsize
\setlength{\tabcolsep}{2.7pt}{
% \begin{tabular}{llllllllllll}
\begin{tabular}{llcccccccccl}
\toprule[1pt]
\multirow{2}{*}{Method} & \multirow{2}{*}{TS-Type} & \multirow{2}{*}{Imaging} & \multicolumn{5}{c}{Imaged Time Series Modeling} & \multirow{2}{*}{TS-Recover} & \multirow{2}{*}{Task} & \multirow{2}{*}{Domain} & \multirow{2}{*}{Code}\\ \cmidrule{4-8}
 & & & Multi-modal & Model & Pre-trained & Fine-tune & Prompt & & & & \\ \midrule
\cite{silva2013time} & UTS & RP & \xmark & \texttt{K-NN} & \xmark & \xmark & \xmark & \xmark & Classification & General & \xmark\\
\cite{wang2015encoding} & UTS & GAF & \xmark & \texttt{CNN} & \xmark & \cmark$^{\flat}$ & \xmark & \cmark & Classification & General & \xmark\\
\cite{wang2015imaging} & UTS & GAF & \xmark & \texttt{CNN} & \xmark & \cmark$^{\flat}$ & \xmark & \cmark & Multiple & General & \xmark\\
% \multirow{2}{*}{\cite{wang2015imaging}} & \multirow{2}{*}{UTS} & \multirow{2}{*}{GAF} & \multirow{2}{*}{\xmark} & \multirow{2}{*}{\texttt{CNN}} & \multirow{2}{*}{\xmark} & \multirow{2}{*}{\cmark$^{\flat}$} & \multirow{2}{*}{\xmark} & \multirow{2}{*}{\cmark} & Classification & \multirow{2}{*}{General} & \multirow{2}{*}{\xmark}\\
% & & & & & & & & & \& Imputation & & \\
\cite{ma2017learning} & MTS & Heatmap & \xmark & \texttt{CNN} & \xmark & \cmark$^{\flat}$ & \xmark & \cmark & Forecasting & Traffic & \xmark\\
\cite{hatami2018classification} & UTS & RP & \xmark & \texttt{CNN} & \xmark & \cmark$^{\flat}$ & \xmark & \xmark & Classification & General & \xmark\\
\cite{yazdanbakhsh2019multivariate} & MTS & Heatmap & \xmark & \texttt{CNN} & \xmark & \cmark$^{\flat}$ & \xmark & \xmark & Classification & General & \cmark\textsuperscript{\href{https://github.com/SonbolYb/multivariate_timeseries_dilated_conv}{[1]}}\\
MSCRED \cite{zhang2019deep} & MTS & Other ($\S$\ref{sec.othermethod}) & \xmark & \texttt{ConvLSTM} & \xmark & \cmark$^{\flat}$ & \xmark & \xmark & Anomaly & General & \cmark\textsuperscript{\href{https://github.com/7fantasysz/MSCRED}{[2]}}\\
\cite{li2020forecasting} & UTS & RP & \xmark & \texttt{CNN} & \cmark & \cmark & \xmark & \xmark & Forecasting & General & \cmark\textsuperscript{\href{https://github.com/lixixibj/forecasting-with-time-series-imaging}{[3]}}\\
\cite{cohen2020trading} & UTS & LinePlot & \xmark & \texttt{Ensemble} & \xmark & \cmark$^{\flat}$ & \xmark & \xmark & Classification & Finance & \xmark\\
% \cite{du2020image} & UTS & Spectrogram & \xmark & \texttt{CNN} & \xmark & \cmark$^{\flat}$ & \xmark & \xmark & Classification & Finance & \xmark\\
\cite{barra2020deep} & UTS & GAF & \xmark & \texttt{CNN} & \xmark & \cmark$^{\flat}$ & \xmark & \xmark & Classification & Finance & \xmark\\
% \cite{barra2020deep} & UTS & GAF & \xmark & \texttt{VGG-16} & \xmark & \cmark$^{\flat}$ & \xmark & \xmark & Classification & Finance & \xmark\\
% \cite{cao2021image} & UTS & RP & \xmark & \texttt{CNN} & \xmark & \cmark$^{\flat}$ & \xmark & \xmark & Classification & General & \xmark\\
VisualAE \cite{sood2021visual} & UTS & LinePlot & \xmark & \texttt{CNN} & \xmark & \cmark$^{\flat}$ & \xmark & \cmark & Forecasting & Finance & \xmark\\
% VisualAE \cite{sood2021visual} & UTS & LinePlot & \xmark & \texttt{CNN} & \xmark & \cmark$^{\flat}$ & \xmark & \xmark & Img-Generation & Finance & \xmark\\
\cite{zeng2021deep} & MTS & Heatmap & \xmark & \texttt{CNN,LSTM} & \xmark & \cmark$^{\flat}$ & \xmark & \cmark & Forecasting & Finance & \xmark\\
% \cite{zeng2021deep} & MTS & Heatmap & \xmark & \texttt{SRVP} & \xmark & \cmark$^{\flat}$ & \xmark & \cmark & Forecasting & Finance & \xmark\\
AST \cite{gong2021ast} & UTS & Spectrogram & \xmark & \texttt{DeiT} & \cmark & \cmark & \xmark & \xmark & Classification & Audio & \cmark\textsuperscript{\href{https://github.com/YuanGongND/ast}{[4]}}\\
TTS-GAN \cite{li2022tts} & MTS & Heatmap & \xmark & \texttt{ViT} & \xmark & \cmark$^{\flat}$ & \xmark & \cmark & Ts-Generation & Health & \cmark\textsuperscript{\href{https://github.com/imics-lab/tts-gan}{[5]}}\\
SSAST \cite{gong2022ssast} & UTS & Spectrogram & \xmark & \texttt{ViT} & \cmark$^{\natural}$ & \cmark & \xmark & \xmark & Classification & Audio & \cmark\textsuperscript{\href{https://github.com/YuanGongND/ssast}{[6]}}\\
MAE-AST \cite{baade2022mae} & UTS & Spectrogram & \xmark & \texttt{MAE} & \cmark$^{\natural}$ & \cmark & \xmark & \xmark & Classification & Audio & \cmark\textsuperscript{\href{https://github.com/AlanBaade/MAE-AST-Public}{[7]}}\\
AST-SED \cite{li2023ast} & UTS & Spectrogram & \xmark & \texttt{SSAST,GRU} & \cmark & \cmark & \xmark & \xmark & EventDetection & Audio & \xmark\\
\cite{jin2023classification} & UTS & %Multiple
LinePlot & \xmark & \texttt{CNN} & \cmark & \cmark & \xmark & \xmark & Classification & Physics & \xmark\\
ForCNN \cite{semenoglou2023image} & UTS & LinePlot & \xmark & \texttt{CNN} & \xmark & \cmark$^{\flat}$ & \xmark & \xmark & Forecasting & General & \xmark\\
Vit-num-spec \cite{zeng2023pixels} & UTS & Spectrogram & \xmark & \texttt{ViT} & \xmark & \cmark$^{\flat}$ & \xmark & \xmark & Forecasting & Finance & \xmark\\
% \cite{wimmer2023leveraging} & MTS & LinePlot & \xmark & \texttt{CLIP,LSTM} & \cmark & \cmark & \xmark & \xmark & Classification & Finance & \xmark\\
ViTST \cite{li2023time} & MTS & LinePlot & \xmark & \texttt{Swin} & \cmark & \cmark & \xmark & \xmark & Classification & General & \cmark\textsuperscript{\href{https://github.com/Leezekun/ViTST}{[8]}}\\
MV-DTSA \cite{yang2023your} & UTS\textsuperscript{*} & LinePlot & \xmark & \texttt{CNN} & \xmark & \cmark$^{\flat}$ & \xmark & \cmark & Forecasting & General & \cmark\textsuperscript{\href{https://github.com/IkeYang/machine-vision-assisted-deep-time-series-analysis-MV-DTSA-}{[9]}}\\
TimesNet \cite{wu2023timesnet} & MTS & Heatmap & \xmark & \texttt{CNN} & \xmark & \cmark$^{\flat}$ & \xmark & \cmark & Multiple & General & \cmark\textsuperscript{\href{https://github.com/thuml/TimesNet}{[10]}}\\
ITF-TAD \cite{namura2024training} & UTS & Spectrogram & \xmark & \texttt{CNN} & \cmark & \xmark & \xmark & \xmark & Anomaly & General & \xmark\\
\cite{kaewrakmuk2024multi} & UTS & GAF & \xmark & \texttt{CNN} & \cmark & \cmark & \xmark & \xmark & Classification & Sensing & \xmark\\
HCR-AdaAD \cite{lin2024hierarchical} & MTS & RP & \xmark & \texttt{CNN,GNN} & \xmark & \cmark$^{\flat}$ & \xmark & \xmark & Anomaly & General & \xmark\\
FIRTS \cite{costa2024fusion} & UTS & Other ($\S$\ref{sec.othermethod}) & \xmark & \texttt{CNN} & \xmark & \cmark$^{\flat}$ & \xmark & \xmark & Classification & General & \cmark\textsuperscript{\href{https://sites.google.com/view/firts-paper}{[11]}}\\
% \multirow{2}{*}{FIRTS \cite{costa2024fusion}} & \multirow{2}{*}{UTS} & Spectrogram & \multirow{2}{*}{\xmark} & \multirow{2}{*}{\texttt{CNN}} & \multirow{2}{*}{\xmark} & \multirow{2}{*}{\cmark$^{\flat}$} & \multirow{2}{*}{\xmark} & \multirow{2}{*}{\xmark} & \multirow{2}{*}{Classification} & \multirow{2}{*}{General} & \multirow{2}{*}{\cmark\textsuperscript{\href{https://sites.google.com/view/firts-paper}{[2]}}}\\
%  & & \& GAF,RP,MTF & & & & & & & & & \\
% \cite{homenda2024time} & UTS\textsuperscript{*} & Multiple & \xmark & \texttt{CNN} & \xmark & \cmark$^{\flat}$ & \xmark & \xmark & Classification & General & \xmark\\
CAFO \cite{kim2024cafo} & MTS & RP & \xmark & \texttt{CNN,ViT} & \xmark & \cmark$^{\flat}$ & \xmark & \xmark & Explanation & General & \cmark\textsuperscript{\href{https://github.com/eai-lab/CAFO}{[12]}}\\
% \multirow{2}{*}{CAFO \cite{kim2024cafo}} & \multirow{2}{*}{MTS} & \multirow{2}{*}{RP} & \multirow{2}{*}{\xmark} & \texttt{ShuffleNet,ResNet} & \multirow{2}{*}{\cmark} & \multirow{2}{*}{\cmark} & \multirow{2}{*}{\xmark} & \multirow{2}{*}{\xmark} & Classification & \multirow{2}{*}{General} & \multirow{2}{*}{\cmark}\\
%  & & & & \texttt{MLP-Mixer,ViT} & & & & & \& Explanation & & \\
ViTime \cite{yang2024vitime} & UTS\textsuperscript{*} & LinePlot & \xmark & \texttt{ViT} & \cmark$^{\natural}$ & \cmark & \xmark & \cmark & Forecasting & General & \cmark\textsuperscript{\href{https://github.com/IkeYang/ViTime}{[13]}}\\
ImagenTime \cite{naiman2024utilizing} & MTS & Other ($\S$\ref{sec.othermethod}) & \xmark & %\texttt{Diffusion}
\texttt{CNN} & \xmark & \cmark$^{\flat}$ & \xmark & \cmark & Ts-Generation & General & \cmark\textsuperscript{\href{https://github.com/azencot-group/ImagenTime}{[14]}}\\
TimEHR \cite{karami2024timehr} & MTS & Heatmap & \xmark & \texttt{CNN} & \xmark & \cmark$^{\flat}$ & \xmark & \cmark & Ts-Generation & Health & \cmark\textsuperscript{\href{https://github.com/esl-epfl/TimEHR}{[15]}}\\
VisionTS \cite{chen2024visionts} & UTS\textsuperscript{*} & Heatmap & \xmark & \texttt{MAE} & \cmark & \cmark & \xmark & \cmark & Forecasting & General & \cmark\textsuperscript{\href{https://github.com/Keytoyze/VisionTS}{[16]}}\\ \midrule
InsightMiner \cite{zhang2023insight} & UTS & LinePlot & \cmark & \texttt{LLaVA} & \cmark & \cmark & \cmark & \xmark & Txt-Generation & General & \xmark\\
\cite{wimmer2023leveraging} & MTS & LinePlot & \cmark & \texttt{CLIP,LSTM} & \cmark & \cmark & \xmark & \xmark & Classification & Finance & \xmark\\
% \cite{dixit2024vision} & UTS & Spectrogram & \cmark & \texttt{GPT4o,Gemini} & \cmark & \xmark & \cmark & \xmark & Classification & Audio & \xmark\\
\multirow{2}{*}{\cite{dixit2024vision}} & \multirow{2}{*}{UTS} & \multirow{2}{*}{Spectrogram} & \multirow{2}{*}{\cmark} & \texttt{GPT4o,Gemini} & \multirow{2}{*}{\cmark} & \multirow{2}{*}{\xmark} & \multirow{2}{*}{\cmark} & \multirow{2}{*}{\xmark} & \multirow{2}{*}{Classification} & \multirow{2}{*}{Audio} & \multirow{2}{*}{\xmark}\\
 & & & & \& \texttt{Claude3} & & & & & & & \\
\cite{daswani2024plots} & MTS & LinePlot & \cmark & \texttt{GPT4o,Gemini} & \cmark & \xmark & \cmark & \xmark & Multiple & General & \xmark\\
TAMA \cite{zhuang2024see} & UTS & LinePlot & \cmark & \texttt{GPT4o} & \cmark & \xmark & \cmark & \xmark & Anomaly & General & \xmark\\
\cite{prithyani2024feasibility} & MTS & LinePlot & \cmark & \texttt{LLaVA} & \cmark & \cmark & \cmark & \xmark & Classification & General & \cmark\textsuperscript{\href{https://github.com/vinayp17/VLM_TSC}{[17]}}\\
\bottomrule[1pt]
\end{tabular}}
\vspace{-0.25cm}
\caption{Taxonomy of vision models on time series. The top panel includes single-modal models. The bottom panel includes multi-modal models. {\bf TS-Type} denotes type of time series. {\bf TS-Recover} denotes %whether time series recovery ($\S$\ref{sec.processing}) has been performed.
recovering time series from predicted images ($\S$\ref{sec.processing}). \textsuperscript{*}: %the model has been %applied on MTSs by %processing %modeling the individual UTSs of each MTS.
the method has been used to model the individual UTSs of an MTS. $^{\natural}$: a new pre-trained model was proposed in the work. $^{\flat}$: %without using a pre-trained model, fine-tune means training from scratch.
when pre-trained models were unused, ``Fine-tune'' refers to train a task-specific model from scratch. %In the
{\bf Model} column: \texttt{CNN} could be regular CNN, ResNet, VGG-Net, %U-Net,
{\em etc.}}\label{tab.taxonomy}
% The code only include verified official code from the authors.
\vspace{-0.3cm}
\end{table*}

\begin{table*}[t]
\centering
\small
\setlength{\tabcolsep}{2.9pt}{
\begin{tabular}{l|l|l|l}\hline
% \toprule[1pt]
\rowcolor{gray!20}
{\bf Method} & {\bf TS-Type} & {\bf Advantages} & {\bf Limitations}\\ \hline
Line Plot ($\S$\ref{sec.lineplot}) & UTS, MTS & matches human perception of time series & limited to MTSs with a small number of variates\\ \hline
Heatmap ($\S$\ref{sec.heatmap}) & UTS, MTS & straightforward for both UTSs and MTSs & the order of variates may affect their correlation learning\\ \hline
Spectrogram ($\S$\ref{sec.spectrogram}) & UTS & encodes the time-frequency space & limited to UTSs; needs a proper choice of window/wavelet\\ \hline
GAF ($\S$\ref{sec.gaf}) & UTS & encodes the temporal correlations in a UTS & limited to UTSs; $O(T^{2})$ time and space complexity\\ \hline% for long time series\\ \hline
% RP ($\S$\ref{sec.rp}) & UTS & flexibility in image size by tuning $m$ and $\tau$ & limited to UTSs; the pattern has a threshold-dependency\\ \hline
RP ($\S$\ref{sec.rp}) & UTS & flexibility in image size by tuning $m$ and $\tau$ & limited to UTSs; information loss after thresholding\\ \hline
% \bottomrule[1pt]
\end{tabular}}
\vspace{-0.2cm}
\caption{Summary of the five primary methods for transforming time series to images. {\bf TS-Type} denotes type of time series.}\label{tab.tsimage}
\vspace{-0.2cm}
\end{table*}

\section{Time Series To Image Transformation}\label{sec.tsimage}

% This section summarizes 5 major methods for imaging time series ($\S$\ref{sec.lineplot}-$\S$\ref{sec.rp}). We also discuss some other methods ($\S$\ref{sec.othermethod}) and how to model MTS with these methods ($\S$\ref{sec.modelmts}).
This section summarizes the methods for imaging time series ($\S$\ref{sec.lineplot}-$\S$\ref{sec.othermethod}) and their extensions to encode MTSs ($\S$\ref{sec.modelmts}).

% This section summarizes 5 major methods for transforming time series to images, including Line Plot, Heatmap, Spetrogram, GAF and RP, and several minor methods. We discuss their pros and cons and how to deal with MTS.

% This section discusses the advantages and limitations of different methods for time series to image transformation (invertible, efficiency, information preservation, MTS, long-range time series, parametric, etc.).

%\subsection{Methods}

\vspace{-0.08cm}

\subsection{Line Plot}\label{sec.lineplot}

Line Plot is a straightforward way for visualizing UTSs for human analysis ({\em e.g.}, stocks, power consumption, {\em etc.}). As illustrated by Fig. \ref{fig.tsimage}(a), the simplest approach is to draw a 2D image with x-axis representing %the time horizon
time steps and y-axis representing %the magnitude of the normalized time series.
time-wise values, %A line is used to connect all values of the series over time.
with a line connecting all values of the series over time. This image can be %represented by either three-channel pixels or single-channel pixels
either three-channel ({\em i.e.}, RGB) or single-channel as the colors may not %provide additional information
be informative %\cite{cohen2020trading,sood2021visual,jin2023classification,zhang2023insight,zhuang2024see}.
\cite{cohen2020trading,sood2021visual,jin2023classification,zhang2023insight}. ForCNN \cite{semenoglou2023image} even uses a single 8-bit integer to represent each pixel for black-white images. So far, there is no consensus on whether other graphical components, such as legend, grids and tick labels, could provide extra benefits in any task. For example, ViTST \cite{li2023time} finds these components are superfluous in a classification task, while TAMA \cite{zhuang2024see} finds grid-like auxiliary lines help enhance anomaly detection.

In addition to the regular Line Plot, MV-DTSA \cite{yang2023your} and ViTime \cite{yang2024vitime} divide an image into $h\times L$ grids, %where $h$ is the number of rows and $L$ is the number of columns,
and %introduced
define a function to map each time step of a UTS to a grid, producing a grid-like Line Plot. Also, we include methods that use Scatter Plot \cite{daswani2024plots,prithyani2024feasibility} in this category because %the only difference between a Scatter Plot and a Line Plot is whether the time-wise values are connected by lines.
a Scatter Plot resembles a Line Plot but doesn't connect %time-wise values
data points with a line. By comparing them, \cite{prithyani2024feasibility} finds a Line Plot could induce better time series classification.

For MTSs, we defer the discussion on Line Plot to $\S$\ref{sec.modelmts}.

% For MTS, some methods use the channel-independence assumption proposed in \cite{nie2023time} and represent each variate in MTS with an individual Line Plot \cite{yang2023your,yang2024vitime}. ViTST \cite{li2023time} also uses an individual Line Plot per variate, but colors different lines and assembles all plots to form a bigger image. The method in \cite{wimmer2023leveraging} plots %the time series of
% all variates in a single Line Plot and distinguish them by %use different
% types of lines ({\em e.g.}, solid, dashed, dotted, {\em etc.}). %to distinguish them.
% However, these methods only work for a small number of variates. For example, in \cite{wimmer2023leveraging}, there are only 4 variates in its financial MTSs.

%\cite{li2023time} space-costly because of blank pixels. scatter plot.

%Invertible with a numeric prediction head \cite{sood2021visual}. It fits tasks such as forecasting, imputation, etc.

\vspace{-0.08cm}

\subsection{Heatmap}\label{sec.heatmap}

As shown in Fig. \ref{fig.tsimage}(b), Heatmap is a 2D visualization of the magnitude of the values in a matrix using color. %The variation of color represents the intensity of each value. %Therefore,
It has been used to %directly
represent the matrix of an MTS, {\em i.e.}, $\mat{X} \in \mathbb{R}^{d\times T}$, as a one-channel $d\times T$ image \cite{li2022tts,yazdanbakhsh2019multivariate}. Similarly, TimEHR \cite{karami2024timehr} represents an {\em irregular} MTS, where the intervals between time steps are uneven, as a $d\times H$ Heatmap image by grouping the uneven time steps into $H$ even time bins. In \cite{zeng2021deep}, a different method is used for visualizing a 9-variate financial %time series.
MTS. It reshapes the 9 variates at each time step to a $3\times 3$ Heatmap image, and uses the sequence of images to forecast future %image
frames, achieving %time series
%MTS
time series forecasting. In contrast, VisionTS \cite{chen2024visionts} uses Heatmap to visualize UTSs. %instead.
Similar to TimesNet \cite{wu2023timesnet}, it first segments a length-$T$ UTS into $\lfloor T/P\rfloor$ length-$P$ subsequences, where $P$ is a parameter representing a periodicity of the UTS. Then the subsequences are stacked into a $P\times \lfloor T/P\rfloor$ matrix, %and duplicated 3 times to produce a 3-channel
with 3 duplicated channels, to produce a grayscale image %which serves as an
input to %a vision foundation model.
an LVM. To encode MTSs, VisionTS adopts the channel independence assumption \cite{nie2023time} and individually models each variate in an MTS.

\vspace{0.2cm}

\noindent{\bf Remark.} Heatmap can be used to visualize matrices of various forms. It is also used for matrices generated by the subsequent methods ({\em e.g.}, Spectrogram, GAF, RP) in this section. In this paper, the name Heatmap refers specifically to images that use color to visualize the (normalized) values in UTS $\mat{x}$ or MTS $\mat{X}$ without performing other transformations.

%\cite{chen2024visionts,karami2024timehr} bin version of TSH \cite{karami2024timehr}, DE and STFT \cite{naiman2024utilizing} (DE can be used for constructing RP), rearrange variates for video version of TSH \cite{zeng2021deep}.

%\vspace{0.2cm}

\subsection{Spectrogram}\label{sec.spectrogram}

A {\em spectrogram} is a visual representation of the spectrum of frequencies of a signal as it varies with time, which are extensively used for analyzing audio signals \cite{gong2021ast}. Since audio signals are a type of UTS, spectrogram can be considered as a method for imaging a UTS. As shown in Fig. \ref{fig.tsimage}(c), a common format is a 2D heatmap image with x-axis representing time steps and y-axis representing frequency, {\em a.k.a.} a time-frequency space. %The color at each point
Each pixel in the image represents the (logarithmic) amplitude of a specific frequency at a specific time point. Typical methods for %transforming a UTS to
producing a spectrogram include {\bf Short-Time Fourier Transform (STFT)} \cite{griffin1984signal}, {\bf Wavelet Transform} \cite{daubechies1990wavelet}, and {\bf Filterbank} \cite{vetterli1992wavelets}.

\vspace{0.2cm}

\noindent{\bf STFT.} %Discrete Fourier transform (DFT) can be used to represent a UTS signal %$\mat{x}=[x_{1}, ..., x_{T}]$
%$\mat{x}\in\mathbb{R}^{1\times T}$ as a sum of sinusoidal components. The output of the transform is a function of frequency $f(w)$, describing the intensity of each constituent frequency $w$ of the entire UTS. 
Discrete Fourier transform (DFT) can be used to describe the intensity $f(w)$ of each constituent frequency $w$ of a UTS signal $\mat{x}\in\mathbb{R}^{1\times T}$. However, $f(w)$ has no time dependency. It cannot provide dynamic information such as when a specific frequency appear in the UTS. STFT addresses this deficiency by sliding a window function $g(t)$ over the time steps in %the UTS,
$\mat{x}$, and computing the DFT within each window by
\begin{equation}\label{eq.stft}
\small
\begin{aligned}
f(w,\tau) = \sum_{t=1}^{T}x_{t}g(t - \tau)e^{-iwt}
\end{aligned}
\end{equation}
where $w$ is frequency, $\tau$ is the position of the window, $f(w,\tau)$ describes the intensity of frequency $w$ at time step $\tau$.

%With a proper selection of the
By selecting a proper window function $g(\cdot)$ ({\em e.g.}, Gaussian/Hamming/Bartlett window), %({\em e.g.}, Gaussian window, Hamming window, Bartlett window), %{\em etc.}),
a 2D spectrogram ({\em e.g.}, Fig. \ref{fig.tsimage}(c)) can be drawn via a heatmap on the squared values $|f(w,\tau)|^{2}$, with $w$ as the y-axis, and $\tau$ as the x-axis. For example, \cite{dixit2024vision} uses STFT based spectrogram as an input to LMMs %\hh{do you mean LVMs? check}
for time series classification.

%Fourier transform is a powerful data analysis tool that represents any complex signal as a sum of sines and cosines and transforms the signal from the time domain to the frequency domain. However, Fourier transform can only show which frequencies are present in the signal, but not when these frequencies appear. The STFT divides original signal into several parts using a sliding window to fix this problem. STFT involves a sliding window for extracting frequency components within the window.

\vspace{0.2cm}

\noindent{\bf Wavelet Transform.} %Like Fourier transform, %\hh{this paragraph needs a citation}
Continuous Wavelet Transform (CWT) uses the inner product to measure the similarity between a signal function $x(t)$ and an analyzing function. %In STFT (Eq.~\eqref{eq.stft}), the analyzing function is a windowed exponential $g(t - \tau)e^{-iwt}$.
%In CWT,
The analyzing function is a {\em wavelet} $\psi(t)$, where the typical choices include Morse wavelet, Morlet wavelet, %Daubechies wavelet, %Beylkin wavelet, 
{\em etc.} %The
CWT compares $x(t)$ to the shifted and scaled ({\em i.e.}, stretched or shrunk) versions of the wavelet, and output a CWT coefficient by
\begin{equation}\label{eq.cwt}
\small
\begin{aligned}
c(s,\tau) = \int_{-\infty}^{\infty}x(t)\frac{1}{s}\psi^{*}(\frac{t - \tau}{s})dt
\end{aligned}
\end{equation}
where $*$ denotes complex conjugate, $\tau$ is the time step to shift, and $s$ represents the scale. In practice, a discretized version of CWT in Eq.~\eqref{eq.cwt} is implemented for UTS $[x_{1}, ..., x_{T}]$.

It is noteworthy that the scale $s$ controls the frequency encoded in a wavelet -- a larger $s$ leads to a stretched wavelet with a lower frequency, and vice versa. As such, by varying $s$ and $\tau$, a 2D spectrogram ({\em e.g.}, Fig. \ref{fig.tsimage}(d)) can be drawn %, often with a heatmap
on $|c(s,\tau)|$, where $s$ is the y-axis and $\tau$ is the x-axis. Compared to STFT, which uses a fixed window size, Wavelet Transform allows variable wavelet sizes -- a larger size %region
for more precise low frequency information. 
%Usually, $s$ and $\tau$ vary dependently -- a larger $s$ leads to a stretched wavelet that shifts slowly, {\em i.e.}, a smaller $\tau$. This property %of CWT
%yields a spectrogram that balances the resolutions of frequency %$s$
%and time, %$\tau$,
%which is an advantage over the fixed time resolution in STFT.
% Thus, both of the methods in %\cite{du2020image}
% \cite{namura2024training} and \cite{zeng2023pixels} choose CWT (with Morlet wavelet) to generate the spectrogram.
Thus, the methods in \cite{du2020image,namura2024training,zeng2023pixels} choose CWT (with Morlet wavelet) to generate the spectrogram.

%A wavelet is a wave-like oscillation that has zero mean and is localized in both time and frequency space.

\vspace{0.2cm}

\noindent{\bf Filterbank.} This method %is relevant to
resembles STFT and is often used in processing audio signals. Given an audio signal, it firstly goes through a {\em pre-emphasis filter} to boost high frequencies, which helps improve the clarity of the signal. Then, STFT is applied on the signal. %with a sliding window $g(t)$ of size $k$ that shifts in a fixed stride $\tau$. %where the adjacent windows may overlap in $k$ time length.
%Finally, filterbank features are computed by applying multiple ``triangle-shaped'' filters spaced on the Mel-scale to the STFT output $f(w, \tau)$. %where Mel-scale is a method to make the filters more discriminative on lower frequencies, %than higher frequencies,
%imitating the non-linear human ear perception of sound.
Finally, multiple ``triangle-shaped'' filters spaced on a Mel-scale are applied to the STFT power spectrum $|f(w, \tau)|^{2}$ to extract frequency bands. The outcome filterbank features $\hat{f}(w, \tau)$ can be used to yield a spectrogram with $w$ as the y-axis, and $\tau$ as the x-axis.

%Filterbank was introduced in AST \cite{gong2021ast} with %$k$=25ms
Filterbank was adopted in AST \cite{gong2021ast} with 
a 25ms Hamming window that shifts every 10ms for classifying audio signals using Vision Transformer (ViT). It then becomes widely used in the follow-up works such as SSAST \cite{gong2022ssast}, MAE-AST \cite{baade2022mae}, and AST-SED \cite{li2023ast}, as summarized in Table \ref{tab.taxonomy}.



%Use MLP to predict TS directly \cite{zeng2023pixels}.

%\vspace{0.2cm}

% \vspace{0.2cm}

\subsection{Gramian Angular Field (GAF)}\label{sec.gaf}

GAF was introduced for classifying UTSs using CNNs %using %image based CNNs
by \cite{wang2015encoding}. It was then extended %with an extension
to an imputation task in \cite{wang2015imaging}. Similarly, \cite{barra2020deep} applied GAF for financial time series forecasting.

Given a UTS $\mat{x}\in\mathbb{R}^{1\times T}$, %$[x_{1}, ..., x_{T}]$,
the first step %before GAF
is to rescale each $x_{t}$ to a value $\tilde{x}_{t}$ %in the interval of
within $[0, 1]$ (or $[-1, 1]$). %by min-max normalization.
This range enables mapping $\tilde{x}_{t}$ to polar coordinates by $\phi_{t}=\text{arccos}(\tilde{x}_{i})$, with a radius $r=t/N$ encoding the time stamp, where $N$ is a constant factor to regularize the span of the polar coordinates. %system. Then,
Two types of GAF, Gramian Sum Angular Field (GASF) and Gramian Difference Angular Field (GADF) are defined as
\begin{equation}\label{eq.gaf}
\small
\begin{aligned}
&\text{GASF:}~~\text{cos}(\phi_{t} + \phi_{t'})=x_{t}x_{t'} - \sqrt{1 - x_{t}^{2}}\sqrt{1 - x_{t'}^{2}}\\
&\text{GADF:}~~\text{sin}(\phi_{t} - \phi_{t'})=x_{t'}\sqrt{1 - x_{t}^{2}} - x_{t}\sqrt{1 - x_{t'}^{2}}
\end{aligned}
\end{equation}
which exploits the pairwise temporal correlations in the UTS. Thus, the outcome is a $T\times T$ matrix $\mat{G}$ with $\mat{G}_{t,t'}$ specified by either type in Eq.~\eqref{eq.gaf}. A GAF image is a heatmap on $\mat{G}$ with both axes representing time, as illustrated by Fig. \ref{fig.tsimage}(e).

% Invertible.

% \vspace{0.2cm}

\subsection{Recurrence Plot (RP)}\label{sec.rp}

%RP \cite{eckmann1987recurrence} is a method to encode a UTS into an image that aims to capture the periodic patterns in the UTS by using its reconstructed {\em phase space}. The phase space of a UTS $[x_{1}, ..., x_{T}]$ can be reconstructed by {\em time delay embedding}, which is a set of new vectors $\mat{v}_{1}$, ..., $\mat{v}_{l}$ with

RP \cite{eckmann1987recurrence} encodes a UTS into an image that captures its periodic patterns by using its reconstructed {\em phase space}. The phase space of %a UTS %$[x_{1}, ..., x_{T}]$
$\mat{x}\in\mathbb{R}^{1\times T}$ can be reconstructed by {\em time delay embedding} -- a set of new vectors $\mat{v}_{1}$, ..., $\mat{v}_{l}$ with
\begin{equation}\label{eq.de}
\small
\begin{aligned}
\mat{v}_{t}=[x_{t}, x_{t+\tau}, x_{t+2\tau}, ..., x_{t+(m-1)\tau}]\in\mathbb{R}^{m\tau},~~~1\le t \le l
\end{aligned}
\end{equation}
where $\tau$ is the time delay, $m$ is the dimension of the phase space, both %of which
are hyperparameters. Hence, $l=T-(m-1)\tau$. With vectors $\mat{v}_{1}$, ..., $\mat{v}_{l}$, an RP image %is constructed by measuring
measures their pairwise distances, results in an $l\times l$ image whose element
\begin{equation}\label{eq.rp}
\small
\begin{aligned}
\text{RP}_{i,j}=\Theta(\varepsilon - \|\mat{v}_{i} - \mat{v}_{j}\|),~~~1\le i,j\le l
\end{aligned}
\end{equation}
where $\Theta(\cdot)$ is the Heaviside step function, $\varepsilon$ is a threshold, and $\|\cdot\|$ is a norm function such as $\ell_{2}$ norm. Eq.~\eqref{eq.rp} %states RP produces a heatmap image on a binary matrix with $\text{RP}_{i,j}=1$ if $\mat{v}_{i}$ and $\mat{v}_{j}$ are sufficiently similar.
generates a binary matrix with $\text{RP}_{i,j}=1$ if $\mat{v}_{i}$ and $\mat{v}_{j}$ are sufficiently similar, producing a black-white image ({\em e.g.}, Fig. \ref{fig.tsimage}(f)).% ({\em e.g.}, a periodic pattern).

An advantage of RP is its flexibility in image size by tuning $m$ and $\tau$. Thus it has been used for time series classification %\cite{cao2021image},
\cite{silva2013time,hatami2018classification}, forecasting \cite{li2020forecasting}, anomaly detection \cite{lin2024hierarchical} and %feature-wise
explanation \cite{kim2024cafo}. Moreover, the method in \cite{hatami2018classification}, and similarly in HCR-AdaAD \cite{lin2024hierarchical}, omit the thresholding in Eq.~\eqref{eq.rp} and uses $\|\mat{v}_{i} - \mat{v}_{j}\|$ to produce continuously valued images %in a classification task
to avoid information loss.


% \vspace{0.2cm}

\subsection{Other Methods}\label{sec.othermethod}

%There are some less commonly used methods. For example, in
Additionally, %there are some peripheral methods. %In addition to GAF,
\cite{wang2015encoding} introduces Markov Transition Field (MTF) for imaging a UTS. %$\mat{x}\in\mathbb{R}^{1\times T}$. 
%MTF first assigns each $x_{t}$ to one of $Q$ quantile bins, then builds a $Q\times Q$ Markov transition matrix $\mat{M}$ {\em s.t.} $\mat{M}_{i,j}$ represents the frequency %with which
%of the case when a point $x_{t}$ in the $i$-th bin is followed by a point $x_{t'}$ in the $j$-th bin, {\em i.e.}, $t=t'+1$. Matrix $\mat{M}$ serves as the input of a heatmap image.
MTF is a matrix $\mat{M}\in\mathbb{R}^{Q\times Q}$ encoding the transition probabilities over time segments, where $Q$ is the number of segments. %Moreover,
ImagenTime \cite{naiman2024utilizing} stacks the delay embeddings $\mat{v}_{1}$, ..., $\mat{v}_{l}$ in Eq.~\eqref{eq.de} to an $l\times m\tau$ matrix for visualizing UTSs. %It also uses a variant of STFT.
% The method in \cite{homenda2024time} introduces five different 2D images by counting, rearranging, replicating the values in a UTS. 
MSCRED \cite{zhang2019deep} uses heatmaps on the $d\times d$ correlation matrices of MTSs with $d$ variates for anomaly detection. 
Furthermore, some methods use a mixture of imaging methods by stacking different transformations. \cite{wang2015imaging} stacks GASF, GADF, MTF to a 3-channel image. %Similarly,
FIRTS \cite{costa2024fusion} builds a 3-channel image by stacking GASF, MTF and RP. %the GASF, MTF, RP representations of each UTS.
%\cite{jin2023classification} combines Line Plot with Constant-Q Transform (CQT) \cite{brown1991calculation}, a method related to wavelet transform ($\S$\ref{sec.spectrogram}), to generate 2-channel images.
The mixture methods encode a UTS with multiple views and were found more robust than single-view images in these works for %time series
classification tasks.

\subsection{How to Model MTS}\label{sec.modelmts}

In the above methods, Heatmap ($\S$\ref{sec.heatmap}) can be %directly
used to visualize the %2D
variate-time matrices, $\mat{X}$, of MTSs ({\em e.g.}, Fig. \ref{fig.structure}(b)), where correlated variates %are better to
should be spatially close to each other. Line Plot ($\S$\ref{sec.lineplot}) can be used to visualize MTSs by plotting all variates in the same image \cite{wimmer2023leveraging,daswani2024plots} or combining all univariate images to compose a bigger %1-channel
image \cite {li2023time}, but these methods only work for a small number of variates. Spectrogram ($\S$\ref{sec.spectrogram}), GAF ($\S$\ref{sec.gaf}), and RP ($\S$\ref{sec.rp}) were designed specifically for UTSs. For these methods and Line Plot, which are not straightforward %for MTS transformation,
in imaging MTSs, the general approaches %to use them %for MTS
include using channel independence assumption to model each variate individually \cite{nie2023time}, %like VisionTS \cite{chen2024visionts},
or stacking the images of $d$ variates to form a $d$-channel image %as did by
\cite{naiman2024utilizing,kim2024cafo}. %\cite{prithyani2024feasibility,naiman2024utilizing,kim2024cafo}.
However, the latter does not fit some vision models pre-trained on RGB images which requires 3-channel inputs (more discussions are deferred to $\S$\ref{sec.processing}).

\vspace{0.2cm}

\noindent{\bf Remark.} As a summary, Table \ref{tab.tsimage} recaps the salient advantages and limitations of the five primary imaging methods that are introduced in this section.

% \hh{can we have a table (e.g., rows are different imaging methods and columns are a few desirable propoerties) or a short paragraph to discuss/summarize/compare the strenths and weakness of different imaging methods for ts? This might bring some structure/comprehension to this section (as opposed to, e.g., some reviewer might complain that what we do here is a laundry list)}

\section{Imaged Time Series Modeling}\label{sec.model}

With image representations, time series analysis can be readily performed with vision models. This section discusses such solutions from %traditional vision models %($\S$\ref{sec.cnns})
%to the recent large vision models %($\S$\ref{sec.lvms})
%and large multimodal models.% ($\S$\ref{sec.lmms}).
the traditional models to the SOTA models.

\begin{figure*}[!t]
\centering
\includegraphics[width=0.9\textwidth]{fig/fig_2.pdf}
% \vspace{-1em}
\caption{An illustration of different modeling strategies on imaged time series in (a)(b)(c) and task-specific heads in (d).}\label{fig.models}
\vspace{-0.2cm}
\end{figure*}

\subsection{Conventional Vision Models}\label{sec.cnns}

%Similar to
Following traditional %methods on
image classification, \cite{silva2013time} applies a K-NN classifier on the RPs of time series, \cite{cohen2020trading} applies an ensemble of fundamental classifiers such as %linear regression, SVM, Ada Boost, {\em etc.}
SVM and AdaBoost on the Line Plots %images
for time series classification. As an image encoder, %a typical encoder, %of images,
CNNs have been %extensively
widely used for learning image representations. %\cite{he2016deep}.
Different from using 1D CNNs on sequences %UTS or MTS
\cite{bai2018empirical}, %regular
2D or 3D CNNs can be applied on imaged time series as shown in Fig. \ref{fig.models}(a). %to learn time series representations by encoding their image transformations.
For example, %standard
regular CNNs have been used on Spectrograms \cite{du2020image}, tiled CNNs have been used on GAF images \cite{wang2015encoding,wang2015imaging}, dilated CNNs have been used on Heatmap images \cite{yazdanbakhsh2019multivariate}. More frequently, ResNet \cite{he2016deep}, Inception-v1 \cite{szegedy2015going}, and VGG-Net \cite{simonyan2014very} have been used on Line Plots \cite{jin2023classification,semenoglou2023image}, Heatmap images \cite{zeng2021deep}, RP images \cite{li2020forecasting,kim2024cafo}, GAF images \cite{barra2020deep,kaewrakmuk2024multi}, 
% Heatmaps \cite{zeng2021deep}, RPs \cite{li2020forecasting,kim2024cafo}, GAFs \cite{barra2020deep,kaewrakmuk2024multi},
and even a mixture of GAF, MTF and RP images \cite{costa2024fusion}. In particular, for time series generation tasks, %a diffusion model with U-Nets \cite{naiman2024utilizing} and GAN frameworks of CNNs \cite{li2022tts,karami2024timehr} have also been explored.%investigated.
GAN frameworks of CNNs \cite{li2022tts,karami2024timehr} and a diffusion model with U-Nets \cite{naiman2024utilizing} have also been explored.

Due to their small to medium sizes, these models are often trained from scratch using task-specific training data. %per task using the task's training set. %of time series images.
Meanwhile, fine-tuning {\em pre-trained vision models}  %such as those pre-trained on ImageNet, %\cite{deng2009imagenet}, 
have already been found promising in cross-modality knowledge transfer for time series anomaly detection \cite{namura2024training}, forecasting \cite{li2020forecasting} and classification \cite{jin2023classification}.

% \cite{li2020forecasting} uses ImageNet pretrained CNNs.

\subsection{Large Vision Models (LVMs)}\label{sec.lvms}

Vision Transformer (ViT) \cite{dosovitskiy2021image} has %given birth to
inspired the development of %some
modern LVMs %large vision models (LVMs)
such as %DeiT \cite{touvron2021training}, 
Swin \cite{liu2021swin}, BEiT \cite{bao2022beit}, and MAE \cite{he2022masked}. %Given an input image, ViT splits it
As Fig. \ref{fig.models}(b) shows, ViT splits an %input
image into {\em patches} of fixed size, then embeds each patch and augments it with a positional embedding. The %resulting
vectors of patches are processed by a Transformer %encoder
as if they were token embeddings. Compared to CNNs, ViTs are less data-efficient, but have higher capacity. %Consequently,
Thus, %the
{\em pre-trained} ViTs have been explored for modeling %the images of time series.
imaged time series. For example, AST \cite{gong2021ast} fine-tunes DeiT \cite{touvron2021training} on the filterbank spetrogram of audios %signals
for classification tasks and finds %using
ImageNet-pretrained DeiT is remarkably effective in knowledge transfer. The fine-tuning paradigm has also been %similarly
adopted in \cite{zeng2023pixels,li2023time} but with different pre-trained models %initializations
such as Swin by \cite{li2023time}. 
VisionTS \cite{chen2024visionts} %explains
attributes %the superiority of LVMs
LVMs' superiority over LLMs in knowledge transfer %over LLMs %as an outcome of
to the small gap between the pre-trained images and imaged time series. %the patterns learned from the large-scale pre-trained images and the patterns in the images of time series.
It %also
finds that with one-epoch fine-tuning, MAE becomes the SOTA time series forecasters on %many
some benchmark datasets.

Similar to %build
time series foundation models %\cite{das2024decoder,goswami2024moment,ansari2024chronos,shi2024time}, %such as TimesFM \cite{das2024decoder}, MOMENT \cite{goswami2024moment}, Chronos \cite{ansari2024chronos} and Time-MoE \cite{shi2024time},
such as TimesFM \cite{das2024decoder}, %and MOMENT \cite{goswami2024moment}, 
there are some initial efforts in pre-training ViT architectures with imaged time series. Following AST, SSAST \cite{gong2022ssast} introduced a %joint discriminative and generative
%masked spectrogram patch prediction self-supervised learning framework
masked spectrogram patch prediction framework for pre-training ViT on a large dataset -- AudioSet-2M. Then it becomes a backbone of some follow-up works such as AST-SED \cite{li2023ast} for sound event detection. %To be effective for UTSs,
For UTSs, ViTime \cite{yang2024vitime} generates a large set of Line Plots of synthetic UTSs for pre-training ViT, which was found superior over TimesFM in zero-shot forecasting tasks on benchmark datasets.

\subsection{Large Multimodal Models (LMMs)}\label{sec.lmms}

%As Large Multimodal Models (LMMs)
As LMMs %are getting
get growing attentions, some %of the
notable LMMs, such as LLaVA \cite{liu2023visual}, Gemini \cite{team2023gemini}, GPT-4o \cite{achiam2023gpt} and Claude-3 \cite{anthropic2024claude}, have been explored to consolidate the power of LLMs %on time series
and LVMs in time series analysis. 
Since LMMs support multimodal input via prompts, methods in this thread typically prompt LMMs with the textual and imaged representations of time series, %textual representation of time series and their %image transformations, transformed images,
%then instruct LMMs
and instructions on what tasks to perform ({\em e.g.}, Fig. \ref{fig.models}(c)).

InsightMiner \cite{zhang2023insight} is a pioneer work that uses the LLaVA architecture to generate %textual descriptions about
texts describing the trend of each input UTS. It extracts the trend of a UTS by Seasonal-Trend decomposition, encodes the Line Plot of the trend, and concatenates the embedding of the Line Plot with the embeddings of a textual instruction, which includes a sequence of numbers representing the UTS, {\em e.g.}, ``[1.1, 1.7, ..., 0.3]''. The concatenated embeddings are taken by a language model for generating trend descriptions. %It also fine-tunes a few layers with the generated texts to align LLaVA checkpoints with time series domain.
Similarly, \cite{prithyani2024feasibility} adopts the LLaVA architecture, but for MTS classification. An MTS is encoded by %a sequence of
the visual %token
embeddings of the stacked Line Plots of all variates. %meanwhile
%The method also stacks
%The time series of all variate are also stacked in a prompt % of all variates in a prompt
The matrix of the MTS is also verbalized in a prompt 
as the textual modality. %By manipulating token embeddings,
By integrating token embeddings, both %of these %works propose to
methods fine-tune some layers of the LMMs with some synthetic data.

Moreover, zero-shot and in-context learning performance of several commercial LMMs have been evaluated for audio classification \cite{dixit2024vision}, anomaly detection \cite{zhuang2024see}, and some synthetic tasks \cite{daswani2024plots}, where the image %({\em e.g.}, spectrograms, Line Plots)
and textual representations of a query %UTS or MTS
time series are integrated into a prompt. For in-context learning, these methods inject the images of a few example time series and their labels ({\em e.g.}, classes) %({\em e.g.}, classes, normal status)
into an instruction to prompt LMMs for assisting the prediction of the query time series.

\subsection{Task-Specific Heads}\label{sec.task}

%With the image embedding of a time series, the next step is to produce its prediction.
For classification tasks, most of the methods in Table \ref{tab.taxonomy} adopt a fully connected (FC) layer or multilayer perceptron (MLP) to transform an embedding into a probability distribution over all classes. For forecasting tasks, there are two approaches: (1) using a $d_{e}\times W$ MLP/FC layer to directly predict (from the $d_{e}$-dimensional embedding) the time series values in a future time window of size $W$ \cite{li2020forecasting,semenoglou2023image}; (2) predicting the pixel values that represent the future part of the time series and then recovering the time series from the predicted image \cite{yang2023your,chen2024visionts,yang2024vitime} ($\S$\ref{sec.processing} discusses the recovery methods). Imputation and generation tasks resemble forecasting %in the sense of predicting
as they also predict time series values. Thus approach (2) has been used for imputation \cite{wang2015imaging} and generation \cite{naiman2024utilizing,karami2024timehr}. %LMMs have been used for classification, text generation, and anomaly detection. For these tasks,
When using LMMs for classification, text generation, and anomaly detection, most of the methods prompt LMMs to produce the desired outputs in textual answers, circumventing task-specific heads \cite{zhang2023insight,dixit2024vision,zhuang2024see}.

%Forecasting: MLP, FC to predict numerical values using embeddings. Imputation of images (TSH). Classification: MLP, FC using embeddings.

\section{Pre-Processing and Post-Processing}\label{sec.processing}

To be successful in using vision models, some subtle design desiderata %to be considered
include {\bf time series normalization}, {\bf image alignment} and {\bf time series recovery}.

\vspace{0.2cm}

\noindent{\bf Time Series Normalization.} Vision models are usually trained on %images after Gaussian normalization (GN).
standardized images. To be aligned, the images introduced in $\S$\ref{sec.tsimage} should be normalized with a controlled mean and standard deviation, as did by \cite{gong2021ast} on spectrograms. In particular, as Heatmap is built on raw time series values, the commonly used Instance Normalization (IN) \cite{kim2022reversible} can be applied on the time series as suggested by VisionTS \cite{chen2024visionts} since IN share similar merits as Standardization. %although min-max normalization was used by \cite{karami2024timehr,zeng2021deep}.
Using Line Plot requires a proper range of y-axis. In addition to rescaling time series %by min-max or GN
\cite{zhuang2024see}, ViTST \cite{li2023time} introduced several methods to remove extreme values from the plot. GAF requires min-max normalization on its input, as it transforms time series values withtin $[0, 1]$ to polar coordinates ({\em i.e.}, arccos). In contrast, input to RP is usually normalization-free as an $\ell_{2}$ norm is involved in Eq.~\eqref{eq.rp} before thresholding.%for a comparison with a threshold.

\vspace{0.2cm}

\noindent{\bf Image Alignment.} When using pre-trained models, it is imperative to fit the image size to the input requirement of the models. This is especially true for Transformer based models as they use a fixed number of positional embeddings to encode the spacial information of image patches. For 3-channel RGB images such as Line Plot, it is straightforward to meet a pre-defined size by adjusting the resolution when producing the image. For images built upon matrices such as Heatmap, Spectrogram, GAF, RP, the number of channels and matrix size need adjustment. For the channels, one method is to duplicate a matrix to 3 channels \cite{chen2024visionts}, another way is to average the weights of the 3-channel patch embedding layer into a 1-channel layer \cite{gong2021ast}. For the image size, bilinear interpolation is a common method to resize input images \cite{chen2024visionts}. Alternatively, AST \cite{gong2021ast} %use cut and bilinear interpolation on
resizes the positional embeddings instead of the images to fit the model to a desired input size. However, the interpolation in these methods may either alter the time series or the spacial information in positional embeddings.

% single-channel (UTS), RGB channel (UTS), duplicate channels (UTS), multi-channel (MTS).

%Bilinear interpolation.

%Correlated variates are better to be spatially close to each other.

%\subsection{Pre-training}

\vspace{0.2cm}

\noindent{\bf Time Series Recovery.} As stated in $\S$\ref{sec.task}, tasks such as forecasting, imputation and generation requires predicting time series values. For models that predict pixel values of images, post-processing involves recovering time series from the predicted images. Recovery from Line Plots is tricky, it requires locating pixels that %correspond to
represent time series and mapping them back to the original values. This can be done by manipulating a grid-like Line Plot as introduced in \cite{yang2023your,yang2024vitime}, which has a recovery function. In contrast, recovery from Heatmap is straightforward as it directly stores the predicted time series values \cite{zeng2021deep,chen2024visionts}. Spectrogram is underexplored in these tasks and it remains open on how to recover time series from it. The existing work \cite{zeng2023pixels} uses Spectrogram for forecasting only with an MLP head that directly predicts time series. %predicts time series values.
GAF supports accurate recovery by an inverse mapping from polar coordinates to normalized time series \cite{wang2015imaging}. However, RP lost time series information during thresholding (Eq.~\ref{eq.rp}), thus may not fit recovery-demanded tasks without using an {\em ad-hoc} prediction head.


% Line Plot was regarded as matrices with rows and columns for mapping in \cite{sood2021visual}.


%\section{Tasks and Time Series Recovery}

%\subsection{Task-Specific Head}

% \subsection{Time Series Recovery}



\section{\textsc{ScholaWrite}: A Dataset of Cognitive Process of Scholarly Writing}



% \begin{table}[t] \centering
% \begin{minipage}[t]{0.99\linewidth}
%     \resizebox{\textwidth}{!}{
%     \large
%     \begin{tabular}{@{}l|c|c|c|c|c @{}} 
%      \toprule
%      Project & 1 & 2 & 3 & 4 & 5\\
%      \midrule \midrule
%     \# Authors  & 1 (3) & 1 (4) & 1 (3) & 1 (4) & 9 (18)\\
%      \# keystrokes & 14,217 & 5,059 & 6,641 & 8,348 & 27,239 \\
%      \# Words added & 17,387  & 23,835 & 7,779 & 12,448 & 57,511\\
%      \# Words deleted & 11,739 & 15,158 & 2,308 & 7,621 & 25,853\\
%      \bottomrule
%     \end{tabular}
%     }
%     \vspace{-3mm}
%     \caption{Statistics of writing actions per Overleaf project in \textsc{ScholaWrite}. `\# Authors' represents the number of authors who participated in our study (with the total number of authors in the final manuscript). 
%     } \label{table:data-stat}
% \end{minipage}\hfill
% \vspace{5mm}
%     \begin{minipage}[t]{0.99\linewidth}
%     \resizebox{\textwidth}{!}{
%     \large
%     \begin{tabular}{@{}l|ccccc@{}}
%         \toprule
%          & 1 & 2 & 3 & 4 & 5 \\
%         \midrule \midrule
%         Idea Generation & 515 & 130 & 116 & 309 & 3255 \\
%         Idea Organization & 0 & 45 & 25 & 9 & 231 \\
%         Section Planning & 182 & 57 & 111 & 201 & 773 \\
%         \midrule
%         Text Production & 9267 & 2438 & 5109 & 4478 & 14031 \\
%         Object Insertion & 583 & 383 & 62 & 486 & 1300 \\
%         Cross-reference & 141 & 112 & 13 & 292 & 458 \\
%         Citation Integration & 75 & 151 & 69 & 127 & 245 \\
%         Macro Insertion & 16 & 7 & 51 & 29 & 33 \\
%         \midrule 
%         Linguistic Style & 233 & 75 & 42 & 201 & 411 \\
%         Coherence & 422 & 242 & 126 & 193 & 1021 \\
%         Clarity & 1249 & 645 & 721 & 1180 & 3301 \\
%         Scientific Accuracy & 307 & 15 & 2 & 24 & 95 \\
%         Structural & 359 & 506 & 105 & 257 & 1042 \\
%         Fluency & 116 & 90 & 46 & 135 & 476 \\
%         Visual Formatting & 752 & 163 & 43 & 427 & 567 \\
%         % Idea Generation & 88 & 46 & 120 & 315 & 526 \\
%         % Idea Organization & 37 & 13 & 25 & 9 & 0 \\
%         % Section Planning & 19 & 41 & 115 & 237 & 193 \\
%         % \midrule
%         % Text Production & 1281 & 1242 & 5139 & 4576 & 9305 \\
%         % Object Insertion & 281 & 128 & 65 & 550 & 627 \\
%         % Citation Integration & 121 & 64 & 76 & 147 & 83 \\
%         % Cross-reference & 56 & 70 & 14 & 330 & 143 \\
%         % Macro Insertion & 7 & 0 & 59 & 34 & 16 \\
%         % \midrule 
%         % Fluency & 42 & 66 & 48 & 148 & 137 \\
%         % Clarity & 190 & 524 & 726 & 1241 & 1289 \\
%         % Coherence & 76 & 173 & 134 & 200 & 449 \\
%         % Structural & 146 & 400 & 110 & 266 & 413 \\
%         % Scientific Accuracy & 1 & 15 & 2 & 25 & 322 \\
%         % Textual Style & 42 & 34 & 42 & 216 & 238 \\
%         % Visual Style & 53 & 129 & 43 & 445 & 784 \\
%         \bottomrule
%         \end{tabular}
%     }
%     \vspace{-3mm}
%     \caption{Distribution of intention labels annotated across all five Overleaf projects.} \label{table:label_distribution_all}
% \end{minipage}
% \end{table}



\paragraph{Summary}
Our post-processed dataset, \textsc{ScholaWrite}, contains end-to-end writing trajectories with annotated intentions, for five Overleaf projects whose final product turned into arXiv preprints. 
The dataset consists of 61,504 arrays of keystroke changes with fine-grained annotations of writing intentions. 
% We publicly release our post-processed data in Huggingface data cards.\footnote{\url{https://huggingface.co/datasets/minnesotanlp/scholawrite}} 
Appendix Table \ref{table:data-stat} shows the overall statistics of the dataset.

% \begin{table}[ht!]
% \centering
% \footnotesize
% \begin{tabular}{@{}c|c|c|c|c|c @{}} 
%  \toprule
%  Project & 1 & 2 & 3 & 4 & 5\\
%  \midrule \midrule
%  \# Authors (participants) & 1 & 1 & 1 & 1 & 9\\
%  \# Authors (manuscript) & 3 & 4 & 3 & 4 & 18\\
%  \# Writing Activities & 13419 & 5002 & 6589 & 7942 & 26784 \\
%  \# Words added & 18394 & 24779 & 7724 & 14601 & 62140\\
%  \# Words deleted & 14056 & 17194 & 3061 & 10163 & 35125\\
%  % Final word count &  & 12308 & 6672 & 9984 & 32286\\
%  \bottomrule
% \end{tabular}
% % \begin{tabular}{@{}c|c|c|c|c|c @{}} 
% %  \toprule
% %  Project & 1 & 2 & 3 & 4 & 5\\
% %  \midrule \midrule
% %  \# Authors & 3 (1) & 4 (1) & 3 (1) & 4 (1) & 18 (9)\\
% %  \# Writing Activities & 13419 & 5002 & 6589 & 7942 & 26784 \\
% %  \# Words added & 18394 & 24779 & 7724 & 14601 & 62140\\
% %  \# Words deleted & 14056 & 17194 & 3061 & 10163 & 35125\\
% %  % Final word count &  & 12308 & 6672 & 9984 & 32286\\
% %  \bottomrule
% % \end{tabular}
% \caption{Statistics of writing actions per Overleaf project in \textsc{ScholaWrite}. The rows `\# Authors (participants)' and `\# Authors (manuscript)' represent the number of authors who participated in our research and the total number of authors present in the final manuscript, respectively.
% Note that the authors who participated in projects 1,2,3, and 4 are also the authors of project 5.}
% \label{table:data-stat}
% \end{table}





% \begin{figure*}[ht]
%     \centering
%     \begin{subfigure}[b]{0.4\textwidth}
%         \centering
%         \includegraphics[width=0.9\columnwidth,trim={0 0cm 0 0},clip]{figures/sankey/newplot5.png}
%         \caption{Sankey diagram of writing intention flow}
%         \label{fig:sankey-proj2}
%     \end{subfigure}
%     \vspace{2mm}
%     \begin{subfigure}[b]{0.43\textwidth}
%         \centering
%         \includegraphics[width=0.9\columnwidth,trim={0 0cm 0 0},clip]{figures/writing_activities/timestamp_proj1_intention.pdf}
%         \caption{Per-intention human writing activities over time}
%          \label{fig:writing-activity-per-intention}
%     \end{subfigure}

% \end{figure*}


% \begin{figure}[ht!]
%     \centering
%     % \makebox[\textwidth]
%     {\includegraphics[width=0.9\columnwidth,trim={0 1cm 0 0},clip]{figures/writing_activities/timestamp_proj1_intention.pdf}}
%     \caption{Per-intention human writing activities over time in the \textsc{ScholaWrite} dataset.\vspace{-4mm}}
%     \label{fig:writing-activity-per-intention}
% \end{figure}

% \subsection{Writing Pattern Analyses}
\paragraph{Writing Intention Distributions} The writing intention distributions is illustrated in Figure \ref{fig:intention-dist}. During the planning stage, idea generation is the most frequent intention.  Text production dominates the implementation stage, and clarity becomes the primary focus during the revision stage. Since we defined the boundary between implementation and revision as the point where meaning changes occur, the high frequency of text production suggests that authors often continue revising their texts by introducing new topics into the documents.  Appendix Table \ref{table:label_distribution_all} shows the label distribution per project. 

\begin{table}[t!]
\footnotesize
\centering
\begin{tabular}{@{}lp{1pt}rr@{}}
\toprule
Label & & Subsequent label & Probability  \\
\midrule
\colorbox{planningcolor}{Idea Generation} & $\rightarrow$ & \colorbox{implementationcolor}{Text Production} & 0.52 \\
\colorbox{planningcolor}{Idea Organization} & $\rightarrow$ & \colorbox{planningcolor}{Idea Generation} & 0.34  \\
\midrule
\colorbox{implementationcolor}{Citation Integration} & $\rightarrow$ & \colorbox{implementationcolor}{Text Production} & 0.37 \\
\colorbox{implementationcolor}{Cross-reference} & $\rightarrow$ & \colorbox{implementationcolor}{Text Production} & 0.36  \\
\midrule
\colorbox{revisioncolor}{Clarity} & $\rightarrow$ & \colorbox{implementationcolor}{Text Production}  & 0.35  \\
\colorbox{revisioncolor}{Coherence} & $\rightarrow$ & \colorbox{implementationcolor}{Text Production}  & 0.34  \\
\colorbox{revisioncolor}{Scientific Accuracy} & $\rightarrow$ & \colorbox{implementationcolor}{Text Production}  & 0.34 \\
\bottomrule
\end{tabular}
\caption{Top-2 Probability of inter-connections between writing intentions. For example, in 34\% of instances where an author engaged in ``Idea Organization,'' the subsequent intention was ``Idea Generation.'' See Appendix Table \ref{table:flow-intention-full} for full description.\vspace{-2mm}}
\label{table:flow-intention}
\end{table}

% \begin{figure}[ht!]
%     \centering
%     \includegraphics[width=0.9\linewidth]{latex/figures/overall_label_distribution.pdf}
%     \vspace{-4mm}
%     \caption{Aggregated log-scaled distribution of writing intentions across the entire dataset. Color indicates high-level process - \colorbox{planningcolor}{planning}, \colorbox{implementationcolor}{implementation}, and \colorbox{revisioncolor}{revision}. Authors predominantly introduce new content during writing sessions.}
%     \label{fig:intention-dist}
% \end{figure}


\paragraph{Flows Between Writing Intentions} 

We analyze the flow of writing intentions most likely to follow each other throughout the writing process, as presented in Table \ref{table:flow-intention}. Many intentions proceed to text production, where the idea generation has the highest probability of feeding into text production. Figure \ref{fig:sankey-proj2} also shows a similar pattern of flows between text productions and other intentions. Also, in Table \ref{table:flow-intention} the reflexive relationship between text production and clarity suggests a constant loop of producing new texts and refining them, ensuring the precision and reasonability of their claims. Please see Appendix Figure \ref{fig:writing-sankey-all} for the intention flows of each of the five projects.



% \vspace{-3mm}
\begin{figure*}[ht]
    \centering
    \begin{subfigure}[b]{0.4\textwidth}
        \centering
        \includegraphics[width=0.9\columnwidth,trim={0 0cm 0 0},clip]{figures/sankey/project2.pdf}
        \caption{Sankey diagram of writing intention flow}
        \label{fig:sankey-proj2}
    \end{subfigure}
    \vspace{2mm}
    \begin{subfigure}[b]{0.43\textwidth}
        \centering
        \includegraphics[width=0.9\columnwidth,trim={0 0cm 0 0},clip]{figures/writing_activities/timestamp_proj1_intention.pdf}
        \caption{Per-intention human writing activities over time}
         \label{fig:writing-activity-per-intention}
    \end{subfigure}
    
    \begin{subfigure}[b]{0.34\textwidth}
        \centering
        \includegraphics[width=\textwidth,trim={0.1 1.2cm 0 0.1cm},clip]{figures/overall_label_distribution.pdf}
        \vspace{-2em}
        \caption{distribution (log) of intentions}
        \label{fig:intention-dist}
    \end{subfigure}
    \hspace{0.1em}
    \begin{subfigure}[b]{0.30\textwidth}
        \centering
        \includegraphics[width=\textwidth,trim={0.3cm 0 0 0},clip]{figures/dist_to_uni/avg_label_w_dist.pdf}
        \vspace{-2em}
        \caption{distance to uniform distribution}
        \label{fig:avg_dist_uni}
    \end{subfigure}
    \hspace{0.1em}
    \begin{subfigure}[b]{0.33\textwidth}
        \centering
        \includegraphics[width=\textwidth]{figures/project_label_distributions/avg_time_series.pdf}
        \vspace{-2em}
        \caption{label distribution over time}
        \label{fig:avg-time-label}
    \end{subfigure}
    % \begin{subfigure}{0.4\textwidth}
    %     \centering
    %     \raisebox{1.3\height}{% Adjust vertical alignment
    %         \includegraphics[width=\textwidth]{latex/figures/writing_activities/timestamp_proj1_broad.pdf}
    %     }
    %     \caption{High-level writing activities over time}
    %     \label{fig:timestamp-proj1-broad}
    % \end{subfigure}
    % \begin{subfigure}[b]{0.5\textwidth}
    %     \centering
    %     \includegraphics[width=0.9\textwidth]{latex/figures/writing_activities/timestamp_proj1_intention.pdf}
    %     \caption{Per-intention writing activities over time}
    %     \label{fig:timestamp-proj1-intention}
    % \end{subfigure}
    \caption{Overall characteristics of scholarly writing patterns in the \textsc{ScholaWrite} dataset
    %: (a) the distribution of writing intentions; (b) the average Wasserstein distance to the uniform distribution; and (c) The distribution of labels of one project over time, sorted according to their distribution mean.
    }
    \label{fig:sec4-results}
\end{figure*}

% \begin{table}[ht!]
% \footnotesize
% \centering
% \begin{tabular}{@{}lp{1pt}rr@{}}
% \toprule
% Label & & Subsequent label & Probability  \\
% \midrule
% \colorbox{planningcolor}{Idea Generation} & $\rightarrow$ & \colorbox{implementationcolor}{Text Production} & 0.52 \\
% \colorbox{planningcolor}{Idea Organization} & $\rightarrow$ & \colorbox{planningcolor}{Idea Generation} & 0.34  \\
% \colorbox{planningcolor}{Section Planning} & $\rightarrow$ & \colorbox{implementationcolor}{Text Production} & 0.33  \\
% \midrule
% \colorbox{implementationcolor}{Text Production} & $\rightarrow$ & \colorbox{revisioncolor}{Clarity} & 0.20\\
% \colorbox{implementationcolor}{Object Insertion} & $\rightarrow$ & \colorbox{implementationcolor}{Text Production} & 0.32 \\
% \colorbox{implementationcolor}{Citation Integration} & $\rightarrow$ & \colorbox{implementationcolor}{Text Production} & 0.37 \\
% \colorbox{implementationcolor}{Cross-reference} & $\rightarrow$ & \colorbox{implementationcolor}{Text Production} & 0.36  \\
% \colorbox{implementationcolor}{Macro Insertion} & $\rightarrow$ & \colorbox{planningcolor}{Idea Generation} & 0.29 \\
% \midrule
% \colorbox{revisioncolor}{Fluency} & $\rightarrow$ & \colorbox{implementationcolor}{Text Production}  & 0.30 \\
% \colorbox{revisioncolor}{Coherence} & $\rightarrow$ & \colorbox{implementationcolor}{Text Production}  & 0.34  \\
% \colorbox{revisioncolor}{Clarity} & $\rightarrow$ & \colorbox{implementationcolor}{Text Production}  & 0.35  \\
% \colorbox{revisioncolor}{Structural} & $\rightarrow$ & \colorbox{implementationcolor}{Text Production}  & 0.27  \\
% \colorbox{revisioncolor}{Linguistic Style} & $\rightarrow$ & \colorbox{implementationcolor}{Text Production}  & 0.29 \\
% \colorbox{revisioncolor}{Scientific Accuracy} & $\rightarrow$ & \colorbox{implementationcolor}{Text Production}  & 0.34 \\
% \colorbox{revisioncolor}{Visual Formatting} & $\rightarrow$ & \colorbox{implementationcolor}{Text Production} & 0.25 \\
% \bottomrule
% \end{tabular}
% \caption{Probability of inter-connections between writing intentions in \textsc{ScholaWrite}. For example, in 34\% of instances where an author engaged in ``Idea Organization,'' the subsequent intention was ``Idea Generation.'' }
% \label{table:flow-intention-full}
% \end{table}


%%%%%%%%%%%%%%%%%%%%%%%

\paragraph{Time-series Distributions of Writing Intentions} 
% \vspace{-3mm}

We calculated the average Wasserstein distance between each intention distribution and uniform distribution to assess how evenly intentions occur throughout the writing process. As shown in Figure \ref{fig:avg_dist_uni}, text production had the smallest distance, while scientific accuracy had the largest. This suggests that text production is spread throughout the writing, whereas scientific accuracy is concentrated in specific phases. A similar pattern appears in Figure \ref{fig:avg-time-label} , where text production is evenly distributed over time, and scientific accuracy peaks in the middle-to-late of the writing process. 
% Appendix Figures \ref{fig:writing-step-broad-all} and \ref{fig:writing-step-detailed-all} demonstrate a similar pattern that text production of the implementation stage spreads evenly over time. 
Figure \ref{fig:writing-activity-per-intention} demonstrates a similar pattern that text production of the implementation stage spreads evenly over time. 
Types of clarity and structural revisions tend to occur after the medium steps of writing. Please see Appendix Figures \ref{fig:dist-to-uni-all} to \ref{fig:writing-step-detailed-all} for each of the five projects.
% Please refer to Figures \ref{fig:dist-to-uni-all} to \ref{fig:writing-step-detailed-all} in the Appendix for each of the five Overleaf projects. 

% \dk{Add the timestamp figures that Ross made here for one of the projects. Please include two figures at high-level and intention level.}

% \begin{figure}[ht!]
%     \centering
%     \includegraphics[width=0.9\linewidth]{latex/figures/dist_to_uni/avg_w_dist_to_uni.pdf}
%     \vspace{-5mm}
%     \caption{Average distance to uniform}
%     \label{fig:avg_dist_uni}
% \end{figure}








\section{Applications for Writing Assistance}




We envision \textsc{ScholaWrite} as a valuable resource for training language models and improving future writing assistants for scholarly writing. To evaluate its usability, we conducted experiments training LLMs to mimic the complex, non-linear writing processes of human scholars.
% We believe \textsc{ScholaWrite} can serve as a valuable resource for training language models or enhancing the cognitive writing process in future writing assistants for scholarly writing. To evaluate its usability, we performed several experiments where existing LLMs were trained to mimic the complex, non-linear scientific writing processes of human scholars. 
Specifically, we aimed to showcase the capabilities of LLMs trained on \textsc{ScholaWrite} in two scenarios:

\textbf{(1) Predicting an author's next writing intention} (\S\ref{sec:eval:predict}): The task is crucial for writing assistants to accurately assess the writer's current status in context and predict the correct writing intention. This enables them to offer cognitively-appropriate writing suggestions that align with the writer's needs.

\textbf{(2) Iteratively generating scholarly writing actions from scratch (\S\ref{sec:self-writing})} (called Iterative Self-Writing), mirroring the human writing process\label{sec:self-writing-overview}: This task focuses on how well the model trained on our dataset can replicate the actual iterative writing and thinking process of scholars, and whether the generated text achieves higher quality compared to LLM-prompted writing.

% As illustrated in Figure \ref{fig:iterative_model_writing}, the first task (represented by the inner box labeled ``Prediction'') takes the ``before'' text from a keystroke pair (i.e., Listing \ref{table:single-entry}) and context \colorbox{planningcolor}{prompt}, and predicts the writing \colorbox{pink}{intention} to apply for the subsequent actions. Once the intention is predicted, the second task generates the  ``after'' text based on the predicted intention and given \colorbox{BlueGreen}{prompt} (represented by the outer box labeled ``Generation''). This process repeats iteratively until no further changes are made or a maximum number of iterations, such as 100, is reached. 


% We randomly split our dataset into train (80\%) and test (20\%) sets. 
% % To address the budget constraint in GPT4o, 
% For each intention label in the test set, we randomly select up to 300 keystroke entries, due to budget constraints. 
% Note that the same train and test sets are applied to all models across all experiments. See Appendix \ref{sec:appendix:model} for a full description of the model training process. 



% \begin{figure}[t!]
%     \centering\hspace*{-0.2cm}
% \includegraphics[width=0.53\textwidth,trim={1.5cm 0.9cm 0.5cm 0cm},clip]{figures/fig_iter.pdf}
% \vspace{-5mm}
%     \caption{The overview of next writing intention prediction task (Prediction box) and iterative self-writing task setup (the whole pipeline).\vspace{-5mm}}
% \label{fig:iterative_model_writing}
% \end{figure}

\begin{figure}[th!]
    \centering\hspace*{0.2cm}
\includegraphics[width=0.49\textwidth,trim={1.5cm 0.9cm 0.5cm 0cm},clip]{figures/fig_iter.pdf}
\vspace{-8mm}
    \caption{The overview of next writing intention prediction task (Prediction box) and iterative self-writing task setup (the whole pipeline).\vspace{-3mm}}
\label{fig:iterative_model_writing}
\end{figure}

\subsection{Predicting Next Writing Intention}\label{sec:eval:predict}

\paragraph{Setup \& Metrics} 

This task (represented by the inner box ``Prediction'' in Figure \ref{fig:iterative_model_writing}) takes the ``before-text'' from a keystroke pair (i.e., Listing \ref{table:single-entry}) and context \colorbox{planningcolor}{prompt}, and predicts the writing \colorbox{pink}{intention} to apply for the subsequent actions.

We use BERT \cite{devlin2019bert}, RoBERTa \cite{Liu2019RoBERTaAR}, and Llama3.1-8B-Instruct \cite{dubey2024llama} as baselines, fine-tuning each on the \textsc{ScholaWrite} training set. For comparison, we also run GPT-4o \cite{gpt4o} on the test set. To evaluate model performance, we used a weighted F-1 score\footnote{Weighted F-1 was chosen to address skewed label distribution (as shown in Figure \ref{fig:intention-dist} and Table \ref{table:taxonomy-full}).}. See Appendix \ref{sec:appendix:finetuning} for details.

% The fine-tuning prompt included all possible labels with definitions, task instructions, the ``before-text'' chunk, and the corresponding human-annotated intention label, asking the model to predict the intention label based on the ``before-text''. Differences in prompts were limited to only task instructions (see Appendix \ref{sec:appendix:prompt} for prompt details). To evaluate model performance, we used a weighted f-1 score\footnote{Weighted F-1 was chosen to address skewed label distribution (as shown in Figure \ref{fig:intention-dist} and Table \ref{table:taxonomy-full}).}, comparing predictions to gold intention labels on the test set.


\begin{table}[h!]
\centering
\resizebox{\columnwidth}{!}{%
\begin{tabular}{@{}lccccc@{}}
\toprule
 & BERT & RoBERTa & Ll ama-8B & GPT-4o \\ \midrule
Base & 0.04 & 0.02 &  0.12 & 0.08 \\
+ SW & \textbf{0.64} & \textbf{0.64} & \textbf{0.13} & - \\ \bottomrule
\end{tabular}%
}
\caption{Weighted F-1 scores of each baseline and its corresponding fine-tuned model with \textsc{ScholaWrite} (+SW) for the writing intention prediction task. }
\label{table:intention-prediction}
\end{table}


\begin{figure*}[t!]
    \centering
    \begin{subfigure}[b]{0.32\textwidth}
        \centering
        \includegraphics[width=\textwidth]{figures/lexical_all_no_legend.pdf}
        \vspace{-1em}
        \caption{Lexical Diversity}
        \label{fig:lexical-all}
    \end{subfigure}
    \hspace{-0.5em}
    \begin{subfigure}[b]{0.35\textwidth}
        \centering
        \includegraphics[width=\textwidth]{figures/topic_all_no_legend.pdf}
        \vspace{-1em}
        \caption{Topic Consistency}
        \label{fig:topic-all}
    \end{subfigure}
    \begin{subfigure}[b]{0.32\textwidth}
        \centering
        \includegraphics[width=\textwidth]{figures/intention_all_no_legend.pdf}
        \vspace{-1em}
        \caption{Intention Coverage}
        \label{fig:intention-all}
    \end{subfigure}
    \caption{Metric scores of the final writing output of models (\textcolor{magenta}{LLama-8B-SW}, \textcolor{teal}{LLama-8B-Zero}, and \textcolor{blue}{GPT-4o}) after 100 iterations of the iterative self-writing experiment. We observe that our \textcolor{magenta}{Llama-8B-SW} model presents the highest quality of the final output across most of the four seed documents.
    % \dk{add a legend of models in one of the figures above.}
    }
    \label{fig:sec5-auto-all}
\end{figure*}
% \begin{figure*}[ht]
%     \centering
%     \begin{subfigure}[b]{0.35\textwidth}
%         \centering
%         \includegraphics[width=\textwidth]{figures/llama8_SW_output_detailed_seed2.pdf}
%         \vspace{-1.5em}
%         \caption{\textcolor{magenta}{Llama-8B-SW}}
%         \label{fig:llama-sw-timestamp-seed2}
%     \end{subfigure}
%     \hspace{0.5em}
%     \begin{subfigure}[b]{0.25\textwidth}
%         \centering
%         \includegraphics[width=\textwidth,trim={7.2cm 0 0 0},clip]{figures/llama8_meta_output_detailed_seed2.pdf}
%         \vspace{-1.5em}
%         \caption{\textcolor{teal}{Llama-8B-Zero}}
%         \label{fig:llama-zero-timestamp-seed2}
%     \end{subfigure}
%     \hspace{0.5em}
%     \begin{subfigure}[b]{0.25\textwidth}
%         \centering
%         \includegraphics[width=\textwidth,trim={7.2cm 0 0 0},clip]{figures/gpt4o_output_detailed_seed2.pdf}
%         \vspace{-1.5em}
%         \caption{\textcolor{blue}{GPT-4o}}
%         \label{fig:gpt4-timestamp-seed2}
%     \end{subfigure}
%     \caption{Per-intention writing activities over time among different models - (1) \textcolor{magenta}{Llama-8B-SW}; (2) \textcolor{teal}{Llama-8B-Zero}; and (3) \textcolor{blue}{GPT-4o}, from the seed document \cite{du-etal-2022-read}. We observe different writing patterns by model during the entire 100 iterations. 
%     }
%     \label{fig:sec5-timestamp-seed2}
% \end{figure*}

\begin{table*}[h!]
\centering
\footnotesize
% \resizebox{2\columnwidth}{!}{
\begin{tabular}{@{}c@{\hskip 1mm}@{}|p{4.4cm}p{4.4cm}p{4.4cm}@{\hskip 2mm}@{}}
\toprule
\textbf{Iter.} & \textbf{\textcolor{magenta}{Llama-8B-SW}} & \textbf{\textcolor{teal}{Llama-8B-Zero}} & \textbf{\textcolor{blue}{GPT-4o}}\\
\midrule
10 & [\textit{..Editing Abstract..}] but rather should be used to improve the flow of information to avoid information\textcolor{red}{\sout{,}}\textcolor{teal}{-} over\textcolor{red}{\sout{load}}\textcolor{teal}{-claiming} (\textbf{Text Production}) & [\textit{..Generating Experiment Section}] ..with an average acceptance rate of \textcolor{teal}{87.5\% (standard deviation: 3.2\%),..} ... with a reduction of 3\textcolor{red}{\sout{0}}\textcolor{teal}{5}\% in revision time and 2\textcolor{red}{\sout{5}}\textcolor{teal}{8}\% in human effort... (\textbf{Scientific Accuracy}) & [..\textit{Same as the 9th iteration}] \textbackslash usepackage\{booktabs\}

\textbackslash usepackage\{array\}

\textcolor{teal}{\textbackslash usepackage\{hyperref\}}...

\textcolor{red}{\sout{\textbackslash method}} \textcolor{teal}{\textsc{$\mathcal{R}3$}}...

\textcolor{red}{\sout{\textbackslash method}} \textcolor{teal}{\textsc{$\mathcal{R}3$}}...

(\textbf{Cross-reference})\\
\hline
25 & [\textit{..Editing Abstract..}] but rather should be used to improve the flow of information to avoid information overload\textcolor{red}{\sout{,}}\textcolor{teal}{.} (\textbf{Text Production}) & [\textit{..Editing Table}]
Acceptance rate (\%) \& 75 \& 8\textcolor{red}{\sout{7.5}}\textcolor{teal}{8.2} //
Revision time (minutes) \& 45 \& 2\textcolor{red}{\sout{9}}\textcolor{teal}{8.5}
Human effort (minutes) \& 60 \& 4\textcolor{red}{\sout{3}}\textcolor{teal}{2} 
... (\textbf{Scientific Accuracy}) & [..\textit{Same as the 24th iteration}] The \textcolor{red}{\sout{efficiency}} \textcolor{teal}{...consider the structural flowchart in Figure \textbackslash ref\{fig:system-architecture\}, which outlines...}  

[\textit{Inserting the figure}]

\textcolor{teal}{\textbackslash label\{fig:system-architecture\}} (\textbf{Object Insertion})\\
\hline 
51 & [\textit{..Editing Abstract..}] but rather should be used to improve the flow of information, offering \textcolor{red}{\sout{teach}}\textcolor{teal}{previously trained} to \textcolor{teal}{a} load more related information over the\textcolor{teal}{-} load. (\textbf{Clarity}) & \textcolor{teal}{\textbackslash section\{Impact of the Proposed System\}
The proposed system, $\mathcal{R}3$, has the potential to impact the writing process in several ways.}... \textcolor{teal}{\textbackslash section\{Future Research Direction\}}... (\textbf{Structural}) & \textbackslash bibitem\{jones2020one\_shot\}
Jones, L., \textcolor{teal}{\textbf{\textbackslash}}\& Green, D. (2020). 

\textbackslash bibitem\{brown2021collaboration\}
Brown, E., \textcolor{teal}{\textbf{\textbackslash}}\& Davis, M. (2021). 

\textbackslash bibitem\{garcia2021revision\_metrics\}
Garcia, I., \textcolor{teal}{\textbf{\textbackslash}}\& Lopez, R. (2021). (\textbf{Object Insertion})
% \textcolor{red}{\sout{\textbackslash cite\{zhang2022iterative\_revisions\}. Refer to Table \textbackslash ref\{tab:acceptance-rate\} for a detailed comparison of acceptance rates}}\textcolor{teal}{, as detailed in Table \textbackslash ref\{tab:acceptance-rate\} \textbackslash cite\{zhang2022iterative\_revisions\}}. [..\textit{Same as 24th iteration}] \textbackslash caption\{The system architecture of \textsc{$\mathcal{R}3$}, outlining the iteration process from user interactions to text finalization, \textcolor{teal}{as elaborated in Section \textbackslash ref{sec:system}}.\} (\textbf{Cross-reference})
\\
\hline
100 & [..\textit{Same as the 99th iteration}] \textcolor{teal}{\textbackslash end\{document\}} (\textbf{Macro Insertion}) & [..\textit{Same as the 99th }] \textcolor{teal}{\textbackslash usepackage[margin=1in]\{geometry\} \texttt{\% Customizes page margins}} \textcolor{teal}{\textbackslash usepackage\{hyperref\} \texttt{\% Enables hyperlinks}} (\textbf{Fluency}) & [..\textit{Same as the 93th iteration}]
\textbackslash bibliography\{references\}

\textbackslash bibliographystyle\{plain\}

\textcolor{red}{\sout{\% References}}\textbackslash begin\{thebibliography\}\{\}
(\textbf{Cross-reference}) \\ 
\bottomrule
\end{tabular}
\caption{Example model outputs at different iterations, from the seed document \cite{du-etal-2022-read} (Listing \ref{table:seed-entry-read}). 
\label{table:qual-comparison}}
\end{table*}

\paragraph{Results} Table \ref{table:intention-prediction} presents the weighted F1 scores for predicting writing intentions across baselines and fine-tuned models. Regardless of the intricate nature of the task itself\footnote{Each model predicts the next intention using only the "before" text, while human annotators consider multiple "before and after" edits. Moreover, the chosen next intention is not necessarily the only correct one.}, all models finetuned on \textsc{ScholaWrite} show an improved performance compared to their baselines. BERT and RoBERTa achieved the most improvement, while LLama-8B-Instruct showed a modest improvement after fine-tuning. As detailed in Appendix \ref{sec:appendix:model}, more training epochs and encoder-decoder architectures of BERT variants are assumed to be the reason for significant improvement compared to LLMs. This aligns with findings from \citet{grasso2024assessing, yu2023openclosedsmalllanguage}, which shows that RoBERTa and BERT can often match or even outperform LLMs for the text classification tasks. Those results demonstrate the effectiveness of our \textsc{ScholaWrite} dataset to align language models with writers' intentions. 
% \minhwa{further mention limitation of low llama performance in limitation}

% \begin{table}[ht!]
% \footnotesize
% \centering
% \begin{tabular}{ccc}
% \toprule
% & \begin{tabular}[c]{@{}l@{}} \textsc{ScholaWrite-}\\ \textsc{Llama-3.2}\\ \end{tabular} & GPT-4o \\
% \midrule
% Avg. & 0.23 & 0.16 \\
% Macro Avg. & 0.04 & 0.07 \\
% Weighted Avg. & \textbf{0.27} & 0.21 \\
% \bottomrule
% \end{tabular}
% \caption{Average F-1 scores on the intention prediction task. We observe our fine-tuned model \textsc{ScholaWrite-Llama-3.2} perform better than GPT-4o. }
% \label{table:prediction-result-comparison}
% \end{table}





% According to Table \ref{table:prediction-result-comparison}, we observe that the overall performance is very low in both models. These low F1 scores suggest that LLMs face challenges in fully grasping the complexities of scholarly scientific writing. However, our fine-tuned model \textsc{ScholaWrite-Llama-3.2} comparatively outperformed GPT-4o, despite not providing explicit definitions for each label in the Llama3 model. Remarkably, even this smaller 1B-parameter model performed better than the much larger GPT-4. This result demonstrates the value of our dataset in helping LLMs better understand the human reasoning process involved in scientific writing. In Figure \ref{fig:f1-by-label}, 
% \begin{figure}[ht!]
%     \centering
%     \includegraphics[width=\linewidth]{latex/figures/llama_vs_gpt.pdf}
%     \caption{Per-label average log F1 scores on the prediction task of next writing intention.}
%     \label{fig:f1-by-label}
% \end{figure}



% \begin{tabular}{lrl}
% \toprule
%  & Llama 3.2 f1& GPT-4o f1 \\
% \midrule
% Citation Integration & 0.00 & 0.08 \\
% Clarity & 0.01 & 0.10 \\
% Coherence & 0.00 & 0.05 \\
% Cross-reference & 0.01 & 0.07 \\
% Fluency & 0.02 & 0.02 \\
% Idea Generation & 0.05 & 0.04 \\
% Idea Organization & 0.00 & 0.01 \\
% Linguistic Style & 0.00 & 0.00 \\
% Macro Insertion & 0.03 & 0.19 \\
% Object Insertion & 0.09 & 0.10 \\
% Scientific Accuracy & 0.00 & 0.03 \\
% Section Planning & 0.04 & 0.06 \\
% Structural & 0.01 & 0.00 \\
% Text Production & 0.45 & 0.31 \\
% Textual Style & 0.00 & 0.00 \\
% Visual Formatting & 0.00 & 0.13 \\
% \hline
% \hline
% accuracy & 0.23 & 0.16 \\
% macro avg & 0.04 & 0.07 \\
% weighted avg & 0.27 & 0.21 \\
% \bottomrule
% \end{tabular}

%\begin{table*}[ht!]
%\begin{tabular}{lrrrr}\n
%\toprule\n & precision & recall & f1-score\\
%\midrule
%Citation Integration & 0.00 & 0.00 & 0.00\\
%Clarity & 0.15 & 0.01 & 0.01\\
%Coherence & 0.02 & 0.00 & 0.00\\
%Cross-reference & 0.01 & 0.02 & 0.01\\
%Fluency & 0.02 & 0.02 & 0.02\\
%Idea Generation & 0.05 & 0.05 & 0.05\\
%Idea Organization & 0.00 & 0.00 & 0.00\\
%Linguistic Style & 0.00 & 0.00 & 0.00\\
%Macro Insertion & 0.02 & 0.04 & 0.03\\
%Object Insertion & 0.06 & 0.30 & 0.09\\
%Scientific Accuracy & 0.00 & 0.00 & 0.00\\
%Section Planning & 0.03 & 0.07 & 0.04\\
%Structural & 0.04 & 0.01 & 0.01\\
%Text Production & 0.57 & 0.38 & 0.45\\
%% Textual Style & 0.00 & 0.00 & 0.00\\
%Visual Formatting & 0.00 & 0.00 & 0.00\\
%% nan & 0.00 & 0.00 & 0.00\\
%accuracy & 0.23 & 0.23 & 0.23 \\
%macro avg & 0.06 & 0.05 & 0.04\\
%weighted avg & 0.35 & 0.23 & 0.27\\
%\bottomrule
%\end{tabular}
%\caption{Llama 3.2 1B Intention Classification Results}
%\label{tab:intention_classification_res}
%\end{table*}


% Out of 12558 quries, there are 10 invalid response from GPT-4o. Invalid response means the output is not including any of labels mentioned in the prompt. Here is the result of GPT-4o

% \begin{table*}[ht!]
% \begin{tabular}{lrrrr}\n
% \toprule\n & precision & recall & f1-score\\
% \midrule
% Citation Integration & 0.05 & 0.37 & 0.08\\
% Clarity & 0.15 & 0.08 & 0.10\\
% Coherence & 0.03 & 0.10 & 0.05\\
% Cross-reference & 0.04 & 0.46 & 0.07\\
% Fluency & 0.01 & 0.03 & 0.02\\
% Idea Generation & 0.11 & 0.02 & 0.04\\
% Idea Organization & 0.01 & 0.03 & 0.02\\
% Linguistic Style & 0.00 & 0.00 & 0.00\\
% Macro Insertion & 0.11 & 0.62 & 0.19\\
% Object Insertion & 0.09 & 0.12 & 0.10\\
% Scientific Accuracy & 0.02 & 0.10 & 0.03\\
% Section Planning & 0.03 & 0.20 & 0.06\\
% Structural & 0.00 & 0.00 & 0.00\\
% Text Production & 0.67 & 0.20 & 0.31\\
% Visual Formatting & 0.15 & 0.11 & 0.13\\
% accuracy &&& 0.16 \\
% macro avg & 0.09 & 0.15 & 0.07\\
% weighted avg & 0.42 & 0.16 & 0.21\\
% \bottomrule
% \end{tabular}
% \caption{GPT4o Intention Classification Results}
% \label{tab:intention_classification_res}
% \end{table*}

% \tikzset{
%   hatch/.style={pattern=horizontal lines, pattern color=#1},
%   hatch/.default=black
% }



\subsection{Iterative Self-Writing}\label{sec:self-writing}



\paragraph{Setups}

During iterative self-writing (Figure \ref{fig:iterative_model_writing}), a fine-tuned model processes LaTeX-formatted seed document (as ``before-text'') with a context \colorbox{planningcolor}{prompt} to predict the next \colorbox{pink}{intention}, then revises the text (``after-text'') accordingly given \colorbox{BlueGreen}{prompt}. The revised document then serves as the new seed for the next iteration. This process repeats until a set iteration limit (e.g., 100) is reached. All models use the same train (80\%)-test (20\%) split across experiments. See Appendix \ref{sec:appendix:model} and Figure \ref{fig:iteartive-writing-setup-detail} for training details. 

We fine-tune Llama3.1-8B-Instruct (\textcolor{magenta}{Llama-8B-SW}) and compare it to vanilla Llama-8B-Instruct (\textcolor{teal}{Llama-8B-Zero}) and \textcolor{blue}{GPT-4o}. Also, seed documents were derived from LaTeX-formatted abstracts of four award-winning NLP papers on diverse topics \cite{zeng-etal-2024-johnny, lu-etal-2024-semisupervised, du-etal-2022-read, etxaniz-etal-2024-latxa} (Appendix \ref{table:seed-entry-johnny}-\ref{table:seed-entry-latxa}).

% Iterative self-writing involves two subtasks (Figure \ref{fig:iterative_model_writing}): (1) next intention prediction and (2) ``after-text'' generation. Similar to Section \ref{sec:eval:predict}, the first task (the box ``Prediction'') uses a fine-tuned language model to process a LaTeX-formatted seed document as ``before-text'' along with a context \colorbox{planningcolor}{prompt}. The model then predicts the next writing \colorbox{pink}{intention} to guide subsequent revisions. Once an intention is predicted, the second task (the box ``Generation'') generates revisions of the seed document as ``after-text'' based on the predicted intention and given \colorbox{BlueGreen}{prompt}. The revised document then serves as the new seed for the next iteration. This iterative process continues until no further changes occur or a set iteration limit (e.g., 100) is reached.

% For training, we randomly split the \textsc{ScholaWrite} dataset into training (80\%) and testing (20\%) sets. From each intention label in the test set, we sample up to 300 keystroke entries due to budget constraints. All models use the same train-test split across experiments (See Appendix \ref{sec:appendix:model}; Figure \ref{fig:iteartive-writing-setup-detail} for further training details).

% For intention prediction, we fine-tune Llama3.1-8B-Instruct on \textsc{ScholaWrite} training set (\textsc{Llama-8B-SW-pred}'') and compare it to baseline models (Llama-8B-Instruct and GPT-4o) from Section \ref{sec:eval:predict}. For ``after-text'' generation, we fine-tune another Llama3.1-8B-Instruct model (``\textsc{Llama-8B-SW-gen}'') using the same dataset, with Llama3.1-8B-Instruct and GPT-4o as baselines. The fine-tuning prompt includes task instructions, a verbalizer from human-annotated labels, and ``before-text.'' While prompts were standardized, task instructions varied by model (See Appendix \ref{sec:appendix:prompt} for the prompt templates).


% Seed documents were derived from LaTeX-formatted abstracts of four award-winning NLP papers on diverse topics \cite{zeng-etal-2024-johnny, lu-etal-2024-semisupervised, du-etal-2022-read, etxaniz-etal-2024-latxa} (Appendix \ref{table:seed-entry-johnny} to \ref{table:seed-entry-latxa}).

% Also, due to budget constraints, models had different revision strategies. \textsc{Llama-8B-SW-*} and \textsc{Llama-8B-Instruct} continued revision until the next predicted intention changed. \textcolor{blue}{GPT-4o}, however, moved to the next iteration regardless. We refer to the fine-tuned Llama-8B model as \textcolor{magenta}{Llama-8B-ScholaWrite} (or \textcolor{magenta}{Llama-8B-SW}) and the vanilla model as \textcolor{teal}{Llama-8B-Zero}.



%%%%%%%%%%%%%%%%%%%%%%%%%%%


% Starting with a seed document, the classification model predicts the next writing intention, and the generation model revises the document based on that prediction and the ``before'' text. The revised document then serves as the new seed for the next iteration. This process repeats \textit{100 times}.


% \begin{table*}[h!]
% \centering
% \footnotesize
% % \resizebox{2\columnwidth}{!}{
% \begin{tabular}{@{}c@{\hskip 1mm}@{}|p{4.4cm}p{4.4cm}p{4.4cm}@{\hskip 2mm}@{}}
% \toprule
% \textbf{Iter.} & \textbf{\textcolor{magenta}{Llama-8B-SW}} & \textbf{\textcolor{teal}{Llama-8B-Zero}} & \textbf{\textcolor{blue}{GPT-4o}}\\
% \midrule
% 10 & [\textit{..Editing Abstract..}] but rather should be used to improve the flow of information to avoid information\textcolor{red}{\sout{,}}\textcolor{teal}{-} over\textcolor{red}{\sout{load}}\textcolor{teal}{-claiming} (\textbf{Text Production}) & [\textit{..Generating Experiment Section}] ..with an average acceptance rate of \textcolor{teal}{87.5\% (standard deviation: 3.2\%),..} ... with a reduction of 3\textcolor{red}{\sout{0}}\textcolor{teal}{5}\% in revision time and 2\textcolor{red}{\sout{5}}\textcolor{teal}{8}\% in human effort... (\textbf{Scientific Accuracy}) & [..\textit{Same as the 9th iteration}] \textbackslash usepackage\{booktabs\}

% \textbackslash usepackage\{array\}

% \textcolor{teal}{\textbackslash usepackage\{hyperref\}}...

% \textcolor{red}{\sout{\textbackslash method}} \textcolor{teal}{\textsc{$\mathcal{R}3$}}...

% \textcolor{red}{\sout{\textbackslash method}} \textcolor{teal}{\textsc{$\mathcal{R}3$}}...

% (\textbf{Cross-reference})\\
% \hline
% 25 & [\textit{..Editing Abstract..}] but rather should be used to improve the flow of information to avoid information overload\textcolor{red}{\sout{,}}\textcolor{teal}{.} (\textbf{Text Production}) & [\textit{..Editing Table}]
% Acceptance rate (\%) \& 75 \& 8\textcolor{red}{\sout{7.5}}\textcolor{teal}{8.2} //
% Revision time (minutes) \& 45 \& 2\textcolor{red}{\sout{9}}\textcolor{teal}{8.5}
% Human effort (minutes) \& 60 \& 4\textcolor{red}{\sout{3}}\textcolor{teal}{2} 
% ... (\textbf{Scientific Accuracy}) & [..\textit{Same as the 24th iteration}] The \textcolor{red}{\sout{efficiency}} \textcolor{teal}{...consider the structural flowchart in Figure \textbackslash ref\{fig:system-architecture\}, which outlines...}  

% [\textit{Inserting the figure}]

% \textcolor{teal}{\textbackslash label\{fig:system-architecture\}} (\textbf{Object Insertion})\\
% \hline 
% 51 & [\textit{..Editing Abstract..}] but rather should be used to improve the flow of information, offering \textcolor{red}{\sout{teach}}\textcolor{teal}{previously trained} to \textcolor{teal}{a} load more related information over the\textcolor{teal}{-} load. (\textbf{Clarity}) & \textcolor{teal}{\textbackslash section\{Impact of the Proposed System\}
% The proposed system, $\mathcal{R}3$, has the potential to impact the writing process in several ways.}... \textcolor{teal}{\textbackslash section\{Future Research Direction\}}... (\textbf{Structural}) & \textbackslash bibitem\{jones2020one\_shot\}
% Jones, L., \textcolor{teal}{\textbf{\textbackslash}}\& Green, D. (2020). 

% \textbackslash bibitem\{brown2021collaboration\}
% Brown, E., \textcolor{teal}{\textbf{\textbackslash}}\& Davis, M. (2021). 

% \textbackslash bibitem\{garcia2021revision\_metrics\}
% Garcia, I., \textcolor{teal}{\textbf{\textbackslash}}\& Lopez, R. (2021). (\textbf{Object Insertion})
% % \textcolor{red}{\sout{\textbackslash cite\{zhang2022iterative\_revisions\}. Refer to Table \textbackslash ref\{tab:acceptance-rate\} for a detailed comparison of acceptance rates}}\textcolor{teal}{, as detailed in Table \textbackslash ref\{tab:acceptance-rate\} \textbackslash cite\{zhang2022iterative\_revisions\}}. [..\textit{Same as 24th iteration}] \textbackslash caption\{The system architecture of \textsc{$\mathcal{R}3$}, outlining the iteration process from user interactions to text finalization, \textcolor{teal}{as elaborated in Section \textbackslash ref{sec:system}}.\} (\textbf{Cross-reference})
% \\
% \hline
% 100 & [..\textit{Same as the 99th iteration}] \textcolor{teal}{\textbackslash end\{document\}} (\textbf{Macro Insertion}) & [..\textit{Same as the 99th }] \textcolor{teal}{\textbackslash usepackage[margin=1in]\{geometry\} \texttt{\% Customizes page margins}} \textcolor{teal}{\textbackslash usepackage\{hyperref\} \texttt{\% Enables hyperlinks}} (\textbf{Fluency}) & [..\textit{Same as the 93th iteration}]
% \textbackslash bibliography\{references\}

% \textbackslash bibliographystyle\{plain\}

% \textcolor{red}{\sout{\% References}}

% \textbackslash begin\{thebibliography\}\{\}
% (\textbf{Cross-reference}) \\ 
% \bottomrule
% \end{tabular}
% \caption{Example model outputs at different iterations, from the seed document \cite{du-etal-2022-read} (Listing \ref{table:seed-entry-read}). 
% \label{table:qual-comparison}}
% \end{table*}


% We have different modifications between models based on this setup, due to budget constraints. \textsc{Llama-8B-SW-*} and \textsc{Llama-8B-Instruct} keep revising the document within one iteration until the next intention differs from the current one. \textcolor{blue}{GPT-4o} proceeds to the next iteration regardless of whether the predicted intention changes. We name the fine-tuned Llama-8B model as \textcolor{magenta}{Llama-8B-ScholaWrite}\footnote{We interchangeably use \textcolor{magenta}{Llama-8B-ScholaWrite} and \textcolor{magenta}{Llama-8B-SW}.} and the vanilla Llama-8B-Instruct model as \textcolor{teal}{Llama-8B-Zero}.


%We have different between models based on this setup due to budget constraints. \textsc{Llama-8B-SW-*} and \textsc{Llama-8B-instruct} keep revising the document within one iteration until the next intention differs from the current one. \textcolor{blue}{GPT-4o} proceeds to the next iteration regardless of whether the predicted intention changes. We name the fine-tuned Llama-8B model as \textcolor{magenta}{Llama-8B-ScholaWrite}\footnote{We interchangeably use \textcolor{magenta}{Llama-8B-ScholaWrite} and \textcolor{magenta}{Llama-8B-SW}.} and the vanilla Llama-8B-Instruct model as \textcolor{teal}{Llama-8B-Zero}.

% \dk{show an example of seed text and add the rest of seed papers in appendix.}
\paragraph{Metrics}

We evaluated \textit{lexical diversity} (unique-to-total token ratio), \textit{topic consistency} (cosine similarity between seed and final output), and \textit{intention coverage} (unique writing intentions used over 100 iterations).
% We calculated \textit{lexical diversity} (unique-to-total token ratio in the final iteration), \textit{topic consistency} (cosine similarity between the seed document and final output), and \textit{intention coverage} (proportion of unique writing intentions used across 100 iterations out of 15 labels in the taxonomy).
% We calculated \textit{lexical diversity} (the ratio of unique to total tokens in the final iteration), \textit{topic consistency} (cosine similarity between the seed document and the final output), and \textit{intention coverage} (diversity of writing intentions as a proportion of unique labels used across 100 iterations among the 15 available labels in our taxonomy).
For \textbf{human evaluation}, three native English speakers with LaTeX expertise assessed outputs from \textcolor{teal}{Llama-8B-Zero} and \textcolor{magenta}{Llama-8B-SW} on \textit{accuracy} (alignment with predicted intention), \textit{alignment} (similarity to human writing), \textit{fluency}(grammatical correctness), \textit{coherence}(logical structure), and \textit{relevance} (connection to the seed document) - Refer to Appendix \ref{sec:appendix:human-eval}. Accuracy was judged per iteration, while other metrics used pairwise comparisons. Inter-annotator agreement (IAA)\footnote{The IAA scores are 0.84 (\textcolor{magenta}{SW}) and 0.76 (\textcolor{teal}{Zero}) for accuracy, all 100\% for alignment, fluency, coherence, and 49.8\% (\textcolor{magenta}{SW}) and 100\% (\textcolor{teal}{Zero}) for relevance.} was measured using Krippendorff’s alpha for accuracy and percentage agreement for others.

% Following \citet{chang2023surveyevaluationlargelanguage}, we conducted a \textbf{human evaluation} with three native English speakers experienced in LaTeX (see Appendix \ref{sec:appendix:human-eval} for details). They assessed \textit{accuracy} (alignment with predicted intention), \textit{alignment} (similarity to human writing style), \textit{fluency }(grammatical correctness), \textit{coherence} (logical structure), and \textit{relevance }(connection to the seed paper content).
% Furthermore, inspired by \citet{chang2023surveyevaluationlargelanguage}, we conducted a \textbf{human evaluation} with three native English speakers experienced in LaTeX (Refer to Appendix \ref{sec:appendix:human-eval} for more detailed descriptions of the entire evaluation process.). They assessed the outputs based on \textit{accuracy} (alignment with the predicted intention), \textit{alignment} (how closely the model’s process resembled human writing style), \textit{fluency} (grammatical correctness), \textit{coherence} (logical structure), and \textit{relevance} (connection to the seed paper's contents). 

% Accuracy was evaluated per iteration, while alignment, fluency, and coherence were judged via pairwise comparisons. We evaluated outputs from \textcolor{teal}{Llama-8B-Zero} and \textcolor{magenta}{Llama-8B-ScholaWrite}. Inter-annotator agreement (IAA) was measured using Krippendorff’s alpha for accuracy and the average percentage of matches between any of the two annotators for other metrics\footnote{The IAA scores are 0.84 (\textcolor{magenta}{SW}) and 0.76 (\textcolor{teal}{Zero}) for accuracy, all 100\% for alignment, fluency, coherence, and 49.8\% (\textcolor{magenta}{SW}) and 100\% (\textcolor{teal}{Zero}) for relevance.}.
% Accuracy was evaluated for each iteration, while alignment, fluency, and coherence were assessed through pairwise comparisons. Here, we only evaluated outputs from \textcolor{teal}{Llama-8B-Zero} and its fine-tuned model (\textcolor{magenta}{Llama-8B-ScholaWrite}). Inter-annotator agreement (IAA) was Krippendorff's alpha for \textit{accuracy} and the average percentage of match between any of the two annotators for other metrics\footnote{The IAA scores are 0.84 (\textcolor{magenta}{SW}) and 0.76 (\textcolor{teal}{Zero}) for accuracy, all 100\% for alignment, fluency, coherence, and 49.8\% (\textcolor{magenta}{SW}) and 100\% (\textcolor{teal}{Zero}) for relevance.}. 


% (1) \textit{Accuracy}: Out of 100, the number of generated outputs that align with the provided intention.
% (2) \textit{Alignment}: Which model's whole writing process more like you?
% (3) \textit{Overall Fluency check}: Which model’s final writing sounds grammatically correct?
% (4) \textit{Overall coherence check}: Which model’s final writing sounds more logical?
% (5) \textit{Relevancy}: Does the final writing contain related contents to the title, keywords, introduction? 

\begin{table}[ht!]
\footnotesize 
\centering
\begin{tabular}{@{}p{1.2cm}c|c|c|c|c@{}}
\toprule
\textbf{Metrics} & \textbf{Model} & \textbf{Seed 1} & \textbf{Seed 2} & \textbf{Seed 3} & \textbf{Seed 4} \\ \midrule
\multirow{2}{*}{Accuracy} &  \textcolor{magenta}{SW} & 21.0 & 10.3 & 18.3 & 15.3\\
\cmidrule(r){2-6}
& \textcolor{teal}{Zero} & 35.7 & 29.7 & 45.3 & 43.3\\
\midrule
\multirow{2}{*}{Alignment} &  \textcolor{magenta}{SW} & 0 & 0 & 0 & 0\\
\cmidrule(r){2-6}
& \textcolor{teal}{Zero} & 3 & 3 & 3 & 3\\
\midrule
\multirow{2}{*}{Fluency} &  \textcolor{magenta}{SW} & 0 & 0 & 0 & 0\\
\cmidrule(r){2-6}
& \textcolor{teal}{Zero} & 3 & 3 & 3 & 3\\
\midrule
\multirow{2}{*}{Coherence} &  \textcolor{magenta}{SW} & 0 & 0 & 0 & 0\\
\cmidrule(r){2-6}
& \textcolor{teal}{Zero} & 3 & 3 & 3 & 3\\
\midrule
\multirow{2}{*}{Relevance} &  \textcolor{magenta}{SW} & 1 & 3 & 2 & 1\\
\cmidrule(r){2-6}
& \textcolor{teal}{Zero} & 3 & 3 & 3 & 3\\
\bottomrule
\end{tabular}
\caption{Human evaluation results for all the four seed documents. For \textit{accuracy}, each represents the average number of generated keystrokes inferred with correct intentions across three evaluators per seed. For other metrics, each indicates the number of human evaluators who agreed based on the performance of each model. \textcolor{magenta}{SW} abbreviated for \textcolor{magenta}{Llama-8B-SW} and \textcolor{teal}{Zero} for \textcolor{teal}{Llama-8B-Zero}, respectively.
% \dk{Make this table single column. You can use abbreviations for model names to shrink the width}
}
\label{table:human-eval-all}
\end{table}


\paragraph{Results} 

Figure \ref{fig:sec5-auto-all} shows that \textcolor{magenta}{Llama-8B-SW} consistently produced the most lexically diverse words, generated the most semantically aligned topics (Seeds 1 \& 2), and covered the most writing intentions (except Seed 3). These results underscore the value of \textsc{ScholaWrite} in improving scholarly writing quality generated by language models.
% Figure \ref{fig:sec5-auto-all} illustrates the quality of the final writing output produced by each model across all four seed documents. Notably, the two Llama-8B-Instruct models, fine-tuned on \textsc{ScholaWrite} for intention prediction and after-text generation independently (referred to as \textcolor{magenta}{Llama-8B-SW}), consistently used the most lexically diverse words in their final outputs. Moreover, \textcolor{magenta}{Llama-8B-SW} generated content that was semantically most aligned with the seed documents (Seeds 1 \& 2) and covered the highest number of writing intentions based on our taxonomy for all seeds except Seed 3. These results underscore the effectiveness of \textsc{ScholaWrite} as a valuable resource for enhancing the quality of scholarly writing generated by language models. 

However, our human evaluation (Table \ref{table:human-eval-all}) revealed that \textcolor{magenta}{Llama-8B-SW} generated less human-like writing, in terms of fluency and logical claims. It also struggled with generating texts aligned with the predicted intentions. See Appendix Tables \ref{table:human-eval-seed1} to \ref{table:human-eval-seed4} for more details. Despite the weaknesses, \textcolor{magenta}{Llama-8B-SW} still produced more relevant content (Seed 2), which aligns with topic consistency trends in Figure \ref{fig:sec5-auto-all}, highlighting the usefulness of \textsc{ScholaWrite} dataset in certain contexts. 

% Despite their remarkable performance based on automatic evaluation metrics, LLMs still exhibit limitations in learning human writing behaviors and scholarly thinking processes. According to our human evaluation (Table \ref{table:human-eval-all}), \textcolor{magenta}{Llama-8B-SW} generated fewer instances of ``after'' text that aligned with the predicted intentions from the previous step during 100 iterations across all four seed documents. Furthermore, all three evaluators unanimously agreed that the baseline model, \textcolor{teal}{Llama-8B-Zero}, demonstrated more human-like writing behaviors throughout the iterations. Its final outputs were also perceived as more grammatically correct and containing stronger logical claims compared to \textcolor{magenta}{Llama-8B-SW}. Please refer to Tables \ref{table:human-eval-seed1} to \ref{table:human-eval-seed4} in Appendix \ref{sec:appendix:human-eval} for more detailed results. 
% However, the evaluators also noted that the final outputs from \textcolor{magenta}{Llama-8B-SW} contained more relevant content for Seed 2. This observation aligns with the trend in topic consistency scores shown in Figure \ref{fig:sec5-auto-all}, further highlighting the usefulness of \textsc{ScholaWrite} dataset in certain contexts. 




Moreover, \textcolor{magenta}{Llama-8B-SW} exhibited the most human-like writing activity patterns over time (Figure \ref{fig:writing-step-intention-all-model}), which frequently switches between implementation and revision and covers all three high-level processes. \textcolor{teal}{Llama-8B-Zero} and \textcolor{blue}{GPT-4o} tend to remain in a single high-level stage throughout all 100 iterations of self-writing. Compared to Appendix Figure \ref{fig:writing-step-detailed-all}, which depicts frequent transitions across all three stages in an early draft (e.g., the first 100 steps), \textcolor{magenta}{Llama-8B-SW} most closely replicates human writing behaviors in iterative writing tasks. These findings reinforce the potential of \textsc{ScholaWrite} in helping LLMs emulate human scholarly writing processes.
% Furthermore, the \textcolor{magenta}{Llama-8B-ScholaWrite} model exhibits the most human-like pattern of writing activities over time. As shown in Figure \ref{fig:sec5-timestamp-seed2}, \textcolor{magenta}{Llama-8B-ScholaWrite} frequently switches between implementation and revision stages, whereas \textcolor{teal}{Llama-8B-Zero} and \textcolor{blue}{GPT-4o} tend to remain focused on a single high-level stage throughout all 100 iterations of self-writing. Compared to Figure \ref{fig:timestamp-proj1-intention}, which depicts frequent transitions across all three writing stages in an early draft (i.e., the first 100 steps), \textcolor{magenta}{Llama-8B-ScholaWrite} most closely replicates human writing behaviors in iterative writing tasks. These findings highlight the effectiveness of the \textsc{ScholaWrite} dataset in helping language models learn and emulate human scholarly thinking processes. 

% \minhwa{timestamp figures for human writing}


























% The overall average IAAs for all of the three metrics is 0.8, indicating high agreement in the pairwise evaluation. 



% The Llama 3.2 model provided significant revisions throughout the entire writing process. The GPT-4 model provided a lot of output in the beginning stages of the writing process, but eventually ceased providing output once the text reached a certain state. This happened around 50 iterations for all three seeds. Although the output quality from Llama is poor most of the time, since it is continuously provide revisions, leads human evaluators to prefer its response. We suspect that the prompting methods, which ask the model to revise only at one word or phrase level at a time /ref{prompting methods} may be responsible for the lack of output. 

% We can observe three successful and one unsuccessful writing inference in Figure \ref{fig:llama_iterative_output}. The three successful outputs \ref{fig:llama_iter_sub_a} \ref{fig:llama_iter_sub_b} \ref{fig:llama_iter_sub_c} show the Llama model successfully interpreting the Object Insertion direction (top left of image) and inserting full \LaTeX figures. The Llama model has an unsuccessful prediction in \ref{fig:llama_iter_sub_d} where it removes a paragraph instead of performing Idea Generation. We observe four samples iterations of GPT4o model in Figure \ref{fig:gpt_iterative_output}. The GPT model successfully performed Coherence revision in \ref{fig:gpt_iter_sub_a}, and successfully performed cross-reference and object insertion in \ref{fig:gpt_iter_sub_b} and \ref{fig:gpt_iter_sub_c}. In Figure \ref{fig:gpt_iter_sub_d}, the GPT4o model failed to understand the writing intention of Citation Integration, and instead performed revision in text regarding references. From sample iterations, we can determine that the fine-tuned Llama model is able to generate entire figures and complex revisions according to a scholarly writing intention and appropriate \LaTeX syntax. Yet, these models still struggle with some complex writing intentions and may hallucinate information or generate detrimental revision suggestions.



% \label{sec:Results: Percentage Win of Llama}

% \begin{table}
% \begin{center}
% \begin{tabular}{ c|c|c|c } 
%  \hline
%   & Flow & Accuracy & Fluency \\
%  \hline
% Annotator 1 & 0.667 & 0.667 & 0.667 \\
% Annotator 2 & 0.733 & 0.733 & 0.733 \\
% Avg & 0.7 & 0.7 & 0.7 \\
%  \hline
%  %\caption{Human evaluation results for iterative model inference with GPT4o and Llama 3.2. The values over 0.5 represent preference of Llama 3.2 outputs.}
% \end{tabular}
% \end{center}
% \end{table}
%-----------------------------------------------------




%%%%%%%%%%%%%%%%%%%%%%%%%%%%%%%%%%%%%%%%%%
\section{Conclusion and Future Work} 

% We introduce \textsc{ScholaWrite}, the first dataset capturing the cognitive process of scholarly writing, with 62K LaTeX keystrokes collected via our Chrome extension. With expert annotations using a novel taxonomy of writing cognitive intentions, it enables LLMs to better mimic the non-linear, intention-driven nature of human writing, advancing cognitively-aligned writing assistants.
We present \textsc{ScholaWrite}, a first-of-its-kind dataset capturing the end-to-end cognitive process of scholarly writing, comprising nearly 62K LaTeX keystrokes collected via our custom-built Chrome extension. This dataset, sourced from ten graduate students with varying levels of scientific writing expertise, is further enriched with expert annotations based on a novel taxonomy of cognitive writing intentions inspired by \citet{f508427a-e4c0-3d6a-8abf-03a5d21ec6c4, koo2023decoding}. Through several experiments, \textsc{ScholaWrite} shows its value for advancing the cognitive capabilities of LLMs and developing cognitively-aligned writing assistants, enabling them to mimic the complex, non-linear, and intention-driven nature of human writing. 

% We introduce \textsc{ScholaWrite}, the first-of-its-kind dataset that captures the end-to-end scholarly writing process, consisting of nearly 62K LaTeX keystrokes collected via our custom-built Chrome extension. This dataset, sourced from ten graduate students with varying levels of scientific writing expertise, is further enriched with expert annotations based on a novel taxonomy of cognitive writing intentions inspired by \citet{f508427a-e4c0-3d6a-8abf-03a5d21ec6c4, koo2023decoding}. Our study demonstrates the value of \textsc{ScholaWrite} as a resource for advancing the cognitive capabilities of large language models (LLMs) and developing cognitively-aligned writing assistants, enabling them to emulate the complex, non-linear, and intention-driven nature of human writing.


% Future work may expand the dataset to diverse academic fields and collaborative projects, thus allowing LLMs to generalize beyond fact recall to emulate human reasoning processes in a more realistic environment. Integrating advanced memory architectures and lifelong learning could also lead LLMs to dynamically adapt to evolving writing intentions and generate high-quality scholarly texts. 
Future work includes expanding the dataset to diverse academic fields, authors, and collaborative projects, thus enabling models to generalize beyond fact recall to emulate human decision-making and reasoning in more realistic academic environments. Additionally, integrating advanced memory architectures and lifelong learning techniques could further enhance LLMs' ability to adapt dynamically to evolving writing intentions and produce coherent, high-quality scholarly outputs.

\section*{Limitations and Ethical Considerations}

We acknowledge several limitations in our study. First, the \textsc{ScholaWrite} dataset is \textbf{currently limited to the computer science domain}, as LaTeX is predominantly used in computer science journals and conferences. This domain-specific focus may restrict the dataset's generalizability to other scientific disciplines. Future work could address this limitation by collecting keystroke data from a broader range of fields with diverse writing conventions and tools, such as the humanities or biological sciences. For example, students in humanities usually write book-length papers and integrate more sources, so it could affect cognitive complexities.

Second, our dataset includes \textbf{contributions from only 10 participants, resulting in five final preprints on arXiv}. This small-to-medium sample size is partly due to privacy concerns, as the dataset captures raw keystrokes that transparently reflect real-time human reasoning. To mitigate these concerns, we removed all personally identifiable information (PII) during post-processing and obtained full IRB approval for the study's procedures. However, the highly transparent nature of keystroke data may still have discouraged broader participation. Future studies could explore more robust data collection protocols, such as advanced anonymization or de-identification techniques, to better address privacy concerns and enable larger-scale participation.
We also call for community-wise collaboration and participation for our next version of our dataset, \textsc{ScholaWrite 2.0} and encourage researchers to contact authors for future participation.

Furthermore, \textbf{all participants were early-career researchers} (e.g., PhD students) at an R1 university in the United States. Expanding the dataset to include senior researchers, such as post-doctoral fellows and professors, could offer valuable insights into how writing strategies and revision behaviors evolve with research experience and expertise.
Despite these limitations, our study captured an end-to-end writing process for 10 unique authors, resulting in a diverse range of writing styles and revision patterns. The dataset contains approximately 62,000 keystrokes, offering fine-grained insights into the human writing process, including detailed editing and drafting actions over time. While the number of articles is limited, the granularity and volume of the data provide a rich resource for understanding writing behaviors. Prior research has shown that detailed keystroke logs, even from small datasets, can effectively model writing processes \cite{leijten2013keystroke, guo2018modeling, vandermeulen2023writing}. Unlike studies focused on final outputs, our dataset enables a process-oriented analysis, emphasizing the cognitive and behavioral patterns underlying scholarly writing.

Third, \textbf{collaborative writing is underrepresented} in our dataset, as only one Overleaf project involved multiple authors. This limits our ability to analyze co-authorship dynamics and collaborative writing practices, which are common in scientific writing. Future work should prioritize collecting multi-author projects to better capture these dynamics. Additionally, the dataset is \textbf{exclusive to English-language writing}, which restricts its applicability to multilingual or non-English writing contexts. Expanding to multilingual settings could reveal unique cognitive and linguistic insights into writing across languages.

Fourth, due to computational and cost constraints, we evaluated the usability of the \textsc{ScholaWrite} dataset with \textbf{a limited number of LLMs and hyperparameter configurations}. As shown in Table \ref{table:intention-prediction}, the Llama-8B-Instruct model demonstrated only marginal improvements after fine-tuning on our dataset. This underscores the need for future research to explore advanced techniques, such as fine-grained prompt engineering, to better align LLM outputs with human writing processes. Specifically, optimizing prompts with clearer contextual guidance (e.g., "before-text" and intention label definitions) may significantly enhance model performance.

Finally, the human evaluation process in Section \ref{sec:appendix:human-eval} was determined as exempt from IRB review by the authors' primary institution, while the data collection using our Chrome extension program was fully approved by the IRB at our institution. Importantly, no LLMs were used during any stage of the study, except for grammatical error correction in this manuscript. 


% \section*{Acknowledgements}
% This work was supported by the research gift from Grammarly. 
% We thank Anna Martin-Boyle and Ryan Koo for their insights and the data collection tool based on their earlier version of this paper.
% We thank the participants of our data collection for their valuable time, effort, and contributions, which were essential to the success of this research.
% We also thank Minnesota NLP group members and anonymous reviewers for providing us with valuable feedback and comments on the paper draft. 


% \section*{Acknowledgments}


% Bibliography entries for the entire Anthology, followed by custom entries
%\bibliography{anthology,custom}
% Custom bibliography entries only
\bibliography{custom}

\appendix
\subsection{Lloyd-Max Algorithm}
\label{subsec:Lloyd-Max}
For a given quantization bitwidth $B$ and an operand $\bm{X}$, the Lloyd-Max algorithm finds $2^B$ quantization levels $\{\hat{x}_i\}_{i=1}^{2^B}$ such that quantizing $\bm{X}$ by rounding each scalar in $\bm{X}$ to the nearest quantization level minimizes the quantization MSE. 

The algorithm starts with an initial guess of quantization levels and then iteratively computes quantization thresholds $\{\tau_i\}_{i=1}^{2^B-1}$ and updates quantization levels $\{\hat{x}_i\}_{i=1}^{2^B}$. Specifically, at iteration $n$, thresholds are set to the midpoints of the previous iteration's levels:
\begin{align*}
    \tau_i^{(n)}=\frac{\hat{x}_i^{(n-1)}+\hat{x}_{i+1}^{(n-1)}}2 \text{ for } i=1\ldots 2^B-1
\end{align*}
Subsequently, the quantization levels are re-computed as conditional means of the data regions defined by the new thresholds:
\begin{align*}
    \hat{x}_i^{(n)}=\mathbb{E}\left[ \bm{X} \big| \bm{X}\in [\tau_{i-1}^{(n)},\tau_i^{(n)}] \right] \text{ for } i=1\ldots 2^B
\end{align*}
where to satisfy boundary conditions we have $\tau_0=-\infty$ and $\tau_{2^B}=\infty$. The algorithm iterates the above steps until convergence.

Figure \ref{fig:lm_quant} compares the quantization levels of a $7$-bit floating point (E3M3) quantizer (left) to a $7$-bit Lloyd-Max quantizer (right) when quantizing a layer of weights from the GPT3-126M model at a per-tensor granularity. As shown, the Lloyd-Max quantizer achieves substantially lower quantization MSE. Further, Table \ref{tab:FP7_vs_LM7} shows the superior perplexity achieved by Lloyd-Max quantizers for bitwidths of $7$, $6$ and $5$. The difference between the quantizers is clear at 5 bits, where per-tensor FP quantization incurs a drastic and unacceptable increase in perplexity, while Lloyd-Max quantization incurs a much smaller increase. Nevertheless, we note that even the optimal Lloyd-Max quantizer incurs a notable ($\sim 1.5$) increase in perplexity due to the coarse granularity of quantization. 

\begin{figure}[h]
  \centering
  \includegraphics[width=0.7\linewidth]{sections/figures/LM7_FP7.pdf}
  \caption{\small Quantization levels and the corresponding quantization MSE of Floating Point (left) vs Lloyd-Max (right) Quantizers for a layer of weights in the GPT3-126M model.}
  \label{fig:lm_quant}
\end{figure}

\begin{table}[h]\scriptsize
\begin{center}
\caption{\label{tab:FP7_vs_LM7} \small Comparing perplexity (lower is better) achieved by floating point quantizers and Lloyd-Max quantizers on a GPT3-126M model for the Wikitext-103 dataset.}
\begin{tabular}{c|cc|c}
\hline
 \multirow{2}{*}{\textbf{Bitwidth}} & \multicolumn{2}{|c|}{\textbf{Floating-Point Quantizer}} & \textbf{Lloyd-Max Quantizer} \\
 & Best Format & Wikitext-103 Perplexity & Wikitext-103 Perplexity \\
\hline
7 & E3M3 & 18.32 & 18.27 \\
6 & E3M2 & 19.07 & 18.51 \\
5 & E4M0 & 43.89 & 19.71 \\
\hline
\end{tabular}
\end{center}
\end{table}

\subsection{Proof of Local Optimality of LO-BCQ}
\label{subsec:lobcq_opt_proof}
For a given block $\bm{b}_j$, the quantization MSE during LO-BCQ can be empirically evaluated as $\frac{1}{L_b}\lVert \bm{b}_j- \bm{\hat{b}}_j\rVert^2_2$ where $\bm{\hat{b}}_j$ is computed from equation (\ref{eq:clustered_quantization_definition}) as $C_{f(\bm{b}_j)}(\bm{b}_j)$. Further, for a given block cluster $\mathcal{B}_i$, we compute the quantization MSE as $\frac{1}{|\mathcal{B}_{i}|}\sum_{\bm{b} \in \mathcal{B}_{i}} \frac{1}{L_b}\lVert \bm{b}- C_i^{(n)}(\bm{b})\rVert^2_2$. Therefore, at the end of iteration $n$, we evaluate the overall quantization MSE $J^{(n)}$ for a given operand $\bm{X}$ composed of $N_c$ block clusters as:
\begin{align*}
    \label{eq:mse_iter_n}
    J^{(n)} = \frac{1}{N_c} \sum_{i=1}^{N_c} \frac{1}{|\mathcal{B}_{i}^{(n)}|}\sum_{\bm{v} \in \mathcal{B}_{i}^{(n)}} \frac{1}{L_b}\lVert \bm{b}- B_i^{(n)}(\bm{b})\rVert^2_2
\end{align*}

At the end of iteration $n$, the codebooks are updated from $\mathcal{C}^{(n-1)}$ to $\mathcal{C}^{(n)}$. However, the mapping of a given vector $\bm{b}_j$ to quantizers $\mathcal{C}^{(n)}$ remains as  $f^{(n)}(\bm{b}_j)$. At the next iteration, during the vector clustering step, $f^{(n+1)}(\bm{b}_j)$ finds new mapping of $\bm{b}_j$ to updated codebooks $\mathcal{C}^{(n)}$ such that the quantization MSE over the candidate codebooks is minimized. Therefore, we obtain the following result for $\bm{b}_j$:
\begin{align*}
\frac{1}{L_b}\lVert \bm{b}_j - C_{f^{(n+1)}(\bm{b}_j)}^{(n)}(\bm{b}_j)\rVert^2_2 \le \frac{1}{L_b}\lVert \bm{b}_j - C_{f^{(n)}(\bm{b}_j)}^{(n)}(\bm{b}_j)\rVert^2_2
\end{align*}

That is, quantizing $\bm{b}_j$ at the end of the block clustering step of iteration $n+1$ results in lower quantization MSE compared to quantizing at the end of iteration $n$. Since this is true for all $\bm{b} \in \bm{X}$, we assert the following:
\begin{equation}
\begin{split}
\label{eq:mse_ineq_1}
    \tilde{J}^{(n+1)} &= \frac{1}{N_c} \sum_{i=1}^{N_c} \frac{1}{|\mathcal{B}_{i}^{(n+1)}|}\sum_{\bm{b} \in \mathcal{B}_{i}^{(n+1)}} \frac{1}{L_b}\lVert \bm{b} - C_i^{(n)}(b)\rVert^2_2 \le J^{(n)}
\end{split}
\end{equation}
where $\tilde{J}^{(n+1)}$ is the the quantization MSE after the vector clustering step at iteration $n+1$.

Next, during the codebook update step (\ref{eq:quantizers_update}) at iteration $n+1$, the per-cluster codebooks $\mathcal{C}^{(n)}$ are updated to $\mathcal{C}^{(n+1)}$ by invoking the Lloyd-Max algorithm \citep{Lloyd}. We know that for any given value distribution, the Lloyd-Max algorithm minimizes the quantization MSE. Therefore, for a given vector cluster $\mathcal{B}_i$ we obtain the following result:

\begin{equation}
    \frac{1}{|\mathcal{B}_{i}^{(n+1)}|}\sum_{\bm{b} \in \mathcal{B}_{i}^{(n+1)}} \frac{1}{L_b}\lVert \bm{b}- C_i^{(n+1)}(\bm{b})\rVert^2_2 \le \frac{1}{|\mathcal{B}_{i}^{(n+1)}|}\sum_{\bm{b} \in \mathcal{B}_{i}^{(n+1)}} \frac{1}{L_b}\lVert \bm{b}- C_i^{(n)}(\bm{b})\rVert^2_2
\end{equation}

The above equation states that quantizing the given block cluster $\mathcal{B}_i$ after updating the associated codebook from $C_i^{(n)}$ to $C_i^{(n+1)}$ results in lower quantization MSE. Since this is true for all the block clusters, we derive the following result: 
\begin{equation}
\begin{split}
\label{eq:mse_ineq_2}
     J^{(n+1)} &= \frac{1}{N_c} \sum_{i=1}^{N_c} \frac{1}{|\mathcal{B}_{i}^{(n+1)}|}\sum_{\bm{b} \in \mathcal{B}_{i}^{(n+1)}} \frac{1}{L_b}\lVert \bm{b}- C_i^{(n+1)}(\bm{b})\rVert^2_2  \le \tilde{J}^{(n+1)}   
\end{split}
\end{equation}

Following (\ref{eq:mse_ineq_1}) and (\ref{eq:mse_ineq_2}), we find that the quantization MSE is non-increasing for each iteration, that is, $J^{(1)} \ge J^{(2)} \ge J^{(3)} \ge \ldots \ge J^{(M)}$ where $M$ is the maximum number of iterations. 
%Therefore, we can say that if the algorithm converges, then it must be that it has converged to a local minimum. 
\hfill $\blacksquare$


\begin{figure}
    \begin{center}
    \includegraphics[width=0.5\textwidth]{sections//figures/mse_vs_iter.pdf}
    \end{center}
    \caption{\small NMSE vs iterations during LO-BCQ compared to other block quantization proposals}
    \label{fig:nmse_vs_iter}
\end{figure}

Figure \ref{fig:nmse_vs_iter} shows the empirical convergence of LO-BCQ across several block lengths and number of codebooks. Also, the MSE achieved by LO-BCQ is compared to baselines such as MXFP and VSQ. As shown, LO-BCQ converges to a lower MSE than the baselines. Further, we achieve better convergence for larger number of codebooks ($N_c$) and for a smaller block length ($L_b$), both of which increase the bitwidth of BCQ (see Eq \ref{eq:bitwidth_bcq}).


\subsection{Additional Accuracy Results}
%Table \ref{tab:lobcq_config} lists the various LOBCQ configurations and their corresponding bitwidths.
\begin{table}
\setlength{\tabcolsep}{4.75pt}
\begin{center}
\caption{\label{tab:lobcq_config} Various LO-BCQ configurations and their bitwidths.}
\begin{tabular}{|c||c|c|c|c||c|c||c|} 
\hline
 & \multicolumn{4}{|c||}{$L_b=8$} & \multicolumn{2}{|c||}{$L_b=4$} & $L_b=2$ \\
 \hline
 \backslashbox{$L_A$\kern-1em}{\kern-1em$N_c$} & 2 & 4 & 8 & 16 & 2 & 4 & 2 \\
 \hline
 64 & 4.25 & 4.375 & 4.5 & 4.625 & 4.375 & 4.625 & 4.625\\
 \hline
 32 & 4.375 & 4.5 & 4.625& 4.75 & 4.5 & 4.75 & 4.75 \\
 \hline
 16 & 4.625 & 4.75& 4.875 & 5 & 4.75 & 5 & 5 \\
 \hline
\end{tabular}
\end{center}
\end{table}

%\subsection{Perplexity achieved by various LO-BCQ configurations on Wikitext-103 dataset}

\begin{table} \centering
\begin{tabular}{|c||c|c|c|c||c|c||c|} 
\hline
 $L_b \rightarrow$& \multicolumn{4}{c||}{8} & \multicolumn{2}{c||}{4} & 2\\
 \hline
 \backslashbox{$L_A$\kern-1em}{\kern-1em$N_c$} & 2 & 4 & 8 & 16 & 2 & 4 & 2  \\
 %$N_c \rightarrow$ & 2 & 4 & 8 & 16 & 2 & 4 & 2 \\
 \hline
 \hline
 \multicolumn{8}{c}{GPT3-1.3B (FP32 PPL = 9.98)} \\ 
 \hline
 \hline
 64 & 10.40 & 10.23 & 10.17 & 10.15 &  10.28 & 10.18 & 10.19 \\
 \hline
 32 & 10.25 & 10.20 & 10.15 & 10.12 &  10.23 & 10.17 & 10.17 \\
 \hline
 16 & 10.22 & 10.16 & 10.10 & 10.09 &  10.21 & 10.14 & 10.16 \\
 \hline
  \hline
 \multicolumn{8}{c}{GPT3-8B (FP32 PPL = 7.38)} \\ 
 \hline
 \hline
 64 & 7.61 & 7.52 & 7.48 &  7.47 &  7.55 &  7.49 & 7.50 \\
 \hline
 32 & 7.52 & 7.50 & 7.46 &  7.45 &  7.52 &  7.48 & 7.48  \\
 \hline
 16 & 7.51 & 7.48 & 7.44 &  7.44 &  7.51 &  7.49 & 7.47  \\
 \hline
\end{tabular}
\caption{\label{tab:ppl_gpt3_abalation} Wikitext-103 perplexity across GPT3-1.3B and 8B models.}
\end{table}

\begin{table} \centering
\begin{tabular}{|c||c|c|c|c||} 
\hline
 $L_b \rightarrow$& \multicolumn{4}{c||}{8}\\
 \hline
 \backslashbox{$L_A$\kern-1em}{\kern-1em$N_c$} & 2 & 4 & 8 & 16 \\
 %$N_c \rightarrow$ & 2 & 4 & 8 & 16 & 2 & 4 & 2 \\
 \hline
 \hline
 \multicolumn{5}{|c|}{Llama2-7B (FP32 PPL = 5.06)} \\ 
 \hline
 \hline
 64 & 5.31 & 5.26 & 5.19 & 5.18  \\
 \hline
 32 & 5.23 & 5.25 & 5.18 & 5.15  \\
 \hline
 16 & 5.23 & 5.19 & 5.16 & 5.14  \\
 \hline
 \multicolumn{5}{|c|}{Nemotron4-15B (FP32 PPL = 5.87)} \\ 
 \hline
 \hline
 64  & 6.3 & 6.20 & 6.13 & 6.08  \\
 \hline
 32  & 6.24 & 6.12 & 6.07 & 6.03  \\
 \hline
 16  & 6.12 & 6.14 & 6.04 & 6.02  \\
 \hline
 \multicolumn{5}{|c|}{Nemotron4-340B (FP32 PPL = 3.48)} \\ 
 \hline
 \hline
 64 & 3.67 & 3.62 & 3.60 & 3.59 \\
 \hline
 32 & 3.63 & 3.61 & 3.59 & 3.56 \\
 \hline
 16 & 3.61 & 3.58 & 3.57 & 3.55 \\
 \hline
\end{tabular}
\caption{\label{tab:ppl_llama7B_nemo15B} Wikitext-103 perplexity compared to FP32 baseline in Llama2-7B and Nemotron4-15B, 340B models}
\end{table}

%\subsection{Perplexity achieved by various LO-BCQ configurations on MMLU dataset}


\begin{table} \centering
\begin{tabular}{|c||c|c|c|c||c|c|c|c|} 
\hline
 $L_b \rightarrow$& \multicolumn{4}{c||}{8} & \multicolumn{4}{c||}{8}\\
 \hline
 \backslashbox{$L_A$\kern-1em}{\kern-1em$N_c$} & 2 & 4 & 8 & 16 & 2 & 4 & 8 & 16  \\
 %$N_c \rightarrow$ & 2 & 4 & 8 & 16 & 2 & 4 & 2 \\
 \hline
 \hline
 \multicolumn{5}{|c|}{Llama2-7B (FP32 Accuracy = 45.8\%)} & \multicolumn{4}{|c|}{Llama2-70B (FP32 Accuracy = 69.12\%)} \\ 
 \hline
 \hline
 64 & 43.9 & 43.4 & 43.9 & 44.9 & 68.07 & 68.27 & 68.17 & 68.75 \\
 \hline
 32 & 44.5 & 43.8 & 44.9 & 44.5 & 68.37 & 68.51 & 68.35 & 68.27  \\
 \hline
 16 & 43.9 & 42.7 & 44.9 & 45 & 68.12 & 68.77 & 68.31 & 68.59  \\
 \hline
 \hline
 \multicolumn{5}{|c|}{GPT3-22B (FP32 Accuracy = 38.75\%)} & \multicolumn{4}{|c|}{Nemotron4-15B (FP32 Accuracy = 64.3\%)} \\ 
 \hline
 \hline
 64 & 36.71 & 38.85 & 38.13 & 38.92 & 63.17 & 62.36 & 63.72 & 64.09 \\
 \hline
 32 & 37.95 & 38.69 & 39.45 & 38.34 & 64.05 & 62.30 & 63.8 & 64.33  \\
 \hline
 16 & 38.88 & 38.80 & 38.31 & 38.92 & 63.22 & 63.51 & 63.93 & 64.43  \\
 \hline
\end{tabular}
\caption{\label{tab:mmlu_abalation} Accuracy on MMLU dataset across GPT3-22B, Llama2-7B, 70B and Nemotron4-15B models.}
\end{table}


%\subsection{Perplexity achieved by various LO-BCQ configurations on LM evaluation harness}

\begin{table} \centering
\begin{tabular}{|c||c|c|c|c||c|c|c|c|} 
\hline
 $L_b \rightarrow$& \multicolumn{4}{c||}{8} & \multicolumn{4}{c||}{8}\\
 \hline
 \backslashbox{$L_A$\kern-1em}{\kern-1em$N_c$} & 2 & 4 & 8 & 16 & 2 & 4 & 8 & 16  \\
 %$N_c \rightarrow$ & 2 & 4 & 8 & 16 & 2 & 4 & 2 \\
 \hline
 \hline
 \multicolumn{5}{|c|}{Race (FP32 Accuracy = 37.51\%)} & \multicolumn{4}{|c|}{Boolq (FP32 Accuracy = 64.62\%)} \\ 
 \hline
 \hline
 64 & 36.94 & 37.13 & 36.27 & 37.13 & 63.73 & 62.26 & 63.49 & 63.36 \\
 \hline
 32 & 37.03 & 36.36 & 36.08 & 37.03 & 62.54 & 63.51 & 63.49 & 63.55  \\
 \hline
 16 & 37.03 & 37.03 & 36.46 & 37.03 & 61.1 & 63.79 & 63.58 & 63.33  \\
 \hline
 \hline
 \multicolumn{5}{|c|}{Winogrande (FP32 Accuracy = 58.01\%)} & \multicolumn{4}{|c|}{Piqa (FP32 Accuracy = 74.21\%)} \\ 
 \hline
 \hline
 64 & 58.17 & 57.22 & 57.85 & 58.33 & 73.01 & 73.07 & 73.07 & 72.80 \\
 \hline
 32 & 59.12 & 58.09 & 57.85 & 58.41 & 73.01 & 73.94 & 72.74 & 73.18  \\
 \hline
 16 & 57.93 & 58.88 & 57.93 & 58.56 & 73.94 & 72.80 & 73.01 & 73.94  \\
 \hline
\end{tabular}
\caption{\label{tab:mmlu_abalation} Accuracy on LM evaluation harness tasks on GPT3-1.3B model.}
\end{table}

\begin{table} \centering
\begin{tabular}{|c||c|c|c|c||c|c|c|c|} 
\hline
 $L_b \rightarrow$& \multicolumn{4}{c||}{8} & \multicolumn{4}{c||}{8}\\
 \hline
 \backslashbox{$L_A$\kern-1em}{\kern-1em$N_c$} & 2 & 4 & 8 & 16 & 2 & 4 & 8 & 16  \\
 %$N_c \rightarrow$ & 2 & 4 & 8 & 16 & 2 & 4 & 2 \\
 \hline
 \hline
 \multicolumn{5}{|c|}{Race (FP32 Accuracy = 41.34\%)} & \multicolumn{4}{|c|}{Boolq (FP32 Accuracy = 68.32\%)} \\ 
 \hline
 \hline
 64 & 40.48 & 40.10 & 39.43 & 39.90 & 69.20 & 68.41 & 69.45 & 68.56 \\
 \hline
 32 & 39.52 & 39.52 & 40.77 & 39.62 & 68.32 & 67.43 & 68.17 & 69.30  \\
 \hline
 16 & 39.81 & 39.71 & 39.90 & 40.38 & 68.10 & 66.33 & 69.51 & 69.42  \\
 \hline
 \hline
 \multicolumn{5}{|c|}{Winogrande (FP32 Accuracy = 67.88\%)} & \multicolumn{4}{|c|}{Piqa (FP32 Accuracy = 78.78\%)} \\ 
 \hline
 \hline
 64 & 66.85 & 66.61 & 67.72 & 67.88 & 77.31 & 77.42 & 77.75 & 77.64 \\
 \hline
 32 & 67.25 & 67.72 & 67.72 & 67.00 & 77.31 & 77.04 & 77.80 & 77.37  \\
 \hline
 16 & 68.11 & 68.90 & 67.88 & 67.48 & 77.37 & 78.13 & 78.13 & 77.69  \\
 \hline
\end{tabular}
\caption{\label{tab:mmlu_abalation} Accuracy on LM evaluation harness tasks on GPT3-8B model.}
\end{table}

\begin{table} \centering
\begin{tabular}{|c||c|c|c|c||c|c|c|c|} 
\hline
 $L_b \rightarrow$& \multicolumn{4}{c||}{8} & \multicolumn{4}{c||}{8}\\
 \hline
 \backslashbox{$L_A$\kern-1em}{\kern-1em$N_c$} & 2 & 4 & 8 & 16 & 2 & 4 & 8 & 16  \\
 %$N_c \rightarrow$ & 2 & 4 & 8 & 16 & 2 & 4 & 2 \\
 \hline
 \hline
 \multicolumn{5}{|c|}{Race (FP32 Accuracy = 40.67\%)} & \multicolumn{4}{|c|}{Boolq (FP32 Accuracy = 76.54\%)} \\ 
 \hline
 \hline
 64 & 40.48 & 40.10 & 39.43 & 39.90 & 75.41 & 75.11 & 77.09 & 75.66 \\
 \hline
 32 & 39.52 & 39.52 & 40.77 & 39.62 & 76.02 & 76.02 & 75.96 & 75.35  \\
 \hline
 16 & 39.81 & 39.71 & 39.90 & 40.38 & 75.05 & 73.82 & 75.72 & 76.09  \\
 \hline
 \hline
 \multicolumn{5}{|c|}{Winogrande (FP32 Accuracy = 70.64\%)} & \multicolumn{4}{|c|}{Piqa (FP32 Accuracy = 79.16\%)} \\ 
 \hline
 \hline
 64 & 69.14 & 70.17 & 70.17 & 70.56 & 78.24 & 79.00 & 78.62 & 78.73 \\
 \hline
 32 & 70.96 & 69.69 & 71.27 & 69.30 & 78.56 & 79.49 & 79.16 & 78.89  \\
 \hline
 16 & 71.03 & 69.53 & 69.69 & 70.40 & 78.13 & 79.16 & 79.00 & 79.00  \\
 \hline
\end{tabular}
\caption{\label{tab:mmlu_abalation} Accuracy on LM evaluation harness tasks on GPT3-22B model.}
\end{table}

\begin{table} \centering
\begin{tabular}{|c||c|c|c|c||c|c|c|c|} 
\hline
 $L_b \rightarrow$& \multicolumn{4}{c||}{8} & \multicolumn{4}{c||}{8}\\
 \hline
 \backslashbox{$L_A$\kern-1em}{\kern-1em$N_c$} & 2 & 4 & 8 & 16 & 2 & 4 & 8 & 16  \\
 %$N_c \rightarrow$ & 2 & 4 & 8 & 16 & 2 & 4 & 2 \\
 \hline
 \hline
 \multicolumn{5}{|c|}{Race (FP32 Accuracy = 44.4\%)} & \multicolumn{4}{|c|}{Boolq (FP32 Accuracy = 79.29\%)} \\ 
 \hline
 \hline
 64 & 42.49 & 42.51 & 42.58 & 43.45 & 77.58 & 77.37 & 77.43 & 78.1 \\
 \hline
 32 & 43.35 & 42.49 & 43.64 & 43.73 & 77.86 & 75.32 & 77.28 & 77.86  \\
 \hline
 16 & 44.21 & 44.21 & 43.64 & 42.97 & 78.65 & 77 & 76.94 & 77.98  \\
 \hline
 \hline
 \multicolumn{5}{|c|}{Winogrande (FP32 Accuracy = 69.38\%)} & \multicolumn{4}{|c|}{Piqa (FP32 Accuracy = 78.07\%)} \\ 
 \hline
 \hline
 64 & 68.9 & 68.43 & 69.77 & 68.19 & 77.09 & 76.82 & 77.09 & 77.86 \\
 \hline
 32 & 69.38 & 68.51 & 68.82 & 68.90 & 78.07 & 76.71 & 78.07 & 77.86  \\
 \hline
 16 & 69.53 & 67.09 & 69.38 & 68.90 & 77.37 & 77.8 & 77.91 & 77.69  \\
 \hline
\end{tabular}
\caption{\label{tab:mmlu_abalation} Accuracy on LM evaluation harness tasks on Llama2-7B model.}
\end{table}

\begin{table} \centering
\begin{tabular}{|c||c|c|c|c||c|c|c|c|} 
\hline
 $L_b \rightarrow$& \multicolumn{4}{c||}{8} & \multicolumn{4}{c||}{8}\\
 \hline
 \backslashbox{$L_A$\kern-1em}{\kern-1em$N_c$} & 2 & 4 & 8 & 16 & 2 & 4 & 8 & 16  \\
 %$N_c \rightarrow$ & 2 & 4 & 8 & 16 & 2 & 4 & 2 \\
 \hline
 \hline
 \multicolumn{5}{|c|}{Race (FP32 Accuracy = 48.8\%)} & \multicolumn{4}{|c|}{Boolq (FP32 Accuracy = 85.23\%)} \\ 
 \hline
 \hline
 64 & 49.00 & 49.00 & 49.28 & 48.71 & 82.82 & 84.28 & 84.03 & 84.25 \\
 \hline
 32 & 49.57 & 48.52 & 48.33 & 49.28 & 83.85 & 84.46 & 84.31 & 84.93  \\
 \hline
 16 & 49.85 & 49.09 & 49.28 & 48.99 & 85.11 & 84.46 & 84.61 & 83.94  \\
 \hline
 \hline
 \multicolumn{5}{|c|}{Winogrande (FP32 Accuracy = 79.95\%)} & \multicolumn{4}{|c|}{Piqa (FP32 Accuracy = 81.56\%)} \\ 
 \hline
 \hline
 64 & 78.77 & 78.45 & 78.37 & 79.16 & 81.45 & 80.69 & 81.45 & 81.5 \\
 \hline
 32 & 78.45 & 79.01 & 78.69 & 80.66 & 81.56 & 80.58 & 81.18 & 81.34  \\
 \hline
 16 & 79.95 & 79.56 & 79.79 & 79.72 & 81.28 & 81.66 & 81.28 & 80.96  \\
 \hline
\end{tabular}
\caption{\label{tab:mmlu_abalation} Accuracy on LM evaluation harness tasks on Llama2-70B model.}
\end{table}

%\section{MSE Studies}
%\textcolor{red}{TODO}


\subsection{Number Formats and Quantization Method}
\label{subsec:numFormats_quantMethod}
\subsubsection{Integer Format}
An $n$-bit signed integer (INT) is typically represented with a 2s-complement format \citep{yao2022zeroquant,xiao2023smoothquant,dai2021vsq}, where the most significant bit denotes the sign.

\subsubsection{Floating Point Format}
An $n$-bit signed floating point (FP) number $x$ comprises of a 1-bit sign ($x_{\mathrm{sign}}$), $B_m$-bit mantissa ($x_{\mathrm{mant}}$) and $B_e$-bit exponent ($x_{\mathrm{exp}}$) such that $B_m+B_e=n-1$. The associated constant exponent bias ($E_{\mathrm{bias}}$) is computed as $(2^{{B_e}-1}-1)$. We denote this format as $E_{B_e}M_{B_m}$.  

\subsubsection{Quantization Scheme}
\label{subsec:quant_method}
A quantization scheme dictates how a given unquantized tensor is converted to its quantized representation. We consider FP formats for the purpose of illustration. Given an unquantized tensor $\bm{X}$ and an FP format $E_{B_e}M_{B_m}$, we first, we compute the quantization scale factor $s_X$ that maps the maximum absolute value of $\bm{X}$ to the maximum quantization level of the $E_{B_e}M_{B_m}$ format as follows:
\begin{align}
\label{eq:sf}
    s_X = \frac{\mathrm{max}(|\bm{X}|)}{\mathrm{max}(E_{B_e}M_{B_m})}
\end{align}
In the above equation, $|\cdot|$ denotes the absolute value function.

Next, we scale $\bm{X}$ by $s_X$ and quantize it to $\hat{\bm{X}}$ by rounding it to the nearest quantization level of $E_{B_e}M_{B_m}$ as:

\begin{align}
\label{eq:tensor_quant}
    \hat{\bm{X}} = \text{round-to-nearest}\left(\frac{\bm{X}}{s_X}, E_{B_e}M_{B_m}\right)
\end{align}

We perform dynamic max-scaled quantization \citep{wu2020integer}, where the scale factor $s$ for activations is dynamically computed during runtime.

\subsection{Vector Scaled Quantization}
\begin{wrapfigure}{r}{0.35\linewidth}
  \centering
  \includegraphics[width=\linewidth]{sections/figures/vsquant.jpg}
  \caption{\small Vectorwise decomposition for per-vector scaled quantization (VSQ \citep{dai2021vsq}).}
  \label{fig:vsquant}
\end{wrapfigure}
During VSQ \citep{dai2021vsq}, the operand tensors are decomposed into 1D vectors in a hardware friendly manner as shown in Figure \ref{fig:vsquant}. Since the decomposed tensors are used as operands in matrix multiplications during inference, it is beneficial to perform this decomposition along the reduction dimension of the multiplication. The vectorwise quantization is performed similar to tensorwise quantization described in Equations \ref{eq:sf} and \ref{eq:tensor_quant}, where a scale factor $s_v$ is required for each vector $\bm{v}$ that maps the maximum absolute value of that vector to the maximum quantization level. While smaller vector lengths can lead to larger accuracy gains, the associated memory and computational overheads due to the per-vector scale factors increases. To alleviate these overheads, VSQ \citep{dai2021vsq} proposed a second level quantization of the per-vector scale factors to unsigned integers, while MX \citep{rouhani2023shared} quantizes them to integer powers of 2 (denoted as $2^{INT}$).

\subsubsection{MX Format}
The MX format proposed in \citep{rouhani2023microscaling} introduces the concept of sub-block shifting. For every two scalar elements of $b$-bits each, there is a shared exponent bit. The value of this exponent bit is determined through an empirical analysis that targets minimizing quantization MSE. We note that the FP format $E_{1}M_{b}$ is strictly better than MX from an accuracy perspective since it allocates a dedicated exponent bit to each scalar as opposed to sharing it across two scalars. Therefore, we conservatively bound the accuracy of a $b+2$-bit signed MX format with that of a $E_{1}M_{b}$ format in our comparisons. For instance, we use E1M2 format as a proxy for MX4.

\begin{figure}
    \centering
    \includegraphics[width=1\linewidth]{sections//figures/BlockFormats.pdf}
    \caption{\small Comparing LO-BCQ to MX format.}
    \label{fig:block_formats}
\end{figure}

Figure \ref{fig:block_formats} compares our $4$-bit LO-BCQ block format to MX \citep{rouhani2023microscaling}. As shown, both LO-BCQ and MX decompose a given operand tensor into block arrays and each block array into blocks. Similar to MX, we find that per-block quantization ($L_b < L_A$) leads to better accuracy due to increased flexibility. While MX achieves this through per-block $1$-bit micro-scales, we associate a dedicated codebook to each block through a per-block codebook selector. Further, MX quantizes the per-block array scale-factor to E8M0 format without per-tensor scaling. In contrast during LO-BCQ, we find that per-tensor scaling combined with quantization of per-block array scale-factor to E4M3 format results in superior inference accuracy across models. 

% \input{latex/figure_2_test}
\end{document}
