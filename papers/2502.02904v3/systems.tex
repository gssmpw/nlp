% \begin{figure*}[h]
%     \centering
%     % \makebox[\textwidth]{\includegraphics[width=2.2\columnwidth]{latex/figures/figure_new_system.pdf}}
%     {\includegraphics[width=0.8\columnwidth]{latex/figures/fig3_chrome.pdf}}
%     {\includegraphics[width=1.1\columnwidth]{latex/figures/fig3_annot.pdf}}
%     \caption{Overview of System Workflow for Dataset Construction Process: The chrome extension interface on the Overleaf project (left); the annotation interface (right).}
%     \label{fig:system-flow}
% \end{figure*}

\section{Dataset Construction}
To collect authentic writing data from writers' natural thoughts, we developed a Chrome extension that records keystroke data in the background. After the study, an in-depth annotation process is conducted to interpret the intent behind each keystroke.
% For open research, we make all the code of our data collection and annotation framework publicly available: \url{https://minnesotanlp.github.io/scholawrite/}.
We describe the detailed process of our data collection, annotation system design, and taxonomy creation in the following sections.

\subsection{Participant Recruitment}\label{sec:recruitment}
We recruited 10 graduate students\footnote{All participants  attend a R1 university in US and are proficient in
English. Appendix \ref{sec:appendix:recruitment} details the recruitment process.} from the computer science department who were actively preparing academic manuscripts in Overleaf LaTeX  editor for submission to peer-reviewed conferences in AI and NLP. 
The data is collected from November 2023 to February 2024, totaling up to 4 months.
Our study has been approved by the IRB institution of the authors.
Please refer to Appendix \ref{sec:appendix:recruitment} for more detailed descriptions. 


\subsection{Chrome Extension for Keystroke Collection in Overleaf}
We designed and implemented a Chrome extension (Appendix Figure \ref{fig:chrome-extension}), which enables the real-time collection of keystroke trajectories in the Overleaf platform. Participants can create their account credentials, and after logging into the system, the extension monitors the user's keystrokes in the background silently without disrupting writing process. See Appendix \ref{sec:appendix:system} for more description of the extension workflow.
% , without disrupting the typical writing process. 





% \begin{figure}[ht!]
%     \centering
%     % \makebox[\textwidth]
%     {\includegraphics[width=0.7\columnwidth]{figures/figure3_diff.pdf}}
%     \caption{The array of differences between two subsequent texts, generated by \texttt{diff\_match\_patch}}
%     \label{fig:system-diff}
% \end{figure}

% \begin{figure*}[h]
%     \centering
%     % First subfigure
%     \begin{subfigure}[b]{0.42\textwidth}
%         \centering
%         \includegraphics[width=\textwidth]{figures/fig4_a_dk.pdf}
%         \caption{Chrome Extension Interface}
%         \label{fig:system-flow-chrome}
%     \end{subfigure}
%     \hspace{0.002\textwidth}
%     \rule{1pt}{0.2\textheight} % Vertical line
%     % Second subfigure
%     \begin{subfigure}[b]{0.55\textwidth}
%         \centering
%         \includegraphics[width=\textwidth]{figures/fig4_b_dk.pdf}
%         \caption{Annotation Interface}
%         \label{fig:system-flow-annot}
%     \end{subfigure}    
%     % Main caption
%     \caption{Overview of \textsc{ScholaWrite} dataset construction process: The Chrome extension interface on the Overleaf project (left) and the annotation interface (right). The Chrome extension (A) collects real-time keystrokes in the Overleaf editor (highlighted). 
%     During the annotation stage, annotators can click a viewing mode of the collected keystroke data (B). By right-clicking to navigate the timeline of keystroke trace in the interactive panel on the right side (E), annotators can choose an intention label under the drop-down menu (C). They can also view the meta-information of each annotated keystroke (D).}
%     \label{fig:system-flow}
% \end{figure*}


% When a key-up event fires in a browser, the extension collects the writer's viewable texts in the code editor panel\footnote{To prevent privacy concerns, the extension filters out keystroke data from any unauthorized Overleaf projects. Please see Appendix \ref{sec:appendix:system} for more details.}. When each of these actions\footnote{Example actions are (1) inserting a space/newline; (2) copy/paste; (3) undo/redo; (4) switching files and (5) scrolling a page.} occurs, the extension uses `\texttt{diff\_match\_patch}' package\footnote{https://github.com/google/diff-match-patch} to generate an array of differences between two subsequent texts (i.e., Figure \ref{fig:system-diff}). Then, the extension will send the array along with metadata (e.g., time stamp, author ID, etc.) to the backend server. 
% % To prevent any issue of private data collection, the backend fetches only the IDs of the Overleaf projects that participants consented to share during the recruitment process and filters out participants' keystroke data from any unauthorized projects\footnote{We used Google Sheet API to retrieve ID information that we collected during the recruitment process.}.

% For any Overleaf project that consists of multiple LaTeX files, we also collected all keystrokes from subfiles associated with the main LaTeX file. Our comprehensive data collection process captures the end-to-end writing processes of the participating authors across all parts of Overleaf projects. This approach ensures that our dataset reflects the full scope of scholarly writing including edits made in auxiliary files such as files of each section, appendix, bibliography, etc. 

%%%%%%%%%%%%%%%%%%%%%%%%%%%%%%%%

% All keystroke data stored in the database were processed for annotation and model training purposes. Here, raw keystrokes from the Chrome extension system are stored as arrays that represent the differences between subsequent keystrokes, along with accompanying metadata.


%%%%%%%%%%%%%%%%%%%%%%%%%%%%%%%%

% For every key-up events fired, the extension collects the writer's viewable texts in the code editor panel. When each of these actions \footnote{insert a space/newline, edit on the different line, copy/paste/cut, undo/redo, hidden, switch file, scroll} occurs, the extension uses `\texttt{diff\_match\_patch}'
% package\footnote{https://github.com/google/diff-match-patch} to generate an array of differences between two texts collected before the event and after the event. (i.e., Figure \ref{fig:system-diff}). Then, the extension will send the difference array along with other metadata (e.g., time stamp, author ID, file name, etc.) to the backend server. To prevent any issue of private data collection, the backend fetches only the IDs of the Overleaf projects that participants consented to share during the recruitment process and filters out participants' keystroke data from unauthorized projects \footnote{We used Google Sheet API to retrieve ID information that we collected during the recruitment process}.

% has multiple viewing modes to allow annotator annotate from different perspectives (see Figure 3(c)(1)). The By Time mode means view all key-stoke activity happened in one project. The By File mode means showing all the key-stoke data happened in one file regardless which participants contribute. The By User mode means following the keystroke data of one participant in the project. Each of the key-stoke data has an index representing ascending order in time series.

% On the right side of the annotation page, see Figure 3(c)(3), it reflects context in participants view window. Text will be highlighted to reflect different actions of writing. Green means text insertion. Red means text deletion. Yellow means text copy. Blue means user requested AI paraphrase feature on that text. purple means user reject AI's suggestion on paraphrase. If user accept AI generated paraphrase, green and red will be used to show the text differences between original and paraphrased version. 

% By clicking left and right arrow keys on the keyboard, annotator can navigate back and forth on timeline to see the trajectory of writing. The determine the appropriate writing intention and corresponding interval.

% In the span information section (see Figure 3(c)(2)), the two text boxes above is for entering the beginning index and ending index that reflect certain participant's writing intention. In the selection box below, all labels are grouped by three categories. The annotator can multi-select appropriate writing intention labels to the key-stoke data happened from beginning to ending index they entered above (see Figure 3(c)(4)).













% The project field; we can track which project this document belongs to

% The file field; As there are multiple files in one OverLeaf project, we need a file name to distinguish the file author is currently editing.
% message or state field; so that we know the most recent action the user made.

% The revision field; It should be present and not an empty array. It contains the difference array for visualization which will be used for annotation.

% The editingLines fields; as text collected only limited to the user's view window, line numbers tell us the location of the text section within the file.

% By utilizing Google Sheet API provided by Google Cloud, the back-end application can pull the consented project IDs from the Google Sheet into a Python list. By checking the existence of the project ID in the consent list, the back-end application is able to only accept requests with a consented project ID in the JSON body and insert them into MongoDB.
% The back-end was running 24/7 to receive incoming writing data sent from Chrome extension installed on participants personal computer. The chrome extension will collect text change in any project at anytime whenever participants write in the Overleaf Code Editor. To protect participants privacy, the back-end will ignore the request if the writing data is not from consented Overleaf project. 

% \subsubsection{AI Paraphrase}
% When the user clicks the AI paragraph button in the tab window, the extension will get a selected/high-lighted sentence and context, then send it to the back-end along with metadata (time, project ID, file name). The back-end application fills selected text into a predefined prompt and uses the mini-chain \cite{https://github.com/srush/MiniChain?tab=readme-ov-file} package to invoke gpt3.5-turbo API. After getting the response, the server parses the response, renders an HTML to illustrate differences with the original text, sent it to the client. A panel showing paraphrased text and the difference between the original text will pop up on the right. The user can reject or accept paraphrase text based on their evaluation. 

% The paraphrase request, GPT's output, and user's choice on it are all stored into MongoDB for later analysis.

% \subsection{Data Pre- and Post-Processing}



% For public availability, we made three version of our dataset. From raw to most post-processed.
% 1. Raw data: Without any processing expect anonymization, all successfully received and stored data.

% 2. annotation data: 
%     the data entries from raw data that meets following criteria
%     Entry with text difference array longer or equal to 6 will be split into two data entries. First one is showing deleted. Another one is showing insertion.
%     - The entries has consetned project id in the "project" field.
%     - The file name should exist. 
%     - The "message" or "state" field must be present to illustrate writers action that triggered the key-stoke logging
%     - The "revision" field must be present. This is the array of text differences. The essential part to visualize the participants' writing trajectory.
%     - the "editingLines" field must be present. This records the line numbers that are viewable to the participants in the overleaf editor.
    
% 3. Fine-tuning data:
%     1. the data entries from annotation data that meets following criteria.
%     2. The length of text difference array is smaller than 700.
%     3. Data entries that have only one label

%%%%%
% \minhwa{artifact post-processing in just one paragraph}

% \paragraph{Post-processing}

% To encourage public use, we conducted several steps of post-processing to increase the usability of our keystroke collection and mitigate any privacy concerns (See details in Appendix \ref{sec:appendix:postprocess}). 
% Listing \ref{table:single-entry} shows an example data entry in our dataset.


% \begin{table}[ht!]
% \centering
% \begin{minipage}{\linewidth}
% \lstset{
%     basicstyle=\ttfamily\footnotesize, % Font and size for the code
%     breaklines=true, % Allows breaking of long lines
%     frame=single, % Adds a frame around the code block
%     columns=fullflexible, % Makes the content fit the column width
%     captionpos=b % Places the caption at the bottom
% }
% \begin{lstlisting}
% {
%     "Project": 1,
%     "timestamp": 1702958491535,
%     "author": "author1",
%     "before text": "One important expct of studying a LLMs is ..",
%     "after text": "One important aspect of studying LLMs is ..",
%     "label": "fluency"
% }
% \end{lstlisting}
% \vspace{-3mm}
% \captionof{lstlisting}{An example entry of the post-processed data}
% \label{table:single-entry}
% \end{minipage}
% \end{table}


% \begin{itemize}
%     \item STEP 1: Split single data entries with multiple labels into separate entries.
%     \item STEP 2: Split single data entries with multiple text revisions into separate entries.
%     \item STEP 3: Compare data entries labeled as ``non-informative artifact'' to the previous edits made by the same author. Retain the entry if, after human evaluation, the text differences are determined to be a result of natural writing changes. Otherwise, remove those entries. 
%     % \minhwa{talk more in details about how we defined removal of those artifact keystrokes later}
%     \item STEP 4: Remove any private author information such as names, affiliations, and contact information from the collected keystrokes. Anonymize username registered through extension.
% \end{itemize}
