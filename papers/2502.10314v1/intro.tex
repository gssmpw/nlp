We consider the problem of online \textit{interval selection}, or \textit{interval scheduling} on a single machine, where real-length intervals arrive online, and we must output a set of non-conflicting intervals. Each interval is associated with a weight, and the goal is to maximize the sum of weights of the intervals in the solution. This problem is equivalent to finding a maximum weight independent set in interval graphs. We focus on two weight functions, \textit{unit} (or constant) weights, and \textit{proportional} weights, where the weight of an interval is equal to its length.  While interval scheduling is often studied under the real-time assumption where intervals arrive in order of non-decreasing starting times, we consider the generalized version of \textit{any-order} arrivals \cite{borodin2023any}. In the traditional online model of irrevocable decisions, no algorithm (even randomized) can achieve a constant competitive ratio (Bachmann et al. \cite{bachmann2013online} for unit weights, Lipton and Tomkins \cite{lipton1994online} for proportional). Because of this, and because some applications permit it, a relaxation of the problem that allows for revocable acceptances has been considered. In this model, every new interval can be accepted by displacing any conflicting intervals in the solution, but every rejection is final. In the area of scheduling, this is sometimes also called \textit{preemption}, although no ``restarts'' are allowed in our problem. In the offline setting, an optimal solution can be easily found in polynomial time, both for unit, and for proportional weights \cite{kleinberg2006algorithm}. The applications of interval scheduling include routing \cite{plotkin1995competitive}, computer wiring \cite{gupta1979optimal}, project selections during space missions \cite{hall1994maximizing}, and satellite photography \cite{gabrel1995scheduling}. A more detailed discussion on the applications of interval scheduling can be found in the surveys by Kolen et al. \cite{kolen2007interval} and Kovalyov et al. \cite{kovalyov2007fixed}.\\\\
Motivated by advancements in machine learning and access to a plethora of data, there has been an effort to equip online algorithms with possibly erroneous predictions about the input instance. Such algorithms are able to achieve much better performance when these predictions are accurate, overcoming some pessimistic bounds of competitive analysis, and helping to bridge the gap between theory and practice. Various classical online problems such as ski rental and non-clairvoyant job scheduling \cite{purohit2018improving}, caching \cite{lykouris2021competitive}, facility location \cite{almanza2021online}, metrical task systems \cite{antoniadis2023online}, and matching \cite{antoniadis2020secretary} have been considered in this model. See Mitzenmacher and Vassilvitskii \cite{DBLP:books/cu/20/MitzenmacherV20} for a more detailed survey on the topic, and \cite{ALPS} for an online repository of relevant papers. Predictions are also a form of \textit{untrusted advice} (Angelopoulos et al. \cite{angelopoulos2024online}), a natural extension of the model of online algorithms with advice (Boyar et al. \cite{boyar2017online}) when the advice is imperfect. Advice research tends to be more information theoretic, focusing on tradeoffs between the number of advice bits and the quality of the solution. Although predictions are often available as offline information, given to the algorithm in advance, we consider a model where a prediction is associated with each input item, and is also given online. This is quite natural and has been considered before for problems such as paging, graph coloring, and packing (\cite{lykouris2021competitive,rohatgi2020near,antoniadis2023paging,antoniadis2024online,grigorescu2024simple}). This setting also allows for an oracle to adapt as more of the input is revealed, enabling research where there are different bounds on the quality of later predictions, and allowing one to tailor the predictor algorithm directly \cite{elias2024learning}. Furthermore, we use binary predictions, which has our model falling in line with work considering limited size, or \textit{succinct} predictions \cite{antoniadis2023paging,berg2024complexity,angelopoulos2023contract}.\\\\
\textbf{Related work.} Table \ref{tab:prev_work} shows the most relevant existing work in the conventional online setting. In the case of irrevocable decisions, no algorithm (even randomized) can achieve a constant competitive ratio. For the relaxed model of revocable acceptances, we use an asterisk to indicate that the competitive ratio is optimal. In the context of randomized algorithms and revocable acceptances, Emek at al. \cite{emek2016space} give a $6$-competitive algorithm for unit weights, while we know of no work improving upon the $(2\phi + 1)$-competitive algorithm.

\begin{table}[h]\centering
	\caption{Online results without predictions: $n$ is the size of the input, $k$ is the number of different lengths, $\Delta$ is the ratio of the longest to shortest interval.}
	\label{tab:prev_work}
	\begin{tabular}{c  c  c}%\toprule
		 & \textit{Unit} & \textit{Proportional} \\ \toprule
		\parbox[c]{2cm}{Irrevocable \\ (randomized)}  & $\Omega(n)$ \cite{bachmann2013online} & $\Omega(\log\Delta)$ \cite{lipton1994online} \\ \midrule
  %\hline
		\parbox[c]{2cm}{Revocable \\ (deterministic)} & $2k^*$ \cite{borodin2023any} & $(2\phi + 1)^*$ \cite{garay1997efficient,tomkins1995lower}  \\
  \bottomrule
	\end{tabular}
\end{table}

Boyar et al. \cite{boyar2023online} is the most closely related work to our problem with predictions, and motivated our study. They consider the case of unit weighted intervals on a line graph, and give an optimal deterministic algorithm in the setting of irrevocable decisions with performance $OPT - \eta$ for a different set of predictions and error measure. We extend their work using (possibly adaptive) predictions of limited size, considering an additional weight function of interest, and initiating the study of these problems with revocable decisions.\\\\
\textit{Structure of the paper.} In section \ref{section:prelim} we formally define the model, including our predictions and error measure. Section \ref{section:irrev} is about the model of irrevocable decisions, whereas in section \ref{section:rev} we allow for revocable acceptances. We conclude with some experiments on real-world data (section \ref{section:exp}) that showcase the usefulness of our predictions, and complement our theoretical results.