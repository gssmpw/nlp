\documentclass{article}

% Recommended, but optional, packages for figures and better typesetting:
\usepackage{microtype}
\usepackage{graphicx}
\usepackage{subcaption}
\usepackage{amsmath}
\usepackage{booktabs} % for professional tables

\usepackage{hyperref}

\usepackage{algorithmic}
\usepackage{enumitem}
% \usepackage{algpseudocode}

\newcommand{\theHalgorithm}{\arabic{algorithm}}

% If accepted, instead use the following line for the camera-ready submission:
\usepackage[accepted]{icml2025}

% For theorems and such
\usepackage{amsmath}
\usepackage{amssymb}
\usepackage{mathtools}
\usepackage{amsthm}
\usepackage{physics}

% if you use cleveref..
\usepackage[capitalize,noabbrev]{cleveref}

%%%%%%%%%%%%%%%%%%%%%%%%%%%%%%%%
% THEOREMS
%%%%%%%%%%%%%%%%%%%%%%%%%%%%%%%%
\theoremstyle{plain}
\newtheorem{theorem}{Theorem}[section]
\newtheorem{proposition}[theorem]{Proposition}
\newtheorem{lemma}[theorem]{Lemma}
\newtheorem{corollary}[theorem]{Corollary}
\theoremstyle{definition}
\newtheorem{definition}[theorem]{Definition}
\newtheorem{assumption}[theorem]{Assumption}
\theoremstyle{remark}
\newtheorem{remark}[theorem]{Remark}


\newcommand{\rev}{\mathsf{rev}}
\usepackage[textsize=tiny]{todonotes}

\usepackage{booktabs}
\usepackage{multirow}

\usepackage[np, autolanguage]{numprint}  % for controlling significant digits
\newcommand{\bftab}{\fontseries{b}\selectfont}
%% for evaluation metrics
\newcommand{\abbfid}{FID$\downarrow$}
\newcommand{\abbmiou}{mIoU$\uparrow$}
\newcommand{\abbalignment}{Ali$\rightarrow$}
\newcommand{\abboverlap}{Ove$\rightarrow$}
%% for task names (currently following layoutdiffusion)
\newcommand{\taskugen}{Unconditioned}
\newcommand{\tasktype}{Cond-C}
\newcommand{\tasktypesize}{Cond-CS}
\newcommand{\taskcompletion}{Completion}
\newcommand{\taskrefinement}{Refinement}
\newcommand{\redenoise}{ReDeNoise}

\icmltitlerunning{Interleaved Gibbs Diffusion for Constrained Generation}

\begin{document}

\twocolumn[
\icmltitle{Interleaved Gibbs Diffusion for Constrained Generation}
\icmlsetsymbol{equal}{*}

\begin{icmlauthorlist}
\icmlauthor{Gautham Govind Anil}{comp}
\icmlauthor{Sachin Yadav}{comp}
\icmlauthor{Dheeraj Nagaraj}{comp}
\icmlauthor{Karthikeyan Shanmugam}{comp}
\icmlauthor{Prateek Jain}{comp}
\end{icmlauthorlist}

\icmlaffiliation{comp}{Google DeepMind}

\icmlcorrespondingauthor{Gautham Govind Anil}{gauthamga@google.com}
\icmlcorrespondingauthor{Dheeraj Nagaraj}{dheerajnagaraj@google.com}

% You may provide any keywords that you
% find helpful for describing your paper; these are used to populate
% the "keywords" metadata in the PDF but will not be shown in the document
\icmlkeywords{Machine Learning, ICML}

\vskip 0.3in
]

% this must go after the closing bracket ] following \twocolumn[ ...

% This command actually creates the footnote in the first column
% listing the affiliations and the copyright notice.
% The command takes one argument, which is text to display at the start of the footnote.
% The \icmlEqualContribution command is standard text for equal contribution.
% Remove it (just {}) if you do not need this facility.

%\printAffiliationsAndNotice{}  % leave blank if no need to mention equal contribution
\printAffiliationsAndNotice{} % otherwise use the standard text.

\begin{abstract}

We introduce \ours, a novel framework for scene-level appearance transfer from a single style image to a real-world scene represented by multiple views. The method combines explicit semantic correspondences with multi-view consistency to achieve precise and coherent stylization.
Unlike conventional stylization methods that apply a reference style globally, \ours uses open-vocabulary segmentation to establish dense, instance-level correspondences between the style and real-world images. This ensures that each object is stylized with semantically matched textures.
\ours first transfers the style to a single view using a training-free semantic-attention mechanism in a diffusion model.
It then lifts the stylization to additional views via a learned warp-and-refine network guided by monocular depth and pixel-wise correspondences.
Experiments show that \ours consistently outperforms prior methods in structure preservation, perceptual style similarity, and multi-view coherence.
User studies further validate its ability to produce photo-realistic, semantically faithful results.
Our code, pretrained models, and dataset will be publicly released, to support new applications in interior design, virtual staging, and 3D-consistent stylization.

\end{abstract}



\begin{figure*}[t!]
    \centering
    \includegraphics[width=0.7\textwidth]{./Comparison.pdf}
    \caption{Comparison between conventional wireless system (left) and PASS (right).}
    \label{comparison}
    \vspace{-0.5cm}
\end{figure*} 

\section{Introduction} \label{sec:intro}

\IEEEPARstart{S}INCE Marconi demonstrated the feasibility of wireless communication in the late 19th century, the technology has undergone significant evolution and remarkable transformations. To address the unpredictable and dynamic nature of wireless channels, numerous advancements have been made in the air interface design, channel coding, source compression, and communication protocols for improving data rates and enhancing reliability. Among these advancements, multiple-input multiple-output (MIMO) has been one of the most important evolutionary techniques for wireless communication over the past few decades. By exploiting antenna arrays, MIMO brings about multiple benefits, such as enhanced signal strength through beamforming, mitigation of multi-path fading, and efficient spatial-domain multiplexing of users~\cite{bjornson2023twenty}. Since the advent of the third generation (3G) system, MIMO has been a fundamental component of wireless communication standards. However, during that era, the size of antenna arrays in MIMO systems was generally limited. The breakthrough came when Marzetta demonstrated the significant benefits of deploying an infinite number of antennas in 2010~\cite{marzetta2010noncooperative}, revealing the potential of MIMO to enhance communication performance while reducing system complexity. This revelation paved the way for the concept of massive MIMO, i.e., employing large-scale antenna arrays at base stations. Over time, massive MIMO has evolved into a key research focus and has become a reality with the deployment of 5G networks. 


However, massive MIMO has faced numerous challenges, as it is expected to transition from “Massive” in 5G (typically with 32-64 antennas) to “Gigantic” in 6G~\cite{Xtext, bjornson2024enabling}, where the number of antennas is expected to scale to hundreds or even thousands. One of the key obstacles is the complexity and cost of implementing massive MIMO since each antenna typically needs to be fed by a dedicated radio-frequency (RF) chain. Exploiting low-resolution analog-to-digital converters in RF chains or hybrid analog-digital antenna arrays with a limited number of RF chains were common methods to address this challenge, especially in the millimeter-wave band~\cite{heath2016overview}. More recently, advancements in metamaterials have paved the way for new antenna technologies, exemplified by waveguide-fed metasurface antennas~\cite{smith2017analysis, shlezinger2021dynamic, di2024reconfigurable}, which facilitate the ultra-dense deployment of antenna elements at a significantly lower cost and making massive MIMO implementation more feasible.

Flexible-antenna technique is a new evolution of MIMO. Unlike massive MIMO focusing on enlarging the wireless channel dimension, the flexible-antenna technique focuses on enabling the reconfiguration of the wireless channel. One of the most well-known approaches in this domain is the reconfigurable intelligent surface (RIS) technique~\cite{huang2019reconfigurable, wu2019intelligent, mu2021simultaneously}. By deploying RIS between transceivers, the wireless channel can be intelligently reconfigured by adjusting the phase shifts of the signals reflected/refracted by the RIS. More recently, fluid antennas~\cite{new2024tutorial} and movable antennas~\cite{zhu2023movable} have emerged as promising flexible-antenna technologies. The fundamental concept behind these approaches is to implement antenna arrays where individual antenna elements can dynamically adjust their positions within a spatial region, thus creating favorable channel conditions to enhance communication performance. 

Nevertheless, as shown on the left of Fig. \ref{comparison}, both massive MIMO and flexible-antenna techniques have limited capability in fundamentally addressing free-space pathloss and line-of-sight (LoS) blockage, two major causes of signal attenuation in wireless communications. While massive MIMO can achieve high beamforming gains to strengthen signals, it cannot combat LoS blockage and to effectively mitigate free-space pathloss, particularly for cell-edge users. RISs have been considered as a promising solution to overcome LoS blockage by creating virtual LoS paths. However, the double fading effect caused by signal reflection results in much higher pathloss compared to a direct LoS channel~\cite{ozdogan2019intelligent}. Additionally, fluid and movable antennas are typically capable of adjusting their positions only within a few wavelengths, making them more effective for mitigating small-scale fading rather than addressing large-scale pathloss. It is worthy to point out that all the aforementioned MIMO systems are lack of antenna array reconfigurability, i.e., once an antenna array is built, adding or removing antennas is no longer possible.

Pinching-Antenna SyStem (PASS) is a revolutionary technique for addressing the challenges of free-space pathloss and LoS blockage encountered by conventional multi-antenna technologies. This technique was originally proposed and prototyped by NTT DOCOMO in 2022~\cite{suzuki2022pinching}. As illustrated on the right of Fig. \ref{comparison}, PASS employs a dielectric waveguide as its primary transmission medium, which is known for its exceptionally low propagation loss (e.g., 0.01 dB/m \cite{pozar2021microwave}). By pinching a small separated dielectric element, referred to as a \emph{pinching antenna}, onto the waveguide, the system enables signal emission from the waveguide into the pinching antenna, which then radiates the signal into free space. Building on this principle, waveguides can be pre-deployed to extend service coverage, allowing pinching antennas to be placed at positions close to users. This strategic placement transforms the wireless system into a \emph{near-wired} system and hence establishes strong LoS links with users, effectively minimizing free-space path loss and mitigating blockage issues. Additionally, unlike existing MIMO systems, PASS allows both the number and positions of pinching antennas to be easily adjusted by simply pinching them to or releasing them from the waveguide~\cite{suzuki2022pinching}. This feature provides a low-cost and scalable approach to implementing MIMO while also facilitating the so-called \emph{pinching beamforming}, which enhances communication performance by dynamically optimizing antenna positions \cite{liu2025pinching}.

Given the successful prototyping of PASS by NTT DOCOMO, theoretical research on this topic has been steadily growing, though it remains in its early stages. The first theoretical study on PASS for the communication system design was presented in \cite{ding2024flexible}, where the authors provided a comprehensive analysis and developed low-complexity pinching beamforming designs for fundamental single-user and two-user scenarios. The array gain achieved by multiple pinching antennas on a waveguide was analyzed in \cite{ouyang2025array}, unveiling the optimal number of antennas and their spacing for maximizing the beamforming gain. 
% The authors of \cite{tegos2024minimum} studied an uplink PASS system and proposed an iterative antenna position optimization algorithm to maximize the sum rate under perfect phase alignment conditions. In \cite{wang2024antenna}, the authors investigated a downlink PASS system and introduced a matching theory-based optimization method for activating pinching antennas at preconfigured discrete positions. Their findings also highlighted the advantages of using non-orthogonal multiple access (NOMA) in PASS. Expanding on this,
The authors of \cite{bereyhi2025downlink} explored a downlink PASS architecture utilizing multiple waveguides, each equipped with a single pinching antenna, and proposed a greedy approach for jointly optimizing the transmit and pinching beamforming. Meanwhile, \cite{guo2025deep} examined a more generalized scenario, where multiple pinching antennas were deployed on each waveguide, and introduced a graph neural network (GNN)-based deep learning method to address the corresponding joint beamforming optimization problem.

Although PASS has attracted growing attention, several key challenges remain unsolved. On the one hand, the physics modeling of PASS is still underdeveloped, which is crucial for establishing an accurate signal model. In existing studies \cite{ding2024flexible, ouyang2025array, bereyhi2025downlink, guo2025deep}, it is commonly assumed that all signal power within the waveguide is fully radiated into free space and that each pinching antenna on a waveguide emits identical radiation power—an assumption analogous to conventional MIMO systems. However, pinching antennas operate fundamentally differently from traditional electronic antennas, and such assumptions may lack a solid physical foundation and fail to accurately reflect real-world behaviors. On the other hand, most existing works design PASS under simplified assumptions \cite{ding2024flexible, ouyang2025array, bereyhi2025downlink}, such as a single user, a single waveguide, a single pinching antenna per waveguide, or perfectly aligned signal phases. Although the GNN-based deep learning model proposed in \cite{guo2025deep} is capable of handling more complex scenarios with arbitrary numbers of users, waveguides, and pinching antennas, it suffers from a key limitation: the model parameters need to be retrained once the system configuration changes, limiting its generalization ability. Motivated by these challenges, this paper aims to develop a fundamental physics-based signal model for PASS and explore joint beamforming designs for more general scenarios. The key contributions of this work are summarized as follows:
\begin{itemize}
    \item We propose a physics-based hardware model for PASS, in which a pinching antenna is modeled as an open-ended directional waveguide coupler to facilitate the adjustment of radiation characteristics and simplify signal modeling. Based on this model, we characterize the relationship between the electromagnetic (EM) fields within the waveguide and those radiated by the pinching antennas using coupled-mode theory.
    \item We derive a novel signal model for PASS based on the proposed physics framework, revealing the inherent coupling effect between the radiation power of pinching antennas deployed on the same waveguide. Leveraging this coupling relationship, we introduce two simplified power models and their respective implementation methods: equal power and proportional power models.
    \item We formulate a joint transmit and pinching beamforming optimization problem to minimize the transmit power in a general PASS system with arbitrary numbers of users, waveguides, and pinching antennas, considering both continuous and discrete activation of pinching antennas. To solve this highly nonconvex, coupled, and multimodal optimization problem, we propose two algorithms: the penalty-based alternating optimization algorithm and the zero-forcing (ZF)-based low-complexity algorithm.
    \item We provide comprehensive numerical results to validate the advantages of PASS and the effectiveness of the proposed algorithm. The results demonstrate that 1) the ZF-based algorithm delivers performance comparable to the penalty-based algorithm but has a low complexity, 2) PASS significantly reduces transmit power, achieving a reduction of over 95\% compared to conventional and massive MIMO, 3) a dense set of available antenna positions is required for discrete activation to achieve similar performance to continuous activation, and 4) the proportional power model exhibits performance comparable to the equal power model.
\end{itemize}

The rest of this paper is structured as follows. Section \ref{sec:model} introduces the proposed physics-based hardware model and signal model for PASS. Section \ref{sec:beamforing} presents the general system model for downlink PASS and introduces a penalty-based alternating optimization method and a ZF-based algorithm for solving the joint beamforming optimization problem. Numerical evaluations and performance comparisons under various system configurations are presented in Section \ref{sec:results}. Finally, Section \ref{sec:conclusion} summarizes the findings and concludes the paper.


\emph{Notations:} Scalars are denoted using regular typeface, vectors and matrices are represented in boldface, and Euclidean subspaces are indicated with calligraphic letters. The set of complex and real numbers are denoted by $\mathbb{C}$ and $\mathbb{R}$, respectively. The inverse, conjugate, transpose, conjugate transpose, and trace operators are denoted by $(\cdot)^{-1}$, $(\cdot)^*$, $(\cdot)^T$, $(\cdot)^H$, and $\mathrm{tr}(\cdot)$, respectively. The absolute value, Euclidean norm, Frobenius norm, and maximum norm are denoted by $|\cdot|$, $\|\cdot\|$, $\|\cdot\|_F$, and $\|\cdot\|_\infty$ respectively. The real part of a complex number of demoted by $\Re \{\cdot\}$. The entry in the $n$-th row and $m$-th column of a matrix $\mathbf{X}$ is denoted by $[\mathbf{X}]_{n,m}$. An identity matrix of dimension $N \times N$ is denoted by $\mathbf{I}_N$. The big-O notation is given by $O(\cdot)$. A diagonal matrix with diagonal entries $x_1,\dots,x_N$ is denoted as $\mathrm{diag}(x_1,\dots,x_N)$.    




% \begin{figure*}[t!]
% \centering
% \begin{subfigure}[t]{0.48\textwidth}
%     \centering
%     \includegraphics[height=0.5\textwidth]{./Comparison_conventional.pdf}
% \end{subfigure}
% \hspace{-1.5cm}
% \begin{subfigure}[t]{0.48\textwidth}
%     \centering
%     \includegraphics[height=0.5\textwidth]{./Comparison_PASSpdf.pdf}
% \end{subfigure}
% \caption{Comparison between conventional wireless system (left) and PASS (right).}
% \end{figure*} 



\section{Preliminaries}
\label{sec:prelim}

\paragraph{Notation}

Let $\mathcal{X}$ be a finite set, let $L$ be the sequence length such that $L = L_1 + L_2$, $L_1, L_2 \in \mathbb{N}\cup\{0\}$. Let $d_{L_1+1},\dots,d_{L} \in \mathbb{N}$ be the continuous dimensions. We let our state space to be $\mathcal{S}_L = \mathcal{X}^{L_1}\times_{i=L_1+1}^{L}\mathbb{R}^{d_i}$. The elements of this set can be represented as a tuple/sequence of length $L$. For any $s \in \mathcal{S}_L$, let $s_i$ denote the element in $s$ at position $i$ in the tuple. Note that, $s_i$ is a discrete token from the set $\mathcal{X}$ if $i \leq L_1$ and it is a continuous vector sampled from $\mathbb{R}^{d_{i}}$ if $L_1 < i \leq L$. Let $s_{-i}$ denote the tuple of length $L-1$ obtained by removing the element at the $i^\text{th}$ position of $s$.

\paragraph{Problem Setup} Given samples $s_1,\dots,s_N$ from the target distribution $\pi$ over $\mathcal{S}_L$, the task is to learn a model which can generate more samples approximately from $\pi$. We will call $\mathcal{D} = \{s_1,\dots,s_N\}$ to be the dataset. 

\section{Interleaved Gibbs Diffusion}
\label{sec:igd}
We now describe the Interleaved Gibbs Diffusion (IGD) framework for sampling from a target distribtuion $\pi$ over $ \mathcal{S}_L$, given access to discrete and continuous denoisers which satisfy certain properties. We first describe the forward noising process and then the reverse denoising process using the given denoisers. In IGD, both the forward noising and reverse denoising processes operate one element at a time. 
Our noising process is illustrated in Figure \ref{fig:noising_process}.

\begin{figure}
    %\centering
    \includegraphics[width = 0.5 \textwidth]{images/figure_noising_process.pdf}
    \caption{\textbf{Interleaved Noising Process: } Sequential noising of discrete tokens ($D_s$) and continuous vectors ($C_s$). Noising occurs one element at a time, keeping other elements unchanged.}
    \label{fig:noising_process}
\end{figure}

\subsection{Forward Noising Process} \label{subsec:fnp}
The forward noising process takes a sample $s$ from the target distribution $\pi$ and applies a discrete time Markov chain to obtain the trajectory $s^{(0)},s^{(1)},\dots,s^{(T)}$, where $T$ is the total number of timesteps. We refer to $t$ as the \textit{sequence time}. Note that $s^{(0)} = s$. For each $t$, we choose a position $i_t \in \{1, 2, \dots, L \}$ to be noised at sequence time $t$. In this work, we choose $i_t$ in a round-robin fashion from some permutation of $\{1, 2, \dots, L \}$ so that all positions are noised exactly once after every $L$ sequence timesteps; we call this permutation the \textit{interleaving pattern}. Given $i_t$, the corresponding sequence element $s_{i_t}$ can either be discrete or continuous, based on which we either perform either discrete noising or continuous noising.

\paragraph{Discrete Noising}
If $s_{i_t}$ is discrete (i.e, $i_t \leq L_1$), following \cite{varma2024glauber}, we consider token $\phi \notin \mathcal{X}$ and define a probability distribution $\Pi_t$ over $\mathcal{X} \cup \{\phi\}$. Note that $\Pi_t$ depends on the sequence time $t$. We refer to $\Pi_t$ as the discrete noise schedule. Then the discrete noising process is as follows:

Sample $z_t \sim \Pi_t$ independent of $s^{(t)}$. Then we have:
\begin{align*}
    s^{(t+1)}_{j} = 
    \begin{cases}
    % s^{(t)}_{j},& \text{if } j \neq  i_t\\
    % s^{(t)}_{j},& \text{if } j =  i_t \text{  and  } z_t = \phi\\
    z_t,& \text{if } j =  i_t \text{  and  } z_t \neq \phi \\
    s^{(t)}_{j},& \text{otherwise }
    \end{cases}
\end{align*}

\paragraph{Continuous Noising}
If $s_{i_t}$ is continuous (i.e, $L_1 < i_t \leq L_2$), we use $m_{i_t}^t$ to denote the number of times position $i_t$ has been visited by sequence time $t$ (including the visit at $t$). Let $m = \max_{i_t} m_{i_t}^T $. Define $[\tilde{\beta}_j]_{j = 1}^{m}$ to be a monotonically increasing sequence, which we refer to as the continuous noise schedule.
Then, the continuous noising process is given by:
$
    s^{(t+1)}_{i_t} = \left(\sqrt{1 - \tilde{\beta}_{m_{i_t}^t}}\right) s^{(t)}_{i_t} + \left(\sqrt{\tilde{\beta}_{m_{i_t}^t}}\right) \epsilon^{(t)}
$
where $\epsilon^{(t)} \sim \mathcal{N}(0, \mathbf{I})$. Note that $s^{(t+1)}_{j} = s^{(t)}_{j} \quad \forall j \neq i_t$.

\begin{lemma}[Mild extension of Lemma $1$ in \cite{varma2024glauber}]
Denote the distribution of $s^{(t)}$ by $P_t$. Suppose $\Pi_t\left(\cdot | \mathcal{X} \right) = \Pi\left(\cdot | \mathcal{X} \right)$ for all $t$, $\Pi_t(\phi) \leq 1 - \epsilon$ for some $\epsilon > 0$ and $\lim_{T \rightarrow \infty} \sum_{j} \log \left(1- \tilde{\beta}_j\right) = -\infty$. As $T \rightarrow \infty$, $P_T$ converges to the product distribution: $\Pi\left( \cdot| \mathcal{X} \right)^{L_1} \times_{i=L_1 + 1}^{L} \mathcal{N}\left(0, \mathbf{I}_{d_i} \right)$ .
\label{lemma:for_process}
\end{lemma}

\paragraph{Co-ordinate wise independent noising:}

The noising process of any element $s_{i_t}^{(t)}$ at any time $t$ is independent of other elements; this allows us to sample $s^{(t)}$ at any time $t$ directly from $s^{(0)}$ without having to compute $s^{(1)}, s^{(2)}, ..., s^{(t)}$ sequentially (Algorithm given in Appendix \ref{app:sec:fwd_prcs}).

\subsection{Reverse Denoising Process}

The reverse denoising process takes a sample $\hat{s}^{(T)}$ from $P_T$ 
% $\Pi\left( \cdot| \mathcal{X} \right)^{\otimes L_1} \otimes \prod_{i = 1}^{L_2} \mathcal{N}\left(0, \mathbf{I}_{d_i} \right)$
as the input and applies a discrete time Markov chain to obtain the trajectory $\hat{s}^{(T)},\hat{s}^{(T-1)},\dots,\hat{s}^{(0)}$, where $T$ is the total number of sequence timesteps.  Recall that $i_t$ denotes the position which was noised at time $t$ during the forward process.\\

Given $\hat{s}^{(t+1)}$, we set $\hat{s}^{(t)}_{-i_t} = \hat{s}^{(t+1)}_{-i_t}$. Depending on whether $\hat{s}^{(t+1)}_{i_t}$ is discrete (resp. continuous) we use the discrete denoiser (resp. continuous denoiser) to sample $\hat{s}^{(t)}_{i_t}$ (${s}^{(t+1)}$ is the sample from the forward process at time $t+1$):

\textbf{Discrete Denoiser} is a (learned) sampling algorithm which can sample from $\hat{P}_{t,i_t}(\cdot|s)$, a probability distribution over $\mathcal{X}$ given $s$ as the input. $\hat{P}_{t,i_t}(\cdot|s = \hat{s}^{(t+1)})$ approximates one of the following:

$\mathbb{P}(s^{(t)}_{i_t} = \cdot| s^{(t+1)}_{-i_t} = \hat{s}^{(t+1)}_{-i_t})$ or $\mathbb{P}(s^{(t)}_{i_t} = \cdot| s^{(t+1)} = \hat{s}^{(t+1)})$

\textbf{Discrete Denoising Step:} $\texttt{DiscDen}(\hat{s}^{(t)},i_t,t)$ outputs a sample $\hat{s}^{(t)}_{i_t} \sim \hat{P}_{t,i_t}\left( \cdot | s = \hat{s}^{(t+1)} \right)$.


\textbf{Continuous Denoiser} is a (learned) sampling algorithm which can sample from the distribution $\hat{P}_{t,i_t}(\cdot|s)$ over $\mathbb{R}^{d_{i_t}}$ given $\hat{s}^{(t+1)}$ as the input. $\hat{P}_{t,i_t}(\cdot|s = \hat{s}^{(t+1)})$ approximates the conditional distribution $\mathbb{P}(s^{(t)}_{i_t} = \cdot| s^{(t+1)} = \hat{s}^{(t+1)})$.

\textbf{Continuous Denoising Step:} $\texttt{ContDen}(\hat{s}^{(t)}_{i_t},i_t,t)$ outputs a sample $\hat{s}^{(t)}_{i_t} \sim \hat{P}_{t,i_t}\left( \cdot |s =  \hat{s}^{(t+1)} \right)$.


\begin{lemma}
Assume $\hat{s}^{(T)} \sim P_T$ and assume we have access to ideal discrete and continuous denoisers. Then, $\hat{s}^{(0)}$ obtained after $T$ steps of reverse denoising process, will be such that $\hat{s}^{(0)} \sim \pi$.
\label{lemma:rev_process}
\end{lemma}

From the definition of the discrete and continuous denoisers, it is clear that unlike the forward process, the reverse process is \textit{not factorizable}. However, by sacrificing factorizability, we are able to achieve \textit{exact reversal of the forward process}, provided we have access to ideal denoisers. The denoising algorithm is detailed in Algorithm \ref{alg:framework}.

\begin{algorithm}[ht]
\begin{algorithmic}
\INPUT { $\hat{s}^{T} \sim P_T$, {discrete denoiser} \texttt{DiscDen} , {continuous denoiser} \texttt{ContDen}, noise positions $\{i_t\}$}
\OUTPUT {$\hat{s}^{0} \sim {\pi}$}

\FOR{$t \in [T, T-1, \dots, 1]$}
 \STATE $\hat{s}^{(t-1)}_{-i_t} = \hat{s}^{(t)}_{-i_t}$
 \IF {$\hat{s}^{(t)}_{i_t}$ is discrete}
    \STATE  $\hat{s}^{(t-1)}_{i_t} = \texttt{DiscDen}(\hat{s}^{(t)},i_t,t)$
 \ELSE
    \STATE  $\hat{s}^{(t-1)}_{i_t} = \texttt{ContDen}(\hat{s}^{(t)},i_t,t)$ 
  \ENDIF
\ENDFOR
\end{algorithmic}
\caption{Interleaved Gibbs Diffusion: Ideal Denoising}
\label{alg:framework}
\end{algorithm}

\subsection{\redenoise~Algorithm}
\label{sec:redenoise}
Inspired by \cite{meng2021sdedit}, we now propose a simple but effective mechanism for quality improvements at inference time. Given a sample obtained through complete reverse process $\hat{s}^{(0)}$, we now repeat the following two steps $N_R$ times: (1) Noise $\hat{s}^{(0)}$ for $T_R$ rounds to obtain $\hat{s}^{(T_R)}$. (2) Denoise $\hat{s}^{(T_R)}$ back to $\hat{s}^{(0)}$.
While $N_R$ decides the number of times the noise-denoise process is repeated, $T_R$ decides how much noising is done each time. These are hyperparameters which can be tuned to suit the task at hand.

\subsection{Conditional Generation}
\label{sec:cond_generation}
We train the model for conditional generation - i.e, generate a subset of the co-ordinates conditioned on the rest. We adopt the state-space doubling strategy, inspired by \cite{levi2023dlt}. A binary mask vector is created indicating whether each element in the sequence is part of the conditioning or not; for vectors in $\mathbb{R}^d$, a mask is created for each element in the vector. The mask is now embedded/projected and added to the discrete/continuous embedding and fed into the model while training. Further, during the forward and reverse processes, the conditioned elements are not noised/denoised.

\section{Training the Denoisers}
\label{sec:training}



Having established the IGD framework, we now describe strategies to train the discrete and continuous denoisers, which have been black boxes in our discussion so far.  

\subsection{Training the Discrete Denoiser}

 Throughout this subsection, we use $g_{\theta}$ to denote a parameterised neural network which is trained to be the discrete denoiser. $g_{\theta}$ takes input from the space $\mathcal{S}_{L} \times \{0, 1, \dots, T-1 \}$ and outputs logits in the space $[0, 1]^{\abs{\mathcal{X}}}$.  We now describe two strategies to train $g_{\theta}$:

\subsubsection{$\abs{\mathcal{X}}$-ary classification}
In this approach, the objective is to learn $\mathbb{P}\left({s}^{(t)}_{i_t} = \cdot | {s}^{(t+1)} = \hat{s}^{(t)} \right)$. So, we directly train the model to predict ${s}^{(t)}_{i_t}$ given ${s}^{(t+1)}$. Since there are $\abs{\mathcal{X}}$ discrete tokens in the vocabulary, this is a $\abs{\mathcal{X}}$-ary classification problem, where the input is ${s}^{(t+1)}$ and the corresponding label is ${s}^{(t)}_{i_t}$. Hence, we minimize the cross-entropy loss: $ \mathcal{L}_{CE}\left(\theta; {s}^{(t+1)}, t \right) = -\log \left( g_{\theta} ^{s_{i_t}} \left({s}^{(t+1)}, t \right) \right)  $
where $g^{s_{i_t}}_{\theta}(\cdot) $ denotes the logit corresponding to token ${s}^{(t)}_{i_t}$.

\subsubsection{Binary classification}
\label{subsec:model_train_binary}
In this approach, the objective is to learn $\mathbb{P}\left({s}^{(t)}_{i_t} = \cdot | {s}^{(t+1)}_{-i_t} = \hat{s}^{(t)}_{-i_t} \right)$. We adapt Lemma 3.1 from \cite{varma2024glauber} to simplify this objective:

\begin{lemma}
Let $s \in \mathcal{S}^L$. Then, for $x \in \mathcal{X}$ and discrete ${s}^{(t)}_{i_t}$, we can write $\mathbb{P}\left({s}^{(t)}_{i_t} = x | {s}^{(t+1)}_{-i_t} = {s}_{-i_t} \right)$ as :
\small
\begin{equation*}
     \frac{\mathbb{P}(z_t = x)}{\mathbb{P}(z_t = \phi)} \left( \frac{1}{\mathbb{P}\left({z_t} = x | {s}^{(t+1)}_{-i_t} = {s}_{-i_t}, {s}^{(t+1)}_{i_t} = x \right)}  - 1 \right)
\end{equation*}
\normalsize
 where $\left(s^{(0)},  \dots s^{(T)}, \right)$ is obtained from forward process.
\end{lemma}

Hence, it is sufficient for the model to learn $\mathbb{P}\left({z_t} = x | {s}^{(t+1)}_{-i_t} = {s}_{-i_t}, {s}^{(t+1)}_{i_t} = x \right)$ for all $x \in \mathcal{X}$. This can be formulated as a binary classification task: Given ${s}^{(t+1)}_{-i_t}$ and ${s}^{(t+1)}_{i_t} = x$ as the input, predict whether $z_t = x$ or $z_t = \phi$. Hence, we minimize the binary cross-entropy loss: $
    \mathcal{L}_{BCE}\left(\theta; {s}^{(t+1)}_{-i_t}, t \right) = -  \mathbf{1}_{z_t \neq \phi}  \log \left( g_{\theta} ^{x} \left({s}^{(t+1)}_{-i_t}, t \right) \right) 
    - \mathbf{1}_{z_t = \phi}  \log \left(1 - g_{\theta} ^{x} \left({s}^{(t+1)}_{-i_t}, t \right) \right)  $
\normalsize
where $g^{x}_{\theta}(\cdot) $ denotes the logit corresponding to token $x$.

Preliminary experiments (Appendix \ref{app:par_xary_binary}) gave better results with the binary classification loss; hence we use binary classification for training the discrete denoiser.


\subsection{Training the Continuous Denoiser}
\label{subsec:con_den}

In continuous diffusion, the noising (and denoising) process follows an SDE; the entire process happens in an uninterrupted fashion. However, in IGD, the noising and denoising happen with interruptions, because of the sequential nature. Thus, in the reverse process, the conditioning surrounding a continuous element changes every time it is picked for denoising. By an adaptation of the standard Tweedie's formula and exploiting the fact that forward noising process for every element is independent of other elements, we show that using the current conditioning and estimating the cumulative noise added from the beginning (across interruptions) still reverses the continuous elements in an interleaved manner. This is the novelty behind Lemma \ref{lemma:score_noise}.


Suppose we are given a sample $s^{(t)}$ from the distribution at time $t$. Let $\stackrel{d}{=}$ denote equality in distribution. Suppose $x_0 = s_{i_t}^{(t)} \in \mathbb{R}^{d_{i_t}}$ and consider the Orstein-Uhlenbeck Process $dx_\tau = -x_\tau d\tau + \sqrt{2}dB_\tau$ with standard Brownian motion $B_\tau$. Then $x_{\tau_0}|s^{(t)} \stackrel{d}{=} s_{i_t}^{(t+1)}|s^{(t)}$ whenever $\tau_0 = \frac{1}{2}\log(\tfrac{1}{1-\beta_{m_{i_t}^t}})$. Based on the observations in \cite{song2020score,ho2020denoising}, the reverse SDE given by 
\begin{equation}\label{eq:rev_sde}
    x_\tau^{\mathsf{rev}} = x_\tau^{\mathsf{rev}}d\tau + 2 \nabla \log q_{\tau_0-\tau}(x^{\rev}_{\tau}|s_{-i_t}^{(t)})\tau + \sqrt{2}dB_\tau
\end{equation} is such that if $x^{\rev}_0 = s_{i_t}^{(t+1)}$ then $x^{\rev}_{\tau_0}|s^{(t+1)} \stackrel{d}{=} s_{i_t}^{(t)}|s^{(t+1)}$ where
$q_{\tau}(\cdot|s^{(t+1)})$ is the conditional density function of $x_{\tau}$. We use DDPM \cite{ho2020denoising} to sample from $\mathbb{P}(s_{i_t}^{(t)} = \cdot|s^{(t+1)})$ by learning the score function $\nabla \log q_{\tau}(\cdot|s_{-i_t}^{(t+1)})$ and then discretizing the reverse SDE in Equation~\eqref{eq:rev_sde}.

To obtain a more precise discretization, we divide the noising at sequence timestep $t$ into $K_{i_t}^t$ \textit{element timesteps} (whenever $s_{i_t}$ is a continuous vector). We define $s_{i_t}^{(t, 0)} = s_{i_t}^{(t)}$, $s_{i_t}^{(t, K_{i_t}^{t})} = s_{i_t}^{(t+1)}$, and for $k \in [0, 1, \dots, K_{i_t}^{t}-1]$:
\small
\begin{align*}
    s^{(t, k+1)}_{i_t} \sim \mathcal{N}\left(\left(\sqrt{1 - {\beta}(t, k)}\right) s^{(t, k)}_{i_t}, \left({{\beta}(t, k)}\right) \mathbf{I}  \right)
\end{align*}
\normalsize
where ${\beta}$ is a continuous noise schedule which outputs a scalar given $(t, k)$ as input. Following the popular DDPM \cite{ho2020denoising} framework, we rewrite the noising process as:
\small
\begin{align}\label{eq:diff_relation}
    s^{(t, k+1)}_{i_t} = \left(\sqrt{\bar{\alpha}(t, k)}\right) s^{(0)}_{i_t} + \left(\sqrt{1 - \bar{\alpha}(t, k)}\right) \epsilon
\end{align}
\normalsize
where $\epsilon \sim \mathcal{N}(0, \mathbf{I})$ and $\bar{\alpha}$ is a cumulative noise schedule obtained from $\hat{\beta}$. The exact relations between $\tilde{\beta}$ (defined in \ref{subsec:fnp}), ${\beta}$ and $\bar{\alpha}$ are given in Appendix \ref{app:sec:beta_connection}. With this discretization, the reverse process becomes:
\small
\begin{align*}
    \hat{s}^{(t, k)}_{i_t} = \frac{\left(\hat{s}^{(t, k+1)}_{i_t} - {\beta}(t, k+1)p({s}^{(t, k+1)}) \right)}{\sqrt{1 - {\beta}(t, k+1)}} 
    + \sqrt{{\beta}(t, k+1)} \epsilon'
\end{align*}
\normalsize
where $\epsilon' \in \mathcal{N}(0, \mathbf{I})$ and  $p({s}^{(t, k+1)})$ is the  score function $\nabla_{{s}_{i_t}^{(t, k+1)}} \log q({s}_{i_t}^{(t, k+1)} | {s}_{-i_t}^{(t, k+1)} )$. Now to learn the score function, we use the following Lemma:
\begin{lemma}
Under the considered forward process where noising occurs independently, we have:
 \begin{align*}
     \nabla_{{s}_{i_t}^{(t, k+1)}} \log q({s}_{i_t}^{(t, k+1)} | {s}_{-i_t}^{(t, k+1)} )  = -\frac{1}{\sqrt{1-\bar{\alpha}}} \mathbb{E} \left[ \epsilon | {s}^{(t,k+1)} \right]
 \end{align*}
\label{lemma:score_noise}
\end{lemma}
Hence, if we can learn $\mathbb{E} \left[ \epsilon | {s}^{(t,k+1)} \right]$ exactly, the forward process can be reversed exactly assuming you start from the stationary distribution. Hence, we minimize the regression loss: $\norm{\epsilon - g\left({s}^{(t, k+1)}, t, k \right)}_2^2$
where $g(\cdot)$ is a neural network which is trained to predict  $\epsilon $ given $\left(\hat{s}^{(t, k+1)}, t, k \right)$. 

Apart from DDPM sampling, we also evaluated DDIM, which is an ODE based method. However, preliminary results (reported in Appendix \ref{app:par:ddpm_ddim}) indicated that DDPM performs better. A detailed description of the exact training and inference algorithms we use is given in Appendix \ref{app:sec:model_train_pseudo}.

\section{Model Architecture}
Inspired by \cite{peebles2023scalable}, we use a transformer-based architecture closely resembling Diffusion Transformers (DiTs) for the model. Since DiT has been designed for handling discrete tokens, we modify the architecture slightly to accommodate continuous vectors as well. However, we keep modifications to a minimum, so that the proposed architecture can still benefit from DiT design principles. Our proposed architecture, which we refer to as Discrete-Continuous (Dis-Co) DiT, is illustrated in Figure \ref{fig:gen_dit}.

Figure \ref{fig:disco_network} gives a high-level overview of the model with $N$ Dis-Co DiT blocks stacked on top of each other. Discrete embeddings, continuous projections and their corresponding time embeddings are passed into the Dis-Co DiT blocks. Figure \ref{fig:disco_block} details the structure of a single Dis-Co DiT block. The discrete embeddings and continuous vectors are processed as in a regular transformer block; however the discrete and continuous time information ($(t,k)$ variables) is incorporated using adaptive layer normalization \cite{xu2019adaptivelayernormalization}. Exact details are given in Appendix \ref{app:sec:model_arch}. 
\sloppy
\begin{figure*}
    \begin{tabular}[c]{lr}
    \begin{subfigure}[c]{0.45\textwidth}
      \includegraphics[width=1.1\textwidth]{images/DisCo_DiT.pdf}
      \caption{}
      \label{fig:disco_network}
    \end{subfigure}&
    \begin{subfigure}[c]{0.45\textwidth}
      \includegraphics[width=1.1\textwidth]{images/DisCo_DiT_Block.pdf}
      \caption{}
      \label{fig:disco_block}
    \end{subfigure}
  \end{tabular} 
  \caption{\textbf{Dis-Co DiT Architecture:} (a) illustrates overall architecture, with both discrete and continuous inputs and outputs (b) shows detailed architecture of a single block, where time information is incorporated through adaptive layer normalization.}
  \label{fig:gen_dit}
\end{figure*}
\section{Experiments}
\label{sec:experiments}

We evaluate the IGD framework on three different tasks: Layout Generation, Molecule Generation and the Boolean Satisfiability problem. While the first two tasks involve generating both discrete tokens and continuous vectors, 3SAT involves only discrete tokens. Nevertheless, all three problems are constrained generation problems and can hence benefit from the exact reversal of IGD framework. 

\subsection{Layout Generation}

\subsubsection{Background}
Layout generation aims to generate coherent arrangements of UI elements (e.g., buttons, text blocks) or document components
(e.g., titles, figures, tables) that satisfy both functional requirements and aesthetic principles. This problem is important 
in various applications of graphic design and interface prototyping.

Formally, each layout is a set of $N$ elements $\{ e_i \}_{i=1}^N$. Each element $e_i$ is represented by a discrete category $t_i \in \mathbb{N}$  and a continuous bounding box vector $\mathbf{p}_i \in \mathbb{R}^4$. We use the parameterization $\mathbf{p}_i = [x_i,\, y_i,\, l_i,\, w_i]^\top$, where $(x_i,\, y_i)$ represents the upper-left corner of the bounding box, and $(l_i,\, w_i)$ its length and width, respectively.

\subsubsection{Experimental Setup}
We adopt a setup similar to \cite{guerreiro2025layoutflow} for standardized comparison to existing layout generation methods.

\paragraph{Datasets:}
We evaluate our method on two popular layout generation datasets:
\\
 1. PubLayNet \cite{zhong2019publaynet}: Contains layouts of scientific documents annotated with 5 element categories.
\\
 2. RICO \cite{deka2017rico}: Provides user-interface (UI) layouts with 25 element categories.\\
Following prior works \cite{jiang2023layoutformer++, zhang2023layoutdiffusion}, layouts containing more than 20 elements are discarded from the datasets.

\paragraph{Evaluation metrics:}
\label{par:layout_gen_eval_metrics}
Following previous works \cite{inoue2023layoutdm, chen2024towards}, we evaluate our method primarily using two metrics described below:
\begin{enumerate}[noitemsep,topsep=0pt]
    \item Frechet Inception Distance (FID) \cite{heusel2017gans}: This metric measures the distance between the generated and real data distributions by comparing the features extracted from a neural network. For FID calculation, we use the feature space from the same network with identical weights as in \cite{zhang2023layoutdiffusion}.
    \item Maximum Intersection over Union (mIoU) \cite{kikuchi2021constrained}: This calculates the maximum IoU between bounding boxes of the generated layouts and real data layouts with the same element category. 
\end{enumerate}
Results on additional evaluation metrics (Alignment and Overlap) are presented in Appendix \ref{app:layout_gen_full_results}. Baseline metrics in Table \ref{tab:layout_results} are reported as given in \cite{guerreiro2025layoutflow}.

\textbf{{Tasks:}}
Results are presented on three common layout generation tasks:
\begin{enumerate}[noitemsep,topsep=0pt]
    \item Unconditional Generation: No constraints.
    \item Category-Conditioned Generation: Element categories are specified.
    \item Category\,+\,Size-Conditioned Generation: Both element categories and sizes are specified.
\end{enumerate}



\paragraph{Baselines:}
We compare with state-of-the-art methods: Diffusion-based approaches include: LayoutDM \cite{inoue2023layoutdm}, which applies discrete diffusion to handle element categories and positions; LayoutDiffusion \cite{zhang2023layoutdiffusion}, employing iterative refinement with tailored noise schedules for layout attributes; and DLT \cite{levi2023dlt}, a hybrid model separating element categories and coordinates into distinct diffusion processes. Flow-based: LayoutFlow \cite{guerreiro2025layoutflow} leverages trajectory learning for efficient sampling. Non-diffusion baselines comprise: LayoutTransformer \cite{gupta2021layouttransformer} (autoregressive sequence generation), LayoutFormer++ \cite{jiang2023layoutformer++} (serializes constraints into token sequences for conditional generation), and NDN-none \cite{lee2020neural} (adversarial training without constraints).


\subsubsection{Results}
Table \ref{tab:layout_results} presents quantitative results across different tasks and datasets. On RICO, we outperform all baselines in category-conditioned and category+size-conditioned generation, with competitive performance on unconditioned generation. On PubLayNet, we achieve the best FID in unconditioned and category+size-conditioned generation.

Notably, IGD outperforms DLT, a discrete-continuous diffusion model which assumes factorizability of the reverse process, on most of the tasks in both datasets by a significant margin. This further demonstrates the effectiveness of our framework in comparison to existing discrete-continuous diffusion models.
We also note that models such as LayoutDM and LayoutDiffusion employ specialized diffusion processes tailored for layout generation, whereas we directly employ our discrete-continuous diffusion framework without further modifications. We refer to Appendix \ref{app:layout_gen} for further implementation details, example generations as well as extensive ablations.

\begin{table*}[t]
    \centering
    \caption{\textbf{Layout Generation:} Quantitative results on the RICO and PubLayNet datasets. Refer to section \ref{par:layout_gen_eval_metrics} for details on evaluation tasks and metrics.}
    \label{tab:layout_results}
    \resizebox{2.00\columnwidth}{!}{
    \begin{tabular}{lrrrrrrrr} 
        \toprule
        \multicolumn{9}{c}{RICO} \\
        \midrule
         &  \multicolumn{2}{c}{\shortstack{Unconditioned}} && \multicolumn{2}{c}{\shortstack{Category\\Conditioned}} && \multicolumn{2}{c}{\shortstack{Category$+$Size\\Conditioned}} \\ 
      Method                                    &  \abbfid                & \abbmiou          & $\quad$ & \abbfid        & \abbmiou &$\quad$ & \abbfid & \abbmiou \\
        \midrule
       LayoutTransformer   & 24.32                       & 0.587             && -         & -  && - & - \\
         LayoutFormer\texttt{\char`+\char`+}           & 20.20                    & \bftab{0.634}     && 2.48               & 0.377 && - & - \\
         NDN-none & - & - && 13.76 & {0.350} && - & -  \\
        % \cmidrule{2-9} 
         LayoutDM                                     & 4.43                        & 0.582             && 2.39                & 0.341  && {1.76} &    0.424   \\
         DLT                                          & 13.02                     & 0.566             &&  6.64                       &  0.326 && 6.27 &  0.424 \\
         LayoutDiffusion                               & 2.49  & 0.620 && {1.56}    &  0.345   && - & - \\
         LayoutFlow                              & \bftab{2.37}           & 0.570             && 1.48      &  0.322 &&   1.03        & 0.470 \\ 
         \midrule
         Ours                              & {2.54}        & 0.594            &&  \bftab{1.06}      &  \bftab{0.385} &&  \bftab{0.96}       & \bftab{0.524}  \\ 
        \bottomrule
    \end{tabular}
    
\quad   
\begin{tabular}{lrrrrrrrr}
        \toprule
        \multicolumn{9}{c}{PubLayNet} \\
        \midrule
        &  \multicolumn{2}{c}{\shortstack[c]{Unconditioned}} && \multicolumn{2}{c}{\shortstack[c]{Category\\Conditioned}} && \multicolumn{2}{c}{\shortstack[c]{Category$+$Size\\Conditioned}} \\  
      Method                                    &  \abbfid                & \abbmiou          & $\quad$ & \abbfid        & \abbmiou &$\quad$ & \abbfid & \abbmiou \\
        \midrule
       LayoutTransformer   & 30.05                       & 0.359             && -         & -  && - & - \\
         LayoutFormer\texttt{\char`+\char`+}           & 47.08                    & 0.401     && 10.15               & 0.333 && - & - \\
         NDN-none & - & - && 35.67 & 0.310 && - & -  \\
         LayoutDM                                     & 36.85                        & 0.382             && 39.12                & 0.348  && 29.91 &    0.436   \\
         DLT            & 12.70         & \bftab{0.431}             &&  7.09                       &  0.349 && 5.35 &  0.426 \\
         LayoutDiffusion                               & 8.63  & 0.417 && 3.73    &  0.343   && - & - \\
         LayoutFlow                              & 8.87           &      0.424         &&       \bftab{3.66} & 0.350 &&   1.26        & 0.454 \\ 
         \midrule
         Ours                              & \bftab{8.32}        & 0.419            &&  4.08      &  \bftab{0.402} &&  \bftab{0.886}       & \bftab{0.553}  \\ 
        \bottomrule
    \end{tabular}
    }
\end{table*}
\subsection{Molecule Generation}

\subsubsection{Background}
Molecule generation aims to synthesize new valid molecular structures from a distribution learned through samples. Recently, generative models trained on large datasets of valid molecules have gained traction. In particular, diffusion-based methods have shown strong capabilities in generating discrete atomic types and their corresponding 3D positions.


We represent a molecule with $n$ atoms by $(z_i, \mathbf{p}_i)_{i=1}^{n}$, where $z_{i} \in \mathbb{N}$ is the atom's atomic number and $\mathbf{p}_i \in \mathbb{R}^{3}$ is the position. We focus on organic molecules with covalent bonding, where bond orders (single, double, triple, or no bond) between atoms are assigned using a distance-based lookup table following \cite{hoogeboom2022equivariant}.

\subsubsection{Experimental Setup}
We closely follow the methodology used in prior works \cite{hua2024mudiff} and \cite{hoogeboom2022equivariant} for 3D molecule generation. Further details are given below:

\paragraph{Datasets:}
We evaluate on the popular QM9 benchmark \cite{ramakrishnan2014quantum} which contains organic molecules with up to 29 atoms and their 3D coordinates. We adopt the standardized 100K/18K/13K train/val/test split to ensure fair comparison with prior works. We generate all atoms, including hydrogen, since this is a harder task.

\paragraph{Evaluation metrics:}
\label{par:mol_gen_eval_metrics}
We adopt four metrics following prior works \cite{hua2024mudiff} and \cite{hoogeboom2022equivariant}: 
\begin{enumerate}[noitemsep,topsep=0pt]
    \item Atom Stability: The fraction of atoms that satisfy their valency. Bond orders (single, double, triple, no bond) are determined from pairwise atomic distances using a distance-based lookup table given in \cite{hoogeboom2022equivariant}.
    \item Molecule Stability: Fraction of molecules where all atoms are stable.
    \item Validity: RDKit-based \cite{landrum2006rdkit} molecular sanitization checks, as in \cite{hoogeboom2022equivariant}. These checks include: chemical plausibility of bond angles and lengths, absence of disconnected components, kekulization of aromatic rings, and more.
    \item Uniqueness: Fraction of unique and valid molecules.
\end{enumerate}

\paragraph{Baselines:}
We compare with state-of-the-art methods: E-NF (equivariant normalizing flows) \cite{pmlr-v119-kohler20a} models molecular generation via invertible flow transformations. G-SchNet \cite{NIPS2019_8974} employs an autoregressive architecture with rotational invariance. Diffusion-based approaches include EDM \cite{hoogeboom2022equivariant} (with SE(3)-equivariant network \cite{fuchs2020se}) , GDM \cite{hoogeboom2022equivariant} (non-equivariant variant of EDM), and DiGress \cite{vignac2023digress} (discrete diffusion for atoms/bonds without 3D geometry). GeoLDM \cite{xu2023geometric} leverages an equivariant latent diffusion process, while MUDiff \cite{hua2024mudiff} unifies discrete (atoms/bonds) and continuous (positions) diffusion with specialized attention blocks. While \cite{peng2023moldiff} and \cite{vignac2023midi} are also diffusion based methods, they are not directly comparable due to reasons we list in \ref{app:subsec:baselines}.
% \sy{Mention about this complimentary thread with NN trained distance lookup for bonds}

\subsubsection{Results}
Table~\ref{tab:qm9_results} compares our method with others on QM9. Notably, \emph{without relying} on specialized equivariant diffusion or domain-specific architectures, IGD attains strong performance across four key metrics. Our model achieves 98.9\% atom stability and 95.4\% molecular validity, equaling or surpassing other methods. Our model achieves a molecule stability of 90.5\%, surpassing the baselines. While not the best in `uniqueness', our approach still yields more than 95\% unique samples among the valid molecules. In addition, we observe noticeably lower standard deviations than most baselines, reflecting consistent performance.

Notably, applying \emph{ReDeNoise} at inference yielded an improvement of 4.99\% in molecular stability. Further implementation details, and ablation studies examining design choices such as the interleaving pattern, discrete and continuous noise schedules are presented in Appendix~\ref{app:mol_gen}.


 \begin{table}[t]
     \vspace{-8pt}
    %\centering
    \caption{\textbf{Molecule Generation:} Quantitative results on QM9 benchmark.
    We report mean (standard deviation) across 3 runs, each with 10K generated samples. Refer to section \ref{par:mol_gen_eval_metrics} for details on evaluation metrics.}
    \label{tab:qm9_results}
        \scalebox{.7}{
    \begin{tabular}{l c c c c}
    \toprule
     Method  & Atom stable (\%) & Mol stable (\%) & Validity (\%) & Uniqueness (\%)\\
      \midrule
        {E-NF}  & 85.0 & 4.9 & 40.2 & 39.4 \\
       {G-Schnet}  & 95.7 & 68.1 & 85.5 & 80.3  \\
        % \midrule % Not sure about this but maybe midrule may help to distinguish them
        GDM  & 97.6 & 71.6 & 90.4 & 89.5  \\ 
        EDM  & ${98.7}\pm{0.1}$ & {82.0}$\pm{0.4}$ & 91.9 $\pm 0.5$ & 90.7 $\pm 0.6$\\ 
        DiGress  & ${98.1}\pm{0.3}$ & {79.8}$\pm{5.6}$ & \textbf{95.4 }$\pm 1.1$ & {97.6} $\pm 0.4$\\ 
        % MDM  & ${98.6}$ & $\textbf{{91.9}}$ & - & -\\ 
        GeoLDM  & \textbf{{98.9}}$\pm{0.1}$ & {89.4}$\pm{0.5}$ & 93.8 $\pm 0.4$ & 92.7 $\pm 0.5$\\ 
        MUDiff  & ${98.8}\pm{0.2}$ & {89.9}$\pm{1.1}$ & 95.3 $\pm 1.5$ & \textbf{99.1} $\pm 0.5$\\ 
        \midrule
        Ours & \textbf{98.9}$ \pm {0.03}$ & \textbf{90.5} $\pm 0.15$ & \textbf{95.4} $\pm 0.2$ & 95.6 $\pm 0.1$ \\
        \midrule
        Data   & 99.0 & 95.2 & 99.3 & 100.0 \\
     \bottomrule
    \end{tabular}}
     \vspace{-15pt}
  \end{table}
\subsection{Boolean Satisfiability Problem}

\subsubsection{Background} The Boolean Satisfiability (SAT) problem is the task of determining whether there exists a binary assignment to the variables of a given Boolean expression (in Conjunctive Normal Form (CNF)) that makes it evaluate to \textit{True}. SAT is a canonical NP-Complete problem~\cite{cook1971complexity} and underlies a broad range of real-world applications in formal hardware/software verification, resource scheduling, and other constraint satisfaction tasks~\cite{clarke2001modelchecking, gomes2008satisfiability, vizel2015satmodelchecking}.

Our goal is to find a valid assignment for the Boolean variables, when the given CNF formula is satisfiable. Let $n$ be the number of variables and $m$ the number of clauses. In Random $k$-SAT, a well-studied variation of SAT, the relative difficulty of an instance is determined by the clause density $\frac{m}{n}$. There is a sharp transition between satisfiable and unsatisfiable instances of random 3-SAT at the critical clause density $\alpha_{\mathrm{sat}}(k=3)$, when m is set close to $m=4.258n + 58.26 n^{-\frac{2}{3}}$ \cite{ding2015satisfiabilityconjecture}. Following the setup of \cite{ye2024autoregressiondiscretediffusioncomplex}, we choose $m$ close to this threshold to focus on relatively hard random 3-SAT instances.

\subsubsection{Experimental Setup}
\textbf{Datasets:}
We consider two experimental setups:
% the first to compare with prior works and the second for exploration of this problem.

\underline{Setup 1} (Single $n$): We follow the train and test partitions from \cite{ye2024autoregressiondiscretediffusioncomplex}, which provide separate datasets for $n \!= 5, 7,$ and $9$, for direct comparison. Specifically, $n=5$ and $n=7$, use a training set of 50K samples each, while for $n=9$, the training set consists of 100K samples.

\underline{Setup 2} (Combined $n$): We then move to a more challenging, large-scale setting by extending the range of $n$ up to 20. Following the same generation procedure, for each $n$ in ${6,7,\dots,20}$, we generate 1M training samples, resulting in a combined dataset of 15M instances. In this setup, we train a single model on the aggregated data covering all $n$ from 6 to 20. Figure \ref{fig:sat_n20_accuracy_plot} illustrates how the model's accuracy varies with $n$ under different model sizes.

More details on data generation and model configuration are provided in Appendix \ref{app:3sat}.

\textbf{Baselines:}
We compare against two types of baseline models: 1) Autoregressive Models with a GPT-2 architecture \cite{radford2019language} trained from scratch and 2) Discrete diffusion models in \cite{ye2024autoregressiondiscretediffusioncomplex} (MDM) that applies adaptive sequence- and token-level reweighting to emphasize difficult subgoals in planning and reasoning. MDM has demonstrated strong performance on tasks such as Sudoku and Boolean Satisfiability compared to standard autoregressive and discrete diffusion approaches.


\begin{table}[htbp]
\centering
\caption{\textbf{SAT:} Accuracy with increasing number of variables $n$. Separate model trained for each $n$}
\begin{tabular}{lcccc}
\toprule
Method                               & Params & $n=5$  & $n=7$  & $n=9$  \\
\midrule
GPT-2 Scratch                        & 6M     & 97.6   & 85.6   & 73.3   \\
MDM                                  & 6M     & 100.0  & 95.9   & 87.0   \\
\hline
\multirow{2}{*}{\textbf{Ours}}       & 6M     & 100.0  & 98.0   & 94.5   \\
                                     & 85M    & -      & 99.9   & 99.9   \\
\bottomrule
\end{tabular}
\label{tab:sat_n_5_7_9_accuracy}
\end{table}

\begin{figure}[t]
  \centering
  \includegraphics[width=\columnwidth]{images/sat_n20_accuracy_plot.pdf}
  \caption{\textbf{SAT:} Accuracy for different number of variables across model sizes trained on a combined dataset for n in $6,7\dots,20$.}
  \label{fig:sat_n20_accuracy_plot}
\end{figure}

\subsubsection{Results}
In Table~\ref{tab:sat_n_5_7_9_accuracy}, we see that our method consistently outperforms the autoregressive (GPT-2) and diffusion-based (MDM) baselines across different choices for $n$. This performance gap is more pronounced for larger $n$: at $n=9$, our model achieves 94.5\% accuracy, compared to 87.0\% for MDM and 73.3\% for GPT-2. Scaling the model to 85M parameters further reaches near-perfect accuracy (99.9\%) for $n=7$ and $n=9$, thus highlighting the crucial role of model capacity in handling complex SAT instances.

For \emph{Setup 2}, Figure \ref{fig:sat_n20_accuracy_plot} reveals a steep accuracy drop for the 6M-parameter model; it starts declining around $n = 8$ and approaches zero by $n = 12$. In contrast, the 85M-parameter model remains robust up to $n = 18$, and an even larger 185M-parameter model sustains high accuracy near $n = 19$. This degradation trend aligns with the theoretical hardness of random 3-SAT, where solution spaces become exponentially sparse as $n$ increases. Larger models postpone this accuracy drop underscoring a direct relationship between parameter count and combinatorial reasoning capacity.
\section{Conclusion and future directions} \label{sec:conclusion}

In this paper we proposed a nested MLMC framework that offers important computational savings by performing most calculations in low precision and exploiting approximate random normal variables for the low precision path calculations. The low precision calculations could be performed in fixed precision on an FPGA for greater efficiency, and we suggested a procedure to optimise the bit-widths of every variable at each Monte Carlo level. This is an important improvement over previous mixed precision MLMC frameworks which held the lower precision fixed \cite{Rounding_error_oliver} or defined uniform bit-width at every level heuristically \cite{brugger2014mixed}. Our numerical results suggest that for the first levels our procedure reduces the cost at these levels by a factor 5 or 7. Hence the overall savings are significant since most paths are calculated on the first levels. Our approach would be even more efficient for the Milstein scheme because its higher order strong convergence leads to a greater proportion of the computational costs being on the coarsest levels.

The next stage of the research project will be to implement the RNG methods and the nested framework on FPGAs to determine the hardware requirements and confirm the extent of the computational savings. It would also be good to compare the performance benefits to using half-precision floating point arithmetic on GPUs or CPUs for the low-accuracy computations.




\newpage
\clearpage
\section*{Impact Statement}
This paper contributes to the advancement of machine learning by introducing a novel constrained sampling algorithm.  While generative machine learning models, particularly in text and image generation, have raised societal concerns, our work focuses on molecule generation, layout generation, and 3-SAT, using publicly available datasets.  At this stage, we foresee no direct societal impact.  However, we acknowledge that future applications of our method in sensitive domains may necessitate a more thorough evaluation of potential societal implications.
\bibliography{main}
\bibliographystyle{icml2025}


\newpage
\appendix
\onecolumn
\section{Proofs}

\subsection{Lemma \ref{lemma:for_process}}

\paragraph{Statement:\\}

Denote the distribution of $s^{(t)}$ by $P_t$. Suppose $\Pi_t\left(\cdot | \mathcal{X} \right) = \Pi\left(\cdot | \mathcal{X} \right)$ for all $t$, $\Pi_t(\phi) \leq 1 - \epsilon$ for some $\epsilon > 0$ and $\lim_{T \rightarrow \infty} \sum_{j} \log \left(1- \tilde{\beta}_j\right) = -\infty$. As $T \rightarrow \infty$, $P_T$ converges to the product distribution: $\Pi\left( \cdot| \mathcal{X} \right)^{L_1} \times_{i=L_1 + 1}^{L} \mathcal{N}\left(0, \mathbf{I}_{d_i} \right)$ .

\paragraph{Proof:\\}

We closely follow the proof of Lemma 1 in \cite{varma2024glauber}.

Note that the forward noising for each element is independent of all other elements.  Hence, it suffices to consider the noising of each element separately.

Consider a discrete element. By assumption, the probability of not choosing $\phi$:
$$ 1 - \Pi(\phi) \geq \epsilon  $$
where $\epsilon>0$ for all. Further, when $\phi$ is not chosen at time $t$, then the distribution of the discrete token is $\Pi( \cdot|\mathcal{X})$ for all time $\geq t$ independent of other tokens. The probability of choosing only $\phi$ until time $t$ is at most $(1-\epsilon)^t$ and this goes to $0$ as $t \rightarrow \infty $. Therefore with probability $1$, asymptotically, every discrete element is noised to the distribution $\Pi\left( \cdot| \mathcal{X} \right)$.

Consider a continuous vector at position $i$. From the definition of the forward process, we have:
\begin{align}
    s^{(t+1)}_{i} = \left(\sqrt{1 - \tilde{\beta}_{m_{i}^t}}\right) s^{(t)}_{i} + \left(\sqrt{\tilde{\beta}_{m_{i}^t}}\right) \epsilon
\end{align}
where $\epsilon \sim \mathcal{N}(0, \mathbf{I})$. Merging the Gaussians, we have:
\begin{align*}
    s^{(t+1)}_{i} = \left(\sqrt{\tilde{\alpha}_{m_{i}^t}}\right) s^{(0)}_{i} + \left(\sqrt{1 - \tilde{\alpha}_{m_{i}^t}}\right) \epsilon
\end{align*}
where:
\begin{align*}
    \tilde{\alpha}_{m_{i}^t} = \prod_{j = 0}^{m_{i}^t} (1 - \tilde{\beta}_j)
\end{align*}
Since $m_i^{t}$ denotes the number of times the position $i$ is visited by sequence time $t$,  $m_i^{t} \rightarrow \infty$ as $t \rightarrow \infty$. Hence, from the assumption $\lim_{T \rightarrow \infty} \sum_{j} \log \left(1- \tilde{\beta}_j\right) = -\infty$, we have $\lim_{t \rightarrow \infty}\tilde{\alpha}_{m_{i}^t} = 0$ and hence the continuous vector will converge to an independent Gaussian with variance $1$ per continuous dimension.

\newpage

\subsection{Lemma \ref{lemma:rev_process}}

\paragraph{Statement:\\}
Assume $\hat{s}^{(T)} \sim P_T$ and assume we have access to ideal discrete and continuous denoisers. Then, $\hat{s}^{(0)}$ obtained after $T$ steps of reverse denoising process, will be such that $\hat{s}^{(0)} \sim \pi$.

\paragraph{Proof:\\}

Recall that $s^{(t)} \in \mathcal{S}_{L}$ denotes the the sequence at sequence time $t$ of the forward process. Further, $P_{t}$ denotes the probability measure of $s^{(t)}$ over $\mathcal{S}_{L}$. $\hat{s}^{(t)} \in \mathcal{S}_{L}$ denotes the the sequence at sequence time $t$ of the reverse process and let $\hat{P}_{t}$ denote the probability measure of $\hat{s}^{(t)}$ over $\mathcal{S}_{L}$.

We now prove the lemma by induction. Assume that $\hat{s}^{(t+1)} \overset{d}{=} {s}^{(t+1)} $, i.e., ${P}_{t+1} = \hat{P}_{t+1}$. Consider a measurable set $\mathcal{A}$ such that $\mathcal{A} \subseteq \mathcal{S}_L$ . Let $y \sim \hat{P}_{t+1}$. From the measure decomposition theorem, we have:
\begin{align*}
    \mathbb{P}(\hat{s}^{(t)} \in \mathcal{A}) = \int_{y} \mathbb{P}\left(\hat{s}^{(t)} \in \mathcal{A}|\hat{s}^{(t+1)} = y \right) d\hat{P}_{t+1}(y)
\end{align*}
From the induction assumption, we can rewrite this as:
\begin{align*}
    \mathbb{P}(\hat{s}^{(t)} \in \mathcal{A}) = \int_{y} \mathbb{P}\left(\hat{s}^{(t)} \in \mathcal{A}| \hat{s}^{(t+1)} = y \right) d{P}_{t+1}(y)
\end{align*}
From the definition of the reverse process, we know that $\hat{s}^{(t)}_{-i_t} = \hat{s}^{(t+1)}_{-i_t} $. Therefore, we have:
\begin{align*}
     \mathbb{P}\left(\hat{s}^{(t)} \in \mathcal{A}|\hat{s}^{(t+1)} = y \right) =  \mathbb{P}\left(\hat{s}^{(t)}_{i_t} \in \mathcal{A}_{-i_t}\left(y_{-i_t} \right)| \hat{s}^{(t+1)} = y \right) 
\end{align*}
where $\mathcal{A}_{-i_t}\left(y_{-i_t} \right) = \{x_{i_t}: x \in \mathcal{A}, x_{-i_t} = y_{-i_t} \}$. Depending on the reverse process chosen, we have:
\begin{align*}
    \mathbb{P}\left(\hat{s}^{(t)}_{i_t} \in \mathcal{A}_{-i_t}\left(y_{-i_t} \right)|\hat{s}^{(t+1)} =  y \right) &= \mathbb{P}\left({s}^{(t)}_{i_t} \in \mathcal{A}_{-i_t}\left(y_{-i_t} \right)|s^{(t+1)} = y \right) \quad \text{or} \\
    \mathbb{P}\left(\hat{s}^{(t)}_{i_t} \in \mathcal{A}_{-i_t}\left(y_{-i_t} \right)|\hat{s}^{(t+1)} =  y \right) &= \mathbb{P}\left({s}^{(t)}_{i_t} \in \mathcal{A}_{-i_t}\left(y_{-i_t} \right)|s^{(t+1)}_{-i_t} = y_{-i_t} \right)
\end{align*}

\textbf{Case 1:} $ \mathbb{P}\left(\hat{s}^{(t)}_{i_t} \in \mathcal{A}_{-i_t}\left(y_{-i_t} \right)|\hat{s}^{(t+1)} =  y \right) = \mathbb{P}\left({s}^{(t)}_{i_t} \in \mathcal{A}_{-i_t}\left(y_{-i_t} \right)|s^{(t+1)} = y \right)  $

We have:
\begin{align*}
    \mathbb{P}\left(\hat{s}^{(t)} \in \mathcal{A}|\hat{s}^{(t+1)} = y \right) =  \mathbb{P}\left({s}^{(t)}_{i_t} \in \mathcal{A}_{-i_t}\left(y_{-i_t} \right)|s^{(t+1)} = y \right)
\end{align*}
And hence:
\begin{align*}
    \mathbb{P}(\hat{s}^{(t)} \in \mathcal{A}) &= \int_y  \mathbb{P}\left({s}^{(t)}_{i_t} \in \mathcal{A}_{-i_t}\left(y_{-i_t} \right)|s^{(t+1)} = y \right) d{P}_{t+1}(y) \\
    &= \mathbb{P}({s}^{(t)} \in \mathcal{A})
\end{align*}

\textbf{Case 2:} $ \mathbb{P}\left(\hat{s}^{(t)}_{i_t} \in \mathcal{A}_{-i_t}\left(y_{-i_t} \right)|\hat{s}^{(t+1)} =  y \right) = \mathbb{P}\left({s}^{(t)}_{i_t} \in \mathcal{A}_{-i_t}\left(y_{-i_t} \right)|s^{(t+1)}_{-i_t} = y_{-i_t} \right)  $

We have:
\begin{align*}
    \mathbb{P}\left(\hat{s}^{(t)} \in \mathcal{A}|\hat{s}^{(t+1)} = y \right) = \mathbb{P}\left({s}^{(t)}_{i_t} \in \mathcal{A}_{-i_t}\left(y_{-i_t} \right)|s^{(t+1)}_{-i_t} = y_{-i_t} \right)
\end{align*}
And hence:
\begin{align*}
    \mathbb{P}(\hat{s}^{(t)} \in \mathcal{A}) &= \int_y  \mathbb{P}\left({s}^{(t)}_{i_t} \in \mathcal{A}_{-i_t}\left(y_{-i_t} \right)|s^{(t+1)}_{-i_t} = y_{-i_t} \right) d{P}_{t+1}(y) \\
\end{align*}
By measure decomposition theorem ${P}_{t+1}(y)$ is factorizable as:
\begin{align*}
    {P}_{t+1}(y) = {P}_{t+1, -i_t}(y_{-i_t}){P}_{t+1, i_t}(y_{i_t}|y_{-i_t})
\end{align*}
Therefore:
\begin{align*}
    \mathbb{P}(\hat{s}^{(t)} \in \mathcal{A}) &= \int_y  \mathbb{P}\left({s}^{(t)}_{i_t} \in \mathcal{A}_{-i_t}\left(y_{-i_t} \right)|s^{(t+1)}_{-i_t} = y_{-i_t} \right) \left(d{P}_{t+1, -i_t}(y_{-i_t})\right) \left(d{P}_{t+1, i_t}(y_{i_t}|y_{-i_t})\right) \\
    &= \int_{y_{-i_t}} \mathbb{P}\left({s}^{(t)}_{i_t} \in \mathcal{A}_{-i_t}\left(y_{-i_t} \right)|s^{(t+1)}_{-i_t} = y_{-i_t} \right) \left(d{P}_{t+1, i_t}(y_{i_t}|y_{-i_t}))\right) \int_{y_{i_t}} \left(d{P}_{t+1, i_t}(y_{i_t})\right) \\
     &= \mathbb{P}({s}^{(t)} \in \mathcal{A})
\end{align*}

Hence, we have  $\hat{s}^{(t)} \overset{d}{=} {s}^{(t)} $, i.e. ${P}_{t} = \hat{P}_{t}$. Therefore, by induction $\hat{P}_{0} = \pi$, provided $\hat{P}_{T} = {P}_{T}$.

\newpage
\subsection{Lemma \ref{lemma:score_noise}}

\paragraph{Statement:}
Under the considered forward process where noising occurs independently, we have:
 \begin{align*}
     \nabla_{{s}_{i_t}^{(t, k+1)}} \log q({s}_{i_t}^{(t, k+1)} | {s}_{-i_t}^{(t, k+1)} )  = -\frac{1}{\sqrt{1-\bar{\alpha}}} \mathbb{E} \left[ \epsilon | {s}^{(t,k+1)} \right]
 \end{align*}

\paragraph{Proof:}

Let us split $\hat{s}^{(t, k+1)} = \left[\hat{s}_{i_t}^{(t, k+1)} \; \hat{s}_{-i_t}^{(t, k+1)}  \right]$. Note that, $\hat{s}_{i_t}^{(t, k+1)}$ is the continuous part that is being de noised.

\begin{align} \label{eq:gradlog}
    \nabla_{\hat{s}_{i_t}^{(t, k+1)}} \log q(\hat{s}_{i_t}^{(t, k+1)} | \hat{s}_{-i_t}^{(t, k+1)} )  
  \hfill & =\frac{\nabla_{\hat{s}_{i_t}^{(t, k+1)}}  q(\hat{s}_{i_t}^{(t, k+1)} | \hat{s}_{-i_t}^{(t, k+1)} )}{ q(\hat{s}_{i_t}^{(t, k+1)} | \hat{s}_{-i_t}^{(t, k+1)} )} \nonumber \\
   \hfill & = \frac{\nabla_{\hat{s}_{i_t}^{(t, k+1)}}  q(\hat{s}_{i_t}^{(t, k+1)} | \hat{s}_{-i_t}^{(t, k+1)} ) q(\hat{s}_{-i_t}^{(t, k+1)})}{ q(\hat{s}_{i_t}^{(t, k+1)} | \hat{s}_{-i_t}^{(t, k+1)} ) q(\hat{s}_{-i_t}^{(t, k+1)})} \nonumber \\
   \hfill & = \frac{\nabla_{\hat{s}_{i_t}^{(t, k+1)}}  q(\hat{s}_{i_t}^{(t, k+1)}, \hat{s}_{-i_t}^{(t, k+1)} )}{ q(\hat{s}_{i_t}^{(t, k+1)} , \hat{s}_{-i_t}^{(t, k+1)} )} 
    \nonumber \\
    \hfill & = \frac{\nabla_{\hat{s}_{i_t}^{(t, k+1)}}  \int q(\hat{s}_{i_t}^{(t, k+1)}, \hat{s}_{-i_t}^{(t, k+1)} | s_{i_t}^{(0)} ) q(s_{i_t}^{(0)}) ds_{i_t}^{(0)} } { q(\hat{s}_{i_t}^{(t, k+1)} , \hat{s}_{-i_t}^{(t, k+1)} )} \nonumber \\
    \hfill & \overset{(a)}{=} \frac{\nabla_{\hat{s}_{i_t}^{(t, k+1)}}  \int q(\hat{s}_{i_t}^{(t, k+1)}| s_{i_t}^{(0)} ) q( \hat{s}_{-i_t}^{(t, k+1)} |  s_{i_t}^{(0)}) q(s_{i_t}^{(0)}) ds_{i_t}^{(0)} } { q(\hat{s}_{i_t}^{(t, k+1)} , \hat{s}_{-i_t}^{(t, k+1)} )} \nonumber \\
    \hfill &\overset{(b)}{=} \frac{\int \frac{-\epsilon}{\sqrt{1-\bar{\alpha}}} q(\hat{s}_{i_t}^{(t, k+1)}| s_{i_t}^{(0)} ) q( \hat{s}_{-i_t}^{(t, k+1)} |  s_{i_t}^{(0)}) q(s_{i_t}^{(0)}) ds_{i_t}^{(0)} } { q(\hat{s}_{i_t}^{(t, k+1)} , \hat{s}_{-i_t}^{(t, k+1)} )}  \nonumber \\
    \hfill &\overset{(a)}{=} \frac{\int \frac{-\epsilon}{\sqrt{1-\bar{\alpha}}} q(\hat{s}_{i_t}^{(t, k+1)}, \hat{s}_{-i_t}^{(t, k+1)},s_{i_t}^{(0)})  ds_{i_t}^{(0)} } { q(\hat{s}_{i_t}^{(t, k+1)} , \hat{s}_{-i_t}^{(t, k+1)} )} \nonumber \\
    \hfill & = \int \frac{-\epsilon}{\sqrt{1-\bar{\alpha}}} q(s_{i_t}^{(0)} |\hat{s}_{i_t}^{(t, k+1)}, \hat{s}_{-i_t}^{(t, k+1)})  ds_{i_t}^{(0)}  
\end{align}
In the RHS of the chain in \eqref{eq:gradlog}, we observe that $\hat{s}^{t,k+1}$ is being conditioned on and given $\hat{s}^{t,k+1}$, $\epsilon$ is a function only of $s_{i_t}^{(0)}$ from \eqref{eq:diff_relation}. Therefore, the above chain yields:
 \begin{align}
     \nabla_{\hat{s}_{i_t}^{(t, k+1)}} \log q(\hat{s}_{i_t}^{(t, k+1)} | \hat{s}_{-i_t}^{(t, k+1)} )  = \frac{1}{\sqrt{1-\bar{\alpha}}} \mathbb{E} \left[ -\epsilon | \hat{s}^{(t,k+1)} \right]
 \end{align}
 
 This is the exactly $\frac{g_{\theta}(\cdot)}{\sqrt{1-\bar{\alpha}}}$ if the estimator was a perfect MMSE estimator.


Justifications:- (a) observe that conditioned on $s_{i_t}^{(0)}$, how $i_t$-th element is noised in the forward process is independent of all other elements. This gives rise to the conditional independence.(b) We exchange the integral and the $\nabla$ operator. Let $q(x|y)$ be conditionally Gaussian, i.e.  $x|y \sim \mathcal{N}(\sqrt{\bar{\alpha}} y ; (1-\bar{\alpha}))$, then it is a property of the conditional Gaussian random variable that $\nabla_x q(x|y) = -\left(\frac{x - \sqrt{\bar{\alpha}}y}{1-\bar{\alpha}} \right) * q(x|y)$. Taking $x = \hat{s}_{i_t}^{t,k}$ and $y = s_{i_t}^{(0)}$ from \eqref{eq:diff_relation}, we see that: $\nabla_{\hat{s}_{i_t}^{(t, k+1)}}  q(\hat{s}_{i_t}^{(t, k+1)}| s_{i_t}^{(0)} ) = \frac{-\epsilon}{\sqrt{1-\bar{\alpha}}}* q(\hat{s}_{i_t}^{(t, k+1)}| s_{i_t}^{(0)} ) $.

\newpage

\section{Connection between $\tilde{\beta}, \beta$ and $\bar{\alpha}$}
\label{app:sec:beta_connection}
From subsection \ref{subsec:fnp}, we have:
\begin{equation}
\label{eq:app:beta_tilde}
    s^{(t+1)}_{i_t} = \left(\sqrt{1 - \tilde{\beta}_{m_{i_t}^t}}\right) s^{(t)}_{i_t} + \left(\sqrt{\tilde{\beta}_{m_{i_t}^t}}\right) \epsilon^{(t)}
\end{equation}
    
And from subsection \ref{subsec:con_den}, we have:
\begin{equation}
\label{eq:app:beta}
    s^{(t, k+1)}_{i_t} = \left(\sqrt{1 - {\beta}(t, k)}\right) s^{(t, k)}_{i_t} + \left(\sqrt{{\beta}(t, k)}\right) \epsilon
\end{equation}

where $s^{(t, 0)}_{i_t} = s^{(t)}_{i_t}$ and $s^{(t, K_{i_t}^t)}_{i_t} = s^{(t+1)}_{i_t}$. Define ${\alpha}(t, k) = 1 - {\beta}(t, k)$. Then, we have:
$$
    s^{(t, k+1)}_{i_t} = \left(\sqrt{\prod_{k' = 0}^k{\alpha}(t, k')}\right) s^{(t, 0)}_{i_t} + \left(\sqrt{1 - {\prod_{k' = 0}^k{\alpha}(t, k')}}\right) \epsilon
$$
where we have merged the Gaussians. Setting $k = K_{i_t}^{t}-1$ and comparing \eqref{eq:app:beta_tilde} and \eqref{eq:app:beta}, we have:
\begin{align}\label{eq:beta_tilde}
    \tilde{\beta}_{m_{i_t}^t} &= 1 - {\prod_{k' = 0}^{K_{i_t}^t - 1}{\alpha}(t, k')}
\end{align}
Recall from subsection \ref{subsec:con_den} that:
\begin{equation}
\label{eq:app:alpha_bar}
    s^{(t, k+1)}_{i_t} = \left(\sqrt{\bar{\alpha}(t, k)}\right) s^{(0)}_{i_t} + \left(\sqrt{1 - \bar{\alpha}(t, k)}\right) \epsilon
\end{equation}
Again rewriting \eqref{eq:app:beta} by merging gaussians, we have:
$$
    s^{(t, k+1)}_{i_t} = \left(\sqrt{\prod_{t' = 0}^{t-1}\prod_{k' = 0}^{K_i^{t'}}{\alpha}(t', k') \prod_{k''=0}^k \alpha(t,k'')}\right) s^{(0, 0)}_{i_t} + \left(\sqrt{1 - {\prod_{t' = 0}^{t-1}\prod_{k' = 0}^{K_i^{t'}}{\alpha}(t', k') \prod_{k''=0}^k \alpha(t,k'')}}\right) \epsilon
$$
Comparing with \eqref{eq:app:alpha_bar}, we have:
\begin{align}
\label{eq:alpha_rel}
    {\bar{\alpha}(t, k)} = \prod_{t' = 0}^{t-1}\prod_{k' = 0}^{K_i^{t'}}{\alpha}(t', k') \prod_{k''=0}^k \alpha(t,k'')
\end{align}

\newpage
\section{Forward process: Generating $s^{(t)}$ directly}
\label{app:sec:fwd_prcs}
For training the model for denoising at sequence time $t$ (and element time $k$ if we are denoising a continuous vector), we need access to:
\begin{itemize}
    \item $(s^{(t-1)}, s^{(t)})$ if $s^{(t)}_{i_t}$ is discrete 
    \item $(s^{(t, k)}, \epsilon)$ if $s^{(t, k)}_{i_t}$ is continuous
\end{itemize}
Note that $\epsilon$ is as defined in \eqref{eq:diff_relation}. Once you have $s^{(t-1)}$, $s^{(t)}$ can be generated by applying one discrete noising step, by sampling $z_{t-1}$. $\epsilon$ is required to compute $s^{(t, k)}_{i_t}$ from $s^{(t, k)}_{i_t}$ and hence having access to $s^{(t, k)}$ would imply access to $\epsilon$. Hence, if we can directly sample $s^{(t, k)}$ without going through all intermediate timesteps, the model can be trained efficiently to denoise at time $t$.

Let us denote by $m_{j}^{(t, k)}$ the total number of times an element at position $j$ has been noised by sequence time $t$ and element time $k$. Further, let $\tau_j^t = \{t' \in \{0, 1, \dots, t\};i_{t'} = j  \}$ denote the set of all sequence timesteps $t'$ at which position $j$ was visited prior to (and including) sequence time $t$.

For discrete elements, $ m_{j}^{(t, k)} = \abs{\tau_j^t}$. That is, for discrete elements, the number of noising steps is equal to the number of visits at that position by  time $t$ (Note that element time $k$ is irrelevant for discrete noising ).

For continuous vectors, $m_{j}^{(t, k)} = \sum_{t' \in \tau_j^t} K^{t'}_{j}$, provided $j \neq i_t$. That is, for continuous vectors which are not being noised at time $t$, the total number of noising steps are obtained by summing up the element times for all prior visits at that position. For $j = i_t$, $m_{j}^{(t, k)} = \sum_{t' \in \tau_j^t - \{t\}} K^{t'}_{j} + k$. That is, if a continuous vector is being noised at time $t$, the number of noising steps for that vector is obtained by summing up the element times for all prior visits as well as the current element time.

Since the noising process for each element is independent of other elements, to describe the generation of $s^{(t)}$(or $s^{(t, k)}$) from $s^{0}$,  it is sufficient to describe generating $s_{j}^{(t)}$(or $s_{j}^{(t, k)}$) from $s_{j}^{(0)}$ individually for each $j$. 

\paragraph{If $s_{j}^{(t)}$ is discrete:\\}

Recall from subsection \ref{subsec:fnp} that $\Pi_{t}(\phi)$ denotes the probability of sampling the token $\phi$ at sequence time $t$. Further, assume $\Pi_t(\cdot | \mathcal{X})$ is same for all $t$. Define $p_{j}^t = 1 - \prod_{t' \in \tau_j^t} (1 - \Pi_{t'}(\phi))$. Sample $z \sim \Pi(\cdot|\mathcal{X})$  Then:
\begin{align*}
    s^{(t)}_{j} = 
    \begin{cases}
    s^{(0)}_{j},& \text{with probability } 1- p_j^t\\
    z,& \text{ with probability } p_j^t\\
    \end{cases}
\end{align*}
The above follows from the fact that each flip for $t' \in \tau_j^t$ is an independent Bernoulli trial and hence, even if there is one success among these $m_{j}^{(t, k)}$ trials, the token is noised to $\Pi(\cdot|\mathcal{X})$.

\paragraph{If $s_{j}^{(t, k)}$ is continuous:\\}

Following subsection \ref{subsec:con_den}, we define the continuous noise schedule for the continuous vector at position $j$ as  $\beta_j = \{ \beta_i; i \in \{0, 1, \dots, m_{j}^{(T, 0)}-1 \} \}$. Let $\beta_j[i]$ denote the $i^\text{th}$ element of $\beta_j$. Define $\bar{\alpha}_j = \{ \prod_{i' \leq i}(1 - \beta_j[i']); i \in \{0, 1, \dots, m_{j}^{(T, 0)}-1 \} \}$.  Let $\bar{\alpha}_j[i]$ denote the $i^\text{th}$ element of $\bar{\alpha}_j$. Then, following \eqref{eq:diff_relation}, we have:
\begin{align*}
    s^{(t, k)}_{i_t} = \left(\sqrt{\bar{\alpha}_{i_t}[m_{i_t}^{(t, k)}]}\right) s^{(0)}_{i_t} + \left(\sqrt{1 - \bar{\alpha}_{i_t}[m_{i_t}^{(t, k)}]}\right) \epsilon
\end{align*}
where $\epsilon \sim \mathcal{N}(0, \mathbf{I} )$.

The forward process can thus be thought of as the block $\texttt{FwdPrcs}$ with the following input and output:\\
\begin{tabular}{l l l}
  Input: $(s^{(0)}, t,  \{i_{\tau}\}_{\tau = 0}^t, \{\Pi_{\tau}\}_{\tau = 0}^t)$ & Output: $(s^{(t)}, z_{t}, s^{(t+1)})$ & if $s^{(0)}_{i_{t}}$ is discrete \\
  Input: $(s^{(0)}, t, k, \{i_{\tau}\}_{\tau = 0}^t, \{\beta_{j}\}_{j = L_1+1}^{L_2})$ & Output: $(s^{(t, k+1)}, \epsilon)$ & if $s^{(0)}_{i_t}$ is continuous \\
\end{tabular}


\newpage

\section{Model Training and Inference: Pseudocode}
\label{app:sec:model_train_pseudo}

We use the binary classification based loss for describing the training of the model to do discrete denoising since this leads to better results. Note that for this, from , the input to the model should be $s_{-i_t}^{(t+1)}$ and the model should predict $\mathbb{P}\left({z_t} = x | {s}^{(t+1)}_{-i_t} = {s}_{-i_t}, {s}^{(t+1)}_{i_t} = x \right)$ for all $x \in \mathcal{X}$. To do this efficiently, we adapt the masking strategy from \cite{varma2024glauber}. Define a token $\omega \notin \mathcal{X}$. Let $\mathcal{\tilde{X}} = \mathcal{X} \cup \omega$. Let $\tilde{s}^{(t+1)} \in \mathcal{\tilde{S}}_{L}$, where $\mathcal{\tilde{S}}_L = \mathcal{\tilde{X}}^{L_1}\times_{i=L_1+1}^{L}\mathbb{R}^{d_i}$, be defined as: $\tilde{s}^{t+1}_{-i_t} = s^{t+1}_{-i_t}$ and $\tilde{s}^{t+1}_{i_t} = \omega$. The neural network $f_{\theta}$ then takes as input: time tuple $(t, k)$, noising position $i_t$, sequence $\tilde{s}^{t+1}$ (or $\tilde{s}^{t, k+1}$ if $i_t$ corresponds to a continuous vector). The time tuple $(t, k)$ is $(t, 0)$ if the element under consideration is discrete since discrete tokens only have one noising step. The model has $\abs{\mathcal{X}}$ logits corresponding to \textit{each} discrete token (and hence a total of $L_1\abs{\mathcal{X}}$ logits) and $\mathbb{R}^{d_i}$ dimensional vectors corresponding to \textit{each} continuous vector (and hence a total of $L_2$ continuous vectors). $i_t$ is necessary for the model to decide which output needs to be sliced out: we use $f_{\theta}^{i_t}$ to denote the output of the model corresponding to the element at position $i_t$ (which could either be discrete or continuous). Further, we use $f_{\theta}^{(i_t, s_{i_t}^{t+1})}$ to denote the logit corresponding to position $i_t$ and token $s_{i_t}^{t+1}$, provided $i_t$ corresponds to a discrete token.

We can then write the pseudocode for training as follows:

\begin{algorithm}[ht]
\begin{algorithmic}
\INPUT { Dataset $\mathcal{D}$, {model} $f_{\theta}$ , {forward process block} \texttt{FwdPrcs}, optimizer \texttt{opt}, total sequence timesteps $T$, noise positions $\{i_t\}_{t = 0}^{T-1}$, discrete noise schedule $\{\Pi_{t}\}_{t = 0}^{T-1}$, continuous noise schedule $\{\beta_{j}\}_{j = L_1+1}^{L_2}$, continuous noising steps  $\{K_{j}^t\}_{j = L_11+1, t = 0}^{j = L_2, t=T-1}$}
\OUTPUT {trained model parameters $\theta$}

\FOR{each iteration:}
 \STATE sample $s^{(0)}$ from $\mathcal{D}$
 \STATE sample $t$ from $[0, 1, \dots, T-1]$
 \IF {$\hat{s}^{(0)}_{i_t}$ is discrete}
    \STATE  $(s^{(t)}, z_{t}, s^{(t+1)}) = \texttt{FwdPrcs}(s^{(0)}, t, \{i_{\tau}\}_{\tau = 0}^t, \{\Pi_{\tau}\}_{\tau = 0}^t)$
    \STATE construct $\tilde{s}^{t+1}$ from ${s}^{t+1}$
    \STATE compute the BCE loss:
    $$ \mathcal{L} =  -  \mathbf{1}_{z_t \neq \phi}  \log \left( f_{\theta} ^{(i_t, s^{(t+1)}_{i_t})} \left(\tilde{s}^{(t+1)}, t, i_{t} \right) \right)
    - \mathbf{1}_{z_t = \phi}  \log \left(1 - f_{\theta} ^{(i_t, s^{(t+1)}_{i_t})} \left(\tilde{s}^{(t+1)}, t, 0, i_{t} \right) \right)  $$
 \ELSE
    \STATE  sample $k$ from $[0, 1, \dots, K_{i_t}^t - 1]$
    \STATE $(s^{(t, k+1)}, \epsilon) = \texttt{FwdPrcs}(s^{(0)}, t, k, \{i_{\tau}\}_{\tau = 0}^t, \{\beta_{j}\}_{j = L_1+1}^{L_2}) $
    \STATE compute the MSE loss:
    $$ \mathcal{L} = \norm{\epsilon - f^{i_t}_{\theta}\left({s}^{(t, k+1)}, t, k, i_t \right)}_2^2 $$
    
 \ENDIF
 \STATE $\theta \leftarrow \texttt{opt.update}(\theta, \nabla_{\theta}\mathcal{L})$
\ENDFOR
\end{algorithmic}
\caption{Model Training}
\label{app:alg:training}
\end{algorithm}

\newpage

Recall that $\hat{s}^{(t)}$ represents the sequence from the reverse process at time $t$ and $P_T = \Pi\left( \cdot| \mathcal{X} \right)^{L_1} \times_{i=L_1 + 1}^{L} \mathcal{N}\left(0, \mathbf{I}_{d_i} \right)$ denotes the stationary distribution of the forward process. If the training of the model is perfect, we will have $\hat{s}^{(0)} \sim \pi$. Then the pseudocode for inference:

\begin{algorithm}[ht]

\begin{algorithmic}
\INPUT {total sequence timesteps $T$, noise positions $\{i_t\}_{t = 0}^{T-1}$, discrete noise schedule $\{\Pi_{t}\}_{t = 0}^{T-1}$, continuous noise schedule $\{\beta_{j}\}_{j = L_1+1}^{L_2}$, continuous noising steps  $\{K_{j}^t\}_{j = L_1+1, t = 0}^{j = L_2, t=T-1}$}
\OUTPUT{$\hat{s}^{(0)}$}

\STATE sample $\hat{s}^{(T)} \sim P_T$
\FOR{ $t$ in $[T-1, T-2, \cdots, 0]$}

\IF{$\hat{s}^{(t+1)}_{i_t}$ is discrete}
\STATE construct $\tilde{s}^{(t+1)}$ from $\hat{s}^{(t+1)}$
\STATE get $\hat{y} = f_{\theta} ^{i_t} \left(\tilde{s}^{(t+1)}, t, i_{t} \right)$ \COMMENT{ $\hat{y}$ denotes the vector of $\abs{\mathcal{X}}$ logits corresponding to position $i_t$}
\STATE compute $\hat{\mathbb{P}}\left(s^{(t)}_{i_t} = a | s^{(t+1)}_{-i_t} \right) = \frac{\Pi_t(a)}{\Pi_t(\phi)} \left( \frac{1}{\hat{y}^{(a)}} - 1 \right)$ for all $a \in
\mathcal{X}$ \COMMENT{$\hat{y}^{(a)}$ denotes logit corresponding to token $a$}
\STATE sample $\hat{s}^{(t)}_{i_t} \sim \hat{\mathbb{P}}\left(s^{(t)}_{i_t} = a | s^{(t+1)}_{-i_t} \right)$
\STATE set $\hat{s}^{(t)}_{-i_t} = \hat{s}^{(t+1)}_{-i_t}$
\ELSE
\STATE set $\hat{s}^{(t, K_{i_t}^t)} = \hat{s}^{(t+1)}$
\FOR{$k$ in $[K_{i_t}^t-1, K_{i_t}^t-2, \cdots, 0]$}
\STATE get $\epsilon_{\theta} = f_{\theta} ^{i_t} \left(\hat{s}^{(t, k+1)}, t, k, i_t \right)$ \COMMENT{ $f_{\theta} ^{i_t}$ denotes the continuous vector corresponding to position $i_t$}
\IF{t = k = 0}
\STATE get $\epsilon = 0$
\ELSE
\STATE get $\epsilon \sim \mathcal{N}(0, \mathbf{I})$
\ENDIF
\STATE set $\hat{s}^{(t, k)}_{i_t}  = \frac{\left(\hat{s}^{(t, k+1)}_{i_t} - {\beta_{i_t}}(t, k+1)\epsilon_{\theta} \right)}{\sqrt{1 - {\beta_{i_t}}(t, k+1)}} 
    + \left(\sqrt{{\beta_{i_t}}(t, k+1)}\right) \epsilon$
\STATE set $\hat{s}^{(t, k)}_{-i_t} = \hat{s}^{(t, k+1)}_{-i_t}$
\ENDFOR
\STATE set $\hat{s}^{(t)} = \hat{s}^{(t, 0)}$

\ENDIF

\ENDFOR

\end{algorithmic}

\end{algorithm}

\newpage

\section{Model Architecture: More Details}
\label{app:sec:model_arch}

Figure \ref{fig:disco_network} gives a high level overview of the proposed Dis-Co DiT architecture. Just like DiT, we feed in the discrete tokens and corresponding discrete time as input to the Dis-Co DiT block; however, we now also feed in the continuous vector inputs and corresponding continuous time. Also note that time is now the tuple $(t, k)$, where $t$ is the sequence time and $k$ is the element time. For discrete elements $k = 0$ always. Time is now embedded through an embedding layer similar to DiT; discrete tokens are also embedded through an embedding layer. Continuous vectors are projected using a linear layer into the same space as the discrete embeddings; these projected vectors are referred to as continuous embeddings. Discrete embeddings, continuous embeddings and their corresponding time embeddings are then passed into the Dis-Co DiT blocks. Following DiT, the outputs from the Dis-Co DiT blocks are then processed using adaptive layer normalization and a linear layer to obtain the discrete logits and continuous predictions.

Figure \ref{fig:disco_block} details the structure of a single Dis-Co DiT block. The discrete and continuous time embeddings are processed by an MLP and are used for adaptive layer normalization, adaLN-Zero, following DiT. The discrete and continuous embedding vectors, after appropriate adaptive layer normalization, are concatenated and passed to the Multi-Head Self Attention Block. The output from the Self Attention block is again split into discrete and continuous parts, and the process is then repeated with a Pointwise Feedforward network instead of Self Attention. This output is then added with the output from Self Attention (after scaling) to get the final output from the Dis-Co DiT block.


\paragraph{Generating Time Embeddings:}
Assume you are embedding the time tuple $(t, k)$ ($k = 0$ for discrete). Following DiT, we compute the vector $d$ whose $i^{th}$ element is given by:
$$ d[i] = k*f^{\frac{-i}{d_{in}-1}} $$
where $k$ is the element time, $f$ is the frequency parameter (set to $10000$ in all our experiments) and $d_{in}$ is the time embedding input dimension (set to $256$ in all our experiments). Similarly, we compute the vector $c$ whose $i^{th}$ element is given by:
$$ c[i] = t*(T_C f)^{\frac{-i}{d_{in}-1}} $$
where $t$ is the sequence time, $T_C$ is a frequency multiplier designed to account for the fact that multiple continuous noising steps happen for a single discrete flip. In our experiments, we set $T_C = K_{i_t}^t$. Once we have these vectors, we construct the following vector:
$$ y = [\sin(d)\; \cos(d)\; \sin(c)\; \cos(c)\; ] $$
i.e., we concatenate the vectors after applying $\sin$ and $\cos$ elementwise. This vector $y$ is then passed through 2 MLP layers to get the final time embedding.

\newpage

\section{Layout Generation}
\label{app:layout_gen}

\subsection{Additional Results}
\label{app:layout_gen_full_results}

\begin{table*}[h]
    \centering
    \caption{\textbf{Layout Generation:} Additional metrics on the RICO and PubLayNet datasets.}
    \label{tab:layout_results_additional}
    \resizebox{1.00\columnwidth}{!}{
    \begin{tabular}{lrrrrrrrr} 
        \toprule
        \multicolumn{9}{c}{RICO} \\
        \midrule
         &  \multicolumn{2}{c}{\shortstack{Unconditioned}} && \multicolumn{2}{c}{\shortstack{Category\\Conditioned}} && \multicolumn{2}{c}{\shortstack{Category$+$Size\\Conditioned}} \\ 
      Method                                    &  \abbalignment               & \abboverlap          & $\quad$ & \abbalignment        & \abboverlap &$\quad$ & \abbalignment & \abboverlap \\
        \midrule
       LayoutTransformer   & 0.037                       & 0.542             && -         & -  && - & - \\
         LayoutFormer\texttt{\char`+\char`+}           & 0.051                    & {0.546}     && \bftab{0.124}               & 0.537 && - & - \\
         NDN-none & - & - && 0.560 & {0.550} && - & -  \\
         LayoutDM                                     & 0.143                        & 0.584             && 0.222                & 0.598  && {0.175} &    0.606   \\
         DLT                                          & 0.271                     & 0.571             &&  0.303                       &  0.616 && 0.332 &  0.609 \\
         LayoutDiffusion                               & 0.069  & 0.502 && \bftab{0.124}    &  0.491   && - & - \\
         LayoutFlow                              & \bftab{0.150}           & 0.498             && 0.176      &  0.517 &&   0.283        & 0.523 \\ 
         \midrule
         Ours                              & {0.198}        & \bftab{0.443}            &&  {0.215}      &  \bftab{0.461} &&  \bftab{0.204}       & \bftab{0.490}  \\ 
         \midrule
        & \multicolumn{4}{c}{Alignment} &  \multicolumn{4}{c}{Overlap}\\
        \midrule\\
        Validation Data & \multicolumn{4}{c}{0.093} & \multicolumn{4}{c}{0.466} \\
        \bottomrule
    \end{tabular}
    % }
    % %\label{tab:layout_results_rico}
    
    %     \resizebox{0.9\columnwidth}{!}{
\quad
\begin{tabular}{lrrrrrrrr}
        \toprule
        \multicolumn{9}{c}{PubLayNet} \\
        \midrule
        &  \multicolumn{2}{c}{\shortstack[c]{Unconditioned}} && \multicolumn{2}{c}{\shortstack[c]{Category\\Conditioned}} && \multicolumn{2}{c}{\shortstack[c]{Category$+$Size\\Conditioned}} \\  
      Method                                    &  \abbalignment               & \abboverlap          & $\quad$ & \abbalignment        & \abboverlap &$\quad$ & \abbalignment & \abboverlap \\
        \midrule
       LayoutTransformer   & 0.067                       & 0.005             && -         & -  && - & - \\
         LayoutFormer\texttt{\char`+\char`+}           & 0.228                    & {0.001}     && \bftab{0.025}               & 0.009 && - & - \\
         NDN-none & - & - && 0.350 & 0.170 && - & -  \\
         LayoutDM                                     & 0.180                        & 0.132             && 0.267                & 0.139  && 0.246 &    0.160   \\
         DLT            & 0.117         & {0.036}             &&  0.097                       &  0.040 && 0.130 &  0.053 \\
         LayoutDiffusion                               & 0.065  & \bftab{0.003} && 0.029    &  \bftab{0.005}   && - & - \\
         LayoutFlow                              & \bftab{0.057}           &      0.009         &&       {0.037} & 0.011 &&   \bftab{0.041}        & 0.031 \\ 
         \midrule
         Ours                              & {0.094}        & 0.008            &&  0.088      &  {0.013} &&  {0.081}       & \bftab{0.027}  \\ 
        \midrule
        & \multicolumn{4}{c}{Alignment} &  \multicolumn{4}{c}{Overlap}\\
        \midrule\\
        Validation Data & \multicolumn{4}{c}{0.022} & \multicolumn{4}{c}{0.003} \\
        \bottomrule
    \end{tabular}
    }
    % \label{tab:layout_results}
\end{table*}

Alignment and Overlap capture the geometric aspects of the generations. As per \cite{guerreiro2025layoutflow}, we judge both metrics with respect to a reference dataset, which in our case is the validation dataset. We see that there is no consistent trend with respect to these metrics among models. Further, note that most of the reported models use specialized losses to ensure better performance with respect to these metrics; our model achieves comparable performance despite not using any specialized losses. Our framework can be used in tandem with domain-specific losses to improve the performance on these geometric metrics.

\subsection{Generated Examples}
Table \ref{tab:layout_gen_examples} shows generated samples on PubLayNet dataset on the three tasks of Unconditioned, Category-conditioned and Category+Size conditioned.

\begin{table}[h!]
    \centering
    \begin{tabular}{ccc}
        \multicolumn{1}{c}{\textbf{Unconditioned Generation}} &  \multicolumn{1}{c}{\textbf{Category-conditioned Generation}} & \multicolumn{1}{c}{\textbf{Category+Size-conditioned Generation}}\\
        % First Column
        \begin{minipage}{0.3\textwidth}
            \centering
            \includegraphics[width=0.5\textwidth]{images/appendix_pub_layout_uncond1.pdf}
            \vspace{0.5cm}
            \includegraphics[width=0.5\textwidth]{images/appendix_pub_layout_uncond2.pdf}
        \end{minipage} &
        % Second Column
        \begin{minipage}{0.3\textwidth}
            \centering
            \includegraphics[width=0.5\textwidth]{images/appendix_pub_layout_cond_c_1.pdf}
            \vspace{0.5cm}
            \includegraphics[width=0.5\textwidth]{images/appendix_pub_layout_cond_c_2.pdf}
        \end{minipage} &
        % Third Column
        \begin{minipage}{0.3\textwidth}
            \centering
            \includegraphics[width=0.5\textwidth]{images/appendix_pub_layout_cond_cs_1.pdf}
        
            \vspace{0.5cm}
            \includegraphics[width=0.5\textwidth]{images/appendix_pub_layout_cond_cs_2.pdf}
        \end{minipage} \\
    \end{tabular}
    \caption{Generated Layouts on PubLayNet Dataset}
    \label{tab:layout_gen_examples}
\end{table}

\subsection{Training Details}
We train a Dis-Co DiT model with the configuration in Table \ref{tab:layout_arch}.

\begin{table}[h]
    \centering
    \begin{tabular}{c c}
    \toprule
    % \multicolumn{2}{c}{} \\
    % \midrule
   Number of Generalized DiT Blocks  & 6  \\
   Number of Heads  & 8 \\
   Model Dimension & 512 \\
   MLP Dimension & 2048 \\
   Time Embedding Input Dimension & 256 \\
   Time Embedding Output Dimension & 128 \\
   \bottomrule
\end{tabular}
    \caption{Model configuration for Layout Generation}
    \label{tab:layout_arch}
\end{table}

We use the AdamW optimizer \cite{loshchilov2018decoupled} (with $\beta_1 = 0.9$, $\beta_2 = 0.999$ and $\epsilon = 10^{-8}$) with no weight decay and with no dropout.  We use EMA with decay $0.9999$.  We set the initial learning rate to 0 and warm it up linearly for 8000 iterations to a peak learning rate of $10^{-4}$; a cosine decay schedule is then applied to decay it to $10^{-6}$ over the training steps. For PubLayNet, we train for $4$ Million iterations with a batch size of $4096$, whereas for RICO, we train for $1.1$ Million iterations with a batch size of $4096$. By default, the sequence is noised for $4$ rounds $(T = 120)$; each continuous vector is noised $200$ times per round. We use pad tokens to pad the number of elements to 20 if a layout has fewer elements.

\paragraph{Data sampling and pre-processing:}
Since we train a single model for all three tasks (unconditional, class conditioned, class and size conditioned), we randomly sample layouts for each task by applying the appropriate binary mask required for the state-space doubling strategy. We begin training by equally sampling for all three tasks; during later stages of training, it may help to increase the fraction of samples for harder tasks to speed up training. For instance, we found that for the RICO dataset, doubling the fraction of samples for unconditional generation after $700$k iterations results in better performance in unconditional generation (while maintaining good performance in the other two tasks) when training for $1.1$ Million iterations. Further, each bounding box is described as $[x_i, y_i, l_i, w_i]$, where $(x_i, y_i)$ denotes the positions of the upper-left corner of the bounding box and $(l_i, w_i)$ denotes the length and width of the bounding box respectively. Note that $0 \leq x_i,y_i,l_i,w_i \leq 1 $ since the dataset is normalized. We further re-parameterize these quantities using the following transformation:
$$ g(x) = \log\left(\frac{x}{1-x} \right) $$
Note that we clip $x$ to $[10^{-5}, 1-10^{-5}]$ so that $g(x)$ is defined throughout. We then use this re-parameterized version as the dataset to train the diffusion model. While inference, the predicted vectors are transformed back using the inverse transformation:
$$ h(x) = g^{-1}(x) = \left(\frac{e^{x}}{1+e^{x}} \right) $$

\subsection{Ablations}
\label{app:subsec:abl_layout}

Unless specified otherwise, all the results reported in ablations use top-$p$ sampling with $p = 0.99$ and do not use the ReDeNoise algorithm at inference. From preliminary experiments, we found top-p sampling and ReDeNoise to only have marginal effects on the FID score; hence, we did not tune this further. For all layout generation experiments, we noise the sequence in a round-robin fashion, and in each round, $\Pi(\phi)$ is constant for discrete tokens across all positions. Similarly, $K_{i_t}^{t}$ which is the number of continuous noising steps per round, is constant across all positions per round. Hence, from here on, we use sequences of length $r$. where $r$ is the total number of noising rounds to denote $\Pi(\phi)$ and $K_{i_t}^{t}$ values for that particular round. By default, we choose $\Pi(\phi)$ to be $[0.5, 0.5, 0.5, 0.5]$, where the $4$ element sequence, which we refer to as the discrete noise schedule, denotes noising for 4 rounds with $\Pi(\phi)$ for the round chosen from the sequence. Similarly, the default value of $K_{i_t}^{t}$ is chosen to be $[200, 200, 200, 200]$, and we refer to this sequence as the continuous noising steps. Let us denote $\sum_{t}K_{i_t}^{t} $ as $K$. Note that $K$ is same across positions since we assume same number of continuous noising steps across positions per round. Given $K$, we define the following as the cosine schedule for $\beta$ (denoted by $\text{cosine}(a, b)$):
$$ \beta(j) = b + 0.5(a - b)(1 + \cos(\left(\frac{j}{K} \right)\pi)) $$
where $j$ is the total number of continuous noising steps at sequence time $t$ and element time $k$. We use $\text{cosine}(0.0001, 0.03)$ as the default schedule. We also define a linear noise schedule for $\beta$ ($\beta$ (denoted by $\text{lin}(a, b)$)):
$$ \beta(j) = a + (b - a)(1 + (\left(\frac{j}{K} \right))) $$
Further, we report only the unconditional FID for PubLayNet/RICO in the ablations as this is the most general setting.
\paragraph{Interleaving pattern:}
We broadly considered two interleaving patterns. In the first pattern, the bounding box vectors of each item was treated as a separate vector to form the interleaving pattern $[t_1, p_1, t_2, p_2, \dots, t_n, p_n ]$, where $t_i \in \mathbb{N}$ is the discrete item type and $p_i \in \mathbb{R}^{4}$ is its corresponding bounding box description ($p_i = [x_i, y_i, l_i, w_i]^\top$).  This interleaving pattern leads to $20$ discrete elements and $20$ continuous vectors per layout, resulting in a sequence of length $40$. In the second pattern, the bounding box vectors of all the $n$ items were bunched together as a single vector to form the interleaving pattern $[t_1, t_2, \dots, t_n, p^c ]$, where $p^c \in \mathbb{R}^{4n}$ is a single vector which is formed by concatenating the bounding box vectors of all $n$ items. This interleaving pattern leads to $20$ discrete elements and $1$ continuous vector per layout, resulting in a sequence of length $21$. We compare FID scores on unconditional generation on PubLayNet with these two interleaving patterns in Table \ref{tab:abl_layout_interleaving}.

\begin{table}[h]
    \centering
    \begin{tabular}{c c c c c}
    \toprule
   Interleaving Pattern & Disc. Noise Schedule & Cont. Noise Schedule & Cont. Noise Steps & FID \\
   \midrule
   Positions separate  &  $[0.5, 0.5, 0.5, 0.5]$ & $\text{cosine}(0.0001, 0.03)$ & $[200, 200, 200, 200]$ & 8.76 \\
   Positions together &  $[0.5, 0.5, 0.5, 0.5]$ & $\text{cosine}(0.0001, 0.03)$ & $[200, 200, 200, 200]$ & 14.21 \\
   Positions together &  $[0.35, 0.5, 0.5, 0.5]$ & $\text{cosine}(0.0001, 0.03)$ & $[200, 200, 200, 200]$ & 13.59 \\
   Positions together &  $[0.75, 0.5, 0.5, 0.5]$ & $\text{cosine}(0.0001, 0.03)$ & $[200, 200, 200, 200]$ & 13.99 \\
   Positions together &  $[0.99, 0.9, 0.8, 0.5, 0.5, 0.5]$ & $\text{cosine}(0.0001, 0.03)$ & $[150, 150, 150, 150, 150, 150]$ & 25.38 \\
   Positions together &  $[0.9, 0.75, 0.5, 0.5, 0.25]$ & $\text{cosine}(0.0001, 0.015)$ & $[500, 500, 500, 500, 500]$ & 17.86 \\
   \bottomrule
\end{tabular}
    \caption{Ablation on Interleaving Pattern}
    \label{tab:abl_layout_interleaving}
\end{table}

We see that despite tuning multiple hyperparameters for noise schedules, having the positions together leads to worse results than having the positions separate. Hence, we use the interleaving pattern of having the positions separate for all further experiments.

\paragraph{$\abs{\mathcal{X}}$-ary classification v/s Binary classification:} \label{app:par_xary_binary}
We compare the two strategies for training the discrete denoiser, $\abs{\mathcal{X}}$-ary classification and Binary classification (as described in \ref{sec:training}), on the unconditional generation task in the RICO dataset. The results are given in Table \ref{tab:abl_layout_loss}.

\begin{table}[h]
    \centering
    \begin{tabular}{c c}
    \toprule
   Discrete Loss Considered & FID \\
   \midrule
    $\abs{\mathcal{X}}$-ary Cross Entropy &  3.51 \\
   Binary Cross Entropy  & 2.62 \\
   \bottomrule
\end{tabular}
    \caption{Ablation on choice of discrete loss function}
    \label{tab:abl_layout_loss}
\end{table}


\paragraph{Discrete and continuous noise schedules:}
We evaluate the unconditional FID scores on PubLayNet and RICO for multiple configurations of discrete and continuous noise schedules. We report the results in Tables \ref{tab:abl_publaynet_noising} and \ref{tab:abl_rico_noising}.

\begin{table}[h]
    \centering
    \begin{tabular}{c c c c c}
    \toprule
   Disc. Noise Schedule & Cont. Noise Schedule & Cont. Noise Steps & FID \\
   \midrule
   $[0.5, 0.5, 0.5, 0.5]$ & $\text{lin}(0.0001, 0.02)$ & $[200, 200, 200, 200]$ & 13.19 \\
    $[0.5, 0.5, 0.5, 0.5]$ & $\text{lin}(0.0001, 0.035)$ & $[200, 200, 200, 200]$ & 10.62 \\
    $[0.5, 0.5, 0.5, 0.5]$ & $\text{cosine}(0.0001, 0.03)$ & $[200, 200, 200, 200]$ & 8.86 \\
    $[0.5, 0.5, 0.5, 0.5]$ & $\text{cosine}(0.0001, 0.03)$ & $[100, 100, 300, 300]$ & 8.32 \\
    $[0.5, 0.5, 0.5, 0.5]$ & $\text{cosine}(0.0001, 0.03)$ & $[25, 25, 50, 700]$ & 8.68 \\
    $[0.5, 0.5, 0.5, 0.5]$ & $\text{cosine}(0.0001, 0.06)$ & $[10, 10, 10, 370]$ & 12.78 \\
    $[0.75, 0.5, 0.25, 0.25]$ & $\text{cosine}(0.0001, 0.03)$ & $[10, 10, 10, 770]$ & 10.06 \\ 
     $[0.5, 0.5, 0.5, 0.5]$ & $\text{cosine}(0.0001, 0.025)$ & $[10, 10, 10, 970]$ & 9.67 \\
    $[0.5, 0.5, 0.5, 0.5]$ & $\text{cosine}(0.0001, 0.02)$ & $[10, 10, 10, 1170]$ & 10.83 \\
    $[0.9, 0.75, 0.5, 0.5, 0.25]$ & $\text{cosine}(0.0001, 0.06)$ & $[50, 50, 50, 50, 50, 50]$ & 9.10 \\
    $[0.5, 0.5, 0.5, 0.5, 0.5, 0.5]$ & $\text{cosine}(0.0001, 0.03)$ & $[10, 10, 10, 10, 10, 850]$ & 10.42 \\
    $[0.99, 0.9, 0.8, 0.5, 0.25, 0.05]$ & $\text{cosine}(0.0001, 0.03)$ & $[400, 400, 70, 10, 10, 10]$ & 17.69 \\
   
    
   \bottomrule
\end{tabular}
    \caption{Ablation on Discrete and Continuous Noise Schedules - PubLayNet}
    \label{tab:abl_publaynet_noising}
\end{table}

\begin{table}[]
    \centering
    \begin{tabular}{c c c c c}
    \toprule
   Disc. Noise Schedule & Cont. Noise Schedule & Cont. Noise Steps & FID \\
   \midrule
   $[0.5, 0.5, 0.5, 0.5]$ & $\text{cosine}(0.0001, 0.03)$ & $[10, 10, 10, 770]$ & 2.54 \\
    $[0.5, 0.5, 0.5, 0.5]$ & $\text{cosine}(0.0001, 0.06)$ & $[10, 10, 10, 370]$ & 3.67 \\
    $[0.5, 0.5, 0.5, 0.5]$ & $\text{cosine}(0.0001, 0.05)$ & $[10, 10, 10, 570]$ & 3.35 \\
    $[0.5, 0.5, 0.5, 0.5]$ & $\text{cosine}(0.0001, 0.03)$ & $[300, 300, 100, 100]$ & 5.13 \\
    $[0.5, 0.5, 0.5, 0.5]$ & $\text{cosine}(0.0001, 0.03)$ & $[100, 100, 300, 300]$ & 4.33 \\
    $[0.9, 0.8, 0.7, 0.5, 0.5, 0.5]$ & $\text{cosine}(0.0001, 0.03)$ & $[10, 10, 10, 10, 380, 380]$ & 3.88 \\
    
   \bottomrule
\end{tabular}
    \caption{Ablation on Discrete and Continuous Noise Schedules - RICO}
    \label{tab:abl_rico_noising}
\end{table}

From the ablations, it seems like for layout generation, noising the discrete tokens faster than the continuous vectors gives better performance. This could be because denoising the bounding boxes faster allows the model to make the element type predictions better. 

\newpage

\paragraph{Best configuration:} We obtain the best results with the configuration in Table \ref{app:tab:layout_config_best}.

\begin{table}[!h]
    \centering
    \begin{tabular}{c c c}
    \toprule
    Hyperparameter & PubLayNet & RICO\\
    \midrule
    % \multicolumn{2}{c}{} \\
    % \midrule
   Interleaving Pattern  & Positions separate & Positions separate  \\
   Discrete Noise Schedule & $[0.5, 0.5, 0.5, 0.5]$ & $[0.5, 0.5, 0.5, 0.5]$ \\
   Continuous Noising Steps & $[100, 100, 300, 300]$ &  $[10, 10, 10, 770]$ \\
   Continuous Noise Schedule &$\text{cosine}(0.0001, 0.03)$  \\
   Top-p & 0.99\\
   \bottomrule
\end{tabular}
    \caption{Best configuration for Layout Generation}
    \label{app:tab:layout_config_best}
\end{table}


\newpage

\section{Molecule Generation}
\label{app:mol_gen}

\subsection{Other Baselines}
\label{app:subsec:baselines}
\cite{vignac2023midi} proposes to generate 2D molecular graphs in tandem with 3D positions to allow better molecule generation. Our numbers cannot be directly compared with this work since they use a different list of allowed bonds, as well as use formal charge information. We also note that our framework can also be used to generate 2D molecular graphs along with 3D positions; we can also make use of the rEGNNs and uniform adaptive schedule proposed in \cite{vignac2023midi}. Hence, our framework can be thought of as complementary to \cite{vignac2023midi}. Similarly, \cite{peng2023moldiff} proposes to use the guidance of a bond predictor to improve molecule generation. Again, we cannot directly compare the numbers since they use a dedicated bond predictor to make bond predictions instead of a look-up table. The idea of bond predictor can also be incorporated in our framework seamlessly; hence our framework is again complementary to this work.

\subsection{Training Details}
\label{app:mol_train_details}

We train a Dis-Co DiT model with the following configuration:

\begin{table}[h]
    \centering
    \begin{tabular}{c c}
    \toprule
    % \multicolumn{2}{c}{} \\
    % \midrule
   Number of Generalized DiT Blocks  & 8  \\
   Number of Heads  & 8 \\
   Model Dimension & 512 \\
   MLP Dimension & 2048 \\
   Time Embedding Input Dimension & 256 \\
   Time Embedding Output Dimension & 128 \\
   \bottomrule
\end{tabular}
    \caption{Model configuration for QM9}
    \label{tab:qm9_arch}
\end{table}


We use the AdamW optimizer (with $\beta_1 = 0.9$, $\beta_2 = 0.999$ and $\epsilon = 10^{-8}$) with no weight decay and with no dropout.  We use EMA with decay $0.9999$.  We set the initial learning rate to 0 and warm it up linearly for 8000 iterations to a peak learning rate of $10^{-4}$; a cosine decay schedule is then applied to decay it to $10^{-6}$ over the training steps. For QM9, we train for $2.5$ Million iterations with a batch size of 2048. We use pad tokens to pad the number of atoms to 29 if a molecule has fewer atoms.

\paragraph{Distance-based embedding for atom positions:}
We adapt the distance embedding part from the EGCL layer proposed in \cite{hoogeboom2022equivariant}. Consider a molecule with $N$ atoms; let us denote the atom position of the $i^\text{th}$ atom as $x_i$. Then, we begin by computing the pairwise distance between the $i^\text{th}$ atom and all the other atoms (including the $i^\text{th}$ atom itself) to get an $N-$ dimensional vector $d_i$. $d_i$ is fed into the Generalized DiT block and embedded to a vector of size $D$, where $D$ is the model dimension, using a linear projection. This $D$ dimensional array is processed as usual by the block and at the end of the block, it is projected back into an $N$ dimensional vector, which we call $m_i$, using another linear layer. Then, we modify $x_i$ as follows:
$$ x_i  \leftarrow x_i + \sum_{j \neq i} \frac{x_i - x_j}{d_{ij} + 1} m_{ij} $$
where $d_{ij}$ denotes the $j^\text{th}$ element of $d_i$ and $m_{ij}$ denotes the $j^\text{th}$ element of $m_i$. The distance $d_i$ is now recomputed using the modified $x_i$ and the process is repeated for each block. After the final block, we subtract out the initial value of $x_i$ from the output. 

\subsection{Ablations}

Unless specified otherwise, all the results reported in ablations use top-$p$ sampling with $p = 0.99$ and do not use the ReDeNoise algorithm at inference. For all molecule generation experiments, we noise the sequence in a round-robin fashion, and in each round, $\Pi(\phi)$ is constant for discrete tokens across all positions. Similarly, $K_{i_t}^{t}$ which is the number of continuous noising steps per round, is constant across all positions per round. By default, we choose $\Pi(\phi)$ to be $[0.5, 0.5, 0.5, 0.5]$, where the $4$ element sequence, which we refer to as the discrete noise schedule, denotes noising for 4 rounds with $\Pi(\phi)$ for the round chosen from the sequence. Similarly, the default value of $K_{i_t}^{t}$ is chosen to be $[200, 200, 200, 200]$, and we refer to this sequence as the continuous noising steps. Let us denote $\sum_{t}K_{i_t}^{t} $ as $K$. Note that $K$ is same across positions since we assume same number of continuous noising steps across positions per round. Given $K$, we use the following noise schedule for $\beta$:
$$ \beta(j) = 0.03 + 0.5(0.0001 - 0.03)(1 + \cos(\left(\frac{j}{K} \right)\pi)) $$
where $j$ is the total number of continuous noising steps at sequence time $t$ and element time $k$. We denote this noise schedule as $\text{cosine}(0.0001, 0.03)$.

\paragraph{Interleaving pattern:}
We broadly considered two interleaving patterns. In the first pattern, the atom positions of each atom was treated as a separate vector to form the interleaving pattern $[z_1, p_1, z_2, p_2, \dots, z_n, p_n ]$, where $z_i \in \mathbb{N}$ is the discrete atomic number and $p_i \in \mathbb{R}^{3}$ is its corresponding atom position. This interleaving pattern results in 29 discrete tokens and 29 continuous vectors. In the second pattern, the atom positions of all the $n$ atoms were bunched together as a single vector to form the interleaving pattern $[z_1, z_2, \dots, z_n, p^c ]$, where $p^c \in \mathbb{R}^{3n}$ is a single vector which is formed by concatenating the atom positions of all $n$ atoms.This interleaving pattern results in 29 discrete tokens and 1 continuous vector. The atom and molecule stability for these two configurations are given in Table \ref{tab:abl_qm9_interleaving}.
\begin{table}[h]
    \centering
    \begin{tabular}{c c c}
    \toprule
   Interleaving Pattern & Atom. Stability & Mol. Stability \\
   \midrule
   Positions separate  &  88.99 & 28.9 \\
   Positions together  & 98.07 & 83.83 \\
   \bottomrule
\end{tabular}
    \caption{Ablation on Interleaving Pattern}
    \label{tab:abl_qm9_interleaving}
\end{table}

As we can see, having the atom positions together helps improve performance by a large margin; we hypothesize that this could be due to the fact that having the positions together allows the model to capture the symmetries of the molecules better. We choose the interleaving pattern with the positions together for all further experiments.

\paragraph{DDPM v/s DDIM:}
\label{app:par:ddpm_ddim}
We evaluate both DDPM and DDIM using the positions together interleaving pattern. The results are given in Table \ref{tab:abl_qm9_sampling}. DDPM outperforms DDIM by a large margin and hence we use DDPM for all experiments.

\begin{table}[h]
    \centering
    \begin{tabular}{c c c}
    \toprule
   Saampling Strategy& Atom. Stability & Mol. Stability \\
   \midrule
   DDIM  &  94.84 & 61.29 \\
   DDPM  & 98.07 & 83.83 \\
   \bottomrule
\end{tabular}
    \caption{Ablation on Sampling Strategy}
    \label{tab:abl_qm9_sampling}
\end{table}

\paragraph{Distance-based atom position embedding:} 
As we discussed in $\ref{app:mol_train_details}$, we use a distance-based embedding for the atom positions. We tried directly using the positions, as well as using both by concatenating distance along with the positions.  The atom and molecule stability for these two configurations are given in Table \ref{tab:abl_qm9_embedding}.
\begin{table}[h]
    \centering
    \begin{tabular}{c c c}
    \toprule
   Embedding & Atom. Stability & Mol. Stability \\
   \midrule
   Position  &  91.87 & 55.93 \\
   Distance  & 98.07 & 83.83 \\
   Position + Distance & 95.54 & 68.15 \\
   \bottomrule
\end{tabular}
    \caption{Ablation on embedding}
    \label{tab:abl_qm9_embedding}
\end{table}

As we can see, using the distance embedding leads to the best results. This could be due to the fact that molecules inherently have rotation symmetry, which distance-based embeddings capture more naturally. This could also be due to the fact that both atom and molecule stability are metrics which rely on the distance between atoms and allowing the model to focus on the distance allows it to perform better. Hence, we choose the distance-based atom position embedding for all further experiments.

\paragraph{Sequence time sampling:}

While the sequence time $t$ is typically sampled uniformly between $0$ and $T-1$, note that for the interleaving pattern with the positions together, only one sequence timestep per round corresponds to noising continuous vectors since we have $n$ discrete tokens and $1$ continuous vector. This may make it slower for the model to learn the reverse process for the continuous vector. Hence, we also try a \textit{balanced} sequence time sampling strategy, where we sample $t$ such that the time steps where continuous vector is noised is sampled with probability $0.5$. For the same number of training steps, performance of both strategies are detailed in Table \ref{tab:abl_qm9_seq_time}.
\begin{table}[h]
    \centering
    \begin{tabular}{c c c}
    \toprule
  Sequence Time Sampling & Atom. Stability & Mol. Stability \\
  \midrule
  Uniform sampling  &  97.92 & 79.78 \\
  Balanced sampling  & 98.24 & 84.47 \\
  \bottomrule
\end{tabular}
    \caption{Ablation on Sequence Time Sampling}
    \label{tab:abl_qm9_seq_time}
\end{table}

Since the balanced sampling strategy leads to better performance, we choose this strategy for all further experiments.

\paragraph{Discrete noise schedule and continuous noising steps:}
We fix the total number of noising rounds in the forward process as $4$, the total number of continuous noising steps as $800$ and the $\beta$ schedule as $\text{cosine}(0.0001, 0.03)$ based on initial experiments. The discrete noise schedule and continuous noising steps are then varied.
\begin{table}[h]
    \centering
    \begin{tabular}{c c c c}
    \toprule
  Discrete Noise Schedule & Continuous Noising Steps &  Atom. Stability & Mol. Stability \\
  \midrule
  $[0.5, 0.5, 0.5, 0.5]$  & $[200, 200, 200, 200]$ & 98.07 & 83.83 \\
  $[0.5, 0.5, 0.5, 0.5]$  & $[100, 100, 300, 300]$ & 97.63 & 79.40 \\
  $[0.5, 0.5, 0.5, 0.5]$  & $[300, 300, 100, 100]$ & 97.93 & 81.37 \\
  $[0.5, 0.5, 0.5, 0.5]$  & $[100, 300, 100, 300]$ & 98.08 & 83.08 \\
  $[0.75, 0.5, 0.5, 0.25]$ & $[100, 200, 200, 300]$ & 98.13 & 83.00 \\
  $[0.85, 0.5, 0.5, 0.25]$ & $[50, 250, 200, 300]$ & 98.14 & 81.99 \\
  \bottomrule
\end{tabular}
    \caption{Ablation on Noise Schedules}
    \label{tab:abl_qm9_noise}
\end{table}
Despite trying out multiple schedules, the default schedule of $[200, 200, 200, 200]$ and $[0.5, 0.5, 0.5, 0.5]$ give the best results; we use these noise schedules for further experiments. Results are given in Table \ref{tab:abl_qm9_noise}.

\paragraph{Effect of ReDeNoise:}
We examine the effect of ReDeNoise algorithm at inference. Preliminary results indicated that noising and denoising for more than one round does not improve performance. Hence, we apply ReDeNoise for one round, but do multiple iterations of the noising and denoising. We observe the following:
\begin{table}[h]
    \centering
    \begin{tabular}{c c c}
    \toprule
  No. of times ReDeNoise is applied & Atom. Stability & Mol. Stability \\
  \midrule
  No ReDeNoise  &  97.94 & 80.24 \\
  1x & 98.23  & 83.42 \\
  2x & 98.37  & 85.17 \\
  3x & 98.46 & 85.78 \\
  4x  & 98.48 & 86.20 \\
  5x & 98.52 & 86.49 \\
  6x & 98.60 & 87.11 \\
  7x & 98.48 & 86.30 \\
  \bottomrule
\end{tabular}
    \caption{Ablation on ReDeNoise (unbalanced sequence time sampling)}
    \label{tab:abl_qm9_redenoise_unbalanced}
\end{table}
\begin{table}[!h]
    \centering
    \begin{tabular}{c c c}
    \toprule
  No. of times ReDeNoise is applied & Atom. Stability & Mol. Stability \\
  \midrule
  No ReDeNoise  &  98.24 & 84.47 \\
  6x  & 98.74 & 89.46 \\
  \bottomrule
\end{tabular}
    \caption{Ablation on ReDeNoise (balanced sequence time sampling)}
    \label{tab:abl_qm9_redenoise}
\end{table}
ReDeNoise improves performance upto 6 iterations, after which the metrics saturate. However, we see that there is a substantial improvement in the moelcular stability metric on using ReDeNoise. Table \ref{tab:abl_qm9_redenoise_unbalanced} gives the results of ReDeNoise in the unbalanced sequence time sampling setting. Since we observed performance improvement till $6$ rounds, we used this for further experiments. The results for balanced sequence time sampling is given in Table \ref{tab:abl_qm9_redenoise}.

\newpage
\paragraph{Effect of Top-p sampling:}
We vary top-p sampling value at inference and examine the effects in Table \ref{tab:abl_qm9_topp}.
\begin{table}[!h]
    \centering
    \begin{tabular}{c c c}
    \toprule
 Top-p & Atom. Stability & Mol. Stability \\
  \midrule
  0.8  &  98.60 &  88.5 \\
  0.9  & 98.90 & 90.74 \\
  0.99  & 98.74 & 89.46\\
  \bottomrule
\end{tabular}
    \caption{Ablation on Top-p}
    \label{tab:abl_qm9_topp}
\end{table}


\paragraph{Best configuration:} After all the above ablations, we obtain the best results with the following configuration:

\begin{table}[!h]
    \centering
    \begin{tabular}{c c}
    \toprule
    % \multicolumn{2}{c}{} \\
    % \midrule
   Interleaving Pattern  & Positions together  \\
   Atom Position Embedding  & Distance-based \\
   Sequence Time Sampling & Balanced \\
   Discrete Noise Schedule & $[0.5, 0.5, 0.5, 0.5]$ \\
   Continuous Noising Steps & $[200, 200, 200, 200]$ \\
   Continuous Noise Schedule &$\text{cosine}(0.0001, 0.03)$  \\
   ReDeNoise & 6x  \\
   Top-p & 0.9\\
   \bottomrule
\end{tabular}
    \caption{Best configuration for QM9}
    \label{app:tab:mol_config_best}
\end{table}

\section{Boolean Satisfiability Problem}
\label{app:3sat}

\subsection{Training Details} 
\label{app:sat_train_details}

We trained models of three different sizes (6M, 85M, and 185M parameters), whose configurations are summarized in Table \ref{tab:sat_model_configuration}. Each model was trained for 1M steps on the combined dataset with $n\in {6,\dots,20}$. For the experiments where a separate model was trained for each $n$ (corresponding to Table~\ref{tab:sat_n_5_7_9_accuracy}), the batch size was increased from 8192 to 16384 and trained for 200K steps.
A gradual noising schedule of $[0.99, 0.9, 0.8, 0.5, 0.5, 0.25]$ was used for the discrete noising process in all SAT experiments.
\begin{table}[ht]
\centering
\begin{tabular}{lccc}
\toprule
\textbf{Parameter} & \textbf{6M} & \textbf{85M} & \textbf{185M} \\
\midrule
Number of DiT Blocks & 4 & 12 & 24 \\
Number of Heads & 8 & 12 & 16 \\
Model Dimension & 336 & 744 & 768 \\
MLP Dimension & 1344 & 2976 & 3072 \\
Time Embedding Input Dim & 256 & 256 & 256 \\
Time Embedding Output Dim & 128 & 128 & 128 \\
Learning Rate & 2e-4 & 7.5e-5 & 5e-5 \\
Batch Size & 8192 & 8192 & 4096 \\
\bottomrule
\end{tabular}
\caption{Model Configurations for Different Parameter Sizes for Boolean Satisfiability Problem}
\label{tab:sat_model_configuration}
\end{table}

Here DiT Block \cite{peebles2023scalable} is a modified transformer block designed to process conditional inputs in diffusion models. For Boolean Satisfiability (SAT), these blocks evolve variable assignments and clause states while incorporating diffusion timestep information through specialized conditioning mechanisms.\\

Adaptive Layer Norm (adaLN-Zero) \cite{xu2019adaptivelayernormalization}: Dynamically adjusts normalization parameters using timestep embeddings:
\begin{equation}
    \text{AdaLN}(h,t) = t_s \cdot \text{LayerNorm}(h) + t_b
\end{equation}
where $t_s$, $t_b$ are learned projections from timestep $t$. The \textit{adaLN-Zero} variant initializes residual weights ($\alpha$) to zero, preserving identity initialization for stable training.\\

Time-conditioned MLP: Processes normalized features with gated linear units (GLU), scaled by the diffusion timestep.

We use the AdamW optimizer \cite{loshchilov2018decoupled} (with $\beta_1 = 0.9$, $\beta_2 = 0.999$ and $\epsilon = 10^{-8}$) with no weight decay and with no dropout.  We use EMA with decay $0.9999$.

\subsection{Data Generation}
We follow the procedure of \citet{ye2024autoregressiondiscretediffusioncomplex} to create a large dataset of 15M satisfiable 3-SAT instances covering $n \in {6,\dots,20}$. Each instance is generated by: 
\begin{enumerate}
\item Sampling clauses where each clause has exactly three variables, chosen uniformly at random from the $n$ available. 
\item Randomly deciding whether each variable in the clause appears in complemented or non-complemented form. 
\end{enumerate} 
After generating the clauses, we run a standard SAT solver to ensure each instance is satisfiable, discarding any unsatisfiable cases. Finally, the data is split into training and test sets, with multiple checks to prevent overlap.


\subsection{Accuracy Trend During Training}
\label{app:sat_accuracy_trend}

\begin{figure}[t]
  \centering
  \includegraphics[width=0.6\columnwidth]{images/appendix_sat_n20_clauses_accuracy_trend.pdf}
  \caption{Evolution of the model’s SAT accuracy and number of satisfied clauses over training for random 3-SAT instances with $n=18$ on 185M model.}
  \label{fig:sat_n_18_accuracy_trend}
\end{figure}

Figure~\ref{fig:sat_n_18_accuracy_trend} illustrates how the SAT accuracy evolves over training for a model trained on instances, showing for $n=18$ as a representative example. In the early stages (roughly the first half of training), the accuracy remains near zero, even as the model steadily improves in satisfying individual clauses. This indicates that the model initially learns partial solutions that satisfy a growing fraction of the clauses. Once the model begins consistently satisfying nearly all clauses in an instance, accuracy jumps sharply, reflecting that the assignments finally meet all the constraints simultaneously.

\newpage
\section{Licenses and Copyrights Across Assets}
\label{app:licenses}

\begin{enumerate}
    \item PubLayNet Benchmark
    \begin{itemize}
        \item Citation:~\cite{zhong2019publaynet}
        \item Asset Link: \href{https://github.com/ibm-aur-nlp/PubLayNet}{[link]}
        \item Lincense: \href{https://github.com/ibm-aur-nlp/PubLayNet/blob/master/LICENSE.md}{[link]}
    \end{itemize}
    \item RICO Benchmark 
    \begin{itemize}
        \item Citation: ~\cite{deka2017rico}
        \item Asset Link: \href{http://www.interactionmining.org/rico.html}{[link]}
        \item License:\href{http://www.interactionmining.org/rico_copyright.txt}{[link]}
    \end{itemize}
    \item QM9 Benchmark
    \begin{itemize} 
        \item Citation: ~\cite{ramakrishnan2014quantum}
        \item Asset Link: \href{http://quantum-machine.org/datasets/}{[link]}
        \item License: \href{https://creativecommons.org/licenses/by/4.0/}{[link]}
    \end{itemize}
    \item Random 3-SAT benchmark
    \begin{itemize}
        \item Citation: ~\cite{ye2024autoregressiondiscretediffusioncomplex}
        \item Asset Link:
        \href{https://github.com/HKUNLP/diffusion-vs-ar?tab=readme-ov-file#usage}{[link]}
        \item License:
        \href{https://github.com/HKUNLP/diffusion-vs-ar/blob/main/LICENSE}{[link]}
    \end{itemize}
    \item PySAT: Python toolkit for prototyping with SAT solvers
    \begin{itemize}
        \item Citation: ~\cite{imms-sat18}
        \item Asset Link: \href{https://github.com/pysathq/pysat}{[link]}
        \item License: \href{https://github.com/pysathq/pysat/blob/master/LICENSE.txt}{[link]}
    \end{itemize}
\end{enumerate}





\end{document}
