\section{Forward process: Generating $s^{(t)}$ directly}
\label{app:sec:fwd_prcs}
For training the model for denoising at sequence time $t$ (and element time $k$ if we are denoising a continuous vector), we need access to:
\begin{itemize}
    \item $(s^{(t-1)}, s^{(t)})$ if $s^{(t)}_{i_t}$ is discrete 
    \item $(s^{(t, k)}, \epsilon)$ if $s^{(t, k)}_{i_t}$ is continuous
\end{itemize}
Note that $\epsilon$ is as defined in \eqref{eq:diff_relation}. Once you have $s^{(t-1)}$, $s^{(t)}$ can be generated by applying one discrete noising step, by sampling $z_{t-1}$. $\epsilon$ is required to compute $s^{(t, k)}_{i_t}$ from $s^{(t, k)}_{i_t}$ and hence having access to $s^{(t, k)}$ would imply access to $\epsilon$. Hence, if we can directly sample $s^{(t, k)}$ without going through all intermediate timesteps, the model can be trained efficiently to denoise at time $t$.

Let us denote by $m_{j}^{(t, k)}$ the total number of times an element at position $j$ has been noised by sequence time $t$ and element time $k$. Further, let $\tau_j^t = \{t' \in \{0, 1, \dots, t\};i_{t'} = j  \}$ denote the set of all sequence timesteps $t'$ at which position $j$ was visited prior to (and including) sequence time $t$.

For discrete elements, $ m_{j}^{(t, k)} = \abs{\tau_j^t}$. That is, for discrete elements, the number of noising steps is equal to the number of visits at that position by  time $t$ (Note that element time $k$ is irrelevant for discrete noising ).

For continuous vectors, $m_{j}^{(t, k)} = \sum_{t' \in \tau_j^t} K^{t'}_{j}$, provided $j \neq i_t$. That is, for continuous vectors which are not being noised at time $t$, the total number of noising steps are obtained by summing up the element times for all prior visits at that position. For $j = i_t$, $m_{j}^{(t, k)} = \sum_{t' \in \tau_j^t - \{t\}} K^{t'}_{j} + k$. That is, if a continuous vector is being noised at time $t$, the number of noising steps for that vector is obtained by summing up the element times for all prior visits as well as the current element time.

Since the noising process for each element is independent of other elements, to describe the generation of $s^{(t)}$(or $s^{(t, k)}$) from $s^{0}$,  it is sufficient to describe generating $s_{j}^{(t)}$(or $s_{j}^{(t, k)}$) from $s_{j}^{(0)}$ individually for each $j$. 

\paragraph{If $s_{j}^{(t)}$ is discrete:\\}

Recall from subsection \ref{subsec:fnp} that $\Pi_{t}(\phi)$ denotes the probability of sampling the token $\phi$ at sequence time $t$. Further, assume $\Pi_t(\cdot | \mathcal{X})$ is same for all $t$. Define $p_{j}^t = 1 - \prod_{t' \in \tau_j^t} (1 - \Pi_{t'}(\phi))$. Sample $z \sim \Pi(\cdot|\mathcal{X})$  Then:
\begin{align*}
    s^{(t)}_{j} = 
    \begin{cases}
    s^{(0)}_{j},& \text{with probability } 1- p_j^t\\
    z,& \text{ with probability } p_j^t\\
    \end{cases}
\end{align*}
The above follows from the fact that each flip for $t' \in \tau_j^t$ is an independent Bernoulli trial and hence, even if there is one success among these $m_{j}^{(t, k)}$ trials, the token is noised to $\Pi(\cdot|\mathcal{X})$.

\paragraph{If $s_{j}^{(t, k)}$ is continuous:\\}

Following subsection \ref{subsec:con_den}, we define the continuous noise schedule for the continuous vector at position $j$ as  $\beta_j = \{ \beta_i; i \in \{0, 1, \dots, m_{j}^{(T, 0)}-1 \} \}$. Let $\beta_j[i]$ denote the $i^\text{th}$ element of $\beta_j$. Define $\bar{\alpha}_j = \{ \prod_{i' \leq i}(1 - \beta_j[i']); i \in \{0, 1, \dots, m_{j}^{(T, 0)}-1 \} \}$.  Let $\bar{\alpha}_j[i]$ denote the $i^\text{th}$ element of $\bar{\alpha}_j$. Then, following \eqref{eq:diff_relation}, we have:
\begin{align*}
    s^{(t, k)}_{i_t} = \left(\sqrt{\bar{\alpha}_{i_t}[m_{i_t}^{(t, k)}]}\right) s^{(0)}_{i_t} + \left(\sqrt{1 - \bar{\alpha}_{i_t}[m_{i_t}^{(t, k)}]}\right) \epsilon
\end{align*}
where $\epsilon \sim \mathcal{N}(0, \mathbf{I} )$.

The forward process can thus be thought of as the block $\texttt{FwdPrcs}$ with the following input and output:\\
\begin{tabular}{l l l}
  Input: $(s^{(0)}, t,  \{i_{\tau}\}_{\tau = 0}^t, \{\Pi_{\tau}\}_{\tau = 0}^t)$ & Output: $(s^{(t)}, z_{t}, s^{(t+1)})$ & if $s^{(0)}_{i_{t}}$ is discrete \\
  Input: $(s^{(0)}, t, k, \{i_{\tau}\}_{\tau = 0}^t, \{\beta_{j}\}_{j = L_1+1}^{L_2})$ & Output: $(s^{(t, k+1)}, \epsilon)$ & if $s^{(0)}_{i_t}$ is continuous \\
\end{tabular}


\newpage
