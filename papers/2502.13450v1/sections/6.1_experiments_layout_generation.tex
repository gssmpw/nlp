\subsection{Layout Generation}

\subsubsection{Background}
Layout generation aims to generate coherent arrangements of UI elements (e.g., buttons, text blocks) or document components
(e.g., titles, figures, tables) that satisfy both functional requirements and aesthetic principles. This problem is important 
in various applications of graphic design and interface prototyping.

Formally, each layout is a set of $N$ elements $\{ e_i \}_{i=1}^N$. Each element $e_i$ is represented by a discrete category $t_i \in \mathbb{N}$  and a continuous bounding box vector $\mathbf{p}_i \in \mathbb{R}^4$. We use the parameterization $\mathbf{p}_i = [x_i,\, y_i,\, l_i,\, w_i]^\top$, where $(x_i,\, y_i)$ represents the upper-left corner of the bounding box, and $(l_i,\, w_i)$ its length and width, respectively.

\subsubsection{Experimental Setup}
We adopt a setup similar to \cite{guerreiro2025layoutflow} for standardized comparison to existing layout generation methods.

\paragraph{Datasets:}
We evaluate our method on two popular layout generation datasets:
\\
 1. PubLayNet \cite{zhong2019publaynet}: Contains layouts of scientific documents annotated with 5 element categories.
\\
 2. RICO \cite{deka2017rico}: Provides user-interface (UI) layouts with 25 element categories.\\
Following prior works \cite{jiang2023layoutformer++, zhang2023layoutdiffusion}, layouts containing more than 20 elements are discarded from the datasets.

\paragraph{Evaluation metrics:}
\label{par:layout_gen_eval_metrics}
Following previous works \cite{inoue2023layoutdm, chen2024towards}, we evaluate our method primarily using two metrics described below:
\begin{enumerate}[noitemsep,topsep=0pt]
    \item Frechet Inception Distance (FID) \cite{heusel2017gans}: This metric measures the distance between the generated and real data distributions by comparing the features extracted from a neural network. For FID calculation, we use the feature space from the same network with identical weights as in \cite{zhang2023layoutdiffusion}.
    \item Maximum Intersection over Union (mIoU) \cite{kikuchi2021constrained}: This calculates the maximum IoU between bounding boxes of the generated layouts and real data layouts with the same element category. 
\end{enumerate}
Results on additional evaluation metrics (Alignment and Overlap) are presented in Appendix \ref{app:layout_gen_full_results}. Baseline metrics in Table \ref{tab:layout_results} are reported as given in \cite{guerreiro2025layoutflow}.

\textbf{{Tasks:}}
Results are presented on three common layout generation tasks:
\begin{enumerate}[noitemsep,topsep=0pt]
    \item Unconditional Generation: No constraints.
    \item Category-Conditioned Generation: Element categories are specified.
    \item Category\,+\,Size-Conditioned Generation: Both element categories and sizes are specified.
\end{enumerate}



\paragraph{Baselines:}
We compare with state-of-the-art methods: Diffusion-based approaches include: LayoutDM \cite{inoue2023layoutdm}, which applies discrete diffusion to handle element categories and positions; LayoutDiffusion \cite{zhang2023layoutdiffusion}, employing iterative refinement with tailored noise schedules for layout attributes; and DLT \cite{levi2023dlt}, a hybrid model separating element categories and coordinates into distinct diffusion processes. Flow-based: LayoutFlow \cite{guerreiro2025layoutflow} leverages trajectory learning for efficient sampling. Non-diffusion baselines comprise: LayoutTransformer \cite{gupta2021layouttransformer} (autoregressive sequence generation), LayoutFormer++ \cite{jiang2023layoutformer++} (serializes constraints into token sequences for conditional generation), and NDN-none \cite{lee2020neural} (adversarial training without constraints).


\subsubsection{Results}
Table \ref{tab:layout_results} presents quantitative results across different tasks and datasets. On RICO, we outperform all baselines in category-conditioned and category+size-conditioned generation, with competitive performance on unconditioned generation. On PubLayNet, we achieve the best FID in unconditioned and category+size-conditioned generation.

Notably, IGD outperforms DLT, a discrete-continuous diffusion model which assumes factorizability of the reverse process, on most of the tasks in both datasets by a significant margin. This further demonstrates the effectiveness of our framework in comparison to existing discrete-continuous diffusion models.
We also note that models such as LayoutDM and LayoutDiffusion employ specialized diffusion processes tailored for layout generation, whereas we directly employ our discrete-continuous diffusion framework without further modifications. We refer to Appendix \ref{app:layout_gen} for further implementation details, example generations as well as extensive ablations.

\begin{table*}[t]
    \centering
    \caption{\textbf{Layout Generation:} Quantitative results on the RICO and PubLayNet datasets. Refer to section \ref{par:layout_gen_eval_metrics} for details on evaluation tasks and metrics.}
    \label{tab:layout_results}
    \resizebox{2.00\columnwidth}{!}{
    \begin{tabular}{lrrrrrrrr} 
        \toprule
        \multicolumn{9}{c}{RICO} \\
        \midrule
         &  \multicolumn{2}{c}{\shortstack{Unconditioned}} && \multicolumn{2}{c}{\shortstack{Category\\Conditioned}} && \multicolumn{2}{c}{\shortstack{Category$+$Size\\Conditioned}} \\ 
      Method                                    &  \abbfid                & \abbmiou          & $\quad$ & \abbfid        & \abbmiou &$\quad$ & \abbfid & \abbmiou \\
        \midrule
       LayoutTransformer   & 24.32                       & 0.587             && -         & -  && - & - \\
         LayoutFormer\texttt{\char`+\char`+}           & 20.20                    & \bftab{0.634}     && 2.48               & 0.377 && - & - \\
         NDN-none & - & - && 13.76 & {0.350} && - & -  \\
        % \cmidrule{2-9} 
         LayoutDM                                     & 4.43                        & 0.582             && 2.39                & 0.341  && {1.76} &    0.424   \\
         DLT                                          & 13.02                     & 0.566             &&  6.64                       &  0.326 && 6.27 &  0.424 \\
         LayoutDiffusion                               & 2.49  & 0.620 && {1.56}    &  0.345   && - & - \\
         LayoutFlow                              & \bftab{2.37}           & 0.570             && 1.48      &  0.322 &&   1.03        & 0.470 \\ 
         \midrule
         Ours                              & {2.54}        & 0.594            &&  \bftab{1.06}      &  \bftab{0.385} &&  \bftab{0.96}       & \bftab{0.524}  \\ 
        \bottomrule
    \end{tabular}
    
\quad   
\begin{tabular}{lrrrrrrrr}
        \toprule
        \multicolumn{9}{c}{PubLayNet} \\
        \midrule
        &  \multicolumn{2}{c}{\shortstack[c]{Unconditioned}} && \multicolumn{2}{c}{\shortstack[c]{Category\\Conditioned}} && \multicolumn{2}{c}{\shortstack[c]{Category$+$Size\\Conditioned}} \\  
      Method                                    &  \abbfid                & \abbmiou          & $\quad$ & \abbfid        & \abbmiou &$\quad$ & \abbfid & \abbmiou \\
        \midrule
       LayoutTransformer   & 30.05                       & 0.359             && -         & -  && - & - \\
         LayoutFormer\texttt{\char`+\char`+}           & 47.08                    & 0.401     && 10.15               & 0.333 && - & - \\
         NDN-none & - & - && 35.67 & 0.310 && - & -  \\
         LayoutDM                                     & 36.85                        & 0.382             && 39.12                & 0.348  && 29.91 &    0.436   \\
         DLT            & 12.70         & \bftab{0.431}             &&  7.09                       &  0.349 && 5.35 &  0.426 \\
         LayoutDiffusion                               & 8.63  & 0.417 && 3.73    &  0.343   && - & - \\
         LayoutFlow                              & 8.87           &      0.424         &&       \bftab{3.66} & 0.350 &&   1.26        & 0.454 \\ 
         \midrule
         Ours                              & \bftab{8.32}        & 0.419            &&  4.08      &  \bftab{0.402} &&  \bftab{0.886}       & \bftab{0.553}  \\ 
        \bottomrule
    \end{tabular}
    }
\end{table*}