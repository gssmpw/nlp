\subsection{Molecule Generation}

\subsubsection{Background}
Molecule generation aims to synthesize new valid molecular structures from a distribution learned through samples. Recently, generative models trained on large datasets of valid molecules have gained traction. In particular, diffusion-based methods have shown strong capabilities in generating discrete atomic types and their corresponding 3D positions.


We represent a molecule with $n$ atoms by $(z_i, \mathbf{p}_i)_{i=1}^{n}$, where $z_{i} \in \mathbb{N}$ is the atom's atomic number and $\mathbf{p}_i \in \mathbb{R}^{3}$ is the position. We focus on organic molecules with covalent bonding, where bond orders (single, double, triple, or no bond) between atoms are assigned using a distance-based lookup table following \cite{hoogeboom2022equivariant}.

\subsubsection{Experimental Setup}
We closely follow the methodology used in prior works \cite{hua2024mudiff} and \cite{hoogeboom2022equivariant} for 3D molecule generation. Further details are given below:

\paragraph{Datasets:}
We evaluate on the popular QM9 benchmark \cite{ramakrishnan2014quantum} which contains organic molecules with up to 29 atoms and their 3D coordinates. We adopt the standardized 100K/18K/13K train/val/test split to ensure fair comparison with prior works. We generate all atoms, including hydrogen, since this is a harder task.

\paragraph{Evaluation metrics:}
\label{par:mol_gen_eval_metrics}
We adopt four metrics following prior works \cite{hua2024mudiff} and \cite{hoogeboom2022equivariant}: 
\begin{enumerate}[noitemsep,topsep=0pt]
    \item Atom Stability: The fraction of atoms that satisfy their valency. Bond orders (single, double, triple, no bond) are determined from pairwise atomic distances using a distance-based lookup table given in \cite{hoogeboom2022equivariant}.
    \item Molecule Stability: Fraction of molecules where all atoms are stable.
    \item Validity: RDKit-based \cite{landrum2006rdkit} molecular sanitization checks, as in \cite{hoogeboom2022equivariant}. These checks include: chemical plausibility of bond angles and lengths, absence of disconnected components, kekulization of aromatic rings, and more.
    \item Uniqueness: Fraction of unique and valid molecules.
\end{enumerate}

\paragraph{Baselines:}
We compare with state-of-the-art methods: E-NF (equivariant normalizing flows) \cite{pmlr-v119-kohler20a} models molecular generation via invertible flow transformations. G-SchNet \cite{NIPS2019_8974} employs an autoregressive architecture with rotational invariance. Diffusion-based approaches include EDM \cite{hoogeboom2022equivariant} (with SE(3)-equivariant network \cite{fuchs2020se}) , GDM \cite{hoogeboom2022equivariant} (non-equivariant variant of EDM), and DiGress \cite{vignac2023digress} (discrete diffusion for atoms/bonds without 3D geometry). GeoLDM \cite{xu2023geometric} leverages an equivariant latent diffusion process, while MUDiff \cite{hua2024mudiff} unifies discrete (atoms/bonds) and continuous (positions) diffusion with specialized attention blocks. While \cite{peng2023moldiff} and \cite{vignac2023midi} are also diffusion based methods, they are not directly comparable due to reasons we list in \ref{app:subsec:baselines}.
% \sy{Mention about this complimentary thread with NN trained distance lookup for bonds}

\subsubsection{Results}
Table~\ref{tab:qm9_results} compares our method with others on QM9. Notably, \emph{without relying} on specialized equivariant diffusion or domain-specific architectures, IGD attains strong performance across four key metrics. Our model achieves 98.9\% atom stability and 95.4\% molecular validity, equaling or surpassing other methods. Our model achieves a molecule stability of 90.5\%, surpassing the baselines. While not the best in `uniqueness', our approach still yields more than 95\% unique samples among the valid molecules. In addition, we observe noticeably lower standard deviations than most baselines, reflecting consistent performance.

Notably, applying \emph{ReDeNoise} at inference yielded an improvement of 4.99\% in molecular stability. Further implementation details, and ablation studies examining design choices such as the interleaving pattern, discrete and continuous noise schedules are presented in Appendix~\ref{app:mol_gen}.


 \begin{table}[t]
     \vspace{-8pt}
    %\centering
    \caption{\textbf{Molecule Generation:} Quantitative results on QM9 benchmark.
    We report mean (standard deviation) across 3 runs, each with 10K generated samples. Refer to section \ref{par:mol_gen_eval_metrics} for details on evaluation metrics.}
    \label{tab:qm9_results}
        \scalebox{.7}{
    \begin{tabular}{l c c c c}
    \toprule
     Method  & Atom stable (\%) & Mol stable (\%) & Validity (\%) & Uniqueness (\%)\\
      \midrule
        {E-NF}  & 85.0 & 4.9 & 40.2 & 39.4 \\
       {G-Schnet}  & 95.7 & 68.1 & 85.5 & 80.3  \\
        % \midrule % Not sure about this but maybe midrule may help to distinguish them
        GDM  & 97.6 & 71.6 & 90.4 & 89.5  \\ 
        EDM  & ${98.7}\pm{0.1}$ & {82.0}$\pm{0.4}$ & 91.9 $\pm 0.5$ & 90.7 $\pm 0.6$\\ 
        DiGress  & ${98.1}\pm{0.3}$ & {79.8}$\pm{5.6}$ & \textbf{95.4 }$\pm 1.1$ & {97.6} $\pm 0.4$\\ 
        % MDM  & ${98.6}$ & $\textbf{{91.9}}$ & - & -\\ 
        GeoLDM  & \textbf{{98.9}}$\pm{0.1}$ & {89.4}$\pm{0.5}$ & 93.8 $\pm 0.4$ & 92.7 $\pm 0.5$\\ 
        MUDiff  & ${98.8}\pm{0.2}$ & {89.9}$\pm{1.1}$ & 95.3 $\pm 1.5$ & \textbf{99.1} $\pm 0.5$\\ 
        \midrule
        Ours & \textbf{98.9}$ \pm {0.03}$ & \textbf{90.5} $\pm 0.15$ & \textbf{95.4} $\pm 0.2$ & 95.6 $\pm 0.1$ \\
        \midrule
        Data   & 99.0 & 95.2 & 99.3 & 100.0 \\
     \bottomrule
    \end{tabular}}
     \vspace{-15pt}
  \end{table}