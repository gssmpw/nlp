\section{Related Work}
We will first review the most relevant online POMDP solvers, followed by those specifically designed for $\rho$POMDPs.
	\subsection{State of the art online POMDP solvers}

		Online POMDP solvers for large or infinite state spaces typically approximate the belief using a set of state samples, commonly referred to as particles. 
		This particle-based representation is flexible and well-suited for capturing complex, multimodal beliefs \cite{Thrun05book}.
		For discussion purposes, these solvers can be broadly categorized into two groups: state simulators and belief simulators.
		See Figure \ref{fig:trees} for a simple illustration.

		\begin{figure}[tb]
			\centering
			\includesvg[width=0.4\textwidth]{Figures/Trees_updated.svg}
			\caption{Illustration of belief tree construction by a state simulator (left) and a belief simulator (right). New particles and new nodes are marked in red. The state simulator updates beliefs by adding new particles along the trajectory, while the belief simulator maintains fixed beliefs once created.}
			\label{fig:trees}
		\end{figure}

		% \begin{figure}[tb]
		% 	\centering
		% 	\includesvg[width=0.5\textwidth]{Figures/Trees_long.svg}
		% 	\caption{Illustration of belief tree construction by a state simulator (left) and a belief simulator (right). New particles and new nodes are marked in red. The state simulator updates beliefs by adding new particles along the trajectory, while the belief simulator maintains fixed beliefs once created.}
		% 	\label{fig:trees}
		% \end{figure}

		\textbf{State Simulators:}
		State simulators focus on simulating state trajectories directly, incrementally updating visited beliefs with new particles at each visitation.
		Examples of state simulators include:
  
		POMCP \cite{Silver10nips}, which extends the UCT algorithm \cite{Kocsis06ecml} to the POMDP framework.  
		DESPOT \cite{Somani13nips} and its successors \cite{Ye17jair,Garg19rss}, which use heuristics to guide the search process.  
		POMCPOW \cite{Sunberg18icaps}, which extends POMCP to continuous action and observation spaces by incorporating progressive widening.  
		LABECOP \cite{Hoerger21icra}, an algorithm for continuous observation spaces, extracts the belief from scratch for each sampled observation sequence.

		A common trait of these algorithms is that each time a belief node is visited, the belief is updated with additional particles. 
		Intuitively, this approach improves the belief representation in frequently visited nodes, aligning with the exploration-exploitation trade-off.

		\textbf{Belief Simulators:}  
		Belief simulators, on the other hand, treat POMDP belief states as nodes in an equivalent Belief-MDP. Examples include:  

		PFT-DPW \cite{Sunberg18icaps}, which represents each belief node with a fixed number of particles. This makes the approach simple to implement and particularly effective for belief-dependent rewards, as rewards are computed once upon node creation.  
		AdaOPS \cite{Wu21nips}, which dynamically adapts the number of particles per belief node and aggregates similar beliefs, achieving competitive results compared to other state-of-the-art solvers.

		A key limitation of belief simulators is their fixed belief representation, which does not improve over time. This inefficiency leads to unpromising regions of the search space receiving the same computational effort as promising ones. 
		Moreover, these algorithms are less flexible when planning times vary.  Given ample time, belief simulators can construct dense trees, but belief representations at individual nodes may remain suboptimal. 
		Under time constraints, however, they often produce shallow, sparse trees, as significant computational effort is spent maintaining fixed belief representations rather than effectively exploring the search space.
				
\subsection{$\rho$POMDP Solvers}

	Several algorithms have been proposed to address the challenges of online planning in $\rho$POMDPs.

	PFT-DPW \cite{Sunberg18icaps} was introduced to accommodate belief-dependent rewards in POMDPs, though it was not demonstrated for this application. 
	Building on PFT-DPW, IPFT \cite{Fischer20icml} introduced the concept of reward shaping using information-theoretic rewards. It reinvigorates particles at each traversal of posterior nodes and estimates information-theoretic rewards by a kernel density estimator (KDE).
	
	AI-FSSS \cite{Barenboim22ijcai} reduces the computational cost of information-theoretic rewards by aggregating observations, providing bounds on the expected reward and value function to guide the search. Despite this improvement, its approach remains constrained by a fixed observation branching factor and a fixed number of particles per belief node. 
	\cite{Zhitnikov24ijrr} introduce an adaptive multilevel simplification paradigm for $\rho$POMDPs, which accelerates planning by computing rewards from a smaller subset of particles while bounding the introduced error. While their current implementation builds upon PFT-DPW, future extensions could complement our approach.
	
	All the above algorithms belong to the belief-simulators family and share the limitation of fixed belief representations.  

	An exception, closely related to our work, is $\rho$POMCP \cite{Thomas21arxiv}, which extends POMCP to handle belief-dependent rewards by propagating a fixed set of particles from the root instead of simulating a single particle per iteration. Their approach includes variants such as Last-Value-Update (LVU), which use the most recent reward estimates to reduce bias, unlike POMCP’s running average.

	However, $\rho$POMCP is limited to discrete spaces and recomputes belief-dependent rewards from scratch whenever a belief node is updated. This is costly in general and especially in continuous spaces, where the number of particles in the belief can grow indefinitely. These limitations highlight the need for efficient incremental updates to avoid full recomputation—an issue directly addressed by our approach.
	% However, $\rho$POMCP is limited to discrete spaces and recomputes belief-dependent rewards from scratch, which is costly in general and especially in continuous spaces where belief nodes grow indefinitely. This underscores the need for efficient incremental updates—an issue our approach directly addresses.

% \subsection{$\rho$POMDP Solvers}

% 	Several algorithms have been proposed to address the challenges of online planning in $\rho$POMDPs.

% 	PFT-DPW \cite{Sunberg18icaps} was introduced to accommodate belief-dependent rewards in POMDPs, though it was not demonstrated for this application. 
% 	Building on PFT-DPW, IPFT \cite{Fischer20icml} reinvigorates particles during each traversal of posterior nodes and estimates the reward function using a kernel density estimator (KDE). 
% 	% However, this approach scales poorly due to the quadratic cost of KDE with the number of particles and lacks theoretical justification for the averaging of multiple reward estimates. 

% 	AI-FSSS \cite{Barenboim22ijcai} reduces the computational cost of information-theoretic rewards by aggregating observations, providing bounds on the expected reward and value function to guide the search. 
% 	Despite this improvement, their method is restricted to a fixed observation branching factor and a fixed number of particles per belief node.
% 	\cite{Zhitnikov24ijrr} present an adaptive multilevel simplification paradigm for $\rho$POMDPs, which accelerates planning with belief-dependent rewards by using a smaller subset of particles for reward calculations while bounding the error introduced by this simplification. 
% 	Although their implementation currently builds upon PFT-DPW, future extensions could complement our approach.

% 	All the algorithms above lie inside the belief-simulator family and thus share the same limitations.
% 	% \RB{Could potentialy remove these if too long}\cite{Flaspohler19ral} utilize Monte Carlo Tree Search (MCTS) with information-theoretic rewards to guide information gathering. Their work assumes fully observed agent states, with beliefs modeled as Gaussian Processes over unknown environmental phenomena. 
% 	% \cite{do2023information} extend POMCP by incorporating entropy to explicitly guide the search process without altering the reward function. While not directly a $\rho$POMDP solver, their approach is complementary to ours.

% 	Closely related to our work is \cite{Thomas21arxiv}, which introduces $\rho$POMCP, an extension of POMCP for belief-dependent rewards. Instead of simulating a single particle, $\rho$POMCP propagates a fixed set of particles from the root during each tree traversal. Their approach includes variants that rely on the most recent reward estimate to reduce bias, contrasting with POMCP’s running average.

% 	However, $\rho$POMCP is limited to discrete spaces and recalculates belief-dependent rewards from scratch whenever a belief node is updated. This is particularly costly in continuous spaces, where belief nodes contain an increasing number of particles. These limitations underscore the need for efficient methods that update belief-dependent rewards incrementally, avoiding full recomputation—an issue directly addressed by our approach