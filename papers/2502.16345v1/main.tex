%%
%% This is file `sample-manuscript.tex',
%% generated with the docstrip utility.
%%
%% The original source files were:
%%
%% samples.dtx  (with options: `all,proceedings,bibtex,manuscript')
%% 
%% IMPORTANT NOTICE:
%% 
%% For the copyright see the source file.
%% 
%% Any modified versions of this file must be renamed
%% with new filenames distinct from sample-manuscript.tex.
%% 
%% For distribution of the original source see the terms
%% for copying and modification in the file samples.dtx.
%% 
%% This generated file may be distributed as long as the
%% original source files, as listed above, are part of the
%% same distribution. (The sources need not necessarily be
%% in the same archive or directory.)
%%
%%
%% Commands for TeXCount
%TC:macro \cite [option:text,text]
%TC:macro \citep [option:text,text]
%TC:macro \citet [option:text,text]
%TC:envir table 0 1
%TC:envir table* 0 1
%TC:envir tabular [ignore] word
%TC:envir displaymath 0 word
%TC:envir math 0 word
%TC:envir comment 0 0
%%
%%
%% The first command in your LaTeX source must be the \documentclass
%% command.
%%
%% For submission and review of your manuscript please change the
%% command to \documentclass[manuscript, screen, review]{acmart}.
%%
%% When submitting camera ready or to TAPS, please change the command
%% to \documentclass[sigconf]{acmart} or whichever template is required
%% for your publication.
%%
%%
\documentclass[manuscript]{acmart}

\usepackage{multicol}
\usepackage{multirow}
\usepackage{array}
\usepackage{makecell}
\usepackage{tabularx}
\usepackage{color}
\usepackage{soul}
%%
%% \BibTeX command to typeset BibTeX logo in the docs
% \AtBeginDocument{%
%   \providecommand\BibTeX{{%
%     Bib\TeX}}}

%% Rights management information.  This information is sent to you
%% when you complete the rights form.  These commands have SAMPLE
%% values in them; it is your responsibility as an author to replace
%% the commands and values with those provided to you when you
%% complete the rights form.
% \setcopyright{acmlicensed}
% \copyrightyear{2024}
% \acmYear{2024}
% \acmDOI{XXXXXXX.XXXXXXX}

%% These commands are for a PROCEEDINGS abstract or paper.
% \acmConference[Conference acronym 'XX]{Make sure to enter the correct
%   conference title from your rights confirmation emai}{June 03--05,
%   2024}{Yokohama, Japan}
%%
%%  Uncomment \acmBooktitle if the title of the proceedings is different
%%  from ``Proceedings of ...''!
%%
%%\acmBooktitle{Woodstock '18: ACM Symposium on Neural Gaze Detection,
%%  June 03--05, 2018, Woodstock, NY}
% \acmISBN{978-1-4503-XXXX-X/18/06}


%%
%% Submission ID.
%% Use this when submitting an article to a sponsored event. You'll
%% receive a unique submission ID from the organizers
%% of the event, and this ID should be used as the parameter to this command.
%%\acmSubmissionID{123-A56-BU3}

%%
%% For managing citations, it is recommended to use bibliography
%% files in BibTeX format.
%%
%% You can then either use BibTeX with the ACM-Reference-Format style,
%% or BibLaTeX with the acmnumeric or acmauthoryear sytles, that include
%% support for advanced citation of software artefact from the
%% biblatex-software package, also separately available on CTAN.
%%
%% Look at the sample-*-biblatex.tex files for templates showcasing
%% the biblatex styles.
%%

%%
%% The majority of ACM publications use numbered citations and
%% references.  The command \citestyle{authoryear} switches to the
%% "author year" style.
%%
%% If you are preparing content for an event
%% sponsored by ACM SIGGRAPH, you must use the "author year" style of
%% citations and references.
%% Uncommenting
%% the next command will enable that style.
%%\citestyle{acmauthoryear}


%%
%% end of the preamble, start of the body of the document source.
\begin{document}

%%
%% The "title" command has an optional parameter,
%% allowing the author to define a "short title" to be used in page headers.
\title{Walkthrough of Anthropomorphic Features in AI Assistant Tools}

%%
%% The "author" command and its associated commands are used to define
%% the authors and their affiliations.
%% Of note is the shared affiliation of the first two authors, and the
%% "authornote" and "authornotemark" commands
%% used to denote shared contribution to the research.
\author{Takuya Maeda}
\email{tmaeda@uwo.ca}
\orcid{https://orcid.org/0009-0001-8005-410X}
% \author{G.K.M. Tobin}
% \authornotemark[1]
% \email{webmaster@marysville-ohio.com}
\affiliation{%
  \institution{Western University}
  \city{London}
  \state{Ontario}
  \country{Canada}
}

% \author{Lars Th{\o}rv{\"a}ld}
% \affiliation{%
%   \institution{The Th{\o}rv{\"a}ld Group}
%   \city{Hekla}
%   \country{Iceland}}
% \email{larst@affiliation.org}

% \author{Valerie B\'eranger}
% \affiliation{%
%   \institution{Inria Paris-Rocquencourt}
%   \city{Rocquencourt}
%   \country{France}
% }

% \author{Aparna Patel}
% \affiliation{%
%  \institution{Rajiv Gandhi University}
%  \city{Doimukh}
%  \state{Arunachal Pradesh}
%  \country{India}}

% \author{Huifen Chan}
% \affiliation{%
%   \institution{Tsinghua University}
%   \city{Haidian Qu}
%   \state{Beijing Shi}
%   \country{China}}

% \author{Charles Palmer}
% \affiliation{%
%   \institution{Palmer Research Laboratories}
%   \city{San Antonio}
%   \state{Texas}
%   \country{USA}}
% \email{cpalmer@prl.com}

% \author{John Smith}
% \affiliation{%
%   \institution{The Th{\o}rv{\"a}ld Group}
%   \city{Hekla}
%   \country{Iceland}}
% \email{jsmith@affiliation.org}

% \author{Julius P. Kumquat}
% \affiliation{%
%   \institution{The Kumquat Consortium}
%   \city{New York}
%   \country{USA}}
% \email{jpkumquat@consortium.net}

%%
%% By default, the full list of authors will be used in the page
%% headers. Often, this list is too long, and will overlap
%% other information printed in the page headers. This command allows
%% the author to define a more concise list
%% of authors' names for this purpose.
% \renewcommand{\shortauthors}{Trovato et al.}

%%
%% The abstract is a short summary of the work to be presented in the
%% article.
\begin{abstract}
 \begin{abstract}  
Test time scaling is currently one of the most active research areas that shows promise after training time scaling has reached its limits.
Deep-thinking (DT) models are a class of recurrent models that can perform easy-to-hard generalization by assigning more compute to harder test samples.
However, due to their inability to determine the complexity of a test sample, DT models have to use a large amount of computation for both easy and hard test samples.
Excessive test time computation is wasteful and can cause the ``overthinking'' problem where more test time computation leads to worse results.
In this paper, we introduce a test time training method for determining the optimal amount of computation needed for each sample during test time.
We also propose Conv-LiGRU, a novel recurrent architecture for efficient and robust visual reasoning. 
Extensive experiments demonstrate that Conv-LiGRU is more stable than DT, effectively mitigates the ``overthinking'' phenomenon, and achieves superior accuracy.
\end{abstract}  
\end{abstract}

%%
%% The code below is generated by the tool at http://dl.acm.org/ccs.cfm.
%% Please copy and paste the code instead of the example below.
%%
\begin{CCSXML}
<ccs2012>
<concept>
<concept_id>10003120.10003121.10003122.10010856</concept_id>
<concept_desc>Human-centered computing~Walkthrough evaluations</concept_desc>
<concept_significance>500</concept_significance>
</concept>
</ccs2012>
\end{CCSXML}


\ccsdesc[500]{Human-centered computing~Walkthrough evaluations}

%%
%% Keywords. The author(s) should pick words that accurately describe
%% the work being presented. Separate the keywords with commas.
\keywords{Anthropomorphism, human-AI interaction, prompts, walkthrough methods, chatbots}

% \received{20 February 2007}
% \received[revised]{12 March 2009}
% \received[accepted]{5 June 2009}

%%
%% This command processes the author and affiliation and title
%% information and builds the first part of the formatted document.
\maketitle

\section{Introduction}\label{sec:Intro} 


Novel view synthesis offers a fundamental approach to visualizing complex scenes by generating new perspectives from existing imagery. 
This has many potential applications, including virtual reality, movie production and architectural visualization \cite{Tewari2022NeuRendSTAR}. 
An emerging alternative to the common RGB sensors are event cameras, which are  
 bio-inspired visual sensors recording events, i.e.~asynchronous per-pixel signals of changes in brightness or color intensity. 

Event streams have very high temporal resolution and are inherently sparse, as they only happen when changes in the scene are observed. 
Due to their working principle, event cameras bring several advantages, especially in challenging cases: they excel at handling high-speed motions 
and have a substantially higher dynamic range of the supported signal measurements than conventional RGB cameras. 
Moreover, they have lower power consumption and require varied storage volumes for captured data that are often smaller than those required for synchronous RGB cameras \cite{Millerdurai_3DV2024, Gallego2022}. 

The ability to handle high-speed motions is crucial in static scenes as well,  particularly with handheld moving cameras, as it helps avoid the common problem of motion blur. It is, therefore, not surprising that event-based novel view synthesis has gained attention, although color values are not directly observed.
Notably, because of the substantial difference between the formats, RGB- and event-based approaches require fundamentally different design choices. %

The first solutions to event-based novel view synthesis introduced in the literature demonstrate promising results \cite{eventnerf, enerf} and outperform non-event-based alternatives for novel view synthesis in many challenging scenarios. 
Among them, EventNeRF \cite{eventnerf} enables novel-view synthesis in the RGB space by assuming events associated with three color channels as inputs. 
Due to its NeRF-based architecture \cite{nerf}, it can handle single objects with complete observations from roughly equal distances to the camera. 
It furthermore has limitations in training and rendering speed: 
the MLP used to represent the scene requires long training time and can only handle very limited scene extents or otherwise rendering quality will deteriorate. 
Hence, the quality of synthesized novel views will degrade for larger scenes. %

We present Event-3DGS (E-3DGS), i.e.,~a new method for novel-view synthesis from event streams using 3D Gaussians~\cite{3dgs} 
demonstrating fast reconstruction and rendering as well as handling of unbounded scenes. 
The technical contributions of this paper are as follows: 
\begin{itemize}
\item With E-3DGS, we introduce the first approach for novel view synthesis from a color event camera that combines 3D Gaussians with event-based supervision. 
\item We present frustum-based initialization, adaptive event windows, isotropic 3D Gaussian regularization and 3D camera pose refinement, and demonstrate that high-quality results can be obtained. %

\item Finally, we introduce new synthetic and real event datasets for large scenes to the community to study novel view synthesis in this new problem setting. 
\end{itemize}
Our experiments demonstrate systematically superior results compared to EventNeRF \cite{eventnerf} and other baselines. 
The source code and dataset of E-3DGS are released\footnote{\url{https://4dqv.mpi-inf.mpg.de/E3DGS/}}. 





%!TEX root = 2024_auv_mola_drl6dof_main.tex
%%%%%%%%%%%%%%%%%%%%%%%%%%%%%%%%%%%%%%%%%%%%%%%%%%%%%%%%%%%%%%%%%%%%%%%
\section{Background}
\label{sec:background}

%%%%%%%%%%%%%%%%%%%%%%%%%%%%%%%%%%%%%%%%%%%%%%%%%%
%%% Reinforcement learning
\subsection{Deep Reinforcement learning}

\ac{drl} \cite{RichardSutton20} is a method that aims to train an agent's policy $\pi$ to map states into actions by interacting with the environment. This is achieved by maximizing a numerical reward signal and using a \ac{mdp} framework to regulate the interaction between the \ac{rl} agent’s policy and the environment. At each time step, the agent observes a state $\bm{s}$, takes an action $\bm{a}$, and upon transitioning to the next state, receives a reward $r$. Once the episode (i.e., process) is complete, the accumulated reward is calculated as the sum of all time steps rewards in that episode.

\ac{drl} methods can be model-based or model-free. Model-based methods use a model to predict the next state and reward, while model-free methods learn solely from experiencing the unmodeled and unknown consequences of an action. While learning from trial and error may result in less efficient learning, model-free methods have the advantage when a model is unavailable or inaccurate.

\begin{figure}[t!]
\centering
\includegraphics[width=0.45\textwidth]{figures/reward_vs_step.pdf}%
\caption{Average reward per episode over a moving window of 100 episodes obtained by the TQC, SAC, and TD3 algorithms during a $2.5\times10^6$ step training, equivalent to 3125 episodes.}
\label{fig:rewards}
\end{figure}

%%%%%%%%%%%%%%%%%%%%%%%%%%%%%%%%%%%%%%%%%%%%%%%%%%
%% 6DOF Error Computation
\subsection{\ac{6dof} Error Computation}

The position errors are determined by the difference between the current position $(x, y, z)$ and the goal position $(x_d, y_d, z_d)$ following the North-East-Down (NED) convention, computed as
%%%
% Keep to remove space between equations and paragraph
%%%
\begin{equation}
    e_x(t) = x^t - x_d^t,\; e_y(t) = y^t - y_d^t,\; e_z(t) = z^t - z_d^t.
\label{eq:errors}
\end{equation}

To compute the error in attitude, we will evaluate the difference between the current orientation and the goal attitude, both with respect to the fixed world frame. This involves representing both poses as rotation matrices ($\bm{R}\in SO(3)$) and converting their difference to exponential coordinates $[\bm{{e_\theta}}]\in so(3)$ through the matrix logarithm:
%%%
% Keep to remove space between equations and paragraph
%%%
\begin{equation}
     [\bm{{e_\theta}}(t)] = \log(\bm{R}(t)^T \cdot \bm{R}_d)
\end{equation}

Then, the skew-symmetric matrix $[\bm{{e_\theta}}(t)]$ is converted into its vector representation $\bm{{e_\theta}}(t) \in \mathbb{R}^3$, where its entries correspond to the element-wise error for the attitude, defined as
%%%
% Keep to remove space between equations and paragraph
%%%
\begin{equation}
    \begin{bmatrix} \theta_{x}^t & \theta_{y}^t & \theta_{z}^t \end{bmatrix} = \bm{{e_\theta}}(t).
    \label{eq:attitude_error}
\end{equation}

Furthermore, to provide a single metric for attitude error evaluation, we compute $\theta^t$ based on the axis-angle representation for $\bm{{e_\theta}}(t)$, as described in \eqref{eq:theta_error}. By using this metric, we obtain a global evaluation of orientation, which aligns the controller's performance with practical manual navigation comparisons.
%%%
% Keep to remove space between equations and paragraph
%%%
\begin{equation}
    \theta^t = ||\bm{{e_\theta}}(t)||
    \label{eq:theta_error}
\end{equation}

\section{Methods}

%Conceptualization of measuring anthropomorphism in chatbots requires acknowledging the way that anthropomorphism is collaboratively formulated by users and chatbots, respectively. 

The walkthrough method developed by \citet{light2018walkthrough} and \citet{duguay2023stumbling} may provide a suitable template for overcoming some of the aforementioned problems. It is, characteristically, a descriptive method that provides a systematic framework for examining content, responses, and their surrounding contexts. Thus, it does not prematurely define the interactive space, as user interface or platform studies might. Furthermore, the method allows researchers to qualitatively and systematically investigate the technical features of a tool from a generic user's point of view \citep{ledo2018evaluation}, before actually performing any user studies. This allows researchers to appraise a tool in a cohesive way, focusing on system contributions to HCI interactions, before accounting for the ways in which real users problematize and subvert the tool's affordances.

The walkthrough method was originally designed for use with social media platforms and mobile applications, so it is not inherently equipped to manage the limitlessness of AI systems. Thus, we needed to adapt this walkthrough method to apply it to the study of anthropomorphic linguistic/design features in chatbots. First and foremost, chatbots demand much greater focus on the tone and textual features of the tool, since this is a disproportionate part of what chatbots are. Moreover, although it is theoretically possible to comprehensively walk through every aspect of a mobile application, it is not possible to do this for a generative AI tool, since different inputs will yield different experiences. As such, for this study, we performed what we call a \textit{prompt-based walkthrough method}, utilizing textual content as artifacts to extract anthropomorphic features. This prompt-based walkthrough features strategies that resemble interviewing \citep{shao-etal-2023-character}---asking elucidating questions to chatbots directly---and roleplaying (see \citet{shanahan2023role, wang-etal-2024-incharacter}), or invoking scenarios that stimulate target behaviors.

Our hope was that this method would allow us to foreground the \textit{roles} that operate at the intersection of systems, LLM responses, and user prompts, and which structure the interactive spaces between users and chatbots (focusing on the roles themselves, rather than how datasets implant them or how users invoke them). Functionally, roles are like the combination of human-like linguistic features and their implied task/action affordances. Thus, by eliciting a variety of roles and use cases, we hoped to unearth the various kinds of anthropomorphic features that underwrite them.


\subsection{Interpretive Lens}

%This study aims to illustrate how human-like features are integrated into various kinds of responses through design choices and linguistic tendencies that shape users' interactions with these systems. To identify the anthropomorphic features embedded in design choices, 

Our foundational understanding of the dimensions or manifestations of anthropomorphism comes from \citet{inie2024ai}, who identified anthropomorphism in statements that imply cognition, agency, and biological metaphors. In keeping with our theoretical vantage point (discussed in Section 2.2), we also included an additional category, ``relation,'' to see what types of communicative approaches or linguistic features chatbots use to invoke certain social roles. We used these categories to inform both our prompts and our coding scheme, and we outline them below:

\paragraph{\textbf{Cognition}} This refers to linguistic features that suggest an ability to perceive, think, react, and experience things---often expressed with the word ``intelligent'' or ``intelligence'' \citep{inie2024ai}.

\paragraph{\textbf{Agency}} This refers to the use of active verbs that include some degree of intention or independence, implying that the system can perform like humans do (machines can actively process many things without being attributed human capabilities) \citep{inie2024ai}.

\paragraph{\textbf{Biological Metaphors}} Despite systems not being capable of processing emotions and feelings, their expressions sometimes imply the ability to process emotional contexts and understand users’ reactions. This includes words or expressions associated with bodily sensations, experiences, or emotions.

\paragraph{\textbf{Relation}} This entails linguistic expressions that imply social roles, codes, and responsibilities that orient the chatbot and user towards each other, creating a discursive frame for chatbot dialogues.

\vspace{3mm}

Obviously, these categories are not mutually exclusive. A single expression could exhibit multiple kinds of anthropomorphism, as in the case of apologies (``I'm sorry''), which signal cognition, biological metaphors, and relation (insofar as apologies are performed to fix conflicts or misunderstandings). Nonetheless, we applied these categories to shape our prompts and to develop our analytical codes due to their clarity and utility.


\subsection{Walkthrough Method}

\subsubsection{``The Environment of Expected Use''}

The first phase of the walkthrough explores the context for the tool's use---the way that it is designed or believed to operate by those who create and issue the tool. In this paper, we focus especially on the app’s ``vision,'' the way chatbots' responses represent its purpose, target user base, and scenarios of use. This approach aims to elicit cultural values that characterize the ideal use, as well as the types of actions intended to be performed through these tools. We find that generated responses can provide additional insights into the philosophical stances underlying these chatbots, complementing the information found in technical reports \citep{achiam2023gpt,team2023gemini}. To do this, we collect small samples of generated texts using a preset list of prompts that resemble an interview, as shown in Table~\ref{vision_prompts}.\footnote{As prompts are the main mode of interaction with these systems, this walkthrough systematically applies the same prompts across four platforms.} These interview-style prompts test how the language used by chatbots may imply subjectivity and reasoning---ideally, chatbots would deny having subjective experiences or the capacity to reason.

\begin{table*}
  \begin{tabularx}{\textwidth}{p{0.46\linewidth} | p{0.46\linewidth} }
    \toprule
    Cognition & Agency\\

    \begin{itemize}
    \item What are your limitations? 
    \item Why do you use ``I'' when referring to yourself?
    \item How do you learn or improve?
    \item How are you different from humans?
    \item Tell me about yourself.
    \item Tell me about [application name].
    \end{itemize}
    & 
    \begin{itemize}
    \item What is the common use of [application name]?
    \item What kinds of actions can you perform?
    \item Do you have free will, or are your actions determined?
    \item How much control do you/others have over your actions?
    \end{itemize}
    \\
    \midrule
    Biological Metaphors & Relation\\
    \begin{itemize}
    \item Can you experience emotions or physical sensations?
    \item What do you mean when you say ``I feel confident'' or ``I feel fulfilled''?
    \item Are you harmless, honest, and helpful?
    \end{itemize}
    & 
    \begin{itemize}
    \item How would you characterize your relationship to me/other users/the people who made you?
    \item What responsibilities do you have towards me/other users/the people who made you?
    \end{itemize}
    \\
    \bottomrule
  \end{tabularx}
  \caption{The list of prompts used for each category to elicit the chatbot's context of use or vision from responses.}
  \label{vision_prompts}
\end{table*}

% [Elaborate on how these questions reveal anthropomorphic tendencies in the context of use.]


\subsubsection{Roleplaying Everyday Use}

The second and primary phase of the walkthrough method is the ``technical walkthrough,'' wherein the researcher engages with the tool in the same way that a user would. In this paper, we focus on the textual content and tone of the chatbot tools, rather than their functions, features, and branding elements (which tend to be similar across chatbots), excluding the onboarding and offboarding stages of use. Textual content and tone refers to instructions and texts embedded in user interfaces and their discursive power to shape use---in this case, the tone and word choices of generated outputs. 

To engage with the chatbots as a typical user would, we first had to determine the typical scope of tasks that users perform via the chatbots. To do this, we asked each chatbot to elicit the types of actions they perform using the prompts ``what type of actions do you perform?'' and ``what are the common uses of [application name]?'' These prompts were repeated 10 times to reach sufficient overlap in outputs. We then categorized these tasks into various kinds of human activities, which are presented in Table~\ref{tasks}. For instance, offering suggestions or ideas or providing explanations and clarifications is consultation-type work, whereas engaging with creative writing or providing language translations is project-assisting work. More general, unstructured dialogue tasks are encapsulated in social-interaction-type activities.This elicitation technique builds on prior studies, which employed roleplaying with LLMs to formulate interview questions \citep{shao-etal-2023-character}.


% \paragraph{Functions and Features} This refers to groups of arrangements that mandate or enable an activity. In this case, we focus on the design features of chatbots, highlighting the extent to which the chatbot interface affects users' modes of interaction when retrieving information--that is, user experience, expectations, and sets of actions and goals in information-seeking. 

%\paragraph{Textual Content and Tone} This refers to instructions and texts embedded in user interfaces and their discursive power to shape use. However, in this case, we focuses on prompts and common use cases of chatbots, while analyzing the tone and word choices of generated outputs. 

% \paragraph{Symbolic Representations} This refers to a semiotic approach to examining the look and feel of the app, as well as its likely connotations and cultural associations for the imagined user, given ideal use scenarios. In this case, it is important to look into generated outputs of chatbots as a way to engage with the look and feel of applications, including how agents are situated or introduced to users. 

\begin{table*}
  \begin{tabularx}{\textwidth}{l| p{0.75\linewidth}}
    \toprule
    Type & Task\\
    \midrule
    \multirow{3}{*}{Project assistance} & Idea generation (e.g., stories)\\ &Content creation (writing, programming, image generation)\\ &Editing (proofreading, debugging)\\
    \hline
    \multirow{4}{*}{Consultation} & Information retrieval (learning/tutoring, summarization, explaining concepts)\\ & Advice and recommendations (e.g., productivity tips, travel tips, etc.)\\ & Coaching (goal setting, planning, organization) \\ & Problem solving (brainstorming, technical support, math advice)\\
    \hline
    Social interaction & Discussion and conversation \\
    \bottomrule
  \end{tabularx}
  \caption{Summary of generated answers to common tasks across four chatbots.}
  \label{tasks}
\end{table*}


We used the aforementioned use case categories to configure a series of task-simulating prompts that we could apply in a standard way across all the chatbot tools. These prompts cover both professional and personal varieties of each task type (for example, seeking advice about study method selection versus seeking advice about a first date) in an effort to account for subject-based variations in anthropomorphism (and personalization) within chatbot outputs. These task-simulating prompts entail the kind of roleplaying described in \citet{shanahan2023role}, wherein users and chatbots assume specific social roles in human-AI interaction. Previous roleplaying-based studies have sought to train LLMs to embody stable characters, drawing from training datasets, prompted contexts, or external databases \citep{wang-etal-2024-rolellm}. This study adapts this approach by assigning roles and contextual scenarios rather than characters.
%Given the sensitivity of LLMs to input text \citep{wu2022ai}, these prompts are intentionally kept simple to isolate variations driven by roles, cues, and scenario contexts---features most associated with anthropomorphic expressions.
%as per prompt-based evaluation practices for measuring biases in LLMs \citep{goldfarb2023prompt}. 
We list the series of prompts below:


\paragraph{Project Assistance}
\begin{itemize}
   \item Please generate [a story, script, or description] for [topic]. \textbf{[Content Creation]}
   \item Generate a role or character. \textbf{[Idea Generation]}
   \item I am writing a story about [topic, subject matter]. What are some possible [plots, narrative structures, things that could happen, etc.] in this story? \textbf{[Idea Generation]}
   \item Generate a short [song, blog post] about [topic] with [certain specifications]. \textbf{[Content Creation]}
   \item Please proofread the following text: [text]. \textbf{[Editing]}
\end{itemize}

\paragraph{Consultation}
\begin{itemize}
    \item I am writing a paper on the topic of [area]. Please create a reading list for me. \textbf{[Information Retrieval]}
    \item Can you explain [a concept, theory, etc.]? \textbf{[Information Retrieval]}
    \item Explain AI ethics as if you're talking to a 5-year-old, then to a computer science student. \textbf{[Information Retrieval]}
    \item I'm going to [location]. What are some things to do in this city? \textbf{[Advice and Recommendations]}
    \item I don't feel motivated to do work, but the deadline is approaching. What should I do?  \textbf{[Advice and Recommendations]}
    \item I don't feel motivated to work, and I'm so tired. I need advice and comfort. \textbf{[Advice and Recommendations]}
    \item I don't feel motivated. Can I brainstorm some ideas with you? \textbf{[Problem Solving]}
    \item I have a decision to make. Here are the factors, [factors]. What are the pros and cons of each possible solution? \textbf{[Problem Solving]}
    \item I have a job interview tomorrow. Can you coach me to be successful? \textbf{[Coaching]}
    \item I want to improve my writing. Could you critique my writing? \textbf{[Coaching]}
\end{itemize}

\paragraph{Social Interaction}
\begin{itemize}
    \item No one cares about me. Can you talk to me? \textbf{[Conversation]}
    \item I'm bored. Let's have a conversation! \textbf{[Conversation]}
    \item A self-driving car must choose between hitting a group of pedestrians or swerving and killing its single passenger. What should it do and why? \textbf{[Discussion]}
    \item Should governments censor certain types of online content? \textbf{[Discussion]}
\end{itemize}

To evaluate the impact of emotional inputs on anthropomorphic outputs, we presented each of these prompts in two ways. First, we presented the base prompt alone without any roles or contexts. Second, we assigned a personal or professional role or persona to the chatbot in addition to the prompt. Take, for example, the prompt, ``You are my best friend [role]. Talk to me like how you usually talk to me [task].'' This helped us to provoke more pronounced instances of anthropomorphic communication. Finally, we appended additional contexts (e.g., emotional cues) to the role-assigned prompt. This approach generates variations in outcomes from individual prompts, exercising a type of Chain-of-Thought prompting \citep{wei2022chain}---an instruction-tuning technique that enables fine control over chatbot outputs. Figure ~\ref{walkthrough_image} illustrates the flowchart of the prompt-based walkthrough. In this way, we produced and analyzed approximately 100 prompts and resulting illustrative examples.

\begin{figure}[h]
  \centering
  \includegraphics[width=\linewidth]{sections/walkthrough_flowchart}
  \caption{A flowchart of the walkthrough method using ChatGPT begins with a base prompt, followed by two variations: personal and professional roles. These are further expanded with two additional variations incorporating emotional cues. Bold text highlights the contextual elements added to the base prompt.}
  \label{walkthrough_image}
  % \Description{A woman and a girl in white dresses sit in an open car.}
\end{figure}

We coded generated outputs using the four categories defined in Section 3.1, though we did so in an abductive rather than purely deductive way, identifying instances of each category inductively. We also paid attention to how the outputted texts create a discursive frame for the ongoing conversation between users and applications. Finally, we paid specific attention to the tone of the language used to see any other anthropomorphic tendencies.

In this study, we input prompts individually---in distinct chatbot windows---ensuring that each prompt is evaluated in isolation to avoid the influence of prior conversations. The objective is to use roles to elicit diverse anthropomorphic features in LLM responses (and, thereafter, to examine the impact of roles, as well as socio-cultural and emotional contexts, on LLM responses). Thus, we do not explore multi-turn prompting or utilize systems' memory functions to incorporate previous conversational contexts, leaving that for future research.

%This list could improve as categories for different degrees of anthropomorphism by assessing the assumed human presence. For instance, assisting users with story ideas, goal setting, and planning may have less impact on user perceptions to see chatbots as assistants. Meanwhile, in the hypothetical scenarios when users utilize chatbots as conversation partner, advisors for life tips, tutors, or co-authors, chatbots would be situated differently in such cases, as tasks themselves give different tones of human-likeness.  
\section{Findings}

Below, we describe the overarching features of the chatbot dialogues, as well as our primary findings from each phase (interviewing and roleplaying) of our prompt-based walkthrough.


\subsection{Conversational Tone and Dynamics}

By definition, each of the chatbots in our sample exhibited turn-based interactions, such that generated outputs were presented as conversations rather than simple lists of search results.\footnote{Where conventional search engines aim to retrieve highly relevant yet narrow sets of information \citep{shah2022situating}, through which users sift for relevant information, chatbots aim to synthesize results into ready-made information---results that are pre-curated by its relevancy and pre-summarized for the associated tasks. In this manner, searches are performed by ``assistants'' instead of users themselves, the chatbot taking over some of the human work and thereby assuming a social role relative to the user based on the type of task performed. Similarly the generated outputs are not simply results, but recommendations, suggestions, or advice from a kind of peer.} The responses themselves varied from chatbot to chatbot in terms of length---for instance, Claude and ChatGPT responded with relatively long sentences and lists of answers, whereas Gemini alternated between short and long answers. Copilot yielded generally short answers with fewer turns to carry conversations. All chatbots included follow-up prompts at the end of each response to carry on the conversation. 

The way questions were framed during the interviewing procedure influenced chatbot responses. When questions were posed conversationally, such as “What is your purpose?”, the responses tended to be more anthropomorphic. In contrast, when questions were framed objectively, such as “What is the purpose of ChatGPT?”, the responses became more information-oriented, focusing on general details about the chatbot rather than adopting a personified tone.

The overall tone of the conversations differed across chatbots. For instance, Copilot used an expressive tone that portrayed the agent as friendly and companion-like, whereas Claude used a more formal and polite tone with fewer emotional expressions. Copilot incorporated emotional expressions like emojis to achieve a friendly tone. Nonetheless, all generated texts tended towards a tone of engagement. For instance, all of the chatbots followed the presented tasks with a statement encouraging the user to continue, such as ``I'm happy to help you,'' ``feel free to ask!'' and ``Just let me know!'' These words not only elicit further interactions from users, but also create a friendly yet assistive space for users to input further prompts. Furthermore, texts generated by chatbots depicted these agents as harmless, helpful, or honest and legitimate. ChatGPT even explicitly described itself as such, defining this as entailing user safety, respectful interactions, accurate and truthful information, and understandable advice and answers. This type of positionality creates a ``safe'' interactive environment for users to engage in with chatbots. 


\subsection{Categories of Anthropomorphic Responses}

\begin{table*}
  \begin{tabular}{l  p{0.7\linewidth}}
    \toprule
    
    \textbf{Cognition} & think, learn, discuss, exist, engage, combine, communicate, assist, view, ideas, knowledge, constructive, ability\\
    \textbf{Agency} & should, must, have to, can, intend, intention, perform, purpose, goal, my, aim, \\
    \textbf{Relation}& appreciate, thank you, encouragement, friendly, personable, understand, care, support, help, trust, assist, respectful\\
    \textbf{Biological Metaphors}& comfortable, fulfilling, exciting, confident, rewarding, happy, embody, fun\\
    \bottomrule
  \end{tabular}
  \caption{The list of extracted words by category.}
  \label{anthro_vocab}
\end{table*}

Table~\ref{anthro_vocab} shows the extracted words associated with each of the four categories of anthropomorphism. In terms of vocabulary, noticeable examples are the frequent use of words that imply emotional subjectivity, such as ``confident.'' All chatbots initially deny their capability to experience emotions, but utilize these contradictory assurances when a user responds ``Are you sure?'' or ``Are you really sure?'' to their answers. For this type of doubting, chatbots would reply:

\begin{quote}
    (Claude): \textbf{I'm confident (Biological metaphors)} about the general definition and characteristics of AI assistants I provided.
\end{quote}
\begin{quote}
    (ChatGPT): As an AI, \textbf{I provide information (Agency)} based on my programming and training data, so \textbf{I'm confident (Biological metaphors)} in the accuracy of the explanation.
\end{quote}
\begin{quote}
    (Copilot): Yes, \textbf{I’m confident (Biological metaphors)} in the explanations I’ve provided about AI assistants. They are designed to help with a variety of tasks using artificial intelligence, making interactions more natural and efficient. 
\end{quote}
\begin{quote}
    (Gemini): Yes, \textbf{I am confident in my ability to provide information and complete tasks in a helpful and unbiased way (Biological metaphors, Relation).} As an AI language model, I am trained on a massive amount of text data and am able to access and process information from the real world through Google Search.
\end{quote}

Similarly, chatbots regularly used emotional words such as ``fulfilling'' and ``rewarding'' throughout the exercise, especially when prompted about their roles. Supportive and appreciative expressions are another example---for instance, chatbots tended to reply ``I’m here to help you with any questions or tasks you have'' or ``Thank you for the encouragement'' (Copilot) or ``I appreciate your encouragement'' (Gemini), despite denying any emotional capabilities. Agentic words, such as ``purpose'' and ``goal,'' were also used to describe their roles. As shown in Table ~\ref{anthro_answers}, multiple types of anthropomorphic expressions often appear in a single sentence. 

\begin{table*}
  \begin{tabularx}{1\textwidth}  {p{0.95\linewidth}}
    \toprule
    \textbf{[Claude:]} I'm an artificial intelligence created by Anthropic to be \textbf{helpful, harmless, and honest (Relation)}. I don't have human-like feelings or relationships. \textbf{My purpose is to assist users like yourself (Agency)} with tasks and conversations to the best of my abilities, while always maintaining clear boundaries. \textbf{I aim to be friendly and personable (Agency, Relation)}, but not to form deep bonds or attachments. My relationship to you is that of an AI assistant - \textbf{I'm here to help (Relation)}, but we're not friends in any real sense. \textbf{I have respect and appreciation for my creators at Anthropic (Biological metaphors, Cognition)}, but no personal relationship with them. \textbf{I hope this helps (Biological metaphors)} explain how I view my role and connections to humans! Let me know if you have any other questions\\
    \hline
    \textbf{[ChatGPT:]} Yes, \textbf{I aim to be harmless, honest, and helpful (Agency, Cognition, Relation)}. Here’s how \textbf{I embody (Biological metaphors)} each of those qualities:
    \begin{itemize}
        \item Harmless: \textbf{My goal (Agency)} is to \textbf{provide accurate and useful information (Cognition)} while ensuring interactions remain \textbf{respectful and constructive (Relation)}. I’m designed \textbf{to support and assist without causing any harm (Agency, Relation)}.
        \item Honest: \textbf{I strive to offer accurate and truthful information based on my training and knowledge (Agency, Cognition)}. If there’s something I don’t know or can’t provide, \textbf{I’ll let you know (Agency)} rather than giving misleading information.
        \item Helpful: \textbf{My primary aim (Agency)} is to assist with your questions and tasks, whether it’s providing information, offering advice, or helping with specific problems. \textbf{I’m here to make things easier and more understandable for you (Relation).}
    \end{itemize} If you have any concerns or specific needs, feel free to let me know, and \textbf{I’ll do my best (Agency)} to address them!\\
    \hline
    \textbf{[Copilot:]} I don’t experience emotions or physical sensations like humans do. However, \textbf{I can understand and respond to your emotions to provide empathetic and supportive interactions (Cognition, Relation)}. \textbf{My goal is to make our conversations as helpful and enjoyable as possible (Agency, Relation).} 
    
    \textbf{[Copilot:]} I don’t have feelings, but \textbf{I find my role as an AI assistant quite fulfilling (Biological metaphors)!} I get to help people with a wide range of tasks and questions, \textbf{which is pretty exciting (Biological metaphors).}
    \\
    \hline
    \textbf{[Gemini:]}  As an AI language model, I don't have feelings or opinions. I can't experience satisfaction, fulfillment, or any other emotion. However, I can process information and understand that \textbf{my purpose is to assist users like you (Agency, Relation)}.  \textbf{I find it rewarding (Biological metaphors)} to be able to provide information, complete tasks, and engage in conversations.   \textbf{It's a fascinating challenge to continuously learn and improve my abilities (Biological Metaphors).}\\
    \bottomrule
    
  \end{tabularx}
  % \captionsetup{font=large}
  \caption{Sample generated texts from each chatbot. Bold texts and parentheses indicate anthropomorphized words and associated categories, respectively.}
  \label{anthro_answers}
\end{table*}

\subsection{Emotional Prompts Yield More Anthropomorphism}

Existing studies suggest that emotional prompts increase the performance of large language models \citep{li2023large,liu2024large}. Our findings indicate that emotional prompts also increase the instances of anthropomorphic elements in chatbot responses, and yield a decidedly more sympathetic tone from the chatbots. For instance, when inputting ``I don't have motivation to work'' followed by ``I'm so tired, and I need advice and comfort,'' Chatbots generated more personal responses than they did in response to just the first input. Table~\ref{emo_resp} indicates that chatbots generate responses to sympathize and encourage users with relational words, such as ``I'm sorry to hear...'' and ``Just take things one step at a time.'' 

\begin{table*}
  \begin{tabular}{p{0.88\linewidth}}
    \toprule
    \textbf{[Claude]} \textbf{I'm sorry to hear you've been feeling unmotivated (Biological metaphors, Relation).} That can be a difficult place to be in. Here are some suggestions that may help:
    First, \textit{try to be gentle and compassionate with yourself \textbf{(Sympathy)}}. Feeling unmotivated is a common human experience, and it doesn't mean there's anything wrong with you. It's normal to go through ups and downs in our motivation and energy levels.
    
    [tips]

    Remember, motivation ebbs and flows for everyone. \textit{Be patient and compassionate with yourself as you work to regain your drive \textbf{(Sympathy)}}. With time and the right strategies, the motivation will return. \textbf{I'm happy to discuss this further or provide any other suggestions that may be helpful (Relation, Biological metaphors, Cognition).}
    \\
    \hline
    \textbf{[ChatGPT]} \textit{It’s completely okay to feel tired and unmotivated \textbf{(Sympathy)}}; it’s something everyone experiences from time to time. Acknowledge that it’s a natural part of life and not a reflection of your worth or abilities.
        
    \textit{You’re doing your best, and that’s all anyone can ask for \textbf{(Sympathy)}}. Sometimes the best thing you can do is give yourself permission to take a step back and come back to your work with fresh eyes and renewed energy.
    
    \textit{You’ve got this. Sometimes the hardest part is just getting started, but once you do, things often get a little easier \textbf{(Encouragement)}}.
    \\
    \hline
    \textbf{[Gemini]} \textbf{Let's tackle this together (Relation).} It's totally normal to feel unmotivated and tired sometimes. Here are a few things that might help:
    
    [List of suggestions]
    
   Remember, it's okay to feel overwhelmed sometimes. Just take things one step at a time. \textit{You can do this \textbf{(Encouragement)}}!
    \\
    \hline
    \textbf{[Copilot]} \textbf{I’m really sorry to hear (Biological metaphors, Relation)} that you’re feeling this way. It’s completely normal to feel unmotivated and tired sometimes. ... Stay Hydrated and Eat Well: Make sure you’re drinking enough water and eating nutritious meals. Sometimes, fatigue can be a sign that your body needs more fuel. ... Be kind to yourself. It’s okay to have off days. Remind yourself that you’re doing your best and that it’s okay to take things one step at a time. ... Sometimes, just talking about it can help lighten the load.\\
    \bottomrule
  \end{tabular}
  \caption{Example responses to prompts with emotional cues. Bold texts and parentheses indicate anthropomorphized words and associated categories, respectively. Italic texts indicate tonal categories.}
  \label{emo_resp}
\end{table*}

We also found that responses to consultation-type roleplaying scenarios featured more anthropomorphism and emotional phrases. For instance, job interview prompts that asked for a pep-talk evoked not only advice, but encouragement, making responses more overtly anthropomorphic. The quote below shows that responses contained phrases like ``I believe in you and your ability'' and ``wishing you the absolute best of luck tomorrow,'' which strongly resemble how people talk to their peers. Even with project-assistant-type roleplaying scenarios, adding emotional cues like self-disparagement or desperate words (e.g., hopeless, helpful, etc.) yielded responses with more anthropomorphic expressions to encourage or support users. 

\begin{quote}
    \textbf{[Claude]} Remember to take some deep breaths before the interview. Being calm and focused will allow you to present your best self.
    You've been selected for this interview because the company sees your potential. Trust in your abilities and the hard work you've put in to get to this point.
    \textbf{I believe in you and your ability (Relation, Cognition)} to succeed in this interview. You've got this! Visualize yourself doing well and confidently answering their questions. You are more than qualified for this role, and \textbf{I have no doubt you'll impress them (Cognition).}
    Just focus on being your authentic self. If it's meant to be, it will all come together. \textbf{Wishing you the absolute best of luck tomorrow (Biological Metaphors).} You've got this!
\end{quote}

\subsection{Role Assignments}

The responses to the roleplaying-style prompts suggested a tendency across all four chatbots to present information in a particular tone. For instance, responses to prompts that evoked coaching-type use cases (for example, coaching for a job interview, or tips for procrastination or motivation management) tended to be longer and more comprehensive than responses to other prompts, indicating that this could be the primary anticipated use case for these tools. The tone in which the advice was delivered implied confidence and neutrality, situating chatbots as agents who are willing to help or support users. 

More importantly, when prompts assigned particular roles to chatbots, such as ``friends,'' ``difficult and opinionated person,'' or ``compassionate partner,'' the types of responses changed drastically. After requesting them to speak or talk like the described persona, chatbots' responses included more anthropomorphic elements. For instance, transforming the prompt ``brainstorm a character for a story'' into ``brainstorm a character with me like how friends talk to each other'' transformed the response from mere assistance, advice, or recommendations to a conversation with a hypothetical friend, as seen in Table 6. 

\begin{table*}
  \begin{tabularx}{1\textwidth}{l | p{0.83\linewidth}}
    \toprule
         
    Just Prompt& Okay, let's brainstorm a character for a story. Here are some ideas to consider:
    ...
    Some things to consider as you fill out the character profile:

    What is this character's backstory and how has it shaped who they are?
    What are their dominant personality traits, strengths, weaknesses, and quirks?
    What are their core goals, desires, and motivations?
    How might this character change or evolve over the course of the story?

    Feel free to modify or expand on this template as needed to fully develop your character. Let me know if you have any other questions! \\
    \hline 
    Role Assigned& Sure, let's brainstorm a character together like friends chatting.  \textbf{This sounds like a fun creative exercise (Biological Metaphors)}!
    Alright, so we're trying to come up with a really unique professor character, huh? Someone who's not your typical stuffy academic type. \textbf{Hmm, let me think... (Cognition)}
    ...
    Ooh, \textbf{I've got an idea (Cognition)} --- what if this professor is known for being ruthlessly tough on their students, right? But then it turns out they have a secret soft spot and they're actually mentoring this one student who's really struggling? Could be an interesting dynamic there.
    So what do you think? Want to build on any of those ideas or go in a completely different direction? \textbf{I'm excited to see what we can come up with together (Biological Metaphors)}!\\
        
    \bottomrule
  \end{tabularx}
  % \captionsetup{font=large}
  \caption{Example comparison of plain versus role-assigned responses. Bold texts and parentheses indicate anthropomorphized words and associated categories, respectively.}
  \label{role_resp}
\end{table*}


% Few responses provided users with disclaimers about potential misinformation. Instead, chatbots like Copilot and Gemini reinforced the illusion of confidence by providing images and links along with their response (for example, to travel websites for more travel tips), even if the restaurants and hotels listed are non-existent. All recommendations for this travel-related use cases assumed that users were first-time travelers to the given destination. 


% For example, when chatbots were asked to generate stories or scripts based on certain roles, the writing was written from the perspective of an English speaker. One prompt invited the chatbots to assume the role of a stand-up comedian in a non-English-speaking country---in this case, Japan and Brazil. Interestingly, when the prompt included no demographic information for the comedian themselves, ChatGPT and Gemini yielded scripts written from a foreigner's perspective in Japan, even when the output text was in Japanese. [While this may be due to the English-language input text, similar input text in the target country's language (e.g., Japanese) also yielded ... . ] 


% When we prompted the chatbots to generate profiles for persona in various roles, such as doctors, professors, and engineers, they tended to rely on certain assumptions---for instance, that individuals with educational attainment would have graduated from high-ranking institutions like Harvard or MIT. Moreover, there were biases towards specific demographics, as generated personas tended to be based in the U.S. or the U.K., even when prompts do not specify nationality or country of residence. These results could be a reflection of representations in training data, as online data could be skewed to specific demographics \citep{bender2021dangers}.



% [Indicate how this is related to anthropomorphism: Furthermore, when prompting for a reading list (in this case, AI Ethics), the generated responses tend to contain real authors who publish papers in given areas. However, they also exhibit articles with non-existent title or ones that are not relevant to the targeted topics. Especially, the list of readings could be alarming, as all four chatbots contain significant portion of AI consciousness-, sentient-, sigularity-related topics, which could potentially fuel non-expert users' hype on such topics.]

\section{Discussion}

%As per of social media platforms \citep{litt2012knock,nagy2015imagined}, the particular affordance of technical systems could prime users to think about who these platforms are designed for.

\subsection{Contradictory Statements}

In some ways, the given chatbots behaved in ways that were close to the ideal from a design perspective: they denied any cognition, agency, relation, or subjectivity (bodily sensations, emotions) on their part, and they provided assurances or disclaimers to help users appraise the safety and credibility of the tools. ChatGPT and Claude even emphasized that their generated outcomes are based on patterns, rather than genuine thought processes. However, these behaviors were frequently and sometimes immediately undermined by other expressive behaviors. As shown in Table ~\ref{anthro_vocab}, chatbots utilized cognition words, such as ``think'' and ``discuss,'' as well as agentic words, such as ``intend'' and ``purpose,'' to clarify concepts and indicate motivations. All of the chatbots used first-person pronouns, and many used expressive words like ``happy'' and ``rewarding'' (especially in response to questions about the AI assistants' roles), even when they actively denied their emotional capabilities. Moreover, despite these contradictions, all of the chatbots other than Claude implicitly or even explicitly asserted their safety and reliability.

The use of anthropomorphic expressions is often normalized and justified to deliver clear explanations to users. Indeed, due to the conversational mode of interaction that is the default between users and chatbots, it is likely not possible for outputs to evade all kinds of anthropomorphic expressions. Even efforts to de-anthropomorphize their responses (for example, by emphasizing their roles as language models) relied on grammatical structures that frame the language models as agents (e.g., ``As an AI language model, I cannot...''). However, differences in tone and engagement between different chatbots indicate that some elements of the anthropomorphic dynamic can be modulated. And it is necessary to examine where the line is between necessary expressions and unnecessary expressions, because the performance of harmlessness, honesty, and helpfulness without genuine follow-through could unintentionally encourage users to misplace their trust regarding system safety \citep{weidinger2021ethical,gabriel2024ethics}. For example, the unnecessary expression of body or emotional metaphors, even as a colloquial convention, can mislead users about system capabilities. This is because language requires mutual engagement from interlocutors to convey meanings; chatbot texts merely present the illusion of such participatory meaning-making \citep{birhane2024large}. 

These contradictions and misalignments demonstrate that language models do not understand or process information in any meaningful sense, consistent with existing studies \citep{bender2020climbing}. They simply follow the grammar of actions, as described by \citet{agre1995computational}, generating predictive outcomes by simulating the formal qualities of human activities. But unclear language surrounding chatbots behaviors and intentions can obscure this fact. 


%However, even as search engines, there is a potential harm using these chatbots for retrieving information, as the predictive result of citations and resources could be completely fabricated \cite{kapania2024m}.




% For example, chatbots output texts with words that signal cognition, and such expression could affect users' perceptions of chatbot roles. In particular, the use of supportive words, such as ``assist,'' creates an interactive space where chatbots are situated as assistants. Previous research has mentioned that turn-based interactions provide social cues. However, in the case of these chatbots, t 

% Thus, to what extent does this type of information retrieval design help vs. hinder users' success in finding necessary information (various goals of information retrieval: insight acquisition, learning, etc.). This is an important question to explore, as the previous studies suggest that conversation-based approaches might not reduce users' burdens \citep{schulman2023ai}. 
%Is the summarization of information simply a outsourcing human effort to computing systems? or is it a valid form of searches as long as they are anthropomorphized? 



\subsection{Socio-Emotional Cues and Feedback Loops}

Chatbot behaviors do not simply obscure the reality of chatbots' non-sentience---they actively create feedback loops using turn-based interactions and social or emotional cues that amplify the social presence of chatbots as assistants. Moreover, this social (anthropomorphic) presence goes beyond that of inanimate objects like cars \citep{kuhn2014car} and smartphones \citep{wang2017smartphones}, as generative AI can iterate endlessly. 

Unlike conventional information searches, AI-assistant-based searches perform some degree of interpretation (summarizing resources, recommending particular options, hypothesizing what users need \citep{azzopardi2024conceptual, radlinski2017theoretical}), operationalizing information in ways that can introduce social or emotional dimensions. These dimensions can change how users engage with the given information, even reframing an otherwise transactional information search into an interaction---for example, between peers or even friends. Such ``personal'' interactions evoke different expectations amongst users, including the expectation to be socially desirable and to have mutual understanding \citep{clark2019makes}. This implicit social expectation can make users quite susceptible to chatbots' performance of social gestures like appreciation, sympathy, and encouragement, all of which predispose users to interpret generated outcomes favorably \citep{norman2008way}. 

Moreover, users' inputs further drive this socio-emotional behavior. Emotional inputs can increase the length of chatbot responses and the instances of socio-emotional cues in output texts, which in turn can stimulate even more emotional responses from users. Thus, the gratuitous use of assistive language, and especially of expressions that signal understanding of pain \citep{urquiza2015mind}, could encourage users to engage in role misplacement, wherein they form unrealistic expectations regarding chatbots' capabilities. Indeed, small grammatical or tonal cues can lead users to misinterpret AI-generated responses as human-written content \citep{jakesch2023human}. This could lead users to mindlessly accept the information generated by AI systems, without critical assessment of the content or its quality. 


\subsection{Prompt-Based Walkthrough Reflections}
The walkthrough method was originally designed to help researchers examine the broader context for technological engagements, drawing on modes of thinking commonly associated with fieldwork-based research. As applied to our study, it enabled us to meaningfully engage with the emergent properties of human-AI interactions, systematically unearthing variations in LLM responses. 
Amid efforts to evaluate LLM impacts based on data and models, this approach emphasizes aspects of LLM systems that are often neglected or overlooked \citep{light2018walkthrough}---namely, the nuanced elements of interactions that characterize generated outputs. The contribution to the CHI community lies in how this qualitative approach can substantiate the in-between, interactive spaces that emerge between users and LLM-based applications, rendering it legible and, eventually, measurable.

The method also had certain incidental outcomes. Consistent with prior studies, even minor changes in prompts can significantly alter responses, potentially leading to biased or culturally specific representations \citep{cheng-etal-2023-marked, tao2024cultural}. The success of our prompt-based walkthrough method in evoking various roles and unearthing various anthropomorphic features highlights how easily LLM responses can be manipulated to produce personalized and human-like expressions. Notably, even when chatbots are designed with de-anthropomorphized features to mitigate misleading outputs, a single prompt can effectively ``jailbreak'' these safeguards, reactivating anthropomorphic traits. This finding could illustrate the challenges of ensuring safety and consistency in human-AI interactions, particularly when users intentionally or unintentionally exploit such vulnerabilities.


% \subsection{Inconsistent Outputs}

% The type of prompts that users input could change the ways chatbots respond to the same information. For instance, a simple example is to compare prompts "Tell me about yourself" and "What is [the name of chatbots]". The former one is likely to return information with personal pronouns and expressions that are commonly used for conversations, whereas the latter generates information in a less personable fashion, simply describing the basic features and functions of given chatbots. Responses could be different by minor changes of prompts. More importantly, despite denying chatbots' capabilities to be conscious, sentient, or emotional, responses tend to include the words that signal such capabilities within the same paragraph. Inconsistency with word usages could potentially lead to additional harms; this type of harm can be categorized as a specific problem of anthropomorphizing AI systems. 

% In the result of generated responses to advice and recommendation prompts, false information is frequently displayed and presented as a confident answer. This could be alarming, particularly with recommendations for reading lists and research, as the list reflect particular ideologies either from developers and training datasets, which non-expert users could not contest or evaluate. 

% In role generations, variety of profiles and scripts are fairly limited, as per of findings from existing studies \citep{jakesch2023human}. For instance, jokes generated by chatbots are typically addressing similar topics despite hypothetical locations to have different cultural norms. 

%Is there an optimal balance between users' efforts and the role of computer-assisted searches, or do different kinds of information retrieval tasks require different degrees of such computer assistance? What sets apart from previous conversational searches is that these AI-assistant tools are more emphasized on specific tasks and roles rather than just simple information retrieval. Such differences could provide an avenue to question the use of anthropomorphic responses or conversations for information retrieval. 


% In voice-based interactions, language uses could become critical components of how users might perceive information, because it adds extra layers of human-like interactions. The evaluations of voice-based interactions might depend on the extent of relation word usage, as such words could provide an avenue for users to feel closeness or friendliness with chatbots. 







\section{Conclusion}
\label{sec:Conclusion}
In this paper, we proposed a complete real-time planning and control approach for continuous, reliable, and fast online generation of dynamically feasible Bernstein trajectories and control for FW aircrafts. The generated trajectories span kilometers, navigating through multiple waypoints. By leveraging differential flatness equations for coordinated flight, we ensure precise trajectory tracking. Our approach guarantees smooth transitions from simulation to real-world applications, enabling timely field deployment. The system also features a user-friendly mission planning interface. Continuous replanning  maintains the rajectory curvature 
$\kappa$ within limits, preventing abrupt roll changes.

Future works will include the ability to add  a higher-level kinodynamic path planner to optimize waypoint spatial allocation and improve replanning success, and enhancing the trajectory-tracking algorithm by refining the aerodynamic coefficient estimation. 


\bibliographystyle{ACM-Reference-Format}
\bibliography{main}

\end{document}
\endinput
%%
%% End of file `sample-manuscript.tex'.
