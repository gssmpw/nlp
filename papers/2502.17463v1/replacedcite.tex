\section{Related Work}
\label{sec3}
% Attributable to factors that are not yet thoroughly elucidated, the rate and extensive occurrence of this Parkinson's disease have surged substantially in the last twenty years____.
% Mahan: PD and gait

% Parkinson's Disease (PD) is a progressive neurodegenerative condition that primarily affects motor functions but also involves several non-motor symptoms. This disease arises from the loss of dopamine-producing brain cells, placing it among a group of motor system disorders. PD is notably the second most common neurodegenerative disease____, with its prevalence estimated at approximately 0.3\% in the general population. This prevalence increases to around 1\% in individuals over 60 years old and approximately 3\% in those aged 80 and above____. Early detection of PD is crucial for effective management and slowing disease progression.


% PD is characterized by a range of early indicators that precede the more obvious motor signs like tremors and rigidity. Vocal impairments, for instance, are among the earliest signs, with approximately 90\% of patients experiencing changes in voice and speech patterns in the initial stages____. This has led to the development of diagnostic tools focusing on speech analysis, which provide non-invasive and cost-effective early detection.

% Gait analysis has emerged as a particularly informative diagnostic tool for Parkinson's Disease (PD). The disease affects gait mechanics, leading to distinctive alterations in walking patterns, such as reduced stride length, decreased speed, and increased gait variability. Advanced technologies involving sensor-based systems and motion capture are now extensively used to quantify these deviations, enabling not only early detection but also monitoring of disease progression and response to treatment.

% In clinical settings, gait analysis is employed to assess and treat individuals with various conditions that impair their walking ability, including poor health, advanced age, body size, weight, and speed of movement. To effectively evaluate gait characteristics and other related phenomena in PD patients, a standardized assessment is necessary. Such assessments typically involve measurements of gait count, walking speed, and step length.

% Pistacchi et al. conducted a comprehensive study on early PD patients using 3D gait analysis, examining several temporal parameters. The study found that the cadence in PD patients was 102.46 ± 13.17 steps/min, compared to 113.84 ± 4.30 steps/min in healthy individuals. Stride duration for PD patients was measured at 1.19 ± 0.18 seconds for the right limb and 1.19 ± 0.19 seconds for the left limb, while for healthy subjects, it was significantly shorter, at 0.426 ± 0.16 seconds for the right limb and 0.429 ± 0.23 seconds for the left limb. The stance duration in PD patients was 0.74 ± 0.14 seconds for the right limb and 0.74 ± 0.16 seconds for the left limb, in contrast to 1.34 ± 1.1 seconds and 0.83 ± 0.6 seconds in healthy subjects, respectively. Additionally, the velocity of PD patients was markedly lower at 0.082 ± 0.29 m/s compared to 1.33 ± 0.06 m/s in healthy individuals ____.

% Further research by Sofuwa et al. indicated that PD significantly reduces step length and walking speed when compared to non-PD controls____. Lescano et al. focused on analyzing various gait parameters, including stance and swing phase duration, and the vertical component of ground reaction force, to determine statistically significant differences between PD patients at stages 2 and 2.5 on the modified Hoehn and Yahr scale____.


% % Sadra: ML and ai in PD
% The use of ground reaction force sensors embedded in shoes to distinguish between healthy individuals and patients with Parkinson's disease has been extensively studied____. Vidya et al.____ developed a decision support system for gait classification employing a multi-class support vector machine (MCSVM) and utilized spatiotemporal features for classification. In a different approach, Daliri____ extracted statistical features, including the minimum, maximum, mean, and standard deviation for each time series, which were then fed into binary machine learning classifiers like Support Vector Machines (SVM). Furthermore, an interpretable method for diagnosing Parkinson’s disease using clinical features from vertical Ground Reaction Force (vGRF) data is presented in ____. The approach employs type-2 fuzzy logic and a novel learning method, achieving high classification performance. El Maachi et al.____ proposed a deep learning-based approach for Parkinson's disease detection and severity prediction using a 1D convolutional neural network (1D-Convnet). The proposed model processes gait data and demonstrates high accuracy and efficiency in classification tasks. Zhao et al.____ used a dual-channel LSTM to diagnose some diseases, including PD, based on gait dynamics, achieving a classification accuracy of 97.43\%. However, their approach was limited by the partial acquisition of gait dynamics due to the use of only one sensor per foot. Balaji et al.____ proposed an LSTM network for diagnosing and rating the severity of Parkinson's disease using gait patterns. The method avoids hand-crafted features and effectively classifies PD with high accuracy, demonstrating a significant improvement over related techniques. Vidya and Sasikumar____ developed a CNN-LSTM model for Parkinson's disease diagnosis using gait analysis. By leveraging VGRF signals and employing empirical mode decomposition. This hybrid approach effectively combines CNN's spatial feature extraction with LSTM's temporal sequence learning, outperforming previous methods in PD severity classification.
% While numerous machine learning approaches have been developed for Parkinson's disease (PD) diagnosis and severity assessment, most rely heavily on hand-crafted features or complex black-box models that lack interpretability. Our proposed method addresses these limitations by introducing an explainable deep learning approach utilizing Sinc filters in a convolutional neural network (CNN). This approach extracts significant frequency bands from raw vertical Ground Reaction Force (vGRF) signals, offering both high accuracy in classification and transparency in the decision-making process. This work aims to enhance the interpretability and reliability of PD diagnosis and severity measurement tools, ensuring that the insights provided by the model are understandable and actionable for medical professionals.
%new edit

Parkinson's Disease (PD) is a progressive neurodegenerative disorder marked by motor symptoms (e.g., tremors, rigidity) and non-motor manifestations. Due to the movement disorders, gait analysis have been one of the most popular approaches to study the Parkinson's disease in the literature. 

Gait analysis has proven valuable for PD diagnosis due to its ability to identify changes such as reduced stride length, slower speed, and increased variability. ____ observed significant differences in cadence, stride and stance duration, and gait velocity in early-stage PD patients compared to controls. ____ reported shorter step length and slower walking speed, while ____ found deviations in stance and swing phases and ground reaction force at Hoehn and Yahr stages 2--2.5.


Numerous sensor-based systems have been developed for automated PD detection via gait analysis, extensively utilizing vGRF captured by in-shoe sensors____.____ employed spatiotemporal features with a multi-class support vector machine. ____ presented a type-2 fuzzy logic approach using vGRF data. and ____ proposed a 1D convolutional neural network for PD detection and severity estimation. ____ utilized a dual-channel LSTM for gait classification, though limited by partial gait acquisition. ____ applied LSTM networks for PD diagnosis and severity rating without hand-crafted features, and ____ introduced a CNN-LSTM model based on empirical mode decomposition of vGRF signals.

Although these machine learning methods have advanced PD diagnosis, many depend on hand-crafted features or lack interpretability.
 To overcome these limitations, we propose an explainable deep learning model using Sinc filters in a convolutional neural network (CNN) to extract salient frequency bands from raw vGRF data. This approach combines high classification accuracy with transparency, improving the clinical relevance of PD diagnosis and severity assessment.




% The deterioration of executive functions and movement disorders in patients with PD have been shown extensively ____. Yogev et al. ____ studied the impacts of different types of dual-tasking and cognitive function on the gait of patients with PD and control subjects. They also showed contrasting measures of gait rhythmicity for patients with PD in comparison to other features. Additionally, in ____ it is investigated that Parkinson's disease has a great impact on the left-right symmetry of gait. Yogev et al. ____ conducted a similar walking condition for both patients with PD and healthy control subjects and they demonstrated that asymmetry of gait increased mainly during the dual-task condition in patients with PD but not in the healthy control subjects. Considering the Hausdorff et al. ____ studies on gait variability and Basal Ganglia disorders, it can be concluded that the ability to maintain a steady gait with low stride-to-stride variability of gait cycle timing, would be decreased in patients with PD. Parkinson's disease symptoms also include speech disorders as well as cognitive impairments ____. In addition, $90\%$ of the patients with PD themselves report speech impairments as one of the most significant symptoms ____.

% To classify the patients with PD and healthy control subjects based on their gait cycles, both wearable ____ and non-wearable ____ sensors have been used in various experiments \citep {GOYAL2020103955}. For instance, Jean et al. ____, conducted the classification using Spatial-Temporal Image of Plantar pressure (STIP) among normal step and patient steps with PD. In ____ the wearable sensors on-shoe along with some algorithms are presented to characterize the Parkinson's disease motor symptoms.

% Accordingly, the classification of healthy control subjects and patients with PD using ground reaction force sensors placed in shoes has been extensively studied ____. In ____, the vGRFs measurements of both left and right foot were used to extract statistical features including minimum, maximum, average, and the standard deviation of each time series. Their extracted features then were fed to machine learning binary classifier including Support Vector Machine (SVM). In ____ the gait asymmetry (GA) was calculated based on the difference of the ground reaction force (GRF) features of the left and right feet. This was done by decomposition of the GRF into components of different frequency sub-bands via the wavelet transform and Multi-Layer Perceptron (MLP) models.  In ____, Lee et al. utilized the gait characteristics of idiopathic PD patients who shuffle their feet while they are walking to classify patients with PD and healthy control subjects. They trained a neural network with weighted fuzzy membership functions (NEWFM) using extracted 40 statistical and wavelet-based features.

% Different supervised and unsupervised methods such as Decision Tree (DT), Support Vector Machine (SVM), K-Nearest Neighbors (KNN), Gaussian-Mixture Model (GMM), and K-Means are extensively utilized to classify patients with PD and healthy control subjects. In ____, automatic noninvasive identification of PD is used with the combination of wavelet analysis and SVM which led to an accuracy of $90.32\%$. Although most prior research focused on time-domain and frequency-domain features, only the clinical features extracted from vertical Ground Reaction Forces (vGRFs) were considered in ____. Accordingly, nineteen statistical features are extracted and used as the input of machine learning-based classifiers.

% A  time-delay neural network classifier learned by a Q-back propagation learning approach was proposed in ____ using temporal information of vGRF time-series to classify the PD patients and healthy subjects. Data from three Parkinson's disease research projects are used for evaluation of this approach ____. The accuracy on the three sub-datasets reached $91.49\%$, $92.19\%$, and $90.91\%$, respectively.

% Although most of the previous works tried to extract feature vectors based on some human knowledge, recently published method tried to extract high-level features based temporal and spatial analysis of the gait-cycle using deep neural networks ____. In ____, we have demonstrated that the spatial correlation among different sensors data during time placed in each left and right foot are useful for diagnosing the patients. To consider the temporal dependencies, a structure using Long-Short Term Memory (LSTM) cell layers has been proposed to build a Recurrent Neural Network (RNN). In ____, a deep neural network consists of one dimensional convolutional layers is used to classify patients. The final accuracy of the method reaches $98.7\%$. In ____ and ____, a combination of Long-short Term Memory (LSTM) and Convolutional Neural Network (CNN) are used to detect patients suffering from PD with accuracy equal to $98.61\%$ and $99.22\%$, respectively. Although these deep methods perform efficiently, they have many parameters that must be determined based on a supervised learning method. Therefore, to learn these deep structures, considering the small size of available datasets, authors of these methods have segmented the sequence related each subject into very small pieces (10 to 20 steps) to increase the number of training samples. Moreover, these deep structures lack the interpretability and their reasoning for classification is not clear and verifiable for an expert.

% Most previous studies have not used clinical features or have not applied an interpretable method for classifying PD patients. Using an interpretable machine learning approach that the base of its decision making is transparent and expert understandable is more agreeable in a clinical application ____. In ____ a self-organizing interval type-2 fuzzy neural network is presented based on analysing the clinical features extracted from the recorded vGRF signal. This method also reports its extracted fuzzy rules that can be verified and modified based on the knowledge of experts. On the other hand, experts can utilize the extracted rules to detect patients suffering from PD.

% Although the method presented in ____ is a an interpretable machine learning method, it is limited to utilize the clinical features extracted from the recorded signals. In this paper, an interpretable deep learning method is proposed based on using adaptive bandpass filters in its first layers to detect patients suffering from PD based on studying their vGRF signals directly. Furthermore, the proposed method is utilized to determine the severity level of PD. 


% -
% -
% -
% all together