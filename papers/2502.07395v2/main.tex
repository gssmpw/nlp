\documentclass[conference]{IEEEtran}
\IEEEoverridecommandlockouts

\usepackage{cite}
\usepackage{amsmath,amssymb,amsfonts}
\usepackage{algorithmic}
\usepackage{graphicx}
\usepackage{textcomp}
\usepackage{xcolor}
\usepackage{graphicx}
\usepackage{booktabs}
\usepackage{url}
\usepackage{multirow}
\usepackage{threeparttable}

\usepackage[most]{tcolorbox}

% \usepackage[belowskip=-15pt,aboveskip=0pt]{caption}
% \setlength{\intextsep}{30pt plus 10pt minus 5pt}

\newcommand{\morakot}[1]{
    \textcolor{blue}{\textbf{TODO:} #1}%
}

\newcommand{\chaiyong}[1]{
    \textcolor{red}{\textbf{~Chaiyong:} #1}%
}

\newcommand{\bee}[1]{
    \textcolor{blue}{\textbf{~Bee:} #1}%
}

\newcommand{\censortext}[1]{\textcolor{black}{\rule{3em}{1ex}}}


\def\BibTeX{{\rm B\kern-.05em{\sc i\kern-.025em b}\kern-.08em
    3T\kern-.1667em\lower.7ex\hbox{E}\kern-.125emX}}
    
\begin{document}
\title{On Categorizing Open Source Software Security Vulnerability Reporting Mechanisms on GitHub}

\author{%
Sushawapak Kancharoendee\textsuperscript{\dag}, Thanat Phichitphanphong\textsuperscript{\dag}, Chanikarn Jongyingyos\textsuperscript{\dag}, Brittany Reid\textsuperscript{*},\\
Raula Gaikovina Kula\textsuperscript{\S}, Morakot Choetkiertikul\textsuperscript{\dag}, Chaiyong Ragkhitwetsagul\textsuperscript{\dag}, Thanwadee Sunetnanta\textsuperscript{\dag}\\
\textsuperscript{\dag}\textit{Faculty of Information and Communication Technology, Mahidol University, Thailand}\\
\textsuperscript{*}\textit{Graduate School of Science and Technology, Nara Institute of Science and Technology (NAIST), Japan}\\
\textsuperscript{\S}\textit{Graduate School of Information Science and Technology, Osaka University, Japan}\\
\{sushawapak.kan, thanat.phi, chanikarn.jon\}@student.mahidol.ac.th, brittany.reid@naist.ac.jp, \\raula-k@ist.osaka-u.ac.jp, \{morakot.cho, chaiyong.rag, thanwadee.sun\}@mahidol.ac.th
}

\maketitle

\begin{abstract}

Open-source projects are essential to software development, but publicly disclosing vulnerabilities without fixes increases the risk of exploitation. The Open Source Security Foundation (OpenSSF) addresses this issue by promoting robust security policies to enhance project security. Current research reveals that many projects perform poorly on OpenSSF criteria, indicating a need for stronger security practices and underscoring the value of SECURITY.md files for structured vulnerability reporting. This study aims to provide recommendations for improving security policies. By examining 679 open-source projects, we find that email is still the main source of reporting. Furthermore, we find that projects without SECURITY.md files tend to be less secure (lower OpenSSF scores). Our analysis also indicates that, although many maintainers encourage private reporting methods, some contributors continue to disclose vulnerabilities publicly, bypassing established protocols. The results from this preliminary study pave the way for understanding how developers react and communicate a potential security threat. Future challenges include understanding the impact and effectiveness of these mechanisms and what factors may influence how the security threat is addressed.

% Open-source projects are essential to software development, yet publicly disclosing vulnerabilities without fixes increases the risk of exploitation. The Open Source Security Foundation (OpenSSF) addresses this issue by offering guidelines that encourage robust security policies to enhance project security. Current research shows that most projects perform poorly on OpenSSF criteria, suggesting a limited commitment to security practices and underscoring the importance of SECURITY.md files for structured vulnerability reporting. The objective of this study is to propose guidelines to improve security policies. By examining 679 open-source projects, we find that those without SECURITY.md files tend to have slightly lower OpenSSF scores. Our analysis indicates that, although many maintainers recommend private reporting methods, some vulnerabilities continue to be disclosed publicly by contributors, bypassing established protocols.
\end{abstract}

\begin{IEEEkeywords}
open-source, security policy, OpenSSF
\end{IEEEkeywords}


\section{Introduction}

Video generation has garnered significant attention owing to its transformative potential across a wide range of applications, such media content creation~\citep{polyak2024movie}, advertising~\citep{zhang2024virbo,bacher2021advert}, video games~\citep{yang2024playable,valevski2024diffusion, oasis2024}, and world model simulators~\citep{ha2018world, videoworldsimulators2024, agarwal2025cosmos}. Benefiting from advanced generative algorithms~\citep{goodfellow2014generative, ho2020denoising, liu2023flow, lipman2023flow}, scalable model architectures~\citep{vaswani2017attention, peebles2023scalable}, vast amounts of internet-sourced data~\citep{chen2024panda, nan2024openvid, ju2024miradata}, and ongoing expansion of computing capabilities~\citep{nvidia2022h100, nvidia2023dgxgh200, nvidia2024h200nvl}, remarkable advancements have been achieved in the field of video generation~\citep{ho2022video, ho2022imagen, singer2023makeavideo, blattmann2023align, videoworldsimulators2024, kuaishou2024klingai, yang2024cogvideox, jin2024pyramidal, polyak2024movie, kong2024hunyuanvideo, ji2024prompt}.


In this work, we present \textbf{\ours}, a family of rectified flow~\citep{lipman2023flow, liu2023flow} transformer models designed for joint image and video generation, establishing a pathway toward industry-grade performance. This report centers on four key components: data curation, model architecture design, flow formulation, and training infrastructure optimization—each rigorously refined to meet the demands of high-quality, large-scale video generation.


\begin{figure}[ht]
    \centering
    \begin{subfigure}[b]{0.82\linewidth}
        \centering
        \includegraphics[width=\linewidth]{figures/t2i_1024.pdf}
        \caption{Text-to-Image Samples}\label{fig:main-demo-t2i}
    \end{subfigure}
    \vfill
    \begin{subfigure}[b]{0.82\linewidth}
        \centering
        \includegraphics[width=\linewidth]{figures/t2v_samples.pdf}
        \caption{Text-to-Video Samples}\label{fig:main-demo-t2v}
    \end{subfigure}
\caption{\textbf{Generated samples from \ours.} Key components are highlighted in \textcolor{red}{\textbf{RED}}.}\label{fig:main-demo}
\end{figure}


First, we present a comprehensive data processing pipeline designed to construct large-scale, high-quality image and video-text datasets. The pipeline integrates multiple advanced techniques, including video and image filtering based on aesthetic scores, OCR-driven content analysis, and subjective evaluations, to ensure exceptional visual and contextual quality. Furthermore, we employ multimodal large language models~(MLLMs)~\citep{yuan2025tarsier2} to generate dense and contextually aligned captions, which are subsequently refined using an additional large language model~(LLM)~\citep{yang2024qwen2} to enhance their accuracy, fluency, and descriptive richness. As a result, we have curated a robust training dataset comprising approximately 36M video-text pairs and 160M image-text pairs, which are proven sufficient for training industry-level generative models.

Secondly, we take a pioneering step by applying rectified flow formulation~\citep{lipman2023flow} for joint image and video generation, implemented through the \ours model family, which comprises Transformer architectures with 2B and 8B parameters. At its core, the \ours framework employs a 3D joint image-video variational autoencoder (VAE) to compress image and video inputs into a shared latent space, facilitating unified representation. This shared latent space is coupled with a full-attention~\citep{vaswani2017attention} mechanism, enabling seamless joint training of image and video. This architecture delivers high-quality, coherent outputs across both images and videos, establishing a unified framework for visual generation tasks.


Furthermore, to support the training of \ours at scale, we have developed a robust infrastructure tailored for large-scale model training. Our approach incorporates advanced parallelism strategies~\citep{jacobs2023deepspeed, pytorch_fsdp} to manage memory efficiently during long-context training. Additionally, we employ ByteCheckpoint~\citep{wan2024bytecheckpoint} for high-performance checkpointing and integrate fault-tolerant mechanisms from MegaScale~\citep{jiang2024megascale} to ensure stability and scalability across large GPU clusters. These optimizations enable \ours to handle the computational and data challenges of generative modeling with exceptional efficiency and reliability.


We evaluate \ours on both text-to-image and text-to-video benchmarks to highlight its competitive advantages. For text-to-image generation, \ours-T2I demonstrates strong performance across multiple benchmarks, including T2I-CompBench~\citep{huang2023t2i-compbench}, GenEval~\citep{ghosh2024geneval}, and DPG-Bench~\citep{hu2024ella_dbgbench}, excelling in both visual quality and text-image alignment. In text-to-video benchmarks, \ours-T2V achieves state-of-the-art performance on the UCF-101~\citep{ucf101} zero-shot generation task. Additionally, \ours-T2V attains an impressive score of \textbf{84.85} on VBench~\citep{huang2024vbench}, securing the top position on the leaderboard (as of 2025-01-25) and surpassing several leading commercial text-to-video models. Qualitative results, illustrated in \Cref{fig:main-demo}, further demonstrate the superior quality of the generated media samples. These findings underscore \ours's effectiveness in multi-modal generation and its potential as a high-performing solution for both research and commercial applications.
\section{Background} \label{section:LLM}

% \subsection{Large Language Model (LLM)}   

Figure~\ref{fig:LLaMA_model}(a) shows that a decoder-only LLM initially processes a user prompt in the “prefill” stage and subsequently generates tokens sequentially during the “decoding” stage.
Both stages contain an input embedding layer, multiple decoder transformer blocks, an output embedding layer, and a sampling layer.
Figure~\ref{fig:LLaMA_model}(b) demonstrates that the decoder transformer blocks consist of a self attention and a feed-forward network (FFN) layer, each paired with residual connection and normalization layers. 

% Differentiate between encoder/decoder, explain why operation intensity is low, explain the different parts of a transformer block. Discuss Table II here. 

% Explain the architecture with Llama2-70B.

% \begin{table}[thb]
% \renewcommand\arraystretch{1.05}
% \centering
% % \vspace{-5mm}
%     \caption{ML Model Parameter Size and Operational Intensity}
%     \vspace{-2mm}
%     \small
%     \label{tab:ML Model Parameter Size and Operational Intensity}    
%     \scalebox{0.95}{
%         \begin{tabular}{|c|c|c|c|c|}
%             \hline
%             & Llama2 & BLOOM & BERT & ResNet \\
%             Model & (70B) & (176B) & & 152 \\
%             \hline
%             Parameter Size (GB) & 140 & 352 & 0.17 & 0.16 \\
%             \hline
%             Op Intensity (Ops/Byte) & 1 & 1 & 282 & 346 \\
%             \hline
%           \end{tabular}
%     }
% \vspace{-3mm}
% \end{table}

% {\fontsize{8pt}{11pt}\selectfont 8pt font size test Memory Requirement}

\begin{figure}[t]
    \centering
    \includegraphics[width=8cm]{Figure/LLaMA_model_new_new.pdf}
    \caption{(a) Prefill stage encodes prompt tokens in parallel. Decoding stage generates output tokens sequentially.
    (b) LLM contains N$\times$ decoder transformer blocks. 
    (c) Llama2 model architecture.}
    \label{fig:LLaMA_model}
\end{figure}

Figure~\ref{fig:LLaMA_model}(c) demonstrates the Llama2~\cite{touvron2023llama} model architecture as a representative LLM.
% The self attention layer requires three GEMVs\footnote{GEMVs in multi-head attention~\cite{attention}, narrow GEMMs in grouped-query attention~\cite{gqa}.} to generate query, key and value vectors.
In the self-attention layer, query, key and value vectors are generated by multiplying input vector to corresponding weight matrices.
These matrices are segmented into multiple heads, representing different semantic dimensions.
The query and key vectors go though Rotary Positional Embedding (RoPE) to encode the relative positional information~\cite{rope-paper}.
Within each head, the generated key and value vectors are appended to their caches.
The query vector is multiplied by the key cache to produce a score vector.
After the Softmax operation, the score vector is multiplied by the value cache to yield the output vector.
The output vectors from all heads are concatenated and multiplied by output weight matrix, resulting in a vector that undergoes residual connection and Root Mean Square layer Normalization (RMSNorm)~\cite{rmsnorm-paper}.
The residual connection adds up the input and output vectors of a layer to avoid vanishing gradient~\cite{he2016deep}.
The FFN layer begins with two parallel fully connections, followed by a Sigmoid Linear Unit (SiLU), and ends with another fully connection.
\section{RoleMRC}
\label{sec:method}

In this section, we build RoleMRC. Figure\,\ref{fig:method} illustrates the overall pipeline of RoleMRC from top to bottom, which is divided into three steps.

\subsection{A Meta-pool of 10k Role Profiles}
\label{sec:meta_pool}
We first collect a meta-pool of 10k role profile using two open-source datasets, with Step 1 and 2.

\paragraph{Step 1: Persona Sampling.} We randomly sample 10.5k one-sentence demographic persona description from PersonaHub\,\cite{ge2024scaling}, such as ``\emph{A local business owner interested in economic trends}'', as shown at the top of Figure\,\ref{fig:method}. 

\paragraph{Step 2: Role Profile Standardization.} Next, we use a well-crafted prompt with gpt-4o\,\cite{gpt4o} to expand each sampled persona into a complete role profile, in reference to the 1-shot standardized example. Illustrated in the middle of Figure\,\ref{fig:method}, we require a standardized role profile consisting of seven components: \emph{Role Name and Brief Description}, \emph{Specific Abilities and Skills}, \emph{Speech Style}, \emph{Personality Characteristics}, \emph{Past Experience and Background}, \emph{Ability and Knowledge Boundaries} and \emph{Speech Examples}. %Setting standard specifications helps convert the generated role profiles into formatted records, which is beneficial for the post quality control. 
Standardizing these profiles ensures structured formatting, simplifying quality control. 
After manual checking and format filtering, we remove 333 invalid responses from gpt-4o, resulting in 10.2k final role profiles. We report complete persona-to-profile standardization prompt and structure tree of final role profiles in Appendix\,\ref{sec:app_prompt_1} and \,\ref{sec:app_tree}, respectively.

Machine Reading Comprehension (MRC) is one of the core tasks for LLMs to interact with human users. Consequently, we choose to synthesize fine-grained role-playing instruction-following data based on MRC. We first generate a retrieval pool containing 808.7k MRC data from the MSMARCO training set\,\cite{bajaj2016ms}. By leveraging SFR-Embedding\,\cite{SFR-embedding-2}, we perform an inner product search to identify the most relevant and least relevant MRC triplets (Passages, Question, Answer) for each role profile. For example, the middle part of Figure\,\ref{fig:method} shows that for the role \emph{Jessica Thompson, a resilient local business owner}, the most relevant question is about \emph{the skill of resiliency}, while the least relevant question is \emph{converting Fahrenheit to Celsius}. After review, we categorise the most relevant MRC triplet as within a role's knowledge boundary, and the least relevant MRC triplet as beyond their expertise.

\begin{figure}[t]
    \centering
    \includegraphics[width=1.0\linewidth]{figures/step3.png}
    \caption{The strategy of gradually synthesizing finer role-playing instructions in step 3 of Figure\,\ref{fig:method}.}
    \vspace{-1.0em}
    \label{fig:step3}
\end{figure}

\subsection{38k Role-playing Instructions}
Based on the role profiles, we then adopt \textbf{Step 3: Multi-stage Dialogue Synthesis} to generate 38k role-playing instructions, progressively increasing granularity across three categories %including three types with gradually finer granularity 
(Figure\,\ref{fig:step3}):
%\begin{itemize}
%[leftmargin=*,noitemsep,topsep=0pt]

\noindent \textbf{\underline{Free Chats.}} The simplest dialogues, free chats, are synthesized at first. Here, we ask gpt-4o to simulate and generate multi-turn open-domain conversations between the role and an imagined user based on the standardized role profile. When synthesizing the conversation, we additionally consider two factors: the \textbf{initial speaker} in the starting round of the conversation, and whether the role's speech has \textbf{a narration wrapped in brackets} at the beginning (e.g., \emph{(Aiden reviews the network logs, his eyes narrowing as he spots unusual activity) I found it!}). The narration refers to a short, vivid description of the role's speaking state from an omniscient perspective, which further strengthens the sense of role's depth and has been adopted in some role-playing datasets\,\cite{tu2024charactereval}. 

As shown on the left side of Figure\,\ref{fig:step3}, based on the aforementioned two factors, we synthesize four variations of Free Chats. In particular, when  narration is omitted, we deleted all the 
narration content in the speech examples from the role profile; %and for the case that 
when narration is allowed, we retain the narration content, and also add instructions to allow appropriate insertion of narration in the task prompt of gpt-4o. It worth to note that, in narration-allowed dialogues, not every response of the role has narration inserted to prevent overfitting. All categories of data in RoleMRC incorporate narration insertion and follow similar control mechanisms. The following sections will omit further details on narration.

\noindent \textbf{\underline{On-scene MRC Dialogues.}} The synthesis of on-scene MRC dialogues can be divided into two parts. The first part is similar to the free chats. As shown by the {\color{lightgreen}{green round rectangle}} in the upper part of Figure\,\ref{fig:step3}, we ask gpt-4o to synthesize a conversation (lower left corner of Figure\,\ref{fig:step3}) between the role and the user focusing on relevant passages. This part of the synthesis and the Free Chats share the entire meta-pool, so each consisting of 5k dialogues.

The remaining part forms eight types of single-turn role-playing Question Answering (QA). In the middle of Figure\,\ref{fig:step3}, we randomly select a group of roles and examined the most relevant MRCs they matched: if the question in the MRC is answerable, then the ground truth answer is stylized to match the role profile; otherwise, a seed script of ``unanswerable'' is randomly selected then stylized. The above process generates four groups of 1k data from type ``[1]'' to type``[4]''. According to the middle right side of Figure\,\ref{fig:step3}, we also select a group of roles and ensure that the least relevant MRCs they matched contain answerable QA pairs. Since the most irrelevant MRCs are outside the knowledge boundary of the roles, the role-playing responses to these questions are ``out-of-mind'' refusal or ``let-me-try'' attempt, thus synthesizing four groups of 1k data, from type ``[5]'' to type ``[8]''.

\noindent \textbf{\underline{Ruled Chats.}} We construct Ruled Chats by extending On-scene MRC Dialogues in categories ``[1]'' to ``[8]'' with incorporated three additional rules, as shown in the right bottom corner of Figure\,\ref{fig:step3}. For the \textbf{multi-turn rules}, we apply them to the four unanswerable scenarios ``[3]'', ``[4]'', ``[5]'', and ``[6]'', adding a user prompt that  forces the role to answer. Among them, data ``[3]'' and ``[4]'' maintain refusal since the questions in MRC are unanswerable; while ``[5]'' and ``[6]'' are transformed into attempts to answer despite knowledge limitations. For the \textbf{nested formatting rules}, we add new formatting instructions to the four categories of data ``[1]'', ``[2]'', ``[3]'', and ``[4]'', such as requiring emojis,  capitalization, specific punctuation marks, and controlling the total number of words, then modify the previous replies accordingly. For the last \textbf{prioritized rules}, we apply them to subsets ``[1]'' and ``[2]'' that contain normal stylized answers, inserting a  global refusal directive from the system, and thus creating a conflict between system instructions and the role's ability boundary.
%\end{itemize}

\begin{table}[t]
\resizebox{\columnwidth}{!}{%
  \begin{tabular}{c|c|c|c|c|c}
    \toprule
    & & \textbf{S*} & \textbf{P*} & \textbf{\#Turns} & \textbf{\#Words} \\ 
    \midrule
    \multirow{13.5}{*}{\textbf{RoleMRC}} 
    & \multicolumn{5}{c|}{\textbf{Free Chats}} \\ 
    \cmidrule(lr){2-6}
    & Chats & 5k & / & 9.47 & 38.62 \\ 
    \cmidrule(lr){2-6}
    & \multicolumn{5}{c|}{\textbf{On-scene MRC Dialogues}} \\ 
    \cmidrule(lr){2-6} 
    & On-scene Chats & 5k & / & 9.2 & 43.18 \\
    & Answer & 2k & 2k & 1 & 39.45 \\ 
    & No Answer & 2k & 2k & 1 & 47.09 \\ 
    & Refusal & 2k & 2k & 1 & 48.41 \\ 
    & Attempt & 2k & 2k & 1 & 47.92 \\ 
    \cmidrule(lr){2-6}
    & \multicolumn{5}{c|}{\textbf{Ruled Chats}} \\ 
    \cmidrule(lr){2-6}
    & Multi-turn & 2k & 2k & 2 & 42.47 \\ 
    & Nested & 1.6k & 1.6k & 1 & 46.17 \\ 
    & Prioritized & 2.4k & 2.4k & 1 & 42.65 \\ 
    \midrule
    & \textbf{Total} & 24k & 14k & 3.5 & 40.6 \\ 
    \midrule
    \multirow{3}{*}{\textbf{-mix}} 
    & RoleBench & 16k & / & 1 & 23.95 \\ 
    & RLHFlow & 40k & / & 1.39 & 111.79 \\ 
    & UltraFeedback & / & 14k & 1 & 199.28 \\ 
    \midrule
    & \textbf{Total} & 80k & 28k & 2 & 67.1 \\ 
    \bottomrule
  \end{tabular}}
  \vspace{-2mm}
  \caption{Statistics of RoleMRC. In particular, the column names S*, P*, \#Turns, and \#Words, stands for size of single-label data, size of pair-label data, average turns, and average number of words per reply, respectively. RoleMRC-mix expands RoleMRC by adding existing role-playing data.}
 \vspace{-3mm}
  \label{tab:roleMRC}
\end{table}

\subsection{Integration and Mix-up}
All the seed scripts and prioritized rules used for constructing On-scene Dialogues and Ruled Chats are reported in Appendix\,\ref{sec:app_scripts}. These raw responses are logically valid manual answers that remain unaffected by the roles' speaking styles, making them suitable as negative labels to contrast with the stylized answers. Thanks to these meticulous seed texts, we obtain high-quality synthetic data with stable output from gpt-4o. After integration, as shown in Table\,\ref{tab:roleMRC}, the final RoleMRC contains 24k single-label data for Supervised Fine-Tuning (SFT) and 14k pair-label data for Human Preference Optimization (HPO)\,\cite{ouyang2022training,rafailov2023direct,sampo,hong2024reference}. Considering that fine-tuning LLMs with relatively fixed data formats may lead to catastrophic forgetting\,\cite{kirkpatrick2017overcoming}, we create RoleMRC-mix as a robust version by incorporating external role-playing data (RoleBench\,\cite{wang2023rolellm}) and general instructions (RLHFlow\,\cite{dong2024rlhf}, UltraFeedback\,\cite{cui2023ultrafeedback}).

\section{Experimental Results}
In this section, we present the main results in~\secref{sec:main}, followed by ablation studies on key design choices in~\secref{sec:ablation}.

\begin{table*}[t]
\renewcommand\arraystretch{1.05}
\centering
\setlength{\tabcolsep}{2.5mm}{}
\begin{tabular}{l|l|c|cc|cc}
type & model     & \#params      & FID$\downarrow$ & IS$\uparrow$ & Precision$\uparrow$ & Recall$\uparrow$ \\
\shline
GAN& BigGAN~\cite{biggan} & 112M & 6.95  & 224.5       & 0.89 & 0.38     \\
GAN& GigaGAN~\cite{gigagan}  & 569M      & 3.45  & 225.5       & 0.84 & 0.61\\  
GAN& StyleGan-XL~\cite{stylegan-xl} & 166M & 2.30  & 265.1       & 0.78 & 0.53  \\
\hline
Diffusion& ADM~\cite{adm}    & 554M      & 10.94 & 101.0        & 0.69 & 0.63\\
Diffusion& LDM-4-G~\cite{ldm}   & 400M  & 3.60  & 247.7       & -  & -     \\
Diffusion & Simple-Diffusion~\cite{diff1} & 2B & 2.44 & 256.3 & - & - \\
Diffusion& DiT-XL/2~\cite{dit} & 675M     & 2.27  & 278.2       & 0.83 & 0.57     \\
Diffusion&L-DiT-3B~\cite{dit-github}  & 3.0B    & 2.10  & 304.4       & 0.82 & 0.60    \\
Diffusion&DiMR-G/2R~\cite{liu2024alleviating} &1.1B& 1.63& 292.5& 0.79 &0.63 \\
Diffusion & MDTv2-XL/2~\cite{gao2023mdtv2} & 676M & 1.58 & 314.7 & 0.79 & 0.65\\
Diffusion & CausalFusion-H$^\dag$~\cite{deng2024causal} & 1B & 1.57 & - & - & - \\
\hline
Flow-Matching & SiT-XL/2~\cite{sit} & 675M & 2.06 & 277.5 & 0.83 & 0.59 \\
Flow-Matching&REPA~\cite{yu2024representation} &675M& 1.80 & 284.0 &0.81 &0.61\\    
Flow-Matching&REPA$^\dag$~\cite{yu2024representation}& 675M& 1.42&  305.7& 0.80& 0.65 \\
\hline
Mask.& MaskGIT~\cite{maskgit}  & 227M   & 6.18  & 182.1        & 0.80 & 0.51 \\
Mask. & TiTok-S-128~\cite{yu2024image} & 287M & 1.97 & 281.8 & - & - \\
Mask. & MAGVIT-v2~\cite{yu2024language} & 307M & 1.78 & 319.4 & - & - \\ 
Mask. & MaskBit~\cite{weber2024maskbit} & 305M & 1.52 & 328.6 & - & - \\
\hline
AR& VQVAE-2~\cite{vqvae2} & 13.5B    & 31.11           & $\sim$45     & 0.36           & 0.57          \\
AR& VQGAN~\cite{vqgan}& 227M  & 18.65 & 80.4         & 0.78 & 0.26   \\
AR& VQGAN~\cite{vqgan}   & 1.4B     & 15.78 & 74.3   & -  & -     \\
AR&RQTran.~\cite{rq}     & 3.8B    & 7.55  & 134.0  & -  & -    \\
AR& ViTVQ~\cite{vit-vqgan} & 1.7B  & 4.17  & 175.1  & -  & -    \\
AR & DART-AR~\cite{gu2025dart} & 812M & 3.98 & 256.8 & - & - \\
AR & MonoFormer~\cite{zhao2024monoformer} & 1.1B & 2.57 & 272.6 & 0.84 & 0.56\\
AR & Open-MAGVIT2-XL~\cite{luo2024open} & 1.5B & 2.33 & 271.8 & 0.84 & 0.54\\
AR&LlamaGen-3B~\cite{llamagen}  &3.1B& 2.18& 263.3 &0.81& 0.58\\
AR & FlowAR-H~\cite{flowar} & 1.9B & 1.65 & 296.5 & 0.83 & 0.60\\
AR & RAR-XXL~\cite{yu2024randomized} & 1.5B & 1.48 & 326.0 & 0.80 & 0.63 \\
\hline
MAR & MAR-B~\cite{mar} & 208M & 2.31 &281.7 &0.82 &0.57 \\
MAR & MAR-L~\cite{mar} &479M& 1.78 &296.0& 0.81& 0.60 \\
MAR & MAR-H~\cite{mar} & 943M&1.55& 303.7& 0.81 &0.62 \\
\hline
VAR&VAR-$d16$~\cite{var}   & 310M  & 3.30& 274.4& 0.84& 0.51    \\
VAR&VAR-$d20$~\cite{var}   &600M & 2.57& 302.6& 0.83& 0.56     \\
VAR&VAR-$d30$~\cite{var}   & 2.0B      & 1.97  & 323.1 & 0.82 & 0.59      \\
\hline
\modelname& \modelname-B    &172M   &1.72&280.4&0.82&0.59 \\
\modelname& \modelname-L   & 608M   & 1.28& 292.5&0.82&0.62\\
\modelname& \modelname-H    & 1.1B    & 1.24 &301.6&0.83&0.64\\
\end{tabular}
\caption{
\textbf{Generation Results on ImageNet-256.}
Metrics include Fréchet Inception Distance (FID), Inception Score (IS), Precision, and Recall. $^\dag$ denotes the use of guidance interval sampling~\cite{guidance}. The proposed \modelname-H achieves a state-of-the-art 1.24 FID on the ImageNet-256 benchmark without relying on vision foundation models (\eg, DINOv2~\cite{dinov2}) or guidance interval sampling~\cite{guidance}, as used in REPA~\cite{yu2024representation}.
}\label{tab:256}
\end{table*}

\subsection{Main Results}
\label{sec:main}
We conduct experiments on ImageNet~\cite{deng2009imagenet} at 256$\times$256 and 512$\times$512 resolutions. Following prior works~\cite{dit,mar}, we evaluate model performance using FID~\cite{fid}, Inception Score (IS)~\cite{is}, Precision, and Recall. \modelname is trained with the same hyper-parameters as~\cite{mar,dit} (\eg, 800 training epochs), with model sizes ranging from 172M to 1.1B parameters. See Appendix~\secref{sec:sup_hyper} for hyper-parameter details.





\begin{table}[t]
    \centering
    \begin{tabular}{c|c|c|c}
      model    &  \#params & FID$\downarrow$ & IS$\uparrow$ \\
      \shline
      VQGAN~\cite{vqgan}&227M &26.52& 66.8\\
      BigGAN~\cite{biggan}& 158M&8.43 &177.9\\
      MaskGiT~\cite{maskgit}& 227M&7.32& 156.0\\
      DiT-XL/2~\cite{dit} &675M &3.04& 240.8 \\
     DiMR-XL/3R~\cite{liu2024alleviating}& 525M&2.89 &289.8 \\
     VAR-d36~\cite{var}  & 2.3B& 2.63 & 303.2\\
     REPA$^\ddagger$~\cite{yu2024representation}&675M &2.08& 274.6 \\
     \hline
     \modelname-L & 608M&1.70& 281.5 \\
    \end{tabular}
    \caption{
    \textbf{Generation Results on ImageNet-512.} $^\ddagger$ denotes the use of DINOv2~\cite{dinov2}.
    }
    \label{tab:512}
\end{table}

\noindent\textbf{ImageNet-256.}
In~\tabref{tab:256}, we compare \modelname with previous state-of-the-art generative models.
Out best variant, \modelname-H, achieves a new state-of-the-art-performance of 1.24 FID, outperforming the GAN-based StyleGAN-XL~\cite{stylegan-xl} by 1.06 FID, masked-prediction-based MaskBit~\cite{maskgit} by 0.28 FID, AR-based RAR~\cite{yu2024randomized} by 0.24 FID, VAR~\cite{var} by 0.73 FID, MAR~\cite{mar} by 0.31 FID, and flow-matching-based REPA~\cite{yu2024representation} by 0.18 FID.
Notably, \modelname does not rely on vision foundation models~\cite{dinov2} or guidance interval sampling~\cite{guidance}, both of which were used in REPA~\cite{yu2024representation}, the previous best-performing model.
Additionally, our lightweight \modelname-B (172M), surpasses DiT-XL (675M)~\cite{dit} by 0.55 FID while achieving an inference speed of 9.8 images per second—20$\times$ faster than DiT-XL (0.5 images per second). Detailed speed comparison can be found in Appendix \ref{sec:speed}.



\noindent\textbf{ImageNet-512.}
In~\tabref{tab:512}, we report the performance of \modelname on ImageNet-512.
Similarly, \modelname-L sets a new state-of-the-art FID of 1.70, outperforming the diffusion based DiT-XL/2~\cite{dit} and DiMR-XL/3R~\cite{liu2024alleviating} by a large margin of 1.34 and 1.19 FID, respectively.
Additionally, \modelname-L also surpasses the previous best autoregressive model VAR-d36~\cite{var} and flow-matching-based REPA~\cite{yu2024representation} by 0.93 and 0.38 FID, respectively.




\noindent\textbf{Qualitative Results.}
\figref{fig:qualitative} presents samples generated by \modelname (trained on ImageNet) at 512$\times$512 and 256$\times$256 resolutions. These results highlight \modelname's ability to produce high-fidelity images with exceptional visual quality.

\begin{figure*}
    \centering
    \vspace{-6pt}
    \includegraphics[width=1\linewidth]{figures/qualitative.pdf}
    \caption{\textbf{Generated Samples.} \modelname generates high-quality images at resolutions of 512$\times$512 (1st row) and 256$\times$256 (2nd and 3rd row).
    }
    \label{fig:qualitative}
\end{figure*}

\subsection{Ablation Studies}
\label{sec:ablation}
In this section, we conduct ablation studies using \modelname-B, trained for 400 epochs to efficiently iterate on model design.

\noindent\textbf{Prediction Entity X.}
The proposed \modelname extends next-token prediction to next-X prediction. In~\tabref{tab:X}, we evaluate different designs for the prediction entity X, including an individual patch token, a cell (a group of surrounding tokens), a subsample (a non-local grouping), a scale (coarse-to-fine resolution), and an entire image.

Among these variants, cell-based \modelname achieves the best performance, with an FID of 2.48, outperforming the token-based \modelname by 1.03 FID and surpassing the second best design (scale-based \modelname) by 0.42 FID. Furthermore, even when using standard prediction entities such as tokens, subsamples, images, or scales, \modelname consistently outperforms existing methods while requiring significantly fewer parameters. These results highlight the efficiency and effectiveness of \modelname across diverse prediction entities.






\begin{table}[]
    \centering
    \scalebox{0.92}{
    \begin{tabular}{c|c|c|c|c}
        model & \makecell[c]{prediction\\entity} & \#params & FID$\downarrow$ & IS$\uparrow$\\
        \shline
        LlamaGen-L~\cite{llamagen} & \multirow{2}{*}{token} & 343M & 3.80 &248.3\\
        \modelname-B& & 172M&3.51&251.4\\
        \hline
        PAR-L~\cite{par} & \multirow{2}{*}{subsample}& 343M & 3.76 & 218.9\\
        \modelname-B&  &172M& 3.58&231.5\\
        \hline
        DiT-L/2~\cite{dit}& \multirow{2}{*}{image}& 458M&5.02&167.2 \\
         \modelname-B& & 172M&3.13&253.4 \\
        \hline
        VAR-$d16$~\cite{var} & \multirow{2}{*}{scale} & 310M&3.30 &274.4\\
        \modelname-B& &172M&2.90&262.8\\
        \hline
        \baseline{\modelname-B}& \baseline{cell} & \baseline{172M}&\baseline{2.48}&\baseline{269.2} \\
    \end{tabular}
    }
    \caption{\textbf{Ablation on Prediction Entity X.} Using cells as the prediction entity outperforms alternatives such as tokens or entire images. Additionally, under the same prediction entity, \modelname surpasses previous methods, demonstrating its effectiveness across different prediction granularities. }%
    \label{tab:X}
\end{table}

\noindent\textbf{Cell Size.}
A prediction entity cell is formed by grouping spatially adjacent $k\times k$ tokens, where a larger cell size incorporates more tokens and thus captures a broader context within a single prediction step.
For a $256\times256$ input image, the encoded continuous latent representation has a spatial resolution of $16\times16$. Given this, the image can be partitioned into an $m\times m$ grid, where each cell consists of $k\times k$ neighboring tokens. As shown in~\tabref{tab:cell}, we evaluate different cell sizes with $k \in \{1,2,4,8,16\}$, where $k=1$ represents a single token and $k=16$ corresponds to the entire image as a single entity. We observe that performance improves as $k$ increases, peaking at an FID of 2.48 when using cell size $8\times8$ (\ie, $k=8$). Beyond this, performance declines, reaching an FID of 3.13 when the entire image is treated as a single entity.
These results suggest that using cells rather than the entire image as the prediction unit allows the model to condition on previously generated context, improving confidence in predictions while maintaining both rich semantics and local details.





\begin{table}[t]
    \centering
    \scalebox{0.98}{
    \begin{tabular}{c|c|c|c}
    cell size ($k\times k$ tokens) & $m\times m$ grid & FID$\downarrow$ & IS$\uparrow$ \\
       \shline
       $1\times1$ & $16\times16$ &3.51&251.4 \\
       $2\times2$ & $8\times8$ & 3.04& 253.5\\
       $4\times4$ & $4\times4$ & 2.61&258.2 \\
       \baseline{$8\times8$} & \baseline{$2\times2$} & \baseline{2.48} & \baseline{269.2}\\
       $16\times16$ & $1\times1$ & 3.13&253.4  \\
    \end{tabular}
    }
    \caption{\textbf{Ablation on the cell size.}
    In this study, a $16\times16$ continuous latent representation is partitioned into an $m\times m$ grid, where each cell consits of $k\times k$ neighboring tokens.
    A cell size of $8\times8$ achieves the best performance, striking an optimal balance between local structure and global context.
    }
    \label{tab:cell}
\end{table}



\begin{table}[t]
    \centering
    \scalebox{0.95}{
    \begin{tabular}{c|c|c|c}
      previous cell & noise time step &  FID$\downarrow$ & IS$\uparrow$ \\
       \shline
       clean & $t_i=0, \forall i<n$& 3.45& 243.5\\
       increasing noise & $t_1<t_2<\cdots<t_{n-1}$& 2.95&258.8 \\
       decreasing noise & $t_1>t_2>\cdots>t_{n-1}$&2.78 &262.1 \\
      \baseline{random noise}  & \baseline{no constraint} &\baseline{2.48} & \baseline{269.2}\\
    \end{tabular}
    }
    \caption{
    \textbf{Ablation on Noisy Context Learning.}
    This study examines the impact of noise time steps ($t_1, \cdots, t_{n-1} \subset [0, 1]$) in previous entities ($t=0$ represents pure Gaussian noise).
    Conditioning on all clean entities (the ``clean'' variant) results in suboptimal performance.
    Imposing an order on noise time steps, either ``increasing noise'' or ``decreasing noise'', also leads to inferior results. The best performance is achieved with the "random noise" setting, where no constraints are imposed on noise time steps.
    }
    \label{tab:ncl}
\end{table}


\noindent\textbf{Noisy Context Learning.}
During training, \modelname employs Noisy Context Learning (NCL), predicting $X_n$ by conditioning on all previous noisy entities, unlike Teacher Forcing.
The noise intensity of previous entities is contorlled by noise time steps $\{t_1, \dots, t_{n-1}\} \subset [0, 1]$, where $t=0$ corresponds to pure Gaussian noise.
We analyze the impact of NCL in~\tabref{tab:ncl}.
When conditioning on all clean entities (\ie, the ``clean'' variant, where $t_i=0, \forall i<n$), which is equivalent to vanilla AR (\ie, Teacher Forcing), the suboptimal performance is obtained.
We also evaluate two constrained noise schedules: the ``increasing noise'' variant, where noise time steps increase over AR steps ($t_1<t_2< \cdots < t_{n-1}$), and the `` decreasing noise'' variant, where noise time steps decrease ($t_1>t_2> \cdots > t_{n-1}$).
While both settings improve over the ``clean'' variant, they remain inferior to our final ``random noise'' setting, where no constraints are imposed on noise time steps, leading to the best performance.




        


\section{Discussion and Conclusion}

% \begin{quote}
% \textit{"We believe it is unethical for social workers not to learn... about technology-mediated social work."} (\citeauthor{singer_ai_2023}, 2023)
% \end{quote}

In this study, we uncovered multiple ways in which GenAI can be used in social service practice. While some concerns did arise, practitioners by and large seemed optimistic about the possibilities of such tools, and that these issues could be overcome. We note that while most participants found the tool useful, it was far from perfect in its outputs. This is not surprising, since it was powered by a generic LLM rather than one fine-tuned for social service case management. However, despite these inadequacies, our participants still found many uses for most of the tool's outputs. Many flaws pointed out by our participants related to highly contextualised, local knowledge. To tune an AI system for this would require large amounts of case files as training data; given the privacy concerns associated with using client data, this seems unlikely to happen in the near future. What our study shows, however, is that GenAI systems need not aim to be perfect to be useful to social service practitioners, and can instead serve as a complement to the critical "human touch" in social service.

We draw both inspiration and comparisons with prior work on AI in other settings. Studies on creative writing tools showed how the "uncertainty" \cite{wan2024felt} and "randomness" \cite{clark2018creative} of AI outputs aid creativity. Given the promise that our tool shows in aiding brainstorming and discussion, future social service studies could consider AI tools explicitly geared towards creativity - for instance, providing side-by-side displays of how a given case would fit into different theoretical frameworks, prompting users to compare, contrast, and adopt the best of each framework; or allowing users to play around with combining different intervention modalities to generate eclectic (i.e. multi-modal) interventions.

At the same time, the concept of supervision creates a different interaction paradigm to other uses of AI in brainstorming. Past work (e.g. \cite{shaer2024ai}) has explored the use of GenAI for ideation during brainstorming sessions, wherein all users present discuss the ideas generated by the system. With supervision in social service practice, however, there is a marked information and role asymmetry: supervisors may not have had the time to fully read up on their supervisee's case beforehand, yet have to provide guidance and help to the latter. We suggest that GenAI can serve a dual purpose of bringing supervisors up to speed quickly by summarising their supervisee's case data, while simultaneously generating a list of discussion and talking points that can improve the quality of supervision. Generalising, this interaction paradigm has promise in many other areas: senior doctors reviewing medical procedures with newer ones \cite{snowdon2017does} could use GenAI to generate questions about critical parts of a procedure to ask the latter, confirming they have been correctly understood or executed; game studio directors could quickly summarise key developmental pipeline concerns to raise at meetings and ensure the team is on track; even in academia, advisors involved in rather too many projects to keep track of could quickly summarise each graduate student's projects and identify potential concerns to address at their next meeting.

In closing, we are optimistic about the potential for GenAI to significantly enhance social service practice and the quality of care to clients. Future studies could focus on 1) longitudinal investigations into the long-term impact of GenAI on practitioner skills, client outcomes, and organisational workflows, and 2) optimising workflows to best integrate GenAI into casework and supervision, understanding where best to harness the speed and creativity of such systems in harmony with the experience and skills of practitioners at all levels.

% GenAI here thus serves as a tool that supervisors can use before rather then using the session, taking just a few minutes of their time to generate a list of discussion points with their supervisees.

% Traditional brainstorming comes up with new things that users discuss. In supervision, supervisors can use AI to more efficiently generate talking points with their supervisees. These are generally not novel ideas, since an experienced worker would be able to come up with these on their own. However, the interesting and novel use of AI here is in its use as a preparation tool, efficiently generating talking and discussion points, saving supervisors' time in preparing for a session, while still serving as a brainstorming tool during the session itself.

% The idea of embracing imperfect AI echoes the findings of \citeauthor{bossen2023batman} (2023) in a clinical decision setting, which examined the successful implementation of an "error-prone but useful AI tool". This study frames human-AI collaboration as "Batman and Robin", where AI is a useful but ultimately less skilled sidekick that plays second fiddle to Batman. This is similar to \citeauthor{yang2019unremarkable}'s (2019) idea of "unremarkable AI", systems designed to be unobtrusive and only visible to the user when they add some value. As compared to \citeauthor{bossen2023batman}, however, we see fewer instances of our AI system producing errors, and more examples of it providing learning and collaborative opportunities and other new use cases. We build on the idea of "complementary performance" \cite{bansal2021does}, which discusses how the unique expertise of AI enhances human decision-making performance beyond what humans can achieve alone. Beyond decision-making, GenAI can now enable "complementary work patterns", where the nature of its outputs enables humans to carry out their work in entirely new ways. Our study suggests that rather being a sidekick - Robin - AI is growing into the role of a "second Batman" or "AI-Batman": an entity with distinct abilities and expertise from humans, and that contributes in its own unique way. There is certainly still a time and place for unremarkable AI, but exploring uses beyond that paradigm uncovers entirely new areas of system design.

% % \cite{gero2022sparks} found AI to be useful for science writers to translate ideas already in their head into words, and to provide new perspectives to spark further inspiration. \textit{But how is ours different from theirs?}

% \subsection{New Avenues of Human-AI Collaboration}
% \label{subsubsec:discussionhaicollaboration}

% Past HCI literature in other areas \cite{nah2023generative} has suggested that GenAI represents a "leap" \cite{singh2023hide} in human-AI collaboration, 
% % Even when an AI system sometimes produces irrelevant outputs, it can still provide users 
% % Such systems have been proposed as ways to 
% helping users discover new viewpoints \cite{singh2023hide}, scour existing literature to suggest new hypotheses 
% \cite{cascella2023evaluating} and answer questions \cite{biswas2023role}, stimulate their cognitive processes \cite{memmert2023towards}, and overcome "writer's block" \cite{singh2023hide, cooper2023examining} (particularly relevant to SSPs and the vast amount of writing required of them). Our study finds promise for AI to help SSPs in all of these areas. By nature of being more verbose and capable of generating large amounts of content, GenAI seems to create a new way in which AI can complement human work and expertise. Our system, as LLMs tend to do, produced a lot of "bullshit" (S6) \cite{frankfurt2005bullshit} - superficially true statements that were often only "tangentially related" and "devoid of meaning" \cite{halloran2023ai}. Yet, many participants cited the page-long analyses and detailed multi-step intervention plans generated by the AI system to be a good starting point for further discussion, both to better conceptualize a particular case and to facilitate general worker growth and development. Almost like throwing mud at a wall to see what sticks, GenAI can quickly produce a long list of ideas or information, before the worker glances through it and quickly identifies the more interesting points to discuss. Playing the proposed role as a "scaffold" for further work \cite{cooper2023examining}, GenAI, literally, generates new opportunities for novel and more effective processes and perspectives that previous systems (e.g., PRMs) could not. This represents an entirely new mode of human-AI collaboration.
% % This represents a new mode of collaboration not possible with the largely quantitative AI models (like PRMs) of the past.

% Our work therefore supports and extends prior research that have postulated the the potential of AI's shifting roles from decision-maker to human-supporter \cite{wang_human-human_2020}. \citeauthor{siemon2022elaborating} (2022) suggests the role of AI as a "creator" or "coordinator", rather than merely providing "process guidance" \cite{memmert2023towards} that does not contribute to brainstorming. Similarly, \citeauthor{memmert2023towards} (2023) propose GenAI as a step forward from providing meta-level process guidance (i.e. facilitating user tasks) to actively contributing content and aiding brainstorming. We suggest that beyond content-support, AI can even create new work processes that were not possible without GenAI. In this sense, AI has come full circle, becoming a "meta-facilitator".

% % --- WIP BELOW ---

% % Our work echoes and extends previous research on HAI collaboration in tasks requiring a human touch. \cite{gero2023social} found AI to be a safe space for creative writers to bounce ideas off of and document their inner thoughts. \cite{dhillon2024shaping} reference the idea of appropriate scaffolding in argumentative writing, where the user is providing with guidance appropriate for their competency level, and also warns of decreased satisfaction and ownership from AI use. 

% Separately, we draw parallels with the field of creative writing, where HAI collaboration has been extensively researched. Writers note the "irreducibly human" aspect of creativity in writing \cite{gero2023social}, similar to the "human touch" core to social service practice (D1); both groups therefore expressed few concerns about AI taking over core aspects of their jobs. Another interesting parallel was how writers often appreciated the "uncertainty" \cite{wan2024felt} and "randomness" \cite{clark2018creative} of AI systems, which served as a source of inspiration. This echoes the idea of "imperfect AI" "expanding [the] perspective[s]" (S4) of our participants when they simply skimmed through what the AI produced. \cite{wan2024felt} cited how the "duality of uncertainty in the creativity process advances the exploration of the imperfection of GenAI models". While social service work is not typically regarded as "creative", practitioners nonetheless go through processes of ideation and iteration while formulating a case. Our study showed hints of how AI can help with various forms of ideation, but, drawing inspiration from creative writing tools, future studies could consider designs more explicitly geared towards creativity - for instance, by attempting to fit a given case into a number of different theories or modalities, and displaying them together for the user to consider. While many of these assessments may be imperfect or even downnright nonsensical, they may contain valuable ideas and new angles on viewing the case that the practitioner can integrate into their own assessment.

% % \cite{foong2024designing}, describing the design of caregiver-facing values elicitation tools, cites the "twin scenarios" that caregivers face - private use, where they might use a tool to discover their patient's values, and collaborative use, where they discuss the resulting values with other parties close to the patient. This closely mirrors how SSPs in our study reference both individual and collaborative uses of our tool. Unlike in \cite{foong2024designing}, however, we do not see a resulting need to design a "staged approach" with distinct interface features for both stages.

% % --- END OF WIP ---

% Having mentioned algorithm aversion previously, we also make a quick point here on the other end of the spectrum - automation bias, or blind trust in an automated system \cite{brown2019toward}. LLMs risk being perceived as an "ultimate epistemic authority" \cite{cooper2023examining} due to their detailed, life-like outputs. While automation bias has been studied in many contexts, including in the social sector or adjacent areas, we suggest that the very nature of GenAI systems fundamentally inhibits automation bias. The tendency of GenAI to produce verbose, lengthy explanations prompts users to read and think through the machine's judgement before accepting it, bringing up opportunities to disagree with the machine's opinion. This guards against blind acceptance of the system's recommendations, particularly in the culture of a social work agency where constant dialogue - including discussing AI-produced work - is the norm.


% % : Perception of AI in Social Service Work ??

% \subsection{Redefining the Boundary}
% \label{subsubsec:discussiontheoretical}

% As \citeauthor{meilvang_working_2023} (2023) describes, the social service profession has sought to distance itself from comprising mostly "administrative work" \cite{abbott2016boundaries}, and workers have long tried to tried to reduce their considerable time \cite{socialraadgiverforening2010notat} spent on such tasks in favour of actual casework with clients \cite{toren1972social}. Our study, however, suggests a blurring of the line between "manual" administrative tasks and "mental" casework that draws on practitioner expertise. Many tasks our participants cited involve elements of both: for instance, documenting a case recording requires selecting only the relevant information to include, and planning an intervention can be an iterative process of drafting a plan and discussing it with colleagues and superiors. This all stems from the fact that GenAI can produce virtually any document required by the user, but this document almost always requires revision under a watchful human eye.

% \citeauthor{meilvang_working_2023} (2023) also describes a more recent shift in the perceived accepted boundary of AI interventions in social service work. From "defending [the] jurisdiction [of social service work] against artificial intelligence" in the early days of PRM and other statistical assessment tools, the community has started to embrace AI as a "decision-support ... element in the assessment process". Our study concurs and frames GenAI as a source of information that can be used to support and qualify the assessments of SSPs \cite{meilvang_working_2023}, but suggests that we can take a step further: AI can be viewed as a \textit{facilitator} rather than just a supporter. GenAI can facilitate a wide range of discussions that promote efficiency, encourage worker learning and growth, and ultimately enhance client outcomes. This entails a much larger scope of AI use, where practitioners use the information provided by AI in a range of new scenarios. 

% Taken together, these suggest a new focus for boundary work and, more broadly, HCI research. GAI can play a role not just in menial documentation or decision-support, but can be deeply ingrained into every facet of the social service workflow to open new opportunities for worker growth, workflow optimisation, and ultimately improved client outcomes. Future research can therefore investigate the deeper, organisational-level effects of these new uses of AI, and their resulting impact on the role of profession discretion in effective social service work.

% % MH: oh i feel this paragraph is quite new to me! Could we elaborate this more, and truncate the first two paragraphs a bit to adjust the word propotion?


% % Our study extensively documents this for the first time in social service practice, and in the process reveals new insights about how AI can play such a role.



% \subsection{Design Implications}

% % Add link from ACE diagram?

% % EJ: it would be interesting to discuss how LLMs could help "hands-on experience" in the discussion section

% Addressing the struggle of integrating AI amidst the tension between machine assessment and expert judgement, we reframe AI as an \textit{facilitator} rather than an algorithm or decision-support tool, alleviating many concerns about trust and explainablity. We now present a high-level framework (Figure \ref{fig:hai-collaboration}) on human-AI collaboration, presenting a new perspective on designing effective AI systems that can be applied to both the social service sector and beyond.

% \begin{figure}
%     \centering
%     \includegraphics[scale=0.15]{images/designframework.png}
%     \caption{Framework for Human-AI Collaboration}
%     \label{fig:hai-collaboration}
%     \Description{An image showing our framework for Human-AI Collaboration. It shows that as stakeholder level increases from junior to senior, the directness of use shifts from co-creation to provision.}
% \end{figure}
% % MH: so this paradigm is proposed by us? I wonder if this could a part of results as well..?

% % \subsubsection{From Creation to Provision}

% In Section \ref{sec:stage2findings}, we uncovered the different ways in which SSPs of varying seniorities use, evaluate, and suggest uses of AI. These are intrinsically tied to the perspectives and levels of expertise that each stakeholder possesses. We therefore position the role of AI along the scale of \textit{creation} to \textit{provision}. 

% With junior workers, we recommend \textbf{designing tools for co-creation}: systems that aid the least experienced workers in creating the required deliverables for their work. Rather than \textit{telling} workers what to do - a difficult task in any case given the complexity of social work solutions - AI systems should instead \textit{co-create} deliverables required of these workers. These encompass the multitude of use cases that junior workers found useful: creating reports, suggesting perspectives from which to formulate a case, and providing a starting template for possible intervention plans. Notably, since AI outputs are not perfect, we emphasise the "co" in "co-creation": AI should only be a part of the workflow that also includes active engagement on the part of the SSPs and proactive discussion with supervisors. 

% For more experienced SSPs, we recommend \textbf{designing tools for provision}. Again, this is not the mere provision of recommendations or courses of action with clients, but rather that of resources which complement the needs of workers with greater responsibilities. This notably includes supplying materials to aid with supervision, a novel use case that to our knowledge has not surfaced in previous literature. In addition, senior workers also benefit greatly from manual tasks such as routine report writing and data processing. Since these workers are more experienced and can better spot inaccuracies in AI output, we suggest that AI can "provide" a more finished product that requires less vetting and corrections, and which can be used more directly as part of required deliverables.

% % MH: can we seperate here? above is about the guidance to paradigm, below is the practical roadmap for implementation
% In terms of concrete design features, given the constant focus on discussing AI outputs between colleagues in our FGDs, we recommend that AI tools, particularly those for junior workers, \textbf{include collaborative features} that facilitate feedback and idea sharing between users. We also suggest that designers work closely with domain experts (i.e. social work practitioners and agencies) to identify areas where the given AI model tends to make more mistakes, and to build in features that \textbf{highlight potential mistakes or inadequacies} in the AI's output to facilitate further discussion and avoid workers adopting suboptimal suggestions. 

% We also point out a fundamental difference between GenAI systems and previous systems: that GenAI can now play an important role in aiding users \textit{regardless of its flaws}. The nature of GenAI means that it promotes discussion and opens up new workflows by nature of its verbose and potentially incomplete outputs. Rather than working towards more accurate or explainable outcomes, which may in any case have minimal improvement on worker outcomes \cite{li2024advanced}, designers can also focus on \textbf{understanding how GenAI outputs can augment existing user flows and create new ones}.

% % for more senior workers...

% % how to differentiate levels of workers?

% % \subsubsection{Provider}

% % The most basic and obvious role of modern AI that we identify leverages the main strength of LLMs. They have the ability to produce high-quality writing from short, point-form, or otherwise messy and disjoint case notes that user often have \textit{[cite participant here]}. 

% Finally, given the limited expertise of many workers at using AI, it is important that systems \textbf{explicitly guide users to the features they need}, rather than simply relying on the ability of GAI to understand complex user instructions. For example, in the case of flexibility in use cases (Section \ref{subsubsec:control}), systems should include user flows that help combine multiple intervention and assessment modalities in order to directly meet the needs of workers.

% \subsection{Limitations and Future Work}

% While we attempt to mimic a contextual inquiry and work environment in our study design, there is no substitute for real data from actual system deployment. The use of an AI system in day-to-day work could reveal a different set of insights. Future studies could in particular study how the longitudinal context of how user attitudes, behaviours, preferences, and work outputs change with extended use of AI. 

% While we tried to include practitioners from different agencies, roles, and seniorities, social service practice may differ culturally or procedurally in other agencies or countries. Future studies could investigate different kinds of social service agencies and in different cultures to see if AI is similarly useful there.

% As the study was conducted in a country with relatively high technology literacy, participants naturally had a higher baseline understanding and acceptance of AI and other computer systems. However, we emphasise that our findings are not contingent on this - rather, we suggest that our proposed lens of viewing AI in the social sector is a means for engaging in relevant stakeholders and ensuring the effective design and implementation of AI in the social sector, regardless of how participants feel about AI to begin with. 



% % \subsection{Notes}

% % 1) safety and risks and 2) privacy - what does the emphasis on this say about a) design recommendations and b) approach to designing/PD of such systems?


% % W9 was presented with "Strengths" and "SFBT" output options. They commented, "solution focus is always building on the person's strengths". W9 therefore requested being able to output strengths and SFBT at the same time. But this would suggest that the SFBT output does not currently emphasise strengths strongly enough. However, W9 did not specifically evaluate that, and only made this comment because they saw the "strengths" option available, and in their head, strengths are key to SFBT.
% % What does this say about system design and UI in relation to user mental models?
\paragraph{Summary}
Our findings provide significant insights into the influence of correctness, explanations, and refinement on evaluation accuracy and user trust in AI-based planners. 
In particular, the findings are three-fold: 
(1) The \textbf{correctness} of the generated plans is the most significant factor that impacts the evaluation accuracy and user trust in the planners. As the PDDL solver is more capable of generating correct plans, it achieves the highest evaluation accuracy and trust. 
(2) The \textbf{explanation} component of the LLM planner improves evaluation accuracy, as LLM+Expl achieves higher accuracy than LLM alone. Despite this improvement, LLM+Expl minimally impacts user trust. However, alternative explanation methods may influence user trust differently from the manually generated explanations used in our approach.
% On the other hand, explanations may help refine the trust of the planner to a more appropriate level by indicating planner shortcomings.
(3) The \textbf{refinement} procedure in the LLM planner does not lead to a significant improvement in evaluation accuracy; however, it exhibits a positive influence on user trust that may indicate an overtrust in some situations.
% This finding is aligned with prior works showing that iterative refinements based on user feedback would increase user trust~\cite{kunkel2019let, sebo2019don}.
Finally, the propensity-to-trust analysis identifies correctness as the primary determinant of user trust, whereas explanations provided limited improvement in scenarios where the planner's accuracy is diminished.

% In conclusion, our results indicate that the planner's correctness is the dominant factor for both evaluation accuracy and user trust. Therefore, selecting high-quality training data and optimizing the training procedure of AI-based planners to improve planning correctness is the top priority. Once the AI planner achieves a similar correctness level to traditional graph-search planners, strengthening its capability to explain and refine plans will further improve user trust compared to traditional planners.

\paragraph{Future Research} Future steps in this research include expanding user studies with larger sample sizes to improve generalizability and including additional planning problems per session for a more comprehensive evaluation. Next, we will explore alternative methods for generating plan explanations beyond manual creation to identify approaches that more effectively enhance user trust. 
Additionally, we will examine user trust by employing multiple LLM-based planners with varying levels of planning accuracy to better understand the interplay between planning correctness and user trust. 
Furthermore, we aim to enable real-time user-planner interaction, allowing users to provide feedback and refine plans collaboratively, thereby fostering a more dynamic and user-centric planning process.

\section{Acknowledgment}
\label{7_acknowledgment}

This work is supported by JSPS KAKENHI JP20H05706, JP23K28065, and Daiichi-Sankyo ``Habataku" Support Program for the Next Generation of Researchers, NAIST Senju Monju Project.

% \bibliographystyle{alpha}
\bibliographystyle{ieeetr}
\bibliography{bibliography}

\end{document}