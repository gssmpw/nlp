\documentclass[conference]{IEEEtran}
\IEEEoverridecommandlockouts

\usepackage{cite}
\usepackage{amsmath,amssymb,amsfonts}
\usepackage{algorithmic}
\usepackage{graphicx}
\usepackage{textcomp}
\usepackage{xcolor}
\usepackage{graphicx}
\usepackage{booktabs}
\usepackage{url}
\usepackage{multirow}
\usepackage{threeparttable}

\usepackage[most]{tcolorbox}

% \usepackage[belowskip=-15pt,aboveskip=0pt]{caption}
% \setlength{\intextsep}{30pt plus 10pt minus 5pt}

\newcommand{\morakot}[1]{
    \textcolor{blue}{\textbf{TODO:} #1}%
}

\newcommand{\chaiyong}[1]{
    \textcolor{red}{\textbf{~Chaiyong:} #1}%
}

\newcommand{\bee}[1]{
    \textcolor{blue}{\textbf{~Bee:} #1}%
}

\newcommand{\censortext}[1]{\textcolor{black}{\rule{3em}{1ex}}}


\def\BibTeX{{\rm B\kern-.05em{\sc i\kern-.025em b}\kern-.08em
    3T\kern-.1667em\lower.7ex\hbox{E}\kern-.125emX}}
    
\begin{document}
\title{On Categorizing Open Source Software Security Vulnerability Reporting Mechanisms on GitHub}

\author{%
Sushawapak Kancharoendee\textsuperscript{\dag}, Thanat Phichitphanphong\textsuperscript{\dag}, Chanikarn Jongyingyos\textsuperscript{\dag}, Brittany Reid\textsuperscript{*},\\
Raula Gaikovina Kula\textsuperscript{\S}, Morakot Choetkiertikul\textsuperscript{\dag}, Chaiyong Ragkhitwetsagul\textsuperscript{\dag}, Thanwadee Sunetnanta\textsuperscript{\dag}\\
\textsuperscript{\dag}\textit{Faculty of Information and Communication Technology, Mahidol University, Thailand}\\
\textsuperscript{*}\textit{Graduate School of Science and Technology, Nara Institute of Science and Technology (NAIST), Japan}\\
\textsuperscript{\S}\textit{Graduate School of Information Science and Technology, Osaka University, Japan}\\
\{sushawapak.kan, thanat.phi, chanikarn.jon\}@student.mahidol.ac.th, brittany.reid@naist.ac.jp, \\raula-k@ist.osaka-u.ac.jp, \{morakot.cho, chaiyong.rag, thanwadee.sun\}@mahidol.ac.th
}

\maketitle

\begin{abstract}

Open-source projects are essential to software development, but publicly disclosing vulnerabilities without fixes increases the risk of exploitation. The Open Source Security Foundation (OpenSSF) addresses this issue by promoting robust security policies to enhance project security. Current research reveals that many projects perform poorly on OpenSSF criteria, indicating a need for stronger security practices and underscoring the value of SECURITY.md files for structured vulnerability reporting. This study aims to provide recommendations for improving security policies. By examining 679 open-source projects, we find that email is still the main source of reporting. Furthermore, we find that projects without SECURITY.md files tend to be less secure (lower OpenSSF scores). Our analysis also indicates that, although many maintainers encourage private reporting methods, some contributors continue to disclose vulnerabilities publicly, bypassing established protocols. The results from this preliminary study pave the way for understanding how developers react and communicate a potential security threat. Future challenges include understanding the impact and effectiveness of these mechanisms and what factors may influence how the security threat is addressed.

% Open-source projects are essential to software development, yet publicly disclosing vulnerabilities without fixes increases the risk of exploitation. The Open Source Security Foundation (OpenSSF) addresses this issue by offering guidelines that encourage robust security policies to enhance project security. Current research shows that most projects perform poorly on OpenSSF criteria, suggesting a limited commitment to security practices and underscoring the importance of SECURITY.md files for structured vulnerability reporting. The objective of this study is to propose guidelines to improve security policies. By examining 679 open-source projects, we find that those without SECURITY.md files tend to have slightly lower OpenSSF scores. Our analysis indicates that, although many maintainers recommend private reporting methods, some vulnerabilities continue to be disclosed publicly by contributors, bypassing established protocols.
\end{abstract}

\begin{IEEEkeywords}
open-source, security policy, OpenSSF
\end{IEEEkeywords}


\section{Introduction}\label{sec:intro}

In computational finance, Monte Carlo simulations are used extensively to estimate the expected value of financial payoffs based on the solution of stochastic differential equations (SDEs) which model the evolution of stock prices, interest rates, exchange rates and other quantities \cite{glasserman04}.  Monte Carlo methods are very general and flexible, but for high accuracy it requires generating a large number of costly SDE path approximations, which has motivated research into a number of variance reduction or, equivalently, cost reduction techniques. One such method is
Multilevel Monte Carlo (MLMC), which was proposed in \cite{GILES2008} and was adapted for various applications that are summarised in \cite{Giles_overview17} and successfully combined with other methods such as quasi-Monte Carlo methods. The main idea of MLMC is to approximate the payoff using different time stepping resolutions when numerically solving the underlying SDE and to generate an optimal number of samples on each level, such that the overall computational cost is minimised subject to the desired bound on the variance. %, such that the total computational cost is minimised. 
The computational savings come from the fact that most samples are computed on the coarser levels and hence are less expensive while only a few samples from the finest levels are required \cite{GILES2008}.


Among the directions in which the computational cost 
of MLMC methods could further be reduced, an important avenue is the use of lower precision calculations, especially for the first Monte Carlo levels where the targeted accuracy is relatively low. 
 An overview of the research on mixed precision for the standard Monte Carlo (MC) framework is provided in \cite{ChowMixedPrecisionStandardMC} but only a few references study the potential of low precision computation in the MLMC framework \cite{Rounding_error_oliver}. To the best of our knowledge, the only MLMC framework with customised precision in the literature is \cite{brugger2014mixed}, but they use a uniform precision for all operations on each Monte Carlo level instead of optimising 
 the precision of each intermediary variable to reduce as much as possible the cost of path generation.
 
An important motivation for an MLMC framework with variable precision would be performing the low precision computations on reconfigurable hardware devices such as Field Programmable Gate Arrays (FPGAs). FPGAs contain customizable logic blocks and connectors that make it easy to adapt the digital circuit architecture for a specific application, leading to a highly parallel and optimised implementation. Therefore they are successfully exploited in applications that require high speed and have high computational workload, such as signal processing \cite{woods2008fpga}, and real time applications like high frequency trading \cite{HFT1,HFT2}. That is why a number of previous works in hardware architecture design implemented the MLMC algorithm to price financial options using FPGAs as accelerators, which resulted in improved speed and power efficiency compared to full CPU architectures \cite{Schryver2013AMM}. The paper \cite{lindsey2016domain} also proposed 
a Domain Specific Language to automate the configuration of FPGAs for this specific application. However, only \cite{brugger2014mixed} proposed a heuristic to reduce the precision in calculations.

In addition, all aforementioned works considered that the random number generation (RNG) is performed in single or double precision. Yet in most cases an important portion of the workload in the overall MLMC simulation comes from the RNG and in \cite{brugger2014mixed} this limited the total computational savings.
To reduce the cost of MLMC simulations in particular those based on the Geometric Brownian Motion (GBM), \cite{approximateICDF_Oliver, NestedOliver} have proposed to use approximate random numbers that are generated by applying an approximation of the inverse CDF to uniform random numbers. In \cite{NestedOliver}, the authors proposed a way to integrate these lower precision random variables into a \textit{nested} MLMC framework and completed a numerical analysis to bound the resulting error at each MC level by a product of the time step and the error in the random number approximation. The same authors show in \cite{approximateICDF_Oliver} that using approximate random variables reduces the cost of path generation by a factor 7.


In this paper we propose a nested MLMC framework that combines the use of approximate random normal variables and lower precision calculations to reduce the computational cost of MLMC even further than \cite{brugger2014mixed,NestedOliver}. We illustrate the efficiency of our framework in Matlab, after making several assumptions on the cost of operations and size of the errors that we carefully justify. We focus on the case of GBM and use the approximate RNG methods presented in \cite{approximateICDF_Oliver} as well as a new slightly modified method that combines CDF inversion and the central limit theorem. To choose the precision of the variables in the low precision path generation, we introduce a novel method to optimise the bit-widths. This optimisation is performed before the main path generation loop is executed and is based on a linear model of the payoff error  
due to rounding when computing in low precision. The error model relies on algorithmic differentiation in a similar manner to \cite{unifying-bwoptim,bitwidth-AD,ADAPT}. The bit-width optimisation procedure can be performed off-line, so this stage can be excluded from the on-line time complexity of our framework. The user specified desired accuracy is then enforced by calculating on-line the number of samples that need to be generated.

In terms of hardware design, we suggest implementing the low precision path generation on FPGAs and the full-precision ones on a CPU or GPU. 
The FPGA offers enough flexibility to define a separate bit-width for every variable in the low precision path generation, and can be reconfigured periodically to update the bit-widths when the market parameters have changed considerably. 


The paper is organized as follows : \Cref{sec:MLMC} introduces MLMC and nested MLMC to make clear the estimator that is implemented in our framework. Then in \Cref{sec:RNG} we detail the methods that could be used to obtain approximate random normally distributed numbers very cheaply for the low precision path generation. In \Cref{sec:error_model} and \Cref{sec:costModel} we propose an error model and a cost model (resp.) that we then use to formulate the optimisation problem that is solved to obtain the optimal bit-widths of fixed point variables in \Cref{sec:optimisation}. Finally we summarise our results and future directions in \Cref{sec:conclusion}.



\section{Background}
\label{sec:background}


\subsection{Preliminaries}

{\color{red}[TODO: LLMs? in-context learning?]}

\subsection{Problem Definition}

{\color{red}[TODO: define the problem of citation intent]}

\section{Method}

\begin{figure*}[t]
    \centering
    \includegraphics[width=\linewidth]{figures/pipeline.png} \hfill

    \caption{An overview of our data synthesis pipeline. Starting from our seed data, we select a reference sample and collect \textsc{Reference-Level Feedback} on both the instruction and response. Instruction feedback is used to synthesize new instructions. We generate their corresponding responses, and then improve it using the response feedback.}
    \label{fig:pipeline}
\end{figure*}

In this section, we present our data synthesis pipeline that leverages \textsc{Reference-Level Feedback} to generate high-quality instruction-response pairs. An overview of the pipeline is presented in Figure \ref{fig:pipeline}, and the steps are detailed in the following subsections. Complete examples for each step can be found in Appendix \ref{sec:appendix_examples}, and the prompts used for each section can be found in Appendix \ref{sec:appendix_prompt_templates}.


\subsection{Feedback Collection}

Our pipeline begins with a seed dataset -- a small collection of carefully curated instruction-response pairs that serve as exemplars for synthesized data samples. It can be either manually crafted by human annotators or automatically selected using quality-based criteria. These reference samples are high-quality and exhibit desirable characteristics such as clarity and relevance, which we aim to replicate in our synthetic data. For \textsc{Reference-Level Feedback}, we systematically identify and capture such qualities through a framework that identifies the strength of each sample, as well as potential areas for improvement.

Unlike traditional approaches that collect feedback on generated responses at the sample-level, our method identifies the qualities that make reference samples high-quality and uses it for feedback. This feedback captures a richer signal than feedback collected at the sample-level, establishing higher quality standards for synthesis and providing more effective guidance for generating training data that exhibits similar properties to the reference samples.

For each reference sample in the seed dataset, we collect \textsc{Reference-Level Feedback} from both the instruction and the response:

\textbf{Instruction Feedback.} To collect feedback from a reference instruction and capture essential features that make it effective for training, we analyze key attributes (e.g., clarity and actionability). We also ensure comprehensive coverage along a wide breadth by collecting feedback along two dimensions: relevant subject areas (e.g. cellular biology, csv file manipulation, legislative processes) and relevant skills necessary to respond to the instruction (e.g. understanding of specific tools, knowledge of processes, analysis). This enables us to systematically identify desirable characteristics of instructions while maximizing the breadth of instruction types.

\textbf{Response Feedback.} When collecting feedback from a reference response, we identify key qualities that make it an effective response to the instruction. We evaluate along multiple critical dimensions, including factual accuracy, relevance to the instruction, and comprehensiveness. This feedback captures both the strengths of the reference response and specific areas where it can be improved upon.


\subsection{Data Synthesis}
Now, we use the collected \textsc{Reference-Level Feedback} from the previous stage to synthesize new data samples, while maintaining the quality standards established by our reference data. For each reference sample and its corresponding feedback, we employ a two-phase synthesis process, as illustrated in Figure \ref{fig:pipeline}:

\begin{enumerate}
    \item \textbf{Instruction Synthesis.} We provide an LLM the reference instruction as an example and the instruction feedback as guidelines to synthesize new instructions that maintain the qualities specified in the feedback. As depicted in Step 2 of Figure \ref{fig:pipeline}, we synthesize 10 new instructions for \textbf{subject-based} feedback, which produces instructions that align with the subject areas of the reference response. We also synthesize 10 new instructions for \textbf{skill-based} feedback, which produces instructions that align with the skills needed to respond to the reference instruction.
    
    \item \textbf{Response Synthesis and Refinement.} For each synthesized instruction, we first generate an initial response. We then enhance this response using the reference response feedback, instructing the language model to analyze the feedback and incorporate the relevant aspects. This process is shown in Step 3 of Figure \ref{fig:pipeline}.
    
    \paragraph{Note on relevance of response feedback.}
    Although the response feedback was originally collected for the reference response, many aspects of it can still remain applicable because of the shared characteristics between the reference and synthesized instructions. We acknowledge that not all feedback elements may transfer, and to account for this, we explicitly instruct the model to selectively apply only the relevant aspects of the feedback and ignore the irrelevant aspects. An example of this can be found in \ref{sec:appendix_examples}.
\end{enumerate}

This synthesis process enables us to synthesize new data, while systematically propagating the high-quality characteristics of reference samples.

\subsection{Theoretical Efficiency Analysis}
Our presented pipeline for data synthesis with \textsc{Reference-Level Feedback} is significantly more efficient than using traditional sample-level feedback methods, specifically in the frequency of feedback collection. While sample-level approaches require feedback for every synthesized sample, our method only requires feedback once for every reference sample. This translates to a reduction from $O(n)$ feedback collections, where $n$ represents the number of synthesized samples, to $O(1)$. However, it is also important to note that this efficiency gain comes with an initial fixed cost of collecting and curating seed data.
\section{Results}
\label{sec:results}
Following \nksr, we evaluate our method using metrics including the standard Chamfer-$L_1$ Distance~(CD-$L_1 \times 10^{-2}$, $\downarrow$) and F-score~($\uparrow$) with a threshold~($\delta{=}0.010$). 
We also report additional metrics proposed in \nksr~including Chamfer-$L_1$ Distance by Completeness (Comp.~$\times 10^{-2}$, $\downarrow$) and Accuracy (Acc.~$\times 10^{-2}$, $\downarrow$) in the \texttt{Supplementary Material}. 
We evaluate our method on multiple datasets, under two settings including in-domain evaluation for accuracy estimation -- training set and test set are from same dataset, and cross-domain evaluation for generalization ability estimation where training set and test set are from different datasets. 
Additionally, for cross-domain evaluation we use the following datasets prepared by the leading voxel-based baseline, \nksr, and one additional dataset from RangeUDF~\cite{wang2022rangeudf}:

\begin{itemize}
    \item \synthetic{}  is a synthetic dataset created from ShapeNet objects~\cite{chang2015shapenet}. Each scene contains 2-3 objects. 
    Following prior works~\cite{wang2022rangeudf,chibane2020ndf}, we re-scale the synthetic rooms to roughly match real-world scale.
    There are 3750 scenes as training set and \ws{995 scenes} as the test set. 
    \item \scannet{} is a real-world indoor scene dataset. We use the setting from previous work~\cite{wang2022rangeudf, tang2021SACon, peng2020convoccnet, boulch2022poco} where we train on 1201 rooms and test on 312 rooms. 
    \item \carla is a large-scale outdoor driving scene prepared by NKSR~\cite{huang2023neural} using the CARLA simulator~\cite{dosovitskiy2017carla}. 
    \ws{Following NSKR~\cite{huang2023neural}, we test on two subsets including the 'Original' subset (10 random drives simulated on 3 towns) and the 'Novel' subset (3 drives from an additional town only for testing).}
    To avoid exploding GPU memory during training, we follow NKSR~\cite{huang2023neural} to divide a large scene into patches. The resultant training set has {3757} patches. 
    \item \scenenn{}  is a real-world indoor dataset prepared by RangeUDF~\cite{wang2022rangeudf} which we used for cross-domain evaluation. We only use its test set which consists of 20 scenes.
\end{itemize}



\begin{table*}
\centering
\resizebox{\linewidth}{!}{
\setlength{\tabcolsep}{3pt}
\begin{tabular}{LccccccccccccC}
\toprule
Methods & & \multicolumn{3}{c}{\ws{{\bf \synthetic}}}  &  \multicolumn{3}{c}{{\bf \scannet}} & \multicolumn{3}{c}{\ws{{\bf \carla(Original)}}} & \multicolumn{3}{c}{\ws{{\bf \carla(Novel)}}} \\ 
 \cmidrule(lr){3-5} \cmidrule(lr){6-8} \cmidrule(lr){9-11} \cmidrule(lr){12-14} 
&Primitive& CD ($10^{-2}$) $\downarrow$ & F-Score  $\uparrow$ & Latency (s) $\downarrow$  & CD ($10^{-2}$) $\downarrow$ & F-Score  $\uparrow$ & Latency (s) $\downarrow$  & CD (cm) $\downarrow$ & F-Score  $\uparrow$ & Latency (s) $\downarrow$ & CD (cm) $\downarrow$ & F-Score  $\uparrow$ & Latency (s) $\downarrow$ \\        
\midrule
SA-CONet~\cite{tang2021SACon} & Voxels & {0.496} & {93.60} & - & - & - & - & - & - & - & - & - & -\\
ConvOcc~\cite{peng2020convoccnet} & Voxels & {0.420} & {96.40} & - & - & - & - & - & - & - & - & - & -\\
NDF~\cite{chibane2020ndf} & Voxels & {0.408} & {95.20} & - & 0.385  & 96.40  & -  & - & - & - & - & - & -\\
RangeUDF~\cite{wang2022rangeudf} & Voxels & {0.348} & {97.80} & {-} & 0.286 & 98.80 & - & - & - & - & - & - & -\\
\ws{TSDF-Fusion~\cite{zeng20163dmatch}} & -  & - & - & - & - & - & - & 8.1 & 80.2 & - & 7.6 & 80.7 & - \\
\ws{POCO~\cite{boulch2022poco}} & - & - & - & - & - & - & - & 7.0 & 90.1 & - & 12.0 & 92.4 & - \\
\ws{SPSR~\cite{kazhdan2013screened}} & - & - & - & - & - & - & - & 13.3 & 86.5 & - & 11.3 & 88.3 & - \\
\nksr & Voxels &  \underline{0.346} &  \underline{97.41} & \underline{0.40} & \underline{0.246} & \underline{99.51} & \underline{1.54} &  \underline{3.9} &  \underline{93.9} &  \underline{2.0} &  \underline{2.9} &  \underline{96.0} &  \underline{1.8} \\
\nksr (more data) & Voxels & - & - & - & - & - & - & {3.6} & {94.0} & {2.0} & {3.0} & {96.0} & {1.8}\\
Ours~(Minkowski)~\cite{choy20194d} \scriptsize{(w/ KNN)} & Voxels & - & \todo{} & \todo{} & 0.254 & 99.41 & 0.46 & 3.4 & 97.2 &1.9 & 2.7 & 98.1 & 2.0 \\
Ours~(Minkowski)~\cite{choy20194d} & Voxels & - & \todo{} & \todo{} & 0.301 & 98.48 & 0.31 & 3.8 & 96.2 & 1.5 & 3.0 & 97.4 & 1.5\\
\rowcolor{1st} Ours \scriptsize{(w/ KNN)} & Points &{0.321} & {98.34} & {0.13} & {0.243} & {99.61} & {0.48} &{3.2} & {97.5} & {3.2} &{2.6} & {98.3} & {3.4}\\
\rowcolor{1st}Ours & Points & {0.360} & {96.32} & 0.14 & 0.257 & 99.33 & 0.49 & {3.3} & {97.4} & 1.7 & {2.7} & {98.2} & 1.7 \\

\bottomrule
\end{tabular}
}
\caption{\textbf{In-domain evaluation} -- We show that our method achieves the best accuracy (CD and F-score) with significantly improved time efficiency~(inference latency).
Note we retrain \nksr (numbers are underlined) for fairer comparison, \ws{as the training data for \nksr is different from ours -- i.e., they reported some models trained on a ``mix'' of datasets, which is impossible to reproduce.
}
}
\label{tab:indomain}
\end{table*}


\paragraph{Evaluation pipeline}
To evaluate our method, we first extract the mesh with Dual Marching Cubes~\cite{schaefer2004dual} on the predicted SDF, and then compute the CD and F-score between 100k points sampled on the mesh, and 100k points sampled from the ground-truth dense point cloud.
We use the same approach as \nksr to prepare the input point clouds for training and evaluation from the ground-truth dense point clouds through downsampling.
Specifically, for indoor datasets (i.e., \synthetic, 
\scannet and \scenenn), we uniformly sample 10K points sampled from the ground truth dense point cloud. 
For outdoor driving scenes~(i.e., \carla), we follow the evaluation pipeline from \nksr.
We sample sparse input point clouds with a sparse 32-beam LiDAR with a ray distance noise of 0-5 cm and pose noise of $0-3^\circ$, and obtain the ground truth from a noise-free dense 256-beam LiDAR.

\begin{figure*}
\centering
\includegraphics[width=\linewidth]{visualizations/test_set_results.pdf}
\caption{
{\textbf{Qualitative results on \carla and \synthetic}} -- our method achieves high quality surface reconstructions which preserve more details than \nksr~which loses information due to quantization for large and non-uniformly sampled datasets like Carla.
}
\label{fig:qual_results_carla_syn}
\end{figure*}
 
\begin{figure*}
\centering
\vspace{-1em}
\includegraphics[width=.95\linewidth]{visualizations/scannet_results_0.pdf}
\caption{
Qualitative results on \scannet: We compare our method with prior SOTA~\cite{huang2023neural} and Ours~(Minkowski)~\cite{choy20194d} that is more comparable as it only differs from ours in the backbone. Our method achieves reconstruction of similar quality to the SOTA. It also \textit{significantly} outperforms Ours~(Minkowski), highlighting the importance of point-based methods. 
% \TODO{callouts too small? almost no zoom? why?}
}
\vspace{-1em}
\label{fig:scannet_results}
\end{figure*}
  

\paragraph{Implementation details}
We base our feature backbone on PointTransformerV3~\cite{wu2024point} with 4-levels.
The PointNet-style network is a 2-layered residual connection MLP, with hidden dimension of $32$ and output feature dimension of $32$.    
The grid size used in neighborhood function is $0.01$ meters.
Following \nksr, we use the similar coefficients for loss terms -- i.e., $\lambda_{\text{SDF}}$ is $300$ and $\lambda_{\text{mask}}$ is $150$.
However, we empirically set $\lambda_{\text{Eikonal}}$ to $10$~(\nksr does not need this regularizer thanks to its specialized surface solver).
We train our model with a batch size of $4$ on either a single \texttt{NVIDIA RTX A6000 ADA} or an \texttt{NVIDIA L40S}, and a learning rate of $10^{-3}$.
We adopt the Adam optimizer with default parameters.
We set the maximum number of epochs to 200 and employ a cosine learning rate decay starting from epoch 120.


\begin{table*}
\centering
\resizebox{\linewidth}{!}{
\setlength{\tabcolsep}{2pt}
\begin{tabular}{LccccccccccC}
\toprule
Methods & & \multicolumn{3}{c}{{\bf \synthetic $\rightarrow$ \scannet}}  &  \multicolumn{3}{c}{{{\bf \scannet $\rightarrow$ \synthetic}}} & \multicolumn{3}{c}{{{\bf \scannet $\rightarrow$ \scenenn}}} \\ 
 \cmidrule(lr){3-5} \cmidrule(lr){6-8} \cmidrule(lr){9-11}
&Primitive& CD ($10^{-2}$) $\downarrow$ & F-Score  $\uparrow$ & {Latency (s) $\downarrow$ } & CD ($10^{-2}$) $\downarrow$ & F-Score  $\uparrow$ & {Latency (s) $\downarrow$ } & CD ($10^{-2}$) $\downarrow$ & F-Score  $\uparrow$ & {Latency (s) }$\downarrow$ \\       
\midrule
SA-CONet~\cite{tang2021SACon} & Voxels & 0.845 & 77.80 & - & - & - & - & - & - & - \\
ConvOcc~\cite{peng2020convoccnet} & Voxels & 0.776 & 83.30  & - & - & - & - & - & - & - \\
NDF~\cite{chibane2020ndf} & Voxels & 0.452 & 96.00 & - & {0.568} & {88.10} & - & 0.425 & 94.80 & - \\
RangeUDF~\cite{wang2022rangeudf} & Voxels & {0.303} & {98.60} & {-} & 0.481& 91.50 & - & 0.324 & 97.80 & - \\
\nksr & Voxels & {0.329} & {97.37} & {2.02} & {0.351} & {97.41} & {0.46} & {0.268} & {99.18} & {1.95} \\
\rowcolor{1st} Ours (w/ KNN) & Points & {0.284} & {98.65} & {0.54} & {0.327} &{98.37} & {0.13} & {0.277} & {99.00} & {0.50} \\
\bottomrule
\end{tabular}
}
\caption{\textbf{Cross-domain evaluation} -- we achieve the best generalization ability in two cases with much better time efficiency. In the other case where we generalize from \scannet to \scenenn, we achieve accuracy on par with the SOTA baseline~\cite{huang2023neural} with less than a half of their latency.  
}
\vspace{-1.4em}
\label{tab:across_domain}
\end{table*}


\paragraph{Reconstruction latency}
For both our models and NKSR, we record the reconstruction latency for all indoor scenes on a single \texttt{NVIDIA RTX 3090}, and for large outdoor scenes on a single \texttt{NVIDIA L40s} given that more GPU memory is required.
We omit data loading time, and only record the average forward pass time. 

\subsection{In-domain evaluation}
We compare against \nksr~(the current state-of-the-art), RangeUDF~\cite{wang2022rangeudf},  SPSR~\cite{kazhdan2013screened}, NDF~\cite{chibane2020ndf}, ConvOcc~\cite{peng2020convoccnet} and SA-CONet~\cite{tang2021SACon}.     
We further include a baseline that replaces our backbone with MinkowskiNet~\cite{choy20194d} (i.e., Ours~(Minkowski)) to show the degraded performance due to the information loss caused by voxelization.

\paragraph{Quantitative results -- \Cref{tab:indomain}}
Across indoor and outdoor datasets, our method outperforms baselines in terms of accuracy and time efficiency. Especially in outdoor datasets, our method achieves the best surface reconstruction with the smallest latency -- nearly \textit{half} of the second best's latency.
In indoor datasets, which have relatively uniform sampling patterns, we achieve accuracy on par with the previous state-of-the-art, but with significantly improved time efficiency.
Note that we achieve this advantage even with KNN because, in smaller indoor point clouds, the highly engineered KNN implementation has similar time efficiency to that of our neighborhood function.
We further detail our analysis on this matter in the \texttt{Supplementary Material}. 
We also note that our approximate neighborhood function is still effective, as it outperforms the directly comparable baseline MinkowskiNet~\cite{choy20194d}, which shares the same structure except for the backbone and neighborhood function.


\paragraph{Qualitative results -- \Cref{fig:qual_results_carla_syn,fig:scannet_results}}
We show that our method tends to reconstruct surfaces of the best quality among the compared methods.
Especially, on the non-uniform large scale \carla, our method tends to preserve more details than the previous state-of-the-art~\cite{huang2023neural}, which voxelizes the point cloud.   

\subsection{Cross-domain evaluation -- \Cref{tab:across_domain}}
We further test the generalization ability of our method with a cross-domain evaluation.
We evaluate models trained with dataset A on other a different dataset B; we denote this as~A $\rightarrow$ B. 
As shown in \Cref{tab:across_domain}, there are three cases in total.
In two cases (i.e., \synthetic $\rightarrow$ \scannet and \scannet $\rightarrow$ \synthetic), our method achieves the best accuracy with the best time efficiency. 
In another case (\scannet $\rightarrow$ \scenenn), we achieve accuracy on par with SOTA~\cite{huang2023neural} with a much better time efficiency, i.e., less than a half of the latency required by the SOTA~\cite{huang2023neural}.

\subsection{Ablation studies}
Our ablations are executed on \scannet, as it is a real-world dataset, and is equipped with precise ground truth surface meshes.

\begin{table}
\centering
\resizebox{.9\columnwidth}{!}{
\begin{tabular}{LccccccC}
\toprule
{\bf Neighbor Num.} & {CD (10\textsuperscript{-2})} $\downarrow$ & {F-score} $\uparrow$ & Latency (s) $\downarrow$ \\ \midrule
 2 & 0.246 & 99.56 & 109 \\
 4 & 0.244 & 99.59 & 127 \\
 \rowcolor{1st} 
8 & {0.243} & 99.61 & 151 \\
16 & 0.256 & 99.28 & 187 \\
\bottomrule
\end{tabular}
}
\caption{{\bf The impact of neighborhood size} -- larger neighborhoods lead to increased computational cost, and we find that 8 neighbors gives the best balance of cost and quality.}
\label{tab:numpts_neighbor}
\vspace{-1em}
\end{table}

\paragraph{Impact of neighborhood size -- \Cref{tab:numpts_neighbor}}
We analyze the impact of neighborhood size on performance. Larger neighborhood size leads to increased computation overhead. 
We show that the 8-nearest neighboring points gives the best trade-off between accuracy and time efficiency.
Considering a large number (e.g., 16) of neighboring points degrades performance as the the aggregation module has limited capacity to predict the precise SDF from a large local point cloud.

\begin{table}
\centering
\resizebox{.95\columnwidth}{!}{
\begin{tabular}{@{}lcccccc@{}}
\toprule
\makecell{\bf Num. of hidden\\\bf layers in $\aggregation$} & CD (10\textsuperscript{-2}) $\downarrow$ & F-score $\uparrow$ & Latency (s) $\downarrow$ \\ \midrule
 2 & 0.257 & 99.33 & 152 \\
 4 & 0.256 & 99.32 & 166 \\
\bottomrule
\end{tabular}
}
\caption{{\bf Impact of capacity of $\aggregation$} -- we find that increasing the number of layers in $\aggregation$ beyond 2 decreases time efficiency without substantially improving the reconstruction quality.}
\label{tab:agg_capacity}
\vspace{-1em}
\end{table}

\paragraph{Impact of capacity of $\aggregation$ -- \Cref{tab:agg_capacity}} 
We report how the capacity of the aggregation module $\aggregation$ (i.e., different number of hidden layers) impacts the performance.
We observe that aggregation modules of higher capacity give better performance but degraded time efficiency. However, as shown in~\Cref{tab:agg_capacity}, a very large capacity (4 layers) for $\aggregation$ does not help.
We show that we we use 2 layers to have a good trade-off between accuracy and time efficiency. 
We supplement~\Cref{tab:agg_capacity} with an analysis across even more levels in the \texttt{Supplementary Material}.

\begin{table}
\centering
\resizebox{.9\columnwidth}{!}{
\begin{tabular}{@{}lcccc@{}}
\toprule
\textbf{Num. of scales} &KNN & Minkowski & Z-order & Hilbert  \\ \midrule
0 & 1.00 & 0.17 & 0.44  & \cellcolor{1st}0.46  \\
1 & 1.00 & 0.29 & 0.48  & \cellcolor{1st}0.50  \\
2 & 1.00 & 0.38 & 0.49  & \cellcolor{1st}0.52  \\
3 & 1.00 & 0.44 & 0.49  & \cellcolor{1st}0.53  \\ %
\bottomrule
\end{tabular}
}
\caption{\textbf{Recall rate of our Hilbert-curve based $\neighbor$} -- we find that the Hilbert curve consistently outperforms both the Z-order curve~\cite{morton1966computer} and the one-ring neighborhood from Minkowski relative to the exact k-nearest neighbors.
}
\vspace{-1em}
\label{tab:locality_neighbor}
\end{table}

\paragraph{Analysis of neighbors retrieved by~$\neighbor$ -- \Cref{tab:locality_neighbor}}
\at{We now investigate the quality of the point neighborhoods retrieved by various possible implementations for $\neighbor$.
In particular, we are interested to experimentally study whether our serialization indeed preserves locality.
To quantify this, we treat the neighborhood retrieved with KNN as the ground-truth.}
We report the recall rate of a local neighborhood by comparing it with this ground truth~(we ignore the precision rate because we remove false positives with a distance threshold).
We also report the recall rate of the one-ring neighborhood retrieved in Minkowski~\cite{choy20194d}.
We show that the recall rate of our Hilbert $\neighbor$ is the best across variants, and across all scales.

\begin{table}[t]
\centering
\resizebox{\columnwidth}{!}{
\begin{tabular}{L rr rR}
\toprule
Methods & \multicolumn{2}{c}{Uniform} & \multicolumn{2}{c}{Non-Uniform}   \\ 
\cmidrule(r){1-1}
\cmidrule(lr){2-3}
\cmidrule(l){4-5}
\nksr & 0.246 & 480s & 0.273 & 668s  \\
Ours~(Minkowski)~\cite{choy20194d}  & 0.301 & 97s & 0.349 & 94s \\
Ours~(Minkowski)~\cite{choy20194d} {(w/ KNN)} & 0.254 & 145s & 0.294 & 155s \\
\rowcolor{1st} Ours~(w/ serialization) & {0.257} & {152s} & {0.296} & {145s} \\
\rowcolor{1st} Ours~(w/ KNN) & \textbf{0.243} & \textbf{151s} & \textbf{0.273} & \textbf{142s}  \\
\bottomrule
\end{tabular}
}
\caption{
\textbf{The impact of sampling} -- we evaluate uniform vs non-uniform sampling on ScanNet. We find that our method achieves the best accuracy (in terms of CD ($10^{-2}$)) and good time efficiency compared to \nksr~for both sampling types.
}
\vspace{-1em}
\label{tab:nonuniform_scannet}
\end{table}

\paragraph{The impact of sampling pattern --~\Cref{tab:nonuniform_scannet}} 
We report the impact of sampling pattern on performance by evaluating models on ScanNet point clouds that are uniformly or non-uniformly sampled. 
{To non-uniformly sample the ScanNet point clouds, we first partitioned the scene into eight blocks and randomly sampled a different number of points from each block. The number of samples followed an arithmetic sequence with a common difference of 200. Finally, we padded the last block to ensure that the total number of points remained 10K.}
 
We show that our method achieves better robustness to non-uniform sampling than the baselines, highlighting the importance of avoiding quantization of the point cloud for high quality surface reconstruction. 


\section{Discussion}
\label{section:discussion}


\subsection{Practical Implications for Feedforward Prompting}

Of course, prompting an LLM continuously before the user submits their prompt is significantly most costly over submitting the prompt just once, once the user is ready.

% But user might not be ready, and the cognitive costs is pretty heavy.


\subsection{}


% Does this work well with Chain of Thought actually?
% Maybe this approach will actually incentivize self-prompt-chaining???
% What are the implications of this?


% A benefit of this is certainly more transparency in the LLM
% LLM is so flexible that adding this kind of structure is still okay for the LLM



% What's more costly, entering a prompt, then responding and saying, no i want this, or typing a prompt, and tuning the prompt/expected output to reduce message exchanges?

% Learning to become a better prompter. One is by trial and error experience. Perhaps another is through this feedforward that tells you what you might be able to anticipate.
\section{Concluding Remarks}
In this paper, we proposed a novel approach utilizing multimodal LLMs to generate gesture-aware speech recognition transcripts for patients with language disorders. Our framework integrates verbal speech and iconic gestures, enabling the generation of enriched transcripts that capture the latent meaning conveyed through both modalities. Through extensive experimentation, we demonstrated that the proposed method effectively contextualizes incomplete or disfluent speech by incorporating gesture information, leading to more accurate and meaningful representations of the speaker's intent. These findings highlight the potential of our approach to significantly contribute to the field of speech and language therapy, offering innovative tools that can enhance the quality of life for individuals with language disorders by facilitating better communication and assessment methods.

\subsection{Ethical Statement} 
Our dataset was obtained from AphasiaBank with the approval of the Institutional Review Board (IRB) and adheres to the data sharing guidelines set by TalkBank\footnote{https://talkbank.org/share/ethics.html}. This includes complying with the Ground Rules for all TalkBank databases, which are based on the American Psychological Association Code of Ethics~\cite{american2002ethical}.

\subsection{Limitation \& Future Work} 
%This study represents a preliminary investigation into using multimodal LLMs to generate gesture-aware speech recognition transcripts. 
While the results are promising, we recognize several limitations and outline our plans to extend this work further.

One primary limitation is the absence of a definitive ground truth for quantitative evaluation. Since our model generates transcripts by synthesizing speech and gesture data from scratch, traditional benchmarks, such as comparisons with standard speech recognition outputs, are insufficient. Moreover, existing original transcripts lack gesture annotations, making direct comparisons challenging. In future work, we aim to address this gap by collaborating with certified pathologists to conduct qualitative assessments, such as A-B preference tests, to evaluate the effectiveness of gesture-enriched transcripts in accurately conveying the speaker's intentions.

To support quantitative evaluations, we plan to develop novel metrics that assess transcript quality, including grammar accuracy, semantic consistency, and the integration of multimodal information. Such metrics will provide a more objective basis for assessing our model's performance and facilitate comparisons with other multimodal and unimodal approaches.

Another limitation of this study is its focus on structured gestures from a specific task, the Peanut Butter Sandwich Task. While this task offers a controlled context for testing our approach, it does not encompass the diversity of gestures and communication patterns seen in everyday scenarios. As part of our future work, we plan to expand the scope of our model to include tasks such as the Cinderella Story Recall Task~\cite{bird1996cinderella}, which involves unstructured and complex narrative gestures. This expansion will allow us to evaluate the adaptability and robustness of our model in handling varied linguistic and gestural contexts.

In summary, while this study establishes a strong foundation for gesture-aware speech recognition, we aim to refine and extend our methods through collaborative qualitative evaluations, the development of robust quantitative metrics, and broader task applications. These efforts will ensure that our approach continues to evolve, ultimately contributing to more effective communication tools and interventions for individuals with language disorders.




\section*{Acknowledgment}

This work was partially supported by a fellowship of the German Academic Exchange Service (DAAD).
Niclas Vödisch acknowledges travel support from the European Union’s Horizon 2020 research and innovation program under ELISE grant agreement No.~951847 and from the ELSA Mobility Program within the project European Lighthouse On Safe And Secure AI (ELSA) under the grant agreement No.~101070617.


% \bibliographystyle{alpha}
\bibliographystyle{ieeetr}
\bibliography{bibliography}

\end{document}