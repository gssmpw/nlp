 \section{Background}
\label{2_background}

% This section discusses the background and related work, beginning with an overview of security policies in GitHub and a brief explanation of the OpenSSF Scorecard for evaluating the quality of security practices in open-source projects. It then reviews related work on security policies, security issues in open-source projects, and studies that assess security practices using the OpenSSF Scorecard.

% \morakot{discuss Github security policy guidline, security.md pattern} \bee{done}

% \morakot{discuss OpenSSF, justify why we use OpenSSF} \bee{done}

This section covers background and related work, starting with an overview of GitHub security policies and the OpenSSF Scorecard for evaluating security practices in open-source projects.

\subsection{Security Policies in GitHub Repositories}
GitHub provides a feature for creating a security policy within a repository through a dedicated \texttt{SECURITY.md} file. This file guides software practitioners, such as contributors and users, on handling security concerns, including the appropriate channels for reporting vulnerabilities in a project. Maintainers can add the security policy to the repository’s root, \texttt{doc}, or \texttt{.github} folder. When creating a security policy using GitHub’s security feature, a guideline suggests including key details to improve its usefulness.\footnote{\url{https://docs.github.com/en/code-security/getting-started/adding-a-security-policy-to-your-repository}} These details include the \textit{Supported Versions} section, listing the project versions that receive security updates, and the \textit{Reporting a Vulnerability} section, which provides instructions for reporting vulnerabilities. This section can also specify communication channels for updates on reported vulnerabilities, outline steps taken upon vulnerability acceptance or rejection, and include additional information about the repository’s security practices. 

GitHub also provides features that enable users to report vulnerabilities, through the GitHub Advisory (GHSA) \cite{GitHubAdvisoryDatabase:online}. GitHub Advisory is a database that publishes information on security vulnerabilities that impact open-source projects. It includes security-related details, such as severity, affected versions, and recommended fixes.


% \morakot{add few sentences explain Github advisory} \bee{done}


% \subsection{Security policies in GitHub repositories}
% GitHub provides the features for creating the security policy in their repository through a dedicated \texttt{SECURITY.md} file for informing users about how to report security vulnerabilities within the project. Software practioner (e.g., maintainers) can add the security policy to the repository's root, \texttt{doc}, or \texttt{.github} folder. When creating a security policy using the GitHub security feature, it has a guidline to include key details to make the policy practical. The key details include the \textit{Supported Versions} section, which lists the project versions receiving security updates. \textit{Reporting a Vulnerability} section, which provides information for developers on how to report vulnerabilities. This section can include the channel for receiving updates on a reported vulnerability, the steps involved after accepting or rejecting the vulnerability, or additional details about the security repository.

\subsection{OpenSSF Scorecard}
The OpenSSF Scorecard is an automated tool developed by the Open Source Security Foundation (OpenSSF) to assess security practices in open-source software repositories \cite{OpenSSF:online}. The Scorecard provides a security practices score for each project, based on 18 criteria that collectively reflect the robustness of security practices in maintaining a project repository. These scores help software practitioners identify and address potential security risks in their projects, offering insights to improve overall security practices. 

% Key criteria include Binary Artifacts, Branch Protection, CII Best Practices, Contributors, Dependency Update Tool, Fuzzing, License, Maintained, SAST, and Vulnerabilities.

Each metric in the OpenSSF Scorecard is classified into one of four risk categories: Critical, High, Medium, and Low, based on its potential security implications. Metrics are scored from 0 to 10, reflecting how closely the project aligns with security best practices. If Scorecard cannot verify a particular practice, it assigns a score of -1, which indicates inconclusive evidence or a possible runtime error. This designation ensures that such cases do not negatively impact the overall project score. For example, the Scorecard evaluates metrics like Branch Protection, which checks if protective measures are in place for critical branches, and Dependency Update Tool, assessing whether tools are used to keep dependencies up-to-date, which helps reduce vulnerability risks. Another metric, License, confirms whether the project has a well-defined, open-source license, contributing to transparency and legal security. These checks help developers understand security gaps and guide improvements.

% The goal of the tool is to assist maintainers and users in evaluating the security prectices of the projects.
% The OpenSSF Scorecard automates this process by evaluating open-source projects across several predefined security practices. 
% \chaiyong{This may be too many metrics. We can just present a few related to our studies to save some space.} \bee{done-only 10 metrics (from 18) were presented}

% Each metric is classified into one of four risk categories; Critical, High, Medium, and Low. It is identified based on the potential security implications. The metric within the OpenSSF Scorecard is rated between 0 and 10, depending on the degree to which the project follows security best practices. It also assigned the notation -1 indicating that Scorecard could not get conclusive evidence of implementing practices, or perhaps an internal error occurred due to a runtime error in Scorecard to avoid the effect on the overall project's score.


% \morakot{move related work here. there are two group of related work that we need mention here 1. a group that study security.md and 2. a group that use Openssf}\chaiyong{Agree. I think it reads better if the related work comes after OpenSSF. I also think there should be some background about GitHub reporting mechanism or security policy.} \bee{done-background and related works were all fixed. added background about security policy, related works for security policy and OpenSSF}

% \subsection{Related works}




% The related work in this study focuses on security policies in open-source projects on GitHub. Ayala et al. \cite{Ayala:SVM2023} analyzed the popular GitHub repositories to evaluate the adoption of security policies and workflows. The study shows only a small portion of packages implemented the security policy, and most still lack structured security measures. This underscores the significance of setting up workflows and security policies of open-source projects. The GitHub security policy feature aims to provide instructions on how to report the vulnerability. It is also important to access security issues as a feature for bug reporting and project discussions. Bühlmann et al. \cite{Noah:SAC2022} conducted an examination of the characteristics of security issues and the methods used by developers to manage these reports. The result showed that security issues comprise only a small proportion of all issues. The number of security issues is increasing over time, with more discussion time than non-security issues.

% Zahan et al. \cite{Zahan:IEEE} conducted an experiment on security practices using OpenSSF Scorecard to identify gaps of security practices in npm and PyPI ecosystems. This study showed gaps in Code Review, Maintenance, Licensing, Branch Protection, and Security Policy practices in the GitHub repository. The paper also identifies practices that can improve the security of the projects. The identified practices include Security Policy, Pinned Dependencies, Maintained, Code Review and Branch Protection. As also shown in the study in a relationship between security practices predictors and the number of the vulnerability \cite{Zahan:ICSE2023}.


\begin{figure*}[ht!]
    \centering
    \includegraphics[width=0.85\linewidth]{figure/workflow.jpeg}
    \caption{Our research methodology}
    \label{fig:workflow}
% \vspace{-.3cm}
\end{figure*}





% However, there is no research looking into the content of the security policies. Our study aims to focus on the content of security policies in order to propose guidelines to improve them. We emphasize the importance of the security policy by identifying the practices that would get the better OpenSSF Scorecard score if the policies were established.

% \chaiyong{We should at least compare and contrast the related studies to ours and show the gap that they did not explore.} \bee{done-described in the paragraph above}