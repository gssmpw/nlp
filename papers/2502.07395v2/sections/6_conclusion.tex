\section{Challengs and Future Outlook}
\label{6_conclusion}

Open-source projects play a vital role in software development, but sharing vulnerabilities before fixes are available poses a significant risk of exploitation. Consequently, open-source projects often establish their own security policies to address these concerns. Our study examined security reporting mechanisms in PyPi packages on GitHub, focusing on the alignment between security policies and actual reporting practices. Interestingly, email remains the primary means of reporting security issues. Our analysis reveals frequent non-compliance in reporting security issues, suggesting that current reporting methods may not be effective for all contributors.
Effective security policies must cover key practices such as secure coding, vulnerability management, data handling, and access control. A crucial takeaway is that email communication requires robust security measures to protect sensitive information. The OpenSSF Scorecard analysis also shows that projects with well-defined security policies tend to adhere more closely to key practices, underscoring the positive impact of effective security policies on project security. At this stage, we observe various methods of reporting vulnerabilities, which warrants further investigation into the reasons behind these choices and their pros and cons.

 % While most maintainers encourage private vulnerability reporting, many external contributors publicly disclose issues, highlighting the need for improved communication and awareness. The OpenSSF Scorecard analysis shows that projects with well-defined security policies tend to adhere more closely to key practices, underscoring the positive impact of effective security policies on project security.

% Future studies can explore the effectiveness of different reporting mechanisms, including the speed and responsiveness of mechanisms for immediate communication. This would involve a developer study to gather feedback on what reporting mechanisms are most effective and how intrusive they are to software development workflows.

% Future research could expand this study by identifying best practices for security policies across diverse open-source ecosystems. This would involve tailoring security policy elements to varying project characteristics, such as project size, complexity, and contributor base, to provide actionable recommendations for software practitioners.

% In the era of AI, further research is needed to explore automated methods for detecting non-compliance with security policies and investigate communication strategies to enhance adherence among external contributors. Such agents would be critical in quickly raising awareness and mitigating potential threats while identifying false positives.
% Moreover, future work could focus on understanding the communities of open-source projects and their practices and compliance that they enforce within their communities. This includes examining software library ecosystems, where regulations in an open-source context remain unclear.
% Ultimately, these challenges and future directions offer new opportunities to keep the open-source community secure from security threats with rapid mitigation.

Future work could focus on understanding the communities of open-source projects and their practices and compliance that they enforce within their communities. This includes examining software library ecosystems, where regulations in an open-source context remain unclear. Examining security policy and identifying best practices for security policies across diverse open-source ecosystems is also needed. This would involve tailoring security policy elements to varying project characteristics, such as project size, complexity, and contributor base, to provide actionable recommendations for software practitioners. 

Moreover, in the era of AI, further research is needed to explore automated methods for detecting non-compliance with security policies and investigate communication strategies to enhance adherence among external contributors, such as tools for preventing vulnerabilities from being discussed publicly. Such agents would be critical in quickly raising awareness and mitigating potential threats while identifying false positives. Ultimately, these challenges and future directions offer new opportunities to keep the open-source community secure from security threats with rapid mitigation.

% Further research could also explore automated methods for detecting non-compliance with security policies and investigate communication strategies to enhance adherence among external contributors. These efforts would help software practitioners define robust, adaptable security policies suited to their unique project needs, ultimately contributing to stronger security practices across the open-source ecosystem.

% In this study, we looked into the process for reporting vulnerabilities, which are defined as reporting mechanisms in the security policy of the PyPi packages. We observed the developers' adherence to the policy by investigating the security issues. The study shows that most PyPI packages on GitHub provide reporting mechanisms for reporting vulnerabilities via personal email, and the security policy also includes external links. Most of the maintainers of PyPI packages understand the importance of privately reporting vulnerabilities and have established guidelines for reporting them through private channels. However, for the projects that have security issues, despite the the policy were created, external contributors still continued to submit security issues. This indicates the limits of the broader developer’s awareness of vulnerabilities being exploited when publicly disclosing the vulnerabilities without fixes. We also access the security practices of the projects using the OpenSSF Scorecard to identify the practical practices that would be improved if the projects had established the security policy. Dependency-Update-Tool and Maintained the project with an established security policy demonstrate a significant difference from the project without a policy. The findings highlight the importance of a security policy, as they show that projects with established security policies implement security practices more than projects without policies.

% \textbf{Future Work:}
% In future work, we plan to collect more data for PyPi packages. This study observed only the projects that have a security report on GitHub advisory to ensure the projects are active in addressing the vulnerability. However, some projects may not have reported on GitHub advisory, but they may have other security practices and reporting mechanisms that were not included in this study. The study highlights the enhancements in security practices that occur when projects implement security policies. Since the goal of this study is to provide guidance for improving the security policies of open-source projects, it is also important to investigate the further relationship between the security policy and security practices in order to find the practices that should be included in the policy.