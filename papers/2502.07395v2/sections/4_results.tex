\section{Results}
\label{4_results}

This section provides a detailed account of the results for each research question, along with a discussion that emphasizes the significance of our findings.


% In this section, we present the investigation for each research question, which includes the approach and the result from our analysis.

% \morakot{consider to add example of issues to support the finding} \bee{example issue is already be described in intro}

\subsection{RQ1: What are the reporting mechanisms in security policies?}

% For this research question, we investigate the ``Reporting Mechanism'' of the packages with a security policy established. 
% To avoid potential exploitation, we should address the vulnerabilities before revealing them to the public. Therefore, it is advisable to refrain from publicly reporting vulnerabilities or security-related issues. Given the significance of reporting mechanisms, we conducted an investigation into the mechanisms outlined in security policies. 
% Our study encompasses the content of 302 security policies. We manually classified the reporting mechanisms defined in the content of security policies to understand the process of reporting vulnerabilities in PyPi packages.

% \chaiyong{I think the Venn diagram might work better here.} \bee{done-Venn diagram added}


Figure \ref{fig:rq1} shows a Venn diagram of the reporting mechanisms defined in security policies. Our analysis of 302 GitHub repositories from PyPi shows that most repositories use email (41.06\%) as the preferred reporting method, followed by external links to other platforms (21.52\%) and the GitHub Advisories feature (12.91\%). Notably, 62 repositories (20.53\%) provide more than one reporting mechanism. Only 0.99\% use GitHub Issues, and 2.98\% do not specify any reporting method. We categorize email, external links, or GitHub advisories as private channels for reporting security vulnerabilities in repositories. In answering RQ1, we found that most security policies define the private channels as reporting mechanisms, as the results show that 285 repositories (94.37\%) define these channels in the security policies. This suggests that most project maintainers are aware of the risks of publicly disclosed security vulnerabilities and take steps to mitigate them.

% \begin{tcolorbox}[colframe=black!75, colback=white]
\textbf{RQ1 Summary:} 
% Given that most security policy reporting mechanisms are private communication channels, our findings suggest that most project maintainers are aware of publicly disclosed security vulnerabilities.
Our findings indicate that most project maintainers are aware of publicly disclosed security, since most security policy reporting mechanisms are private communication channels.
% \end{tcolorbox}

% \begin{tcolorbox}[colback=gray!5,colframe=black,title= RQ1 Summary]
% Given that most security policy reporting mechanisms are private communication channels, our findings suggest that most project maintainers are aware of publicly disclosed security vulnerabilities.
% \end{tcolorbox}

% Figure \ref{fig:rq1} shows the vein diagram of the reporting mechanism defined in the security policy. Our study shows that among 302 GitHub repositories from PyPi mostly use email (45.69\%) as a reporting method. Followed by providing the external links to access another platform (20.86\%), and submitting the vulnerabilities using the GitHub Advisories feature (14.24\%). There are 50 repositories that provide more than 1 reporting mechanism (16.56\%). Only 0.99\% use GitHub issue and 1.66\% do not define reporing method. To answer RQ1, we found that most security policies define email, external links, or GitHub advisories as reporting mechanisms. All these practices are classified as private channels. This indicated that most project maintainers are aware of publicly disclosed security vulnerabilities. To answer RQ1, we found that most security policies define email, external links, or GitHub advisories as reporting mechanisms. All these practices are classified as private channels. This indicated that most project maintainers are aware of publicly disclosed security vulnerabilities.

% Followed by providing the external links to access another platform (20.86\%), and submitting the vulnerabilities using the GitHub Advisories feature (14.24\%). Some packages provided more than 1 reporting mechanism, such as allowing developers to send an email or submit the GitHub Advisory to the project. Some packages either did not define a process for reporting vulnerabilities in their security policies, or they provided reporting mechanism as a GitHub issue. However, these only present a small portion of the study, for 1.66\% and 0.99\% respectively.

% \chaiyong{To answer RQ1, we found that ...} \bee{done}


\begin{figure}[!]
    \centering
    \includegraphics[width=0.95\linewidth]{figure/rq1.jpeg}
    \caption{The Reporting Mechanism Defined In Security Policy}
    \label{fig:rq1}
    \vspace*{-.5cm}
\end{figure}

% \begin{table}[h!]
% \centering
% \caption{Statistics of the reporting mechanism defined in security policy}
% \label{tab:reporting_mechanism_statistic}
% \begin{tabular}{@{}lrr@{}}
% \toprule
% \textbf{Reporting mechanism} & \textbf{\# Packages} & \textbf{Percentage} \\ \midrule
% Email & 138 & 45.69 \\
% External link & 63 & 20.86 \\
% GitHub Advisory & 43 & 14.24 \\
% Email and GitHub Advisory & 24 & 7.95 \\
% Email and External link & 18 & 5.96 \\
% Email and GitHub issue & 7 & 2.32 \\
% Not mentioned & 5 & 1.66 \\
% GitHub issue & 3 & 0.99 \\
% Email, GitHub issue, and External link & 1 & 0.33 \\
% \bottomrule
% \end{tabular}
% \end{table}

\subsection{RQ2: Do the developer's practices align with the security policy?}

We investigate the alignment between reporting mechanisms defined in security policies and actual practices. In repositories specifying Email, External Link, or GitHub Advisory as reporting channels, we found a significant decrease in security-related issues reported through GitHub Issues, suggesting that practitioners are generally adhering to the defined reporting mechanisms. However, 787 security-related issues were still reported through GitHub Issues across 58 repositories after private reporting methods were established, indicating non-compliance with the policies. 

Table \ref{tab:issue_reporting_mechaism_submitter} shows that of 787 non-compliance issues, these are primarily occurring in repositories using Email as the reporting method. 
% This may suggest that Email may not be the most convenient channel for reporting security-related issues. 
We also investigated the roles of submitters of these security issues identified by the \texttt{author\_association} attribute of the analyzed repositories. This attribute includes four categories: \textit{OWNER}, \textit{CONTRIBUTOR}, and \textit{MEMBER}, while \textit{NONE} represents external contributors with no formal association to the project. 
% The results show that the majority of non-compliant security-related issues are created by external contributors, who may be unaware of or disregard the policy. 
The results indicate that most developers follow the defined reporting mechanism, as the results show a small number of security issues do not comply with the security policy. However, for these noncompliant security issues, the majority of them were created by external contributors (i.e., the \texttt{author\_association} is \textit{NONE}), who may either be unaware of the policy or intentionally disregard it.

% , which indicates the submitter's relationship to the project.

% After the declaration of reporint mechanism in security policy, we can identified 787 security-related issues from 58 repositories.

% For the GitHub issues of open source projects, all GitHub users can access, create, or submit issues, and the information about these issues will be shown to the public. In case of security issues, this practice gives more chances for the package vulnerabilities to be seen by attackers and increases in chance for the vulnerability to be exploited. To find whether developers follow the security policy or not. 
% This research question aims to investigate the consistency of the reporting mechanism defined in the security policies and the security issues. We analyzed GitHub discussion issues labeled with security keywords with the security issues of the repositories that have the security policy. Consequently, the analysis includes 1,567 security issues created after their security policy establishment across 63 repositories. 

% We excluded the security issues that were created before the security policy creation to ensure that there was no conflict between the issues and the instructions before and after the issues were created. After excluding these issues, there still have security issues created after the security policy establishment including 787 issues across 58 repositories.

% After the declaration of reporint mechanism in security policy, we can identified 787 security-related issues from 58 repositories.

% Table \ref{tab:issue_reporting_mechaism_submitter} shows the statistics of the collected security issues of PyPi packages. The findings show that there are 776 security issues that did not follow the security policy (i.e., security issues were created even though the policy provided the instructions to report the vulnerabilities through another channel). 

% We also investigated the submitters of these security issues, by analyzing the \texttt{author\_association} of each security issue. The attribute \texttt{author\_association} of each issue presents the association of the issue's creator with the project, which is represented in four categories. ``\textit{OWNER}", ``\textit{CONTRIBUTOR}", and ``\textit{MEMBER}", represent the project maintainer of the project. "``\textit{NONE}" represents an external contributor of the project (i.e., contributors have no association role with the projects). The \textit{author\_association} of the security issue's creators for the security issues created after policy establishment is also presented. The result shows that most security issues were created by users with ``\textit{NONE}" association (349 issues), followed by ``\textit{MEMBER}" (202 issues) and ``\textit{CONTRIBUTOR}" (182 issues). This indicates that most developers have an awareness about disclosing vulnerabilities publicly, as only a small number of security issues were created across a small proportion of packages. However, for the packages with the security issues, despite defining a security policy to report vulnerabilities privately, security issues still be created. Developers who are not affiliated with the packages (i.e., contributors outside the packages) continue to actively create security issues.

% \chaiyong{What is this? Is it an attribute in each issue?} \bee{done-description added}

% \begin{table}[h!]
% \centering
% \caption{Security-related issues created after the establishment of security policies}
% \label{tab:number_issues_statistic}
% \begin{tabular}{@{}lrr@{}}
% \toprule
% \textbf{Reporting mechanism} & \textbf{\# Issue} & \textbf{\# Pull requests} \\ \midrule
% Email & 476 & 335 \\
% GitHub Advisory & 167 & 41 \\
% Email and External link & 73 & 41 \\
% External link & 54 & 48 \\
% Email and GitHub Advisory & 6 & 19 \\
% Not mentioned & 11 & 0 \\
% Email and GitHub Issue & 0 & 2 \\
% \bottomrule
% \end{tabular}
% \end{table}

% \begin{table}[h!]
% \centering
% \caption{Security-related issues created after the establishment of security policies}
% \label{tab:number_issues_statistic}
% \begin{tabular}{@{}lrr@{}}
% \toprule
% \textbf{Reporting mechanism} & \textbf{\# Issue} \\ \midrule
% Email & 476 \\
% GitHub Advisory & 167 \\
% Email and External link & 73 \\
% External link & 54 \\
% Email and GitHub Advisory & 6 \\
% Not mentioned & 11 \\
% \bottomrule
% \end{tabular}
% \end{table}

% \begin{table}[h!]
% \centering
% \caption{Issue submitters association to the project}
% \label{tab:issue_submitter_association}
% \begin{tabular}{@{}lrr@{}}
% \toprule
% \textbf{Submitter association to the project} & \textbf{\# Issues} & \textbf{\# Pull Requests} \\ \midrule
% NONE & 349 & 30 \\
% MEMBER & 202 & 257 \\
% CONTRIBUTOR & 182 & 170 \\
% COLLABORATOR & 53 & 29 \\
% OWNER & 1 & 0 \\ \bottomrule
% \end{tabular}
% \end{table}


% \begin{table}[h!]
% \centering
% \caption{Non-compliance in security-related issues based on reporting mechanisms and reporter roles}
% \label{tab:issue_reporting_mechaism_submitter}
% \begin{tabular}{@{}lrlr@{}}
% \toprule
% \textbf{Reporting mechanism} & \textbf{\# Issue} & \textbf{Submitter association} & \textbf{\# Issue} \\ \midrule
% Email & 311 & NONE & 349 \\
% External link & 207 & MEMBER & 202 \\
% GitHub Advisory & 166 & CONTRIBUTOR & 182 \\
% Email and External link & 76 & COLLABORATOR & 53 \\
% Not mentioned & 14 & OWNER & 1 \\
% Email and GitHub Advisory & 6 &  & \\
% Email and GitHub issue & 6 & & \\ 
% GitHub Advisory and External link & 1 & & \\ \bottomrule
% \end{tabular}
% \end{table}

% \begin{table}[h!]
% \centering
% \caption{Non-compliance in security-related issues based on reporting mechanisms and reporter roles}
% \label{tab:issue_reporting_mechaism_submitter}
% \begin{tabular}{p{2.9cm}r p{2.0cm}r}
% \toprule
% \textbf{Reporting mechanism} & \textbf{\# Issue} & \textbf{Submitter association} & \textbf{\# Issue} \\ \midrule
% Email & 311 & NONE & 349 \\
% External link & 207 & MEMBER & 202 \\
% GitHub (GH) Advisory & 166 & CONTRIBUTOR & 182 \\
% Email and External link & 76 & COLLABORATOR & 53 \\
% Not mentioned & 14 & OWNER & 1 \\
% Email \& GH Advisory & 6 &  &  \\
% Email \& GH issues & 6 &  &  \\ 
% GH Advisory \& External link & 1 &  &  \\ \bottomrule
% \end{tabular}
% \end{table}

\begin{table}[h!]
\centering
\caption{Non-compliance in security-related issues based on reporting mechanisms and reporter roles}
\label{tab:issue_reporting_mechaism_submitter}
\begin{tabular}{p{2.9cm}r p{2.0cm}r}
\toprule
\textbf{Reporting mechanism} & \textbf{\%} & \textbf{Submitter} & \textbf{\%} \\ \midrule
Email & 39.52 & NONE & 44.34 \\
External link & 26.30 & MEMBER & 25.67 \\
GitHub (GH) Advisory & 21.09 & CONTRIBUTOR & 23.13 \\
Email and External link & 9.66 & COLLABORATOR & 6.73 \\
Not mentioned & 1.78 & OWNER & 0.13 \\
Email \& GH Advisory & 0.76 &  &  \\
Email \& GH issues & 0.76 &  &  \\ 
GH Advisory \& External link & 0.13 &  &  \\ \bottomrule
\end{tabular}
\end{table}

% \begin{table}[h!]
% \centering
% \caption{Non-compliance in security-related issues based on reporting mechanisms and reporter roles}
% \label{tab:issue_reporting_mechaism_submitter}
% \begin{tabular}{p{2.0cm}r}
% \toprule
% \textbf{Submitter association} & \textbf{Percentage} \\ \midrule
% NONE & 44.34 \\
% MEMBER & 25.67 \\
% CONTRIBUTOR & 23.13 \\
% COLLABORATOR & 6.73 \\
% OWNER & 0.13 \\ \bottomrule
% \end{tabular}
% \end{table}


% \begin{tcolorbox}[colframe=black!75, colback=white]
 \textbf{RQ2 Summary:} 
%  % The majority of non-compliant security-related issues are created by external contributors. 
Most developers follow the defined reporting mechanism, as a small number of security issues that do not comply with the policy are created by external contributors.
% \end{tcolorbox}

% \begin{tcolorbox}[colback=gray!5,colframe=black,title= RQ2 Summary]
% The findings suggests that the majority of non-compliant security-related issues are created by external contributors.
% \end{tcolorbox}

% \begin{table}[h!]
% \centering
% \caption{Non-compliance in security-related issues based on reporting mechanisms and reporter roles}
% \label{tab:issue_reporting_mechaism_submitter}
% \begin{tabular}{@{}lrlr@{}}
% \toprule
% \textbf{Reporting mechanism} & \textbf{\# Issue} & \textbf{Submitter association} & \textbf{\# Issue} \\ \midrule
% Email & 476 & NONE & 349 \\
% GitHub Advisory & 167 & MEMBER & 202 \\
% Email and External link & 73 & CONTRIBUTOR & 182 \\
% External link & 54 & COLLABORATOR & 53 \\
% Email and GitHub Advisory & 6 & OWNER & 1 \\
% Not mentioned & 11 & & \\ \bottomrule
% \end{tabular}
% \end{table}

% The finding indicated there are some packages the security issues still being submitted even though they define the security policy to report the vulnerabilities privately, including the majority of the submitters being contributors from outside the packages. 
% This shows that the external contributors may not be aware of the security policy. However, all these security issues show up in a small proportion of the packages that define the security policy.

\subsection{RQ3: Do projects with a security policy differ in OpenSSF Scorecard scores compared to those without one?}

We used the OpenSSF Scorecard tool to assess 302 repositories with a security policy and 376 without one across 10 measurable security practices from OpenSSF Scorecard: Binary Artifacts, Branch Protection, CII Best Practices, Contributors, Dependency Update Tool, Fuzzing, License, Maintained, SAST, and Vulnerabilities. 

Table \ref{tab:SSF_score_statistics} shows the comparison results. Repositories with a security policy in place show higher adherence to security practices across most metrics. Notably, the mean scores for eight practices such as Branch Protection, CII Best Practices, and Maintained, are significantly higher in repositories with a security policy. This is shown by a higher mean score with \textit{p-value} less than 0.001 for the repositories with the security policy. This indicates that repositories with security policies are more proactive about implementing strong security practices.

% \begin{tcolorbox}[colframe=black!75, colback=white]
\textbf{RQ3 Summary:} Repositories with security policies are more proactive in implementing security practices, as evidenced by the higher security practices score.
% \end{tcolorbox}

% \begin{tcolorbox}[colback=gray!5,colframe=black,title= RQ3 Summary]
% The findings indicate that repositories with security policies are more proactive in implementing security practices, as evidenced by the higher security practices score.
% \end{tcolorbox}

% To answer RQ3, we analyzed the security practices of packages with and without a security policy, using the OpenSSF Scorecard tool, to determine whether there is a relationship between establishing security and the security practices of the packages. 

% We use the OpenSSF Scorecard tool to assess 303 repositories with a security policy and 376 without one across 10 practices measurable on GitHub: Binary Artifacts, Branch Protection, CII Best Practices, Contributors, Dependency Update Tool, Fuzzing, License, Maintained, SAST, and Vulnerabilities. Table \ref{tab:SSF_score_statistics} shows the comparison results. The repository with a security policy established demonstrate higher adherence to security practices in most metrics. The mean score of 8 practices of the repository with a security policy significantly higher than the without one such as Branch-Protection, CII-Best-Practices, and Maintained.

% statistics for the security practices score from the OpenSSF Scorecard of packages with and without a security policy established. The statistics include minimum, maximum, and average scores, along with the p-value of the observed difference score between the packages. 

% The repository with a security policy established demonstrate higher adherence to security practices in some metrics like the Dependency Update Tool and Maintained, as the mean score of each practice shows a statistically significant difference with a p-value lower than 0.001. 


% \begin{table}[h!]
% \centering
% \caption{mean OpenSSF Scorecard score of packages with and without security policy \morakot{update this with the more statistical measures}}
% \label{tab:SSF_score_statistics}
% \begin{tabular}{@{}lrr@{}}
% \toprule
% \textbf{Security practics} & \textbf{With Policy} & \textbf{Without Policy} \\ \midrule
% \textbf{Overall Scorecard Score} & 5.93 & 3.95 \\
% \textbf{Binary-Artifacts} & 9.50 & 9.64 \\
% \textbf{Branch-Protection} & 3.53 & 1.69 \\
% \textbf{CII-Best-Practices} & 0.29 & 0.01 \\
% \textbf{Contributors} & 9.48 & 8.25 \\
% \textbf{Dependency-Update-Tool} & 7.02 & 3.19 \\
% \textbf{Fuzzing} & 1.75 & 0.61 \\
% \textbf{License} & 9.45 & 9.24 \\
% \textbf{Maintained} & 7.49 & 4.04 \\
% \textbf{SAST} & 3.57 & 4.04 \\
% \textbf{Security-Policy} & 9.40 & 0.00 \\
% \textbf{Vulnerabilities} & 7.45 & 7.43 \\
% \bottomrule
% \end{tabular}
% \end{table}

% \begin{table*}[h!]
% \centering
% \caption{Statistics of OpenSSF Scorecard scores for packages with and without security policies \morakot{update this with the more statistical measures} \bee{done}}
% \label{tab:SSF_score_statistics}
% \begin{tabular}{@{}lrrrrrrr@{}}
% \toprule
% \multirow{2}{*}{\textbf{Security Practice}} & \multicolumn{3}{c}{\textbf{With Policy}} & \multicolumn{3}{c}{\textbf{Without Policy}} & \multirow{2}{*}{\textbf{p-value}} \\ \cmidrule(lr){2-4} \cmidrule(lr){5-7}
%  & \textbf{Min} & \textbf{Max} & \textbf{Mean} & \textbf{Min} & \textbf{Max} & \textbf{Mean} &  \\ \midrule
% \textbf{Aggregate Score*} & 3.00 & 9.40 & 5.93 & 1.30 & 7.30 & 3.95 & $<$0.001 \\
% \textbf{Binary-Artifacts} & 0.00 & 10.00 & 9.50 & 0.00 & 10.00 & 9.64 & 0.26 \\
% \textbf{Branch-Protection*} & 0.00 & 10.00 & 3.53 & 0.00 & 8.00 & 1.69 & $<$0.001 \\
% \textbf{CII-Best-Practices*} & 0.00 & 10.00 & 0.29 & 0.00 & 2.00 & 0.01 & $<$0.001 \\
% \textbf{Contributors*} & 0.00 & 10.00 & 9.48 & 0.00 & 10.00 & 8.25 & $<$0.001 \\
% \textbf{Dependency-Update-Tool*} & 0.00 & 10.00 & 7.02 & 0.00 & 10.00 & 3.19 & $<$0.001 \\
% \textbf{Fuzzing*} & 0.00 & 10.00 & 1.75 & 0.00 & 10.00 & 0.61 & $<$0.001 \\
% \textbf{License} & 0.00 & 10.00 & 9.45 & 0.00 & 10.00 & 9.24 & 0.20 \\
% \textbf{Maintained*} & 0.00 & 10.00 & 7.49 & 0.00 & 0.00 & 4.04 & $<$0.001 \\
% \textbf{SAST*} & 0.00 & 10.00 & 3.57 & 0.00 & 0.00 & 1.06 & $<$0.001 \\
% \textbf{Vulnerabilities} & 0.00 & 10.00 & 7.45 & 0.00 & 10.00 & 7.43 & 0.95 \\ \bottomrule
% \end{tabular}
% \centering
% \parbox{0.8\textwidth}{\centering * indicates that projects with established policies receive a statistically significant higher mean score than the projects without policy.}
% \end{table*}

\begin{table}[h!]
\centering
\caption{Comparison of OpenSSF Scorecard scores between repositories with and without security policies}
\label{tab:SSF_score_statistics}
\begin{tabular}{@{}lrrrrrrr@{}}
\toprule
\multirow{2}{*}{\textbf{Security Practice}} & \multicolumn{2}{c}{\textbf{With Policy}} & \multicolumn{2}{c}{\textbf{Without Policy}} & \multirow{2}{*}{\textbf{p-value}} \\ \cmidrule(lr){2-3} \cmidrule(lr){4-5}
 & \textbf{Mean} & \textbf{SD} & \textbf{Mean} & \textbf{SD} &  \\ \midrule
\textbf{Aggregate Score*} & 5.93 & 1.14 & 3.95 & 1.29 & $<$0.001 \\
\textbf{Binary-Artifacts} & 9.50 & 1.73 & 9.64 & 1.61 & 0.26 \\
\textbf{Branch-Protection*} & 3.53 & 2.94 & 1.69 & 2.57 & $<$0.001 \\
\textbf{CII-Best-Practices*} & 0.29  & 1.21 & 0.01 & 0.1 & $<$0.001 \\
\textbf{Contributors*} & 9.48 & 1.97 & 8.25 & 3.37 & $<$0.001 \\
\textbf{Dependency-Update-Tool*} & 7.02 & 4.58 & 3.19 & 4.67 & $<$0.001 \\
\textbf{Fuzzing*} & 1.75 & 3.81 & 0.61 & 2.40 & $<$0.001 \\
\textbf{License} & 9.45 & 1.71 & 9.24 & 2.29 & 0.20 \\
\textbf{Maintained*} & 7.49 & 4.12 & 4.04 & 4.62 & $<$0.001 \\
\textbf{SAST*} & 3.57 & 4.44 & 1.06 & 2.84 & $<$0.001 \\
\textbf{Vulnerabilities} & 7.45 & 3.9 & 7.43 & 3.95 & 0.95 \\ \bottomrule
\end{tabular}
\centering
% \parbox{0.8\textwidth}{\centering * indicates that projects with established policies receive a statistically significant higher mean score than the projects without policy.}
\begin{tablenotes}
\item{* indicates practices where the score for repositories with a security policy is significantly higher than those without.}
\end{tablenotes}
    % \vspace*{-.67cm}

\end{table}

