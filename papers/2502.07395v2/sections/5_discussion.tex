

% Our study highlights the need for effective, inclusive security policies that address diverse roles within the developer community. The observed non-compliance in reporting security issues suggests that current reporting methods may not fully support all practitioners, especially external contributors who may lack secure communication access or familiarity with project-specific protocols. This underscores the importance of flexible, accessible security policies that enable all developers to follow best practices. Moreover, compliance challenges reveal a broader need for clear, reinforced security practices across all levels of project participation. This opens an opportunity for further research to optimize security policies, tailoring reporting mechanisms to diverse roles and improving communication strategies to support consistent adherence. By enhancing policy inclusivity, we can strengthen security practices across open-source projects.


% This section discusses the implications of our findings. Including the factors that might impact our study results.

% The findings from RQ1 showed that most reporting mechanisms provided in the security policy of PyPi packages prefer to report the vulnerability through email, followed by proving the external links and using GitHub Advisories. Some packages also request to avoid reporting the vulnerability through GitHub issues. Only a small number of packages provide instructions for reporting vulnerabilities through GitHub issues. This finding suggests that project maintainers are aware of publicly disclosed security vulnerabilities and avoid them by reporting them privately. Security issues account for only a small proportion of all issues and only a small proportion of affected packages. However, the findings from RQ2 show that the majority of these packages continue to have security issues, despite the establishment of a security policy that instructs developers to report vulnerabilities privately. This indicated the limits of the broader developer's awareness of vulnerabilities and potential fixes, especially for contributors outside the projects. This highlights that project maintainers should consider enhancing their public communication strategies to better inform developers about existing security concerns and improvements.

% The findings from RQ3 demonstrate the relationship between establishing a security policy and implementing security practices, which are evaluated using the OpenSSF Scorecard. Most security practices in packages with and without a security policy do not differ. However, some practices are obviously different scores, such as Dependency-Update-Tool and Maintained. This finding underscores the significance of a security policy, as it demonstrates that projects with established security policies adopt security practices more effectively than those without projects without policies.