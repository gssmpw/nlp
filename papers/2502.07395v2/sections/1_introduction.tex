\section{Introduction}
\label{1_introduction}




% Ensuring software security is crucial, particularly for open source projects, which are often accessible to a wide range of contributors and end users. The awareness of software security has emerged around software vulnerabilities cataloged in frameworks like Common Weakness Enumerations (CWE) and Common Vulnerabilities and Exposures (CVE), a database of publicly disclosed vulnerabilities in software and hardware and being documented in the National Vulnerability Database (NVD) \cite{NVD:online}. 

% \morakot{This paragraph needs supporting evidence. Please find literature that backs up your argument or adds a related perspective.} \bee{done-all cited}

Open-source software has become essential in modern development, enabling developers to reuse and modify code. However, with its growing use, the risk of vulnerabilities has increased \cite{Alfadel:EMSE2023}. As security threats rise, developers' security practices are essential \cite{Zahan:ICSE2023}. Effective management is key, especially for open-source projects, where developers' responses to vulnerabilities are vital \cite{Prana:EMSE2021}. Publicly disclosing vulnerabilities without fixes heightens the risk of exploitation. To address this, platforms like GitHub provide security measures, such as a \texttt{SECURITY.md} file \cite{GitHubSecurityPolicy:online} for reporting vulnerabilities and GitHub Advisories \cite{GitHubAdvisoryDatabase:online} for documenting security issues. 

To address bugs or discuss their projects, developers use GitHub Issues, a task-tracking feature available to all GitHub users for open-source projects. However, the traditional practice of reporting security-related issues through GitHub Issues can pose risks of vulnerability being known by attackers, as it makes vulnerabilities publicly known before they are fixed, giving attackers more time to exploit them. For this reason, some GitHub projects request that developers avoid reporting vulnerabilities through Issues. For example, Figure \ref{fig:example_issue} shows a security issue in an open-source project posted publicly as a GitHub issue, detailing a problem with token handling. By publically posting this issue, the unaddressed token leak could allow attackers to perform unauthorized actions beyond the intended scope before the developers can fix the issue.

% Figure \ref{fig:example_issue} shows an example of the problem in security issues specified in an open-source project. The issue provides information about the problem with token handling in the project. Since the project is open-source, any GitHub user can access this issue. This scenario increases the risks when the attackers are able to specify the problem from this issue. If attackers identify a token leak that remains unaddressed, it could enable them to take unauthorized actions beyond the scope of the intended task.

% \morakot{leverage this as a motivating example, add example of that issue}  \chaiyong{I think ``specific task'' is quite confusing here. Do you mean the ``intended task''?} \bee{done-added example issue as an example, fixed wording for specific tasks to be intended task}



% Open-source software has become essential to modern software development, allowing developers to reuse and modify existing code for software development. However, as open-source projects have become more widely used, the risk of software vulnerabilities has also been increasing over time \cite{Alfadel:EMSE2023}. 
% As security threats grow, the developer's practices toward security are essential \cite{Zahan:ICSE2023}. Improper management can exploit vulnerabilities within the projects. Especially for the open source project, how developers react to the vulnerability is essential \cite{Prana:EMSE2021}. Publicly disclosing vulnerabilities without fixes increases the risk of attackers exploiting them. To address these concerns in security practices, platforms like GitHub have integrated measures to enhance open-source project security. By providing the security policies through a dedicated SECURITY.md file \cite{GitHubSecurityPolicy:online}. The security policies in GitHub projects outline the process for reporting security vulnerabilities in projects. GitHub also provides GitHub Advisories \cite{GitHubAdvisoryDatabase:online}, a channel to disclose, address, and document security issues through structured advisories. 


% Platforms like GitHub have integrated practices to improve open source project security with the addition of security policies through a dedicated SECURITY.md file. Security policies outline the process for reporting the security vulnerabilities of the projects. GitHub also offers GitHub Advisories (\cite{GitHubAdvisoryDatabase:online}), a channel to disclose, address, and document security issues through structured advisories linked to CVEs and CWEs.

% To address the bugs or engage in discussions about their projects, developers can use GitHub Issues, a task-tracking feature, to report or discuss their projects. For open source projects, all GitHub users can access, create, or submit issues. When it comes to security-related issues in the projects, this practice made vulnerabilities known publicly before the vulnerabilities were fixed. Giving attackers more time and increases the likelihood of vulnerabilities being exploited. Thus, some GitHub projects ask developers not to report the vulnerability by creating an issue.

\begin{figure}[!]
    \centering
    \includegraphics[width=0.95\linewidth]{figure/example_issue_cut.png}
    \caption{A security issue created in open source project}
    
    \label{fig:example_issue}
    \vspace*{-.75cm}
% \vspace{-.3cm}
\end{figure}

% \morakot{discuss existing work and identify gaps} \bee{done}

Recent studies showed that only a small proportion of open-source projects have defined security policies \cite{Ayala:SVM2023}. However, defining security policy in the projects is considered an important practice for developers to improve the project's security \cite{Zahan:ICSE2023}. These highlight a gap in the security practices of the open-source projects. However, no study has focused on analyzing the characteristics and effectiveness of the security policies, including the extent to which developers' practices align with the provided guidelines.
The related work in this study examines security policies in open-source projects on GitHub. Ayala et al. \cite{Ayala:SVM2023} analyzed popular GitHub repositories, finding that only a small portion had implemented security policies, with many lacking structured security measures. This highlights the importance of workflows and security policies in open-source projects. The GitHub security policy feature guides vulnerability reporting, while security issues serve as a feature for bug tracking and project discussions. Bühlmann et al. \cite{Noah:SAC2022} studied security issues, noting that they comprise a small yet growing portion of issues and require more discussion time than non-security issues.
Zahan et al. \cite{Zahan:IEEE} used the OpenSSF Scorecard to assess security practices in the npm and PyPI ecosystems. The study identified gaps in areas such as Code Review, Maintenance, Licensing, Branch Protection, and Security Policy. The study also recommended practices for enhancing such areas. Additionally, a relationship between security practices and vulnerability levels was identified \cite{Zahan:ICSE2023}.

Despite increased attention on security policies for open-source projects, no research has yet examined the actual content of these policies. Our study addresses this gap by analyzing specific elements within security policies to propose guidelines for improvement. By identifying practices that could improve OpenSSF Scorecard scores if properly implemented, we set the stage for further guidance on defining effective security policies.


This motivated us to investigate the characteristics of security policies in open-source projects, aiming to provide software practitioners with guidelines for enhancing security policies. This paper presents our preliminary investigation into GitHub projects' security policies, focusing on two main aspects. First, we examine how developers report vulnerabilities by comparing the practices outlined in security policies with actual developer actions. To do this, we analyze the content of security policies to identify vulnerability reporting processes and manually review developer practices. Second, we conduct a preliminary assessment of security practices' quality using the Open Source Security Foundation's (OpenSSF) Scorecard \cite{Scorecard:online}. Through a comparative study, we evaluate security practices in projects with and without defined security policies to determine whether the presence of a security policy strengthens overall security practices. Thus, we define the following research questions for our preliminary investigation:

\begin{enumerate}
    \item \textbf{RQ1:} What are the reporting mechanisms in security policies?
    \item \textbf{RQ2:} Do the developer's practices align with the security policy?
    \item \textbf{RQ3:} Do projects with a security policy differ in OpenSSF Scorecard scores compared to those without one?
\end{enumerate}


% including how well developers' practices align with the provided guidelines.


% \morakot{need to strengthen the objective and a big goal of the study. and say what is the scope of our ERA paper.}

 % focusing on the PyPi packages available on GitHub\chaiyong{May need some support on why we're interested in the PyPi packages}. Given Python is a widely used programming language in the software development, as indicated in the 2023 Stack-Overflow survey \cite{TopProgrammingLanguage:online}.

 % to establish the guidelines for improving the security practices of the project

% This motivated us to investigate the characteristics of an open-source project's security policy . We aim to establish guidelines for software practitioners to enhance their security policies. This paper thus servs as our preliminary investigation on the security policy of GitHub projects by focusing on two main aspects. First, we perform the investigation on how developers report vulnerabilities in both the practices defined in security policies and the developers' actual practices. To investigate the practices outlined in security policies, we mine the security policy's content and identify the process for reporting vulnerabilities and manually check contributor's practices. Second, to preliminary assess quality of security practices in those projects, we use the Open Source Security Foundation's (OpenSSF) Scorecard \cite{Scorecard:online}. We conduct a comparison study to evaluates security practices on open-source project that have security policy and do not have security policy to check whether the existing of security policy strengthen the project's secrutiy practice in overall.


% Since the security policy of GitHub projects primarily aims to provide instructions on how to report project vulnerabilities, so we aim to investigate how developers report vulnerabilities in both the practices defined in security policies and the developers' actual practices. To investigate the practices outlined in security policies, we look into the security policy's content and identify the process for reporting vulnerabilities. We define this process as a ``reporting mechanism''.\chaiyong{$\leftarrow$ I do not understand this sentence. It may need to be elaborated.} We then expand the study to other security practices of the projects, which would be improved if the project established the security policy. To access the security practices that can improve security policies, we use the Open Source Security Foundation's (OpenSSF) Scorecard \cite{Scorecard:online} for the projects with and without security policy. The OpenSSF Scorecard evaluates security practices, resulting in 18 metrics with an aggregate score. \chaiyong{I think this whole paragraph presents many things but are not connected well. You may need to link better between security policies, reporting mechanism, and the OpenSSF.} \bee{done-fixed whole paragraph above and added supporting reason for studying in PyPI}


To address our research questions, we analyzed 679 PyPi packages hosted on GitHub. We focused on PyPI ecosystem for Python language since Python is a widely used programming language in software development, as indicated in the 2024 IEEE Spectrum survey \cite{TopProgrammingLanguage:online}. We examined how developers report security vulnerabilities by reviewing security policy instructions and analyzing security issues in these projects. We then compared Scorecard scores between projects with and without a defined security policy. Our study shows that most PyPi package maintainers recognize the need for private channels like email or GitHub advisories to report vulnerabilities, as outlined in many security policies. However, external contributors often bypass these policies and submit security issues publicly. Projects with established security policies show notable improvements in the Scorecard's Dependency Update Tool and Maintained Practices scores. This underscores the importance of improving public communication strategies or refining security policies to enhance developer awareness and adherence to security practices.

% \morakot{we compare openssf score between projects with and without security.md to ....} \bee{done- description added in the below paragraph}

% To answer the research questions, we analyzed 679 PyPi packages hosted on GitHub since. We examine how developers report security vulnerabilities in the projects by exploring the instructions outlined in the security policies and analyze the security issues created in the projects. We then compare Scorecard scores between projects with and without a security policy. Our study thus reveal that most PyPi package maintainers are aware of the publicly discussed vulnerability information, as most security policies outline the process for reporting vulnerabilities through private channels, such as email or GitHub advisories. However, external contributors still submit security issues without adhering to the security policy. For the security practices of the projects, the Dependency Update Tool and Maintained Practices show the significant difference in the projects with the security policy established. Emphasize the significance of improving public communication strategies or security policy to enhance developers' awareness of security practices.

% . To investigate the developer's practices regarding the security policy, we analyze the security issues created in the projects with policy established. We then compare Scorecard scores between projects with and without a security policy to find the practices that would be improved if the policy were established, highlighting the impact of the policy on the security practices of the projects.

% \morakot{briefly discuss finding here} \bee{done}

% Most PyPi package maintainers are aware of the publicly discussed vulnerability information, as most security policies outline the process for reporting vulnerabilities through private channels, such as email or GitHub advisories. However, external contributors still submit security issues without adhering to the security policy. However, this practice is observe in only a few packages. For the security practices of the projects, the Dependency Update Tool and Maintained Practices show the significant difference in the projects with the security policy established. Emphasize the significance of improving public communication strategies or security policy to enhance developers' awareness of security practices.

% \morakot{dev practices that related to security, since reporting is the main focus of github sec.md, we thus study how dev report it, then we expand the study to other practices in openssf score byusing openssf score as a proxy to measure practices. we thus start by compare with and without sec.md project. this is aim to imrpove the adotpion of sec policy.} \bee{done-described in objective paragraph (This motivated us to...)}