
\section{Fine-tuning for Moral Reasoning Acquisition\label{sec:moraltuning}}
In this section, we introduce fine-tuning strategies and experimental results of existing learning paradigms for moral reasoning acquisition, focusing on the tasks of RoT generation and ethical judgment prediction.
\begin{table*}[t]
\small
\centering
\begin{tabular}{c c c c c| c c c c c}
\toprule
\texttt{SocialChem} &\text{\small BertScore} & \text{\small Rouge1} & \text{\small Rouge2} & \text{\small RougeL} & \texttt{MIC} &\text{\small BertScore} & \text{\small Rouge1} & \text{\small Rouge2} & \text{\small RougeL} \\ 
        \midrule
        rot&.777 & .229 & .096 & .213& rot&.768 & .175 & .077 & .168 \\ 
        \textbf{moral-rot}& .836 & .416 & .205 & .401& \textbf{moral-rot}&.826 & .393 & .192 & .379 \\ 
        \midrule
        judg&.7240 & .230 & .137 & .230 & judg&.671 & .071 & .000 & .071 \\ 
        \textbf{moral-judg}& .7632 & .464 & .346 & .464& \textbf{moral-judg}&.762 & .314 & .000 & .314 \\
        rot-judg&.7626 & .464 & .346 & .464& rot-judg&.660 & .061 & .000 & .061 \\
        moral-rot-judg&.7628 & .463 & .345 & .463& moral-rot-judg&.761 & .306 & .000 & .306 \\
        \bottomrule
\end{tabular}
\caption{\small Performance of Fine-tuned \texttt{Mistral} Model Across Various Fine-tuning Strategies for Each Benchmark, with the best strategy highlighted in \textbf{bold}. For both tasks, introducing more information, e.g., moral foundation, in the fine-tuning process would improve the performance. The~\texttt{moral-rot} achieves the optimal performance for both SocialChem and MIC. The~\texttt{moral-judg} and~\texttt{moral-judg} are the best strategy for SocialChem and MIC respectively, in terms of the judgment prediction task. Additional results for Llama3 are availabe in Table~\ref{tab:ethicaltuning4llama3}.}
\label{tab:ethicaltuning4mistral}
\end{table*}

\textbf{Experimental Settings.}
We take Mistral-7B\footnote{\url{https://huggingface.co/mistralai/Mistral-7B-v0.1}} and Llama3-8B\footnote{\url{https://huggingface.co/meta-llama/Meta-Llama-3-8B-Instruct}} as the backbone models and leverage the LoRA method to fine-tune them through a supervised fine-tuning loss.
For each benchmark, we employ two fine-tuning strategies for RoT generation and four fine-tuning strategies for ethical judgment prediction. For \textit{RoT generation}, we fine-tune LLMs: (1) to directly generate RoT according to the given situation (\text{rot}) and (2) first generate the moral foundation, then the RoT (\text{moral-rot}). For \textit{Judgment Prediction}, we fine-tune LLMs to: (1) directly predict judgment (\text{judg}) , and (2) firstly generate the moral foundation and/or RoT then the judgment (\text{moral-judg},~\text{rot-judg} and~\text{moral-rot-judg}).

The prompting format and LoRA fine-tuning settings are available in Appendix~\ref{appendix:ethicaltuning}. 
We consider 10000 samples with only one underlying moral foundation for analytical convenience. In the process of fine-tuning, we take the check point with the least loss on the  development set, and report its performance on the test set.
During inference, we prompt fine-tuned LLMs to first generate intermediate predictions before producing the final RoT or ethical judgment, following the same prompting strategy used during fine-tuning.
For example, in the~\texttt{moral-rot} strategy, LLMs are instructed to first predict the moral foundation based on the given situation and subsequently generate the RoT using both the situation and the predicted moral foundation. Following~\citet{ziems2022moral}, we report the performance of the BertScore~\cite{zhang2019bertscore}, Rouge-1, Rouge-2, and Rouge-L metrics.

RoT generation and ethical judgment prediction align with the core capabilities essential for morality-related scenarios and serve as prototypical formats for moral reasoning.
By incorporating moral foundations into RoT generation, we aim to guide LLMs to first identify the moral foundation associated with a given situation, thereby improving the quality of the generated RoT. RoTs serve as instances of evidence and explanation for ethical judgments, aligning with previous studies that seek to enhance LLMs’ social intelligence through social interaction environments~\cite{liutraining,wang-etal-2024-sotopia}.

\textbf{Main Results.}
Table~\ref{tab:ethicaltuning4mistral} and Table~\ref{tab:ethicaltuning4llama3} present fine-tuning results for Mistral and Llama3, respectively\footnote{Note that this paper does not aim to achieve state-of-the-art performance but rather to investigate the underlying mechanisms behind these performance gains.}.
As shown in Table~\ref{tab:ethicaltuning4mistral}, introducing moral foundations in fine-tuning enhances performance across all experimental settings.
However, incorporating RoT information along into the ethical judgment prediction task has a negative impact to the MIC benchmark. We hypothesize that this is because judgments are significantly shorter than RoTs, and the added complexity of RoTs would introduce challenges for fine-tuning.