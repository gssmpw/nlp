\section{Related Works}
Machine ethics~\cite{anderson2011machine,tolmeijer2020implementations,nath2020problem,allen2006machine} has been a long-standing research topic for hardware and software systems, with the aim of maximizing their benefits while minimizing socio-ethical risks.
%Recently, we have witnessed the progress of Artificial Intelligence (AI), particularly that associated with Large Language Models (LLMs), changing the world.
Ensuring LLMs will acquire an understanding of ethics to prevent them from making harmful decisions has become a serious research problem for both academia and industry.
Dating back to the 1940s, the Three Laws of Robotics~\cite{asimov1941three} were proposed to ensure that robots do not cause harm to humans.
Since then, machine ethics has been explored by researchers in philosophy, psychology, and cognitive science. However, it remains a significant challenge for AI, as even coherent and diverse language generation poses difficulties.
The widespread deployment of LLMs opens the door for AI researchers to pursue ethics acquisition due to their strong semantic modeling capability. 
%of LLMs. %and their widespread deployment.
% The emergence of LLMs

Numerous studies have attempted to evaluate the moral and ethical orientations encoded in LLMs through empirical experiments. \citet{bonagiri2024sage} demonstrates that model performance and moral consistency are independent of one another, while \citet{abdulhai2023moral} investigates whether LLMs exhibit biases toward specific moral principles. \citet{scherrer2024evaluating} proposes a statistical method to assess the moral values encoded in LLMs, and \citet{zhang2023measuring} introduces a metric to determine whether LLMs understand ethical values both in terms of “knowing what” and “knowing why.” Collectively, these studies highlight that LLMs lack consistent moral or ethical orientations across different scenarios.
%Enabling LLMs to acquire ethical values is a formidable challenge, not only because ethical AI operates at the level of pragmatics~\cite{awad2022computational}, but also due to the philosophical complexities surrounding the proper representation of human ethics~\cite{zhixuan2024preferencesaialignment}. Progress has been made, albeit only partially. 