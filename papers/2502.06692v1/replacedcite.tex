\section{Related work}
\paragraph{Language identification} The task of identifying the language of a text is an ``old'' NLP task dating back to the 1960s. Simple but relatively powerful tools have been available since the 1990s ____.

In recent years, the main focus of NLP research has shifted towards large language models, and especially towards extending their coverage to an increasing number of languages. As training data for underrepresented languages is mostly found in web crawls, reliable LID systems covering a large number of languages are more important than ever. While the earliest LID systems were restricted to a dozen languages, recent systems cover hundreds ____ and even thousands ____ of languages.

In terms of methods, simple linear classifiers with character-level and word-level features have often outperformed more sophisticated neural models ____. Most currently available large-coverage LID models are based on the FastText architecture ____, a multinomial logistic regression classifier with character n-gram embeddings as input features. These include FastText-176 ____, NLLB-218 ____, OpenLID ____ and GlotLID ____. Different approaches are used by HeLI-OTS ____, which bases its decisions on a combination of character n-gram and word unigram language models, and \texttt{gpt2-lang-ident}\footnote{\url{https://huggingface.co/nie3e/gpt2-lang-ident}}, which is a fine-tuned decoder-only model ____.

In practice, LID is most often applied to individual sentences, even though the tools can work with longer or shorter segments of text.


\paragraph{LID for closely related and Nordic languages}

To our knowledge, the only publication focusing specifically on LID for Nordic languages is \newcite{haas-derczynski-2021-discriminating}. They compile a dataset for the six languages (including both Norwegian standards) from Wikipedia and evaluate a range of LID models on it. They find that the languages mostly cluster into three groups: Danish--Bokmål--Nynorsk, Swedish, and Icelandic--Faroese. Their models were not available online as of writing this paper. Besides this, \newcite{nbnordiclid} present two FastText-based LID models: one containing only the 12 most common languages of the Nordic countries (including several Sámi languages, Finnish, and English), and one with an extended coverage of 159 languages.

Futhermore, the previously mentioned off-the-shelf LID systems (NLLB-218, OpenLID, GlotLID, HeLI-OTS) cover all six Nordic languages, with the exception of FastText-176, which does not include Faroese.


\paragraph{Multi-label language identification}

Most existing LID training and evaluation corpora are not manually labeled. Instead, they are based on the assumption that the language is determined by the source it is retrieved from. If a sentence is retrieved from a Danish newspaper, it is assumed to be only Danish. 
But when dealing with closely related languages, it is often the case that an instance cannot be unambiguously assigned to a single language ____.

Recent proposals address this issue by framing LID between similar languages as a multi-label task ____ and by manually annotating the evaluation data ____. However, these works do not include studies of Scandinavian languages.