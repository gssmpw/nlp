\section{Related Works}
\label{sec:related_works}
% how did the field of FSA start 
Sentiment analysis in financial markets emerged from the growing recognition that market psychology plays a crucial role in market movements \cite{adaptive_lo_2004}. As Natural Language Processing (NLP) techniques advanced, they enabled the extraction of sentiments and emotions from large volumes of unstructured text, providing a means to quantify market psychology \cite{twitter_bollen_2011}. Social media platforms, particularly Twitter, became valuable sources of real-time insights into investor sentiment, which could be used to predict financial market behavior. Sentiment analysis, or opinion mining, had already been established as a method within NLP for extracting emotional responses \cite{murfi2024bert}, making them well-suited for quantifying investor psychology and applying these signals to financial markets. The power of sentiment analysis in forecasting market trends, volatility, volume, and risk contributed to the birth of FSA \cite{man2019financial, du2024financial}. The primary goal of FSA has been to utilize NLP methods to capture investor sentiment and psychology, providing valuable insights for predicting market movements \cite{du2024financial}. Over time, the field has evolved, and various methods have been developed to refine sentiment classification, improve prediction accuracy, and integrate more nuanced factors such as market context. In the subsequent sections, we will explore the current state of the art in FSA, examine existing gaps, and highlight how this study aims to address these challenges.

\subsection{Financial Sentiment Analysis Scope}
FSA spans a wide range of methods, including lexicon-based approaches \cite{du2023finsenticnet}, word representations \cite{cryptocurrency_lamon_2017}, and pre-trained language models \cite{teng2025bert, huang2023finbert, sentiment_kulakowski_2023, fingpt_yang_2023, fineas_gutirrezfandio_2021}. It also draws on various data sources such as news headlines \cite{cryptocurrency_lamon_2017}, social media posts \cite{singh2020sentiment}, and financial documents \cite{man2019financial}, with applications in volatility prediction, price prediction, and trend forecasting \cite{du2024financial}. The core goal of FSA is to quantify investor sentiment from textual data to inform decision-making in financial markets. In this study, we focus on predicting Bitcoin’s short-term market movements by quantifying sentiment from Bitcoin-related tweets. This task has been addressed by various studies, but we consider PreBit \cite{prebit_zou_2023} to be the state-of-the-art in this area. Unlike many other studies, PreBit advances the field by passing the actual tweet representations, rather than their sentiment class labels, to the trend prediction model. This allows the model to learn patterns between the words in the tweets and the market movements they ought to predict. By retaining the tweet’s representation, the model can identify nuanced relationships between specific words and their effect on market behavior, bridging the gap between textual sentiment and market outcomes.

Despite advancements like these, FSA models are still often used in conjunction with price-based prediction models because they do not perform adequately on their \cite{man2019financial}. This limitation is not necessarily due to the inferiority of textual data compared to price-based features. Even when domain-specific, pre-trained word embeddings are used, the issue lies in how we still assign sentimental value to words based on subjective perceptions, rather than allowing the market’s reaction to define these relationships. This leads to our core hypothesis: market-driven labels, learned from actual market movements, can better inform sentiment analysis models. By letting the market itself teach the model the significance of words, we aim to bridge the gap between textual content and market reactions, moving away from subjective interpretations. Our hypothesis also points to a second major challenge: contextualization. The context in which a tweet is shared—such as market conditions, sentiment trends, or external factors—can be more influential than the content itself. Previous research supports this idea, showing that expanded feature spaces, including factors like author credibility \cite{elon_ante_2023, GARCIAMENDEZ2024124515, ask_akbiyik_2021}, post volume \cite{bitcoin_critien_2022}, and extracted topics of discussion \cite{hu2021stock}, significantly improve the performance of FSA models. In the next section, we discuss the latest developments in NLP and FSA that have inspired our research and highlight how our approach builds on these advances.

\subsection{Innovations in NLP, and The FSA Horizon}
Large Language Models (LLMs) have increasingly been leveraged for financial market analysis, with prominent examples such as FinBERT \cite{huang2023finbert}, CryptoBERT \cite{sentiment_kulakowski_2023}, and FinEAS \cite{fineas_gutirrezfandio_2021}. These models have set a strong foundation for improving FSA tasks. FinBERT, for instance, adapts to the finance domain by effectively incorporating contextual information from financial texts, outperforming traditional machine learning algorithms in sentiment classification. Similarly, FinEAS, building upon Sentence-BERT, generates high-quality sentence embeddings that substantially enhance FSA task performance. CryptoBERT, which serves as the foundation for our study, is fine-tuned on cryptocurrency-related social media data, leveraging the BERTweet architecture and domain-specific training to accurately analyze crypto-related tweets. On the other hand, FinGPT \cite{fingpt_yang_2023} represents a breakthrough as a general-purpose financial advisor, demonstrating how LLMs can synthesize market data and provide accessible insights for developing financial models. In our work, we build upon CryptoBERT by incorporating market and temporal context into tweets, enabling the model to better detect patterns and improve prediction accuracy, ultimately offering more actionable insights for cryptocurrency investors. Our study is further inspired by research that examined the temporal window of tweet propagation during events like the COVID-19 and H1N1 pandemics, which suggested that the most impactful tweet propagation occurs within a window of 8 to 15 days \cite{valle2022does, whether_aysan_2023}. This finding informed our decision to focus on short-term trend prediction in the market, aligning with the intuition that market psychology (driven by fear and greed) manifests strongly in short-term fluctuations.

A key innovation in improving text classification tasks like FSA has been prompt engineering \cite{promptlearning_zhu_2022}. Specifically, studies such as Dhar et al. \cite{analysis_dhar_2023} have demonstrated the transformative impact of prompt engineering on financial decision-making, enhancing operational efficiency, risk management, and customer service. Ahmed et al. \cite{leveraging_ahmed_2024} further explored prompt engineering in FSA, showing how carefully designed prompts can significantly boost model performance. In our research, we use prompt fine-tuning to incorporate both temporal and market context, enhancing the model's capacity to identify patterns and improve the quality of insights for financial forecasting. We were also inspired by Dhingra et al. \cite{timeaware_dhingra_2021}, who proposed jointly modeling text with timestamps to improve temporal knowledge. Additionally, Agarwal et al. \cite{natural_xing_2018} demonstrated the benefits of temporal domain adaptation in improving model predictions. By integrating both temporal and market information into the tweet context, we enhance our model's ability to accurately predict market movements and generate actionable insights.

% why did we get here and what are we trying to cover.
Despite the current limitations and challenges in FSA, the future of this field holds tremendous promise. While there are still gaps to bridge, the continuous advancements in NLP, particularly through LLMs and innovative approaches like prompt engineering, offer a clear path forward. By leveraging market-driven labels, enhancing models with contextual and temporal information, and incorporating new data sources, we can address existing shortcomings and unlock the full potential of sentiment analysis in financial forecasting. As more studies explore these avenues and integrate market-specific context, the scope for FSA to play a central role in predicting market trends, managing risk, and enhancing decision-making in finance continues to expand. With these ongoing innovations, the future of FSA is indeed bright, offering exciting opportunities for more accurate and actionable financial predictions.