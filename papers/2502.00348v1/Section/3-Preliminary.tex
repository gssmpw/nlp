

We begin by providing a formal description of the task. Following~\cite{he2020lightgcn, he2024double}, we define a set of users as $\mathcal{U} = \{u\}$ and a set of items as $\mathcal{V} = \{v\}$, along with an observed interaction set, $\mathcal{I} = \{(u, v) \mid u \in \mathcal{U}, v \in \mathcal{V}\}$, where the pair \((u, v)\) indicates that user \(u\) has interacted with item \(v\). Generally, recommendation methods based on implicit feedback are trained on interaction data, treating \((u, v) \in \mathcal{I}\) as positive samples and \((u, v) \in (\mathcal{U} \times \mathcal{V}) \setminus \mathcal{I}\) as negative samples to learn the parameters \(\Theta\). The training of the recommendation model is formulated as:
\begin{equation*}
    \Theta^{*} = \arg\min_{\Theta} \mathcal{L}(\mathcal{U},\mathcal{V},\mathcal{I}),
\end{equation*}
where \(\mathcal{L}\) denotes the recommendation loss. However, the observed interaction set \(\mathcal{I}\) may contain noise, leading to a deviation between the learned parameters \(\Theta^{*}\) and the ideal parameters \(\Theta^{\mathrm{Ideal}}\), which represent the optimal model parameters in the absence of noise. The goal of denoising methods is to mitigate the impact of noise in the observed interaction set \(\mathcal{I}\) on the model parameters~\cite{zhang2023robust, zhao2024denoising, yu2020sampler}. 


