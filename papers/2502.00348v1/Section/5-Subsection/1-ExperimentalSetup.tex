\begin{table}[t]
  \centering
    \caption{Dataset statistics}
    \resizebox{0.49\textwidth}{!}{
        % \begin{threeparttable}

\begin{tabular}{lrrrrr}
    \toprule
    \textbf{ DATASET } & \textbf{ \#Users } & \textbf{ \#Items } & \textbf{\#Interactions}  & \textbf{Avg.Inter.} & \textbf{Sparsity}\\
    \midrule
     Gowalla  & 29,858 & 40,981& 1,027,370 & 34.4 & 99.92\% \\
     Yelp2018  & 31,668 & 38,048 & 1,561,406 & 49.3 & 99.88\% \\ 
     MIND  & 38,441 & 38,000 & 1,210,953 & 31.5 & 99.92\% \\ 
     MIND-Large  & 111,664 & 54,367 & 3,294,424 & 29.5 & 99.95\% \\
    \bottomrule
    \end{tabular}
        % \end{threeparttable}
    }
  \label{tab:datasets}%
\end{table}%




\begin{table*}[t]
    \centering
    \caption{Recommendation performance of different denoising methods. The highest scores are in bold, and the runner-ups are with underlines. A significant improvement over the runner-up is marked with * (i.e., two-sided t-test with $0.05 \le p < 0.1$) and ** (i.e., two-sided t-test with $p < 0.05$).}
    \resizebox{\textwidth}{!}{

\begin{tabular}{lcccccccccccc}
    \toprule
    \multicolumn{1}{c}{\multirow{3}{*}{\textbf{Model}}} & \multicolumn{4}{c}{\textbf{Gowalla} } & \multicolumn{4}{c}{\textbf{Yelp2018} } & \multicolumn{4}{c}{\textbf{MIND} }  \\
    \cmidrule(lr){2-5} \cmidrule(lr){6-9} \cmidrule(lr){10-13}
    & \multicolumn{2}{c}{\textbf{Recall} } & \multicolumn{2}{c}{\textbf{NDCG} } & \multicolumn{2}{c}{\textbf{Recall} } & \multicolumn{2}{c}{\textbf{NDCG} } & \multicolumn{2}{c}{\textbf{Recall} } & \multicolumn{2}{c}{\textbf{NDCG} } \\
    \cmidrule(lr){2-3} \cmidrule(lr){4-5} \cmidrule(lr){6-7} \cmidrule(lr){8-9} \cmidrule(lr){10-11} \cmidrule(lr){12-13}
    & \textbf{@20} & \textbf{@50} & \textbf{@20} & \textbf{@50} & \textbf{@20} & \textbf{@50} & \textbf{@20} & \textbf{@50} & \textbf{@20} & \textbf{@50} & \textbf{@20} & \textbf{@50} \\
    % \cmidrule(lr){2-5} \cmidrule(lr){6-9} \cmidrule(lr){10-13}
    % & \textbf{Recall@20} & \textbf{Recall@50} & \textbf{NDCG@20} & \textbf{NDCG@50} & \textbf{Recall@20} & \textbf{Recall@50} & \textbf{NDCG@20} & \textbf{NDCG@50} & \textbf{Recall@20} & \textbf{Recall@50} & \textbf{NDCG@20} & \textbf{NDCG@50} \\
    \midrule
    \textbf{MF}& 0.1486 & 0.2410 & 0.1073 & 0.1370 & 0.0621 & 0.1187 & 0.0483 & 0.0704 & 0.0658 & 0.1219 & 0.0430 & 0.0615 \\
    ~+\textbf{R-CE}& 0.1456 & 0.2362 & 0.1053 & 0.1343 &\underline{0.0654} &\underline{0.1239} & 0.0506 & 0.0733 &\underline{0.0716} & 0.1311 & 0.0468 & 0.0663 \\
    ~+\textbf{T-CE}& 0.1326 & 0.2197 & 0.0920 & 0.1197 & 0.0571 & 0.1113 & 0.0430 & 0.0639 & 0.0359 & 0.0812 & 0.0215 & 0.0363 \\
    ~+\textbf{DeCA}& 0.1463 & 0.2356 & 0.1068 & 0.1355 & 0.0645 & 0.1225 & 0.0502 & 0.0729 & 0.0714 &\underline{0.1312} & 0.0471 &\underline{0.0668} \\
    ~+\textbf{BOD}&\underline{0.1489} &\underline{0.2415} &\underline{0.1079} &\underline{0.1376} & 0.0654 & 0.1235 &\underline{0.0511} &\underline{0.0738} & 0.0713 & 0.1300 &\underline{0.0473} & 0.0665 \\
    ~+\textbf{DCF}& 0.1489 & 0.2413 & 0.1073 & 0.1367 & 0.0635 & 0.1208 & 0.0493 & 0.0715 & 0.0710 & 0.1297 & 0.0472 & 0.0665 \\
    \cmidrule(lr){2-13}
    ~+\textbf{PLD (ours)}&\textbf{0.1520**} &\textbf{0.2475**} &\textbf{0.1097} &\textbf{0.1404*} &\textbf{0.0677**} &\textbf{0.1264**} &\textbf{0.0527**} &\textbf{0.0755**} &\textbf{0.0769**} &\textbf{0.1379**} &\textbf{0.0513*} &\textbf{0.0713**} \\
    \multicolumn{1}{c}{Gain}& +2.04\% $\uparrow$& +2.47\% $\uparrow$& +1.71\% $\uparrow$& +1.98\% $\uparrow$& +3.46\% $\uparrow$& +2.02\% $\uparrow$& +3.09\% $\uparrow$& +2.33\% $\uparrow$& +7.36\% $\uparrow$& +5.09\% $\uparrow$& +8.46\% $\uparrow$& +6.83\% $\uparrow$\\
    \midrule
    \textbf{LightGCN}& 0.1553 & 0.2509 & 0.1142 & 0.1449 & 0.0665 & 0.1270 & 0.0516 & 0.0750 &\underline{0.0817} &\underline{0.1485} &\underline{0.0538} &\underline{0.0757} \\
    ~+\textbf{R-CE}& 0.1536 & 0.2481 & 0.1131 & 0.1434 & 0.0554 & 0.1042 & 0.0428 & 0.0617 & 0.0723 & 0.1315 & 0.0478 & 0.0670 \\
    ~+\textbf{T-CE}& 0.1146 & 0.1859 & 0.0859 & 0.1088 & 0.0532 & 0.1004 & 0.0412 & 0.0595 & 0.0674 & 0.1222 & 0.0447 & 0.0626 \\
    ~+\textbf{DeCA}& 0.1540 & 0.2495 & 0.1133 & 0.1440 &\underline{0.0678} &\underline{0.1298} &\underline{0.0526} &\underline{0.0766} & 0.0812 & 0.1480 & 0.0532 & 0.0751 \\
    ~+\textbf{BOD}&\underline{0.1560} &\underline{0.2519} &\underline{0.1154} &\underline{0.1461} & 0.0672 & 0.1280 & 0.0523 & 0.0758 & 0.0809 & 0.1475 & 0.0532 & 0.0750 \\
    ~+\textbf{DCF}& 0.1276 & 0.2072 & 0.0948 & 0.1203 & 0.0619 & 0.1180 & 0.0482 & 0.0699 & 0.0734 & 0.1342 & 0.0483 & 0.0681 \\
    \cmidrule(lr){2-13}
    ~+\textbf{PLD (ours)}&\textbf{0.1580**} &\textbf{0.2558**} &\textbf{0.1157} &\textbf{0.1472*} &\textbf{0.0693**} &\textbf{0.1325**} &\textbf{0.0538**} &\textbf{0.0783**} &\textbf{0.0837**} &\textbf{0.1516**} &\textbf{0.0551**} &\textbf{0.0774**} \\
    \multicolumn{1}{c}{Gain}& +1.23\% $\uparrow$& +1.53\% $\uparrow$& +0.32\% $\uparrow$& +0.71\% $\uparrow$& +2.31\% $\uparrow$& +2.02\% $\uparrow$& +2.44\% $\uparrow$& +2.22\% $\uparrow$& +2.43\% $\uparrow$& +2.06\% $\uparrow$& +2.47\% $\uparrow$& +2.33\% $\uparrow$\\
    \bottomrule
\end{tabular}
    }
\label{tab:performance}%
\end{table*}


\begin{table}[t]
    \centering
    \caption{Recommendation performance of different denoising methods on MIND-Large.}
    \resizebox{0.45\textwidth}{!}{

\begin{tabular}{lcccc}
    \toprule
    \multicolumn{1}{c}{\multirow{2}{*}{\textbf{Model}}} & \multicolumn{2}{c}{\textbf{Recall} } & \multicolumn{2}{c}{\textbf{NDCG} }\\
    \cmidrule(lr){2-3} \cmidrule(lr){4-5}
    & \textbf{@20} & \textbf{@50} & \textbf{@20} & \textbf{@50}\\
    \midrule
    \textbf{MF}& 0.0788 & 0.1441 & 0.0501 & 0.0710 \\
    ~+\textbf{R-CE}& 0.0790 & 0.1453 & 0.0502 & 0.0715 \\
    ~+\textbf{T-CE}& 0.0329 & 0.0788 & 0.0194 & 0.0340 \\
    ~+\textbf{DeCA}& 0.0793 & 0.1447 & 0.0507 & 0.0717 \\
    ~+\textbf{BOD}& 0.0794 & 0.1435 & 0.0516 & 0.0722 \\
    ~+\textbf{DCF}&\underline{0.0818} &\underline{0.1482} &\underline{0.0528} &\underline{0.0741} \\
    \cmidrule(lr){2-5}
    ~+\textbf{PLD (ours)}&\textbf{0.0846**} &\textbf{0.1524**} &\textbf{0.0543**} &\textbf{0.0760**} \\
    \multicolumn{1}{c}{Gain}& +3.36\% $\uparrow$& +2.85\% $\uparrow$& +2.67\% $\uparrow$& +2.58\% $\uparrow$\\
    \bottomrule
\end{tabular}
    }
\label{tab:mindl}%
\end{table}



\subsubsection{Datasets}
We utilize four widely recognized datasets: the \textbf{Gowalla} check-in dataset~\cite{liang2016modeling}, the \textbf{Yelp2018} business dataset, and the \textbf{MIND} and \textbf{MIND-Large} news recommendation datasets~\cite{wu2020mind}. The Gowalla and Yelp2018 datasets include all users, while for the MIND dataset, we sample two subsets of users, constructing MIND and MIND-Large, following~\cite{zhang2024lorec2}. Consistent with~\cite{zhang2024improving, zhangunderstanding}, we exclude users and items with fewer than 10 interactions from our analysis. We allocate 80\% of each user's historical interactions to the training set, reserving the remainder for testing. Additionally, 10\% of the training set is randomly selected to form a validation set for hyperparameter tuning. Detailed statistics for the datasets are summarized in Table~\ref{tab:datasets}.



\subsubsection{Baselines}
We incorporate various denoising methods, including four reweight-based approaches and one self-supervised method. Specifically, we evaluate R-CE, T-CE~\cite{wang2021denoising}, BOD~\cite{wang2023efficient}, and DCF~\cite{he2024double} as reweight-based methods, and DeCA~\cite{wang2022learning} as a self-supervised method.
\begin{itemize}[leftmargin=*]
   \item \textbf{R-CE}~\cite{wang2021denoising}: R-CE assigns reduced training weight to high-loss interactions.
   \item \textbf{T-CE}~\cite{wang2021denoising}: T-CE drops interactions with the highest loss values at a predefined drop rate.
   \item \textbf{BOD}~\cite{wang2023efficient}: BOD treats the process of determining interaction weights as a bi-level optimization problem to learn more effective denoising weights.
   \item \textbf{DCF}~\cite{he2024double}: DCF addresses the challenges posed by hard positive samples and the data sparsity introduced by dropping interactions in T-CE.
   \item \textbf{DeCA}~\cite{wang2022learning}: DeCA posits that clean samples tend to yield consistent predictions across different models, incorporating two recommendation models during training to better differentiate between clean and noisy interactions.
\end{itemize}

\subsubsection{Evaluation Metrics}
We adopt standard metrics widely employed in the field. The primary metrics for evaluating recommendation performance are the top-$k$ metrics: Recall at $K$ ($\mathrm{Recall}@K$) and Normalized Discounted Cumulative Gain at $K$ ($\mathrm{NDCG}@K$), as described in~\cite{zhang2023robust, he2020lightgcn, herlocker2004evaluating}. For evaluation, we set $K=20$ and $K=50$, following~\cite{wang2019neural, zhang2024improving}.

\subsubsection{Implementation Details} 
In our study, we employ two commonly used backbone recommendation models: MF~\cite{koren2009matrix} and LightGCN~\cite{he2020lightgcn}. The configuration of both denoising methods and recommendation models involves selecting a learning rate from \{0.1, 0.01, $\dots$, $1 \times 10^{-5}$\}, and a weight decay from \{0, 0.1, $\dots$, $1 \times 10^{-5}$\}. For PLD, the candidate pool size $k$ is selected from \{2, 3, 5, 10, 20\}, and the temperature coefficient $\tau$ is chosen from \{0.01, 0.05, 0.1, 0.2, 0.3, 0.4, 0.5\}. For the baselines, hyperparameter settings follow those specified in the original publications. Our implementation code is available at the following link\footnote{\url{https://github.com/Kaike-Zhang/PLD}}.





