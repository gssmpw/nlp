To investigate the causes of the overlap between normal and noisy interactions in the overall loss distribution, we conduct an experimental analysis. Using the MIND dataset~\cite{wu2020mind} as a case study, we introduce additional noise ratios of 10\%, 20\%, 30\%, and 40\% into user interactions and evaluate the impact on LightGCN~\cite{he2020lightgcn}\footnote{Similar experiments are conducted on different datasets and models, yielding consistent results. Due to space constraints, we present only the results for this configuration here.}. Detailed description of the experimental setup can be found in Section~\ref{sec:exp_setup}.

To facilitate this analysis, we introduce the following notation:
\begin{itemize}[leftmargin=*]
    \item $\mathcal{I}_{\text{normal}}$: the set of normal interactions.
    \item $\mathcal{I}_{\text{noise}}$: the set of noisy interactions.
    \item $l_{u,v}$: the loss corresponding to the interaction between user $u$ and item $v$.
    \item $\mathcal{O}$: the overlap region containing noisy interactions with lower losses and normal interactions with higher losses in the \textbf{overall loss distribution}, as depicted in Figure~\ref{fig:user_loss_org}.
    \item $\mathcal{O}_{u}$: the overlap region in user $u$'s \textbf{personal loss distribution}.
\end{itemize}
\textbf{Note that} Quartiles are used instead of max-min values to mitigate the influence of extreme values when determining overlap regions.
We further define the following sets to analyze interactions within the overlap regions:
\begin{itemize}[leftmargin=*]
    \item $\mathcal{I}^{\mathcal{G}}_{\text{normal}} = \{ (u, v) \mid (u, v) \in \mathcal{I}_{\text{normal}} \land l_{u,v} \in \mathcal{O}\}$: the set of normal interactions that fall within the overlap region of the overall loss distribution.
    % $\mathcal{I}^{\mathcal{G}}_{\text{normal}} \coloneqq \mathcal{I}_{\text{normal}} \cap \mathcal{O}$: the set of normal interactions that fall within the overlap region in the overall loss distribution.
    \item $\mathcal{I}^{\mathcal{G}}_{\text{noise}} = \{ (u, v) \mid (u, v) \in \mathcal{I}_{\text{noise}} \land l_{u,v} \in \mathcal{O} \}$: the set of noisy interactions that fall within the overlap region of the overall loss distribution.
    \item $\mathcal{I}^{\mathcal{P}}_{\text{normal}} = \{ (u, v) \mid (u, v) \in \mathcal{I}_{\text{normal}} \land l_{u,v} \in \mathcal{O}_{u} \}$: the set of normal interactions that fall within the overlap region of the personal loss distribution.
    \item $\mathcal{I}^{\mathcal{P}}_{\text{noise}} = \{ (u, v) \mid (u, v) \in \mathcal{I}_{\text{noise}} \land l_{u,v} \in \mathcal{O}_{u} \}$: the set of noisy interactions that fall within the overlap region of the personal loss distribution.
\end{itemize}



\begin{figure}
    \centering
    \includegraphics[width=0.485\textwidth]{Imgs/user_loss_org.pdf}
    \caption{Probability Distribution of losses.}
    \label{fig:user_loss_org}
\end{figure}

% \begin{table}[t]
%   \centering
%     \caption{Interaction-wise loss statistics}
%     \resizebox{0.47\textwidth}{!}{

% \begin{tabular}{cccc}
%     \toprule
%     \textbf{ Noise Ration } & \textbf{ $\vert \mathcal{U}^{\text{any}, +}_{\text{normal}} \vert$ } & \textbf{ $\vert \mathcal{U}^{\text{any}, -}_{\text{noise}} \vert$  } & \textbf{$\vert \mathcal{U}^{\text{any}, +}_{\text{normal}} \cap \mathcal{U}^{\text{any}, -}_{\text{noise}} \vert$} \\
%     \midrule
%     0.1 &19,312 &11,250 &5,943 \\
%     0.2 &23,354 &20,092 &10,254 \\
%     0.3 &23,512 &24,813 &13,181 \\
%     0.4 &23,588 &27,452 &15,079 \\
%     \bottomrule
%     \end{tabular}
%     }
%   \label{tab:inter_user}%
% \end{table}%


\begin{table}[t]
  \centering
    \caption{Statistics of overall loss distribution}
    \resizebox{0.475\textwidth}{!}{

\begin{tabular}{ccccc}
    \toprule
    \textbf{ Noise Ratio } & $\vert \mathcal{I}^{\mathcal{G}}_{\text{normal}} \vert$  & $\vert \mathcal{I}^{\mathcal{G}}_{\text{normal}} \vert / \vert \mathcal{I}_{\text{normal}} \vert $ & $\vert \mathcal{I}^{\mathcal{G}}_{\text{noise}} \vert$    & $\vert \mathcal{I}^{\mathcal{G}}_{\text{noise}} \vert / \vert \mathcal{I}_{\text{noise}} \vert$\\
    \midrule
    0.1 &152,892 &18.40\% &12,976 &15.61\% \\
    0.2 &162,700 &19.58\% &29,130 &17.52\% \\
    0.3 &162,763 &19.58\% &45,275 &18.16\% \\
    0.4 &159,454 &19.19\% &59,486 &17.89\% \\
    \bottomrule
    \end{tabular}
    }
  \label{tab:inter_user}%
\end{table}%


\textbf{Overall Loss Distribution.} The overall loss distribution consists of the loss of all interactions. For clarity, we separate the overall loss distribution into normal and noisy interaction loss distributions. Figure~\ref{fig:user_loss_org} illustrates that normal and noisy interactions exhibit significant overlap in the overall loss distribution across varying noise ratios. As shown in Table~\ref{tab:inter_user}, across different noise ratios, the values of $\vert \mathcal{I}^{\mathcal{G}}_{\text{normal}} \vert / \vert \mathcal{I}_{\text{normal}} \vert$ and $\vert \mathcal{I}^{\mathcal{G}}_{\text{noise}} \vert / \vert \mathcal{I}_{\text{noise}} \vert$ are generally high. This makes it difficult to distinguish between normal and noisy interactions based on the overall loss distribution, increasing the likelihood of denoising errors in existing methods~\cite{wang2021denoising, he2024double, gao2022selfguided, lin2023autodenoise}. Consequently, relying solely on the overall loss distribution may not be an effective approach for differentiating between normal and noisy interactions.

\textbf{User Case in Personal Loss Distribution.} To further analyze these interactions, we randomly select five users from the dataset and display their personal loss distributions across varying noise ratios. As shown in Figure~\ref{fig:user_loss}, for each user, the losses of normal interactions consistently remain lower than those of noisy interactions. However, due to significant variance in users' personal loss distributions, the overlap depicted in Figure~\ref{fig:user_loss_org} is primarily attributed to certain users exhibiting normal interaction losses that exceed other users' noisy interaction losses. For example, at a noise ratio of 0.2, the noisy interaction losses of user 2 differ from corresponding normal interaction losses but are similar to the normal interaction losses of user 5.


\begin{figure}
    \centering
    \includegraphics[width=0.485\textwidth]{Imgs/user_loss_user.pdf}
    \caption{Personal loss distribution for five users.}
    \label{fig:user_loss}
\end{figure}

\textbf{Statistics of Personal Loss Distribution.} Building on the previous user case analysis, we propose that noisy interactions can be more effectively identified by analyzing users' personal loss distributions. To further illustrate this, we examine the statistical differences between users' normal and noisy interaction losses. For each user, we compute the difference between the lower quartile of their normal interaction losses and the upper quartile of their noisy interaction losses. As depicted in Figure~\ref{fig:user_diff}, most users exhibit higher noisy interaction losses compared to their normal interaction losses, a trend that persists across all noise levels. As shown in Table~\ref{tab:inter_person}, compared to $\vert \mathcal{I}^{\mathcal{G}}_{\text{normal}} \vert$ and $\vert \mathcal{I}^{\mathcal{G}}_{\text{noise}} \vert$, $\vert \mathcal{I}^{\mathcal{P}}_{\text{normal}} \vert$ and $\vert \mathcal{I}^{\mathcal{P}}_{\text{noise}} \vert$ decrease significantly, further validating the effectiveness of personal loss distributions for distinguishing normal interactions from noisy ones. This observation offers valuable insights for potential improvements in denoising strategies.

