\begin{table}[t]
  \centering
    \caption{Method complexity comparison.}
    \resizebox{0.425\textwidth}{!}{

\begin{tabular}{ccc}
    \toprule
    \textbf{ Methods } & \textbf{Space Complexity} & \textbf{Time Complexity} \\
    \midrule
    Base & $M$ & $\mathcal{O}(N)$ \\
    T-CE & $M$ & $\mathcal{O}(N\log(N))$ \\
    BOD  & $M + d_1 \times d_2$  & $\mathcal{O}((d_1 \times d_2)N)$ \\
    DCF & $M$ & $\mathcal{O}(N\log(N))$ \\
    \textbf{PLD (ours)} & $M$ & $\mathcal{O}(kN)$ \\
    \bottomrule
    \end{tabular}
    }
  \label{tab:complex}%
\end{table}%


This section compares various reweight-based denoising methods, including T-CE~\cite{wang2021denoising}, BOD~\cite{wang2023efficient}, DCF~\cite{he2024double}, and our PLD, focusing on space and time complexities. The comparison is summarized in Table~\ref{tab:complex}.

\textbf{Space Complexity.}
The space complexity of the base model is determined by the number of parameters, denoted as \(M\). T-CE, DCF, and our PLD do not introduce any additional modules, so their space complexity remains unchanged. In contrast, BOD introduces extra components, specifically a generator and decoder (i.e., EN \(\in \mathbb{R}^{d_1 \times d_2}\) and DE \(\in \mathbb{R}^{d_2}\)), which significantly increases its complexity.

\textbf{Time Complexity.}
The time complexity of the base model is determined by the number of interactions, denoted as \(N\), resulting in a complexity of \(\mathcal{O}(N)\). Both T-CE and DCF require sorting the loss values, increasing their complexity to \(\mathcal{O}(N \log N)\). BOD needs to encode and decode the weights of each edge, leading to a time complexity of \(\mathcal{O}((d_1 \times d_2 + d_1)N)\). Our PLD introduces a resampling process, adding an additional \(\mathcal{O}(2kN)\) to the time complexity, where \(k \ll N\).

In summary, our PLD does not significantly increase the space or time complexity of the base model. Compared to other reweight-based denoising methods, our approach demonstrates clear advantages.
