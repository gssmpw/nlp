

\begin{abstract}
Resource limitations often constrain the parameter counts of Large Language Models (LLMs), hindering their performance. While existing methods employ parameter sharing to reuse the same parameter set under fixed budgets, such approaches typically force each layer to assume multiple roles with a predetermined number of iterations, restricting efficiency and adaptability.
In this work, we propose the {Zero Token Transformer (ZTT)}, which features a {head-tail decoupled parameter cycling} method. We disentangle the first (head) and last (tail) layers from parameter cycling and iteratively refine only the intermediate layers. Furthermore, we introduce a {Zero-Token Mechanism}, an internal architectural component rather than an input token, to guide layer-specific computation. At each cycle, the model retrieves a zero token (with trainable key values) from a Zero-Token Pool, integrating it alongside regular tokens in the attention mechanism. The corresponding attention scores not only reflect each layer’s computational importance but also enable dynamic early exits without sacrificing overall model accuracy.
Our approach achieves superior performance under tight parameter budgets, effectively reduces computational overhead via early exits, and can be readily applied to fine-tune existing pre-trained models for enhanced efficiency and adaptability.
\end{abstract}
