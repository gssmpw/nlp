\documentclass[10pt,conference]{IEEEtran}
\IEEEoverridecommandlockouts

\usepackage{cite}
\usepackage{amsmath,amssymb,amsfonts}
\usepackage{algorithmic}
\usepackage{textcomp}
\usepackage{xcolor}
\usepackage{xspace}
\usepackage{pifont}%
\usepackage[inkscapelatex=false]{svg}
\usepackage{booktabs}
\usepackage{listings}
\usepackage{graphicx,xfp,subcaption}
\usepackage{tabularray}
\usepackage{tabularx}
\usepackage{tcolorbox}
\usepackage{multirow}
\usepackage{caption,array}
\usepackage{url}
\usepackage{hyperref}
\usepackage{array}
\usepackage{balance}
\usepackage[normalem]{ulem}
\usepackage{pgfplots, pgfplotstable}
\pgfplotsset{compat=1.18}

\newcommand{\rowgroup}[1]{\hspace{-1em}#1}
\def\BibTeX{{\rm B\kern-.05em{\sc i\kern-.025em b}\kern-.08em
    T\kern-.1667em\lower.7ex\hbox{E}\kern-.125emX}}

\begin{figure*}
    \centering
    \includegraphics[width=1\linewidth]{bar2.pdf}
    \caption{(a) shows the bar chart of the raw data, (b) presents the results of applying Moving Average Smoothing to reduce anomalies in prediction percentages, and (c) highlights the reduction of visual clutter and emphasizes sequential behavior patterns after merging behaviors of the same category.}
    \label{fig:bar}
    \Description{(a) shows the bar chart of the raw data, (b) presents the results of applying Moving Average Smoothing to reduce anomalies in prediction percentages, and (c) highlights the reduction of visual clutter and emphasizes sequential behavior patterns after merging behaviors of the same category.}
\end{figure*}

\section{Data Collection and Processing}
\label{sec:data}
\RR{In this section, we provided an overview of the data collection context and introduced the collaborative programming performance framework along with its metric quantification methods.}

\subsection{Data Collection}
We collaborated with Professor E1, an expert in programming education, and teaching assistants (TA1 and TA2), experienced in Python, to collect data from E1's Spring 2023 Python course with 66 non-computer science freshmen in 22 groups. Using non-intrusive methods, we recorded group discussions, screen activities (without audio), and code submissions. Session lengths ranged from 10 to 60 minutes based on question completion. 
Due to data quality issues, we selected data from 19 groups (57 students) for analysis.


\subsection{Data Preprocessing}
In collaborative programming analysis, students' spoken content was key to understanding discussion and evaluating collaboration. We used the Faster-Whisper model~\cite{fasterwhisper} for speech recognition and the Pyannote-audio model~\cite{pyannoteaudio} for speaker diarization. 
For groups lacking clear problem-solving strategies, we used Tesseract OCR~\cite{tesseract} to analyze screen recordings and extract key frames through screenshots.

\subsection{Scope of Collaborative Programming Performance Framework}
Evaluating student and group performance in collaborative programming required considering multiple dimensions~\cite{hawlitschek2023empirical}.  
Building on literature and expert input (E1), we proposed the following comprehensive analytical framework to assess performance. 



\subsubsection{Student Performance Assessment}
\label{shema}
Previous research demonstrated that students' skills, backgrounds, and personalities in the classroom vary significantly, affecting their engagement and learning outcomes~\cite{wu2019analysing}. 
Therefore, we focus on each student's \textit{background} (prior academic performance and major), \textit{role transitions}, \textit{behavioral engagement}, and \textit{cognitive engagement}.






\textbf{Problem-solving Categorization:}
Based on previous frameworks~\cite{wu2019analysing}, team theory~\cite{zhao2023analysing}, and collaborative coding processes~\cite{sun2021three}, we developed a coding scheme (Fig.~\ref{fig:scheme}) to capture group problem-solving in collaborative programming. 
The scheme used four color-coded categories to represent discussion types. 
The first three categories followed a hierarchical structure, indicating discussion depth, while the green category focuses on situation awareness and specific behaviors.

Building on the scheme, we used tailored prompts with the ChatGPT-4o model~\cite{gpt4o} to classify behavioral patterns in transcribed dialogue \RR{(More details are in appendix B)}. 
\RR{The model provided a prediction percentage of uncertainty for each classification, improving result interpretability. }
To minimize anomalies, we applied a ``moving window'' technique with Moving Average Smoothing~\cite{chang2022muse}, stabilizing prediction percentages (Fig.\ref{fig:bar}-b). To reduce visual clutter in long time-series data, we aggregated consecutive instances of the same category, averaging prediction percentages (Fig.\ref{fig:bar}-c). These results were displayed in the timeline panel's progress bar, enabling detailed analysis by zooming into specific behavior categories in Sec.~\ref{barchart}. 




\textbf{Roles Extraction:}
We analyzed each speaker's dynamic roles (Driver, Navigator, and Monitor) during programming~\cite{lewis2011pair}. Using ChatGPT-4o and prompts based on the Thought Chain Model~\cite{wei2022chain}, we guided the model through step-by-step reasoning to generate role classifications. Prompts were iterated for clarity, and the model's responses were structured hierarchically and returned in JSON format. Each query was repeated ten times, with the majority result adopted for classification.

\RR{\textbf{Behavioral Engagement:} reflected the level of effort and participation students invested in learning~\cite{fredricks2022measurement}. 
In our study, we focused on the duration and frequency of student speech.} 
We extracted conversation data, excluding irrelevant chat, and divided each conversation into two parts: the first half and the full conversation. We then measured speaking duration, frequency, and degree centrality using co-occurrence networks~\cite{ng1999toward}. For each question, we created and normalized two networks, followed by Non-negative Matrix Factorization (NMF)~\cite{lee2000algorithms} to identify key behavioral patterns for dynamic group comparison.


\RR{\textbf{Cognitive Engagement:} referred to the cognitive investment students made in their learning. We highlighted the role changes and behavior frequencies of students during the collaborative process. }
To capture dynamic changes in student cognitive engagement, we split the dialogue for each question into two segments: the first half and the full dialogue. We extracted the frequency of each speaker's 14 behavioral categories and their roles at each timestamp. After normalizing these features for consistency, we applied NMF to reduce dimensionality and assess each speaker's cognitive engagement.

\begin{figure*}
  \includegraphics[width=\textwidth]{CPVis.pdf}
  \caption{\RR{A screenshot of Group 10 view.} \textit{CPVis} applies multimodal learning analysis to provide instructors with evidence for evaluating group and student performance. It consists of three views:
Filter View (A) Provides an overview and allows group selection. The selected groups appear in the lasso selection area (A2), and the similarity panel (A3) displays the most similar and different groups based on the search (A1a).
Content View (B) Displays group performance, with the B1 panel showing completed codes, the B3a panel illustrating the behavior sequence, and the B3b panel showing student engagement over time.
Detail View (C) Presents the group's collaborative programming video (C1) and raw conversation data (C2).}
  \Description{A screenshot of Group 10 view. \textit{CPVis} applies multimodal learning analysis to provide instructors with evidence for evaluating group and student performance. It consists of three views:
Filter View (A) Provides an overview and allows group selection. The selected groups appear in the lasso selection area (A2), and the similarity panel (A3) displays the most similar and different groups based on the search (A1a).
Content View (B) Displays group performance, with the B1 panel showing completed codes, the B3a panel illustrating the behavior sequence, and the B3b panel showing student engagement over time.
Detail View (C) Presents the group's collaborative programming video (C1) and raw conversation data (C2).}
  \label{fig:teaser}
  \end{figure*}

\subsubsection{Group Performance Assessment}
We evaluated group performance based on three dimensions: code quality, collaborative problem-solving, and teacher scaffolding. 
Through in-depth discussions with domain experts, we assessed how each dimension was valued and measured in the context of our study.




\label{code}
\textbf{Code quality}, reflecting students' mastery of course concepts, was a key metric for evaluating group performance. To assess student submissions, we used ChatGPT-4o~\cite{gpt4o} to evaluate dimensions such as problem-solving, code integrity, accuracy, and algorithmic innovation, scoring each on a 1–5 scale. After refining evaluation prompts, we ran the assessment ten times per submission, averaging the results to ensure consistency and reliability.





\textbf{Collaborative Problem-Solving (CPS):} 
Earlier studies categorized CPS into team effectiveness and task effectiveness~\cite{rosen2020towards}. Team effectiveness was measured by student engagement, while task effectiveness was assessed through code quality. %Our analysis captured problem-solving behaviors by frequency and sequence.
To evaluate CPS, we examined task effectiveness, represented by the average question score (\(\bar{s}\)), and team effectiveness, assessed through the standard deviation of engagement (\(\sigma_e\)) and the average engagement score (\(\bar{e}\)) as shown in Equation \ref{eq:1}. We then used the coefficient of variation (\(CV_e\)) \RR{to account for both engagement variability and engagement}. Finally, the overall collaboration quality was calculated using Equation \ref{eq:2}, combining question performance and engagement balance. 
\begin{equation}
\sigma_e = \sqrt{\frac{1}{n} \sum_{i=1}^{n} (e_i - \bar{e})^2}, \quad CV_e = \frac{\sigma_e}{\bar{e}}
\label{eq:1}
\end{equation}

\begin{equation}
Quality = \bar{s} \cdot (1 - CV_e)
\label{eq:2}
\end{equation}
As shown in Table \ref{table:comparison}, Group 19, despite achieving a respectable average score, exhibited imbalanced engagement, leading to a lower collaboration quality score. In contrast, Group 20 demonstrated more balanced and higher engagement, resulting in a better overall collaboration quality.
\begin{table}[htbp]
\centering
\begin{tabular}{cccccc}
\toprule
\textbf{Group} & \(\bar{s}\) & \textbf{Engagement Levels} & \(\sigma_e\) & \(\text{CV}_e\) & \textbf{CQ} \\
\midrule
Group 19 & \(4.11\) & (10.515, 9.725, 4.575) & \(2.80\) & \(0.24\) & \(2.80\) \\
Group 20 & \(4.14\) & (10.06, 9.32, 8.62) & \(0.73\) & \(0.08\) & \(3.88\) \\
\bottomrule
\end{tabular}
\caption{Comparison of Group 19 and Group 20 on Collaboration Quality (CQ).}
\label{table:comparison}
\end{table}

\textbf{Teacher Scaffolding,} categorized into cognitive (low, medium, high-control) and metacognitive forms~\cite{ouyang2022applying}, reflected the level of support provided to a group and its impact on programming performance. We evaluated four scaffolding dimensions, leveraging GPT-4o for annotation. By using targeted prompts and examples, we improved classification accuracy, while teacher scaffolding was categorized according to the type of support based on a semantic analysis of interactions.



\begin{document}


\NewDocumentCommand{\codeword}{v}{%
\small{\texttt{{#1}}}%
}


\newcommand{\ie}{\emph{i.e.,}\xspace}
\newcommand{\eg}{\emph{e.g.,}\xspace}
\newcommand{\cmark}{\ding{51}}%
\newcommand{\xmark}{\ding{55}}

%%%%% THIS PART HAS BEEN MOVED TO `macros.tex` %%%%%
\newcommand{\thought}[1]{{\color[rgb]{0.2,0.39,0.66}(#1)}}
\newcommand{\todo}[1]{{\color[rgb]{1.0,0.0,0.0}(#1)}}
\newcommand{\hsh}[1]{{\color{green!50!black} Henrik: #1}}
\newcommand{\st}[1]{{\color{red!50!black} Sebastian: #1}}

\newcommand{\ulm}[1]{_{\scaleto{\mathrm{#1}}{3pt}}}
\newcommand\at[2]{\left.#1\right|_{#2}}











\newtheorem{assumption}{Assumption}

\DeclareMathOperator*{\argmax}{arg\,max}
\DeclareMathOperator*{\argmin}{arg\,min}

\newcommand{\swname}[1]{\texttt{#1}}
\newcommand{\ie}{i\/.\/e\/.,\/~}
\newcommand{\eg}{e\/.\/g\/.,\/~}
\newcommand{\cf}{cf\/.\/~}

\newcommand{\fig}{Fig\/.\/~}
\newcommand{\defn}{Def\/.\/~}
\newcommand{\sect}{Sec\/.\/~}
\newcommand{\tabl}{Tab\/.\/~}
\newcommand{\algo}{Algorithm~}
\newcommand{\theo}{Theorem~}

\newcommand{\bnnl}{3 hidden layers}
\newcommand{\bnnn}{50 neurons}
\newcommand{\bnna}{tanh activations}

\newcommand{\capt}[1]{\mdseries{\emph{#1}}}

\newcommand{\videolink}{at \url{https://youtu.be/_d7AqTRjz6g}}
\newcommand{\codelink}{\url{https://github.com/wheelbot/mini-wheelbot}}

\newcommand{\fakepar}[1]{\vspace{0mm}\noindent\textbf{#1.}}

\newcommand{\needref}{\textcolor{red}{[REF]}}

\newcommand{\plotfontsize}{9pt}

%%%%%%%%%%%%%%%%%%%%%%%%%%%%%%%%%%%%%%%%%%%%%%%%%%%%


\newcommand{\gulzar}[1]{\textcolor{red}{G: #1}}
\newcommand{\tien}[1]{\textcolor{blue}{T: #1}}
\newcommand{\question}[1]{\textcolor{brown}{Question: #1}}
\newcommand{\waris}[1]{\textcolor{orange}{W: #1}}
\newcommand{\edit}[1]{\textcolor{cyan}{#1}}

\newcommand{\todo}[1]{\textcolor{green}{TODO: #1}}

\title{Are the Majority of Public Computational Notebooks Pathologically Non-Executable?}

\author{\IEEEauthorblockN{Tien Nguyen, Waris Gill, Muhammad Ali Gulzar}
\IEEEauthorblockA{Virginia Tech\\
Blacksburg, USA \\
\{tiennguyen, waris, gulzar\}@vt.edu}
}

\maketitle
\begin{abstract}
Retrieval-Augmented Generation (RAG) is often used with Large Language Models (LLMs) to infuse domain knowledge or user-specific information. In RAG, given a user query, a retriever extracts chunks of relevant text from a knowledge base. These chunks are sent to an LLM as part of the input prompt. Typically, any given chunk is repeatedly retrieved across user questions. However, currently, for every question, attention-layers in LLMs fully compute the key values (KVs) repeatedly for the input chunks, as state-of-the-art methods cannot reuse KV-caches when chunks appear at arbitrary locations with arbitrary contexts. Naive reuse leads to output quality degradation.  This leads to potentially redundant computations on expensive GPUs and increases latency. In this work, we propose \sys, a system for managing and reusing precomputed KVs corresponding to the text chunks (we call \textit{chunk-caches}) in RAG-based systems. We present how to identify \hl{\textit{chunk-caches} that are reusable}, how to efficiently perform a small fraction of recomputation to \textit{fix} the cache to maintain output quality, and how to efficiently store and evict \textit{chunk-caches} in the hardware for maximizing reuse while masking any overheads. With real production workloads as well as synthetic datasets, we show that \sys reduces redundant computation by \textbf{51\%} over SOTA prefix-caching and \textbf{75\%} over full recomputation.
\hl{Additionally, with continuous batching on a real production workload, we get a \textbf{1.6$\times$} speedup in throughput and a \textbf{2$\times$} reduction in end-to-end response latency over prefix-caching while maintaining quality, for both the \llama-3-8B and \llama-3-70B models. 
}
\end{abstract}





\begin{IEEEkeywords}
Computational notebooks; non-executability; restoration; mining software repositories
\end{IEEEkeywords}

\section{Introduction}
Graph Neural Networks (GNNs) have emerged as the defacto approach for machine learning over graph-structured inputs~\cite{chami2021machine}; GNN-based models are currently used in navigation apps~\cite{derrow2021eta}, to predict protein structures~\cite{jumper2021highly}, and to create weather forecasts (GraphCast~\cite{lam2022graphcast}). These impressive results, however, require training GNNs over massive amounts of graph data. For example, GraphCast was trained on 53TB over four weeks using 32 Cloud TPU v4 nodes (10/2024 est. cost: \$70K), limiting the development of such a model to those with sufficient resources. 

Motivated by the above, this work focuses on scalable, cost-effective, distributed GNN training over large graphs using common cloud offerings. While recent works~\cite{salient++, distDGL, distdglv2} have sought to address this need, we find that existing pipelines face scalability challenges when graphs have billions of nodes or edges and when training with multiple GPUs. These challenges arise from the unique properties of the GNN workload itself.

In particular, distributed GNN training necessitates that the graph is partitioned across machines; yet, the partitioning has a direct impact on the subsequent training efficiency, as GNN systems must communicate across machines to sample the neighborhood of graph nodes~\cite{shao2024distributed}. This communication can be reduced using \textit{min-edge-cut partitioning} algorithms that minimize the number of edges with endpoints in different partitions (machines) (called \textit{cut edges}). Thus, min-edge-cut partitioning is widely used in GNN systems, and has been shown to lead to an order of magnitude faster training compared to random partitioning~\cite{merkel2023experimental, distdglv2}. 

Min-edge-cut partitioning, however, becomes increasingly expensive with graph size. For instance, many systems utilize the offline algorithm METIS~\cite{karypis1997metis} due to its ability to effectively minimize edge cuts by iteratively refining partitions across the whole graph and its comparatively efficient implementation~\cite{merkel2023experimental, shao2024distributed, lin2023comprehensive}; yet, METIS takes 8000s and requires a special machine with 630GB of memory to partition a common benchmark graph (the 1.6B edge OGBN-Papers100M), whereas GNN training takes only 549s (10 epochs, one GPU) and can run on cloud machines with 244GB of memory~\cite{mariusgnn} (details in Section~\ref{sec:eval}). Although the partitioning overhead can be amortized across models, it still presents a bottleneck to GNN training. To address this issue, streaming algorithms iterate over the graph and assign vertices to partitions greedily~\cite{abbas2018streaming}. While these algorithms offer improved scalability, they tend to result in more edge cuts than offline methods~\cite{zhang2018akin}; e.g., we find a streaming greedy approach cuts up to 4$\times$ more edges than METIS.

In this work, we introduce Armada, a new end-to-end system for large-scale distributed GNN training that aims to address the bottleneck of partitioning in existing GNN pipelines. To overcome this challenge, Armada's key contribution is a novel memory-efficient min-edge-cut partitioning algorithm called \partitioning (Greedy plus Refinement for Edge-cut Minimization). \partitioning can efficiently scale to massive graphs on common hardware by processing streaming chunks of graph edges, yet it still returns partitions with edge cuts comparable to METIS. For example, in the same setting in which METIS requires 8000s and 630GB, \partitioning can partition the graph with similar edge cuts in 175s using 9.3GB.

\partitioning's partitioning algorithm builds on existing streaming greedy approaches. 
Specifically, \partitioning iterates over the graph edges in chunks and greedily assigns the vertices in each chunk to partitions. The key idea behind \partitioning, however, is that it allows prior vertex assignments to be modified throughout the process, rather than freezing them after an initial greedy selection (as in existing algorithms~\cite{abbas2018streaming}). This approach, inspired by offline algorithms, refines the partitioning by leveraging lightweight statistics accumulated during streaming (these statistics provide estimates of the number of neighbors per node in each partition).

We analyze theoretically \partitioning's expected number of edge cuts versus chunk size, providing insight into its expected behavior. This analysis, confirmed by experiments, shows that refinement is critical for minimizing edge cuts when using small chunk sizes (e.g., $\le$10\% of the edges) and thus for minimizing \partitioning's computational requirements (which are proportional to chunk size): We show that \partitioning with a chunk size of 10\% and METIS cut a similar number of edges, but \partitioning does so with 8$\times$ less memory and runtime (see Section~\ref{sec:eval}). \partitioning even achieves comparable results with a chunk size of 1\%, leading to further reductions and enabling \partitioning to partition the largest public graphs (e.g., Hyperlink-2012~\cite{hyperlink}; 3.5B nodes, 128B edges) with only 500GB of memory.

Given a partitioned graph, Armada's second main contribution is the introduction of a new distributed architecture, that disaggregates the CPU resources used for neighborhood sampling from the GPU resources used for model computation, in order to achieve scalable, memory-efficient, and cost-effective GNN training on common hardware. Concretely, Armada consists of: 1) A partitioning layer that implements \partitioning. 2) A storage layer to store the partitioned graph, implemented over cheap disk-based storage. 3) A distributed mini batch preparation layer consisting of a set of workers running on cheap CPU-only machines; workers read graph partitions from storage and prepare batches (i.e., perform neighborhood sampling) for training. 4) A distributed model computation layer that utilizes a set of GPU machines to perform training over the prepared batches.

We chose a disaggregated architecture to optimize resource utilization during training. On common cloud machines, we find that even with zero communication, mini batch preparation can be up to an order of magnitude slower than mini batch computation (Figure~\ref{fig:armada_breakdown}). Disaggregation allows Armada to overcome this imbalance. By independently scaling the batch preparation layer, we can ensure that GPUs in the computation layer remain saturated during training. In contrast, existing systems, which rely only on the fixed set of CPU resources attached to the GPU machines used for training to prepare batches, are unable to parallelize mini batch preparation and suffer from sublinear speedups as compute resources are scaled. For example, on a cloud GPU machine, we find that two SoTA systems~\cite{salient++, distDGL} yield only 4.3$\times$ and 1.7$\times$ speedup when using eight instead of one GPU (Table~\ref{tab:runtime_nc} left). Sublinear speedups lead to higher than necessary total training cost and runtime over massive graphs, as expensive compute resources sit idle. Yet in the same setting, \systemname achieves a 7.5$\times$ speedup with eight instead of one GPU.

Despite the flexibility of disaggregation, challenges arise due to the communication overhead between various components. Thus, we carefully design Armada with a focus on minimizing communication between and within layers. In particular, Armada includes two optimizations to reduce the data sent between batch preparation and compute workers: 1) batch workers group mini batches destined for different GPUs on the same compute worker and transfer them together, rather than independently, in order to enable greater compression (mini batch grouping), and 2) compute workers in Armada maintain a cache of frequently accessed data in their local CPU memory (feature caching). Together, these optimizations enable Armada to scale each layer in the architecture independently without communication bottlenecks.

We evaluate Armada's disaggregated architecture for GNN training and compare against existing SoTA systems. Using popular GNN architectures, we show that while existing systems scale sublinearly, Armada does not, leading to runtime improvements up to 4.5$\times$ and monetary cost reductions up to 3.1$\times$ compared to existing systems.

\section{Motivation}
\label{sec:motivation}


\begin{figure*}[ht!] % Figure 2
\centerline{\includegraphics[width=0.8\textwidth]{images/full_workflow.pdf}}
    \caption{LLM-based error-driven notebook executability analysis and restoration workflow.}
    \label{fig:main_workflow}
    % \vspace{-4ex}
\end{figure*}


This section presents two case studies demonstrating that seemingly non-executable notebooks can potentially be restored with minor reconfiguration, and pathologically non-executable notebooks still offer valuable partially executable code. 


\subsection{Case Study 1: Restoring Complete Executability}
\label{case_study_1}

    The notebook {\small{\texttt{random\_forest\_algorithm.ipynb}}} \cite{girlscript2024} implements the random forest classifier algorithm. This notebook is hosted by the GirlScript Foundation's GitHub open-source repository ``Winter of Contributing" with 881 stars. It consists of 24 code cells. When this notebook is executed as is, it results in a ``FileNotFound" error in cell 2, attempting to read input file {\small{\texttt{Social\_Network\_Ads.csv}}}. This input file is accessible in the original author's environment, suggesting that the author intended this notebook to be executable before uploading it to GitHub. However, this input file is neither included in the repository nor available in the adopter's environment. Prior studies \cite{Pimentel2019, Pimentel2021} would classify this notebook as non-executable. This classification is analogous to considering ``{\small{\texttt{javac}}}" non-executable if no input {\small{\texttt{.java}}} file is provided. This notebook aims to demonstrate the logical steps required to create a classifier. Our observation is that for such a notebook, improving executability can help serve the notebook's purpose, which can be done by generating a synthetic input file. To do so, we package the notebook's code, Python error description, and the notebook's documentation into a prompt and query Llama-3 to generate an input file with a correct relative path to the notebook with synthetic content. This synthetically generated input file results in full executability of all 24 cells, i.e., improving the executability of the previously non-executable notebook by over 95\% as illustrated in Figure \ref{fig:cs1-full-exec}. This is a prime example of how non-executability is often over-classified when the input file is not provided but may be available locally to the original author. 
    

\subsection{Case Study 2: Improving Partial Executability}
\label{case_study_2}

    The notebook {\small{\texttt{DinosaurusIsland--Character level language model final-v3.ipynb}}} from GitHub repository ``deep-learning-coursera" \cite{gemaatienza} is demonstrating a deep learning tutorial. It has 129 stars and a total of 14 code cells. When executed in linear order, the notebook first encounters a ``ModuleNotFound" error in cell 1 due to the absence of module {\small{\texttt{utils}}}. We resolve this error by installing the required module. A follow-up execution raises a ``NameError" in cell 6 due to the undefined function {\small{\texttt{softmax()}}}. After reviewing the notebook, we noted that the markdown cell claimed the function was provided, but it is missing. The original notebook shows a valid output, meaning the author successfully executed it. This is a common case where program states are shared between different executions, even if the corresponding code is moved or deleted. Even if the notebook is executed end-to-end before pushing on GitHub, the error would not have surfaced as the function definition is still present in the current Python kernel. However, such program states are not shared with the notebook itself, resulting in a ``NameError" in a new Python Kernel. 
    
    Our analysis indicates that defining the undefined name with an appropriate implementation could significantly improve the notebook's executability. Therefore, we prompt Llama-3 with the error report and the code cells to generate a definition for this function and insert a new code cell containing the definition right before its usage. This LLM-generated code cell requires a module import for {\small{\texttt{tensorflow}}}. Again, we were able to install it in the environment. These simple addition and module installations improve the original notebook's executability by 7 cells until it encounters ``AttributeError" at cell 8. This is a tutorial notebook, and the first half illustrates valuable deep-learning knowledge in sequence models. Thus, partial executability offers added value.



\section{Research Questions}
\label{research questions}

Under the traditional definition of executability, a notebook is considered \textbf{executable} if it does not trigger any error throughout its complete top-down execution. A \textbf{non-executable} notebook, on the other hand, fails to execute due to an error or exception. In this paper, we relax this notion of executability by dividing it further into {\bf pathologically non-executable}, {\bf non-executable but restorable}, and {\bf executable}. When a notebook's executability is hindered by unresolvable errors, e.g., syntax or indentation errors, it is \textbf{pathologically non-executable}. If a notebook is executable in the original author's execution environment but fails to execute in another environment due to issues such as missing input files, execution orders, or execution environments, it is called a  {\bf non-executable but restorable} or {\bf misconfigured} notebook which is not executable but can be fully or partially restored. 

Similar to different notions of executability, we also introduce different degrees of executability. Instead of considering it as a binary metric, we measure executability on a continuous spectrum, defined as a ratio between the number of cells executed until the first error and the total number of cells in a notebook. 100\% executability refers to fully executable notebook, whereas $<$100\% refers to partially executable notebooks. A fully executable notebook does not guarantee reproducibility, i.e., a notebook must produce the same outputs as the original notebook. Reproducibility is beyond the scope of this work and is often not required for data-centric notebooks as they are expected to produce different outputs on different data. We explore the following research questions:

\begin{itemize}
    \item \textbf{RQ1:} What are the common causes of non-executability in notebooks?
    \item \textbf{RQ2:} How many non-executable notebooks are pathologically non-executable, and how many can be restored?
    \item \textbf{RQ3:} To what extent can pathologically non-executable notebooks be executed?
    \item \textbf{RQ4:} Can LLM-based restoration strategies enhance notebook executability?
\end{itemize}







\section{Empirical Analysis Methodology}
\label{sec:approach}



Figure \ref{fig:main_workflow} (a) illustrates the intermediate stages of our analysis process. Starting with a repository, we retrieve all notebooks and conduct a lightweight static error check to identify issues that may impede notebook execution. Next, we automatically analyze the repository to extract any provided environment requirement files, enabling us to configure the appropriate environment for notebook execution. During execution, we log the first-encountered error and incrementally apply targeted restoration strategies based on the error type, measuring executability improvements at each stage. 


%%%%%%%%%%%%%%%%%%%%%%%%%%%%%%%%%%%%%%%%%%%%%%%%%%%%%%%%%%%%
\subsection{Analysis Approach}
\label{analysis_approach}



    \subsubsection{Initial Error Checking}
        For a notebook to be executable, it first needs to be compilable. As a first step, we look for compilation errors by applying a lightweight error checking on all notebooks within each repository to uncover any compiling issues, such as syntax or indentation errors. This preliminary filtering step assesses executability without running the code, allowing us to exclude notebooks that require extensive code-level modifications for restoration and are thus deemed pathologically non-executable. To access the content of the notebooks, we use the {\small{\texttt{nbformat}}} module. While performing static analysis of the notebook's code, we detect and omit notebooks from further investigation that (1) cannot be read due to corruption and encoding problems, (2) contain no code cells (with or without markdown cells), or all code cells do not hold any content; and (3) written in programming languages other than Python 3 (such as Python 2, Julia, R, or C). 
    

    \subsubsection{Dynamic Error Checking} 
        \label{dynamic-error-checking}
        In step \ding{203} of Figure \ref{fig:main_workflow} (b), we perform dynamic error checking to detect and categorize potential runtime errors. We use Papermill \cite{papermill}, which allows us to execute the notebook and detect and report any errors encountered during execution. We use Python 3 as the execution kernel and document the first error that halts the execution. We categorize the execution status of each notebook into the following groups:
    
        \begin{enumerate}
            \item \textbf{Executable:} The notebook is fully executable without encountering errors.
            \item \textbf{FileNotFound:} The notebook writes to or reads from a file or directory that is unavailable during execution.
            \item \textbf{ModuleNotFound:} The notebook imports unavailable modules, packages, or libraries.
            \item \textbf{NameError:} The notebook uses a variable, function, or class without defining or importing it.
            \item \textbf{Others/Non-Fixable:} The notebook encounters errors not covered by the above categories. We also identify and categorize these errors accordingly.
        \end{enumerate}

        \begin{figure}[t!] % Figure 3
            \includegraphics[width=\columnwidth]{images/def_use.pdf}
            \caption{Def-use lists for the first two cells in notebook \cite{pytopia}.}
            \label{fig:def-use}
        \end{figure}

        \noindent The categories reflect a natural progression of notebook adoptions and fixes, as shown in Figure \ref{fig:main_workflow}. 
        %This begins with checking executability and then categorizing non-executable notebooks into fixable and non-fixable errors. “Fixable errors” are further categorized into three subcategories based on the most common issues identified in prior literature \cite{Pimentel2019}.
        If a notebook is found fully executable after this execution process, we mark it as executable. If it is non-executable due to missing input files, required modules, or undefined names, we iteratively apply restoration strategies (Figure \ref{fig:main_workflow} (b)) and perform dynamic error checking to remove as many executability issues as possible.

        
        For notebooks with ``NameError," we perform a static {\em def-use} analysis to localize the root cause behind undefined names. Like traditional {\em def-use} analysis, we build a specialized AST visitor that tracks variable definition and usage with precise scope awareness within each cell. Figure \ref{fig:def-use} shows how \textit{def} and \textit{use} sets look for two notebook cells. Note that while maintaining the definition set, we consider variable binding, access, and scoping rules for Python to avoid any false positives and false negatives during the analysis. Our definition set extends beyond variables to include imports, functions, classes, and control flow elements. 
        
        Next, we use {\em def-use} sets to isolate the location of undefined names in the notebook and search nearby cells to locate the definition. Finally, the framework returns any detected definition of the undefined variable and its cell location. If the variable's definition appears later in the notebook than its initial use, we categorize the notebook as {\em defined after use} and return the cell number where the definition is found. If no definition exists throughout the notebook, we classify it as {\em undefined}. 

        % The framework first analyzes the entire notebook using the AST visitor to create detailed maps of all variable definitions and uses across all cells. For each cell, it keeps track of both global and local scope variables. When it finds variable uses and definitions, it records not just the cell number but also the specific scope within that cell. The framework then attempts to locate the cell where the given undefined variable might be defined in the subsequent cells. 
        %For each potential definition, we check if that definition would be accessible from where the variable is used while following Python's scoping rules. For example, a variable defined inside a function is not accessible to code outside that function.     
        

    %\tien{To accomplish this, each cell is analyzed to generate a set of \textit{definitions} (def), comprising variables (or names) that are defined within the cell. To construct this set, we recursively traverse the AST nodes, extracting all instances of {\small{\texttt{AST.Name}}} referring to variables. When encountering a node of type {\small{\texttt{Name}}}, we determine its context within the notebook. If the context is {\small{\texttt{Store()}}}, indicating the definition of a variable, we add it to the definition set. Note that while maintaining the definition set, we consider variable binding, access, and scoping rules for Python to avoid any false positives and false negatives during the analysis.}
    
    %In other programming files, certain errors can be easily detected during the compilation phase. Similarly, in notebooks, we can statically identify some issues without the need for execution. 
    %For a notebook to be executable, it first needs to be compilable. As a first step, we look for compilation errors by applying static error checking on all notebooks to uncover any compiling issues, such as syntax errors or indentation errors. Additionally, we can detect runtime NameError instances by examining the definition and usage \cite{Wang2021, Chen2014, Ryder1994} of variables within a notebook to pinpoint any undefined variables. This preemptive process aids in assessing executability early without executing the code. Notebooks that fail this early screen fall into pathologically non-executable notebooks since restoring them requires a code-level rewrite that may change the semantics of the original notebook.  


    %Next, in \ding{203} of Figure \ref{fig:main_workflow}, we use the AST of each cell to identify potential NameError issues arising from undefined variable use. To accomplish this, we first identify all the definitions and usage instances of variables (or names). Each cell is analyzed to generate two distinct sets: the \textit{definition} (def), comprising variables that are defined within the cell, and the \textit{use}, containing variables that are referred to in the cell. Figure \ref{fig:reorder_workflow} shows how def and use sets look like for two cells in a public notebook. To construct these two sets, we recursively traverse the AST nodes, extracting all instances of \codeword{AST.Name} referring to variables. When encountering a node of type \codeword{Name}, we determine its context within the notebook. If the context is \codeword{Store()}, indicating the definition of a variable, we add it to the definition set. Conversely, if the context is \codeword{Load()} or \codeword{Del()}, signifying the usage of variable values, we append it to the use set. Note that while maintaining def-use sets, we consider the variable binding, access, and scoping rules for Python to avoid any false positives and false negatives during error checking. 

    %\tien{Our definition set extends beyond variables to include imports, functions, classes, and control flow elements. For imports, we incorporate module names and aliases to the definition set for potential use elsewhere in the notebook. Similarly, we extract function names and argument lists from different types of defined functions (regular, async, and lambda). Class names are also added to class definitions. In control flow constructs, such as loops, we include variables referenced in conditions or defined within the loop body. This comprehensive definition set supports the accurate identification of dependencies throughout the notebook.}
    
    %We expand our def and use set beyond variables. For import statements, including attributes indicating module names and their aliases, we add them to the set as potential future usage within the notebook. Similarly, for function definitions (including regular, async, and lambda functions), we extract their names and argument lists, including them into the use set. We exclusively gather user-defined names, omitting built-in functions to avoid triggering NameError messages. Class names are extracted from class definitions and appended to the use set as well. In control flow constructs such as for loops or while loops, we include variable names in conditional statements and those defined within the body in the definition set. Variables utilized within the body are added to the use set.

    
    % Considering the scope of each name or variable is crucial, as variables can be defined and used in different scopes, ranging from global to more local scopes. Variables defined in outer scopes can be accessed in inner scopes or the same scope, whereas variables defined within a local scope cannot be accessed outside of that scope. For instance, functions cannot be utilized outside of their respective functions, and variables defined within a class can be accessed within functions within that class or along with the class. Consequently, for each extracted name, a scope is assigned. Function definitions, class definitions, and control flow constructs narrow the scope of variables or names defined or used within their respective bodies.

    %\tien{Given the undefined variable and its location in the notebook, we examine the cells to locate its definition using the established definition list, noting the position of the first occurrence. If the variable's definition appears later in the notebook than its initial use, this suggests a possible execution order issue due to disordered cells. In such cases, we categorize the notebook under “defined after use” and return the cell number where the definition is found. If no definition exists throughout the notebook, we classify it as “undefined.” This categorization guides our approach in the restoration stage.}

    %Assuming the top-down execution of the notebook, we use def-use sets of notebook cells to identify any undefined names, classifying them based on their first encounter: undefined in the entire notebook, defined after their usage, or both. If a variable is in a def set of current or any of the prior cells, we conclude that no NameError occurs for that variable. If a variable is not in any of those sets, we explore whether it is defined after that cell, indicating a potential issue due to disordered cell order. If the name is defined in any subsequent cell, we categorize the notebook into the "defined after use" group. Otherwise, it is placed in the "undefined" group.  When a cell contains undefined and defined-after variable names, we assign that notebook to the "both" category. This classification approach ensures thorough identification and categorization of potential NameError issues within the notebook. At the end of static error checking, we categorize notebooks as pathologically non-executable, non-executable but restorable, and potentially executable notebooks.   
    
    %For each cell in a notebook, we extract its usage list and compare each name against a combination of names defined prior to that cell. This combination comprises all names defined within the scope of usage and outer scopes within the same cell, as well as the definition lists of predecessor cells at scope 0. Why specifically scope 0? Considering that if a variable is defined in a deeper inner scope of a cell, the subsequent cell cannot access that name within a similar inner scope. Therefore, we only need to verify against the list of definition names at scope 0 of the cells that follow.    

    %Through static analysis, we can systematically categorize notebooks based on identified static errors, allowing for incremental refinement of our understanding of potential issues. This methodical step provides a structured framework for addressing errors and ensures a comprehensive examination of notebooks prior to dynamic analysis.

    % After this process, we collected a total of 22,998 notebooks containing NameError. Among those, 18,633 have at least one undefined variable in the first error encounter, 3,828 have variables defined after their usage, and 537 have both. 

    
%%%%%%%%%%%%%%%%%%%%%%%%%%%%%%%%%%%%%%%%%%%%%%%%%%%%%%%%%%%%%%%%%%%%%%

\begin{figure*}[t!] % Figure 4
        \centerline{\includegraphics[width=1\textwidth]{images/LLM_Prompts.pdf}}
        \caption{Examples of prompts for LLM and responses for different error types. }
        \label{fig:LLM-prompts}
        % \vspace{-3ex}
\end{figure*}

    
    \subsection{LLM-Based Error-Driven Restoration}
        %In contrast to prior studies on notebook executability \cite{Head2019, Wang2020, Zhu2021, Wang2021} that consider a notebook non-executable if they result in an error, we attempt to recover the executability of notebooks from missing appropriate execution environment. These notebooks are still executable in the original author's environment and, therefore, intrinsically executable. 
        This section explains the lightweight use of LLMs to synthesize the execution environments, enabling improved measurements of notebook restorability. This addresses undefined name issues from incorrect execution orders or the absence of definitions, generates synthetic yet syntactically valid input, and installs external modules. 
        %Note that the goal of this work is not to create new notebook restoration techniques but to use existing tools to separate pathologically non-executable notebooks from misconfiguration.
        
%       LLMs have shown remarkable success in various data generation tasks, including program synthesis, comment generation, test generation, and debugging.
        
        


         %   In the pursuit of restoring the executability of non-executable notebooks, our approach hinges on the implementation of fundamental restoration strategies. Our overarching objective is to seamlessly recover the executability of these notebooks while meticulously preserving the integrity and semantics of their contents. Through employing these strategies, we strive to ensure that the essential essence and structure of the notebooks remain unaltered, thus facilitating their continued utility as valuable resources for future studies and analyses.

    
   
    %Our static-error checking phase identifies notebooks that are non-executable due to NameError issues. Among those, we restore the executability of notebooks where NameError occurs due to define-after-use issues. Notebooks in which a variable is undefined throughout are pathologically non-executable as there is no straightforward method to restore it without altering the notebook's semantics.
  
    %The nearest cell containing the variable's definition is extracted and placed immediately before the cell where the variable is utilized, as illustrated in Figure \ref{fig:reorder_workflow}. Cell reordering is performed in step \ding{204} of the analysis as shown in Figure \ref{fig:main_workflow}.
    
    %Moving a cell may either fix the NameError or lead to a new NameError. We apply static and dynamic error checking to the refactored notebook to evaluate its executability again. If the NameError does not appear, the executability of the reordered notebook is compared to that of the original file, where no reordering occurs. If a new NameError occurs, we consider such notebooks restorable, but we do not attempt to restore them further, as it requires a meticulous search for the right cell execution order. When a NameError occurs due to define-after-use within a cell, we classify such notebooks as pathologically non-executable.  This restoration phase concludes by addressing NameErrors that do not require intra-cell code refactoring. 

        \subsubsection{ModuleNotFound Error}
            Most of the repositories do not provide environment requirement information, and if they do, those modules are often outdated \cite{Zhu2021, Wang2021}. Notebooks with ``ModuleNotFound" errors are restorable since they were originally executable but failed to execute in a new environment. We attempt to re-establish the execution environment for the notebooks by inspecting the REQUIREMENTS/INSTALL instructions and installing the required dependencies in the execution environment before executing any notebooks. During execution, if a notebook returns a ``ModuleNotFound" error, we extract the missing module name from the error message and construct a terminal command {\small{\texttt{pip install <missing module>}}} (\ding{204}-D of Figure \ref{fig:main_workflow}). If the installation fails due to an incorrect or deprecated module name, we use LLM to obtain the correct and updated name for the corresponding missing module (\ding{204}-E of Figure \ref{fig:main_workflow} and LLM prompt in Figure \ref{fig:LLM-prompts}-A) and retry the installation. 
            
            %\tien{Figure \ref{fig:LLM-prompts}-B shows an example of the LLM prompt and response for a failed module installation.} After missing modules are installed into the execution environment, notebooks are subsequently re-executed to evaluate their executability (step \ding{205} in Figure \ref{fig:main_workflow})
            
        %We structure the prompt to facilitate the LLM in generating accurate and concise responses within our tailored response format. We then extract the updated module name from the LLM response and re-attempt the module installation with the new module name. We solely aim to assess whether the executability of the notebook improves through this intervention. 
         

   % Notebooks are frequently shared and reused among multiple users, leading to potential issues stemming from disparities in execution environments. Notebooks may include execution environment information; however, it is often inaccurate or outdated \cite{Wang2021}. The non-availability of correct external modules and packages is a major reason for notebook non-executability.  We consider notebooks with ModuleNotFound errors as restorable since they were originally executable but failed to execute in a new environment due to misconfiguration. Thus, we devise a lightweight strategy to restore the executability of such notebooks instead of classifying them as non-executable.   
    
    %When a notebook execution results in ModuleNotFound error, we extract the missing module name from the error message and construct a terminal command \codeword{pip install <missing module>} with the corresponding missing module. This installs the detected missing module into the current execution environment. We do not alter or update the missing module names due to deprecated APIs, as this could compromise the notebook's semantics. Our aim is solely to assess whether the executability of the notebook improves through this intervention.  Notebooks exhibiting this type of error are subsequently re-executed to evaluate their executability (step \ding{206} in Figure \ref{fig:exec_workflow}) after missing modules are installed into the execution environment.

        \subsubsection{FileNotFound Error}
            ``FileNotFound" errors often occur when notebooks attempt to access and read unavailable input files or directories. We use LLM to generate synthetic input data tailored to the notebook's semantics, attempting to resolve ``FileNotFound" errors while avoiding additional runtime issues. Our insight is that the code cells in notebooks contain sufficient information for LLMs to generate syntactically correct input data needed for execution. Since our goal is executability rather than reproducibility, syntactically correct synthetic data is sufficient to address the ``FileNotFound" error and restore notebook executability.
            %Given the absence of the actual data for the missing input files, using LLM is the optimal approach to generate suitable synthetic sample content, thereby enabling the restoration of notebook executability.         
            Concise contexts improve the LLM's performance \cite{Ramlochan2024}. Thus, we provide contexts of the notebook error, such as the missing file path and type, which guides LLM on the file's syntax, correct data type, and content. Instead of the {\small{\texttt{.ipynb}}} file, we provide the source code of all cells in the prompt. This conversion from {\small{\texttt{.ipynb}}} to Python also removes noise that may misguide or overwhelm the LLMs. Figure \ref{fig:LLM-prompts}-B shows this prompt and the response by the LLM for the example mentioned in \ref{case_study_1}. 
            %The three restoration strategies are completely automated to help scale the empirical evaluation.  




     
        \subsubsection{NameError}
            %NameError is one of the most common notebook executability errors arising from variable undefined issues when executed top-down linearly. Such errors often do not occur for a notebook's original authors, who know the correct non-linear execution of cells. However, notebooks are rarely accompanied by the execution order of cells when made publicly available.
            For a non-executable notebook due to undefined names, we extract the name and the specific cell in which the error occurs and the cell where the name definition is located or indicate if it cannot be found.  We then prompt LLM to address this issue as illustrated in \ding{204}-A of Figure \ref{fig:main_workflow} (b). The prompt also includes the code cells and the error type (see Figure \ref{fig:LLM-prompts}-C). The LLM's response is used to rewrite the notebook, which is analyzed again for additional errors. However, the changes introduced by LLMs may change the notebook's semantics. Since test cases are almost non-existent in notebooks (only 1.54\% notebooks have tests \cite{Pimentel2021}), it is challenging to verify semantic accuracy. Even if test cases are available, we must first make the notebook executable to enable testing and detect any semantic inconsistencies introduced by LLM-based modifications. Our position is that addressing executability is a prerequisite for reproducibility, which requires cell executability even if it causes semantic changes.  
     
     
%     Even if LLM-based restorations introduce minor logical errors, achieving executability is the supports future reproducibility techniques, creating opportunities to correct any semantically inconsistent fixes to NameError.

%    Jupyter notebooks are formatted as JSON-based files containing extensive information about the execution environment, metadata, and all cells even when they contain no content. The source code for each cell is stored as a single string, making it challenging for the LLM to extract accurate information. Therefore, we need the source code of the notebook as a whole. Additionally, supplying the LLM engine with the notebook's JSON-based format would be counterproductive due to the excess of unnecessary information. 

 %   To provide the aforementioned context information to the LLM, we use a regular expression to extract the full path of the missing input file from the reported error message during the dynamic execution of a notebook. To acquire the notebook's source code, we employ the PythonExplorer module to convert the JSON-based data into Python-typed data. This procedure integrates the source code of individual notebook cells into cohesive Python content.

   % We structure the prompt to facilitate the LLM in generating accurate data within our tailored response format, also shown in Figure \ref{fig:LLM-prompts}.
% {\em "Generate a sample input file \codeword{<missing_file_path>} for the source code below. Format the response with only the needed data between \texttt{\textasciigrave\textasciigrave\textasciigrave} and \texttt{\textasciigrave\textasciigrave\textasciigrave}. Just data and No fluff."<Notebook source code in Python>}
        %
    
    
    
    % To do so, we ensure that the Large Language Model (LLM) generates only the necessary data by including the directive ``Just data and No fluff." \waris{in the intro I mentioned we use regular expressions to sanitize. we can remove that point or add this insight there.}
    % This helps prevent the LLM from producing any superfluous or irrelevant content, eliminating the need to sanitize and refine the response. The response is then injected in the input files with the appropriate file names and directories. 
    
    % Once the missing input files have been created and correctly placed, we perform dynamic error checking again to determine whether the error has been resolved or if it remains unchanged, as shown in \ding{205} in Figure \ref{fig:main_workflow}. If the issue persists, it may be due to discrepancies in how the notebooks read the file directories versus how they were created. 
    
    
    % \waris{i believe we try to create the file path given in the filename. The error might me be that sometimes due to window machines path or username paths we cannot create the file path. We also do not want to edit the path in the notebook. It also quite possible that LLM did not generate the correct data. example it might miss a column in csv while generating data.} \gulzar{This ca}
    
    
    % In such cases, we halt the process and report the partial executability of the notebook, noting the specific issues preventing full execution.

\section{Empirical Results}
The following sections present the empirical findings of the notebook executability analysis.

\begin{figure}[t]
    \centering
    \begin{tikzpicture}
        \begin{axis}[
            width=1\columnwidth, % Fit to one column
            height=0.40\columnwidth, % Adjust height for proportion
            ybar,
            symbolic x coords={$\geq$1000, 500-999, 300-499, 200-299,150-199,125-149,100-124,90-99,80-89,70-79,60-69,55-59,50-54,45-49,40-44,35-39,30-34,25-29,20-24,15-19,10-14,4-9},
            xtick=data,
            ylabel={\# of Notebooks},
            xlabel={GitHub Stars},
            ymin=0,
            ymax=14000, 
            ytick={0, 2000, 4000, 6000, 8000, 10000, 12000},
            minor tick num=0, % Disable minor ticks
            nodes near coords,
            every node near coord/.append style={font=\scriptsize, yshift=-2pt, xshift=-3pt, rotate=75, anchor=west},
            enlarge x limits=0.05,
            bar width=0.17cm,
            xticklabel style={font=\scriptsize, rotate=75},
            yticklabel style={font=\scriptsize}, 
            label style={font=\small}
        ]
        \addplot[
            fill=blue,
            draw=none
        ] table[x=Star,y=Count,col sep=space]{starVScount.csv};

        \end{axis}
    \end{tikzpicture}
    \caption{Distribution of notebooks and their GitHub stars.}
    \label{fig:notebook_count_vs_star}
    % \vspace{-4ex}
\end{figure} % Figure 5

    \noindent{\textbf{Dataset.}} We focus on two criteria to collect notebooks to investigate our research questions. First, the notebooks should be popular, actively shared, and reused. Second, the notebooks should be the most recent version at the time of the notebook dataset collection, as the computational notebook ecosystem has changed in the last few years with the advent of AI and new notebook tooling. We exclude irrelevant files that are either unreadable, written in non-Python languages, or authored in Python 2, which has been deprecated since 2020. Based on this, we use GitHub API to search for Jupyter notebooks with parameters, including language and star range. 
    %For example, language = “Jupyter Notebook” and stars = “500..999” will retrieve Jupyter notebooks that have 500 to 999 stars.
    GitHub stars are commonly used to indicate popularity \cite{Borges2016}.  We then take a stratified sample of up to 1000 repositories based on GitHub star "tiers" (e.g., $\geq$1000, 500-999, 300-499, and so on) with logarithmic tier sizes to balance the repository count per tier due to long-tailed distribution of repositories against stars. This leads to approximately 318,000 notebooks. Due to prohibitive compute costs from LLMs, we take a 13\%  sample of total notebooks,  resulting in \totalNotebooksInDataset notebooks from \totalRepos repositories.    Figure \ref{fig:notebook_count_vs_star} shows the distribution of sampled notebooks based on stars.



    
    %    By including different ranges, we aim to obtain a representative sample of notebooks. Another reason to focus on popularity repositories is that they are actively maintained and frequently utilized by a diverse range of users, ensuring the relevance of our analysis \tien{Again, we should be careful with this claim if the reviewers don't agree as some repositories are a few years old}.
    %Leveraging notebooks from these repositories enables us to effectively filter out obsolete, poor-quality, and unmanaged notebooks, thereby enhancing the quality of our dataset. Note that our primary focus is analyzing notebooks with a specified Python 3 kernel. Consequently, 

    
 %   \tien{Maybe move this up a little to indicate that we applied these filters to get the total notebook number that we show}. 
    
%For each notebook, we clone the entire repositories, including documentation and data files, to make the best effort to recreate the execution environment of the original authors of the notebook \tien{It may not make sense much here. We cloned all repos first then applied filters to collect valid notebooks' info, i.e., their paths relative to the program but still kept them in their repos}.  


% \tienWe leverage the GitHub search engine to inquire about public repositories with Jupyter Notebook as their dominant programming language. We also include star ranges in descending order as a search criteria for more specific results.     
    
    
%We gather a corpus comprising \totalNotebooksInDataset notebooks sourced from two distinct datasets. The first dataset encompasses \totalNotebooksFromGitHub notebooks, which is a 5\% random sample of all notebooks with 10+ GitHub stars. These repositories are actively maintained and frequently utilized by a diverse range of users, ensuring the relevance of our analysis. Leveraging notebooks from these repositories enables us to effectively filter out obsolete, poor-quality, and unmanaged notebooks, thereby enhancing the quality of our dataset.
%Additionally, we have incorporated approximately 5\% random sample dataset utilized in previous studies conducted by Pimentel et al. \cite{Pimentel2019, Pimentel2021}, comprising \totalNotebooksFromDrive notebooks. We pick a 5\% sample from this dataset for two reasons. First, the original data contains approximately \totalALLNotebooksFromDrive notebooks, including many low-quality, incomplete ones. Second, the entire dataset is inaccessible due to Google Drive limits, and GitHub has recently imposed rate limits on the automated crawling of the GitHub repository. By combining these two datasets, we aim to obtain a more representative sample of notebooks. Note that our primary focus is analyzing notebooks with a specified Python 3 kernel. Consequently, we exclude any irrelevant files that are either unreadable, written in non-Python languages, or authored in Python 2, as it has not been supported since 2020. For the feasibility and scalability of this study, we also set an execution timeout of 5 minutes—any notebook executing beyond 5 minutes is excluded from the dataset. For example, experimental notebooks that demonstrate an infinite loop are omitted using timeouts. 

% \subsection{}
\noindent{\textbf{Analysis Environment.}} We establish a separate virtual environment for each repository. Sandboxing the execution environments eliminates the risk of mismatched versions of Python modules, side effects from executing the notebooks, and conflicts stemming from shared dependencies. We use Python version 3 as the execution kernel. Before analyzing a notebook, we locate the requirements file (if available) listing necessary modules or packages and install them in the execution environment (in \ding{202} of Figure \ref{fig:main_workflow}). 
For the feasibility and scalability of this study, we also set an execution timeout of 5 minutes.
%—any notebook executing beyond 5 minutes is excluded from the dataset. 
%For example, experimental notebooks that demonstrate an infinite loop are omitted using timeouts. We employ a cluster of six NVIDIA DGX A100 \cite{NVIDIADG22:online} nodes, each with 2048 GB of memory and a minimum of 128 cores, to conduct our analysis. 
We use a state-of-the-art open-source LLM called Llama-3 \cite{llama3, touvron2023llama}.% for LLM-based synthetic data generation. 
    
    
%\waris{Based on the approach section in addition to input generation we also used LLM for error analysis. We should make it generic or write all the use cases of LLM in our analysis.}

    
% We establish a distinct virtual environment for each notebook, ensuring that the analysis is conducted under identical conditions. Within these environments, we use Python version 3.12.3 as the execution kernel, providing a standardized framework for execution. By employing consistent environments, we mitigate the risk of conflicts or discrepancies stemming from shared dependencies or variations in environmental settings. Furthermore, to enhance efficiency and prevent potential issues such as infinite loops, we impose a time limit of 5 minutes for each notebook's execution. We employed an enterprise-level cluster consisting of six NVIDIA DGX A100 \cite{NVIDIADG22:online} nodes to conduct our analysis. Each node has 2048 GB of memory and a minimum of 128 cores. Additionally, every node includes an A100 GPU, which boasts 80 GB of memory. This configuration provided a robust platform for our computational tasks.
% We use a state-of-the-art open-source LLM called Llama-3 \cite{llama3, touvron2023llama} for LLM-based synthetic input generation.
  
%  \gulzar{Waris: write detail spec of the server} that is equipped with GPU. For LLM-based synthetic input generation, we use state-of-the-art open-source LLM called \gulzar{LLM name and write details of that.}.

\subsection{RQ1: Causes of Non-Executability}

        \begin{table}[t]
    	\centering
    	\caption{Top 10 Common Errors in Notebooks.}
        \begin{tabular}{|m{3cm}|>{\centering\arraybackslash}m{1.3cm}|>{\centering\arraybackslash}m{1.6cm}|>{\centering\arraybackslash}m{1cm}|}
    		\toprule
    		\textbf{Error Type}       & \textbf{\# of \hspace{10pt}Notebooks}  & \textbf{\% w.r.t Non-Executable}                    & \textbf{\% w.r.t Dataset}              \\
    		\midrule
    		ModuleNotFound Error      & \totalModuleNotFound            & \percentModuleNotFoundInNonExecutable      & \percentModuleNotFoundInTotal \\ 
    		FileNotFound Error        & \totalFileNotFound              & \percentFileNotFoundInNonExecutable        & \percentFileNotFoundInTotal   \\ 
    		AttributeError            & \totalAttributeError            & \percentAttributeErrorInNonExecutable      & \percentAttributeErrorInTotal \\ 
    		ImportError               & \totalImportError               & \percentImportErrorInNonExecutable         & \percentImportErrorInTotal    \\ 
    		ValueError                & \totalValueError                & \percentValueErrorInNonExecutable          & \percentValueErrorInTotal     \\
            TypeError                 & \totalTypeError & \percentTypeErrorInNonExecutable          & \percentTypeErrorInTotal \\
            KeyError                  & \totalKeyError                  & \percentKeyErrorInNonExecutable            & \percentKeyErrorInTotal       \\
            StdinNotImplementedError  & \totalUserInputs                & \percentUserInputsInNonExecutable          & \percentUserInputsInTotal     \\
            IndexError                & \totalIndexError                & \percentIndexErrorInNonExecutable          & \percentIndexErrorInTotal    \\ 
            NameError                 & \totalNameError                 & \percentNameErrorInNonExecutable           & \percentNameErrorInTotal      \\
    	   \bottomrule
	   \end{tabular}
	\label{common_errors}
 % \vspace{-3ex}
\end{table}

    Our first goal is to identify the non-executable notebooks and the reason behind their non-executability. %While prior work \cite{Pimentel2019, Pimentel2021, gigascience2024, Zhu2021, Wang2021}, have performed similar investigations, they either use skewed notebook corpus, use highly strict execution criteria, or fail to follow the requirement/install instructions \tien{these work did similar investigations, but they did look into requirement files. Yes, they use binary executability notion and don't apply star range on the dataset}. 
    We first set up the execution environment by incorporating the installation instructions in the parent repository. This attempt at executing the notebooks results in \totalExecutable (\percentExecutable) executable notebooks, and \totalNonExecutable (\percentNonExecutable) non-executable notebooks. %This finding is consistent with the prior work \cite{Pimentel2019} that less than 25\% are fully executable. 
    There are two reasons why non-executability is still more than previously reported 76\% \cite{Pimentel2019}. First, repositories often do not provide complete and detailed requirements files. 
    %Pimentel \cite{Pimentel2019} mentioned that only 13.72\% of their dataset had required information.
    Later, we show that even after following the requirements file, we still encounter environment misconfiguration errors in nearly \percentModuleNotFoundInTotal of the notebooks.
    %Prior work \cite{Pimentel2019} work shows that about 65\% failed to install. 
    Second, prior work studied a dataset of notebooks containing a vast amount of obsolete, simple, and unmanaged notebooks; these are experimental notebooks used for dabbling with notebook environments. Such notebooks may not depend on external packages and libraries, which lowers the overall non-executability. Low popularity notebooks tend to be less dependent on external packages and libraries and are often shorter, reducing the chances of ``FileNotFound" and ``ModuleNotFound" errors but exhibiting higher ``NameError" rates, as shown in \cite{Pimentel2019}. We hypothesize that the executability of the low-quality notebooks will be lower once environment configuration errors are addressed. %Therefore, if methods from prior studies are followed, the non-executability would be much higher than recorded by our analysis. %\gulzar{do we have several notebooks that benefit from requirements files?} \tien{No, but we may be able to extract how many repos have requirement files (approximately, because they may be named something customized to their project, but not high)}

     
     Next, we investigate and categorize why \percentNonExecutable notebooks cannot execute and find the top reasons that hinder their execution. Table \ref{common_errors} presents the top 10 issues encountered in the notebooks. Among them, ModuleNotFound and FileNotFound errors are the most frequent, making up \percentModuleNotFoundInTotal and \percentFileNotFoundInTotal of the entire dataset, respectively. Other common exceptions include ``AttributeError" (\percentAttributeErrorInTotal), ``ValueError" (\percentValueErrorInTotal), ``TypeError" (\percentTypeErrorInTotal), and ``StdinNotImplementedError" (\percentUserInputsInTotal).
    %, which also appears among the most frequent errors encountered during notebook execution.
    
    % \gulzar{Tien: write about all other errors and tell which errors are non-restorable and which are not. And why we are only talking about 4?}
    
    % \gulzar{We need a detailed comparison for each of the following error types with the error numbers in prior work vs ours; if different, should we highlight and justify?}

    \subsubsection{Execution Environment}
        In our analysis, \totalModuleNotFound notebooks encountered ``ModuleNotFound" errors during execution when executed in a fresh environment. This accounts for \percentModuleNotFoundInNonExecutable of the total non-executable notebooks. These errors arise when the required modules or packages are not available in the package manager (e.g., {\small{\texttt {pip}}}) or cannot be located within the execution environment in the case of custom packages and libraries. Such mismatches highlight the importance of ensuring compatibility and consistency across execution environments to facilitate seamless notebook execution across different computing setups. Prior work \cite{Pimentel2019} reports that 20.7\% of notebooks failed due to ``ImportError," whereas our analysis finds that only \percentIndexErrorInTotal of notebooks in our dataset encounter this exception. ImportError can occur for various reasons, such as issues within the module or missing dependencies, while the ``ModuleNotFound" error specifically occurs when Python cannot locate the module at all. Introduced in Python 3.6, the ``ModuleNotFound" error clarifies this distinction, making it easier for developers to handle missing modules.

    \subsubsection{External Input Data}
        %In scenarios where notebooks use external input files, these files are typically local to the developers' machines, unless provided alongside the notebooks. When executing notebooks that depend on such input files, users may encounter ``FileNotFound" errors if the necessary input files are missing. 
        We find \percentFileNotFoundInNonExecutable of the total non-executable notebooks affected by the ``FileNotFound" error. Additionally, when the notebook requires user-typed inputs, notebook execution will raise a ``StdinNotImplementedError." We do not attempt to restore the executability of those notebooks, as they necessitate dynamic interactions, which are not feasible in an automated execution environment. We detect a total of \percentUserInputsInNonExecutable initially non-executable notebooks that required user inputs during their execution. 
        
        %We further compare our findings with what is reported in prior work \cite{Pimentel2019}.  %\tien{Prior work shows that approximately 1\% of notebooks encounter \textit{StdinNotImplementedError}, which aligns with our finding. Similarly, \textit{FileNotFoundError} affects \percentFileNotFoundInTotal of the notebooks in our dataset, closely matching the 8.5\% reported by \cite{Pimentel2019}.} 

        % \input{figures/FileNotFound_vs_Stars}
        
    \subsubsection{Missing/Disordered Variable Definitions}

        % \begin{table}[htbp]
%     \centering
%     \pgfplotstabletypeset[
%         col sep=comma,          % CSV column separator
%         string type,            % Treat all columns as strings
%         every head row/.style={ % Style for the header row
%             before row=\toprule, 
%             after row=\midrule
%         },
%         every last row/.style={after row=\bottomrule}, % Bottom rule after the last row
%         columns/Star Range/.style={string type, column name={Star Range}}, % Custom label
%         columns/NameError/.style={column name={\% NameError}, column type={c}},
%         columns/No NameError/.style={column name={\% No NameError}, column type={c}}
%     ]{NameError_vs_Stars_normalized.csv} % Path to the CSV file  

%     \caption{Percentage of NameErrors in Notebooks by Star Range} % Table caption
%     \label{tab:nameerror_table} % Optional: label for referencing the table
% \end{table}

\begin{table}[t]
    \centering
    % \caption{\tien{Normalized percentage of notebooks with NameError in different star ranges.}}
    \resizebox{\columnwidth}{!}{ 
        \begin{tabular}{|l|c|c|c|c|c|}
            \toprule
            \textbf{Star Range} & \textbf{$\geq$1000} & \textbf{500-999} & \textbf{100-499} & \textbf{10-99} & \textbf{4-9} \\
            \midrule
            \textbf{Percentage} & 0.65\% & 0.41\% & 1.11\% & 1.05\% & 2.09\% \\
            \bottomrule
        \end{tabular}
    }
    \caption{Percentage of notebooks with NameError in different star ranges.}
    \label{tab:nameerror_star_range}
\end{table}


 % Table 2
        
        ``NameError" in a notebook can arise due to the use of a variable, library, function, or class name that was not previously defined. Among the \totalNonExecutable notebooks that are non-executable, we identify \totalNameError notebooks containing at least one instance of this error, accounting for \percentNameErrorInNonExecutable (\percentNameErrorInTotal of total), as shown in Table \ref{common_errors}. We further investigate these notebooks to explore the relationship between GitHub stars and ``NameError" occurrences. Table \ref{tab:nameerror_star_range} shows the normalized number of notebooks affected by ``NameError" in different star ranges. Our findings indicate that ``NameError" is less prevalent in more popular notebooks, with its occurrence decreasing as the star increases. While ``NameError" affects only a small portion of our dataset, prior work \cite{Pimentel2019} ranks it among the top two errors, reporting it in 14.53\% of notebooks. We observe that ``NameError" is predominantly found in low popularity notebooks, which are less actively reused and frequently maintained, allowing such errors to persist. This provides concrete evidence that ``NameError" tends to exist in less frequently used notebooks. Therefore, a higher ``NameError" rate reported in \cite{Pimentel2019} suggests that {\em the dataset of notebooks studied in earlier work includes a significant number of low-popularity and rarely managed and reused notebooks.} For example, the notebook \cite{JadFatTail}, with only four stars, resembles a trial notebook that defines numerous functions without any documentation, ultimately leading to an undefined function error.
        
          
    
    \subsubsection{Experimentation Notebooks}
        Notebooks are often used as programming playgrounds, and users who are new to programming often use them to dabble with programming. Notebooks resulting from such exercises are not managed and not expected to be shared and, thus, by definition, are incomplete. Similar to these notebooks, we find that \percentTimeout of notebooks in our dataset contains an infinite while loop to demonstrate concepts such as infinite loops, large-scale training campaigns, or I/O operations over the network. Executing such notebooks may result in ``TimeoutError" issues. Notebooks with ``TimeoutErrors" are not analyzable; thus, we exclude those from Table \ref{common_errors}. %\tien{What can we say more about this part? How does it relate to the table \ref{common_errors}}


\begin{tcolorbox}[left=0mm, right=0mm, top=0mm, bottom=0mm]
\textbf{Takeaway RQ1.} Our analysis shows that the primary causes of non-executable notebooks are missing dependencies and external data. High-quality notebooks have fewer fundamental errors like ``NameError," indicating better code maintenance and frequent usage. %This emphasizes the importance of providing complete requirement files and necessary resources to improve notebook executability.
\end{tcolorbox}

        

\subsection{RQ2: Pathological Non-Executability} 
\begin{figure}[t!] % Figure 6
	\centerline{\includegraphics[width=\columnwidth]{images/sankeymatic.png}}
	\caption{Summary of our investigation results. This shows different notions of executability in computational notebooks.}
	\label{fig:overall_results}
 % \vspace{-3ex}
\end{figure}
 
Resolving errors like compilation errors, syntax errors, indentation errors, and runtime errors like ``OSError'' and ``HTTPError'' require significant code refactoring to align with the intended semantics. Without the knowledge of such semantics, the executability of such notebooks cannot be correctly restored. Notebooks requiring extensive intervention to restore executability are categorized as {\em pathologically non-executable}. Notebooks with errors that can be resolved by configuring the execution environment are classified as non-executable but {\em restorable}. For instance, as in Case Study 1, a missing input data file may trigger a runtime error, rendering the notebook non-executable. However, providing the correct execution environment and data file enables the notebook to run successfully without altering its semantics. 


We identify \percentExecutable (\totalExecutable\hspace{1pt}/ \totalNotebooksInDataset) notebooks initially executable, indicating end-to-end execution without errors. Among the rest, shown in Figure \ref{fig:overall_results}, we find \totalPathological (\percentPathological) pathologically non-executable notebooks, distinguished by errors that significantly impede their execution and require additional specification about the intended goal of the notebook to recover. \totalRestorable (\percentRestorableInNonExecutable) non-executable notebooks fall into the restorable category, suggesting errors that could be remedied through appropriate restoration strategies. 

\begin{tcolorbox} [left=0mm, right=0mm, top=0mm, bottom=0mm]
    \textbf{Takeaway RQ2.} Surprisingly, only \percentPathological of all notebooks are pathologically non-executable; the rest are either executable or potentially executable given a suitable execution environment. This validates our hypothesis that pathologically non-executable notebooks are significantly lower than previously found. 
\end{tcolorbox}

\subsection{RQ3: Degree of Executability of Pathologically Non-executable Notebooks}

\begin{figure}
    \centering
    \begin{tikzpicture}
        \begin{axis}[
            width=1\columnwidth, % Fit to one column
            height=0.45\columnwidth, % Adjust height for proportion
            ybar,
            symbolic x coords={$>$0-9, 10-19, 20-29, 30-39, 40-49, 50-59, 60-69, 70-79, 80-89, 90-99},
            xtick=data,
            ylabel={\# of Notebooks},
            xlabel={Executable Cells (\%)},
            ymin=0,
            ymax=1200, 
            ytick={0, 200, 400, 600, 800, 1000},
            % xtick={0, 10, 20, 30, 40, 50, 60, 70, 80, 90, 100},
            minor tick num=0, % Disable minor ticks
            nodes near coords,
            every node near coord/.append style={font=\scriptsize, yshift=-2pt},
            enlarge x limits=0.1,
            % bar width=0.4cm,
            bar width=0.4cm,
            % tick label style={font=\scriptsize, rotate=45},
            label style={font=\small},
            % xlabel style={yshift=2pt},
            xticklabel style={font=\scriptsize, rotate=45}, % Rotate only x-axis tick labels
            yticklabel style={font=\scriptsize}, 
        ]
        \addplot[
            fill=blue,
            draw=none
        ] table[x=Executable,y=Count,col sep=space]{pathological.csv};
        \end{axis}
    \end{tikzpicture}
    \caption{Executability of pathologically non-executable notebooks in terms of the percentage of successfully executed cells. }
    \label{fig:path-non-exec-results}
    % \vspace{-3ex}
\end{figure}

% \begin{figure*}[!t]
%     \centering
%     \begin{minipage}{0.48\textwidth}
%     \begin{tikzpicture}
%         \begin{axis}[
%             width=1\columnwidth, % Fit to one column
%             height=0.45\columnwidth, % Adjust height for proportion
%             ybar,
%             symbolic x coords={$>$0-9, 10-19, 20-29, 30-39, 40-49, 50-59, 60-69, 70-79, 80-89, 90-99},
%             xtick=data,
%             ylabel={\# of Notebooks},
%             xlabel={Executable Cells (\%)},
%             ymin=0,
%             ymax=1200, 
%             ytick={0, 200, 400, 600, 800, 1000},
%             % xtick={0, 10, 20, 30, 40, 50, 60, 70, 80, 90, 100},
%             minor tick num=0, % Disable minor ticks
%             nodes near coords,
%             every node near coord/.append style={font=\scriptsize, yshift=-2pt},
%             enlarge x limits=0.1,
%             % bar width=0.4cm,
%             bar width=0.4cm,
%             % tick label style={font=\scriptsize, rotate=45},
%             label style={font=\small},
%             xlabel style={yshift=2pt},
%             xticklabel style={font=\scriptsize, rotate=45}, % Rotate only x-axis tick labels
%             yticklabel style={font=\scriptsize}, 
%         ]
%         \addplot[
%             fill=blue,
%             draw=none
%         ] table[x=Executable,y=Count,col sep=space]{pathological.csv};
%         \end{axis}
%     \end{tikzpicture}
%     \caption{Executability of pathologically non-executable notebooks in terms of the percentage of successfully executed cells. }
%     \label{fig:path-non-exec-results}
%     % \vspace{-3ex}
%     \end{minipage}
%     \hfill
%     \begin{minipage}{0.48\textwidth}
%             \begin{tikzpicture}
%         \begin{axis}[
%             width=1\columnwidth, % Fit to one column
%             height=0.45\columnwidth, % Adjust height for proportion
%             ybar,
%             symbolic x coords={$>$0-10, 11-20, 21-30, 31-40, 41-50, 51-60, 61-70, 71-80, 81-90, 91-100},            
%             xtick=data,
%             ylabel={\# of Notebooks},
%             xlabel={Executable Cells (\%)},
%             % xlabel style={yshift=7pt},
%             ymin=0,
%             ymax=970, 
%             ytick={0, 200, 400, 600, 800},
%             minor tick num=0, % Disable minor ticks
%             nodes near coords,
%             % every node near coord/.append style={font=\tiny, rotate=90, yshift=-5pt, xshift=5pt},
%             every node near coord/.append style={font=\scriptsize, yshift=-2pt},
%             enlarge x limits=0.1,
%             bar width=0.4cm,
%             % bar width=0.2cm,
%              % tick label style={font=\scriptsize, rotate=45},
%             label style={font=\small},
%             xlabel style={yshift=2pt},
%             xticklabel style={font=\scriptsize, rotate=45}, % Rotate only x-axis tick labels
%             yticklabel style={font=\scriptsize}, 
%         ]
%         \addplot[
%             fill=red,
%             draw=none
%         ] table[x=Executable,y=Count,col sep=space]{modulenotfound.csv};


%         \end{axis}
%     \end{tikzpicture}
%     \caption{Improvement in executability by addressing ModuleNotFound errors.}
%     \label{fig:module-found-results}
%     % \vspace{-4ex}
%     \end{minipage}
% \end{figure*}

 % Figure 7

Notebooks are developed and executed incrementally and interactively, unlike traditional software, which is built and run atomically. Even if a notebook is pathologically non-executable as a whole, it may still contain valuable code snippets that are executable. This introduces a fine-grained notion of executability, where errors may occur only in cells toward the later stages of the notebook. Our position is that these executable cells provide valuable code fragments that can be used for dynamic analysis and supporting tasks such as code reuse, comprehension, repair, and model training. With this new, fine-grained executability, we measure the number of cells a pathologically non-executable notebook successfully executes. We then divide this number by the total number of cells in the notebooks to measure the degree of executability (partial execution) in pathologically non-executable notebooks. 


Our analysis reveals that, on average, \averagePercentPartialPathological of the cells in pathologically non-executable notebooks are executable. For example, although the notebook \cite{jarodHAN} encounters ``ValueError" during linear execution, it is still partially executable up to 10 cells or in 76.9\% of total cells. Figure \ref{fig:path-non-exec-results} shows the frequency of pathologically non-executable notebooks with different levels of partial executability. The Y-axis represents the number of notebooks, and the X-axis represents the percentage of cells successfully executed in those notebooks. For instance, we observe that nearly \percentPathologicalOverFiftyPercent of the pathologically non-executable notebooks (\totalPathologicalOverFiftyPercent) successfully execute more than half of the cells ($>$50\%). On the other hand, \totalPathologicalZeroPercent (\percentPathologicalZeroPercent) notebooks have 0\% partial executability due to the pathological error in the first cell. Partial execution in nearly three-quarters of pathological non-executable notebooks demonstrates that even non-executable notebooks can be partially adopted, and dynamic analysis techniques can be applied to improve their executability and reproducibility further.


\begin{tcolorbox} [left=0mm, right=0mm, top=0mm, bottom=0mm]
    % \textbf{Takeaway:} Even pathologically non-executable notebooks often contain a significant portion of executable cells. Our analysis shows that, on average, \averagePercentPartialPathological of the cells in these notebooks are executable, with nearly \percentPathologicalOverFiftyPercent of them successfully executing more than half of their cells. This demonstrates that partially executable notebooks hold valuable code fragments but prior work strictly categorizes these notebooks as non-executable.  

    \textbf{Takeaway RQ3.} Even pathologically non-executable notebooks contain, on average, \averagePercentPartialPathological executable cells. 
    %, with nearly \percentPathologicalOverFiftyPercent of them successfully executing more than half of their cells. 
    This demonstrates that partially executable notebooks hold valuable code fragments for downstream restoration steps.
\end{tcolorbox}



%\waris{This is a good result. Do not directly state the result. First, write about the figure (e.g., what is the X-axis and what is the Y-axis). Next, explain 1-2 results (bars) from the figure. For instance, the bar with the x-axis 20-30 shows 829 notebooks are executable with between 20 to 30 cells in them. 

%If priors studies do not execute these notebooks then state it alongside why our approach is executing these notebooks. 
	%After explaining the results, write a concluding takeaway stating that we found 34\% of cells in X number of non-executable notebooks are executable suggesting that even such notebooks are partially restorable and can be utilized by the developers (e.g., data analysts, etc)}




% \gulzar{We should keep it but tone it down a little bit}
        
% \tien{Mention the NameError results here}

% \waris{We should merge RQ3 and RQ4 because both of these types represent 2nd type of non-executablity. Next, we can discuss each of these in separate subsubsection. Similar to RQ2 the title of the new section should represent the 2nd type of non-executability category.}
        
          
%Among the \totalUndefinedVarsDetected notebooks rendered non-executable due to undefined variables, \totalUndefinedAnywhere contain at least one undefined variable that remains undefined both before and after the respective cell. Conversely, the remaining \totalDefinedAfter notebooks feature defined-after-use variables. In such cases, we employ a def-use cell reordering strategy to correct the cell sequence and resolve the issue of undefined variables. Post cell reordering, we execute the newly arranged notebook alongside other restoration techniques, comparing the results with those obtained from the original notebook execution. 
        
%Among the \totalDefinedAfter notebooks featuring define-after-use variables, \totalDefinedAfterRestored exhibit improved execution outcomes compared to their original executions. This strategy leads to an enhancement of \averagePercentIncreaseAfterReordering in the number of executable cells while fully restoring the executability of \percentNotebooksFullyRestoredDefinedAfterReordering of the notebooks. However, \totalDefinedAfterNotRestored notebooks fail to produce better outcomes as they still contain the wrong cell execution order resulting from the movement of the definition cell preceding the usage cell. Figure  \ref{fig:reorder-results} shows the frequency of notebooks that achieve different levels of improvements in executability. 

\subsection{RQ4: LLM-Based Error-Driven Restoration}
\label{sec:results}
    %\edit{. Leveraging these capabilities, we envision that LLMs can also assist in restoring the executability of notebooks by addressing errors that prevent their successful execution. To this end, we leverage LLMs to analyze the code snippets in non-executable notebooks and identify the root causes of their non-executability.}
    
    After executing notebooks (Section \ref{dynamic-error-checking}), we categorize \totalNonExecutable non-executable into two main categories: \totalPathological pathologically non-executable notebooks and \totalRestorable restorable notebooks (see Figure \ref{fig:overall_results}). They account for \percentPathological and \percentRestorableInNonExecutable of total non-executable notebooks, respectively.
    %We call a notebook restorable when it is expected to execute fully in the original author's environment but fails to execute in others, typically due to a misconfigured environment, missing input data, or incorrect execution order.    
    This section analyzes the executability of restorable notebooks when restored using lightweight LLM-based error-driven strategies.

    \subsubsection{Addressing ModuleNotFound}   
        Our study reveals that \totalModuleNotFound notebooks encountered ``ModuleNotFound" errors out of \totalRestorable notebooks, accounting for \percentModuleNotFoundInRestorable of the restorable notebooks. We use an error-driven, code-aware prompt to request LLM to infer the missing module name, and then we package the missing model name to install the correct environment for each notebook. During this restoration process, we discover that \totalInvalidModuleNotFound (\percentInvalidModuleNotFoundInAllModuleNotFound) encounter ``ModuleNotFound" errors due to deprecated modules, libraries, and packages \cite{Wang2021, Zhu2021}, changes in installation procedures, or missing custom packages. Old versions of certain packages are sometimes removed from {\small{\texttt{pip}}} repositories due to security vulnerabilities, licensing issues, or to encourage the use of newer, stable releases. This removal makes these versions unavailable, preventing us from resolving dependencies. 
        
        Among the restorable notebook that imports a valid public module, we successfully installed the required modules into the execution environment for \totalModuleNotFoundRestored (resulting in \percentModuleNotFoundRestored success rate), while \totalModuleNotFoundNotRestored failed to do so. Module installation failures occur because only \percentRequirementInTotal of notebooks include an environment requirement file, leading to potential dependency conflicts due to unspecified module versions. For the notebooks where modules were successfully installed, the overall notebook executability increased by \averagePercentModuleNotFoundRestoredIncrease, on average. This improvement underscores the importance of adapting to changes in module availability and installation processes to enhance the executability of notebooks.
    
        Figure \ref{fig:module-found-results} shows the number of notebooks and the corresponding improvements in executability after applying this strategy. The Y-axis represents the number of restored notebooks, and the X-axis represents the increase in executable cells. The figure shows that 500 notebooks have $>$90\% improvements in executability. For example, the notebook \cite{Visualize-ML} initially encounters a ``ModuleNotFound" error due to the absence of {\small{\texttt{networkx}}} module in the environment at cell 1. The environment requirement information is not provided in the repository. This approach restores the full executability of that notebook.


    \subsubsection{Addressing FileNotFound} 

        \begin{figure}[t]
    \centering
    \begin{tikzpicture}
        \begin{axis}[
            width=1\columnwidth, % Fit to one column
            height=0.45\columnwidth, % Adjust height for proportion
            ybar,
            symbolic x coords={$>$0-10, 11-20, 21-30, 31-40, 41-50, 51-60, 61-70, 71-80, 81-90, 91-100},            
            xtick=data,
            ylabel={\# of Notebooks},
            xlabel={Executable Cells (\%)},
            % xlabel style={yshift=7pt},
            ymin=0,
            ymax=970, 
            ytick={0, 200, 400, 600, 800},
            minor tick num=0, % Disable minor ticks
            nodes near coords,
            % every node near coord/.append style={font=\tiny, rotate=90, yshift=-5pt, xshift=5pt},
            every node near coord/.append style={font=\scriptsize, yshift=-2pt},
            enlarge x limits=0.1,
            bar width=0.4cm,
            % bar width=0.2cm,
             % tick label style={font=\scriptsize, rotate=45},
            label style={font=\small},
            xlabel style={yshift=2pt},
            xticklabel style={font=\scriptsize, rotate=45}, % Rotate only x-axis tick labels
            yticklabel style={font=\scriptsize}, 
        ]
        \addplot[
            fill=red,
            draw=none
        ] table[x=Executable,y=Count,col sep=space]{modulenotfound.csv};


        \end{axis}
    \end{tikzpicture}
    \caption{Improvement in executability by addressing ModuleNotFound errors.}
    \label{fig:module-found-results}
    % \vspace{-4ex}
\end{figure}
 % Figure 8
        % \begin{figure}[!t]
%     \centering
%     \begin{tikzpicture}
%         \begin{axis}[
%             width=\columnwidth, % Fit to one column
%             height=0.45\columnwidth, % Adjust height for proportion
%             ybar,
%             symbolic x coords={$\geq$1000, 500-999, 300-499, 200-299,150-199,125-149,100-124,90-99,80-89,70-79,60-69,55-59,50-54,45-49,40-44,35-39,30-34,25-29,20-24,15-19,10-14,4-9},
%             xtick=data,
%             ylabel={\shortstack{\# of Executable \\ Notebooks}},
%             xlabel={GitHub Stars},
%             ylabel style={yshift=0pt},
%             xlabel style={yshift=5pt},
%             ymin=0,
%             ymax=3300, 
%             ytick={0, 1000, 2000, 3000},
%             yticklabels={0,1,2,3}, % Display them as 0, 1, 2, 3
%             extra y tick style={tick label style={anchor=south, yshift=5pt}}, % Adjust position of extra tick label
%             extra y ticks={3000}, % Add an extra tick at the top
%             extra y tick labels={$ .10^3$},
%             minor tick num=0, % Disable minor ticks
%             nodes near coords,
%             every node near coord/.append style={font=\scriptsize, rotate=90, anchor=west},
%             enlarge x limits=0.05,
%             bar width=0.05cm,
%             xticklabel style={font=\scriptsize, rotate=45}, % Rotate only x-axis tick labels
%             yticklabel style={font=\scriptsize}, 
%             label style={font=\small, yshift=0},
%             legend style={
%                 at={(0.7,0.7)}, % Position at the center, above the plot
%                 anchor=south, % Anchor the legend to the bottom center
%                 font=\scriptsize,
%                 legend columns=2, % Display in a single row
%             }
%         ]
      
%         \addplot[
%             fill=blue,
%             draw=none
%         ] table[x=Star,y=Count,col sep=space]{InitialstarVScount.csv};
%         \addlegendentry{Initial Count};

%         \addplot[
%             fill=orange,
%             draw=none
%         ] table[x=Star,y=Count,col sep=space]{FinalstarVScount.csv};
%         \addlegendentry{Final Count};
        
%         \end{axis}
%     \end{tikzpicture}
%     \caption{Fully executable notebooks before/after restorations. 
%     }
%     \label{fig:executable_vs_star}
%     \vspace{-3ex}
% \end{figure}

\begin{figure*}[!t]
    \centering
    \begin{minipage}{0.32\textwidth}
        \centering
        \begin{tikzpicture}
        \begin{axis}[
            width=\columnwidth, % Fit to one column
            height=0.65\columnwidth, % Adjust height for proportion
            ybar,
            symbolic x coords={$>$0-10, 11-20, 21-30, 31-40, 41-50, 51-60, 61-70, 71-80, 81-90, 91-100},
            xtick=data,
            ylabel={\# of Notebooks},
            xlabel={Executable Cells (\%)},
            ymin=0,
            ymax=450, 
            ytick={0,100,200, 300, 400},
            % xtick={0, 10, 20, 30, 40, 50, 60, 70, 80, 90, 100},
            minor tick num=0, % Disable minor ticks
            nodes near coords,
            every node near coord/.append style={font=\scriptsize, yshift=-2pt},
            enlarge x limits=0.1,
            % bar width=0.4cm,
            bar width=0.25cm,
            label style={font=\small},
            xlabel style={yshift=2pt},
            xticklabel style={font=\scriptsize, rotate=45}, % Rotate only x-axis tick labels
            yticklabel style={font=\scriptsize}, 
        ]
        \addplot[
            fill=purple,
            draw=none
        ] table[x=Executable,y=Count,col sep=space]{filenotfound.csv};
        \end{axis}
    \end{tikzpicture}
    \caption{Improvement in executability with synthetic input. %The X-axis represents the percentage increase in the number of successfully executed cells. \waris{mention that this figure is related to restoreable-non-executalble-notebooks. Highlight the significance. For instance, in previous approach 115 notebooks which are complety restored were completely categorized  as non-executable.
    }
    \label{fig:llm-input-results}
    \vspace{-5ex}
    \end{minipage}
    \hfill
    \begin{minipage}{0.668\textwidth}
        \centering
        \begin{tikzpicture}
        \begin{axis}[
            width=\columnwidth, % Fit to one column
            height=0.35\columnwidth, % Adjust height for proportion
            ybar,
            symbolic x coords={$\geq$1000, 500-999, 300-499, 200-299,150-199,125-149,100-124,90-99,80-89,70-79,60-69,55-59,50-54,45-49,40-44,35-39,30-34,25-29,20-24,15-19,10-14,4-9},
            xtick=data,
            ylabel={\shortstack{\# of Executable \\ Notebooks}},
            xlabel={GitHub Stars},
            ylabel style={yshift=0pt},
            xlabel style={yshift=5pt},
            ymin=0,
            ymax=3100, 
            ytick={0, 1000, 2000, 3000},
            yticklabels={0,1,2,3}, % Display them as 0, 1, 2, 3
            extra y tick style={tick label style={anchor=south, yshift=5pt}}, % Adjust position of extra tick label
            extra y ticks={3000}, % Add an extra tick at the top
            extra y tick labels={$ .10^3$},
            minor tick num=0, % Disable minor ticks
            nodes near coords,
            every node near coord/.append style={font=\scriptsize, rotate=75, anchor=west},
            enlarge x limits=0.03,
            bar width=0.12cm,
            xticklabel style={font=\scriptsize, rotate=45}, % Rotate only x-axis tick labels
            yticklabel style={font=\scriptsize}, 
            label style={font=\small, yshift=0},
            legend style={
                at={(0.7,0.7)}, % Position at the center, above the plot
                anchor=south, % Anchor the legend to the bottom center
                font=\scriptsize,
                legend columns=2, % Display in a single row
            }
        ]
      
        \addplot[
            fill=blue,
            draw=none
        ] table[x=Star,y=Count,col sep=space]{InitialstarVScount.csv};
        \addlegendentry{Initial Count};

        \addplot[
            fill=orange,
            draw=none
        ] table[x=Star,y=Count,col sep=space]{FinalstarVScount.csv};
        \addlegendentry{Final Count};
        
        \end{axis}
    \end{tikzpicture}
    \caption{Fully executable notebooks before/after restorations. 
    }
    \label{fig:executable_vs_star}
    % \vspace{-3ex}
    \end{minipage}
\end{figure*}
 % Figure 9 and Figure 10
        Most of the notebooks rely on input data files, which, in most cases, are not packaged with the notebook due to privacy concerns about the data or size. We find that \totalFileNotFound popular notebooks encounter ``FileNotFound" errors, accounting for \percentFileNotFoundInRestorable of non-executable but restorable notebooks. To restore the executability of such notebooks or reveal other execution errors, we use LLM to generate synthetic data for the missing input files automatically.
        As mentioned in Section \ref{sec:approach}, we do not expect synthetic data to reproduce the notebook results. Instead, it can help validate the dynamic behavior of a notebook in order to understand and analyze it better. 
        %This can potentially increase the chances of fully reproducing the intended behavior of the notebook. For instance, program repair techniques, including GenProg~\cite{genprog}, assume code executability - a requirement synthetic data can help achieve in notebooks. Thus, a notebook's executable portion offers more repairs and comprehension opportunities than non-executable ones.} \question{Should this part be here? More above? Or in Discussion?}
        %Our insight is that LLMs can generate synthetic data, similar to how they generate data for training newer, smaller LLMs \cite{}. We aim to leverage LLMs to generate the exact synthetic data required by a given notebook facing a file not found error.
        %This task is challenging as it requires the LLM to understand the code, infer the data type, and generate the appropriate data. This is the initial work to utilize  LLMs capabilities to generate complex structured data inputs for resolving notebook file not found errors.} 
        
        Among the notebooks with ``FileNotFound" errors, we successfully generate input files via LLMs for \totalSuccessfulToGenerateFiles (\percentSuccessfulToGenerateFiles) notebooks. The generated inputs are then placed in a directory structure mentioned in the notebook automatically, and the notebook is executed again. In \totalSuccessfulToGenerateFiles notebooks that were rerun with the synthetic input files, the executability of \totalFileNotFoundRestored (\percentFileNotFoundRestoredInSuccessfullyGenerated) notebooks are fully or partially restored.
        
        Our dataset of notebooks performs a wide range of tasks (statistics, ML, AI, and data analysis) across a wide range of domains, which leads to a large variety of input files being processed by these notebooks. Figure \ref{fig:cs1-full-exec} shows a real-world scenario where a notebook is fully restored after generating synthetic input file {\small{\texttt{Social\_Network\_Ads.csv}}}. The complexity of the content of this file (\eg textual data as column names, numerical data as column values, precision in generating binary labels for the {\small{\texttt{Target}}} column, and avoiding alphabetic words in numerical columns) is shown in Figure \ref{fig:LLM-prompts}. Thus, \percentFileNotFoundRestoredInSuccessfullyGenerated increase in notebook execution with synthetic input files is significant and a promising direction to improve notebook executability (Section \ref{sec:discussion}). 
        

  % \begin{figure}
    \centering
    \begin{tikzpicture}
        \begin{axis}[
            width=\columnwidth, % Fit to one column
            height=0.45\columnwidth, % Adjust height for proportion
            ybar,
            symbolic x coords={$>$0-10, 11-20, 21-30, 31-40, 41-50, 51-60, 61-70, 71-80, 81-90, 91-100},
            xtick=data,
            ylabel={\# of Notebooks},
            xlabel={Executable Cells (\%)},
            ymin=0,
            ymax=450, 
            ytick={0,100,200, 300, 400},
            % xtick={0, 10, 20, 30, 40, 50, 60, 70, 80, 90, 100},
            minor tick num=0, % Disable minor ticks
            nodes near coords,
            every node near coord/.append style={font=\scriptsize, yshift=-2pt},
            enlarge x limits=0.1,
            % bar width=0.4cm,
            bar width=0.4cm,
            label style={font=\small},
            xlabel style={yshift=2pt},
            xticklabel style={font=\scriptsize, rotate=45}, % Rotate only x-axis tick labels
            yticklabel style={font=\scriptsize}, 
        ]
        \addplot[
            fill=purple,
            draw=none
        ] table[x=Executable,y=Count,col sep=space]{filenotfound.csv};
        \end{axis}
    \end{tikzpicture}
    \caption{Improvement in executability with synthetic input. %The X-axis represents the percentage increase in the number of successfully executed cells. \waris{mention that this figure is related to restoreable-non-executalble-notebooks. Highlight the significance. For instance, in previous approach 115 notebooks which are complety restored were completely categorized  as non-executable.
    }
    \label{fig:llm-input-results}
    \vspace{-5ex}
\end{figure}


        
%\waris{we should write that even restoring 553 notebooks is quite a good and promising direction to further explore and improve the notebook executability. We can refer to the discussion section here where we discussed how we can enhance the use of LLMs in notebook restoration. Also, refer to the figure here from the motivation where LLM generated data we were able to execute it. Write that that previously without LLM it may require manual effort to inspect the whole program and then either find such data on the internet or manually generate dummy data ...}
    
%\sout{While LLM excels at generating synthetic text, it faces limitations when it comes to generating data for file types such as images, zip files, or files without specified extensions.}
%\waris{Give reason here. Why LLMs e.g., llama cannot work here? Because they are trained on the textual and tabular data. Their tokenizer cannot tokenize the image pixels and thus LLMs like Llama will fail to generate the data for image and audio video files. Mention that we discuss it more in the discussion section.  We will discuss multi-modal LLMs which can interpret both textual and image data. }
%\sout{In scenarios requiring files in common formats like {\texttt .json}, {\texttt .csv}, or {\texttt .txt}, LLM can efficiently generate content based on the provided source code and file path as prompts to the LLM engine. \gulzar{Waris: Same here}}
%\waris{Explain the reason. It is similar to the above. It is because during training these LLMs have seen similar training data \ie contains .csv files which can be tokenized and fed to LLM.}


% In scenarios requiring input files in formats like {\tt .json}, {\tt .csv}, or {\tt .txt}, LLMs can efficiently generate content
% %based on the provided source code and file path as prompts. 
% This is because these LLMs have been trained on similar data formats, allowing them to tokenize and process such data effectively. However, LLMs like Llama-3 face limitations when generating data for file types such as images, audio, or zip files, as they are trained on textual and tabular data.
% %and cannot process image pixels, audio frequencies, and video frames. 
% This limitation is discussed further in the discussion section, where we highlight the potential of multimodal LLMs to enhance notebook executability to generate data for image, audio, or video files.



    Figure \ref{fig:llm-input-results} shows the enhancement in executability achieved in non-executable notebooks with LLM-based input generation. The X-axis represents the percentage of the increase of successfully executed cells. The Y-axis is the corresponding number of notebooks restored. For instance, 51 notebooks see more than 90\% increase in executability, whereas \totalFileNotFoundLessFivePercent show less than 5\%. Among notebooks for which we were able to generate the synthetic input files, we partially restore \percentFileNotFoundPartiallyRestoredInSuccessfullyGenerated of the notebooks and fully recover \percentFileNotFoundFullyRestoredInSuccessfullyGenerated of them using the LLM-based input generation. On average, we observe an incremental improvement of \avgIncreaseAfterFileFixed in the number of executable cells in a notebook after employing the LLM-based restoration.

        
\subsubsection{Addressing NameError}
   %NameError is one of the most common exceptions encountered during notebook execution \cite{Pimentel2019}. Our investigation shows that \totalNameError notebooks encounter this error during our dynamic analysis, comprising \percentNameErrorInRestorable of restorable notebooks. 
   We find only \totalNameError notebook encountering ``NameError," among which only \totalDefinedAfter had define-after-use issues. Using LLM to address this issue in a few notebooks will not offer statistically significant insights. Instead, we use LLM to assist in generating proper definitions for the undefined variables or names. Among \totalNameError, we enhance the executability of \totalNameErrorFixed notebooks (\percentNameErrorFixedInAllNameErrors) by applying this approach. This comprises \percentNameErrorFixedFullInAllNameErrors and \percentNameErrorFixedPartialInAllNameErrors notebooks fully and partially restored in the same order. Case Study 2 in Section \ref{case_study_2} provides an example of a notebook facing ``NameError" due to an undefined function, which prevents full execution. We use LLM to identify a definition for this function under the {\small{\texttt{tensorflow}}} module, improving executability by 50\%. This increase is an indication of LLM's potential for restoring functionality in notebooks with missing definitions.

    %\edit{IF WE NEED TO TALK ABOUT DEFINE AFTER: For X notebooks that have define-after-user issues, we use \textit{def-user} reordering approach to rearrange the definition of the undefined name preceding its usage.}

\begin{tcolorbox} [left=0mm, right=0mm, top=0mm, bottom=0mm]
    \textbf{Takeaway RQ4.} Utilizing LLM-based error-driven restoration strategies can potentially enhance the executability of restorable notebooks. For notebooks encountering errors such as ``ModuleNotFound," ``FileNotFound," and ``NameError," we achieve improvements of \averagePercentModuleNotFoundRestoredIncrease, \avgIncreaseAfterFileFixed, and \avgIncreaseAfterNameFixed in executability for each respective exception.
\end{tcolorbox}

\noindent Finally, we summarize the impact of restoration in Figure \ref{fig:executable_vs_star}, which shows the increase in the number of notebooks that achieved full executability before and after all restorations have been applied. 














% \subsection{Results Summary} 
% \waris{If we provide a summary/takeaway at the end of each subsection we may not need this. If we want to summarize all the findings we can do it without the subsection.}
% \tien{Figure  \ref{fig:overall_results} demonstrates our investigation results with a breakdown of executable, non-executable but restorable, and pathologically non-executable, as well as fully and partially executable after the restoration. Initially, according to the traditional notion of executability, \percentNonExecutable of the notebooks are non-executable. After our analysis and lightweight restoration, we find that \percentPathological of the non-executable notebooks are, in fact, pathologically non-executable with \averagePercentPartialPathological cells executable on average. Among the non-executable but restorable notebooks, \percentFullyRestored end up being fully executable, and \percentPartiallyRestored partially executable. Previously, notebooks that are now fully restored had only a limited portion executable, averaging \percentFullyRestoredBefore cells per notebook. Additionally, for notebooks that remained partially executable, we improved their executability from \percentPartiallyRestoredBefore to \percentPartiallyRestoredAfter. Finally, across all notebooks, on average, \overallExecutableCells of the cells per notebook are still executable, which can utilized by non-author developers.}

\section{Discussion}
\label{sec:discussion}

    \begin{table*}[t!]
    \centering
    \caption{Summary of some related works with ours in computational notebook topic.}
    \begin{tabular}{|m{2cm}|>{\raggedright}m{1.6cm}|>{\centering\arraybackslash}m{0.8cm}|>{\centering\arraybackslash}m{1cm}|>{\centering\arraybackslash}m{1.5cm}|>{\centering\arraybackslash}m{1.3cm}|>{\centering\arraybackslash}m{2.0cm}|>{\raggedright\arraybackslash}m{4.5cm}|}
        \toprule
        
        \textbf{} & \textbf{Purpose} & \textbf{Study Dataset} & \textbf{Module Error} & \textbf{NameError / Cell Order} & \textbf{Input File Error} & \textbf{Non-Executable Found} & \textbf{Dataset Characteristics}\\ 
        \midrule
        
        RELANCER \cite{Zhu2021}                          & Executability        & 4,043             & \cmark    & \xmark    & \xmark    & 47\%      & Collected from Meta Kaggle \cite{kaggle}\\ 
        \hline
        SnifferDog\cite{Wang2021}                        & Executability        & 2,646             & \cmark    & \xmark    & \xmark    & 72.6\%    & Random sample from \cite{Pimentel2019} only those with installable dependency\\ 
        \hline
        Osiris\cite{Wang2020}                            & Reproducibility      & 5,393             & \xmark    & \cmark    & \xmark    & 82.6\%    & Random sample from \cite{Pimentel2019}\\ 
        \hline
        Pimentel et al.\cite{Pimentel2019, Pimentel2021} & Empirical            & $>$1.4M    & N/A       & N/A       & N/A       & 76\%      & Includes high-number of low-starred, rarely reused notebooks from GitHub\\ 
        \hline
        \textbf{This work}                               & Executability        & 42.5K              & \cmark    & \cmark    & \cmark    & \percentPathological & Highly-starred notebooks from GitHub\\ 
        \bottomrule
    \end{tabular}
    \label{related_work_table}
    % \vspace{-3ex}
\end{table*}

    In this section, we distill key findings in a {\em FAQ} format, demonstrating the value of accurate executability categorization, the utility of a notebook corpus once marked unusable, and practices that could prevent non-executability or restore executability in future notebooks. 
    
%\noindent{\em Why do notebook executability measurements in this work differ from prior investigations?} We identify two key factors that distinguish our findings on notebook executability from prior empirical investigations. First, our notebook corpus is primarily sourced from popular repositories—actively shared, reused, and maintained—whereas prior work focused more broadly, including notebooks that are less actively managed or of lower quality. While previous studies provide insights into a general range of public notebooks, our study offers new perspectives on those designed for sharing, adaptation, and reuse. Second, our dataset was collected in 2024, while the most recent prior investigation was conducted in 2021. Over the past three years, several tools aimed at improving notebook quality have emerged \cite{}\textcolor{red}{CITE}, likely contributing to the increase in executability.

\noindent{\em Why are \percentPathological notebooks pathologically non-executable and \percentPartiallyRestored of restorable notebooks still only partially executable?} 
    Prior work has identified that notebooks have alarmingly low numbers of test cases \cite{Pimentel2019}. Developers releasing their notebooks for public use have very limited testing tools to verify reproducibility and executability. For example, notebooks with undefined variables ``NameError" and ``AttributeError" often go unnoticed due to unintended dependencies on session states saved from prior cell execution, leading to those errors being masked on the developer's machine, similar to Case Study 2 in Section \ref{case_study_2}. In a new environment, such states are unavailable, resulting in these errors. 
    %\tien{This may be right, but notebooks originally have execution orders that they can use to execute, even though they may not be consistent. Pimentel \cite{Pimentel2019} and Osiris \cite{Wang2020} looked into this method.} 
    Our findings on pathological errors in notebooks demonstrate the need for static and dynamic analysis tools to catch such errors early in notebook development. 
    %\waris{btw VSCode Jupyter notebook extension addresses this issue.}


\noindent{\em How can LLMs further restore notebook executability with a high success rate?} 
    In the first case study (Section \ref{case_study_1}), Llama-3 successfully generated synthetic data for a `.csv' file. However, in another case \cite{tirthajyoti}, it failed to produce a valid `PNG' file due to limitations in multi-modal generation. This is because most LLMs have been trained on similar textual data formats. Therefore, Llama-3 also faces limitations when generating images, audio, video, or zip files. Multi-modal models such as GPT-4o can generate richer and more diverse inputs. %Our results with language-specific LLMs indicate that these models have the potential to generate synthetic data that promises high executability. 
    This capability can benefit developers who want to share their notebooks without disclosing the input data files. They can leverage LLMs to generate synthetic data (or scripts for synthetic data generation similar to database benchmarks \cite{tpcds}), enhancing executability in remote environments. Additionally, LLM performance tends to degrade when provided with large contexts, which aligns with recent research on LLMs \cite{Leng2024}. Moreover, notebook executability can be enhanced by generating feedback-driven fixes (e.g., multi-shot) for non-executable cells. This includes generating new input files based on error feedback.  
    %For example, if a cell error is caused by a missing file, an LLM could generate the necessary data or adjust paths to match the environment. 
    % Such a framework would greatly benefit notebook developers, enabling quick execution of public notebooks and addressing issues like missing files with minimal manual intervention.


%Moreover, we can further improve the execution of notebook cells by generating feedback-driven fixes for non-executable cells. For instance, we can achieve this by allowing code edits in notebook cells or generating newer input based on the error related to generating input files.  If we get an error in a cell, we can ask the LLM to generate newer file data if it's related to the generated file in the same context or if it requires a code edit (e.g., renaming the file path from a Windows machine path to a Linux machine path) to make it executable. Such a framework will be highly beneficial for notebook developers. It allows them to quickly run public notebooks and automatically address issues such as missing files, which would otherwise require manual effort.
%  Also, if an LLM generates six columns of a CSV correctly but the name or the data in the last column is not correct, we can ask the LLM to generate the last column data correctly. 



\noindent{\em What use is there for partially executable notebooks?}
    Prolonging notebook execution leads to more cells being executed successfully, which translates into more code surfaces that can be executed and understood. This inherently improves code reuse and enhances the chances of notebooks being fully restored. Furthermore, dynamic analysis techniques are limited to executable code only. By improving notebook executability, we increase the application surface of dynamic analysis tools such as dynamic taint analysis \cite{clause2007dytan}, runtime tracing \cite{meliou2011tracing}, automated debugging \cite{parnin2011automated}, symbolic execution \cite{baldoni2018survey}, and Osiris \cite{Wang2020} that can play vital roles in recovering notebook reproducibility. %Additionally, code corpora have proven extremely valuable in training new emerging LLMs. A big challenge is verifying the correctness of training data. More executable cells offer pathways to verified/tested code snippets that can be included in model training. 
    Partial executability metrics can also provide fine-grained measures of the effectiveness of reproducibility and execution-restoration tools. Additionally, it can serve as valuable feedback when applying code repair techniques in notebooks, such as automated cell reordering \cite{Wang2020}. 
    
    
\noindent {\em How do the findings of this study improve notebooks?} 
    We explored several design choices in building our measurement framework. We identify gaps in the notebook ecosystem for code review, debugging, test generation, and build tools. For existing public notebooks, which are growing exponentially \cite{Rule2018}, this measurement framework can be extended to improve fine-grained executability, feeding into dynamic analysis tools to enhance reproducibility. 
    
    Our findings, showing that only \totalRequirement (\percentRequirementInTotal) repositories have REQUIREMENTS files, indicate that there is no standardized build and continuous integration mechanism for notebooks. While this may not apply to exploratory, standalone, one-off notebooks, the most popular repositories provide a range of notebooks that must be properly orchestrated to ensure executability. We believe Python-based build tools, such as {\small{\texttt{disutils}}} \cite{distutil} and {\small{\texttt{setuptools}}} \cite{setuptools}, can be extended to support notebook extensions at the browser level to enforce correct build and dependency management for notebooks.


 \noindent {\em Threats to validity.}
    Outdated, textual, and empty notebooks can cause variations in the results of this study. To mitigate that, we exclude notebooks written in Python 2 since support for Python 2 was discontinued in 2020. We also filter empty or instructional notebooks from our dataset. Similarly, focusing purely on GitHub repositories can bias the results towards a specific class of notebooks. Upon investigation, we identified numerous HuggingFace and Kaggle notebooks that are also hosted on GitHub. While there is always room to increase the scale of the analysis, our emphasis on popular, actively reusable notebooks deprioritizes the need to expand the scale. Our static and dynamic analyses rely on the Python AST parser and Python interpreter to capture errors. These tools can potentially miss or misclassify errors, which can impact our categorization. 
        %We do not capture semantic errors, nor do we measure reproducibility. 
    Since most of our runtime errors are captured through dynamic error checking, an earlier fatal runtime error (that we cannot resolve) may hinder our ability to capture ``FileNotFound" or ``ModuleNotFound" errors. Lastly,  we use a timeout of 5 minutes to execute each notebook. 
        %Any notebook taking more than 5 minutes is considered unanalyzable; thus, we exclude it from our dataset. 
    This approach may exclude notebooks like the ones requiring expensive machine learning training. However, in our notebook corpus, only \percentTimeout of notebooks encountered timeout errors.

\vspace{-0.5ex}
\section{Related Work}
\label{related work}

    With the increasing popularity of notebooks \cite{Kluyver2016}, a growing body of research has focused on understanding the unique characteristics of notebooks \cite{Chattopadhyay2020, Rule2018, Rule2018CellFolding, In2024, gigascience2024} and their diverse applications in various fields \cite{Wang2019, Randles2017, Li2021}. Recent works have also proposed potential enhancements to these tools, aiming to improve notebooks' functionality and usability for data scientists and other users \cite{Chattopadhyay2020, Wang2024, McNutt2023, gigascience2024}. Table \ref{related_work_table} summarizes and compares this work with prior investigations. %We also explain the dataset's characteristics in each work to show the underlying reason for varying rates of non-executable notebooks reported.

    \subsection{Empirical Studies on Notebook Executability}
        Pimental et al. \cite{Pimentel2019, Pimentel2021} conducted an analysis of 1.4 million notebooks sourced from GitHub, aiming to examine their nature, quality, and reproducibility. They report notebook quality results, covering various characteristics and insights into notebook execution processes, as well as common issues encountered within these notebooks. Similarly, Wang et al. \cite{Wang2019BetterCode} assesses the quality of code present in notebooks, ultimately concluding that notebooks often exhibit suboptimal coding practices, thus highlighting the importance of enhancing notebook code quality. Both studies only view notebook executability as an atomic measure. In particular, they lack a fine-grained, deeper analysis of a large portion of notebooks that are (1) non-executable only on non-author environments due to misconfiguration, thus fully restorable, and (2) offer reusable, valuable code that is partially executable. 

    

    \subsection{Notebook Restoration and Reproducibility Tools}
        Numerous tools are proposed to restore notebook executability and reproducibility. For instance, RELANCER automatically updates deprecated APIs in non-executable notebooks by gathering APIs from GitHub and documentation \cite{Zhu2021}. Similarly, SnifferDog restores the execution environment by providing a collection of APIs from Python packages and libraries to update required packages in notebooks \cite{Wang2021}. These approaches complement our analysis. However, due to the highly complex and heavyweight nature of these techniques, they are infeasible for large-scale notebook executability analysis. 
        %Their use of machine learning is also likely to apply code rewrite that may change the notebook's semantics. 
        Osiris \cite{Wang2020} restores the reproducibility of only fully executable notebooks that also contain output cells. It addresses cell dependencies and generates possible execution orders, similar to our def-use set-based reordering. Osiris demonstrates that without executable notebooks, reproducibility can not feasibly be restored. Similar to our investigation, a transparent, fine-grained view of the state of notebook executability can facilitate further reproducibility studies or tool development.

    
    \subsection{Notebook Development Assistance Tools}
        Previous research has proposed improving software development practices with extended user interfaces and advanced programming assistant features. Fork It \cite{Weinman2021} is a forking and backtracking extension designed to investigate various alternatives and traverse different states within a single notebook. Similarly, numerous notebook tools \cite{Rule2018CellFolding, Li2024, Zhu2024Facilitating, Wang202Conflict, SuperNOVA} aim to streamline collaborative and exploratory experiments, often incorporating new visualization and conflict resolution for collaborative work in notebooks. To assist in code debugging and cleaning,  Robinson et al. \cite{Robinson2022} examined error identification strategies used for Python notebooks, while Head et al. \cite{Head2019} utilized program slicing to extract relevant code cells producing specific outputs.  
        %The most appealing feature of the notebooks is their simple and interactive execution environment that facilitates rapid experimental programming.       
        Adding new features and complicated user interfaces significantly intervenes with rapid, interactive programming. Most of these tools have seen resistance in adoption by both users and popular notebook platform providers. With the emergence of LLMs, recent work has explored using LLMs in assisting notebook development \cite{McNutt2023, Wang2024, Weber2024Computational, grotov2024untangling}. These results resonate with our findings.  

        
    


\section{Conclusion}
\label{sec:conclusion}

Computational notebooks, widely used for data science and ML/AI tasks, continually suffer from executability issues. Prior studies reporting the non-executability of notebooks rely on a rigid definition of executability, leading to over-estimating non-executable notebooks. For the first time, we introduce the notions of partial executability and pathological non-executability for notebooks, contextualizing executability to the notebook's interactive computing paradigm. Our investigation finds \totalPathological out of \totalNotebooksInDataset notebooks are pathologically non-executable, while \totalRestorable can be restored given suitable execution environments. We leverage LLM-based error-driven restoration techniques to fully restored \percentFullyRestored and partially restored \percentPartiallyRestored of previously non-executable notebooks. These results offer key evidence that notebooks can benefit from LLM-based restoration, and partial executable notebooks are still valuable for broader code reuse practices. 


\section*{Acknowledgement} We thank anonymous reviewers for providing valuable and constructive feedback to help improve the quality of this work. This work was supported in part by Amazon - Virginia Tech Initiative in Efficient and Robust Machine Learning, 4-VA, and the National Science Foundation award 2106420. We also thank the Advanced Research Computing Center at Virginia Tech for their support in building and evaluating this work.



\balance

\documentclass{MITstyle}

%\usepackage[table]{xcolor}
\usepackage{chngcntr}
\usepackage{hyperref}
\usepackage{microtype}

\title{A Lightweight and Extensible Cell Segmentation and Classification Model for Whole Slide Images}

\author{Nikita Shvetsov~$^{1, }$\footnote{Correspondence e-mail: nikita.shvetsov@uit.no}, Thomas K. Kilvaer~$^{2, 3}$, Masoud Tafavvoghi~$^{4}$, Anders Sildnes~$^{1}$, \\ Kajsa Møllersen~$^{4}$, Lill-Tove Rasmussen Busund~$^{5, 6}$, Lars Ailo Bongo~$^{1}$ \\
%
\vspace{1em} % Space between authors and afilliations
%
\normalfont{\small $^{1}$Department of Computer Science, UiT The Arctic University of Norway}\\
\normalfont{\small $^{2}$Department of Oncology, University Hospital of North Norway}\\
\normalfont{\small $^{3}$Department of Clinical Medicine, UiT The Arctic University of Norway}\\
\normalfont{\small $^{4}$Department of Community Medicine, UiT The Arctic University of Norway}\\
\normalfont{\small $^{5}$Department of Medical Biology, UiT The Arctic University of Norway} \\
\normalfont{\small $^{6}$Department of Clinical Pathology, University Hospital of North Norway} %\vspace{2em}
}

\begin{document}
\maketitle

\section*{Abstract}

% \begin{abstract}
% Developing clinically useful cell-level analysis tools in digital pathology remains challenging due to limitations in dataset granularity, inconsistent annotations, computational demands of advanced models, and difficulties in integrating new technologies into clinical workflows. To address these challenges, we propose a multi-faceted solution that enhances data quality, model performance, and usability to create a lightweight and extensible cell segmentation and classification model.

% First, we update data labels by employing a cross-relabeling process that refines the labels of two existing datasets, PanNuke and MoNuSAC, to create a new unified dataset with enhanced granularity, encompassing seven distinct cell types. Second, we leverage the H-Optimus foundation model as a fixed encoder to improve feature representation for simultaneous cell segmentation and classification tasks. Third, to address the computational demands of foundation models, we employ knowledge distillation to reduce model size and complexity while maintaining comparable performance. Finally, to facilitate integration into clinical workflows, we integrate the distilled model into the QuPath software, a widely used open-source platform in digital pathology.

% Our results demonstrate improvements in cell segmentation and classification performance using the H‑Optimus-based model compared to a CNN-based model. Specifically, the average $R^2$ improved from 0.575 to 0.871, and the average $PQ$ score improved from 0.450 to 0.492, indicating better alignment with actual cell counts and enhanced segmentation and classification quality. Furthermore, the distilled student model maintains performance comparable to the larger foundation model while reducing the parameter count by a factor of 48.
% Overall, by reducing computational complexity and integrating it into existing workflows, the proposed approach may significantly impact diagnostic processes, reduce the workload of pathologists, and contribute to improved patient outcomes. Though our approach shows potential enhancements in efficiency and usability of cell segmentation and classification models in digital pathology, extensive validation is needed to deploy these models in clinical practice.
% \end{abstract}

%%% shortened abstract
\begin{abstract}
Developing clinically useful cell-level analysis tools in digital pathology remains challenging due to limitations in dataset granularity, inconsistent annotations, high computational demands, and difficulties integrating new technologies into workflows. To address these issues, we propose a solution that enhances data quality, model performance, and usability by creating a lightweight, extensible cell segmentation and classification model. 

First, we update data labels through cross-relabeling to refine annotations of PanNuke and MoNuSAC, producing a unified dataset with seven distinct cell types. Second, we leverage the H-Optimus foundation model as a fixed encoder to improve feature representation for simultaneous segmentation and classification tasks. Third, to address foundation models' computational demands, we distill knowledge to reduce model size and complexity while maintaining comparable performance. Finally, we integrate the distilled model into QuPath, a widely used open-source digital pathology platform. 

Results demonstrate improved segmentation and classification performance using the H-Optimus-based model compared to a CNN-based model. Specifically, average $R^2$ improved from 0.575 to 0.871, and average $PQ$ score improved from 0.450 to 0.492, indicating better alignment with actual cell counts and enhanced segmentation quality. The distilled model maintains comparable performance while reducing parameter count by a factor of 48. By reducing computational complexity and integrating into workflows, this approach may significantly impact diagnostics, reduce pathologist workload, and improve outcomes. Although the method shows promise, extensive validation is necessary prior to clinical deployment.
\end{abstract}
\clearpage

\section{Introduction}
In digital pathology, accurate segmentation and classification of cells are crucial for many diagnostic, prognostic, and predictive analyses \cite{Jaber_Beziaeva_etal._2019,Lin_Pan_etal._2022,Park_Ock_etal._2022,Shen_Choi_etal._2024}. Nowadays, developments in computational pathology offer multiple solutions \cite{H._Qu_P._Wu_etal._2020,Javed_Mahmood_etal._2020} to utilize cell-level datasets to train machine learning models that solve these problems. The quality and specificity of training datasets are critical for robust and accurate models. Adhering to the principle of "garbage in, garbage out", it is essential to ensure that these datasets are extensively and accurately labeled with distinct classes that reflect the diverse biological characteristics of different cell types. Unfortunately, the number of open-source datasets comprising such high-quality annotations is limited. Existing cell segmentation datasets \cite{Gamper_Koohbanani_etal._2019,Graham_Vu_etal._2019,Verma_Kumar_etal._2021} may offer extensive annotations for certain cell types while providing more general labels for others. For example, in PanNuke, which is one of the largest open-source datasets comprising labeled cells, various types of morphologically and functionally different inflammatory cells like macrophages and lymphocytes are clustered in a broad "inflammatory" class. Consequently, these classes are frequently omitted from analyses or aggregated into broader meta-classes \cite{Gamper_Koohbanani_etal._2020} and likely interfere with other cell classes included in the dataset. This and similar inconsistencies in annotation granularity limit the ability of machine learning models to learn the comprehensive and nuanced features necessary for accurate cell segmentation and classification. To address these challenges, methods for refining and standardizing dataset annotations are essential to enhance the quality of training data.

A complementary approach to mitigate the absence of high-quality training data is the use of foundation models. Foundation models as encoders are defined as large-scale, versatile networks pre-trained on vast, diverse datasets using self-supervised learning, contrasting with convolutional neural network (CNN) pre-trained encoders that rely on supervised learning with labeled data. In practice, foundation models leverage enormous amounts of weakly or unlabeled data from millions of whole slide images (WSIs) and employ self-attention mechanisms to capture long-range dependencies and global context \cite{Chen_Ding_etal._2024,Saillard_Jenatton_etal._2024,Vorontsov_Bozkurt_etal._2024,Xu_Usuyama_etal._2024}. As a consequence, foundation models are able to produce transferable feature representations across different cell types and tissue environments. The feature representations can be leveraged by decoder networks to produce segmentation masks and pixel-level classifications. Because foundation models have comprehensive feature representations, they can be effectively fine-tuned using much smaller amounts of cell-level data compared to the large datasets needed to train models from scratch. Furthermore, foundation models incorporate adversarial training elements or contrastive learning \cite{Chen_Ding_etal._2024,Xu_Usuyama_etal._2024}, enhancing their resilience and adaptability by exposing them to challenging and varied scenarios during training. This may result in more generalizable models, often making them well-suited for diverse and complex tasks in digital pathology.

Despite the inherent advantages of foundation models, their deployment for practical use faces its own obstacles. In particular, they require substantial computational power, financial investments and rigorous testing to ensure reliability and efficacy for a given task \cite{Akkus_Dangott_etal._2022,Dragomir_Cocuz_etal._2022,Go_2022,Jafri_Farooqui_etal._2024}. Moreover, while foundation models enhance feature representation and performance, they depend on the quality of available annotations for decoder fine-tuning and, like any other model, cannot resolve existing inconsistencies or ambiguities in data labels. Therefore, there remains a critical need for solutions that address both data quality and practical deployment considerations.
Further, integrating new technologies into existing clinical workflows often encounters resistance, as it necessitates adjustments to established diagnostic processes. So, there is a need to develop solutions that could be integrated into current practices, minimizing the burden on medical professionals to adopt new tools \cite{King_Williams_etal._2023}.

Existing solutions \cite{Goldsborough_Philps_etal._2024,Hörst_Rempe_etal._2024}, while addressing some aspects of these challenges, fall short in providing a comprehensive approach. To address the data quality and clinical deployment issues, we propose a multi-faceted solution that encompasses data refinement, model optimization, and integration with existing pathology tools (\hyperref[fig:fig1]{Figure 1}). The outcome is a lightweight cell segmentation and classification model that can be integrated into digital pathology workflows for practical clinical use.

\begin{figure}[h!]
    \centering
    \includegraphics[width=\textwidth, height=0.82\textheight, keepaspectratio]{images/Figure_1.pdf}
    \caption{Overview of the proposed solution, including 1) Data refinement using cross-relabeling, 2) Teacher model development and fine tuning, 3) Student model optimization with knowledge distillation and 4) Student model and QuPath integration}
    \label{fig:fig1}
\end{figure}
\clearpage

Our approach begins with preparing the data for the fine-tuning and training of the machine learning models. We create a refined dataset, acquired via cross-relabeling two cell-level datasets, enhancing annotation specificity and consistency of the labeled data. Subsequently, we create a cell segmentation and classification model based on the foundation model. We leverage the foundation model as a fixed encoder and fine-tune a decoder using the refined dataset to improve generalization across diverse tissue- and cell types.
To ensure that the model remains lightweight and deployable in a possibly resource-constrained environment, we employ knowledge distillation to approximate the functionality of the foundation model. Finally, to facilitate the practical application of our model in digital pathology workflows, we integrate it with the QuPath \cite{Bankhead_Loughrey_etal._2017} application. Each methodological component contributes to the overarching goal of enhancing model performance, generalizability, and usability in clinical settings.

The primary contributions of this paper are:
\begin{enumerate}
    \item \textit{Data labels refinement through cross-relabeling:}
    
    We propose a new method for refining labels of cell-level datasets through cross-relabeling. This method employs classification models to re-label broad and ambiguous instances, resulting in a more diverse dataset. Our evaluation demonstrates that these classification models achieve high accuracy on test subsets, indicating the reliability of the method for label refinement.

    \item \textit{Enhanced model performance via foundation models:}
    
    We employ a foundation model as a feature extractor for the cell segmentation and classification task. In comparison with training a CNN model from scratch, the foundation model backbone only needs fine-tuning, which significantly reduces training time, computational resources and data requirements. We show that using a foundation model encoder leads to better performance in cell segmentation and classification networks than using a CNN-based encoder. This improvement may enable the model to generalize more effectively across various tissue types and imaging methods.
    
    \item \textit{Model optimization through knowledge distillation:}
    
    We show that a smaller student model trained using knowledge distillation on the refined dataset obtained via our cross-relabeling approach from a foundation model achieves comparable performance in cell segmentation and quantification tasks. As a result, this model is more suitable for deployment in environments without high-performance computing resources.
    
    \item \textit{Integration with QuPath:}
    
    We integrate the distilled cell segmentation and classification model into QuPath, a widely used open-source digital pathology platform, to accelerate clinical adaptation by enabling pathologists to more easily incorporate advanced computational tools into their existing workflows.
\end{enumerate}

Through these methodological steps, we aim to bridge the gap between advanced machine learning techniques and practical clinical applications, making accurate and efficient digital pathology accessible in a broader range of healthcare settings.

\section{Refining Existing Datasets Using Cross-Relabeling}
To address the limitations of sparse and ambiguous labeling of cell-level datasets, we propose a generalizable cross-relabeling strategy that can be applied to any dataset containing broadly categorized or imprecisely labeled cell types. This approach involves training and subsequently leveraging classification models to refine broad categories into more specific or biologically relevant classes.
When applied to cell-level data, the methodology includes extracting individual cell images from the dataset patches, preprocessing these images to standardize the size and accommodate partial cells, and then training deep learning classifiers capable of distinguishing between the finer cell subtypes within the coarser categories. 
To illustrate our approach, we focus on the PanNuke \cite{Gamper_Koohbanani_etal._2020, Gamper_Koohbanani_etal._2019} and MoNuSAC \cite{Verma_Kumar_etal._2021} datasets that we have used to train models for cell quantification in our previous works \cite{Shvetsov_Grønnesby_etal._2022,Shvetsov_Sildnes_etal._2024}. We find that for better cell differentiation we have to introduce more granular labels. PanNuke includes a broad classification of "inflammatory" cells, encompassing lymphocytes, macrophages, and neutrophils. Each cell type differs significantly in structure, function, and clinical relevance. Conversely, MoNuSAC uses the label "epithelial" for a class that comprises both benign epithelial cells and malignant neoplastic cells. This practice makes it challenging to differentiate between benign and malignant epithelial cells in the dataset, which is a critical distinction when identifying tumor areas within tissue samples. To address these issues, we implement a cross-relabeling strategy as shown in \hyperref[fig:fig2]{Figure 2}. The key components are two classification models: one is trained on singular cell images from PanNuke data to classify the epithelial meta-class into epithelial and neoplastic classes. The other is trained on MoNuSAC to refine the inflammatory class into lymphocytes, neutrophils, and macrophages.

\begin{figure}[h!]
    \centering
    \includegraphics[width=\textwidth]{images/Figure_2.pdf}
    \caption{Refined dataset generation via cross relabeling}
    \label{fig:fig2}
\end{figure}

The refining approach consists of three consecutive steps. The first is the preprocessing step, in which we extract individual cells from both datasets (\hyperref[fig:fig3]{Figure 3}). The specifics of PanNuke and MoNuSAC patch preparation before cell preprocessing are provided in \hyperref[chap:S1]{Appendix S1}.

\begin{figure}[h!]
    \centering
    \includegraphics[width=\textwidth]{images/Figure_3.pdf}
    \caption{Cell instances preprocessing including (1) cell map extraction, (2) bounding box delineation, (3) adjusting cell boxes and (4) cropping and resizing of cell images}
    \label{fig:fig3}
\end{figure}

During preprocessing, we extract cell type maps from the ground truth label mask and calculate bounding boxes around each cell instance. To accommodate partial cells at patch borders, a common issue in cropped patch images, we employ mirror padding and extend the field of view of the cell label by 15 pixels to capture adjacent cells. We then crop and resize the identified regions to $64 \times 64$ pixels using bicubic interpolation.

The preprocessed PanNuke dataset comprises 68,031 neoplastic and 23,207 epithelial cell images, while MoNuSAC comprises  33,104 lymphocytes, 1,252 neutrophils, and 1,695 macrophages, which we subsequently use in training cell classification models and classifying the cell image data \hyperref[fig:S2]{Appendix Figure S2 (1)}. 

The next step is to train two distinct ResNet50-based classifiers tailored to address the specific labeling challenges inherent in each dataset. We use ResNet50 for classification models due to its proven effectiveness for image classification tasks in histopathology \cite{pan2022reviewmachinelearningapproaches}, and its compatibility with small images. For the PanNuke dataset, we design the classifier, trained on MoNuSAC data, to disaggregate the heterogeneous "inflammatory" cell category into distinct subtypes: lymphocytes, macrophages, and neutrophils. Similarly, for the MoNuSAC dataset, the classifier is trained on PanNuke data and distinguishes between benign and malignant epithelial cells within the overarching "epithelial" label. By applying these targeted classifiers to their respective datasets, we assign more specific labels to individual cell instances, thus enabling us to create a unified dataset.
To ensure a balanced representation of classes, we train both models on datasets that had been equalized to match the size of the least represented class. Thus, we obtain datasets comprising 23,207 samples per class for PanNuke and 1,252 samples per class for MoNuSAC data. Next, we partition both of them into training (70\%), validation (20\%), and testing (10\%) subsets. To mitigate the risk of overfitting, we use a single dropout layer with a rate of p=0.5 in both models and data augmentation using randomized color perturbations, rotation, and horizontal and vertical flipping. We employ AdamW optimizer and the cross-entropy loss function for the training criterion.

To evaluate the two trained models, we measure the classification accuracy on the respective test subsets. The accuracies on the test subset for both classifiers are presented in \hyperref[tab:1]{Table 1}. The PanNuke model achieves an average accuracy of 93.57\%, with higher accuracy for neoplastic cells (96.06\%) compared to epithelial cells (86.26\%). The confusion matrix in Figure A3.1 shows that the model predominantly distinguishes accurately between epithelial and neoplastic tissues, with a substantial number of correct classifications and relatively few misclassifications. The MoNuSAC model demonstrates an average accuracy of 98.92\%, excelling in classifying lymphocytes (99.67\%) and macrophages (94.12\%), with lower performance for neutrophils (85.71\%). The confusion matrix in Figure A3.2 shows that the model identifies lymphocytes and performs reasonably well with macrophages and neutrophils.

\begin{table}[h!]
\renewcommand{\arraystretch}{1.5}
  \centering
  \caption{Cell classification results for PanNuke and MoNuSAC trained models (CI 95\%).}
  \label{tab:1}
  \begin{tabular}{|l|c|c|}
   \hline
   %\rowcolor{gray!30}
    Accuracy               & PanNuke model              & MoNuSAC model              \\
    \hline
    Average      & 0.936 (0.931--0.941)         & 0.989 (0.986--0.993)        \\
    \hline
    Neoplastic   & 0.961 (0.956--0.965)         & -                          \\
    \hline
    Epithelial   & 0.863 (0.849--0.877)         & -                          \\
    \hline
    Lymphocytes  & -                          & 0.997 (0.995--0.999)        \\
    \hline
    Neutrophils  & -                          & 0.857 (0.796--0.918)        \\
    \hline
    Macrophages  & -                          & 0.941 (0.906--0.976)        \\
    \hline
  \end{tabular}
\end{table}

Finally, during the last step, we use the model trained on PanNuke data for epithelial cells in MoNuSAC and the model trained on MoNuSAC for the inflammatory cells class in PanNuke. Specifically, we use classifier models to relabel epithelial cells in MoNuSAC and inflammatory cells in PanNuke data. Then we combine cells with refined labels and the rest of the cells in both datasets to create a refined dataset (\hyperref[fig:S2]{Appendix Figure S2 (2)}). The process of relabeling cells and visualizing them on a patch is shown in \hyperref[fig:fig4]{Figure 4}. The cell counts in the refined dataset are provided in \hyperref[tab:S4]{Appendix Table S4}.

\begin{figure}[h!]
    \centering
    \includegraphics[width=\textwidth, height=0.42\textheight, keepaspectratio]{images/Figure_4.pdf}
    \caption{Cell relabeling procedure for epithelial and inflammatory cell classes}
    \label{fig:fig4}
\end{figure}

%\hfill

Relabeling and combining datasets have been explored in a prior study \cite{Parulekar_Kanwat_etal._2023}, where consecutive fine-tuning on multiple datasets was employed to account for hierarchical class label structures. While the method presented in \cite{Parulekar_Kanwat_etal._2023} is intuitive, it often lacks consistency and requires multiple fine-tuning runs, which can be cumbersome and time-consuming. 
In contrast, cross-relabeling simplifies this process by using specialized classification models tailored to each dataset's specific labeling challenges. This approach provides better transparency and produces a unified dataset encompassing seven distinct cell types across multiple tissue samples, enhancing data diversity for further model training or fine-tuning.

Despite these improvements, cross-relabeling does not entirely resolve issues related to poor labeling quality or the amount of labeled data. Specifically, our results show lower accuracies persist for underrepresented classes, such as macrophages, which may stem from a limited sample availability and intrinsic challenges in distinguishing these cells based solely on H\&E staining. Furthermore, while our method enhances label specificity, it relies on the initial quality of the broad labels; thus, any fundamental inaccuracies in the original annotations can propagate through the relabeling process. Addressing the overall problem of limited data labels may require integrating additional data sources or utilizing complementary immunohistochemical staining methods.
Although the reported performance metrics are obtained from evaluations on the native test sets of each dataset, it is important to note that the primary application of these classifiers is to perform cross-relabeling, where a model trained on one dataset (e.g., PanNuke) is applied to another (e.g., MoNuSAC) and vice versa. We acknowledge that a more systematic evaluation of cross-dataset generalization is needed and could be performed in future work.

Overall, the refined dataset produced by our approach can enhance the supervised training or fine-tuning of cell segmentation and classification models, especially those that utilize pre-trained foundation models to improve feature extraction robustness. In addition, these models can detect nuanced classes that enable researchers to conduct more detailed analyses of biological processes in computational pathology.

\section{Foundation models for robust cell segmentation and classification}

Accurate cell segmentation and classification in digital pathology are hindered by limited labeled data and the fact that conventional CNNs are unable to capture global contextual information due to their local receptive field constraints \cite{Gheflati_Rivaz_2022,Yang_Marcus_etal.}. Traditional approaches in cell quantification have predominantly relied on CNN encoders, such as ResNet50, given their proven effectiveness in semantic segmentation tasks \cite{Deshmane_2023,Graham_Vu_etal._2019,Mukasheva_Koishiyeva_etal._2024,Stringer_Wang_etal._2021}. However, approaches that include fine-tuning of pretrained CNNs, data augmentation, and stain normalization to partially increase data variability and address staining differences often fail to achieve the necessary generalization and robustness across diverse tissue types and staining conditions \cite{G._Wang_W._Li_etal._2018,Gao_Bagci_etal._2018,Karim_El_Khoury_Martin_Fockedey_etal._2021}.

To overcome these challenges, we leverage an encoder-decoder network that uses a foundation model as the encoder and a CNN upsampling decoder (\hyperref[fig:fig5]{Figure 5}) for simultaneous cell segmentation and classification in 2D patches extracted from WSIs. Foundation models with transformer-based architectures are viable alternatives to CNN-based encoders \cite{Shamshad_Khan_etal._2023,Sourget_2023}. They enable the creation of more advanced architectures that can decode or transform learned features more effectively \cite{Chen_Duan_etal._2023,Cheng_Misra_etal._2022,Xie_Wang_etal._2021}.

\begin{figure}[h!]
    \centering
    \includegraphics[width=\textwidth]{images/Figure_5.pdf}
    \caption{UNETR-like model with foundational model as backbone}
    \label{fig:fig5}
\end{figure}

By utilizing a transformer-based encoder, we incorporate global contextual information into the feature extraction process, which is a key advantage of such architectures \cite{Chen_Lu_etal._2021}. This foundation model integration facilitates accurate pixel-wise segmentation and classification without the need for extensive encoder training, thereby potentially improving generalization across varied cellular structures and tissue types.
In our implementation, we employ a modified UNETR \cite{Hatamizadeh_Tang_etal._2021} architecture that combines a vision transformer (ViT) \cite{Dosovitskiy_Beyer_etal._2021} encoder with a CNN-based decoder. The encoder utilizes the pretrained H-Optimus foundation model, which contains 1.1 billion parameters and is trained on over 500,000 H\&E stained WSIs \cite{Saillard_Jenatton_etal._2024}. We extract outputs from four evenly spaced transformer blocks $Z_i$, where $i \in [1, 14, 26, 38]$, to serve as residual connections for the CNN decoder. We select these blocks based on our observation that features from non-adjacent levels of the encoder lead to better overall performance on the test subset.

The CNN decoder upsamples the feature representations, acquired from the transformer blocks, to generate an intermediate vector that is handled by two task-specific layers that generate cell segmentation and classification masks. The first task-specific layer is the ‘Cellpose head’,  which is used to delineate cell instances. The layer generates horizontal and vertical gradient maps to form vector fields that are refined through gradient tracking in a post-processing step using the Cellpose algorithm \cite{Stringer_Wang_etal._2021}, known for its efficacy in cell segmentation tasks and generalizability across multiple domains \cite{Pachitariu_Stringer_2022,Stringer_Pachitariu_2024}. The second task-specific layer is the "Cell type head", which assigns labels to individual pixels. In the post-processing step, we determine the output classification label of each segmented cell instance by majority voting over the labeled pixels that comprise the cell in the segmentation map.

To evaluate model performance and measure the impact of adding a foundation model as backbone, we compare it to a ResNet50-based model. ResNet50 is a widely used solution for encoders in segmentation architectures in the medical domain \cite{Deshmane_2023,Graham_Vu_etal._2019,Mukasheva_Koishiyeva_etal._2024,Stringer_Wang_etal._2021}. For the H-Optimus-based model, we utilize frozen weights for the encoder and only fine-tune the decoder to take advantage of the extensive pre-training of the foundation model. For the ResNet50-based model we start with ImageNet \cite{Deng_Dong_etal.} weights and train both encoder and decoder parts. Hyperparameters for the training step are set to be identical, where possible, for comparable evaluation. 
For this evaluation, we deliberately use the PanNuke dataset to provide a standardized and controlled comparison between the H‑Optimus and ResNet50-based models (\hyperref[fig:S2]{Appendix Figure S2 (3)}). Specifically, we use two of the default PanNuke dataset splits (66\%) for training and validation, and reserve the third split (33\%) for testing.

To address the challenge of cell class imbalance in the PanNuke dataset, which is a common characteristic in most cell-level H\&E patch datasets, both models’ training processes employ a weighted loss function comprising cross-entropy and focal loss \cite{Lin_Goyal_etal._2018}. The focal loss component is adjusted with coefficients derived from each cell class' instance frequency, emphasizing learning from underrepresented classes and enhancing the model's sensitivity to rare but significant cellular patterns. The cross-entropy loss is augmented with spectral decoupling regularization \cite{Pezeshki_Kaba_etal._2021,Pohjonen_Stürenberg_etal._2022} and spatially varying label smoothing \cite{Islam_Glocker_2021}, which potentially stabilizes training and improves generalization in case of complex tissue morphologies. For optimization, we employ the \textit{AdamW} \cite{Loshchilov_Hutter_2019} to counter unbalanced class scenarios, with cosine annealing learning rate scheduler.

We utilize the scikit-learn library \cite{Van_der_Walt_Schönberger_etal._2014} and HoVer-Net \cite{Graham_Vu_etal._2019} implementations of $R^2$ (the coefficient of determination) and $PQ$ (panoptic quality) to evaluate our experiments. Complete mathematical formulations and detailed explanations of these metrics are provided in \hyperref[chap:S5]{Appendix S5}. To compute confidence intervals, we use nonparametric bootstrapping, where after calculating the metric on the full sample, we generated 1000 bootstrap replicates by resampling with replacement and then determined the 95\% confidence intervals as the 2.5th and 97.5th percentiles of the resulting empirical distribution.

%\hfill

The model comparisons are summarized in \hyperref[tab:2]{Table 2}. The H‑Optimus-based model achieves higher $R^2$ across all cell classes compared to the ResNet50-based model, which means that its predictions are more closely aligned with the PanNuke cell counts, indicating a stronger correlation with the observed data. Notably, the improvement of $R^2_{dead}$ may be an indicator of better global contextual representations provided by the foundation model backbone. In terms of segmentation and classification quality combined, measured by the PQ score, the H‑Optimus-based model demonstrates notable improvements across most cell classes. Overall, the average $R^2$ improved from 0.575 to 0.871, while the average $PQ$ score improved from 0.450 to 0.492, demonstrating better performance of the H-Optimus-based model.

\begin{table}[h!]
\renewcommand{\arraystretch}{1.5}
  \centering
  \caption{Cell quantification metrics for baseline and proposed models (CI 95\%).}
  \label{tab:2}
  \begin{tabular}{|l|c|c|}
    \hline
    %\rowcolor{gray!30}
    Metric             & Resnet50-based            & H-optimus-based              \\
    \hline
    $R^2_{neoplastic}$    & 0.681 (0.576--0.769)       & \textbf{0.941 (0.917--0.960)} \\
    \hline
    $R^2_{inflammatory}$  & 0.863 (0.778--0.903)       & \textbf{0.949 (0.918--0.966)} \\
    \hline
    $R^2_{connective}$    & 0.600 (0.488--0.698)       & 0.609 (0.436--0.772)          \\
    \hline
    $R^2_{dead}$          & 0.097 (-11.389--0.669)     & 0.925 (0.404--0.982)          \\
    \hline
    $R^2_{epithelial}$    & 0.635 (0.490--0.747)       & \textbf{0.930 (0.886--0.964)} \\
    \hline
    $PQ_{neoplastic}$       & 0.517 (0.499--0.535)       & \textbf{0.589 (0.575--0.604)} \\
    \hline
    $PQ_{inflammatory}$     & 0.455 (0.429--0.482)       & \textbf{0.528 (0.507--0.549)} \\
    \hline
    $PQ_{connective}$       & 0.416 (0.400--0.431)       & \textbf{0.451 (0.436--0.465)} \\
    \hline
    $PQ_{dead}$             & 0.374 (0.342--0.408)       & 0.292 (0.209--0.365)          \\
    \hline
    $PQ_{epithelial}$       & 0.488 (0.460--0.519)       & \textbf{0.599 (0.579--0.618)} \\
    \hline
  \end{tabular}
\end{table}

Our results  show that integrating the H‑Optimus foundation model within the UNETR architecture enhances the model's ability to segment and classify cells across diverse tissues from PanNuke data. The pretrained transformer encoder provides robust feature representations, resulting in higher average $R^2$ and $PQ$ scores compared to the CNN-based model. This leads to more reliable cell quantification and more accurate downstream analysis. Additionally, the streamlined fine-tuning process reduces computational overhead and training time, making the model more adaptable for new data.

Despite these advancements, the foundation model-based approach does not fully resolve all challenges related to cell segmentation and classification. We observe lower metric scores for underrepresented classes in the training data. Furthermore, foundation models typically encompass billions of parameters, resulting in substantial computational and memory requirements. It therefore poses challenges for deployment in resource-constrained environments, limiting their practical applicability in certain clinical settings.

\section{Model optimization via Knowledge Distillation}

To address the limitations posed by the extensive size of foundation models, we implement knowledge distillation — a model compression technique that leverages the teacher-student paradigm \cite{Hinton_Vinyals_etal._2015}. By training a smaller, more efficient student model to replicate the output of a larger, pre-trained teacher model, we retain performance while significantly reducing the model's complexity and resource requirements (\hyperref[fig:fig6]{Figure 6}).

\begin{figure}[h!]
    \centering
    \includegraphics[width=\textwidth, height=0.45\textheight, keepaspectratio]{images/Figure_6.pdf}
    \caption{Knowledge distillation framework for training a student model using a pre-trained teacher}
    \label{fig:fig6}
\end{figure}

We employ knowledge distillation to compress the H‑Optimus-based teacher model into a more efficient student model. The teacher model is the modified UNETR architecture with the H‑Optimus foundation model described in the previous chapter. The student model is based on a UNet architecture augmented with residual connections and incorporates a smaller ViT encoder with 9 million parameters \cite{Steiner_Kolesnikov_etal._2022,Wightman_2019}. 

First, we fine-tune the teacher model using the refined dataset from the cross-relabeling procedure (Section 2). Initially we train the decoder of the teacher model while keeping the encoder weights frozen. We split the refined dataset into train (70\%), validation (20\%) and test (10\%) subsets (\hyperref[fig:S2]{Appendix Figure S2 (4)}). During fine-tuning, we use the train and validation subsets, while leaving the test subset for model evaluation. We set the training procedure and model hyperparameters to be identical to those that were used to demonstrate the utility of foundation models for the simultaneous cell segmentation and classification task.

Next, we perform knowledge distillation from teacher to student using the refined dataset used to fine-tune the teacher model. The student model is trained to replicate the teacher model's outputs. We utilize a specialized loss function that aligns the student's predicted probability distribution with the teacher's, incorporating the teacher's class probability distribution derived from the output. Following the methodology of Hinton et al. \cite{Hinton_Vinyals_etal._2015}, we experiment with various hyperparameter settings for the temperature ($T$) and the balancing coefficients ($\alpha$ and $\beta$) in the loss function. We vary $T$ from 1 to 20 and adjust $\alpha$ and $\beta$ to balance the distillation and student losses. Through iterative tuning and evaluation, we identify that setting $T=14$, $\alpha=0.3$, and $\beta=0.7$ yields a configuration that converges and closely approximates the teacher model's performance during training.

Finally, we assess the performance of both models using the $R^2$ and $PQ$ (defined in \hyperref[chap:S5]{Appendix S5}) on the test set of the refined dataset (\hyperref[tab:3]{Table 3}). We observe that the 95\% confidence intervals overlap for most cell types, so we cannot claim statistically significant performance differences between the teacher and student models. One exception appears in the neoplastic class. The teacher model produces an $R^2$ of 0.919, while the student model shows an $R^2$ of 0.852. In addition, the student model achieves higher $PQ$ values for the neoplastic and connective classes, though the confidence intervals show overlap.

\begin{table}[h!]
\renewcommand{\arraystretch}{1.5}
  \centering
  \caption{Cell quantification metrics for teacher and distilled student models (CI 95\%).}
  \label{tab:3}
  \begin{tabular}{|l|c|c|}
    \hline
    %\rowcolor{gray!30}
    Metric & Teacher & Student \\
    \hline
    $R^2_{neoplastic}$    & \textbf{0.919} (0.898--0.939) & 0.852 (0.800--0.891) \\
    \hline
    $R^2_{lymphocyte}$    & 0.969 (0.956--0.977)         & 0.969 (0.956--0.978) \\
    \hline
    $R^2_{connective}$    & 0.694 (0.548--0.809)         & 0.618 (0.469--0.741) \\
    \hline
    $R^2_{dead}$          & 0.755 (0.400--0.908)         & 0.424 (0.100--0.731) \\
    \hline
    $R^2_{epithelial}$    & 0.922 (0.870--0.958)         & 0.843 (0.738--0.917) \\
    \hline
    $R^2_{macrophage}$    & 0.384 (-0.369--0.724)        & 0.704 (0.352--0.859) \\
    \hline
    $R^2_{neutrofil}$     & 0.854 (0.578--0.929)         & 0.833 (0.502--0.925) \\
    \hline
    $PQ_{neoplastic}$       & 0.581 (0.569--0.593)         & 0.601 (0.588--0.613) \\
    \hline
    $PQ_{lymphocyte}$       & 0.536 (0.520--0.553)         & 0.563 (0.544--0.579) \\
    \hline
    $PQ_{connective}$       & 0.436 (0.421--0.451)         & 0.457 (0.441--0.474) \\
    \hline
    $PQ_{dead}$             & 0.272 (0.235--0.315)         & 0.279 (0.201--0.369) \\
    \hline
    $PQ_{epithelial}$       & 0.522 (0.500--0.545)         & 0.530 (0.506--0.555) \\
    \hline
    $PQ_{macrophage}$       & 0.524 (0.459--0.588)         & 0.474 (0.405--0.543) \\
    \hline
    $PQ_{neutrofil}$        & 0.541 (0.490--0.592)         & 0.565 (0.522--0.607) \\
    \hline
  \end{tabular}
\end{table}


We further decompose the $PQ$ metric into its $SQ$ and $DQ$ components (\hyperref[tab:S6]{Appendix Table S6}). Both models produce nearly identical $SQ$ values, which indicates that they predict instance boundaries with similar precision. Although the student model shows some improvement in $DQ$ scores for certain classes, the confidence intervals overlap and do not confirm a statistically significant difference.

We observe that the student and teacher models yield comparable detection performance despite the student model using a much smaller and simpler architecture. A model with fewer parameters reduces the risk of overfitting when training data are scarce relative to the model’s complexity \cite{Farias_Ludermir_etal._2022}. The knowledge distillation process also encourages the student model to focus on the most generalizable detection features learned from the teacher. These factors enable the student model to achieve similar detection performance across different cell types.

Additionally, considering the model sizes reported in \hyperref[tab:4]{Table 4}, the distilled model achieves a significant reduction compared to the teacher model, with a 48-fold decrease in parameter count and a 5.5-fold reduction in on-disk size. In inference mode, the teacher model requires 16 GB of VRAM for a batch size of 32, while the distilled model only needs 3 GB of VRAM for the same batch size. These reductions make the distilled model significantly more practical for fine-tuning and deployment in resource-constrained environments.

\begin{table}[h!]
\renewcommand{\arraystretch}{1.5}
  \centering
  \caption{Parameter counts and size of teacher and distilled model}
  \label{tab:4}
  \adjustbox{max width=\textwidth}{%
  \begin{tabular}{|l|c|c|c|}
    \hline
    %\rowcolor{gray!30}
    Metric & H-optimus-based (Teacher) & mobileViT-based (Student) & Magnitude of difference \\
    \hline
    Parameters count       & 1,158,917,906   & \textbf{24,093,393}   & \textbf{48x}  \\
    \hline
    Estimated Total Size (MB) & 87,912       & \textbf{15,935}    & \textbf{5.5x} \\
    \hline
  \end{tabular}%
}
\end{table}

%\hfill

With recent advancements in complex network architectures and the use of pretrained encoders to achieve state-of-the-art performance \cite{Baumann_Dislich_etal._2024,Hörst_Rempe_etal._2024} in cell segmentation and classification tasks, model size, computational complexity, and processing times have increased. This limits the scalability and accessibility of these models. As we demonstrate, this may be mitigated using knowledge distillation. Studies in the field of natural language processing have demonstrated the efficacy of knowledge distillation in retaining the capabilities of the teacher model while achieving significant reductions in size and complexity \cite{Huangpu_Gao_2024,Sun_Yu_etal.}. 

We demonstrate the feasibility of knowledge distillation in digital pathology, specifically for cell segmentation and classification tasks. Moreover, we achieve this performance while also significantly reducing the parameter count. In addressing the challenge of knowledge transfer, we found that distillation from a transformer-based model to a smaller transformer is more straightforward than attempting to map transformer features to CNN blocks. In our experiments, using a CNN-based network as a student results in worse cell quantification performance due to the structural constraints of CNN feature space dimensions. 

Although our primary approach relies on a transformer-based student model that performs well, it can be further optimized to incorporate advantages from CNN architectures. For example, employing alternative techniques such as using ViT adapters \cite{Chen_Duan_etal._2023} or $1 \times 1$ convolutions to adjust feature map sizes may be beneficial for harnessing CNN advantages like enhanced local feature extraction. Moreover, if additional performance improvements are desired, the process can be further enhanced by applying supplementary knowledge distillation techniques, such as self-distillation \cite{Zhang_Song_etal._2019} or online distillation \cite{Houyon_Cioppa_etal._2023}.

Despite these promising results, further validation on independent datasets is necessary to fully understand the model's limitations. Underrepresented classes may pose challenges when addressing complex cases. Pathologists need to validate these models to adopt them in clinical settings. While the distilled models are smaller and more deployable, a technological gap persists because pathologists traditionally rely on established methods for inspecting WSIs and diagnosing diseases. Addressing the complexities involved in deploying models for inference and supporting pathologists in adopting new tools is essential for integrating these models into clinical workflows.

\section{Model integration with QuPath}
Digital pathology tools with graphical user interfaces are essential for visualizing and analyzing WSIs. To make our student model useful in clinical pathology workflows, it needs to be integrated into a tool that enables inspecting regions, creating annotations, and providing quantitative analyses of biomarkers. Therefore, we integrate the trained student model from the previous chapter into the QuPath open‑source platform \cite{Bankhead_Loughrey_etal._2017}. QuPath provides the required annotation, visualization, and analysis tools to interpret complex histological data, including workflows for cell segmentation, classification, and quantification (\hyperref[fig:fig7]{Figure 7}). 

\begin{figure}[h!]
    \centering
    \includegraphics[width=\textwidth]{images/Figure_7.pdf}
    \caption{Visualization of model-generated cell quantification annotations (left) and the corresponding unannotated slide (right) in QuPath}
    \label{fig:fig7}
\end{figure}

To identify the regions in a WSI critical for prognosticating tumor development, such as specific tumor areas or border regions without overlapping healthy tissue, the pathologist uses QuPath to outline these regions. Then, the pathologist initiates a cell segmentation and classification script through the QuPath interface for the selected regions. The resulting annotations and quantified cell information are then directly overlaid onto the WSI in the QuPath interface. Additional design and implementation details are in \hyperref[chap:S7]{Appendix S7}. 

Two common approaches for integrating deep learning models into QuPath are Java‑based native QuPath extensions \cite{Goldsborough_Philps_etal._2024} and the execution of RESTful API requests to a model server coupled with handling the response via an extension, as demonstrated in the application of cell segmentation models applied to immunofluorescence images \cite{Sugawara_2023}. While the community is actively working on these integration strategies, there is currently no universal solution that fully addresses all integration and performance requirements.

Extensions may offer better integration with QuPath, allowing slightly improved performance and more widespread usage of the built-in QuPath models, but they lack the flexibility to customize models and modify their behavior. For example, the newest version of QuPath includes models such as StarDist \cite{Weigert_Schmidt} and InstanSeg \cite{Goldsborough_Philps_etal._2024} that can perform cell segmentation. Both models pose limitations when applied to simultaneous cell segmentation and classification. StarDist performs well only on convex, round shapes by design, whereas some neoplastic, inflammatory, and connective cells exhibit complex and non-convex shapes. InstanSeg provides only semantic segmentation without assigning classes to the segmented cells.

%\hfill

In contrast, our approach offers an alternative integration strategy. It utilizes the paquo library to directly interact with QuPath’s internal application programming interface from within Python. This enables data exchange and processing without the need for intermediate conversion steps and provides greater control over model customization, retraining, and the incorporation of custom processing steps.

The integration of our custom model with QuPath underscores its potential to significantly enhance the diagnostic process by reducing the time burden on pathologists and enabling them to focus on more complex interpretative tasks using familiar software. Leveraging a tool that is already well-established among pathologists increases the likelihood of its adoption into daily clinical workflows. The quantitative data generated through the automated workflow is critical for both clinical decision-making and research, facilitating more accurate biomarker analysis, enabling robust statistical evaluations, and supporting hypothesis generation and testing. Additionally, by streamlining cell segmentation and classification, the tool enhances the scalability and reproducibility of pathological assessments, ultimately contributing to improved diagnostic accuracy and patient outcomes.

\section{Conclusion and future work}

In this study, we address critical challenges in digital pathology and tackle the usability and deployment issues of the developed models in standard computing environments without the need for high-performance computing systems. Our multi-faceted approach encompasses data refinement through cross-relabeling, leveraging foundation models for robust cell segmentation and classification, optimizing model performance via knowledge distillation, and integrating the optimized model into the QuPath software for practical application. This approach is used to construct a capable, versatile, and adjustable model for cell segmentation and classification, with enhanced performance and usability.

\begin{sloppypar}
While our approach shows potential in the field of computational pathology, certain limitations persist. 
For example, our implementation currently exhibits lower performance in detecting macrophages. 
This serves as an instance of the broader challenge of accurately identifying complex cell types. In order to address this issue, extending our approach to incorporate additional data sources, exploring alternative modeling approaches, and integrating other imaging modalities such as immunohistochemical staining may help improve detection accuracy. Moreover, although the distilled model reduces computational demands, integrating advanced deep learning models into clinical practice requires addressing technological gaps and potential resistance to adopting new tools within established diagnostic processes.
\end{sloppypar}

Future work could focus on several key areas to refine the proposed approach and facilitate its adoption in clinical environments. Enhancing the cell-relabeling process with additional datasets \cite{Graham_Jahanifar_etal._2021} could improve the representation of underrepresented cell types and enhance overall model performance. Also, incorporating additional data sources, such as multi-modal imaging or complementary staining methods, may address limitations related to cell type differentiation and class imbalance. Exploring other foundation models \cite{Vorontsov_Bozkurt_etal._2024,Zimmermann_Vorontsov_etal._2024} or introducing additional modalities \cite{Ding_Wagner_etal._2024,Vaidya_Zhang_etal._2025} may provide alternative architectures better suited to specific tasks or offer improved efficiency. Implementing more complex knowledge distillation techniques \cite{Houyon_Cioppa_etal._2023,Zhang_Song_etal._2019} could further optimize the model's performance and adaptability. Additionally, deeper integration with QuPath or other digital pathology software could provide pathologists more control over cell quantification analysis directly within the QuPath interface, thereby increasing accessibility and usability. Such enhancements would not only refine model performance but also ensure greater adaptability and scalability within various clinical environments. Finally, extensive validation of the model by pathologists and benchmarking against independent datasets are essential steps toward establishing the model's reliability and fostering confidence in its clinical utility.

\section*{Acknowledgments} 
This work was funded in part by the Research Council of Norway grant no. 309439 SFI Visual Intelligence, and the North Norwegian Health Authority grant no. HNF1521-20.

\bibliographystyle{IEEEtran}
\begin{sloppypar}
\begin{thebibliography}{99}

\bibitem{chaplot2020neural} Chaplot, Devendra Singh, et al. "Neural topological slam for visual navigation." Proceedings of the IEEE/CVF conference on computer vision and pattern recognition. 2020.

\bibitem{maksymets2021thda} Maksymets, Oleksandr, et al. "Thda: Treasure hunt data augmentation for semantic navigation." Proceedings of the IEEE/CVF International Conference on Computer Vision. 2021.

\bibitem{mezghan2022memory} Mezghan, Lina, et al. "Memory-augmented reinforcement learning for image-goal navigation." 2022 IEEE/RSJ International Conference on Intelligent Robots and Systems (IROS). IEEE, 2022.

\bibitem{al2022zero} Al-Halah, Ziad, Santhosh Kumar Ramakrishnan, and Kristen Grauman. "Zero experience required: Plug \& play modular transfer learning for semantic visual navigation." Proceedings of the IEEE/CVF Conference on Computer Vision and Pattern Recognition. 2022.

\bibitem{ye2021auxiliary} Ye, Joel, et al. "Auxiliary tasks and exploration enable objectgoal navigation." Proceedings of the IEEE/CVF international conference on computer vision. 2021.

\bibitem{chaplot2020object} Chaplot, Devendra Singh, et al. "Object goal navigation using goal-oriented semantic exploration." Advances in Neural Information Processing Systems 33 (2020)

\bibitem{ramakrishnan2022poni} Ramakrishnan, Santhosh Kumar, et al. "Poni: Potential functions for objectgoal navigation with interaction-free learning." Proceedings of the IEEE/CVF Conference on Computer Vision and Pattern Recognition. 2022.

\bibitem{ramrakhya2022habitat} Ramrakhya, Ram, et al. "Habitat-web: Learning embodied object-search strategies from human demonstrations at scale." Proceedings of the IEEE/CVF Conference on Computer Vision and Pattern Recognition. 2022.

\bibitem{mousavian2019visual} Mousavian, Arsalan, et al. "Visual representations for semantic target driven navigation." 2019 International Conference on Robotics and Automation (ICRA). IEEE, 2019.

\bibitem{dhariwal2021diffusion} Dhariwal, Prafulla, and Alexander Nichol. "Diffusion models beat gans on image synthesis." Advances in neural information processing systems 34 (2021)

\bibitem{ho2022classifier} Ho, Jonathan, and Tim Salimans. "Classifier-free diffusion guidance." arXiv preprint arXiv:2207.12598 (2022).

\bibitem{nichol2021glide} Nichol, Alex, et al. "Glide: Towards photorealistic image generation and editing with text-guided diffusion models." arXiv preprint arXiv:2112.10741 (2021)

\bibitem{brooks2023instructpix2pix} Brooks, Tim, Aleksander Holynski, and Alexei A. Efros. "Instructpix2pix: Learning to follow image editing instructions." Proceedings of the IEEE/CVF Conference on Computer Vision and Pattern Recognition. 2023.

\bibitem{fu2023guiding} Fu, Tsu-Jui, et al. "Guiding instruction-based image editing via multimodal large language models." arXiv preprint arXiv:2309.17102 (2023).

\bibitem{geng2024instructdiffusion} Geng, Zigang, et al. "Instructdiffusion: A generalist modeling interface for vision tasks." Proceedings of the IEEE/CVF Conference on Computer Vision and Pattern Recognition. 2024.

\bibitem{zhou2024minedreamer} Zhou, Enshen, et al. "Minedreamer: Learning to follow instructions via chain-of-imagination for simulated-world control." arXiv preprint arXiv:2403.12037 (2024).

\bibitem{zhou2023esc} Zhou, Kaiwen, et al. "Esc: Exploration with soft commonsense constraints for zero-shot object navigation." International Conference on Machine Learning. PMLR, 2023.

\bibitem{yu2023l3mvn} Yu, Bangguo, Hamidreza Kasaei, and Ming Cao. "L3mvn: Leveraging large language models for visual target navigation." 2023 IEEE/RSJ International Conference on Intelligent Robots and Systems (IROS). IEEE, 2023.

\bibitem{gadre2023cows} Gadre, Samir Yitzhak, et al. "Cows on pasture: Baselines and benchmarks for language-driven zero-shot object navigation." Proceedings of the IEEE/CVF Conference on Computer Vision and Pattern Recognition. 2023.

\bibitem{shah2023navigation} Shah, Dhruv, et al. "Navigation with large language models: Semantic guesswork as a heuristic for planning." Conference on Robot Learning. PMLR, 2023.

\bibitem{cai2024bridging} Cai, Wenzhe, et al. "Bridging zero-shot object navigation and foundation models through pixel-guided navigation skill." 2024 IEEE International Conference on Robotics and Automation (ICRA). IEEE, 2024.

\bibitem{yu2023co} Yu, Bangguo, Hamidreza Kasaei, and Ming Cao. "Co-NavGPT: Multi-robot cooperative visual semantic navigation using large language models." arXiv preprint arXiv:2310.07937 (2023).

\bibitem{wu2024voronav} Wu, Pengying, et al. "Voronav: Voronoi-based zero-shot object navigation with large language model." arXiv preprint arXiv:2401.02695 (2024).

\bibitem{qin2023mp5} Qin, Yiran, et al. "Mp5: A multi-modal open-ended embodied system in minecraft via active perception." arXiv preprint arXiv:2312.07472 (2023).

\bibitem{du2024learning} Du, Yilun, et al. "Learning universal policies via text-guided video generation." Advances in Neural Information Processing Systems 36 (2024).

\bibitem{ajay2024compositional} Ajay, Anurag, et al. "Compositional foundation models for hierarchical planning." Advances in Neural Information Processing Systems 36 (2024).

\bibitem{liang2024skilldiffuser} Liang, Zhixuan, et al. "Skilldiffuser: Interpretable hierarchical planning via skill abstractions in diffusion-based task execution." Proceedings of the IEEE/CVF Conference on Computer Vision and Pattern Recognition. 2024.

\bibitem{heusel2017gans} Heusel, Martin, et al. "Gans trained by a two time-scale update rule converge to a local nash equilibrium." Advances in neural information processing systems 30 (2017).

\bibitem{zhang2018unreasonable} Zhang, Richard, et al. "The unreasonable effectiveness of deep features as a perceptual metric." Proceedings of the IEEE conference on computer vision and pattern recognition. 2018.

\bibitem{brown2020language} Brown, Tom B. "Language models are few-shot learners." arXiv preprint arXiv:2005.14165 (2020).

\bibitem{podell2023sdxl} Podell, Dustin, et al. "Sdxl: Improving latent diffusion models for high-resolution image synthesis." arXiv preprint arXiv:2307.01952 (2023).

\bibitem{brohan2022rt} Brohan, Anthony, et al. "Rt-1: Robotics transformer for real-world control at scale." arXiv preprint arXiv:2212.06817 (2022).

\bibitem{brohan2023rt} Brohan, Anthony, et al. "Rt-2: Vision-language-action models transfer web knowledge to robotic control." arXiv preprint arXiv:2307.15818 (2023).

\bibitem{li2024manipllm} Li, Xiaoqi, et al. "Manipllm: Embodied multimodal large language model for object-centric robotic manipulation." Proceedings of the IEEE/CVF Conference on Computer Vision and Pattern Recognition. 2024.

\bibitem{shah2023vint} Shah, Dhruv, et al. "ViNT: A foundation model for visual navigation." arXiv preprint arXiv:2306.14846 (2023).

\bibitem{liu2024visual} Liu, Haotian, et al. "Visual instruction tuning." Advances in neural information processing systems 36 (2024).

\bibitem{hu2021lora} Hu, Edward J., et al. "Lora: Low-rank adaptation of large language models." arXiv preprint arXiv:2106.09685 (2021).

\bibitem{qin2023supfusion} Qin, Yiran, et al. "SupFusion: Supervised LiDAR-camera fusion for 3D object detection." Proceedings of the IEEE/CVF International Conference on Computer Vision. 2023.

\bibitem{qin2024worldsimbench} Qin, Yiran, et al. "Worldsimbench: Towards video generation models as world simulators." arXiv preprint arXiv:2410.18072 (2024).

\bibitem{yu2025gamefactory} Yu, Jiwen, et al. "GameFactory: Creating New Games with Generative Interactive Videos." arXiv preprint arXiv:2501.08325 (2025).

\bibitem{zhou2024code} Zhou, Enshen, et al. "Code-as-Monitor: Constraint-aware Visual Programming for Reactive and Proactive Robotic Failure Detection." arXiv preprint arXiv:2412.04455 (2024).

\bibitem{zhang2024ad} Zhang, Zaibin, et al. "AD-H: Autonomous Driving with Hierarchical Agents." arXiv preprint arXiv:2406.03474 (2024).

\bibitem{wang2024toward} Wang, Chaoqun, et al. "Toward Accurate Camera-based 3D Object Detection via Cascade Depth Estimation and Calibration." arXiv preprint arXiv:2402.04883 (2024).

\bibitem{huang2024story3d} Huang, Yuzhou, et al. "Story3d-agent: Exploring 3d storytelling visualization with large language models." arXiv preprint arXiv:2408.11801 (2024).

\bibitem{savinov2018semi} Savinov, Nikolay, Alexey Dosovitskiy, and Vladlen Koltun. "Semi-parametric topological memory for navigation." arXiv preprint arXiv:1803.00653 (2018).

\bibitem{majumdar2022zson} Majumdar, Arjun, et al. "Zson: Zero-shot object-goal navigation using multimodal goal embeddings." Advances in Neural Information Processing Systems 35 (2022): 32340-32352.

\bibitem{yadav2023offline} Yadav, Karmesh, et al. "Offline visual representation learning for embodied navigation." Workshop on Reincarnating Reinforcement Learning at ICLR 2023. 2023.

\bibitem{yadav2023ovrl} Yadav, Karmesh, et al. "Ovrl-v2: A simple state-of-art baseline for imagenav and objectnav." arXiv preprint arXiv:2303.07798 (2023).

\bibitem{sun2024fgprompt} Sun, Xinyu, et al. "FGPrompt: fine-grained goal prompting for image-goal navigation." Advances in Neural Information Processing Systems 36 (2024).

\bibitem{zhu2017target} Zhu, Yuke, et al. "Target-driven visual navigation in indoor scenes using deep reinforcement learning." 2017 IEEE international conference on robotics and automation (ICRA). IEEE, 2017.

\bibitem{koh2024generating} Koh, Jing Yu, Daniel Fried, and Russ R. Salakhutdinov. "Generating images with multimodal language models." Advances in Neural Information Processing Systems 36 (2024).

\bibitem{krantz2022instance} Krantz, Jacob, et al. "Instance-specific image goal navigation: Training embodied agents to find object instances." arXiv preprint arXiv:2211.15876 (2022).

\bibitem{schulman2017proximal} Schulman, John, et al. "Proximal policy optimization algorithms." arXiv preprint arXiv:1707.06347 (2017).

\bibitem{anderson2018evaluation} Anderson, Peter, et al. "On evaluation of embodied navigation agents." arXiv preprint arXiv:1807.06757 (2018).

\bibitem{lin2024navcot} Lin, Bingqian, et al. "NavCoT: Boosting LLM-Based Vision-and-Language Navigation via Learning Disentangled Reasoning." arXiv preprint arXiv:2403.07376 (2024).

\bibitem{NavGPT} Zhou, Gengze, Yicong Hong, and Qi Wu. "Navgpt: Explicit reasoning in vision-and-language navigation with large language models." Proceedings of the AAAI Conference on Artificial Intelligence.

\bibitem{hahn2021no} Hahn, Meera, et al. "No rl, no simulation: Learning to navigate without navigating." Advances in Neural Information Processing Systems 34 (2021): 26661-26673.

\bibitem{li2025t2isafety} Li, Lijun, et al. "T2ISafety: Benchmark for Assessing Fairness, Toxicity, and Privacy in Image Generation." arXiv preprint arXiv:2501.12612 (2025).

\bibitem{an2024agfsync} An, Jingkun, et al. "AGFSync: Leveraging AI-Generated Feedback for Preference Optimization in Text-to-Image Generation." arXiv preprint arXiv:2403.13352 (2024).


\end{thebibliography}
\end{sloppypar}

\clearpage
\beginsupplement
\section*{Appendix}
\renewcommand{\thesubsection}{S\arabic{subsection}}

\subsection{\label{chap:S1}PanNuke and MoNuSAC preprocessing}
The PanNuke dataset comprises a set of 7,901 RGB patches, each with dimensions of $256 \times 256$ pixels, which we set as the standard patch size for our analysis. In contrast, the MoNuSAC dataset encompasses 294 images of heterogeneous dimensions. To standardize the MoNuSAC images with our experiments, we implement a standardization protocol. Specifically, for images exceeding the dimensions of $256 \times 256$ pixels, we segment them into equal-sized patches and apply mirror padding to the remaining portions to avoid information loss at the peripherals. Patches with dimensions less than $128 \times 128$ pixels are excluded from the dataset due to the insufficient resolution to capture relevant cellular details. For patches where either dimension falls between 128 and 256 pixels, we employ upsampling to achieve the standard patch size. As a result, we obtain a total of 2,823 RGB patches derived from the MoNuSAC dataset for subsequent analysis. For additional details on the MoNuSAC data preparation process, refer to the source code \cite{Shvetsov_2025a}.
\clearpage

\subsection{\label{chap:S2}Data usage for the methodology}

\counterwithin{figure}{subsection}
\renewcommand{\thefigure}{S\arabic{subsection}}

\begin{figure}[h!]
    \centering
    \includegraphics[width=\textwidth, height=0.85\textheight, keepaspectratio]{images/A2.pdf}
    \caption{Overview of the methodology for cross-labeling, dataset refinement, and model comparison. (1) Cross-relabeling - training and testing cell classification models, (2) Cross-relabeling - using cell classification models to create refined dataset, (3) Fine-tuning and training models for comparison, (4) Student knowledge distillation with refined dataset}
    \label{fig:S2}
\end{figure}
\clearpage

\subsection{\label{chap:S3}Confusion matrices for classification models}
\counterwithin{figure}{subsection}
\renewcommand{\thefigure}{S\arabic{subsection}.\arabic{figure}}

\begin{figure}[h!]
    \centering
    \includegraphics[width=\textwidth, height=0.4\textheight, keepaspectratio]{images/A3_1.pdf}
    \caption{Confusion matrix for PanNuke trained model}
    \label{fig:S3.1}
\end{figure}

\begin{figure}[h!]
    \centering
    \includegraphics[width=\textwidth, height=0.4\textheight, keepaspectratio]{images/A3_2.pdf}
    \caption{Confusion matrix for MoNuSAC trained model}
    \label{fig:S3.2}
\end{figure}

\clearpage

\subsection{\label{chap:S4}Datasets cell counts}

\counterwithin{table}{subsection}
\renewcommand{\thetable}{S\arabic{subsection}}

\begin{table}[h!]
\renewcommand{\arraystretch}{2.0}
\centering
\caption{\label{tab:S4}Cell counts for PanNuke, MoNuSAC and refined datasets. Numbers in parentheses indicate preprocessed cell counts for cell classifier models training and testing.}
%\adjustbox{max width=\textwidth}{%
\begin{tabular}{|l|c|c|c|}
\hline
%\rowcolor{gray!30}
Cell type & PanNuke & MoNuSAC & Refined \\
\hline
Neoplastic & 77,403 (68,031) & - & 105,451 \\
\hline
Epithelial & 26,572 (23,207) & - & 29,926 \\
\hline
Epithelial (benign and malignant) & - & 31,402 & - \\
\hline
Inflammatory & 32,276 & - & - \\
\hline
Lymphocytes & - & 37,045 (33,104) & 65,275 \\
\hline
Neutrophils & - & 1,355 (1,252) & 3,833 \\
\hline
Macrophage & - & 1,842 (1,695) & 3,410 \\
\hline
Dead & 2,908 & - & 2,908 \\
\hline
Connective & 50,585 & - & 50,585 \\
\hline
\end{tabular}
%
%}
\end{table}



\clearpage

\subsection{\label{chap:S5}Definition of validation metrics}
\counterwithin{equation}{subsection}
\renewcommand{\theequation}{\arabic{equation}}

\subsubsection{\label{chap:S5.1}R\textsuperscript{2}}
The coefficient of determination, denoted as $R^2$, is a statistical measure that represents the proportion of variance in the dependent variable that is predictable from the independent variables. In the context of cell quantification in pathology, $R^2$ is used to assess how well the predicted quantities of different cell types in a patch align with the actual quantities observed in the ground truth data, with higher values representing more accurate quantification. $R^2$ is defined as
\begin{equation*}
R^2 = 1 - \frac{\sum_{i=1}^n (y_i - \hat{y}_i)^2}{\sum_{i=1}^n (y_i - \bar{y})^2},
\end{equation*}
where $y_i$ represents the actual number of cells of a specific type in the $i$-th image, $\hat{y}_i$ represents the predicted number of cells of that type in the $i$-th image, $\bar{y}$ is the mean of the actual numbers across all images, and $n$ is the total number of images in the dataset.

The $R^2$ metric has a range of $(-\infty, 1]$. An $R^2$ of 1 indicates perfect prediction, where all predicted values exactly match the actual values. An $R^2$ of 0 suggests that the model explains none of the variability of the response data around its mean. If $R^2$ is negative, it indicates that the model performs worse than a model that simply predicts the mean of the actual values for all observations.

\subsubsection{\label{chap:S5.2}PQ}
Panoptic Quality ($PQ$) is a comprehensive metric used to evaluate the performance of segmentation models in tasks that require both instance segmentation and classification. $PQ$ provides a single score that encapsulates both the detection accuracy (i.e., how many objects were correctly identified) and the segmentation quality (i.e., how accurately the objects' boundaries were delineated). This metric is particularly useful in multiclass scenarios where each pixel is classified into distinct categories, such as different cell types in pathology images.

$PQ$ is calculated as the product of two terms: Detection Quality ($DQ$) and Segmentation Quality ($SQ$). It can be expressed as
\begin{equation*}
PQ = DQ \cdot SQ,
\end{equation*}
where
\begin{equation*}
DQ = \frac{TP}{TP + 0.5\, FP + 0.5\, FN},
\end{equation*}
\begin{equation*}
SQ = \frac{\sum_{(p, g) \in \mathcal{M}} IoU(p, g)}{TP}.
\end{equation*}
In these formulas, $TP$ denotes the number of correctly matched instances between ground truth and prediction, $FP$ denotes the predicted instances that have no corresponding ground truth, $FN$ denotes the ground truth instances that were not detected, $IoU(p, g)$ is the Intersection over Union for a pair of matched instances $p$ (prediction) and $g$ (ground truth), and $\mathcal{M}$ is the set of matched pairs.

The $PQ$ metric is calculated for each class and is averaged across classes to provide a global performance measure.

The $PQ$ score has a range of $[0, 1.0]$, where a higher score indicates better performance in both detecting and segmenting the instances correctly. A $PQ$ of 1 signifies perfect identification and segmentation of all instances, whereas a $PQ$ of 0 indicates that no instances were correctly identified and segmented.

\clearpage

\subsection{\label{chap:S6}Segmentation and Detection quality metrics for teacher and student models}

\begin{table}[h!]
\renewcommand{\arraystretch}{2.0}
\centering
\caption{Segmentation and detection quality for student and teacher models (CI 95\%)}
\label{tab:S6}
%\adjustbox{max width=\textwidth}{%
\begin{tabular}{|l|c|c|}
\hline
%\rowcolor{gray!30}
Metric & Teacher & Student \\
\hline
$SQ_{neoplastic}$ & 0.819 (0.815--0.823) & 0.824 (0.819--0.828) \\
\hline
$SQ_{lymphocyte}$ & 0.795 (0.788--0.802) & 0.790 (0.783--0.796) \\
\hline
$SQ_{connective}$ & 0.770 (0.762--0.776) & 0.780 (0.772--0.786) \\
\hline
$SQ_{dead}$ & 0.659 (0.623--0.688) & 0.657 (0.624--0.695) \\
\hline
$SQ_{epithelial}$ & 0.780 (0.770--0.790) & 0.788 (0.779--0.797) \\
\hline
$SQ_{macrophage}$ & 0.788 (0.760--0.810) & 0.757 (0.730--0.783) \\
\hline
$SQ_{neutrofil}$ & 0.782 (0.761--0.801) & 0.775 (0.759--0.792) \\
\hline
$DQ_{neoplastic}$ & 0.706 (0.692--0.719) & 0.727 (0.712--0.741) \\
\hline
$DQ_{lymphocyte}$ & 0.675 (0.656--0.698) & 0.713 (0.691--0.734) \\
\hline
$DQ_{connective}$ & 0.566 (0.546--0.584) & 0.583 (0.565--0.602) \\
\hline
$DQ_{dead}$ & 0.410 (0.361--0.465) & 0.435 (0.306--0.561) \\
\hline
$DQ_{epithelial}$ & 0.668 (0.639--0.694) & 0.673 (0.644--0.702) \\
\hline
$DQ_{macrophage}$ & 0.657 (0.583--0.727) & 0.615 (0.531--0.703) \\
\hline
$DQ_{neutrofil}$ & 0.691 (0.625--0.753) & 0.729 (0.679--0.778) \\
\hline
\end{tabular}
%
%}
\end{table}

\clearpage

\subsection{\label{chap:S7}QuPath integration method}
We adopt an integration strategy leveraging the paquo \cite{Bayer_AG} library, a Python package that enables direct interaction with QuPath’s internal API, thereby facilitating seamless data exchange without intermediate conversion steps. The data processing pipeline (\hyperref[fig:S7]{Appendix Figure S7}) begins with the acquisition of WSIs and their associated annotations from QuPath, which are represented as Shapely \cite{Gillies_Wel_etal._2024} polygons. Utilizing paquo, we directly read, create, and modify these annotations and detections within a QuPath project in the Python environment. Images are then cropped using these polygons and processed by cell segmentation and classification models employing standard vision processing toolkits such as OpenCV, pyvips, and PyTorch. Additionally, QuPath employs Groovy scripts to initiate a Python process that starts the entire pipeline from QuPath graphical interface: fetching polygons, extracting images from them, and running deep learning model inference on the cropped images. 
The results are returned to QuPath, leveraging paquo's Python bindings to manipulate QuPath data while minimizing the computational overhead typically associated with cross-environment communication.

\counterwithin{figure}{subsection}
\renewcommand{\thefigure}{S\arabic{subsection}}

\begin{figure}[h!]
    \centering
    \includegraphics[width=\textwidth]{images/A7.pdf}
    \caption{QuPath integration workflow using Python environment}
    \label{fig:S7}
\end{figure}

Compared to traditional workflows that involve exporting annotations as GeoJSON, classifying them in Python, and reimporting them into QuPath, our approach offers several advantages. We eliminate the need to switch between programming languages, providing a cohesive and streamlined development process entirely within QuPath software and removing the necessity to use other tools. Meanwhile, we avoid storing annotations as intermediate JSON files unless required for external use or archiving. By conducting the entire inference and post-processing workflow within the Python environment, we leverage the power and flexibility of Python libraries for image processing and machine learning. This approach also enables adjustments to any set of labels and models, thereby improving its applicability.

%\hfill

The distilled model and QuPath integration code are packaged into a Docker container, enabling streamlined execution with the Docker engine. Detailed integration code and deployment instructions can be found in the GitHub repository \cite{Shvetsov_2025b}.

Despite these benefits, we acknowledge that the paquo library is a proof‑of‑concept project in its early development stage and has not been tested across all versions of QuPath.

\clearpage

\subsection{\label{chap:S8}Data and code availability statement}
All datasets, models, and code used in this study are publicly available and can be obtained from the repositories listed below. 
The PanNuke \cite{Gamper_Koohbanani_etal._2019} and MoNuSAC \cite{Verma_Kumar_etal._2021} datasets are publicly accessible, and download information along with detailed descriptions can be found in their respective articles. Preprocessing scripts for PanNuke and MoNuSAC data, as well as individual cell extraction scripts, are available on GitHub \cite{Shvetsov_2025a}. The H-Optimus foundation model used in our experiments can be downloaded from the HuggingFace repository \cite{hoptimus2024}, and model information is available on GitHub \cite{Saillard_Jenatton_etal._2024}. In addition, the integration code for QuPath and the distilled model packaged in a Docker container are provided in the repository \cite{Shvetsov_2025b}, and paquo Python library is available from the authors GitHub repository \cite{Bayer_AG}.
\clearpage

\end{document}

\end{document}
