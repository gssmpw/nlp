\documentclass[10pt,conference]{IEEEtran}
\IEEEoverridecommandlockouts

\usepackage{cite}
\usepackage{amsmath,amssymb,amsfonts}
\usepackage{algorithmic}
\usepackage{textcomp}
\usepackage{xcolor}
\usepackage{xspace}
\usepackage{pifont}%
\usepackage[inkscapelatex=false]{svg}
\usepackage{booktabs}
\usepackage{listings}
\usepackage{graphicx,xfp,subcaption}
\usepackage{tabularray}
\usepackage{tabularx}
\usepackage{tcolorbox}
\usepackage{multirow}
\usepackage{caption,array}
\usepackage{url}
\usepackage{hyperref}
\usepackage{array}
\usepackage{balance}
\usepackage[normalem]{ulem}
\usepackage{pgfplots, pgfplotstable}
\pgfplotsset{compat=1.18}

\newcommand{\rowgroup}[1]{\hspace{-1em}#1}
\def\BibTeX{{\rm B\kern-.05em{\sc i\kern-.025em b}\kern-.08em
    T\kern-.1667em\lower.7ex\hbox{E}\kern-.125emX}}

% TODO
% more detailed introduction of dataset creation
% the rumour label in such datasets
\section{Data} \label{sec:data}
We use three rumour datasets in this work, namely: PHEME~\citep{pheme2015,kochkina-etal-2018-one}, Twitter15, and Twitter16~\citep{ma-etal-2017-detect}:

% TJB: how can the number of threads be greater than the number of tweets? these numbers don't make sense
% RX: fixed, the numbers were incorrect
\paragraph{PHEME}~\citet{pheme2015} contains 6,425 tweet posts of rumours and non-rumours related to 9 events. To avoid using specific a priori keywords to search for tweet posts, PHEME used the Twitter (now X) steaming API to identify newsworthy events from breaking news and then selected from candidate rumours that met rumour criteria, finally they collected associated conversations and annotate them. They engaged journalists to annotate the threads. The data were collected between 2014 and 2015. The 9 events are split into two groups, the first being breaking news that contains rumours, including Ferguson unrest, Ottawa shooting, Sydney siege, Charlie Hebdo shooting, and Germanwings plane crash. The rest are specific rumours, namely Prince to play in Toronto, Gurlitt collection, Putin missing, and Michael Essien contracting Ebola.
% TJB: say something about the time period when this data was collected
% RX: added

\paragraph{Twitter 15}~\citet{twitter15} was constructed by collecting rumour and non-rumour posts from the tracking websites snopes.com and emergent.info. They then used the Twitter API to gather corresponding posts, resulting in 94 true and 446 false posts. This dataset further includes 1,490 root posts and their follow posts, comprising 1,116 rumours and 374 non-rumours.
% TJB: the "tweet" vs. "comment" terminology is potentially confusing and needs to be clarified
% RX: unified, used root and follow posts to refer to root posts and the comment posts, posts are used to describe tweets in general.

\paragraph{Twitter 16}
Similarly to Twitter 15, \citet{twitter16} collected rumours and non-rumours from snopes.com, resulting in 778 reported events, 64\% of which are rumours. For each event, keywords were extracted from the final part of the Snopes URL and refined manually---adding, deleting, or replacing words iteratively---until the composed queries yielded precise Twitter search results. The final dataset includes 1,490 root tweet posts and their follow posts, comprising 613 rumours and 205 non-rumours.

\begin{table*}[!t]
    \centering
    \small
    \begin{tabular}{p{0.05\linewidth}p{0.9\linewidth}}
    \toprule
    Task & Prompt \\
    \midrule
    V-oc & Categorize the text into an ordinal class that best characterizes the writer's mental state, considering various degrees of positive and negative sentiment intensity. 3: very positive mental state can be inferred. 2: moderately positive mental state can be inferred. 1: slightly positive mental state can be inferred. 0: neutral or mixed mental state can be inferred. -1: slightly negative mental state can be inferred. -2: moderately negative mental state can be inferred. -3: very negative mental state can be inferred.\\
    \midrule
    E-c & Categorize the text's emotional tone as either `neutral or no emotion' or identify the presence of one or more of the given emotions (anger, anticipation, disgust, fear, joy, love, optimism, pessimism, sadness, surprise, trust).\\
    \midrule
    E-i & Assign a numerical value between 0 (least E) and 1 (most E) to represent the intensity of emotion E expressed in the text.\\
    \bottomrule
    \end{tabular}
    \caption{Prompts used for EmoLLM to detect emotion information in tweets. V-oc = Valence Ordinal Classification, E-c = Emotion Classification, and E-i = Emotion Intensity Regression.}
    \label{tab:emollm_ins}
\end{table*}


  
%%% Local Variables:
%%% mode: latex
%%% TeX-master: "../main_anonymous"
%%% End:

\begin{document}


\NewDocumentCommand{\codeword}{v}{%
\small{\texttt{{#1}}}%
}


\newcommand{\ie}{\emph{i.e.,}\xspace}
\newcommand{\eg}{\emph{e.g.,}\xspace}
\newcommand{\cmark}{\ding{51}}%
\newcommand{\xmark}{\ding{55}}

%%%%% THIS PART HAS BEEN MOVED TO `macros.tex` %%%%%

%
\setlength\unitlength{1mm}
\newcommand{\twodots}{\mathinner {\ldotp \ldotp}}
% bb font symbols
\newcommand{\Rho}{\mathrm{P}}
\newcommand{\Tau}{\mathrm{T}}

\newfont{\bbb}{msbm10 scaled 700}
\newcommand{\CCC}{\mbox{\bbb C}}

\newfont{\bb}{msbm10 scaled 1100}
\newcommand{\CC}{\mbox{\bb C}}
\newcommand{\PP}{\mbox{\bb P}}
\newcommand{\RR}{\mbox{\bb R}}
\newcommand{\QQ}{\mbox{\bb Q}}
\newcommand{\ZZ}{\mbox{\bb Z}}
\newcommand{\FF}{\mbox{\bb F}}
\newcommand{\GG}{\mbox{\bb G}}
\newcommand{\EE}{\mbox{\bb E}}
\newcommand{\NN}{\mbox{\bb N}}
\newcommand{\KK}{\mbox{\bb K}}
\newcommand{\HH}{\mbox{\bb H}}
\newcommand{\SSS}{\mbox{\bb S}}
\newcommand{\UU}{\mbox{\bb U}}
\newcommand{\VV}{\mbox{\bb V}}


\newcommand{\yy}{\mathbbm{y}}
\newcommand{\xx}{\mathbbm{x}}
\newcommand{\zz}{\mathbbm{z}}
\newcommand{\sss}{\mathbbm{s}}
\newcommand{\rr}{\mathbbm{r}}
\newcommand{\pp}{\mathbbm{p}}
\newcommand{\qq}{\mathbbm{q}}
\newcommand{\ww}{\mathbbm{w}}
\newcommand{\hh}{\mathbbm{h}}
\newcommand{\vvv}{\mathbbm{v}}

% Vectors

\newcommand{\av}{{\bf a}}
\newcommand{\bv}{{\bf b}}
\newcommand{\cv}{{\bf c}}
\newcommand{\dv}{{\bf d}}
\newcommand{\ev}{{\bf e}}
\newcommand{\fv}{{\bf f}}
\newcommand{\gv}{{\bf g}}
\newcommand{\hv}{{\bf h}}
\newcommand{\iv}{{\bf i}}
\newcommand{\jv}{{\bf j}}
\newcommand{\kv}{{\bf k}}
\newcommand{\lv}{{\bf l}}
\newcommand{\mv}{{\bf m}}
\newcommand{\nv}{{\bf n}}
\newcommand{\ov}{{\bf o}}
\newcommand{\pv}{{\bf p}}
\newcommand{\qv}{{\bf q}}
\newcommand{\rv}{{\bf r}}
\newcommand{\sv}{{\bf s}}
\newcommand{\tv}{{\bf t}}
\newcommand{\uv}{{\bf u}}
\newcommand{\wv}{{\bf w}}
\newcommand{\vv}{{\bf v}}
\newcommand{\xv}{{\bf x}}
\newcommand{\yv}{{\bf y}}
\newcommand{\zv}{{\bf z}}
\newcommand{\zerov}{{\bf 0}}
\newcommand{\onev}{{\bf 1}}

% Matrices

\newcommand{\Am}{{\bf A}}
\newcommand{\Bm}{{\bf B}}
\newcommand{\Cm}{{\bf C}}
\newcommand{\Dm}{{\bf D}}
\newcommand{\Em}{{\bf E}}
\newcommand{\Fm}{{\bf F}}
\newcommand{\Gm}{{\bf G}}
\newcommand{\Hm}{{\bf H}}
\newcommand{\Id}{{\bf I}}
\newcommand{\Jm}{{\bf J}}
\newcommand{\Km}{{\bf K}}
\newcommand{\Lm}{{\bf L}}
\newcommand{\Mm}{{\bf M}}
\newcommand{\Nm}{{\bf N}}
\newcommand{\Om}{{\bf O}}
\newcommand{\Pm}{{\bf P}}
\newcommand{\Qm}{{\bf Q}}
\newcommand{\Rm}{{\bf R}}
\newcommand{\Sm}{{\bf S}}
\newcommand{\Tm}{{\bf T}}
\newcommand{\Um}{{\bf U}}
\newcommand{\Wm}{{\bf W}}
\newcommand{\Vm}{{\bf V}}
\newcommand{\Xm}{{\bf X}}
\newcommand{\Ym}{{\bf Y}}
\newcommand{\Zm}{{\bf Z}}

% Calligraphic

\newcommand{\Ac}{{\cal A}}
\newcommand{\Bc}{{\cal B}}
\newcommand{\Cc}{{\cal C}}
\newcommand{\Dc}{{\cal D}}
\newcommand{\Ec}{{\cal E}}
\newcommand{\Fc}{{\cal F}}
\newcommand{\Gc}{{\cal G}}
\newcommand{\Hc}{{\cal H}}
\newcommand{\Ic}{{\cal I}}
\newcommand{\Jc}{{\cal J}}
\newcommand{\Kc}{{\cal K}}
\newcommand{\Lc}{{\cal L}}
\newcommand{\Mc}{{\cal M}}
\newcommand{\Nc}{{\cal N}}
\newcommand{\nc}{{\cal n}}
\newcommand{\Oc}{{\cal O}}
\newcommand{\Pc}{{\cal P}}
\newcommand{\Qc}{{\cal Q}}
\newcommand{\Rc}{{\cal R}}
\newcommand{\Sc}{{\cal S}}
\newcommand{\Tc}{{\cal T}}
\newcommand{\Uc}{{\cal U}}
\newcommand{\Wc}{{\cal W}}
\newcommand{\Vc}{{\cal V}}
\newcommand{\Xc}{{\cal X}}
\newcommand{\Yc}{{\cal Y}}
\newcommand{\Zc}{{\cal Z}}

% Bold greek letters

\newcommand{\alphav}{\hbox{\boldmath$\alpha$}}
\newcommand{\betav}{\hbox{\boldmath$\beta$}}
\newcommand{\gammav}{\hbox{\boldmath$\gamma$}}
\newcommand{\deltav}{\hbox{\boldmath$\delta$}}
\newcommand{\etav}{\hbox{\boldmath$\eta$}}
\newcommand{\lambdav}{\hbox{\boldmath$\lambda$}}
\newcommand{\epsilonv}{\hbox{\boldmath$\epsilon$}}
\newcommand{\nuv}{\hbox{\boldmath$\nu$}}
\newcommand{\muv}{\hbox{\boldmath$\mu$}}
\newcommand{\zetav}{\hbox{\boldmath$\zeta$}}
\newcommand{\phiv}{\hbox{\boldmath$\phi$}}
\newcommand{\psiv}{\hbox{\boldmath$\psi$}}
\newcommand{\thetav}{\hbox{\boldmath$\theta$}}
\newcommand{\tauv}{\hbox{\boldmath$\tau$}}
\newcommand{\omegav}{\hbox{\boldmath$\omega$}}
\newcommand{\xiv}{\hbox{\boldmath$\xi$}}
\newcommand{\sigmav}{\hbox{\boldmath$\sigma$}}
\newcommand{\piv}{\hbox{\boldmath$\pi$}}
\newcommand{\rhov}{\hbox{\boldmath$\rho$}}
\newcommand{\upsilonv}{\hbox{\boldmath$\upsilon$}}

\newcommand{\Gammam}{\hbox{\boldmath$\Gamma$}}
\newcommand{\Lambdam}{\hbox{\boldmath$\Lambda$}}
\newcommand{\Deltam}{\hbox{\boldmath$\Delta$}}
\newcommand{\Sigmam}{\hbox{\boldmath$\Sigma$}}
\newcommand{\Phim}{\hbox{\boldmath$\Phi$}}
\newcommand{\Pim}{\hbox{\boldmath$\Pi$}}
\newcommand{\Psim}{\hbox{\boldmath$\Psi$}}
\newcommand{\Thetam}{\hbox{\boldmath$\Theta$}}
\newcommand{\Omegam}{\hbox{\boldmath$\Omega$}}
\newcommand{\Xim}{\hbox{\boldmath$\Xi$}}


% Sans Serif small case

\newcommand{\Gsf}{{\sf G}}

\newcommand{\asf}{{\sf a}}
\newcommand{\bsf}{{\sf b}}
\newcommand{\csf}{{\sf c}}
\newcommand{\dsf}{{\sf d}}
\newcommand{\esf}{{\sf e}}
\newcommand{\fsf}{{\sf f}}
\newcommand{\gsf}{{\sf g}}
\newcommand{\hsf}{{\sf h}}
\newcommand{\isf}{{\sf i}}
\newcommand{\jsf}{{\sf j}}
\newcommand{\ksf}{{\sf k}}
\newcommand{\lsf}{{\sf l}}
\newcommand{\msf}{{\sf m}}
\newcommand{\nsf}{{\sf n}}
\newcommand{\osf}{{\sf o}}
\newcommand{\psf}{{\sf p}}
\newcommand{\qsf}{{\sf q}}
\newcommand{\rsf}{{\sf r}}
\newcommand{\ssf}{{\sf s}}
\newcommand{\tsf}{{\sf t}}
\newcommand{\usf}{{\sf u}}
\newcommand{\wsf}{{\sf w}}
\newcommand{\vsf}{{\sf v}}
\newcommand{\xsf}{{\sf x}}
\newcommand{\ysf}{{\sf y}}
\newcommand{\zsf}{{\sf z}}


% mixed symbols

\newcommand{\sinc}{{\hbox{sinc}}}
\newcommand{\diag}{{\hbox{diag}}}
\renewcommand{\det}{{\hbox{det}}}
\newcommand{\trace}{{\hbox{tr}}}
\newcommand{\sign}{{\hbox{sign}}}
\renewcommand{\arg}{{\hbox{arg}}}
\newcommand{\var}{{\hbox{var}}}
\newcommand{\cov}{{\hbox{cov}}}
\newcommand{\Ei}{{\rm E}_{\rm i}}
\renewcommand{\Re}{{\rm Re}}
\renewcommand{\Im}{{\rm Im}}
\newcommand{\eqdef}{\stackrel{\Delta}{=}}
\newcommand{\defines}{{\,\,\stackrel{\scriptscriptstyle \bigtriangleup}{=}\,\,}}
\newcommand{\<}{\left\langle}
\renewcommand{\>}{\right\rangle}
\newcommand{\herm}{{\sf H}}
\newcommand{\trasp}{{\sf T}}
\newcommand{\transp}{{\sf T}}
\renewcommand{\vec}{{\rm vec}}
\newcommand{\Psf}{{\sf P}}
\newcommand{\SINR}{{\sf SINR}}
\newcommand{\SNR}{{\sf SNR}}
\newcommand{\MMSE}{{\sf MMSE}}
\newcommand{\REF}{{\RED [REF]}}

% Markov chain
\usepackage{stmaryrd} % for \mkv 
\newcommand{\mkv}{-\!\!\!\!\minuso\!\!\!\!-}

% Colors

\newcommand{\RED}{\color[rgb]{1.00,0.10,0.10}}
\newcommand{\BLUE}{\color[rgb]{0,0,0.90}}
\newcommand{\GREEN}{\color[rgb]{0,0.80,0.20}}

%%%%%%%%%%%%%%%%%%%%%%%%%%%%%%%%%%%%%%%%%%
\usepackage{hyperref}
\hypersetup{
    bookmarks=true,         % show bookmarks bar?
    unicode=false,          % non-Latin characters in AcrobatÕs bookmarks
    pdftoolbar=true,        % show AcrobatÕs toolbar?
    pdfmenubar=true,        % show AcrobatÕs menu?
    pdffitwindow=false,     % window fit to page when opened
    pdfstartview={FitH},    % fits the width of the page to the window
%    pdftitle={My title},    % title
%    pdfauthor={Author},     % author
%    pdfsubject={Subject},   % subject of the document
%    pdfcreator={Creator},   % creator of the document
%    pdfproducer={Producer}, % producer of the document
%    pdfkeywords={keyword1} {key2} {key3}, % list of keywords
    pdfnewwindow=true,      % links in new window
    colorlinks=true,       % false: boxed links; true: colored links
    linkcolor=red,          % color of internal links (change box color with linkbordercolor)
    citecolor=green,        % color of links to bibliography
    filecolor=blue,      % color of file links
    urlcolor=blue           % color of external links
}
%%%%%%%%%%%%%%%%%%%%%%%%%%%%%%%%%%%%%%%%%%%


%%%%%%%%%%%%%%%%%%%%%%%%%%%%%%%%%%%%%%%%%%%%%%%%%%%%


\newcommand{\gulzar}[1]{\textcolor{red}{G: #1}}
\newcommand{\tien}[1]{\textcolor{blue}{T: #1}}
\newcommand{\question}[1]{\textcolor{brown}{Question: #1}}
\newcommand{\waris}[1]{\textcolor{orange}{W: #1}}
\newcommand{\edit}[1]{\textcolor{cyan}{#1}}

\newcommand{\todo}[1]{\textcolor{green}{TODO: #1}}

\title{Are the Majority of Public Computational Notebooks Pathologically Non-Executable?}

\author{\IEEEauthorblockN{Tien Nguyen, Waris Gill, Muhammad Ali Gulzar}
\IEEEauthorblockA{Virginia Tech\\
Blacksburg, USA \\
\{tiennguyen, waris, gulzar\}@vt.edu}
}

\maketitle
\begin{abstract}  
Test time scaling is currently one of the most active research areas that shows promise after training time scaling has reached its limits.
Deep-thinking (DT) models are a class of recurrent models that can perform easy-to-hard generalization by assigning more compute to harder test samples.
However, due to their inability to determine the complexity of a test sample, DT models have to use a large amount of computation for both easy and hard test samples.
Excessive test time computation is wasteful and can cause the ``overthinking'' problem where more test time computation leads to worse results.
In this paper, we introduce a test time training method for determining the optimal amount of computation needed for each sample during test time.
We also propose Conv-LiGRU, a novel recurrent architecture for efficient and robust visual reasoning. 
Extensive experiments demonstrate that Conv-LiGRU is more stable than DT, effectively mitigates the ``overthinking'' phenomenon, and achieves superior accuracy.
\end{abstract}  
\begin{IEEEkeywords}
Computational notebooks; non-executability; restoration; mining software repositories
\end{IEEEkeywords}

\section{Introduction}\label{sec:intro}
\begin{figure}[h]  
    \centering
    \includegraphics[width=1.0\textwidth]{figures/fig1_new.pdf}  
    \caption{
    Web AI agents exhibit a significantly higher jailbreak rate (46.6\%) compared to standalone LLMs (0\%), highlighting their increased vulnerability in real-world deployment.}
    \label{fig:experiment_email}
\end{figure}

Recent advancements in Large Language Models (LLMs) have demonstrated impressive reasoning capabilities and proficiency in solving complex problems. These capabilities are increasingly being extended to multi-step tasks, driving the evolution of LLM-based AI agent systems \citep{shen2024scribeagentspecializedwebagents,yang2024agentoccamsimplestrongbaseline,yang2024swe,putta2024agent,zhang2024webpilot}. 
One such system is the Web (browser) AI agent, which integrates an LLM with software tools and APIs to execute sequences of actions aimed at achieving specific goals within a web environment.
These agents leverage LLM capabilities for planning \citep{zheng2024naturalplanbenchmarkingllms}, reflection \citep{Pallagani_2024}, and effective tool utilization \citep{yao2024taubenchbenchmarktoolagentuserinteraction,shi2024learningusetoolscooperative}, enabling more autonomous and adaptive web-based interactions.


Many previous studies \citep{openhands,shen2024scribeagentspecializedwebagents,su2025learnbyinteractdatacentricframeworkselfadaptive} have highlighted significant advancements in autonomous web agents. 
However, despite their promising potential, their safety and security vulnerabilities have not yet been systematically studied. 
Given their direct integration with web browsers, these agents could be exploited to distribute malware or send phishing Emails to extract personal information, posing serious security risks (as shown in Fig. \ref{fig:experiment_email}). 


In this study, we highlight the heightened vulnerability of Web AI agent frameworks to jailbreaking compared to traditional LLMs. Through comprehensive experiments, we demonstrate that web agents, by design, exhibit a significant higher susceptibility to following malicious commands due to fundamental component-level differences from standalone LLMs. 
Notably, while a standalone LLM (such as a regular chatbot)
refuses malicious requests with a 0\% success rate, the Web AI agent follows them at a rate of 46.6\% (Fig. \ref{fig:experiment_email}).


Importantly, we identify three primary factors contributing to the increased vulnerabilities of Web AI agents:
\textbf{(1)} Directly embedding user input into the LLM system prompt, 
\textbf{(2)} Generating
actions in a multi-turn manner, and 
\textbf{(3)} Processing observations and action histories, which increases the likelihood of executing harmful instructions and weakens the system’s ability to assess risks. 
\textbf{Additionally,} we find that mock-up testing environments may inadvertently distort security evaluations by oversimplifying real-world interactions, potentially leading to misleading conclusions about an agent's robustness.



To better understand the heightened vulnerability of Web AI agents to jailbreaking and their increased susceptibility to executing malicious commands, we introduce a 5-level fine-grained ablative metric that goes beyond the conventional binary assessments of LLM vulnerabilities, offering a more nuanced evaluation of jailbreak signals. 
Ultimately, our study 
raises awareness of the security challenges posed by Web AI agents and advocates for proactive measures to design safer, more resilient agent frameworks. 















\textbf{Our contributions:}
\begin{itemize}[leftmargin=*]
\item \textbf{Empirical evidence of Web AI agents’ heightened vulnerability: }We systematically compare Web AI agents with standalone LLM chatbots, revealing that Web AI agents are significantly more susceptible to jailbreaking and executing malicious commands.
\item \textbf{Root cause analysis of Web AI agent vulnerabilities:} We investigate the design-level differences between Web AI agents and standalone LLMs, identifying key factors—such as system prompt manipulation, multi-turn action generation, and reliance on historical observations—that contribute to their increased vulnerability.
\item \textbf{A fine-grained evaluation protocol for jailbreak susceptibility:} We introduce a structured, five-level harmfulness evaluation framework that goes beyond binary assessments, enabling a more detailed and nuanced analysis of Web AI agent vulnerabilities.
\item \textbf{Actionable insights for targeted defense strategies:} Based on our findings, we provide recommendations for mitigating security risks in Web AI agents, focusing on improving system prompt handling, action generation mechanisms, and contextual awareness in agent architectures.

    
    
\end{itemize}






\section{Motivation}
\label{sec:motivation}


\begin{figure*}[ht!] % Figure 2
\centerline{\includegraphics[width=0.8\textwidth]{images/full_workflow.pdf}}
    \caption{LLM-based error-driven notebook executability analysis and restoration workflow.}
    \label{fig:main_workflow}
    % \vspace{-4ex}
\end{figure*}


This section presents two case studies demonstrating that seemingly non-executable notebooks can potentially be restored with minor reconfiguration, and pathologically non-executable notebooks still offer valuable partially executable code. 


\subsection{Case Study 1: Restoring Complete Executability}
\label{case_study_1}

    The notebook {\small{\texttt{random\_forest\_algorithm.ipynb}}} \cite{girlscript2024} implements the random forest classifier algorithm. This notebook is hosted by the GirlScript Foundation's GitHub open-source repository ``Winter of Contributing" with 881 stars. It consists of 24 code cells. When this notebook is executed as is, it results in a ``FileNotFound" error in cell 2, attempting to read input file {\small{\texttt{Social\_Network\_Ads.csv}}}. This input file is accessible in the original author's environment, suggesting that the author intended this notebook to be executable before uploading it to GitHub. However, this input file is neither included in the repository nor available in the adopter's environment. Prior studies \cite{Pimentel2019, Pimentel2021} would classify this notebook as non-executable. This classification is analogous to considering ``{\small{\texttt{javac}}}" non-executable if no input {\small{\texttt{.java}}} file is provided. This notebook aims to demonstrate the logical steps required to create a classifier. Our observation is that for such a notebook, improving executability can help serve the notebook's purpose, which can be done by generating a synthetic input file. To do so, we package the notebook's code, Python error description, and the notebook's documentation into a prompt and query Llama-3 to generate an input file with a correct relative path to the notebook with synthetic content. This synthetically generated input file results in full executability of all 24 cells, i.e., improving the executability of the previously non-executable notebook by over 95\% as illustrated in Figure \ref{fig:cs1-full-exec}. This is a prime example of how non-executability is often over-classified when the input file is not provided but may be available locally to the original author. 
    

\subsection{Case Study 2: Improving Partial Executability}
\label{case_study_2}

    The notebook {\small{\texttt{DinosaurusIsland--Character level language model final-v3.ipynb}}} from GitHub repository ``deep-learning-coursera" \cite{gemaatienza} is demonstrating a deep learning tutorial. It has 129 stars and a total of 14 code cells. When executed in linear order, the notebook first encounters a ``ModuleNotFound" error in cell 1 due to the absence of module {\small{\texttt{utils}}}. We resolve this error by installing the required module. A follow-up execution raises a ``NameError" in cell 6 due to the undefined function {\small{\texttt{softmax()}}}. After reviewing the notebook, we noted that the markdown cell claimed the function was provided, but it is missing. The original notebook shows a valid output, meaning the author successfully executed it. This is a common case where program states are shared between different executions, even if the corresponding code is moved or deleted. Even if the notebook is executed end-to-end before pushing on GitHub, the error would not have surfaced as the function definition is still present in the current Python kernel. However, such program states are not shared with the notebook itself, resulting in a ``NameError" in a new Python Kernel. 
    
    Our analysis indicates that defining the undefined name with an appropriate implementation could significantly improve the notebook's executability. Therefore, we prompt Llama-3 with the error report and the code cells to generate a definition for this function and insert a new code cell containing the definition right before its usage. This LLM-generated code cell requires a module import for {\small{\texttt{tensorflow}}}. Again, we were able to install it in the environment. These simple addition and module installations improve the original notebook's executability by 7 cells until it encounters ``AttributeError" at cell 8. This is a tutorial notebook, and the first half illustrates valuable deep-learning knowledge in sequence models. Thus, partial executability offers added value.



\section{Research Questions}
\label{research questions}

Under the traditional definition of executability, a notebook is considered \textbf{executable} if it does not trigger any error throughout its complete top-down execution. A \textbf{non-executable} notebook, on the other hand, fails to execute due to an error or exception. In this paper, we relax this notion of executability by dividing it further into {\bf pathologically non-executable}, {\bf non-executable but restorable}, and {\bf executable}. When a notebook's executability is hindered by unresolvable errors, e.g., syntax or indentation errors, it is \textbf{pathologically non-executable}. If a notebook is executable in the original author's execution environment but fails to execute in another environment due to issues such as missing input files, execution orders, or execution environments, it is called a  {\bf non-executable but restorable} or {\bf misconfigured} notebook which is not executable but can be fully or partially restored. 

Similar to different notions of executability, we also introduce different degrees of executability. Instead of considering it as a binary metric, we measure executability on a continuous spectrum, defined as a ratio between the number of cells executed until the first error and the total number of cells in a notebook. 100\% executability refers to fully executable notebook, whereas $<$100\% refers to partially executable notebooks. A fully executable notebook does not guarantee reproducibility, i.e., a notebook must produce the same outputs as the original notebook. Reproducibility is beyond the scope of this work and is often not required for data-centric notebooks as they are expected to produce different outputs on different data. We explore the following research questions:

\begin{itemize}
    \item \textbf{RQ1:} What are the common causes of non-executability in notebooks?
    \item \textbf{RQ2:} How many non-executable notebooks are pathologically non-executable, and how many can be restored?
    \item \textbf{RQ3:} To what extent can pathologically non-executable notebooks be executed?
    \item \textbf{RQ4:} Can LLM-based restoration strategies enhance notebook executability?
\end{itemize}







\section{Empirical Analysis Methodology}
\label{sec:approach}



Figure \ref{fig:main_workflow} (a) illustrates the intermediate stages of our analysis process. Starting with a repository, we retrieve all notebooks and conduct a lightweight static error check to identify issues that may impede notebook execution. Next, we automatically analyze the repository to extract any provided environment requirement files, enabling us to configure the appropriate environment for notebook execution. During execution, we log the first-encountered error and incrementally apply targeted restoration strategies based on the error type, measuring executability improvements at each stage. 


%%%%%%%%%%%%%%%%%%%%%%%%%%%%%%%%%%%%%%%%%%%%%%%%%%%%%%%%%%%%
\subsection{Analysis Approach}
\label{analysis_approach}



    \subsubsection{Initial Error Checking}
        For a notebook to be executable, it first needs to be compilable. As a first step, we look for compilation errors by applying a lightweight error checking on all notebooks within each repository to uncover any compiling issues, such as syntax or indentation errors. This preliminary filtering step assesses executability without running the code, allowing us to exclude notebooks that require extensive code-level modifications for restoration and are thus deemed pathologically non-executable. To access the content of the notebooks, we use the {\small{\texttt{nbformat}}} module. While performing static analysis of the notebook's code, we detect and omit notebooks from further investigation that (1) cannot be read due to corruption and encoding problems, (2) contain no code cells (with or without markdown cells), or all code cells do not hold any content; and (3) written in programming languages other than Python 3 (such as Python 2, Julia, R, or C). 
    

    \subsubsection{Dynamic Error Checking} 
        \label{dynamic-error-checking}
        In step \ding{203} of Figure \ref{fig:main_workflow} (b), we perform dynamic error checking to detect and categorize potential runtime errors. We use Papermill \cite{papermill}, which allows us to execute the notebook and detect and report any errors encountered during execution. We use Python 3 as the execution kernel and document the first error that halts the execution. We categorize the execution status of each notebook into the following groups:
    
        \begin{enumerate}
            \item \textbf{Executable:} The notebook is fully executable without encountering errors.
            \item \textbf{FileNotFound:} The notebook writes to or reads from a file or directory that is unavailable during execution.
            \item \textbf{ModuleNotFound:} The notebook imports unavailable modules, packages, or libraries.
            \item \textbf{NameError:} The notebook uses a variable, function, or class without defining or importing it.
            \item \textbf{Others/Non-Fixable:} The notebook encounters errors not covered by the above categories. We also identify and categorize these errors accordingly.
        \end{enumerate}

        \begin{figure}[t!] % Figure 3
            \includegraphics[width=\columnwidth]{images/def_use.pdf}
            \caption{Def-use lists for the first two cells in notebook \cite{pytopia}.}
            \label{fig:def-use}
        \end{figure}

        \noindent The categories reflect a natural progression of notebook adoptions and fixes, as shown in Figure \ref{fig:main_workflow}. 
        %This begins with checking executability and then categorizing non-executable notebooks into fixable and non-fixable errors. “Fixable errors” are further categorized into three subcategories based on the most common issues identified in prior literature \cite{Pimentel2019}.
        If a notebook is found fully executable after this execution process, we mark it as executable. If it is non-executable due to missing input files, required modules, or undefined names, we iteratively apply restoration strategies (Figure \ref{fig:main_workflow} (b)) and perform dynamic error checking to remove as many executability issues as possible.

        
        For notebooks with ``NameError," we perform a static {\em def-use} analysis to localize the root cause behind undefined names. Like traditional {\em def-use} analysis, we build a specialized AST visitor that tracks variable definition and usage with precise scope awareness within each cell. Figure \ref{fig:def-use} shows how \textit{def} and \textit{use} sets look for two notebook cells. Note that while maintaining the definition set, we consider variable binding, access, and scoping rules for Python to avoid any false positives and false negatives during the analysis. Our definition set extends beyond variables to include imports, functions, classes, and control flow elements. 
        
        Next, we use {\em def-use} sets to isolate the location of undefined names in the notebook and search nearby cells to locate the definition. Finally, the framework returns any detected definition of the undefined variable and its cell location. If the variable's definition appears later in the notebook than its initial use, we categorize the notebook as {\em defined after use} and return the cell number where the definition is found. If no definition exists throughout the notebook, we classify it as {\em undefined}. 

        % The framework first analyzes the entire notebook using the AST visitor to create detailed maps of all variable definitions and uses across all cells. For each cell, it keeps track of both global and local scope variables. When it finds variable uses and definitions, it records not just the cell number but also the specific scope within that cell. The framework then attempts to locate the cell where the given undefined variable might be defined in the subsequent cells. 
        %For each potential definition, we check if that definition would be accessible from where the variable is used while following Python's scoping rules. For example, a variable defined inside a function is not accessible to code outside that function.     
        

    %\tien{To accomplish this, each cell is analyzed to generate a set of \textit{definitions} (def), comprising variables (or names) that are defined within the cell. To construct this set, we recursively traverse the AST nodes, extracting all instances of {\small{\texttt{AST.Name}}} referring to variables. When encountering a node of type {\small{\texttt{Name}}}, we determine its context within the notebook. If the context is {\small{\texttt{Store()}}}, indicating the definition of a variable, we add it to the definition set. Note that while maintaining the definition set, we consider variable binding, access, and scoping rules for Python to avoid any false positives and false negatives during the analysis.}
    
    %In other programming files, certain errors can be easily detected during the compilation phase. Similarly, in notebooks, we can statically identify some issues without the need for execution. 
    %For a notebook to be executable, it first needs to be compilable. As a first step, we look for compilation errors by applying static error checking on all notebooks to uncover any compiling issues, such as syntax errors or indentation errors. Additionally, we can detect runtime NameError instances by examining the definition and usage \cite{Wang2021, Chen2014, Ryder1994} of variables within a notebook to pinpoint any undefined variables. This preemptive process aids in assessing executability early without executing the code. Notebooks that fail this early screen fall into pathologically non-executable notebooks since restoring them requires a code-level rewrite that may change the semantics of the original notebook.  


    %Next, in \ding{203} of Figure \ref{fig:main_workflow}, we use the AST of each cell to identify potential NameError issues arising from undefined variable use. To accomplish this, we first identify all the definitions and usage instances of variables (or names). Each cell is analyzed to generate two distinct sets: the \textit{definition} (def), comprising variables that are defined within the cell, and the \textit{use}, containing variables that are referred to in the cell. Figure \ref{fig:reorder_workflow} shows how def and use sets look like for two cells in a public notebook. To construct these two sets, we recursively traverse the AST nodes, extracting all instances of \codeword{AST.Name} referring to variables. When encountering a node of type \codeword{Name}, we determine its context within the notebook. If the context is \codeword{Store()}, indicating the definition of a variable, we add it to the definition set. Conversely, if the context is \codeword{Load()} or \codeword{Del()}, signifying the usage of variable values, we append it to the use set. Note that while maintaining def-use sets, we consider the variable binding, access, and scoping rules for Python to avoid any false positives and false negatives during error checking. 

    %\tien{Our definition set extends beyond variables to include imports, functions, classes, and control flow elements. For imports, we incorporate module names and aliases to the definition set for potential use elsewhere in the notebook. Similarly, we extract function names and argument lists from different types of defined functions (regular, async, and lambda). Class names are also added to class definitions. In control flow constructs, such as loops, we include variables referenced in conditions or defined within the loop body. This comprehensive definition set supports the accurate identification of dependencies throughout the notebook.}
    
    %We expand our def and use set beyond variables. For import statements, including attributes indicating module names and their aliases, we add them to the set as potential future usage within the notebook. Similarly, for function definitions (including regular, async, and lambda functions), we extract their names and argument lists, including them into the use set. We exclusively gather user-defined names, omitting built-in functions to avoid triggering NameError messages. Class names are extracted from class definitions and appended to the use set as well. In control flow constructs such as for loops or while loops, we include variable names in conditional statements and those defined within the body in the definition set. Variables utilized within the body are added to the use set.

    
    % Considering the scope of each name or variable is crucial, as variables can be defined and used in different scopes, ranging from global to more local scopes. Variables defined in outer scopes can be accessed in inner scopes or the same scope, whereas variables defined within a local scope cannot be accessed outside of that scope. For instance, functions cannot be utilized outside of their respective functions, and variables defined within a class can be accessed within functions within that class or along with the class. Consequently, for each extracted name, a scope is assigned. Function definitions, class definitions, and control flow constructs narrow the scope of variables or names defined or used within their respective bodies.

    %\tien{Given the undefined variable and its location in the notebook, we examine the cells to locate its definition using the established definition list, noting the position of the first occurrence. If the variable's definition appears later in the notebook than its initial use, this suggests a possible execution order issue due to disordered cells. In such cases, we categorize the notebook under “defined after use” and return the cell number where the definition is found. If no definition exists throughout the notebook, we classify it as “undefined.” This categorization guides our approach in the restoration stage.}

    %Assuming the top-down execution of the notebook, we use def-use sets of notebook cells to identify any undefined names, classifying them based on their first encounter: undefined in the entire notebook, defined after their usage, or both. If a variable is in a def set of current or any of the prior cells, we conclude that no NameError occurs for that variable. If a variable is not in any of those sets, we explore whether it is defined after that cell, indicating a potential issue due to disordered cell order. If the name is defined in any subsequent cell, we categorize the notebook into the "defined after use" group. Otherwise, it is placed in the "undefined" group.  When a cell contains undefined and defined-after variable names, we assign that notebook to the "both" category. This classification approach ensures thorough identification and categorization of potential NameError issues within the notebook. At the end of static error checking, we categorize notebooks as pathologically non-executable, non-executable but restorable, and potentially executable notebooks.   
    
    %For each cell in a notebook, we extract its usage list and compare each name against a combination of names defined prior to that cell. This combination comprises all names defined within the scope of usage and outer scopes within the same cell, as well as the definition lists of predecessor cells at scope 0. Why specifically scope 0? Considering that if a variable is defined in a deeper inner scope of a cell, the subsequent cell cannot access that name within a similar inner scope. Therefore, we only need to verify against the list of definition names at scope 0 of the cells that follow.    

    %Through static analysis, we can systematically categorize notebooks based on identified static errors, allowing for incremental refinement of our understanding of potential issues. This methodical step provides a structured framework for addressing errors and ensures a comprehensive examination of notebooks prior to dynamic analysis.

    % After this process, we collected a total of 22,998 notebooks containing NameError. Among those, 18,633 have at least one undefined variable in the first error encounter, 3,828 have variables defined after their usage, and 537 have both. 

    
%%%%%%%%%%%%%%%%%%%%%%%%%%%%%%%%%%%%%%%%%%%%%%%%%%%%%%%%%%%%%%%%%%%%%%

\begin{figure*}[t!] % Figure 4
        \centerline{\includegraphics[width=1\textwidth]{images/LLM_Prompts.pdf}}
        \caption{Examples of prompts for LLM and responses for different error types. }
        \label{fig:LLM-prompts}
        % \vspace{-3ex}
\end{figure*}

    
    \subsection{LLM-Based Error-Driven Restoration}
        %In contrast to prior studies on notebook executability \cite{Head2019, Wang2020, Zhu2021, Wang2021} that consider a notebook non-executable if they result in an error, we attempt to recover the executability of notebooks from missing appropriate execution environment. These notebooks are still executable in the original author's environment and, therefore, intrinsically executable. 
        This section explains the lightweight use of LLMs to synthesize the execution environments, enabling improved measurements of notebook restorability. This addresses undefined name issues from incorrect execution orders or the absence of definitions, generates synthetic yet syntactically valid input, and installs external modules. 
        %Note that the goal of this work is not to create new notebook restoration techniques but to use existing tools to separate pathologically non-executable notebooks from misconfiguration.
        
%       LLMs have shown remarkable success in various data generation tasks, including program synthesis, comment generation, test generation, and debugging.
        
        


         %   In the pursuit of restoring the executability of non-executable notebooks, our approach hinges on the implementation of fundamental restoration strategies. Our overarching objective is to seamlessly recover the executability of these notebooks while meticulously preserving the integrity and semantics of their contents. Through employing these strategies, we strive to ensure that the essential essence and structure of the notebooks remain unaltered, thus facilitating their continued utility as valuable resources for future studies and analyses.

    
   
    %Our static-error checking phase identifies notebooks that are non-executable due to NameError issues. Among those, we restore the executability of notebooks where NameError occurs due to define-after-use issues. Notebooks in which a variable is undefined throughout are pathologically non-executable as there is no straightforward method to restore it without altering the notebook's semantics.
  
    %The nearest cell containing the variable's definition is extracted and placed immediately before the cell where the variable is utilized, as illustrated in Figure \ref{fig:reorder_workflow}. Cell reordering is performed in step \ding{204} of the analysis as shown in Figure \ref{fig:main_workflow}.
    
    %Moving a cell may either fix the NameError or lead to a new NameError. We apply static and dynamic error checking to the refactored notebook to evaluate its executability again. If the NameError does not appear, the executability of the reordered notebook is compared to that of the original file, where no reordering occurs. If a new NameError occurs, we consider such notebooks restorable, but we do not attempt to restore them further, as it requires a meticulous search for the right cell execution order. When a NameError occurs due to define-after-use within a cell, we classify such notebooks as pathologically non-executable.  This restoration phase concludes by addressing NameErrors that do not require intra-cell code refactoring. 

        \subsubsection{ModuleNotFound Error}
            Most of the repositories do not provide environment requirement information, and if they do, those modules are often outdated \cite{Zhu2021, Wang2021}. Notebooks with ``ModuleNotFound" errors are restorable since they were originally executable but failed to execute in a new environment. We attempt to re-establish the execution environment for the notebooks by inspecting the REQUIREMENTS/INSTALL instructions and installing the required dependencies in the execution environment before executing any notebooks. During execution, if a notebook returns a ``ModuleNotFound" error, we extract the missing module name from the error message and construct a terminal command {\small{\texttt{pip install <missing module>}}} (\ding{204}-D of Figure \ref{fig:main_workflow}). If the installation fails due to an incorrect or deprecated module name, we use LLM to obtain the correct and updated name for the corresponding missing module (\ding{204}-E of Figure \ref{fig:main_workflow} and LLM prompt in Figure \ref{fig:LLM-prompts}-A) and retry the installation. 
            
            %\tien{Figure \ref{fig:LLM-prompts}-B shows an example of the LLM prompt and response for a failed module installation.} After missing modules are installed into the execution environment, notebooks are subsequently re-executed to evaluate their executability (step \ding{205} in Figure \ref{fig:main_workflow})
            
        %We structure the prompt to facilitate the LLM in generating accurate and concise responses within our tailored response format. We then extract the updated module name from the LLM response and re-attempt the module installation with the new module name. We solely aim to assess whether the executability of the notebook improves through this intervention. 
         

   % Notebooks are frequently shared and reused among multiple users, leading to potential issues stemming from disparities in execution environments. Notebooks may include execution environment information; however, it is often inaccurate or outdated \cite{Wang2021}. The non-availability of correct external modules and packages is a major reason for notebook non-executability.  We consider notebooks with ModuleNotFound errors as restorable since they were originally executable but failed to execute in a new environment due to misconfiguration. Thus, we devise a lightweight strategy to restore the executability of such notebooks instead of classifying them as non-executable.   
    
    %When a notebook execution results in ModuleNotFound error, we extract the missing module name from the error message and construct a terminal command \codeword{pip install <missing module>} with the corresponding missing module. This installs the detected missing module into the current execution environment. We do not alter or update the missing module names due to deprecated APIs, as this could compromise the notebook's semantics. Our aim is solely to assess whether the executability of the notebook improves through this intervention.  Notebooks exhibiting this type of error are subsequently re-executed to evaluate their executability (step \ding{206} in Figure \ref{fig:exec_workflow}) after missing modules are installed into the execution environment.

        \subsubsection{FileNotFound Error}
            ``FileNotFound" errors often occur when notebooks attempt to access and read unavailable input files or directories. We use LLM to generate synthetic input data tailored to the notebook's semantics, attempting to resolve ``FileNotFound" errors while avoiding additional runtime issues. Our insight is that the code cells in notebooks contain sufficient information for LLMs to generate syntactically correct input data needed for execution. Since our goal is executability rather than reproducibility, syntactically correct synthetic data is sufficient to address the ``FileNotFound" error and restore notebook executability.
            %Given the absence of the actual data for the missing input files, using LLM is the optimal approach to generate suitable synthetic sample content, thereby enabling the restoration of notebook executability.         
            Concise contexts improve the LLM's performance \cite{Ramlochan2024}. Thus, we provide contexts of the notebook error, such as the missing file path and type, which guides LLM on the file's syntax, correct data type, and content. Instead of the {\small{\texttt{.ipynb}}} file, we provide the source code of all cells in the prompt. This conversion from {\small{\texttt{.ipynb}}} to Python also removes noise that may misguide or overwhelm the LLMs. Figure \ref{fig:LLM-prompts}-B shows this prompt and the response by the LLM for the example mentioned in \ref{case_study_1}. 
            %The three restoration strategies are completely automated to help scale the empirical evaluation.  




     
        \subsubsection{NameError}
            %NameError is one of the most common notebook executability errors arising from variable undefined issues when executed top-down linearly. Such errors often do not occur for a notebook's original authors, who know the correct non-linear execution of cells. However, notebooks are rarely accompanied by the execution order of cells when made publicly available.
            For a non-executable notebook due to undefined names, we extract the name and the specific cell in which the error occurs and the cell where the name definition is located or indicate if it cannot be found.  We then prompt LLM to address this issue as illustrated in \ding{204}-A of Figure \ref{fig:main_workflow} (b). The prompt also includes the code cells and the error type (see Figure \ref{fig:LLM-prompts}-C). The LLM's response is used to rewrite the notebook, which is analyzed again for additional errors. However, the changes introduced by LLMs may change the notebook's semantics. Since test cases are almost non-existent in notebooks (only 1.54\% notebooks have tests \cite{Pimentel2021}), it is challenging to verify semantic accuracy. Even if test cases are available, we must first make the notebook executable to enable testing and detect any semantic inconsistencies introduced by LLM-based modifications. Our position is that addressing executability is a prerequisite for reproducibility, which requires cell executability even if it causes semantic changes.  
     
     
%     Even if LLM-based restorations introduce minor logical errors, achieving executability is the supports future reproducibility techniques, creating opportunities to correct any semantically inconsistent fixes to NameError.

%    Jupyter notebooks are formatted as JSON-based files containing extensive information about the execution environment, metadata, and all cells even when they contain no content. The source code for each cell is stored as a single string, making it challenging for the LLM to extract accurate information. Therefore, we need the source code of the notebook as a whole. Additionally, supplying the LLM engine with the notebook's JSON-based format would be counterproductive due to the excess of unnecessary information. 

 %   To provide the aforementioned context information to the LLM, we use a regular expression to extract the full path of the missing input file from the reported error message during the dynamic execution of a notebook. To acquire the notebook's source code, we employ the PythonExplorer module to convert the JSON-based data into Python-typed data. This procedure integrates the source code of individual notebook cells into cohesive Python content.

   % We structure the prompt to facilitate the LLM in generating accurate data within our tailored response format, also shown in Figure \ref{fig:LLM-prompts}.
% {\em "Generate a sample input file \codeword{<missing_file_path>} for the source code below. Format the response with only the needed data between \texttt{\textasciigrave\textasciigrave\textasciigrave} and \texttt{\textasciigrave\textasciigrave\textasciigrave}. Just data and No fluff."<Notebook source code in Python>}
        %
    
    
    
    % To do so, we ensure that the Large Language Model (LLM) generates only the necessary data by including the directive ``Just data and No fluff." \waris{in the intro I mentioned we use regular expressions to sanitize. we can remove that point or add this insight there.}
    % This helps prevent the LLM from producing any superfluous or irrelevant content, eliminating the need to sanitize and refine the response. The response is then injected in the input files with the appropriate file names and directories. 
    
    % Once the missing input files have been created and correctly placed, we perform dynamic error checking again to determine whether the error has been resolved or if it remains unchanged, as shown in \ding{205} in Figure \ref{fig:main_workflow}. If the issue persists, it may be due to discrepancies in how the notebooks read the file directories versus how they were created. 
    
    
    % \waris{i believe we try to create the file path given in the filename. The error might me be that sometimes due to window machines path or username paths we cannot create the file path. We also do not want to edit the path in the notebook. It also quite possible that LLM did not generate the correct data. example it might miss a column in csv while generating data.} \gulzar{This ca}
    
    
    % In such cases, we halt the process and report the partial executability of the notebook, noting the specific issues preventing full execution.

%% !!!!!! \todo{New results on summarization: Last row in Table 5 (shall this be ablation or understanding?)}
%% !!!!!! \todo{Change "user" to "consumer"?)}

\subsection{Data}
We conduct a series of experiments on industrial datasets to compare the performance of LR-Recsys against state-of-the-art recommender systems in the literature. We start with introducing the three datasets used. 

\begin{itemize}
\item \textbf{Amazon Movie}: This dataset captures consumer purchasing behavior in the Movies \& TV category on Amazon \citep{ni2019justifying}\footnote{\url{https://nijianmo.github.io/amazon/index.html}}. Collected in 2023, it contains 17,328,314 records from 6,503,429 users and 747,910 unique movies. Each record includes the user ID, movie ID, movie title, user rating (on a 1-5 scale), purchasing timestamp, user’s past purchasing history (as a sequence of movie IDs), and three aspect terms summarizing key movie attributes (e.g., ``thriller'', ``exciting'', ``director'').

\item \textbf{Yelp Restaurant}: This dataset documents users' restaurant check-ins on the Yelp platform\footnote{\url{https://www.yelp.com/dataset}}. It spans 11 metropolitan areas in the United States and comprises 6,990,280 check-in records from 1,987,897 users across 150,346 restaurants. Each record includes the user ID, restaurant ID, check-in timestamp, user rating (on a 1-5 scale), user’s historic visits (as a sequence of restaurant IDs), and three aspect terms summarizing key restaurant features (e.g., ``atmosphere'', ``service'',  ``expensive'').

\item \textbf{TripAdvisor Hotel}: This dataset captures users’ hotel stays on the TripAdvisor platform \citep{li2023personalized}. Collected in 2019, it contains 343,277 hotel stay records from 9,765 users and 6,280 unique hotels. Each record includes the user ID, hotel ID, check-in timestamp, user rating, the list of hotels previously visited by the user, and three aspect terms describing key hotel attributes (e.g., ``beach'', ``price'', ``service''). 
\end{itemize}


% \begin{itemize}
% \item \textbf{Amazon Movie}, which records the consumer purchasing actions in the Movies \& TV department of Amazon \citep{ni2019justifying}\footnote{https://nijianmo.github.io/amazon/index.html}. This dataset was collected in the year of 2023, including 17,328,314 records of 650,3429 users and 747,910 unique movies in total. Each individual record in this dataset contains the information of user ID, movie ID, movie title, user rating (on the scale of 1-5), user purchasing timestamp, user past purchasing history in this department (as a sequence of movie IDs), and three aspect terms that summarize the most important properties of the movie (such as ''thriller'', ''exciting'', and ''director''). 
% \item \textbf{Yelp Restaurant}, which records the users' restaurant check-in behaviors on the Yelp platform \footnote{https://www.yelp.com/dataset}. This dataset spans across 11 metropolitan areas in the United States, and includes 6,990,280 check-in records of 150,346 restaurants from 1,987,897 users. For each individual record, we collect the information of user ID, restaurant ID, check-in timestamp, user rating (on the scale of 1-5), user historic visits (as a sequence of restaurant IDs), and three aspect terms that summarize the most important features of the restaurant (such as ''atmosphere'', ''service'', and ''expensive'').
% \item \textbf{TripAdvisor Hotel}, which records the users' hotel stays on the TripAdvisor platform \citep{li2023personalized}. This dataset was collected in the year of 2019, and it includes 343,277 hotel staying records from 9,765 unique users of 6.280 unique hotels. Each individual record collects the information of user ID, hotel ID, hotel check-in timestamp, user rating, the list of hotels that the user has stayed before, as well as three aspect terms that describe the most unique part of the hotel (such as ''beach'', ''concierge'', and ''breakfast'').
% \end{itemize}

For all three datasets, the recommender system aims to predict the outcome for a given candidate product using the inputs of user ID, candidate item ID, and the user's past purchasing history (specifically, the last five items purchased).\endnote{For the aspect-based recommendation baselines introduced below, the aspect terms are also included as inputs to the model.} The outcome of interest for all three datasets is the product rating (on a 1-5 scale). 

To evaluate the efficacy and robustness of our framework, we examine two prediction settings: (1) a regression task, where the goal is to predict the exact rating value, and (2) a classification task, where the objective is to categorize ratings as either high (4 or 5) or low (1, 2, or 3). In line with established practices in the recommender system literature, we use root mean squared error (RMSE) and mean absolute error (MAE) as evaluation metrics for the regression task. For the classification task, we use the area under the ROC curve (AUC) \citep{hanley1982meaning}. These three datasets represent three distinct business applications with significantly different statistical distributions, making them excellent testbeds for evaluating the generalizability and flexibility of LR-Recsys.

\subsection{Baseline Models}
We compare LR-Recsys against state-of-the-art black-box recommender systems, LLM-based recommender systems, and a wide range of explainable recommender systems. Specifically, we identify the following three groups of 14 state-of-the-art baselines from recent marketing and computer science literature:

% \begin{itemize}
\begin{enumerate}
    \item \textbf{Aspect-Based Recommender Systems}: These models utilize ground-truth aspect terms as additional information to facilitate preference reasoning and generate recommendations.
    \begin{itemize}
        \item \textbf{A3NCF} \citep{cheng20183ncf} constructs a topic model to extract user preferences and item characteristics from reviews, and capture user attention on specific item aspects via an attention network.
        \item \textbf{SULM} \citep{bauman2017aspect} predicts the sentiment of a user about an item’s aspects, identifies the most valuable aspects of their potential experience, and recommends items based on these aspects.
        \item \textbf{AARM} \citep{guan2019attentive} models interactions between similar aspects to enrich aspect connections between users and products, using an attention network to focus on aspect-level importance.
        \item \textbf{MMALFM} \citep{cheng2019mmalfm} applies a multi-modal aspect-aware topic model to estimate aspect importance and predict overall ratings as a weighted linear combination of aspect ratings.
        \item \textbf{ANR} \citep{chin2018anr} learns aspect-based user and item representations through an attention mechanism and models multi-faceted recommendations using a neural co-attention framework.
        \item \textbf{MTER} \citep{le2021explainable} generates aspect-level comparisons between target and reference items, producing recommendations based on these comparative explanations.
    \end{itemize}

    \item \textbf{Sequential Recommender Systems}: These models use sequences of past user behaviors to predict the next likely purchase, leveraging various neural network architectures.
    \begin{itemize}
        \item \textbf{SASRec} \citep{kang2018self} utilizes self-attention to capture long-term semantics in user actions and identify relevant items in a user’s history.
        \item \textbf{DIN} \citep{zhou2018deep} adopts a local activation unit to adaptively learn user interest representations from historical behaviors and predict preferences for candidate items.
        \item \textbf{BERT4Rec} \citep{sun2019bert4rec} adopts bidirectional self-attention and the Cloze objective to model user behavior sequences and avoid information leakage, enhancing recommendation efficiency.
        \item \textbf{UniSRec} \citep{hou2022towards} uses contrastive pre-training to learn universal sequence representations of user preferences, improving recommendation accuracy.
    \end{itemize}

    \item \textbf{Interpretable Recommender Systems}: These models focus on generating high-quality recommendations accompanied by intuitive explanations.
    \begin{itemize}
        \item \textbf{AMCF} \citep{pan2021explainable} maps uninterpretable general features to interpretable aspect features, optimizing for both recommendation accuracy and explanation clarity through dual-loss minimization.
        \item \textbf{PETER} \citep{li2021personalized} predicts words in target explanations using IDs, endowing them with linguistic meaning to generate personalized recommendations.
        \item \textbf{UCEPic} \citep{li2023ucepic} combines aspect planning and lexical constraints to produce personalized explanations through insertion-based generation, improving recommendation performance.
        \item \textbf{PARSRec} \citep{gholami2022parsrec} leverages common and individual behavior patterns via an attention mechanism to tailor recommendations and generate explanations based on these patterns.
    \end{itemize}
% \end{itemize}
\end{enumerate}

\begin{comment}

\begin{itemize}
\item \textbf{Aspect-Based Recommender Systems}, which utilize the ground-truth aspect terms as additional information to facilitate the preference reasoning and to provide recommendations accordingly. These models include: (1) \textbf{A3NCF \citep{cheng20183ncf}}, which constructs a topic model to extract user preferences and item characteristics from user reviews. This topic model guides the representation learning of users and items, and also captures a user’s special attention on each aspect of the targeted item with an attention network. These user/item representations and aspect attention values will then be used for generating recommendations; (2) \textbf{SULM \citep{bauman2017aspect}}, which first predicts the sentiment that the user may have about the item based on what she/he might express about the aspects of the item, and then identifies the most valuable aspects of the user’s potential experience with that item. It further recommends items based on those most important aspects accordingly; (3) \textbf{AARM \citep{guan2019attentive}}, which models the interactions between synonymous and similar aspects to enrich the aspect connections between user and product. It also contains a neural attention network to capture a user’s attention toward aspects when examining different products, and produces recommendations accordingly; (4) \textbf{MMALFM \citep{cheng2019mmalfm}}, which applies a multi-modal aspect-aware topic model to model users’ preferences and items’ features from different aspects, and also estimate the aspect importance of a user toward an item. The overall rating is then predicted via a linear combination of the aspect ratings, which are weighted by the importance of the corresponding aspect; (5) \textbf{ANR \citep{chin2018anr}}, which performs aspect-based representation learning for both users and items via an attention-based component. It also models the multi-faceted process for recommendations by estimating the aspect-level user and item importance based on the neural co-attention mechanism; and (6) \textbf{MTER \citep{le2021explainable}}, which formulates comparative explanations involving aspect-level comparisons between the target item and the reference items, and produces recommendations accordingly.
\item \textbf{Sequential Recommender System}, which utilizes the sequence of user past behaviors to predict the next product that the consumer is likely to purchase using various types of neural network architectures. These models include: (1) \textbf{SASRec \citep{kang2018self}}, which proposes a self-attention-based sequential model to capture long-term semantics in recommendations, and to identify which items are ''relevant'' from a user’s action history; (2) \textbf{DIN \citep{zhou2018deep}}, which designs a local activation unit to adaptively learn the user interest representation from historical behaviors, and then infers the user preference on the candidate item accordingly; (3) \textbf{BERT4Rec \citep{sun2019bert4rec}}, which employs the deep bidirectional self-attention mechanism to model user behavior sequences, and also adopts the Cloze objective to predict the random masked items in the sequence to avoid the information leakage, resulting in more efficient recommendation performance; (4) \textbf{UniSRec \citep{hou2022towards}}, which utilizes the contrastive pre-training technique to learn universal sequence representations that represent the user preferences, and then provide subsequent recommendations accordingly.
\item \textbf{Interpretable Recommender System}, which attempts to develop models that generate not only high-quality recommendations but also intuitive explanations for those recommendations. These models include: (1) \textbf{AMCF \citep{pan2021explainable}}, which presents a novel feature mapping approach that maps the uninterpretable general features onto the interpretable aspect features, achieving both satisfactory accuracy and explainability in the recommendations by simultaneous minimization of rating prediction loss and interpretation loss; (2) \textbf{PETER \citep{li2021personalized}}, which designs a simple and effective learning objective that utilizes the IDs to predict the words in the target explanation, endowing the IDs with linguistic meanings and producing personalized recommendations; (3) \textbf{UCEPic \citep{li2023ucepic}}, which generates high-quality personalized explanations for recommendation results by unifying aspect planning and lexical constraints in an insertion-based generation manner, and these explanations can be subsequently used for improving the recommendation performance; and (4) \textbf{PARSRec \citep{gholami2022parsrec}}, which relies on common behavior patterns as well as individual behaviors to tailor the recommendation strategy for each person through the attention mechanism, and produce the explanations based on these behavioral patterns.
\end{itemize}
\end{comment}

We split each dataset into training and test sets using an 80-20 ratio at the user-temporal level. To ensure a fair comparison, we adopted Grid Search \citep{bergstra2011algorithms} and allocated equal computational resources—in terms of training time and memory usage—to optimize the hyperparameters for both our proposed approach and all baseline models. Detailed hyperparameter settings are provided in Appendix \ref{appen:hyper_param}. We independently ran LR-Recsys and each baseline model ten times and reported the average performance metrics along with their standard deviations.


\subsection{Main Results}
\label{sec:main_results}
Table \ref{main_results} presents the main results. LR-Recsys consistently outperforms all three groups of 14 baseline models across two recommendation tasks and three datasets. Specifically, LR-Recsys achieves an improvement of 5-20\% in RMSE, 15-30\% in MAE, and 2.9-3.7\% in AUC compared to the best-performing baselines. These results demonstrate the efficacy of LR-Recsys and the value of LLM-based contrastive explanations in improving the recommendation performance. 

% Yelp: 
% RMSE 11.3
% MAE 18.6
% AUC 3

% Amazon: 
% RMSE 20
% MAE 33 
% AUC 3.7

% It is also important to note, however, that these performance improvements that we have achieved in the experiments, ranging from 3\% to 15\% in terms of the RMSE, MAE, and AUC metrics, are indeed economically significant according to the discussions in the literature \citep{gunawardana2022evaluating}. In fact, according to a series of recent online experiments conducted at major e-commerce, including Google \citep{zhang2023empowering}, Amazon \citep{chen2024shopping}, Alibaba \citep{zhou2018deep}, and LinkedIn \citep{wang2024limaml}, performance improvements on the accuracy-based metrics (such as RMSE, MAE, and AUC that we use in our paper) in the offline experiments will typically lead to significant business performance increases in the online experiments as well. For example, \citet{li2024variety} proposed a novel recommender system design that manages to achieve a significant improvement of 3\% on average in terms of the AUC and Hit Rate@10 metrics on the three offline public datasets (Yelp, MovieLens, and Alibaba). When deployed at the production system of a major video streaming platform, the authors observe a similar level of 3\% improvement over Click-Through Rate and Video View metrics, which led to an additional \$30 million in revenue for the company. Additionally, according to the industrial practices at Netflix \citep{gomez2015netflix}, even a tiny 0.1\% improvement on the business performance metric would lead to significant economic and business values for the company, while in our experiments, our proposed model consistently achieves performance improvements around or above 3\% in terms of the AUC metric, which is clearly beneficial to the business performance \citep{gunawardana2022evaluating}.

It is important to emphasize that the performance improvements achieved in our experiments are economically significant \citep{gunawardana2022evaluating}. Recent online experiments conducted by major e-commerce platforms, including Google \citep{zhang2023empowering}, Amazon \citep{chen2024shopping}, Alibaba \citep{zhou2018deep}, and LinkedIn \citep{wang2024limaml}, consistently show that improvements in accuracy-based metrics (such as RMSE, MAE, and AUC) during offline testing often translate into substantial business performance gains when deployed in production systems. For example, \citet{li2024variety} proposed a novel recommender system that achieved a 3\% improvement in AUC and Hit Rate@10 on public datasets (Yelp, MovieLens, and Alibaba). When deployed on a major video streaming platform, this improvement translated to a 3\% increase in Click-Through Rate and Video Views, resulting in an additional \$30 million in annual revenue. Netflix highlight that even a 0.1\% improvement in business performance metrics can deliver significant economic value \citep{gomez2015netflix}. Notably, our proposed model consistently achieves performance gains of approximately 3\% or more, validating its efficacy to drive meaningful business impact \citep{gunawardana2022evaluating}.


% Finally, to give the audience a better understanding of how our proposed model works, we present the following case example from our Yelp dataset, where we aim to suggest a premium Thai restaurant for a particular consumer. However, after weighing the positive and negative reasons generated by the contrastive-explanation generator in our LR-Recsys, it seems that an alternative Japanese restaurant would be a better fit for the consumer. Therefore, our proposed recommendation model only predicts a 1.14 rating for this recommended Thai restaurant, which matches the ground-truth rating of 1 and significantly outperforms the prediction of 1.89 by the best-performing baseline model PETER \citep{li2021personalized} in our experiments.

To illustrate how our proposed model operates, we present a case study from the Yelp dataset below. In this scenario, the task is to recommend a premium Thai restaurant to a specific consumer. However, after incorporating the positive and negative reasoning generated by the contrastive-explanation generator in our LR-Recsys, the model determines that an alternative Japanese restaurant would be a better fit for the consumer. Consequently, our recommendation system predicts a rating of 1.14 for the Thai restaurant, closely aligning with the ground-truth rating of 1. This prediction significantly outperforms the baseline model PETER \citep{li2021personalized}, which predicts a rating of 1.89.
\newline

\fbox{%
% \vspace{-0.1in}
    \parbox{\textwidth}{%
\textbf{Consumer Past Visiting History}: O-Ku Sushi, Zen Japanese, MGM Grand Hotel, Sen of Japan, Sushi Bong

\textbf{Restaurant Profile}: ''Siam Thai Kitchen is a Thai restaurant that offers a unique dining experience in the city. The restaurant is known for its authentic Thai cuisine and its warm and inviting atmosphere. The menu features a variety of traditional Thai dishes, as well as some modern twists on classic Thai flavors. The restaurant is perfect for couples, families, and groups of friends who are looking for a delicious and authentic Thai dining experience.''

\textbf{Generated Positive Explanation}: ''The consumer is looking for a unique and flavorful dining experience and the restaurant offers a variety of Asian cuisine.''

\textbf{Generated Negative Explanation}: ''The consumer is looking for a traditional Japanese experience and wants to escape the busy city life, while the restaurant is not a traditional Japanese experience and is located in a city''

\textbf{Positive Explanation Attention Value}: 0.23

\textbf{Negative Explanation Attention Value}: 0.87

\textbf{Predicted Consumer Rating}: 1.14

\textbf{Ground-Truth Consumer Rating}: 1

\textbf{PETER-Predicted Consumer Rating}: 1.89
    }%
}
\newline
% In the remainder of this section, we will conduct a series of additional analyses to understand where the performance improvements that we achieved in the experiments come from, as well as conduct additional ablation studies and robustness checks to demonstrate the flexibility and generalizability of LR-Recsys.

% In the remainder of this section, we conduct a series of additional analyses to explore the sources of the performance gains observed in our experiments. Additionally, we perform ablation studies and robustness checks to further demonstrate the flexibility and generalizability of LR-Recsys.

\begin{table}
\centering
\footnotesize
   \setlength\extrarowheight{2pt}
\resizebox{0.95\textwidth}{!}{
\begin{tabular}{|c|ccc|ccc|ccc|} \hline
 & \multicolumn{3}{c|}{TripAdvisor} & \multicolumn{3}{c|}{Yelp} & \multicolumn{3}{c|}{Amazon Movie} \\ \hline
 & RMSE $\downarrow$ & MAE $\downarrow$ & AUC $\uparrow$ & RMSE & MAE & AUC $\uparrow$ & RMSE $\downarrow$ & MAE $\downarrow$ & AUC $\uparrow$ \\ \hline
\textbf{LR-Recsys (Ours)} & \textbf{0.1889} & \textbf{0.1444} & \textbf{0.7289} & \textbf{0.2149} & \textbf{0.1685} & \textbf{0.7229} & \textbf{0.1673} & \textbf{0.1180} & \textbf{0.7500} \\
 & (0.0010) & (0.0008) & (0.0018) & (0.0010) & (0.0009) & (0.0017) & (0.0010) & (0.0009) & (0.0018) \\
\textbf{\% Improved} & +5.36\%*** & +15.11\%*** & +2.88\%*** & +11.31\%*** & +18.64\%*** & +3.01\%*** & +20.30\%*** & +33.33\%*** & +3.65\%*** \\ \hline
A3NCF & 0.2103 & 0.1811 & 0.6879 & 0.2607 & 0.2181 & 0.6785 & 0.2241 & 0.1903 & 0.6971 \\
 & (0.0019) & (0.0013) & (0.0027) & (0.0023) & (0.0016) & (0.0029) & (0.0029) & (0.0018) & (0.0032) \\
SULM & 0.2191 & 0.1872 & 0.6736 & 0.2823 & 0.2258 & 0.6614 & 0.2477 & 0.1980 & 0.6855 \\
 & (0.0021) & (0.0013) & (0.0027) & (0.0019) & (0.0015) & (0.0029) & (0.0027) & (0.0019) & (0.0027)\\
AARM & 0.2083 & 0.1803 & 0.6901 & 0.2582 & 0.2162 & 0.6801 & 0.2162 & 0.1845 & 0.7032 \\
 & (0.0019) & (0.0014) & (0.0030) & (0.0021) & (0.0015) & (0.0029) & (0.0027) & (0.0018) & (0.0029) \\
MMALFM & 0.2117 & 0.1820 & 0.6894 & 0.2591 & 0.2167 & 0.6801 & 0.2301 & 0.1931 & 0.6931 \\
 & (0.0019) & (0.0014) & (0.0029) & (0.0020) & (0.0016) & (0.0030) & (0.0028) & (0.0020) & (0.0036) \\
ANR & 0.2083 & 0.1804 & 0.6905 & 0.2575 & 0.2145 & 0.6817 & 0.2275 & 0.1915 & 0.6960 \\
 & (0.0017) & (0.0014) & (0.0027) & (0.0021) & (0.0017) & (0.0031) & (0.0026) & (0.0018) & (0.0026)\\ 
MTER & 0.2099 & 0.1825 & 0.6889 & 0.2614 & 0.2169 & 0.6809 & 0.2283 & 0.1906 & 0.6967 \\
 & (0.0019) & (0.0014) & (0.0029) & (0.0021) & (0.0016) & (0.0031) & (0.0026) & (0.0017) & (0.0026) \\ \hline
SASRec & 0.2089 & 0.1731 & 0.7005 & 0.2491 & 0.2135 & 0.6897 & 0.2176 & 0.1869 & 0.7025 \\
 & (0.0007) & (0.0006) & (0.0015) & (0.0011) & (0.0009) & (0.0016) & (0.0013) & (0.0008) & (0.0013) \\
DIN & 0.2022 & 0.1709 & 0.7076 & 0.2479 & 0.2116 & 0.6917 & 0.2155 & 0.1853 & 0.7046 \\
 & (0.0009) & (0.0007) & (0.0017) & (0.0009) & (0.0008) & (0.0015) & (0.0009) & (0.0007) & (0.0013) \\
BERT4Rec & 0.2003 & \underline{0.1701} & \underline{0.7085} & 0.2460 & 0.2101 & 0.6928 & 0.2126 & 0.1832 & 0.7088 \\
 & (0.0009) & (0.0006) & (0.0017) & (0.0009) & (0.0008) & (0.0015) & (0.0009) & (0.0008) & (0.0015) \\
UniSRec & 0.2026 & 0.1720 & 0.7066 & 0.2448 & 0.2093 & 0.6956 & 0.2103 & 0.1810 & 0.7133 \\
 & (0.0015) & (0.0010) & (0.0023) & (0.0013) & (0.0011) & (0.0020) & (0.0011) & (0.0009) & (0.0017) \\ \hline
AMCF & 0.2088 & 0.1755 & 0.6989 & 0.2501 & 0.2123 & 0.6928 & 0.2376 & 0.1863 & 0.7035 \\
 & (0.0019) & (0.0013) & (0.0027) & (0.0016) & (0.0013) & (0.0023) & (0.0013) & (0.0010) & (0.0019) \\
PETER & \underline{0.1996} & 0.1715 & 0.7078 & \underline{0.2423} & \underline{0.2071} & 0.7003 & \underline{0.2099} & \underline{0.1770} & \underline{0.7226} \\
 & (0.0019) & (0.0013) & (0.0027) & (0.0015) & (0.0013) & (0.0022) & (0.0013) & (0.0010) & (0.0019) \\ 
UCEPic & 0.2035 & 0.1723 & 0.7066 & 0.2477 & 0.2099 & \underline{0.7018} & 0.2228 & 0.1801 & 0.7080 \\
 & (0.0015) & (0.0011) & (0.0023) & (0.0015) & (0.0012) & (0.0023) & (0.0011) & (0.0009) & (0.0017) \\ 
PARSRec & 0.2008 & 0.1703 & 0.7080 & 0.2471 & 0.2106 & 0.6923 & 0.2133 & 0.1837 & 0.7069 \\
 & (0.0009) & (0.0007) & (0.0017) & (0.0009) & (0.0008) & (0.0015) & (0.0009) & (0.0007) & (0.0013) \\ \hline 
\end{tabular}
}
\caption{Recommendation performance on three datasets. ``\%Improved'' represents the performance gains of LR-Recsys (ours) compared to the best-performing baseline (underlined). Metrics with $\downarrow$ indicate that lower values are better (e.g., RMSE, MAE), while metrics with $\uparrow$ indicate that higher values are better (e.g., AUC). ***p$<$0.01;**p$<$0.05.}
\label{main_results}
\end{table}



\subsection{Understanding the Improvements} 
In this section, we present additional analyses to decompose and better understand the significant performance gains observed with LR-Recsys in the previous section.

\subsubsection{Improved learning efficiency.} 

Based on the theoretical insights discussed in Section \ref{sec:theory}, incorporating explanations into the recommendation process is expected to significantly improve learning efficiency. This implies that our proposed LR-Recsys should require \emph{less} training data to achieve recommendation performance comparable to the baselines. To demonstrate this, we randomly sample subsets of the three datasets, keeping 12\%, 25\%, and 50\% of the original training data, and train LR-Recsys on these subsets while keeping the same test set for evaluation. The results, presented in Table \ref{efficiency}, show that our model achieves performance equivalent to the best-performing baseline, PETER \citep{li2021personalized}, using as little as 25\% of the training data. These findings validate the improved learning efficiency of LR-Recsys, matching the theoretical insights in Section \ref{sec:theory}.

\begin{table}
\centering
\footnotesize
   \setlength\extrarowheight{3pt}
\resizebox{0.95\textwidth}{!}{
\begin{tabular}{|c|ccc|ccc|ccc|} \hline
 & \multicolumn{3}{c|}{TripAdvisor} & \multicolumn{3}{c|}{Yelp} & \multicolumn{3}{c|}{Amazon Movie} \\ \hline
 & RMSE $\downarrow$ & MAE $\downarrow$  & AUC $\uparrow$ & RMSE $\downarrow$ & MAE $\downarrow$ & AUC $\uparrow$ & RMSE $\downarrow$ & MAE $\downarrow$ & AUC $\uparrow$ \\ \hline
 LR-Recsys & 0.1889 & 0.1444 & 0.7289 & 0.2149 & 0.1685 & 0.7229 & 0.1673 & 0.1180 & 0.7500 \\
 100\% Training Data & (0.0010) & (0.0008) & (0.0018) & (0.0010) & (0.0009) & (0.0017) & (0.0010) & (0.0009) & (0.0018) \\ \hline
LR-Recsys & 0.1938 & 0.1503 & 0.7144 & 0.2188 & 0.1774 & 0.7133 & 0.1791 & 0.1308 & 0.7276 \\
 50\% Training Data & (0.0017) & (0.0013) & (0.0029) & (0.0018) & (0.0017) & (0.0031) & (0.0021) & (0.0017) & (0.0033) \\ \hline
LR-Recsys & 0.2017 & 0.1679 & 0.7020 & 0.2356 & 0.1958 & 0.7004 & 0.1997 & 0.1703 & 0.7173 \\
25\% Training Data  & (0.0026) & (0.0021) & (0.0041) & (0.0027) & (0.0027) & (0.0047) & (0.0039) & (0.0028) & (0.0049) \\ \hline
LR-Recsys & 0.2098 & 0.1796 & 0.6912 & 0.2557 & 0.2140 & 0.6822 & 0.2175 & 0.1866 & 0.7015 \\
12\% Training Data & (0.0036) & (0.0030) & (0.0054) & (0.0039) & (0.0038) & (0.0068) & (0.0066) & (0.0044) & (0.0063) \\ \hline
PETER & 0.1996 & 0.1715 & 0.7078 & 0.2423 & 0.2071 & 0.7003 & 0.2099 & 0.1770 & 0.7226 \\
100\% Training Data & (0.0019) & (0.0013) & (0.0027) & (0.0015) & (0.0013) & (0.0022) & (0.0013) & (0.0010) & (0.0019) \\ \hline
\end{tabular}
}
\caption{Recommendation performance across three datasets using varying percentages of training data for LR-Recsys.}
\label{efficiency}
\end{table}

\subsubsection{The gain is from LLM's reasoning capability.}  
\label{res_understanding_reasoning}
The theoretical insights in Section \ref{sec:theory} highlight that the advantage of LR-Recsys lies in leveraging LLMs' strong \emph{reasoning} capabilities to identify the important variables. Therefore, LLMs with better reasoning capabilities are expected to lead to better recommendation performance. To validate this, we conduct additional experiments within the LR-Recsys framework, using different LLMs with varying reasoning capabilities. As shown in Table \ref{llm_models}, the performance of LR-Recsys with Llama 3.1 is significantly better than LR-Recsys with Llama 3, Mixtral-8$\times$7b, Vicuna-7b-v1.5, Qwen2-7B, or GPT-2. This aligns with the reasoning capability leaderboard at \url{https://huggingface.co/spaces/allenai/ZebraLogic}, where Llama 3.1 demonstrates the highest reasoning capabilities among the tested models. Furthermore, Llama 3 and Mixtral-8$\times$7b also outperform Vicuna-7b-v1.5, Qwen2-7B, and GPT-2 in the reasoning leaderboard, which is also aligned with the results observed in LR-Recsys. These results confirm that better \emph{reasoning} capabilities in LLMs directly translate to improved performance within the LR-Recsys framework. 

Moreover, LR-Recsys significantly outperforms an alternative approach that uses LLMs directly for recommendations—without generating explicit explanations (``LLM Direct Recommendation (with Llama 3.1)'' row in Table \ref{llm_models}). This suggests that only using LLMs for recommendation without tapping into their reasoning abilities is insufficient. Additionally, we find that including LLM-generated product profile information plays only a minor role in the overall effectiveness of the model, as LR-Recsys continues to significantly outperform baseline models even when these augmented profiles are removed (``LR-Recsys w/o Profile Augmentation'' row in Table \ref{llm_models}).

Furthermore, we confirm that the observed performance improvements are not due to information leakage or pre-existing dataset knowledge. For example, the Amazon Movie dataset was collected in 2023, while GPT-2 was pre-trained on data available only up to 2019. Despite this, when GPT-2 is used within the LR-Recsys framework, our approach still outperforms other baselines.



Finally, we also tested utilizing LLMs' summarization capabilities instead of their reasoning abilities within LR-Recsys. Specifically, we replace the positive and negative explanation prompts in the contrastive-explanation generator with the following prompt: 
\begin{quote}
\emph{
``Given the profiles of the watching history of this consumer \{movie\_profile\_seq\}, can you provide a summary of the consumer preference of candidate movies?'' 
}
\end{quote} 
In other words, we leverage the LLM's summarization skills to condense the user's consumption history, and then use this summarized information as input to the DNN instead of the positive and negative explanations. As shown in the last row of Table \ref{llm_models} (``Consumption History Summarization''), the performance of using LLM for summarization is significantly worse than that of LR-Recsys using LLM for explanations. This suggests that the gain from LR-Recsys specifically comes from the \emph{reasons} (positive and negative explanations) provided by LLMs, rather than their ability to summarize consumption history.

Collectively, these findings support the conclusion that the performance gains observed with LR-Recsys are primarily driven by the LLMs' reasoning capabilities, \emph{not} their external dataset knowledge or summarization skills.







% To validate this point, we conduct additional experiments in this section, where we implement multiple versions of LLMs with different levels of reasoning capabilities to construct our proposed model, as well as using LLMs directly to produce recommendations (i.e. without explicitly asking for explanations). As we can observe from Table \ref{llm_models}, recommendation performance obtained from using Llama 3.1 is significantly better than that of Llama 3 and Mixtral-8$\times$7b, and even more so than that of Vicuna-7b-v1.5, Qwen2-7B, and GPT-2. This is consistent with the reasoning capability leaderboard shown at \url{https://huggingface.co/spaces/allenai/ZebraLogic}, where Llama 3.1 is shown to possess the best reasoning capability compared to Llama 3 and Mixtral-8$\times$7b. Furthermore, these three LLMs perform significantly better than Vicuna-7b-v1.5, Qwen2-7B, and GPT-2 in terms of reasoning capability. Therefore, our results confirm that better \emph{reasoning} capabilities in LLMs directly translate to improved performance in our LR-Recsys framework. Furthermore, our proposed model also significantly outperforms the alternative model that uses LLM directly for recommendations, since it does not utilize any benefits coming from the LLM's reasoning capability. We can also verify that the incorporation of the LLM-generated product profile information only plays a small part in terms of the effectiveness of our proposed model, since LR-Recsys still significantly outperforms the baseline models even when the LLM-augmented profile is removed (last row in Table \ref{llm_models}). Lastly, we would like to point out that the performance improvements are not a result of the information/knowledge leakage, since one of our offline datasets, Amazon Movie, was collected in 2023, while GPT-2 uses pre-training data prior to 2019. Even when we use the GPT-2 as the backbone model, we can still achieve performance improvements over prior models in the literature.

\begin{table}
\centering
\footnotesize
   \setlength\extrarowheight{3pt}
\resizebox{1.00\textwidth}{!}{
\begin{tabular}{|c|ccc|ccc|ccc|} \hline
 & \multicolumn{3}{c|}{TripAdvisor} & \multicolumn{3}{c|}{Yelp} & \multicolumn{3}{c|}{Amazon Movie} \\ \hline
 & RMSE $\downarrow$ & MAE $\downarrow$ & AUC $\uparrow$ & RMSE $\downarrow$ & MAE $\downarrow$ & AUC $\uparrow$ & RMSE $\downarrow$ & MAE $\downarrow$ & AUC $\uparrow$ \\ \hline
LR-Recsys with \textbf{Llama 3.1} (Ours) & 0.1889 & 0.1444 & 0.7289 & 0.2149 & 0.1685 & 0.7229 & 0.1673 & 0.1180 & 0.7500 \\
 & (0.0010) & (0.0008) & (0.0018) & (0.0010) & (0.0009) & (0.0017) & (0.0010) & (0.0009) & (0.0018) \\ \hline
LR-Recsys with \textbf{Llama 3}  & 0.1934 & 0.1491 & 0.7260 & 0.2166 & 0.1697 & 0.7210 & 0.1695 & 0.1203 & 0.7472 \\
 & (0.0010) & (0.0008) & (0.0018) & (0.0010) & (0.0009) & (0.0017) & (0.0010) & (0.0009) & (0.0018) \\ \hline
LR-Recsys with \textbf{Mixtral-8} $\times$7b & 0.1910 & 0.1462 & 0.7271 & 0.2163 & 0.1693 & 0.7218 & 0.1691 & 0.1199 & 0.7480 \\
 & (0.0010) & (0.0008) & (0.0018) & (0.0010) & (0.0009) & (0.0017) & (0.0010) & (0.0009) & (0.0018) \\ \hline
LR-Recsys with \textbf{Vicuna-7b-v1.5} & 0.1949 & 0.1502 & 0.7243 & 0.2175 & 0.1703 & 0.7196 & 0.1703 & 0.1210 & 0.7455 \\
 & (0.0010) & (0.0008) & (0.0018) & (0.0010) & (0.0009) & (0.0017) & (0.0010) & (0.0009) & (0.0018) \\ \hline
LR-Recsys with \textbf{Qwen2-7B}  & 0.1966 & 0.1520 & 0.7202 & 0.2193 & 0.1724 & 0.7170 & 0.1727 & 0.1235 & 0.7419 \\
 & (0.0010) & (0.0008) & (0.0018) & (0.0010) & (0.0009) & (0.0017) & (0.0010) & (0.0009) & (0.0018) \\ \hline
LR-Recsys with \textbf{GPT-2}  & 0.1940 & 0.1582 & 0.7169 & 0.2211 & 0.1801 & 0.7144 & 0.1799 & 0.1304 & 0.7288 \\
 & (0.0014) & (0.0010) & (0.0023) & (0.0015) & (0.0013) & (0.0023) & (0.0015) & (0.0013) & (0.0024) \\ \hline
LLM Direct Recommendation (with Llama 3.1)& 0.2233 & 0.1838 & 0.6735 & 0.2976 & 0.2402 & 0.6528 & 0.2680 & 0.2055 & 0.6736 \\
 & (0.0033) & (0.0020) & (0.0036) & (0.0044) & (0.0029) & (0.0046) & (0.0046) & (0.0031) & (0.0055) \\ \hline
LR-Recsys w/o Profile Augmentation & 0.1973 & 0.1520 & 0.7211 & 0.2193 & 0.1719 & 0.7173 & 0.1744 & 0.1239 & 0.7430 \\
 & (0.0013) & (0.0011) & (0.0021) & (0.0012) & (0.0012) & (0.0021) & (0.0013) & (0.0012) & (0.0021) \\ \hline
Consumption History Summarization & 0.1971 & 0.1700 & 0.7055 & 0.2409 & 0.2051 & 0.6990 & 0.2077 & 0.1751 & 0.7236 \\
 & (0.0011) & (0.0008) & (0.0018) & (0.0011) & (0.0009) & (0.0017) & (0.0011) & (0.0010) & (0.0018) \\ \hline
\end{tabular}
}
\caption{Recommendation performance across three datasets using LLMs with varying levels of reasoning capability.}
\label{llm_models}
\end{table}

\subsubsection{``Harder'' examples benefit more from LR-Recsys.} 
% Harder Examples Benefit More.

By leveraging LLMs' reasoning capabilities to identify important variables, LR-Recsys should intuitively provide greater benefits for ``harder'' examples where consumers' decisions are less obvious. To test this hypothesis, we compute the prediction uncertainty for each observation in our datasets, measured as the variance—or “disagreement”—across the predictions made by our model and all baseline models from Table \ref{main_results}. Intuitively, higher prediction uncertainty indicates a more challenging, or ``harder'', prediction task.

In Fig.\ref{fig:uncertainty}, we created plots for each dataset, where the x-axis represents the normalized uncertainty level (scaled between 0 and 1 using min-max normalization \citep{patro2015normalization}), and the y-axis represents the performance improvement of our LR-Recsys over the best-performing baseline (measured by RMSE). As shown by the regression lines in Figures \ref{fig:uncertainty:amazon}, \ref{fig:uncertainty:tripadvisor}, and \ref{fig:uncertainty:yelp}, there is a statistically significant \emph{positive correlation} between uncertainty and performance improvements. Specifically, the performance gains from incorporating explanations are consistently larger for high-uncertainty examples across all three datasets, validating the insight that our LR-Recsys is more beneficial for examples that are ``harder'' or more uncertain.

This observation aligns with the theoretical insights in Section \ref{sec:theory}. For ``harder'' examples, the model is likely uncertain about which input variables to rely on for making predictions, leading to higher prediction uncertainty. In such cases, the knowledge provided by LLMs about the important variables becomes particularly valuable, allowing the model to focus on the most relevant features. Consequently, the performance gains of our LR-Recsys are larger for these more challenging cases.

\begin{figure}[hbtp!]
% \vspace{-0.2in}
    \begin{subfigure}[b]{0.45\textwidth}
        \centering
        \includegraphics[width=\textwidth]{figures/uncertainty_line.jpg}
        \caption{Amazon Dataset.}
        \label{fig:uncertainty:amazon}
    \end{subfigure}
    \hspace{0.5mm}
     \centering
    \begin{subfigure}[b]{0.45\textwidth}
        \centering
        \includegraphics[width=\textwidth]{figures/hotel_uncertainty_line.jpg}
        \caption{TripAdvisor Dataset.}
        \label{fig:uncertainty:tripadvisor}
    \end{subfigure}
    \begin{subfigure}[b]{0.45\textwidth}
        \centering
        \includegraphics[width=\textwidth]{figures/restaurant_uncertainty_line.jpg}
        \caption{Yelp Dataset.}
        \label{fig:uncertainty:yelp}
    \end{subfigure}
   \caption{Performance improvement of LR-Recsys against (normalized) pediction uncertainty.}
  \label{fig:uncertainty}
% \vspace{-0.2in}
\end{figure}


% \begin{figure}[hbtp!]
% \centering
% \includegraphics[width=0.5\linewidth]{figures/uncertainty_line.jpg}
% \caption{Uncertainty Analysis for the Amazon Dataset}\label{fig:uncertainty:amazon}
% \vspace{-0.2in}
% \end{figure}

% \begin{figure}[hbtp!]
% \centering
% \includegraphics[width=0.5\linewidth]{figures/hotel_uncertainty_line.jpg}
% \caption{Uncertainty Analysis for the TripAdvisor Dataset}\label{fig:uncertainty:tripadvisor}
% \vspace{-0.2in}
% \end{figure}

% \begin{figure}[hbtp!]
% \centering
% \includegraphics[width=0.5\linewidth]{figures/restaurant_uncertainty_line.jpg}
% \caption{Uncertainty Analysis for the Yelp Dataset}\label{fig:uncertainty:yelp}
% \vspace{-0.2in}
% \end{figure}

\subsection{Understanding the Role of Contrastive Explanations} 
\label{sec:role_pos_neg}
% We conducted additional analysis to understand how both the positive explanations and the negative explanations contribute to the significant performance improvements from LR-Recsys. To begin with, we implement four variants of our proposed model and test their performance respectively: (1) \textbf{Aspect Terms Only}, where we use the LLM to generate only a few aspect terms that represent the most important properties of the candidate item that the consumer might consider, rather than ask for explicit explanations by leveraging LLM's reasoning capability; (2) \textbf{Positive Explanations Only}, where we use the LLM to generate only positive explanations (without negative explanations); (3) \textbf{General Explanations Only}, where we let the LLM to infer whether the use may or may not like the product, and generate the associated explanations accordingly; and (4) \textbf{Purchashing History Summarization}, where we use the LLM to summarize the purchasing history of the user and use the summarized information for producing recommendations, rather than generating the explanations for recommendations. The results are summarized in Table \ref{generation}. As we can observe from Table \ref{generation}, all of these variants do not perform as well as our proposed LR-Recsys, demonstrating the significant role of both the positive and the negative explanations.

\subsubsection{The need for both positive and negative explanations.} 
\label{sec:dual_explanation}

We conducted analyses to understand how explanations, and in particular both positive and negative explanations, contribute to the significant performance improvements of LR-Recsys. We compared three variants of the contrastive-explanation generator in LR-Recsys:
(1) \textbf{Aspect Terms Only}: The LLM only generates a few aspect terms representing the most important properties of the candidate item that the consumer might consider, without leveraging explicit reasoning-based explanations;
(2) \textbf{Positive Explanations Only}: The LLM generates only positive explanations;
(3) \textbf{General Explanations Only}: The LLM infers whether the user may or may not like the product and generates corresponding general explanations, without distinguishing between positive and negative reasoning;
% (4) \textbf{Purchasing History Summarization}: The LLM summarizes the user's purchasing history and uses this summarized information instead of the generated explanations as input to the DNN.

The detailed prompts used for each variant are listed in Appendix \ref{appen:variants_prompt}. The results, summarized in Table \ref{generation}, show that none of these variants match the performance of our proposed LR-Recsys with the contrastive-explanation generator. This confirms the value of incorporating both positive and negative explanations to improve predictive performance.


\begin{table}
\centering
\footnotesize
   \setlength\extrarowheight{3pt}
\resizebox{0.95\textwidth}{!}{
\begin{tabular}{|c|ccc|ccc|ccc|} \hline
 & \multicolumn{3}{c|}{TripAdvisor} & \multicolumn{3}{c|}{Yelp} & \multicolumn{3}{c|}{Amazon Movie} \\ \hline
 & RMSE $\downarrow$ & MAE $\downarrow$ & AUC $\uparrow$ & RMSE $\downarrow$ & MAE $\downarrow$ & AUC $\uparrow$ & RMSE $\downarrow$ & MAE $\downarrow$ & AUC $\uparrow$ \\ \hline
LR-Recsys (Ours) & \textbf{0.1889} & \textbf{0.1444} & \textbf{0.7289} & \textbf{0.2149} & \textbf{0.1685} & \textbf{0.7229} & \textbf{0.1673} & \textbf{0.1180} & \textbf{0.7500} \\
 & (0.0010) & (0.0008) & (0.0018) & (0.0010) & (0.0009) & (0.0017) & (0.0010) & (0.0009) & (0.0018) \\ \hline
Aspect Terms Only & 0.1975 & 0.1709 & 0.7071 & 0.2413 & 0.2053 & 0.6991 & 0.2083 & 0.1757 & 0.7243 \\
 & (0.0010) & (0.0008) & (0.0018) & (0.0010) & (0.0009) & (0.0017) & (0.0011) & (0.0010) & (0.0018) \\ \hline
Positive Explanations Only & 0.1928 & 0.1480 & 0.7258 & 0.2179 & 0.1708 & 0.7168 & 0.1703 & 0.1209 & 0.7456 \\
 & (0.0010) & (0.0008) & (0.0018) & (0.0010) & (0.0009) & (0.0018) & (0.0010) & (0.0009) & (0.0018) \\ \hline
General Explanations Only & 0.1961 & 0.1633 & 0.7006 & 0.2499 & 0.2279 & 0.6710 & 0.2136 & 0.1797 & 0.7076 \\
 & (0.0014) & (0.0010) & (0.0023) & (0.0015) & (0.0016) & (0.0023) & (0.0015) & (0.0013) & (0.0027) \\ \hline
% Purchasing History Summarization & 0.1971 & 0.1700 & 0.7055 & 0.2409 & 0.2051 & 0.6990 & 0.2077 & 0.1751 & 0.7236 \\
%  & (0.0011) & (0.0008) & (0.0018) & (0.0011) & (0.0009) & (0.0017) & (0.0011) & (0.0010) & (0.0018) \\ \hline
\end{tabular}
}
\caption{Recommendation performance across three datasets using LLMs for alternative generation tasks.}
\label{generation}
\end{table}


\subsubsection{Attention weights on positive and negative explanations.} 
\label{sec:attention_analysis}

% For the classification task, the goal of the recommender system is to predict whether a consumer will give a high rating to a particular product. In Sections \ref{sec:main_results} and \ref{sec:dual_explanation}, we demonstrated that both positive and negative explanations contribute to the improved prediction accuracy of this task. 

We further analyze the contribution of each type of explanation to different predictions. One hypothesis is that high ratings depend more on positive explanations (i.e., reasons the consumer likes the product), while low ratings depend more on negative explanations (i.e., reasons the consumer does not like the product).

To test this, we conduct an \emph{attention value analysis} to quantify the contributions of positive and negative explanations to the classification task. The attention value for the positive explanation ${\bar{\alpha}}_{pos}$ is computed as the average of all the relevant pairwise attention values in the attention layer of the DNN component. Following the notation in Section \ref{sec:framework_dnn_loss}, we define:
\begin{equation}
\label{eq:alpha_pos_neg}
\begin{aligned}
{\bar{\alpha}}_{pos} &= \frac{1}{|I|}{\sum_{i\in I}\alpha_{i, pos}}, \;\;\;\;\;
{\bar{\alpha}}_{neg} &= \frac{1}{|I|}{\sum_{i\in I}\alpha_{i, neg}},
\end{aligned}
\end{equation}
where $I = [\text{pos}, \text{neg}, c, p, \text{seq}, \text{context}]$ represents the index set corresponding to each element in the input $X_{\text{input}}$. In other words, ${\bar{\alpha}}_{pos}$ captures the average ``attention'' that the model puts on the positive explanation embedding when generating the final prediction, and ${\bar{\alpha}}_{neg}$ captures the average ``attention'' put on the negative explanation embedding. Therefore, ${\bar{\alpha}}_{pos}$ and ${\bar{\alpha}}_{neg}$ estimates the relative importance of the positive and negative explanations in producing the final recommendation results. 

% We visualize the distribution of attention values ${\bar{\alpha}}_{pos}$ and ${\bar{\alpha}}_{neg}$ across three datasets in Fig.\ref{fig:attention}. The results show that when a product receives a high rating, LR-Recsys assigns \emph{more} attention to positive explanations than negative ones, as indicated by the distribution of attention weights for positive explanations (blue bars) being skewed further to the \emph{right} compared to negative explanations (orange bars) (Fig.\ref{fig:positive:yelp}, \ref{fig:positive:amazon}, and \ref{fig:positive:tripadvisor}). Conversely, for products receiving a low rating, LR-Recsys assigns more attention to negative explanations, with the distribution of attention weights for positive explanations shifting further to the \emph{left} compared to negative explanations (Fig. \ref{fig:negative:yelp}, \ref{fig:negative:amazon}, and \ref{fig:negative:tripadvisor}). In Appendix \ref{appen:attn_pie_charts}, we also plot the distribution of the attenion values on the other input components (i.e. consumer, item and context embeddings) and find that these explanations together account for a significant (30-40\%) of total model attention values.

% 
We visualize the distribution of attention values on positive explanations (${\bar{\alpha}}_{pos}$) and negative explanations (${\bar{\alpha}}_{neg}$) for the Yelp Dataset in Fig.\ref{fig:attention}, while the results for all three datasets are provided in Appendix \ref{appen:attn_pos_neg}. The results across all three datasets consistently show that when a product receives a high rating, LR-Recsys assigns \emph{more} attention to positive explanations than negative ones, as indicated by the distribution of attention weights for positive explanations (blue bars) being skewed further to the \emph{right} compared to negative explanations (orange bars) (Fig.\ref{fig:positive:yelp}). Conversely, for products receiving a low rating, LR-Recsys assigns more attention to negative explanations, with the distribution of attention weights for positive explanations shifting further to the \emph{left} compared to negative explanations (Fig. \ref{fig:negative:yelp},). In Appendix \ref{appen:attn_pie_charts}, we also plot the distribution of attention values on other input components (i.e., consumer, item, and context embeddings) and find that these explanations together account for a significant (30-40\%) of total attention values. 

The insights from these plots are significant. In addition to confirming the importance of both types of explanations in generating predictions, the separation of attention weight distributions between positive and negative explanations highlights how LR-Recsys effectively leverages contrastive explanations for the prediction task. Across all three datasets, we observe that positive predictions rely more on positive explanations, while negative predictions depend more on negative explanations. This behavior is intuitive: When a product is predicted to receive a high rating, the framework selectively focuses on positive explanations (reasons the consumer may like the product); conversely, when a product is predicted to receive a low rating, the framework is focused on negative explanations (reasons the consumer may \emph{not} like the product). By providing both positive and negative explanations through the contrastive-explanation generator, LR-Recsys is able to intelligently decide which type of explanation to rely on, in order to generate the most accurate predictions. This explains why both types of explanations are critical for the prediction, and why LR-Recsys outperforms other LLM-based recommendation system variants as demonstrated above.

% when producing positive recommendations, the recommender system will pay more attention to positive explanations, as evidenced by their higher attention values compared to the negative explanations; similarly, when producing negative recommendations, the recommender system will pay more attention to negative explanations. Therefore, both explanations are indeed crucial components for determining the final decisions of the recommendation model.

\begin{figure}[hbtp!]
% \vspace{-0.2in}
    \begin{subfigure}[b]{0.45\textwidth}
        \centering
        \includegraphics[width=\textwidth]{figures/restaurant_positive_record.jpg}
        \caption{Positive examples (Yelp Dataset).}
        \label{fig:positive:yelp}
    \end{subfigure}
    \hspace{0.5mm}
     \centering
    \begin{subfigure}[b]{0.45\textwidth}
        \centering
        \includegraphics[width=\textwidth]{figures/restaurant_negative_record.jpg}
        \caption{Negative examples (Yelp Dataset).}
        \label{fig:negative:yelp}
    \end{subfigure}
    
    % \begin{subfigure}[b]{0.45\textwidth}
    %     \centering
    %     \includegraphics[width=\textwidth]{figures/movie_positive_record.jpg}
    %     \caption{Positive examples (Amazon Movie Dataset).}
    %     \label{fig:positive:amazon}
    % \end{subfigure}
    % \hspace{0.5mm}
    %  \centering
    % \begin{subfigure}[b]{0.45\textwidth}
    %     \centering
    %     \includegraphics[width=\textwidth]{figures/movie_negative_record.jpg}
    %     \caption{Negative examples (Amazon Movie Dataset).}
    %     \label{fig:negative:amazon}
    % \end{subfigure}
    
    % \begin{subfigure}[b]{0.45\textwidth}
    %     \centering
    %     \includegraphics[width=\textwidth]{figures/hotel_positive_record.jpg}
    %     \caption{Positive examples (TripAdvisor Dataset).}
    %     \label{fig:positive:tripadvisor}
    % \end{subfigure}
    % \hspace{0.5mm}
    %  \centering
    % \begin{subfigure}[b]{0.45\textwidth}
    %     \centering
    %     \includegraphics[width=\textwidth]{figures/hotel_negative_record.jpg}
    %     \caption{Negative examples (TripAdvisor Dataset).}
    %     \label{fig:negative:tripadvisor}
    % \end{subfigure}
    
   \caption{(Color online) Distribution of attention values on positive and negative explanations for positive and negative examples.}
  \label{fig:attention}
% \vspace{-0.2in}
\end{figure}

% \begin{figure}[hbtp!]
% \centering
% \includegraphics[width=0.5\linewidth]{figures/movie_positive_record.jpg}
% \caption{Attention Value Analysis Positive Explanations in the Amazon Dataset}\label{fig:positive:amazon}
% \vspace{-0.2in}
% \end{figure}

% \begin{figure}[hbtp!]
% \centering
% \includegraphics[width=0.5\linewidth]{figures/movie_negative_record.jpg}
% \caption{Attention Value Analysis Negative Explanations in the Amazon Dataset}\label{fig:negative:amazon}
% \vspace{-0.2in}
% \end{figure}

% \begin{figure}[hbtp!]
% \centering
% \includegraphics[width=0.5\linewidth]{figures/restaurant_positive_record.jpg}
% \caption{Attention Value Analysis Positive Explanations in the Yelp Dataset}\label{fig:positive:yelp}
% \vspace{-0.2in}
% \end{figure}

% \begin{figure}[hbtp!]
% \centering
% \includegraphics[width=0.5\linewidth]{figures/restaurant_negative_record.jpg}
% \caption{Attention Value Analysis Negative Explanations in the Yelp Dataset}\label{fig:negative:yelp}
% \vspace{-0.2in}
% \end{figure}

% \begin{figure}[hbtp!]
% \centering
% \includegraphics[width=0.5\linewidth]{figures/hotel_positive_record.jpg}
% \caption{Attention Value Analysis Positive Explanations in the TripAdvisor Dataset}\label{fig:positive:tripadvisor}
% \vspace{-0.2in}
% \end{figure}

% \begin{figure}[hbtp!]
% \centering
% \includegraphics[width=0.5\linewidth]{figures/hotel_negative_record.jpg}
% \caption{Attention Value Analysis Negative Explanations in the TripAdvisor Dataset}\label{fig:negative:tripadvisor}
% \vspace{-0.2in}
% \end{figure}


\subsubsection{Actionable business insights from aggregated contrastive explanations.} 
\label{sec:actionable_insights}

Another benefit of LR-Recsys is its ability to provide actionable insights for business owners or content creators. For instance, we aggregate all positive and negative explanations associated with a specific restaurant, Siam Thai Kitchen, from the Yelp dataset. A summary of the top keywords (based on word frequencies) reveals that the restaurant excels at offering an authentic Thai dining experience. However, the negative explanations highlight opportunities for improvement, such as expanding the menu with healthier, more upscale options to attract a broader audience. These actionable insights, generated through contrastive explanations in LR-Recsys, are not possible with traditional black-box recommender systems.
\newline
% Another benefit of LR-Recsys is its potential to provide actionable insights for business owners or content creators. As an example, we collect all interaction records associated with a specific restaurant, Siam Thai Kitchen, in the Yelp dataset, and aggregate all of the positive and negative explanations generated in our LR-Recsys framework.

% A summary of the top keywords, based on word frequencies in the generated explanations, is listed below. We see that the restaurant is doing well in terms of offering a unique and authentic Thai dining experience. However, the negative explanations also suggest opportunities for improvement, such as improving its menu to include healthier and more upscale options to attract more consumers. Such actionable insights are not possible with traditional black-box recommender systems, but are made possible by the contrastive explanations generated through our LR-Recsys framework.



\fbox{%
    \parbox{\textwidth}{%
\textbf{Restaurant Name}: Siam Thai Kitchen

\textbf{Restaurant Profile}: ''Siam Thai Kitchen is a Thai restaurant that offers a unique dining experience in the city. The restaurant is known for its authentic Thai cuisine and its warm and inviting atmosphere. The menu features a variety of traditional Thai dishes, as well as some modern twists on classic Thai flavors. The restaurant is perfect for couples, families, and groups of friends who are looking for a delicious and authentic Thai dining experience.''

\textbf{Top Keywords in Positive Explanations}: Unique, Thai, Traditional, Atmosphere, Dining, Experience, Service, Authentic

\textbf{Top Keywords in Negative Explanations}: Healthy, Luxurious, Fine Dining, Upscale, Quick, Snack, Modern 
    }%
}
\newline


\subsection{Ablation Studies and Robustness Checks}
We conduct a series of additional ablation studies and robustness checks to further demonstrate the effectiveness and generalizability of LR-Recsys. We provide the detailed results in Appendix \ref{appen:ablation_robustness}.



\begin{comment}
\subsection{Descriptive analysis of the generated explanations} \hfill\\
\todo{use LLM to summarize}

Profile Prompt: ''Create a succinct profile for a hotel based on its name. This profile should be tailored for use in recommendation systems and must identify the types of consumers who would enjoy the hotel.''

Input: ''Hilton Newark Airport''

Output: ''Stay at the Hilton Newark Airport and experience a comfortable and convenient stay. Ideal for business and leisure travelers, this hotel is located just minutes away from the airport and offers easy access to the city's top attractions. Enjoy the hotel's amenities, including a fitness center, restaurant, and bar.''

Explanation Prompt: ''Given the profiles of restaurants visited by this consumer in the past, can you provide a reason for why this consumer watched the following recommended movie with the following profile? Answer with one sentence with the following format: The consumer visited this restaurant because''

Positive Explanation Output: The consumer visited this restaurant because the consumer is looking for a unique and flavorful dining experience and the restaurant offers a variety of Asian cuisine.

Negative Explanation Output: The consumer did not visit this restaurant because the consumer is looking for a traditional Japanese experience and wants to escape the busy city life and the restaurant is not a traditional Japanese experience and is located in a city.

\todo{Pan: any description analysis / summary statistics you can show here to give insights on (1) what are common positive explanations, and (2) what are common negative explanations? }
\end{comment}
\section{Discussion}
\label{sec:discussion}

    \begin{table*}[t!]
    \centering
    \caption{Summary of some related works with ours in computational notebook topic.}
    \begin{tabular}{|m{2cm}|>{\raggedright}m{1.6cm}|>{\centering\arraybackslash}m{0.8cm}|>{\centering\arraybackslash}m{1cm}|>{\centering\arraybackslash}m{1.5cm}|>{\centering\arraybackslash}m{1.3cm}|>{\centering\arraybackslash}m{2.0cm}|>{\raggedright\arraybackslash}m{4.5cm}|}
        \toprule
        
        \textbf{} & \textbf{Purpose} & \textbf{Study Dataset} & \textbf{Module Error} & \textbf{NameError / Cell Order} & \textbf{Input File Error} & \textbf{Non-Executable Found} & \textbf{Dataset Characteristics}\\ 
        \midrule
        
        RELANCER \cite{Zhu2021}                          & Executability        & 4,043             & \cmark    & \xmark    & \xmark    & 47\%      & Collected from Meta Kaggle \cite{kaggle}\\ 
        \hline
        SnifferDog\cite{Wang2021}                        & Executability        & 2,646             & \cmark    & \xmark    & \xmark    & 72.6\%    & Random sample from \cite{Pimentel2019} only those with installable dependency\\ 
        \hline
        Osiris\cite{Wang2020}                            & Reproducibility      & 5,393             & \xmark    & \cmark    & \xmark    & 82.6\%    & Random sample from \cite{Pimentel2019}\\ 
        \hline
        Pimentel et al.\cite{Pimentel2019, Pimentel2021} & Empirical            & $>$1.4M    & N/A       & N/A       & N/A       & 76\%      & Includes high-number of low-starred, rarely reused notebooks from GitHub\\ 
        \hline
        \textbf{This work}                               & Executability        & 42.5K              & \cmark    & \cmark    & \cmark    & \percentPathological & Highly-starred notebooks from GitHub\\ 
        \bottomrule
    \end{tabular}
    \label{related_work_table}
    % \vspace{-3ex}
\end{table*}

    In this section, we distill key findings in a {\em FAQ} format, demonstrating the value of accurate executability categorization, the utility of a notebook corpus once marked unusable, and practices that could prevent non-executability or restore executability in future notebooks. 
    
%\noindent{\em Why do notebook executability measurements in this work differ from prior investigations?} We identify two key factors that distinguish our findings on notebook executability from prior empirical investigations. First, our notebook corpus is primarily sourced from popular repositories—actively shared, reused, and maintained—whereas prior work focused more broadly, including notebooks that are less actively managed or of lower quality. While previous studies provide insights into a general range of public notebooks, our study offers new perspectives on those designed for sharing, adaptation, and reuse. Second, our dataset was collected in 2024, while the most recent prior investigation was conducted in 2021. Over the past three years, several tools aimed at improving notebook quality have emerged \cite{}\textcolor{red}{CITE}, likely contributing to the increase in executability.

\noindent{\em Why are \percentPathological notebooks pathologically non-executable and \percentPartiallyRestored of restorable notebooks still only partially executable?} 
    Prior work has identified that notebooks have alarmingly low numbers of test cases \cite{Pimentel2019}. Developers releasing their notebooks for public use have very limited testing tools to verify reproducibility and executability. For example, notebooks with undefined variables ``NameError" and ``AttributeError" often go unnoticed due to unintended dependencies on session states saved from prior cell execution, leading to those errors being masked on the developer's machine, similar to Case Study 2 in Section \ref{case_study_2}. In a new environment, such states are unavailable, resulting in these errors. 
    %\tien{This may be right, but notebooks originally have execution orders that they can use to execute, even though they may not be consistent. Pimentel \cite{Pimentel2019} and Osiris \cite{Wang2020} looked into this method.} 
    Our findings on pathological errors in notebooks demonstrate the need for static and dynamic analysis tools to catch such errors early in notebook development. 
    %\waris{btw VSCode Jupyter notebook extension addresses this issue.}


\noindent{\em How can LLMs further restore notebook executability with a high success rate?} 
    In the first case study (Section \ref{case_study_1}), Llama-3 successfully generated synthetic data for a `.csv' file. However, in another case \cite{tirthajyoti}, it failed to produce a valid `PNG' file due to limitations in multi-modal generation. This is because most LLMs have been trained on similar textual data formats. Therefore, Llama-3 also faces limitations when generating images, audio, video, or zip files. Multi-modal models such as GPT-4o can generate richer and more diverse inputs. %Our results with language-specific LLMs indicate that these models have the potential to generate synthetic data that promises high executability. 
    This capability can benefit developers who want to share their notebooks without disclosing the input data files. They can leverage LLMs to generate synthetic data (or scripts for synthetic data generation similar to database benchmarks \cite{tpcds}), enhancing executability in remote environments. Additionally, LLM performance tends to degrade when provided with large contexts, which aligns with recent research on LLMs \cite{Leng2024}. Moreover, notebook executability can be enhanced by generating feedback-driven fixes (e.g., multi-shot) for non-executable cells. This includes generating new input files based on error feedback.  
    %For example, if a cell error is caused by a missing file, an LLM could generate the necessary data or adjust paths to match the environment. 
    % Such a framework would greatly benefit notebook developers, enabling quick execution of public notebooks and addressing issues like missing files with minimal manual intervention.


%Moreover, we can further improve the execution of notebook cells by generating feedback-driven fixes for non-executable cells. For instance, we can achieve this by allowing code edits in notebook cells or generating newer input based on the error related to generating input files.  If we get an error in a cell, we can ask the LLM to generate newer file data if it's related to the generated file in the same context or if it requires a code edit (e.g., renaming the file path from a Windows machine path to a Linux machine path) to make it executable. Such a framework will be highly beneficial for notebook developers. It allows them to quickly run public notebooks and automatically address issues such as missing files, which would otherwise require manual effort.
%  Also, if an LLM generates six columns of a CSV correctly but the name or the data in the last column is not correct, we can ask the LLM to generate the last column data correctly. 



\noindent{\em What use is there for partially executable notebooks?}
    Prolonging notebook execution leads to more cells being executed successfully, which translates into more code surfaces that can be executed and understood. This inherently improves code reuse and enhances the chances of notebooks being fully restored. Furthermore, dynamic analysis techniques are limited to executable code only. By improving notebook executability, we increase the application surface of dynamic analysis tools such as dynamic taint analysis \cite{clause2007dytan}, runtime tracing \cite{meliou2011tracing}, automated debugging \cite{parnin2011automated}, symbolic execution \cite{baldoni2018survey}, and Osiris \cite{Wang2020} that can play vital roles in recovering notebook reproducibility. %Additionally, code corpora have proven extremely valuable in training new emerging LLMs. A big challenge is verifying the correctness of training data. More executable cells offer pathways to verified/tested code snippets that can be included in model training. 
    Partial executability metrics can also provide fine-grained measures of the effectiveness of reproducibility and execution-restoration tools. Additionally, it can serve as valuable feedback when applying code repair techniques in notebooks, such as automated cell reordering \cite{Wang2020}. 
    
    
\noindent {\em How do the findings of this study improve notebooks?} 
    We explored several design choices in building our measurement framework. We identify gaps in the notebook ecosystem for code review, debugging, test generation, and build tools. For existing public notebooks, which are growing exponentially \cite{Rule2018}, this measurement framework can be extended to improve fine-grained executability, feeding into dynamic analysis tools to enhance reproducibility. 
    
    Our findings, showing that only \totalRequirement (\percentRequirementInTotal) repositories have REQUIREMENTS files, indicate that there is no standardized build and continuous integration mechanism for notebooks. While this may not apply to exploratory, standalone, one-off notebooks, the most popular repositories provide a range of notebooks that must be properly orchestrated to ensure executability. We believe Python-based build tools, such as {\small{\texttt{disutils}}} \cite{distutil} and {\small{\texttt{setuptools}}} \cite{setuptools}, can be extended to support notebook extensions at the browser level to enforce correct build and dependency management for notebooks.


 \noindent {\em Threats to validity.}
    Outdated, textual, and empty notebooks can cause variations in the results of this study. To mitigate that, we exclude notebooks written in Python 2 since support for Python 2 was discontinued in 2020. We also filter empty or instructional notebooks from our dataset. Similarly, focusing purely on GitHub repositories can bias the results towards a specific class of notebooks. Upon investigation, we identified numerous HuggingFace and Kaggle notebooks that are also hosted on GitHub. While there is always room to increase the scale of the analysis, our emphasis on popular, actively reusable notebooks deprioritizes the need to expand the scale. Our static and dynamic analyses rely on the Python AST parser and Python interpreter to capture errors. These tools can potentially miss or misclassify errors, which can impact our categorization. 
        %We do not capture semantic errors, nor do we measure reproducibility. 
    Since most of our runtime errors are captured through dynamic error checking, an earlier fatal runtime error (that we cannot resolve) may hinder our ability to capture ``FileNotFound" or ``ModuleNotFound" errors. Lastly,  we use a timeout of 5 minutes to execute each notebook. 
        %Any notebook taking more than 5 minutes is considered unanalyzable; thus, we exclude it from our dataset. 
    This approach may exclude notebooks like the ones requiring expensive machine learning training. However, in our notebook corpus, only \percentTimeout of notebooks encountered timeout errors.

\vspace{-0.5ex}
\section{Related Work}
\label{related work}

    With the increasing popularity of notebooks \cite{Kluyver2016}, a growing body of research has focused on understanding the unique characteristics of notebooks \cite{Chattopadhyay2020, Rule2018, Rule2018CellFolding, In2024, gigascience2024} and their diverse applications in various fields \cite{Wang2019, Randles2017, Li2021}. Recent works have also proposed potential enhancements to these tools, aiming to improve notebooks' functionality and usability for data scientists and other users \cite{Chattopadhyay2020, Wang2024, McNutt2023, gigascience2024}. Table \ref{related_work_table} summarizes and compares this work with prior investigations. %We also explain the dataset's characteristics in each work to show the underlying reason for varying rates of non-executable notebooks reported.

    \subsection{Empirical Studies on Notebook Executability}
        Pimental et al. \cite{Pimentel2019, Pimentel2021} conducted an analysis of 1.4 million notebooks sourced from GitHub, aiming to examine their nature, quality, and reproducibility. They report notebook quality results, covering various characteristics and insights into notebook execution processes, as well as common issues encountered within these notebooks. Similarly, Wang et al. \cite{Wang2019BetterCode} assesses the quality of code present in notebooks, ultimately concluding that notebooks often exhibit suboptimal coding practices, thus highlighting the importance of enhancing notebook code quality. Both studies only view notebook executability as an atomic measure. In particular, they lack a fine-grained, deeper analysis of a large portion of notebooks that are (1) non-executable only on non-author environments due to misconfiguration, thus fully restorable, and (2) offer reusable, valuable code that is partially executable. 

    

    \subsection{Notebook Restoration and Reproducibility Tools}
        Numerous tools are proposed to restore notebook executability and reproducibility. For instance, RELANCER automatically updates deprecated APIs in non-executable notebooks by gathering APIs from GitHub and documentation \cite{Zhu2021}. Similarly, SnifferDog restores the execution environment by providing a collection of APIs from Python packages and libraries to update required packages in notebooks \cite{Wang2021}. These approaches complement our analysis. However, due to the highly complex and heavyweight nature of these techniques, they are infeasible for large-scale notebook executability analysis. 
        %Their use of machine learning is also likely to apply code rewrite that may change the notebook's semantics. 
        Osiris \cite{Wang2020} restores the reproducibility of only fully executable notebooks that also contain output cells. It addresses cell dependencies and generates possible execution orders, similar to our def-use set-based reordering. Osiris demonstrates that without executable notebooks, reproducibility can not feasibly be restored. Similar to our investigation, a transparent, fine-grained view of the state of notebook executability can facilitate further reproducibility studies or tool development.

    
    \subsection{Notebook Development Assistance Tools}
        Previous research has proposed improving software development practices with extended user interfaces and advanced programming assistant features. Fork It \cite{Weinman2021} is a forking and backtracking extension designed to investigate various alternatives and traverse different states within a single notebook. Similarly, numerous notebook tools \cite{Rule2018CellFolding, Li2024, Zhu2024Facilitating, Wang202Conflict, SuperNOVA} aim to streamline collaborative and exploratory experiments, often incorporating new visualization and conflict resolution for collaborative work in notebooks. To assist in code debugging and cleaning,  Robinson et al. \cite{Robinson2022} examined error identification strategies used for Python notebooks, while Head et al. \cite{Head2019} utilized program slicing to extract relevant code cells producing specific outputs.  
        %The most appealing feature of the notebooks is their simple and interactive execution environment that facilitates rapid experimental programming.       
        Adding new features and complicated user interfaces significantly intervenes with rapid, interactive programming. Most of these tools have seen resistance in adoption by both users and popular notebook platform providers. With the emergence of LLMs, recent work has explored using LLMs in assisting notebook development \cite{McNutt2023, Wang2024, Weber2024Computational, grotov2024untangling}. These results resonate with our findings.  

        
    


\section{Conclusion}
\label{sec:conclusion}

Computational notebooks, widely used for data science and ML/AI tasks, continually suffer from executability issues. Prior studies reporting the non-executability of notebooks rely on a rigid definition of executability, leading to over-estimating non-executable notebooks. For the first time, we introduce the notions of partial executability and pathological non-executability for notebooks, contextualizing executability to the notebook's interactive computing paradigm. Our investigation finds \totalPathological out of \totalNotebooksInDataset notebooks are pathologically non-executable, while \totalRestorable can be restored given suitable execution environments. We leverage LLM-based error-driven restoration techniques to fully restored \percentFullyRestored and partially restored \percentPartiallyRestored of previously non-executable notebooks. These results offer key evidence that notebooks can benefit from LLM-based restoration, and partial executable notebooks are still valuable for broader code reuse practices. 


\section*{Acknowledgement} We thank anonymous reviewers for providing valuable and constructive feedback to help improve the quality of this work. This work was supported in part by Amazon - Virginia Tech Initiative in Efficient and Robust Machine Learning, 4-VA, and the National Science Foundation award 2106420. We also thank the Advanced Research Computing Center at Virginia Tech for their support in building and evaluating this work.



\balance

% This must be in the first 5 lines to tell arXiv to use pdfLaTeX, which is strongly recommended.
\pdfoutput=1
% In particular, the hyperref package requires pdfLaTeX in order to break URLs across lines.

\documentclass[11pt]{article}

% Change "review" to "final" to generate the final (sometimes called camera-ready) version.
% Change to "preprint" to generate a non-anonymous version with page numbers.
\usepackage{acl}

% Standard package includes
\usepackage{times}
\usepackage{latexsym}

% Draw tables
\usepackage{booktabs}
\usepackage{multirow}
\usepackage{xcolor}
\usepackage{colortbl}
\usepackage{array} 
\usepackage{amsmath}

\newcolumntype{C}{>{\centering\arraybackslash}p{0.07\textwidth}}
% For proper rendering and hyphenation of words containing Latin characters (including in bib files)
\usepackage[T1]{fontenc}
% For Vietnamese characters
% \usepackage[T5]{fontenc}
% See https://www.latex-project.org/help/documentation/encguide.pdf for other character sets
% This assumes your files are encoded as UTF8
\usepackage[utf8]{inputenc}

% This is not strictly necessary, and may be commented out,
% but it will improve the layout of the manuscript,
% and will typically save some space.
\usepackage{microtype}
\DeclareMathOperator*{\argmax}{arg\,max}
% This is also not strictly necessary, and may be commented out.
% However, it will improve the aesthetics of text in
% the typewriter font.
\usepackage{inconsolata}

%Including images in your LaTeX document requires adding
%additional package(s)
\usepackage{graphicx}
% If the title and author information does not fit in the area allocated, uncomment the following
%
%\setlength\titlebox{<dim>}
%
% and set <dim> to something 5cm or larger.

\title{Wi-Chat: Large Language Model Powered Wi-Fi Sensing}

% Author information can be set in various styles:
% For several authors from the same institution:
% \author{Author 1 \and ... \and Author n \\
%         Address line \\ ... \\ Address line}
% if the names do not fit well on one line use
%         Author 1 \\ {\bf Author 2} \\ ... \\ {\bf Author n} \\
% For authors from different institutions:
% \author{Author 1 \\ Address line \\  ... \\ Address line
%         \And  ... \And
%         Author n \\ Address line \\ ... \\ Address line}
% To start a separate ``row'' of authors use \AND, as in
% \author{Author 1 \\ Address line \\  ... \\ Address line
%         \AND
%         Author 2 \\ Address line \\ ... \\ Address line \And
%         Author 3 \\ Address line \\ ... \\ Address line}

% \author{First Author \\
%   Affiliation / Address line 1 \\
%   Affiliation / Address line 2 \\
%   Affiliation / Address line 3 \\
%   \texttt{email@domain} \\\And
%   Second Author \\
%   Affiliation / Address line 1 \\
%   Affiliation / Address line 2 \\
%   Affiliation / Address line 3 \\
%   \texttt{email@domain} \\}
% \author{Haohan Yuan \qquad Haopeng Zhang\thanks{corresponding author} \\ 
%   ALOHA Lab, University of Hawaii at Manoa \\
%   % Affiliation / Address line 2 \\
%   % Affiliation / Address line 3 \\
%   \texttt{\{haohany,haopengz\}@hawaii.edu}}
  
\author{
{Haopeng Zhang$\dag$\thanks{These authors contributed equally to this work.}, Yili Ren$\ddagger$\footnotemark[1], Haohan Yuan$\dag$, Jingzhe Zhang$\ddagger$, Yitong Shen$\ddagger$} \\
ALOHA Lab, University of Hawaii at Manoa$\dag$, University of South Florida$\ddagger$ \\
\{haopengz, haohany\}@hawaii.edu\\
\{yiliren, jingzhe, shen202\}@usf.edu\\}



  
%\author{
%  \textbf{First Author\textsuperscript{1}},
%  \textbf{Second Author\textsuperscript{1,2}},
%  \textbf{Third T. Author\textsuperscript{1}},
%  \textbf{Fourth Author\textsuperscript{1}},
%\\
%  \textbf{Fifth Author\textsuperscript{1,2}},
%  \textbf{Sixth Author\textsuperscript{1}},
%  \textbf{Seventh Author\textsuperscript{1}},
%  \textbf{Eighth Author \textsuperscript{1,2,3,4}},
%\\
%  \textbf{Ninth Author\textsuperscript{1}},
%  \textbf{Tenth Author\textsuperscript{1}},
%  \textbf{Eleventh E. Author\textsuperscript{1,2,3,4,5}},
%  \textbf{Twelfth Author\textsuperscript{1}},
%\\
%  \textbf{Thirteenth Author\textsuperscript{3}},
%  \textbf{Fourteenth F. Author\textsuperscript{2,4}},
%  \textbf{Fifteenth Author\textsuperscript{1}},
%  \textbf{Sixteenth Author\textsuperscript{1}},
%\\
%  \textbf{Seventeenth S. Author\textsuperscript{4,5}},
%  \textbf{Eighteenth Author\textsuperscript{3,4}},
%  \textbf{Nineteenth N. Author\textsuperscript{2,5}},
%  \textbf{Twentieth Author\textsuperscript{1}}
%\\
%\\
%  \textsuperscript{1}Affiliation 1,
%  \textsuperscript{2}Affiliation 2,
%  \textsuperscript{3}Affiliation 3,
%  \textsuperscript{4}Affiliation 4,
%  \textsuperscript{5}Affiliation 5
%\\
%  \small{
%    \textbf{Correspondence:} \href{mailto:email@domain}{email@domain}
%  }
%}

\begin{document}
\maketitle
\begin{abstract}
Recent advancements in Large Language Models (LLMs) have demonstrated remarkable capabilities across diverse tasks. However, their potential to integrate physical model knowledge for real-world signal interpretation remains largely unexplored. In this work, we introduce Wi-Chat, the first LLM-powered Wi-Fi-based human activity recognition system. We demonstrate that LLMs can process raw Wi-Fi signals and infer human activities by incorporating Wi-Fi sensing principles into prompts. Our approach leverages physical model insights to guide LLMs in interpreting Channel State Information (CSI) data without traditional signal processing techniques. Through experiments on real-world Wi-Fi datasets, we show that LLMs exhibit strong reasoning capabilities, achieving zero-shot activity recognition. These findings highlight a new paradigm for Wi-Fi sensing, expanding LLM applications beyond conventional language tasks and enhancing the accessibility of wireless sensing for real-world deployments.
\end{abstract}

\section{Introduction}

In today’s rapidly evolving digital landscape, the transformative power of web technologies has redefined not only how services are delivered but also how complex tasks are approached. Web-based systems have become increasingly prevalent in risk control across various domains. This widespread adoption is due their accessibility, scalability, and ability to remotely connect various types of users. For example, these systems are used for process safety management in industry~\cite{kannan2016web}, safety risk early warning in urban construction~\cite{ding2013development}, and safe monitoring of infrastructural systems~\cite{repetto2018web}. Within these web-based risk management systems, the source search problem presents a huge challenge. Source search refers to the task of identifying the origin of a risky event, such as a gas leak and the emission point of toxic substances. This source search capability is crucial for effective risk management and decision-making.

Traditional approaches to implementing source search capabilities into the web systems often rely on solely algorithmic solutions~\cite{ristic2016study}. These methods, while relatively straightforward to implement, often struggle to achieve acceptable performances due to algorithmic local optima and complex unknown environments~\cite{zhao2020searching}. More recently, web crowdsourcing has emerged as a promising alternative for tackling the source search problem by incorporating human efforts in these web systems on-the-fly~\cite{zhao2024user}. This approach outsources the task of addressing issues encountered during the source search process to human workers, leveraging their capabilities to enhance system performance.

These solutions often employ a human-AI collaborative way~\cite{zhao2023leveraging} where algorithms handle exploration-exploitation and report the encountered problems while human workers resolve complex decision-making bottlenecks to help the algorithms getting rid of local deadlocks~\cite{zhao2022crowd}. Although effective, this paradigm suffers from two inherent limitations: increased operational costs from continuous human intervention, and slow response times of human workers due to sequential decision-making. These challenges motivate our investigation into developing autonomous systems that preserve human-like reasoning capabilities while reducing dependency on massive crowdsourced labor.

Furthermore, recent advancements in large language models (LLMs)~\cite{chang2024survey} and multi-modal LLMs (MLLMs)~\cite{huang2023chatgpt} have unveiled promising avenues for addressing these challenges. One clear opportunity involves the seamless integration of visual understanding and linguistic reasoning for robust decision-making in search tasks. However, whether large models-assisted source search is really effective and efficient for improving the current source search algorithms~\cite{ji2022source} remains unknown. \textit{To address the research gap, we are particularly interested in answering the following two research questions in this work:}

\textbf{\textit{RQ1: }}How can source search capabilities be integrated into web-based systems to support decision-making in time-sensitive risk management scenarios? 
% \sq{I mention ``time-sensitive'' here because I feel like we shall say something about the response time -- LLM has to be faster than humans}

\textbf{\textit{RQ2: }}How can MLLMs and LLMs enhance the effectiveness and efficiency of existing source search algorithms? 

% \textit{\textbf{RQ2:}} To what extent does the performance of large models-assisted search align with or approach the effectiveness of human-AI collaborative search? 

To answer the research questions, we propose a novel framework called Auto-\
S$^2$earch (\textbf{Auto}nomous \textbf{S}ource \textbf{Search}) and implement a prototype system that leverages advanced web technologies to simulate real-world conditions for zero-shot source search. Unlike traditional methods that rely on pre-defined heuristics or extensive human intervention, AutoS$^2$earch employs a carefully designed prompt that encapsulates human rationales, thereby guiding the MLLM to generate coherent and accurate scene descriptions from visual inputs about four directional choices. Based on these language-based descriptions, the LLM is enabled to determine the optimal directional choice through chain-of-thought (CoT) reasoning. Comprehensive empirical validation demonstrates that AutoS$^2$-\ 
earch achieves a success rate of 95–98\%, closely approaching the performance of human-AI collaborative search across 20 benchmark scenarios~\cite{zhao2023leveraging}. 

Our work indicates that the role of humans in future web crowdsourcing tasks may evolve from executors to validators or supervisors. Furthermore, incorporating explanations of LLM decisions into web-based system interfaces has the potential to help humans enhance task performance in risk control.






\section{Related Work}
\label{sec:relatedworks}

% \begin{table*}[t]
% \centering 
% \renewcommand\arraystretch{0.98}
% \fontsize{8}{10}\selectfont \setlength{\tabcolsep}{0.4em}
% \begin{tabular}{@{}lc|cc|cc|cc@{}}
% \toprule
% \textbf{Methods}           & \begin{tabular}[c]{@{}c@{}}\textbf{Training}\\ \textbf{Paradigm}\end{tabular} & \begin{tabular}[c]{@{}c@{}}\textbf{$\#$ PT Data}\\ \textbf{(Tokens)}\end{tabular} & \begin{tabular}[c]{@{}c@{}}\textbf{$\#$ IFT Data}\\ \textbf{(Samples)}\end{tabular} & \textbf{Code}  & \begin{tabular}[c]{@{}c@{}}\textbf{Natural}\\ \textbf{Language}\end{tabular} & \begin{tabular}[c]{@{}c@{}}\textbf{Action}\\ \textbf{Trajectories}\end{tabular} & \begin{tabular}[c]{@{}c@{}}\textbf{API}\\ \textbf{Documentation}\end{tabular}\\ \midrule 
% NexusRaven~\citep{srinivasan2023nexusraven} & IFT & - & - & \textcolor{green}{\CheckmarkBold} & \textcolor{green}{\CheckmarkBold} &\textcolor{red}{\XSolidBrush}&\textcolor{red}{\XSolidBrush}\\
% AgentInstruct~\citep{zeng2023agenttuning} & IFT & - & 2k & \textcolor{green}{\CheckmarkBold} & \textcolor{green}{\CheckmarkBold} &\textcolor{red}{\XSolidBrush}&\textcolor{red}{\XSolidBrush} \\
% AgentEvol~\citep{xi2024agentgym} & IFT & - & 14.5k & \textcolor{green}{\CheckmarkBold} & \textcolor{green}{\CheckmarkBold} &\textcolor{green}{\CheckmarkBold}&\textcolor{red}{\XSolidBrush} \\
% Gorilla~\citep{patil2023gorilla}& IFT & - & 16k & \textcolor{green}{\CheckmarkBold} & \textcolor{green}{\CheckmarkBold} &\textcolor{red}{\XSolidBrush}&\textcolor{green}{\CheckmarkBold}\\
% OpenFunctions-v2~\citep{patil2023gorilla} & IFT & - & 65k & \textcolor{green}{\CheckmarkBold} & \textcolor{green}{\CheckmarkBold} &\textcolor{red}{\XSolidBrush}&\textcolor{green}{\CheckmarkBold}\\
% LAM~\citep{zhang2024agentohana} & IFT & - & 42.6k & \textcolor{green}{\CheckmarkBold} & \textcolor{green}{\CheckmarkBold} &\textcolor{green}{\CheckmarkBold}&\textcolor{red}{\XSolidBrush} \\
% xLAM~\citep{liu2024apigen} & IFT & - & 60k & \textcolor{green}{\CheckmarkBold} & \textcolor{green}{\CheckmarkBold} &\textcolor{green}{\CheckmarkBold}&\textcolor{red}{\XSolidBrush} \\\midrule
% LEMUR~\citep{xu2024lemur} & PT & 90B & 300k & \textcolor{green}{\CheckmarkBold} & \textcolor{green}{\CheckmarkBold} &\textcolor{green}{\CheckmarkBold}&\textcolor{red}{\XSolidBrush}\\
% \rowcolor{teal!12} \method & PT & 103B & 95k & \textcolor{green}{\CheckmarkBold} & \textcolor{green}{\CheckmarkBold} & \textcolor{green}{\CheckmarkBold} & \textcolor{green}{\CheckmarkBold} \\
% \bottomrule
% \end{tabular}
% \caption{Summary of existing tuning- and pretraining-based LLM agents with their training sample sizes. "PT" and "IFT" denote "Pre-Training" and "Instruction Fine-Tuning", respectively. }
% \label{tab:related}
% \end{table*}

\begin{table*}[ht]
\begin{threeparttable}
\centering 
\renewcommand\arraystretch{0.98}
\fontsize{7}{9}\selectfont \setlength{\tabcolsep}{0.2em}
\begin{tabular}{@{}l|c|c|ccc|cc|cc|cccc@{}}
\toprule
\textbf{Methods} & \textbf{Datasets}           & \begin{tabular}[c]{@{}c@{}}\textbf{Training}\\ \textbf{Paradigm}\end{tabular} & \begin{tabular}[c]{@{}c@{}}\textbf{\# PT Data}\\ \textbf{(Tokens)}\end{tabular} & \begin{tabular}[c]{@{}c@{}}\textbf{\# IFT Data}\\ \textbf{(Samples)}\end{tabular} & \textbf{\# APIs} & \textbf{Code}  & \begin{tabular}[c]{@{}c@{}}\textbf{Nat.}\\ \textbf{Lang.}\end{tabular} & \begin{tabular}[c]{@{}c@{}}\textbf{Action}\\ \textbf{Traj.}\end{tabular} & \begin{tabular}[c]{@{}c@{}}\textbf{API}\\ \textbf{Doc.}\end{tabular} & \begin{tabular}[c]{@{}c@{}}\textbf{Func.}\\ \textbf{Call}\end{tabular} & \begin{tabular}[c]{@{}c@{}}\textbf{Multi.}\\ \textbf{Step}\end{tabular}  & \begin{tabular}[c]{@{}c@{}}\textbf{Plan}\\ \textbf{Refine}\end{tabular}  & \begin{tabular}[c]{@{}c@{}}\textbf{Multi.}\\ \textbf{Turn}\end{tabular}\\ \midrule 
\multicolumn{13}{l}{\emph{Instruction Finetuning-based LLM Agents for Intrinsic Reasoning}}  \\ \midrule
FireAct~\cite{chen2023fireact} & FireAct & IFT & - & 2.1K & 10 & \textcolor{red}{\XSolidBrush} &\textcolor{green}{\CheckmarkBold} &\textcolor{green}{\CheckmarkBold}  & \textcolor{red}{\XSolidBrush} &\textcolor{green}{\CheckmarkBold} & \textcolor{red}{\XSolidBrush} &\textcolor{green}{\CheckmarkBold} & \textcolor{red}{\XSolidBrush} \\
ToolAlpaca~\cite{tang2023toolalpaca} & ToolAlpaca & IFT & - & 4.0K & 400 & \textcolor{red}{\XSolidBrush} &\textcolor{green}{\CheckmarkBold} &\textcolor{green}{\CheckmarkBold} & \textcolor{red}{\XSolidBrush} &\textcolor{green}{\CheckmarkBold} & \textcolor{red}{\XSolidBrush}  &\textcolor{green}{\CheckmarkBold} & \textcolor{red}{\XSolidBrush}  \\
ToolLLaMA~\cite{qin2023toolllm} & ToolBench & IFT & - & 12.7K & 16,464 & \textcolor{red}{\XSolidBrush} &\textcolor{green}{\CheckmarkBold} &\textcolor{green}{\CheckmarkBold} &\textcolor{red}{\XSolidBrush} &\textcolor{green}{\CheckmarkBold}&\textcolor{green}{\CheckmarkBold}&\textcolor{green}{\CheckmarkBold} &\textcolor{green}{\CheckmarkBold}\\
AgentEvol~\citep{xi2024agentgym} & AgentTraj-L & IFT & - & 14.5K & 24 &\textcolor{red}{\XSolidBrush} & \textcolor{green}{\CheckmarkBold} &\textcolor{green}{\CheckmarkBold}&\textcolor{red}{\XSolidBrush} &\textcolor{green}{\CheckmarkBold}&\textcolor{red}{\XSolidBrush} &\textcolor{red}{\XSolidBrush} &\textcolor{green}{\CheckmarkBold}\\
Lumos~\cite{yin2024agent} & Lumos & IFT  & - & 20.0K & 16 &\textcolor{red}{\XSolidBrush} & \textcolor{green}{\CheckmarkBold} & \textcolor{green}{\CheckmarkBold} &\textcolor{red}{\XSolidBrush} & \textcolor{green}{\CheckmarkBold} & \textcolor{green}{\CheckmarkBold} &\textcolor{red}{\XSolidBrush} & \textcolor{green}{\CheckmarkBold}\\
Agent-FLAN~\cite{chen2024agent} & Agent-FLAN & IFT & - & 24.7K & 20 &\textcolor{red}{\XSolidBrush} & \textcolor{green}{\CheckmarkBold} & \textcolor{green}{\CheckmarkBold} &\textcolor{red}{\XSolidBrush} & \textcolor{green}{\CheckmarkBold}& \textcolor{green}{\CheckmarkBold}&\textcolor{red}{\XSolidBrush} & \textcolor{green}{\CheckmarkBold}\\
AgentTuning~\citep{zeng2023agenttuning} & AgentInstruct & IFT & - & 35.0K & - &\textcolor{red}{\XSolidBrush} & \textcolor{green}{\CheckmarkBold} & \textcolor{green}{\CheckmarkBold} &\textcolor{red}{\XSolidBrush} & \textcolor{green}{\CheckmarkBold} &\textcolor{red}{\XSolidBrush} &\textcolor{red}{\XSolidBrush} & \textcolor{green}{\CheckmarkBold}\\\midrule
\multicolumn{13}{l}{\emph{Instruction Finetuning-based LLM Agents for Function Calling}} \\\midrule
NexusRaven~\citep{srinivasan2023nexusraven} & NexusRaven & IFT & - & - & 116 & \textcolor{green}{\CheckmarkBold} & \textcolor{green}{\CheckmarkBold}  & \textcolor{green}{\CheckmarkBold} &\textcolor{red}{\XSolidBrush} & \textcolor{green}{\CheckmarkBold} &\textcolor{red}{\XSolidBrush} &\textcolor{red}{\XSolidBrush}&\textcolor{red}{\XSolidBrush}\\
Gorilla~\citep{patil2023gorilla} & Gorilla & IFT & - & 16.0K & 1,645 & \textcolor{green}{\CheckmarkBold} &\textcolor{red}{\XSolidBrush} &\textcolor{red}{\XSolidBrush}&\textcolor{green}{\CheckmarkBold} &\textcolor{green}{\CheckmarkBold} &\textcolor{red}{\XSolidBrush} &\textcolor{red}{\XSolidBrush} &\textcolor{red}{\XSolidBrush}\\
OpenFunctions-v2~\citep{patil2023gorilla} & OpenFunctions-v2 & IFT & - & 65.0K & - & \textcolor{green}{\CheckmarkBold} & \textcolor{green}{\CheckmarkBold} &\textcolor{red}{\XSolidBrush} &\textcolor{green}{\CheckmarkBold} &\textcolor{green}{\CheckmarkBold} &\textcolor{red}{\XSolidBrush} &\textcolor{red}{\XSolidBrush} &\textcolor{red}{\XSolidBrush}\\
API Pack~\cite{guo2024api} & API Pack & IFT & - & 1.1M & 11,213 &\textcolor{green}{\CheckmarkBold} &\textcolor{red}{\XSolidBrush} &\textcolor{green}{\CheckmarkBold} &\textcolor{red}{\XSolidBrush} &\textcolor{green}{\CheckmarkBold} &\textcolor{red}{\XSolidBrush}&\textcolor{red}{\XSolidBrush}&\textcolor{red}{\XSolidBrush}\\ 
LAM~\citep{zhang2024agentohana} & AgentOhana & IFT & - & 42.6K & - & \textcolor{green}{\CheckmarkBold} & \textcolor{green}{\CheckmarkBold} &\textcolor{green}{\CheckmarkBold}&\textcolor{red}{\XSolidBrush} &\textcolor{green}{\CheckmarkBold}&\textcolor{red}{\XSolidBrush}&\textcolor{green}{\CheckmarkBold}&\textcolor{green}{\CheckmarkBold}\\
xLAM~\citep{liu2024apigen} & APIGen & IFT & - & 60.0K & 3,673 & \textcolor{green}{\CheckmarkBold} & \textcolor{green}{\CheckmarkBold} &\textcolor{green}{\CheckmarkBold}&\textcolor{red}{\XSolidBrush} &\textcolor{green}{\CheckmarkBold}&\textcolor{red}{\XSolidBrush}&\textcolor{green}{\CheckmarkBold}&\textcolor{green}{\CheckmarkBold}\\\midrule
\multicolumn{13}{l}{\emph{Pretraining-based LLM Agents}}  \\\midrule
% LEMUR~\citep{xu2024lemur} & PT & 90B & 300.0K & - & \textcolor{green}{\CheckmarkBold} & \textcolor{green}{\CheckmarkBold} &\textcolor{green}{\CheckmarkBold}&\textcolor{red}{\XSolidBrush} & \textcolor{red}{\XSolidBrush} &\textcolor{green}{\CheckmarkBold} &\textcolor{red}{\XSolidBrush}&\textcolor{red}{\XSolidBrush}\\
\rowcolor{teal!12} \method & \dataset & PT & 103B & 95.0K  & 76,537  & \textcolor{green}{\CheckmarkBold} & \textcolor{green}{\CheckmarkBold} & \textcolor{green}{\CheckmarkBold} & \textcolor{green}{\CheckmarkBold} & \textcolor{green}{\CheckmarkBold} & \textcolor{green}{\CheckmarkBold} & \textcolor{green}{\CheckmarkBold} & \textcolor{green}{\CheckmarkBold}\\
\bottomrule
\end{tabular}
% \begin{tablenotes}
%     \item $^*$ In addition, the StarCoder-API can offer 4.77M more APIs.
% \end{tablenotes}
\caption{Summary of existing instruction finetuning-based LLM agents for intrinsic reasoning and function calling, along with their training resources and sample sizes. "PT" and "IFT" denote "Pre-Training" and "Instruction Fine-Tuning", respectively.}
\vspace{-2ex}
\label{tab:related}
\end{threeparttable}
\end{table*}

\noindent \textbf{Prompting-based LLM Agents.} Due to the lack of agent-specific pre-training corpus, existing LLM agents rely on either prompt engineering~\cite{hsieh2023tool,lu2024chameleon,yao2022react,wang2023voyager} or instruction fine-tuning~\cite{chen2023fireact,zeng2023agenttuning} to understand human instructions, decompose high-level tasks, generate grounded plans, and execute multi-step actions. 
However, prompting-based methods mainly depend on the capabilities of backbone LLMs (usually commercial LLMs), failing to introduce new knowledge and struggling to generalize to unseen tasks~\cite{sun2024adaplanner,zhuang2023toolchain}. 

\noindent \textbf{Instruction Finetuning-based LLM Agents.} Considering the extensive diversity of APIs and the complexity of multi-tool instructions, tool learning inherently presents greater challenges than natural language tasks, such as text generation~\cite{qin2023toolllm}.
Post-training techniques focus more on instruction following and aligning output with specific formats~\cite{patil2023gorilla,hao2024toolkengpt,qin2023toolllm,schick2024toolformer}, rather than fundamentally improving model knowledge or capabilities. 
Moreover, heavy fine-tuning can hinder generalization or even degrade performance in non-agent use cases, potentially suppressing the original base model capabilities~\cite{ghosh2024a}.

\noindent \textbf{Pretraining-based LLM Agents.} While pre-training serves as an essential alternative, prior works~\cite{nijkamp2023codegen,roziere2023code,xu2024lemur,patil2023gorilla} have primarily focused on improving task-specific capabilities (\eg, code generation) instead of general-domain LLM agents, due to single-source, uni-type, small-scale, and poor-quality pre-training data. 
Existing tool documentation data for agent training either lacks diverse real-world APIs~\cite{patil2023gorilla, tang2023toolalpaca} or is constrained to single-tool or single-round tool execution. 
Furthermore, trajectory data mostly imitate expert behavior or follow function-calling rules with inferior planning and reasoning, failing to fully elicit LLMs' capabilities and handle complex instructions~\cite{qin2023toolllm}. 
Given a wide range of candidate API functions, each comprising various function names and parameters available at every planning step, identifying globally optimal solutions and generalizing across tasks remains highly challenging.



\section{Preliminaries}
\label{Preliminaries}
\begin{figure*}[t]
    \centering
    \includegraphics[width=0.95\linewidth]{fig/HealthGPT_Framework.png}
    \caption{The \ourmethod{} architecture integrates hierarchical visual perception and H-LoRA, employing a task-specific hard router to select visual features and H-LoRA plugins, ultimately generating outputs with an autoregressive manner.}
    \label{fig:architecture}
\end{figure*}
\noindent\textbf{Large Vision-Language Models.} 
The input to a LVLM typically consists of an image $x^{\text{img}}$ and a discrete text sequence $x^{\text{txt}}$. The visual encoder $\mathcal{E}^{\text{img}}$ converts the input image $x^{\text{img}}$ into a sequence of visual tokens $\mathcal{V} = [v_i]_{i=1}^{N_v}$, while the text sequence $x^{\text{txt}}$ is mapped into a sequence of text tokens $\mathcal{T} = [t_i]_{i=1}^{N_t}$ using an embedding function $\mathcal{E}^{\text{txt}}$. The LLM $\mathcal{M_\text{LLM}}(\cdot|\theta)$ models the joint probability of the token sequence $\mathcal{U} = \{\mathcal{V},\mathcal{T}\}$, which is expressed as:
\begin{equation}
    P_\theta(R | \mathcal{U}) = \prod_{i=1}^{N_r} P_\theta(r_i | \{\mathcal{U}, r_{<i}\}),
\end{equation}
where $R = [r_i]_{i=1}^{N_r}$ is the text response sequence. The LVLM iteratively generates the next token $r_i$ based on $r_{<i}$. The optimization objective is to minimize the cross-entropy loss of the response $\mathcal{R}$.
% \begin{equation}
%     \mathcal{L}_{\text{VLM}} = \mathbb{E}_{R|\mathcal{U}}\left[-\log P_\theta(R | \mathcal{U})\right]
% \end{equation}
It is worth noting that most LVLMs adopt a design paradigm based on ViT, alignment adapters, and pre-trained LLMs\cite{liu2023llava,liu2024improved}, enabling quick adaptation to downstream tasks.


\noindent\textbf{VQGAN.}
VQGAN~\cite{esser2021taming} employs latent space compression and indexing mechanisms to effectively learn a complete discrete representation of images. VQGAN first maps the input image $x^{\text{img}}$ to a latent representation $z = \mathcal{E}(x)$ through a encoder $\mathcal{E}$. Then, the latent representation is quantized using a codebook $\mathcal{Z} = \{z_k\}_{k=1}^K$, generating a discrete index sequence $\mathcal{I} = [i_m]_{m=1}^N$, where $i_m \in \mathcal{Z}$ represents the quantized code index:
\begin{equation}
    \mathcal{I} = \text{Quantize}(z|\mathcal{Z}) = \arg\min_{z_k \in \mathcal{Z}} \| z - z_k \|_2.
\end{equation}
In our approach, the discrete index sequence $\mathcal{I}$ serves as a supervisory signal for the generation task, enabling the model to predict the index sequence $\hat{\mathcal{I}}$ from input conditions such as text or other modality signals.  
Finally, the predicted index sequence $\hat{\mathcal{I}}$ is upsampled by the VQGAN decoder $G$, generating the high-quality image $\hat{x}^\text{img} = G(\hat{\mathcal{I}})$.



\noindent\textbf{Low Rank Adaptation.} 
LoRA\cite{hu2021lora} effectively captures the characteristics of downstream tasks by introducing low-rank adapters. The core idea is to decompose the bypass weight matrix $\Delta W\in\mathbb{R}^{d^{\text{in}} \times d^{\text{out}}}$ into two low-rank matrices $ \{A \in \mathbb{R}^{d^{\text{in}} \times r}, B \in \mathbb{R}^{r \times d^{\text{out}}} \}$, where $ r \ll \min\{d^{\text{in}}, d^{\text{out}}\} $, significantly reducing learnable parameters. The output with the LoRA adapter for the input $x$ is then given by:
\begin{equation}
    h = x W_0 + \alpha x \Delta W/r = x W_0 + \alpha xAB/r,
\end{equation}
where matrix $ A $ is initialized with a Gaussian distribution, while the matrix $ B $ is initialized as a zero matrix. The scaling factor $ \alpha/r $ controls the impact of $ \Delta W $ on the model.

\section{HealthGPT}
\label{Method}


\subsection{Unified Autoregressive Generation.}  
% As shown in Figure~\ref{fig:architecture}, 
\ourmethod{} (Figure~\ref{fig:architecture}) utilizes a discrete token representation that covers both text and visual outputs, unifying visual comprehension and generation as an autoregressive task. 
For comprehension, $\mathcal{M}_\text{llm}$ receives the input joint sequence $\mathcal{U}$ and outputs a series of text token $\mathcal{R} = [r_1, r_2, \dots, r_{N_r}]$, where $r_i \in \mathcal{V}_{\text{txt}}$, and $\mathcal{V}_{\text{txt}}$ represents the LLM's vocabulary:
\begin{equation}
    P_\theta(\mathcal{R} \mid \mathcal{U}) = \prod_{i=1}^{N_r} P_\theta(r_i \mid \mathcal{U}, r_{<i}).
\end{equation}
For generation, $\mathcal{M}_\text{llm}$ first receives a special start token $\langle \text{START\_IMG} \rangle$, then generates a series of tokens corresponding to the VQGAN indices $\mathcal{I} = [i_1, i_2, \dots, i_{N_i}]$, where $i_j \in \mathcal{V}_{\text{vq}}$, and $\mathcal{V}_{\text{vq}}$ represents the index range of VQGAN. Upon completion of generation, the LLM outputs an end token $\langle \text{END\_IMG} \rangle$:
\begin{equation}
    P_\theta(\mathcal{I} \mid \mathcal{U}) = \prod_{j=1}^{N_i} P_\theta(i_j \mid \mathcal{U}, i_{<j}).
\end{equation}
Finally, the generated index sequence $\mathcal{I}$ is fed into the decoder $G$, which reconstructs the target image $\hat{x}^{\text{img}} = G(\mathcal{I})$.

\subsection{Hierarchical Visual Perception}  
Given the differences in visual perception between comprehension and generation tasks—where the former focuses on abstract semantics and the latter emphasizes complete semantics—we employ ViT to compress the image into discrete visual tokens at multiple hierarchical levels.
Specifically, the image is converted into a series of features $\{f_1, f_2, \dots, f_L\}$ as it passes through $L$ ViT blocks.

To address the needs of various tasks, the hidden states are divided into two types: (i) \textit{Concrete-grained features} $\mathcal{F}^{\text{Con}} = \{f_1, f_2, \dots, f_k\}, k < L$, derived from the shallower layers of ViT, containing sufficient global features, suitable for generation tasks; 
(ii) \textit{Abstract-grained features} $\mathcal{F}^{\text{Abs}} = \{f_{k+1}, f_{k+2}, \dots, f_L\}$, derived from the deeper layers of ViT, which contain abstract semantic information closer to the text space, suitable for comprehension tasks.

The task type $T$ (comprehension or generation) determines which set of features is selected as the input for the downstream large language model:
\begin{equation}
    \mathcal{F}^{\text{img}}_T =
    \begin{cases}
        \mathcal{F}^{\text{Con}}, & \text{if } T = \text{generation task} \\
        \mathcal{F}^{\text{Abs}}, & \text{if } T = \text{comprehension task}
    \end{cases}
\end{equation}
We integrate the image features $\mathcal{F}^{\text{img}}_T$ and text features $\mathcal{T}$ into a joint sequence through simple concatenation, which is then fed into the LLM $\mathcal{M}_{\text{llm}}$ for autoregressive generation.
% :
% \begin{equation}
%     \mathcal{R} = \mathcal{M}_{\text{llm}}(\mathcal{U}|\theta), \quad \mathcal{U} = [\mathcal{F}^{\text{img}}_T; \mathcal{T}]
% \end{equation}
\subsection{Heterogeneous Knowledge Adaptation}
We devise H-LoRA, which stores heterogeneous knowledge from comprehension and generation tasks in separate modules and dynamically routes to extract task-relevant knowledge from these modules. 
At the task level, for each task type $ T $, we dynamically assign a dedicated H-LoRA submodule $ \theta^T $, which is expressed as:
\begin{equation}
    \mathcal{R} = \mathcal{M}_\text{LLM}(\mathcal{U}|\theta, \theta^T), \quad \theta^T = \{A^T, B^T, \mathcal{R}^T_\text{outer}\}.
\end{equation}
At the feature level for a single task, H-LoRA integrates the idea of Mixture of Experts (MoE)~\cite{masoudnia2014mixture} and designs an efficient matrix merging and routing weight allocation mechanism, thus avoiding the significant computational delay introduced by matrix splitting in existing MoELoRA~\cite{luo2024moelora}. Specifically, we first merge the low-rank matrices (rank = r) of $ k $ LoRA experts into a unified matrix:
\begin{equation}
    \mathbf{A}^{\text{merged}}, \mathbf{B}^{\text{merged}} = \text{Concat}(\{A_i\}_1^k), \text{Concat}(\{B_i\}_1^k),
\end{equation}
where $ \mathbf{A}^{\text{merged}} \in \mathbb{R}^{d^\text{in} \times rk} $ and $ \mathbf{B}^{\text{merged}} \in \mathbb{R}^{rk \times d^\text{out}} $. The $k$-dimension routing layer generates expert weights $ \mathcal{W} \in \mathbb{R}^{\text{token\_num} \times k} $ based on the input hidden state $ x $, and these are expanded to $ \mathbb{R}^{\text{token\_num} \times rk} $ as follows:
\begin{equation}
    \mathcal{W}^\text{expanded} = \alpha k \mathcal{W} / r \otimes \mathbf{1}_r,
\end{equation}
where $ \otimes $ denotes the replication operation.
The overall output of H-LoRA is computed as:
\begin{equation}
    \mathcal{O}^\text{H-LoRA} = (x \mathbf{A}^{\text{merged}} \odot \mathcal{W}^\text{expanded}) \mathbf{B}^{\text{merged}},
\end{equation}
where $ \odot $ represents element-wise multiplication. Finally, the output of H-LoRA is added to the frozen pre-trained weights to produce the final output:
\begin{equation}
    \mathcal{O} = x W_0 + \mathcal{O}^\text{H-LoRA}.
\end{equation}
% In summary, H-LoRA is a task-based dynamic PEFT method that achieves high efficiency in single-task fine-tuning.

\subsection{Training Pipeline}

\begin{figure}[t]
    \centering
    \hspace{-4mm}
    \includegraphics[width=0.94\linewidth]{fig/data.pdf}
    \caption{Data statistics of \texttt{VL-Health}. }
    \label{fig:data}
\end{figure}
\noindent \textbf{1st Stage: Multi-modal Alignment.} 
In the first stage, we design separate visual adapters and H-LoRA submodules for medical unified tasks. For the medical comprehension task, we train abstract-grained visual adapters using high-quality image-text pairs to align visual embeddings with textual embeddings, thereby enabling the model to accurately describe medical visual content. During this process, the pre-trained LLM and its corresponding H-LoRA submodules remain frozen. In contrast, the medical generation task requires training concrete-grained adapters and H-LoRA submodules while keeping the LLM frozen. Meanwhile, we extend the textual vocabulary to include multimodal tokens, enabling the support of additional VQGAN vector quantization indices. The model trains on image-VQ pairs, endowing the pre-trained LLM with the capability for image reconstruction. This design ensures pixel-level consistency of pre- and post-LVLM. The processes establish the initial alignment between the LLM’s outputs and the visual inputs.

\noindent \textbf{2nd Stage: Heterogeneous H-LoRA Plugin Adaptation.}  
The submodules of H-LoRA share the word embedding layer and output head but may encounter issues such as bias and scale inconsistencies during training across different tasks. To ensure that the multiple H-LoRA plugins seamlessly interface with the LLMs and form a unified base, we fine-tune the word embedding layer and output head using a small amount of mixed data to maintain consistency in the model weights. Specifically, during this stage, all H-LoRA submodules for different tasks are kept frozen, with only the word embedding layer and output head being optimized. Through this stage, the model accumulates foundational knowledge for unified tasks by adapting H-LoRA plugins.

\begin{table*}[!t]
\centering
\caption{Comparison of \ourmethod{} with other LVLMs and unified multi-modal models on medical visual comprehension tasks. \textbf{Bold} and \underline{underlined} text indicates the best performance and second-best performance, respectively.}
\resizebox{\textwidth}{!}{
\begin{tabular}{c|lcc|cccccccc|c}
\toprule
\rowcolor[HTML]{E9F3FE} &  &  &  & \multicolumn{2}{c}{\textbf{VQA-RAD \textuparrow}} & \multicolumn{2}{c}{\textbf{SLAKE \textuparrow}} & \multicolumn{2}{c}{\textbf{PathVQA \textuparrow}} &  &  &  \\ 
\cline{5-10}
\rowcolor[HTML]{E9F3FE}\multirow{-2}{*}{\textbf{Type}} & \multirow{-2}{*}{\textbf{Model}} & \multirow{-2}{*}{\textbf{\# Params}} & \multirow{-2}{*}{\makecell{\textbf{Medical} \\ \textbf{LVLM}}} & \textbf{close} & \textbf{all} & \textbf{close} & \textbf{all} & \textbf{close} & \textbf{all} & \multirow{-2}{*}{\makecell{\textbf{MMMU} \\ \textbf{-Med}}\textuparrow} & \multirow{-2}{*}{\textbf{OMVQA}\textuparrow} & \multirow{-2}{*}{\textbf{Avg. \textuparrow}} \\ 
\midrule \midrule
\multirow{9}{*}{\textbf{Comp. Only}} 
& Med-Flamingo & 8.3B & \Large \ding{51} & 58.6 & 43.0 & 47.0 & 25.5 & 61.9 & 31.3 & 28.7 & 34.9 & 41.4 \\
& LLaVA-Med & 7B & \Large \ding{51} & 60.2 & 48.1 & 58.4 & 44.8 & 62.3 & 35.7 & 30.0 & 41.3 & 47.6 \\
& HuatuoGPT-Vision & 7B & \Large \ding{51} & 66.9 & 53.0 & 59.8 & 49.1 & 52.9 & 32.0 & 42.0 & 50.0 & 50.7 \\
& BLIP-2 & 6.7B & \Large \ding{55} & 43.4 & 36.8 & 41.6 & 35.3 & 48.5 & 28.8 & 27.3 & 26.9 & 36.1 \\
& LLaVA-v1.5 & 7B & \Large \ding{55} & 51.8 & 42.8 & 37.1 & 37.7 & 53.5 & 31.4 & 32.7 & 44.7 & 41.5 \\
& InstructBLIP & 7B & \Large \ding{55} & 61.0 & 44.8 & 66.8 & 43.3 & 56.0 & 32.3 & 25.3 & 29.0 & 44.8 \\
& Yi-VL & 6B & \Large \ding{55} & 52.6 & 42.1 & 52.4 & 38.4 & 54.9 & 30.9 & 38.0 & 50.2 & 44.9 \\
& InternVL2 & 8B & \Large \ding{55} & 64.9 & 49.0 & 66.6 & 50.1 & 60.0 & 31.9 & \underline{43.3} & 54.5 & 52.5\\
& Llama-3.2 & 11B & \Large \ding{55} & 68.9 & 45.5 & 72.4 & 52.1 & 62.8 & 33.6 & 39.3 & 63.2 & 54.7 \\
\midrule
\multirow{5}{*}{\textbf{Comp. \& Gen.}} 
& Show-o & 1.3B & \Large \ding{55} & 50.6 & 33.9 & 31.5 & 17.9 & 52.9 & 28.2 & 22.7 & 45.7 & 42.6 \\
& Unified-IO 2 & 7B & \Large \ding{55} & 46.2 & 32.6 & 35.9 & 21.9 & 52.5 & 27.0 & 25.3 & 33.0 & 33.8 \\
& Janus & 1.3B & \Large \ding{55} & 70.9 & 52.8 & 34.7 & 26.9 & 51.9 & 27.9 & 30.0 & 26.8 & 33.5 \\
& \cellcolor[HTML]{DAE0FB}HealthGPT-M3 & \cellcolor[HTML]{DAE0FB}3.8B & \cellcolor[HTML]{DAE0FB}\Large \ding{51} & \cellcolor[HTML]{DAE0FB}\underline{73.7} & \cellcolor[HTML]{DAE0FB}\underline{55.9} & \cellcolor[HTML]{DAE0FB}\underline{74.6} & \cellcolor[HTML]{DAE0FB}\underline{56.4} & \cellcolor[HTML]{DAE0FB}\underline{78.7} & \cellcolor[HTML]{DAE0FB}\underline{39.7} & \cellcolor[HTML]{DAE0FB}\underline{43.3} & \cellcolor[HTML]{DAE0FB}\underline{68.5} & \cellcolor[HTML]{DAE0FB}\underline{61.3} \\
& \cellcolor[HTML]{DAE0FB}HealthGPT-L14 & \cellcolor[HTML]{DAE0FB}14B & \cellcolor[HTML]{DAE0FB}\Large \ding{51} & \cellcolor[HTML]{DAE0FB}\textbf{77.7} & \cellcolor[HTML]{DAE0FB}\textbf{58.3} & \cellcolor[HTML]{DAE0FB}\textbf{76.4} & \cellcolor[HTML]{DAE0FB}\textbf{64.5} & \cellcolor[HTML]{DAE0FB}\textbf{85.9} & \cellcolor[HTML]{DAE0FB}\textbf{44.4} & \cellcolor[HTML]{DAE0FB}\textbf{49.2} & \cellcolor[HTML]{DAE0FB}\textbf{74.4} & \cellcolor[HTML]{DAE0FB}\textbf{66.4} \\
\bottomrule
\end{tabular}
}
\label{tab:results}
\end{table*}
\begin{table*}[ht]
    \centering
    \caption{The experimental results for the four modality conversion tasks.}
    \resizebox{\textwidth}{!}{
    \begin{tabular}{l|ccc|ccc|ccc|ccc}
        \toprule
        \rowcolor[HTML]{E9F3FE} & \multicolumn{3}{c}{\textbf{CT to MRI (Brain)}} & \multicolumn{3}{c}{\textbf{CT to MRI (Pelvis)}} & \multicolumn{3}{c}{\textbf{MRI to CT (Brain)}} & \multicolumn{3}{c}{\textbf{MRI to CT (Pelvis)}} \\
        \cline{2-13}
        \rowcolor[HTML]{E9F3FE}\multirow{-2}{*}{\textbf{Model}}& \textbf{SSIM $\uparrow$} & \textbf{PSNR $\uparrow$} & \textbf{MSE $\downarrow$} & \textbf{SSIM $\uparrow$} & \textbf{PSNR $\uparrow$} & \textbf{MSE $\downarrow$} & \textbf{SSIM $\uparrow$} & \textbf{PSNR $\uparrow$} & \textbf{MSE $\downarrow$} & \textbf{SSIM $\uparrow$} & \textbf{PSNR $\uparrow$} & \textbf{MSE $\downarrow$} \\
        \midrule \midrule
        pix2pix & 71.09 & 32.65 & 36.85 & 59.17 & 31.02 & 51.91 & 78.79 & 33.85 & 28.33 & 72.31 & 32.98 & 36.19 \\
        CycleGAN & 54.76 & 32.23 & 40.56 & 54.54 & 30.77 & 55.00 & 63.75 & 31.02 & 52.78 & 50.54 & 29.89 & 67.78 \\
        BBDM & {71.69} & {32.91} & {34.44} & 57.37 & 31.37 & 48.06 & \textbf{86.40} & 34.12 & 26.61 & {79.26} & 33.15 & 33.60 \\
        Vmanba & 69.54 & 32.67 & 36.42 & {63.01} & {31.47} & {46.99} & 79.63 & 34.12 & 26.49 & 77.45 & 33.53 & 31.85 \\
        DiffMa & 71.47 & 32.74 & 35.77 & 62.56 & 31.43 & 47.38 & 79.00 & {34.13} & {26.45} & 78.53 & {33.68} & {30.51} \\
        \rowcolor[HTML]{DAE0FB}HealthGPT-M3 & \underline{79.38} & \underline{33.03} & \underline{33.48} & \underline{71.81} & \underline{31.83} & \underline{43.45} & {85.06} & \textbf{34.40} & \textbf{25.49} & \underline{84.23} & \textbf{34.29} & \textbf{27.99} \\
        \rowcolor[HTML]{DAE0FB}HealthGPT-L14 & \textbf{79.73} & \textbf{33.10} & \textbf{32.96} & \textbf{71.92} & \textbf{31.87} & \textbf{43.09} & \underline{85.31} & \underline{34.29} & \underline{26.20} & \textbf{84.96} & \underline{34.14} & \underline{28.13} \\
        \bottomrule
    \end{tabular}
    }
    \label{tab:conversion}
\end{table*}

\noindent \textbf{3rd Stage: Visual Instruction Fine-Tuning.}  
In the third stage, we introduce additional task-specific data to further optimize the model and enhance its adaptability to downstream tasks such as medical visual comprehension (e.g., medical QA, medical dialogues, and report generation) or generation tasks (e.g., super-resolution, denoising, and modality conversion). Notably, by this stage, the word embedding layer and output head have been fine-tuned, only the H-LoRA modules and adapter modules need to be trained. This strategy significantly improves the model's adaptability and flexibility across different tasks.


\section{Experiment}
\label{s:experiment}

\subsection{Data Description}
We evaluate our method on FI~\cite{you2016building}, Twitter\_LDL~\cite{yang2017learning} and Artphoto~\cite{machajdik2010affective}.
FI is a public dataset built from Flickr and Instagram, with 23,308 images and eight emotion categories, namely \textit{amusement}, \textit{anger}, \textit{awe},  \textit{contentment}, \textit{disgust}, \textit{excitement},  \textit{fear}, and \textit{sadness}. 
% Since images in FI are all copyrighted by law, some images are corrupted now, so we remove these samples and retain 21,828 images.
% T4SA contains images from Twitter, which are classified into three categories: \textit{positive}, \textit{neutral}, and \textit{negative}. In this paper, we adopt the base version of B-T4SA, which contains 470,586 images and provides text descriptions of the corresponding tweets.
Twitter\_LDL contains 10,045 images from Twitter, with the same eight categories as the FI dataset.
% 。
For these two datasets, they are randomly split into 80\%
training and 20\% testing set.
Artphoto contains 806 artistic photos from the DeviantArt website, which we use to further evaluate the zero-shot capability of our model.
% on the small-scale dataset.
% We construct and publicly release the first image sentiment analysis dataset containing metadata.
% 。

% Based on these datasets, we are the first to construct and publicly release metadata-enhanced image sentiment analysis datasets. These datasets include scenes, tags, descriptions, and corresponding confidence scores, and are available at this link for future research purposes.


% 
\begin{table}[t]
\centering
% \begin{center}
\caption{Overall performance of different models on FI and Twitter\_LDL datasets.}
\label{tab:cap1}
% \resizebox{\linewidth}{!}
{
\begin{tabular}{l|c|c|c|c}
\hline
\multirow{2}{*}{\textbf{Model}} & \multicolumn{2}{c|}{\textbf{FI}}  & \multicolumn{2}{c}{\textbf{Twitter\_LDL}} \\ \cline{2-5} 
  & \textbf{Accuracy} & \textbf{F1} & \textbf{Accuracy} & \textbf{F1}  \\ \hline
% (\rownumber)~AlexNet~\cite{krizhevsky2017imagenet}  & 58.13\% & 56.35\%  & 56.24\%& 55.02\%  \\ 
% (\rownumber)~VGG16~\cite{simonyan2014very}  & 63.75\%& 63.08\%  & 59.34\%& 59.02\%  \\ 
(\rownumber)~ResNet101~\cite{he2016deep} & 66.16\%& 65.56\%  & 62.02\% & 61.34\%  \\ 
(\rownumber)~CDA~\cite{han2023boosting} & 66.71\%& 65.37\%  & 64.14\% & 62.85\%  \\ 
(\rownumber)~CECCN~\cite{ruan2024color} & 67.96\%& 66.74\%  & 64.59\%& 64.72\% \\ 
(\rownumber)~EmoVIT~\cite{xie2024emovit} & 68.09\%& 67.45\%  & 63.12\% & 61.97\%  \\ 
(\rownumber)~ComLDL~\cite{zhang2022compound} & 68.83\%& 67.28\%  & 65.29\% & 63.12\%  \\ 
(\rownumber)~WSDEN~\cite{li2023weakly} & 69.78\%& 69.61\%  & 67.04\% & 65.49\% \\ 
(\rownumber)~ECWA~\cite{deng2021emotion} & 70.87\%& 69.08\%  & 67.81\% & 66.87\%  \\ 
(\rownumber)~EECon~\cite{yang2023exploiting} & 71.13\%& 68.34\%  & 64.27\%& 63.16\%  \\ 
(\rownumber)~MAM~\cite{zhang2024affective} & 71.44\%  & 70.83\% & 67.18\%  & 65.01\%\\ 
(\rownumber)~TGCA-PVT~\cite{chen2024tgca}   & 73.05\%  & 71.46\% & 69.87\%  & 68.32\% \\ 
(\rownumber)~OEAN~\cite{zhang2024object}   & 73.40\%  & 72.63\% & 70.52\%  & 69.47\% \\ \hline
(\rownumber)~\shortname  & \textbf{79.48\%} & \textbf{79.22\%} & \textbf{74.12\%} & \textbf{73.09\%} \\ \hline
\end{tabular}
}
\vspace{-6mm}
% \end{center}
\end{table}
% 

\subsection{Experiment Setting}
% \subsubsection{Model Setting.}
% 
\textbf{Model Setting:}
For feature representation, we set $k=10$ to select object tags, and adopt clip-vit-base-patch32 as the pre-trained model for unified feature representation.
Moreover, we empirically set $(d_e, d_h, d_k, d_s) = (512, 128, 16, 64)$, and set the classification class $L$ to 8.

% 

\textbf{Training Setting:}
To initialize the model, we set all weights such as $\boldsymbol{W}$ following the truncated normal distribution, and use AdamW optimizer with the learning rate of $1 \times 10^{-4}$.
% warmup scheduler of cosine, warmup steps of 2000.
Furthermore, we set the batch size to 32 and the epoch of the training process to 200.
During the implementation, we utilize \textit{PyTorch} to build our entire model.
% , and our project codes are publicly available at https://github.com/zzmyrep/MESN.
% Our project codes as well as data are all publicly available on GitHub\footnote{https://github.com/zzmyrep/KBCEN}.
% Code is available at \href{https://github.com/zzmyrep/KBCEN}{https://github.com/zzmyrep/KBCEN}.

\textbf{Evaluation Metrics:}
Following~\cite{zhang2024affective, chen2024tgca, zhang2024object}, we adopt \textit{accuracy} and \textit{F1} as our evaluation metrics to measure the performance of different methods for image sentiment analysis. 



\subsection{Experiment Result}
% We compare our model against the following baselines: AlexNet~\cite{krizhevsky2017imagenet}, VGG16~\cite{simonyan2014very}, ResNet101~\cite{he2016deep}, CECCN~\cite{ruan2024color}, EmoVIT~\cite{xie2024emovit}, WSCNet~\cite{yang2018weakly}, ECWA~\cite{deng2021emotion}, EECon~\cite{yang2023exploiting}, MAM~\cite{zhang2024affective} and TGCA-PVT~\cite{chen2024tgca}, and the overall results are summarized in Table~\ref{tab:cap1}.
We compare our model against several baselines, and the overall results are summarized in Table~\ref{tab:cap1}.
We observe that our model achieves the best performance in both accuracy and F1 metrics, significantly outperforming the previous models. 
This superior performance is mainly attributed to our effective utilization of metadata to enhance image sentiment analysis, as well as the exceptional capability of the unified sentiment transformer framework we developed. These results strongly demonstrate that our proposed method can bring encouraging performance for image sentiment analysis.

\setcounter{magicrownumbers}{0} 
\begin{table}[t]
\begin{center}
\caption{Ablation study of~\shortname~on FI dataset.} 
% \vspace{1mm}
\label{tab:cap2}
\resizebox{.9\linewidth}{!}
{
\begin{tabular}{lcc}
  \hline
  \textbf{Model} & \textbf{Accuracy} & \textbf{F1} \\
  \hline
  (\rownumber)~Ours (w/o vision) & 65.72\% & 64.54\% \\
  (\rownumber)~Ours (w/o text description) & 74.05\% & 72.58\% \\
  (\rownumber)~Ours (w/o object tag) & 77.45\% & 76.84\% \\
  (\rownumber)~Ours (w/o scene tag) & 78.47\% & 78.21\% \\
  \hline
  (\rownumber)~Ours (w/o unified embedding) & 76.41\% & 76.23\% \\
  (\rownumber)~Ours (w/o adaptive learning) & 76.83\% & 76.56\% \\
  (\rownumber)~Ours (w/o cross-modal fusion) & 76.85\% & 76.49\% \\
  \hline
  (\rownumber)~Ours  & \textbf{79.48\%} & \textbf{79.22\%} \\
  \hline
\end{tabular}
}
\end{center}
\vspace{-5mm}
\end{table}


\begin{figure}[t]
\centering
% \vspace{-2mm}
\includegraphics[width=0.42\textwidth]{fig/2dvisual-linux4-paper2.pdf}
\caption{Visualization of feature distribution on eight categories before (left) and after (right) model processing.}
% 
\label{fig:visualization}
\vspace{-5mm}
\end{figure}

\subsection{Ablation Performance}
In this subsection, we conduct an ablation study to examine which component is really important for performance improvement. The results are reported in Table~\ref{tab:cap2}.

For information utilization, we observe a significant decline in model performance when visual features are removed. Additionally, the performance of \shortname~decreases when different metadata are removed separately, which means that text description, object tag, and scene tag are all critical for image sentiment analysis.
Recalling the model architecture, we separately remove transformer layers of the unified representation module, the adaptive learning module, and the cross-modal fusion module, replacing them with MLPs of the same parameter scale.
In this way, we can observe varying degrees of decline in model performance, indicating that these modules are indispensable for our model to achieve better performance.

\subsection{Visualization}
% 


% % 开始使用minipage进行左右排列
% \begin{minipage}[t]{0.45\textwidth}  % 子图1宽度为45%
%     \centering
%     \includegraphics[width=\textwidth]{2dvisual.pdf}  % 插入图片
%     \captionof{figure}{Visualization of feature distribution.}  % 使用captionof添加图片标题
%     \label{fig:visualization}
% \end{minipage}


% \begin{figure}[t]
% \centering
% \vspace{-2mm}
% \includegraphics[width=0.45\textwidth]{fig/2dvisual.pdf}
% \caption{Visualization of feature distribution.}
% \label{fig:visualization}
% % \vspace{-4mm}
% \end{figure}

% \begin{figure}[t]
% \centering
% \vspace{-2mm}
% \includegraphics[width=0.45\textwidth]{fig/2dvisual-linux3-paper.pdf}
% \caption{Visualization of feature distribution.}
% \label{fig:visualization}
% % \vspace{-4mm}
% \end{figure}



\begin{figure}[tbp]   
\vspace{-4mm}
  \centering            
  \subfloat[Depth of adaptive learning layers]   
  {
    \label{fig:subfig1}\includegraphics[width=0.22\textwidth]{fig/fig_sensitivity-a5}
  }
  \subfloat[Depth of fusion layers]
  {
    % \label{fig:subfig2}\includegraphics[width=0.22\textwidth]{fig/fig_sensitivity-b2}
    \label{fig:subfig2}\includegraphics[width=0.22\textwidth]{fig/fig_sensitivity-b2-num.pdf}
  }
  \caption{Sensitivity study of \shortname~on different depth. }   
  \label{fig:fig_sensitivity}  
\vspace{-2mm}
\end{figure}

% \begin{figure}[htbp]
% \centerline{\includegraphics{2dvisual.pdf}}
% \caption{Visualization of feature distribution.}
% \label{fig:visualization}
% \end{figure}

% In Fig.~\ref{fig:visualization}, we use t-SNE~\cite{van2008visualizing} to reduce the dimension of data features for visualization, Figure in left represents the metadata features before model processing, the features are obtained by embedding through the CLIP model, and figure in right shows the features of the data after model processing, it can be observed that after the model processing, the data with different label categories fall in different regions in the space, therefore, we can conclude that the Therefore, we can conclude that the model can effectively utilize the information contained in the metadata and use it to guide the model for classification.

In Fig.~\ref{fig:visualization}, we use t-SNE~\cite{van2008visualizing} to reduce the dimension of data features for visualization.
The left figure shows metadata features before being processed by our model (\textit{i.e.}, embedded by CLIP), while the right shows the distribution of features after being processed by our model.
We can observe that after the model processing, data with the same label are closer to each other, while others are farther away.
Therefore, it shows that the model can effectively utilize the information contained in the metadata and use it to guide the classification process.

\subsection{Sensitivity Analysis}
% 
In this subsection, we conduct a sensitivity analysis to figure out the effect of different depth settings of adaptive learning layers and fusion layers. 
% In this subsection, we conduct a sensitivity analysis to figure out the effect of different depth settings on the model. 
% Fig.~\ref{fig:fig_sensitivity} presents the effect of different depth settings of adaptive learning layers and fusion layers. 
Taking Fig.~\ref{fig:fig_sensitivity} (a) as an example, the model performance improves with increasing depth, reaching the best performance at a depth of 4.
% Taking Fig.~\ref{fig:fig_sensitivity} (a) as an example, the performance of \shortname~improves with the increase of depth at first, reaching the best performance at a depth of 4.
When the depth continues to increase, the accuracy decreases to varying degrees.
Similar results can be observed in Fig.~\ref{fig:fig_sensitivity} (b).
Therefore, we set their depths to 4 and 6 respectively to achieve the best results.

% Through our experiments, we can observe that the effect of modifying these hyperparameters on the results of the experiments is very weak, and the surface model is not sensitive to the hyperparameters.


\subsection{Zero-shot Capability}
% 

% (1)~GCH~\cite{2010Analyzing} & 21.78\% & (5)~RA-DLNet~\cite{2020A} & 34.01\% \\ \hline
% (2)~WSCNet~\cite{2019WSCNet}  & 30.25\% & (6)~CECCN~\cite{ruan2024color} & 43.83\% \\ \hline
% (3)~PCNN~\cite{2015Robust} & 31.68\%  & (7)~EmoVIT~\cite{xie2024emovit} & 44.90\% \\ \hline
% (4)~AR~\cite{2018Visual} & 32.67\% & (8)~Ours (Zero-shot) & 47.83\% \\ \hline


\begin{table}[t]
\centering
\caption{Zero-shot capability of \shortname.}
\label{tab:cap3}
\resizebox{1\linewidth}{!}
{
\begin{tabular}{lc|lc}
\hline
\textbf{Model} & \textbf{Accuracy} & \textbf{Model} & \textbf{Accuracy} \\ \hline
(1)~WSCNet~\cite{2019WSCNet}  & 30.25\% & (5)~MAM~\cite{zhang2024affective} & 39.56\%  \\ \hline
(2)~AR~\cite{2018Visual} & 32.67\% & (6)~CECCN~\cite{ruan2024color} & 43.83\% \\ \hline
(3)~RA-DLNet~\cite{2020A} & 34.01\%  & (7)~EmoVIT~\cite{xie2024emovit} & 44.90\% \\ \hline
(4)~CDA~\cite{han2023boosting} & 38.64\% & (8)~Ours (Zero-shot) & 47.83\% \\ \hline
\end{tabular}
}
\vspace{-5mm}
\end{table}

% We use the model trained on the FI dataset to test on the artphoto dataset to verify the model's generalization ability as well as robustness to other distributed datasets.
% We can observe that the MESN model shows strong competitiveness in terms of accuracy when compared to other trained models, which suggests that the model has a good generalization ability in the OOD task.

To validate the model's generalization ability and robustness to other distributed datasets, we directly test the model trained on the FI dataset, without training on Artphoto. 
% As observed in Table 3, compared to other models trained on Artphoto, we achieve highly competitive zero-shot performance, indicating that the model has good generalization ability in out-of-distribution tasks.
From Table~\ref{tab:cap3}, we can observe that compared with other models trained on Artphoto, we achieve competitive zero-shot performance, which shows that the model has good generalization ability in out-of-distribution tasks.


%%%%%%%%%%%%
%  E2E     %
%%%%%%%%%%%%


\section{Conclusion}
In this paper, we introduced Wi-Chat, the first LLM-powered Wi-Fi-based human activity recognition system that integrates the reasoning capabilities of large language models with the sensing potential of wireless signals. Our experimental results on a self-collected Wi-Fi CSI dataset demonstrate the promising potential of LLMs in enabling zero-shot Wi-Fi sensing. These findings suggest a new paradigm for human activity recognition that does not rely on extensive labeled data. We hope future research will build upon this direction, further exploring the applications of LLMs in signal processing domains such as IoT, mobile sensing, and radar-based systems.

\section*{Limitations}
While our work represents the first attempt to leverage LLMs for processing Wi-Fi signals, it is a preliminary study focused on a relatively simple task: Wi-Fi-based human activity recognition. This choice allows us to explore the feasibility of LLMs in wireless sensing but also comes with certain limitations.

Our approach primarily evaluates zero-shot performance, which, while promising, may still lag behind traditional supervised learning methods in highly complex or fine-grained recognition tasks. Besides, our study is limited to a controlled environment with a self-collected dataset, and the generalizability of LLMs to diverse real-world scenarios with varying Wi-Fi conditions, environmental interference, and device heterogeneity remains an open question.

Additionally, we have yet to explore the full potential of LLMs in more advanced Wi-Fi sensing applications, such as fine-grained gesture recognition, occupancy detection, and passive health monitoring. Future work should investigate the scalability of LLM-based approaches, their robustness to domain shifts, and their integration with multimodal sensing techniques in broader IoT applications.


% Bibliography entries for the entire Anthology, followed by custom entries
%\bibliography{anthology,custom}
% Custom bibliography entries only
\bibliography{main}
\newpage
\appendix

\section{Experiment prompts}
\label{sec:prompt}
The prompts used in the LLM experiments are shown in the following Table~\ref{tab:prompts}.

\definecolor{titlecolor}{rgb}{0.9, 0.5, 0.1}
\definecolor{anscolor}{rgb}{0.2, 0.5, 0.8}
\definecolor{labelcolor}{HTML}{48a07e}
\begin{table*}[h]
	\centering
	
 % \vspace{-0.2cm}
	
	\begin{center}
		\begin{tikzpicture}[
				chatbox_inner/.style={rectangle, rounded corners, opacity=0, text opacity=1, font=\sffamily\scriptsize, text width=5in, text height=9pt, inner xsep=6pt, inner ysep=6pt},
				chatbox_prompt_inner/.style={chatbox_inner, align=flush left, xshift=0pt, text height=11pt},
				chatbox_user_inner/.style={chatbox_inner, align=flush left, xshift=0pt},
				chatbox_gpt_inner/.style={chatbox_inner, align=flush left, xshift=0pt},
				chatbox/.style={chatbox_inner, draw=black!25, fill=gray!7, opacity=1, text opacity=0},
				chatbox_prompt/.style={chatbox, align=flush left, fill=gray!1.5, draw=black!30, text height=10pt},
				chatbox_user/.style={chatbox, align=flush left},
				chatbox_gpt/.style={chatbox, align=flush left},
				chatbox2/.style={chatbox_gpt, fill=green!25},
				chatbox3/.style={chatbox_gpt, fill=red!20, draw=black!20},
				chatbox4/.style={chatbox_gpt, fill=yellow!30},
				labelbox/.style={rectangle, rounded corners, draw=black!50, font=\sffamily\scriptsize\bfseries, fill=gray!5, inner sep=3pt},
			]
											
			\node[chatbox_user] (q1) {
				\textbf{System prompt}
				\newline
				\newline
				You are a helpful and precise assistant for segmenting and labeling sentences. We would like to request your help on curating a dataset for entity-level hallucination detection.
				\newline \newline
                We will give you a machine generated biography and a list of checked facts about the biography. Each fact consists of a sentence and a label (True/False). Please do the following process. First, breaking down the biography into words. Second, by referring to the provided list of facts, merging some broken down words in the previous step to form meaningful entities. For example, ``strategic thinking'' should be one entity instead of two. Third, according to the labels in the list of facts, labeling each entity as True or False. Specifically, for facts that share a similar sentence structure (\eg, \textit{``He was born on Mach 9, 1941.''} (\texttt{True}) and \textit{``He was born in Ramos Mejia.''} (\texttt{False})), please first assign labels to entities that differ across atomic facts. For example, first labeling ``Mach 9, 1941'' (\texttt{True}) and ``Ramos Mejia'' (\texttt{False}) in the above case. For those entities that are the same across atomic facts (\eg, ``was born'') or are neutral (\eg, ``he,'' ``in,'' and ``on''), please label them as \texttt{True}. For the cases that there is no atomic fact that shares the same sentence structure, please identify the most informative entities in the sentence and label them with the same label as the atomic fact while treating the rest of the entities as \texttt{True}. In the end, output the entities and labels in the following format:
                \begin{itemize}[nosep]
                    \item Entity 1 (Label 1)
                    \item Entity 2 (Label 2)
                    \item ...
                    \item Entity N (Label N)
                \end{itemize}
                % \newline \newline
                Here are two examples:
                \newline\newline
                \textbf{[Example 1]}
                \newline
                [The start of the biography]
                \newline
                \textcolor{titlecolor}{Marianne McAndrew is an American actress and singer, born on November 21, 1942, in Cleveland, Ohio. She began her acting career in the late 1960s, appearing in various television shows and films.}
                \newline
                [The end of the biography]
                \newline \newline
                [The start of the list of checked facts]
                \newline
                \textcolor{anscolor}{[Marianne McAndrew is an American. (False); Marianne McAndrew is an actress. (True); Marianne McAndrew is a singer. (False); Marianne McAndrew was born on November 21, 1942. (False); Marianne McAndrew was born in Cleveland, Ohio. (False); She began her acting career in the late 1960s. (True); She has appeared in various television shows. (True); She has appeared in various films. (True)]}
                \newline
                [The end of the list of checked facts]
                \newline \newline
                [The start of the ideal output]
                \newline
                \textcolor{labelcolor}{[Marianne McAndrew (True); is (True); an (True); American (False); actress (True); and (True); singer (False); , (True); born (True); on (True); November 21, 1942 (False); , (True); in (True); Cleveland, Ohio (False); . (True); She (True); began (True); her (True); acting career (True); in (True); the late 1960s (True); , (True); appearing (True); in (True); various (True); television shows (True); and (True); films (True); . (True)]}
                \newline
                [The end of the ideal output]
				\newline \newline
                \textbf{[Example 2]}
                \newline
                [The start of the biography]
                \newline
                \textcolor{titlecolor}{Doug Sheehan is an American actor who was born on April 27, 1949, in Santa Monica, California. He is best known for his roles in soap operas, including his portrayal of Joe Kelly on ``General Hospital'' and Ben Gibson on ``Knots Landing.''}
                \newline
                [The end of the biography]
                \newline \newline
                [The start of the list of checked facts]
                \newline
                \textcolor{anscolor}{[Doug Sheehan is an American. (True); Doug Sheehan is an actor. (True); Doug Sheehan was born on April 27, 1949. (True); Doug Sheehan was born in Santa Monica, California. (False); He is best known for his roles in soap operas. (True); He portrayed Joe Kelly. (True); Joe Kelly was in General Hospital. (True); General Hospital is a soap opera. (True); He portrayed Ben Gibson. (True); Ben Gibson was in Knots Landing. (True); Knots Landing is a soap opera. (True)]}
                \newline
                [The end of the list of checked facts]
                \newline \newline
                [The start of the ideal output]
                \newline
                \textcolor{labelcolor}{[Doug Sheehan (True); is (True); an (True); American (True); actor (True); who (True); was born (True); on (True); April 27, 1949 (True); in (True); Santa Monica, California (False); . (True); He (True); is (True); best known (True); for (True); his roles in soap operas (True); , (True); including (True); in (True); his portrayal (True); of (True); Joe Kelly (True); on (True); ``General Hospital'' (True); and (True); Ben Gibson (True); on (True); ``Knots Landing.'' (True)]}
                \newline
                [The end of the ideal output]
				\newline \newline
				\textbf{User prompt}
				\newline
				\newline
				[The start of the biography]
				\newline
				\textcolor{magenta}{\texttt{\{BIOGRAPHY\}}}
				\newline
				[The ebd of the biography]
				\newline \newline
				[The start of the list of checked facts]
				\newline
				\textcolor{magenta}{\texttt{\{LIST OF CHECKED FACTS\}}}
				\newline
				[The end of the list of checked facts]
			};
			\node[chatbox_user_inner] (q1_text) at (q1) {
				\textbf{System prompt}
				\newline
				\newline
				You are a helpful and precise assistant for segmenting and labeling sentences. We would like to request your help on curating a dataset for entity-level hallucination detection.
				\newline \newline
                We will give you a machine generated biography and a list of checked facts about the biography. Each fact consists of a sentence and a label (True/False). Please do the following process. First, breaking down the biography into words. Second, by referring to the provided list of facts, merging some broken down words in the previous step to form meaningful entities. For example, ``strategic thinking'' should be one entity instead of two. Third, according to the labels in the list of facts, labeling each entity as True or False. Specifically, for facts that share a similar sentence structure (\eg, \textit{``He was born on Mach 9, 1941.''} (\texttt{True}) and \textit{``He was born in Ramos Mejia.''} (\texttt{False})), please first assign labels to entities that differ across atomic facts. For example, first labeling ``Mach 9, 1941'' (\texttt{True}) and ``Ramos Mejia'' (\texttt{False}) in the above case. For those entities that are the same across atomic facts (\eg, ``was born'') or are neutral (\eg, ``he,'' ``in,'' and ``on''), please label them as \texttt{True}. For the cases that there is no atomic fact that shares the same sentence structure, please identify the most informative entities in the sentence and label them with the same label as the atomic fact while treating the rest of the entities as \texttt{True}. In the end, output the entities and labels in the following format:
                \begin{itemize}[nosep]
                    \item Entity 1 (Label 1)
                    \item Entity 2 (Label 2)
                    \item ...
                    \item Entity N (Label N)
                \end{itemize}
                % \newline \newline
                Here are two examples:
                \newline\newline
                \textbf{[Example 1]}
                \newline
                [The start of the biography]
                \newline
                \textcolor{titlecolor}{Marianne McAndrew is an American actress and singer, born on November 21, 1942, in Cleveland, Ohio. She began her acting career in the late 1960s, appearing in various television shows and films.}
                \newline
                [The end of the biography]
                \newline \newline
                [The start of the list of checked facts]
                \newline
                \textcolor{anscolor}{[Marianne McAndrew is an American. (False); Marianne McAndrew is an actress. (True); Marianne McAndrew is a singer. (False); Marianne McAndrew was born on November 21, 1942. (False); Marianne McAndrew was born in Cleveland, Ohio. (False); She began her acting career in the late 1960s. (True); She has appeared in various television shows. (True); She has appeared in various films. (True)]}
                \newline
                [The end of the list of checked facts]
                \newline \newline
                [The start of the ideal output]
                \newline
                \textcolor{labelcolor}{[Marianne McAndrew (True); is (True); an (True); American (False); actress (True); and (True); singer (False); , (True); born (True); on (True); November 21, 1942 (False); , (True); in (True); Cleveland, Ohio (False); . (True); She (True); began (True); her (True); acting career (True); in (True); the late 1960s (True); , (True); appearing (True); in (True); various (True); television shows (True); and (True); films (True); . (True)]}
                \newline
                [The end of the ideal output]
				\newline \newline
                \textbf{[Example 2]}
                \newline
                [The start of the biography]
                \newline
                \textcolor{titlecolor}{Doug Sheehan is an American actor who was born on April 27, 1949, in Santa Monica, California. He is best known for his roles in soap operas, including his portrayal of Joe Kelly on ``General Hospital'' and Ben Gibson on ``Knots Landing.''}
                \newline
                [The end of the biography]
                \newline \newline
                [The start of the list of checked facts]
                \newline
                \textcolor{anscolor}{[Doug Sheehan is an American. (True); Doug Sheehan is an actor. (True); Doug Sheehan was born on April 27, 1949. (True); Doug Sheehan was born in Santa Monica, California. (False); He is best known for his roles in soap operas. (True); He portrayed Joe Kelly. (True); Joe Kelly was in General Hospital. (True); General Hospital is a soap opera. (True); He portrayed Ben Gibson. (True); Ben Gibson was in Knots Landing. (True); Knots Landing is a soap opera. (True)]}
                \newline
                [The end of the list of checked facts]
                \newline \newline
                [The start of the ideal output]
                \newline
                \textcolor{labelcolor}{[Doug Sheehan (True); is (True); an (True); American (True); actor (True); who (True); was born (True); on (True); April 27, 1949 (True); in (True); Santa Monica, California (False); . (True); He (True); is (True); best known (True); for (True); his roles in soap operas (True); , (True); including (True); in (True); his portrayal (True); of (True); Joe Kelly (True); on (True); ``General Hospital'' (True); and (True); Ben Gibson (True); on (True); ``Knots Landing.'' (True)]}
                \newline
                [The end of the ideal output]
				\newline \newline
				\textbf{User prompt}
				\newline
				\newline
				[The start of the biography]
				\newline
				\textcolor{magenta}{\texttt{\{BIOGRAPHY\}}}
				\newline
				[The ebd of the biography]
				\newline \newline
				[The start of the list of checked facts]
				\newline
				\textcolor{magenta}{\texttt{\{LIST OF CHECKED FACTS\}}}
				\newline
				[The end of the list of checked facts]
			};
		\end{tikzpicture}
        \caption{GPT-4o prompt for labeling hallucinated entities.}\label{tb:gpt-4-prompt}
	\end{center}
\vspace{-0cm}
\end{table*}
% \section{Full Experiment Results}
% \begin{table*}[th]
    \centering
    \small
    \caption{Classification Results}
    \begin{tabular}{lcccc}
        \toprule
        \textbf{Method} & \textbf{Accuracy} & \textbf{Precision} & \textbf{Recall} & \textbf{F1-score} \\
        \midrule
        \multicolumn{5}{c}{\textbf{Zero Shot}} \\
                Zero-shot E-eyes & 0.26 & 0.26 & 0.27 & 0.26 \\
        Zero-shot CARM & 0.24 & 0.24 & 0.24 & 0.24 \\
                Zero-shot SVM & 0.27 & 0.28 & 0.28 & 0.27 \\
        Zero-shot CNN & 0.23 & 0.24 & 0.23 & 0.23 \\
        Zero-shot RNN & 0.26 & 0.26 & 0.26 & 0.26 \\
DeepSeek-0shot & 0.54 & 0.61 & 0.54 & 0.52 \\
DeepSeek-0shot-COT & 0.33 & 0.24 & 0.33 & 0.23 \\
DeepSeek-0shot-Knowledge & 0.45 & 0.46 & 0.45 & 0.44 \\
Gemma2-0shot & 0.35 & 0.22 & 0.38 & 0.27 \\
Gemma2-0shot-COT & 0.36 & 0.22 & 0.36 & 0.27 \\
Gemma2-0shot-Knowledge & 0.32 & 0.18 & 0.34 & 0.20 \\
GPT-4o-mini-0shot & 0.48 & 0.53 & 0.48 & 0.41 \\
GPT-4o-mini-0shot-COT & 0.33 & 0.50 & 0.33 & 0.38 \\
GPT-4o-mini-0shot-Knowledge & 0.49 & 0.31 & 0.49 & 0.36 \\
GPT-4o-0shot & 0.62 & 0.62 & 0.47 & 0.42 \\
GPT-4o-0shot-COT & 0.29 & 0.45 & 0.29 & 0.21 \\
GPT-4o-0shot-Knowledge & 0.44 & 0.52 & 0.44 & 0.39 \\
LLaMA-0shot & 0.32 & 0.25 & 0.32 & 0.24 \\
LLaMA-0shot-COT & 0.12 & 0.25 & 0.12 & 0.09 \\
LLaMA-0shot-Knowledge & 0.32 & 0.25 & 0.32 & 0.28 \\
Mistral-0shot & 0.19 & 0.23 & 0.19 & 0.10 \\
Mistral-0shot-Knowledge & 0.21 & 0.40 & 0.21 & 0.11 \\
        \midrule
        \multicolumn{5}{c}{\textbf{4 Shot}} \\
GPT-4o-mini-4shot & 0.58 & 0.59 & 0.58 & 0.53 \\
GPT-4o-mini-4shot-COT & 0.57 & 0.53 & 0.57 & 0.50 \\
GPT-4o-mini-4shot-Knowledge & 0.56 & 0.51 & 0.56 & 0.47 \\
GPT-4o-4shot & 0.77 & 0.84 & 0.77 & 0.73 \\
GPT-4o-4shot-COT & 0.63 & 0.76 & 0.63 & 0.53 \\
GPT-4o-4shot-Knowledge & 0.72 & 0.82 & 0.71 & 0.66 \\
LLaMA-4shot & 0.29 & 0.24 & 0.29 & 0.21 \\
LLaMA-4shot-COT & 0.20 & 0.30 & 0.20 & 0.13 \\
LLaMA-4shot-Knowledge & 0.15 & 0.23 & 0.13 & 0.13 \\
Mistral-4shot & 0.02 & 0.02 & 0.02 & 0.02 \\
Mistral-4shot-Knowledge & 0.21 & 0.27 & 0.21 & 0.20 \\
        \midrule
        
        \multicolumn{5}{c}{\textbf{Suprevised}} \\
        SVM & 0.94 & 0.92 & 0.91 & 0.91 \\
        CNN & 0.98 & 0.98 & 0.97 & 0.97 \\
        RNN & 0.99 & 0.99 & 0.99 & 0.99 \\
        % \midrule
        % \multicolumn{5}{c}{\textbf{Conventional Wi-Fi-based Human Activity Recognition Systems}} \\
        E-eyes & 1.00 & 1.00 & 1.00 & 1.00 \\
        CARM & 0.98 & 0.98 & 0.98 & 0.98 \\
\midrule
 \multicolumn{5}{c}{\textbf{Vision Models}} \\
           Zero-shot SVM & 0.26 & 0.25 & 0.25 & 0.25 \\
        Zero-shot CNN & 0.26 & 0.25 & 0.26 & 0.26 \\
        Zero-shot RNN & 0.28 & 0.28 & 0.29 & 0.28 \\
        SVM & 0.99 & 0.99 & 0.99 & 0.99 \\
        CNN & 0.98 & 0.99 & 0.98 & 0.98 \\
        RNN & 0.98 & 0.99 & 0.98 & 0.98 \\
GPT-4o-mini-Vision & 0.84 & 0.85 & 0.84 & 0.84 \\
GPT-4o-mini-Vision-COT & 0.90 & 0.91 & 0.90 & 0.90 \\
GPT-4o-Vision & 0.74 & 0.82 & 0.74 & 0.73 \\
GPT-4o-Vision-COT & 0.70 & 0.83 & 0.70 & 0.68 \\
LLaMA-Vision & 0.20 & 0.23 & 0.20 & 0.09 \\
LLaMA-Vision-Knowledge & 0.22 & 0.05 & 0.22 & 0.08 \\

        \bottomrule
    \end{tabular}
    \label{full}
\end{table*}




\end{document}

\end{document}
