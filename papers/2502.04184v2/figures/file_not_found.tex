\begin{figure}
    \centering
    \begin{tikzpicture}
        \begin{axis}[
            width=\columnwidth, % Fit to one column
            height=0.45\columnwidth, % Adjust height for proportion
            ybar,
            symbolic x coords={$>$0-10, 11-20, 21-30, 31-40, 41-50, 51-60, 61-70, 71-80, 81-90, 91-100},
            xtick=data,
            ylabel={\# of Notebooks},
            xlabel={Executable Cells (\%)},
            ymin=0,
            ymax=450, 
            ytick={0,100,200, 300, 400},
            % xtick={0, 10, 20, 30, 40, 50, 60, 70, 80, 90, 100},
            minor tick num=0, % Disable minor ticks
            nodes near coords,
            every node near coord/.append style={font=\scriptsize, yshift=-2pt},
            enlarge x limits=0.1,
            % bar width=0.4cm,
            bar width=0.4cm,
            label style={font=\small},
            xlabel style={yshift=2pt},
            xticklabel style={font=\scriptsize, rotate=45}, % Rotate only x-axis tick labels
            yticklabel style={font=\scriptsize}, 
        ]
        \addplot[
            fill=purple,
            draw=none
        ] table[x=Executable,y=Count,col sep=space]{filenotfound.csv};
        \end{axis}
    \end{tikzpicture}
    \caption{Improvement in executability with synthetic input. %The X-axis represents the percentage increase in the number of successfully executed cells. \waris{mention that this figure is related to restoreable-non-executalble-notebooks. Highlight the significance. For instance, in previous approach 115 notebooks which are complety restored were completely categorized  as non-executable.
    }
    \label{fig:llm-input-results}
    \vspace{-5ex}
\end{figure}
