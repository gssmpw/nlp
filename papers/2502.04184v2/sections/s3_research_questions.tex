\section{Research Questions}
\label{research questions}

Under the traditional definition of executability, a notebook is considered \textbf{executable} if it does not trigger any error throughout its complete top-down execution. A \textbf{non-executable} notebook, on the other hand, fails to execute due to an error or exception. In this paper, we relax this notion of executability by dividing it further into {\bf pathologically non-executable}, {\bf non-executable but restorable}, and {\bf executable}. When a notebook's executability is hindered by unresolvable errors, e.g., syntax or indentation errors, it is \textbf{pathologically non-executable}. If a notebook is executable in the original author's execution environment but fails to execute in another environment due to issues such as missing input files, execution orders, or execution environments, it is called a  {\bf non-executable but restorable} or {\bf misconfigured} notebook which is not executable but can be fully or partially restored. 

Similar to different notions of executability, we also introduce different degrees of executability. Instead of considering it as a binary metric, we measure executability on a continuous spectrum, defined as a ratio between the number of cells executed until the first error and the total number of cells in a notebook. 100\% executability refers to fully executable notebook, whereas $<$100\% refers to partially executable notebooks. A fully executable notebook does not guarantee reproducibility, i.e., a notebook must produce the same outputs as the original notebook. Reproducibility is beyond the scope of this work and is often not required for data-centric notebooks as they are expected to produce different outputs on different data. We explore the following research questions:

\begin{itemize}
    \item \textbf{RQ1:} What are the common causes of non-executability in notebooks?
    \item \textbf{RQ2:} How many non-executable notebooks are pathologically non-executable, and how many can be restored?
    \item \textbf{RQ3:} To what extent can pathologically non-executable notebooks be executed?
    \item \textbf{RQ4:} Can LLM-based restoration strategies enhance notebook executability?
\end{itemize}






