\begin{abstract}
Computational notebooks are the de facto platform for exploratory data science, offering an interactive programming environment where users can create, modify, and execute code cells in any sequence. However, this flexibility often introduces code quality issues, with prior studies showing that approximately 76\% of public notebooks are non-executable, raising significant concerns about reusability. We argue that the traditional notion of executability—requiring a notebook to run {\em fully} and {\em without error}—is overly rigid, misclassifying many notebooks and overestimating their non-executability.

This paper investigates pathological executability issues in public notebooks under varying {\em notions} and {\em degrees} of executability. Notebooks, by construction, are incrementally and interactively executed, where each cell execution advances logic toward the notebook's goal. Even partially improving executability can improve code comprehension and offer a pathway for dynamic analyses. With this insight, we first categorize notebooks into potentially restorable and pathological non-executable notebooks and then measure how removing misconfiguration and superficial execution issues in notebooks can improve their executability (i.e., additional cells executed without error). For instance, we use a Large Language Model (LLM) to generate synthetic input data to restore non-executable notebooks with ``FileNotFound" errors. In a dataset of \totalNotebooksInDataset popular public notebooks, containing \totalNonExecutable non-executable notebooks, only \percentPathological are truly pathologically non-executable. For restorable notebooks, LLM-based methods fully restore \percentFullyRestored of previously non-executable notebooks. Among the partially restored, it improves the notebooks' executability by \averagePercentModuleNotFoundRestoredIncrease and \avgIncreaseAfterFileFixed by installing the correct modules and generating synthetic data. These findings challenge prior assumptions, suggesting that notebooks have higher executability than previously reported, many of which offer valuable partial execution, and that their executability should be evaluated within the interactive notebook paradigm rather than through traditional software executability standards.
\end{abstract}
