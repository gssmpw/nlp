

\section{Conclusion}
\label{sec:conclusion}

Computational notebooks, widely used for data science and ML/AI tasks, continually suffer from executability issues. Prior studies reporting the non-executability of notebooks rely on a rigid definition of executability, leading to over-estimating non-executable notebooks. For the first time, we introduce the notions of partial executability and pathological non-executability for notebooks, contextualizing executability to the notebook's interactive computing paradigm. Our investigation finds \totalPathological out of \totalNotebooksInDataset notebooks are pathologically non-executable, while \totalRestorable can be restored given suitable execution environments. We leverage LLM-based error-driven restoration techniques to fully restored \percentFullyRestored and partially restored \percentPartiallyRestored of previously non-executable notebooks. These results offer key evidence that notebooks can benefit from LLM-based restoration, and partial executable notebooks are still valuable for broader code reuse practices. 


\section*{Acknowledgement} We thank anonymous reviewers for providing valuable and constructive feedback to help improve the quality of this work. This work was supported in part by Amazon - Virginia Tech Initiative in Efficient and Robust Machine Learning, 4-VA, and the National Science Foundation award 2106420. We also thank the Advanced Research Computing Center at Virginia Tech for their support in building and evaluating this work.


