\documentclass[preprint,3p,sort]{elsarticle}

%\journal{Journal of Computational Physics}

% Packages and macros go here
\usepackage{amsfonts}
\usepackage{amsthm}
\usepackage{graphicx}
\usepackage{epstopdf}
\usepackage{algorithmic}
\usepackage{amsmath}
\usepackage{amssymb}
\usepackage{hyperref}
\usepackage[T1]{fontenc}
\usepackage{soul,color}
\usepackage{cases}
%%
%  color for highlighting changes
%\definecolor{chcol}{rgb}{0.4,0.,0.9}
%\newcommand{\change}[1]{\textcolor{chcol}{#1}}
%%
%  convenience commands for analysis section
%
\newcommand{\dbtilde}[1]{\widehat{#1}}
\newcommand{\BCMat}{\mathsf{R}}

% Add a serial/Oxford comma by default.
\newcommand{\creflastconjunction}{, and~}

\newcommand{\w}{\vec{}}
\newcommand{\eul}{\mathcal{L}}


%%
%. theorem, lemma, remark etc. styles
%%
\newtheorem{theorem}{Theorem}[section]
\newtheorem{proposition}[theorem]{Proposition}
\newtheorem{lemma}[theorem]{Lemma}
\newtheorem{corollary}[theorem]{Corollary}
\newtheorem{definition}{Definition}[section]
%
\theoremstyle{remark}
\newtheorem{remark}[theorem]{Remark}
%
\theoremstyle{example}
\newtheorem{example}[theorem]{Example}
%
% used to mark the end of a remark
%\newcommand\xqed[1]{%
%  \leavevmode\unskip\penalty9999 \hbox{}\nobreak\hfill
%  \quad\hbox{#1}}
%\newcommand{\RemarkQED}{\xqed{$\triangle$}}

% Sets running headers as well as PDF title and authors

\DeclareMathOperator{\diag}{diag}

%%
%. equation numbering
\numberwithin{equation}{section}
%\usepackage{refcheck}   % check for unused labels and equations

%%
\begin{document}

%\end{document}

\begin{frontmatter}

%
% ORCIDs Jan:  0000-0002-7972-6183, Andrew: 0000-0002-5902-1522

\title{Open Boundary Conditions for Nonlinear Initial Boundary Value Problems}


% or include affiliations in footnotes:
\author[sweden,southafrica]{Jan Nordstr\"{o}m}%\corref{firstcorrespondingauthor}}
%\cortext[firstcorrespondingauthor]{Corresponding author}
%\ead{jan.nordstrom@liu.se}
%\author[sweden]{Andrew R.~Winters\corref{secondcorrespondingauthor}}
\cortext[secondcorrespondingauthor]{Corresponding author}
\ead{jan.nordstrom@liu.se}
\address[sweden]{Department of Mathematics, Applied Mathematics, Link\"{o}ping University, SE-581 83 Link\"{o}ping, Sweden}
\address[southafrica]{Department of Mathematics and Applied Mathematics, University of Johannesburg, P.O. Box 524, Auckland Park 2006, Johannesburg, South Africa}

%  abstract P.O. Box 524, Auckland Park 2006, Johannesburg, South Africa
\begin{abstract}
We present a straightforward energy stable weak implementation procedure of open boundary conditions for nonlinear initial boundary value problems.
It simplifies previous work  and its practical implementation.
 \end{abstract}

% keywords
\begin{keyword}
Nonlinear boundary conditions  \sep shallow water equations  \sep Euler equations \sep Navier-Stokes equations \sep  energy stability  \sep summation-by-parts
\end{keyword}

% REQUIRED
%\begin{AMS}
%   35L50, 35L60
%\end{AMS}

\end{frontmatter}

%\end{document}


%%%%%%%%%%%%%%%%%%%%%%%%%%%%%%%%%%%%%%%%%%%%%%%%%%%%%%%%%%%%%%%%%%%%%%%%
\section{Introduction}
%%%%%%%%%%%%%%%%%%%%%%%%%%%%%%%%%%%%%%%%%%%%%%%%%%%%%%%%%%%%%%%%%%%%%%%%
To bound nonlinear initial boundary value problems (IBVPs) with inflow-outflow boundary conditions involving non-zero data is maybe the most crucial task when aiming for  stability, see \cite{HEDSTROM1979222,Dubois198893,NYCANDER2008108,Nordstrom2013,svard2014,nordstrom2019,svard2021entropy,nordstrom2022linear} for previous efforts on this subject.
%The energy method \cite{kreiss1970,gustafsson1995time,nordstrom2020,nordstrom2005} applied to linear IBVPs provides $L_2 $ energy estimates 
%in an $L_2 $ equivalent norm 
%and well posed boundary conditions. For nonlinear problems \cite{nordstrom2019, nordstrom2020spatial, Lauren2021} it require certain symmetry properties of the matrices involved.  %limit the use of the energy method and occasionally
%This prompted interest in entropy and kinetic energy estimates. Entropy estimates occasionally provide $L_2 $ estimates \cite{harten1983,Tadmor2003,nordstrom2022linear}  and boundary conditions while kinetic energy estimates \cite{Jameson2008188,PIROZZOLI20107180} do not. In most papers boundary conditions of inflow-outflow with non-zero data are avoided.
In  \cite{NORDSTROM2024_BC}, new nonlinear boundary conditions and boundary procedures leading to energy bounds were presented. 
%The procedure was exemplified on  the shallow water (SW), Euler and Navier-Stokes equations \cite{nordstrom2022linear-nonlinear,Nordstrom2022_Skew_Euler,NORDSTROM2024_BC,nordstrom2024skewsymmetric_jcp}.
Following \cite{nordstrom_roadmap}, the continuous procedure  was mimicked (using summation-by-parts operators and weak numerical boundary conditions) leading to provable energy stability. However, both the boundary condition and weak implementation technique in \cite{NORDSTROM2024_BC} were described in a rather complicated way. In this note we simplify the formulation and present a more user friendly procedure. 
%We also show that  the proofs in  \cite{NORDSTROM2024_BC} still hold.

%The main theory is untouched by the simplifications and the previous proofs still hold.
 
%The rest of the note is organised as follows: In Section~\ref{sec:theory} summarize the main content of the previously published boundary procedure in \cite{NORDSTROM2024_BC}.  In Section~\ref{numerics} we show that the energy bounded continuous formulation leads to  nonlinear stability of an SBP-SAT based scheme, including non-zero boundary data. A summary is provided in Section~\ref{sec:conclusion}.

%%%%%%%%%%%%%%%%%%%%%%%%%%%%%%%%%%%%%%%%%%%%%%%%%%%%%%%%%%%%%%%%%%%%%%%%
\section{Short summary of  previous results}\label{sec:theory}
%%%%%%%%%%%%%%%%%%%%%%%%%%%%%%%%%%%%%%%%%%%%%%%%%%%%%%%%%%%%%%%%%%%%%%%%
Following \cite{NORDSTROM2024_BC} we consider the nonlinear IBVP posed on the domain $ \Omega$ with open boundary $\partial\Omega$
%We restrict the analysis to the hyperbolic part of IBVPs, where the nonlinearity normally resides. 
%The parabolic (viscous) part could be added on and properly posed provide dissipation or damping effects (although it can complicate the boundary treatment \cite{kreiss1989initial,Gustafsson1978,nordstrom2020,nordstrom2005,nordstrom2019,nordstrom2020spatial,Lauren2021}).
\begin{equation}\label{eq:nonlin}
P U_t + (A_i(U)U)_{x_i}+A_i(U)^TU_{x_i}+C(U)U= \epsilon (D_i(U))_{x_i},  \quad t \geq 0,  \quad  \vec x=(x_1,x_2,..,x_k) \in \Omega
\end{equation}
augmented with the initial condition $U(\vec x,0)=F(\vec x)$ in $\Omega$ and  the non-homogeneous boundary condition
\begin{equation}\label{eq:nonlin_BC}
B(U) U - G(\vec x,t)=0,  \quad t \geq 0,  \quad  \vec x=(x_1,x_2,..,x_k) \in  \partial\Omega.
\end{equation}
In (\ref{eq:nonlin_BC}), $B$ is the nonlinear boundary operator and $G$ the boundary data. In (\ref{eq:nonlin}), Einsteins summation convention is used and $P$ is a symmetric positive definite (or semi-definite) time-independent matrix that defines an energy norm (or semi-norm) $\|U\|^2_P= \int_{\Omega} U^T P U d\Omega$. 
%Note in particular that $P$  is not a function of the solution vector $U$. 
The $n \times n$ matrices $A_i,C$ and the boundary operator $B$  are smooth functions of the smooth $n$ component vector $U$. The viscous fluxes (or stresses) with first derivatives are included in  $D_i$ which satisfies $U_{x_i} ^T D_i  \geq 0$ while $\epsilon$ 
%(the inverse of the Reynolds number) 
is a non-negative constant parameter.
% measuring the influence of viscous forces. 
\begin{remark}
The formulation  (\ref{eq:nonlin}) can be derived for the shallow water, Euler and Navier-Stokes equations  \cite{nordstrom2022linear-nonlinear,Nordstrom2022_Skew_Euler,nordstrom2024skewsymmetric_jcp}. It enables the direct use of the Green-Gauss theorem leading to an energy-rate as shown in \cite{NORDSTROM2024_BC}. However, we stress that the new boundary treatment that we will present below can be applied to any nonlinear problem as long as it's energy rate only depends the nonlinear surface term.



 %\cite{Nordstrom2022_Skew_Euler,NORDSTROM2024_BC,nordstrom2024skewsymmetric_jcp}.
%the shallow water, Euler and Navier-Stokes equations \cite{nordstrom2022linear-nonlinear,Nordstrom2022_Skew_Euler,NORDSTROM2024_BC,nordstrom2024skewsymmetric_jcp}.
\end{remark}



%do not include specific details, except for the skew-symmetric form of the inviscid terms and the symmetric form of the viscous terms. 
%Formulation  (\ref{eq:nonlin} )includes the shallow water, Euler and Navier-Stokes equations \cite{nordstrom2022linear-nonlinear,Nordstrom2022_Skew_Euler,NORDSTROM2024_BC,nordstrom2024skewsymmetric_jcp}.
%\end{remark}
%%%%%%%%%%%%%%%%%%%%%%%%%%%%%%%%%%%%%%%%%%%%%%%%%%%%%%%%%%%%%%%%%%%%%%%%
\subsection{The energy rates}\label{sec:theory_inviscid}
% $\epsilon \rightarrow 0$
%%%%%%%%%%%%%%%%%%%%%%%%%%%%%%%%%%%%%%%%%%%%%%%%%%%%%%%%%%%%%%%%%%%%%%%%
For completeness and clarity we repeat the proof (from  \cite{NORDSTROM2024_BC}) of the following proposition.
\begin{proposition}\label{lemma:Matrixrelation_complemented}
The IBVP  (\ref{eq:nonlin}) with  $C+C^T = 0$ has an energy rate that only depends on (\ref{eq:nonlin_BC}). 
%The volume term in the energy rate is dissipative.
\end{proposition}
%\begin{proof}
%It follows directly by IBP, for details see \cite{nordstrom2022linear-nonlinear}.
%\end{proof}
\begin{proof}
The energy method applied to (\ref{eq:nonlin}) and Green-Gauss formula  leads to the following energy rate
\begin{equation}\label{eq:boundaryPart1}
\frac{1}{2} \frac{d}{dt}\|U\|^2_P +  \epsilon  \int\limits_{\Omega}U_{x_i} ^T D_i d \Omega + \oint\limits_{\partial\Omega}U^T  (n_i A_i)  \\\ U  -   \epsilon U^T (n_i D_i) \\\ ds= \int\limits_{\Omega}(U_{x_i}^T  A_i U - U^T A_i^T U_{x_i})-U^T  C U \\\ d \Omega,
%\int\limits_{\Omega}(U_{x_i}^T  A_i U - U^T A_i^T U_{x_i}) \\\ d \Omega -\int\limits_{\Omega} U^T  C U \\\ d \Omega,
\end{equation}
where $\vec n =(n_1,..,n_k)^T$ is the outward pointing unit normal. The first volume term on the right-hand side of (\ref{eq:boundaryPart1}) cancel by the skew-symmetric formulation and the second one by the condition $C+C^T = 0$.
%which proves the claim.
\end{proof}
\begin{remark}
Since the right-hand side of (\ref{eq:boundaryPart1}) vanish and $ U_{x_i} ^T D_i \geq 0$ provide dissipation, Proposition \ref{lemma:Matrixrelation_complemented} show that an energy-estimate is obtained if the surface terms are bounded by external data as shown in  \cite{NORDSTROM2024_BC}. 
\end{remark}
\begin{remark}
We stress again that the new boundary treatment that we will present in Section 3, is not specifically dependent on the formulation (\ref{eq:nonlin}). It can be applied to any nonlinear problem as long as one can derive an energy rate with the structure in (\ref{eq:boundaryPart1}), i.e. which depends only on a nonlinear surface term.
\end{remark}

%%%%%%%%%%%%%%%%%%%%%%%%%%%%%%%%%%%%%%%%%%%%%%%%%%%%%%%%%%%%%%%%%%%%%%%%
 \subsection{The nonlinear boundary conditions}\label{BC_theory} 
%%%%%%%%%%%%%%%%%%%%%%%%%%%%%%%%%%%%%%%%%%%%%%%%%%%%%%%%%%%%%%%%%%%%%%%%
We start with  the case $\epsilon =  0$ and consider the surface term $U^T  (n_i A_i) U = U^T  \tilde A  U $
%\begin{equation}\label{1Dprimalstab_trans_2}
%\oint\limits_{\partial\Omega}U^T  (n_i A_i)   \\\ U \\\ ds = \frac{1}{2}\oint\limits_{\partial\Omega} U^T ((n_i A_i)  +(n_i A_i )^T) U \\\ ds = \oint\limits_{\partial\Omega}U^T   \tilde A   \\\ U \\\ ds,
%\end{equation}
where $\tilde A$ is symmetric (the skew-symmetric part vanish). 
%Recall that if $V=U$ we are dealing with a nonlinear problem, otherwise a variable coefficient problem. 
%In the CFD problems we consider, the Cartesian velocity
%field is transformed to the normal and tangential ones leading to the new vectors $U_n=NU$. 
Next, we transform $\tilde A$ to diagonal form as $ T^T \tilde A  T =  \Lambda = diag( \lambda_i)$ giving
%new transformed variables $W = T^{-1} U$ 
%$W = (N  \tilde T)^{-1}U=T^{-1} U$ 
%and
%The final boundary term becomes
\begin{equation}\label{1Dprimalstab_trans_final}
U^T   \tilde A  U = W^T   \Lambda W  = (W^+)^T   \Lambda^+  W^+ + (W^-)^T   \Lambda^- W^-  \,\ \text{where} \,\ W(U) = T^{-1} U.
\end{equation}
%\begin{equation}\label{1Dprimalstab_trans_final}
%\oint\limits_{\partial\Omega}U^T   \tilde A   \\\ U \\\ ds  \\\ ds = \oint\limits_{\partial\Omega}W^T   \Lambda   \\\ W\\\ ds = \oint\limits_{\partial\Omega}(W^+)^T   \Lambda^+   \\\ W^+ + (W^-)^T   \Lambda^-   \\\ W^-\\\ ds=0.
%\oint\limits_{\partial\Omega}(W^+)^T   \Lambda^+   \\\ W^+ + (W^-)^T   \Lambda^-   \\\ W^-\\\ ds.
%\oint\limits_{\partial\Omega}  \lambda_i W_i^2 \\\ ds.
%\end{equation}
The matrix $T$ could be the standard eigenvector matrix  {\it or} another non-singular matrix that transform $\tilde A$ to diagonal form. In the latter case, Sylvester's Criterion \cite{horn2012} guarantee that the number of positive and negative diagonal entries are the same as the number of positive and negative eigenvalues.
% i.e. the same as the number of positive and negative eigenvalues. 
In the eigenvalue case $T^{-1}=T^T$, but not so in the  general  case.
%In (\ref{1Dprimalstab_trans_final}),  
$\Lambda^+ $ and  $\Lambda^-$  are the positive and negative parts of $\Lambda$ while $W^+$  and $W^-$ are the corresponding variables. Possible zero diagonal entries are included in $\Lambda^+$. 

Note that the diagonal matrix  $\Lambda=\Lambda(U)$ is solution dependent and not a priori bounded (as for linear IBVPs). For linear problems, the number of boundary conditions at a surface position is equal to the number of eigenvalues (or diagonal entries) of $\tilde A$ with the wrong (in this case negative) sign  \cite {nordstrom2020}. 
%Sylvester's Criterion \cite{horn2012}, shows that the number of boundary conditions is equal to the number of $\lambda_i(V)$  with the wrong sign if the rotation matrix  $T$ is non-singular.  
In the nonlinear case, it is more complicated (and still not known) since multiple forms of $W^T   \Lambda W$ may exist \cite{nordstrom2022linear,nordstrom2022linear-nonlinear,Nordstrom2022_Skew_Euler}.
%,  see Section \ref{Eulerex} below and \cite{nordstrom2022linear-nonlinear,Nordstrom2022_Skew_Euler} for examples. 
\begin{remark}\label{Sylvester}
The nonlinear boundary procedure based on (\ref{1Dprimalstab_trans_final}) has similarities with the linear characteristic boundary procedure \cite{kreiss1970,MR436612}. Hence, with a slight abuse of notation we will refer to $\Lambda(U)$ as eigenvalues and to the variables $W(U)$ as characteristic variables, even in the general transformation case.
% even though they play a similar role.
\end{remark}

\section{Previous and updated new general formulation of nonlinear boundary conditions}\label{nonlinear_BC_char}
%\subsection{Linear boundary conditions and estimates}\label{linear_BC}
The starting point for the derivation of nonlinear
 %(and linear)  
 boundary conditions  
 % (\ref{eq:nonlin_BC}) 
 is the diagonal form (\ref{1Dprimalstab_trans_final}) of the boundary term. First we find the formulation (\ref{1Dprimalstab_trans_final}) with a {\it minimal} number of entries in $\Lambda^-$ (as mentioned above, there might be more than one formulation). Next, we specify the transformed characteristic variables $W^-$ in terms of $W^+$ and external data and
% need a way to combine different ingoing characteristic variables $W^-$ and 
 add a scaling possibility. The previous formulation in  \cite{NORDSTROM2024_BC} read
\begin{equation}\label{Gen_BC_form}
 BU-G=S^{-1}(\sqrt{|\Lambda^-|}W^--R \sqrt{\Lambda^+}W^+) - G =0 \,\ \text{or equivalently} \,\  \sqrt{|\Lambda^-|}W^- - R \sqrt{\Lambda^+}W^+ - SG =0
\end{equation}
where $R$ is a matrix combining values of $W^-$ and $W^+$,  $S$ an non-singular scaling matrix and $G$ is the external data. 
%We have used the notation $ |\Lambda|=diag( |\lambda_i|)$ and $ \sqrt{|\Lambda|} =diag(  \sqrt{|\lambda_i|})$.  
The boundary condition (\ref{Gen_BC_form}) where we used the notation $ |\Lambda|=diag( |\lambda_i|)$ and $ \sqrt{|\Lambda|} =diag(  \sqrt{|\lambda_i|})$ is general in the sense that it  can include all components of $W$ suitably combined by the matrices $S$ and $R$. 

We will impose the boundary conditions both strongly and weakly. For the weak imposition we introduce a lifting operator $L_C$
 that enforce the boundary conditions weakly in our governing equation (\ref{eq:nonlin}) 
 %for $\epsilon  \rightarrow  0$ 
 as 
%follows
\begin{equation}\label{eq:nonlin_lif}
P U_t + (A_i U)_{x_i}+A^T_i U_{x_i}+C U+L_C(\tilde \Sigma(BU-G))=\epsilon (D_i)_{x_i}.
%,  \quad t \geq 0,  \quad  \vec x=(x_1,x_2,..,x_k) \in \Omega.
\end{equation}
%Similar to the scalar case, 
The lifting operator exist only at the surface of the domain and for two smooth vector functions  $\phi, \psi$ it satisfies $\int\limits \phi^T   L_C(\psi) d \Omega = \oint\limits \phi^T  \psi ds$
% $\Phi, \Psi$ satisfies $\int\limits_{\Omega} \Phi^T   L_C(\Psi) d \Omega = \oint\limits_{\partial\Omega} \Phi^T  \Psi ds$
%\begin{equation}\label{def_lifting}
%\int\limits_{\Omega} \Phi^T   L_C(\Psi) \\\ d \Omega  \\\  = \oint\limits_{\partial\Omega} \Phi^T  \Psi \\\ ds
%\end{equation}
and enables development  of the numerical boundary procedure in the continuous setting \cite{Arnold20011749,nordstrom_roadmap}.
%Note that the boundary condition (\ref{Gen_BC_form}) including the external data $G$ is nonlinear.
%Also, we observe that the data $G$ must  represent some nonlinear interaction since the boundary terms coming from the equations after applying the energy method are cubic, not quadratic {(as discussed in Section \ref{sec:theory_il} for the scalar case).  
The previous weak implementation of (\ref{Gen_BC_form}) using a lifting operator was given by
\begin{equation}\label{Pen_term_Gen_BC_form}
L_C(\tilde \Sigma(BU-G))= L_C(2( I^-T^{-1})^T \Sigma ( \sqrt{|\Lambda^-|}W^--R \sqrt{\Lambda^+}W^+ - SG )),
\end{equation}
where
%$W=T^{-1}U$, $W^-=I^-W$, $W^+=I^+W$, $I^-+I^+=I$ and 
$\tilde \Sigma = 2( I^-T^{-1})^T \Sigma S$ is a penalty matrix parametrized by the matrix coefficient $S,\Sigma$ and $I^-W=W^-$.
%The matrices $J^-$ and $J^+$ picks out the negative and positive characteristic variables from  $W$. 
% After the derivation of the stability conditions we will return to the boundary condition formulation (\ref{eq:nonlin_BC}) in the original variables.
%\begin{remark}
%To impose nonlinear boundary conditions, we must allow for some nonlinear scaling or interaction since the boundary terms coming from the equations are cubic, not quadratic.
%The lifting operator $L_C$ in (\ref{Pen_term_Gen_BC_form}) is a penalty term that forces $U$ to satisfy the boundary condition.
%\end{remark}
%The previous procedure in \cite{NORDSTROM2024_BC} for a stable nonlinear inhomogeneous boundary condition included the following steps for the determination of the unknowns $R,S,\Sigma$ in (\ref{Gen_BC_form}) and (\ref{Pen_term_Gen_BC_form}).
In \cite{NORDSTROM2024_BC}, the following steps were required in order to determine the unknowns $R,S,\Sigma$ in (\ref{Gen_BC_form}) and (\ref{Pen_term_Gen_BC_form}). \begin{enumerate}

\item Boundedness for strong homogeneous ($G=0$) boundary conditions lead to conditions on $R$.

\item Boundedness for strong inhomogeneous ($G\neq 0$) boundary conditions lead to conditions on $S$.

\item Boundedness for weak homogeneous ($G=0$) boundary conditions lead to conditions on $\Sigma$.

\item Boundedness for weak inhomogeneous ($G\neq 0$) boundary conditions followed  from conditions 1-3.
%The previous conditions on $R, S, \Sigma$ suffice
\end{enumerate}

We will simplify the previous formulation (\ref{Pen_term_Gen_BC_form}), and start by rewriting the boundary term in (\ref{1Dprimalstab_trans_final}) as
\begin{equation}\label{1Dprimalstab_trans_final_simp}
U^T   \tilde A   \\\ U = (A^+U)^T  (A^+ U) - (A^-U)^T  (A^-U)
\end{equation}
where  $A^+=\sqrt{\Lambda^+} T^{-1}$ and $A^-=\sqrt{|\Lambda^-|} T^{-1}$. This reformulation simplifies the boundary condition (\ref{Gen_BC_form}) to
\begin{equation}\label{Gen_BC_form_simp}
 BU-G=S^{-1}(A^- - R A^+)U - G =0 \,\ \text{or equivalently} \,\  (A^--R A^+) U - SG =0
\end{equation}
Note that the new boundary condition (\ref{Gen_BC_form_simp}) is identical to the old one in (\ref{Gen_BC_form}), with the boundary operator $B$  reformulated.  Also the lifting operator in  (\ref{Pen_term_Gen_BC_form}) simplifies by using $\tilde \Sigma=2 (A^-)^T S$ to yield
\begin{equation}\label{Pen_term_Gen_BC_form_simp}
L_C(\tilde \Sigma(BU-G))= L_C(2(A^-)^T  ( (A^--R A^+) U - SG )).
\end{equation}
%Note that the penalty matrix is now predetermined (or removed) leaving a  less parameter dependent form.
\begin{remark}\label{simplified}
 The new boundary procedure simplifies the previous one by {\it i)} reducing the algebra in (\ref{Gen_BC_form}) and  (\ref{Pen_term_Gen_BC_form}) to (\ref{Gen_BC_form_simp}) and (\ref{Pen_term_Gen_BC_form_simp}) respectively, {\it ii)} reducing the number of parameters from three ($R,S,\Sigma$) to two ($R,S$) and  {\it iii)} by operating with $B(U)$ explicitly on the original dependent variable $U$. 
\end{remark}
%\begin{remark}\label{simplified}
 %The new boundary procedure in (\ref{Gen_BC_form_simp}) and (\ref{Pen_term_Gen_BC_form_simp}) simplifies the previous one by {\it i)} reducing the algebra, {\it ii)} reducing the parameters and  {\it iii)} operating only on the original dependent variable $U$. 
%\end{remark}
The following Lemma is the main result of this note and it  replaces Lemma 3.1 in \cite{NORDSTROM2024_BC}.


\begin{lemma}\label{lemma:GenBC}
Consider the boundary term (\ref{1Dprimalstab_trans_final_simp}), the boundary condition (\ref{Gen_BC_form_simp}) and the lifting operator (\ref{Pen_term_Gen_BC_form_simp}).
%Furthermore, let $ |\Lambda^- |=diag( |\lambda^-_i |)$ and $  |\Lambda^- |^{1/2}=diag(  \sqrt{|\lambda_i^-|})$. 

The boundary term augmented with {\bf 1. strong nonlinear homogeneous boundary conditions} is positive semi-definite if the matrix $R$ is such that
%$W^T   \Lambda   W  = (W^+)^T ( \Lambda^+ - R^T  |\Lambda^- |  R)W^+  \geq 0$
%\[
% W^T   \Lambda   W  = (W^+)^T ( \Lambda^+ - R^T  |\Lambda^- |  R)W^+
%\]
\begin{equation}\label{R_condition}
I- R^T  R  \geq 0.
\end{equation}

The boundary term augmented with {\bf 2. strong nonlinear inhomogeneous boundary conditions} is bounded by the data $G$ if
the matrix  $R$ satisfies (\ref{R_condition}) with strict inequality and the matrix $S$ is such that
\begin{equation}\label{S_condition}
I- S^T  S - (R^T S)^T (I- R^T  R)^{-1} (R^T S)  \geq 0 \quad  \text{where}  \quad (I- R^T  R)^{-1}=\sum^{\infty}_{k=0}(R^T R)^k.
%S =   \tilde S^{-1} |\Lambda^- |^{1/2} \ \mbox{with 
%$\tilde S$ sufficiently small in an absolute sense.}
\end{equation}

The boundary term augmented with {\bf 3. weak nonlinear homogeneous boundary conditions} is positive semi-definite if the matrix $R$
satisfies (\ref{R_condition}).

The boundary term augmented with {\bf 4. weak nonlinear inhomogeneous boundary conditions} is bounded by the data $G$ if
the matrix $R$ satisfies (\ref{R_condition}) with strict inequality and the matrix $S$ satisfies  (\ref{S_condition}).
% and the matrix $\Sigma$ is given by (\ref{Sigma_condition}).
\end{lemma}
\begin{remark}\label{explain_notation_nonstadard} 
We have used the notation $A \geq 0$ to indicate that the matrix $A$ is positive semi-definite above.
%  and $A > 0$  to indicate positive definiteness in (\ref{R_condition}) and (\ref{S_condition}).
%, and will continue to use it below.
\end{remark}
\begin{proof} 
%We proceed in the step-by-step manner indicated in the Lemma above.
The proof of Lemma \ref{lemma:GenBC} is step-by-step identical to the proof of Lemma 3.1 in \cite{NORDSTROM2024_BC} by replacing $\sqrt{|\Lambda^-|}W^-,\sqrt{\Lambda^+}W^+$ with  $A^- U, A^+ U$ respectively and $\tilde \Sigma = 2( I^-T^{-1})^T \Sigma S$ with $\tilde \Sigma=2 (A^-)^T S$.
%\begin{equation}\label{Pen_term_Gen_BC_form}
%L_C(\tilde \Sigma(BU-G))= L_C(2( I^-T^{-1})^T \Sigma ( \sqrt{|\Lambda^-|}W^--R \sqrt{\Lambda^+}W^+ - SG )),
%\end{equation}
%where
%$W=T^{-1}U$, $W^-=I^-W$, $W^+=I^+W$, $I^-+I^+=I$ and 
%$\tilde \Sigma = 2( I^-T^{-1})^T \Sigma S$ 
%where  $A^+=\sqrt{\Lambda^+} T^{-1}$ and $A^-=\sqrt{|\Lambda^-|} T^{-1}$
%$\tilde \Sigma=2 (A^-)^T S$ 
\end{proof}
%\begin{remark}\label{explain_proof_simplified} 
%The new formulation of the weak boundary condition now contains on two matrix parameters $R$ and $S$
%\end{remark}

%\begin{remark}
%The proof of Lemma \ref{lemma:GenBC} is step-by-step identical to the proof of Lemma 3.1 in \cite{NORDSTROM2024_BC}, but simplified.
%can be used to prove that the estimates (\ref{1Dprimalstab}) and (\ref{1Dprimalstab_strong}) in Proposition \ref{lemma:Matrixrelation} holds.
%\end{remark}
%The matrices $R,S$ satisfies the conditions in  Lemma 3.1 in \cite{NORDSTROM2024_BC}.
%This concludes the analysis of the formulation of nonlinear boundary conditions for first derivative terms.
%\begin{remark}
%The new boundary procedure presented above generalise the well known linear characteristic boundary procedure  \cite{kreiss1970,MR436612} by inserting the additional scaling with the solution dependent eigenvalues. We show below that the new nonlinear boundary procedure can also be used to bound IBVPs involving second derivatives similar to what has been done in the linear case \cite{nordstrom2005,MR1339182,MR1669660}.
%\cite{MR1339182,MR1669660,MR2177803}. 
%\end{remark}

%%%%%%%%%%%%%%%%%%%%%%%%%%%%%%%%%%%%%%%%%%%%%%%%%%%%%%%%%%%%%%%%%%%%%%%%
\subsection{Extension to include viscous terms}\label{nonlinear_BC_visc}
%%%%%%%%%%%%%%%%%%%%%%%%%%%%%%%%%%%%%%%%%%%%%%%%%%%%%%%%%%%%%%%%%%%%%%%%
The boundary terms to consider in the case when $\epsilon >  0$  are given in Proposition \ref{lemma:Matrixrelation_complemented}. The argument in the surface integral
can be reformulated into the first derivative setting.  By introducing the notation $n_i D_i/2=\tilde F$ for the viscous flux we rewrite the boundary terms as
\begin{equation}\label{newflux_form_1}
U^T  (n_i A_i) U -   \epsilon U^T (n_i D_i) =  U^T \tilde A U -  \epsilon U^T  \tilde F - \epsilon \tilde F^T  U=
\begin{bmatrix}
U \\
\epsilon \tilde F
\end{bmatrix}^T
\begin{bmatrix}
\tilde A & - I \\
 - I &  0
\end{bmatrix}
\begin{bmatrix}
U \\
 \epsilon \tilde F
\end{bmatrix}
% (U-\epsilon  \tilde A^{-1} \tilde F)^T \tilde A (U-\epsilon  \tilde A^{-1} \tilde F)-(\epsilon  \tilde A^{-1} \tilde F)^T  \tilde A  (\epsilon  \tilde A^{-1} \tilde F).
\end{equation}
where $I$ is the identity matrix. We can now formally diagonalise the boundary terms in (\ref{newflux_form_1}), apply the boundary condition (\ref{Gen_BC_form_simp}) in combination with (\ref{Pen_term_Gen_BC_form_simp}) and  Lemma \ref{lemma:GenBC} to obtain energy bounds.

%%%%%%%%%%%%%%%%%%%%%%%%%%%%%%%%%%%%%%%%%%%%%%%%%%%%%%%%%%%%%%%%%%%%%%%%
\section{Summary}\label{sec:conclusion}
%%%%%%%%%%%%%%%%%%%%%%%%%%%%%%%%%%%%%%%%%%%%%%%%%%%%%%%%%%%%%%%%%%%%%%%%
We have simplified and removed unnecessary parameter dependencies in previous work on nonlinear boundary conditions in \cite{NORDSTROM2024_BC}. This will remove complexities from its practical numerical implementation.
%%%%%%%%%%%%%%%%%%%%%%%%%%%%%%%%%%%%%%%%%%%%%%%%%%%%%%%%%%%%%%%%%%%%%%%%
\section*{Acknowledgment}
%%%%%%%%%%%%%%%%%%%%%%%%%%%%%%%%%%%%%%%%%%%%%%%%%%%%%%%%%%%%%%%%%%%%%%%%
J.N.  was supported by Vetenskapsr{\aa}det, Sweden [no.~2021-05484 VR] and University of Johannesburg Global Excellence and Stature Initiative Funding.
 

%%%%%%%%%%%%%%%%%%%%%%%%%%%%%%%%%%%%%%%%%%%%%%%%%%%%%%%%%%%%%%%%%%%%%%%%
%\section*{References}
%\section{References}
%%%%%%%%%%%%%%%%%%%%%%%%%%%%%%%%%%%%%%%%%%%%%%%%%%%%%%%%%%%%%%%%%%%%%%%%
\bibliographystyle{elsarticle-num}
%\bibliography{References_Jan,References_andrew,References_Fredrik}
%\bibliography{References_Jan,References_Fredrik}
\bibliography{References_Jan}


\end{document}

\it 1. The homogeneous boundary condition (\ref{Gen_BC_form_simp}) implemented strongly} (with $G=0$) in (\ref{1Dprimalstab_trans_final_simp}) lead to 
\begin{equation}\label{R1_derivation}
U^T   \tilde A   \\\ U  = (A^+U)^T  (A^+ U) - (A^-U)^T  (A^-U) =  (A^+ U)^T (I-R^TR) (A^+ U) \geq 0
\end{equation}
since $A^-U=R A^+ U $ and condition (\ref{R_condition}) guarantees positive semi-definiteness. 
%This determines the bounds on matrix $R$
%an energy bounded solution as defined in Proposition \ref{lemma:Matrixrelation}.

{\it 2. The inhomogeneous boundary condition (\ref{Gen_BC_form_simp}) implemented strongly} (with $G\neq0$)  lead to
\begin{equation}\label{S1_derivation}
U^T   \tilde A   \\\ U  = (A^+U)^T  (A^+ U)  -  (R  (A^+ U) + S G)^T (R  (A^+ U) + S G).
\end{equation}
Expanding (\ref{S1_derivation}), adding and subtracting $G^T G$ lead to the result
\begin{equation}
\label{estimate_2}
U^T   \tilde A   \\\ U =
\begin{bmatrix}
A^+ U \\
G
\end{bmatrix}^T
\begin{bmatrix}
I- R^T R & - R^T S \\
-S^T R  &  I-S^T S
\end{bmatrix}
\begin{bmatrix}
A^+ U \\
G
\end{bmatrix} - G^T G. 
\end{equation}
The matrix in (\ref{estimate_2}) can be rotated into block-diagonal form with the upper left block preserved and the lower right block being
$I- S^T  S - (R^T S)^T\left( I- R^T R \right)^{-1}  (R^T S)$ since (\ref{R_condition}) holds strictly. Next we choose $S$ such that (\ref{S_condition}) holds which guarantees that the left term on the righthand side  in (\ref{estimate_2}) is non-negative.
%\begin{equation}\label{S_condition-derivation}
%I- S^T  S - (R^T S)^T\left( I- R^T R \right)^{-1}  (R^T S)=I- S^T  S - (R^T S)^T\left(\sum^{\nifty}_{k=0}(R^T R)^k \right) (R^T S), 
%S =   \tilde S^{-1} |\Lambda^- |^{1/2} \ \mbox{with 
%$\tilde S$ sufficiently small in an absolute sense.}
%\end{equation}
%which is bounded from below by the data if (\ref{R_condition}) is satisfied with strict inequality and condition (\ref{S_condition}) holds.
%\begin{equation}
%\begin{bmatrix}
%\Lambda^+ - R^T  |\Lambda^- |  R & - R^T  |\Lambda^- |^{-1/2} \tilde S \\
%-\tilde S^T  |\Lambda^- |^{-1/2} R  &  I-\tilde S^T \tilde S
%\end{bmatrix} \geq 0,
%\end{equation}

{\it 3. The homogeneous boundary condition (\ref{Gen_BC_form_simp}) implemented weakly} (with $G=0$)  using the lifting operator in (\ref{Pen_term_Gen_BC_form_simp}) and the energy method lead to the boundary term
\begin{equation}
\label{Sigma1_derivation}
U^T   \tilde A   \\\ U   + 2 (A^- U)^T (A^- U - R A^+ U)= (A^+ U)^T (A^+ U) + (A^- U)^T (A^- U)- 2 (A^- U)^T  R (A^+ U).
 \end{equation}
% where we let $\Sigma=\Sigma^T.$ 
 %Collecting similar terms and making the choice (\ref{Sigma_condition}) for  $\Sigma$ transforms the right hand side to
%\begin{equation}\label{Sigma2_derivation}
% (\sqrt{\Lambda^+}W^+)^T  (\sqrt{\Lambda^+}W^+) +( \sqrt{|\Lambda^-|}W^-)^T ( \sqrt{|\Lambda^-|}W^-) -2 ( \sqrt{|\Lambda^-|}W^-)^T R (\sqrt{\Lambda^+}W^+).
%\end{equation}
Adding and subtracting  $(R A^+ U)^T  (R A^+ U)$ together with  (\ref{R_condition}) guarantee positivity since  (\ref{Sigma1_derivation}) goes to
%(\ref{Sigma2_derivation}) 
\begin{equation}\label{Sigma4_derivation}
  (A^+ U)^T (I - R^T R)(A^+ U) + (A^- U-R A^+ U)^T (A^- U-R A^+ U) \geq 0.
\end{equation}
%which lead to positive semi-definite boundary term by using condition (\ref{R_condition}).
% to an energy bounded solution as defined in Proposition \ref{lemma:Matrixrelation} if (\ref{R_condition}) holds.

{\it 4. The inhomogeneous boundary condition (\ref{Gen_BC_form_simp}) implemented weakly} (with $G\neq0$)  using the lifting operator in (\ref{Pen_term_Gen_BC_form_simp}) and the energy method lead to the boundary terms
%By choosing  $R$ as in (\ref{R_condition}),  $S$ as in (\ref{S_condition}) and $\Sigma$ as in (\ref{Sigma_condition}) followed by adding and subtracting $G^TG$ provides the following boundary terms in the weak inhomogeneous case
%\[
\begin{equation}\label{Sigma3_derivation}
U^T   \tilde A   \\\ U   + 2 (A^- U)^T (A^- U - R A^+ U-SG).
%W^T \Lambda   W   + 2 (\sqrt{|\Lambda^-|}W^-)^T  (\sqrt{|\Lambda^-|} W^- - R \sqrt{\Lambda^+}W^+  -  SG)).
\end{equation}
%\]
By adding and subtracting $G^T G $  and rearranging, the boundary terms (\ref{Sigma3_derivation}) above can be written as
%\begin{equation}\label{final _gen_weak_result_1}
%( \sqrt{\Lambda^+}W^+)^T (I - R^T R) ( \sqrt{\Lambda^+}W^+) +   (W^- - R W^+)^T  |\Lambda^- | (W^- - R W^+) - 2 (W^-)^T  |\Lambda^- |  S^{-1}G.
%\end{equation}
%The first term in (\ref{final _gen_weak_result_1}) is positive if condition (\ref{R_condition}) on $R$ holds with strict inequality. 
%By rearranging  (\ref{final _gen_weak_result_1}) we find that it is equivalent to
%\begin{equation}\label{final _gen_weak_result_2}
%(W^+)^T (\Lambda^+ - R^T  |\Lambda^- | R) W^+ +   (W^- - R W^+)^T  |\Lambda^- | (W^- - R W^+) - 2 (W^-)^T  |\Lambda^- |  S^{-1}G.
%\end{equation}
%\begin{equation}\label{final _gen_weak_result_1}
%W^T \Lambda   W &+ 2 U^T  ( E^-(NT)^{-1})^T \Sigma (W^- - R W^+ - S^{-1} G))=W^T \Lambda   W + 2 (W^-)^T  |\Lambda^- | (W^- - R W^+  -  S^{-1}G))
%\end{equation}
%\begin{equation}\nonumber
%\label{final _gen_weak_result_2}
% (\sqrt{|\Lambda^-|}W^- - R  \sqrt{\Lambda^+}W^+ - SG)^T  (\sqrt{|\Lambda^-|}W^- - R  \sqrt{\Lambda^+}W^+ - SG) + 
% \begin{bmatrix}
%\sqrt{\Lambda^+} W^+ \\
%G
%\end{bmatrix}^T
%\begin{bmatrix}
%I- R^T R & - R^T S \\
%-S^T R  &  I-S^T S
%\end{bmatrix}
%\begin{bmatrix}
%\sqrt{\Lambda^+} W^+ \\
%G
%\end{bmatrix}
%\end{equation}
\begin{align}
& (A^- U - R A^+ U - SG)^T  (A^- U - R A^+ U - SG) +  \begin{bmatrix}
A^+ U  \\
G
\end{bmatrix}^T
\begin{bmatrix}
I- R^T R & - R^T S \\
-S^T R  &  I-S^T S
\end{bmatrix}
\begin{bmatrix}
A^+ U  \\
G
\end{bmatrix} - G^TG.
%\\ 
%&  \begin{bmatrix}
%A^+ U  \\
%G
%\end{bmatrix}^T
%\begin{bmatrix}
%I- R^T R & - R^T S \\
%-S^T R  &  I-S^T S
%\end{bmatrix}
%\begin{bmatrix}
%A^+ U  \\
%G
%\end{bmatrix} - G^TG 
%\label{generic_discrete_BT_lifting_details_2}
\end{align}
The first term is obviously positive semi-definite and the remaining part is identical to the term in (\ref{estimate_2}).


%%%%%%%%%%%%%%%%%%%%%%%%%%%%%%%%%%%%%%%%%%%%%%%%%%%%%%%%%%%%%%%%%%%%%%%%
\section{Stability of numerical schemes built on skew-symmetric formulations}\label{numerics}
%%%%%%%%%%%%%%%%%%%%%%%%%%%%%%%%%%%%%%%%%%%%%%%%%%%%%%%%%%%%%%%%%%%%%%%%
To exemplify the straightforward construction of stable linear and nonlinear schemes based on the skew-symmetric formulation in Proposition \ref{lemma:Matrixrelation} we consider an SBP approximation of the 1D SWEs obtained by cancelling relevant terms in (\ref{eq:swNoncons_new_matrix_ansatz_sol_A}). The boundary conditions (normally imposed through SAT terms or numerical flux functions) are ignored and we focus on  the continuous equation 
\begin{equation}\label{SWE_Cont}
U_t+(A U)_x+A^T U_x  =0 \quad \text{where}  \quad
A = \begin{bmatrix}
     a_{11}& a_{12} \\[0.05cm]
     a_{21}& a_{22}\\[0.05cm]
     \end{bmatrix}=
       \begin{bmatrix}
     \alpha \frac{U_2}{\sqrt{U_1}} & (1-3 \alpha) \sqrt{U_1}  \\[0.05cm]
     2 \alpha  \sqrt{U_1} &   \frac{1}{2} \frac{U_2}{\sqrt{U_1}}                          \\[0.05cm]
       \end{bmatrix}
\end{equation}
which is semi-discretised in space using summation-by-parts operators as
\begin{equation}\label{SWE_Disc}
 \vec U_t+{\bf D_x} ({\bf A}  \vec U)+{\bf A}^T {\bf D_x}  (\vec U) =0  \quad \text{where}  \quad
 \vec U= \begin{bmatrix}
   \vec U_1  \\[0.05cm]
   \vec U_2 \\[0.05cm]
     \end{bmatrix}
 \quad \text{and}  \quad
{\bf A} = \begin{bmatrix}
     {\bf a_{11}} & {\bf a_{12}} \\[0.05cm]
     {\bf a_{21}} & {\bf a_{22}}\\[0.05cm]
     \end{bmatrix}.
\end{equation}
In (\ref{SWE_Disc}),  $(\vec U_1, \vec U_2)$ are approximations of  $(U_1,U_2)$ in each node, ${\bf a_{11}}, {\bf a_{12}}, {\bf a_{21}}, {\bf a_{22}}$ are diagonal matrices with node values of $ a_{11}, a_{12}, a_{21}, a_{22}$ injected. Moreover ${\bf D_x}=I_2 \otimes D_{x}$ where $D_{x}=P^{-1}_{x}Q_{x}$ is the SBP difference operator, $P_{x}$ is a positive definite diagonal quadrature matrix, $Q_{x}$ satisfies the SBP constraint $Q_{x}+Q_{x}^T=B=diag[-1,0,...,0,1]$, $\otimes$ denote the Kronecker product  and $I_2$  is the $2 \times 2$ identity matrix.

The discrete energy method yields 
\begin{equation}\label{SWE_Disc_energy}
 \vec U^T (I_2 \otimes P_x) \vec U_t+ \vec U^T (I_2 \otimes Q_x) {\bf A}  \vec U)+ \vec U^T (I_2 \otimes P_x){\bf A}^T {\bf D_x}  (\vec U) =0.
\end{equation}
Next we note that $(I_2 \otimes P_{x}){\bf A}^T={\bf A}^T(I_2 \otimes P_{x})$ since the matrices involved are diagonal. The SBP property of $Q_{x}$ now allows us to discretely integrate-by-parts using $Q_{x}=B-Q_{x}^T$ and arrive at
\begin{equation}\label{SWE_Disc_energy_final}
\dfrac{d}{dt} \|U\|^2_{I_2 \otimes P_x}+ \vec U^T (I_2 \otimes B) {\bf A}  \vec U=({\bf D_x}  \vec U)^T (I_2 \otimes P_x){\bf A}\vec U-({\bf A}\vec U)^T (I_2 \otimes P_x) {\bf D_x}  \vec U,
\end{equation}
which mimics (\ref{eq:boundaryPart1}) perfectly. Consequently, since the right hand side of (\ref{SWE_Disc_energy_final}) vanish, only the energy rate in terms of boundary terms remain. Note that it is irrelevant whether the matrix $\bf A$ is a function of the solution or not. The skew-symmetry and proper boundary conditions are all that matters for stability.

