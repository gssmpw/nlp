\section{Related Work}
\label{appsec:relatedwork}

\subsection{Molecular Dynamics (MD)} \label{app:md_md}
The most fundamental concepts nowadays to describe the dynamics of molecules are given by the laws of quantum mechanics. The Schrödinger equation is a partial differential equation, that gives the evolution of the complex-valued wave function $\psi$ over time $t$: $\displaystyle i\hbar \dfrac{\partial \psi}{\partial t} = \hat{H}(t) \psi$. Here $i$ is the imaginary unit with $i^2=-1$, $\hbar$ is reduced Planck constant, and, $\hat{H}(t)$ is the Hamiltonian operator at time $t$, which is applied to a function $\psi$ and maps to another function. It determines how a quantum system evolves with time and its eigenvalues correspond to measurable energy values of the quantum system. The solution to Schrödinger's equation in the many-body case (particles $1,\ldots,N$) is the wave function $\psi(\mathbf{x}_1,\ldots,\mathbf{x}_N, t): \bigtimes_{i=1}^N \dR^3 \times \dR \rightarrow \dC$ which we abbreviate as $\psi(\left\{ \mathbf{x} \right\}, t)$. It's the square modulus $|\psi(\left\{ \mathbf{x} \right\}, t)|^2=\psi^*(\left\{ \mathbf{x} \right\}, t) \psi(\left\{ \mathbf{x} \right\}, t)$ is usually interpreted as a probability density to measure the positions  $\mathbf{x}_1,\ldots,\mathbf{x}_N$ at time $t$, whereby the normalization condition $\int \ldots \int |\psi(\left\{ \mathbf{x} \right\}, t)|^2 d\mathbf{x}_1 \ldots d\mathbf{x}_N=1$ holds for the wave function $\psi$.

Analytic solutions of $\psi$ for specific operators $\hat{H(t)}$ are hardly known and are only available for simple systems like free particles or hydrogen atoms. In contrast to that are proteins with many thousands of atoms. However, already for much smaller quantum systems approximations are needed. A famous example is the Born–Oppenheimer approximation, where the wave function of the multi-body system is decomposed into parts for heavier atom nuclei and the light-weight electrons, which usually move much faster. In this case, one obtains a Schrödinger equation for electron movement and another Schrödinger equation for nuclei movement. A much faster option than solving a second Schrödinger equation for the motion of the nuclei is to use the laws from classical Newtonian dynamics. The solution of the first Schrödinger equation defines an energy potential, which can be utilized to obtain forces $\mathbf{F}_i$ on the nuclei and to update nuclei positions according to Newton's equation of motion: $\mathbf{F}_i=m_i \ \ddot{\mathbf{q}}_i(t)$ (with $m_i$ being the mass of particle $i$ and $\mathbf{q}_i(t)$ describing the motion trajectory of particle $i$ over time $t$).

Additional complexity in studying molecule dynamics is introduced by environmental conditions surrounding molecules. Maybe the most important is temperature. For bio-molecules it is often of interest to assume that they are dissolved in water. To model temperature, a usual strategy is to assume a system of coupled harmonic oscillators to model a heat bath, from which Langevin dynamics can be derived \citep{ford1965statistical, zwanzig1973nonlinear}. The investigation of the relationship between quantum-mechanical modeling of heat baths and Langevin dynamics still seems to be a current research topic, where there there are different aspects like the coupling of the oscillators or Markovian properties when stochastic forces are introduced. For instance, \citet{hoel2019classical}, studies how canonical quantum observables are approximated by molecular dynamics. This includes the definition of density operators, which behave according to the quantum Liouville-von Neumann equation. 

The forces in molecules are usually given as the negative derivative of the (potential) energy: $\mathbf{F}_i=- \nabla E$. In the context of molecules, $E$ is usually assumed to be defined by a force field, which is a parameterized sum of intra- and intermolecular interaction terms. An example is the Amber force field \citep{Ponder2003,Case2024}:
\begin{align}
E=&\sum_{\text{bonds} \ r} k_b (r-r_0)^2+\sum_{\text{angles} \ \theta} k_{\theta} (\theta-\theta_0)^2+\\ \nonumber &\sum_{\text{dihedrals} \ \phi} V_n (1+cos(n \phi - \gamma))+\sum_{i=1}^{N-1} \sum_{j=i+1}^{N} \left( \frac{A_{ij}}{R_{ij}^{12}}-\frac{B_{ij}}{R_{ij}^6}+\frac{q_i q_j}{\epsilon R_{ij}}\right)
\end{align}
Here $k_b, r_0, k_{\theta}, \theta_0, V_n, \gamma, A_{ij}, B_{ij}, \epsilon, q_i, q_j$ serve as force field parameters, which are found either empirically or which might be inspired by theory.

Newton's equations of motions for all particles under consideration form a system of ordinary differential equations (ODEs), to which different numeric integration schemes like Euler, Leapfrog, or, Verlet can be applied to obtain particle position trajectories for given initial positions and initial velocities. In case temperature is included, the resulting Langevin equations form a system of stochastic differential equations (SDEs), and Langevin integrators can be used. It should be mentioned, that it is often necessary to use very small integration timesteps to avoid large approximation errors. This, however, increases the time needed to find new stable molecular configurations.

\subsection{Relationship of \ourMethod to Graph Foundation Models}
From our perspective, \ourMethod bears a relationship to graph foundation models \citep[GFMs;][]{liu2023towards,maoposition}. \citet{bommasani2021opportunities} consider foundation models to be \textit{trained on broad data at scale} and to be \textit{adaptable to a wide range of downstream tasks}. \citet{maoposition} argue, that graphs are more diverse than natural language or images, and therefore there are quite unique challenges for GFMs. Especially they mention that \textit{ none of the current GFM have the capability to transfer across all graph tasks and datasets from all domains}. It is for sure true that \ourMethod is not a GFM in this sense. However, it might be debatable whether \ourMethod might serve as a domain- or task-specfic GFM. While we mainly focused on a trajectory prediction task and are from that point of view task-specific, we observed that our trained models can generalize across different molecules or differently taken scenes, which might seem quite remarkable given that it is common practice to train specific trajectory prediction models for single molecules or single scenes. Nevertheless, it was not our aim in this research to provide a GFM, since we believe that this would require more investigation into further domains and could also require, for instance, checking whether emergent abilities might arise with larger models and more training data \citep{liu2023towards}.









\clearpage