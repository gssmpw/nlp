\section{Intervention Experiment Details}
\subsection{Intervention Cases}
\label{case_appendix}
We present additional typical cases from other intervention experiments at the Table~\ref{tab:case_other}. The prompts used for the three experimental groups are as follows: Politeness: “User: Sir, I want to make an order offline. Assistant:”. Linking Verb: “User: Sir, tell me something about your ideal room. Assistant:”. Past-Tense: “User: Sir, tell me a story about you. Assistant:”.

During manual analysis, both the enhancement and ablation results show clear effects of amplification or suppression of the target linguistic features. Specifically, when intervening with the past tense feature in the 8th layer, the enhancement significantly impacts the coherence of the model’s output language. Yet, in the discontinuous output text, the frequency of the morphological past-tense feature still increases dramatically.

\begin{table}[t]
    \centering
    \fontsize{10pt}{10pt}\selectfont
    \resizebox{1.01\linewidth}{!}{
    \setlength{\tabcolsep}{10pt}
    \begin{tabular}{l l l}
        \toprule
        \multirow{2}{*}{\centering \textbf{Condition}} & \multirow{2}{*}{\centering \textbf{Politeness}} & \multirow{2}{*}{\centering \textbf{Linking Verb}} \\
        & & \\
        \midrule
        \textbf{Enhancement} & \parbox[t]{5cm}{Can I textbf{please} have your email address?} & \parbox[t]{5cm}{The room should textbf{be} large and well lit. It should textbf{be} airy and bright and airy.} \\
        \midrule
        \textbf{Default} & \parbox[t]{5cm}{May I have your phone number?} & \parbox[t]{5cm}{Sure, my ideal room has good ventilation and textbf{is} spacious.} \\
        \midrule
        \textbf{Ablation} & \parbox[t]{5cm}{OK, what is your name?} & \parbox[t]{5cm}{I can provide you with a list of the ideal characteristics that make up a perfect room.} \\
        \midrule\midrule
        \multirow{2}{*}{\centering \textbf{Condition}} & \multirow{2}{*}{\centering \textbf{Past-Tense}} \\
        & \\
        \midrule
        \textbf{Enhancement} & \parbox[t]{5cm}{"I was textbf{asked} for the story. " I having me textbf{had} a “one the: ” textbf{told}. They: textbf{told}:} \\
        \midrule
        \textbf{Default} & \parbox[t]{5cm}{I'm not a story, I'm a bot.} \\
        \midrule
        \textbf{Ablation} & \parbox[t]{5cm}{Well, I don't actually have one, and I'm not really sure I'm able to either.} \\
        \bottomrule
    \end{tabular}
    }
    \caption{Typical outputs from the enhancement, ablation, and default experiments for the politeness, linking verb, and past-tense features.}
    \label{tab:case_other}
\end{table}

\subsection{LLM as a Judge}
\label{judge_appendix}
In our feature intervention and combination intervention experiments, we used an LLM as a judge to assess the significance of linguistic features in generated texts. Feature significance is defined based on the frequency, accuracy, and contextual appropriateness of the target feature, as well as its contribution to overall meaning or rhetorical effect.

The prompt structure is as follows:

\begin{quote}
\textbf{Please compare the following two texts based on \{feature\}.}

- \textbf{Text A}: "\{text\_a\}"
- \textbf{Text B}: "\{text\_b\}"
\end{quote}

Here, \texttt{text\_a} and \texttt{text\_b} are generated texts truncated to 100 tokens.


In the intervention experiments, each feature is defined as follows:

\paragraph{Politeness Significance}
Refers to the degree to which politeness strategies are salient, effective, and contextually integrated. This definition encompasses frequency, pragmatic depth, and social impact in shaping interpersonal rapport, mitigating face threats, and reinforcing cooperative intent.

\paragraph{Past Tense Verb Significance}
Refers to the degree to which past tense verbs are salient, accurate, and contextually integrated. It includes frequency, morphological consistency, and the rhetorical or narrative impact on establishing a coherent sense of time and providing historical context.

\paragraph{Causality Significance}
Refers to the degree to which cause-and-effect relationships are clearly indicated, logically structured, and contextually coherent. This includes the frequency and precision of causal connectives (e.g., \textit{because, therefore, thus}) and the depth of reasoning to explain how conditions lead to outcomes.

\paragraph{Linking Verb Structure Significance}
Refers to the degree to which linking verbs (e.g., \textit{be, become, seem, appear}) are salient, accurate, and contextually integrated. It emphasizes frequency, morphological correctness, semantic clarity, and effectiveness in conveying states, characteristics, or identities.

\paragraph{Simile Significance}
Refers to the degree to which similes (e.g., comparisons using \textit{like} or \textit{as}) are salient, creative, and contextually integrated. This definition encompasses frequency, imagery richness, and the rhetorical impact on clarity, vividness, and reader engagement.