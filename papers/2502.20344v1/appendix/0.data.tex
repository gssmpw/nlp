\section{Dataset Construction}


\subsection{Linguistic Structure Definition}
\label{app:ling}

\begin{enumerate}
\item \textbf{Phonetics} examines the physical production and acoustic properties of speech sounds. 
\item \textbf{Phonology} investigates the abstract sound systems and patterns in a language. 
\item \textbf{Morphology} focuses on the internal structure of words. 
\item \textbf{Syntax} deals with sentence structure and the rules governing the arrangement of words into phrases and clauses. 
\item \textbf{Semantics} explores the meaning of words and sentences at a literal or denotational level. 
\item \textbf{Pragmatics} considers how context influences meaning, covering phenomena like implicature, presupposition, and speech acts.
\end{enumerate}

\subsection{Construction Process}
\label{app:data_construction}
When constructing the dataset, we first manually create 5–10 Chinese and English sentences that strongly exhibit the target linguistic feature, along with 5–10 counterfactual sentences. Next, we use GPT-o1 to generate the remaining portion of the dataset, which includes 80 feature-consistent Chinese and English sentences and 80 counterfactual sentences. Minimal pairs are constructed by manually modifying each feature-consistent sentence. After dataset construction, all sentences are manually reviewed and corrected.

\subsection{Construction Examples}
We present three examples from morphology, syntax, and semantics.

\vspace{3ex}

\noindent\textbf{Plural Noun – Minimal Pairs:} Directly change the plural form to singular, disregarding grammar.\\[1ex]
The books on the shelf are all about history.\\
The book on the shelf are all about history.\\
She bought several flowers to decorate the house.\\
She bought several flower to decorate the house.

\vspace{2ex}

\noindent\textbf{Plural Noun – Counterfactual Sentence:} Maintain grammatical correctness while removing the plural form.\\[1ex]
The books on the shelf are all about history.\\
The book on the shelf is all about history.\\
She bought several flowers to decorate the house.\\
She bought a flower to decorate the house.

\vspace{2ex}

\noindent\textbf{Genitive – Minimal Pair:} Remove ``of'' while keeping the rest unchanged.\\[1ex]
The roof of the house was damaged in the storm.\\
The roof the house was damaged in the storm.\\
The color of the sky changed at sunset.\\
The color the sky changed at sunset.

\vspace{2ex}

\noindent\textbf{Genitive – Counterfactual Sentence:} Retain ``of'' but use a non-genitive context.\\[1ex]
The roof of the house was damaged in the storm.\\
She is afraid of spiders.\\
The color of the sky changed at sunset.\\
He ran out of time.

\vspace{2ex}

\noindent\textbf{Contrast – Minimal Pair:} Remove the contrast marker, disregarding grammar.\\[1ex]
I wanted to go out, but it was raining.\\
I wanted to go out, it was raining.\\
She was very tired, yet she kept working.\\
She was very tired, she kept working.

\vspace{2ex}

\noindent\textbf{Contrast – Counterfactual Sentence:} Change the meaning by altering the logical relation in the second clause to a continuation.\\[1ex]
I wanted to go out, but it was raining.\\
I wanted to go out, then I grabbed an umbrella.\\
She was very tired, yet she kept working.\\
She was very tired, then she took a short nap.


\subsection{Special Cases for Minimal Pairs}
In cases involving transitive verbs, intransitive verbs, linking verbs, and preposed verbs in inversion, direct deletion results in sentences that lose their predicate meaning and cannot convey a complete semantic unit. In such cases, examining activations is meaningless, and minimal contrast pair datasets are not constructed for these features. In the polite speech dataset, the minimal contrast pairs obtained by removing the politeness marker are identical to the counterfactual sentences converted to non-polite sentences; hence, the two datasets are the same.