\section{Metric Calculation}

\subsection{Feature Representation Confidence (FRC)}
\label{app:frc}

In our feature analysis experiments, we introduce two key causal probabilities that serve as the basis for computing the Feature Representation Confidence (FRC). 

The first measure, the \emph{Probability of Necessity} (PN), is defined as \(PN = \frac{P(Y=1\mid do(X=1)) - P(Y=1\mid do(X=0))}{P(Y=1\mid do(X=1))}\). This metric quantifies the extent to which the presence of a linguistic feature is necessary for the activation of a corresponding base vector. Here, \(P(Y=1\mid do(X=1))\) represents the probability that the base vector is activated when the feature is present, whereas \(P(Y=1\mid do(X=0))\) indicates the probability of activation when the feature is deliberately suppressed via intervention. The numerator, \(P(Y=1\mid do(X=1))-P(Y=1\mid do(X=0))\), captures the net increase in activation due to the feature, and dividing by \(P(Y=1\mid do(X=1))\) normalizes this increase relative to the activation when the feature is present.

Similarly, the second measure, the \emph{Probability of Sufficiency} (PS) is expressed as \(PS = \frac{P(Y=1\mid do(X=1)) - P(Y=1\mid do(X=0))}{1-P(Y=1\mid do(X=0))}\). PS measures the likelihood that the introduction of the feature is sufficient to trigger the activation of the base vector. In this formulation, the denominator \(1-P(Y=1\mid do(X=0))\) represents the maximum possible increase in activation probability (i.e., the probability that the base vector is not activated in the absence of the feature). Thus, PS reflects the proportion of this potential increase that is realized when the feature is present.

Finally, the Feature Representation Confidence (FRC) is computed as the harmonic mean of PN and PS: \(FRC = \frac{2\, PN\, PS}{PN + PS}\). The harmonic mean is chosen because it ensures that FRC remains low if either PN or PS is low, thereby providing a balanced measure that only yields a high score when both necessity and sufficiency are strong. This approach allows us to robustly quantify the ability of the SAE latent space's base vectors to represent the targeted linguistic features.

\subsection{Feature Intervention Confidence (FIC)}
\label{app:fic}

In our methodology, the Feature Intervention Confidence (FIC) score is computed as the harmonic mean of the normalized ablation effect \(E_{abl}\) and the normalized enhancement effect \(E_{enh}\):
\[
FIC = \frac{2\, E_{abl}\, E_{enh}}{E_{abl} + E_{enh}}.
\]
This formulation ensures that FIC is high only when both the ablation and enhancement interventions yield strong effects.

In practice, however, it is possible that one or both of these effects are negative, indicating that an intervention produces an effect opposite to the intended direction. Moreover, even if only one effect is significant while the other is near zero, the feature may still exhibit causal influence. Simply setting an effect that is near zero or negative to 0 would result in an FIC score of 0, which does not adequately capture the underlying causality.

To address this, we introduce a penalty coefficient \(w\) to adjust for negative or near-zero effects. Specifically, we define the penalized effect \(E'\) for each intervention as follows:
\[
E' =
\begin{cases}
E, & \text{if } E \geq 0, \\
w \cdot |E|, & \text{if } E < 0.
\end{cases}
\]
Here, \(w\) is empirically set to 0.5. Thus, if one of the normalized effects (either \(E_{abl}\) or \(E_{enh}\)) is negative, we compute its penalized value as \(0.5\) times its absolute value rather than setting it directly to 0. This approach ensures that even when one of the effects is weak or slightly negative, the FIC score does not vanish entirely, preserving the indication of causality.

Accordingly, the FIC score is then computed as:
\[
FIC = \frac{2\, E_{abl}'\, E_{enh}'}{E_{abl}' + E_{enh}'}.
\]

In our experiments (see Table\ref{tab:main_intervention}), only the metaphor feature shows a slightly negative ablation effect, while the enhancement and ablation effects for the other features are positive. The introduction of the penalty coefficient \(w\) effectively moderates the impact of the negative effect for the metaphor feature, resulting in a more balanced and meaningful FIC score.

This penalty mechanism is crucial because even when only one of the interventions (ablation or enhancement) shows a significant effect, it still provides evidence of the feature’s causal role. By incorporating \(w\), we ensure that such cases are not misrepresented by an FIC score of 0, thus offering a more robust measure of the overall causal strength.