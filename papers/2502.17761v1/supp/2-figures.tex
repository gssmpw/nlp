
\begin{figure*}[!ht]
\centering  
\includegraphics[width=\textwidth]{figures/EDF1_gene_ablation.pdf}
\caption{\textbf{$\ours$ ST prediction analysis on additional gene sets}. In addition to the analysis for 250 highly-expressed genes (HEG) in \textbf{Figure~\ref{fig:prostate}}, we analyze $\ours$ for gene sets with \textbf{(a)} 1,000 HEG and \textbf{(b)} 250 highly variable genes (HVG) over three different scenarios. Error bars indicate one standard deviation from the mean, over ten sections across five patients. In addition to the analysis for 250 highly-expressed genes (HEG) in \textbf{Extended Figure \ref{fig:breast_crc}}, we analyze $\ours$ for gene sets with 1,000 HEG over two different scenarios for \textbf{(c)} the breast cancer cohort and \textbf{(d)} the colorectal cancer cohort. Statistical significance was assessed with the Wilcoxon signed-rank test. $^{\ast\ast}p\leq 0.01$, $^{\ast\ast\ast}p\leq 0.001$, $^{\ast\ast\ast\ast}p\leq 0.0001$. PCC: Pearson Correlation Coefficient. SSIM: Structural Similarity Index Measure.}
\label{fig:ext_1000HEG_and_HVG}
\end{figure*}


\clearpage

\begin{figure*}[!ht]
\centering  
\includegraphics[width=\textwidth]{figures/EDF_variance.pdf}
\caption{\textbf{$\ours$ ST prediction analysis on gene expression variance}. The correlation Spearman's $\rho$ between the variance of $\ours$-predicted expression levels (orange) and the variance of measured ST expression levels (blue) across all Visium ST spots in each tissue section (refer to \textbf{Online Methods} in section \textbf{ST spot filtering and expression normalization}). Genes are ranked based on measured ST expression variance, from the smallest to the largest. The variance measures are shown across three different scenarios for three exemplar sections.}
\label{fig:ext_variance_ablation}
\end{figure*}



\clearpage

\begin{figure*}[!ht]
\centering  
\includegraphics[width=\textwidth]{figures/Extended_Prostate_3D.pdf}
\caption{\textbf{$\ours$ analysis on prostate cancer cohort}. The predicted 3D ST heatmaps for four representative genes (\textit{SPON2}, \textit{ACTA2}, \textit{MSMB}, and \textit{EPCAM}) along with 3D tissue images captured with microCT, 3D morphological segmentation, and representative
2D H\&E-stained histological sections for each of the two patients on which Visium ST is measured. The bottom row for each patient shows the measured ST expression and $\ours$-predicted expression (\textit{3D+VOI} setting) along with Pearson Correlation Coefficient (PCC) for evaluating prediction capacity. During \textit{VOI} fine-tuning, ST data from one representative section is first used for training while evaluation is performed on the second section. The process is then reversed, with the second plane used for training and the first plane for evaluation. We observe that \textit{SPON2} and \textit{EPCAM} genes are overexpressed in tumor regions, \textit{MSMB} gene is downregulated in prostatic tumor glands compared to benign glands, and \textit{ACTA2} gene is overexpressed in stromal tissue regions, aligning with previous findings in literature\cite{berglund2018spatial,song2022single}. Examples for P1 are in \textbf{Figure~\ref{fig:prostate}}. Scalebar is 1 mm.}
\label{fig:ext_prostate_3D}
\end{figure*}


\clearpage

\begin{figure*}[!ht]
\centering  
\includegraphics[width=\textwidth]{figures/EDF_BCR.pdf}
\caption{\textbf{$\ours$ captures inter-tumoral heterogeneity}. 3D ST prediction by VORTEX of \textit{AZGP1} gene on three samples with different biochemical recurrence (BCR) status. \textit{AZGP1} downregulation in prostate adenocarcinoma is associated with shorter time to BCR\cite{kristensen2019predictive,burdelski2016reduced}. 
$\ours$ captures the inter-tumoral heterogeneity that agrees with patient BCR status and predicts high expression of \textit{AZGP1} in tumoral regions for low-risk sample P1, and low expression for high-risk samples P3 and P4. Scalebar is 1 mm.}
\label{fig:ext_intertumoral}
\end{figure*}





\clearpage
\begin{figure*}[!ht]
\centering  
\includegraphics[width=\textwidth]{figures/EDF_spatial_clustering.pdf}
\caption{\textbf{3D Spatial Domain identification with $\ours$}. Spatial domains across the tissue volumes for two patients (P1 and P3). The \textit{3D + VOI} setting shows higher degree of agreement with the manually annotated morphology by a pathologist. Adjusted Rand Index (ARI) scores are displayed. Scalebar is 1~mm.}
\label{fig:ext_spat_cluster}
\end{figure*}

\clearpage

\begin{figure*}[!ht]
\centering  
\includegraphics[width=\textwidth]{figures/EDF_largeFOV.pdf}
\caption{\textbf{$\ours$ on large prostate cancer tissue}. 
3D ST prediction by $\ours$ on large prostate cancer tissue volumes for \textit{MSMB} and \textit{AMACR} genes in sample P3 and 
\textit{SPON2} and \textit{MSMB} in sample P4. Cross-sections at different depths are shown, along with the spatial domains identified by $\ours$. In P3, spatial domain (S.D.) 1 predominantly corresponds to adenocarcinoma, S.D. 2 to hyperplastic benign glands, S.D. 3 to stroma, and S.D. 4 to luminal areas of benign glands with cystic change and adventitia. In P4, S.D. 1 predominantly corresponds to adenocarcinoma, S.D. 2 to benign prostatic glands, S.D. 3 to intratumoral stroma and S.D. 4 to luminal areas and tissue edges. Scalebar is 1 mm.}
\label{fig:ext_prostate_3D_largeFOV}
\end{figure*}

\clearpage
\begin{figure*}[!ht]
\centering  
\includegraphics[width=0.9\textwidth]{figures/Extended_OTLS.pdf}
\caption{\textbf{$\ours$ with prostate cancer sample imaged with open-top lightsheet microscopy (OTLS)}. Simulated prostate core needle biopsy was imaged with OTLS at $1 \mu m$/voxel, which was converted to provide H\&E-like appearance of the 3D tissue sample. $\ours$ pretrained on 2D H\&E image and ST pairs is applied to OTLS. The agreement with the tumoral and stromal regions with up-regulation of \textit{KLK3} and \textit{COL1A1} respectively demonstrates its generalizability across imaging modalities.}
\label{fig:ext_otls}
\end{figure*}

\clearpage

\begin{figure*}[]
\centering
\includegraphics[width=\textwidth]{figures/Fig_BREAST_CRC.pdf}
\caption{\textbf{$\ours$ analysis on breast and colorectal cancer cohort}. 
\textbf{(a)} H\&E tissue images and ST are obtained from serial tissue sections sampled from breast cancer volumes. 2.5D tissue image is constructed by registering serial sections\cite{gatenbee2023virtual}.
\textbf{(b)} Schematic for evaluation of performance change with increasing ST sections within VOI.
\textbf{(c)} PCC and SSIM between the predicted and the measured ST expression for four breast cancer patients. For fine-tuning with VOI, performance is shown over an increasing number of sections with ST measurements without cohort-level pretraining (yellow) and with pretraining (red) as illustrated in (b).
\textbf{(d)} PCC and SSIM between the predicted and the measured ST expression for three gene sets averaged across six colorectal cancer patients with two sections each.
\textbf{(e)} 2.5D ST heatmap of \textit{ESR1} with measured expression, H\&E, and morphology segmentation.
\textbf{(f)} H\&E, measured, and predicted \textit{EpCAM} expression from two CRC samples. PCC between the measured and predicted values is displayed. 
\textbf{(g)} 2.5D ST heatmaps for large colorectal cancer volume with 22 serial sections\cite{lin2023multiplexed}, morphology segmentation, spatial domains, and predicted expression profiles. Additional examples can be found in \textbf{Extended Data Figure~\ref{fig:ext_CRC_3D}}. 
\textbf{(h)} Zoomed-in region with `cord-like' structure of stroma (left half) and normal mucosa (right half). S: Section. PCC: Pearson correlation coefficient. SSIM: Structural Similarity Index Measure. VOI: Volume of interest.}
\label{fig:breast_crc}
\end{figure*} 

\clearpage


% Extended Figure CRC
\begin{figure*}[!ht]
\centering  
\includegraphics[width=\textwidth]{figures/Extended_CRC_3D.pdf}
\caption{\textbf{$\ours$ analysis on breast and colorectal cancer cohorts: additional visualizations}. \textbf{(a)} 2.5D ST heatmaps of predicted and measured gene \textit{COX6C}, which plays a crucial role in
the identification of hormone-responsive breast cancer\cite{west2001predicting}. The four central sections, with 2.5D morphological context (three sections) comprised of a section of interest and a neighboring section above and below, out of the six total planes for each patient are shown.
\textbf{(b)} 2.5D ST heatmaps of predicted and measured gene \textit{CEACAM5} and \textit{KRT8}, which are upregulated in tumoral tissue compared to normal colonic mucosa\cite{xiao2024integrating}. PCC between the measured and the predicted ST expressions is displayed. \textbf{(c)} 2.5D ST heatmaps obtained with $\ours$ for publicly-available CRC sample with 22 serial tissue sections and tissue segmentation with representative 2D axial section. 
Unless specified otherwise, scalebar is 1 mm. PCC: Pearson correlation coefficient. Morph. segmentation: morphology segmentation.}
\label{fig:ext_CRC_3D}
\end{figure*}

\clearpage

\begin{figure*}[!ht]
\centering  
\includegraphics[width=\textwidth]{figures/EDF_CRC_queries.pdf}
\caption{\textbf{Cross-modal morphology retrieval with $\ours$ on colorectal cancer cohort.} \textbf{(a)} Molecular query analysis for large 2.5D colorectal cancer tissue volume.
We design two molecular queries defined by up-regulation of \textit{EPCAM}, \textit{KRT8}, \textit{CEACAM5} and up-regulation of \textit{COL1A1}, \textit{SPARC}, \textit{S100A4}. The first and second molecular queries largely correspond to adenocarcinoma and stroma, respectively, as can be seen with the complementary similarity heatmap around adenocarcinoma and the top similar patches. Additionally, we introduce a third molecular query as a control for the first query, by using low-expression of \textit{EPCAM}, \textit{KRT8}, \textit{CEACAM5}. \textbf{(b)} Close-up view of an image region containing colon adenocarcinoma, mucin, and stroma. Complementary heatmap of molecular query 1 and 2 is observed between adenocarcinoma and surrounding stroma. \textbf{(c)} Examples of top similar and dissimilar 2D patches for each molecular query are shown. For molecular query 1, tumor patches characterize the high similarity group while low similarity is observed in patches containing stroma and smooth muscle (muscularis propria). For molecular query 2, stroma patches represent the high similarity group while other tissue components such as adenocarcinoma and muscularis propria constitute the low similarity set. For molecular query 3 (control for molecular query 1), high similarity patches are composed of heterogeneous morphology, while tumor and secreted mucin constitute the low similarity group. Unless specified otherwise, scalebar is 1 mm.
}
\label{fig:ext_CRC_query}
\end{figure*}
