

\begin{table}[h]
    \centering
    \scalebox{0.7}{
    \begin{tabular}{|l|c|}
        \hline
        \multirow{2}{*}{\textbf{Condition}} & 
        \textbf{Pearson's Correlation with} \\
        & \textbf{Keeping the Privacy Choice (r-value)} \\ \hline
        Participant gets a favorable outcome in the first round of the game. & \textbf{.375} \\ \hline

        Participant's choice is not supported by the majority of the group. & -.217 \\ \hline
        
        Participant selects ``Not Share'' in the first round of the game. & .153 \\ \hline
        
        Participant initially tend to ``Not Share'' based on Likert score. & .141 \\ \hline
        
        \end{tabular}}
    \caption{Pearson's Correlations for different conditions in scenarios compared to keeping the privacy choice. All of the listed correlations are weak (|r-value| < .4) but significant with p-value < .05.}
    \label{tab:pearson_s_r}
    \Description{Pearson's r value for different factors in scenarios compared to keeping the privacy choice. The strongest correlation is with the condition of ``Participant gets a favorable outcome in the first round of the game.'' with the r-value of .375. Remaining correlations are listed below:
    The condition: ``Participant's choice is not supported by the majority of the group.'' - r-value: -.217.
    The condition: ``Participant selects ``Not Share'' in the first round of the game.'' - r-value: .153.
    The condition: ``Participant initially tend to ``Not Share'' based on Likert score.'' - r-value: .141.
    All correlations are weak with the absolute values of r being less than .4 but also significant with p-values being less than .05.}
\end{table}

