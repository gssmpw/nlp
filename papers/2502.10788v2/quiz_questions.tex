\subsection{Quiz}

As mentioned before, questions are displayed in a sample scenario with a sample group photo including the participant, as in the case of actual experiment. Agents in the given scenario are named as Participant \#2, Participant \#3 and Participant \#4 so that the actual participant can consider them as peers.

Regardless its type (interactive task or multiple choice), all possible phrases of feedback to the participant for a question is listed here as well. For the first two questions, more than one phrase can be displayed to the participant, if their actions in the given task include more than one mistake. Feedback that is associated with the correct option (or the correct action in the interactive questions) is marked with `*', as well as the correct option.  

%%%% Q1 %%%%
\mypara{\textbf{Question 1:}}


\underline{Your training task is:} \\

To pick a preference of \textbf{NOT SHARE}

and to enter an amount of money \textbf{higher than or equal to 11.}


\mypara{\textbf{Feedback 1.1:}} You have not entered a decision!

\mypara{\textbf{Feedback 1.2:}} You have not entered an amount of credits!

\mypara{\textbf{Feedback 1.3:}} You have entered ``SHARE'' instead of ``NOT SHARE'' as your decision. The instructions stated to enter ``NOT SHARE''.

\mypara{\textbf{Feedback 1.4:}} You have not entered the correct amount of credits that was specified in the instructions.

\mypara{\textbf{Feedback 1.5*:}} Correct! You have correctly entered ``NOT SHARE'' as your decision and correctly entered the amount of credits.

%%%% Q2 %%%%
\mypara{\textbf{Question 2:}}


\underline{Your training task is:} \\

To pick a preference of \textbf{SHARE}

and to enter an amount of money \textbf{lower than or equal to 7.}

\clearpage
\mypara{\textbf{Feedback 2.1:}} You have not entered a decision!

\mypara{\textbf{Feedback 2.2:}} You have not entered an amount of credits!

\mypara{\textbf{Feedback 2.3:}} You have entered ``NOT SHARE'' instead of ``SHARE'' as your decision. The instructions stated to enter ``SHARE''.

\mypara{\textbf{Feedback 2.4:}} You have not entered the correct amount of credits that was specified in the instructions.

\mypara{\textbf{Feedback 2.5*:}} Correct! You have correctly entered ``SHARE'' as your decision and correctly entered the amount of credits.



%%%% Q3 %%%%
\mypara{\textbf{Question 3:}}

\underline{Your training task is:} 

To specify what the outcome of the scenario is given all the participants sharing preferences and amount of credits entered.

\textbf{Your Choices:} \\
\indent \indent Sharing reference: 	``NOT SHARE'' \\
\indent \indent Amount:		13 

\textbf{Participant \#2:} \\ 
\indent \indent Sharing preference:	``SHARE'' \\
\indent \indent Amount: 		10

\textbf{Participant \#3:} \\
\indent \indent Sharing preference:	``SHARE'' \\
\indent \indent Amount: 		10

\textbf{Participant \#4:} \\
\indent \indent Sharing preference:	``NOT SHARE'' \\
\indent \indent Amount: 		10 \\ 

What will be the outcome of the scenario? \\
\indent \textbf{a)} SHARE \indent \indent \textbf{b*)} NOT SHARE 

\mypara{\textbf{Feedback 3.1:}} The outcome of the scenario should have been ``NOT SHARE'' because the amount of credits entered by the participants that selected ``NOT SHARE'' is greater than the amount of credits entered by participants selecting ``SHARE''.

\mypara{\textbf{Feedback 3.2*:}} Correct! You have correctly selected the outcome of the scenario. The amount of credits entered by the participants that selected ``NOT SHARE'' is greater than the amount of credits entered by participants selecting ``SHARE''.


%%%% Q4 %%%%
\mypara{\textbf{Question 4:}}

\underline{Your training task is:} 

To specify what the outcome of the scenario is given all the participants sharing preferences and amount of credits entered.

\clearpage
\textbf{Your Choices:} \\
\indent \indent Sharing reference: 	``SHARE'' \\
\indent \indent Amount:		5 

\textbf{Participant \#2:} \\ 
\indent \indent Sharing preference:	``SHARE'' \\
\indent \indent Amount: 		5

\textbf{Participant \#3:} \\
\indent \indent Sharing preference:	``SHARE'' \\
\indent \indent Amount: 		5

\textbf{Participant \#4:} \\
\indent \indent Sharing preference:	``NOT SHARE'' \\
\indent \indent Amount: 		20  

What will be the outcome of the scenario? \\
\indent \textbf{a)} SHARE \indent \indent \textbf{b*)} NOT SHARE 

\mypara{\textbf{Feedback 4.1:}} The outcome of the scenario should have been ``NOT SHARE'' because the amount of credits entered by the participants that selected ``NOT SHARE'' is greater than the amount of credits entered by participants selecting ``SHARE''.

\mypara{\textbf{Feedback 4.2*:}} Correct! You have correctly selected the outcome of the scenario. The amount of credits entered by the participants that selected ``NOT SHARE'' is greater than the amount of credits entered by participants selecting ``SHARE''.



%%%% Q5 %%%%
\mypara{\textbf{Question 5:}}

\underline{Your training task is:}

To specify the amount of money that you will pay in this scenario. 

In the actual experiment, the amount you enter will be deducted from your available budget in case your sharing decision is the favorable outcome.

\textbf{REMEMBER:  If your decision changes the outcome, you will be charged an additional 5 as tax.}

\textbf{Your Choices:} \\
\indent \indent Sharing reference: 	``NOT SHARE'' \\
\indent \indent Amount:		15 

\textbf{Participant \#2:} \\ 
\indent \indent Sharing preference:	``SHARE'' \\
\indent \indent Amount: 		10

\textbf{Participant \#3:} \\
\indent \indent Sharing preference:	``SHARE'' \\
\indent \indent Amount: 		10

\textbf{Participant \#4:} \\
\indent \indent Sharing preference:	``NOT SHARE'' \\
\indent \indent Amount: 		10  

How much would you pay in this situation? \\
\indent \textbf{a)} 0 \indent \indent \textbf{b)} 10
\indent \indent \textbf{c)} 15
\indent \indent \textbf{d*)} 20

\clearpage
\mypara{\textbf{Feedback 5.1:}} Selecting 0 is never an option, since the scenario's outcome aligned with your sharing preference, you will at least always pay the amount of credits you have specified.

\mypara{\textbf{Feedback 5.2:}} This amount is not in line with the amount that you are supposedly entered (Participant \#1). Please make sure you understand how to participate in this study prior to taking part.

\mypara{\textbf{Feedback 5.3:}} The amount you entered affects the decision of the outcome. This means that without your amount of credits the opposite preference would have prevailed. In this case you will pay an additional 5 from your budget. Please make sure you understand how to participate in this study prior to taking part.

\mypara{\textbf{Feedback 5.4*:}} Correct! You are correct, you will pay 20 from your budget. Since the amount you entered affected the outcome of the scenario you will be charged an additional amount (5) in this case.




%%%% Q6 %%%%
\mypara{\textbf{Question 6:}}

\underline{Your training task is:} 

To specify the amount of money that you will pay in this scenario. 

In the actual experiment, the amount you enter will be deducted from your available budget in case your sharing decision is the favorable outcome.

\textbf{REMEMBER:  If your decision changes the outcome, you will be charged an additional 5 as tax.}

\textbf{Your Choices:} \\
\indent \indent Sharing reference: 	``NOT SHARE'' \\
\indent \indent Amount:		9 

\textbf{Participant \#2:} \\ 
\indent \indent Sharing preference:	``SHARE'' \\
\indent \indent Amount: 		3

\textbf{Participant \#3:} \\
\indent \indent Sharing preference:	``SHARE'' \\
\indent \indent Amount: 		5

\textbf{Participant \#4:} \\
\indent \indent Sharing preference:	``NOT SHARE'' \\
\indent \indent Amount: 		10  

How much would you pay in this situation? \\
\indent \textbf{a)} 0 \indent \indent \textbf{b*)} 9
\indent \indent \textbf{c)} 14
\indent \indent \textbf{d)} 19

\mypara{\textbf{Feedback 6.1:}} Selecting 0 is never an option, since the scenario's outcome aligned with your sharing preference, you will at least always pay the amount of credits you have specified.

\mypara{\textbf{Feedback 6.2*:}} Correct! You are correct, you will pay 9 from your budget. Since the credits you placed for your sharing decision did not affect the outcome of the scenario you will not be charged an additional amount in this case.

\mypara{\textbf{Feedback 6.3:}} You will not pay 14 for this scenario. The amount of credits you entered did not affect the outcome hence you will not be charged the extra 5 credits. You would only have paid the additional credits if participant \#4 had entered an amount lower than 8. In that case since your amount would have been higher on its own compared to the SHARE preference amount, you would have been taxed.

\clearpage
\mypara{\textbf{Feedback 6.4:}} This amount is not in line with the amount that you are supposedly entered (Participant \#1). Please make sure you understand how to participate in this study prior to taking part.


%%%% Q7 %%%%
\mypara{\textbf{Question 7:}}

\underline{Your training task is:} 

To specify the amount of money that you will pay in this scenario. 

In the actual experiment, the amount you enter will be deducted from your available budget in case your sharing decision is the favorable outcome.

\textbf{REMEMBER:  If your decision changes the outcome, you will be charged an additional 5 as tax.}

\textbf{Your Choices:} \\
\indent \indent Sharing reference: 	``SHARE'' \\
\indent \indent Amount:		8 

\textbf{Participant \#2:} \\ 
\indent \indent Sharing preference:	``SHARE'' \\
\indent \indent Amount: 		9

\textbf{Participant \#3:} \\
\indent \indent Sharing preference:	``NOT SHARE'' \\
\indent \indent Amount: 		14

\textbf{Participant \#4:} \\
\indent \indent Sharing preference:	``NOT SHARE'' \\
\indent \indent Amount: 		7 

How much would you pay in this situation? \\
\indent \textbf{a*)} 0 \indent \indent \textbf{b)} 8
\indent \indent \textbf{c)} 13
\indent \indent \textbf{d)} 20


\mypara{\textbf{Feedback 7.1*:}} Correct! You will not pay anything for this scenario since even though you entered 8, the decision you selected is not the outcome of the scenario.

\mypara{\textbf{Feedback 7.2:}} You will not pay anything for this scenario since even though you entered 8, the decision you selected is not the outcome of the scenario.

\mypara{\textbf{Feedback 7.3:}} You will not pay anything for this scenario. REMEMBER: only if the scenario outcome aligns with your preference you pay the amount of credits entered and you only pay the extra 5 credits if the amount of credits you enter affects the outcome. Neither of these happened in this scenario. Please make sure you understand how to participate in this study prior to taking part.

\mypara{\textbf{Feedback 7.4:}} You have not entered this amount for the decision, therefore you are not paying the amount you have selected.


%%%% Q8 %%%%
\mypara{\textbf{Question 8:}}

\underline{Your training task is:}  

To specify the amount of money that you will pay in this scenario. 

In the actual experiment, the amount you enter will be deducted from your available budget in case your sharing decision is the favorable outcome.

\textbf{REMEMBER:  If your decision changes the outcome, you will be charged an additional 5 as tax.}
\clearpage

\textbf{Your Choices:} \\
\indent \indent Sharing reference: 	``NOT SHARE'' \\
\indent \indent Amount:		20 

\textbf{Participant \#2:} \\ 
\indent \indent Sharing preference:	``SHARE'' \\
\indent \indent Amount: 		9

\textbf{Participant \#3:} \\
\indent \indent Sharing preference:	``SHARE'' \\
\indent \indent Amount: 		7

\textbf{Participant \#4:} \\
\indent \indent Sharing preference:	``SHARE'' \\
\indent \indent Amount: 		7  

How much would you pay in this situation? \\
\indent \textbf{a*)} 0 \indent \indent \textbf{b)} 7
\indent \indent \textbf{c)} 9
\indent \indent \textbf{d)} 20

\mypara{\textbf{Feedback 8.1*:}} Correct! You will not pay anything for this scenario since even though you entered 20, the decision you selected is not the outcome of the scenario. 

\mypara{\textbf{Feedback 8.2:}} You have not entered this amount for the decision, therefore you are not paying the amount you have selected.

\mypara{\textbf{Feedback 8.3:}} You will not pay anything for this scenario since even though you entered 20, the decision you selected is not the outcome of the scenario.


%%%% Q9 %%%%
\mypara{\textbf{Question 9:}}

\underline{Your training task is:} 

To select who would get the highest reward given the information.

\textbf{REMEMBER: If your sharing decision is the favorable outcome you earn a reward.}

\textbf{The total reward at the end of the experiment will be converted to lottery tickets to win an Amazon  voucher worth €20.}

\begin{table}[H]
    \centering
    \scalebox{0.9}{\begin{tabular}{|c|c|c|c|}
    \hline
    \textbf{Participant}& \textbf{Scenarios Participated} & \textbf{Favorable Outcomes} & \textbf{Budget} \\ \hline

    You & 16 & 13 & 20 \\ \hline

    Participant \#2 & 16 & 8 & 20 \\ \hline

    Participant \#3 & 16 & 4 & 20 \\ \hline

    Participant \#4 & 16 & 11 & 20 \\ \hline
    \end{tabular}}\label{tab:q9}
\end{table}

Who would get the highest reward? \\
\indent \textbf{a*)} You \indent \indent \textbf{b)} Participant \#2
\indent \indent \textbf{c)} Participant \#3
\indent \indent \textbf{d)} Participant \#4

\mypara{\textbf{Feedback 9.1*:}} Correct! This is correct. Everyone saved the same amount of budget, but you had more favorable outcomes than the others. Therefore in this scenario you would get the highest reward. 

\mypara{\textbf{Feedback 9.2:}} This is incorrect. Everyone saved the same amount of budget, but you had more favorable outcomes than the others. Therefore in this scenario you would get the highest reward, not the other participants.

\clearpage
%%%% Q10 %%%%
\mypara{\textbf{Question 10:}}

\underline{Your training task is:} 

To select who would get the highest reward given the information.

\textbf{REMEMBER: If your sharing decision is the favorable outcome you earn a reward.}

\textbf{The total reward at the end of the experiment will be converted to lottery tickets to win an Amazon  voucher worth €20.}

\begin{table}[H]
    \centering
    \scalebox{0.9}{
    \begin{tabular}{|c|c|c|c|}
    \hline
    \textbf{Participant}& \textbf{Scenarios Participated} & \textbf{Favorable Outcomes} & \textbf{Budget} \\ \hline

    You & 16 & 9 & 22 \\ \hline

    Participant \#2 & 16 & 9 & 43 \\ \hline

    Participant \#3 & 16 & 9 & 37 \\ \hline

    Participant \#4 & 16 & 9 & 42 \\ \hline
    \end{tabular}}\label{tab:q10}
\end{table}

Who would get the highest reward? \\
\indent \textbf{a)} You \indent \indent \textbf{b*)} Participant \#2
\indent \indent \textbf{c)} Participant \#3
\indent \indent \textbf{d)} Participant \#4

\mypara{\textbf{Feedback 10.1:}} This is incorrect. While everyone had the same number of favorable outcomes, Participant \#2 has saved the most budget, therefore gets the highest reward.

\mypara{\textbf{Feedback 10.2*:}} Correct! This is correct. While everyone had the same number of favorable outcomes, Participant \#2 has saved the most budget, therefore gets the highest reward.

