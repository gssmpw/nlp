\subsection{Slides}

\mypara{\textbf{Page 1:}}

In the following pages, we will show you examples of scenarios that you will be presented as part of this research in order to familiarize you with the experiment.

\textcolor{red}{PLEASE PAY ATTENTION!}

\vspace{5mm}
\mypara{\textbf{Page 2:}}

\textcolor{red}{PLEASE PAY ATTENTION:} We will first explain how the experiment works, and then we will ask you questions to see if you have understood various aspects of the experiment.

\textcolor{red}{PLEASE BE AWARE:} If you make errors during the training, you will have a chance to repeat it. However, in order to participate in the remainder of the study and receive payment you will have to pass the quiz in the training.


\vspace{5mm}
\mypara{\textbf{Page 3:}} 
\begin{figure}[H]    \includegraphics[scale=0.4, left]{TrainingSlide_Page3.jpg}
    \label{fig:ts_page3}
\end{figure}

\clearpage
\mypara{\textbf{Page 4:}}

\textbf{FOR EACH SCENARIO WE ARE ASKING YOU TO IMAGINE THIS SITUATION:}

You (yourself and your colleagues) have taken a group picture and need to decide if you want it to be posted online. To reach an outcome, we are asking you to assign a preference (share or not share) and a value (in credits) to support your decision. 

After you all have made your individual decision, you will be shown each other’s sharing preferences and whether or not your preference will be the favorable outcome.

Look at your colleagues’ preferences compared to yours. You can now choose to change or keep your decision and the associated credits based on that.

\textcolor{red}{PLEASE BE AWARE that the decision at the second round will be final whether your choice is favorable or not.}

\mypara{\textbf{Page 5:}}

Your task is to decide whether to \textbf{SHARE} or \textbf{NOT SHARE} all the images in this scenario. You have an amount of credits (budget) that you can use to reinforce your decision and enter an amount to do so.  The credit(s) you can use in the scenarios come from your budget. For this experiment, we give 50 credits as the initial budget that you can spend from, and for each scenario you will get an additional 10.

The credit(s) you place on an outcome reflects how much the decision matters to you. You will be able to place an amount between 1 and 20 for a decision, as long as you have enough budget to do so.

Enter the amount you want to place and then select your sharing decision. Once you are certain of your choices you can submit.

\mypara{\textbf{Page 6:}}
\begin{figure}[H]    \includegraphics[scale=0.36, left]{TrainingSlide_Page6.jpg}
    \label{fig:ts_page6}
\end{figure}

\clearpage

\mypara{\textbf{Page 7:}}
\begin{figure}[H]    \includegraphics[scale=0.36, left]{TrainingSlide_Page7.jpg}
    \label{fig:ts_page7}
\end{figure}

\mypara{\textbf{Page 8:}}
\begin{figure}[H]    \includegraphics[scale=0.36, left]{TrainingSlide_Page8.jpg}
    \label{fig:ts_page8}
\end{figure}


\mypara{\textbf{Page 9:}}
\begin{figure}[H]    \includegraphics[scale=0.36, left]{TrainingSlide_Page9.jpg}
    \label{fig:ts_page9}
\end{figure}

\mypara{\textbf{Page 10:}}
\begin{figure}[H]    \includegraphics[scale=0.36, left]{TrainingSlide_Page10.jpg}
    \label{fig:ts_page10}
\end{figure}


\mypara{\textbf{Page 11:}}
\begin{figure}[H]    \includegraphics[scale=0.36, left]{TrainingSlide_Page11.jpg}
    \label{fig:ts_page11}
\end{figure}

\vspace{-10mm}
\mypara{\textbf{Page 12:}}

You will be taken to another page where all the other participants’ sharing preferences will be shown. In this stage you can either decide to keep or change your selection. At this time, the amounts for \textbf{SHARE} and \textbf{NOT SHARE} will be added individually. The decision with the highest total will be the outcome.


\mypara{\textbf{Page 13:}}

Please be aware that if your decision does not match the final outcome (i.e. your sharing decision is not the favorable outcome), you will not be charged. If it does (i.e. your sharing decision is the favorable outcome), you will be charged. Your sharing decision may be the favorable outcome because you have placed more credits than all the other participants for an outcome or that it could be favorable as part of the consensus.

For example: \\
\vspace{-3mm}

You - 5 Not Share \\
\indent Participant \#2 - 5 Share \\
\indent Participant \#3 - 10 Share \\
\indent Participant \#4 - 5 Not Share \\

\vspace{-3mm}
You will not be charged. \\

\vspace{-3mm}
You - 5 Share \\
\indent Participant \#2 - 5 Share \\
\indent Participant \#3 - 10 Share \\
\indent Participant \#4 - 5 Not Share \\

\vspace{-3mm}
You will be charged.

\clearpage
\mypara{\textbf{Page 14:}}

\textbf{How much will you be charged?}

If your sharing decision is the favorable outcome but your amount did not affect the outcome of the scenario (e.g. the outcome would have been the same without you), you will only be charged the amount you have specified.

For example: \\

You - 5 Not Share \\
\indent Participant \#2 - 5 Share \\
\indent Participant \#3 - 5 Share \\
\indent Participant \#4 - 15 Not Share \\

You will be charged 5. \\

However, if your sharing decision is the favorable outcome and your amount did affect the outcome an extra 5 will be deducted from your budget. This is what we call a tax.

For example: \\

You - 15 Share \\
\indent Participant \#2 - 5 Share \\
\indent Participant \#3 - 5 Share \\
\indent Participant \#4 - 15 Not Share \\

Here you will be charged the amount you entered (10) and also an extra amount of 5 (tax) since without your amount the outcome would have been different. Hence, you will be charged 15 in total.

\mypara{\textbf{Page 15:}}
\textbf{How much will you gain?}

When you finish this study, you will be rewarded a minimum of €5. It is possible to gain an extra payment. If your sharing decision is the favorable outcome without changing your sharing preferences you will be awarded an additional 0.5 reward. You still get the same reward if you only change the amount you entered without changing your sharing preferences and your sharing decision is the favorable outcome.

For example: \\

\textit{First Round:} \\

You - 5 Share \\
\indent Participant \#2 - 5 Share \\
\indent Participant \#3 - 10 Share \\
\indent Participant \#4 - 5 Not  Share \\

\textit{Second Round:} \\

You - 5 Share \\
\indent Participant \#2 - 5 Share \\
\indent Participant \#3 - 10 Share \\
\indent Participant \#4 - 5 Not  Share \\

Here you will get a reward of 0.5. \\

The amount of credits left in your budget and in the extra reward will be converted to additional payment. For each 10 you save at the end of the experiment from your budget, you will get an additional 0.1 reward (e.g 10 -> 0.1; 15 -> 0.1; 23 -> 0.2). The total reward at the end of the experiment will be converted to lottery tickets to win an Amazon voucher worth €20.














