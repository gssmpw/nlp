\section{Introduction} 

\begin{figure*}
    \centering
    %\includegraphics[width=\linewidth]{images/overview.png}
    \includegraphics[width=.8\linewidth,trim=0.3in 1.4in 0.3in 1.4in]{images/tinytenants_teaser}
    \caption{A sequence diagram illustrating a conversation in \asyncfw{} framework, where Alice issues a request to her agent. Documents available to Alice's agent are insufficient to answer the user request. The agent uses a people search tool, after which it decides what subset of people to contact, in which order, what questions to pose, etc. The temporal ordering of tool calls and message exchanges is denoted by \#i. }
    \label{fig:intro}
\end{figure*}


% Paragraph 1 -- general
% \emph{-- General discussions, use-cases and promise of bot-mediated collaboration -- }
In today's fast-paced and interconnected world, effective collaboration is essential for achieving complex tasks and making informed decisions \cite{papachristou2023leveraging,gemp2024states}. 
Many decision-making, content creation, and information-gathering tasks require collecting information from multiple people. For example, preparing a list of interns across teams in an organization by reaching out to the leader of each team; preparing a newsletter for project updates might necessitate coordinating with multiple contributors; identifying a suitable time to meet might require several rounds of negotiations \cite{lin2024decision}. Identifying what information is available, judiciously determining who to contact, asking precise questions, and compiling research results can be a challenging and time-consuming process---especially when real-time interaction between team members is difficult to coordinate.

At the same time, recent large language models (LLMs), such as GPT-4 \cite{openai2023gpt4}, Phi-3 \cite{abdin2024phi}, LLaMa \cite{touvron2023llama}, and Gemini \cite{team2023gemini}, are becoming a crucial building block in developing automated agents that can assist human users with complex tasks
\cite{xi2023rise,wang2024survey,butler2023microsoft}.
%Such LLMs are increasingly being used to 
These tasks include chat applications for assisting individual users with searching and summarizing information (such as in Microsoft Copilot Chat\footnote{\url{https://copilot.microsoft.com/}}), and even supporting these users in workplace decision-making \citep{butler2023microsoft,kim2024leveraginglargelanguagemodels}.
Could these agents be extended to improve collaboration among multiple users?

In this paper, we introduce
\textbf{\asyncfw{}},
an evaluation framework for studying effectiveness of LLM-powered agents to assist with multi-user collaboration tasks.
Each \asyncfw{} task takes place within a fictitious organization with 2--20 employees, some of whom possess a collection of documents necessary to solve some task.
One of the users (the \emph{initiating user}) communicates the task to an \emph{agent} (Fig.~\ref{fig:intro}).
Agents have direct access to the initiating user's documents, and can engage in conversations with other users to gather relevant information. 
They must rely on limited descriptions of other users, and potentially previous interactions, to determine who to contact for a given task.
\asyncfw{} comprises two families of tasks: \textbf{\dataspider{}} and \textbf{\datanews{}}, derived from the \textsc{spider} \citep{yu2018spider} and \textsc{multinews} \citep{fabbri2019multi} datasets respectively. It evaluates 
agents' ability to answer questions involving complex relational reasoning and document summarization.

Our initial benchmark release also includes reference agent implementations based on popular
prompting and orchestration strategies, and a suite of evaluation metrics.
We report evaluation results using Phi-3-medium \cite{abdin2024phi}, GPT-4-turbo and GPT-4o \cite{openai2023gpt4} language models  to implement these agents. Our results indicate that LM-powered agents can struggle to coordinate with multiple users to correctly address information seeking and document authoring requests. Major research questions remain around how to optimally determine which people to contact and in what order, how to ask high-quality questions, and how to learn and adapt to the structure of an organization. 
\asyncfw{} thus provide a test-bed for building AI-driven systems that can enhance human collaboration, and will also enable future work on learning from interaction, distributing tasks equitably, and maintaining user privacy in such agentic systems.
