\section{Conclusions and Future Directions}

\asyncfw{} is  a new benchmark designed to evaluate the role of language model (LM) agents in facilitating collaborative information gathering within multi-user environments. %By simulating real-world collaboration scenarios across synthetic organizations, \asyncfw{} explores the ability of LM agents to asynchronously route information, identify relevant collaborators, and compile accurate, task-relevant responses. 
It comprises two domains, \dataspider{} and \datanews{}, which challenge LM  agents to handle tasks related to question-answering and document creation. Experiments with popular LM agent architectures revealed both their potential and limitations in accurately and efficiently completing complex collaborative tasks. 
% \asyncfw{} provides a valuable platform for advancing LM-mediated collaboration and sets the stage for future innovations in communication and teamwork enhancement through artificial intelligence.

Future work could consider AI agents that learn over time from interactions for improving their performance over time \cite{lewis1998designing}. By analyzing past conversations, they can improve information source selection and communication strategies, making future interactions more efficient. 
Secondly, privacy risks emerge when agents access personal documents, as large language models may not fully adhere to privacy guidelines \cite{mireshghallah2023can}. Future work could focus on privacy-centric evaluations and explore new information access models to mitigate such risks.
Finally, an excessive number of AI-initiated requests can overwhelm users, hindering productivity. Building agents that can minimize human effort and prioritize urgent requests remains a challenge. %, and explore fully autonomous setups where users’ agents interact directly with each other, reducing the need for continuous user involvement.

\section*{Limitations}

\asyncfw{} consists of two tasks and is in one language (English). Future work could explore further expanding the domains and supported languages.
We make the simplifying assumption that an agent in our setup can engage only in dyadic conversations. Exploring more topologies such as group chats \cite{wu2023autogen} would bring-in additional challenges. 
We designed the domains and the experiment setup to study the effectiveness of the LM agents on a diverse set of information gathering behaviors. However, our analysis did not model all the possible factors in a real-world. Future work can explore additional factors such as turn-around speed and reliability of the response from a collaborator, how busy a person is, and various social dynamics that can be at play in organizations. %as we wanted to focus on ...



\section*{Ethics Statement}
Allowing AI agents the capability to send messages to other users without fine-grained supervision presents a trade-off between saving user time and maintaining control. While autonomy can streamline workflows by eliminating the need for constant user confirmation, verifying key actions helps ensure accuracy and user oversight. While we studied the task in a sand-boxed environment, practitioners should carefully choose the degree of autonomy granted (for example, a more conservative approach would be to get user confirmation before every message that is sent). %, and employ reasonable protection against prompt injection and other attacks.

\section*{Acknowledgements}
We thank Jason Eisner and Hao Fang for thoughtful discussions. We thank Chris Kedzie, Patrick Xia, Justin Svegliato and Soham Dan for feedback on the paper.

