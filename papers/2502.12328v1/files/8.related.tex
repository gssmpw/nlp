\section{Related Work}


\textbf{AI-mediated collaboration and negotiations:}
Recent research in human-AI collaboration has explored various strategies for facilitating decision-making and negotiations among multiple users. \citet{lin2024decision} examines how AI assistants can assist humans through natural language interactions to make complex decisions, such as planning a multi-city itinerary or negotiating travel arrangements among friends. %This work highlights the potential of AI to enhance collaborative decision-making through dialogue. 
\citet{gemp2024states} focus on how game-theoretic approaches that can guide LLMs in tasks like meeting scheduling and resource allocation. %This research introduces a framework where game-theoretic solvers help AI agents strategize and steer conversations toward optimal outcomes, demonstrating the relevance of structured reasoning in multi-user collaboration scenarios.
Past work \cite{papachristou2023leveraging} has also explored the role of LLMs in facilitating group decisions, such as selecting a meeting time or venue, where LLM agents analyze individual preferences from conversations. % and iteratively suggest options that satisfy the group’s needs. 
In contrast, \asyncfw{} focuses on LM agents for coordinating multi-user information gathering.

\textbf{Multi-hop reasoning and task decomposition:}
In our setup, an agent needs to compile information from multiple sources, a theme shared with prior work in multi-hop QA \cite{welbl2018constructing, yang2018hotpotqa} and multi-document summarization \cite{liu2018generating,fabbri2019multi}. %there are additional steps of finding the relevant users, posing apt questions, compiling the gathered information, with minimum communication overhead possible. % It may not be clear upfront who is the best user to reach out. agent needs to pose sutiable questions, and possibly have followup conversations. Also, we care about optimizing number of messages/ overhead/ efficiency of communication. 
Past work on solving complex tasks by decomposing them (via prompting) into simpler sub-tasks \cite{wolfson2020break, khotdecomposed, jhamtani2023natural} is also relevant.
Compared to such past work, our setup requires additional steps of finding the relevant users, posing apt questions, compiling the gathered information, and doing so with minimum communication overhead possible.


