
\section{Baseline Agent Architectures}

To demonstrate the usefulness of \asyncfw{} as a research platform, we develop and evaluate a reference  
LM-powered agent implementation to perform tasks by coordinating interactions, retrieving relevant information, and posing targeted queries to other organization members. 
We consider an event-based reactive agent, which is triggered by user actions: upon getting a message from any organization member, the agent follows ReAct-style prompting loop \cite{DBLP:conf/iclr/YaoZYDSN023}, taking actions, making observations, and performing reflection, until it decides to pause and wait for a next event, or terminate the session.




\subsection{Actions}

The agent can perform a few types of actions.
\textbf{Document Retrieval}:
agents have access to documents accessible to the initiating user, by invoking a function \texttt{search\_documents(query: str)}. Documents are indexed using a standard BM25 index, and the tool call returns a fixed number (upto 3) of documents with the highest matching score. 
\textbf{People Retrieval}:
agents can search through a repository of employee profiles and knowledge areas, by invoking a function \texttt{search\_relevant\_people(query: str)}. 
However, these expertise profiles may be outdated or imprecise, requiring the agent to navigate uncertainty while coordinating queries. As in document retrieval, descriptions are retrieved using a standard BM25 index. A fixed number (up to 10) of highest-scoring results are returned.
\textbf{Sending Messages:} 
the agent is capable of exchanging messages with any person in the organization. 
%One simplifying design choice we make is that all communications happen between exactly two participants at a time: the agent and one person. 
\textbf{Person Resolution}:
the agent can resolve a person name to get their user ids, to be used to send messages to them.
\textbf{Turn and Session Completion}: agent can mark the current turn or the entire session as completed. 

Signatures of Python functions corresponding to the allowed actions are provided in the prompt. See Appendix \ref{sec:action_descriptions} for the full set of action descriptions. %Appendix \ref{sec:action_descriptions} has detailed function signatures.



\subsection{Observations and Reflection}

After each action is taken, the agent receives a textual observation.
These include retrieved documents or descriptions of collaborators. 
%Send message action does not produce an immediate return value. 
As is typical in LLM-based agent architectures, these observations are simply appended to the agent's prompt. Before invoking additional actions, the agent may perform \emph{reflection} actions, corresponding to text-based (``scratchpad'' or ``chain-of-thought'') reasoning about its future plans. Our agent represents reflection as tool calls that return no value but remain in the agent's prompt at future timesteps.



%\begin{table}[h!]
%\centering
%\small 
%\begin{tabular}{l}
%\toprule
%% \textbf{Prompt Structure} \\
%% \hline
%\texttt{Action Descriptions} \\
%\midrule
%\texttt{Exemplars}  \\
%\midrule
%\texttt{<Current Interaction>} \\
%\verb|# {received-message}| \\
%\verb|>>> {action-1}| \\
%\verb|{observation-1}| \\
%\verb|>>> {reflection-1}| \\
%\verb|>>> {action-2}| \\
%\ldots \\
%\verb|>>> {turn-complete-action}| \\
%\verb|# {received-message}| \\
%\verb|>>> {action-1}| \\
%\ldots \\
%\hline
%\end{tabular}
%\caption{Overview of the prompt structure. Expanded prompts available in Appendix \ref{sec:appendix-approach}.}
%\label{tab:prompt}
%\end{table}


\subsection{Prompt Structure}

The prompt has 3 parts: action descriptions (outlined above); exemplars; and interaction history.

\textbf{Exemplars:} In each domain, we manually annotated four exemplars (See Appendix \ref{sec:exemplars} for a full exemplar) with events, actions, and observations. The exemplars are designed to reflect all relevant phenomena in the domain in question, such as dealing with fragmented information, handling unanswerable questions, and managing redirection. %(where one user suggests contacting another for the relevant information), etc.

\textbf{Interaction History:} An event (receiving a message from an employee) triggers LLM into a loop of action prediction, observation, and reflection, till an end of turn or session is predicted. 
Actions are executed immediately after they are predicted;
%A predicted action is parsed into an allowed Python function and its parameters, and immediately executed.  
events, action, and observation are incrementally appended in the prompt in the order in which they occur (see Appendix~\ref{sec:appendix-approach}).


%(Table \ref{tab:prompt}). %(Event -> Action-1 -> Observation-1 (if not None) -> Reflection-1 -> Action-2 -> ... -> Reflection-n -> Turn-completed). 

