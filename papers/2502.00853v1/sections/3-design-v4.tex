\section{Designing Spatial Hybrid Interfaces for Visual Sensemaking}
\label{sec:pc_vr_design}



Following the methodology established by Horak et al.~\cite{horak2018david} for identifying design requirements of hybrid interfaces, we first selected a representative task scenario. We then analyzed the core design characteristics of PC and VR within this context. Based on these characteristics, we defined the design criteria essential for a spatial hybrid user interface tailored to visual sensemaking tasks.

\subsection{Task Domain}
\label{secc:task}
We chose a classic visual sensemaking task from the literature~\cite{mahyar2014supporting,balakrishnan2008visualizations,tong2023towards,yang2024putting} where the user acts as a detective tasked with investigating a hidden illegal activity against wildlife using a set of text documents. 
Similar to a real-world scenario, the detective must extract key entities from documents and construct a node-link diagram connecting different entities with relationships. 
To complete the task, the user must identify the who, what, where, when, how, and why of the event.
We chose this scenario for several reasons: node-link diagrams are familiar to users and require low visualization literacy; they are justified in both 2D and 3D; and the sensemaking task is complex enough to provide a deeper understanding of the visualization system's user experience.

\subsection{PC and VR for Visual Sensemaking}
\label{sec:pc-vr-only-design}
To design the spatial hybrid interface combining PC and VR for visual sensemaking, we first systematically analyzed the characteristics of each platform.

\para{PC}s generally offer visual interfaces with relatively high-resolution input and output~\cite{feiner1991hybrid}. 
In particular, high-resolution input means that the mouse and keyboard provide a mature and accurate interaction mechanism. Moreover, many well-designed systems have utilized high-resolution displays to compact information into a standard PC screen. 
Additionally, 2D visualization used in PC provides users with a familiar visual representation~\cite{riegler2020cross} and visual exploration workflow.
In general, 2D node-link diagrams have been commonly used to reveal relationships between entities in visual sensemaking~\cite{mahyar2014supporting,balakrishnan2008visualizations,lin2021taxthemis,perer2011visual}. 
However, the PC's small physical display space may limit the scalability of the visualization. 


\para{VR} can provide the same \retvcg{views}{interfaces} and functionalities as in the PC interface\retvcg{, except input devices}{}. 
Additionally, an optimized VR interface can leverage its display and interaction modalities---such as a large workspace, 3D visualization, embodied interaction, and spatial navigation---for visual sensemaking tasks.
Specifically, the virtually large display area accommodates more visualizations~\cite{horak2018david} and other content, such as documents~\cite{tong2023towards}, enhancing visual content management~\cite{satriadi2020maps,lisle2020evaluating,yang2020virtual}.
Moreover, 3D node-link diagrams have been shown to outperform traditional 2D node-link diagrams~\cite{kwon2016study}.
Additionally, AR/VR introduces more intuitive and novel interactions for data visualization through physical body movements~\cite{cordeil2017imaxes,liu2023datadancing,tong2022exploring,yang2020tilt,in2023table}.
\re{Lastly, spatial ability can be utilized and beneficial to the sensemaking process in VR~\cite{lisle2020evaluating}, as well as data analytics in VR~\cite{hayatpur2020datahop,li2023gestureexplorer,in2024evaluating}.}{}
Nonetheless, the relatively low-resolution output and input capability limit the effectiveness of visual sensemaking~\cite{feiner1991hybrid}. 
In particular, the text is hard to read, and the input interaction in VR is not precise enough.



\subsection{PC+VR Hybrid System for Visual Sensemaking}

Guided by prior studies~\cite{lisle2020evaluating,hubenschmid2022relive,hubenschmid2021towards,davidson2022exploring,tong2023towards} and our specific research objectives concerning crime-solving tasks, we delineate five key design requirements.

\para{R1: Supporting a movable hybrid user interface.}\label{req:r1}
One of the benefits of the immersive environment is the spatial navigation~\cite{lisle2020evaluating,hayatpur2020datahop,li2023gestureexplorer}. However, the current design of hybrid interfaces of PC and immersive devices does not fully utilize this characteristic. 
Most of the work~\cite{immersed2023,wang2020towards} used the AR/VR HMD to provide an extended 3D view for the PC monitor, and users are mainly seated without any navigation in space. 
To fully engage users in ``\textit{Immersive Space to Think}''~\cite{lisle2020evaluating}, supporting spatial navigation in the hybrid system should be considered. 

\para{R2: Design optimized interface for each device.}\label{req:r2}
Each device's design should be optimized to increase user experience~\cite{tong2023towards}.
For example, as demonstrated in previous studies~\cite{davidson2022exploring,lisle2021sensemaking}, the large 3D space in VR should be utilized to facilitate the spatial sensemaking process. Moreover, Saffo~\etal{}~\cite{saffo2023eyes} suggested that abstract data visualizations are better suited for interpretation and interaction on a PC, while natural spatial mapping visualizations are more advantageous in VR. Thus, the design in PC should make use of the existing well-established design, and the design in VR should make good use of the large 3D space.


\para{R3: Provide the same context in both interfaces.}\label{req:r3}
\re{The PC and VR interfaces should support the same context to avoid losing context when switching devices.
People may prefer to perform the task even in a less efficient environment to avoid the trouble caused by switching devices~\cite{hubenschmid2022relive}.
Therefore, previous work often offers users similar or duplicated views in both devices to provide the same context~\cite{hubenschmid2022relive,jansen2023autovis}.
Especially in our work, we are interested in investigating when users choose to use a PC and when to use VR in the PC+VR system and want to ensure the investigation is unbiased.
As a result, we aim to design the spatial hybrid PC+VR system with identical information and functionalities in both environments.}{}

\para{R4: \retvcg{Reduce transition cost between}{} PC and VR interfaces.}\label{req:r4} \retvcg{Frequent transitions between devices could interrupt the sensemaking process and create disorientation~\cite{hubenschmid2022relive}.
Therefore, we consider minimizing the transition time between PC and VR usage during visual sensemaking.}{}
Inspired by Davidson~\etal{}~\cite{davidson2022exploring} and suggested by Hubenschimid~\etal{}~\cite{hubenschmid2022relive}, we aim to render the PC screen inside VR, \ie{}, a simulated PC in VR~\cite{jetter2020vr}, so that the user could view the PC and VR interfaces at the same time\retvcg{, avoiding the change in devices to reduce transition time.}{}
Thus, we expected the hybrid interface to be located near Augmented Virtuality in reality–virtuality continuum~\cite{milgram1995augmented} (\Cref{fig:rv}), meaning that users are situated in VR but they can still see partial objects in reality, \ie{}, the physical table, keyboard, and mouse.

\para{R5: Allow easy-to-switch input modality and cross-device interaction.} \label{req:r5}
Tedious switching between different input devices could potentially increase the cost of using different devices. For example, although VR controllers could provide more precise control and more functionality compared to hand gestures, they did not complement well with the mouse and keyboard. Users are required to find an empty space to put down the VR controllers whenever they switch to a PC. Therefore, the hybrid system should allow an easy-to-switch input modality, such as hand.
Moreover, cross-device linking and brushing should be supported~\cite{hubenschmid2022relive}. Users should be able to see the highlighted marks on both devices. It could reduce interruption and misorientation during visual sensemaking.
