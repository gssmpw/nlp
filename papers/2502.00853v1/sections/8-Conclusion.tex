\section{Conclusion}
\label{sec:conclusion}


This paper presents a spatial hybrid PC+VR system that seamlessly combines the benefits of both interfaces for visual sensemaking by integrating a simulated PC with a movable mouse and keyboard to support immersive spatial navigation with a VR headset. 
To minimize the effort involved in transitioning between PC and VR, we introduced three techniques: a simulated PC in PC+VR, synchronized states between devices, and hand gestures as the primary modality. 
We conducted a user study with 18 participants to explore user behaviors, usage patterns, and preferences for the PC+VR system compared to PC-only and VR-only conditions during a visual sensemaking task.
\retvcg{We found the following key insights from our exploratory study: 
1) PC+VR was overall most preferred, particularly for interaction; 
2) PC+VR with the movable PC did not negatively impact performance; 
3) PC+VR helped reduce physical demand compared to VR-only; 
4) Participants were more willing to transit between PC and VR when the transition cost was lower, as shown by an increase in temporary transition at the interaction level; and 
5) Participants engaged in more spatial navigation in PC+VR, utilizing features such as moving the simulated PC and rotating around it to enhance their experience.
}{}
We believe our findings could inspire future designs of spatial hybrid systems to enhance visual sensemaking.

