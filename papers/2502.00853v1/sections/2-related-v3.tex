\section{Related Work}

\para{Data Visualization for Visual Sensemaking.}
Data visualization is shown to be effective for sensemaking by visualizing the complex relationships among entities~\cite{card1999readings}. For example, Hendrix~\etal{}~\cite{hendrix2000visualizations} visualizes the hierarchical structure of Java code for better program comprehension. Perer~\etal{}~\cite{perer2011visual} utilized a node-link diagram to understand the complex relationships between different enterprises. Moreover, researchers~\cite{balakrishnan2008visualizations, lin2021taxthemis, mahyar2014supporting} implemented visualization systems to assist in solving crime.

However, the abovementioned systems are designed for the desktop environment, which provide only a limited 2D display space. Recently, Lee~\etal{}~\cite{lee2021post} tried to bring Post-it notes with links from reality to VR for ideation, and Tong~\etal{}~\cite{tong2023towards} explored visual sensemaking with node-link diagrams in VR.
In this work, we aim to explore the potential of using AR/VR HMDs for visual sensemaking.

\para{Immersive Analytics.} Immersive analytics
is an emerging research field that uses
immersive technologies, such as VR and AR, to interact with and explore complex data in new and more intuitive ways~\cite{marriott2018immersive,ens2021grand}.
A series of studies have been conducted to explore the benefits and drawbacks of Immersive Analytics, and these have been comprehensively reviewed in recent surveys~\cite{fonnet2021survey,kraus2022immersive,skarbez2019immersive,klein2022immersive}.
While we do not aim to provide a comprehensive list of all such studies, we introduce some representative ones to offer an informative overview.

There are several frequently reported benefits of Immersive analytics from the literature: 3D rendering, large display space, embodied interaction\re{, and spatial navigation}{}.
\emph{Firstly}, VR/AR facilitates the rendering of 3D graphics in any location around the user, making it particularly useful for analyzing data with inherent 3D information~\cite{brath20143d}, such as scientific visualizations~\cite{el2019virtual}. 
The additional space and dimension provided by VR/AR also enable the de-cluttering of visualizations with dense data and facilitate the examination of spatial patterns, as seen in geographic visualizations~\cite{yang2018origin}, scatterplots~\cite{kraus2019impact,bach2017hologram}, and network visualizations~\cite{kwon2016study}.
\emph{Secondly}, VR/AR provides an almost unlimited screen that creates new possibilities for creating novel techniques, such as room-sized visualizations~\cite{yang2018maps,kraus2019impact}. 
\emph{Thirdly}, embodied physical movements can be more intuitive and expressive than the conventional PC user interface~\cite{cordeil2017imaxes,yang2020tilt,liu2023datadancing,tong2022exploring,hurter2018fiberclay}. 
\re{Lastly, physical moving in space and head rotation has proven to be an effective navigation method~\cite{yang2020embodied,ball2007move} and is readily available in immersive environments.
Additionally, the spatial arrangement of content leverages spatial memory and leads to more efficient wayfinding~\cite{yang2020virtual,satriadi2020maps,lisle2020evaluating,hayatpur2020datahop,in2024evaluating,li2023gestureexplorer}.}{}

While Immersive Analytics can offer many benefits, there are significant reservations about using VR/AR.
For example, although numerous efforts have been made to enhance the text input experience in VR/AR, its efficiency and usability still lag behind keyboard input in a desktop environment~\cite{speicher2018selection}.
Additionally, the laser pointer is the most commonly used interaction in VR/AR, allowing users to access distant objects. However, such mid-air interactions unavoidably result in fatigue and imprecise interaction with data points~\cite{cordeil2020embodied}. 
Given these limitations, we propose to address its disadvantages by providing users with a PC computing environment.

\para{\re{Hybrid User Interface with PC and Immersive Technology}{Hybrid User Interface} for Data Visualization.}
Since different devices have unique advantages, the hybrid user interface has been introduced to leverage benefits from different devices~\cite{feiner1991hybrid}. However, building hybrid systems poses inherent and distinct challenges, such as loss of context~\cite{hubenschmid2021towards}. 
Furthermore, interaction design across devices continues to be one of the significant challenges in the field~\cite{ens2021grand}.

\re{One popular hybrid user interface design for data visualization would be combining PC and immersive technologies (AR/VR) because visual analytics systems and visualizations are designed and tailored for a PC environment. On the one hand, researchers explore transitional and collaborative interfaces for data exploration~\cite{frohler2022survey}.}{} 
For example, ReLive has been introduced to provide ex-situ analysis on PC and in-situ analysis in VR~\cite{hubenschmid2022relive}. However, the proposed system requires users to put on and remove the VR HMD to switch between complete reality and virtuality, disrupting workflow continuity. AutoVis~\cite{jansen2023autovis} includes a virtual tablet showing part of the visualization from the PC view. It potentially reduces context switching. Yet, interactions with the desktop view are not fully supported in the VR environment. \retvcg{It still requires putting on and off the VR headset to switch computing environments. Therefore, the cost of transition remains high.}{}

On the other hand, researchers are actively exploring non-transitional interfaces (\eg{}, using AR and PC simultaneously for visual analytics).
For example, Wang~\etal{}~\cite{wang2020towards,wang2022understanding} has explored the combination of AR and PC for 3D scientific visualization. \re{Seraji~\etal{}~\cite{seraji2024analyzing} enables users to transfer data visualizations between AR and PC environments.}{}

\re{Although the involvement of the AR could make the integration of the PC setup easier, the interference of the background context makes the color~\cite{whitlock2020graphical} and text~\cite{zhou2024did} in visualizations harder to design and read. In particular, our tasks do not incorporate real-world context; therefore, instead of using AR, which may introduce background distractions, VR offers a fully immersive visual analytics experience that enhances focus and minimizes distractions~\cite{lisle2023different}}{}.
\re{Furthermore, in the previously mentioned work, AR/VR primarily serves as an extended 3D display for the PC workspace, where the user remains seated and stationary~\cite{wang2022understanding,pavanatto2021we,immersed2023}}{}. This approach does not fully exploit the spatial capabilities offered by the immersive environment. 
\retvcg{Wang~\etal{}~\cite{wang2020towards} investigated the potential of spatial movement in visual exploration and discovered that navigating through data by walking is intuitive. However, in their study, the PC remained stationary, which meant that users could not utilize the computer interface while moving around, leading to interruptions in their workflow.}{} 

\retvcg{As a result, we aim to design a hybrid spatial system that reduces the cost of transition and the interference of the reality background and can fully leverage the benefits of immersive techniques (\eg{}, large display and spatial navigation).}{}
