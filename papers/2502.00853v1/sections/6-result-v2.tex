\section{Results}
\label{sec:result}

\begin{figure*}
\centering
\includegraphics[width=\textwidth]{figs/user_experience.png}
\caption{Results of the NASA Task Load questionnaire (a) and two questions for users' concentration (b). Error bars show the 95\% confidence interval (CI).
Significance values are reported as $p {<} .05(*)$ and $p {<} .001({*}{*}*)$. \re{The table presents the statistical data. Significance values ($p {<}.05$) are highlighted in green.}{}}
\label{fig:user_experience}
\end{figure*}

\begin{figure*}
\centering
\includegraphics[width=\textwidth]{figs/interaction_CHI24_new.pdf}
\caption{The average number of interactions logged for each condition. Error bars show the 95\% confidence interval (CI). 
Significance values are reported as $p {<} .05(*)$ and $p {<} .01(**)$. \re{The table presents the statistical data. Significance values ($p {<}.05$) are highlighted in green.}{}}
\label{fig:interaction_result}
\end{figure*}

For \textit{time} and \textit{number of interactions}, we first applied a log transformation to check the normality assumption. We then employed \textit{repeated measure ANOVA} to evaluate the effect of the three study conditions on the dependent variables. 
We determined the significance of including an independent variable or interaction terms using a log-likelihood ratio.
Additionally, we performed post-hoc pairwise comparisons using \textit{t-test with Bonferroni correction}.
For other non-parametric data, such as accuracy, user experience ratings, and rankings, we conducted \textit{Friedman tests} with \textit{Nemenyi post-hoc} analysis.
For qualitative feedback, we used affinity diagramming~\cite{hartson2012ux} to organize and analyze the transcripted recordings.

\para{Time}, \textbf{Accuracy} and \textbf{User Experience Ratings}.
With repeated measure ANOVA, we did not find significance for \textit{time} ($F(2,34){=}1.53, p{=}0.230, \eta^2_G{=}0.0537$).
With the Friedman test, we also did not find significance for \textit{accuracy} ($\chi^2(2){=}0.426,p{=}0.808,W{=}0.0118$) and \textit{user experience ratings}, \ie, concentration and NASA Task Load except physical demand, as shown in \Cref{fig:user_experience}. VR ($avg{=}4.33, CI{=}0.950$) is significantly more physically demanding than PC+VR ($avg{=}2.83, CI{=}0.857, p{=}0.0164$) and PC ($avg{=}1.67, CI{=}0.418, p{=}0.00100$).

\para{Number of Interactions.}
We did not find that our testing conditions significantly affected the total number of interactions, \Cref{fig:interaction_result}(a).
However, when we subdivided the number of interactions into specific types, see \Cref{fig:interaction_result}(b), we did find significant differences in adding nodes ($F(2,34){=}6.01, p{=}0.00581, \eta^2_G{=}0.111$) and links ($F(2,34){=}5.56, p{=}0.00815, \eta^2_G{=}0.110$) reflected.
These two interactions were the primary interactions performed by the participants to build the graphs.
By testing the pairwise significance, we found that participants added more nodes in PC-only ($avg{=}22,CI{=}4.76$) than VR-only ($avg{=}14.5,CI{=}4.83,p{=}0.00846$) and PC+VR ($avg{=}16.6,CI{=}6.63,p{=}0.0262$), as well as more links in PC-only ($avg{=}24.1, CI{=}5.89$) than VR-only ($avg{=}15.5, CI{=}5.26, p{=}0.00892$) and PC+VR ($avg{=}17.7,CI{=}6.97,p{=}0.0367$).


\begin{figure*}
\centering
\includegraphics[width=\textwidth]{figs/ranking_CHI24_new.pdf}
\caption{Ranking results in terms of Authoring, Exploring (finding a target), Discovering (finding an insight), Interaction, and Overall ranking of the post-study survey for the three conditions in the formal study.
Significance values are reported as $p {<} .05(*)$ and $p {<} .01(**)$.
Marginal significance values are reported as $.05{<}p{<}.1(\cdot)$
\re{The table presents the statistical data. Significance values ($p{<}.05$) and marginal significance values ($.05{<}p{<}.1$) are highlighted in green and yellow, respectively.}{}}
\label{fig:ranking}
\end{figure*}

\para{User Preference.} 
As shown in \Cref{fig:ranking}, there are significant differences in terms of overall preference ($\chi^2(2){=}6.33,p{=}0.0421,W{=}0.176$), interaction ($\chi^2(2){=}11.4,p{=}0.0033,W{=}0.318$), and authoring experiences ($\chi^2(2){=}8.11,p{=}0.0173,W{=}0.225$). \retvcg{Overall, participants liked PC+VR more than VR-only (12/18,$p{=}0.0333$). In particular, in terms of interaction, PC+VR is ranked higher than VR-only (11/18,$p{=}0.00247$).}{} We also found that PC-only is preferred in interaction slightly more than VR-only (5/18,$p{=}0.0771$).
\re{Moreover, more participants preferred authoring with PC-only than VR-only (11/18, $p{=}0.0128$).
Though we did not observe further significant differences, there is a marginal difference in terms of discovering ($\chi^2(2){=}5.33,p{=}0.0694$), and more participants preferred discovering with PC+VR (11/18).}{}


\para{Qualitative Comments and Different Strategies Used.}
Participants raised different opinions towards different interfaces.
PC interface was described as \userquote{easy and simple to interact with the graph precisely} (16/18), VR interface was \userquote{suitable for reading documents} (15/18), and PC+VR could \userquote{leverage benefits from both of them and overcome their weaknesses} (12/18).
With the improvement and involvement of more participants than in the preliminary study, we received new points for our conditions. 
Though with a higher resolution of text, more participants (from 25\% to 44\%) considered the PC interface unsuitable for reading documents (8/18) for two major reasons. The first one is that the document list is inefficient in switching between documents (5/8), and the small screen of the simulated PC interface is unsuitable for reading (4/8). For example, P3 commented that \userquote{I need to click and jump to the different articles to find the association in PC.} P16 added that \userquote{the PC screen size is too small [compared to VR] which is not good for reading.} For the PC+VR condition, we did not receive complaints about our simulated PC regarding its latency and low resolution.

\re{Moreover, on the one hand, 13 participants mentioned that they had used a different strategy in using different interfaces for completing the task. Specifically, ten participants mentioned that they utilized graphs more in PC-only, while in VR-only and PC+VR mode, they tended to read all documents first and create the graph only with key documents. For example, P3 mentioned that \userquote{I tend to create [graphs] on the PC. In VR, I only need to remember spatial positions and don't want to create graphs.} P18 added that \userquote{The operation in PC is very familiar and precise, and I tend to record more information (using the graph) because the interaction cost is relatively small, and I hope I won’t have to go back and switch documents. In VR, I didn't create graphs first; I read all the documents first and then only created graphs for key information.} On the other hand, four participants mentioned that the strategy used in all conditions is similar. For example, P7 stated that she analyzed the time first, then read documents [based on the time], and created the graph for all three conditions. Moreover, P6 mentioned that the strategy used was independent of the interface but experience of using visual sensemaking tools.}{}

\para{PC+VR Hybrid User Strategies.}
\re{To analyze the user strategies of the PC+VR interfaces, we visualize the interaction logs and}{We} group participants into different strategy categories from the perspectives of the \emph{temporal} and \emph{spatial}.

\begin{figure}
\centering
\includegraphics[width=\columnwidth]{figs/segment_CHI24.pdf}
\caption{The distribution of time for all participants in using PC and VR in the PC+VR condition in the formal study: primarily use PC (time of PC ${>} 75\%$) (a), primarily use simulated VR (time of VR ${>} 75\%$) (b), used both VR and simulated PC with non-frequent switching (c), and used both VR and simulated PC with frequent switching (d). The orange and grey dots represent the events related to documents and graphs.}
\label{fig:time}
\end{figure}

\begin{figure*}
\centering
\includegraphics[width=0.8\linewidth]{figs/pattern_CHI24.pdf}
\caption{This figure shows four representative patterns of how users solved the task with an example of corresponding user trajectories (a-d) in the formal study. The black arc represents the position of the documents in VR. Some participants did not move the position of the PC and themselves (a) or just themselves (b). Others moved the position of the PC but remained stationary (c) or together (d).}
\label{fig:patterns}
\end{figure*}

\vspace{1mm}\noindent\textit{Temporal strategies.}
We were interested in how much time the participants spent in the different environments and how frequently they switched between environments. 
We identified four different strategies from our participants based on 1) the time spent on PC or VR and 2) the frequency of switching between devices:
\begin{enumerate}[noitemsep,topsep=0pt,parsep=0pt,partopsep=0pt,leftmargin=*]
    \item \textbf{PC-only}: 1/18 participant spent most of their time (${>} 75\%$) in the PC environment, \Cref{fig:time}(a).
    \item \textbf{VR-only}: 6/18 participants spent most of their time (${>} 75\%$) in the VR environment, \Cref{fig:time}(b).
    \item \textbf{VR-then-PC}: 4/18 participants spent noticeable time in each environment without frequently switching, especially having a pattern of VR first, then PC, \Cref{fig:time}(c).
    \item \textbf{Frequent Switch}: 7/18 participants spent similar time in each environment and frequently switched environments, \Cref{fig:time}(d).
\end{enumerate}

\vspace{1mm}\noindent\textit{Spatial strategies.}
We also wanted to know how participants moved the PC as well as how they moved in space.
We also identified four different strategies from our 18 participants based on 1) the movement of the user (${\geq} 290m$ or ${<} 290m$) and 2) the movement of the table (${\geq} 5.5m$ or ${<} 5.5m$):
\begin{enumerate}[noitemsep,topsep=0pt,parsep=0pt,partopsep=0pt,leftmargin=*]
    \item \textbf{Stationary User and PC}: 3/18 participants almost did not move the PC and primarily stood near the initial starting position, \Cref{fig:patterns}(a).
    \item \textbf{Stationary PC}: 8/18 participants almost did not move the PC but moved themselves to use the space in VR, \Cref{fig:patterns}(b). With these eight participants, we also observed three subpatterns: \textit{\underline{PC Side}} (3/8, move the PC to the side, \Cref{fig:patterns}(b1)),  \textit{\underline{User Side}} (3/8, move towards right or left sides, \Cref{fig:patterns}(b2)), and \textit{\underline{Circle}} (2/8, move around the PC, \Cref{fig:patterns}(b3)).
    \item \textbf{Self-Rotation}: 5/18 participants remained stationary while they moved the PC in the space, \Cref{fig:patterns}(c).
    \item \textbf{Carrying}: 2/18 participants constantly moved the PC with them in the VR space, \Cref{fig:patterns}(d).
\end{enumerate}
