\section{Experimental Setup\protect\footnote{Dataset and training details in Appendix~\ref{app:dataset_n_train}; all prompts (\eg prompts of user simulator and LLM judges) in Appendix~\ref{app:prompts}.}}

\begin{comment}
We conducted extensive and large-scale experiments to answer the following research questions:
\begin{itemize}
    \item How \textit{effective} are the \objects in collaborating with users? (Sec.~\ref{sec:quantitative})
    \item How do \objects outperform? What component is \textit{essential} to their success? 
    (Sec.~\ref{sec:ablation})
    \item What are the \textit{insights} from the generation results from interactive-trained LLMs? (Sec.~\ref{sec:case})
    \item Can the capabilities of interactive-trained LLMs \textit{generalize} to untrained data domains? (Sec.~\ref{sec:extended})
\end{itemize}
\end{comment}




For fine-tuning and evaluation, we create three multiturn datasets using publicly available data across diverse domains~\citep{math, bigcodebench, medium}: collaborative document editing, coding problem assistance, and multiturn mathematics problem solving.

To build a multiturn environment (Figure~\ref{fig:evaluation}), we employ GPT-4o-mini as a user simulator LLM to role-play realistic user behaviors, given the target problem and conversation history. Our simulation-based evaluations are designed to closely mimic real-world interactions~\cite{simulate1000}. 
Unlike traditional single-turn tasks, our setup requires dynamic interactions over multiple turns to achieving a goal. 
The three interactive datasets are: 

\noindent \textbf{\doct}: Document editing requires iterative feedback and refinements across multiple turns to ensure coherence and alignment with user intent. We sample 100 Medium articles as goal documents, which are summarized into target problems to guide the user simulator. After each interaction, task performance is evaluated using the \textbf{BLEU} score, measuring similarity between the extracted document and the original articles.

\noindent \textbf{\codet}: Coding tasks inherently require multiturn interactions, such as clarifying requirements and debugging. We sample 600 coding problems from BigCodeBench~\citep{bigcodebench} as the target problems given to the user simulator. For evaluation, we compute the average \textbf{Pass Rate (PR)} of code at the end of the interactions.

\noindent \textbf{\mathct}: Math problem solving often requires addressing implicit assumptions, verifying intermediate steps, and clarifying reasoning. We sample 200 level-5 math problems from MATH~\citep{math} to prompt the user simulator, which interacts with the LLMs. Task success is measured by the \textbf{accuracy (ACC)} of the final solution, as evaluated by an LLM judge.

    

In addition to the above task-specific metrics, we incorporate two task-agnostic scores across all datasets: \textbf{1) Average Token Count}, which quantifies the average number of tokens generated by the LLM per conversation, reflecting interaction efficiency. \textbf{2) Interactivity (ITR)}, which evaluates engagement levels using an LLM judge (Claude-3.5-Sonnet), with scores rescaled to the range [0, 1]. 



    
    
\begin{table*}[t!]
    \centering
    \small
    
    \scalebox{0.90}{
    \setlength{\tabcolsep}{1.0pt}
    \begin{tabular}{l c c c r | c c c c c c |c  c c }
    \toprule
    \multirow{1}{*}{Method} & \multirow{1}{*}{Recipe} & \multirow{1}{*}{Complexity} & \multirow{1}{*}{\# P.} & \multirow{1}{*}{\# T.P.}& MME & MMB &POPE & \multicolumn{1}{c} {SEED} & MMMU & MM-Vet& TQA & SQA-I  & \multicolumn{1}{c}{GQA} \\
    \midrule
    \rowcolor{gray!14}
    \multicolumn{14}{l}{\textbf{\textit{Encoder-based VLMs}}} \\ 
    OpenFlamingo~\cite{openflamingo} & \underline{PT, SFT}& Quadratic & 9B& 96.6\%  & - & 4.6 & - & - & - & - & 33.6 & - & - \\
    MiniGPT-4~\cite{minigpt} & \underline{PT, SFT}& Quadratic & 13B& 94.8\%  & 581.7 & 23.0 & - & - & -& 22.1 & - & - & 32.2  \\
    Qwen-VL~\cite{qwenvl} & \underline{PT, SFT}& Quadratic & 7B& 100.0\%  & - & 38.2 & - & 56.3 & - & - & 63.8 & 67.1 & 59.3\\ 
    LLaVA-Phi~\cite{llavaphi}  & \underline{PT, SFT}& Quadratic & 3B& 90.0\%  & 1335.1 & 59.8 & 85.0 & - & - & 28.9& 48.6 & 68.4 & - \\
    MobileVLM-3B~\cite{mobilevlm} & \underline{PT, SFT}& Quadratic & 3B& 90.0\%  & 1288.9 & 59.6 & 84.9 & - & - & - & 47.5 & 61.0 & 59.0  \\
    VisualRWKV~\cite{visualrwkv} & \underline{PT, SFT}&  \textbf{Linear} & 3B& 90.0\%  & 1369.2 & 59.5 & 83.1 & - & - & - & 48.7 & 65.3 & 59.6 \\
    VL-Mamba~\cite{vlmamba} & \underline{PT, SFT}&  \textbf{Linear} & 3B& 90.0\%  & 1369.6 & 57.0 & 84.4 & - & -& 32.6 & 48.9 & 65.4 & 56.2 \\
    Cobra~\cite{cobra} & \underline{PT, SFT}&  \textbf{Linear} & 3.5B& 82.6\%  & - & - & \textbf{88.4} & - & - & - & 58.2 & - & \textbf{62.3}\\
    \midrule
    \rowcolor{gray!14}
    \multicolumn{14}{l}{\textbf{\textit{Decoder-only VLMs}}} \\
    Fuyu-8B (HD)~\cite{fuyu} & \underline{PT, SFT}& Quadratic & 8B& 100.0\%  & 728.6 & 10.7 & 74.1 & - & - & 21.4 & - & - & -\\
    SOLO~\cite{solo} & \underline{PT, SFT}& Quadratic &  7B& 100.0\%   & 1001.3 & - & - & 64.4 & - & - & - & 73.3 & -   \\    
    Chameleon-7B~\cite{chameleon}  & \underline{PT, SFT}& Quadratic &  7B& 100.0\%   & 170 & 31.1 & - & 30.6 & 25.4 & 8.3 & 4.8 & 47.2 & -\\  
    EVE-7B~\cite{eve}  & \underline{PT, SFT}& Quadratic &  7B& 100.0\%  & 1217.3 & 49.5 & 83.6 & 61.3 & \underline{32.3} & 25.6& 51.9 & 63.0 & 60.8 \\
    Emu3~\cite{emu3} & \underline{PT, SFT}& Quadratic & 8B& 100.0\%  & - & 58.5 & 85.2 & \underline{68.2} & 31.6 & \underline{37.2} & \underline{64.7} & \underline{89.2} & 60.3\\
    HoVLE~\cite{hovle} & DT, PT, SFT & Quadratic & \textbf{2.6B}& 100.0\%  & \textbf{1433.5} & \textbf{71.9} & \underline{87.6} & \textbf{70.7} & \textbf{33.7} & \textbf{44.3} & \textbf{66.0} & \textbf{94.8} & \underline{60.9} \\
    \rowcolor{green!15}
    \name{} & \textbf{DT} & \textbf{Linear} & \underline{2.7B}& \underline{14.7\%}  &1303.5 & 57.2 & 85.2 & 62.9& 30.7  & 31.1 &47.7 & 79.2 & 57.4 \\
    \rowcolor{yellow!15}
    \name{} & \textbf{DT} & \underline{Hybrid} & \underline{2.7B}& \textbf{11.2\%}  & \underline{1371.1} & \underline{63.7} & 86.7 & 66.3 & \underline{32.3} & 36.9 & 55.1 & 86.9 & 59.3  \\
    
    \bottomrule
    \end{tabular}
    }
    \vspace{-1em}
    \caption{\textbf{Comparison with existing VLMs on general VLM benchmarks.} ``Recipe'' denotes the adopted training recipe. ``PT'', ``SFT'', and ``DT'' denote the pre-training, supervised fine-tuning, and distillation training, respectively. ``Complexity'' denotes the model computation complexity with respect to the number of tokens. ``\# P.'' denotes the number of total parameters. ``\# T.P.'' denotes the percentage of trainable parameters ($\frac{\text{trainable paramters}}{\text{total parameters}}$). The best performance is highlighted in \textbf{bold} and the second-best result is \underline{underlined}.}
    \label{tab:results_general}
    \end{table*}

\begin{table*}
  [t]
  \centering
  \resizebox{\textwidth}{!}{%
  \begin{tabular}{cccccccccccc}
    \toprule \multicolumn{2}{c}{Components}                                                             & \multicolumn{5}{c}{Re-executability Rate (\%)} & \multicolumn{5}{c}{Readability (\#)} \\
    \cmidrule(lr){1-2} \cmidrule(lr){3-7} \cmidrule(lr){8-12}        \hspace{8pt}\labelemoji\hspace{8pt}                                                                & \hspace{8pt}\toolemoji\hspace{8pt}                                      & O0                                 & O1             & O2             & O3             & AVG            & O0             & O1             & O2             & O3             & AVG            \\
    \hline
    \rowcolor[rgb]{0.93,0.93,0.93}\multicolumn{12}{c}{\textbf{Initialize with LLM4Decompile-End-6.7B~\citep{llm4decompile}}}   \\
    \xmark                                                                                              & \xmark                                    & 69.51                              & 46.95          & 50.61          & 46.34          & 53.35          & 3.98 & 3.41 & 3.44 & 3.38 & 3.55 \\
    \cmark                                                                                              & \xmark                                    & 75.61                              & 50.61          & 50.00          & 50.00          & 56.55          & 4.01 & 3.44 & 3.39 & \textbf{3.49} & 3.58 \\
    \xmark                                                                                              & \cmark                                    & 83.54                     & \textbf{56.10}          & 51.22          & 50.61 & 60.37 & 4.05 & 3.51 & 3.51 & 3.42 & 3.62 \\
    \cmark                                                                                              & \cmark                                    & \textbf{85.37}                            & \textbf{56.10}                     & \textbf{51.83} & \textbf{52.43}          & \textbf{61.43} & \textbf{4.13} & \textbf{3.60} & \textbf{3.54} & \textbf{3.49} & \textbf{3.69} \\

    \rowcolor[rgb]{0.93,0.93,0.93}\multicolumn{12}{c}{\textbf{Initialize with Deepseek-Coder-6.7B-base~\citep{deepseekcoder}}} \\
    \xmark                                                                                              & \xmark                                    & 59.15                              & 35.98          & 39.02          & 37.80          & 42.99          & 3.71 & 3.05 & 3.16 & 3.05 & 3.24 \\
    \cmark                                                                                              & \xmark                                    & 66.46                              & 41.46          & 38.41          & 36.59          & 45.73          & 3.76 & 3.17 & \textbf{3.21} & 3.08 & 3.31 \\
    \xmark                                                                                              & \cmark                                    & 70.73                              & 39.63          & 39.02          & 40.24          & 47.41          & 3.90 & 3.17 & 3.08 & 3.11 & 3.31 \\
    \cmark                                                                                              & \cmark                                    & \textbf{79.88}                     & \textbf{45.73} & \textbf{43.90} & \textbf{42.68} & \textbf{53.05} & \textbf{3.96} & \textbf{3.21} & 3.18 & \textbf{3.19} & \textbf{3.38} \\
    \bottomrule
  \end{tabular}%
  }
  \caption{The ablation study of different methods across four optimization levels
  (O0, O1, O2, O3), as well as their average scores (AVG). The results in bold represent the optimal performance. The ~\labelemoji~ and ~\toolemoji~ means Relabedling and Function Call. \textbf{Bold} denotes the best performance.}
  \label{tab:ablation}
\end{table*}

\xhdr{Fine-tuning \name{}s}  \name{}s are based on \llama{}~\citep{metallama} with LoRA finetuning~\citep{lora}. We train four model variants: \textbf{1)~Offline models}: SFT and Offline DPO are fine-tuned on pre-generated multiturn conversational datasets guided by Multiturn-aware Rewards (MR) (\cf Section~\ref{sec:optimization}). \textbf{2) Online models}: PPO and Online DPO are further trained from the SFT and Offline DPO models, respectively. The model during online fine-tuning is involved in the collaborative simulation to compute MRs, which, in turn, dynamically adjust the model preference. 


\xhdr{Baselines} We compare \name{}s against (1) the pretrained \llama (\textit{Base}), (2) the base model with proactive prompt engineering (\textit{Proactive Base}), which encourages follow-up and clarification questions. 




\begin{figure*}[t]
    \centering
    \includegraphics[width=0.9\linewidth]{figures/coding_v2}
    \vspace{-5pt}
    \caption{Case study on \codet. The non-collaborative LLM assumes user needs, adding unnecessary steps like punctuation and stopword removal. In contrast, \name{} clarifies tokenizer preferences, error handling, and package installation, leading to a solution that precisely aligns with user intent.}
    \label{fig:coding}
\end{figure*}
\vspace{-5pt}

\begin{figure*}[t]
\centering
\begin{minipage}{0.38\textwidth}
    \centering
    \includegraphics[width=0.92\textwidth]{figures/cropped/case_study.pdf}
    \caption{Reward comparison for response \texttt{A} and \texttt{B} of Figure~\ref{fig:coding} shows different preferences.}
    \label{fig:reward_preference}
\end{minipage}%
\hfill
\begin{minipage}{0.58\textwidth}
    \centering
    \resizebox{1.0\textwidth}{!}{
    \begin{tabular}{ccccc}
        \toprule 
        & \multicolumn{2}{c}{Action-level Accuracy} & \multicolumn{2}{c}{Macro Metric}  \\
        & Ambiguous   & Non-Ambiguous & Accuracy & F1\\
        \midrule
        GPT-4o & 15.44\% & 95.60\% & 55.52\% & 56.62\% \\
        \midrule
        Base (\llama{}) & 16.26\% & 90.40\% & 53.33\% & 53.31\%\\
        \name{} & 52.84\% & 72.32\% & 62.58\% & 55.08\%\\
        \bottomrule
    \end{tabular}
    }
    \captionof{table}{Zero-shot generalization to \ambcoqa{}, a conversational QA benchmark to identify ambiguity. We assess action-level accuracy, measuring whether the model asks a question for ambiguous inputs and provides a direct answer for non-ambiguous ones. 
    }
    \label{tab:abg_coqa}
\end{minipage}
\vspace{-5pt}
\end{figure*}



\section{Results of Simulated Experiments}
\label{sec:quantitative}

We present the results in Table~\ref{tab:results} and the takeaways are:


\xhdr{Prompt engineering is helpful, but limited in terms of performance gains and flexibility}
Proactive Base improves base model performance by encouraging follow-up questions and clarifications. For example, it increases BLEU on \doc from 32.2\% to 35.0\% and reduces read tokens by 0.31k compared to the base model. However, these gains are modest and do not fully address the challenges of multiturn collaboration. We observe that prompting strategies remain rigid, relying on predefined instructions rather than adapting dynamically to user needs. For instance, the model sometimes asks clarification questions even when unnecessary, leading to redundant interactions that disrupt conversation flow.

\xhdr{\name{} increases task performance, conversational efficiency, and engagement}
\name{} achieves \taskimprov superior task-specific performance, \efficiencyimprov more efficient conversations, and \itrimprov enhanced interactivity compared to the best baselines.
We highlight that \name{} engage in more meaningful collaborations, with ITR shows substantial gains. For \doct, the Online DPO model increases ITR from 0.46 to 0.92. 
Moreover, our framework significantly improves conversational efficiency by minimizing the content users need to review to arrive at the final solution. For \mathct, Online DPO decreases token count per conversation by 1.03k compared to the base model.

\subsection{Ablations on Reward Mechanisms (Figure~\ref{fig:ablation})}
\label{sec:ablation}

To investigate how components contribute to \name{}'s superior performance, we conduct an ablation study focusing on the reward mechanisms used during fine-tuning. 
We evaluate the following reward mechanisms:
\begin{itemize}
    \item \textbf{Variants of Multiturn-aware Reward}: We vary the forward sampling window size $w=1,2,3$ to assess their ability to capture long-term conversational effects through simulated collaborations.
     \item \textbf{Immediate Rewards} evaluate the model's immediate response based on:
        \textit{1) Helpfulness}: Assessed by an LLM judge;
        \textit{2) Extrinsic Reward}: Focuses on task-specific metrics like BLEU while ignoring intrinsic factors such as efficiency;
        \textit{3) Extrinsic + Intrinsic Reward}: Combines task-specific metrics with efficiency and interactivity measures. This can be seen as a special case of the multiturn-aware reward function with $w=0$.
    
    
\end{itemize}

\section{User evaluation with frequent users of mobile ASR: Lab study and online survey }
To evaluate the usability of our approach, we decided to conduct an in-person lab evaluation of the SpeechCompass phone case and the speech-to-text application (described in Section~\ref{subsection:app}), with frequent users of mobile transcription technology. We first conducted a large-scale online pilot study to inform the design of the in-person lab evaluation, which we conducted with eight deaf or hard-of-hearing participants, set up to mimic a realistic conversation scenario. 

\begin{figure*}
  \centering
  \includegraphics[width=0.75\linewidth]{images/second_study.pdf}
  \caption{Participants' preferences for different visualization techniques in the online survey. A) Results indicating how valuable the specific indicator would be for the user. B) Preferences for the specific indicators for speech direction.} 
  \label{fig:user_preferences_online} 
\end{figure*}


\subsection{Large-scale, online survey (n=494)} In this survey, we use screenshots of our interactive UI prototypes to solicit initial user
feedback on the potential for our proposed approach, to guide the design of a more realistic in-person lab study.

The study was conducted using the same Google Surveys deployment and screening methodology as for the foundational study, detailed in Section 3. The participants were shown different UI renderings and were asked to rate them. The large-scale online survey could only show static images of the interfaces, due to limitations of the survey tool. Out of 985 respondents we focus our analysis on the 494 participants who use captioning technology multiple times per week or more frequently. 

As shown in Figure~\ref{fig:user_preferences_online}A, the colored text was found to be valuable by 60\% of participants. Glyph indicators for speech direction, which included arrow and circle+line indicators, were found valuable by 70\%. The Edge indicator and the mini map had a less positive reception. 

To better understand which glyph indicators were favored, we also asked targeted questions about them, as shown in Figure~\ref{fig:user_preferences_online}B. \emph{Circle + line} was preferred by 13.1\% more respondents than the \emph{highlight box} (45.1\% vs 32.0\%), and the \emph{arrow} was preferred by 21.9\% more respondents than the \emph{circle + line} (51.2\% vs 29.3\%).


\subsection{Lab study (n=8)}
\alex{explain and emphasize intention}
We recruited 8 participants from our institution who were frequent users of captioning technology. Five were female, three were male, and all were deaf or hard of hearing. One participant was 25--34 years old, four were 34--44, one was 45--53, and two were 65+ (we are only allowed to collect age ranges at our institution). 


% setup: https://docs.google.com/document/d/1akr5HVMgJb8Kd9KaEZJcdXn2S0IbHhd8JdBPTE0TiA0/edit?usp=sharing
The study took place in a quiet lab over approximately 60 minutes and used the phone-case prototype (Figure~\ref{fig:pcb_design}) with our mobile ASR application (Figure~\ref{fig:phone_interfaces}). First, the participant was introduced to the technology, prototype, and the purpose of the study. Then, the participant was asked to fill out a background survey, which included demographic questions and their current use and experienced challenges with transcription technology. Afterward, the participant was introduced to different visualization scenarios with the SpeechCompass application. The participant used the SpeechCompass transcription while sitting between the two experimenters, as they all sat around a small table with the SpeechCompass phone case in the center. In each of the seven conditions, which ran for 5 minutes, the experimenters sat across from each other and had short conversations about different topics. The participants were instructed to turn off hearing aid devices if they used any, and were asked to use the SoundCompass UI and transcript to follow the conversation. The experimenters' casual conversations included topics like weekend plans, hobbies, and the weather. The seven conditions, which used the ASR, diarization, and localization functionality for different visualization techniques, are shown in Figure~\ref{fig:ui_options} and presented with more UI context in Figure~\ref{fig:phone_interfaces}. The conditions were:
\begin{enumerate}
    \item \textbf{Transcription only}. The transcribed text is shown in white on a black background. 
    
    \item \textbf{Edge indicator}. A circle (``dot'') that moves around the edge of the screen to point to the currently active speaker. The color of the dot changes based on the direction. 
    
    \item \textbf{Arrow indicator}. A glyph using a colored arrow next to a white text block. The glyph points in the direction of the associated speech. 
    
    \item \textbf{Circle + line indicator}. A glyph using a circle with a directional line next to a white text block. The glyph points in the direction of the speech associated with the text. 
    
    \item \textbf{Mini map}. A colored circle with a smaller circle (``dot'') moves around its edge to point to the currently active speaker. The color of the dot changes based on the direction. 
    
    \item \textbf{Colored text}. The text is colored based on the direction that the associated speech was coming from. 
    
    \item \textbf{Everything on}. All indicators are turned on (except the Circle + line, as it couldn't be used simultaneously with the arrow). 
\end{enumerate}

%five isolated visualization techniques, baseline with just text transcription (no speaker information), and with all visualization turned one. Minimap was shown with an arrow, since we envisioned it would be combined with other techniques. 
After participants had completed all conditions, they filled out a form that asked them to rate how desirable each of the five visual indicator styles (\textit{Edge indicator}, \textit{Arrow}, \textit{Circle  + line}, \textit{Colored map}, and \textit{Colored text}) were on a 7-point Likert scale, from \emph{-3: Strongly dislike} to \emph{+3: Strongly like}. Finally, they were asked to rate the overall value of directional feedback to the transcription experience, how strongly they would recommend these features to users of mobile captioning, and whether they had any general free-form feedback about SpeechCompass. 

\begin{figure*}
  \centering
  \includegraphics[width=0.65\linewidth]{images/study_setup.png}
  \caption{Examples of seven visualization scenarios that participants experienced in the in-person study.} 
  \label{fig:ui_options} 
\end{figure*}

%After running the scenarios, participants filled out the second part of the survey, which asked them to rate each scenario and overall impression on a scale from -3:strongly dislike to +3:strongly like. Finally, the participants filled out free form feedback about the study. 

\begin{figure*}
  \centering
  \includegraphics[width=0.65\linewidth]{images/box_plot_in_person_study_results.png}
  \caption{Boxplots of results of the in-person study. A) Participants' preferences for different visualization techniques. B) Overall opinions about augmented mobile ASR application.\alex{love these plots -- maybe to B you could also add the question about multi-people conversations as the leftmost, since it is also on same scale?} } 
  \label{fig:user_preferences} 
\end{figure*}

\subsection{Results}
Mobile transcription apps (e.g., Android Live Transcribe) were the most used communication technology for the participants. Specifically, three used them multiple times per day, one used them daily, three used them multiple times per week, and one used them rarely. 

75\% of participants frequently experienced the scenario where multiple people would get mixed up in the transcript (two multiple times per day, two daily, two multiple times per week). All participants agreed that it was challenging to participate in conversations when speech was combined from multiple people. 
%Similarly to the online survey, we asked participants to select the biggest challenges they experienced in their use of transcription technology (same options as in Figure~\ref{fig: survey-challenges}). where the majority (6/8) selected \textit{"Have to look away from the person I am talking to"}.  
\\

A Kruskal-Wallis (KW) test found a significant effect
on participant preferences for visualization techniques (P=.014).
The post-hoc pair-wise analyses using the Wilcoxon test with Bonferroni correction did, however, not show statistical significance between any pairs.
Of the five visual indicator styles that participants experienced, \emph{Colored text} was the most well-received (mean ($\bar{x})=2.625$), as it was rated positively by all the participants. %, with six strong like (+3), one like (+2), and one slight like (+1). 
The \emph{Arrow} indicator was also well-received ($\bar{x}=1.125$), with six positive, one negative, and one neutral participant.
%(one strong like (+3), three like (+2) and one slight like (+1)) and one dislike (-1) and one neutral (0)). 
Several participants noted that \emph{Arrow} and \emph{Colored text} worked well together: \emph{"Arrows + color seem to be most easier way to indicate the direction." (P2)} and \emph{"The combination of the colored text with the arrow was the most effective for me." (P7)}.

The other indicator styles received more mixed feedback. The feedback for both \emph{Edge indicator} ($\bar{x}=0.25$) and \emph{Circle + line} ($\bar{x}=-0.125$) was split between four negative and four positive participants. 
Some participants were concerned that \emph{Edge indicator} was distracting and not sufficiently discreet: \textit{"I do prefer the tool be as discrete as possible and would perhaps choose to avoid bright colored things moving around since this would be eye-catching and this kind of attention is often undesired" (P3)} and 
\textit{"Indicator moving around the edge was distracting and causing a bit of eye strain" (P2)}.
On the other hand, another participant found this style particularly useful: \textit{"the color dot moving to the speaker direction worked REALLY well" (P1)}. 
For \emph{Circle + line}, some participants struggled with its legibility: \textit{"If the analog direction indicators were larger (and translucent, or set behind)" (P8)} and \textit{"The lines in a circle were a bit slower and not as accurate (buggy)" (P5)}.
The \emph{Mini map} was rated positively by five participants and negatively by three. The most favorable participant stated: \emph{"this is also great for environmental awareness for those with single-sided hearing or no hearing at all." (P3)} and a participant who disliked the \emph{Edge indicator} commented: \emph{"steady map in the corner worked a bit better (P5)"}.

Overall, all participants agreed with the value of directional feedback ($\bar{x}=2.88$, seven Strongly agree:+3 and one Agree:+2) and would recommend these features to other users of captioning technology ($\bar{x}=2.63$, five Strongly agree:+3 and three Agree:+2): \textit{"I really liked that almost immediately I could tell that there was a speaker change, so that as soon as the text started to show up, I could better contextualize that text as attributed to a new speaker." (P1)}, \textit{"I'm very happy to see this tool being developed, it's a great addition to other speech recognition tools!" (P3)}, and \textit{"This prototype is definitely a life changer and I strongly believe that it will improve the quality of access to communication with speakers for many users" (P6)}.

\subsection{Discussion}
Consistent with the large-scale survey, the value of the diarization and localization features was immediate to all users. The participants were asked if directional guidance would be valuable in their mobile transcription experience. All eight users agreed. Also, all eight users would recommend this feature to mobile captioning users. 

While the large-scale survey helped inform our testing and exclude conditions (e.g., \emph{Highlight box}), the lab study allowed us to more rigorously evaluate the techniques in a realistic scenario. This difference became significant for the \emph{Edge indicator} and \emph{Mini map}, where issues, such as discreetness and distracting aspects, became evident during live usage. 

The results suggest that the combination of \textit{Colored text} and \textit{Arrow} would meet the preferences of most users, thanks to the balance of directional encoding and clarity. The arrow has redundant benefits too, since colored text might not always be reliably visible depending on lighting and screen conditions (e.g., strong sunlight, or dim display) and might also not be usable for colorblind users. The mixed feedback for other techniques indicates that the interface may also benefit from mechanisms that would allow users to customize the visualization style. Such customization could also apply to rendering properties, such as color, transparency, and line thickness, as some participants found \textit{Circle + line} particularly difficult to interpret. In both the large-scale survey and the in-person lab study, the \textit{Arrow} was preferred over \textit{Circle + line}. Through more customization options and extended usage in their daily lives, participants will be able to provide more nuanced feedback about these techniques. 


% Edge indicator and mini map had a less positive reception. However, they were rated more positively than those in the in-person study. Since participants didn't experience the working prototype, the discreet and distracting aspects that were observed in the in-person study were not captured. 

% In both online and in-person study, the arrow directional glyph was preferred to circle+line.



% This dichotomy demonstrates that users should be given a way to customize their experience. For example, the edge indicator received strong likes and dislikes from different participants. 


% This indicates that the interface designers should make the directional glyphs as easy to read as possible.


% The results of the online survey followed what was observed in the in-person study. Edge indicator and mini map had a less positive reception. However, they were rated more positively than those in the in-person study. Since participants didn't experience the working prototype, the discreet and distracting aspects that were observed in the in-person study were not captured. In both online and in-person study, the arrow directional glyph was preferred to circle+line.

% As indicated in the survey, the value of the diarization and localization features was immediate to all users. The participants were asked if directional feedback is valuable in their mobile transcription experience. All eight users agreed. Also, all eight users would recommend this feature to mobile captioning users. 


% \textit{"I really liked that almost immediately I could tell that there was a speaker change, so that as soon as the text started to show up, I could better contextualize that text as attributed to a new speaker." (P1)}

% P3
% Arrows + color seem to be most easier way to indicate the direction.
% \emph{"Arrows + color seem to be most easier way to indicate the direction." (P2)}
% P4
% \textit{"I'm very happy to see this tool being developed, it's a great addition to other speech recognition tools!" (P3)
% }
% \textit{"it was great to see so many options being offered" (P3)
% }
% P6 
% \textit{"This prototype is definitely a life changer and I strongly beleve that it will improve the quality of access to communication with speakers for many users" (P6)}

% P8
% The combination of the colored text with the arrow was the most effective for me.

% \emph{"The combination of the colored text with the arrow was the most effective for me." (P7)}


We present results in Figure~\ref{fig:ablation}.
Interestingly,
expanding the forward sampling window $w$ within the range generally enhances performance and efficiency by better capturing future interactions. Notably, MR with $w=2$ balances the gains and additional costs to conduct forward sampling, making it well-suited for large-scale fine-tuning. In contrast, immediate rewards, even with extrinsic and intrinsic components, fall short as they ignore long-term impact. 
These findings validate the positive impact of the forward sampling strategy in MRs.


\subsection{Case Study (Figure~\ref{fig:coding} \& ~\ref{fig:reward_preference})}
\label{sec:case}
We now offer a deeper insight into \name{}'s behavior as shown in Figure~\ref{fig:coding}. In this example,
the user request to tokenize a text file is inherently open-ended due to unspecified factors, such as the NLTK environment, tokenizer selection, and optional preprocessing steps. The base LLM makes several arbitrary assumptions, applying lowercase conversion and stopword removal without user confirmation. The user simulator later corrects these assumptions, but the final solution remains incorrect due to missing stopwords.
In contrast, \name{} actively clarifies user intent by seeking confirmation on key decisions, ensuring an aligned final solution with a 100\% Pass Rate. This approach also reduces user effort with lower token usage.

In Figure \ref{fig:reward_preference}, we compare different reward mechanisms for responses A and B of Figure~\ref{fig:coding}, to confirm that these rewards work as intended. The helpfulness rewards favor response A due to its seemingly more well-round output. Extrinsic rewards assign zero scores to both, as A provides an incorrect solution and B defers answering. Extrinsic + Intrinsic rewards slightly favor B for efficiency and engagement. Interestingly, MR assigns significantly higher rewards to B, especially at $w=2$ and $w=3$, since the response obtains useful information and provide a precise answer within the future interaction window.


\begin{table*}[t]
\centering
\vspace{-5pt}
\caption{Representative Feedback from Human Participants.}
\vspace{-10pt}
\begin{tabularx}{\textwidth}{|p{1.6cm}|X|X|}
    \hline
    \footnotesize \textbf{Model} & \footnotesize \textbf{Strengths} & \footnotesize \textbf{Weaknesses} \\
    \hline
    Base & \textit{``Follows great instruction and does exactly what I'm asking it to do.'', ``It can create a nice form of an outline to work with.''} & \textit{``The AI just agreed with me on pretty much everything. There was no discussion'', ``I didn't really like that it kept coming up with different options''} \\
    \hline
    Proactive Base & \textit{``It is very organized and it actually asks you for feedback after writing the revision.''} & \textit{``The AI seemed to be very redundant and asked me the same questions over and over.''} \\
    \hline
    \namewithspace{} & \textit{``Asking questions and making you think of things you never thought of'', ``The AI really helped me with focusing on one part of the story at a time.'', ``It helped really well to navigate what to say and what information is needed''} & \textit{``The AI assistant was not up to date enough to help with this recent sporting event.  The AI assistant also asked me to repeat information I had already given it.''} \\
    \hline
\end{tabularx}
\vspace{-5pt}
\label{tab:user_study}
\end{table*}

\subsection{Model Generalization (Table~\ref{tab:abg_coqa})}
\label{sec:generalization}
Modern foundation models are expected to generalize across a diverse range of tasks beyond their training domain. A key question is whether collaborative behaviors learned by \name{} during fine-tuning transfer effectively to new tasks without additional adaptation. 

We assess \name{}, trained with online DPO on \code (the coding assistance task), on Abg-CoQA~\cite{abg_coqa}, a question-answering (QA) benchmark where questions are labeled as ambiguous or non-ambiguous (\cf Appendix~\ref{app:abg_coqa}).
We categorize the model’s responses into two actions—asking a clarifying question or providing a direct answer—and evaluate action-level accuracy within each question type. 
As shown in Table~\ref{tab:abg_coqa}, 
GPT-4o and \llama{} rarely ask clarifying questions regardless of ambiguity. 
In contrast, 
\name{} proactively asks questions about 50\% of the time while maintaining high accuracy on unambiguous inputs.
This behavior leads to the highest Macro Accuracy across both ambiguous and non-ambiguous sets and improves Macro F1 over the base model, while leaving room for further improvement against GPT-4o. These results suggest that \textbf{\name{} effectively generalizes its learned collaborative strategies beyond its training domain}.

\section{Real-world User Study}

\xhdr{Setup}
We conduct a large-scale user study using Amazon Mechanical Turk with \numturker{} participants. Each participant is assigned a document type---randomly selected to be either blog post, creative writing, or personal statement---and chooses a topic from a predefined set. To simulate real-world scenarios where users have only a rough idea of the task, they are first asked to provide brief responses to topic-related questions.
Participants then engage in at least eight turns of conversation with an anonymized AI assistant, which can be Base, Proactive Base, or \name{}. Every three turns, they provide an interaction rating based on their experience so far. After the conversation, participants rate the final document quality and overall interaction. All ratings are in a scale from 1 to 10. We also record the total interaction duration to assess efficiency.
The detailed user study setup is provided in Appendix~\ref{app:user_study}.

\xhdr{Quantitative Results (Figure~\ref{fig:user_study})} Across multiple metrics, \name{} consistently outperforms the baselines. It achieves an average document quality score of 8.50. Specifically, 91.4\% of participants rate \name{}'s \textbf{document quality} as ``good'' (score 8–9), and 56.9\% as ``very good'' (score 9–10), compared to 88.5\% and 39.3\% for Base (\llama{}), respectively. Similarly, 63.8\% of participants find \name{} \textbf{highly engaging}, while only 42.6\% report the same for \llama{}. 

Interestingly, for \textbf{multiturn interaction}, the Base model shows a declining trend in ratings from turns 6–9, indicating reduced user experience in longer conversations. In contrast, both \name{} and Proactive Base exhibit increasing ratings over time, with \name{} consistently achieving higher average ratings every three turns compared to Proactive Base. This suggests that \name{} maintains sustained engagement more effectively.  

Moreover, \name{} improves task efficiency, reducing \textbf{time spent} by \realtimeimprov{} compared to the Base model and by 15.6\% relative to Proactive Base. While Proactive Base is prompted to maintain conciseness, it frequently asks unnecessary questions, causing lower efficiency. In contrast, \name{} strikes a more streamlined user experience.


\xhdr{Qualitative Results (Table~\ref{tab:user_study})} We collected a total of 180 strengths and 180 weaknesses across the three models. Table~\ref{tab:user_study} presents representative feedback, while we summarize here the mddels' strengths and weaknesses:
The base model generates coherent content while effectively follow user instructions, but it sometimes struggles with maintaining context in long texts, and can be overly verbose or repetitive in its responses. 
Proactive Base excels in responsiveness and adapting to user input but struggles with memory retention, and could produce repetitive or overly structured content.
On the other hand, \name{} is highly engaging, effectively guiding users through writing, adapting seamlessly to feedback. However, users also point out that \name{} can occasionally feel bland, lack of up to date information, and require additional effort to personalize the output. 
Overall, \name{} enhances collaboration by guiding users through an interactive and iterative refinement process, yet future improvements should focus on increasing personalization, creativity, and real-time knowledge integration to further optimize human-LLM collaboration.
    
