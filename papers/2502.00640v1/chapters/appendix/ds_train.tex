
\section{Dataset and Training Details}
\label{app:dataset_n_train}

\subsection{Dataset Generation for Offline Training}
\begin{figure}[t]
    \centering
    \includegraphics[width=\linewidth]{images/entity_label.pdf}
    \vspace{-1.5pc}
    \caption{Illustration of our entity-level dataset construction. We form entity-level hallucination labels according to the atomic facts extracted by \textsc{FActScore}.}
    \label{fig:data_generation}
    \vspace{-1pc}
\end{figure}

The Multiturn-aware Reward (MR) function enables the generation of high-quality synthetic conversation datasets for training. Given a user query, multiple LLM responses are sampled and ranked based on their MR scores, with higher-ranked responses designated as \textit{Chosen} and lower-ranked as \textit{Rejected}. To simulate natural conversational flow, the first turn from the chosen response's forward interaction window is appended to the prompt for the next turn, iteratively extending the conversation until completion. Solid red arrows denote data collection for Supervised Fine-Tuning (SFT), while dashed blue arrows indicate preference data construction for Direct Preference Optimization (DPO). This approach systematically curates multiturn conversations that enhance both response quality and collaborative efficiency, both of which are explicitly captured by MR.

Given (1) a user simulator LLM, \eg GPT-4o-mini, (2) an assistant LLM, GPT-4o, and (3) arbitrary tasks with defined task-specific metric, we can simulated and generate high-quality conversations following Figure~\ref{fig:flow}. We create the following training datasets in this simulated environments. 
\small
\begin{tabular}{p{6.8cm}p{.8cm}p{.8cm}p{.8cm}crr}
\toprule
\multirow{2}{*}{\textbf{Metric}} & \multicolumn{3}{c}{\textbf{Median}} & \multicolumn{3}{c}{\textbf{Statistics}} \\
\cline{2-7}
 & \textbf{Iter. 1} & \textbf{Iter. 2} & \textbf{Iter. 3} & \textbf{Comparison} & \textbf{r} & \textbf{p-value} \\
\midrule
\multirow{3}{*}{UMUX-LITE (SUS)} & \multirow{3}{*}{60.82} & \multirow{3}{*}{66.23} & \multirow{3}{*}{82.48} & 1 vs 2 & 1.429 & 0.279 \\
 &  &  &  & 2 vs 3 & 0.000 & 0.066 \\
 &  &  &  & 1 vs 3 & 0.000 & 0.031* \\
\hline
\multirow{3}{*}{NASA-TLX Score} & \multirow{3}{*}{3.92} & \multirow{3}{*}{2.83} & \multirow{3}{*}{1.92} & 1 vs 2 & 2.041 & 0.500 \\
 &  &  &  & 2 vs 3 & 0.817 & 0.138 \\
 &  &  &  & 1 vs 3 & 0.000 & 0.094 \\
\midrule
\multicolumn{7}{c}{\textbf{Self-Defined Likert Scale Questions}} \\
\midrule
\multirow{3}{*}{\makecell[l]{Iterating on my sketches was easy}} & \multirow{3}{*}{4.0} & \multirow{3}{*}{5.0} & \multirow{3}{*}{6.5} & 1 vs 2 & 3.674 & 0.844 \\
 &  &  &  & 2 vs 3 & 0.000 & 0.039* \\
 &  &  &  & 1 vs 3 & 1.429 & 0.156 \\
\hline
\multirow{3}{*}{\makecell[l]{The sketches encapsulated what I intended to achieve}} & \multirow{3}{*}{5.5} & \multirow{3}{*}{5.5} & \multirow{3}{*}{5.5} & 1 vs 2 & 2.654 & 0.783 \\
 &  &  &  & 2 vs 3 & 0.408 & 0.655 \\
 &  &  &  & 1 vs 3 & 1.414 & 0.688 \\
\hline
\multirow{3}{*}{\makecell[l]{The re-generated code aligned with my intended changes}} & \multirow{3}{*}{4.5} & \multirow{3}{*}{5.0} & \multirow{3}{*}{6.0} & 1 vs 2 & 1.225 & 0.892 \\
 &  &  &  & 2 vs 3 & 0.000 & 0.141 \\
 &  &  &  & 1 vs 3 & 0.707 & 0.063 \\
\hline
\multirow{3}{*}{\makecell[l]{I felt more control over the AI model and generated results}} & \multirow{3}{*}{5.5} & \multirow{3}{*}{5.0} & \multirow{3}{*}{5.5} & 1 vs 2 & 2.236 & 0.786 \\
 &  &  &  & 2 vs 3 & 0.000 & 0.785 \\
 &  &  &  & 1 vs 3 & 0.500 & 1.000 \\
\hline
\multirow{3}{*}{\makecell[l]{I felt more control over the whole code editing process}} & \multirow{3}{*}{5.0} & \multirow{3}{*}{5.0} & \multirow{3}{*}{6.0} & 1 vs 2 & 0.000 & 1.000 \\
 &  &  &  & 2 vs 3 & 0.866 & 0.059 \\
 &  &  &  & 1 vs 3 & 1.000 & 0.102 \\
\bottomrule
\end{tabular}


\subsection{Training Details}

We provide the hyperparameters for \name{} fine-tuning in Table~\ref{tab:hyper}. 

Notably, \name{} relies on a minimal set of hyperparameters, using the same window size and sample size for computing MRs across multiple datasets. The penalty factor on token count, $\lambda$, is set lower for \doc compared to \code and \mathc, as document lengths in \doc can vary significantly and may be easily bounded by 1 in Eq.~\ref{eq:intrinsic} if $\lambda$ is too large.
\begin{table}[h!]
    \caption{Hyperparameters}
    \label{tab:TrainingParams}
    \centering
    \begin{tabular}{l|l}
        \textbf{Parameters} & \textbf{Tuning} \\% & \textbf{Estimation}  \\
        \hline
        Sampling time                   & $0.05$  \\   %& $0.05$     \\
        Reward discount factor $\gamma$ & $0.99$  \\   %& $0.99$     \\
        Learning rate for actor         & $10^{-3}$ \\% & $10^{-3}$  \\
        Learning rate for critic        & $10^{-3}$ \\% & $10^{-3}$  \\
        $L_2$ Regularization factor     & $10^{-5}$ \\% & $10^{-4}$  \\
        Optimizer parameter $\epsilon$  & $10^{-8}$ \\% & $10^{-8}$  \\
        Minimum batch size              & $1024$  \\%   & $64$     \\
        Experience buffer length        & $10^{6}$  \\% & $10^{6}$   \\
    \end{tabular}
\end{table}