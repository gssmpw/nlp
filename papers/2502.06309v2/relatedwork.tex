\section{Literature Review}
\label{section:review}
This section briefly reviews literature that is related to this paper, as complementary to Section \ref{section:introduction}.

\textbf{Resistive element.}
A series of works seek various resistive elements that have near-constant or at least symmetric responses.
The leading candidates currently include {PCM}~\citep{Burr2016,2020legalloJPD}, {ReRAM}~\citep{Jang2014, Jang2015, stecconi2024analog}, 
{CBRAM}~\citep{Lim2018, Fuller2019}, {ECRAM}~\citep{tang2018ecram, onen2022nanosecond}, MRAM \citep{jung2022crossbar, xiao2024adapting}, FTJ \citep{guo2020ferroic} or flash memory \citep{wang2018three, xiang2020efficient, merrikh2017high}. 

However, the resistive element possessing symmetric updates may not be the best option in terms of manufacturing. 
For example, although ECRAM provides almost symmetric updates, it is still less competitive than ReRAM as ReRAM has a faster response speed and lower pulse voltage \citep{stecconi2024analog}.
The suitability of the resistive elements is evaluated based on metrics across multiple dimensions, such as the number of conductance states, retention, material endurance, switching energy, response speed, manufacturing cost, and cell size. 
Among them, this paper is only interested in the impact of response functions in the training.

\textbf{Gradient-based training on AIMC hardware.} A series of works focuses on implementing back-propagation (BP) and gradient-based training on AIMC hardware.
The seminal work \citep{gokmen2016acceleration, gokmen2017cnn} leverages the rank-one structure of the gradient and implements \AnalogSGD~by a stochastic pulse update scheme, \emph{rank-update}. Rank-update significantly accelerates the gradient descent step by avoiding dealing the gradients with $O(N^2)$ elements directly but using two $O(N)$ vectors for update instead, where $N$ is the numbers of matrix row and column.
To alleviate the {\em asymmetric update issue}, researchers also design various of \AnalogSGD~variants, \texttt{Tiki-Taka} algorithm family \citep{gokmen2020, gokmen2021, rasch2024fast}. The key components of \TT~are introducing a \emph{residual array} to stabilize the training.
Apart from the rank-update, a hybrid scheme that performs forward and backward in the analog domain but computes the gradients in the digital domain has been proposed in \citep{nandakumar2018, nandakumar2020}. Their solution, referred to as \emph{mixed-precision update}, provides a more accurate gradient signal but requires 5$\times$-10$\times$ higher overhead compared to the rank-update scheme \citep{rasch2024fast}.

Attributed to these efforts, analog training has empirically shown great promise in achieving a similar level of accuracy as digital training on chip prototype, with reduced energy consumption and training time \citep{wang2018fully, gong2022deep}.
Simultaneously, the parallel acceleration solution with AIMC hardware is also under exploration \citep{wu2024pipeline}.
Despite its good performance, it is still mysterious about when and why they work.


\textbf{Theoretical foundation of gradient-based training.} The closely related result comes from the convergence study of \TT~\citep{wu2024towards}. Similar to our work, they attempt to model the dynamic and provide the convergence properties of \AnalogSGD~and \TT. However, their work is limited to a special linear response function.
Furthermore, their paper considers a simplified version of \TT, with a hyper-parameter $\gamma=0$ (see \Cref{section:TT}). As we will show empirically and theoretically, \TT~benefits from a non-zero $\gamma$. Consequently,
We compare the results briefly in Table \ref{table:convergence-digital-vs-analog} and comprehensively in Appendix \ref{section:relation-with-wu2024}.

\begin{table*}[!h]
    \centering
    \setlength{\tabcolsep}{1.0em} % for the horizontal padding
    {\renewcommand{\arraystretch}{1.3}% for 
        \begin{tabular}{c | c | c | c }
            \toprule
                 & $\gamma$ & Generic response & Linear response \\
            \midrule
            \TT~\citep{wu2024towards}  & $=0$ & \XSolidBrush & $O\lp\sqrt{\frac{1}{K}}\frac{1}{1-33P_{\max}^2/\tau^2}\rp$  \\
            \cmidrule{1-4}
            \TT~
            [Corollary \ref{theorem:TT-convergence-scvx-final}]
            & $\ne 0$
            & $O\lp\sqrt{\frac{1}{K}}\frac{1}{H^{\texttt{TT}}_{\min}}\rp$ &$O\lp\sqrt{\frac{1}{K}}\frac{1}{1-P_{\max}^2/\tau^2}\rp$ \\
            \bottomrule
            \end{tabular}
    }
    \caption{
        Comparison between our paper and \cite{wu2024towards}.
        Mixing-coefficient $\gamma$ is a hyper-parameter of \TT.
        ``Generic response'' and ``Linear response'' columns are the convergence rates in the corresponding settings.
        $K$ represents the number of iterations.
        $H^{\texttt{TT}}_{\min}$ and $P_{\max}^2/\tau^2 < 1$ measure the saturation while the former one reduces to the latter on linear response functions.
      }
    \label{table:convergence-digital-vs-analog}
\end{table*}
  
\textbf{Energy-based models and equilibrium propagation.}
Apart from achieving explicit gradient signals by the BP, there are also attempts to train models based on \emph{equilibrium propagation} (EP, \cite{scellier2017equilibrium}), which provides a biologically plausible alternative to traditional BP. EP is applicable to a series of energy-based models, where the forward pass is performed by minimizing an energy function \citep{watfa2023energy, scellier2024energy}. The update signal in EP is computed by measuring the output difference between a free phase and an active phase. EP eliminates the need for BP non-local weight transport mechanism, making it more compatible with neuromorphic and energy-efficient hardware \citep{kendall2020training, ernoult2020equilibrium}. We highlight here that the approach to attain update signals (BP or EP) is orthogonal to the update mechanism (pulse update). Their difference lies in the objective $f(W_k)$, which is hidden in this paper. Therefore, building upon the pulse update, our work is applicable to both BP and EP.

\textbf{Physical neural network.}
The model executing on AIMC hardware, which leverages resistive crossbar array to accelerate MVM operation, is a concrete implementation of physical neural networks (PNNs, \cite{wright2022deep, momeni2024training}). 
PNN is a generic concept of implementing neural networks via a physical system in which a set of tunable parameters, such as holographic grating \citep{psaltis1990holography}, wave-based systems \citep{hughes2019wave}, and photonic networks \citep{tait2017neuromorphic}. Our work particularly focuses on training with AIMC hardware, but the methodology developed in this paper can be transferred to the study of other PNNs.