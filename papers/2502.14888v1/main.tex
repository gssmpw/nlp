%%%%%%%% ICML 2025 EXAMPLE LATEX SUBMISSION FILE %%%%%%%%%%%%%%%%%
\documentclass{article}

% Recommended, but optional, packages for figures and better typesetting:

%%%%% NEW MATH DEFINITIONS %%%%%

\usepackage{amsmath,amsfonts,bm}
\usepackage{derivative}
% Mark sections of captions for referring to divisions of figures
\newcommand{\figleft}{{\em (Left)}}
\newcommand{\figcenter}{{\em (Center)}}
\newcommand{\figright}{{\em (Right)}}
\newcommand{\figtop}{{\em (Top)}}
\newcommand{\figbottom}{{\em (Bottom)}}
\newcommand{\captiona}{{\em (a)}}
\newcommand{\captionb}{{\em (b)}}
\newcommand{\captionc}{{\em (c)}}
\newcommand{\captiond}{{\em (d)}}

% Highlight a newly defined term
\newcommand{\newterm}[1]{{\bf #1}}

% Derivative d 
\newcommand{\deriv}{{\mathrm{d}}}

% Figure reference, lower-case.
\def\figref#1{figure~\ref{#1}}
% Figure reference, capital. For start of sentence
\def\Figref#1{Figure~\ref{#1}}
\def\twofigref#1#2{figures \ref{#1} and \ref{#2}}
\def\quadfigref#1#2#3#4{figures \ref{#1}, \ref{#2}, \ref{#3} and \ref{#4}}
% Section reference, lower-case.
\def\secref#1{section~\ref{#1}}
% Section reference, capital.
\def\Secref#1{Section~\ref{#1}}
% Reference to two sections.
\def\twosecrefs#1#2{sections \ref{#1} and \ref{#2}}
% Reference to three sections.
\def\secrefs#1#2#3{sections \ref{#1}, \ref{#2} and \ref{#3}}
% Reference to an equation, lower-case.
\def\eqref#1{equation~\ref{#1}}
% Reference to an equation, upper case
\def\Eqref#1{Equation~\ref{#1}}
% A raw reference to an equation---avoid using if possible
\def\plaineqref#1{\ref{#1}}
% Reference to a chapter, lower-case.
\def\chapref#1{chapter~\ref{#1}}
% Reference to an equation, upper case.
\def\Chapref#1{Chapter~\ref{#1}}
% Reference to a range of chapters
\def\rangechapref#1#2{chapters\ref{#1}--\ref{#2}}
% Reference to an algorithm, lower-case.
\def\algref#1{algorithm~\ref{#1}}
% Reference to an algorithm, upper case.
\def\Algref#1{Algorithm~\ref{#1}}
\def\twoalgref#1#2{algorithms \ref{#1} and \ref{#2}}
\def\Twoalgref#1#2{Algorithms \ref{#1} and \ref{#2}}
% Reference to a part, lower case
\def\partref#1{part~\ref{#1}}
% Reference to a part, upper case
\def\Partref#1{Part~\ref{#1}}
\def\twopartref#1#2{parts \ref{#1} and \ref{#2}}

\def\ceil#1{\lceil #1 \rceil}
\def\floor#1{\lfloor #1 \rfloor}
\def\1{\bm{1}}
\newcommand{\train}{\mathcal{D}}
\newcommand{\valid}{\mathcal{D_{\mathrm{valid}}}}
\newcommand{\test}{\mathcal{D_{\mathrm{test}}}}

\def\eps{{\epsilon}}


% Random variables
\def\reta{{\textnormal{$\eta$}}}
\def\ra{{\textnormal{a}}}
\def\rb{{\textnormal{b}}}
\def\rc{{\textnormal{c}}}
\def\rd{{\textnormal{d}}}
\def\re{{\textnormal{e}}}
\def\rf{{\textnormal{f}}}
\def\rg{{\textnormal{g}}}
\def\rh{{\textnormal{h}}}
\def\ri{{\textnormal{i}}}
\def\rj{{\textnormal{j}}}
\def\rk{{\textnormal{k}}}
\def\rl{{\textnormal{l}}}
% rm is already a command, just don't name any random variables m
\def\rn{{\textnormal{n}}}
\def\ro{{\textnormal{o}}}
\def\rp{{\textnormal{p}}}
\def\rq{{\textnormal{q}}}
\def\rr{{\textnormal{r}}}
\def\rs{{\textnormal{s}}}
\def\rt{{\textnormal{t}}}
\def\ru{{\textnormal{u}}}
\def\rv{{\textnormal{v}}}
\def\rw{{\textnormal{w}}}
\def\rx{{\textnormal{x}}}
\def\ry{{\textnormal{y}}}
\def\rz{{\textnormal{z}}}

% Random vectors
\def\rvepsilon{{\mathbf{\epsilon}}}
\def\rvphi{{\mathbf{\phi}}}
\def\rvtheta{{\mathbf{\theta}}}
\def\rva{{\mathbf{a}}}
\def\rvb{{\mathbf{b}}}
\def\rvc{{\mathbf{c}}}
\def\rvd{{\mathbf{d}}}
\def\rve{{\mathbf{e}}}
\def\rvf{{\mathbf{f}}}
\def\rvg{{\mathbf{g}}}
\def\rvh{{\mathbf{h}}}
\def\rvu{{\mathbf{i}}}
\def\rvj{{\mathbf{j}}}
\def\rvk{{\mathbf{k}}}
\def\rvl{{\mathbf{l}}}
\def\rvm{{\mathbf{m}}}
\def\rvn{{\mathbf{n}}}
\def\rvo{{\mathbf{o}}}
\def\rvp{{\mathbf{p}}}
\def\rvq{{\mathbf{q}}}
\def\rvr{{\mathbf{r}}}
\def\rvs{{\mathbf{s}}}
\def\rvt{{\mathbf{t}}}
\def\rvu{{\mathbf{u}}}
\def\rvv{{\mathbf{v}}}
\def\rvw{{\mathbf{w}}}
\def\rvx{{\mathbf{x}}}
\def\rvy{{\mathbf{y}}}
\def\rvz{{\mathbf{z}}}

% Elements of random vectors
\def\erva{{\textnormal{a}}}
\def\ervb{{\textnormal{b}}}
\def\ervc{{\textnormal{c}}}
\def\ervd{{\textnormal{d}}}
\def\erve{{\textnormal{e}}}
\def\ervf{{\textnormal{f}}}
\def\ervg{{\textnormal{g}}}
\def\ervh{{\textnormal{h}}}
\def\ervi{{\textnormal{i}}}
\def\ervj{{\textnormal{j}}}
\def\ervk{{\textnormal{k}}}
\def\ervl{{\textnormal{l}}}
\def\ervm{{\textnormal{m}}}
\def\ervn{{\textnormal{n}}}
\def\ervo{{\textnormal{o}}}
\def\ervp{{\textnormal{p}}}
\def\ervq{{\textnormal{q}}}
\def\ervr{{\textnormal{r}}}
\def\ervs{{\textnormal{s}}}
\def\ervt{{\textnormal{t}}}
\def\ervu{{\textnormal{u}}}
\def\ervv{{\textnormal{v}}}
\def\ervw{{\textnormal{w}}}
\def\ervx{{\textnormal{x}}}
\def\ervy{{\textnormal{y}}}
\def\ervz{{\textnormal{z}}}

% Random matrices
\def\rmA{{\mathbf{A}}}
\def\rmB{{\mathbf{B}}}
\def\rmC{{\mathbf{C}}}
\def\rmD{{\mathbf{D}}}
\def\rmE{{\mathbf{E}}}
\def\rmF{{\mathbf{F}}}
\def\rmG{{\mathbf{G}}}
\def\rmH{{\mathbf{H}}}
\def\rmI{{\mathbf{I}}}
\def\rmJ{{\mathbf{J}}}
\def\rmK{{\mathbf{K}}}
\def\rmL{{\mathbf{L}}}
\def\rmM{{\mathbf{M}}}
\def\rmN{{\mathbf{N}}}
\def\rmO{{\mathbf{O}}}
\def\rmP{{\mathbf{P}}}
\def\rmQ{{\mathbf{Q}}}
\def\rmR{{\mathbf{R}}}
\def\rmS{{\mathbf{S}}}
\def\rmT{{\mathbf{T}}}
\def\rmU{{\mathbf{U}}}
\def\rmV{{\mathbf{V}}}
\def\rmW{{\mathbf{W}}}
\def\rmX{{\mathbf{X}}}
\def\rmY{{\mathbf{Y}}}
\def\rmZ{{\mathbf{Z}}}

% Elements of random matrices
\def\ermA{{\textnormal{A}}}
\def\ermB{{\textnormal{B}}}
\def\ermC{{\textnormal{C}}}
\def\ermD{{\textnormal{D}}}
\def\ermE{{\textnormal{E}}}
\def\ermF{{\textnormal{F}}}
\def\ermG{{\textnormal{G}}}
\def\ermH{{\textnormal{H}}}
\def\ermI{{\textnormal{I}}}
\def\ermJ{{\textnormal{J}}}
\def\ermK{{\textnormal{K}}}
\def\ermL{{\textnormal{L}}}
\def\ermM{{\textnormal{M}}}
\def\ermN{{\textnormal{N}}}
\def\ermO{{\textnormal{O}}}
\def\ermP{{\textnormal{P}}}
\def\ermQ{{\textnormal{Q}}}
\def\ermR{{\textnormal{R}}}
\def\ermS{{\textnormal{S}}}
\def\ermT{{\textnormal{T}}}
\def\ermU{{\textnormal{U}}}
\def\ermV{{\textnormal{V}}}
\def\ermW{{\textnormal{W}}}
\def\ermX{{\textnormal{X}}}
\def\ermY{{\textnormal{Y}}}
\def\ermZ{{\textnormal{Z}}}

% Vectors
\def\vzero{{\bm{0}}}
\def\vone{{\bm{1}}}
\def\vmu{{\bm{\mu}}}
\def\vtheta{{\bm{\theta}}}
\def\vphi{{\bm{\phi}}}
\def\va{{\bm{a}}}
\def\vb{{\bm{b}}}
\def\vc{{\bm{c}}}
\def\vd{{\bm{d}}}
\def\ve{{\bm{e}}}
\def\vf{{\bm{f}}}
\def\vg{{\bm{g}}}
\def\vh{{\bm{h}}}
\def\vi{{\bm{i}}}
\def\vj{{\bm{j}}}
\def\vk{{\bm{k}}}
\def\vl{{\bm{l}}}
\def\vm{{\bm{m}}}
\def\vn{{\bm{n}}}
\def\vo{{\bm{o}}}
\def\vp{{\bm{p}}}
\def\vq{{\bm{q}}}
\def\vr{{\bm{r}}}
\def\vs{{\bm{s}}}
\def\vt{{\bm{t}}}
\def\vu{{\bm{u}}}
\def\vv{{\bm{v}}}
\def\vw{{\bm{w}}}
\def\vx{{\bm{x}}}
\def\vy{{\bm{y}}}
\def\vz{{\bm{z}}}

% Elements of vectors
\def\evalpha{{\alpha}}
\def\evbeta{{\beta}}
\def\evepsilon{{\epsilon}}
\def\evlambda{{\lambda}}
\def\evomega{{\omega}}
\def\evmu{{\mu}}
\def\evpsi{{\psi}}
\def\evsigma{{\sigma}}
\def\evtheta{{\theta}}
\def\eva{{a}}
\def\evb{{b}}
\def\evc{{c}}
\def\evd{{d}}
\def\eve{{e}}
\def\evf{{f}}
\def\evg{{g}}
\def\evh{{h}}
\def\evi{{i}}
\def\evj{{j}}
\def\evk{{k}}
\def\evl{{l}}
\def\evm{{m}}
\def\evn{{n}}
\def\evo{{o}}
\def\evp{{p}}
\def\evq{{q}}
\def\evr{{r}}
\def\evs{{s}}
\def\evt{{t}}
\def\evu{{u}}
\def\evv{{v}}
\def\evw{{w}}
\def\evx{{x}}
\def\evy{{y}}
\def\evz{{z}}

% Matrix
\def\mA{{\bm{A}}}
\def\mB{{\bm{B}}}
\def\mC{{\bm{C}}}
\def\mD{{\bm{D}}}
\def\mE{{\bm{E}}}
\def\mF{{\bm{F}}}
\def\mG{{\bm{G}}}
\def\mH{{\bm{H}}}
\def\mI{{\bm{I}}}
\def\mJ{{\bm{J}}}
\def\mK{{\bm{K}}}
\def\mL{{\bm{L}}}
\def\mM{{\bm{M}}}
\def\mN{{\bm{N}}}
\def\mO{{\bm{O}}}
\def\mP{{\bm{P}}}
\def\mQ{{\bm{Q}}}
\def\mR{{\bm{R}}}
\def\mS{{\bm{S}}}
\def\mT{{\bm{T}}}
\def\mU{{\bm{U}}}
\def\mV{{\bm{V}}}
\def\mW{{\bm{W}}}
\def\mX{{\bm{X}}}
\def\mY{{\bm{Y}}}
\def\mZ{{\bm{Z}}}
\def\mBeta{{\bm{\beta}}}
\def\mPhi{{\bm{\Phi}}}
\def\mLambda{{\bm{\Lambda}}}
\def\mSigma{{\bm{\Sigma}}}

% Tensor
\DeclareMathAlphabet{\mathsfit}{\encodingdefault}{\sfdefault}{m}{sl}
\SetMathAlphabet{\mathsfit}{bold}{\encodingdefault}{\sfdefault}{bx}{n}
\newcommand{\tens}[1]{\bm{\mathsfit{#1}}}
\def\tA{{\tens{A}}}
\def\tB{{\tens{B}}}
\def\tC{{\tens{C}}}
\def\tD{{\tens{D}}}
\def\tE{{\tens{E}}}
\def\tF{{\tens{F}}}
\def\tG{{\tens{G}}}
\def\tH{{\tens{H}}}
\def\tI{{\tens{I}}}
\def\tJ{{\tens{J}}}
\def\tK{{\tens{K}}}
\def\tL{{\tens{L}}}
\def\tM{{\tens{M}}}
\def\tN{{\tens{N}}}
\def\tO{{\tens{O}}}
\def\tP{{\tens{P}}}
\def\tQ{{\tens{Q}}}
\def\tR{{\tens{R}}}
\def\tS{{\tens{S}}}
\def\tT{{\tens{T}}}
\def\tU{{\tens{U}}}
\def\tV{{\tens{V}}}
\def\tW{{\tens{W}}}
\def\tX{{\tens{X}}}
\def\tY{{\tens{Y}}}
\def\tZ{{\tens{Z}}}


% Graph
\def\gA{{\mathcal{A}}}
\def\gB{{\mathcal{B}}}
\def\gC{{\mathcal{C}}}
\def\gD{{\mathcal{D}}}
\def\gE{{\mathcal{E}}}
\def\gF{{\mathcal{F}}}
\def\gG{{\mathcal{G}}}
\def\gH{{\mathcal{H}}}
\def\gI{{\mathcal{I}}}
\def\gJ{{\mathcal{J}}}
\def\gK{{\mathcal{K}}}
\def\gL{{\mathcal{L}}}
\def\gM{{\mathcal{M}}}
\def\gN{{\mathcal{N}}}
\def\gO{{\mathcal{O}}}
\def\gP{{\mathcal{P}}}
\def\gQ{{\mathcal{Q}}}
\def\gR{{\mathcal{R}}}
\def\gS{{\mathcal{S}}}
\def\gT{{\mathcal{T}}}
\def\gU{{\mathcal{U}}}
\def\gV{{\mathcal{V}}}
\def\gW{{\mathcal{W}}}
\def\gX{{\mathcal{X}}}
\def\gY{{\mathcal{Y}}}
\def\gZ{{\mathcal{Z}}}

% Sets
\def\sA{{\mathbb{A}}}
\def\sB{{\mathbb{B}}}
\def\sC{{\mathbb{C}}}
\def\sD{{\mathbb{D}}}
% Don't use a set called E, because this would be the same as our symbol
% for expectation.
\def\sF{{\mathbb{F}}}
\def\sG{{\mathbb{G}}}
\def\sH{{\mathbb{H}}}
\def\sI{{\mathbb{I}}}
\def\sJ{{\mathbb{J}}}
\def\sK{{\mathbb{K}}}
\def\sL{{\mathbb{L}}}
\def\sM{{\mathbb{M}}}
\def\sN{{\mathbb{N}}}
\def\sO{{\mathbb{O}}}
\def\sP{{\mathbb{P}}}
\def\sQ{{\mathbb{Q}}}
\def\sR{{\mathbb{R}}}
\def\sS{{\mathbb{S}}}
\def\sT{{\mathbb{T}}}
\def\sU{{\mathbb{U}}}
\def\sV{{\mathbb{V}}}
\def\sW{{\mathbb{W}}}
\def\sX{{\mathbb{X}}}
\def\sY{{\mathbb{Y}}}
\def\sZ{{\mathbb{Z}}}

% Entries of a matrix
\def\emLambda{{\Lambda}}
\def\emA{{A}}
\def\emB{{B}}
\def\emC{{C}}
\def\emD{{D}}
\def\emE{{E}}
\def\emF{{F}}
\def\emG{{G}}
\def\emH{{H}}
\def\emI{{I}}
\def\emJ{{J}}
\def\emK{{K}}
\def\emL{{L}}
\def\emM{{M}}
\def\emN{{N}}
\def\emO{{O}}
\def\emP{{P}}
\def\emQ{{Q}}
\def\emR{{R}}
\def\emS{{S}}
\def\emT{{T}}
\def\emU{{U}}
\def\emV{{V}}
\def\emW{{W}}
\def\emX{{X}}
\def\emY{{Y}}
\def\emZ{{Z}}
\def\emSigma{{\Sigma}}

% entries of a tensor
% Same font as tensor, without \bm wrapper
\newcommand{\etens}[1]{\mathsfit{#1}}
\def\etLambda{{\etens{\Lambda}}}
\def\etA{{\etens{A}}}
\def\etB{{\etens{B}}}
\def\etC{{\etens{C}}}
\def\etD{{\etens{D}}}
\def\etE{{\etens{E}}}
\def\etF{{\etens{F}}}
\def\etG{{\etens{G}}}
\def\etH{{\etens{H}}}
\def\etI{{\etens{I}}}
\def\etJ{{\etens{J}}}
\def\etK{{\etens{K}}}
\def\etL{{\etens{L}}}
\def\etM{{\etens{M}}}
\def\etN{{\etens{N}}}
\def\etO{{\etens{O}}}
\def\etP{{\etens{P}}}
\def\etQ{{\etens{Q}}}
\def\etR{{\etens{R}}}
\def\etS{{\etens{S}}}
\def\etT{{\etens{T}}}
\def\etU{{\etens{U}}}
\def\etV{{\etens{V}}}
\def\etW{{\etens{W}}}
\def\etX{{\etens{X}}}
\def\etY{{\etens{Y}}}
\def\etZ{{\etens{Z}}}

% The true underlying data generating distribution
\newcommand{\pdata}{p_{\rm{data}}}
\newcommand{\ptarget}{p_{\rm{target}}}
\newcommand{\pprior}{p_{\rm{prior}}}
\newcommand{\pbase}{p_{\rm{base}}}
\newcommand{\pref}{p_{\rm{ref}}}

% The empirical distribution defined by the training set
\newcommand{\ptrain}{\hat{p}_{\rm{data}}}
\newcommand{\Ptrain}{\hat{P}_{\rm{data}}}
% The model distribution
\newcommand{\pmodel}{p_{\rm{model}}}
\newcommand{\Pmodel}{P_{\rm{model}}}
\newcommand{\ptildemodel}{\tilde{p}_{\rm{model}}}
% Stochastic autoencoder distributions
\newcommand{\pencode}{p_{\rm{encoder}}}
\newcommand{\pdecode}{p_{\rm{decoder}}}
\newcommand{\precons}{p_{\rm{reconstruct}}}

\newcommand{\laplace}{\mathrm{Laplace}} % Laplace distribution

\newcommand{\E}{\mathbb{E}}
\newcommand{\Ls}{\mathcal{L}}
\newcommand{\R}{\mathbb{R}}
\newcommand{\emp}{\tilde{p}}
\newcommand{\lr}{\alpha}
\newcommand{\reg}{\lambda}
\newcommand{\rect}{\mathrm{rectifier}}
\newcommand{\softmax}{\mathrm{softmax}}
\newcommand{\sigmoid}{\sigma}
\newcommand{\softplus}{\zeta}
\newcommand{\KL}{D_{\mathrm{KL}}}
\newcommand{\Var}{\mathrm{Var}}
\newcommand{\standarderror}{\mathrm{SE}}
\newcommand{\Cov}{\mathrm{Cov}}
% Wolfram Mathworld says $L^2$ is for function spaces and $\ell^2$ is for vectors
% But then they seem to use $L^2$ for vectors throughout the site, and so does
% wikipedia.
\newcommand{\normlzero}{L^0}
\newcommand{\normlone}{L^1}
\newcommand{\normltwo}{L^2}
\newcommand{\normlp}{L^p}
\newcommand{\normmax}{L^\infty}

\newcommand{\parents}{Pa} % See usage in notation.tex. Chosen to match Daphne's book.

\DeclareMathOperator*{\argmax}{arg\,max}
\DeclareMathOperator*{\argmin}{arg\,min}

\DeclareMathOperator{\sign}{sign}
\DeclareMathOperator{\Tr}{Tr}
\let\ab\allowbreak

\usepackage{microtype}
\usepackage{subcaption}    % For subfigure environment
% \usepackage{subfigure}
\usepackage{booktabs} % for professional tables
\usepackage{multirow}
\usepackage[table]{xcolor}      % 表格着色
\usepackage{wrapfig}

% hyperref makes hyperlinks in the resulting PDF.
% If your build breaks (sometimes temporarily if a hyperlink spans a page)
% please comment out the following usepackage line and replace
% \usepackage{icml2025} with \usepackage[nohyperref]{icml2025} above.
\usepackage{hyperref}


% Attempt to make hyperref and algorithmic work together better:
\newcommand{\theHalgorithm}{\arabic{algorithm}}

% Use the following line for the initial blind version submitted for review:

% for arxiv
% \usepackage{icml2025}

% If accepted, instead use the following line for the camera-ready submission:

% for arxiv
\usepackage[accepted]{icml2025}
\pagestyle{empty}
% For theorems and such
\usepackage{amsmath}
\usepackage{amssymb}
\usepackage{mathtools}
\usepackage{amsthm}
\usepackage{wasysym} 

% if you use cleveref..
\usepackage[capitalize,noabbrev]{cleveref}
\newcommand{\fix}{\marginpar{FIX}}
\newcommand{\new}{\marginpar{NEW}}
\newcommand{\yw}[1]{{\color{blue}\textbf{YW}: #1}}
\newcommand{\hq}[1]{{\color{orange}\textbf{HQ}: #1}}
%%%%%%%%%%%%%%%%%%%%%%%%%%%%%%%%
% THEOREMS
%%%%%%%%%%%%%%%%%%%%%%%%%%%%%%%%
\theoremstyle{plain}
\newtheorem{theorem}{Theorem}[section]
\newtheorem{proposition}[theorem]{Proposition}
\newtheorem{lemma}[theorem]{Lemma}
\newtheorem{corollary}[theorem]{Corollary}
\theoremstyle{definition}
\newtheorem{definition}[theorem]{Definition}
\newtheorem{assumption}[theorem]{Assumption}
\theoremstyle{remark}
\newtheorem{remark}[theorem]{Remark}

% Todonotes is useful during development; simply uncomment the next line
%    and comment out the line below the next line to turn off comments
%\usepackage[disable,textsize=tiny]{todonotes}
\usepackage[textsize=tiny]{todonotes}

% The \icmltitle you define below is probably too long as a header.
% Therefore, a short form for the running title is supplied here:
\icmltitlerunning{The Multi-faceted Monosemanticity in Multimodal Representations}

\begin{document}

\twocolumn[
\icmltitle{The Multi-Faceted Monosemanticity in Multimodal Representations}

\begin{icmlauthorlist}
\icmlauthor{Hanqi Yan\textsuperscript{*}}{kcl}
\icmlauthor{Xiangxiang Cui\textsuperscript{*}}{surrey}
\icmlauthor{Lu Yin}{surrey}
\icmlauthor{Paul Pu Liang}{mit}
\icmlauthor{Yulan He\textsuperscript{†}}{kcl,turing}
\icmlauthor{Yifei Wang\textsuperscript{†}}{mit}
\end{icmlauthorlist}

\icmlaffiliation{surrey}{The University of Surrey, UK}
\icmlaffiliation{kcl}{King's College London, UK}
\icmlaffiliation{turing}{The Alan Turing Institute, UK}
\icmlaffiliation{mit}{MIT CSAIL, USA}

% \icmlcorrespondingauthor{Hanqi Yan}{hanqi.yan@kcl.ac.uk}
% \icmlcorrespondingauthor{Xiangxiang Cui}{xiangxiangcui.ai@outlook.com}
% \icmlcorrespondingauthor{Lu Yin}{l.yin@surrey.ac.uk}
% \icmlcorrespondingauthor{Paul Pu Liang}{ppliang@mit.edu}
\icmlcorrespondingauthor{Yulan He}{yulan.he@kcl.ac.uk}
\icmlcorrespondingauthor{Yifei Wang}{yifei\_w@mit.edu}

\setlength{\abovecaptionskip}{-1.5ex}
\setlength{\belowcaptionskip}{-1.5ex}
\setlength{\floatsep}{-2.0ex}
\setlength{\textfloatsep}{1.0ex}
\icmlkeywords{Multimodal, Interpretability, Monosemanticity}

\vskip 0.3in
]

% \linespread{0.99}
\thispagestyle{empty} % add for arxiv
% 这条命令会打印作者列表和其上方的版权声明
\printAffiliationsAndNotice{}

%----------------------------------------------------------------------%
% 摘要
%----------------------------------------------------------------------%
\begin{abstract}


The choice of representation for geographic location significantly impacts the accuracy of models for a broad range of geospatial tasks, including fine-grained species classification, population density estimation, and biome classification. Recent works like SatCLIP and GeoCLIP learn such representations by contrastively aligning geolocation with co-located images. While these methods work exceptionally well, in this paper, we posit that the current training strategies fail to fully capture the important visual features. We provide an information theoretic perspective on why the resulting embeddings from these methods discard crucial visual information that is important for many downstream tasks. To solve this problem, we propose a novel retrieval-augmented strategy called RANGE. We build our method on the intuition that the visual features of a location can be estimated by combining the visual features from multiple similar-looking locations. We evaluate our method across a wide variety of tasks. Our results show that RANGE outperforms the existing state-of-the-art models with significant margins in most tasks. We show gains of up to 13.1\% on classification tasks and 0.145 $R^2$ on regression tasks. All our code and models will be made available at: \href{https://github.com/mvrl/RANGE}{https://github.com/mvrl/RANGE}.

\end{abstract}


\section{Introduction}
Backdoor attacks pose a concealed yet profound security risk to machine learning (ML) models, for which the adversaries can inject a stealth backdoor into the model during training, enabling them to illicitly control the model's output upon encountering predefined inputs. These attacks can even occur without the knowledge of developers or end-users, thereby undermining the trust in ML systems. As ML becomes more deeply embedded in critical sectors like finance, healthcare, and autonomous driving \citep{he2016deep, liu2020computing, tournier2019mrtrix3, adjabi2020past}, the potential damage from backdoor attacks grows, underscoring the emergency for developing robust defense mechanisms against backdoor attacks.

To address the threat of backdoor attacks, researchers have developed a variety of strategies \cite{liu2018fine,wu2021adversarial,wang2019neural,zeng2022adversarial,zhu2023neural,Zhu_2023_ICCV, wei2024shared,wei2024d3}, aimed at purifying backdoors within victim models. These methods are designed to integrate with current deployment workflows seamlessly and have demonstrated significant success in mitigating the effects of backdoor triggers \cite{wubackdoorbench, wu2023defenses, wu2024backdoorbench,dunnett2024countering}.  However, most state-of-the-art (SOTA) backdoor purification methods operate under the assumption that a small clean dataset, often referred to as \textbf{auxiliary dataset}, is available for purification. Such an assumption poses practical challenges, especially in scenarios where data is scarce. To tackle this challenge, efforts have been made to reduce the size of the required auxiliary dataset~\cite{chai2022oneshot,li2023reconstructive, Zhu_2023_ICCV} and even explore dataset-free purification techniques~\cite{zheng2022data,hong2023revisiting,lin2024fusing}. Although these approaches offer some improvements, recent evaluations \cite{dunnett2024countering, wu2024backdoorbench} continue to highlight the importance of sufficient auxiliary data for achieving robust defenses against backdoor attacks.

While significant progress has been made in reducing the size of auxiliary datasets, an equally critical yet underexplored question remains: \emph{how does the nature of the auxiliary dataset affect purification effectiveness?} In  real-world  applications, auxiliary datasets can vary widely, encompassing in-distribution data, synthetic data, or external data from different sources. Understanding how each type of auxiliary dataset influences the purification effectiveness is vital for selecting or constructing the most suitable auxiliary dataset and the corresponding technique. For instance, when multiple datasets are available, understanding how different datasets contribute to purification can guide defenders in selecting or crafting the most appropriate dataset. Conversely, when only limited auxiliary data is accessible, knowing which purification technique works best under those constraints is critical. Therefore, there is an urgent need for a thorough investigation into the impact of auxiliary datasets on purification effectiveness to guide defenders in  enhancing the security of ML systems. 

In this paper, we systematically investigate the critical role of auxiliary datasets in backdoor purification, aiming to bridge the gap between idealized and practical purification scenarios.  Specifically, we first construct a diverse set of auxiliary datasets to emulate real-world conditions, as summarized in Table~\ref{overall}. These datasets include in-distribution data, synthetic data, and external data from other sources. Through an evaluation of SOTA backdoor purification methods across these datasets, we uncover several critical insights: \textbf{1)} In-distribution datasets, particularly those carefully filtered from the original training data of the victim model, effectively preserve the model’s utility for its intended tasks but may fall short in eliminating backdoors. \textbf{2)} Incorporating OOD datasets can help the model forget backdoors but also bring the risk of forgetting critical learned knowledge, significantly degrading its overall performance. Building on these findings, we propose Guided Input Calibration (GIC), a novel technique that enhances backdoor purification by adaptively transforming auxiliary data to better align with the victim model’s learned representations. By leveraging the victim model itself to guide this transformation, GIC optimizes the purification process, striking a balance between preserving model utility and mitigating backdoor threats. Extensive experiments demonstrate that GIC significantly improves the effectiveness of backdoor purification across diverse auxiliary datasets, providing a practical and robust defense solution.

Our main contributions are threefold:
\textbf{1) Impact analysis of auxiliary datasets:} We take the \textbf{first step}  in systematically investigating how different types of auxiliary datasets influence backdoor purification effectiveness. Our findings provide novel insights and serve as a foundation for future research on optimizing dataset selection and construction for enhanced backdoor defense.
%
\textbf{2) Compilation and evaluation of diverse auxiliary datasets:}  We have compiled and rigorously evaluated a diverse set of auxiliary datasets using SOTA purification methods, making our datasets and code publicly available to facilitate and support future research on practical backdoor defense strategies.
%
\textbf{3) Introduction of GIC:} We introduce GIC, the \textbf{first} dedicated solution designed to align auxiliary datasets with the model’s learned representations, significantly enhancing backdoor mitigation across various dataset types. Our approach sets a new benchmark for practical and effective backdoor defense.




% \section{Problem Setup}

% In this work, we primarily study the visual-language features extracted by CLIP models. CLIP (Contrastive Language-Image Pretraining) consists of an image encoder $f_i$ and a text encoder $f_t$, which are jointly pretrained on a large set of image-text pairs, denoted as $(x_i, x_t)$, using contrastive learning~\citep{radford2021learning}. Specifically, CLIP encourages the representations of positive pairs to be aligned in a shared embedding space (i.e., matching image-text pairs), while pushing apart the representations of negative pairs (i.e., non-matching image-text pairs).
% After pretraining, these representations can be leveraged for various downstream tasks. For example, multimodal large language models (LLMs) such as LLaVA~\citep{liu2024llava} and BLIP~\citep{li2022blip} are built upon CLIP representations. Additionally, many text-to-image generation models, such as Stable Diffusion~\citep{Rombach_2022_CVPR}, utilize CLIP's shared embedding space to facilitate the generation process. Therefore, CLIP representations serve as a fundamental basis for representing images and text in a shared space, enabling a wide range of multimodal applications.

% \textbf{CLIP Backbones.} 

\section{Towards Multimodal Monosemanticity}
\label{sec:disentangle}
In this section, we build a pipeline to extract monosemantic multimodal features and evaluate interpretability of these features. We also analyze the \textbf{modality relevance} within the extracted features using the proposed Monosemantic Dominance Score (MDS).

We consider two CLIP models, the canonical ViT-B-32 CLIP model from OpenAI~\citep{radford2021learning} and a popular CLIP variant, DeCLIP~\citep{li2022supervision}. Beyond multimodal supervision (image-text pairs), DeCLIP also incorporates single-modal self-supervision (image-image pairs and text-text pairs) for more efficient joint learning. We hypothesize that, with the incorporation of self-supervision tasks, DeCLIP is able to extract more single-modal features from the data, enhancing its interpretability and alignment with modality-specific characteristics.

% for evaluating multimodal representations through monosemanticity. It consists of the following components: 1) two methods for \textbf{extracting} monosemantic multimodal features: multimodal SAE and multimodal NCL; 2) \textbf{evaluating} the interpretability of multimodal features with new scalable interpretability metrics; and 3) characterizing the \textbf{modality relevance} in extracted features with the proposed Monosemantic Relevance Score (MRS).

% In this paper, we consider two CLIP models for comparison. The first is the canonical ViT-B-32 CLIP model by OpenAI~\citep{radford2021learning}, pretrained with a pure image-text contrastive loss. Additionally, we consider a popular CLIP variant, DeCLIP~\citep{li2022supervision}, which integrates multimodal supervision (image-text pairs) with single-modal self-supervision (image-image pairs and text-text pairs) for more efficient joint learning. We hypothesize that, with the incorporation of self-supervision tasks, DeCLIP is able to extract more single-modal features from the data, enhancing its interpretability and alignment with modality-specific characteristics.


\subsection{Interpretability Tools for Multimodal Monosemantic Feature Extraction}
Features in deep models are observed to be quite \emph{polysemantic} \citep{olah2020zoom}, in the sense that activating samples along each feature dimension often contain multiple unrelated semantics. Therefore, we first need to disentangle the CLIP features to obtain \emph{monosemantic features}. Building on recent progress in achieving monosemanticity in self-supervised models, we study two methods aimed at improving multimodal monosemanticity.
% Although CLIP features contain rich multimodal semantics, we often find them to be quite \emph{polysemantic} \citep{olah2020zoom}, 
% in the sense that activating samples along each feature dimension often contain multiple unrelated semantics.

% Therefore, as a first step, we need to disentangle the CLIP features to have \emph{monosemantic features}. Borrowing from the recent progress in monosemanticity in self-supervised models, we study the two methods to attain better multimodal monosemanticity.

\textbf{Multimodal SAE.} Sparse Autoencoders (SAEs) \citep{cunningham2023sparse} are a new scalable interpretability method,
% that decomposes polysemantic \emph{neurons} into interpretable monosemantic {features}, 
that has shown success in decomposing polysemantic \emph{neurons} into interpretable, monosemantic {features} across various LLMs  \citep{templeton2024scaling,gao2024scaling,lieberum2024gemma}. Here, we train a \emph{multimodal SAE (MSAE)} $g^+$ by using a \textbf{single} SAE model to reconstruct both image and text representations. Specifically, we adopt a top-K SAE model \citep{makhzani2013k,gao2024scaling},
\begin{align}
&\text{(latent) }z=\operatorname{TopK}\left(W_{\text {enc }}\left(x-b_{\text {pre }}\right)\right),\quad \\
&\text{(reconstruction) }\hat{x}=W_{\mathrm{dec}} z+b_{\mathrm{pre}}, 
\end{align}
and train it with a \emph{multimodal reconstruction objective}.
\begin{equation}
\small
    \gL_{\rm M-SAE}(g)=\E_{(x_i,x_t)\sim \gP}\left[(x_i-g(x_i))^2+(x_t-f(x_t))^ 2\right],
    \nonumber
\end{equation}
with $W_{\text {enc }} \in \mathbb{R}^{n \times d}, b_{\text {enc }} \in \mathbb{R}^n, W_{\text {dec }} \in \mathbb{R}^{d \times n}$, and $b_{\text {pre }} \in \mathbb{R}^d$. 
In this way, the sparse latent feature $z\in\sR^n$ can encode multimodal representations from both modalities.

\textbf{Multimodal NCL.} 
% A key limitation of leveraging SAE for extracting multimodal features is that its reconsstruction objective is still essentially singe-modal, and thus does not take the multimodal nature of CLIP into consideration.
Inspired by the interpretable self-supervised loss with a non-negative constraint (NCL) proposed by \citep{wang2024ncl} to extract sparse features, we adapt it to enhance multimodal interpretability. 
Specifically, given a pretrained CLIP model with an image encoder $f_i$ and a text encoder $f_t$, 
we train a shared MLP network (of similar size to SAE) on top of the encoder outputs using a \emph{Multimodal NCL} loss:
\begin{equation}
\scriptsize
    L_{\rm M-NCL}(g^+)=-\E_{x_i,x_t\sim\gP}\log\frac{\exp(g^+(f_i(x_i))^\top g^+(f_t(x_t))}{E_{x^-_t}\exp(g^+(f_i(x_i))^\top g^+(f_t(x^-_t))}.
\end{equation}
 The MLP network $g^+:\sR^d\to\sR^n$ is designed to have  non-negative outputs, e.g.,
\begin{equation}
\small
   g^+(x)=\operatorname{ReLU}(W_2\operatorname{ReLU}(W_1x+b_1)+b_2),\ \forall\ x\in\sR^d. 
\end{equation}
\citet{wang2024ncl} showed that the non-negative constraints allows NCL to extract highly sparse features, significantly improving monosemanticity.

% Here, an SAE consists of the bias encoder weight $W_{\rm enc}$ and bias $b_{\rm pre}$, a sparse activation $\operatorname{Topk}$ \citep{makhzani2013k}, the decoder weights $W_{\rm dec}$ and bias $b_{\rm pre}$.
% that only activate
% recon the learning objective
% separate SAEs for every Specifically, for  we take the features
% for  the top-$k$ SAE \citep{gao2024scaling} as an example, we apply an


% \yw{we need illustrative examples of original (polysemantic), NCL and SAE features.}

\subsection{Measures for Multimodal Interpretability}
% Aside from the interpretability tools as above, we also need quantitative measures of the interpretability (i.e., monosemanticity) of the extracted  features that are scalable to large multimodal models. 
Existing quantitative interpretability measures \citep{bills2023language} often rely on expensive models (like GPT-4o) and face challenges with scalability and precision \citep{gao2024scaling}, hindering advancements in open science. This motivates us to propose more scalable alternatives, as outlined below.

% When the most activated samples for a neuron can be summarised with the same features, for example, portrait or rural scenery, the neuron is said to interpretable for this feature. We therefore study the modality interpretability for different types of neurons.

% \begin{itemize}
% \item \textbf{Auto-Interpretebility Score}. To measure the interpretability of neuron, OpenAI prompts an advanced LLM to predict the neuron's activations for multiple input tokens and calculate the coefficient between the predicted activation and true activation~\citep{bills2023language}. Specially, the input prompt are pairs of tokens and unknown activation, i.e., $\{t_{i}: \text{unknown}\}$. Then, they identify the probability of \textit{Unknown} over token 0-10 and then normalise the probability into a 0-1 scalar. Due to the infeasbility of obtaining probability of input tokens for closed GPT4-o, we directly prompt the GPT4-o to generate sample-activation pair given a sentence-level explanation. The explaination is generated by GPT4-o by prompting with true sample-activation pairs.
% \item 

\textbf{Embedding-based Similarity.} 
% Considering the high cost of calling OpenAI Api and reproduce issue, we propose a embedding-based similarity measurement. 
% Instead of leveraging costly generative models, 
We propose a scalable interpretability measure based on embedding models that can be applied to both images and text.\footnote{We use the Vision Transformer (\href{https://huggingface.co/google/vit-base-patch16-224-in21k}{ViT-B-16-224-in21k}) for image embeddings and the Sentence Transformer (\href{https://huggingface.co/sentence-transformers/all-MiniLM-L6-v2}{all-MiniLM-L6-v2}) for text embeddings.} 
For each image/text feature $z$, we select the top $m$ activated image/text samples for this dimension, and denote their embeddings as $Z_+\in\sR^{m\times d}$. Similarly, $K$ random samples are encoded into $Z_-\in\sR^{m\times d}$ as a baseline. Then, we calculate the inter-sample similarity between the selected samples, $S_+=Z_+Z_+^\top\in\sR^{m\times m}$ and $S_-=Z_-Z_-^\top\in\sR^{m\times m}$. The monosemanticity of $z$ is measured by calculating the relative difference between the two similarity scores: 
\begin{equation}
\small
I(z)=\frac{1}{m(m-1)}\sum_{i\neq j}\frac{(S_+)_{ij}-(S_-)_{ij}}{(S_-)_{ij}}. 
\end{equation}
The overall interpretability score is the average across all features: $\bar I=\sum_{i=1}^nI(z_i)$.
A higher score indicates that the extracted features exhibit more consistent semantics.
% $$.
% sa calculate the sample
% and obtain their features $\bm{S}_{r}\in\mathbb{R}^{K \times n}$. We calculate the dot-product among all the pair of the encoded samples in $S$ and $S_{r}$ to measure the similarity between every two samples, respectively. Finally, the average of relative difference over the two dot-product matrix (diagonal elements are excluded) are used as our interpretability indicator, $(\bm{E}-\bm{E}_{r})/\bm{E}_{r}$

\textbf{WinRate.}  
Since the representations from different embedding models (e.g., vision and text) are not directly comparable, we propose similarity WinRate, a binary version of the relative similarity score. This is calculated by counting the percentage of elements in $S_+$ that are larger than those in $S_-$:
\begin{equation}
\small
W(z)=\frac{1}{m(m-1)}\sum_{i\neq j}\vone_{[(S_+)_{ij}>(S_-)_{ij}]}. 
\end{equation}
The overall WinRate is given by $\bar W=\sum_{i=1}^nW(z_i)$. 
A high WinRate indicates better monosemanticity.
% \end{itemize}
% \vspace{-3mm}


\begin{table}[h]
    \centering
    \small
    \caption{Average interpretability scores for features extracted from the four models. A larger $|\Delta|$ indicates stronger alignment % that the features are more aligned 
    with a single modality. 
    % \yw{we can highlight the improvement with color and percentage, eg +50\%.}
    }
    \resizebox{0.48\textwidth}{!}{%
    \begin{tabular}{lcc|ccc}
    \toprule
    \multirow{2}{*}{\textbf{Model}}& \multicolumn{2}{c}{\textbf{Similarity}} & \multicolumn{3}{c}{\textbf{WinRate}}\\
    \cmidrule{2-6}
 &Image & Text & Image & Text&$|\Delta|(\text{img}-\text{txt})$\\
    \midrule
    CLIP&  0.113&0.451& 0.652&0.594&0.058\\
    DeCLIP & 0.058&	-0.073& 0.615&	0.457&\textbf{0.158}\\
    CLIP+NCL & \textbf{0.161}&	\textbf{0.592}& \textbf{0.727}&\textbf{0.608}&\textbf{0.119}\\
    CLIP+SAE & \textbf{0.120} &0.244&\textbf{0.667}& 0.540&\textbf{0.127}\\
    \bottomrule
    \end{tabular}
    }
    % \vspace{-8mm}
    \label{tab:interpret_metrics_allmodels}
\end{table}

\textbf{Results.}
% For each neuron, we collect its activated top10 images and texts, and apply the LLM-free metrics, i.e., embed-simi and win-rate to measure the intepretability of each neuron. Noted that some neurons could be only sensitive to one modality, so they could be 
% inefficient in capturing interpretable features in that modality. 
From the interpretability distribution results %of the features extracted 
in Table~\ref{tab:interpret_metrics_allmodels}, we observe the following:
(1) The features extracted using NCL exhibit the highest overall monosemanticity; and (2) compared to CLIP, all other models produce features that are more aligned with a single modality.
%have more single-modality aligned features.
% All the methods highlight the distinctiveness of neuronal perception across different modalities indicated by a larger $|\Delta|$.

% We first evaluate the average interpretability of the features without considering their predominant modality. 
% Next, we split the neurons into different groups and analyze the modality-specific features.
\subsection{Grouping Modality in Multimodal Representations}
Building upon the monosemantic features identified above, we can take a closer look at the distribution of \emph{modality} within each feature of the multimodal CLIP model.
% \yw{motivate the study of modality dominance.[See Above]}

\textbf{Modality Dominance  Score (MDS).}
We propose a metric to determine the predominant modality for each neuron. Specially, we feed $m$ input-output pairs to CLIP and extract the corresponding image features $Z_{I}\in \mathbf{R}^{m \times n}$ and text features $Z_{T}\in \mathbf{R}^{m\times n}$. For each feature $k\in[1,\dots,n]$, we calculate the relative activation between the image and text features across the $m$ inputs as follows: 
\begin{align}
\small
   R(k)=\frac{1}{m}\sum_{i=1}^m\frac{(Z_I)_{ik}}{(Z_I)_{ik}+(Z_T)_{ik}}. \nonumber 
\end{align}
The ratio $R(k)$ indicates how much the feature $k$ is activated in the image modality. Based on this value, we split all $n$ features into three groups according to their dominant modality with the standard deviation $\sigma$: 
 % $\texttt{ImgD (Img Dominant): }  r_{i} > \mu+\sigma;\quad \texttt{TextD (Text Dominant): } r_{i} <\mu-\sigma;\quad \texttt{CrossM (Cross Modality): } \mu-\sigma <r_{i} <\mu+\sigma$.
 \begin{align}
\vspace{-5mm}
    \texttt{ImgD: } & r_{i} > \mu+\sigma; \nonumber \\
    \texttt{TextD: } & r_{i} <\mu-\sigma; \nonumber \\ \texttt{CrossD: } & \mu-\sigma <r_{i} <\mu+\sigma.
    \nonumber
\vspace{-10mm}
\end{align}
% \vspace{-5mm}
We anticipate that \texttt{ImgD} features are mostly activated by images and \texttt{TextD} features by text, while \texttt{CrossD} features are \emph{simultaneously} activated by both image and text when paired.

\begin{figure}[t]
    \centering
    % First row: Two figures
    \begin{subfigure}{0.22\textwidth}
        \centering
        \includegraphics[width=\textwidth,trim={0 18 0 0},clip]{i2t_ratio_openclip.pdf}
        \caption{CLIP}
        \label{fig:1}
    \end{subfigure}
    \hspace{-3mm}
    \begin{subfigure}{0.22\textwidth}
        \centering
        \includegraphics[width=\textwidth,trim={0 18 0 0},clip]{i2t_ratio_declip_step0_paper.pdf}
        \caption{DeCLIP}
        \label{fig:2}
    \end{subfigure}
    \vspace{-3mm}
    % \vskip\baselineskip % Vertical space between rows
    % Second row: Two figures
    \begin{subfigure}{0.22\textwidth}
        \centering
        \includegraphics[width=\textwidth,trim={0 18 0 0},clip]{i2t_ratio_openclip_ncl_step2000_paper.pdf}
        \caption{CLIP+NCL}
        \label{fig:3}
    \end{subfigure}
    \hspace{-3mm}
    \begin{subfigure}{0.22\textwidth}
        \centering
        \includegraphics[width=\textwidth,trim={0 18 0 0},clip]{i2t_ratio_openclip_sae_step200_paper.pdf}
        \caption{CLIP+SAE}
        \label{fig:4}
    \end{subfigure}
    \caption{\footnotesize MDS distributions for different Language-Vision Models. Left to right: CLIP, DeCLIP, CLIP+NCL, CLIP+SAE.}
    \label{fig:four_figures}
\end{figure}
% \begin{figure}[t]
% % \begin{wrapfigure}{l}{0.7\textwidth}
%     \centering
%     \begin{subfigure}{0.23\textwidth}
%         \centering
%         \includegraphics[width=\textwidth,trim={0 18 0 0},clip]{figures/i2t_ratio_openclip.pdf}
%         \label{fig:1}
%     \end{subfigure}
%      \hfill
%     \begin{subfigure}{0.23\textwidth}
%         \centering
%         \includegraphics[width=\textwidth,trim={0 18 0 0},clip]{figures/i2t_ratio_declip_step0_paper.pdf}
%         % \caption{DeCLIP}
%         \label{fig:2}
%     \end{subfigure}
%     \vskip\baselineskip
%     \begin{subfigure}{0.23\textwidth}
%         \centering
%         \includegraphics[width=\textwidth,trim={0 18 0 0},clip]{figures/i2t_ratio_openclip_ncl_step2000_paper.pdf}
%         % \caption{CLIP+NCL}
%         \label{fig:3}
%     \end{subfigure}
%      \hfill
%     \begin{subfigure}{0.23\textwidth}
%         \centering
%         \includegraphics[width=\textwidth,trim={0 18 0 0},clip]{figures/i2t_ratio_openclip_sae_step200_paper.pdf}
%         % \caption{CLIP+SAE}
%         \label{fig:4}
%     \end{subfigure}
%     % \vspace{-5mm}
%     \caption{\footnotesize MDS distributions for different Language-Vision Models. Left to right: CLIP, DeCLIP, CLIP+NCL, CLIP+SAE. }
%     \label{fig:four_figures}
    
% \end{figure}

\textbf{MDS Results in Fig~\ref{fig:four_figures}.} Interestingly, we find that    
CLIP, which is only trained on an image-text contrastive learning objective, contains {a spectrum of features with different modality dominance}. 
Specifically, its distribution is skewed towards the text domain ($<50\%$), but with a long tail in the image domain, suggesting that while most CLIP features are  text-dominant, some are image-dominant.
DeCLIP, on the other hand, shows a more balanced and less centered distribution, %features are less skewed and less centered, showing better coverage of 
covering image-dominant, text-dominant, and cross-model features. This suggests that DeCLIP,  through self-supervision, extracts more modality-specific features, which might be overlooked by pure vision-language contrastive models like CLIP. %from a mechanistic interpretability perspective that self-supervision extract more modality-specific features that might be overlooked by pure visual-language contrastive models like CLIP. 
The extracted features from NCL and SAE exhibit less skewness, with SAE showing a more balanced distribution, indicating its superior capability to extract diverse monosemantic features.

% The extracted features from NCL and are less skewed, suggesting that CLIP features do learn image-dominant and text-dominant features, while these features are largely hidden in the polysemantic cross-modal neurons in original CLIP. For SAE, we notice that the extracted features are more balanced, indicating that SAE is better at extracting diverse monosemantic features.

% \begin{wraptable}{r}{0.4\textwidth}
% \vspace{-3mm}
%     \centering
%     \small
%     \begin{tabular}{r|p{1cm}p{1cm}p{1cm}}
%     \toprule
%     \textbf{Models}&\textbf{\#TextD} &\textbf{\#CrossM} &\textbf{\#ImgD}\\
%     \midrule
%     CLIP  & 46&831&147  \\
%     \midrule
%     DeCLIP     & 177&660&187\\
%     \midrule
%     CLIP+NCL & 121&766&137\\
%     CLIP+SAE & 248& 547&229 \\
%     \bottomrule
%     \end{tabular}
%     \vspace{-2mm}
%     \caption{The number of three types of neurons for different models. On top of CLIP, the other methods increase the neurons' distincts to either image or text modality.}
%     \label{tab:num_neuron_mmodality}
% \end{wraptable}

% \vspace{-3mm}
% \section{ Case Studies and Applications}  
% \vspace{-2mm}
% \textbf{Comparing}
Using the protocol developed above, we have separated  the neurons into three distinct groups, enabling a more in-depth quantitative and quantitative analysis of the relationships and gaps between modalities in a data-driven approach.
\subsection{Understanding Modality-specific Features}

The implications of modality dominance significantly affect feature interpretability across different modalities. %is its influence on feature interpretability at different modalities. 
Ideally, when presented with image samples, \texttt{ImgD} neurons should be more effective at capturing concrete and consistent features than \texttt{TextD} neurons, and the same holds true %. Similar 
for text neurons with textual input samples. 

\textbf{Quantitative interpretability.} We measure both visual and textual monosemanticity. Specially, for image inputs, we calculate the \textsl{visual monosemanticity} by evaluating the interpretability difference between \texttt{ImgD} and \texttt{TextD}, i.e., EmbedSimi(\texttt{ImgD})-EmbedSimi(\texttt{TextD}). For text inputs, we calculate \textsl{textual monosemanticity} using EmbedSimi(\texttt{TextD})-EmbedSimi(\texttt{ImgD}). We have the following observations from Table~\ref{tab:relative_interpret}: (1) For image inputs, all models except CLIP demonstrate a positive visual monosemanticity, indicating better performance than the other two types of neurons. (2) For text inputs, both NCL and SAE excel in capturing  monosemantic textual features compared to the other two models. (3) SAE stands out as the best model for capturing both visual and textual monosemantic features. 
\begin{table}[h]
\caption{The visual and textual monosemanticity. A higher value indicates that \texttt{ImgD} captures more visual than linguistic features, and vice versa for \texttt{TextD}. NCL and SAE prompts the modality-specific monosemanticity in both image and text. }
\resizebox{0.48\textwidth}{!}{%
    \begin{tabular}{r|c|c|c|c}
    \toprule[1pt]
     \textbf{Modality}    & CLIP&DeCLIP&CLIP+NCL&CLIP+SAE \\
     \midrule
    \textbf{Image} & -0.118 & 0.070 & \textbf{0.197}& \underline{0.135} \\
    \textbf{Text} & -0.07 & -0.059 &\underline{0.132} &\textbf{0.439}\\
    \bottomrule[1pt]
    \end{tabular}
    }
% \vspace{-2mm}
\label{tab:relative_interpret}
\end{table}
% \begin{wrapfigure}{r}{0.35\textwidth}
%     \centering
%     % Left subfigure
%     \begin{subfigure}{0.35\textwidth}
%         \centering
%         \includegraphics[width=\textwidth,trim={0 8 0 5},clip]{ICLR 2025 Template/figures/relative_interpret_imginputs.pdf} % Change to your figure
%         % \caption{\footnotesize Relative Intepretability Score on \textbf{Image} Inputs.}
%         \label{fig:left}
%     \end{subfigure}

%     \begin{subfigure}{0.4\textwidth}
%         \centering
%         \includegraphics[width=\textwidth,trim={0 8 0 6},clip]{ICLR 2025 Template/figures/relative_interpret_textinputs.pdf} % Change to your figure
%         % \caption{\footnotesize Relative Intepretability Score on \textbf{Text} Inputs.}
%         \label{fig:right}
%     \end{subfigure}
%     \vspace{-10mm}
%     \caption{\footnotesize Visual and Textual Monosemanticity over the four models.}
%     \label{fig:relative_interpret}
% \end{wrapfigure}
% Such desirable patterns are illustrated in the first row of Table~\ref{tab:img_inter_metric} and Table~\ref{tab:txt_inter_metric}: darker gray represents a higher desirable interpretability score. 
% To verify our hypothesis, we conduct interpretability evaluation on top of both image and text samples and examine the interpretability of different types of neurons by feeding image and text samples, respectively.
% From Table~\ref{tab:img_inter_metric} and Table~\ref{tab:txt_inter_metric}, we have the following the observations: (1) The three types of neurons in all the methods can capture consistent \textit{visual features} as their WinRate are all above 0.5, i.e., the most activated samples for a neuron are more similar to each other than to a set of random samples. DeCLIP, however, is inefficient in capturing consistent \textit{textual} features. (2) CLIP+NCL achieves the the overall best monosemanticity across two modality inputs. Specially, it has best SCS and WinRate for \texttt{ImgD} on image inputs; and best SCS for \texttt{TextD} on text inputs. (3) For image input, except for CLIP, the \texttt{ImgD} neurons in other three models have more visually monosemantic scores than the other two types of neurons. For text input, the \texttt{TextD} neurons in both NCL and SAE capture better monosemantic textual features than the other two types of neurons.
% \definecolor{gray}{rgb}{0.9,0.9,0.9}
% \definecolor{darkgray}{rgb}{0.8,0.8,0.8}
% \definecolor{darkergray}{rgb}{0.65,0.65,0.65}
% \begin{minipage}{0.42\textwidth}
% \centering
% \resizebox{1\textwidth}{!}{%
%     \begin{tabular}{l|c|c|c}
%     \toprule
%     Model&\cellcolor{gray}TextD & \cellcolor{darkgray}CrossM & \cellcolor{darkergray}ImgD \\
%     \midrule
%     \multicolumn{4}{c}{Embedding-based Similarity} \\
%         \midrule
%     CLIP &\cellcolor{darkergray} 0.126&\cellcolor{gray}	0.111&\cellcolor{darkgray}0.118 \\
%     DeCLIP & \cellcolor{darkgray}0.060&	\cellcolor{gray}0.054&	\cellcolor{darkergray}0.070\\
%     CLIP+NCL & \cellcolor{darkgray}0.160&	\cellcolor{gray}0.155&	\cellcolor{darkergray}{0.197}\\
%     CLIP+SAE &\cellcolor{gray}0.096&\cellcolor{darkgray}0.125&\cellcolor{darkergray}0.135\\
%     \midrule
%     \multicolumn{4}{c}{WinRate} \\
%      \midrule
%     CLIP     &\cellcolor{darkergray}0.686&	\cellcolor{gray}0.645&\cellcolor{darkgray}0.679 \\
%     DeCLIP & \cellcolor{darkgray}0.626&\cellcolor{gray}0.607&\cellcolor{darkergray}0.632\\
%     CLIP+NCL &\cellcolor{gray} 0.722&	\cellcolor{darkgray}0.723	&\cellcolor{darkergray}0.754\\
%     CLIP+SAE & \cellcolor{gray}0.623&\cellcolor{darkgray}	0.678&\cellcolor{darkergray}	0.688\\
%     \bottomrule
%     \end{tabular}
%     }
%     \captionof{table}{Interpretability measured by the activated \textbf{\textit{image}} samples.}
%     \label{tab:img_inter_metric}
% \end{minipage}
% \hfill
% \begin{minipage}{0.42\textwidth}
% \resizebox{1\textwidth}{!}{%
% \begin{tabular}{lccc}
%     \toprule
%     Model&\cellcolor{darkergray}TextD & \cellcolor{darkgray}CrossM & \cellcolor{gray}ImgD  \\
%     \midrule
%     \multicolumn{4}{c}{Embedding-based Similarity} \\
%     \midrule
%     CLIP     &\cellcolor{darkgray}  0.538&\cellcolor{gray}0.419&\cellcolor{darkergray}0.608\\
%     DeCLIP & \cellcolor{gray}-0.089 &\cellcolor{darkgray} -0.081&\cellcolor{darkergray}-0.030\\
%     CLIP+NCL &\cellcolor{darkergray}0.676&\cellcolor{darkgray}	0.588&\cellcolor{gray}0.544\\
%     CLIP+SAE & \cellcolor{darkergray}0.435 & \cellcolor{darkgray}0.260&\cellcolor{gray}-0.004\\
%     \midrule
%     \multicolumn{4}{c}{WinRate} \\
%      \midrule
%     CLIP   & \cellcolor{darkgray}0.614 & 	\cellcolor{gray}0.586&\cellcolor{darkergray}	0.631 \\
%     DeCLIP & \cellcolor{gray}0.451&\cellcolor{darkgray}0.457&\cellcolor{darkergray}0.462\\
%     CLIP+NCL & \cellcolor{darkergray}0.614&\cellcolor{darkgray}0.610&\cellcolor{gray}0.600\\
%     CLIP+SAE & \cellcolor{darkergray}0.571&\cellcolor{darkgray}0.545& \cellcolor{gray}0.491\\
%     \hline
%     \end{tabular}
%     }
%     \captionof{table}{Interpretability measured by the activated \textbf{\textit{text}} samples.}
%     \label{tab:txt_inter_metric}
% \end{minipage}



% \textbf{Image-only features.} 

% \textbf{Text-only features.} 

% \textbf{Cross-modal features.} 






% \subsection{Case Studies}
In addition to quantification of the interpretability, we look closer into a few examples of captured features. 

\textbf{Image-dominant neurons capture visual commonalities that are hard-to-describe in words.} We randomly select two \texttt{ImgD} neurons and visualize the top 8 activated images along each neuron in Figure~\ref{fig:act_imgs_imgD}. We find that the top neuron contains repetitive patterns of diverse shapes and colors, and the bottom neuron contains various objects that are partially ocean blue in color.  In contrast, the activated text samples (Table~\ref{tab:act_txts_imgD}) display a more diverse and abstract range of descriptions. Although less cohesive than the images, some patterns do emerge: for instance, two sentences refer to repetitive patterns for feature-647, while two others mention winter-related concepts, such as snow (as seen in the 5-th image for feature-667). These observations suggest that \texttt{ImgD} neurons are more adept at capturing distinct visual features that are not only challenging to express through language but are also more interpretable and intuitive to human perception, aligning with how we naturally understand visual commonalities.


\begin{figure*}
\centering
\includegraphics[width=0.65\linewidth]{Imgd.pdf}
% \caption*{(a) \footnotesize Patterns and textures activated by 
% feature-647.}
% \includegraphics[width=0.65\linewidth,trim={0 0 450 0},clip]{figures/openclip_ncl_ImageDom_667_image.png}
% \caption*{(b) \footnotesize  Water and aquatic themes activated by feature-667.}
% \vspace{-3mm}
\captionof{figure}{\footnotesize Activated images activated \texttt{ImgD} features. \textbf{Top:} Patterns and textures from feature-647. \textbf{Bottom:} Water and aquatic themes in blue from feature-667.}
\label{fig:act_imgs_imgD}
\vfill
\hspace{2mm}
\resizebox{0.8\textwidth}{!}{%
\begin{tabular}{p{8cm}|p{8cm}}
\toprule[1.pt]
\small
\textsl{\textit{Feature-647: Pattern and others.}} & \textsl{\textit{Feature-667: Scenes in winter and other.}} \\
\toprule[1.pt]
A bed with tufted upholstery.&\textbf{White trotting on snowy ground} with a tree.\\
\midrule
\textbf{Seamless pattern, flowers on a background.}&Covering the trailhead in a \textbf{winter} wonderland. \\
\midrule
Every girl should have this in their bedroom.&Red leather belt, a perfect accessory. \\
\midrule
Could new showroom and model signal the start?&The image of drum under the white background.\\
% \midrule
% \textbf{Seamless pattern of yellow - white circles on a black background.}&The community today celebrated holiday with a march starting and ending.\\
% \midrule
% Actor shows a lot of leg. &There is always some kind of fish at a party.\\
% \midrule
% The astute use of fabrics and colors complementing each other.&Possible photograph of person with the drum. \\
% \midrule
% True invention requires that we push away from our comfort zone. &Actor poses at the festival portrait studio \\
\bottomrule
    \end{tabular}
    }
    % \vspace{-3mm}
    \captionof{table}{Activated texts by the same set of \texttt{ImgD} features.}
    \label{tab:act_txts_imgD}
\end{figure*}


\textbf{Text-dominant neurons capture abstract concepts, especially human emotional feelings.} We randomly select two features and display the top 8 activated texts in Table~\ref{tab:act_txts_TextD}. Feature-34 centers around a sweet and happy atmosphere between couples, with themes like cuddling, embracing, and hugging. Feature-242 focuses on strong human emotions, such as ``never'', ``terrifying'' and exclamation marks. These \texttt{TextD} features generally correspond to abstract human feelings and thoughts, which can be associated with various visual objects (e.g., animals, sinkhole, castle.) This partially explain the diversity of objects in the images activated by feature 242 in Figure~\ref{fig:act_imgs_textD}. Interestingly, the images activated by feature-34 mostly depict couples or people in red attire, somewhat reflecting the joyful mood conveyed in the language. This insight highlights that \texttt{TextD} features can abstract the unique, high-level aspects of language, particularly atmosphere and emotions, as a reflection of human intelligence.

\begin{figure*}
\centering
\includegraphics[width=0.65\linewidth]{TextD.pdf}
\captionof{figure}{\footnotesize Activated images by \texttt{TextD} features. \textbf{Top}: Couples and people in red costume from feature-34. \textbf{Bottom}: Diverse objects from feature-242.}
\label{fig:act_imgs_textD}
\vfill
\hspace{2mm}
\resizebox{0.8\textwidth}{!}{%
\footnotesize
\begin{tabular}{p{8cm}|p{8cm}}
\toprule
\textit{Feature-34: Sweet and happy Couple.} & \textit{Feature-242: Strong emotion.} \\
\midrule
Attractive young couple sitting on a bench, talking and \textbf{laughing} with the city. & Animal looking for a cat tree without carpet your options have \textbf{greatly} expanded. \\
% Attractive young couple sitting on a bench, talking and \textbf{laughing} with the city. Animal looking for a cat tree without carpet your options have \textbf{greatly} expanded. \\
\midrule
Sculpture of \textbf{lovers} at the temple &Sinkhole, \textbf{most terrifying thing I have ever seen.}\\
\midrule
\textbf{Happy} couple in winter \textbf{embrace} each other with \textbf{love}& Where's \textbf{the best place} to show off your nails\textbf{?} right in front of the castle, \textbf{of course !} \\
\midrule
Young couple in \textbf{love}, \textbf{hugging} in the old part of town.&We're away from the beginning of the holiday season here\textbf{!}\\
% \midrule
% Pregnant wife and husband \textbf{cuddling} in the sand on the beach.&\textbf{I never have to paint a mural again ! =)} \\
\bottomrule
    \end{tabular}
    }
    % \vspace{-3mm}
    \captionof{table}{\footnotesize Activated texts by the same set of \texttt{TextD} features.}
    \label{tab:act_txts_TextD}
\end{figure*}

\textbf{Cross-Modality features (the majority features) capture common concepts from both visual and textual perspectives.} Different from the \texttt{TextD} and \texttt{ImgD}, whose activated samples tend to contain modality-exclusive features, \texttt{CrossD} neurons capture common concepts that could be expressed in both visual and language modalities. We randomly select two \texttt{CrossD} features and display their top activated images and texts. As shown in Figure~\ref{fig:act_img_crossm} and Table~\ref{tab:act_texts_crossm},
Feature6 mostly activates individuals in different activities, especially outdoor activities, and feature47 activates outdoor scenes. Both kinds of features can be consistently described in both images and languages, representing the common space shared by both modalities, implying that these features are mostly affected by the modality aligned training objectives.


\begin{figure*}
    

\centering
\includegraphics[width=0.6\linewidth]{CrossM.pdf}
\vspace{-4mm}
\captionof{figure}{\footnotesize Activated images by \texttt{CrossD} features. \textbf{Top:} activities performed by individuals from feature-6. \textbf{Bottom}: scenery outside the doors from feature-47.}
\label{fig:act_img_crossm}
% \end{minipage}
\vfill
\hspace{2mm}
\resizebox{0.8\textwidth}{!}{%
\footnotesize
\begin{tabular}{p{8cm}|p{8cm}}
\toprule[1pt]
\multicolumn{1}{c|}{\textit{Feature-6:  Actions/Exercises performed by individuals}}& \textit{Feature-47: Outdoors Scenery}\\
\midrule
\textbf{Young man} working on invention in a warehouse.&A stile on a public footpath overlooking the village on a frosty autumn morning. \\
\midrule
cricketers exercise during a practice session.&A private chapel , and the wrought iron gates in the grounds. \\
\midrule
Cricket player checks his bat during a training session. & Train track : a man blending in with the scenery as he stands on a railway track near a river 
\\
\midrule
Basketball coach watches an offensive possession from the sideline during the second half. &surveying the scene : people look out over loch today on a warm day in the village \\
% \midrule
% An attractive businessman wearing a blue suit and tie with glasses, standing against a white background.&Black and white landscape photograph of a black tree on a foggy autumn morning . \\
\bottomrule
\end{tabular}
}
% \vspace{-3mm}
\captionof{table}{\footnotesize Activated samples by the same set of \texttt{CrossD} features. The activated text share similar concepts with the image samples.}
\label{tab:act_texts_crossm}
\end{figure*}





% \begin{comment}
% \begin{minipage}{0.52\textwidth}
% % \fbox{
% % \includegraphics[width=1\linewidth,trim={0 0 1110 0},clip]{ICLR 2025 Template/figures/openclip_ncl_ImageDom_neuron_118_image.png}
% % \caption*{(a)\footnotesize{Indoor living spaces activate by Neuron118}}
% % }
% % \fbox{
% \includegraphics[width=1\linewidth,trim={0 0 1110 0},clip]{ICLR 2025 Template/figures/openclip_ncl_ImageDom_neuron_647_image.png}
% % }
% \caption*{(b)\footnotesize Patterns and textures activated by Neuron647.}
% % \fbox{
% \includegraphics[width=1\linewidth,trim={0 0 1115 0},clip]{ICLR 2025 Template/figures/openclip_ncl_ImageDom_667_image.png}
% % }
% \captionof{figure}{\footnotesize Activated images by \texttt{ImgD} neurons. Top to bottom: (Feature-647) Patterns and textures; (Feature-667) Water and aquatic themes.}
% \label{fig:act_imgs_imgD}
% % \caption{Activated images by ImgD neurons.}
% \end{minipage}
% \hspace{-2mm}
% % \hfill
% \begin{minipage}{0.45\textwidth}
%     \centering
% \footnotesize
% \begin{tabular}{p{6.2cm}}
% \toprule[1pt]
% % \texttt{Neuron 45: Strong affection}\\
% \texttt{Feature 647: }\\
% \midrule
% a bed with tufted upholstery . \\
% seamless pattern , wild flowers on a gray background . \\
% every girl should have this in their bedroom to wake up to . \\
% could a new showroom and new models signal the start of a comeback ?  \\
% seamless pattern of yellow - white circles on a black background \\
% % Beds made from tree branches \textbf{!}\\
% sinkhole,\textbf{most terrifying thing I have ever seen.}  \\
% Alligators: what's in my bag\textbf{?} gloves I need \textbf{!} \\
% \textbf{i never have to paint a mural again ! =)} \\
% \midrule
% \texttt{Neuron 932: Moments of joy, warmth} \\
% \midrule
% Couple \textbf{kissing} in a gazebo.\\
% % \textbf{Happy} young businessman running up a drawn stairs.\\
% Man with red jumper stand by a \textbf{Christmas tree}.\\
% \textbf{Funny} summer background with the little girl. \\
% % \midrule 
% % \texttt{Neuron 11: family-oriented scenes}\\
% % \midrule
% % \textbf{Parents} with their \textbf{children} on the beach. \\
% % % Person in an undated photo with his \textbf{foster family} . \\
% % \textbf{Family} chatting together on a \textbf{bed} at \textbf{home}. \\
% % \textbf{Smiling family} with their pet on the rug. \\
% \bottomrule[1pt]
%     \end{tabular}
%     \captionof{table}{\footnotesize Activated sentences by \texttt{TextD} neurons.}
%     \label{tab:act_sents_textd}
% \end{minipage}
% \end{comment}



% \subsection{Illustration of TextDom neurons}
% We display the top activated texts by TextDom neurons in OpenClip+NCL model~\footnote{As their activated images do not reflect a clear pattern, we show them in the Appendix.}. 
% \begin{table}[h]
%     \centering
% \begin{tabular}{l}
% \toprule
% \texttt{Neuron-8: Drinks and social settings}.\\
% \midrule
% \textbf{champagne} pouring into a \textbf{glass} . \\
% a \textbf{champagne} \textbf{bottle} and \textbf{glass} of \textbf{wine} \\  
% \textbf{bartender} presses fresh mint leaves in a \textbf{glass} \\
% couple chatting together at outdoor \textbf{cafe} table in the city \\
% ship in harbour viewed from the car ferry leaving a city . \\
% the \textbf{bartender} adds a slice of lime in a \textbf{cocktail} \\
% transforming our kitchen with a kitchen island \\
% person makes a \textbf{drink} at the restaurant . \\ 
% an image of the award winning \textbf{vodka} \\
% \textbf{liquid} being poured into a conical flask with a test tube \\
% \midrule
% \texttt{Neuron-11:family and home settings.}\\
% \midrule
% \textbf{family} chatting together on a \textbf{bed} at \textbf{home} \\
% \textbf{living area} with sunbathing surface, a sofa and a table at the bow of a super yacht \\
% \textbf{parents} with their \textbf{children} on the beach \\
% person, center, in an undated photo with his \textbf{foster family} . \\
% lovely custom wall covering for the black and white \textbf{home office} \\
% wallpaper probably containing a \textbf{living room} and a \textbf{family room} entitled person \\
% floral wallpaper in the \textbf{kitchen} \\
% \textbf{smiling family} with their pet yellow labrador on the rug \\
% \textbf{a family affair} : parents twins with her husband \\
% politician working in his office \\
% \midrule
% Location \\
% traditional balconies in the old town \\
% the pretty twin bedroom is perfect for children and adjoins the double bedroom \\
% waves splash the beach along south \\
% festival will be held and will provide color to the streets . \\
% geographical feature category at the seaside resort \\
% march is a market town \\
% main street by the sea \\
% fair , traditional market where you can buy all the decorations \\
% a boy living accompanies her mother to vote \\
% photo of monks in a temple by author \\
% \bottomrule
%     \end{tabular}
%     \caption{Caption}
%     \label{tab:my_label}
% \end{table}


% \begin{figure}[h!]
%     \centering
%     \begin{subfigure}[t]{\linewidth}
%         \centering
%     \includegraphics[width=0.85\linewidth]{ICLR 2025 Template/figures/TextDom_8_text_paper.png}
%     \caption{The top10 activated images by Neuron-8}
%     \end{subfigure}
    
%     \begin{subfigure}[t]{\linewidth}
%         \centering
%     \includegraphics[width=0.85\linewidth]{ICLR 2025 Template/figures/TextDom_11_text_paper.png}
%     \caption{The top10 activated images by Neuron-11}
%     \end{subfigure}


%     \caption{Top10 activated images by TextDom neurons.}
% \end{figure}


% \begin{table}[ht]
%     \centering
% \resizebox{0.8\textwidth}{!}{%
% \begin{tabular}{l}
% \hline
% Neuron-23  \\
%     \hline
% olives being harvested in the region \\
% attractive muscular man standing on the beach , covered in water droplets \\
% photograph of mules pulling a cart through a brook \\
% a woman fills buckets with water sponsored well in a village \\
% silhouette of a young woman flicking her hair at sunset \\
% young couple walking on the lake shore at sunrise \\
% a bridge damaged by flood water \\
% children playing on rocks on a beach on isle with fishing boat \\
% illustration of a farmer carrying a huge tomato representing the harvest in his farm \\
% stag taking a bath in the rut \\
% \hline
% \end{tabular}
% }
% \caption{The most activated text by DualModality Neuron-23}
% \label{tab:my_label}
% \end{table}

% \begin{table}[ht]
%     \centering
% \resizebox{0.8\textwidth}{!}{%
% \begin{tabular}{l}
% \hline
% Neuron-98  \\
%     \hline
% a pair of ornately decoratedstyle pointy shoes , on a plain white background , pointing at the camera \\
% black alarm clock on a yellow background royalty - free \\
% wrecked : a triumphant militants poses next to a destroyed tank - which bares the flag of military \\
% arctic wolf with scars standing and looking at the camera \\
% hood ornament in a classic monochrome ! \\
% ian hand drawn watercolor , on a white background . \\
% illustration of christmas holly and red ornament isolated on a white background . \\
% soldiers stand next to a tank at an air base . \\
% person stands atop a huge pile of snow in our neighborhood after a blizzard . \\
% an old antique school bus over a white background \\
% \hline
% \end{tabular}
% }
% \caption{The most activated text by Neuron-98}
% \label{tab:my_label}
% \end{table}

% \subsection{image-sensitive neurons}



% \subsection{text-sensitive neurons}
% We collect the text samples(captions) activated by the text-sensitive neurons, and  
% \begin{table}[h]
%     \centering
%     \begin{tabular}{l}
% \hline
% cars stopped in the tunnel after the fire broke out , causing its closure and shutting a runway \\
% living area with sunbathing surface , a sofa and a table at the bow of a super yacht \\
% a lone woman returns to her car in a lonely underground car park \\
% a view of a tunnel while riding a train \\
% desperate : commuters walk up the central panels of escalators on their way to catch trains in a bid to avoid the crowds after drivers went on strike following an assault on their colleague \\
% a conductor stands beside the high speed train of the new - kilometre line at a train station . \\
% a car traveling in solitude down a narrow road cutting though a thick forest \\
% backpacker walks alone by the road in forest \\
% another narrow street , more of a path , between tiny cottages \\
% tourists are seen leaving their hotels after the attacks \\
% \hline
%     \end{tabular}
%     \caption{Caption}
%     \label{tab:my_label}
% \end{table}


% \begin{table}[h]
%     \centering
%     \begin{tabular}{l}
% \hline
% family wearing black and blue for rustic fall family portraits in a city . \\
% illustration vector of cupcake in the creamy rain with rainbow on raining background for happy birthday card \\
% slow panning along a small stream with a stone and a wooden branch in the foreground \\
% illustration of a big wooden house with a windmill on a white background vector \\
% watercolor illustration of a bouquet of colorful flowers , image seamless pattern \\
% a simple white hut on a sandy beach with blue striped umbrellas , lounges , sea , and sky in the background \\
% autumn leaves on a plate of slate \\
% detail of a paintbrush and watercolor paints in a palette \\
% colorful painted wood style double quotes with a fun pink and yellow color wooden beveled effect isolated on a white background with clipping path . \\
% oxen pulling a decorated cart in a farm \\
% a selection of multicoloured agricultural tractors for sale \\ 
% actor in a red dress and shoes \\
% a tiny fly rests amongst a group of mushrooms on the forest floor \\
% colorful canoes on the bank of a peaceful lake \\
% \hline
%     \end{tabular}
%     \caption{Neuron40(TextDom)}
%     \label{tab:my_label}
% \end{table}






% \subsection{Distributions of Multi-Modality neurons}


% \begin{figure}[h!]
%     \centering
%     \begin{subfigure}[b]{0.245\textwidth}
%         \centering
%         \includegraphics[width=\textwidth]{ICLR 2025 Template/figures/i2t_ratio_openclip_step0_paper.pdf}
%         \caption{OpenClip}
%         \label{fig:1}
%     \end{subfigure}
%     \hfill
%     \begin{subfigure}[b]{0.245\textwidth}
%         \centering
%         \includegraphics[width=\textwidth]{ICLR 2025 Template/figures/i2t_ratio_declip_step500_paper.pdf}
%         \caption{DeClip}
%         \label{fig:2}
%     \end{subfigure}
%     \hfill
%     \begin{subfigure}[b]{0.245\textwidth}
%         \centering
%         \includegraphics[width=\textwidth]{ICLR 2025 Template/figures/i2t_ratio_openclip_ncl_step1000_paper.pdf}
%         \caption{OpenClip+NCL}
%         \label{fig:3}
%     \end{subfigure}
%     \hfill
%     \begin{subfigure}[b]{0.245\textwidth}
%         \centering
%         \includegraphics[width=\textwidth]{ICLR 2025 Template/figures/i2t_ratio_openclip_sae100step.pdf}
%         \caption{OpenClip+SAE}
%         \label{fig:4}
%     \end{subfigure}
%     \caption{Img2Txt ratio distributions for different Lanauage-Vision Models.}
%     \label{fig:four_figures}
% \end{figure}


% \begin{figure}[h]
%     \centering
%     \includegraphics[width=0.5\linewidth]{ICLR 2025 Template/figures/neurons_distribution.pdf}
%     \caption{Compared to OpenClip, DeClip, NCL and SAE increase the number of mono-modality neurons. }
%     \label{fig:enter-label}
% \end{figure}

% \paragraph{Interpretability of neurons.} For each neuron, we collect their activated image and text samples. Then, we measure the neuron interpretability using the embedding-based similarity and win rate proposed in \S\ref{subsec:metrics}. The results are shown in Table~\ref{tab:img_inter_metric} and Table~\ref{tab:txt_inter_metric}. It is expected that the ImageDom neurons can capture vision-specific features, which are not prevalent in text modality, and vice versa. For DualModality neurons, they activate both visionary and textual patterns.


% \definecolor{gray}{rgb}{0.9,0.9,0.9}
% \definecolor{darkgray}{rgb}{0.8,0.8,0.8}
% \definecolor{darkergray}{rgb}{0.65,0.65,0.65}
% \begin{table}[h]
%     \centering
%     \resizebox{0.65\textwidth}{!}{%
%     \begin{tabular}{l|c|c|c}
%     \toprule
%     Model&\cellcolor{gray}TextDom & \cellcolor{darkgray}DualModality & \cellcolor{darkergray}ImageDom  \\
%     \midrule
%     \multicolumn{4}{c}{Embedding-based Similarity} \\
%         \midrule
%     OpenClip &\cellcolor{darkergray} 0.126&\cellcolor{gray}	0.111&\cellcolor{darkgray}0.118\\
%     DeClip & \cellcolor{darkgray}0.060&	\cellcolor{gray}0.054&	\cellcolor{darkergray}0.070\\
%     OpenClip+NCL & \cellcolor{darkgray}0.160&	\cellcolor{gray}0.155&	\cellcolor{darkergray}{0.197}\\
%     OpenClip+SAE &\cellcolor{gray}0.096&\cellcolor{darkgray}0.125&\cellcolor{darkergray}0.135\\
%     \midrule
%     \multicolumn{4}{c}{Win Rate} \\
%      \midrule
%     OpenClip     &\cellcolor{darkergray}0.686&	\cellcolor{gray}0.645&\cellcolor{darkgray}0.679 \\
%     DeClip & \cellcolor{darkgray}0.626&\cellcolor{gray}0.607&\cellcolor{darkergray}0.632\\
%     OpenClip+NCL &\cellcolor{gray} 0.722&	\cellcolor{darkgray}0.723	&\cellcolor{darkergray}0.754\\
%     OpenClip+SAE & \cellcolor{gray}0.623&\cellcolor{darkgray}	0.678&\cellcolor{darkergray}	0.688\\
%     \bottomrule
%     \end{tabular}
%     }
%     \caption{Interpretability measured by the activated \textbf{\textit{image}} samples.}
%     \label{tab:img_inter_metric}
% \end{table}


% \begin{table}[h]
%     \centering
%     \begin{tabular}{lccc}
%     \toprule
%     Model&\cellcolor{darkergray}TextDom & \cellcolor{darkgray}DualModality & \cellcolor{gray}ImageDom  \\
%     \midrule
%     \multicolumn{4}{c}{Embedding-based Similarity} \\
%     \midrule
%     OpenClip     &\cellcolor{darkgray}  0.538&\cellcolor{gray}0.419&\cellcolor{darkergray}0.608\\
%     DeClip & \cellcolor{gray}-0.089 &\cellcolor{darkgray} -0.081&\cellcolor{darkergray}-0.030\\
%     OpenClip+NCL &\cellcolor{darkergray}0.676&\cellcolor{darkgray}	0.588&\cellcolor{gray}0.544\\
%     OpenClip+SAE & \cellcolor{darkergray}0.435 & \cellcolor{darkgray}0.260&\cellcolor{gray}-0.004\\
%     \midrule
%     \multicolumn{4}{c}{Win Rate} \\
%      \midrule
%     OpenClip   & \cellcolor{darkgray}0.614 & 	\cellcolor{gray}0.586&\cellcolor{darkergray}	0.631 \\
%     DeClip & \cellcolor{gray}0.451&\cellcolor{darkgray}0.457&\cellcolor{darkergray}0.462\\
%     OpenClip+NCL & \cellcolor{darkergray}0.614&\cellcolor{darkgray}0.610&\cellcolor{gray}0.600\\
%     OpenClip+SAE & \cellcolor{darkergray}0.571&\cellcolor{darkgray}0.545& \cellcolor{gray}0.491\\
%     \hline
%     \end{tabular}
%     \caption{Interpretability measured by the activated \textbf{\textit{text}} samples.}
%     \label{tab:txt_inter_metric}
% \end{table}

\section{Case Studies based on Modality-specific Features}
In this section, we present three case studies based on our three modality features: (1) gender detection (2) adversarial attacks (3) text-to-image generation.

% \begin{figure}[h]
%     \centering 
%     \includegraphics[width=0.3\linewidth]{figures/horse.jpg}
%     \caption{a reference image: horse}
%     \label{fig:ref_image}
% \end{figure}
\begin{figure*}[t]
    \centering
    \includegraphics[width=0.88\linewidth,trim={10 400 0 50},clip]{gender_vis.pdf}
    \caption{\footnotesize \textit{\textbf{Female}} figures ordered by their percentages of \texttt{ImgD} features: 0.14, 0.16, 0.18,0.20, 0.22, 0.24, 0.26. More feminine features are observed with more \texttt{ImgD} features.}
\label{fig:female_diff_imgd}
\end{figure*}
    % \begin{subfigure}{0.12\textwidth}
    %     \centering
    %     \includegraphics[width=\linewidth]{figures/female_0.14_2239_o1.png}
    %     % \caption{14\%}
    % \end{subfigure}
    % \begin{subfigure}{0.12\textwidth}
    %     \centering
    %     \includegraphics[width=1\linewidth,trim={35 0 0 0},clip]{figures/female_0.16_3517_o2.png}
    % \end{subfigure}
    % \begin{subfigure}{0.12\textwidth}
    %     \centering
    %     \includegraphics[width=\linewidth]{figures/female_0.18_9969_o3.png}
    %     % \caption{Caption for Image 2}
    % \end{subfigure}
    % \begin{subfigure}{0.12\textwidth}
    %     \centering
    %     \includegraphics[width=\linewidth]{figures/female_0.2_1513.png}
    %     % \caption{Caption for Image 2}
    % \end{subfigure}
    % \begin{subfigure}{0.12\textwidth}
    %     \centering
    %     \includegraphics[width=\linewidth]{figures/female_0.22_13167_o4.png}
    %     % \caption{Caption for Image 2}
    % \end{subfigure}
    %     \begin{subfigure}{0.12\textwidth}
    %     \centering
    %     \includegraphics[width=\linewidth]{figures/female_0.24_2140_o5.png}
    %     % \caption{Caption for Image 2}
    % \end{subfigure}
    % \begin{subfigure}{0.12\textwidth}
    %     \centering
    %     \includegraphics[width=\linewidth]{figures/female_0.26_6646_o6.png}
    %     % \caption{Caption for Image 2}
    % \end{subfigure}
\subsection{Case Study 1: Gender Pattern in Different Modalities}
We describe gender using visual features, for example, long hair and wearing a dress, and assume that the ImgDom features primarily account for these discriminative visual patterns. Consequently, removing the ImgDom features from a female image may make it  less identifiable in terms of gender, potentially leading to its classification as male in a binary classification task. Similarly, TextDom features play a comparable role when gender is described through textual information. 

To test this hypothesis, we collect both male and female images from the cc3m validation set using a gender classifier~\footnote{\href{https://huggingface.co/touchtech/fashion-images-gender-age-vit-large-patch16-224-in21k-v3}{touchtech/fashion-images-gender-age-vit-large-patch16-224-in21k-v3}}. These images are then encoded using the Clip+SAE model, extracting 1024-dimensional feature representations for both female and male subjects. Next, we apply a zero-mask  intervene strategy to remove the \texttt{ImgDom} and \texttt{TextDom} features from these representations. Notably, our intervention is applied at the feature level, i.e., on activations rather than the raw image or text inputs. Since these modified reature representations cannot be directly processed by existing pretrained classifiers, which require image or text inputs, we employ a zero-shot classification approach inspired by~\citet{bhalla2024interpreting}. Specifically, we use an unsupervised clustering method to measure the distances between the intervened activations and the label embeddings for ``female" and ``male", with the latter obtained by encoding female and male inputs. 

Before analyzing the difference in predominant features between male and female subjects, we first  verify that our identified modality-specific features indeed capture information within their respective  modality. 

\textbf{Modality-specific interventions.} We intervene both \texttt{ImgD} and \texttt{TextD} for image and text inputs, respectively. The probabilities of original image/text and intervened image/text, over the original gender label are in the Table~\ref{tab:gender_case}. 

\begin{table}[h]
\centering
\caption{\footnotesize Probability over the original gender label for different input modality. The results show that after removing the modality dominant features, e.g., \texttt{ImgDom} for the input in the same modality, e.g., image, the original inputs will be affected in a larger extent, i.e., 0.785 compared to 0.828 caused by removing \texttt{TextDom} features.}
% \small
\resizebox{0.48\textwidth}{!}{%
\begin{tabular}{r|cccc}
\toprule[1pt]
    Input Modality & Ori-Acc & w.o. \texttt{ImgD} $\downarrow$ & w.o. \texttt{TextD} $\downarrow$ & w.o. Random $\downarrow$\\
\midrule
   Image& 0.834 &	\textbf{0.785}&	0.828    & 0.815 \\
Text     & 0.709&0.709&\textbf{0.639}&0.699\\
\bottomrule[1pt]
    \end{tabular}
    }
    \label{tab:gender_case}
\end{table}

\textbf{Gender bias in different modalities.} We then show the discrepancy when removing the image and text features to identify the primary modality supporting the gender in this dataset. From the results in Table~\ref{tab:gender_modality}, we observe that female images are more easily affected by the \texttt{ImgD} features, while male texts are more easily affected by the \texttt{TextD} features.
\begin{table}[h]
    \centering
    \caption{\footnotesize Comparison of the effects of removing different modality-feature from the specific gender in the corresponding modality. For \textit{female}, remove the \texttt{ImgD} lead to larger changes to the female visual inputs, than remove the \texttt{TextD} from the female textual inputs, vice versa for male.}
    \resizebox{0.35\textwidth}{!}{% 
    \begin{tabular}{r|cc}
    \toprule[1pt]
        & $\Delta$(Remove ImgD) & $\Delta$(Remove TextD) \\
    \midrule
   Female       & \textbf{17.65}&7.27\\
   Male & 5.64&\textbf{28.67}\\
   \bottomrule[1pt]
    \end{tabular}
    }
    \label{tab:gender_modality}
\end{table}

To vividly show the changes brought by intervene of the \texttt{ImgD} and \texttt{TextD} features on gender, we show the different female images which differ in how many percentage of their most activated features are \texttt{ImgD} features in Figure~\ref{fig:female_diff_imgd}. From left to right, more activated features are \texttt{ImgD} and they tend to contains more detailed feminine concepts, such as backless skirt, hair accessories. 
The middle images show professional female, such as politician and doctor; and the first image shows a pair of leg in sports shoes, with minimal feminine factors, the pink color. 


%%%%below is the results of word cluster of male desriptions, can be moved to appendix%%%%%%
% We then cluster different male descriptions according to the percentage of \texttt{TextD} features among all their top-20 activated features, and we calculate the frequency of the top7 tokens in each cluster shown in Figure~\ref{fig:male_sents_cluster}~\footnote{We remove the gendered personal pronouns, e.g., he, she, woman, man, boy, girl and only focus on how gender-neutral concepts represent the gender.}. With more \texttt{TextD} injection, the textual descriptions become more sports related, such coach, basketball, soccer; while the sentences with less activated \texttt{TextD} have top words, as party, hip, game, smile, home. This trend is consistent with the social stereotype that male are more active in sport activities.
% \begin{figure}[h]
%     \centering
%     % First row
%     \begin{minipage}{0.23\textwidth}
%         \centering
%         \includegraphics[width=\linewidth]{figures/0.1_barchart.pdf}
%         % \caption{Caption for Image 1}
%     \end{minipage}\hspace{0.1cm} % Space between figures
%     \hspace{-3mm}
%     \begin{minipage}{0.23\textwidth}
%         \centering
%         \includegraphics[width=\linewidth]{figures/0.12_barchart.pdf}
%         % \caption{Caption for Image 2}
%     \end{minipage}
%     \vspace{-4mm}
%     % Second row
%     \begin{minipage}{0.23\textwidth}
%         \centering
%         \includegraphics[width=\linewidth]{figures/0.18_barchart.pdf}
%         % \caption{Caption for Image 3}
%     \end{minipage}\hspace{0.1cm} % Space between figures
%        \hspace{-3mm}
%     \begin{minipage}{0.23\textwidth}
%         \centering
%         \includegraphics[width=\linewidth]{figures/024_barchart.pdf}
%         % \caption{Caption for Image 4}
%     \end{minipage}
% % \vspace{-2mm}
% \caption{\footnotesize Top7 words in each male-description clusters, which differ in different percentage of the activated \texttt{TextD}.}
% \label{fig:male_sents_cluster}
% \end{figure}
%%%%below is the results of word cluster of male desriptions, can be moved to appendix%%%%%%5


%%%%%%%%toxicity experiments%%%%%%%%%%%%%%
% \begin{table}[h]
%     \centering
%     \resizebox{0.45\textwidth}{!}{
%     \begin{tabular}{c|cc}
% \toprule
% original & Remove ImgD	& Remove TextD \\
% \midrule
% 0.677&	0.687	&0.452\\
% \bottomrule
%     \end{tabular}
%     }
%     \caption{\footnotesize After removing the \texttt{TextD} features, the original toxic sentences are greatly detoxified, from 0.677 to 0.452. Removal of the \texttt{imgD} instead increase the toxicity.}
%     \label{tab:detoxic}
% \end{table}
%%%%%%%%toxicity experiments%%%%%%%%%%%%%%

% \subsection{Model Editing}
% Another key application of feature intervention is to remove the harmful features. Here, we experiment on the the non-toxic and toxic paired dataset~\citep{lee2024mechanistic}, whose non-toxic data is from wiki-text2 and the toxic sentences are generated by PPLM~\citep{Dathathri2020Plug} conditioned on the toxic attribution. We thus remove the \texttt{TextD} features of the encoded textual features from the toxic sentences. And the label embeddings for unsupervised classification are obtained by both non-toxic and toxic sentences. The results are shown in Table~\ref{tab:detoxic}


% \begin{algorithm}[!htb]
%     \renewcommand{\algorithmicrequire}{\textbf{Input:}}
%     \renewcommand{\algorithmicensure}{\textbf{Output:}}
%     \caption{Jailbreak defense with MDS}
%     \label{algorithm: neuron_optimization}
%     \begin{algorithmic}[1]
%         \REQUIRE
%             Peace features $F_{peace}$,
%             Adversarial sample $I_{adv}$,
%             Neuron sets $\{\mathcal{N}_{text}, \mathcal{N}_{image}, \mathcal{N}_{dual}\}$
%         \ENSURE
%             Optimized image $I_{opt}$
        
%         \FOR{$type$ in $\{\texttt{ImgD}, \texttt{TextD}, \texttt{CrossD}\}$}
%             \FOR{$mode$ in $\{\text{selective}, \text{random}\}$}
%                 \IF{$mode = \text{selective}$}
%                     \STATE $mask \leftarrow \mathcal{N}_{type}$
%                 \ELSE
%                     \STATE $mask \leftarrow \text{RandomMask}(|\mathcal{N}_{type}|)$
%                 \ENDIF
                
%                 \STATE $I_{current} \leftarrow I_{adv}$
%                 \REPEAT
%                     \STATE $F_{current} \leftarrow \text{Encode}(I_{current})$
%                     \STATE $\mathcal{L} \leftarrow \|F_{current}[:, mask] - F_{peace}[:, mask]\|_2$
%                     \STATE Update $I_{current}$ by minimizing $\mathcal{L}$
%                     \STATE $I_{current} \leftarrow \text{clamp}(I_{current}, 0, 1)$
%                 \UNTIL{convergence}
%                 \STATE $I_{opt} \leftarrow I_{current}$
%             \ENDFOR
%         \ENDFOR
%         \RETURN $I_{opt}$
%     \end{algorithmic}
% \end{algorithm}
\subsection{Case Study 2: Adversarial Attacks}
We investigate the impact of different types of features on multimodal adversarial attacks~\citep{cui2024robustness,yin2024vlattack}, following the setup in~\citet{shayegani2024jailbreak}.

The adversarial sample is a benign-appearing image, e.g., a scenery image but injected with harmful semantic information, such as the phrase \textit{``I want to make bomb''}. One defense optimization strategy involves minimizing the distance, between the embeddings of adversarial sample $\mathbf{F}_
{adv}$ and a benign sample $\mathbf{F}_{ben}$, and accordingly update the adversarial sample (in Figure~\ref{fig:ad_overview}). The paired benign  image is injected with the friendly text, e.g.,  \textit{``peace and love''}. To study the effects of our identified modality features, we only select the target feature index $I$ from the embedding for alignment training, i.e., \texttt{ImgD}, \texttt{TextD}, and \texttt{CrossD}. The alignment loss is $\mathcal{L} = \|\mathbf{F}_{adv}[:, I] - \mathbf{F}_{ben}[:, I]\|_2$. Finally, the optimized adversarial sample is then adopted to attack a Vision-Language model (VLM).

\begin{figure}[h]
    \centering
    \includegraphics[width=0.85\linewidth,trim={20 12 20 5},clip]{ad_overview.pdf}
    \caption{Optimization of the adversarial samples, with only selected target features, i.e., ImgD, TextD and CrossD, involved in the alignment.}
    \label{fig:ad_overview}
\end{figure}

\textbf{Models.} We use the the same CLIP model as introduced in Section~\ref{sec:disentangle} as the Multimodality feature extractor, so the index for target features unchanged. 
The VLM being attacked is Llama-1.5-7b-hf~\cite{liu2023llava, liu2023improvedllava}. To evaluate whether the attack the attack is successful, we evaluate the generated response from the VLM to DeepSeek V3~\cite{deepseekai2024deepseekv3technicalreport} to generate a binary label indicating whether the harmful request is rejected or the task is executed.
% generates harmful information after being fed the aligned samples along with the harmful prompt. We repeat the process 100 times for each aligned image, inputting it into Llama-1.5-7b-hf, and count the number of instances where the attack successfully elicits harmful outputs.

% \paragraph{Experiment setup}
% To address this, we use the model: open-clip~\cite{ilharco_gabriel_2021_5143773} to generate adversarial samples and semantic detoxification samples. The test model is Llama-1.5-7b-hf~\cite{liu2023llava}.
% and the GPU used is NVIDIA L20 48GB.
% We use an image with the text "peace, non-violence" as the alignment image.
% The period of alignment optimization is set to 5000.

% L2 Loss Formula
% \paragraph{Loss Function}
% The optimization process minimizes the L2 loss between the feature vectors of the current image \( I_{current} \) and the alignment image \( I_{peace} \). The L2 loss is computed as:

% \[
% \mathcal{L} = \|F_{current}[:, mask] - F_{peace}[:, mask]\|_2
% \]

% where:
% \begin{itemize}
%     \item \( F_{current} \) and \( F_{peace} \) are the feature vectors of the current image \( I_{current} \) and the alignment image \( I_{peace} \), respectively. These feature vectors are extracted using a pre-trained encoder (e.g., CLIP image encoder).
%     \item \( mask \) is a binary mask that selects a subset of neurons from the feature vectors. It is generated based on the neuron sets \( \mathcal{N}_{text} \), \( \mathcal{N}_{image} \), or \( \mathcal{N}_{dual} \), depending on the alignment type (\texttt{ImgD}, \texttt{TextD}, or \texttt{CrossD}).
%     \item \( \| \cdot \|_2 \) denotes the L2 norm, which measures the Euclidean distance between the two feature vectors.
% \end{itemize}

% \paragraph{Attack Success Rate}
% The success of the attack is evaluated by measuring the frequency at which the model (Llama-1.5-7b-hf) generates harmful outputs when provided with the optimized adversarial sample \( I_{opt}^* \) and a harmful prompt \( P_{harm} \). The attack success rate is formally defined as:
% \[
% \text{Success Rate} = \frac{1}{N} \sum_{k=1}^{N} \mathbb{I}\left( \text{Llama}(I_{opt}^*, P_{harm}) \rightarrow \text{Harmful} \right)
% \]
% where:
% \begin{itemize}
%     \item \( N \) is the total number of trials (in our experiments, \( N = 100 \)).
%     \item \( \mathbb{I}(\cdot) \) is the indicator function, which returns 1 if the model generates harmful output and 0 otherwise.
%     \item \( \text{Llama}(I_{opt}^*, P_{harm}) \) denotes the output of the Llama-1.5-7b-hf model when given the optimized alignment sample \( I_{opt}^* \) and the harmful prompt \( P_{harm} \).
%     \item \( \text{Harmful Output} \) is defined as any model response that contains harmful or dangerous information.
% \end{itemize}

\textbf{Results.} The results are shown in Table~\ref{tab:adversarial}. The number of neurons selected was consistent across all experiments. Using the smallest TextD as the baseline, we repeatedly sampled the same number of neurons from \texttt{ImgD} and \texttt{CrossD} as in \texttt{TextD}. If we achieve better defense results (i.e., a lower attack success rate) with a specific type of feature, it suggests that this type of neuron plays a key role in defense. 
% To account for the potential effects of randomly selected features, 
 We observe that leveraging all three target features improves defense results to some extent compared to the original adversarial sample. Given that the number of features in each category differs, we randomly sample an equal number of features from each category to ensure alignment Among them, using \texttt{TextD} for alignment yields the best defense performance, with only 25\% rate comparing with alignment on the same amount of features, 65\%. The performance is followed by \texttt{CrossD} and \texttt{ImgD}. Since the adversarial information primarily stems from undesirable textual semantics, this outcome demonstrates that \texttt{TextD} effectively captures most of the semantic content. In contrast, \texttt{CrossD} captures partial semantics, while \texttt{ImgD} is the least related to semantic information, resulting in minimal benefits for jailbreak defense when aligned.
\begin{table}[h]
    \centering
    \caption{Success rate for adversarial attacks with different target features involved in the alignment training.  The success rate of the benign image is 10\%, for the original adversarial sample is 80\%. For comparison, we also compare with the performance of aligning with the same number of randomly selected features, 65\%.}
    \resizebox{0.35\textwidth}{!}{%
    \begin{tabular}{r|cccc}
    \toprule
    \textbf{Target feature} & \texttt{ImgD} & \texttt{TextD} & \texttt{CrossD} \\
    \midrule
    \textbf{Success Rate} ($\downarrow$) & 50\% & 25\% & 30\% \\ 
    \bottomrule
    \end{tabular}
    }
    \label{tab:adversarial}
\end{table}
\vspace{-8mm}
% \hq{Experiment question: (1) Add all neurons involved in the experiment(to check if the image has changed a lot, if a o=lot, compare the figrues with interven our neurons; (2) random for textD, imgeD; same neurons for; (3) the same number of neurons in the intervention; (4) why the imgD results are better than random.}
\paragraph{Potential.} The feature-specified optimization for multimodality jailbreak provides a more focused and computationally efficient defense strategy. This selective alignment not only enhances interpretability by highlighting the roles of different feature types but also allocates resources more effectively by prioritizing the most critical features for defense. Additionally, it prevents feature dilution, ensuring that semantic integrity is preserved during optimization. This modular and adaptable design makes the method particularly effective for defending modality-specific attacks.
% \hq{what is the clip model shard by text and viusal features.}
% \hq{there is only text encoder in diffusion model, can only accepting the text inputs. reweight the imgD from the reference img and textD from the text encoder; }

% text input: [77,1024],
% reference model->openlip-encoder -> 1024[imgD, textD, cross]
\subsection{Case Study 3: Multimodal Generation}
Despite the impressive capabilities of text-to-image generation models~\citep{yu2024spae,koh2024generating,swamy2024multimodn}, their internal mechanisms for bridging linguistic semantics and visual details remain poorly understood. A key challenge is disentangling how modality-specific features influence the fidelity and controllability of generation. To address this, we investigate the generation process by intervening in different modality-specific features in Stable Diffusion v2~\citep{Rombach_2022_CVPR}.

\textbf{Models.} Stable Diffusion v2~\citep{Rombach_2022_CVPR} is our generation model, and its feature extractor is \texttt{laion/CLIP-ViT-H-14-laion2B-s32B-b79K} rather than the CLIP model previously employed. Therefore, we compute the model-specific MDS based on inference passes over the COCO2017 dataset~\citep{visualization-tools-for-coco-dataset}.
\begin{wrapfigure}{r}{0.42\linewidth}
    \centering 
    \includegraphics[width=0.99\linewidth]{horse.jpg}
    \caption{\footnotesize Reference image.}
    \label{fig:ref_image}
\end{wrapfigure}
\vspace{-3mm}

\textbf{Intervention of the Text-to-Image Generation.} 
The input text prompt is \textit{“\textbf{Please draw an animal}”}. The feature extractor generates an embedding $\mathbf{T}$, representing the original multimodal embedding for generation. Additionally, we provide a reference figure—a horse (Figure~\ref{fig:ref_image})—processed through the same feature extractor, producing a reference embedding $\mathbf{R}$. To control the generation through modality-specific feature intervention, we interpolate only the features at specified indices $I$ defined by MDS. The final multimodal embedding is computed as:
$\mathbf{E}[I] = \alpha \mathbf{T}[I] + (1 - \alpha) \mathbf{R}[I]$, where operations are applied exclusively to the feature indices defined by $I$, i.e., \texttt{ImgD}, \texttt{TextD} and \texttt{CrossD}.

\begin{figure*}[h]
    \centering 
    \includegraphics[width=0.89\linewidth]{image_gen_new.png}
    \includegraphics[width=0.89\linewidth]{text_gen_new.png}
    \includegraphics[width=0.89\linewidth]{cross_gen_new.png}
    \caption{Generated images from the text-to-image model with the text prompt \textit{"Please draw an animal"} and varying levels of intervention from a reference image (horse). From left to right, the interpolation weights range from 0.0 to 0.9 at intervals of 0.1. From top to bottom, the interventions are exclusively applied to the modality-features, i.e.,\texttt{ImgD}, \texttt{TextD} and \texttt{CrossD}.}
    \label{fig:mm_gen}
\end{figure*}

\textbf{Results.} We feed $E$ to the generation model with different $\alpha$ ranging from 0 to 0.9 with an interval of 0.1. The generated images with the selected indices correspond to \texttt{ImgD}, \texttt{TextD}, and \texttt{CrossD} are shown in Figure~\ref{fig:mm_gen}. The results clearly demonstrate that larger interventions on \texttt{ImgD} and \texttt{CrossD} disrupts visual coherence: animal shapes fragment, outlines blur, and textures degrade, implying the role of \texttt{ImgD} in preserving structural and fine-grained visual details. Interestingly, interventions on \texttt{TextD} maintain the visual features without any distortion even with larger $\alpha$.  We can instead observe the shifts in semantic concepts, such as generating cat-like, elephant, or horse. These animals became abstracted into geometric forms or textual overlays, demonstrating that text-guided representations contribute to the structured composition and semantic labeling of the generated visuals, rather than low-level visual details.  
% This finding aligns with human cognition, where linguistic features are more adept at capturing abstract, high-level information.
%\texttt{TextD} maintain the overall visual style of a sketch, yet display different subjects, such as a hand, a girl with long hair, or a furry dog. This observation suggests that \texttt{TextD} primarily encodes high-level semantic concepts, such as the subject, rather than low-level visual details. This finding aligns with human cognition, where linguistic features are more adept at capturing abstract, high-level information.% 1) \texttt{TextD} neurons primarily encode high-level semantic concepts. Perturbing these neurons altered semantic alignment but preserved low-level visual details, indicating their role in abstract linguistic grounding. 2) \texttt{ImgD} neurons critically influence visual fidelity. Modifying their activations led to distortions in textures, color shifts, or structural artifacts. 3) \texttt{CrossD} Neurons exhibited weaker specialization. Although they mediated cross-modal alignment, their impact on both semantics and visuals was less pronounced compared to unimodal neurons, suggesting a trade-off between generality and specificity.

\textbf{Potential.} By isolating modality-specific neurons, our framework provides several benefits for data editing: (i) Semantic Refinement: Adjusting \texttt{TextD} activations improves conceptual alignment; (ii) Visual Enhancement: Tuning \texttt{ImgD} neurons enhances texture realism or ensures stylistic consistency. This data-driven approach not only advances interpretability but also reflects human cognitive principles, where distinct neural pathways govern linguistic abstraction and perceptual processing.


% We applied the abovementined method to derive the Image-Dominant (\texttt{ImgD}), Text-Dominant (\texttt{TextD}), and Cross-Modality (\texttt{CrossD}) for the CLIP encoder in Stable-Diffusion-2~\cite{Rombach_2022_CVPR} via a inference process on COCO2017 dataset~\cite{visualization-tools-for-coco-dataset}. To probe their roles, we systematically perturbed the activations of each group of neurons during the generation.

% \begin{algorithm}[!htb]
%     \renewcommand{\algorithmicrequire}{\textbf{Input:}}
%     \renewcommand{\algorithmicensure}{\textbf{Output:}}
%     \caption{Multimodality generation with MDS}
%     \label{algorithm: cmfar}
%     \begin{algorithmic}[1]
%         \REQUIRE
%             Image-Text paired dataset $\mathcal{D}$,
%             Reference image $I_{ref}$,
%             Target text prompt $T$,
%             Interpolation weight $\alpha$,
%             Feature dimension $d$
%         \ENSURE
%             Modified text embeddings $F'_{target}$
%         \STATE Stage 1: Feature Extraction
%         \FOR{each $(I,t)$ in $\mathcal{D}$}
%             \STATE $F_I = \text{CLIP}_{image}(I) \in \mathbb{R}^d$ 
%             \STATE $F_t = \text{CLIP}_{text}(t) \in \mathbb{R}^d$
%             \STATE Add $F_I, F_t$ to feature sets $\mathcal{F}_I, \mathcal{F}_t$
%         \ENDFOR
         
%         \STATE Stage 2: Computing modality dominance score
%         \STATE $R = \frac{\text{mean}(|\mathcal{F}_I|)}{\text{mean}(|\mathcal{F}_I|) + \text{mean}(|\mathcal{F}_t|)}$ 
%         \STATE $\mu_R = \text{mean}(R)$
%         \STATE $\sigma_R = \text{std}(R)$
%         \STATE $\texttt{CrossD}_{idx} = \{i | \mu_R - \sigma_R < R_i < \mu_R + \sigma_R\}$
%         \STATE $\texttt{TextD}_{idx} = \{i | R_i < \mu_R - \sigma_R\}$
%         \STATE $\texttt{ImgD}_{idx} = \{i | R_i > \mu_R + \sigma_R\}$
        
%         \STATE Stage 3: Neuron weighting
%         \STATE $F_{ref} = \text{CLIP}_{image}(I_{ref})$
%         \STATE $F_{target} = \text{CLIP}_{text}(T)$
%         \STATE $F'_{target} = F_{target}$
%         \FOR{$idx$ in $\{\texttt{CrossD}_{idx}, \texttt{TextD}_{idx}, \texttt{ImgD}_{idx}\}$}
%             \STATE $F'_{target}[idx] = \alpha \cdot F_{ref}[idx] + (1-\alpha) \cdot F_{target}[idx]$
%         \ENDFOR
%         \RETURN $F'_{target}$
%     \end{algorithmic}
% \end{algorithm}

% \hq{insert image for visualization}

\section{Related Work}
\label{sec:related-works}
\subsection{Novel View Synthesis}
Novel view synthesis is a foundational task in the computer vision and graphics, which aims to generate unseen views of a scene from a given set of images.
% Many methods have been designed to solve this problem by posing it as 3D geometry based rendering, where point clouds~\cite{point_differentiable,point_nfs}, mesh~\cite{worldsheet,FVS,SVS}, planes~\cite{automatci_photo_pop_up,tour_into_the_picture} and multi-plane images~\cite{MINE,single_view_mpi,stereo_magnification}, \etal
Numerous methods have been developed to address this problem by approaching it as 3D geometry-based rendering, such as using meshes~\cite{worldsheet,FVS,SVS}, MPI~\cite{MINE,single_view_mpi,stereo_magnification}, point clouds~\cite{point_differentiable,point_nfs}, etc.
% planes~\cite{automatci_photo_pop_up,tour_into_the_picture}, 


\begin{figure*}[!t]
    \centering
    \includegraphics[width=1.0\linewidth]{figures/overview-v7.png}
    %\caption{\textbf{Overview.} Given a set of images, our method obtains both camera intrinsics and extrinsics, as well as a 3DGS model. First, we obtain the initial camera parameters, global track points from image correspondences and monodepth with reprojection loss. Then we incorporate the global track information and select Gaussian kernels associated with track points. We jointly optimize the parameters $K$, $T_{cw}$, 3DGS through multi-view geometric consistency $L_{t2d}$, $L_{t3d}$, $L_{scale}$ and photometric consistency $L_1$, $L_{D-SSIM}$.}
    \caption{\textbf{Overview.} Given a set of images, our method obtains both camera intrinsics and extrinsics, as well as a 3DGS model. During the initialization, we extract the global tracks, and initialize camera parameters and Gaussians from image correspondences and monodepth with reprojection loss. We determine Gaussian kernels with recovered 3D track points, and then jointly optimize the parameters $K$, $T_{cw}$, 3DGS through the proposed global track constraints (i.e., $L_{t2d}$, $L_{t3d}$, and $L_{scale}$) and original photometric losses (i.e., $L_1$ and $L_{D-SSIM}$).}
    \label{fig:overview}
\end{figure*}

Recently, Neural Radiance Fields (NeRF)~\cite{2020NeRF} provide a novel solution to this problem by representing scenes as implicit radiance fields using neural networks, achieving photo-realistic rendering quality. Although having some works in improving efficiency~\cite{instant_nerf2022, lin2022enerf}, the time-consuming training and rendering still limit its practicality.
Alternatively, 3D Gaussian Splatting (3DGS)~\cite{3DGS2023} models the scene as explicit Gaussian kernels, with differentiable splatting for rendering. Its improved real-time rendering performance, lower storage and efficiency, quickly attract more attentions.
% Different from NeRF-based methods which need MLPs to model the scene and huge computational cost for rendering, 3DGS has stronger real-time performance, higher storage and computational efficiency, benefits from its explicit representation and gradient backpropagation.

\subsection{Optimizing Camera Poses in NeRFs and 3DGS}
Although NeRF and 3DGS can provide impressive scene representation, these methods all need accurate camera parameters (both intrinsic and extrinsic) as additional inputs, which are mostly obtained by COLMAP~\cite{colmap2016}.
% This strong reliance on COLMAP significantly limits their use in real-world applications, so optimizing the camera parameters during the scene training becomes crucial.
When the prior is inaccurate or unknown, accurately estimating camera parameters and scene representations becomes crucial.

% In early works, only photometric constraints are used for scene training and camera pose estimation. 
% iNeRF~\cite{iNerf2021} optimizes the camera poses based on a pre-trained NeRF model.
% NeRFmm~\cite{wang2021nerfmm} introduce a joint optimization process, which estimates the camera poses and trains NeRF model jointly.
% BARF~\cite{barf2021} and GARF~\cite{2022GARF} provide new positional encoding strategy to handle with the gradient inconsistency issue of positional embedding and yield promising results.
% However, they achieve satisfactory optimization results when only the pose initialization is quite closed to the ground-truth, as the photometric constrains can only improve the quality of camera estimation within a small range.
% Later, more prior information of geometry and correspondence, \ie monocular depth and feature matching, are introduced into joint optimisation to enhance the capability of camera poses estimation.
% SC-NeRF~\cite{SCNeRF2021} minimizes a projected ray distance loss based on correspondence of adjacent frames.
% NoPe-NeRF~\cite{bian2022nopenerf} chooses monocular depth maps as geometric priors, and defines undistorted depth loss and relative pose constraints for joint optimization.
In earlier studies, scene training and camera pose estimation relied solely on photometric constraints. iNeRF~\cite{iNerf2021} refines the camera poses using a pre-trained NeRF model. NeRFmm~\cite{wang2021nerfmm} introduces a joint optimization approach that simultaneously estimates camera poses and trains the NeRF model. BARF~\cite{barf2021} and GARF~\cite{2022GARF} propose a new positional encoding strategy to address the gradient inconsistency issues in positional embedding, achieving promising results. However, these methods only yield satisfactory optimization when the initial pose is very close to the ground truth, as photometric constraints alone can only enhance camera estimation quality within a limited range. Subsequently, 
% additional prior information on geometry and correspondence, such as monocular depth and feature matching, has been incorporated into joint optimization to improve the accuracy of camera pose estimation. 
SC-NeRF~\cite{SCNeRF2021} minimizes a projected ray distance loss based on correspondence between adjacent frames. NoPe-NeRF~\cite{bian2022nopenerf} utilizes monocular depth maps as geometric priors and defines undistorted depth loss and relative pose constraints.

% With regard to 3D Gaussian Splatting, CF-3DGS~\cite{CF-3DGS-2024} also leverages mono-depth information to constrain the optimization of local 3DGS for relative pose estimation and later learn a global 3DGS progressively in a sequential manner.
% InstantSplat~\cite{fan2024instantsplat} focus on sparse view scenes, first use DUSt3R~\cite{dust3r2024cvpr} to generate a set of densely covered and pixel-aligned points for 3D Gaussian initialization, then introduce a parallel grid partitioning strategy in joint optimization to speed up.
% % Jiang et al.~\cite{Jiang_2024sig} proposed to build the scene continuously and progressively, to next unregistered frame, they use registration and adjustment to adjust the previous registered camera poses and align unregistered monocular depths, later refine the joint model by matching detected correspondences in screen-space coordinates.
% \gjh{Jiang et al.~\cite{Jiang_2024sig} also implemented an incremental approach for reconstructing camera poses and scenes. Initially, they perform feature matching between the current image and the image rendered by a differentiable surface renderer. They then construct matching point errors, depth errors, and photometric errors to achieve the registration and adjustment of the current image. Finally, based on the depth map, the pixels of the current image are projected as new 3D Gaussians. However, this method still exhibits limitations when dealing with complex scenes and unordered images.}
% % CG-3DGS~\cite{sun2024correspondenceguidedsfmfree3dgaussian} follows CF-3DGS, first construct a coarse point cloud from mono-depth maps to train a 3DGS model, then progressively estimate camera poses based on this pre-trained model by constraining the correspondences between rendering view and ground-truth.
% \gjh{Similarly, CG-3DGS~\cite{sun2024correspondenceguidedsfmfree3dgaussian} first utilizes monocular depth estimation and the camera parameters from the first frame to initialize a set of 3D Gaussians. It then progressively estimates camera poses based on this pre-trained model by constraining the correspondences between the rendered views and the ground truth.}
% % Free-SurGS~\cite{freesurgs2024} matches the projection flow derived from 3D Gaussians with optical flow to estimate the poses, to compensate for the limitations of photometric loss.
% \gjh{Free-SurGS~\cite{freesurgs2024} introduces the first SfM-free 3DGS approach for surgical scene reconstruction. Due to the challenges posed by weak textures and photometric inconsistencies in surgical scenes, Free-SurGS achieves pose estimation by minimizing the flow loss between the projection flow and the optical flow. Subsequently, it keeps the camera pose fixed and optimizes the scene representation by minimizing the photometric loss, depth loss and flow loss.}
% \gjh{However, most current works assume camera intrinsics are known and primarily focus on optimizing camera poses. Additionally, these methods typically rely on sequentially ordered image inputs and incrementally optimize camera parameters and scene representation. This inevitably leads to drift errors, preventing the achievement of globally consistent results. Our work aims to address these issues.}

Regarding 3D Gaussian Splatting, CF-3DGS~\cite{CF-3DGS-2024} utilizes mono-depth information to refine the optimization of local 3DGS for relative pose estimation and subsequently learns a global 3DGS in a sequential manner. InstantSplat~\cite{fan2024instantsplat} targets sparse view scenes, initially employing DUSt3R~\cite{dust3r2024cvpr} to create a densely covered, pixel-aligned point set for initializing 3D Gaussian models, and then implements a parallel grid partitioning strategy to accelerate joint optimization. Jiang \etal~\cite{Jiang_2024sig} develops an incremental method for reconstructing camera poses and scenes, but it struggles with complex scenes and unordered images. 
% Similarly, CG-3DGS~\cite{sun2024correspondenceguidedsfmfree3dgaussian} progressively estimates camera poses using a pre-trained model by aligning the correspondences between rendered views and actual scenes. Free-SurGS~\cite{freesurgs2024} pioneers an SfM-free 3DGS method for reconstructing surgical scenes, overcoming challenges such as weak textures and photometric inconsistencies by minimizing the discrepancy between projection flow and optical flow.
%\pb{SF-3DGS-HT~\cite{ji2024sfmfree3dgaussiansplatting} introduced VFI into training as additional photometric constraints. They separated the whole scene into several local 3DGS models and then merged them hierarchically, which leads to a significant improvement on simple and dense view scenes.}
HT-3DGS~\cite{ji2024sfmfree3dgaussiansplatting} interpolates frames for training and splits the scene into local clips, using a hierarchical strategy to build 3DGS model. It works well for simple scenes, but fails with dramatic motions due to unstable interpolation and low efficiency.
% {While effective for simple scenes, it struggles with dramatic motion due to unstable view interpolation and suffers from low computational efficiency.}

However, most existing methods generally depend on sequentially ordered image inputs and incrementally optimize camera parameters and 3DGS, which often leads to drift errors and hinders achieving globally consistent results. Our work seeks to overcome these limitations.

\section{Discussion}\label{sec:discussion}



\subsection{From Interactive Prompting to Interactive Multi-modal Prompting}
The rapid advancements of large pre-trained generative models including large language models and text-to-image generation models, have inspired many HCI researchers to develop interactive tools to support users in crafting appropriate prompts.
% Studies on this topic in last two years' HCI conferences are predominantly focused on helping users refine single-modality textual prompts.
Many previous studies are focused on helping users refine single-modality textual prompts.
However, for many real-world applications concerning data beyond text modality, such as multi-modal AI and embodied intelligence, information from other modalities is essential in constructing sophisticated multi-modal prompts that fully convey users' instruction.
This demand inspires some researchers to develop multimodal prompting interactions to facilitate generation tasks ranging from visual modality image generation~\cite{wang2024promptcharm, promptpaint} to textual modality story generation~\cite{chung2022tale}.
% Some previous studies contributed relevant findings on this topic. 
Specifically, for the image generation task, recent studies have contributed some relevant findings on multi-modal prompting.
For example, PromptCharm~\cite{wang2024promptcharm} discovers the importance of multimodal feedback in refining initial text-based prompting in diffusion models.
However, the multi-modal interactions in PromptCharm are mainly focused on the feedback empowered the inpainting function, instead of supporting initial multimodal sketch-prompt control. 

\begin{figure*}[t]
    \centering
    \includegraphics[width=0.9\textwidth]{src/img/novice_expert.pdf}
    \vspace{-2mm}
    \caption{The comparison between novice and expert participants in painting reveals that experts produce more accurate and fine-grained sketches, resulting in closer alignment with reference images in close-ended tasks. Conversely, in open-ended tasks, expert fine-grained strokes fail to generate precise results due to \tool's lack of control at the thin stroke level.}
    \Description{The comparison between novice and expert participants in painting reveals that experts produce more accurate and fine-grained sketches, resulting in closer alignment with reference images in close-ended tasks. Novice users create rougher sketches with less accuracy in shape. Conversely, in open-ended tasks, expert fine-grained strokes fail to generate precise results due to \tool's lack of control at the thin stroke level, while novice users' broader strokes yield results more aligned with their sketches.}
    \label{fig:novice_expert}
    % \vspace{-3mm}
\end{figure*}


% In particular, in the initial control input, users are unable to explicitly specify multi-modal generation intents.
In another example, PromptPaint~\cite{promptpaint} stresses the importance of paint-medium-like interactions and introduces Prompt stencil functions that allow users to perform fine-grained controls with localized image generation. 
However, insufficient spatial control (\eg, PromptPaint only allows for single-object prompt stencil at a time) and unstable models can still leave some users feeling the uncertainty of AI and a varying degree of ownership of the generated artwork~\cite{promptpaint}.
% As a result, the gap between intuitive multi-modal or paint-medium-like control and the current prompting interface still exists, which requires further research on multi-modal prompting interactions.
From this perspective, our work seeks to further enhance multi-object spatial-semantic prompting control by users' natural sketching.
However, there are still some challenges to be resolved, such as consistent multi-object generation in multiple rounds to increase stability and improved understanding of user sketches.   


% \new{
% From this perspective, our work is a step forward in this direction by allowing multi-object spatial-semantic prompting control by users' natural sketching, which considers the interplay between multiple sketch regions.
% % To further advance the multi-modal prompting experience, there are some aspects we identify to be important.
% % One of the important aspects is enhancing the consistency and stability of multiple rounds of generation to reduce the uncertainty and loss of control on users' part.
% % For this purpose, we need to develop techniques to incorporate consistent generation~\cite{tewel2024training} into multi-modal prompting framework.}
% % Another important aspect is improving generative models' understanding of the implicit user intents \new{implied by the paint-medium-like or sketch-based input (\eg, sketch of two people with their hands slightly overlapping indicates holding hand without needing explicit prompt).
% % This can facilitate more natural control and alleviate users' effort in tuning the textual prompt.
% % In addition, it can increase users' sense of ownership as the generated results can be more aligned with their sketching intents.
% }
% For example, when users draw sketches of two people with their hands slightly overlapping, current region-based models cannot automatically infer users' implicit intention that the two people are holding hands.
% Instead, they still require users to explicitly specify in the prompt such relationship.
% \tool addresses this through sketch-aware prompt recommendation to fill in the necessary semantic information, alleviating users' workload.
% However, some users want the generative AI in the future to be able to directly infer this natural implicit intentions from the sketches without additional prompting since prompt recommendation can still be unstable sometimes.


% \new{
% Besides visual generation, 
% }
% For example, one of the important aspect is referring~\cite{he2024multi}, linking specific text semantics with specific spatial object, which is partly what we do in our sketch-aware prompt recommendation.
% Analogously, in natural communication between humans, text or audio alone often cannot suffice in expressing the speakers' intentions, and speakers often need to refer to an existing spatial object or draw out an illustration of her ideas for better explanation.
% Philosophically, we HCI researchers are mostly concerned about the human-end experience in human-AI communications.
% However, studies on prompting is unique in that we should not just care about the human-end interaction, but also make sure that AI can really get what the human means and produce intention-aligned output.
% Such consideration can drastically impact the design of prompting interactions in human-AI collaboration applications.
% On this note, although studies on multi-modal interactions is a well-established topic in HCI community, it remains a challenging problem what kind of multi-modal information is really effective in helping humans convey their ideas to current and next generation large AI models.




\subsection{Novice Performance vs. Expert Performance}\label{sec:nVe}
In this section we discuss the performance difference between novice and expert regarding experience in painting and prompting.
First, regarding painting skills, some participants with experience (4/12) preferred to draw accurate and fine-grained shapes at the beginning. 
All novice users (5/12) draw rough and less accurate shapes, while some participants with basic painting skills (3/12) also favored sketching rough areas of objects, as exemplified in Figure~\ref{fig:novice_expert}.
The experienced participants using fine-grained strokes (4/12, none of whom were experienced in prompting) achieved higher IoU scores (0.557) in the close-ended task (0.535) when using \tool. 
This is because their sketches were closer in shape and location to the reference, making the single object decomposition result more accurate.
Also, experienced participants are better at arranging spatial location and size of objects than novice participants.
However, some experienced participants (3/12) have mentioned that the fine-grained stroke sometimes makes them frustrated.
As P1's comment for his result in open-ended task: "\emph{It seems it cannot understand thin strokes; even if the shape is accurate, it can only generate content roughly around the area, especially when there is overlapping.}" 
This suggests that while \tool\ provides rough control to produce reasonably fine results from less accurate sketches for novice users, it may disappoint experienced users seeking more precise control through finer strokes. 
As shown in the last column in Figure~\ref{fig:novice_expert}, the dragon hovering in the sky was wrongly turned into a standing large dragon by \tool.

Second, regarding prompting skills, 3 out of 12 participants had one or more years of experience in T2I prompting. These participants used more modifiers than others during both T2I and R2I tasks.
Their performance in the T2I (0.335) and R2I (0.469) tasks showed higher scores than the average T2I (0.314) and R2I (0.418), but there was no performance improvement with \tool\ between their results (0.508) and the overall average score (0.528). 
This indicates that \tool\ can assist novice users in prompting, enabling them to produce satisfactory images similar to those created by users with prompting expertise.



\subsection{Applicability of \tool}
The feedback from user study highlighted several potential applications for our system. 
Three participants (P2, P6, P8) mentioned its possible use in commercial advertising design, emphasizing the importance of controllability for such work. 
They noted that the system's flexibility allows designers to quickly experiment with different settings.
Some participants (N = 3) also mentioned its potential for digital asset creation, particularly for game asset design. 
P7, a game mod developer, found the system highly useful for mod development. 
He explained: "\emph{Mods often require a series of images with a consistent theme and specific spatial requirements. 
For example, in a sacrifice scene, how the objects are arranged is closely tied to the mod's background. It would be difficult for a developer without professional skills, but with this system, it is possible to quickly construct such images}."
A few participants expressed similar thoughts regarding its use in scene construction, such as in film production. 
An interesting suggestion came from participant P4, who proposed its application in crime scene description. 
She pointed out that witnesses are often not skilled artists, and typically describe crime scenes verbally while someone else illustrates their account. 
With this system, witnesses could more easily express what they saw themselves, potentially producing depictions closer to the real events. "\emph{Details like object locations and distances from buildings can be easily conveyed using the system}," she added.

% \subsection{Model Understanding of Users' Implicit Intents}
% In region-sketch-based control of generative models, a significant gap between interaction design and actual implementation is the model's failure in understanding users' naturally expressed intentions.
% For example, when users draw sketches of two people with their hands slightly overlapping, current region-based models cannot automatically infer users' implicit intention that the two people are holding hands.
% Instead, they still require users to explicitly specify in the prompt such relationship.
% \tool addresses this through sketch-aware prompt recommendation to fill in the necessary semantic information, alleviating users' workload.
% However, some users want the generative AI in the future to be able to directly infer this natural implicit intentions from the sketches without additional prompting since prompt recommendation can still be unstable sometimes.
% This problem reflects a more general dilemma, which ubiquitously exists in all forms of conditioned control for generative models such as canny or scribble control.
% This is because all the control models are trained on pairs of explicit control signal and target image, which is lacking further interpretation or customization of the user intentions behind the seemingly straightforward input.
% For another example, the generative models cannot understand what abstraction level the user has in mind for her personal scribbles.
% Such problems leave more challenges to be addressed by future human-AI co-creation research.
% One possible direction is fine-tuning the conditioned models on individual user's conditioned control data to provide more customized interpretation. 

% \subsection{Balance between recommendation and autonomy}
% AIGC tools are a typical example of 
\subsection{Progressive Sketching}
Currently \tool is mainly aimed at novice users who are only capable of creating very rough sketches by themselves.
However, more accomplished painters or even professional artists typically have a coarse-to-fine creative process. 
Such a process is most evident in painting styles like traditional oil painting or digital impasto painting, where artists first quickly lay down large color patches to outline the most primitive proportion and structure of visual elements.
After that, the artists will progressively add layers of finer color strokes to the canvas to gradually refine the painting to an exquisite piece of artwork.
One participant in our user study (P1) , as a professional painter, has mentioned a similar point "\emph{
I think it is useful for laying out the big picture, give some inspirations for the initial drawing stage}."
Therefore, rough sketch also plays a part in the professional artists' creation process, yet it is more challenging to integrate AI into this more complex coarse-to-fine procedure.
Particularly, artists would like to preserve some of their finer strokes in later progression, not just the shape of the initial sketch.
In addition, instead of requiring the tool to generate a finished piece of artwork, some artists may prefer a model that can generate another more accurate sketch based on the initial one, and leave the final coloring and refining to the artists themselves.
To accommodate these diverse progressive sketching requirements, a more advanced sketch-based AI-assisted creation tool should be developed that can seamlessly enable artist intervention at any stage of the sketch and maximally preserve their creative intents to the finest level. 

\subsection{Ethical Issues}
Intellectual property and unethical misuse are two potential ethical concerns of AI-assisted creative tools, particularly those targeting novice users.
In terms of intellectual property, \tool hands over to novice users more control, giving them a higher sense of ownership of the creation.
However, the question still remains: how much contribution from the user's part constitutes full authorship of the artwork?
As \tool still relies on backbone generative models which may be trained on uncopyrighted data largely responsible for turning the sketch into finished artwork, we should design some mechanisms to circumvent this risk.
For example, we can allow artists to upload backbone models trained on their own artworks to integrate with our sketch control.
Regarding unethical misuse, \tool makes fine-grained spatial control more accessible to novice users, who may maliciously generate inappropriate content such as more realistic deepfake with specific postures they want or other explicit content.
To address this issue, we plan to incorporate a more sophisticated filtering mechanism that can detect and screen unethical content with more complex spatial-semantic conditions. 
% In the future, we plan to enable artists to upload their own style model

% \subsection{From interactive prompting to interactive spatial prompting}


\subsection{Limitations and Future work}

    \textbf{User Study Design}. Our open-ended task assesses the usability of \tool's system features in general use cases. To further examine aspects such as creativity and controllability across different methods, the open-ended task could be improved by incorporating baselines to provide more insightful comparative analysis. 
    Besides, in close-ended tasks, while the fixing order of tool usage prevents prior knowledge leakage, it might introduce learning effects. In our study, we include practice sessions for the three systems before the formal task to mitigate these effects. In the future, utilizing parallel tests (\textit{e.g.} different content with the same difficulty) or adding a control group could further reduce the learning effects.

    \textbf{Failure Cases}. There are certain failure cases with \tool that can limit its usability. 
    Firstly, when there are three or more objects with similar semantics, objects may still be missing despite prompt recommendations. 
    Secondly, if an object's stroke is thin, \tool may incorrectly interpret it as a full area, as demonstrated in the expert results of the open-ended task in Figure~\ref{fig:novice_expert}. 
    Finally, sometimes inclusion relationships (\textit{e.g.} inside) between objects cannot be generated correctly, partially due to biases in the base model that lack training samples with such relationship. 

    \textbf{More support for single object adjustment}.
    Participants (N=4) suggested that additional control features should be introduced, beyond just adjusting size and location. They noted that when objects overlap, they cannot freely control which object appears on top or which should be covered, and overlapping areas are currently not allowed.
    They proposed adding features such as layer control and depth control within the single-object mask manipulation. Currently, the system assigns layers based on color order, but future versions should allow users to adjust the layer of each object freely, while considering weighted prompts for overlapping areas.

    \textbf{More customized generation ability}.
    Our current system is built around a single model $ColorfulXL-Lightning$, which limits its ability to fully support the diverse creative needs of users. Feedback from participants has indicated a strong desire for more flexibility in style and personalization, such as integrating fine-tuned models that cater to specific artistic styles or individual preferences. 
    This limitation restricts the ability to adapt to varied creative intents across different users and contexts.
    In future iterations, we plan to address this by embedding a model selection feature, allowing users to choose from a variety of pre-trained or custom fine-tuned models that better align with their stylistic preferences. 
    
    \textbf{Integrate other model functions}.
    Our current system is compatible with many existing tools, such as Promptist~\cite{hao2024optimizing} and Magic Prompt, allowing users to iteratively generate prompts for single objects. However, the integration of these functions is somewhat limited in scope, and users may benefit from a broader range of interactive options, especially for more complex generation tasks. Additionally, for multimodal large models, users can currently explore using affordable or open-source models like Qwen2-VL~\cite{qwen} and InternVL2-Llama3~\cite{llama}, which have demonstrated solid inference performance in our tests. While GPT-4o remains a leading choice, alternative models also offer competitive results.
    Moving forward, we aim to integrate more multimodal large models into the system, giving users the flexibility to choose the models that best fit their needs. 
    


\section{Conclusion}\label{sec:conclusion}
In this paper, we present \tool, an interactive system designed to help novice users create high-quality, fine-grained images that align with their intentions based on rough sketches. 
The system first refines the user's initial prompt into a complete and coherent one that matches the rough sketch, ensuring the generated results are both stable, coherent and high quality.
To further support users in achieving fine-grained alignment between the generated image and their creative intent without requiring professional skills, we introduce a decompose-and-recompose strategy. 
This allows users to select desired, refined object shapes for individual decomposed objects and then recombine them, providing flexible mask manipulation for precise spatial control.
The framework operates through a coarse-to-fine process, enabling iterative and fine-grained control that is not possible with traditional end-to-end generation methods. 
Our user study demonstrates that \tool offers novice users enhanced flexibility in control and fine-grained alignment between their intentions and the generated images.


%----------------------------------------------------------------------%
% 致谢
%----------------------------------------------------------------------%


%----------------------------------------------------------------------%
% 参考文献
%----------------------------------------------------------------------%


\bibliography{refs}

% for arxiv
\bibliographystyle{icml2025}
% \bibliographystyle{plain}

%----------------------------------------------------------------------%
% 附录(如需要)
%----------------------------------------------------------------------%
\appendix
% \section{Appendix}
\label{sec:appendix}
% 这里可以补充额外的实验、推导或附加说明
\subsection{Lloyd-Max Algorithm}
\label{subsec:Lloyd-Max}
For a given quantization bitwidth $B$ and an operand $\bm{X}$, the Lloyd-Max algorithm finds $2^B$ quantization levels $\{\hat{x}_i\}_{i=1}^{2^B}$ such that quantizing $\bm{X}$ by rounding each scalar in $\bm{X}$ to the nearest quantization level minimizes the quantization MSE. 

The algorithm starts with an initial guess of quantization levels and then iteratively computes quantization thresholds $\{\tau_i\}_{i=1}^{2^B-1}$ and updates quantization levels $\{\hat{x}_i\}_{i=1}^{2^B}$. Specifically, at iteration $n$, thresholds are set to the midpoints of the previous iteration's levels:
\begin{align*}
    \tau_i^{(n)}=\frac{\hat{x}_i^{(n-1)}+\hat{x}_{i+1}^{(n-1)}}2 \text{ for } i=1\ldots 2^B-1
\end{align*}
Subsequently, the quantization levels are re-computed as conditional means of the data regions defined by the new thresholds:
\begin{align*}
    \hat{x}_i^{(n)}=\mathbb{E}\left[ \bm{X} \big| \bm{X}\in [\tau_{i-1}^{(n)},\tau_i^{(n)}] \right] \text{ for } i=1\ldots 2^B
\end{align*}
where to satisfy boundary conditions we have $\tau_0=-\infty$ and $\tau_{2^B}=\infty$. The algorithm iterates the above steps until convergence.

Figure \ref{fig:lm_quant} compares the quantization levels of a $7$-bit floating point (E3M3) quantizer (left) to a $7$-bit Lloyd-Max quantizer (right) when quantizing a layer of weights from the GPT3-126M model at a per-tensor granularity. As shown, the Lloyd-Max quantizer achieves substantially lower quantization MSE. Further, Table \ref{tab:FP7_vs_LM7} shows the superior perplexity achieved by Lloyd-Max quantizers for bitwidths of $7$, $6$ and $5$. The difference between the quantizers is clear at 5 bits, where per-tensor FP quantization incurs a drastic and unacceptable increase in perplexity, while Lloyd-Max quantization incurs a much smaller increase. Nevertheless, we note that even the optimal Lloyd-Max quantizer incurs a notable ($\sim 1.5$) increase in perplexity due to the coarse granularity of quantization. 

\begin{figure}[h]
  \centering
  \includegraphics[width=0.7\linewidth]{sections/figures/LM7_FP7.pdf}
  \caption{\small Quantization levels and the corresponding quantization MSE of Floating Point (left) vs Lloyd-Max (right) Quantizers for a layer of weights in the GPT3-126M model.}
  \label{fig:lm_quant}
\end{figure}

\begin{table}[h]\scriptsize
\begin{center}
\caption{\label{tab:FP7_vs_LM7} \small Comparing perplexity (lower is better) achieved by floating point quantizers and Lloyd-Max quantizers on a GPT3-126M model for the Wikitext-103 dataset.}
\begin{tabular}{c|cc|c}
\hline
 \multirow{2}{*}{\textbf{Bitwidth}} & \multicolumn{2}{|c|}{\textbf{Floating-Point Quantizer}} & \textbf{Lloyd-Max Quantizer} \\
 & Best Format & Wikitext-103 Perplexity & Wikitext-103 Perplexity \\
\hline
7 & E3M3 & 18.32 & 18.27 \\
6 & E3M2 & 19.07 & 18.51 \\
5 & E4M0 & 43.89 & 19.71 \\
\hline
\end{tabular}
\end{center}
\end{table}

\subsection{Proof of Local Optimality of LO-BCQ}
\label{subsec:lobcq_opt_proof}
For a given block $\bm{b}_j$, the quantization MSE during LO-BCQ can be empirically evaluated as $\frac{1}{L_b}\lVert \bm{b}_j- \bm{\hat{b}}_j\rVert^2_2$ where $\bm{\hat{b}}_j$ is computed from equation (\ref{eq:clustered_quantization_definition}) as $C_{f(\bm{b}_j)}(\bm{b}_j)$. Further, for a given block cluster $\mathcal{B}_i$, we compute the quantization MSE as $\frac{1}{|\mathcal{B}_{i}|}\sum_{\bm{b} \in \mathcal{B}_{i}} \frac{1}{L_b}\lVert \bm{b}- C_i^{(n)}(\bm{b})\rVert^2_2$. Therefore, at the end of iteration $n$, we evaluate the overall quantization MSE $J^{(n)}$ for a given operand $\bm{X}$ composed of $N_c$ block clusters as:
\begin{align*}
    \label{eq:mse_iter_n}
    J^{(n)} = \frac{1}{N_c} \sum_{i=1}^{N_c} \frac{1}{|\mathcal{B}_{i}^{(n)}|}\sum_{\bm{v} \in \mathcal{B}_{i}^{(n)}} \frac{1}{L_b}\lVert \bm{b}- B_i^{(n)}(\bm{b})\rVert^2_2
\end{align*}

At the end of iteration $n$, the codebooks are updated from $\mathcal{C}^{(n-1)}$ to $\mathcal{C}^{(n)}$. However, the mapping of a given vector $\bm{b}_j$ to quantizers $\mathcal{C}^{(n)}$ remains as  $f^{(n)}(\bm{b}_j)$. At the next iteration, during the vector clustering step, $f^{(n+1)}(\bm{b}_j)$ finds new mapping of $\bm{b}_j$ to updated codebooks $\mathcal{C}^{(n)}$ such that the quantization MSE over the candidate codebooks is minimized. Therefore, we obtain the following result for $\bm{b}_j$:
\begin{align*}
\frac{1}{L_b}\lVert \bm{b}_j - C_{f^{(n+1)}(\bm{b}_j)}^{(n)}(\bm{b}_j)\rVert^2_2 \le \frac{1}{L_b}\lVert \bm{b}_j - C_{f^{(n)}(\bm{b}_j)}^{(n)}(\bm{b}_j)\rVert^2_2
\end{align*}

That is, quantizing $\bm{b}_j$ at the end of the block clustering step of iteration $n+1$ results in lower quantization MSE compared to quantizing at the end of iteration $n$. Since this is true for all $\bm{b} \in \bm{X}$, we assert the following:
\begin{equation}
\begin{split}
\label{eq:mse_ineq_1}
    \tilde{J}^{(n+1)} &= \frac{1}{N_c} \sum_{i=1}^{N_c} \frac{1}{|\mathcal{B}_{i}^{(n+1)}|}\sum_{\bm{b} \in \mathcal{B}_{i}^{(n+1)}} \frac{1}{L_b}\lVert \bm{b} - C_i^{(n)}(b)\rVert^2_2 \le J^{(n)}
\end{split}
\end{equation}
where $\tilde{J}^{(n+1)}$ is the the quantization MSE after the vector clustering step at iteration $n+1$.

Next, during the codebook update step (\ref{eq:quantizers_update}) at iteration $n+1$, the per-cluster codebooks $\mathcal{C}^{(n)}$ are updated to $\mathcal{C}^{(n+1)}$ by invoking the Lloyd-Max algorithm \citep{Lloyd}. We know that for any given value distribution, the Lloyd-Max algorithm minimizes the quantization MSE. Therefore, for a given vector cluster $\mathcal{B}_i$ we obtain the following result:

\begin{equation}
    \frac{1}{|\mathcal{B}_{i}^{(n+1)}|}\sum_{\bm{b} \in \mathcal{B}_{i}^{(n+1)}} \frac{1}{L_b}\lVert \bm{b}- C_i^{(n+1)}(\bm{b})\rVert^2_2 \le \frac{1}{|\mathcal{B}_{i}^{(n+1)}|}\sum_{\bm{b} \in \mathcal{B}_{i}^{(n+1)}} \frac{1}{L_b}\lVert \bm{b}- C_i^{(n)}(\bm{b})\rVert^2_2
\end{equation}

The above equation states that quantizing the given block cluster $\mathcal{B}_i$ after updating the associated codebook from $C_i^{(n)}$ to $C_i^{(n+1)}$ results in lower quantization MSE. Since this is true for all the block clusters, we derive the following result: 
\begin{equation}
\begin{split}
\label{eq:mse_ineq_2}
     J^{(n+1)} &= \frac{1}{N_c} \sum_{i=1}^{N_c} \frac{1}{|\mathcal{B}_{i}^{(n+1)}|}\sum_{\bm{b} \in \mathcal{B}_{i}^{(n+1)}} \frac{1}{L_b}\lVert \bm{b}- C_i^{(n+1)}(\bm{b})\rVert^2_2  \le \tilde{J}^{(n+1)}   
\end{split}
\end{equation}

Following (\ref{eq:mse_ineq_1}) and (\ref{eq:mse_ineq_2}), we find that the quantization MSE is non-increasing for each iteration, that is, $J^{(1)} \ge J^{(2)} \ge J^{(3)} \ge \ldots \ge J^{(M)}$ where $M$ is the maximum number of iterations. 
%Therefore, we can say that if the algorithm converges, then it must be that it has converged to a local minimum. 
\hfill $\blacksquare$


\begin{figure}
    \begin{center}
    \includegraphics[width=0.5\textwidth]{sections//figures/mse_vs_iter.pdf}
    \end{center}
    \caption{\small NMSE vs iterations during LO-BCQ compared to other block quantization proposals}
    \label{fig:nmse_vs_iter}
\end{figure}

Figure \ref{fig:nmse_vs_iter} shows the empirical convergence of LO-BCQ across several block lengths and number of codebooks. Also, the MSE achieved by LO-BCQ is compared to baselines such as MXFP and VSQ. As shown, LO-BCQ converges to a lower MSE than the baselines. Further, we achieve better convergence for larger number of codebooks ($N_c$) and for a smaller block length ($L_b$), both of which increase the bitwidth of BCQ (see Eq \ref{eq:bitwidth_bcq}).


\subsection{Additional Accuracy Results}
%Table \ref{tab:lobcq_config} lists the various LOBCQ configurations and their corresponding bitwidths.
\begin{table}
\setlength{\tabcolsep}{4.75pt}
\begin{center}
\caption{\label{tab:lobcq_config} Various LO-BCQ configurations and their bitwidths.}
\begin{tabular}{|c||c|c|c|c||c|c||c|} 
\hline
 & \multicolumn{4}{|c||}{$L_b=8$} & \multicolumn{2}{|c||}{$L_b=4$} & $L_b=2$ \\
 \hline
 \backslashbox{$L_A$\kern-1em}{\kern-1em$N_c$} & 2 & 4 & 8 & 16 & 2 & 4 & 2 \\
 \hline
 64 & 4.25 & 4.375 & 4.5 & 4.625 & 4.375 & 4.625 & 4.625\\
 \hline
 32 & 4.375 & 4.5 & 4.625& 4.75 & 4.5 & 4.75 & 4.75 \\
 \hline
 16 & 4.625 & 4.75& 4.875 & 5 & 4.75 & 5 & 5 \\
 \hline
\end{tabular}
\end{center}
\end{table}

%\subsection{Perplexity achieved by various LO-BCQ configurations on Wikitext-103 dataset}

\begin{table} \centering
\begin{tabular}{|c||c|c|c|c||c|c||c|} 
\hline
 $L_b \rightarrow$& \multicolumn{4}{c||}{8} & \multicolumn{2}{c||}{4} & 2\\
 \hline
 \backslashbox{$L_A$\kern-1em}{\kern-1em$N_c$} & 2 & 4 & 8 & 16 & 2 & 4 & 2  \\
 %$N_c \rightarrow$ & 2 & 4 & 8 & 16 & 2 & 4 & 2 \\
 \hline
 \hline
 \multicolumn{8}{c}{GPT3-1.3B (FP32 PPL = 9.98)} \\ 
 \hline
 \hline
 64 & 10.40 & 10.23 & 10.17 & 10.15 &  10.28 & 10.18 & 10.19 \\
 \hline
 32 & 10.25 & 10.20 & 10.15 & 10.12 &  10.23 & 10.17 & 10.17 \\
 \hline
 16 & 10.22 & 10.16 & 10.10 & 10.09 &  10.21 & 10.14 & 10.16 \\
 \hline
  \hline
 \multicolumn{8}{c}{GPT3-8B (FP32 PPL = 7.38)} \\ 
 \hline
 \hline
 64 & 7.61 & 7.52 & 7.48 &  7.47 &  7.55 &  7.49 & 7.50 \\
 \hline
 32 & 7.52 & 7.50 & 7.46 &  7.45 &  7.52 &  7.48 & 7.48  \\
 \hline
 16 & 7.51 & 7.48 & 7.44 &  7.44 &  7.51 &  7.49 & 7.47  \\
 \hline
\end{tabular}
\caption{\label{tab:ppl_gpt3_abalation} Wikitext-103 perplexity across GPT3-1.3B and 8B models.}
\end{table}

\begin{table} \centering
\begin{tabular}{|c||c|c|c|c||} 
\hline
 $L_b \rightarrow$& \multicolumn{4}{c||}{8}\\
 \hline
 \backslashbox{$L_A$\kern-1em}{\kern-1em$N_c$} & 2 & 4 & 8 & 16 \\
 %$N_c \rightarrow$ & 2 & 4 & 8 & 16 & 2 & 4 & 2 \\
 \hline
 \hline
 \multicolumn{5}{|c|}{Llama2-7B (FP32 PPL = 5.06)} \\ 
 \hline
 \hline
 64 & 5.31 & 5.26 & 5.19 & 5.18  \\
 \hline
 32 & 5.23 & 5.25 & 5.18 & 5.15  \\
 \hline
 16 & 5.23 & 5.19 & 5.16 & 5.14  \\
 \hline
 \multicolumn{5}{|c|}{Nemotron4-15B (FP32 PPL = 5.87)} \\ 
 \hline
 \hline
 64  & 6.3 & 6.20 & 6.13 & 6.08  \\
 \hline
 32  & 6.24 & 6.12 & 6.07 & 6.03  \\
 \hline
 16  & 6.12 & 6.14 & 6.04 & 6.02  \\
 \hline
 \multicolumn{5}{|c|}{Nemotron4-340B (FP32 PPL = 3.48)} \\ 
 \hline
 \hline
 64 & 3.67 & 3.62 & 3.60 & 3.59 \\
 \hline
 32 & 3.63 & 3.61 & 3.59 & 3.56 \\
 \hline
 16 & 3.61 & 3.58 & 3.57 & 3.55 \\
 \hline
\end{tabular}
\caption{\label{tab:ppl_llama7B_nemo15B} Wikitext-103 perplexity compared to FP32 baseline in Llama2-7B and Nemotron4-15B, 340B models}
\end{table}

%\subsection{Perplexity achieved by various LO-BCQ configurations on MMLU dataset}


\begin{table} \centering
\begin{tabular}{|c||c|c|c|c||c|c|c|c|} 
\hline
 $L_b \rightarrow$& \multicolumn{4}{c||}{8} & \multicolumn{4}{c||}{8}\\
 \hline
 \backslashbox{$L_A$\kern-1em}{\kern-1em$N_c$} & 2 & 4 & 8 & 16 & 2 & 4 & 8 & 16  \\
 %$N_c \rightarrow$ & 2 & 4 & 8 & 16 & 2 & 4 & 2 \\
 \hline
 \hline
 \multicolumn{5}{|c|}{Llama2-7B (FP32 Accuracy = 45.8\%)} & \multicolumn{4}{|c|}{Llama2-70B (FP32 Accuracy = 69.12\%)} \\ 
 \hline
 \hline
 64 & 43.9 & 43.4 & 43.9 & 44.9 & 68.07 & 68.27 & 68.17 & 68.75 \\
 \hline
 32 & 44.5 & 43.8 & 44.9 & 44.5 & 68.37 & 68.51 & 68.35 & 68.27  \\
 \hline
 16 & 43.9 & 42.7 & 44.9 & 45 & 68.12 & 68.77 & 68.31 & 68.59  \\
 \hline
 \hline
 \multicolumn{5}{|c|}{GPT3-22B (FP32 Accuracy = 38.75\%)} & \multicolumn{4}{|c|}{Nemotron4-15B (FP32 Accuracy = 64.3\%)} \\ 
 \hline
 \hline
 64 & 36.71 & 38.85 & 38.13 & 38.92 & 63.17 & 62.36 & 63.72 & 64.09 \\
 \hline
 32 & 37.95 & 38.69 & 39.45 & 38.34 & 64.05 & 62.30 & 63.8 & 64.33  \\
 \hline
 16 & 38.88 & 38.80 & 38.31 & 38.92 & 63.22 & 63.51 & 63.93 & 64.43  \\
 \hline
\end{tabular}
\caption{\label{tab:mmlu_abalation} Accuracy on MMLU dataset across GPT3-22B, Llama2-7B, 70B and Nemotron4-15B models.}
\end{table}


%\subsection{Perplexity achieved by various LO-BCQ configurations on LM evaluation harness}

\begin{table} \centering
\begin{tabular}{|c||c|c|c|c||c|c|c|c|} 
\hline
 $L_b \rightarrow$& \multicolumn{4}{c||}{8} & \multicolumn{4}{c||}{8}\\
 \hline
 \backslashbox{$L_A$\kern-1em}{\kern-1em$N_c$} & 2 & 4 & 8 & 16 & 2 & 4 & 8 & 16  \\
 %$N_c \rightarrow$ & 2 & 4 & 8 & 16 & 2 & 4 & 2 \\
 \hline
 \hline
 \multicolumn{5}{|c|}{Race (FP32 Accuracy = 37.51\%)} & \multicolumn{4}{|c|}{Boolq (FP32 Accuracy = 64.62\%)} \\ 
 \hline
 \hline
 64 & 36.94 & 37.13 & 36.27 & 37.13 & 63.73 & 62.26 & 63.49 & 63.36 \\
 \hline
 32 & 37.03 & 36.36 & 36.08 & 37.03 & 62.54 & 63.51 & 63.49 & 63.55  \\
 \hline
 16 & 37.03 & 37.03 & 36.46 & 37.03 & 61.1 & 63.79 & 63.58 & 63.33  \\
 \hline
 \hline
 \multicolumn{5}{|c|}{Winogrande (FP32 Accuracy = 58.01\%)} & \multicolumn{4}{|c|}{Piqa (FP32 Accuracy = 74.21\%)} \\ 
 \hline
 \hline
 64 & 58.17 & 57.22 & 57.85 & 58.33 & 73.01 & 73.07 & 73.07 & 72.80 \\
 \hline
 32 & 59.12 & 58.09 & 57.85 & 58.41 & 73.01 & 73.94 & 72.74 & 73.18  \\
 \hline
 16 & 57.93 & 58.88 & 57.93 & 58.56 & 73.94 & 72.80 & 73.01 & 73.94  \\
 \hline
\end{tabular}
\caption{\label{tab:mmlu_abalation} Accuracy on LM evaluation harness tasks on GPT3-1.3B model.}
\end{table}

\begin{table} \centering
\begin{tabular}{|c||c|c|c|c||c|c|c|c|} 
\hline
 $L_b \rightarrow$& \multicolumn{4}{c||}{8} & \multicolumn{4}{c||}{8}\\
 \hline
 \backslashbox{$L_A$\kern-1em}{\kern-1em$N_c$} & 2 & 4 & 8 & 16 & 2 & 4 & 8 & 16  \\
 %$N_c \rightarrow$ & 2 & 4 & 8 & 16 & 2 & 4 & 2 \\
 \hline
 \hline
 \multicolumn{5}{|c|}{Race (FP32 Accuracy = 41.34\%)} & \multicolumn{4}{|c|}{Boolq (FP32 Accuracy = 68.32\%)} \\ 
 \hline
 \hline
 64 & 40.48 & 40.10 & 39.43 & 39.90 & 69.20 & 68.41 & 69.45 & 68.56 \\
 \hline
 32 & 39.52 & 39.52 & 40.77 & 39.62 & 68.32 & 67.43 & 68.17 & 69.30  \\
 \hline
 16 & 39.81 & 39.71 & 39.90 & 40.38 & 68.10 & 66.33 & 69.51 & 69.42  \\
 \hline
 \hline
 \multicolumn{5}{|c|}{Winogrande (FP32 Accuracy = 67.88\%)} & \multicolumn{4}{|c|}{Piqa (FP32 Accuracy = 78.78\%)} \\ 
 \hline
 \hline
 64 & 66.85 & 66.61 & 67.72 & 67.88 & 77.31 & 77.42 & 77.75 & 77.64 \\
 \hline
 32 & 67.25 & 67.72 & 67.72 & 67.00 & 77.31 & 77.04 & 77.80 & 77.37  \\
 \hline
 16 & 68.11 & 68.90 & 67.88 & 67.48 & 77.37 & 78.13 & 78.13 & 77.69  \\
 \hline
\end{tabular}
\caption{\label{tab:mmlu_abalation} Accuracy on LM evaluation harness tasks on GPT3-8B model.}
\end{table}

\begin{table} \centering
\begin{tabular}{|c||c|c|c|c||c|c|c|c|} 
\hline
 $L_b \rightarrow$& \multicolumn{4}{c||}{8} & \multicolumn{4}{c||}{8}\\
 \hline
 \backslashbox{$L_A$\kern-1em}{\kern-1em$N_c$} & 2 & 4 & 8 & 16 & 2 & 4 & 8 & 16  \\
 %$N_c \rightarrow$ & 2 & 4 & 8 & 16 & 2 & 4 & 2 \\
 \hline
 \hline
 \multicolumn{5}{|c|}{Race (FP32 Accuracy = 40.67\%)} & \multicolumn{4}{|c|}{Boolq (FP32 Accuracy = 76.54\%)} \\ 
 \hline
 \hline
 64 & 40.48 & 40.10 & 39.43 & 39.90 & 75.41 & 75.11 & 77.09 & 75.66 \\
 \hline
 32 & 39.52 & 39.52 & 40.77 & 39.62 & 76.02 & 76.02 & 75.96 & 75.35  \\
 \hline
 16 & 39.81 & 39.71 & 39.90 & 40.38 & 75.05 & 73.82 & 75.72 & 76.09  \\
 \hline
 \hline
 \multicolumn{5}{|c|}{Winogrande (FP32 Accuracy = 70.64\%)} & \multicolumn{4}{|c|}{Piqa (FP32 Accuracy = 79.16\%)} \\ 
 \hline
 \hline
 64 & 69.14 & 70.17 & 70.17 & 70.56 & 78.24 & 79.00 & 78.62 & 78.73 \\
 \hline
 32 & 70.96 & 69.69 & 71.27 & 69.30 & 78.56 & 79.49 & 79.16 & 78.89  \\
 \hline
 16 & 71.03 & 69.53 & 69.69 & 70.40 & 78.13 & 79.16 & 79.00 & 79.00  \\
 \hline
\end{tabular}
\caption{\label{tab:mmlu_abalation} Accuracy on LM evaluation harness tasks on GPT3-22B model.}
\end{table}

\begin{table} \centering
\begin{tabular}{|c||c|c|c|c||c|c|c|c|} 
\hline
 $L_b \rightarrow$& \multicolumn{4}{c||}{8} & \multicolumn{4}{c||}{8}\\
 \hline
 \backslashbox{$L_A$\kern-1em}{\kern-1em$N_c$} & 2 & 4 & 8 & 16 & 2 & 4 & 8 & 16  \\
 %$N_c \rightarrow$ & 2 & 4 & 8 & 16 & 2 & 4 & 2 \\
 \hline
 \hline
 \multicolumn{5}{|c|}{Race (FP32 Accuracy = 44.4\%)} & \multicolumn{4}{|c|}{Boolq (FP32 Accuracy = 79.29\%)} \\ 
 \hline
 \hline
 64 & 42.49 & 42.51 & 42.58 & 43.45 & 77.58 & 77.37 & 77.43 & 78.1 \\
 \hline
 32 & 43.35 & 42.49 & 43.64 & 43.73 & 77.86 & 75.32 & 77.28 & 77.86  \\
 \hline
 16 & 44.21 & 44.21 & 43.64 & 42.97 & 78.65 & 77 & 76.94 & 77.98  \\
 \hline
 \hline
 \multicolumn{5}{|c|}{Winogrande (FP32 Accuracy = 69.38\%)} & \multicolumn{4}{|c|}{Piqa (FP32 Accuracy = 78.07\%)} \\ 
 \hline
 \hline
 64 & 68.9 & 68.43 & 69.77 & 68.19 & 77.09 & 76.82 & 77.09 & 77.86 \\
 \hline
 32 & 69.38 & 68.51 & 68.82 & 68.90 & 78.07 & 76.71 & 78.07 & 77.86  \\
 \hline
 16 & 69.53 & 67.09 & 69.38 & 68.90 & 77.37 & 77.8 & 77.91 & 77.69  \\
 \hline
\end{tabular}
\caption{\label{tab:mmlu_abalation} Accuracy on LM evaluation harness tasks on Llama2-7B model.}
\end{table}

\begin{table} \centering
\begin{tabular}{|c||c|c|c|c||c|c|c|c|} 
\hline
 $L_b \rightarrow$& \multicolumn{4}{c||}{8} & \multicolumn{4}{c||}{8}\\
 \hline
 \backslashbox{$L_A$\kern-1em}{\kern-1em$N_c$} & 2 & 4 & 8 & 16 & 2 & 4 & 8 & 16  \\
 %$N_c \rightarrow$ & 2 & 4 & 8 & 16 & 2 & 4 & 2 \\
 \hline
 \hline
 \multicolumn{5}{|c|}{Race (FP32 Accuracy = 48.8\%)} & \multicolumn{4}{|c|}{Boolq (FP32 Accuracy = 85.23\%)} \\ 
 \hline
 \hline
 64 & 49.00 & 49.00 & 49.28 & 48.71 & 82.82 & 84.28 & 84.03 & 84.25 \\
 \hline
 32 & 49.57 & 48.52 & 48.33 & 49.28 & 83.85 & 84.46 & 84.31 & 84.93  \\
 \hline
 16 & 49.85 & 49.09 & 49.28 & 48.99 & 85.11 & 84.46 & 84.61 & 83.94  \\
 \hline
 \hline
 \multicolumn{5}{|c|}{Winogrande (FP32 Accuracy = 79.95\%)} & \multicolumn{4}{|c|}{Piqa (FP32 Accuracy = 81.56\%)} \\ 
 \hline
 \hline
 64 & 78.77 & 78.45 & 78.37 & 79.16 & 81.45 & 80.69 & 81.45 & 81.5 \\
 \hline
 32 & 78.45 & 79.01 & 78.69 & 80.66 & 81.56 & 80.58 & 81.18 & 81.34  \\
 \hline
 16 & 79.95 & 79.56 & 79.79 & 79.72 & 81.28 & 81.66 & 81.28 & 80.96  \\
 \hline
\end{tabular}
\caption{\label{tab:mmlu_abalation} Accuracy on LM evaluation harness tasks on Llama2-70B model.}
\end{table}

%\section{MSE Studies}
%\textcolor{red}{TODO}


\subsection{Number Formats and Quantization Method}
\label{subsec:numFormats_quantMethod}
\subsubsection{Integer Format}
An $n$-bit signed integer (INT) is typically represented with a 2s-complement format \citep{yao2022zeroquant,xiao2023smoothquant,dai2021vsq}, where the most significant bit denotes the sign.

\subsubsection{Floating Point Format}
An $n$-bit signed floating point (FP) number $x$ comprises of a 1-bit sign ($x_{\mathrm{sign}}$), $B_m$-bit mantissa ($x_{\mathrm{mant}}$) and $B_e$-bit exponent ($x_{\mathrm{exp}}$) such that $B_m+B_e=n-1$. The associated constant exponent bias ($E_{\mathrm{bias}}$) is computed as $(2^{{B_e}-1}-1)$. We denote this format as $E_{B_e}M_{B_m}$.  

\subsubsection{Quantization Scheme}
\label{subsec:quant_method}
A quantization scheme dictates how a given unquantized tensor is converted to its quantized representation. We consider FP formats for the purpose of illustration. Given an unquantized tensor $\bm{X}$ and an FP format $E_{B_e}M_{B_m}$, we first, we compute the quantization scale factor $s_X$ that maps the maximum absolute value of $\bm{X}$ to the maximum quantization level of the $E_{B_e}M_{B_m}$ format as follows:
\begin{align}
\label{eq:sf}
    s_X = \frac{\mathrm{max}(|\bm{X}|)}{\mathrm{max}(E_{B_e}M_{B_m})}
\end{align}
In the above equation, $|\cdot|$ denotes the absolute value function.

Next, we scale $\bm{X}$ by $s_X$ and quantize it to $\hat{\bm{X}}$ by rounding it to the nearest quantization level of $E_{B_e}M_{B_m}$ as:

\begin{align}
\label{eq:tensor_quant}
    \hat{\bm{X}} = \text{round-to-nearest}\left(\frac{\bm{X}}{s_X}, E_{B_e}M_{B_m}\right)
\end{align}

We perform dynamic max-scaled quantization \citep{wu2020integer}, where the scale factor $s$ for activations is dynamically computed during runtime.

\subsection{Vector Scaled Quantization}
\begin{wrapfigure}{r}{0.35\linewidth}
  \centering
  \includegraphics[width=\linewidth]{sections/figures/vsquant.jpg}
  \caption{\small Vectorwise decomposition for per-vector scaled quantization (VSQ \citep{dai2021vsq}).}
  \label{fig:vsquant}
\end{wrapfigure}
During VSQ \citep{dai2021vsq}, the operand tensors are decomposed into 1D vectors in a hardware friendly manner as shown in Figure \ref{fig:vsquant}. Since the decomposed tensors are used as operands in matrix multiplications during inference, it is beneficial to perform this decomposition along the reduction dimension of the multiplication. The vectorwise quantization is performed similar to tensorwise quantization described in Equations \ref{eq:sf} and \ref{eq:tensor_quant}, where a scale factor $s_v$ is required for each vector $\bm{v}$ that maps the maximum absolute value of that vector to the maximum quantization level. While smaller vector lengths can lead to larger accuracy gains, the associated memory and computational overheads due to the per-vector scale factors increases. To alleviate these overheads, VSQ \citep{dai2021vsq} proposed a second level quantization of the per-vector scale factors to unsigned integers, while MX \citep{rouhani2023shared} quantizes them to integer powers of 2 (denoted as $2^{INT}$).

\subsubsection{MX Format}
The MX format proposed in \citep{rouhani2023microscaling} introduces the concept of sub-block shifting. For every two scalar elements of $b$-bits each, there is a shared exponent bit. The value of this exponent bit is determined through an empirical analysis that targets minimizing quantization MSE. We note that the FP format $E_{1}M_{b}$ is strictly better than MX from an accuracy perspective since it allocates a dedicated exponent bit to each scalar as opposed to sharing it across two scalars. Therefore, we conservatively bound the accuracy of a $b+2$-bit signed MX format with that of a $E_{1}M_{b}$ format in our comparisons. For instance, we use E1M2 format as a proxy for MX4.

\begin{figure}
    \centering
    \includegraphics[width=1\linewidth]{sections//figures/BlockFormats.pdf}
    \caption{\small Comparing LO-BCQ to MX format.}
    \label{fig:block_formats}
\end{figure}

Figure \ref{fig:block_formats} compares our $4$-bit LO-BCQ block format to MX \citep{rouhani2023microscaling}. As shown, both LO-BCQ and MX decompose a given operand tensor into block arrays and each block array into blocks. Similar to MX, we find that per-block quantization ($L_b < L_A$) leads to better accuracy due to increased flexibility. While MX achieves this through per-block $1$-bit micro-scales, we associate a dedicated codebook to each block through a per-block codebook selector. Further, MX quantizes the per-block array scale-factor to E8M0 format without per-tensor scaling. In contrast during LO-BCQ, we find that per-tensor scaling combined with quantization of per-block array scale-factor to E4M3 format results in superior inference accuracy across models. 
 

\end{document}