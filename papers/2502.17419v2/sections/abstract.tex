\begin{abstract}
Achieving human-level intelligence requires refining the transition from the fast, intuitive System 1 to the slower, more deliberate System 2 reasoning. 
While System 1 excels in quick, heuristic decisions, System 2 relies on logical reasoning for more accurate judgments and reduced biases. 
Foundational Large Language Models (LLMs) excel at fast decision-making but lack the depth for complex reasoning, as they have not yet fully embraced the step-by-step analysis characteristic of true System 2 thinking. 
Recently, reasoning LLMs like OpenAI's o1/o3 and DeepSeek's R1 have demonstrated expert-level performance in fields such as mathematics and coding, closely mimicking the deliberate reasoning of System 2 and showcasing human-like cognitive abilities. 
This survey begins with a brief overview of the progress in foundational LLMs and the early development of System 2 technologies, exploring how their combination has paved the way for reasoning LLMs. 
Next, we discuss how to construct reasoning LLMs, analyzing their features, the core methods enabling advanced reasoning, and the evolution of various reasoning LLMs. 
Additionally, we provide an overview of reasoning benchmarks, offering an in-depth comparison of the performance of representative reasoning LLMs. 
Finally, we explore promising directions for advancing reasoning LLMs and maintain a real-time \href{https://github.com/zzli2022/Awesome-Slow-Reason-System}{{GitHub Repository}} to track the latest developments. 
We hope this survey will serve as a valuable resource to inspire innovation and drive progress in this rapidly evolving field.


%This survey offers a comprehensive review of the evolution of key technologies in reasoning LLMs and the challenges they face. 


% Achieving human-level intelligence requires refining the transition from System 1 to System 2 reasoning. System 1 is fast and intuitive, enabling quick, heuristic-driven decisions, while System 2 is slower and more deliberate, relying on logical reasoning and systematic analysis to solve complex problems. This shift from intuition to analysis allows for more accurate judgments and mitigates the cognitive biases inherent in System 1. 
% This survey begins with a concise overview of the key developmental phases of fast-thinking Artificial Intelligence (AI) systems, emphasizing advancements in symbolic logic, statistical methods, deep neural networks, and foundational Large Language Models (LLMs). 
% Each phase marks significant progress, contributing to the overall evolution of AI. 
% While foundational LLMs demonstrate impressive performance across various tasks, they do not yet embody true System 2 thinking and lack the depth of human intelligence, especially in complex reasoning scenarios. 
% Recently, reasoning LLMs, such as OpenAI's o1, have achieved expert-level performance in domains like mathematics and coding, closely mirroring the slow-thinking nature of System 2 and exhibiting human-like reasoning abilities. 












\end{abstract}

% as well as the technical frameworks 


% 1. 主线还是 system1 to system2,不用大改, 聚焦近十几年 从dnn 到llm 到resoning llms的过程; 
% dnn 和基础llm 实际上主要是fast-thinking, 感知任务
% reasoning llm slow-thinking  在复杂推理任务取得了突破

% reasoning llm的这个过程 其实是 融合了之前的一些slow-thinking思想,,,介绍一下这些slow思想的发展过程,他如何影响了后续的reasoning llm,以及和现在reasoning llm技术的联系,

% 3. slow发展其实是 纯slow的符号逻辑推理,后来是概率图, 搜索算法,rl;后来dnn兴起后,slow其实和dnn有过结合,比如神经符号,侧重推理的gnn,深度rl(alphago,alphazero,alphastar这些),

% 4. o1 算是 slow思想和llm(包含丰富的世界知识)结合的高级slow/ system2, 和之前slow+dnn不同的是,,这个更open world,之前主要是closed world 或者specific tasks
