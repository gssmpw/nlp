\newcommand{\spdm}{
    \left[
        \vec\mu_0,
        \vec\sigma_0,
        \dots,
        \vec\mu_G,
        \vec\sigma_G
        \right]^T
}
\newcommand{\spath}{
    \left(
    \vec x ~\|~\mathrm{0}^{2\times N_\mathrm{waypoints}-t}
    \right)^T
}

\section{Environment Modelling}
\label{sect:method}

\subsection{Agent Dynamics}

The agent within the environment is modelled as a heading control model with a fixed step size $s \unit{\meter}$.
It is assumed that any physical vehicle, such as a drone in the case of WiSAR, executing this mission can accurately track the waypoints via its controller or operator.
The agent's state is represented by the 2D position vector $\vec x = [x,y]^T \in \mathbb{R}^2$ and it's dynamics are described by the nonlinear system
\begin{align}
    \dot{\vec x}
     & =
    s
    \begin{bmatrix}
        \cos (\pi (a_t+1)) \\
        \sin (\pi (a_t+1))
    \end{bmatrix} \\
    \vec y
     & =
    \begin{bmatrix}
        1 & 0 \\ 0& 1
    \end{bmatrix}
    \begin{bmatrix}
        x(t) \\ y(t)
    \end{bmatrix}
\end{align}
where $a_t \in [-1,1]$ is the agent's action at time-step $t$ and the Euler integration scheme, $\vec x_{t+1} = \vec x_t + \delta t \dot{\vec x}$,  is used with $\delta t = 1 \unit{\second}$.

\subsection{Probability Distribution Map}

% \begin{figure}[htbp]
%     \centering
%     \begin{tikzpicture}[]
%         \begin{axis}[
%                 xlabel=$x$,
%                 ylabel=$y$,
%                 zlabel=$z$,
%                 hide axis,
%             ]
%             \addplot3[
%                 mesh,
%                 samples=30,
%                 domain=-5:5,
%             ]
%             {
%                 (1/2) * bivariate_gaussian(x,y, 2.5,2.5, 1, 1, 0) + bivariate_gaussian(x,y, -2.5,-2.5, 2, 2, 0)
%             };
%         \end{axis}
%     \end{tikzpicture}
%     \caption{Example $p(\vec x)$ with $\ngauss = 2$, $\vec \mu_0 = [2.5,2.5]$, $\vec\sigma_0 = \mathrm{diag}([0,0])$, $\vec \mu_1 = [-2.5,-2.5]$, and $\vec\sigma_1 = \mathrm{diag}([2,2])$}
%     \label{fig:method:pdm}
% \end{figure}

It is assumed that the \nomdef{Probability Distribution Map}{PDM} is known and is modelled as a sum of $\ngauss$ bivariate Gaussians \citep{ewers_enhancing_2024}.
\citet{yao_gaussian_2017} and \citet{yao_gaussian_2022} showed that bivariate Gaussians are effective at approximating PDMs.
The agent at position $\vec x$ thus has a probability of being above the true search objective, a missing person for WiSAR, given by
\begin{equation}
    p(\vec x) =
    \frac{1}{\ngauss}
    \sum^\ngauss_{i=0}
    \frac
    {
        \exp{
            \left[
                -\frac
                {1}
                {2}
                (\vec x - \vec \mu_i )^T\vec\sigma_i^{-1}(\vec x - \vec \mu_i)
                \right]
        }
    }
    {\sqrt{4\pi^2\det{\vec\sigma_i}}}\label{eqn:sum_of_bivariate_gaussians}
    \\
\end{equation}
where $\vec \mu_i \in \mathbb{R}^2$
and
$\vec \sigma_i \in \mathbb{R}^{2 \times 2}$
are the mean location and covariance matrix of the $i$th bivariate Gaussian respectively. The mean is reset at the start of every episode with
\begin{equation}
    \vec \mu_i \sim \mathcal{U}([x_\min , x_\max],
    [y_\min, y_\max]
    ),\forall i \in [0,\ngauss]\label{eqn:method:sampling_mus}
\end{equation}

The covariance is left unchanged to avoid a further tunable simulation parameter. \autoref{fig:method:pdm} shows how the various PDMs are still highly irregular even with constant covariance due to the randomness introduced by \autoref{eqn:method:sampling_mus}.

\begin{figure}[htbp]
    \centering
    \newcommand{\threetwodgauss}[4]{
        \begin{tikzpicture}[
            ]
            \begin{axis}[
                    axis equal image,
                    scale only axis,
                    xlabel=$x$,
                    ylabel=$y$,
                    zlabel=$z$,
                    hide axis,
                    view={0}{90},
                    width=#4,
                    colormap={bw}{
                            gray(0cm)=(0);
                            gray(1cm)=(1);
                        },
                ]
                \addplot3[
                    surf,
                    shader=flat,
                    samples=30,
                    domain=0:15,
                ]
                {
                    (1/3)*(
                    bivariate_gaussian(x, y, #1, #1, 2.5, 2.5, 0) +
                    bivariate_gaussian(x, y, #2, #2, 2.5, 2.5, 0) +
                    bivariate_gaussian(x, y, #3, #3, 2.5, 2.5, 0)
                    )
                };
            \end{axis}
        \end{tikzpicture}
    }

    \subfloat[The three modes are far apart with the saddle being close to $0$.]{
        \threetwodgauss{0}{7.5}{15}{0.24\linewidth}
    }
    \hfill
    \subfloat[The saddle is almost the same value as the maxima of the modes.]{
        \threetwodgauss{2.5}{7.5}{12.5}{0.24\linewidth}
    }
    \hfill
    \subfloat[All three maxima have merged into a pill-shaped area of high value.]{
        \threetwodgauss{5}{7.5}{10}{0.24\linewidth}
    }
    \hfill
    \subfloat[All maxima are aligned leading to a single hotspot.]{
        \threetwodgauss{7.5}{7.5}{7.5}{0.24\linewidth}
    }
    \caption{Example PDM $p(\vec x)$ with $\ngauss = 3$ and constant covariance. Lighter areas are of higher probability whilst darker ones have lower probability. During search planning the agent would avoid lower probability regions whilst targeting high probability ones to maximize total seen probability.}
    \label{fig:method:pdm}
\end{figure}

\subsection{Reward Architecture}
\label{sect:method:rewards}

As the agent moves a constant distance $s\si\meter$ every step, it is assumed that the camera follows this path continuously at a fixed height whilst pointing straight down at all times.
Therefore, to represent the \tech{seen area} for a given path at time-step $t$, the path is buffered by $\rbuffer\si\meter$ to give the polygon $H_t$.
All probability from the PDM enclosed within $H_t$ is then \tech{seen} and denoted by $p_t$.
This value, the seen probability, is calculated through
\begin{equation}
    I(C) = \oint_C f(\vec x) dC
\end{equation}
Substituting $C=H_t$ and \autoref{eqn:sum_of_bivariate_gaussians} gives
\begin{equation}
    p_t = \oint_{H_t} p(\vec x) d H_t
    \label{eqn:accumulated_probability_at_t}
\end{equation}

Our goal is to maximize the captured probability mass.  To gain insight into how the agent's path affects this, we first focus on the local behaviour of $p_t$. We analyze the area covered by the agent in two steps, as this provides a foundation for understanding more complex paths.  The following lemma demonstrates a crucial property of this two-step area in the simplified case of a uniform PDM.
\begin{lemma}
    \label{lemma:method:pdm_optimal_theta_0}
    For a uniform PDM, $p(\vec x) = 1$, the area $A(\theta)$ of the region $H$ after two steps, as defined by
    \begin{equation}
        \begin{gathered}
            A_2(\theta) =
            \overbrace{4 s \rbuffer}^\textrm{Main Area}
            +\overbrace{\pi \rbuffer^2}^\textrm{Semi-Circle}
            +\overbrace{\frac{1}{2} \rbuffer^2 \theta}^{\textrm{Rounded corner}~g(\theta)}
            -\overbrace{
                \rbuffer^2 \tan
                \left(
                \frac{\theta}{2}
                \right)
            }^{\mathrm{Overlap}~f(\theta)} \\
            \forall~\theta \in \left[0, 2 \arctan\left(\frac{s}{\rbuffer}\right)\right]
            \label{eqn:method:area_of_two_steps}
        \end{gathered}
    \end{equation}
    where $\rbuffer$ and $s$ are constants, is maximized when $\theta = 0$.
\end{lemma}
\begin{proof}
    Assume, for the sake of contradiction, that there exists a $\theta^* \in \left[0, 2 \arctan\left(\frac{s}{\rbuffer}\right)\right]$ such that $A_2(\theta^*) > A_2(0)$.
    The derivative of $A_2$ with respect to $\theta$ is
    \begin{equation}
        \dv{A_2}{\theta}
        = \frac{1}{2} \rbuffer^2 \left(1-\sec^2\left(\frac{\theta}{2}\right)\right)
    \end{equation}
    Since $1- \sec^2(x) \leq 0~\forall~ x \in \mathbb{R}$, we have $\dv{A_2}{\theta} \leq 0~\forall~\theta \in \left[0, 2 \arctan\left(\frac{s}{\rbuffer}\right)\right]$.
    Because the derivative is non-positive, $A_2(\theta)$ is a monotonically decreasing function on the given interval.
    Since $A_2(\theta)$ is monotonically decreasing, for any $\theta^* > 0$, it must be the case that  $A_2(\theta^*) \leq A_2(0)$.
    Our assumption that there exists a $\theta^*$ such that $A_2(\theta^*)>A_2(0)$ must be false. Therefore, the maximum value of $A_2(\theta)$ is achieved when $\theta=0$.
\end{proof}
Further insights can be garnered by applying Green's theorem
\begin{equation}
    \oint_C (Ldx+Mdy)=\int \int_D (\pdv{M}{x} - \pdv{L}{y}) dA
\end{equation}
with $\pdv{M}{x} - \pdv{L}{y}=1$. This shows that decreasing the boundary $C=H_2$ reduces the area of the region $D$ bounded by $C$. With a uniform PDM, maximizing the geometric area is equivalent to maximizing the captured probability mass. From \autoref{lemma:method:pdm_optimal_theta_0}, the buffered polygon formulation maximizes the integral when $\theta=0$ for a uniform PDF. However, if $\pdv{M}{x} - \pdv{L}{y}$ is not constant it could be beneficial to increase $\theta$ and therefore reducing the area in order to maximize the encapsulated values.

Special consideration must be taken for the case where $\frac{s}{\rbuffer} < \frac{\pi}{2}$ as \autoref{lemma:method:pdm_optimal_theta_0} does not hold and must be further explored. This constraint, however, is always met in this work.

% The area of $H_t$, $A_2(\theta)$, after taking two steps with control input $\theta$ is
% \begin{equation}
%     \begin{gathered}
%         A_2(\theta) =
%         \overbrace{4 s \rbuffer}^\textrm{Main Area}
%         +\overbrace{\pi \rbuffer^2}^\textrm{Semi-Circle}
%         +\overbrace{\frac{1}{2} \rbuffer^2 \theta}^{\textrm{Rounded corner}~g(\theta)}
%         -\overbrace{
%             \rbuffer^2 \tan
%             \left(
%                 \frac{\theta}{2}
%             \right)
%         }^{\mathrm{Overlap}~f(\theta)} \\
%         \forall \theta \in \left(0, 2 \arctan\left(\frac{\rbuffer}{s}\right)\right]
%     \end{gathered}
% \end{equation}
% Differentiating $A_2$ with respect to $\theta$ gives
% \begin{equation}
%     \dv{A_2}{\theta}
%     = \frac{1}{2} \rbuffer^2 \left(1-\sec^2\left(\frac{\theta}{2}\right)\right)
% \end{equation}
% Since $1- \sec^2(x) \leq 0~\forall~ x \in \mathbb{R}$ the maximum area is then
% \begin{equation}
%     A_{2,\max} = \lim_{\theta \to 0^+} A_2(\theta).
% \end{equation}
% Applying Green's theorem
% \begin{equation}
%     \oint_C (Ldx+Mdy)=\int \int_D (\pdv{M}{x} - \pdv{L}{y}) dA
% \end{equation}
% with $\pdv{M}{x} - \pdv{L}{y}=1$ shows that decreasing the boundary $C=H_2$, reduces the area of the region $D$ bounded by $C$. With a uniform PDM, maximizing the geometric area is equivalent to maximizing the captured probability mass. Therefore, the buffered polygon formulation, which maximizes the area as
% $\theta \to 0$, is optimal for a uniform PDF. However, if $\pdv{M}{x} - \pdv{L}{y}$ is not constant it could be beneficial to increase $\theta$ and therefore reducing the area in order to maximize the encapsulated values.

\begin{figure}[htbp]
    \newcommand{\bufferedlinestring}[3]{
        \begin{tikzpicture}[]
            \def\angle{#1};
            \def\radius{#2};
            \def\length{#3};

            \node[rectangle, draw, dotted, minimum height=2*\radius cm, minimum width=\length cm, anchor=west] (A)  {};
            \node[rectangle, draw, dotted, minimum height=2*\radius cm, minimum width=\length cm, rotate=\angle, anchor=west] (B) at (A.east) {};

            \node[rectangle, fit=(A) (B)] (surround) {};
            \begin{scope}
                \path[clip] (A.north west) -- (A.north east) -- (A.south east) -- (A.south west) --cycle;
                \path[clip] (B.north west) -- (B.north east) -- (B.south east) -- (B.south west) --cycle;
                \fill[red!50] (surround.south west) rectangle (surround.north east);
            \end{scope}
            \fill[green!50,] (A.east) -- (A.south east) arc[start angle=270, delta angle=\angle, radius=\radius] --cycle;
            \draw[dotted] (B.south west) -- +(\length/2,0);
            \draw[latex-latex] (B.south) arc[start angle=\angle, delta angle=-\angle, radius=\length/2] node[midway,left] {$\theta$};
            \coordinate (ABNorthIntersect) at (intersection of  A.north west--A.north east and B.north west--B.north east);

            \path[draw] (A.north west)
            -- (ABNorthIntersect)
            -- (B.north east) arc[start angle=90+\angle, delta angle=-180, radius=\radius]
            -- (B.south west) arc[start angle=270+\angle, delta angle=-\angle, radius=\radius]
            -- (A.south west) arc[start angle=270, delta angle=-180, radius=\radius]
            --cycle;
            \draw[latex-latex] (A.north west) -- node[right,right] {$\rbuffer$} (A.west);
            \draw[latex-latex] (A.west) -- node[midway,below] {$s$} (A.east);
        \end{tikzpicture}
    }
    \centering
    \subfloat[
        $\theta=\frac{\pi}{4}$
    ]{
        \bufferedlinestring{45}{0.75}{2.5}
    }
    \hfill
    \subfloat[
        $\theta = 2\arctan(\frac{\pi}{2})$
    ]{
        \bufferedlinestring{2*atan(pi/2)}{0.75}{2.5}
    }
    \hfill
    \subfloat[
        $\theta = 2\arctan(\frac{s}{\rbuffer})$
    ]{
        \bufferedlinestring{2*atan(2.5/0.75)}{0.75}{2.5}
    }
    \caption{Visualizing $H_2$ ($H_t$ after two steps) with different $\theta$. The areas coloured in red and green represent $f(\theta)$ and $g(\theta)$ from \autoref{eqn:method:area_of_two_steps} respectively.}
    \label{fig:method:buffered_linestring}
\end{figure}

The action is correlated to the reward by considering the change in accumulated probability at time $t$, defined as
\begin{equation}
    \Delta p_t = p_t - p_{t-1}\label{eqn:delta_p_t}
\end{equation}
To normalize this change in accumulated probability we introduce the scaling constants $k$ and $p_A$.
Constant $k$ relates the area of a single isolated step to the total search area, $A_\mathrm{area}\unit{\meter^2}$.
Simplifying \autoref{eqn:method:area_of_two_steps} to the single step case using the constants defined in \autoref{fig:method:buffered_linestring}, gives the ratio
\begin{equation}
    k = \frac{A_\textit{area}}{\rbuffer(\pi\rbuffer+2s)}
    \label{eqn:k}
\end{equation}
Constant $p_A$ is the total probability enclosed within the total search area given by substituting $H_t = A$ in \autoref{eqn:accumulated_probability_at_t}.
This is gives the scaled probability efficiency reward
\begin{equation}
    r = \frac{k}{p_A} \Delta p_t
    \label{qen:method:scaled_probability_effectiency_reward}
\end{equation}
The ratio of change in accumulated probability to total probability enclosed within the search area satisfies the constraint that $0 \leq \frac{\Delta p_t}{p_A} \leq 1$. This ratio is the probability efficiency, $e_{p,t}$, and provides a useful insight into the performance of a given path.

Further reward shaping is applied to discourage future out-of-bounds actions ($w_{oob}$), and to penalize visiting areas of very low probability ($w_0$).
The augmented reward $r'$ is given by
% Reward
\begin{equation}
    r' =
    \begin{cases}
        -w_{oob},   & \vec x_t \notin [x_\min,x_\max] \times [y_\min,y_\max] \\
        w_r r ,     & \Delta p_t > \epsilon                                  \\
        -w_{0}    , & \textit{else}
    \end{cases}
    \label{eqn:reward_with_cases}
\end{equation}

\subsection{Observation Processing}

The observation vector at time $t\unit{\second}$ is denoted by $s_t$. To ensure flexibility when designing the architectures, the available sub-states are given in \autoref{tbl:observation_states} with architecture-specific observation space definitions given in \autoref{sect:architecture}.

\begin{table}[htb]
    \centering
    \caption{Definition of the five state observations}
    \label{tbl:observation_states}
    \begin{tabular}{@{}lll@{}}
        \toprule
        Sub-state       & Symbol             & Definition                                            \\ \midrule
        Path            & $s_\mathrm{path}$  & $\spath$                                              \\
        PDM             & $s_\mathrm{PDM}$   & $\spdm$                                               \\
        Position        & $s_\mathrm{pos}$   & $\vec x_t$                                            \\
        Out-of-bounds   & $s_\mathrm{oob}$   & $\vec x_t \in [x_\min,x_\max] \times [y_\min,y_\max]$ \\
        Number of steps & $s_\mathrm{steps}$ & $t$                                                   \\ \bottomrule
    \end{tabular}
\end{table}
