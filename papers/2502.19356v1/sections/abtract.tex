\begin{abstract}
    Wilderness search and rescue operations are often carried out over vast landscapes. The search efforts, however, must be undertaken in minimum time to maximize the chance of survival of the victim. Whilst the advent of cheap multicopters in recent years has changed the way search operations are handled, it has not solved the challenges of the massive areas at hand. The problem therefore is not one of complete coverage, but one of maximizing the information gathered in the limited time available.
    In this work we propose that a combination of a recurrent autoencoder and deep reinforcement learning is a more efficient solution to the search problem than previous pure deep reinforcement learning or optimisation approaches. The autoencoder training paradigm efficiently maximizes the information throughput of the encoder into its latent space representation which deep reinforcement learning is primed to leverage. Without the overhead of independently solving the problem that the  recurrent autoencoder is designed for, it is more efficient in learning the control task. We further implement three additional architectures for a comprehensive comparison of the main proposed architecture. Similarly, we apply both soft actor-critic and proximal policy optimisation to provide an insight into the performance of both in a highly non-linear and complex application with a large observation space.
    Results show that the proposed architecture is vastly superior  to the benchmarks, with soft actor-critic achieving the best performance. This model further outperformed work from the literature whilst having below a fifth of the total learnable parameters and training in a quarter of the time.
\end{abstract}