\section{Conclusion}
\label{sect:conclusion}

In this study, we have presented a novel approach to enhancing WiSAR missions through the integration of an LSTM-based RAE with DRL. The challenges associated with traditional search planning methods, particularly in large and complex environments, necessitate solutions that can optimize both efficiency and effectiveness in locating missing persons.

Our proposed framework addresses key limitations of existing methodologies by decoupling feature extraction from the policy training phase, thereby significantly reducing the number of learnable parameters and improving training speeds. By employing a RAE architecture, we have demonstrated that it is possible to maintain high performance while also ensuring the model is lightweight enough for deployment on resource-constrained devices. The empirical results indicate that our approach enhances the probability of detection and thus accelerates the overall search process, which is critical in time-sensitive rescue scenarios.

Furthermore, this work contributes to the growing body of literature on DRL applications in search planning by providing a comprehensive evaluation of various architectures, including comparisons between PPO and SAC algorithms in large observation domains. SAC outperformed PPO for the proposed architecture whereas PPO and SAC were similar for the rest. However, \fsppoconvtwod was the best non-recurrent architecture whilst \fssacconvtwod was the worst showing that this result is application specific. These findings underscore that DRL is not a golden bullet and the importance of careful model engineering when tackling difficult problems.
