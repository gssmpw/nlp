\documentclass[twoside]{article}

\usepackage{aistats2025}

\usepackage{microtype}
\usepackage{graphicx}
\usepackage{svg}
\usepackage{subcaption}
\usepackage{booktabs} 
\usepackage{hyperref}
\usepackage{amssymb}
\usepackage{amsmath}
\usepackage{commath}
\usepackage{graphicx,bm}
\usepackage{verbatim}
\usepackage{csquotes}
\usepackage{multirow}
% If your paper is accepted, change the options for the package
% aistats2025 as follows:
%
%\usepackage[accepted]{aistats2025}
%
% This option will print headings for the title of your paper and
% headings for the authors names, plus a copyright note at the end of
% the first column of the first page.

% If you set papersize explicitly, activate the following three lines:
%\special{papersize = 8.5in, 11in}
%\setlength{\pdfpageheight}{11in}
%\setlength{\pdfpagewidth}{8.5in}

% If you use natbib package, activate the following three lines:
\usepackage[round]{natbib}
\renewcommand{\bibname}{References}
\renewcommand{\bibsection}{\subsubsection*{\bibname}}

% If you use BibTeX in apalike style, activate the following line:
\bibliographystyle{apalike}

\begin{document}

% If your paper is accepted and the title of your paper is very long,
% the style will print as headings an error message. Use the following
% command to supply a shorter title of your paper so that it can be
% used as headings.
%
%\runningtitle{I use this title instead because the last one was very long}

% If your paper is accepted and the number of authors is large, the
% style will print as headings an error message. Use the following
% command to supply a shorter version of the authors names so that
% they can be used as headings (for example, use only the surnames)
%
%\runningauthor{Surname 1, Surname 2, Surname 3, ...., Surname n}

\twocolumn[

\aistatstitle{Multi-level Supervised Contrastive Learning}

\aistatsauthor{Naghmeh Ghanooni$^{1}$ \And  
    Barbod Pajoum$^{1}$\And  
    Harshit Rawal$^{1}$ \AND
    Sophie Fellenz$^{1}$ \And  
    Vo Nguyen Le Duy$^{2}$ \And  
    Marius Kloft$^{1}$
    }
 \vspace{10pt}
\aistatsaddress{$^{1}$RPTU Kaiserslautern-Landau \And $^{2}$RIKEN, Japan}]
% \footnotemark[*]
% \let\thefootnote\relax\footnotetext{* Indicates equal contribution}



\begin{abstract}
Contrastive learning is a well-established paradigm in representation learning. The standard framework of contrastive learning minimizes the distance between \enquote{similar} instances and maximizes the distance between dissimilar ones in the projection space, disregarding the various aspects of similarity that can exist between two samples. Current methods rely on a single projection head, which fails to capture the full complexity of different aspects of a sample, leading to sub-optimal performance, especially in scenarios with limited training data.
In this paper, we present a novel supervised contrastive learning method in a unified framework called multi-level contrastive learning (MLCL), that can be applied to both multi-label and hierarchical classification tasks. The key strength of the proposed method is the ability to capture similarities between samples across different labels and/or hierarchies using multiple projection heads. Extensive experiments on text and image datasets demonstrate that the proposed approach outperforms state-of-the-art contrastive learning methods. 
\end{abstract}

\section{Introduction}
\label{introduction}

% RL and supervised RL
Contrastive learning is a key framework in representation learning, with the primary objective of learning adjacent representations for \enquote{similar} samples. One widely adopted approach is supervised contrastive learning \citep{khosla2020supervised}, known as SupCon, where similarity is defined based on groundtruth labels. In SupCon, the model learns representations by bringing samples from the same class closer in the representation space, while distancing those from different classes. However, it relies on a single label per sample to define similarity, posing limitations for multi-label classification and datasets with hierarchical class structures. Existing approaches typically address these challenges separately, using specific loss functions \citep{zhang2022use} tailored for hierarchical structures or learning label-level embeddings \citep{dao2021contrast} for multi-label datasets \citep{zhang2024multi}. In contrast, our method introduces a unified framework called multi-level contrastive learning, which effectively captures both multi-label and hierarchical aspects of data within a single representation space.

\begin{figure*}
\subfloat[\label{fig:text-rep}]{%
  \includegraphics[width=.49\linewidth]{images/nlp_multi_level.png}%
}
\hfill
\subfloat[\label{fig:img-rep}]{%
  \includegraphics[width=.49\linewidth]{images/img_multi_level.png}%
}
\caption{An illustration of two projection spaces using different examples: a) TripAdvisor reviews and b) CIFAR-100. Similarity can be defined across various dimensions, and a single projection space is insufficient to capture the full spectrum of feature levels.}
\end{figure*}

In this work, we incorporate multiple projection heads into the standard contrastive learning architecture, each tailored to learn representations from a specific aspect of the instance. These projection heads are connected to the encoder to capture different levels of representation. To enhance the effect of the projection heads, a temperature hyperparameter is used to control class discrimination at each level in the projection space, as the concept of similarity differs across levels. By tuning the temperature, the model focuses on \enquote{hard negatives} at the relevant level while reducing their impact on others. This significantly improves performance, particularly in scenarios with limited training samples. 

The main contributions are summarized as follows:
\begin{itemize}
    \item This paper introduces a novel unified framework that combines hierarchical and multi-label settings into a single multi-level approach.
    \item We propose an end-to-end architecture that generalizes standard contrastive learning through multiple projection heads. 
    \item These projection heads capture label-specific similarities and class hierarchies or serve as a regularizer to prevent overfitting.
    \item We validate the effectiveness of the proposed method on multi-label and hierarchical classification tasks within the domains of natural language processing (NLP) and computer vision.
\end{itemize}

\section{Related Work}
\label{relatedwork}
\paragraph{Contrastive learning}
Learning a high-quality representation of the target data that can be transferred to various downstream tasks is a major challenge in both 
Natural Language Processing(NLP) and computer vision. Contrastive learning is a powerful technique in representation learning. SimCLR~\citep{chen2020simple} introduced a novel framework for acquiring high-quality representations from unlabeled images, achieving state-of-the-art results in the field. Its popularity stems from its simplicity, as it operates without the need for specialized architectures \citep{bachman2019learning, henaff2020data} or memory banks \citep{he2020momentum, wu2018unsupervised}, while demonstrating remarkable performance across diverse downstream tasks. SimCLR discriminates positive and negative pairs in the projection space, and large batch sizes with many negative pairs and various data augmentations are required for optimal performance. 
Several subsequent papers attempted to improve SimCLR’s performance and overcome its limitations. For example, \citet{chuang2020debiased} presents a debiased contrastive loss that corrects the sampling bias of negative examples, while BYOL \citep{grill2020bootstrap} relies only on positive samples for training and is more robust to batch size and data augmentations. Furthermore, \citet{kalantidis2020hard} emphasizes the importance of hard negatives in contrastive learning and proposes a hard negative mixing strategy that improves performance and avoids the need for large batch sizes and memory banks \citep{mocov2}. Recently, \citet{wang2023adaptive} introduced an adaptive multiple projection head mechanism for self-supervised contrastive learning models to address better intra- and inter-sample similarities between different views of the samples. They also incorporate an adaptive temperature mechanism to re-weight each similarity pair. In contrast, our approach employs multiple projection heads to capture different aspects of the samples within a supervised framework, making it suitable for downstream tasks such as multi-label and hierarchical classifications.

\begin{figure*}
\subfloat[Two projection heads $g_1$ and $g_2$ are attached to the encoder's output to capture subclass and superclass similarities respectively.\label{fig:img-arch}]{%
  \includegraphics[height=4cm,width=.49\linewidth]{images/img_arch.png}%
}
\hfill
\subfloat[We employ an individual projection head for each label and an additional global head to capture the overall similarities between the two reviews. \label{fig:text-arch}]{%
  \includegraphics[height=4cm,width=.49\linewidth]{images/nlp_arch.png}%
}
\caption{
Our proposed architecture with multiple projection heads for a) hierarchical classification and b) multi-label classification. 
The final loss is computed as a weighted sum of the losses from each head, with an additional cross-entropy loss applied in the case of text classification.
}
%\label{}
\end{figure*}

\paragraph{Supervised Contrastive Learning}
The self-supervised contrastive learning approach can be extended to supervised settings with minimal modifications. SupCon~\citep{khosla2020supervised} developed a supervised contrastive learning method that leverages labels by modifying the contrastive loss function. Instead of comparing each sample with all other samples, as in SimCLR, SupCon pulls samples from the same class closer and pushes samples from different classes farther apart. Several studies have aimed to enhance SupCon. For example, \citet{barbano2022unbiased} focused on controlling the minimal distance between positive and negative samples and introduced a new debiasing regularization loss. SupCon suffers from imbalanced data distribution, leading to poor uniformity in the feature distribution on the hypersphere. \citet{li2022targeted} addressed this issue by proposing a targeted supervised contrastive learning loss to maintain a uniform feature distribution even with long-tail distribution data. Additionally, \citet{chen2022perfectly} suggested a weighted class-conditional InfoNCE loss to enhance the robustness of SupCon. Furthermore, certain methods leverage contrastive representation to learn from noisy labels \citep{ciortan2021framework, ghosh2021contrastive, li2021learning, li2020mopro, xue2022investigating, yi2022learning}.  While these approaches indicate the suitability of contrastive learning in the supervised setting, they do not provide the possibility of contrasting multiple aspects.


\paragraph{Multi-label and Hierarchical Contrastive Learning}
Recent research has extended supervised contrastive learning (SCL) to multi-label contrastive learning \citep{dao2021multi, malkinski2022multi, dao2021contrast, zaigrajewcontrastive, zhang2024multi}. These extensions aim to bring samples with shared labels closer together while separating those with different labels. For instance, \citet{zaigrajewcontrastive} utilized the Jaccard similarity measure to assess label overlap between sample pairs and \citet{zhang2024multi} introduced multiple positive sets to address this problem. Meanwhile, \citet{sajedi2023end} introduced a kernel-based approach that transforms embedded features into a mixture of exponential kernels in Gaussian RKHS for multi-label contrastive learning. Additionally, \citet{dao2021contrast} proposed a module that learns multiple label-level representations using multiple attention blocks. However, the method by \citet{dao2021contrast}, which trains label-specific features, involves two-stage training and produces task-specific representations not universally applicable across all downstream tasks.  In contrast, our approach considers comprehensive similarity aspects among samples beyond mere label overlap, ensuring applicability across a wide range of classification tasks. Incorporating the class hierarchies, \citet{landrieu2021leveraging} (
Guided) integrate class distance into a prototypical network to model the hierarchical class structure and \citet{zhang2022use} (HiMulConE) introduced a hierarchical framework for multi-label representation learning with a new loss function penalizing sample distance in label hierarchies. Existing approaches often overlook diverse similarity perspectives, tailoring solutions to specific scenarios rather than broader applications.

\section{Approach}
\label{approach}

\subsection{Problem Formulation}
\label{sec:formulation}
For simplicity, we adopt a unified formulation for both multi-label and hierarchical classification. In each training iteration, a batch of randomly sampled inputs is given to the network, where each input is associated with \( L \) levels: \(\{x_k, y_k^l\}\), for \( k \in \{1, \dots, B\} \) and \( l \in \{1, \dots, L\} \). In multi-label classification, \( y_k^l \) corresponds to the labels assigned to sample \( x_k \), while for datasets with a hierarchical structure, \( y_k^l \) represents labels at different levels of the hierarchy. The objective is to train an encoder \( f(.) \) that learns semantically meaningful representations of the training data, \( f(x) \), which can subsequently be utilized for downstream tasks.

\subsection{Framework}
\label{subsec:framework}
Our approach is structurally similar to the standard supervised contrastive learning method \citep{khosla2020supervised}, with the key modification being the addition of multiple projection heads.
\begin{enumerate}
    \item For every sample $x_k$ where $1 \leq k \leq N$, two random augmentations $\tilde{x}_{2k}$ and $\tilde{x}_{2k-1}$ are generated, which both have the same label as $x_k$.
    
    \item The encoder network $f(.)$ takes samples $\tilde{x}_i$  and produces their respective embedding vectors where $1 \leq i \leq 2N$.
    
    \item The embedding $f(\tilde{x}_i)$ is mapped to multiple projection spaces by the projection networks $g_h$ where $h \in \{1, \dots , H\}$ and $H$ is the total number of projection heads , resulting in $z_i^h=g_h(f(\tilde{x}_i))$.
    
    \item The supervised contrastive loss is used to pull positive sample pairs together and push negative sample pairs apart in the projection spaces. Positive pairs are defined independently for each projection head based on a similarity criterion. We will elaborate on this part more in the following.
\end{enumerate}

Each projection space is dedicated to a specific similarity notion, where positive pairs are drawn closer and negative pairs are pushed apart. The similarity criteria may be based on the input samples' labels. For instance, if a pair shares the same $l$-th label, it is considered a positive pair; otherwise, it is treated as negative. Alternatively, similarity can be determined by aggregating all labels, with pairs exhibiting substantial label overlap considered positive. In this work, we focus on these two similarity notions, but other criteria could be explored in future research.

For each projection space $h$ the corresponding loss function is defined as $L_h$ inspired from \cite{khosla2020supervised}:
\begin{equation}
\label{supcon-loss}
L_h = \sum_{i \in I} \frac{-1}{|P_h(i)|} \sum_{p\in P_h(i)} \log \frac{\exp\left(\frac{z_i^h \cdot z_p^h}{\tau_h}\right)}{\sum_{a \in A(i)} \exp\left(\frac{z_i^h \cdot z_a^h}{\tau_h}\right)}.
\end{equation}

For each augmented sample, $\tilde{x}_i$, the loss function aims to increase the relative similarity between $z_i^h$ and the positive samples associated with $\tilde{x}_i$, which is defined by the indices $P_h(i)$. The similarity is measured by the dot product between the two projection vectors $z_i^h\cdot z_p^h$. We denote by $A(i)$ the set of indices for all samples, excluding sample $i$, which has a cardinality of $2N-1$. The scalar temperature hyperparameter $\tau_h$ regulates the influence of hard negatives on the final representation as discussed by \citet{chen2020simple} and its influence is further analyzed in the ablation studies.

Our final loss function is defined as $L$:
\begin{equation}
\label{mlcl-loss}
L = \sum_{h=1}^H \alpha_{h} L_{\tau_h}^h,
\end{equation}
where $\sum_{h=1}^H\alpha_{h}=1$.

% The temperature $\tau_h$ plays a crucial role, as it determines the degree of separation of the samples in the projection space corresponding to the head $h$. 

We now describe each of the two tasks, multi-label classification and hierarchical classification, in more detail below.

\begin{figure}
\label{tsne-fig}
\subfloat[SupCon\label{fig:supcon_rep}]{%
  \includegraphics[height=3cm,width=.49\linewidth]{images/x_normal.png}%
}
\hfill
\subfloat[MLCL\label{fig:mulhead_rep}]{%
  \includegraphics[height=3cm,width=.49\linewidth]{images/x_multihead.png}%
}
\caption{The t-SNE visualization of samples from five different classes in the representation space. The SupCon representation indicates a lack of meaningful structure, as samples from class \textit{camel} are positioned far from \textit{chimpanzee}, \textit{kangaroo}, and \textit{cattle} compared to the \textit{bottle} class. In contrast, our proposed method, MLCL, groups animal classes more closely in the representation space.}
\label{fig:representations}
\end{figure}
\subsection{Hierarchical Classification}
\label{sec:single-label}
To demonstrate this, we use the CIFAR-100 dataset \citep{cifar100} as an example. CIFAR-100 provides a two-level class hierarchy, where each sample has two labels: one for the subclass and one for the superclass. Each superclass consists of five subclasses—such as the flower superclass, which includes orchids, poppies, roses, sunflowers, and tulips. In this paper, we focus on the 100-class classification task but utilize the available superclass information to enhance the quality of the learned representations. We use multiple projection heads to deal with class hierarchies, with each head corresponding to a hierarchy level. Taking the CIFAR-100 dataset as an example with two hierarchy levels, our framework utilizes two projection heads: one for the standard subclass labels and one for the superclasses. Figure \ref{fig:img-arch} shows the architecture with the two projection heads. 

During training, given an input batch, we apply the first augmentation module to create a duplicate batch. Both batches pass through an encoder to obtain normalized embeddings. In the training stage, these embeddings are given to the two projection heads. The first head pulls samples with the same subclass label together using the loss function defined in Equation \eqref{supcon-loss}, while the second head pulls the samples with the same superclass together. We set a higher temperature for the second head to achieve a more distinct representation for samples belonging to different superclasses. According to \citet{wang2021understanding}, increasing the temperature leads to greater separation among the distinct classes, while decreasing the temperature results in a more evenly distributed set of classes, making the model more tolerant to negative samples.
In the inference stage, these two heads are removed, leaving only a single representation that has integrated information from both similarity criteria. For evaluation, a linear classifier is trained on these frozen representations using a cross-entropy loss, which is also known as the linear evaluation protocol in representation learning~\citep{chen2020simple, khosla2020supervised}.

\begin{table*}[t]
\caption{Overview of evaluation datasets: 
CIFAR-100 and DeepFashion each have two hierarchical levels. However, classification within the subclass level is the primary focus, while the superclass serves as an auxiliary signal during training.}
\label{overview}
\vskip 0.15in
\hspace{-1cm} % Adjust the value as needed
\begin{center}
\begin{small}
\begin{sc}
\begin{tabular}{cccccccc}
\toprule
Dataset & Task  & \#Train & \#Test & \#Levels & \#Classes \\
\midrule
CIFAR-100    & image classification& 50K& 10K& 2 & 100 \\
DeepFashion  & image classification & 200K & 40K & 2 & 50 \\
TripAdvisor    & sentiment analysis& 10K & 2K & 7 & 3 \\
BeerAdvocate & sentiment analysis& 10K & 2K & 5 & 3 \\
\bottomrule
\end{tabular}
\end{sc}
\end{small}
\end{center}
\vskip 0.15in
\end{table*}
\subsection{Multi-label Classification}
\label{sec:multi-label}
Multi-label classification aims to predict $L$ labels for each sample $x_k$, where each sample has a multi-hot label vector $y_k\in \{0,1\}^L$ and $y_k^l$ denotes the one-hot representation of the $l$-th label. Similarly to the method described in Section~\ref{sec:single-label}, we assign a distinct projection head for each label and include an additional global projection head to capture broader semantic relationships which will be elaborated on later.
We demonstrate our method for multi-label classification using the TripAdvisor review dataset \citep{wang2010latent}. Each review has $L=7$ labels that indicate user ratings for different aspects such as service, location, etc. We employ $H=8$ total heads, where seven heads are dedicated to the seven different aspects, and the final eighth head (global head) is used to focus on global similarities between two hotel reviews that have a similar overall sentiment, combining all aspects together. This illustrates another application of projection heads, where they are used as regularizers, especially in scenarios with limited training samples. Figure \ref{fig:text-arch} shows the architecture of the model.

\paragraph{Global Projection Head}
For datasets with ordinal class numbers, we can define a global similarity perspective that combines the information from all classes. For instance, in the hotel review dataset, if two reviews have 4 or 5 ratings for all aspects, then they are generally similar, and we expect them to have a similar representation, even if the specific rating varies for different aspects, such as one review giving four to service and the other giving five. We use the Jaccard similarity metric, inspired by \citet{zaigrajewcontrastive} to focus on these global similarities. 

The Jaccard similarity measure $s_{i,j}$ is defined for two samples $i$ and $j$ as: 
$$s_{i, j} = \frac{\sum_{l=1}^{L} \min(y_{i}^l, y_{j}^l)}{\sum_{l=1}^{L} \max(y_{i}^l, y_{j}^l)}$$

We introduce a hyperparameter $t$ to define all samples $i$ and $j$ with $s_{i,j} > t$ as positive samples. The loss function defined in Equation \eqref{supcon-loss} is then adjusted to put more weight on samples with higher Jaccard similarity as proposed by \citet{zaigrajewcontrastive}: 
\begin{equation*}
L_{gl} = \sum_{i \in I} \frac{-1}{|P(i)|} \sum_{p\in P(i)} s_{i, p}\log \frac{\exp\left(\frac{z_i \cdot z_p}{\tau}\right)}{\sum_{a \in A(i)} \exp\left(\frac{z_i \cdot z_a}{\tau}\right)}.
\end{equation*}

The training procedure for multi-label classification closely follows the approach used for single-label classification in Section \ref{sec:single-label}. However, since our target datasets are textual, we follow the common practice of combining contrastive learning with the cross-entropy loss, as demonstrated by \citet{gunel2020supervised}. The remainder of the training process is consistent with Section~\ref{sec:single-label}.

The linear classifier and cross-entropy loss are trained simultaneously with the projection heads and the pretrained Bert encoder. The final loss $L_{\text{MLCL}}$ is  defined as a weighted average of the loss in each projection head plus the cross-entropy loss, denoted as 
\begin{equation}
\label{multi-label-loss}
L = \sum_{h=1}^{H} \alpha_{h} L^h_{\tau_h} + (1-\sum_{h=1}^{H}\alpha_h) L_{\text{ce}},
\end{equation}
where $L_{ce}$ denotes the cross-entropy loss used for linear classification.

\subsection{Temperature Analysis}
\label{temperature-analysis}
Temperature plays a crucial role in controlling the impact of hard negative samples during training. Hard negatives are particularly challenging to distinguish from positive samples, as they often have high similarity to the anchor in the representation space. The temperature parameter helps to adjust the sensitivity of the model to subtle differences, allowing it to focus more effectively on more difficult distinctions. As temperature decreases, the model concentrates more on a smaller set of the closest hard negative samples, applying a stronger penalty to these examples. In contrast, when the temperature is large, the model distributes the penalty more evenly across all negative samples. To demonstrate this, we analyze two extreme cases of $\tau \to 0^+$  and $\tau \to +\infty$, inspired by \cite{wang2021understanding}. Based on Equation \eqref{supcon-loss}, the loss function for a given sample $i$ in the projection space $h$ is denoted as $L_h^i$:
\begin{equation*}  
L^i_h = \frac{-1}{|P_h(i)|} \sum_{p\in P_h(i)} \log \frac{\exp\left(\frac{z_i^h \cdot z_p^h}{\tau_h}\right)}{\sum_{a \in A(i)} \exp\left(\frac{z_i^h \cdot z_a^h}{\tau_h}\right)}.
\end{equation*}

For the sake of better readability, the index $h$ is omitted in the following derivation:
\begin{equation*}
\begin{split}
&\lim_{\tau \to 0^+} |P(i)| \times L^i = 
\\
&\sum_{p\in P(i)} \lim_{\tau \to 0^+} \Bigg[-\log \frac{\exp(z_i \cdot z_p/\tau)}{\sum_{a \in A(i)} \exp(z_i \cdot z_a/\tau)}\Bigg] = 
\\
&\sum_{p\in P(i)} \lim_{\tau \to 0^+} \Bigg[-\log \frac{\exp\big(z_i \cdot (z_p-z_i^{\max})/\tau\big)}{\sum_{a \in A(i)} \exp\big(z_i \cdot (z_a-z_i^{\max})/\tau\big)}\Bigg]
 \\
  &=\sum_{p\in P(i)} \lim_{\tau \to 0^+} \Bigg[ - \big(zi \cdot (z_p - z_i^{\max})/\tau\big) 
+ \\
 &  \hspace{34pt} \log \bigg(1 + \sum_{a \in A(i) \backslash p }\exp \big(z_i \cdot (z_a - z_i^{\max})/\tau\big)\bigg) \Bigg]=
  \\
  &\sum_{p\in P(i)} \lim_{\tau \to 0^+} \Bigg[
   - \big(z_i \cdot (z_p - z_i^{\max})/\tau\big)\Bigg] =
  \\
  & = \lim_{\tau \to 0^+} \frac{1}{\tau} \bigg[|P(i)| z_i \cdot z_i^{\max} - \sum_{p\in P(i)} z_i \cdot z_p\bigg],
\end{split}
\end{equation*}
where $z_i^{\max}=\arg\max_{a \in A} z_i \cdot z_a$.  We can infer that as \( \tau \to 0^+ \), the loss function focuses primarily on the positive samples and the negative sample with the maximum similarity to $z_i$, meaning only the hardest negative is considered. In contrast, as $\tau$ increases, negative samples contribute more uniformly to the loss function (for the derivation as $\tau \to +\infty$ please refer to the appendix). In hierarchical classification, the primary task is to classify at the lowest level (subclass), while higher-level labels in the hierarchy (superclass) are used to create a more structured feature space. To achieve this, we use a lower temperature for the projection space corresponding to the lowest level of the hierarchy to fully separate different subclasses, while using a higher temperature for superclasses to uniformly learn from all negative samples. Gradient analysis in the appendix demonstrates a similar effect for hard positives. At low temperatures, the loss function concentrates on the positive sample with the lowest similarity to the anchor, which is not ideal for higher levels of the hierarchy. Two positive samples at the superclass level may belong to different subclasses, requiring some degree of separation. Hence, focusing solely on hard positives at higher hierarchical levels can be counterproductive. This provides a second rationale for using a lower temperature for subclasses to ensure complete separation and a higher temperature for superclasses to promote more uniform learning across all samples. Further empirical analysis is provided in Section~\ref{experiments}.

\setlength{\tabcolsep}{6pt} % General space between cols (6pt standard)
\begin{table*}[t]
\caption{Top-1 classification accuracy on CIFAR-100 and DeepFashion test sets. We compare MLCL with ResNet50 trained with cross-entropy (CE), SimCLR~\citep{chen2020simple}, SupCon~\citep{khosla2020supervised}, Guided~\citep{landrieu2021leveraging}, and HiMulConE~\citep{zhang2022use} losses. The highest test accuracy is marked in bold.}
\label{deepfashion-cifar100-results}
\vskip 0.15in
\begin{center}
\begin{small}
\begin{sc}
\begin{tabular}{cccccccc}
\toprule
Dataset & SimCLR & CE & SupCon & Guided & HiMulConE & MLCL (ours) \\
\midrule
Cifar-100 & 70.70 & 75.30 & 76.50 & 76.40 & - & \textbf{77.70} \\
DeepFashion & 70.38 &  72.44 & 72.82 & 72.61 & 73.21 & \textbf{73.90} \\
\bottomrule
\end{tabular}
\end{sc}
\end{small}
\end{center}
\vskip -0.1in
\end{table*}

\begin{table*}[ht]
\caption{Top-1 classification accuracy on the test set of CIFAR-100 dataset using a subset of training samples. The highest test accuracy is marked in bold.}
\label{cifar100-results}
\vskip 0.15in
\begin{center}
\begin{small}
\begin{sc}
\begin{tabular}{rrrrr}
\toprule
\#Training Samples & SimCLR & Cross-Entropy & SupCon & MLCL (ours) \\
\midrule
% 50,000 & 70.70 &  75.30 & 76.50 & \textbf{77.51} \\
1,000 & 19.31 & 20.50 & 26.82 & \textbf{34.74} \\
5,000 & 34.51 &  21.56 & 46.15 & \textbf{56.02} \\
10,000 & 40.20 & 42.25 & 49.87 & \textbf{59.32} \\
20,000 & 54.73 &  59.04 & 65.21 & \textbf{69.36} \\
30,000 & 59.26 &  63.47 & 71.49 & \textbf{72.50} \\
40,000 & 63.21 &  67.10 & 74.80 & \textbf{75.60} \\
\bottomrule
\end{tabular}
\end{sc}
\end{small}
\end{center}
\vskip -0.1in
\end{table*}

\section{Experiments}
\label{experiments}
We evaluate the presented method using four benchmark datasets, two from the image domain and two text datasets. The first task is image classification on CIFAR-100 \citep{cifar100}, a widely used benchmark for image classification, and DeepFashion \citep{liuLQWTcvpr16DeepFashion}, a benchmark for hierarchical classification. We selected CIFAR-100 and DeepFashion because they offer class hierarchies. The second application focuses on multi-label classification using two text datasets for aspect-based sentiment analysis: TripAdvisor \citep{wang2010latent} and BeerAdvocate \citep{mcauley2012learning}.

\subsection{Datasets}
CIFAR-100 comprises 50,000 training images and 10,000 testing images, categorized into 100 classes, which are further grouped into 20 superclasses. The DeepFashion \citep{liuLQWTcvpr16DeepFashion} dataset, a large-scale clothing dataset, contains 200,000 training images and 40,000 testing images, with labels spanning 50 categories and organized into three superclasses. We intentionally chose these datasets because the widely used ImageNet \citep{deng2009imagenet} lacks groundtruth superclasses. This limitation makes ImageNet less appropriate for our multi-level setting, as it would reduce the problem to the standard supervised contrastive approach (SupCon). For multi-label classification, we conduct experiments on two textual datasets, TripAdvisor and BeerAdvocate. TripAdvisor is a hotel review dataset, with each review having seven different ratings based on various aspects (value, room, location, cleanliness, service, overall). We map the ratings to sentiments by changing ratings four and five to positive, three to neutral, and the others to negative. Similarly, BeerAdvocate consists of five aspects (appearance, aroma, palate, taste, overall), each associated with three sentiments, mirroring the structure of the other dataset. The summary of the datasets is given in Table \ref{overview}.
\setlength{\tabcolsep}{2pt} % General space between cols (6pt standard)
\begin{table*}[ht]
\caption{Top-1 classification accuracy on TripAdvisor and BeerAdvocate datasets. A direct comparison with SupCon \citep{khosla2020supervised} is not feasible due to its single-label classification nature. Instead, we compare our results with fine-tuned BERT using cross-entropy (CE) loss and our proposed MLCL loss (without and with the global projection head). The highest accuracy is highlighted in bold.}
\label{multi-label-results}
\vskip 0.15in
\begin{center}
\begin{small}
\begin{sc}
\begin{tabular}{ccccc}
\toprule
Dataset & \#Samples & CE & MLCL w/o $h_{global}$ & MLCL (ours) \\
\midrule
\multirow{4}{*}{TripAdvisor} & 30  & 71.30 &  72.50 & \textbf{72.9} \\
 & 60 & 74.19 &  74.81 & \textbf{75.1} \\
 & 180  & 77.00 &  77.30 & \textbf{77.9} \\
 & 360 & 78.10 &  78.44 & \textbf{79.0} \\
\midrule
\multirow{4}{*}{BeerAdvocate} & 30 & 65.45 & 65.95 & \textbf{66.90} \\
 & 60 & 66.37 &  66.71 & \textbf{67.40} \\
 & 180 & 68.70 &  69.25 & \textbf{69.61} \\
 & 360 & 70.54 &  71.22 & \textbf{71.81} \\
\bottomrule
\end{tabular}
\end{sc}
\end{small}
\end{center}
\vskip -0.1in
\end{table*}
\subsection{Implementation Details}
\paragraph{Hierarchical Classification} We conduct experiments on the CIFAR-100 and DeepFashion datasets with $H=2$ projection heads, assigning the first head to the subclass and the second to the superclass labels. The parameters for the first head are set as $\tau_1=0.1$ and $\alpha_1=0.5$, and for the second head as $\tau_2=0.5$, $\alpha_2=0.5$, following the notation in Equation \eqref{mlcl-loss}. The remaining hyperparameters are set as proposed in the SupCon paper \citep{khosla2020supervised} to ensure a fair comparison with this established baseline. Our model architecture includes a ResNet-50 encoder \citep{he2016deep} and two multi-layer perceptions (MLPs) with a single hidden layer serving as projection heads. The model is trained for 250 epochs, which is a quarter of the 1,000 epochs used by SupCon, with a batch size of 512. We use stochastic gradient descent (SGD) with momentum optimizer~\citep{ruder2016overview}. After training, the projection heads are discarded, and a linear classifier is trained on the frozen learned representation from the encoder's output to obtain the final accuracy on the subclass labels. This standard evaluation method, known as the linear evaluation protocol, is commonly used to assess the representations learned in contrastive learning studies \citep{khosla2020supervised, chen2020simple}. The procedure is depicted in Figure \ref{fig:img-arch}.

\paragraph{Multi-label Classification} We evaluate our approach on two text datasets for multi-label classification: TripAdvisor hotel reviews and BeerAdvocate reviews. The former consists of seven labels per review, and the latter has five labels. The task is to predict the sentiment for each aspect, a multi-label classification problem. We employ a pretrained BERT encoder \citep{devlin2018bert} with 512 embedding dimensions and fine-tune it by back-propagating the loss from multiple projection heads and a fully connected layer linked to the output of the encoder. The framework of our approach is illustrated in Figure \ref{fig:text-arch}. For the TripAdvisor and BeerAdvocate datasets, we use $H=8$ and $H=6$ projection heads, respectively. Each projection head corresponds to one label, with an additional global projection head for incorporating global similarities into the learned representation. The cross-entropy loss contribution is maintained at $0.7$ for both datasets. For the TripAdvisor dataset, we set $\alpha_i=0.03$ for the first seven heads and $\alpha_8=0.1$ for the final head. For the BeerAdvocate dataset, we apply $\alpha_i=0.04$ for the first five heads and $\alpha_6=0.1$ for the global head. The model is fine-tuned for 100 epochs using the Adam optimizer \citep{kingma2014adam} with a learning rate of $1e-5$ and a batch size of 16. We found that larger batch sizes did not significantly improve performance.


\subsection{Results}

\paragraph{Hierarchical Classification} 
Our method demonstrates a marginal improvement of 1\% over SupCon when trained on the complete CIFAR-100 and DeepFashion training datasets, as presented in Table \ref{deepfashion-cifar100-results}. Notably, this performance is achieved significantly faster, requiring only 250 training epochs compared to SupCon's 1,000 epochs. Furthermore, our approach exhibits superior performance with smaller training sets, achieving an enhancement of 9\% to 10\% for dataset sizes of 10K and 5K samples, as detailed in Table \ref{cifar100-results}. Additionally, our method surpasses the performance of models trained exclusively with cross-entropy loss by a substantial margin.

\paragraph{Multi-label Classification} Our results on the TripAdvisor and BeerAdvocate datasets are reported in Table~\ref{multi-label-results}. The numbers represent the average accuracies across ten different seeds. Using a pretrained BERT encoder, we focus on scenarios with limited training samples ranging from 30 to 360 while maintaining a test dataset size of 2K for all experiments. Our findings indicate that our approach enhances the quality of the learned representation, even with a robust pretrained encoder. The positive impact of the global projection head is also evident when comparing the third and fourth columns in Table~\ref{multi-label-results}.

% \setlength{\tabcolsep}{9pt} % General space between cols (6pt standard)

\subsection{Ablation Study}
\label{ablation}
The feature spaces of our approach and SupCon are visualized in Figure \ref{fig:representations} using t-SNE \citep{van2008visualizing}. The second projection head helps maintain the proximity of representations for samples within the same superclass. The class \textit{camel} is expected to be closer to \textit{cattle}, \textit{chimpanzee}, and \textit{kangaroo}, while remaining distant from \textit{bottle}, which belongs to a different superclass. Furthermore, Figure \ref{fig:temp-analysis} illustrates the impact of the superclass projection head's temperature on the final accuracy using a subset of training samples. Both excessively high and low temperatures negatively affect performance, while $\tau = 0.5$ yields the best results. The subclass temperature is fixed at $0.1$. For additional ablation studies on performance under label noise and transfer learning, please refer to the appendix.

\begin{figure}[ht] 
  \centering
  \includegraphics[width=\linewidth]{images/x_10000.png}
  \caption{Accuracy of MLCL as a function of the superclass projection head temperature. Increasing the temperature to 0.5 reduces the network's focus on hard negatives, leading to improved accuracy.}
  \label{fig:temp-analysis}
\end{figure}

\section{Conclusion}
\label{conclusion}

We introduced a novel supervised contrastive learning method within a unified framework called multi-level contrastive learning, which generalizes to tasks such as hierarchical and multi-label classification. Our approach incorporates multiple projection heads into the encoder network to learn representations at different levels of the hierarchy or for each label. This allows the method to capture label-specific similarities, exploit class hierarchies, and act as a regularizer to prevent overfitting. Although the parameter size for MLCL is equivalent to that of SupCon, it converges more rapidly and learns more effective representations, especially when training samples are limited. We have analyzed the impact of temperature on each projection head and showed that a higher superclass temperature improves the representations by allowing the model to focus less on hard negative samples. As shown in Section \ref{experiments}, this approach improves representation quality across various settings and datasets. Notably, the approach integrates seamlessly with existing contrastive learning methods without relying on any specific network architecture, making it flexible and applicable to a wide range of downstream tasks. Future work could explore the interpretability and explainability of the learned representations.

% \section{GENERAL FORMATTING INSTRUCTIONS}

% Submissions are limited to \textbf{8 pages} excluding references. 
% There will be an additional page for camera-ready versions of the accepted papers.

% Papers are in 2 columns with the overall line width of 6.75~inches (41~picas).
% Each column is 3.25~inches wide (19.5~picas).  The space
% between the columns is .25~inches wide (1.5~picas).  The left margin is 0.88~inches (5.28~picas).
% Use 10~point type with a vertical spacing of
% 11~points. Please use US Letter size paper instead of A4.

% Paper title is 16~point, caps/lc, bold, centered between 2~horizontal rules.
% Top rule is 4~points thick and bottom rule is 1~point thick.
% Allow 1/4~inch space above and below title to rules.

% Author descriptions are center-justified, initial caps.  The lead
% author is to be listed first (left-most), and the Co-authors are set
% to follow.  If up to three authors, use a single row of author
% descriptions, each one center-justified, and all set side by side;
% with more authors or unusually long names or institutions, use more
% rows.

% Use one-half line space between paragraphs, with no indent.

% \section{FIRST LEVEL HEADINGS}

% First level headings are all caps, flush left, bold, and in point size
% 12. Use one line space before the first level heading and one-half line space
% after the first level heading.

% \subsection{Second Level Heading}

% Second level headings are initial caps, flush left, bold, and in point
% size 10. Use one line space before the second level heading and one-half line
% space after the second level heading.

% \subsubsection{Third Level Heading}

% Third level headings are flush left, initial caps, bold, and in point
% size 10. Use one line space before the third level heading and one-half line
% space after the third level heading.

% \paragraph{Fourth Level Heading}

% Fourth level headings must be flush left, initial caps, bold, and
% Roman type.  Use one line space before the fourth level heading, and
% place the section text immediately after the heading with no line
% break, but an 11 point horizontal space.

% %%%
% \subsection{Citations, Figure, References}


% \subsubsection{Citations in Text}

% Citations within the text should include the author's last name and
% year, e.g., (Cheesman, 1985). 
% %Apart from including the author's last name and year, citations can follow any style, as long as the style is consistent throughout the paper.  
% Be sure that the sentence reads
% correctly if the citation is deleted: e.g., instead of ``As described
% by (Cheesman, 1985), we first frobulate the widgets,'' write ``As
% described by Cheesman (1985), we first frobulate the widgets.''


% The references listed at the end of the paper can follow any style as long as it is used consistently.

% %Be sure to avoid
% %accidentally disclosing author identities through citations.

% \subsubsection{Footnotes}

% Indicate footnotes with a number\footnote{Sample of the first
%   footnote.} in the text. Use 8 point type for footnotes. Place the
% footnotes at the bottom of the column in which their markers appear,
% continuing to the next column if required. Precede the footnote
% section of a column with a 0.5 point horizontal rule 1~inch (6~picas)
% long.\footnote{Sample of the second footnote.}

% \subsubsection{Figures}

% All artwork must be centered, neat, clean, and legible.  All lines
% should be very dark for purposes of reproduction, and art work should
% not be hand-drawn.  Figures may appear at the top of a column, at the
% top of a page spanning multiple columns, inline within a column, or
% with text wrapped around them, but the figure number and caption
% always appear immediately below the figure.  Leave 2 line spaces
% between the figure and the caption. The figure caption is initial caps
% and each figure should be numbered consecutively.

% Make sure that the figure caption does not get separated from the
% figure. Leave extra white space at the bottom of the page rather than
% splitting the figure and figure caption.
% \begin{figure}[h]
% \vspace{.3in}
% \centerline{\fbox{This figure intentionally left non-blank}}
% \vspace{.3in}
% \caption{Sample Figure Caption}
% \end{figure}

% \subsubsection{Tables}

% All tables must be centered, neat, clean, and legible. Do not use hand-drawn tables.
% Table number and title always appear above the table.
% See Table~\ref{sample-table}.

% Use one line space before the table title, one line space after the table title,
% and one line space after the table. The table title must be
% initial caps and each table numbered consecutively.

% \begin{table}[h]
% \caption{Sample Table Title} \label{sample-table}
% \begin{center}
% \begin{tabular}{ll}
% \textbf{PART}  &\textbf{DESCRIPTION} \\
% \hline \\
% Dendrite         &Input terminal \\
% Axon             &Output terminal \\
% Soma             &Cell body (contains cell nucleus) \\
% \end{tabular}
% \end{center}
% \end{table}

% \section{SUPPLEMENTARY MATERIAL}

% If you need to include additional appendices during submission, you can include them in the supplementary material file.
% You can submit a single file of additional supplementary material which may be either a pdf file (such as proof details) or a zip file for other formats/more files (such as code or videos). 
% Note that reviewers are under no obligation to examine your supplementary material. 
% If you have only one supplementary pdf file, please upload it as is; otherwise gather everything to the single zip file.

% You must use \texttt{aistats2025.sty} as a style file for your supplementary pdf file and follow the same formatting instructions as in the main paper. 
% The only difference is that it must be in a \emph{single-column} format.
% You can use \texttt{supplement.tex} in our starter pack as a starting point.
% Alternatively, you may append the supplementary content to the main paper and split the final PDF into two separate files.

% \section{SUBMISSION INSTRUCTIONS}

% To submit your paper to AISTATS 2025, please follow these instructions.

% \begin{enumerate}
%     \item Download \texttt{aistats2025.sty}, \texttt{fancyhdr.sty}, and \texttt{sample\_paper.tex} provided in our starter pack. 
%     Please, do not modify the style files as this might result in a formatting violation.
    
%     \item Use \texttt{sample\_paper.tex} as a starting point.
%     \item Begin your document with
%     \begin{flushleft}
%     \texttt{\textbackslash documentclass[twoside]\{article\}}\\
%     \texttt{\textbackslash usepackage\{aistats2025\}}
%     \end{flushleft}
%     The \texttt{twoside} option for the class article allows the
%     package \texttt{fancyhdr.sty} to include headings for even and odd
%     numbered pages.
%     \item When you are ready to submit the manuscript, compile the latex file to obtain the pdf file.
%     \item Check that the content of your submission, \emph{excluding} references and reproducibility checklist, is limited to \textbf{8 pages}. The number of pages containing references and reproducibility checklist only is not limited.
%     \item Upload the PDF file along with other supplementary material files to the CMT website.
% \end{enumerate}

% \subsection{Camera-ready Papers}

% %For the camera-ready paper, if you are using \LaTeX, please make sure
% %that you follow these instructions.  
% % (If you are not using \LaTeX,
% %please make sure to achieve the same effect using your chosen
% %typesetting package.)

% If your papers are accepted, you will need to submit the camera-ready version. Please make sure that you follow these instructions:
% \begin{enumerate}
%     %\item Download \texttt{fancyhdr.sty} -- the
%     %\texttt{aistats2022.sty} file will make use of it.
%     \item Change the beginning of your document to
%     \begin{flushleft}
%     \texttt{\textbackslash documentclass[twoside]\{article\}}\\
%     \texttt{\textbackslash usepackage[accepted]\{aistats2025\}}
%     \end{flushleft}
%     The option \texttt{accepted} for the package
%     \texttt{aistats2025.sty} will write a copyright notice at the end of
%     the first column of the first page. This option will also print
%     headings for the paper.  For the \emph{even} pages, the title of
%     the paper will be used as heading and for \emph{odd} pages the
%     author names will be used as heading.  If the title of the paper
%     is too long or the number of authors is too large, the style will
%     print a warning message as heading. If this happens additional
%     commands can be used to place as headings shorter versions of the
%     title and the author names. This is explained in the next point.
%     \item  If you get warning messages as described above, then
%     immediately after $\texttt{\textbackslash
%     begin\{document\}}$, write
%     \begin{flushleft}
%     \texttt{\textbackslash runningtitle\{Provide here an alternative
%     shorter version of the title of your paper\}}\\
%     \texttt{\textbackslash runningauthor\{Provide here the surnames of
%     the authors of your paper, all separated by commas\}}
%     \end{flushleft}
%     Note that the text that appears as argument in \texttt{\textbackslash
%       runningtitle} will be printed as a heading in the \emph{even}
%     pages. The text that appears as argument in \texttt{\textbackslash
%       runningauthor} will be printed as a heading in the \emph{odd}
%     pages.  If even the author surnames do not fit, it is acceptable
%     to give a subset of author names followed by ``et al.''

%     %\item Use the file sample\_paper.tex as an example.

%     \item The camera-ready versions of the accepted papers are \textbf{9
%       pages}, plus any additional pages needed for references and reproducibility checklist.

%     \item If you need to include additional appendices,
%       you can include them in the supplementary
%       material file.

%     \item Please, do not change the layout given by the above
%       instructions and by the style file.

% \end{enumerate}

% \subsubsection*{Acknowledgements}
% All acknowledgments go at the end of the paper, including thanks to reviewers who gave useful comments, to colleagues who contributed to the ideas, and to funding agencies and corporate sponsors that provided financial support. 
% To preserve the anonymity, please include acknowledgments \emph{only} in the camera-ready papers.


% \subsubsection*{References}

% References follow the acknowledgements.  Use an unnumbered third level
% heading for the references section.  Please use the same font
% size for references as for the body of the paper---remember that
% references do not count against your page length total.

\bibliography{sample_paper}

\clearpage
\appendix
\section*{Appendix}
\section{Discussion: Scope and Ethics}
\label{appendix:scope}
In this work, we evaluate our method on six core scene-aware tasks: existence, count, position, color, scene, and HOI reasoning. We select these tasks as they represent core aspects of multimodal understanding which are essential for many applications. Meanwhile, we do not extend our evaluation to more complex reasoning tasks, such as numerical calculations or code generation, because SOTA diffusion models like SDXL are not yet capable of handling these tasks effectively. Fine-tuning alone cannot overcome the fundamental limitations of these models in generating images that require symbolic logic or complex reasoning. Additionally, we avoid tasks with ethical concerns, such as generating images of specific individuals (e.g., for celebrity recognition task), to mitigate risks related to privacy and misuse. Our goal was to ensure that our approach focuses on technically feasible and responsible AI applications. Expanding to other tasks will require significant advancements in diffusion model capabilities and careful consideration of ethical implications.

\section{Limitations and Future Work}
While our Multimodal Context Evaluator proves effective in enhancing the fidelity of generated images and maintaining diversity, \method is built using pre-trained diffusion models such as SDXL and MLLMs like LLaVA, it inherently shares the limitations of these foundation models. \method still faces challenges with complex reasoning tasks such as numerical calculations or code generation due to the symbolic logic limitations inherent to SDXL. Additionally, during inference, the MLLM context descriptor occasionally generates incorrect information or ambiguous descriptions initially, which can lead to lower fidelity in the generated images. Figure~\ref{fig:failure} further illustrates these observations.

\method currently focuses on single attributes like count, position, and color as part of the multimodal context. This is because this task alone poses significant challenges to existing methods, which \method effectively addresses. A potential direction for future work is to broaden the applicability of \method to synthesize images with multiple scene attributes in the multimodal context as part of compositional reasoning tasks.


\begin{figure}[!h]
    \centering
    \includegraphics[width=\linewidth]{figures/failures.pdf}
    \caption{Failure cases of \method. (a) Our method fails due to the symbolic logic limitation of existing pre-trained SDXL. (b) Initially incorrect descriptions generated by MLLMs lead to low fidelity of generated images. (c) Context description generated by MLLMs is ambiguous and does not directly relate to the text guidance, the spoon can be both inside or outside the bowl.}
    \label{fig:failure}
\end{figure}

\section{Prompt Templates}
\label{appendix:prompts}
Figure~\ref{fig:prompt_templates}~(a-c) showcases the prompt templates used by \method to fine-tune diffusion models specifically on each task including VQA, HOI Reasoning, and Object-Centric benchmarks. It's worth noting that we designed the prompt such that it provides detailed instruction to MLLMs on which scene attributes to focus. We also evaluate the effectiveness of our designed prompt templates by fine-tuning \method with a generic prompt as illustrated in Figure~\ref{fig:prompt_templates}~(d). Table~\ref{table:prommpt} indicates that without using our designed prompt template, the MLLM is not properly instructed to generate specific context description thus leading to reduced performance after fine-tuning on MME tasks. We believe that when using a generic prompt, MLLM is not able to receive sufficient grounding about the multimodal context leading to information loss on key scene attributes.


\begin{table}[!h]
\centering
\footnotesize
\caption{Effectiveness of the prompt template on fine-tuning \method on MME Perception.}
\resizebox{1\linewidth}{!}{
\begin{tabular}{clcccccccccc}
\toprule
 \textbf{MLLM} & \multirow{2}{*}{\textbf{\method}} & \multicolumn{2}{c}{\textbf{Existence}} & \multicolumn{2}{c}{\textbf{Count}} & \multicolumn{2}{c}{\textbf{Position}} & \multicolumn{2}{c}{\textbf{Color}} & \multicolumn{2}{c}{\textbf{Scene}} \\
 \textbf{Name} & & ACC & ACC+ & ACC & ACC+ & ACC & ACC+ & ACC & ACC+ & ACC & ACC+ \\
 \midrule
 \multirow{3}{*}{\makecell{\textbf{LLaVA }  \\ \textbf{v1.6 7B} \\ \citep{liu2024improved}}}
 &w/ prompt template & \textbf{96.67}  & \textbf{93.33}  & \textbf{83.33}  & \textbf{70.00}  & \textbf{81.67}  & \textbf{66.67} & \textbf{95.00}  & \textbf{93.33}  & \textbf{87.75} & \textbf{74.00} \\
 \cmidrule{2-12}
 & \multirow{2}{*}{w/ generic prompt} & 91.67 & 83.33 & 75.00 & 56.67 & \textbf{81.67} & 63.33 & 91.67 & 83.33 & 87.25 & 73.00 \\
 & & {\scriptsize \color{red}\textbf{$\downarrow$ 5.00}} & {\scriptsize \color{red}\textbf{$\downarrow$ 10.00}} & {\scriptsize \color{red}\textbf{$\downarrow$ 8.33}} & {\scriptsize \color{red}\textbf{$\downarrow$ 13.33}} & - &  {\scriptsize \color{red}\textbf{$\downarrow$ 3.34}} & {\scriptsize \color{red}\textbf{$\downarrow$ 3.33}} & {\scriptsize \color{red}\textbf{$\downarrow$ 10.00}} & {\scriptsize \color{red}\textbf{$\downarrow$ 0.50}} & {\scriptsize \color{red}\textbf{$\downarrow$ 1.00}}\\
 \midrule
 \multirow{3}{*}{\makecell{\textbf{InternVL }  \\ \textbf{2.0 8B}\\ \citep{chen2024internvl}}} 
 &w/ prompt template & \textbf{98.33}  & \textbf{96.67} & \textbf{86.67} & \textbf{73.33}  & \textbf{78.33}  & \textbf{63.33}  & \textbf{98.33}  & \textbf{96.67}  & \textbf{86.25} & \textbf{71.00} \\
 \cmidrule{2-12}
 & \multirow{2}{*}{w/ generic prompt} & 91.67 & 83.33 & 80.00 & 60.00 & 71.67 & 50.00 & 91.67 & 83.33 & 84.50 & 69.00 \\
 & & {\scriptsize \color{red}\textbf{$\downarrow$ 6.66}} &  {\scriptsize \color{red}\textbf{$\downarrow$ 13.34}} & {\scriptsize \color{red}\textbf{$\downarrow$ 6.67}} & {\scriptsize \color{red}\textbf{$\downarrow$ 13.33}} & {\scriptsize \color{red}\textbf{$\downarrow$ 6.66}} & {\scriptsize \color{red}\textbf{$\downarrow$ 13.33}} & {\scriptsize \color{red}\textbf{$\downarrow$ 6.66}} & {\scriptsize \color{red}\textbf{$\downarrow$ 13.34}} & {\scriptsize \color{red}\textbf{$\downarrow$ 1.75}} & {\scriptsize \color{red}\textbf{$\downarrow$ 2.00}}\\
\bottomrule
\end{tabular}
}
\label{table:prommpt}
\end{table}

\begin{figure}[!h]
    \centering
    \includegraphics[width=\linewidth]{figures/prompt_template.pdf}
    \caption{Prompt templates (a-c) used by \method to fine-tune the diffusion model on each task including VQA, HOI Reasoning, and Object Centric benchmarks. The generic prompt (d) is also included to evaluate the effectiveness of prompt template.}
    \label{fig:prompt_templates}
\end{figure}
\section{Inference Pipeline}
\label{appendix:inference}
In the inference pipeline of \method (Figure~\ref{fig:inference}), the text guidance $\mathbf{g}$ includes only the question corresponding to the reference image $\mathbf{x}$. The answer is excluded for fair evaluation. Moreover, we remove Multimodal Context Evaluator, and the generated image $\hat{\mathbf{x}}$ is the final output.
\begin{figure}[!h]
    \centering
    \includegraphics[width=\linewidth]{figures/inference.pdf}
    \caption{Inference pipeline of \method}
    \label{fig:inference}
\end{figure}

\begin{figure}[!h]
    \centering
    \includegraphics[width=\linewidth]{figures/diversity_compact_caption.pdf}
    \vspace{-5mm}
    \caption{Examples of context description from MLLM in the inference pipeline where answers are not included in text guidance.}
    \label{fig:diversity_compact_caption}
\end{figure}



\section{Ablation Study on BLIP-2 QFormer}
Our design choice to leverage BLIP-2 QFormer in \method as the multimodal context evaluator facilitates the formulation of our novel Global Semantic and Fine-grained Consistency Rewards. These rewards enable \method to be effective across all tasks as seen in Table~\ref{table:clip}. On replace with a less powerful multimodal context encoder such as CLIP ViT-G/14, we can only implement the global semantic reward as the cosine similarity between the text features and generated image features. As a result, while the setting can maintain performance on coarse-level tasks such as Scene and Existence, there is a noticeable decline on fine-grained tasks like Count and Position. This demonstrates the effectiveness of our design choices in \method and shows that using less powerful alternatives, without the ability to provide both global and fine-grained alignment, affects the fidelity of generated images.

\begin{figure*}[t]
\centering
\includegraphics[width=15.5cm]{figures/clip_zeroshot.png}\\
\caption{CLIP a) training and b) zero-shot inference framework}
\label{fig:clip} 
\end{figure*}


\section{Additional Evaluation on MME Artwork}

To explore the method's ability to work on tasks involving more nuanced or abstract text guidance beyond factual scene attributes, we evaluate \method on an additional task of MME Artwork. This task focuses on image style attributes that are more nuanced/abstract such as the following question-answer pair -- Question: ``Does this artwork exist in the form of mosaic?'', Answer: ``No''.

Table~\ref{table:artwork_reasoning} summarizes the evaluation. We can observe that \method outperforms all existing methods on both ACC and ACC+, implying its higher effectiveness in generating images with high fidelity (in this case, image style preservation) compared to existing methods. This provides evidence that \method can generalize to tasks involving abstract/nuanced attributes such as image style. Figure~\ref{fig:artwork} further shows qualitative comparison between image generation methods on the MME Artwork task.

\begin{table}[h]
\centering
\caption{Comparison on Artwork benchmark and Visual Reasoning task. \method outperforms SOTA image generation and augmentation techniques.}
\resizebox{\linewidth}{!}{
\begin{tabular}{@{}l@{ }ccccccc@{}}
\toprule
\textbf{Method} & \textbf{Real only} & \textbf{RandAugment} &  \textbf{Image Variation} & \textbf{Image Translation} & \textbf{Textual Inversion} & \textbf{I2T2I SDXL} & \textbf{\method} \\
\midrule
\textbf{Artwork ACC} & 69.50 & 69.25 & 69.00 & 67.00 & 66.75 & 68.00 & \textbf{70.25} \\
\textbf{Artwork ACC+} & 41.00 & 41.00 & 40.00 & 38.00 & 37.50 & 38.00 & \textbf{41.50} \\
\midrule
\textbf{Reasoning ACC} & 69.29 & 67.86 & 69.29 & 69.29 & 67.14 & 72.14 & \textbf{72.86} \\
\textbf{Reasoning ACC+} & 42.86 & 40.00 & 41.40 & 40.00 & 37.14 & 47.14 & \textbf{48.57} \\

\bottomrule
\end{tabular}
}
\label{table:artwork_reasoning}
\end{table}


\begin{figure}[!h]
    \centering
    \includegraphics[width=\linewidth]{figures/artwork.pdf}
    \caption{Qualitative comparison on the Artwork task between image generation method. \method can preserve both diversity and fidelity of the reference image in a more abstract domain.}
    \label{fig:artwork}
\end{figure}


\section{Additional Evaluation on MME Commonsense Reasoning}
We have additionally performed our evaluation to more complex tasks such as Visual Reasoning using the MME Commonsense Reasoning benchmark. Results in Table~\ref{table:artwork_reasoning} highlight \method's ability to generalize effectively across diverse domains and complex reasoning tasks, demonstrating its broader applicability. Figure~\ref{fig:reasoning} further shows qualitative comparison between image generation methods on the MME Commonsense Reasoning task.

\begin{figure}[!h]
    \centering
    \includegraphics[width=\linewidth]{figures/reasoning.pdf}
    \caption{Qualitative comparison on the Commonsense Reasoning task between image generation method. \method can preserve both diversity and fidelity of the reference image in a more abstract domain.}
    \label{fig:reasoning}
\end{figure}
\section{FID Scores}
% \textcolor{blue}{We compute FID scores of traditional augmentation and image generation methods. Table~\ref{table:fid} shows that the data distribution of generated images by RandAugment and Image Translation are closer to the real distribution as these methods only change images minimally. We also want to emphasize that even though the FID metric evaluates the quality of generated images, it can not measure the diversity of generated images. \method with rewards fine-tuning achieves a competitive score. As we showed in the diversity analysis in Table~\ref{table:diversity} in the main paper, \method performs significantly better than these ``minimal change" methods while still achieving a competitive FID score. We believe this is a worldwide trade-off.}

We compute FID scores for \method and the different baselines (traditional augmentation and image generation methods) and tabulate the numbers in Table~\ref{table:fid}. FID is a valuable metric for assessing the quality of generated images and how closely the distribution of generated images matches the real distribution. However, \textit{FID does not account for the diversity among the generated images}, which is a critical aspect of the task our work targets~(i.e., how can we generate high fidelity images, preserving certain scene attributes, while still maintaining high diversity?). We also illustrate the shortcomings of FID for the task in Figure~\ref{fig:fid_diversity} where we compare generated images across methods. We observe that RandAugment and Image Translation achieve lower FID scores than \method~(w/ finetuning) because they compromise on diversity by only minimally changing the input image, allowing their generated image distribution to be much closer to the real distribution. While \method has a higher FID score than RandAugment and Image Translation, Figure~\ref{fig:fid_diversity} shows that it is able to preserve the scene attribute w.r.t.~multimodal context while generating an image that is significantly different from than original input image. Therefore, it accomplishes the targeted task more effectively, with both high fidelity and high diversity.

\begin{table}[h]
\centering
\caption{FID scores of traditional augmentation and image generation methods. Lower is better.}
\resizebox{\linewidth}{!}{
\begin{tabular}{@{}l@{ }ccccccc@{}}
\toprule
\multirow{2}{*}{\textbf{Method}} & \multirow{2}{*}{\textbf{RandAugment}} & \multirow{2}{*}{\textbf{I2T2I SDXL}} & \multirow{2}{*}{\textbf{Image Variation}} & \multirow{2}{*}{\textbf{Image Translation}} & \multirow{2}{*}{\textbf{Textual Inversion}} & \multicolumn{2}{c}{\textbf{\method}} \\
& & & & & & \ding{55} fine-tuning & \ding{51} fine-tuning\\
\midrule
\textbf{FID score $\downarrow$} & \textbf{15.93} & 18.35 & 17.66 & 16.29 & 20.84 & 17.78 & 16.55 \\
\bottomrule
\end{tabular}
}
\label{table:fid}
\end{table}

\begin{figure}[!h]
    \centering
    \includegraphics[width=\linewidth]{figures/fid_diversity.pdf}
    \caption{While RandAugment and Image Translation achieve lower FID scores, \method balances fidelity and diversity effectively.}
    \label{fig:fid_diversity}
\end{figure}

\section{User Study}
% \textcolor{blue}{We created a survey form with 50 questions (10 questions per MME task). In each survey question, users were shown: a reference image, a related question, and two generated images from different methods (I2T2I SDXL vs. \method). Users are asked to select the generated image(s) that preserve the attribute referred to by the question in relation to reference image. We collected form responses from 70 people. Table~\ref{table:user_study} shows that \method significantly outperforms I2T2I SDXL in terms of fidelity across all tasks on MME benchmark. We have some examples of survey questions in Figure~\ref{fig:user_study_examples}.}

We conduct a user study where we create a survey form with 50 questions (10 questions per MME Perception task). In each survey question, we show users a reference image, a related question, and a generated image each from two different methods (baseline I2T2I SDXL vs \method). We ask users to select the generated images(s) (either one or both or neither of them) that preserve the attribute referred to by the question in relation to the reference image. If an image is selected, it denotes high fidelity in generation. We collect form responses from 70 people for this study. We compute the percentage of total generated images for each method that were selected by the users as a measure of fidelity. Table~\ref{table:user_study} summarizes the results and shows that \method significantly outperforms I2T2I SDXL in terms of fidelity across all tasks on the MME Perception benchmark. We have some examples of survey questions in Figure~\ref{fig:user_study_examples}.

\begin{figure}[htp]
  \centering
   \includegraphics[width=\columnwidth]{Assets/userstudy_grid.pdf}
   
   \caption{\textbf{User study results.} Users preference percentage of our method compared to other methods in terms of text alignment, visual quality, and overall preference.
   }
   \label{fig:user_study}
\end{figure}
\begin{figure}[!h]
    \centering
    \includegraphics[width=\linewidth]{figures/user_study_examples.pdf}
    \caption{Some examples of our survey questions to evaluate the fidelity of generated images from I2T2I SDXL and \method.}
    \label{fig:user_study_examples}
\end{figure}
\section{Training Performance on Bongard HOI Dataset}
% \textcolor{blue}{We conducted an additional experiment by training a CNN baseline ResNet50 \citep{he2016deep} model on the Bongard-HOI training set with traditional augmentation and other image generation methods, using the same number of training iterations. As shown in Table~\ref{table:hoi_training}, \method consistently outperforms other methods across all test splits. However, as discussed in Subsection~\ref{sec:benchmark_formulation}, our primary focus on test-time evaluation ensures fair comparisons by avoiding variability in training behavior caused by differences in model architectures, data distributions, and training configurations.}

Following the existing method \citep{shu2022testtime}, we conduct an additional experiment by training a ResNet50 \citep{he2016deep} model on the Bongard-HOI \citep{jiang2022bongard} training set with traditional augmentation and Hummingbird generated images. We compare the performance with other image generation methods, using the same
number of training iterations. As shown in Table~\ref{table:hoi_training}, \method consistently outperforms all the baselines across all test splits. In the paper, as discussed in Section~\ref{sec:benchmark_formulation}, we focus primarily on test-time evaluation because it eliminates the variability introduced by model training due to multiple external variables such as model architecture, data distribution, and training configurations, and allows for a fairer comparison where the evaluation setup remains fixed.

\begin{table}[!h]
\centering
\footnotesize
\caption{Comparison on Human-Object Interaction~(HOI) Reasoning by training a CNN-baseline ResNet50 with image augmentation and generation methods. \method outperforms SOTA methods on all $4$ test splits of Bongard-HOI dataset.}
\resizebox{0.8\linewidth}{!}{
\begin{tabular}{lccccc}
\toprule
\multirow{3}{*}{Method} & \multicolumn{4}{c}{Test Splits} & \multirow{3}{*}{Average} \\
\cmidrule{2-5}
 & seen act., & unseen act., & seen act., & unseen act., &  \\
 & seen obj. & seen obj. & unseen obj. & unseen obj. & \\
  % & seen act., seen obj. & unseen act., seen obj. & seen act., unseen obj. & unseen act., unseen obj. &  \\
 % &  &  &  & & \\
\midrule
CNN-baseline (ResNet50) & 50.03\xspace\xspace\xspace\xspace\xspace\xspace\xspace\xspace\xspace\xspace & 49.89\xspace\xspace\xspace\xspace\xspace\xspace\xspace\xspace\xspace\xspace & 49.77\xspace\xspace\xspace\xspace\xspace\xspace\xspace\xspace\xspace\xspace & 50.01\xspace\xspace\xspace\xspace\xspace\xspace\xspace\xspace\xspace\xspace & 49.92\xspace\xspace\xspace\xspace\xspace\xspace\xspace\xspace\xspace\xspace \\
RandAugment \citep{cubuk2020randaugment} & 51.07 {\scriptsize \color{ForestGreen}$\uparrow$ 1.04} & 51.14 {\scriptsize \color{ForestGreen}$\uparrow$ 1.25} & 51.79 {\scriptsize \color{ForestGreen}$\uparrow$ 2.02} & 51.73 {\scriptsize \color{ForestGreen}$\uparrow$ 1.72} & 51.43 {\scriptsize \color{ForestGreen}$\uparrow$ 1.51} \\
Image Variation \citep{xu2023versatile} & 41.78 {\scriptsize \color{red}$\downarrow$ 8.25} & 41.29 {\scriptsize \color{red}$\downarrow$ 8.60} & 41.15 {\scriptsize \color{red}$\downarrow$ 8.62} & 41.25 {\scriptsize \color{red}$\downarrow$ 8.76} & 41.37 {\scriptsize \color{red}$\downarrow$ 8.55} \\
Image Translation \citep{pan2023boomerang} & 46.60 {\scriptsize \color{red}$\downarrow$ 3.43} & 46.94 {\scriptsize \color{red}$\downarrow$ 2.95} & 46.38 {\scriptsize \color{red}$\downarrow$ 3.39} & 46.50 {\scriptsize \color{red}$\downarrow$ 3.51} & 46.61 {\scriptsize \color{red}$\downarrow$ 3.31} \\
Textual Inversion \citep{gal2022image} & \xspace37.67 {\scriptsize \color{red}$\downarrow$ 12.36} & \xspace37.52 {\scriptsize \color{red}$\downarrow$ 12.37} & \xspace38.12 {\scriptsize \color{red}$\downarrow$ 11.65} & \xspace38.06 {\scriptsize \color{red}$\downarrow$ 11.95} & \xspace37.84 {\scriptsize \color{red}$\downarrow$ 12.08} \\
I2T2I SDXL \citep{podell2023sdxl} & 51.92 {\scriptsize \color{ForestGreen}$\uparrow$ 1.89} & 52.18 {\scriptsize \color{ForestGreen}$\uparrow$ 2.29} & 52.25 {\scriptsize \color{ForestGreen}$\uparrow$ 2.48} & 52.15 {\scriptsize \color{ForestGreen}$\uparrow$ 2.14} & 52.13 {\scriptsize \color{ForestGreen}$\uparrow$ 2.21}\\
\textbf{\method} & \textbf{53.71 {\scriptsize \color{ForestGreen}$\uparrow$ 3.68}} & \textbf{53.55 {\scriptsize \color{ForestGreen}$\uparrow$ 3.66}} & \textbf{53.69 {\scriptsize \color{ForestGreen}$\uparrow$ 3.92}} & \textbf{53.41 {\scriptsize \color{ForestGreen}$\uparrow$ 3.40}} & \textbf{53.59 {\scriptsize \color{ForestGreen}$\uparrow$ 3.67}} \\
\bottomrule
\end{tabular}
}
\label{table:hoi_training}
\end{table}



\section{Random Seeds Selection Analysis}
We conduct an additional experiment, varying the number of random seeds from $10$ to $100$. The results are presented in the boxplot in Figure~\ref{fig:boxplot}, which shows the distribution of the mean L2 distances of generated image features from Hummingbird across different numbers of seeds.


The figure demonstrates that the difference in the distribution of the diversity (L2) scores across the different numbers of random seeds is statistically insignificant. So while it is helpful to increase the number of seeds for improved confidence, we observe that it stabilizes at 20 random seeds. This analysis suggests that using $20$ random seeds also suffices to capture the diversity of generated images without significantly affecting the robustness of the analysis.

% We conduct an additional experiment where we vary the number of seeds from 10 to 100. We present the results as a boxplot in Appendix K, Figure 15 which shows the distribution of the mean L2 distances of generated image features from Hummingbird across different numbers of seeds.

% The figure demonstrates that the difference in the distribution of the diversity (L2) scores across the different numbers of random seeds is statistically insignificant. So while it is helpful to increase the number of seeds for improved confidence, we observe that it stabilizes at 20 random seeds. This analysis suggests that using 20 random seeds also suffices to capture the diversity of generated images without significantly affecting the robustness of the analysis.

\begin{figure}[!h]
    \centering
    \includegraphics[width=0.8\linewidth]{figures/diversity_boxplot_rectangular.pdf}
    \caption{Diversity analysis across varying numbers of random seeds (10 to 100) using mean L2 distances of generated image features from \method. The box plot demonstrates consistent diversity scores as the number of seeds increases, indicating that performance stabilizes around 20 random seeds.}
    \label{fig:boxplot}
\end{figure}

\section{Further Explanation of Multimodal Context Evaluator}
The Global Semantic Reward, \(\mathcal{R}_\textrm{global}\), ensures alignment between the global semantic features of the generated image \(\mathbf{\hat{x}}\) and the textual context description \(\mathcal{C}\). This reward leverages cosine similarity to measure the directional alignment between two feature vectors, which can be interpreted as maximizing the mutual information \(I(\mathbf{\hat{x}}, \mathcal{C})\) between the generated image \(\mathbf{\hat{x}}\) and the context description \(\mathcal{C}\). Mutual information quantifies the dependency between the joint distribution \(p_{\theta}(\mathbf{\hat{x}}, \mathcal{C})\) and the marginal distributions. In conditional diffusion models, the likelihood \(p_{\theta}(\mathbf{\hat{x}} \vert \mathcal{C})\) of generating \(\mathbf{\hat{x}}\) given \(\mathcal{C}\) is proportional to the joint distribution:
\[
p_{\theta}(\mathbf{\hat{x}} \vert \mathcal{C}) = \frac{p_{\theta}(\mathbf{\hat{x}}, \mathcal{C})}{p(\mathcal{C})} \propto p_{\theta}(\mathbf{\hat{x}}, \mathcal{C}),
\]
where \(p(\mathcal{C})\) is the marginal probability of the context description, treated as a constant during optimization. By maximizing \(\mathcal{R}_\textrm{global}\), which aligns global semantic features, the model indirectly maximizes the mutual information \(I(\mathbf{\hat{x}}, \mathcal{C})\), thereby enhancing the likelihood \(p_{\theta}(\mathbf{\hat{x}} \vert \mathcal{C})\) in the conditional diffusion model.


The Fine-Grained Consistency Reward, $\mathcal{R}_{\textrm{fine-grained}}$, captures detailed multimodal alignment between the generated image $\mathbf{\hat{x}}$ and the textual context description $\mathcal{C}$. It operates at a token level, leveraging bidirectional self-attention and cross-attention mechanisms in the BLIP-2 QFormer, followed by the Image-Text Matching (ITM) classifier to maximize the alignment score.

\textbf{Self-Attention on Text Tokens:}
    Text tokens $\mathcal{T}_{\mathrm{tokens}}$ undergo self-attention, allowing interactions among words to capture intra-text dependencies:
    \begin{equation}
        \mathcal{T}_{\mathrm{attn}} = \tt{SelfAttention}(\mathcal{T}_{\mathrm{tokens}})
    \end{equation}

\textbf{Self-Attention on Image Tokens:}
    Image tokens $\mathcal{Z}$ are derived from visual features of the generated image $\mathbf{\hat{x}}$ using a cross-attention mechanism:
    \begin{equation}
        \mathcal{Z} = \tt{CrossAttention}(\mathcal{Q}_{\mathrm{learned}}, \mathcal{I}_{\mathrm{tokens}}(\mathbf{\hat{x}}))
    \end{equation}
    These tokens then pass through self-attention to extract intra-image relationships:
    \begin{equation}
        \mathcal{Z}_{\mathrm{attn}} = \tt{SelfAttention}(\mathcal{Z})
    \end{equation}

\textbf{Cross-Attention between Text and Image Tokens:}
    The text tokens $\mathcal{T}_{\mathrm{attn}}$ and image tokens $\mathcal{Z}_{\mathrm{attn}}$ interact through cross-attention to focus on multimodal correlations:
    \begin{equation}
        \mathcal{F} = \tt{CrossAttention}(\mathcal{T}_{\mathrm{attn}}, \mathcal{Z}_{\mathrm{attn}})
    \end{equation}

\textbf{ITM Classifier for Alignment:}
    The resulting multimodal features $\mathcal{F}$ are fed into the ITM classifier, which outputs two logits: one for positive match ($j=1$) and one for negative match ($j=0$). The positive class ($j=1$) indicates strong alignment between the image-text pair, while the negative class ($j=0$) indicates misalignment:
    \begin{equation}
        \mathcal{R}_{\textrm{fine-grained}} = \tt{ITM\_Classifier}(\mathcal{F})_{j=1}
    \end{equation}

The ITM classifier predicts whether the generated image and the textual context description match. Maximizing the logit for the positive class $j=1$ corresponds to maximizing the log probability $\log p(\mathbf{\hat{x}}, \mathcal{C})$ of the joint distribution of image and text. This process aligns the fine-grained details in $\mathbf{\hat{x}}$ with $\mathcal{C}$, increasing the mutual information between the generated image and the text features.

\textbf{Improving fine-grained relationships of CLIP.} While the CLIP Text Encoder, at times, struggles to accurately capture spatial features when processing longer sentences in the Multimodal Context Description, \method addresses this limitation by distilling the global semantic and fine-grained semantic rewards from BLIP-2 QFormer into a specific set of UNet denoiser layers, as mentioned in the implementation details under Appendix~\ref{appendix:impl}~(i.e., Q, V transformation layers including $\tt{to\_q, to\_v, query, value}$). This strengthens the alignment between the generated image tokens~(Q) and input text tokens from the Multimodal Context Description~(K, V) in the cross-attention mechanism of the UNet denoiser. As a result, we obtain generated images with improved fidelity, particularly w.r.t.~spatial relationships, thereby helping to mitigate the shortcomings of vanilla CLIP Text Encoder in processing the long sentences of the Multimodal Context Description.

To illustrate further, a Context Description like “the dog under the pool” is processed in three steps: (1) self-attention is applied to the text tokens (K, V), enabling spatial terms like “dog,” “under,” and “pool” to interact; (2) self-attention is applied to visual features represented by the generated image tokens (Q) to extract intra-image relationships (3) cross-attention aligns this text features with visual features. The resulting alignment scores are used to compute the mean and select the positive class for the reward. Our approach to distill this reward into the cross-attention layers therefore ensures that spatial relationships and other fine-grained attributes are effectively captured, improving the fidelity of generated images.


\section{The Choice of Text Encoder in SDXL and BLIP-2 QFormer}

The choice of text encoder in our pipeline is to leverage pre-trained models for their respective strengths. SDXL inherently uses the CLIP Text Encoder for its generative pipeline, as it is designed to process text embeddings aligned with the CLIP Image Encoder. In the Multimodal Context Evaluator, we use the BLIP-2 QFormer, which is pre-trained with a BERT-based text encoder.

\section{Textual Inversion for Data Augmentation}
In our experiments, we applied Textual Inversion for data augmentation as follows: given a reference image, Textual Inversion learns a new text embedding that captures the context of the reference image (denoted as $<$context$>$). This embedding is then used to generate multiple augmented images by employing the prompt: ``a photo of $<$context$>$". This approach allows Textual Inversion to create context-relevant augmentations for comparison in our experiments.

\section{Convergence Curve}
To evaluate convergence, we monitor the training process using the Global Semantic Reward and Fine-Grained Consistency Reward as criteria. Specifically, we observe the stabilization of these rewards over training iterations. Figure~\ref{fig:convergence} presents the convergence curves for both rewards, illustrating their gradual increase followed by stabilization around 50k iterations. This steady state indicates that the model has learned to effectively align the generated images with the multimodal context.

\begin{figure}[!h]
    \centering
    \includegraphics[width=\linewidth]{figures/convergence.pdf}
    \caption{Convergence curves of Global Semantic and Fine-Grained Consistency Rewards}
    \label{fig:convergence}
\end{figure}


\section{Fidelity Evaluation using GPT-4o}
In addition to the results above, we compute additional metrics for fidelity, which measure how well the model preserves scene attributes when generating new images from a reference image. For this, we use GPT-4o (model version: 2024-05-13) as the MLLM oracle for a VQA task on the MME Perception benchmark \citep{fu2024mme}. 
% We use a MLLM as an oracle for a visual question-answering (VQA) task on the MME Perception benchmark \citep{fu2024mme}. In this experiment, we use GPT-4o (model version: 2024-05-13) as the oracle. 
We evaluate \method with and without fine-tuning process.

The MME dataset consists of Yes/No questions, with a positive and a negative question for every reference image. To measure fidelity, we measure the rate at which the oracle's answer remains consistent across the reference and the generated image for every image in the dataset. We run the experiment multiple times and report the average numbers in Table~\ref{table:fidelity_comparison}. We see that fine-tuning the base SDXL with our novel rewards results in an average increase of $2.99\%$ in fidelity.

\begin{table}[!h]
\centering
\footnotesize
\caption{Fidelity between reference and generated images from \method with and without fine-tuning.}
\resizebox{0.9\linewidth}{!}{
\begin{tabular}{clccc}
\toprule
 \textbf{MLLM Oracle} & \textbf{\method} & \textbf{Fidelity on ``Yes"} & \textbf{Fidelity on ``No"} & \textbf{Overall Fidelity} \\
 \midrule
 \multirow{2}{*}{\makecell{\textbf{GPT-4o}\\\textbf{Ver: 2024-05-13}}}
 & w/o fine-tuning & 68.33\xspace\xspace\xspace\xspace\xspace\xspace\xspace\xspace\xspace\xspace & 70.55\xspace\xspace\xspace\xspace\xspace\xspace\xspace\xspace\xspace\xspace & 71.18\xspace\xspace\xspace\xspace\xspace\xspace\xspace\xspace\xspace\xspace \\
 \cmidrule{2-5}
 % \cmidrule{2-12}
 & w/ fine-tuning & \textbf{69.72} {\scriptsize \color{ForestGreen}\textbf{$\uparrow$ 1.39}}  & \textbf{73.61} {\scriptsize \color{ForestGreen}\textbf{$\uparrow$ 3.06}}  & \textbf{74.17} {\scriptsize \color{ForestGreen}\textbf{$\uparrow$ 2.99}} \\
\bottomrule
\end{tabular}
}
\label{table:fidelity_comparison}
\end{table}


\section{Implementation Details}
\label{appendix:impl}
We implement \method using PyTorch \citep{paszke2019pytorch} and HuggingFace diffusers \citep{huggingface2023diffusers} libraries. For the generative model, we utilize the SDXL Base $1.0$ which is a standard and commonly used pre-trained diffusion model in natural images domain. In the pipeline, we employ CLIP ViT-G/14 as image encoder and both CLIP-L/14 \& CLIP-G/14 as text encoders \citep{radford2021learning}. We perform LoRA fine-tuning on the following modules of SDXL UNet denoiser including $Q$, $V$ transformation layers, fully-connected layers ($\tt{to\_q, to\_v, query, value, ff.net.0.proj}$) with rank parameter $r = 8$, which results in $11$M trainable parameters $\approx 0.46\%$ of total $2.6$B parameters. The fine-tuning is done on $8$ NVIDIA A100 80GB GPUs using AdamW \citep{loshchilov2017decoupled} optimizer, a learning rate of \texttt{5e-6}, and gradient accumulation steps of $8$.

\section{Additional Qualitative Results}
\label{appendix:visuals}
Figure~\ref{fig:diversity_compact_caption} showcases two examples of context description from MLLM in the inference pipeline where answers are not included in text guidance. Figure~\ref{fig:diversity_full} illustrates additional qualitative results highlighting the diversity and multimodal context fidelity between reference and synthetic images, as well as across images generated by \method with different random seeds. Figure~\ref{fig:qualitative_full} shows additional qualitative comparisons between \method and SOTA image generation methods on VQA and HOI Reasoning tasks.
\begin{figure}[!h]
    \centering
    \includegraphics[width=\linewidth]{figures/diversity_full.pdf}
    \vspace{-5mm}
    \caption{Diversity and multimodal context fidelity between reference and synthetic image and across generated ones from \method with different random seeds.}
    \label{fig:diversity_full}
\end{figure}
\begin{figure}[!h]
    \centering
    \includegraphics[width=\linewidth]{figures/qualitative_full_v1.pdf}
    \vspace{-5mm}
    \caption{Qualitative comparison between \method and other image generation methods on MME Perception and HOI Reasoning benchmarks.}
    \label{fig:qualitative_full}
\end{figure}
% \begin{thebibliography}{}

% \setlength{\itemindent}{-\leftmargin}
% \makeatletter\renewcommand{\@biblabel}[1]{}\makeatother
% \bibitem{} J.~Alspector, B.~Gupta, and R.~B.~Allen (1989).
%     \newblock Performance of a stochastic learning microchip.
%     \newblock In D. S. Touretzky (ed.),
%     \textit{Advances in Neural Information Processing Systems 1}, 748--760.
%     San Mateo, Calif.: Morgan Kaufmann.

% \bibitem{} F.~Rosenblatt (1962).
%     \newblock \textit{Principles of Neurodynamics.}
%     \newblock Washington, D.C.: Spartan Books.

% \bibitem{} G.~Tesauro (1989).
%     \newblock Neurogammon wins computer Olympiad.
%     \newblock \textit{Neural Computation} \textbf{1}(3):321--323.
% \end{thebibliography}

%%%%%%%%%%%%%%%%%%%%%%%%%%%%%%%%%%%%%%%%%%%%%%%%%%%%%%%%%%%%
% \section*{Checklist}


% %%% BEGIN INSTRUCTIONS %%%
% The checklist follows the references. For each question, choose your answer from the three possible options: Yes, No, Not Applicable.  You are encouraged to include a justification to your answer, either by referencing the appropriate section of your paper or providing a brief inline description (1-2 sentences). 
% Please do not modify the questions.  Note that the Checklist section does not count towards the page limit. Not including the checklist in the first submission won't result in desk rejection, although in such case we will ask you to upload it during the author response period and include it in camera ready (if accepted).

% \textbf{In your paper, please delete this instructions block and only keep the Checklist section heading above along with the questions/answers below.}
% %%% END INSTRUCTIONS %%%




\end{document}
