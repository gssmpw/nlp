\documentclass[twoside]{article}

\usepackage{aistats2025}

\usepackage{microtype}
\usepackage{graphicx}
\usepackage{svg}
\usepackage{subcaption}
\usepackage{booktabs} 
\usepackage{hyperref}
\usepackage{amssymb}
\usepackage{amsmath}
\usepackage{commath}
\usepackage{graphicx,bm}
\usepackage{verbatim}
\usepackage{csquotes}
\usepackage{multirow}
% If your paper is accepted, change the options for the package
% aistats2025 as follows:
%
%\usepackage[accepted]{aistats2025}
%
% This option will print headings for the title of your paper and
% headings for the authors names, plus a copyright note at the end of
% the first column of the first page.

% If you set papersize explicitly, activate the following three lines:
%\special{papersize = 8.5in, 11in}
%\setlength{\pdfpageheight}{11in}
%\setlength{\pdfpagewidth}{8.5in}

% If you use natbib package, activate the following three lines:
\usepackage[round]{natbib}
\renewcommand{\bibname}{References}
\renewcommand{\bibsection}{\subsubsection*{\bibname}}

% If you use BibTeX in apalike style, activate the following line:
\bibliographystyle{apalike}

\begin{document}

% If your paper is accepted and the title of your paper is very long,
% the style will print as headings an error message. Use the following
% command to supply a shorter title of your paper so that it can be
% used as headings.
%
%\runningtitle{I use this title instead because the last one was very long}

% If your paper is accepted and the number of authors is large, the
% style will print as headings an error message. Use the following
% command to supply a shorter version of the authors names so that
% they can be used as headings (for example, use only the surnames)
%
%\runningauthor{Surname 1, Surname 2, Surname 3, ...., Surname n}

\twocolumn[

\aistatstitle{Multi-level Supervised Contrastive Learning}

\aistatsauthor{Naghmeh Ghanooni$^{1}$ \And  
    Barbod Pajoum$^{1}$\And  
    Harshit Rawal$^{1}$ \AND
    Sophie Fellenz$^{1}$ \And  
    Vo Nguyen Le Duy$^{2}$ \And  
    Marius Kloft$^{1}$
    }
 \vspace{10pt}
\aistatsaddress{$^{1}$RPTU Kaiserslautern-Landau \And $^{2}$RIKEN, Japan}]
% \footnotemark[*]
% \let\thefootnote\relax\footnotetext{* Indicates equal contribution}



\begin{abstract}
Contrastive learning is a well-established paradigm in representation learning. The standard framework of contrastive learning minimizes the distance between \enquote{similar} instances and maximizes the distance between dissimilar ones in the projection space, disregarding the various aspects of similarity that can exist between two samples. Current methods rely on a single projection head, which fails to capture the full complexity of different aspects of a sample, leading to sub-optimal performance, especially in scenarios with limited training data.
In this paper, we present a novel supervised contrastive learning method in a unified framework called multi-level contrastive learning (MLCL), that can be applied to both multi-label and hierarchical classification tasks. The key strength of the proposed method is the ability to capture similarities between samples across different labels and/or hierarchies using multiple projection heads. Extensive experiments on text and image datasets demonstrate that the proposed approach outperforms state-of-the-art contrastive learning methods. 
\end{abstract}

\section{Introduction}
\label{introduction}

% RL and supervised RL
Contrastive learning is a key framework in representation learning, with the primary objective of learning adjacent representations for \enquote{similar} samples. One widely adopted approach is supervised contrastive learning \citep{khosla2020supervised}, known as SupCon, where similarity is defined based on groundtruth labels. In SupCon, the model learns representations by bringing samples from the same class closer in the representation space, while distancing those from different classes. However, it relies on a single label per sample to define similarity, posing limitations for multi-label classification and datasets with hierarchical class structures. Existing approaches typically address these challenges separately, using specific loss functions \citep{zhang2022use} tailored for hierarchical structures or learning label-level embeddings \citep{dao2021contrast} for multi-label datasets \citep{zhang2024multi}. In contrast, our method introduces a unified framework called multi-level contrastive learning, which effectively captures both multi-label and hierarchical aspects of data within a single representation space.

\begin{figure*}
\subfloat[\label{fig:text-rep}]{%
  \includegraphics[width=.49\linewidth]{images/nlp_multi_level.png}%
}
\hfill
\subfloat[\label{fig:img-rep}]{%
  \includegraphics[width=.49\linewidth]{images/img_multi_level.png}%
}
\caption{An illustration of two projection spaces using different examples: a) TripAdvisor reviews and b) CIFAR-100. Similarity can be defined across various dimensions, and a single projection space is insufficient to capture the full spectrum of feature levels.}
\end{figure*}

In this work, we incorporate multiple projection heads into the standard contrastive learning architecture, each tailored to learn representations from a specific aspect of the instance. These projection heads are connected to the encoder to capture different levels of representation. To enhance the effect of the projection heads, a temperature hyperparameter is used to control class discrimination at each level in the projection space, as the concept of similarity differs across levels. By tuning the temperature, the model focuses on \enquote{hard negatives} at the relevant level while reducing their impact on others. This significantly improves performance, particularly in scenarios with limited training samples. 

The main contributions are summarized as follows:
\begin{itemize}
    \item This paper introduces a novel unified framework that combines hierarchical and multi-label settings into a single multi-level approach.
    \item We propose an end-to-end architecture that generalizes standard contrastive learning through multiple projection heads. 
    \item These projection heads capture label-specific similarities and class hierarchies or serve as a regularizer to prevent overfitting.
    \item We validate the effectiveness of the proposed method on multi-label and hierarchical classification tasks within the domains of natural language processing (NLP) and computer vision.
\end{itemize}

\section{Related Work}
\label{relatedwork}
\paragraph{Contrastive learning}
Learning a high-quality representation of the target data that can be transferred to various downstream tasks is a major challenge in both 
Natural Language Processing(NLP) and computer vision. Contrastive learning is a powerful technique in representation learning. SimCLR~\citep{chen2020simple} introduced a novel framework for acquiring high-quality representations from unlabeled images, achieving state-of-the-art results in the field. Its popularity stems from its simplicity, as it operates without the need for specialized architectures \citep{bachman2019learning, henaff2020data} or memory banks \citep{he2020momentum, wu2018unsupervised}, while demonstrating remarkable performance across diverse downstream tasks. SimCLR discriminates positive and negative pairs in the projection space, and large batch sizes with many negative pairs and various data augmentations are required for optimal performance. 
Several subsequent papers attempted to improve SimCLR’s performance and overcome its limitations. For example, \citet{chuang2020debiased} presents a debiased contrastive loss that corrects the sampling bias of negative examples, while BYOL \citep{grill2020bootstrap} relies only on positive samples for training and is more robust to batch size and data augmentations. Furthermore, \citet{kalantidis2020hard} emphasizes the importance of hard negatives in contrastive learning and proposes a hard negative mixing strategy that improves performance and avoids the need for large batch sizes and memory banks \citep{mocov2}. Recently, \citet{wang2023adaptive} introduced an adaptive multiple projection head mechanism for self-supervised contrastive learning models to address better intra- and inter-sample similarities between different views of the samples. They also incorporate an adaptive temperature mechanism to re-weight each similarity pair. In contrast, our approach employs multiple projection heads to capture different aspects of the samples within a supervised framework, making it suitable for downstream tasks such as multi-label and hierarchical classifications.

\begin{figure*}
\subfloat[Two projection heads $g_1$ and $g_2$ are attached to the encoder's output to capture subclass and superclass similarities respectively.\label{fig:img-arch}]{%
  \includegraphics[height=4cm,width=.49\linewidth]{images/img_arch.png}%
}
\hfill
\subfloat[We employ an individual projection head for each label and an additional global head to capture the overall similarities between the two reviews. \label{fig:text-arch}]{%
  \includegraphics[height=4cm,width=.49\linewidth]{images/nlp_arch.png}%
}
\caption{
Our proposed architecture with multiple projection heads for a) hierarchical classification and b) multi-label classification. 
The final loss is computed as a weighted sum of the losses from each head, with an additional cross-entropy loss applied in the case of text classification.
}
%\label{}
\end{figure*}

\paragraph{Supervised Contrastive Learning}
The self-supervised contrastive learning approach can be extended to supervised settings with minimal modifications. SupCon~\citep{khosla2020supervised} developed a supervised contrastive learning method that leverages labels by modifying the contrastive loss function. Instead of comparing each sample with all other samples, as in SimCLR, SupCon pulls samples from the same class closer and pushes samples from different classes farther apart. Several studies have aimed to enhance SupCon. For example, \citet{barbano2022unbiased} focused on controlling the minimal distance between positive and negative samples and introduced a new debiasing regularization loss. SupCon suffers from imbalanced data distribution, leading to poor uniformity in the feature distribution on the hypersphere. \citet{li2022targeted} addressed this issue by proposing a targeted supervised contrastive learning loss to maintain a uniform feature distribution even with long-tail distribution data. Additionally, \citet{chen2022perfectly} suggested a weighted class-conditional InfoNCE loss to enhance the robustness of SupCon. Furthermore, certain methods leverage contrastive representation to learn from noisy labels \citep{ciortan2021framework, ghosh2021contrastive, li2021learning, li2020mopro, xue2022investigating, yi2022learning}.  While these approaches indicate the suitability of contrastive learning in the supervised setting, they do not provide the possibility of contrasting multiple aspects.


\paragraph{Multi-label and Hierarchical Contrastive Learning}
Recent research has extended supervised contrastive learning (SCL) to multi-label contrastive learning \citep{dao2021multi, malkinski2022multi, dao2021contrast, zaigrajewcontrastive, zhang2024multi}. These extensions aim to bring samples with shared labels closer together while separating those with different labels. For instance, \citet{zaigrajewcontrastive} utilized the Jaccard similarity measure to assess label overlap between sample pairs and \citet{zhang2024multi} introduced multiple positive sets to address this problem. Meanwhile, \citet{sajedi2023end} introduced a kernel-based approach that transforms embedded features into a mixture of exponential kernels in Gaussian RKHS for multi-label contrastive learning. Additionally, \citet{dao2021contrast} proposed a module that learns multiple label-level representations using multiple attention blocks. However, the method by \citet{dao2021contrast}, which trains label-specific features, involves two-stage training and produces task-specific representations not universally applicable across all downstream tasks.  In contrast, our approach considers comprehensive similarity aspects among samples beyond mere label overlap, ensuring applicability across a wide range of classification tasks. Incorporating the class hierarchies, \citet{landrieu2021leveraging} (
Guided) integrate class distance into a prototypical network to model the hierarchical class structure and \citet{zhang2022use} (HiMulConE) introduced a hierarchical framework for multi-label representation learning with a new loss function penalizing sample distance in label hierarchies. Existing approaches often overlook diverse similarity perspectives, tailoring solutions to specific scenarios rather than broader applications.

\section{Approach}
\label{approach}

\subsection{Problem Formulation}
\label{sec:formulation}
For simplicity, we adopt a unified formulation for both multi-label and hierarchical classification. In each training iteration, a batch of randomly sampled inputs is given to the network, where each input is associated with \( L \) levels: \(\{x_k, y_k^l\}\), for \( k \in \{1, \dots, B\} \) and \( l \in \{1, \dots, L\} \). In multi-label classification, \( y_k^l \) corresponds to the labels assigned to sample \( x_k \), while for datasets with a hierarchical structure, \( y_k^l \) represents labels at different levels of the hierarchy. The objective is to train an encoder \( f(.) \) that learns semantically meaningful representations of the training data, \( f(x) \), which can subsequently be utilized for downstream tasks.

\subsection{Framework}
\label{subsec:framework}
Our approach is structurally similar to the standard supervised contrastive learning method \citep{khosla2020supervised}, with the key modification being the addition of multiple projection heads.
\begin{enumerate}
    \item For every sample $x_k$ where $1 \leq k \leq N$, two random augmentations $\tilde{x}_{2k}$ and $\tilde{x}_{2k-1}$ are generated, which both have the same label as $x_k$.
    
    \item The encoder network $f(.)$ takes samples $\tilde{x}_i$  and produces their respective embedding vectors where $1 \leq i \leq 2N$.
    
    \item The embedding $f(\tilde{x}_i)$ is mapped to multiple projection spaces by the projection networks $g_h$ where $h \in \{1, \dots , H\}$ and $H$ is the total number of projection heads , resulting in $z_i^h=g_h(f(\tilde{x}_i))$.
    
    \item The supervised contrastive loss is used to pull positive sample pairs together and push negative sample pairs apart in the projection spaces. Positive pairs are defined independently for each projection head based on a similarity criterion. We will elaborate on this part more in the following.
\end{enumerate}

Each projection space is dedicated to a specific similarity notion, where positive pairs are drawn closer and negative pairs are pushed apart. The similarity criteria may be based on the input samples' labels. For instance, if a pair shares the same $l$-th label, it is considered a positive pair; otherwise, it is treated as negative. Alternatively, similarity can be determined by aggregating all labels, with pairs exhibiting substantial label overlap considered positive. In this work, we focus on these two similarity notions, but other criteria could be explored in future research.

For each projection space $h$ the corresponding loss function is defined as $L_h$ inspired from \cite{khosla2020supervised}:
\begin{equation}
\label{supcon-loss}
L_h = \sum_{i \in I} \frac{-1}{|P_h(i)|} \sum_{p\in P_h(i)} \log \frac{\exp\left(\frac{z_i^h \cdot z_p^h}{\tau_h}\right)}{\sum_{a \in A(i)} \exp\left(\frac{z_i^h \cdot z_a^h}{\tau_h}\right)}.
\end{equation}

For each augmented sample, $\tilde{x}_i$, the loss function aims to increase the relative similarity between $z_i^h$ and the positive samples associated with $\tilde{x}_i$, which is defined by the indices $P_h(i)$. The similarity is measured by the dot product between the two projection vectors $z_i^h\cdot z_p^h$. We denote by $A(i)$ the set of indices for all samples, excluding sample $i$, which has a cardinality of $2N-1$. The scalar temperature hyperparameter $\tau_h$ regulates the influence of hard negatives on the final representation as discussed by \citet{chen2020simple} and its influence is further analyzed in the ablation studies.

Our final loss function is defined as $L$:
\begin{equation}
\label{mlcl-loss}
L = \sum_{h=1}^H \alpha_{h} L_{\tau_h}^h,
\end{equation}
where $\sum_{h=1}^H\alpha_{h}=1$.

% The temperature $\tau_h$ plays a crucial role, as it determines the degree of separation of the samples in the projection space corresponding to the head $h$. 

We now describe each of the two tasks, multi-label classification and hierarchical classification, in more detail below.

\begin{figure}
\label{tsne-fig}
\subfloat[SupCon\label{fig:supcon_rep}]{%
  \includegraphics[height=3cm,width=.49\linewidth]{images/x_normal.png}%
}
\hfill
\subfloat[MLCL\label{fig:mulhead_rep}]{%
  \includegraphics[height=3cm,width=.49\linewidth]{images/x_multihead.png}%
}
\caption{The t-SNE visualization of samples from five different classes in the representation space. The SupCon representation indicates a lack of meaningful structure, as samples from class \textit{camel} are positioned far from \textit{chimpanzee}, \textit{kangaroo}, and \textit{cattle} compared to the \textit{bottle} class. In contrast, our proposed method, MLCL, groups animal classes more closely in the representation space.}
\label{fig:representations}
\end{figure}
\subsection{Hierarchical Classification}
\label{sec:single-label}
To demonstrate this, we use the CIFAR-100 dataset \citep{cifar100} as an example. CIFAR-100 provides a two-level class hierarchy, where each sample has two labels: one for the subclass and one for the superclass. Each superclass consists of five subclasses—such as the flower superclass, which includes orchids, poppies, roses, sunflowers, and tulips. In this paper, we focus on the 100-class classification task but utilize the available superclass information to enhance the quality of the learned representations. We use multiple projection heads to deal with class hierarchies, with each head corresponding to a hierarchy level. Taking the CIFAR-100 dataset as an example with two hierarchy levels, our framework utilizes two projection heads: one for the standard subclass labels and one for the superclasses. Figure \ref{fig:img-arch} shows the architecture with the two projection heads. 

During training, given an input batch, we apply the first augmentation module to create a duplicate batch. Both batches pass through an encoder to obtain normalized embeddings. In the training stage, these embeddings are given to the two projection heads. The first head pulls samples with the same subclass label together using the loss function defined in Equation \eqref{supcon-loss}, while the second head pulls the samples with the same superclass together. We set a higher temperature for the second head to achieve a more distinct representation for samples belonging to different superclasses. According to \citet{wang2021understanding}, increasing the temperature leads to greater separation among the distinct classes, while decreasing the temperature results in a more evenly distributed set of classes, making the model more tolerant to negative samples.
In the inference stage, these two heads are removed, leaving only a single representation that has integrated information from both similarity criteria. For evaluation, a linear classifier is trained on these frozen representations using a cross-entropy loss, which is also known as the linear evaluation protocol in representation learning~\citep{chen2020simple, khosla2020supervised}.

\begin{table*}[t]
\caption{Overview of evaluation datasets: 
CIFAR-100 and DeepFashion each have two hierarchical levels. However, classification within the subclass level is the primary focus, while the superclass serves as an auxiliary signal during training.}
\label{overview}
\vskip 0.15in
\hspace{-1cm} % Adjust the value as needed
\begin{center}
\begin{small}
\begin{sc}
\begin{tabular}{cccccccc}
\toprule
Dataset & Task  & \#Train & \#Test & \#Levels & \#Classes \\
\midrule
CIFAR-100    & image classification& 50K& 10K& 2 & 100 \\
DeepFashion  & image classification & 200K & 40K & 2 & 50 \\
TripAdvisor    & sentiment analysis& 10K & 2K & 7 & 3 \\
BeerAdvocate & sentiment analysis& 10K & 2K & 5 & 3 \\
\bottomrule
\end{tabular}
\end{sc}
\end{small}
\end{center}
\vskip 0.15in
\end{table*}
\subsection{Multi-label Classification}
\label{sec:multi-label}
Multi-label classification aims to predict $L$ labels for each sample $x_k$, where each sample has a multi-hot label vector $y_k\in \{0,1\}^L$ and $y_k^l$ denotes the one-hot representation of the $l$-th label. Similarly to the method described in Section~\ref{sec:single-label}, we assign a distinct projection head for each label and include an additional global projection head to capture broader semantic relationships which will be elaborated on later.
We demonstrate our method for multi-label classification using the TripAdvisor review dataset \citep{wang2010latent}. Each review has $L=7$ labels that indicate user ratings for different aspects such as service, location, etc. We employ $H=8$ total heads, where seven heads are dedicated to the seven different aspects, and the final eighth head (global head) is used to focus on global similarities between two hotel reviews that have a similar overall sentiment, combining all aspects together. This illustrates another application of projection heads, where they are used as regularizers, especially in scenarios with limited training samples. Figure \ref{fig:text-arch} shows the architecture of the model.

\paragraph{Global Projection Head}
For datasets with ordinal class numbers, we can define a global similarity perspective that combines the information from all classes. For instance, in the hotel review dataset, if two reviews have 4 or 5 ratings for all aspects, then they are generally similar, and we expect them to have a similar representation, even if the specific rating varies for different aspects, such as one review giving four to service and the other giving five. We use the Jaccard similarity metric, inspired by \citet{zaigrajewcontrastive} to focus on these global similarities. 

The Jaccard similarity measure $s_{i,j}$ is defined for two samples $i$ and $j$ as: 
$$s_{i, j} = \frac{\sum_{l=1}^{L} \min(y_{i}^l, y_{j}^l)}{\sum_{l=1}^{L} \max(y_{i}^l, y_{j}^l)}$$

We introduce a hyperparameter $t$ to define all samples $i$ and $j$ with $s_{i,j} > t$ as positive samples. The loss function defined in Equation \eqref{supcon-loss} is then adjusted to put more weight on samples with higher Jaccard similarity as proposed by \citet{zaigrajewcontrastive}: 
\begin{equation*}
L_{gl} = \sum_{i \in I} \frac{-1}{|P(i)|} \sum_{p\in P(i)} s_{i, p}\log \frac{\exp\left(\frac{z_i \cdot z_p}{\tau}\right)}{\sum_{a \in A(i)} \exp\left(\frac{z_i \cdot z_a}{\tau}\right)}.
\end{equation*}

The training procedure for multi-label classification closely follows the approach used for single-label classification in Section \ref{sec:single-label}. However, since our target datasets are textual, we follow the common practice of combining contrastive learning with the cross-entropy loss, as demonstrated by \citet{gunel2020supervised}. The remainder of the training process is consistent with Section~\ref{sec:single-label}.

The linear classifier and cross-entropy loss are trained simultaneously with the projection heads and the pretrained Bert encoder. The final loss $L_{\text{MLCL}}$ is  defined as a weighted average of the loss in each projection head plus the cross-entropy loss, denoted as 
\begin{equation}
\label{multi-label-loss}
L = \sum_{h=1}^{H} \alpha_{h} L^h_{\tau_h} + (1-\sum_{h=1}^{H}\alpha_h) L_{\text{ce}},
\end{equation}
where $L_{ce}$ denotes the cross-entropy loss used for linear classification.

\subsection{Temperature Analysis}
\label{temperature-analysis}
Temperature plays a crucial role in controlling the impact of hard negative samples during training. Hard negatives are particularly challenging to distinguish from positive samples, as they often have high similarity to the anchor in the representation space. The temperature parameter helps to adjust the sensitivity of the model to subtle differences, allowing it to focus more effectively on more difficult distinctions. As temperature decreases, the model concentrates more on a smaller set of the closest hard negative samples, applying a stronger penalty to these examples. In contrast, when the temperature is large, the model distributes the penalty more evenly across all negative samples. To demonstrate this, we analyze two extreme cases of $\tau \to 0^+$  and $\tau \to +\infty$, inspired by \cite{wang2021understanding}. Based on Equation \eqref{supcon-loss}, the loss function for a given sample $i$ in the projection space $h$ is denoted as $L_h^i$:
\begin{equation*}  
L^i_h = \frac{-1}{|P_h(i)|} \sum_{p\in P_h(i)} \log \frac{\exp\left(\frac{z_i^h \cdot z_p^h}{\tau_h}\right)}{\sum_{a \in A(i)} \exp\left(\frac{z_i^h \cdot z_a^h}{\tau_h}\right)}.
\end{equation*}

For the sake of better readability, the index $h$ is omitted in the following derivation:
\begin{equation*}
\begin{split}
&\lim_{\tau \to 0^+} |P(i)| \times L^i = 
\\
&\sum_{p\in P(i)} \lim_{\tau \to 0^+} \Bigg[-\log \frac{\exp(z_i \cdot z_p/\tau)}{\sum_{a \in A(i)} \exp(z_i \cdot z_a/\tau)}\Bigg] = 
\\
&\sum_{p\in P(i)} \lim_{\tau \to 0^+} \Bigg[-\log \frac{\exp\big(z_i \cdot (z_p-z_i^{\max})/\tau\big)}{\sum_{a \in A(i)} \exp\big(z_i \cdot (z_a-z_i^{\max})/\tau\big)}\Bigg]
 \\
  &=\sum_{p\in P(i)} \lim_{\tau \to 0^+} \Bigg[ - \big(zi \cdot (z_p - z_i^{\max})/\tau\big) 
+ \\
 &  \hspace{34pt} \log \bigg(1 + \sum_{a \in A(i) \backslash p }\exp \big(z_i \cdot (z_a - z_i^{\max})/\tau\big)\bigg) \Bigg]=
  \\
  &\sum_{p\in P(i)} \lim_{\tau \to 0^+} \Bigg[
   - \big(z_i \cdot (z_p - z_i^{\max})/\tau\big)\Bigg] =
  \\
  & = \lim_{\tau \to 0^+} \frac{1}{\tau} \bigg[|P(i)| z_i \cdot z_i^{\max} - \sum_{p\in P(i)} z_i \cdot z_p\bigg],
\end{split}
\end{equation*}
where $z_i^{\max}=\arg\max_{a \in A} z_i \cdot z_a$.  We can infer that as \( \tau \to 0^+ \), the loss function focuses primarily on the positive samples and the negative sample with the maximum similarity to $z_i$, meaning only the hardest negative is considered. In contrast, as $\tau$ increases, negative samples contribute more uniformly to the loss function (for the derivation as $\tau \to +\infty$ please refer to the appendix). In hierarchical classification, the primary task is to classify at the lowest level (subclass), while higher-level labels in the hierarchy (superclass) are used to create a more structured feature space. To achieve this, we use a lower temperature for the projection space corresponding to the lowest level of the hierarchy to fully separate different subclasses, while using a higher temperature for superclasses to uniformly learn from all negative samples. Gradient analysis in the appendix demonstrates a similar effect for hard positives. At low temperatures, the loss function concentrates on the positive sample with the lowest similarity to the anchor, which is not ideal for higher levels of the hierarchy. Two positive samples at the superclass level may belong to different subclasses, requiring some degree of separation. Hence, focusing solely on hard positives at higher hierarchical levels can be counterproductive. This provides a second rationale for using a lower temperature for subclasses to ensure complete separation and a higher temperature for superclasses to promote more uniform learning across all samples. Further empirical analysis is provided in Section~\ref{experiments}.

\setlength{\tabcolsep}{6pt} % General space between cols (6pt standard)
\begin{table*}[t]
\caption{Top-1 classification accuracy on CIFAR-100 and DeepFashion test sets. We compare MLCL with ResNet50 trained with cross-entropy (CE), SimCLR~\citep{chen2020simple}, SupCon~\citep{khosla2020supervised}, Guided~\citep{landrieu2021leveraging}, and HiMulConE~\citep{zhang2022use} losses. The highest test accuracy is marked in bold.}
\label{deepfashion-cifar100-results}
\vskip 0.15in
\begin{center}
\begin{small}
\begin{sc}
\begin{tabular}{cccccccc}
\toprule
Dataset & SimCLR & CE & SupCon & Guided & HiMulConE & MLCL (ours) \\
\midrule
Cifar-100 & 70.70 & 75.30 & 76.50 & 76.40 & - & \textbf{77.70} \\
DeepFashion & 70.38 &  72.44 & 72.82 & 72.61 & 73.21 & \textbf{73.90} \\
\bottomrule
\end{tabular}
\end{sc}
\end{small}
\end{center}
\vskip -0.1in
\end{table*}

\begin{table*}[ht]
\caption{Top-1 classification accuracy on the test set of CIFAR-100 dataset using a subset of training samples. The highest test accuracy is marked in bold.}
\label{cifar100-results}
\vskip 0.15in
\begin{center}
\begin{small}
\begin{sc}
\begin{tabular}{rrrrr}
\toprule
\#Training Samples & SimCLR & Cross-Entropy & SupCon & MLCL (ours) \\
\midrule
% 50,000 & 70.70 &  75.30 & 76.50 & \textbf{77.51} \\
1,000 & 19.31 & 20.50 & 26.82 & \textbf{34.74} \\
5,000 & 34.51 &  21.56 & 46.15 & \textbf{56.02} \\
10,000 & 40.20 & 42.25 & 49.87 & \textbf{59.32} \\
20,000 & 54.73 &  59.04 & 65.21 & \textbf{69.36} \\
30,000 & 59.26 &  63.47 & 71.49 & \textbf{72.50} \\
40,000 & 63.21 &  67.10 & 74.80 & \textbf{75.60} \\
\bottomrule
\end{tabular}
\end{sc}
\end{small}
\end{center}
\vskip -0.1in
\end{table*}

\section{Experiments}
\label{experiments}
We evaluate the presented method using four benchmark datasets, two from the image domain and two text datasets. The first task is image classification on CIFAR-100 \citep{cifar100}, a widely used benchmark for image classification, and DeepFashion \citep{liuLQWTcvpr16DeepFashion}, a benchmark for hierarchical classification. We selected CIFAR-100 and DeepFashion because they offer class hierarchies. The second application focuses on multi-label classification using two text datasets for aspect-based sentiment analysis: TripAdvisor \citep{wang2010latent} and BeerAdvocate \citep{mcauley2012learning}.

\subsection{Datasets}
CIFAR-100 comprises 50,000 training images and 10,000 testing images, categorized into 100 classes, which are further grouped into 20 superclasses. The DeepFashion \citep{liuLQWTcvpr16DeepFashion} dataset, a large-scale clothing dataset, contains 200,000 training images and 40,000 testing images, with labels spanning 50 categories and organized into three superclasses. We intentionally chose these datasets because the widely used ImageNet \citep{deng2009imagenet} lacks groundtruth superclasses. This limitation makes ImageNet less appropriate for our multi-level setting, as it would reduce the problem to the standard supervised contrastive approach (SupCon). For multi-label classification, we conduct experiments on two textual datasets, TripAdvisor and BeerAdvocate. TripAdvisor is a hotel review dataset, with each review having seven different ratings based on various aspects (value, room, location, cleanliness, service, overall). We map the ratings to sentiments by changing ratings four and five to positive, three to neutral, and the others to negative. Similarly, BeerAdvocate consists of five aspects (appearance, aroma, palate, taste, overall), each associated with three sentiments, mirroring the structure of the other dataset. The summary of the datasets is given in Table \ref{overview}.
\setlength{\tabcolsep}{2pt} % General space between cols (6pt standard)
\begin{table*}[ht]
\caption{Top-1 classification accuracy on TripAdvisor and BeerAdvocate datasets. A direct comparison with SupCon \citep{khosla2020supervised} is not feasible due to its single-label classification nature. Instead, we compare our results with fine-tuned BERT using cross-entropy (CE) loss and our proposed MLCL loss (without and with the global projection head). The highest accuracy is highlighted in bold.}
\label{multi-label-results}
\vskip 0.15in
\begin{center}
\begin{small}
\begin{sc}
\begin{tabular}{ccccc}
\toprule
Dataset & \#Samples & CE & MLCL w/o $h_{global}$ & MLCL (ours) \\
\midrule
\multirow{4}{*}{TripAdvisor} & 30  & 71.30 &  72.50 & \textbf{72.9} \\
 & 60 & 74.19 &  74.81 & \textbf{75.1} \\
 & 180  & 77.00 &  77.30 & \textbf{77.9} \\
 & 360 & 78.10 &  78.44 & \textbf{79.0} \\
\midrule
\multirow{4}{*}{BeerAdvocate} & 30 & 65.45 & 65.95 & \textbf{66.90} \\
 & 60 & 66.37 &  66.71 & \textbf{67.40} \\
 & 180 & 68.70 &  69.25 & \textbf{69.61} \\
 & 360 & 70.54 &  71.22 & \textbf{71.81} \\
\bottomrule
\end{tabular}
\end{sc}
\end{small}
\end{center}
\vskip -0.1in
\end{table*}
\subsection{Implementation Details}
\paragraph{Hierarchical Classification} We conduct experiments on the CIFAR-100 and DeepFashion datasets with $H=2$ projection heads, assigning the first head to the subclass and the second to the superclass labels. The parameters for the first head are set as $\tau_1=0.1$ and $\alpha_1=0.5$, and for the second head as $\tau_2=0.5$, $\alpha_2=0.5$, following the notation in Equation \eqref{mlcl-loss}. The remaining hyperparameters are set as proposed in the SupCon paper \citep{khosla2020supervised} to ensure a fair comparison with this established baseline. Our model architecture includes a ResNet-50 encoder \citep{he2016deep} and two multi-layer perceptions (MLPs) with a single hidden layer serving as projection heads. The model is trained for 250 epochs, which is a quarter of the 1,000 epochs used by SupCon, with a batch size of 512. We use stochastic gradient descent (SGD) with momentum optimizer~\citep{ruder2016overview}. After training, the projection heads are discarded, and a linear classifier is trained on the frozen learned representation from the encoder's output to obtain the final accuracy on the subclass labels. This standard evaluation method, known as the linear evaluation protocol, is commonly used to assess the representations learned in contrastive learning studies \citep{khosla2020supervised, chen2020simple}. The procedure is depicted in Figure \ref{fig:img-arch}.

\paragraph{Multi-label Classification} We evaluate our approach on two text datasets for multi-label classification: TripAdvisor hotel reviews and BeerAdvocate reviews. The former consists of seven labels per review, and the latter has five labels. The task is to predict the sentiment for each aspect, a multi-label classification problem. We employ a pretrained BERT encoder \citep{devlin2018bert} with 512 embedding dimensions and fine-tune it by back-propagating the loss from multiple projection heads and a fully connected layer linked to the output of the encoder. The framework of our approach is illustrated in Figure \ref{fig:text-arch}. For the TripAdvisor and BeerAdvocate datasets, we use $H=8$ and $H=6$ projection heads, respectively. Each projection head corresponds to one label, with an additional global projection head for incorporating global similarities into the learned representation. The cross-entropy loss contribution is maintained at $0.7$ for both datasets. For the TripAdvisor dataset, we set $\alpha_i=0.03$ for the first seven heads and $\alpha_8=0.1$ for the final head. For the BeerAdvocate dataset, we apply $\alpha_i=0.04$ for the first five heads and $\alpha_6=0.1$ for the global head. The model is fine-tuned for 100 epochs using the Adam optimizer \citep{kingma2014adam} with a learning rate of $1e-5$ and a batch size of 16. We found that larger batch sizes did not significantly improve performance.


\subsection{Results}

\paragraph{Hierarchical Classification} 
Our method demonstrates a marginal improvement of 1\% over SupCon when trained on the complete CIFAR-100 and DeepFashion training datasets, as presented in Table \ref{deepfashion-cifar100-results}. Notably, this performance is achieved significantly faster, requiring only 250 training epochs compared to SupCon's 1,000 epochs. Furthermore, our approach exhibits superior performance with smaller training sets, achieving an enhancement of 9\% to 10\% for dataset sizes of 10K and 5K samples, as detailed in Table \ref{cifar100-results}. Additionally, our method surpasses the performance of models trained exclusively with cross-entropy loss by a substantial margin.

\paragraph{Multi-label Classification} Our results on the TripAdvisor and BeerAdvocate datasets are reported in Table~\ref{multi-label-results}. The numbers represent the average accuracies across ten different seeds. Using a pretrained BERT encoder, we focus on scenarios with limited training samples ranging from 30 to 360 while maintaining a test dataset size of 2K for all experiments. Our findings indicate that our approach enhances the quality of the learned representation, even with a robust pretrained encoder. The positive impact of the global projection head is also evident when comparing the third and fourth columns in Table~\ref{multi-label-results}.

% \setlength{\tabcolsep}{9pt} % General space between cols (6pt standard)

\subsection{Ablation Study}
\label{ablation}
The feature spaces of our approach and SupCon are visualized in Figure \ref{fig:representations} using t-SNE \citep{van2008visualizing}. The second projection head helps maintain the proximity of representations for samples within the same superclass. The class \textit{camel} is expected to be closer to \textit{cattle}, \textit{chimpanzee}, and \textit{kangaroo}, while remaining distant from \textit{bottle}, which belongs to a different superclass. Furthermore, Figure \ref{fig:temp-analysis} illustrates the impact of the superclass projection head's temperature on the final accuracy using a subset of training samples. Both excessively high and low temperatures negatively affect performance, while $\tau = 0.5$ yields the best results. The subclass temperature is fixed at $0.1$. For additional ablation studies on performance under label noise and transfer learning, please refer to the appendix.

\begin{figure}[ht] 
  \centering
  \includegraphics[width=\linewidth]{images/x_10000.png}
  \caption{Accuracy of MLCL as a function of the superclass projection head temperature. Increasing the temperature to 0.5 reduces the network's focus on hard negatives, leading to improved accuracy.}
  \label{fig:temp-analysis}
\end{figure}

\section{Conclusion}
\label{conclusion}

We introduced a novel supervised contrastive learning method within a unified framework called multi-level contrastive learning, which generalizes to tasks such as hierarchical and multi-label classification. Our approach incorporates multiple projection heads into the encoder network to learn representations at different levels of the hierarchy or for each label. This allows the method to capture label-specific similarities, exploit class hierarchies, and act as a regularizer to prevent overfitting. Although the parameter size for MLCL is equivalent to that of SupCon, it converges more rapidly and learns more effective representations, especially when training samples are limited. We have analyzed the impact of temperature on each projection head and showed that a higher superclass temperature improves the representations by allowing the model to focus less on hard negative samples. As shown in Section \ref{experiments}, this approach improves representation quality across various settings and datasets. Notably, the approach integrates seamlessly with existing contrastive learning methods without relying on any specific network architecture, making it flexible and applicable to a wide range of downstream tasks. Future work could explore the interpretability and explainability of the learned representations.

% \section{GENERAL FORMATTING INSTRUCTIONS}

% Submissions are limited to \textbf{8 pages} excluding references. 
% There will be an additional page for camera-ready versions of the accepted papers.

% Papers are in 2 columns with the overall line width of 6.75~inches (41~picas).
% Each column is 3.25~inches wide (19.5~picas).  The space
% between the columns is .25~inches wide (1.5~picas).  The left margin is 0.88~inches (5.28~picas).
% Use 10~point type with a vertical spacing of
% 11~points. Please use US Letter size paper instead of A4.

% Paper title is 16~point, caps/lc, bold, centered between 2~horizontal rules.
% Top rule is 4~points thick and bottom rule is 1~point thick.
% Allow 1/4~inch space above and below title to rules.

% Author descriptions are center-justified, initial caps.  The lead
% author is to be listed first (left-most), and the Co-authors are set
% to follow.  If up to three authors, use a single row of author
% descriptions, each one center-justified, and all set side by side;
% with more authors or unusually long names or institutions, use more
% rows.

% Use one-half line space between paragraphs, with no indent.

% \section{FIRST LEVEL HEADINGS}

% First level headings are all caps, flush left, bold, and in point size
% 12. Use one line space before the first level heading and one-half line space
% after the first level heading.

% \subsection{Second Level Heading}

% Second level headings are initial caps, flush left, bold, and in point
% size 10. Use one line space before the second level heading and one-half line
% space after the second level heading.

% \subsubsection{Third Level Heading}

% Third level headings are flush left, initial caps, bold, and in point
% size 10. Use one line space before the third level heading and one-half line
% space after the third level heading.

% \paragraph{Fourth Level Heading}

% Fourth level headings must be flush left, initial caps, bold, and
% Roman type.  Use one line space before the fourth level heading, and
% place the section text immediately after the heading with no line
% break, but an 11 point horizontal space.

% %%%
% \subsection{Citations, Figure, References}


% \subsubsection{Citations in Text}

% Citations within the text should include the author's last name and
% year, e.g., (Cheesman, 1985). 
% %Apart from including the author's last name and year, citations can follow any style, as long as the style is consistent throughout the paper.  
% Be sure that the sentence reads
% correctly if the citation is deleted: e.g., instead of ``As described
% by (Cheesman, 1985), we first frobulate the widgets,'' write ``As
% described by Cheesman (1985), we first frobulate the widgets.''


% The references listed at the end of the paper can follow any style as long as it is used consistently.

% %Be sure to avoid
% %accidentally disclosing author identities through citations.

% \subsubsection{Footnotes}

% Indicate footnotes with a number\footnote{Sample of the first
%   footnote.} in the text. Use 8 point type for footnotes. Place the
% footnotes at the bottom of the column in which their markers appear,
% continuing to the next column if required. Precede the footnote
% section of a column with a 0.5 point horizontal rule 1~inch (6~picas)
% long.\footnote{Sample of the second footnote.}

% \subsubsection{Figures}

% All artwork must be centered, neat, clean, and legible.  All lines
% should be very dark for purposes of reproduction, and art work should
% not be hand-drawn.  Figures may appear at the top of a column, at the
% top of a page spanning multiple columns, inline within a column, or
% with text wrapped around them, but the figure number and caption
% always appear immediately below the figure.  Leave 2 line spaces
% between the figure and the caption. The figure caption is initial caps
% and each figure should be numbered consecutively.

% Make sure that the figure caption does not get separated from the
% figure. Leave extra white space at the bottom of the page rather than
% splitting the figure and figure caption.
% \begin{figure}[h]
% \vspace{.3in}
% \centerline{\fbox{This figure intentionally left non-blank}}
% \vspace{.3in}
% \caption{Sample Figure Caption}
% \end{figure}

% \subsubsection{Tables}

% All tables must be centered, neat, clean, and legible. Do not use hand-drawn tables.
% Table number and title always appear above the table.
% See Table~\ref{sample-table}.

% Use one line space before the table title, one line space after the table title,
% and one line space after the table. The table title must be
% initial caps and each table numbered consecutively.

% \begin{table}[h]
% \caption{Sample Table Title} \label{sample-table}
% \begin{center}
% \begin{tabular}{ll}
% \textbf{PART}  &\textbf{DESCRIPTION} \\
% \hline \\
% Dendrite         &Input terminal \\
% Axon             &Output terminal \\
% Soma             &Cell body (contains cell nucleus) \\
% \end{tabular}
% \end{center}
% \end{table}

% \section{SUPPLEMENTARY MATERIAL}

% If you need to include additional appendices during submission, you can include them in the supplementary material file.
% You can submit a single file of additional supplementary material which may be either a pdf file (such as proof details) or a zip file for other formats/more files (such as code or videos). 
% Note that reviewers are under no obligation to examine your supplementary material. 
% If you have only one supplementary pdf file, please upload it as is; otherwise gather everything to the single zip file.

% You must use \texttt{aistats2025.sty} as a style file for your supplementary pdf file and follow the same formatting instructions as in the main paper. 
% The only difference is that it must be in a \emph{single-column} format.
% You can use \texttt{supplement.tex} in our starter pack as a starting point.
% Alternatively, you may append the supplementary content to the main paper and split the final PDF into two separate files.

% \section{SUBMISSION INSTRUCTIONS}

% To submit your paper to AISTATS 2025, please follow these instructions.

% \begin{enumerate}
%     \item Download \texttt{aistats2025.sty}, \texttt{fancyhdr.sty}, and \texttt{sample\_paper.tex} provided in our starter pack. 
%     Please, do not modify the style files as this might result in a formatting violation.
    
%     \item Use \texttt{sample\_paper.tex} as a starting point.
%     \item Begin your document with
%     \begin{flushleft}
%     \texttt{\textbackslash documentclass[twoside]\{article\}}\\
%     \texttt{\textbackslash usepackage\{aistats2025\}}
%     \end{flushleft}
%     The \texttt{twoside} option for the class article allows the
%     package \texttt{fancyhdr.sty} to include headings for even and odd
%     numbered pages.
%     \item When you are ready to submit the manuscript, compile the latex file to obtain the pdf file.
%     \item Check that the content of your submission, \emph{excluding} references and reproducibility checklist, is limited to \textbf{8 pages}. The number of pages containing references and reproducibility checklist only is not limited.
%     \item Upload the PDF file along with other supplementary material files to the CMT website.
% \end{enumerate}

% \subsection{Camera-ready Papers}

% %For the camera-ready paper, if you are using \LaTeX, please make sure
% %that you follow these instructions.  
% % (If you are not using \LaTeX,
% %please make sure to achieve the same effect using your chosen
% %typesetting package.)

% If your papers are accepted, you will need to submit the camera-ready version. Please make sure that you follow these instructions:
% \begin{enumerate}
%     %\item Download \texttt{fancyhdr.sty} -- the
%     %\texttt{aistats2022.sty} file will make use of it.
%     \item Change the beginning of your document to
%     \begin{flushleft}
%     \texttt{\textbackslash documentclass[twoside]\{article\}}\\
%     \texttt{\textbackslash usepackage[accepted]\{aistats2025\}}
%     \end{flushleft}
%     The option \texttt{accepted} for the package
%     \texttt{aistats2025.sty} will write a copyright notice at the end of
%     the first column of the first page. This option will also print
%     headings for the paper.  For the \emph{even} pages, the title of
%     the paper will be used as heading and for \emph{odd} pages the
%     author names will be used as heading.  If the title of the paper
%     is too long or the number of authors is too large, the style will
%     print a warning message as heading. If this happens additional
%     commands can be used to place as headings shorter versions of the
%     title and the author names. This is explained in the next point.
%     \item  If you get warning messages as described above, then
%     immediately after $\texttt{\textbackslash
%     begin\{document\}}$, write
%     \begin{flushleft}
%     \texttt{\textbackslash runningtitle\{Provide here an alternative
%     shorter version of the title of your paper\}}\\
%     \texttt{\textbackslash runningauthor\{Provide here the surnames of
%     the authors of your paper, all separated by commas\}}
%     \end{flushleft}
%     Note that the text that appears as argument in \texttt{\textbackslash
%       runningtitle} will be printed as a heading in the \emph{even}
%     pages. The text that appears as argument in \texttt{\textbackslash
%       runningauthor} will be printed as a heading in the \emph{odd}
%     pages.  If even the author surnames do not fit, it is acceptable
%     to give a subset of author names followed by ``et al.''

%     %\item Use the file sample\_paper.tex as an example.

%     \item The camera-ready versions of the accepted papers are \textbf{9
%       pages}, plus any additional pages needed for references and reproducibility checklist.

%     \item If you need to include additional appendices,
%       you can include them in the supplementary
%       material file.

%     \item Please, do not change the layout given by the above
%       instructions and by the style file.

% \end{enumerate}

% \subsubsection*{Acknowledgements}
% All acknowledgments go at the end of the paper, including thanks to reviewers who gave useful comments, to colleagues who contributed to the ideas, and to funding agencies and corporate sponsors that provided financial support. 
% To preserve the anonymity, please include acknowledgments \emph{only} in the camera-ready papers.


% \subsubsection*{References}

% References follow the acknowledgements.  Use an unnumbered third level
% heading for the references section.  Please use the same font
% size for references as for the body of the paper---remember that
% references do not count against your page length total.

\bibliography{sample_paper}

\clearpage
% \documentclass[twoside]{article}

% \usepackage{aistats2025}
% If your paper is accepted, change the options for the package
% aistats2025 as follows:
%
%\usepackage[accepted]{aistats2025}
%
% This option will print headings for the title of your paper and
% headings for the authors names, plus a copyright note at the end of
% the first column of the first page.

% If you set papersize explicitly, activate the following three lines:
%\special{papersize = 8.5in, 11in}
%\setlength{\pdfpageheight}{11in}
%\setlength{\pdfpagewidth}{8.5in}

% If you use natbib package, activate the following three lines:
%\usepackage[round]{natbib}
%\renewcommand{\bibname}{References}
%\renewcommand{\bibsection}{\subsubsection*{\bibname}}

% If you use BibTeX in apalike style, activate the following line:
%\bibliographystyle{apalike}

% \begin{document}

% If your paper is accepted and the title of your paper is very long,
% the style will print as headings an error message. Use the following
% command to supply a shorter title of your paper so that it can be
% used as headings.
%
%\runningtitle{I use this title instead because the last one was very long}

% If your paper is accepted and the number of authors is large, the
% style will print as headings an error message. Use the following
% command to supply a shorter version of the authors names so that
% they can be used as headings (for example, use only the surnames)
%
%\runningauthor{Surname 1, Surname 2, Surname 3, ...., Surname n}

% Supplementary material: To improve readability, you must use a single-column format for the supplementary material.
\onecolumn
\appendix
\aistatstitle{From Deep Additive Kernel Learning to Last-Layer \\ Bayesian Neural Networks via Induced Prior Approximation: \\
Supplementary Materials}

\section{SPARSE CHOLESKY DECOMPOSITION}
\label{sec:sparse chol decompose}
In this section, we present the algorithm for constructing the induced grids $\mathbf{U}$ as defined in \cref{eq:GPlayer} by using sorted dyadic points, and obtaining the sparse Choleksy decomposition of the Laplace kernel in one dimension, as proposed in \citep{ding2024sparse}.

A set of one-dimensional level-$L$ dyadic points $\Xv_L$ in increasing order over the interval $[0,1]$ is defined as:
\begin{align}
    \Xv_{L}:= \left\{ \frac{1}{2^{L}}, \frac{2}{2^{L}}, \frac{3}{2^{L}}, \ldots, \frac{2^{L}-1}{2^{L}} \right\}.
\end{align}
However, this increasing order does not yield a sparse representation of the Markov kernel $k(\cdot,\cdot)$ on the points $\Xv_L$, i.e., Cholesky decomposition of the covariance matrix $k(\Xv_L, \Xv_L)$ is not sparse. To achieve a sparse hierarchical expansion, we first sort the dyadic points $\Xv_L$ according to their levels.

\paragraph{Sorted Dyadic Points}
For level-$\ell$ dyadic points $\Xv_{\ell}$ where $ \ell=1,\ldots,L$, we first define the set $\rho(\ell)$ consisting of odd numbers as follows:
\begin{align}
    \rho(\ell) = \left\{ 1,3,5,\ldots,2^{\ell}-1 \right\}.
\end{align}
Next, we define the sorted incremental set $\Dv_{\ell}$ (with $\Xv_{0}:= \varnothing$) as:
\begin{align}
    \Dv_{\ell} = 
    \left\{ \frac{i}{2^{\ell}}: i\in \rho(\ell) \right\} = \Xv_{\ell} - \Xv_{\ell-1}, \quad  \ell=1,\ldots L.
\end{align}
Thus, the level-$L$ dyadic points $\Xv_L$ can be decomposed into disjoint incremental sets $\{ \Dv_{\ell} \}_{\ell=1}^{L}$:
\begin{align}
    \Xv_{L} = \cup_{\ell=1}^{L} \Dv_{\ell}, \quad \Dv_{i} \cap \Dv_{j} = \varnothing \text{ for $i\neq j$}.
\end{align}
Therefore, we can define the sorted level-$L$ dyadic points using these incremental sets as:
\begin{align}\label{eq:sorted dyadic}
    \Xv_{L}^{\text{sort}}:= \left\{ \Dv_1,\Dv_2, \ldots, \Dv_{L} \right\} 
    = \left\{ \frac{i \in \rho(\ell) }{2^{\ell}}, \ell=1,\ldots,L \right\}.
\end{align}
For example, the sorted level-3 dyadic points are given by:
\begin{align}
    \Xv_{3}^{\text{sort}} 
    = \bigg\{ 
    \begingroup
        \color{blue}
        \underbracket{
            \color{black}
            \frac{1}{2^1}
        }_{\color{blue}
            \Dv_1
        }
    \endgroup
    , 
    \begingroup
        \color{blue}
        \underbracket{
            \color{black}
            \frac{1}{2^2}, \frac{3}{2^2}
        }_{\color{blue}
            \Dv_2
        }
    \endgroup
    ,
    \begingroup
        \color{blue}
        \underbracket{
            \color{black}
            \frac{1}{2^3}, \frac{3}{2^3}, \frac{5}{2^3}, \frac{7}{2^3}
        }_{\color{blue}
            \Dv_3
        }
    \endgroup
     \bigg\}.
\end{align}

\paragraph{Algorithm}
We now present the algorithm for computing the inverse of the upper triangular Cholesky factor $[ \Lv_{\Xv_{L}^{\text{sort}}}^{\top} ]^{-1}$ of the covariance matrix $k(\Xv_{L}^{\text{sort}}, \Xv_{L}^{\text{sort}})$ in \Cref{alg:cholesky}, where $\Lv_{\Xv_{L}^{\text{sort}}} \Lv_{\Xv_{L}^{\text{sort}}}^{\top} = k(\Xv_{L}^{\text{sort}}, \Xv_{L}^{\text{sort}})$.. The corresponding proof can be found in \citep{ding2024sparse}. The output of \Cref{alg:cholesky} is a sparse matrix with $\Oc(3 \cdot (2^{L}-1))$ nonzero entries. Since each iteration of the for-loop only requires solving a $3 \times 3$ linear system, which costs $\Oc(3^3)$ time, the total computational complexity of \Cref{alg:cholesky} is $\Oc(2^L-1)$. This implies that the complexity of computing $\left[ \Lv_{\Uv}^{\top} \right]^{-1}$ in \cref{eq:GPlayer} is $\Oc(M)$ when $\Uv$, the induced grid of size $M$, consists of sorted dyadic points as defined in \cref{eq:sorted dyadic}.

\begin{algorithm}[hbt!]
\caption{Computation of the inverse Cholesky factor for the Markov kernel $k(\cdot, \cdot)$ on sorted one-dimensional level-$L$ dyadic points $\Xv_L^{\text{sort}}$.}
\label{alg:cholesky}
\setstretch{0.99} % set the line spacing to 0.99
\begin{algorithmic}[1]
    \STATE {\bfseries Input:} Markov kernel $k(\cdot,\cdot)$, sorted level-$L$ dyadic points $\Xv_{L}^{\text{sort}}$
    \STATE {\bfseries Output:} inverse of the upper triangular Cholesky factor $\Rv:= [ \Lv_{\Xv_{L}^{\text{sort}}}^{\top} ]^{-1}$, s.t. $\Lv_{\Xv_{L}^{\text{sort}}} \Lv_{\Xv_{L}^{\text{sort}}}^{\top} = k(\Xv_{L}^{\text{sort}}, \Xv_{L}^{\text{sort}})$
    \STATE Initialize $\Rv \leftarrow \text{zeros($2^L-1$,$2^L-1$)}$;
    \STATE Define $k(\pm \infty, \cdot) = k(\cdot, \pm \infty) = 0$;
    \FOR{$\ell=1$ {\bfseries to} $L$}
        \FOR{$i \in \rho(\ell)=\{1,3,\ldots,2^{\ell}-1\}$}
            \STATE $x_{\text{mid}} := \frac{i}{2^{\ell}}$;\quad
            $x_{\text{left}}:=\frac{i-1}{2^{\ell}}$ {\bfseries if} $i>1$ {\bfseries else} $-\infty$;\quad
            $x_{\text{right}}:=\frac{i+1}{2^{\ell}}$ {\bfseries if} $i<2^{\ell}-1$ {\bfseries else} $+\infty$;
            \STATE Get $i_{\text{mid}}$, $i_{\text{left}}$, $i_{\text{right}}$, the indices of the points $x_{\text{mid}}$, $x_{\text{left}}$, $x_{\text{right}}$ in the sorted set $\Xv_{L}^{\text{sort}}$ respectively;
            \STATE Get the coefficients $c_1$, $c_2$, $c_3$ by solving the following linear system:
            \begin{align}
                \begin{bmatrix}
                     & k(x_{\text{left}}, x_{\text{left}})
                     & k(x_{\text{left}}, x_{\text{mid}})
                     & k(x_{\text{left}}, x_{\text{right}}) \\
                     & k(x_{\text{mid}}, x_{\text{left}})
                     & k(x_{\text{mid}}, x_{\text{mid}})
                     & k(x_{\text{mid}}, x_{\text{right}}) \\
                     & k(x_{\text{right}}, x_{\text{left}})
                     & k(x_{\text{right}}, x_{\text{mid}})
                    &k(x_{\text{right}}, x_{\text{right}})
                \end{bmatrix}
                \begin{bmatrix}
                    c1\\
                    c2\\
                    c3
                \end{bmatrix}=
                \begin{bmatrix}
                    0\\
                    1\\
                    0
                \end{bmatrix}.
            \end{align}
            \STATE $[c_1,c_2,c_3] := [c_1,c_2,c_3] / \sqrt{c_2}$;
            \STATE {\bfseries if} $x_{\text{left}} \neq - \infty$, 
            {\bfseries then} $\Rv[i_{\text{left}} ,i_{\text{mid}}] = c_1$; \quad
            {\bfseries if} $x_{\text{right}} \neq + \infty$, 
            {\bfseries then} $\Rv[i_{\text{right}} ,i_{\text{mid}}] = c_3$;
            \STATE $\Rv[i_{\text{mid}} ,i_{\text{mid}}] = c_2$;
        \ENDFOR
    \ENDFOR
\end{algorithmic}
\end{algorithm}


\section{REPARAMETERIZATION OF KERNEL LENGTHSCALES}
\label{sec:theo}
Considering the additive Laplace kernel with fixed lengthscale $\tilde{\theta}$ for all base kernels, applying linear projections $\left\{ \wv_{p}^{\top}\xv \right\}_{p=1}^{P}$ on inputs $\xv\in \Rb^D$ will give:
\begin{align}
    &\sum_{p=1}^{P}\sigma^2_p k_p\left( \wv^{\top}_{p}\xv,\wv^{\top}_{p}\xv^{\prime} \right)\nonumber \\
    = & \sum_{p=1}^{P} \sigma^2_p\exp \left( -  \frac{\sum_{d=1}^{D} \left| w_{p,d}\left( x_{d}-x_{d}^{\prime} \right) \right|}{\tilde{\theta}} \right)\nonumber \\
    = & \sum_{p=1}^{P} \prod_{d=1}^{D} \sigma^2_p\exp \left( - \frac{\left| x_{d}-x_{d}^{\prime} \right|}{\tilde{\theta} / \left| w_{p,d}\right| } \right)\nonumber \\
    = & \sum_{p=1}^{P} \prod_{d=1}^{D} \sigma^2_p\exp \left( - \frac{\left| x_{d}-x_{d}^{\prime} \right|}{\theta_{p,d}} \right),
\end{align}
This still leads to an additive Laplace kernel but with adaptive lengthscale $\theta_{p,d}$ for base kernels. The resulting kernel also retains \emph{sparse} Cholesky decomposition by the properties of Markov kernels so that the complexity of inference is $\Oc(M)$.

\section{INFERENCE OF PREDICTIVE DISTRIBUTION}
\label{sec:uq of inference}
Given an input $\xv \in \Rb^D$, the prediction of the DAK model can be written in the following equation according to \cref{eq:DAK prediction}: 
\begin{align}
    \tilde{f}_{\xv}
    &= \sum_{p=1}^{P}
    \sigma_p \Big(
        \phi(h_{\psi}^{[p]}(\xv)) \zv_p
    \Big) + \mu \nonumber\\
    &= \sum_{p=1}^{P}
    \sigma_p \Big(
        \bm{\phi}_{p}^{\top} \zv_p
    \Big) + \mu,
\end{align}
where $\bm{\phi}_{p}^{\top}:=\phi(h_{\psi}^{[p]}(\xv)) \in \Rb^{1 \times M}$
% , $\mu_p:=\mu_p(h_{\psi}^{[p]}(\xv)) \in \Rb$
. We assume the variational distribution over the independent Gaussian weights $\zv_p \sim \Nc(\bm{m}_{\zv_p}, \Sv_{\zv_p})$ and the bias $\mu \sim \Nc(m_{\mu}, \sigma_{\mu}^2)$. Then it's straighforward to deduce that
\begin{align}
    \bm{\phi}_{p}^{\top} \zv_p + \mu 
    &\sim
    \Nc\left(
    \bm{\phi}_{p}^{\top} \bm{m}_{\zv_p} + m_{\mu},\hspace{0.2em}
    \bm{\phi}_{p}^{\top} \Sv_{\zv_p} \bm{\phi}_{p} + \sigma_{\mu}^2
    \right), \\
    \sigma_p \left(
    \bm{\phi}_{p}^{\top} \zv_p 
    \right) + \mu
    & \sim
    \Nc\left(
    \sigma_p ( \bm{\phi}_{p}^{\top} \bm{m}_{\zv_p} )+ m_{\mu} ,\hspace{0.2em}
    \sigma_p^2( \bm{\phi}_{p}^{\top} \Sv_{\zv_p} \bm{\phi}_{p}) + \sigma_{\mu}^2
    \right), \\
    \tilde{f}_{\xv} = 
    \sum_{p=1}^{P}
    \sigma_p \left(
    \bm{\phi}_{p}^{\top} \zv_p
    \right) + \mu
    & \sim
    \Nc\left(
    \sum_{p=1}^{P}
    \sigma_p ( \bm{\phi}_{p}^{\top} \bm{m}_{\zv_p}) + m_{\mu} ,\hspace{0.2em}
    \sum_{p=1}^{P}
    \sigma_p^2( \bm{\phi}_{p}^{\top} \Sv_{\zv_p} \bm{\phi}_{p} ) + \sigma_{\mu}^2
    \right).
\end{align}
Therefore, we obtain the predictive distribution of the $\tilde{f}(\xv)$ at the point $\xv \in \Rb^D$ and its mean and variance are given by:
\begin{subequations}
\label{eq:dak inference closed form}
\begin{align}
    \Eb\left[ \tilde{f}_{\xv} \right]
        = \sum_{p=1}^{P}
        \sigma_p ( \bm{\phi}_{p}^{\top} \bm{m}_{\zv_p}) + m_{\mu},
\end{align}
\begin{align}
    \text{Var}\left[ \tilde{f}_{\xv} \right]
        =\sum_{p=1}^{P}
        \sigma_p^2( \bm{\phi}_{p}^{\top} \Sv_{\zv_p} \bm{\phi}_{p}) + \sigma_{\mu}^2.
\end{align}
\end{subequations}
% \begin{subequations}
% \label{eq:dak inference closed form}
%     \begin{align}
%         \Eb\left[ \tilde{f}(\xv) \right]
%         = \sum_{p=1}^{P}
%         \sigma_p ( \bm{\phi}_{p}^{\top} \bm{m}_{\zv_p} + m_{\mu_p} ),
%     \end{align}
%     \begin{align}
%         \text{Var}\left[ \tilde{f}(\xv) \right]
%         =\sum_{p=1}^{P}
%         \sigma_p^2( \bm{\phi}_{p}^{\top} \Sigma_{\zv_p} \bm{\phi}_{p} + \sigma_{\mu_p}^2).
%     \end{align}
% \end{subequations}


\section{TRAINING OF VARIATIONAL INFERENCE}
\label{sec:training}
Given the dataset $\mathcal{D}=\{ \Xv, \yv \}$ where $\Xv:=\{ \xv_i \}_{i=1}^N$, $\yv=(y_1,\ldots,y_N)^{\top}$, $\xv_i \in \Rb^D$, $y_i\in\Rb$, the prediction $\tilde{f}_{\Xv}\in \Rb^N$ of DAK is given by all the parameters $\bm{\theta}=\left\{ \psi, \bm{\sigma} \right\}$, $\bm{\eta}=\left\{ \{ \mv_{\zv_{p}},\Sv_{\zv_{p}}\}_{p=1}^{P} , \{m_{\mu},\sigma_{\mu} \} \right\}$ according to \cref{eq:DAK prediction}:
\begin{align}
    \tilde{f}_{\Xv}:= \tilde{f}(\Xv; \bm{\theta}, \bm{\eta})
    = \sum_{p=1}^{P}
    \sigma_p \Big(
        \phi(h_{\psi}^{[p]}(\Xv)) \zv_p
    \Big) + \mu,
\end{align}
where $\zv_{p} \sim \mathcal{N} (\bm{m}_{\zv_p} ,\Sv_{\zv_p})$, $p=1,\ldots,P$, and $\mu \sim \mathcal{N} ( m_{\mu},\sigma^2_{\mu} )$ are variational variables $\Theta_{\text{var}}$ parameterized by $\bm{\eta}$. The variational distribution is denoted by $q_{\bm{\eta}}(\Theta_{\text{var}})= q(\mu)\prod_{p=1}^{P} q(\zv_{p}) = \Nc ( m_{\mu} ,\sigma_{\mu}^2 )\prod_{p=1}^{P} 
\Nc ( \bm{m}_{\zv_p} ,\Sv_{\zv_p} )$, and the variational prior is denoted by $p(\Theta_{\text{var}})$.

We consider the KL divergence between $q_{\bm{\eta}}(\Theta_{\text{var}})$ and the true posterior $p(\Theta_{\text{var}}\vert \yv, \Xv, \bm{\theta})$:
\begin{align}
& \qquad \text{KL} \left[ q_{\bm{\eta}}(\Theta_{\text{var}}) \| p(\Theta_{\text{var}} \vert \yv,\Xv, \bm{\theta} ) \right] \nonumber \\
= & \int q_{\bm{\eta}}(\Theta_{\text{var}} )\log \frac{q_{\bm{\eta}}(\Theta_{\text{var}} )}{p(\Theta_{\text{var}} \vert \yv,\Xv,\bm{\theta} )} d\Theta_{\text{var}} \nonumber \\
= & \int q_{\bm{\eta}}(\Theta_{\text{var}} )\log \frac{q_{\bm{\eta}}(\Theta_{\text{var}} )p(\yv \vert \Xv,\bm{\theta})}{p(\yv \vert \Xv,\bm{\theta} ,\Theta_{\text{var}} )p(\Theta_{\text{var}} )} d\Theta_{\text{var}} \nonumber \\
= & \int q_{\bm{\eta}}(\Theta_{\text{var}} )\log \frac{q_{\bm{\eta}}(\Theta_{\text{var}} )}{p(\Theta_{\text{var}} )} d\Theta_{\text{var}} -\int q_{\bm{\eta}}(\Theta_{\text{var}} )\log p(\yv \vert \tilde{f}_{\Xv} )d\Theta_{\text{var}} +\log p(\yv\vert \Xv,\bm{\theta}).
\end{align}
Using the fact that $\text{KL}[\cdot \| \cdot] \geq 0$, we have
\begin{align}
\label{eq:variational lower bound}
    \log p(\yv\vert \Xv,\bm{\theta}) & \geq \int q_{\bm{\eta}}(\Theta_{\text{var}} )\log p(\yv \vert \tilde{f}_{\Xv} )d\Theta_{\text{var}} - \text{KL} \left[ q_{\bm{\eta}}(\Theta_{\text{var}} ) \| p(\Theta_{\text{var}}) \right] \nonumber \\
    & = \Eb_{q_{\bm{\eta}}(\Theta_{\text{var}} )} \left[ \log p(\yv \vert \tilde{f}_{\Xv} ) \right] - \text{KL} \left[ q_{\bm{\eta}}(\Theta_{\text{var}} ) \| p(\Theta_{\text{var}}) \right].
\end{align}

\paragraph{Full-training.}
Firstly, we present the joint training of $\bm{\theta}$ and $\bm{\eta}$. The most common approach optimizes the marginal log-likelihood (the left-hand side of \cref{eq:variational lower bound}):
\begin{align}
    \bm{\theta}^{\ast} &=\argmax_{\bm{\theta}} \log p(\yv\vert \Xv,\bm{\theta} ) \\
    &= \argmax_{\bm{\theta}} \log \int p\left( y\vert X,\bm{\theta},\Theta_{\text{var}} \right) p(\Theta_{\text{var}})d\Theta_{\text{var}},
\end{align}
which involves intractable integral in some tasks such as classification. Instead, we optimize the variational lower bound (the right-hand side of \cref{eq:variational lower bound}):
\begin{align}
    \Theta^{\ast} := \argmax_{\bm{\theta},\bm{\eta}} \mathcal{L}(\bm{\theta},\bm{\eta}) =\argmax_{\bm{\theta},\bm{\eta}}\left\{ E_{q_{\bm{\eta}}(\Theta_{\text{var}} )}\left[ \log p(\yv|\tilde{f}_{\Xv} ) \right] -\text{KL} \left[ q_{\bm{\eta}}(\Theta_{\text{var}} )\| p(\Theta_{\text{var}} ) \right] \right\}.
\end{align}

\paragraph{Fine-tuning.}
An alternative training approach is to firstly pre-train the deterministic parameters of feature extractor by standard neural network training, with mean squared error for regression or cross-entropy for classification as the loss function, and then fine-tune the last layer additive GP with fixed features. The objective function is identical to \cref{eq:elbo}, but $\bm{\theta}$ is learned during the pre-training step and is no longer optimized during fine-tuning.


\section{ELBO}%{DERIVATION OF ELBO}
\label{sec:elbo}
\subsection{Assumptions}
Consider the model $y_i = \tilde{f}(\xv_i) + \epsilon_i$ with the i.i.d. noise $\epsilon_i \overset{\text{i.i.d.}}{\sim} \Nc(0, \sigma_{f}^2)$ and $\tilde{f} : \Rb^D \rightarrow \Rb$ is defined in \cref{eq:DAK prediction}. The training dataset is $\mathcal{D} = \{ \Xv, \yv \}$ where $\Xv:=\{ \xv_i \}_{i=1}^N$, $\yv=(y_1,\ldots,y_N)^{\top}$, $\xv_i \in \Rb^D$, $y_i\in\Rb$. $\Theta_{\text{var}}:= \{ \mu ,\{ \zv_{p}\}_{p=1}^{P} \}$ are the variational random variables consisting of Gaussian weights and bias of $P$ units, $\psi$ are the parameters of the NN, $\bm{\sigma}:=(\sigma_1, \ldots, \sigma_p)^{\top}$ are the scale parameters of base GP layers. The variational distributions are $q(\mu)=\Nc(m_{\mu}, \sigma_{\mu}^2)$, $q(\zv_p)=\Nc(\bm{m}_{\zv_p}, \Sv_{\zv_p})$ and the variational priors are $p(\mu)=\Nc(\check{m}_{\mu} ,\check{\sigma}^2_{\mu})$, $p(\zv_p)=\Nc(\check{\bm{m}}_{\zv_p} ,\check{\Sv}_{\zv_p})$. Note that $\Sv_{\zv_p}\in\Rb^{M \times M}$ is a diagonal covariance matrix due to the independence of $\zv_p$, $M$ is the number of inducing points $\Uv$ defined in \cref{eq:GPlayer}, and $\bm{m}_{\zv_p} \in \Rb^M$, $m_{\mu} \in \Rb$, $\sigma_{\mu}^2 \in \Rb$. We derive the ELBO in VI to learn the preditive posterior over the variational variables $\Theta_{\text{var}}:= \{ \mu ,\{ \zv_{p}\}_{p=1}^{P} \}$ parameterized by $\bm{\eta}:=\left\{ \{ \mv_{\zv_{p}},\Sv_{\zv_{p}}\}_{p=1}^{P} , \{m_{\mu},\sigma_{\mu} \} \right\}$, and optimize the deterministic parameters $\bm{\theta}:=\{\psi, \bm{\sigma}\}$.

\subsection{Expected Log Likelihood}
\paragraph{Closed Form}
The \emph{expected log likelihood}, which is the first term in ELBO defined in \cref{eq:elbo}, is given by 
\begin{align}
    {\Eb}_{q_{\bm{\eta}}(\Theta_{\text{var}})} \left[ \log \text{Pr} (\yv \vert \tilde{f}_{\Xv} ) \right]
    &= {\Eb}_{q_{\bm{\eta}}(\Theta_{\text{var}})} \left[ 
    \log \prod_{i=1}^{N} 
    p (y_i \vert \tilde{f}_{\xv_i} )
    \right] \nonumber\\
    &= \sum_{i=1}^{N} 
    {\Eb}_{q_{\bm{\eta}}(\Theta_{\text{var}})} \left[ 
    \log
    p (y_i \vert \tilde{f}_{\xv_i} )
    \right] \nonumber\\
    &= \sum_{i=1}^{N} 
    {\Eb}_{q_{\bm{\eta}}(\Theta_{\text{var}})} \left[ 
    \log
    \Nc( \tilde{f}_i,\hspace{0.2em} \sigma_{f}^2 )
    \right] \nonumber\\
    &= \sum_{i=1}^{N} 
    {\Eb}_{q_{\bm{\eta}}(\Theta_{\text{var}})} \left[ 
    \log \left(
    (2\pi \sigma_{f}^2)^{-\frac{1}{2}}
    \exp\left\{  
        -\frac{ (y_i - \tilde{f}_i)^2 }{2 \sigma_{f}^2}
    \right\}
    \right)
    \right] \nonumber\\
    &= \sum_{i=1}^{N} 
    {\Eb}_{q_{\bm{\eta}}(\Theta_{\text{var}})} \left[
    -\frac{1}{2} \log(2\pi) 
    - \frac{1}{2}\log(\sigma_{f}^2)
    - \frac{1}{2 \sigma_{f}^2}
    (y_i - \tilde{f}_i)^2
    \right] \nonumber\\
    &= - \frac{N}{2} \log(2\pi)
    - \frac{N}{2} \log(\sigma_{f}^2)
    - \frac{1}{2 \sigma_{f}^2}
    \sum_{i=1}^{N}
    {\Eb}_{q_{\bm{\eta}}(\Theta_{\text{var}})} \left[
    (y_i - \tilde{f}_i)^2
    \right] \nonumber\\
    &= - \frac{N}{2} \log(2\pi)
    - \frac{N}{2} \log(\sigma_{f}^2)
    - \frac{1}{2 \sigma_{f}^2}
    \sum_{i=1}^{N} \left(
    \left({\Eb}_{q(\Theta_{\text{var}})} \left[
    (y_i - \tilde{f}_i)
    \right] \right)^2
    + \text{Var}_{q(\Theta_{\text{var}})} \left[
    (y_i - \tilde{f}_i)
    \right]
    \right) \label{eq:evidence halfway},
\end{align}
where
\begin{align}
    \tilde{f}_i
    % \mu_{\tilde{f}_i} &:= \tilde{f}(\xv_i;\Theta_{\text{var}}, \Theta_{\text{det}} ) \nonumber\\
    &= \sum_{p=1}^{P} \sigma_p \Big(
    \begingroup
        \color{blue}
        \underbracket{
            \color{black}
            \phi(h_{\psi}^{[p]}(\xv_i))
        }_{\color{blue}
            :=\bm{\phi}_{i,p}^{\top} \in \Rb^{1 \times M}
        }
    \endgroup
    \zv_p
    \Big)
    + \mu
    % \begingroup
    %     \color{blue}
    %     \underbracket{
    %         \color{black}
    %         \mu_{p}(h_{\psi}^{[p]}(\xv_i))
    %     }_{\color{blue}
    %         :=\mu_{i,p} \in \Rb
    %     }
    % \endgroup 
    \nonumber\\
    &= \sum_{p=1}^{P} \sigma_p \left(
    \bm{\phi}_{i,p}^{\top} \zv_p 
    \right) + \mu.
\end{align}
Recall that the variational assumptions $q(\zv_p)=\Nc(\bm{m}_{\zv_p}, \Sv_{\zv_p})$ and $q(\mu)=\Nc(m_{\mu}, \sigma_{\mu}^2)$, we can infer that
\begin{align}
    \bm{\phi}_{i,p}^{\top} \zv_p + \mu 
    &\sim
    \Nc\left(
    \bm{\phi}_{i,p}^{\top} \bm{m}_{\zv_p} + m_{\mu},\hspace{0.2em}
    \bm{\phi}_{i,p}^{\top} \Sv_{\zv_p} \bm{\phi}_{i,p} + \sigma_{\mu}^2
    \right), \\
    \sigma_p \left(
    \bm{\phi}_{i,p}^{\top} \zv_p 
    \right) + \mu
    & \sim
    \Nc\left(
    \sigma_p ( \bm{\phi}_{i,p}^{\top} \bm{m}_{\zv_p} ) + m_{\mu},\hspace{0.2em}
    \sigma_p^2( \bm{\phi}_{i,p}^{\top} \Sv_{\zv_p} \bm{\phi}_{i,p} ) + \sigma_{\mu}^2
    \right), \\
    \tilde{f}_i = 
    \sum_{p=1}^{P}
    \sigma_p \left(
    \bm{\phi}_{i,p}^{\top} \zv_p 
    \right)+ \mu
    & \sim
    \Nc\left(
    \sum_{p=1}^{P}
    \sigma_p ( \bm{\phi}_{i,p}^{\top} \bm{m}_{\zv_p} )+ m_{\mu},\hspace{0.2em}
    \sum_{p=1}^{P}
    \sigma_p^2( \bm{\phi}_{i,p}^{\top} \Sv_{\zv_p} \bm{\phi}_{i,p} ) + \sigma_{\mu}^2
    \right), \\
    y_i - \tilde{f}_i
    & \sim 
    \Nc\left(
    y_i - 
    \sum_{p=1}^{P}
    \sigma_p ( \bm{\phi}_{i,p}^{\top} \bm{m}_{\zv_p} ) -m_{\mu},\hspace{0.2em}
    \sum_{p=1}^{P}
    \sigma_p^2( \bm{\phi}_{i,p}^{\top} \Sv_{\zv_p} \bm{\phi}_{i,p} ) + \sigma_{\mu}^2
    \right).
\end{align}
Therefore, 
\begin{subequations}\label{eq:exp and var in evidence}
    \begin{align}
        \left({\Eb}_{q(\Theta_{\text{var}})} \left[
        (y_i - \tilde{f}_i)
        \right] \right)^2
        = \left(
         y_i - 
        \sum_{p=1}^{P}
        \sigma_p ( \bm{\phi}_{i,p}^{\top} \bm{m}_{\zv_p} ) -m_{\mu}
        \right)^2,
    \end{align}
    \begin{align}
        \text{Var}_{q(\Theta_{\text{var}})}
        \left[
        (y_i - \tilde{f}_i)
        \right]
        = \sum_{p=1}^{P}
        \sigma_p^2( \bm{\phi}_{i,p}^{\top} \Sv_{\zv_p} \bm{\phi}_{i,p} ) + \sigma_{\mu}^2.
    \end{align}
\end{subequations}
By applying \cref{eq:exp and var in evidence} to \cref{eq:evidence halfway}, we derive the analytical formula for the expected evidence, expressed as
\begin{align}
    {\Eb}_{q_{\bm{\eta}}(\Theta_{\text{var}})} \left[ \log \text{Pr} (\yv \vert \tilde{f}_{\Xv} ) \right]
    &= - \frac{N}{2} \log(2\pi)
    - \frac{N}{2} \log(\sigma_{f}^2) \nonumber\\
    &- \frac{1}{2 \sigma_{f}^2}
    \sum_{i=1}^{N} \left(
        \Big(
         y_i - 
        \sum_{p=1}^{P}
        \sigma_p ( \bm{\phi}_{i,p}^{\top} \bm{m}_{\zv_p} ) -m_{\mu}
        \Big)^2
        + \sum_{p=1}^{P}
        \sigma_p^2( \bm{\phi}_{i,p}^{\top} \Sv_{\zv_p} \bm{\phi}_{i,p} )+ \sigma_{\mu}^2
    \right). \label{eq:evidence final}
\end{align}

\paragraph{Monte Carlo Approximation}
For comparison, we provide the equation for computing the Monte Carlo estimate of the ELBO in the paragraph that follows.
\begin{align}
    {\Eb}_{q_{\bm{\eta}}(\Theta_{\text{var}})} \left[ \log \text{Pr} (\yv \vert \tilde{f}_{\Xv} ) \right]
    % &= {\Eb}_{q(\Theta)} \left[ 
    % \log \prod_{i=1}^{N} 
    % p (y_i \vert \xv_i,\Theta, \psi, \bm{\sigma})
    % \right] \nonumber\\
    &= \sum_{i=1}^{N} 
    {\Eb}_{q_{\bm{\eta}}(\Theta_{\text{var}} )} \left[ 
    \log
    p (y_i \vert \tilde{f}_{\xv_i} )
    \right] \nonumber\\
    & \approx \sum_{i=1}^{N}
    \frac{1}{S}
     \sum_{s=1}^{S}
    \log
    p (y_i \vert \xv_i,\tilde{\Theta}^{(s)}_{\text{var}}, \bm{\theta} ) \nonumber\\
    &= \frac{1}{S} \sum_{i=1}^{N} 
    \sum_{s=1}^{S} 
    \log
    \Nc(y_i \left\vert\right. \tilde{f}_{i}^{(s)},\hspace{0.2em} \sigma_{f}^2 )
    \nonumber\\
    &= \frac{1}{S} \sum_{i=1}^{N} 
    \sum_{s=1}^{S} 
    \log \left(
    (2\pi \sigma_{f}^2)^{-\frac{1}{2}}
    \exp\left\{  
        -\frac{ (y_i - \tilde{f}_{i}^{(s)})^2 }{2 \sigma_{f}^2}
    \right\}
    \right)
    \nonumber\\
    &= \frac{1}{S} \sum_{i=1}^{N} 
    \sum_{s=1}^{S} \left(
    -\frac{1}{2} \log(2\pi) 
    - \frac{1}{2}\log(\sigma_{f}^2)
    - \frac{1}{2 \sigma_{f}^2}
    (y_i - \tilde{f}_{i}^{(s)})^2
    \right) \nonumber\\
    &= - \frac{N}{2} \log(2\pi)
    - \frac{N}{2} \log(\sigma_{f}^2)
    - \frac{1}{2 \sigma_{f}^2}
    \sum_{i=1}^{N}
    \frac{1}{S} \sum_{s=1}^{S}
    (y_i - \tilde{f}_{i}^{(s)})^2, \label{eq:evidence halfway mc approx}
\end{align}
where $S$ is the number of Monte Carlo samples, $\{  \tilde{\mu}^{(s)} ,\{ \tilde{\zv}_{p}^{(s)} \}_{p=1}^{P} \} := \tilde{\Theta}^{(s)}_{\text{var}}$ are the $s$-th Monte Carlo samplings over the variational parameters $\Theta_{\text{var}}$ and $\tilde{\Theta}^{(s)}_{\text{var}} \sim q_{\bm{\eta}}(\Theta_{\text{var}})$, $\tilde{f}_{i}^{(s)}$ is given as follows:
\begin{align}
    \tilde{f}_{i}^{(s)} &:= \tilde{f}(\xv_i;\tilde{\Theta}^{(s)}_{\text{var}},\bm{\theta} ) \nonumber\\
    &= \sum_{p=1}^{P} \sigma_p \Big(
    \begingroup
        \color{blue}
        \underbracket{
            \color{black}
            \phi(h_{\psi}^{[p]}(\xv_i))
        }_{\color{blue}
            :=\bm{\phi}_{i,p}^{\top} \in \Rb^{1 \times M}
        }
    \endgroup
    \tilde{\zv}_p^{(s)} 
    \Big) + \tilde{\mu}^{(s)} \nonumber\\
    &= \sum_{p=1}^{P} \sigma_p \left(
    \bm{\phi}_{i,p}^{\top} \tilde{\zv}_p^{(s)} 
    \right)+ \tilde{\mu}^{(s)}. \label{eq:mc approx mean}
\end{align}
Therefore, we plug \cref{eq:mc approx mean} into \cref{eq:evidence halfway mc approx} and get the the Monte Carlo estimate of the ELBO written in the following formula:
\begin{align}
    {\Eb}_{q_{\bm{\eta}}(\Theta_{\text{var}})} \left[ \log \text{Pr} (\yv \vert \tilde{f}_{\Xv} ) \right]
    &\approx
    - \frac{N}{2} \log(2\pi)
    - \frac{N}{2} \log(\sigma_{f}^2)
    - \frac{1}{2 \sigma_{f}^2}
    \sum_{i=1}^{N}
    \frac{1}{S} \sum_{s=1}^{S}
    \Big(y_i - 
    \sum_{p=1}^{P} \sigma_p \left(
    \bm{\phi}_{i,p}^{\top} \tilde{\zv}_p^{(s)} 
    \Big)- \tilde{\mu}^{(s)}
    \right)^2, \label{eq:evidence final mc approx} \\
    \tilde{\zv}_p^{(s)} &\sim \Nc(\bm{m}_{\zv_p}, \Sv_{\zv_p}),\qquad
    \tilde{\mu}^{(s)} \sim \Nc(m_{\mu}, \sigma_{\mu}^2).
\end{align}


\subsection{KL Divergence}
Since we place Gaussian assumptions over the variational parameters $\Theta_{\text{var}}$,  the \emph{KL divergence}, which is the second term in ELBO defined in \cref{eq:elbo}, is then given by
\begin{align}
    \text{KL} \left[ q(\Theta_{\text{var}} ) \| p(\Theta_{\text{var}}) \right]
    &= \text{KL} \left[ q( \mu ,\{ \zv_{p}\}_{p=1}^{P} ) \Vert p( \mu ,\{ \zv_{p}\}_{p=1}^{P}) \right] \nonumber\\
    & =  
    \text{KL} \left[ q(\mu) \Vert p(\mu) \right] 
    + \sum_{p=1}^{P} 
    \text{KL} \left[ q(\zv_{p}) \Vert p(\zv_{p}) \right],
\end{align}

\begin{align}
     \text{KL} \left[ q(\mu) \Vert p(\mu) \right]
     = \frac{1}{2} \left(
     \frac{\sigma_{\mu}^2}{\check{\sigma}_{\mu}^2} 
     + \frac{(m_{\mu} - \check{m}_{\mu})^2}{\check{\sigma}_{\mu}^2} 
     -\log\left( \frac{\sigma_{\mu}^2}{\check{\sigma}_{\mu}^2} \right)
     -1
     \right),
\end{align}

\begin{align}
    \text{KL} \left[ q(\zv_{p}) \Vert p(\zv_{p}) \right]
    = \frac{1}{2} \sum_{i=1}^{M} \left(
     \frac{[\Sv_{\zv_p}]_{ii}}{[\check{\Sv}_{\zv_p}]_{ii}} 
     + \frac{([\bm{m}_{\zv_p}]_{i} - [\check{\bm{m}}_{\zv_p}]_i)^2}{[\check{\Sv}_{\zv_p}]_{ii}}
     -\log\left( 
     \frac{[\Sv_{\zv_p}]_{ii}}{[\check{\Sv}_{\zv_p}]_{ii}}  
     \right)
     -1
     \right),
\end{align}
where $[\Sv_{\zv_p}]_{ii}$ is the $(i,i)$-th element of the diagonal covariance matrix $\Sv_{\zv_p} \in \Rb^{M \times M}$, $[\bm{m}_{\zv_p}]_{i}$ is the $i$-th element of the mean vector $\bm{m}_{\zv_p} \in \Rb^M$, the approximated posteriors are $q(\mu)=\Nc(m_{\mu}, \sigma_{\mu}^2)$, $q(\zv_p)=\Nc(\bm{m}_{\zv_p}, \Sv_{\zv_p})$ and the priors are $p(\mu)=\Nc(\check{m}_{\mu} ,\check{\sigma}^2_{\mu})$, $p(\zv_p)=\Nc(\check{\bm{m}}_{\zv_p} ,\check{\Sv}_{\zv_p})$.

% \subsection{Performance Comparison}
% \label{sec:toy exp compare}
% We compare the perforamce of computing the ELBO in \cref{eq:elbo} by using closed form in \cref{eq:evidence final} and using Monte Carlo approximation in \cref{eq:evidence final mc approx} in a toy example.
% \textcolor{red}{Table or Figure to add if time available}


\subsection{Limitations of the Closed-Form ELBO}

The closed-form ELBO is only applicable to regression problems. In classification, applying the softmax function to $\tilde{f}(\xv;\bm{\theta}, \bm{\eta})$ results in a non-analytic predictive distribution, meaning the ELBO must still be computed via Monte Carlo sampling during training. Similarly, the closed-form expressions for the predictive mean and variance, as provided in \cref{eq:dak inference closed form} in \Cref{sec:uq of inference}, are not applicable to classification but only apply to regression problems.


\section{COMPUTATIONAL COMPLEXITY}
\label{sec:complexity}
In this section, we discuss the computational complexity of various DKL models compared to the proposed DAK method, focusing on the GP layer as the most computationally demanding component. \Cref{tab:complexity supp} shows the computational complexity of our model compared to other state-of-the-art GP and DKL methods.

\begin{table}[ht]
    \caption{Computational complexity of the DKL models for $N$ training points. The reported training complexity is for one iteration. $\hat{M}$ is the number of inducing points in SVGP and KISS-GP, while $M$ is the size of induced grids in DAK, $M < \hat{M}$. $S$ is the number of Monte Carlo samples, $B$ is the size of mini-batch, $D_w$ is the dimension of the NN outputs in DKL, $P$ is the dimension of the outputs after applying linear transformations to the NN outputs in the proposed DAK model. DAK-MC refers to the DAK model using Monte Carlo approximation, while DAK-CF refers to the DAK model using closed-form inference and ELBO.}
    \centering
    \begin{tabular}{lcc}
    \toprule[1pt]
                  & \textbf{Inference}       & \textbf{Training} (per iteration) \\
    \midrule[0.5pt]
    NN + SVGP     & $\Oc(\hat{M}^2 N)$    & $\Oc( S D_w MB + \hat{M}^3)$ \\
    NN + KISS-GP  & $\Oc(D_w \hat{M}^{1+\frac{1}{D_w}})$  & $\Oc(S D_w MB + D_w \hat{M}^{\frac{3}{D_w}})$ \\
    DAK-MC (ours) & $\Oc(SM)$       & $\Oc(SPMB + PM)$   \\
    DAK-CF (ours) & $\Oc(M)$        & $\Oc(PMB + PM)$    \\
    \bottomrule[1pt]
    \end{tabular}
    \label{tab:complexity supp}
\end{table}

\paragraph{Inference Complexity.}
In inference based on induced approximation, computing the multiplication of the inverse of the covariance matrix $k(\Uv, \Uv)$ and a vector takes $\Oc(\hat{M}^2N)$ time for $\hat{M}$ inducing points $\Uv$ and $N$ training points when using SVGP. This cost is reduced by KISS-GP to $\Oc(D \hat{M}^{1+\frac{1}{D}})$ by decomposing the covariance matrix into a Kronecker product of $D$ one-dimensional covariance matrices of the inducing points: $k(\Uv, \Uv) = \bigotimes_{d=1}^{D} k(\Uv^{[d]}, \Uv^{[d]})$. Despite the significant reduction on complexity, it requires inducing points $\Uv$ arranged on a Cartesian grid of size $\hat{M} = \prod_{d=1}^{D} \hat{M}_d$, where $\hat{M}_d$ is the number of inducing points in the $d$-th dimension. In high-dimensional spaces, fixing $\hat{M}$ leads to very small $\hat{M}_d$ per dimension, which can degrade model performance. To address this, we propose the DAK model via sparse finite-rank approximation, which employs an additive Laplace kernel for GPs. The inverse Cholesky factor $\Lv_{\Uv}^{\top}$ for one-dimensional induced grids $\Uv$ of size $M$, where $M < \hat{M}$, as defined in \cref{eq:GPlayer}, is sparse and can be computed in $\Oc(M)$ time.

\paragraph{Training Complexity.}
In training, VI requires computing the ELBO as described in \cref{eq:elbo}, which consists of two terms: the \emph{expected log likelihood} and the \emph{KL divergence} between the variational distributions and priors. 

1) The \emph{expected log likelihood} is usually approximated via Monte Carlo sampling at a cost of $\Oc(S N_{\Theta} N)$, where $S$ is the number of Monte Carlo samples, $N_{\Theta}$ is the total number of variational parameters $\Theta_{\text{var}}$, and $N$ is the number of training points. This complexity can be reduced to $\Oc(S N_{\Theta} B)$ by applying stochastic variational inference with a mini-batch of size $B \ll N$. For DKL models using SVGP and KISS-GP, $\Theta_{\text{var}}$ are inducing variables, and the expectation does not have a closed form, requiring Monte Carlo sampling. In contrast, in the proposed DAK model, $\Theta_{\text{var}}= \{ \{ \zv_{p}\}_{p=1}^{P}, \mu \}$ consists of independent Gaussian weights $\zv_p\in \Rb^M$ and bias $\mu$. This allows us to derive an analytical form for this term, as shown in \cref{eq:evidence final} in \Cref{sec:elbo}, reducing the computational cost to $\Oc(N_{\Theta} B) = \Oc(PM B)$ when using a mini-batch of size $B$.

2) The \emph{KL divergence} between two Gaussian distributions can be computed in closed form. This leads to a linear time complexity of $\Oc(N_{\Theta})$ if the parameters $\Theta_{\text{var}}$ are independent, or cubic time $\Oc(N_{\Theta}^3)$ if they are fully correlated. In SVGP and KISS-GP, $\Theta_{\text{var}}$ represents fully correlated Gaussian distributed inducing variables, so computing the KL divergence takes $\Oc(\hat{M}^3)$ for SVGP. In KISS-GP, this can be reduced to $\Oc(D \hat{M}^{\frac{3}{D}})$ using fast eigendecomposition of Kronecker matrices. In the DAK model, the weights $\{\zv_p\}_{p=1}^{P}$ as defined in \cref{eq:GPlayer} are independent Gaussian random variables, allowing the KL divergence to be computed in $\Oc(N_{\Theta}) = \Oc(PM)$ time, where $P$ is the number of base GP layers.


\section{ADDITIONAL DISCUSSIONS}

Although interpretability is one advantage of additive models, the main motivation for replacing a GP layer with an additive GP layer in our work is to handle high-dimensional data. When the input dimension is low, it is reasonable that GPs are superior to additive GPs since the additive kernel is an approximated and restrictive kernel. However, when the input dimension increases, the computational complexity grows considerably even in GPs with sparse approximation. For example, in DKL, the output dimension of NN encoder is usually chosen as small as 2, while in pixel data experiments, DKL cannot handle the computation associated with the dimensionality when the output dimension of ResNet is 512 or more. Although DKL is superior in low-dimensional and simple cases, we view additive structure as a necessary component to achieve scalability and good performance with high-dimensional data.

\subsection{Why choosing the induced grids instead of learning the inducing points?}

From an approximation accuracy point of view, there are two separate strategies to increase the accuracy. The first one is to learn the inducing point locations. The second one, however, is to simply increase the number of inducing points on a pre-specified finer grid. The second method is much easier to implement and has a theoretical guarantee by the GP regression theory: as the inducing points become dense in the input region, the approximation will become exact. In contrast, the first approach does not have such a favorable theoretical guarantee. 

The second approach would become difficult to use for many existing methodologies as in general the computational cost would scale as $\mathcal{O}(M^3)$ with $M$ inducing points, which is particularly problematic in high dimensions. 
% The first approach can be viewed as a compromise in those situations, and that is why many existing methods chose to learn the locations of the inducing points instead.
This difficulty is resolved by additive GPs, since approximating an additive GP boils down to approximating one dimensional GPs, which can be accomplished by using a set of pre-specified inducing points on a fine grid in 1-D. One major benefit of the proposed methodology is that the computation now scales at $\mathcal{O}(M)$, enabled by the Markov kernel and the additive kernel. Therefore, a large number of inducing points can be used in an efficient way. 

The proposed method also has several additional benefits: 1) It can decouple to some extent the neural network component and GP component by avoiding learning the inducing points, which may help reduce overfitting/overconfidence; 2) The equivalence to BNN holds exactly with the fixed inducing points, whereas for learned inducing points, this BNN equivalence breaks down, and the proposed computation/training framework would not be possible to carry through; 3) It can simplify the overall optimization since there is no need to learn the inducing points.

\subsection{Limitations and future directions}

Generally, a finer grid will lead to better approximations, but the number of parameters to be trained will also increase. Therefore, there is a trade-off between the accuracy and the computational cost that we can afford. This current work is using a specific Laplace kernel, which can utilize sparse Cholesky decomposition. More general kernels may result in more computational complexity but better representation power of the model. In addition, the current variational family is restricted under mean-field assumptions. A more general variational family, e.g. full/low-rank covariance, may lead to superior performance in some applications. 


\section{EXPERIMENTAL DETAILS}
\label{sec:expdetail}
In this section, we provide additional details regarding the experiments.

\subsection{Benchmarks for Regression}
\label{subsec:regression supp}
\paragraph{Experiment Setup}
For all models, the NN architecture is a fully connected NN with rectified linear unit (ReLU) activation function \citep{nair2010rectified} and two hidden layers containing 64 and 32 neurons, respectively, structured as $D \rightarrow 64 \rightarrow 32 \rightarrow D_w$, where $D$ is the input feature size (also the size of input $\Xv$) and $D_w$ is the output feature size. The models are evaluated with $D_w=16$, 64, and 256, respectively. The number of Monte Carlo samples is set to 8 during training and 20 during inference.

The NN is a deterministic model, and we use the negative Gaussian log-likelihood as the loss function to quantify the uncertainty of the NN outputs and compute the NLPD.

For NN+SVGP, the inducing points are set to the size of 64 in $D_w$ dimension. We implement the \texttt{ApproximateGP} model in GPyTorch \citep{gardner2018gpytorch}, defining the inducing variables as variational parameters, and use \texttt{VariationalELBO} in GPyTorch to perform variational inference and compute the loss.

SV-DKL is originally designed for classification, so for a fair comparison in regression tasks, we modify it by first applying a linear embedding layer $\Wv: \Rb^{D_w} \rightarrow \Rb^P$ with $P=16$ and normalizing the outputs to the interval $[0,1]$ for each base GP, similar to the DAK model. To adapt the additive GP layer for regression, we remove the softmax function from the model in eq. (1) of \citep{wilson2016stochastic}. Given training data $\{ \xv_i, \yv_i \}_{i=1}^{N}$, the model is modified as follows:
\begin{align}
    p(\yv_i \vert \fv_i, A) = \mathcal{A}(\fv_i)^{\top} \yv_i
\end{align}
where $\fv_i \in \Rb^P$ is a vector of independent GPs followed by a linear mixing layer $\mathcal{A}(\fv_i) = A \fv_i$, with $A \in \Rb^{C \times P}$ as the transformation matrix. Here, $C=1$ for single-task regression. For each $p$-th GP ($1 \leq p \leq P$) in the additive GP layer, the corresponding inducing variables $\uv_p$ are set to the size of 64 and treated as variational parameters for training. We use the \texttt{GridInterpolationVariationalStrategy} model with \texttt{LMCVariationalStrategy} in GPyTorch to perform KISS-GP with variational inducing variables, augmented by a linear mixing layer.

For AV-DKL, the inducing points are set to size of 64 in $D_{w}$ dimension. We implement the AV-DKL model based on the source code~\cite{matias2024amortized}.

Both DAK-MC and DAK-CF use the same additive GP layer size as SV-DKL, with $P=16$, and employ fixed induced grids $\Uv = \{1/8, 2/8, \ldots, 7/8\}$ of size 7 for each base GP, which is much smaller than that of SV-DKL.

\paragraph{Metrics}
Let $\{\xv_t, y_t\}_{t=1}^{T}$ represent a test dataset of size $T$, where $\mu_t$ and $\sigma_t^2$ are the predictive mean and variance. We evaluate model performance using two common metrics: Root Mean Squared Error (RMSE) and Negative Log Predictive Density (NLPD).

RMSE is widely used to assess the accuracy of predictions, measuring how far predictions deviate from the true target values. It is calculated as:
\begin{align}
    \text{RMSE} = \sqrt{ \frac{1}{T} \sum_{t=1}^{T}(y_t - \mu_t)^2 }.
\end{align}

NLPD is a standard probabilistic metric for evaluating the quality of a model's uncertainty quantification. It represents the negative log likelihood of the test data given the predictive distribution. For GPs, NLPD is calculated as:
\begin{align}
    \text{NLPD}
    &= - \sum_{t=1}^{T} \log p(y_t = \mu_t \vert \xv_t) \\
    &= \frac{1}{T}
    \sum_{t=1}^{T} \Big[
    \frac{(y_t - \mu_t)^2}{2\sigma_t^2} + \frac{1}{2} \log(2\pi \sigma_t^2)
    \Big].
\end{align}
Both RMSE and NLPD are widely used in the GP regression literature, where smaller values indicate better model performance.

\paragraph{Computing Infrastructure}
The experiments for regression were run on Macbook Pro M1 with 8 cores and 16GB RAM.

\subsection{Benchmarks for Classification}
\label{subsec:classification supp}
We use PyTorch \citep{paszke2019pytorch} baseline of NN models, GPyTorch \citep{gardner2018gpytorch} baseline of SVGP and SV-DKL models. In classification tasks, we apply a softmax likelihood to normalize the output digits to probability distributions. The NN is a deterministic model trained via negative log-likelihood loss, while DKL and DAK models are trained via ELBO loss. The setting of all training tasks are described in \Cref{tab:model classification} and \Cref{tab:optimizer classification}.

SVGP is originally designed for single-output regression. To make it fit for multi-output classification, we used \texttt{IndependentMultitaskVariationalStrategy} in GPyTorch to implement the multi-task \texttt{ApproximateGP} model, and use \texttt{VariationalELBO} with \texttt{SoftmaxLikelihood} in GPyTorch to perform variational inference and compute the loss. 

For SV-DKL, we employed the same \texttt{VariationalELBO} with \texttt{SoftmaxLikelihood} as the variational loss objective. \texttt{GridInterpolationVariationalStrategy} is applied within \texttt{IndependentMultitaskVariationalStrategy} to perform additive KISS-GP approximation. For each KISS-GP unit, we used $64$ variational inducing points initialized on a grid of size $[-1,1]$. 

For DAK, we implemented DAK-MC using Monte Carlo estimation given the intractable softmax likelihood. We employed fixed induced grids $\Uv=\{ -31/32, -30/32, \ldots, 30/32, 31/32 \}$ of size 63 for each base GP component.

\begin{table}[ht]
\caption{Model architectures for image classification on MNIST, CIFAR-10 and CIFAR-100.}
\centering
\resizebox{0.7\linewidth}{!}{
\begin{tabular}{l|l|ccc}
\toprule[1pt]
Model                   & Hyper-parameter          & MNIST       & CIFAR-10    & CIFAR-100   \\
\midrule[0.5pt]
\multirow{4}{*}{NN+SVGP}   & Feature extractor        & CNN         & ResNet-18   & ResNet-34   \\
                        & NN out features $D_w$         & 128         & 512         & 512         \\
                        & Embedding features $P$               & 16          & 64          & 128         \\
                        & \# inducing points $\hat{M}$      & 512         & 512         & 512         \\
                        & \# epochs       & 20         & 200         & 200         \\
                        & Training strategy      & Full-training         & Full-training         & Fine-tuning         \\
\midrule[0.5pt]
\multirow{5}{*}{SV-DKL} & Feature extractor        & CNN         & ResNet-18   & ResNet-34   \\
                        & NN out features $D_w$         & 128         & 512         & 512         \\
                        & Embedding features $P$               & 16          & 64          & 128         \\
                        & \# inducing points $\hat{M}$      & 64          & 64          & 64          \\
                        & Grid bounds              & {[}-1,1{]} & {[}-1,1{]} & {[}-1,1{]} \\
                        & \# epochs       & 20         & 200         & 200         \\
                        & Training strategy       & Full-training         & Full-training         & Fine-tuning         \\
\midrule[0.5pt]
\multirow{4}{*}{DAK}    & Feature extractor        & CNN         & ResNet-18   & ResNet-34   \\
                        & NN out features $D_w$         & 128         & 512         & 512         \\
                        & Embedding features $P$               & 16          & 64          & 128         \\
                        & \# induced interpolation $M$ & 63          & 63          & 63         \\
                        & \# epochs       & 20         & 200         & 200         \\
                        & Training strategy      & Full-training         & Full-training         & Full-training         \\
\bottomrule[1pt]
\end{tabular}

}
\label{tab:model classification}
\end{table}

\paragraph{MNIST} We used a CNN implemented in PyTorch as the feature extractor: \texttt{Conv2d}(1,32,3) $\rightarrow$ \texttt{Conv2d}(32,64,3) $\rightarrow$ \texttt{MaxPool2d}(2) $\rightarrow$ \texttt{Dropout}(0.25) $\rightarrow$ \texttt{Linear}(9216,128) $\rightarrow$ \texttt{Dropout}(0.5). To make a fair comparison, for both SV-DKL and DAK, we applied an embedding module through a linear layer that transform $128$ output features into $P=16$ base GP channels. 

\paragraph{CIFAR-10} We used a ResNet-18 as the feature extractor followed by a linear embedding layer that compressed the $512$ output features into $P=64$ base GP channels. 

\paragraph{CIFAR-100} We used a pretrained ResNet-34 as the feature extractor for SV-DKL and fine-tuned GP output layers since SV-DKL struggled to fit using full-training. For proposed DAK, we used full-training. The number of base GP channels is selected as $P=128$. 

\begin{table}[ht]
\caption{Details of training optimizer for image classification on MNIST, CIFAR-10 and CIFAR-100.}
\centering
\resizebox{0.7\linewidth}{!}{

\begin{tabular}{l|ccc}
\toprule[1pt]
Optimization      & MNIST                                                             & CIFAR-10                                                                                                  & CIFAR-100                                                                                                 \\
\midrule[0.5pt]
Optimizer         & Adadelta                                                          & SGD                                                                                                       & SGD                                                                                                       \\
Initial lr.       & 1.0                                                               & 0.1                                                                                                       & 0.1                                                                                                       \\
Weight decay      & 0.0001                                                            & 0.0001                                                                                                    & 0.0001                                                                                                    \\
Scheduler         & StepLR                                                            & CosineAnnealingLR                                                                                         & CosineAnnealingLR                                                                                         \\
\midrule[0.5pt]
Data Augmentation & MNIST                                                             & CIFAR-10                                                                                                  & CIFAR-100                                                                                                 \\
\midrule[0.5pt]
RandomCrop        & -                                                                 & size=32, padding=4                                                                                        & size=32, padding=4                                                                                        \\
HorizontalFlip    & -                                                                 & p=0.5                                                                                                     & p=0.5                                                                                                     \\
% Normalization     & \begin{tabular}[c]{@{}l@{}}mean=0.1307,\\ std=0.3081\end{tabular} & \begin{tabular}[c]{@{}l@{}}mean={[}0.4914,0.4822,0.4465{]},\\ std={[}0.2023,0.1994,0.2010{]}\end{tabular} & \begin{tabular}[c]{@{}l@{}}mean={[}0.5071,0.4867,0.4408{]},\\ std={[}0.2675,0.2565,0.2761{]}\end{tabular} \\
\bottomrule[1pt]
\end{tabular}
}
\label{tab:optimizer classification}
\end{table}

\paragraph{Additional Benchmark.}  \citet{matias2024amortized} proposed Amortized Variational DKL (AV-DKL), which is a variant SV-DKL using amortization network to compute the inducing locations and variational parameters, thus attenuating the overcorrelation of NN extracted features. AV-DKL is included as the additional benchmark for classification tasks in \Cref{tab:img avdkl}. The training recipe is the same with SV-DKL. 


\begin{table*}[ht]
\caption{\small{Accuracy, NLL, ECE for AV-DKL, SV-DKL, DAK-MC on CIFAR-10/100 averaged over 3 runs. CIFAR-10 uses ResNet-18 with 64 features extracted; CIFAR-100 uses ResNet-34 with 512 features. The best results are highlighted in \textbf{bold}; the second best results are highlighted by \underline{underline}.}}
\centering
\vspace{-0.1cm}
\resizebox{\linewidth}{!}{%
\begin{tabular}{rccclccc}
\toprule[1pt]
\multicolumn{1}{l}{} & \multicolumn{3}{c}{Batch size: 128}  &  & \multicolumn{3}{c}{Batch size: 1024} \\ \cline{2-4} \cline{6-8} \vspace{-8pt} \\
\multicolumn{1}{l}{} & AV-DKL & SV-DKL & \cellcolor{Gray} DAK-MC &   & AV-DKL  & SV-DKL & \cellcolor{Gray} DAK-MC \\ 
\midrule[1pt]
CIFAR-10 - Acc. (\%) $\uparrow$    & \underline{94.23 $\pm$ 0.65}  & 93.44 $\pm$ 0.28    &  \cellcolor{Gray} \textbf{94.81 $\pm$ 0.13}   &     &  \textbf{93.32} $\pm$ \textbf{0.13}        & 90.22 $\pm$ 1.42       & \cellcolor{Gray} \underline{93.02 $\pm$ 0.18}        \\
NLL $\downarrow$     & 0.352 $\pm$ 0.084    & \underline{0.312 $\pm$ 0.033}       &  \cellcolor{Gray} \textbf{0.256} $\pm$ \textbf{0.014}     &      & \underline{0.439 $\pm$ 0.022}         & 0.485 $\pm$ 0.061       & \cellcolor{Gray} \textbf{0.345 $\pm$ 0.001}    \\
ECE $\downarrow$      & 0.048 $\pm$ 0.006    & \underline{0.046 $\pm$ 0.003}       &  \cellcolor{Gray} \textbf{0.039 $\pm$ 0.002}          &     & \underline{0.054 $\pm$ 0.001}       & 0.060 $\pm$ 0.004       & \cellcolor{Gray} \textbf{0.052 $\pm$ 0.001}           \\
\midrule[1pt]
CIFAR-100 -  Acc. (\%) $\uparrow$    & \textbf{77.47 $\pm$ 0.19}  & 74.52 $\pm$ 0.13       & \cellcolor{Gray}  \underline{76.75 $\pm$ 0.18}     &     &  \textbf{77.07 $\pm$ 0.10}        & 66.54 $\pm$ 0.74       & \cellcolor{Gray} \underline{70.38 $\pm$ 1.25}        \\
NLL $\downarrow$     & 1.787 $\pm$ 0.011    & \underline{1.041 $\pm$ 0.007}       & \cellcolor{Gray}  \textbf{1.001 $\pm$ 0.027}     &      & 2.326 $\pm$ 0.030    & \underline{1.738 $\pm$  0.058}      & \cellcolor{Gray} \textbf{1.203 $\pm$ 0.040}        \\
ECE $\downarrow$      & 0.166 $\pm$ 0.002    & \underline{0.049 $\pm$ 0.002}       & \cellcolor{Gray}  \textbf{0.041 $\pm$ 0.004}        &     & 0.175 $\pm$ 0.001         & \underline{0.148 $\pm$ 0.007}       &\cellcolor{Gray}  \textbf{0.056 $\pm$ 0.006}           \\
\bottomrule[1pt]
\end{tabular}
}
\vspace{-0.2cm}
\label{tab:img avdkl}
\end{table*}

\paragraph{Metrics} 
We evaluate model performance using four common metrics: Top-1 accuracy, ELBO, Negative Log Likelihood (NLL), and Expected Calibration Error (ECE). 

ECE is a metric used to quantify the degree of ``calibration'' of a probabilistic model in UQ, specifically for classification problems. It is defined as the weighted average of the absolute difference between the model's predicted probability (confidence) and the actual outcome (accuracy) over several bins of predicted probability. Mathematically, ECE is given by:
\begin{align}
    \text{ECE} =\sum_{m=1}^{M} \frac{\left| B_{m} \right|}{n} \left| \text{acc} (B_{m})-\text{conf} (B_{m}) \right|,
\end{align}
where $M$ is the number of bins into which the confidence values are partitioned, $B_m$ is the set of indices of samples whose predicted confidence falls into the $m$-th bin, $n$ is the total number of samples.

\paragraph{Computing Infrastructure}
The experiments for classification were run on a Linux machine with NVIDIA RTX4080 GPU, and 32GB of RAM.




\subsection{Additional Tables and Figures}
\label{sec:additional exp results}

\paragraph{Choices of learning rates.}
We evaluate the choices of learning rates on 1D regression examples. DKL requires a separate tuning of the learning rate of the GP covariance parameters, which differs from the learning rate of the NN feature extractor. In \Cref{fig:dkl lr}, we choose the learning rate of the NN feature extractor as $0.01$, while the learning rate of the GP covariance is set to different values. (a)-(c) show that different learning rates of covariance in DKL result in different predictive posterior. In particular, although the training losses for DKL in both (a) and (b) are minimal, the regressions do not fit well. On the other hand, DAK does not need a distinct recipe for tuning GP covariances because of the BNN interpretation. Furthermore, the poor posterior is indicated by the higher training loss, as illustrated in (d)-(f).

\begin{figure}[ht]
\centering
\subfloat[$\begin{gathered}\text{DKL: last-layer lr} =0.01.\\ \text{Training loss:} -0.21.\end{gathered}$]{\includegraphics[width=.3\textwidth]{toy_dkl_lr_01.pdf}}
\subfloat[$\begin{gathered}\text{DKL: last-layer lr} =0.001.\\ \text{Training loss: } -0.07.\end{gathered}$]{\includegraphics[width=.3\textwidth]{toy_dkl_lr_001.pdf}}
\subfloat[$\begin{gathered}\text{DKL: last-layer lr} =0.0001.\\ \text{Training loss: } 0.22.\end{gathered}$]{\includegraphics[width=.3\textwidth]{toy_dkl_lr_0001.pdf}}

\subfloat[$\begin{gathered}\text{DAK: last-layer lr} =0.1.\\ \text{Training loss: } 0.10.\end{gathered}$]{\includegraphics[width=.3\textwidth]{toy_dak_lr_1.pdf}}
\subfloat[$\begin{gathered}\text{DAK: last-layer lr} =0.01.\\ \text{Training loss: } 0.10.\end{gathered}$]{\includegraphics[width=.3\textwidth]{toy_dak_lr_01.pdf}}
\subfloat[$\begin{gathered}\text{DAK: last-layer lr} =0.001.\\ \text{Training loss: } 0.22.\end{gathered}$]{\includegraphics[width=.3\textwidth]{toy_dak_lr_001.pdf}}

\caption{Results on 1D regression with different last-layer learning rates. The learning rate of NN feature extractor is set as $0.01$. (a)--(f) shows the regression fits and corresponding training losses. DAK fits for the same learning rate strategy with NN feature extractor (lr=0.01), while DKL requires a separate tuning for last-layer learning rate of GPs. Additionally, a better training loss does not necessarily prevent overfitting for DKL.}
\label{fig:dkl lr}
\end{figure}


\paragraph{Learning curves.} We plot the learning curves of CIFAR-10/100 in \Cref{fig:cifar10 curves} and \ref{fig:cifar100 curves}. The learning curves of SVDKL in \Cref{fig:cifar10 curves} is more unstable, with many significant spikes, and the convergence is slower than DAK. Futhermore, SVDKL struggles to fit with full-training in CIFAR-100, and a pretrained feature extractor is used in CIFAR-100. Therefore, the learning curves of SVDKL look smoothing, but DAK fits well with full-training in CIFAR-100.


\begin{figure}[ht]
\centering
\subfloat[Test Error (\%).]{\includegraphics[width=.3\textwidth]{CIFAR_10_test_error.pdf}}
\subfloat[Test NLL.]{\includegraphics[width=.3\textwidth]{CIFAR_10_nll.pdf}}
\subfloat[ELBO.]{\includegraphics[width=.3\textwidth]{CIFAR_10_elbo.pdf}}
\caption{Test errors, test NLLs, ELBOs of NN, SVDKL, and DAK curves with batch size of 128/1024 for CIFAR-10 averaged on 3 runs. DAK outperforms SVDKL on both test error and NLL along the training epochs. Additionally, SVDKL degrades more and struggles to fit when the batch size becomes larger.}
\label{fig:cifar10 curves}
\end{figure}

\begin{figure}[ht]
\centering
\subfloat[Test Error (\%).]{\includegraphics[width=.3\textwidth]{CIFAR_100_test_error.pdf}}
\subfloat[Test NLL.]{\includegraphics[width=.3\textwidth]{CIFAR_100_nll.pdf}}
\subfloat[ELBO.]{\includegraphics[width=.3\textwidth]{CIFAR_100_elbo.pdf}}
\caption{Test errors, test NLLs, ELBOs of NN, SVDKL, and DAK curves with batch size of 128/1024 for CIFAR-100 averaged on 3 runs. DAK trained NN and last-layer additive GPs jointly, while SVDKL used the pre-trained NN and fine-tuned the last-layer GP since SVDKL struggles to fit using full-training. DAK outperforms SVDKL on both test error and NLL along the training epochs. SVDKL struggled to fit in high-dimensional multitask cases, indicating the necessity of pre-training in SVDKL. However, DAK fitted well with high dimensionality and large batch sizes.}
\label{fig:cifar100 curves}
\end{figure}







% \end{document}

% \begin{thebibliography}{}

% \setlength{\itemindent}{-\leftmargin}
% \makeatletter\renewcommand{\@biblabel}[1]{}\makeatother
% \bibitem{} J.~Alspector, B.~Gupta, and R.~B.~Allen (1989).
%     \newblock Performance of a stochastic learning microchip.
%     \newblock In D. S. Touretzky (ed.),
%     \textit{Advances in Neural Information Processing Systems 1}, 748--760.
%     San Mateo, Calif.: Morgan Kaufmann.

% \bibitem{} F.~Rosenblatt (1962).
%     \newblock \textit{Principles of Neurodynamics.}
%     \newblock Washington, D.C.: Spartan Books.

% \bibitem{} G.~Tesauro (1989).
%     \newblock Neurogammon wins computer Olympiad.
%     \newblock \textit{Neural Computation} \textbf{1}(3):321--323.
% \end{thebibliography}

%%%%%%%%%%%%%%%%%%%%%%%%%%%%%%%%%%%%%%%%%%%%%%%%%%%%%%%%%%%%
% \section*{Checklist}


% %%% BEGIN INSTRUCTIONS %%%
% The checklist follows the references. For each question, choose your answer from the three possible options: Yes, No, Not Applicable.  You are encouraged to include a justification to your answer, either by referencing the appropriate section of your paper or providing a brief inline description (1-2 sentences). 
% Please do not modify the questions.  Note that the Checklist section does not count towards the page limit. Not including the checklist in the first submission won't result in desk rejection, although in such case we will ask you to upload it during the author response period and include it in camera ready (if accepted).

% \textbf{In your paper, please delete this instructions block and only keep the Checklist section heading above along with the questions/answers below.}
% %%% END INSTRUCTIONS %%%




\end{document}
