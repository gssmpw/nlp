%%
%% This is file `sample-sigconf-authordraft.tex',
%% generated with the docstrip utility.
%%
%% The original source files were:
%%
%% samples.dtx  (with options: `all,proceedings,bibtex,authordraft')
%% 
%% IMPORTANT NOTICE:
%% 
%% For the copyright see the source file.
%% 
%% Any modified versions of this file must be renamed
%% with new filenames distinct from sample-sigconf-authordraft.tex.
%% 
%% For distribution of the original source see the terms
%% for copying and modification in the file samples.dtx.
%% 
%% This generated file may be distributed as long as the
%% original source files, as listed above, are part of the
%% same distribution. (The sources need not necessarily be
%% in the same archive or directory.)
%%
%%
%% Commands for TeXCount
%TC:macro \cite [option:text,text]
%TC:macro \citep [option:text,text]
%TC:macro \citet [option:text,text]
%TC:envir table 0 1
%TC:envir table* 0 1
%TC:envir tabular [ignore] word
%TC:envir displaymath 0 word
%TC:envir math 0 word
%TC:envir comment 0 0
%%
%% The first command in your LaTeX source must be the \documentclass
%% command.
%%
%% For submission and review of your manuscript please change the
%% command to \documentclass[manuscript, screen, review]{acmart}.
%%
%% When submitting camera ready or to TAPS, please change the command
%% to \documentclass[sigconf]{acmart} or whichever template is required
%% for your publication.
%%
\documentclass[sigconf]{acmart}
% \documentclass[sigconf,review]{acmart}
%% NOTE that a single column version may required for 
%% submission and peer review. This can be done by changing
%% the \doucmentclass[...]{acmart} in this template to 
%% \documentclass[manuscript,screen]{acmart}
%% 
%% To ensure 100% compatibility, please check the white list of
%% approved LaTeX packages to be used with the Master Article Template at
%% https://www.acm.org/publications/taps/whitelist-of-latex-packages 
%% before creating your document. The white list page provides 
%% information on how to submit additional LaTeX packages for 
%% review and adoption.
%% Fonts used in the template cannot be substituted; margin 
%% adjustments are not allowed.

%%
%% \BibTeX command to typeset BibTeX logo in the docs
%% \BibTeX command to typeset BibTeX logo in the docs
\AtBeginDocument{%
  \providecommand\BibTeX{{%
    Bib\TeX}}}

%% Rights management information.  This information is sent to you
%% when you complete the rights form.  These commands have SAMPLE
%% values in them; it is your responsibility as an author to replace
%% the commands and values with those provided to you when you
%% complete the rights form.
\setcopyright{acmlicensed}
\copyrightyear{2018}
\acmYear{2018}
\acmDOI{XXXXXXX.XXXXXXX}
%% These commands are for a PROCEEDINGS abstract or paper.
\acmConference[Conference acronym 'XX]{Make sure to enter the correct
  conference title from your rights confirmation emai}{June 03--05,
  2018}{Woodstock, NY}
%%
%%  Uncomment \acmBooktitle if the title of the proceedings is different
%%  from ``Proceedings of ...''!
%%
%%\acmBooktitle{Woodstock '18: ACM Symposium on Neural Gaze Detection,
%%  June 03--05, 2018, Woodstock, NY}
\acmISBN{978-1-4503-XXXX-X/18/06}


%%
%% Submission ID.
%% Use this when submitting an article to a sponsored event. You'll
%% receive a unique submission ID from the organizers
%% of the event, and this ID should be used as the parameter to this command.
%%\acmSubmissionID{123-A56-BU3}

%%
%% For managing citations, it is recommended to use bibliography
%% files in BibTeX format.
%%
%% You can then either use BibTeX with the ACM-Reference-Format style,
%% or BibLaTeX with the acmnumeric or acmauthoryear sytles, that include
%% support for advanced citation of software artefact from the
%% biblatex-software package, also separately available on CTAN.
%%
%% Look at the sample-*-biblatex.tex files for templates showcasing
%% the biblatex styles.
%%

%%
%% The majority of ACM publications use numbered citations and
%% references.  The command \citestyle{authoryear} switches to the
%% "author year" style.
%%
%% If you are preparing content for an event
%% sponsored by ACM SIGGRAPH, you must use the "author year" style of
%% citations and references.
%% Uncommenting
%% the next command will enable that style.
%%\citestyle{acmauthoryear}

\usepackage{graphicx}
\usepackage{multirow}
\usepackage{hyperref}       % hyperlinks
\usepackage{url}            % simple URL typesetting
\usepackage{booktabs}       % professional-quality tables
\usepackage{amsfonts}       % blackboard math symbols
\usepackage{nicefrac}       % compact symbols for 1/2, etc.
\usepackage{microtype}      % microtypography
\usepackage{xcolor}         % colors
% \usepackage{minipage}

\usepackage{mathrsfs}
\usepackage{amsmath}


\usepackage{bm}
\usepackage{algorithm}
\usepackage{algorithmic}
%\usepackage{algorithmicx}
\usepackage{float}  
\usepackage{lipsum}

\usepackage{xcolor}    % 用于 \colorbox
\usepackage{varwidth}  % 用于 varwidth 环境

\let\Bbbk\relax         %%redefined in newtxmath.sty
\usepackage{amssymb}
\usepackage{pifont}
 \usepackage{enumitem}
\usepackage{multirow}
\usepackage{amsthm}
\theoremstyle{definition}
\newtheorem{definition}{Definition}
\usepackage{wrapfig}

% \documentclass{article}
\usepackage{booktabs}  % 提供更好看的横线
\usepackage{array}     % 提供更多的列格式控制
% \usepackage{booktabs}  % 提供更好看的横线
\usepackage{graphicx}  % 允许图片插入
\usepackage{subfigure}
\usepackage{multirow}  % 允许跨行合并
\usepackage{multicol}  % 允许跨列合并

\usepackage{colortbl}
\usepackage{xcolor}
\usepackage{array}
\usepackage{xspace}
\newcommand{\ourmethod}{\textsc{{MPhil}}\xspace}
%%
%% end of the preamble, start of the body of the document source.
\begin{document}

%%
%% The "title" command has an optional parameter,
%% allowing the author to define a "short title" to be used in page headers.
\title{Raising the Bar in Graph OOD Generalization: \\ Invariant Learning Beyond Explicit Environment Modeling}

%%
%% The "author" command and its associated commands are used to define
%% the authors and their affiliations.
%% Of note is the shared affiliation of the first two authors, and the
%% "authornote" and "authornotemark" commands
%% used to denote shared contribution to the research.
\author{Xu Shen }
\authornote{Both authors contributed equally to this research.}

\affiliation{%
  \institution{Jilin University}
 \streetaddress{}
  \city{Changchun}
  \country{China}}
  \email{shenxu23@mails.jlu.edu.cn}

\author{Yixin Liu }
\authornotemark[1]
\affiliation{%
  \institution{Griffith University}
  \streetaddress{}
  \city{Goldcoast}
  \country{Australia}}
\email{yixin.liu@griffith.edu.au}

\author{Yili Wang }

\affiliation{%
  \institution{Jilin University}
  \streetaddress{}
  \city{Changchun}
  \country{China}}
\email{wangyl21@mails.jlu.edu.cn}

\author{Rui Miao}
\affiliation{%
  \institution{Jilin University}
  \streetaddress{}
  \city{Changchun}
  \country{China}}
\email{ruimiao20@mails.jlu.edu.cn}

\author{Yiwei Dai}
\affiliation{%
  \institution{Jilin University}
  \streetaddress{}
  \city{Changchun}
  \country{China}}
\email{daiyw23@mails.jlu.edu.cn}

\author{Shirui Pan }
\affiliation{%
  \institution{Griffith University}
  \streetaddress{}
  \city{Goldcoast}
  \country{Australia}}
\email{s.pan@griffith.edu.au}

\author{Xin Wang}
\authornote{Corresponding author.}
% \authornotemark[1]
\affiliation{%
  \institution{Jilin University}
  \streetaddress{}
  \city{Changchun}
  \country{China}}
\email{xinwang@jlu.edu.cn}

%%
%% By default, the full list of authors will be used in the page
%% headers. Often, this list is too long, and will overlap
%% other information printed in the page headers. This command allows
%% the author to define a more concise list
%% of authors' names for this purpose.
\renewcommand{\shortauthors}{Xu Shen et al.}

%%
%% The abstract is a short summary of the work to be presented in the
%% article.

%%
%% The "title" command has an optional parameter,
%% allowing the author to define a "short title" to be used in page headers.


%%
%% The "author" command and its associated commands are used to define
%% the authors and their affiliations.
%% Of note is the shared affiliation of the first two authors, and the
%% "authornote" and "authornotemark" commands
%% used to denote shared contribution to the research.

%%
%% The abstract is a short summary of the work to be presented in the
%% article.
\begin{abstract}
Out-of-distribution (OOD) generalization has emerged as a critical challenge in graph learning, as real-world graph data often exhibit diverse and shifting environments that traditional models fail to generalize across. A promising solution to address this issue is graph invariant learning (GIL), which aims to learn invariant representations by disentangling label-correlated invariant subgraphs from environment-specific subgraphs. However, existing GIL methods face two major challenges: (1) the difficulty of \textbf{capturing and modeling diverse environments} in graph data, and (2) the \textbf{semantic cliff}, where invariant subgraphs from different classes are difficult to distinguish, leading to poor class separability and increased misclassifications. 
To tackle these challenges, we propose a novel method termed \textbf{M}ulti-\textbf{P}rototype \textbf{H}yperspherical \textbf{I}nvariant \textbf{L}earning (\ourmethod), which introduces two key innovations: (1) \textit{hyperspherical invariant representation extraction}, enabling robust and highly discriminative hyperspherical invariant feature extraction, and (2) \textit{multi-prototype hyperspherical classification}, which employs class prototypes as intermediate variables to eliminate the need for explicit environment modeling in GIL and mitigate the semantic cliff issue. Derived from the theoretical framework of GIL, we introduce two novel objective functions: the \textit{invariant prototype matching loss} to ensure samples are matched to the correct class prototypes, and the \textit{prototype separation loss} to increase the distinction between prototypes of different classes in the hyperspherical space.
Extensive experiments on 11 OOD generalization benchmark datasets demonstrate that \ourmethod achieves state-of-the-art performance, significantly outperforming existing methods across graph data from various domains and with different distribution shifts. The source code of \ourmethod is available at \href{https://anonymous.4open.science/r/MPHIL-23C0/}{https://anonymous.4open.science/r/MPHIL-23C0/}.
\end{abstract}

%%
%% The code below is generated by the tool at http://dl.acm.org/ccs.cfm.
%% Please copy and paste the code instead of the example below.
%%
\begin{CCSXML}
<ccs2012>
   <concept>
       <concept_id>10002950.10003624.10003633.10010917</concept_id>
       <concept_desc>Mathematics of computing~Graph algorithms</concept_desc>
       <concept_significance>500</concept_significance>
   </concept>
   <concept>
       <concept_id>10010147.10010341.10010366.10010367</concept_id>
       <concept_desc>Computing methodologies~Machine learning</concept_desc>
       <concept_significance>500</concept_significance>
   </concept>
   <concept>
       <concept_id>10010405.10010406.10010430.10010431</concept_id>
       <concept_desc>Applied computing</concept_desc>
       <concept_significance>300</concept_significance>
   </concept>
   <concept>
       <concept_id>10002951.10003227.10003241</concept_id>
       <concept_desc>Information systems~Data mining</concept_desc>
       <concept_significance>300</concept_significance>
   </concept>
</ccs2012>
\end{CCSXML}

\ccsdesc[500]{Mathematics of computing~Graph algorithms}
\ccsdesc[500]{Computing methodologies~Machine learning}
\ccsdesc[300]{Applied computing}
\ccsdesc[300]{Information systems~Data mining}
%%
%% Keywords. The author(s) should pick words that accurately describe
%% the work being presented. Separate the keywords with commas.
\keywords{Graph out-of-distribution generalization, invariant learning, hyperspherical space}
%% A "teaser" image appears between the author and affiliation
%% information and the body of the document, and typically spans the
%% page.


\received{20 February 2007}
\received[revised]{12 March 2009}
\received[accepted]{5 June 2009}

%%
%% This command processes the author and affiliation and title
%% information and builds the first part of the formatted document.

\maketitle

\section{Introduction} \label{sec:intro}
\section{Introduction}


\begin{figure}[t]
  \centering
  \includegraphics[width=7cm]{head_figure.pdf}
  \vspace{-1em}
  \caption{A conceptual comparison between our proposed ChorusCVR and existing CVR models on the perspective of the discrimination spaces of soft labels.}
  \label{intro}
  \vspace{-2em}
\end{figure}
Recommender systems are crafted to provide users with personalized content (videos, products and ads, \emph{etc.,}) that match their preferences \cite{youtubenet,sim,mmoe,din}. Generally, industrial RecSys typically divided into two major stages. 1) Retrieval stage, which aims to search thousands of related candidates from massive item pool. 2) Ranking stage, which aims to estimate interaction probability, \emph{e.g.,} click-through rate (CTR) and post-click conversion rate (CVR), for each user-item pair for retrieved candidates, and select a set of best items for users. In this paper, we focus on the post-click conversion rate (CVR) estimation task during ranking stage.

\textit{Problem statement.} Typically, a positive CVR sample follows the following data funnel: 
%
\textit{exposure} $\mathcal{D}$$\to$\textit{click} $\mathcal{O}$$\to$\textit{conversion} $\mathcal{R}$, where the \textit{click} space $\mathcal{O}$ is around about $4\sim6\%$ of \textit{exposure} space $\mathcal{D}$ and the \textit{conversion} space $\mathcal{R}$ takes up $2\sim4\%$ of \textit{click} space $\mathcal{O}$.
%
Different with CTR which is learned using exposure space samples, CVR is typically learned using only click space samples because we are unaware the un-clicked samples would be converted or not. 
% 
%
However, during online inference, the CTR and CVR scores are utilized in the same assumed exposure space, which leads to a well-known mismatch sample selected bias (SSB) issue \cite{ssb1,ssb2,ssb3,dcmt}, that CVR learning module is trained in \textit{click} space $\mathcal{O}$ but is used for inference at  \textit{exposure} space $\mathcal{D}$.
%









\textit{Motivation.} To alleviate the SSB problem, previous wisdom introduce several techniques to extend CVR task to \textit{exposure} space.
%
Specifically, ESMM \cite{essm} propose a click-through \& conversion rate (CTCVR) task to merge two CVR and CTR scores as one score to supervised it in \textit{exposure} space, which successfully extend the CVR to entire space to solve space inconsistency between training and inference.
%
Unfortunately, the CTCVR loss made a strong assumption that \textbf{\textbf{un-clicked} training samples are hard negative samples in CTCVR training}. This assumption overlooks some ambiguous negative samples which may be easy for users to buy after he/she clicked, but without chance to be clicked yet \cite{multiipw,dcmt}.
%
To alleviate this false negative sample issue, the recent works are dedicated to find reasonable pseudo soft labels to for \textbf{un-clicked} sample learning.
%
Specifically, the DCMT \cite{dcmt} propose to regularize the CVR objectives by a complementary constraint with a novel counterfactual CVR objective. 
%
For counterfactual CVR learning, DCMT first assumes all un-clicked items as positive samples while all converted items are negative samples, and then apply a $CVR = 1 - counterfactualCVR$ constraints for CVR learning module, as shown in Figure~\ref{intro}(b). 
%
Besides, the NISE \cite{nise} and DDPO \cite{ddpo} first utilize the outputs of an additional CVR tower learned in click space to act as pseudo soft label, and then employ a cross-entropy constraints $CVR\thickapprox extraCVR$ in un-click space, as shown in Figure~\ref{intro}(c).
% 

It has come to our attention that the quality of soft CVR labels is the key to mitigating SSB issues.  So we ask, \textit{what requirements should an ideal soft CVR label satisfies}? Our key insight is, an ideal soft label should at least satisfy two requirements: \textbf{R1. Discriminability}: for a clicked sample, the label can discriminate it would be converted or not; \textbf{R2. Robustness}: for un-converted sample, the label can separate the factually un-converted sample in click space from those ambiguous un-converted sample in un-click space. With this in mind, we  present the discrimination surface of soft labels in existing methods (DCMT, NISE and DDPO) in Figure~\ref{intro}. We find their discrimination surface are fully overlapped with certain part of the surfaces of  ESMM (CTCVR task w/o soft labels), which miss either \textbf{R1} or \textbf{R2}. Thus, few of existing methods can meet all these requirements.

To fill this gap, we present a novel entire-space dual multi-task learning model, namly \textbf{ChorusCVR}, to realize effective CVR learning in un-click space that  fulfills both \textbf{R1} and \textbf{R2} (see Figure~\ref{intro}(d)).  The ChorusCVR consists of two modules, \emph{i.e.,} \textbf{N}egative sample \textbf{D}iscrimination \textbf{M}odule (NDM) and \textbf{S}oft \textbf{A}lignment \textbf{M}odule (SAM). In NDM, we introduce a novel CTunCVR auxiliary task, to provide robust soft CVR labels with the ability to discriminate factual CVR negative samples (clicked but un-converted) and ambiguous CVR negative samples (un-clicked).  In SAM, we utilize generated CTunCVR soft outputs to supervise  CVR learning with several alignment objectives, to realize debiased CVR learning in entire-space. Our contributions can be summarized as follows:
\begin{itemize}[leftmargin=*,align=left]
    \item We introduce a novel CTunCVR auxiliary task to provide soft CVR labels with both discriminability and robustness in entire space.
    \item We propose a novel ChorusCVR model with effective alignment objectives for debiased CVR modelling in entire space.
    \item We conduct extensive experiments on both public and production environment datasets and online A/B testing to verify the efficacy of our method, which show that our ChorusCVR achieves superior performance over all existing state-of-the-art methods.  
\end{itemize}
\section{Related Works} \label{appe:rw}
\section{Related Work}\label{sec:relatedwork}

Internet of Things (IoT) has seen rapid advancements in recent years, becoming an integral part of various domains, such as smart industries and homes, and serving as a key enabler in modern society.
However, despite its growth, IoT continues to face numerous security challenges, prompting significant research efforts aimed at improving IoT security.
With the rise of artificial intelligence (AI), machine learning (ML) and deep learning (DL)-based approaches have become increasingly popular in designing defense mechanisms for IoT devices, including malicious traffic classification~\cite{luo2022transformer,shafiq2020corrauc}, malware detection~\cite{vasan2020mthael,chaganti2022deep,aung2022atlas}, vulnerability discovery~\cite{neshenko2019demystifying}, and others~\cite{al2020survey,otoum2022dl,tambe2019detection}.

More recently, inspired by the success of large language models (LLMs), researchers have begun exploring the potential of LLMs to enhance IoT-related security tasks.
For instance, LLMs have been applied to existing IoT security challenges such as threat detection and fuzzing. Ferrag \etal~\cite{sokiotllm} introduced a BERT-based model, SecurityBERT, to achieve better cyber threat detection accuracy over traditional ML and DL-based methods. 
Similarly, Ma \etal~\cite{ma} and Wang \etal~\cite{llmiotfuz} proposed LLM-assisted fuzzing methods to uncover hidden bugs in IoT devices, enabling the detection of complex vulnerabilities that traditional techniques might miss.
Additionally, Yang \etal~\cite{yang2023iot} combined LLMs with static code analysis using prompt engineering to create a cost-effective solution for IoT vulnerability detection.
\cite{ji2024sevenllm} collected cybersecurity raw texts to train cybersecurity LLM to augment the analysis of cybersecurity events, and \cite{llmtikg} made use of a larger LLM to build knowledge graphs from public threat intelligence and use GPT to create datasets to fine-tune a smaller LLM to extract entities and TTPs from attack description.
Ferraris \etal~\cite{ferraris2024ici} proposed utilizing ChatGPT to enhance IoT trust semantics, aligning with W3C Web of Things (WoT) recommendations\footnote{\scriptsize \url{https://www.w3.org/WoT/}}.
This work extends the TrUStAPIS framework~\cite{ferraris2020trustapis}.

Beyond the above tasks, LLMs have been employed in other IoT challenges.
Meyuhas \etal~\cite{meyuhas2024iotlabel} used LLMs to address the problem of labeling previously unseen IoT devices.
\cite{llmiotcontrol,cui2024llmind} explored leveraging LLMs to control IoT devices and facilitate effective collaboration among them.
Mo \etal~\cite{mo2024iot} collected IoT sensor-natural language paired data and trained IoT-LM to interpret and interact with physical IoT sensors.
Xu \etal~\cite{xu2024penetrative} employed ChatGPT to interpret IoT sensor data and reason over tasks in the physical realm, introducing novel ways of integrating human knowledge into cyber-physical systems. 

Recently, Deldari \etal~\cite{deldari2024auditnet} proposed AuditNet, a conversational AI-based security assistant, which is most similar to \chatiot\ and also augmented by external knowledge.
However, AuditNet focused on standards, policies, and regulations of portable document format (PDF), and aimed to reduce the manual effort of security experts involved in compliance checks of IoT. 
On the other hand, we integrate IoT threat intelligence of various sources into \chatiot\ and can assist multiple kinds of users. Besides, we provide an end-to-end toolkit to process data in various formats, not limited to PDF. 

Together, these studies indicate that LLMs have great potential to improve the security of IoT systems in various domains, from vulnerability discovery to trustworthiness management. 
By integrating LLMs with IoT-specific threat intelligence, these models can be guided to meet the unique challenges posed by the IoT ecosystem.
Moreover, the continuous advancements in the LLM community, combined with increasingly accessible IoT datasets, are likely to further drive the adoption of LLMs in IoT-related research and practical applications.

\section{Preliminaries and Background}
In this section, we introduce the preliminaries and background of this work, including the formulation of the graph OOD generalization problem, graph invariant learning, and hyperspherical embeddings.

% \subsection{Problem Formulation}
\subsection{Problem Formulation} 
In this paper, we focus on the OOD generalization problem on graph classification tasks~\citep{li2022out,jia2024graph,fan2022debiasing,wu2022discovering}. We denote a graph data sample as $(G,y)$, where $G \in \mathcal{G}$ represents a graph instance and $y \in \mathcal{Y}$ represents its label. The dataset collected from a set of environments $\mathcal{E}$ is denoted as $\mathcal{D} = \{\mathcal{D}^{e}\}_{e \in \mathcal{E}}$, where $\mathcal{D}^{e} =\{({G}^{e}_{i},{y}^{e}_{i})\}^{n^{e}}_{i=1}$ represents the data from environment $e$, and $n^e$ is the number of instances in environment $e$. Each pair $({G}^{e}_{i}, {y}^{e}_{i})$ is sampled independently from the joint distribution $P_{e}(\mathcal{G}, \mathcal{Y}) = P(\mathcal{G}, \mathcal{Y} | e)$. 
In the context of graph OOD generalization, the difficulty arises from the discrepancy between the training data distribution $P_{e_{tr}}(\mathcal{G}, \mathcal{Y})$ from environments $e_{tr} \in \mathcal{E}_{tr}$, and the testing data distribution $P_{e_{te}}(\mathcal{G}, \mathcal{Y})$ from unseen environments $e_{te} \in \mathcal{E}_{test}$, where $\mathcal{E}_{te} \neq \mathcal{E}_{tr}$. The goal of OOD generalization is to learn an optimal predictor $f: \mathcal{G} \rightarrow \mathcal{Y}$ that performs well across both training and unseen environments, $\mathcal{E}_{all} = \mathcal{E}_{tr} \cup \mathcal{E}_{te}$, i.e., 
\begin{equation}
\label{eq: OOD_target} \min_{f \in \mathcal{F}} \max_{e \in \mathcal{E}_{\mathrm{all}}} \mathbb{E}_{(G^{e}, y^{e}) \sim P_{e}}[\ell(f(G^e), y^e)], 
\end{equation}
where $\mathcal{F}$ denotes the hypothesis space, and $\ell(\cdot,\cdot)$ represents the empirical risk function. 

\subsection{Graph Invariant Learning (GIL)}
Invariant learning focuses on capturing representations that preserve consistency across different environments, ensuring that the learned invariant representation $\mathbf{z}_{inv}$ maintains consistency with the label $y$~\citep{mitrovic2020representation,wu2022discovering,chen2022learning}. Specifically, for graph OOD generalization, the objective of GIL is to learn an invariant GNN $f:= f_{c} \circ g$, where $g: \mathcal{G} \rightarrow \mathcal{Z}_{inv}$ is an encoder that extracts the invariant representation from the input graph $G$, and $f_{c}: \mathcal{Z}_{inv} \rightarrow \mathcal{Y}$ is a classifier that predicts the label $y$ based on $\mathbf{z}_{inv}$. From this perspective, the optimization objective of OOD generalization, as stated in Eq.~(\ref{eq: OOD_target}), can be reformulated as: 
\begin{equation}
\label{eq: causal} \max_{f_{c}, g} I(\mathbf{z}_{inv}; y), \text{ s.t. } \mathbf{z}_{inv} \perp e,\forall e \in \mathcal{E}_{tr}, \mathbf{z}_{inv} = g(G),
\end{equation}
{where $I(\mathbf{z}_{inv}; y)$ denotes the mutual information between the invariant representation $\mathbf{z}_{inv}$ and the label $y$.}
This objective ensures that $\mathbf{z}_{inv}$ is independent of the environment $e$, focusing solely on the most relevant information for predicting $y$. 

\subsection{Hyperspherical Embedding} 
Hyperspherical learning enhances the discriminative ability and generalization of deep learning models by mapping feature vectors onto a unit sphere~\citep{liu2017deep}. 
To learn a hyperspherical embedding for the input sample, its representation vector $\mathbf{z}$ is mapped into hyperspherical space with arbitrary linear or non-linear projection functions, followed by normalization to ensure that the projected vector $\hat{\mathbf{z}}$ lies on the unit hypersphere ($\|\hat{\mathbf{z}}\|^{2}=1$). 
To make classification prediction, the hyperspherical embeddings $\hat{\mathbf{z}}$ are modeled using the von Mises-Fisher (vMF) distribution~\citep{ming2022exploit}, with the probability density for a unit vector in class $c$ is given by:
\begin{equation}
\label{eq: vMF} p(\hat{\mathbf{z}}; \boldsymbol{\mu}^{(c)}, \kappa ) = Z(\kappa) \exp(\kappa {\boldsymbol{\mu}^{(c)}}^\top \hat{\mathbf{z}}), 
\end{equation} 
where $\boldsymbol{\mu}^{(c)}$ denotes the prototype vector of class $c$ with the unit norm, serving as the mean direction for class $c$, while $\kappa$ controls the concentration of samples around $\boldsymbol{\mu}_c$.
The term $Z(\kappa)$ serves as the normalization factor for the distribution. Given the probability model in Eq.(\ref{eq: vMF}), the hyperspherical embedding $\hat{\mathbf{z}}$ is assigned to class $c$ with the following probability:
\begin{equation} \label{eq: prototpye_1}
    \begin{aligned}
\mathbb{P}\left(y = c \mid \hat{\mathbf{z}}; \{\kappa, \boldsymbol{\mu}^{(i)}\}_{i = 1}^{C}\right) &= \frac{Z(\kappa) \exp \left(\kappa {\boldsymbol{\mu}^{(c)}}^{\top} \hat{\mathbf{z}}\right)}{\sum_{i = 1}^{C} Z(\kappa) \exp \left(\kappa {\boldsymbol{\mu}^{(i)}}^{\top} \hat{\mathbf{z}}\right)}\\ &= \frac{\exp \left({\boldsymbol{\mu}^{(c)}}^{\top} \hat{\mathbf{z}} / \tau\right)}{\sum_{i = 1}^{C} \exp \left({\boldsymbol{\mu}^{(i)}}^{\top} \hat{\mathbf{z}}/ \tau\right)},
    \end{aligned}
\end{equation}
where $\tau = 1/\kappa$ is a temperature parameter. In this way, the classification problem is transferred to the distance measurement between the graph embedding and the prototype of each class in hyperspherical space, where the class prototype is usually defined as the embedding centroid of each class.

\section{Methodology}
\section{Methodology}


\begin{figure*}[t]
  \centering
\includegraphics[width=0.70\textwidth]{main_figure_v2.png}
  \vspace{-1em}
  \caption{Systematic overview of our Chorus CVR model.}
  \label{choruscvr}
  \vspace{-1em}
\end{figure*}


\subsection{Preliminary}
In the ranking stage of industrial recommendation system, all \textit{exposure} user-item pairs will be collected and formed as a data-streaming for model training, i.e., $\mathcal{D}$.
Specifically, each user-item sample in $\mathcal{D}$ could represent as $(u, i, \{\mathbf{x}_u, \mathbf{x}_i,$ $\mathbf{x}_{ui}\}, o_{ui}, r_{ui}) \in \mathcal{D}$, where $u$/$i$ denotes the user-item pair, $\mathbf{x}_u\in\mathbb{R}^{d_u}, \mathbf{x}_i\in\mathbb{R}^{d_i}, \mathbf{x}_{ui}\in\mathbb{R}^{d_{ui}}$ are the user-side features (e.g., user ID), item-side features (e.g., item ID), and item-aware cross features (e.g., SIM \cite{sim}).
%
The $o_{ui}\in\{0,1\}$ and $r_{ui}\in\{0,1\}$ are user-item ground-truth interacted labels, where $o_{ui}$ denotes whether user $u$ clicked item $i$ and $r_{ui}$ denotes whether user $u$ converted item $i$. 
%
According to the entire \textbf{\textit{exposure} space} $\mathcal{D}$, we could further obtain several subset spaces:
%
\begin{itemize}[leftmargin=*,align=left]
\item \textbf{\textit{Click} space} $\mathcal{O}\in\mathcal{D}$, if click label $o_{ui} = 1$.
\item \textbf{\textit{un-Click} space} $\mathcal{N}=\mathcal{D} - \mathcal{O}$, if click label $o_{ui} = 0$.
\item \textbf{\textit{Conversion} space} $\mathcal{R}\in\mathcal{O}$: if label $o_{ui}=1$ and $r_{ui}=1$.
\item \textbf{\textit{un-Conversion} space} $\mathcal{M}=\mathcal{O}-\mathcal{R}$: if label $o_{ui}=1$ and $r_{ui}=0$.
\end{itemize}
%
Based on them, a simple ranking model can be formed as:
\begin{equation}
% \small
\begin{split}
&\hat{y}^{ctr}_{ui} = \texttt{MLP}^{ctr}(\mathbf{x}_{ui}),\quad \hat{y}^{cvr}_{ui} = \texttt{MLP}^{cvr}(\mathbf{x}_{ui}),\\
&\mathbf{x}_{ui} = \texttt{Multi-Task-Encoder}(\mathbf{x}_u\oplus \mathbf{x}_i\oplus \mathbf{x}_{ui}),\\
\end{split}
\label{base}
\end{equation}
where the $\oplus$ denotes the concatenate operator, $\mathbf{x}\in\mathbb{R}^d$ is the encoded hidden states, and $\texttt{MLP}(\cdot)$ denotes a stacked neural-network. We use a share-bottom based multi-task paradigm to predict CTR and CVR scores, $\hat{y}^{ctr},\hat{y}^{cvr}$.
%
Next, we directly minimize the cross-entropy binary classification loss to train CTR tower and CVR towers with corresponding space samples:
%
%
%
\begin{equation}
% \small
% \footnotesize
\begin{split}
&\mathcal{L}^{ctr} = - \frac{1}{|\mathcal{D}|}\big(\sum_{(u,i)\in\mathcal{D}}\delta(\hat{y}^{ctr}_{ui}, o_{ui})\big),\\
&\mathcal{L}^{cvr} = - \frac{1}{|\mathcal{O}|}\big(\sum_{(u,i)\in\mathcal{O}}\delta(\hat{y}^{cvr}_{ui}, r_{ui})\big).
\end{split}
\label{crossentropy}
\end{equation}
%
% 
%
In inference, given the hundreds item candidates in a certain user request, we could obtain predicted CTCVR by  $\hat{y}^{ctcvr}_{ui} =\hat{y}^{ctr}_{ui} \cdot \hat{y}^{cvr}_{ui}$ for each item, which is used for final ranking. Then top K highest items will be returned and shown to user. The CVR are learned in click space during training but be predicted in an assumed explore space during inference, which brings up the question of sample selection bias problem.
% 





To alleviate sample selection bias,  ESMM \cite{essm} expand the click-space CVR learning task to exposure-space CTCVR learning task, to directly solve the inconsistency between training and inference:
\begin{equation}
% \small
\begin{split}
\mathcal{L}^{ctcvr} = &- \frac{1}{|\mathcal{D}|}\Big(\sum_{(u,i)\in\mathcal{D}}\delta(\hat{y}^{ctr}_{ui}\cdot\hat{y}^{cvr}_{ui}, o_{ui}\cdot r_{ui})\Big)
\end{split}
\label{ctxcvr}
\end{equation}
which treats all un-clicked samples as negative samples of CTCVR task. However those un-clicked samples that would be converted if clicked, which are falsely negative samples,  still leads to missing not at random (MNAR) problem  \cite{multiipw}. To mitigate this problem, inverse propensity weighting (IPW) \cite{multiipw,escm2} based method inversely weight the CVR loss in click space by propensity score of observing  $(u,i)$ in click space $\mathcal{O}$, to eliminate the influence of click event to CVR estimation in entire space $D$
\begin{equation}
\small
\begin{split}
\mathcal{L}^{cvr}_{IPW} =& - \frac{1}{|\mathcal{O}|}\Big(\sum_{(u,i)\in\mathcal{O}}\frac{\delta(\hat{y}^{cvr}_{ui}, r_{ui})}{\hat{y}^{ctr}_{ui}}\Big),
\end{split}
\label{cvripw}
\end{equation}
% 
%
Our method is based on above ESMM with IPW  framework. Although alleviating SSB and MNAR problem, IPW-based methods still lack reasonable labels for \textit{un-clicked} samples, which we solve by generating discriminative and robust soft labels.






\subsection{ChorusCVR}
In this section, we dive into ChorusCVR and explain how we realize entire-space debiased CVR learning by generating discriminative and robust soft CVR labels (as shown in Figure~\ref{choruscvr}).

\subsubsection{Negative sample Discrimination Module (NDM)}

As mentioned before, the soft labels introduced by previous works are suboptimal for lack either discriminability or robustness. As shown in Figure~\ref{intro} (d), an ideal discrimination surface should separate the factual negative samples (clicked but un-converted) from positive samples (clicked \& converted), and factual negative samples from ambiguous negative samples (un-clicked). With this in mind, we find the ideal discrimination surface implies a new task, CTunCVR prediction. We formulate CTunCVR labels as:
\begin{equation}
y^{ctuncvr} = o_{ui} * (1-r_{ui}) = 
\begin{cases} 
1 & o_{ui}=1~\&~r_{ui} = 0, \\
0 &  o_{ui}=0, \\
0 & o_{ui}=1~\&~r_{ui} = 1,
\end{cases}
\end{equation}
where only \textit{clicked but un-converted} samples are positive samples, both \textit{clicked \& converted} and \textit{un-clicked} samples are negative samples. Instead of directly predicting CTunCVR score in exposure space, we follow a typical two-stage prediction paradigm to obtain CTunCVR to 
reduce cumulative error. We firstly introduce an additional unCVR tower to predict unCVR score $\hat{y}^{uncvr}$, then combine it with $\hat{y}^{ctr}$ to form CTunCVR score:
% 
\begin{equation}
% \small
\begin{split}
\hat{y}^{uncvr}_{ui} = \hat{y}^{ctr}_{ui}\cdot\hat{y}^{cvr}_{ui}\quad \quad
\hat{y}^{ctuncvr}_{ui} =\hat{y}^{ctr}_{ui}\cdot\hat{y}^{uncvr}_{ui}
\end{split}
\label{uncvr}
\end{equation}
Then we can naturally optimize CTunCVR objective in exposure space by cross entropy loss: 
\begin{equation}
% \small
\begin{split}
\mathcal{L}^{ctuncvr} = - \frac{1}{|\mathcal{D}|}\Big(\sum_{(u,i)\in\mathcal{D}}\delta(\hat{y}^{ctuncvr}_{ui}, o_{ui} * (1-r_{ui})\Big).
\end{split}
\label{uncvr}
\end{equation}
% With the help of this formulation, we can narrow down the problem to the accurate estimation of unCVR. 
With the help of $\mathcal{L}^{ctuncvr}$ and an extra \textbf{unCVR prediction result} $\hat{y}^{uncvr}_{ui}$, we can narrow down the aforementioned problem to consider \textbf{R1. Discriminability} and \textbf{R2. Robustness} problem at same time.
%
For the $\hat{y}^{uncvr}_{ui}$ generation, we add an mirror unCVR tower which similar with the Eq.(\ref{base}) and (\ref{crossentropy}):
%
%
%
% 
%
%
% $
% 
\begin{equation}
% \smalluncvr
\begin{split}
\hat{y}^{uncvr}_{ui} &= \texttt{MLP}^{uncvr}(\mathbf{x}_{ui}), \\
\mathcal{L}^{uncvr} = - \frac{1}{|\mathcal{O}|}\Big(&\sum_{(u,i)\in\mathcal{O}}\delta\big(\hat{y}^{uncvr}_{ui}, 1-r_{ui})\big)\Big)
\end{split}
\label{uncvr}
\end{equation}
% 
Next, analogously with the Eq.(\ref{cvripw}), we then adopt the predicted click $\hat{y}^{ctr}_{ui}$ to inversely weight the unCVR error, to $\mathcal{L}^{uncvr}$ as:
% 
\begin{equation}
% \small
\begin{split}
\mathcal{L}^{uncvr}_{IPW} = - \frac{1}
{|\mathcal{O}|}\Big(\sum_{(u,i)\in\mathcal{O}}\frac{\delta(\hat{y}^{uncvr}_{ui}, 1-r_{ui})}{\hat{y}^{ctr}_{ui}}\Big)
\end{split}
\label{uncvr}
\end{equation}
In this way, the \textit{click} space tendency can be alleviated that higher/lower $pCTR$ sample will declined/enhanced for a fair training. So far we obtain debiased unCVR soft labels, which we will utilize to help the CTunCVR training and CVR component supervision.



\subsubsection{Soft Alignment Module (SAM)}
Up to now, we fulfill the initial goal of obtain high-quality soft labels in un-clicked space. In this section we present the solution to utilize the unCVR score as soft labels to supervise CVR learning, which we call \emph{soft alignment mechanism}. We first use $1-unCVR$ manner as soft labels for entropy-based CVR learning. In the same time, we also use $1-CVR$ manner to generate soft labels for unCVR learning, in a mutual supervision fashion to align unCVR predictions to CVR. All these objectives are inversely weighted by predicted CTR in a IPW paradigm (see $\mathcal{L}^{align1}_{IPW}$ and $\mathcal{L}^{align2}_{IPW}$ in Figure.~\ref{choruscvr}). To further alleviate SSB for un-click space, we also propose a \textit{un-click space IPW} approach, to inversely weight the un-click samples with $1-pCTR$ for CTR and unCVR alignment objectives (see $\mathcal{L}^{align3}_{IPW}$ and $\mathcal{L}^{align4}_{IPW}$ in Figure.~\ref{choruscvr}). Overall, all alignment objectives are as follows:
\begin{equation}
% \small
\begin{split}
\mathcal{L}^{align}_{IPW} = &- \frac{1}{|\mathcal{O}|}\big(\frac{\delta(\hat{y}^{cvr}_{ui}, 1-\texttt{sg}(\hat{y}^{uncvr}_{ui}))}{\hat{y}^{ctr}_{ui}}\big)
- \frac{1}{|\mathcal{N}|}\big(\frac{\delta(\hat{y}^{cvr}_{ui}, 1-\texttt{sg}(\hat{y}^{uncvr}_{ui}))}{1-\hat{y}^{ctr}_{ui}}\big)\\
&- \frac{1}{|\mathcal{O}|}\big(\frac{\delta(\hat{y}^{uncvr}_{ui}, 1-\texttt{sg}(\hat{y}^{cvr}_{ui}))}{\hat{y}^{ctr}_{ui}}\big)
- \frac{1}{|\mathcal{N}|}\big(\frac{\delta(\hat{y}^{uncvr}_{ui}, 1-\texttt{sg}(\hat{y}^{cvr}_{ui}))}{1-\hat{y}^{ctr}_{ui}}\big)
\end{split}
\label{soft}
\end{equation}

where the $\texttt{sg}(\cdot)$ means the stop gradient function, the $\hat{y}^{ctr}_{ui}, (1 - \hat{y}^{ctr}_{ui})$ denote the click propensity in the \textit{click} and \textit{un-click} space, respectively.
% 
%
All losses of our ChorusCVR are as follows:
\begin{equation}
% \small
\begin{split}
\mathcal{L} = \mathcal{L}^{ctcvr} + \mathcal{L}^{cvr}_{IPW} + \mathcal{L}^{ctuncvr} + \mathcal{L}^{uncvr}_{IPW} + \mathcal{L}^{align}_{IPW}
\end{split}
\label{soft}
\end{equation}
In this way, our ChorusCVR  make CVR and unCVR supervise each other during training, which results in an equilibrium. 

\section{Experiments}

In this section, we present our experimental setup  (Sec.~\ref{subsec:setup}) and showcase the results in (Sec.~\ref{subsec:results}). For each experiment, we first highlight the research question being addressed, followed by a detailed discussion of the findings.


\documentclass{MITstyle}

%\usepackage[table]{xcolor}
\usepackage{chngcntr}
\usepackage{hyperref}
\usepackage{microtype}

\title{A Lightweight and Extensible Cell Segmentation and Classification Model for Whole Slide Images}

\author{Nikita Shvetsov~$^{1, }$\footnote{Correspondence e-mail: nikita.shvetsov@uit.no}, Thomas K. Kilvaer~$^{2, 3}$, Masoud Tafavvoghi~$^{4}$, Anders Sildnes~$^{1}$, \\ Kajsa Møllersen~$^{4}$, Lill-Tove Rasmussen Busund~$^{5, 6}$, Lars Ailo Bongo~$^{1}$ \\
%
\vspace{1em} % Space between authors and afilliations
%
\normalfont{\small $^{1}$Department of Computer Science, UiT The Arctic University of Norway}\\
\normalfont{\small $^{2}$Department of Oncology, University Hospital of North Norway}\\
\normalfont{\small $^{3}$Department of Clinical Medicine, UiT The Arctic University of Norway}\\
\normalfont{\small $^{4}$Department of Community Medicine, UiT The Arctic University of Norway}\\
\normalfont{\small $^{5}$Department of Medical Biology, UiT The Arctic University of Norway} \\
\normalfont{\small $^{6}$Department of Clinical Pathology, University Hospital of North Norway} %\vspace{2em}
}

\begin{document}
\maketitle

\section*{Abstract}

% \begin{abstract}
% Developing clinically useful cell-level analysis tools in digital pathology remains challenging due to limitations in dataset granularity, inconsistent annotations, computational demands of advanced models, and difficulties in integrating new technologies into clinical workflows. To address these challenges, we propose a multi-faceted solution that enhances data quality, model performance, and usability to create a lightweight and extensible cell segmentation and classification model.

% First, we update data labels by employing a cross-relabeling process that refines the labels of two existing datasets, PanNuke and MoNuSAC, to create a new unified dataset with enhanced granularity, encompassing seven distinct cell types. Second, we leverage the H-Optimus foundation model as a fixed encoder to improve feature representation for simultaneous cell segmentation and classification tasks. Third, to address the computational demands of foundation models, we employ knowledge distillation to reduce model size and complexity while maintaining comparable performance. Finally, to facilitate integration into clinical workflows, we integrate the distilled model into the QuPath software, a widely used open-source platform in digital pathology.

% Our results demonstrate improvements in cell segmentation and classification performance using the H‑Optimus-based model compared to a CNN-based model. Specifically, the average $R^2$ improved from 0.575 to 0.871, and the average $PQ$ score improved from 0.450 to 0.492, indicating better alignment with actual cell counts and enhanced segmentation and classification quality. Furthermore, the distilled student model maintains performance comparable to the larger foundation model while reducing the parameter count by a factor of 48.
% Overall, by reducing computational complexity and integrating it into existing workflows, the proposed approach may significantly impact diagnostic processes, reduce the workload of pathologists, and contribute to improved patient outcomes. Though our approach shows potential enhancements in efficiency and usability of cell segmentation and classification models in digital pathology, extensive validation is needed to deploy these models in clinical practice.
% \end{abstract}

%%% shortened abstract
\begin{abstract}
Developing clinically useful cell-level analysis tools in digital pathology remains challenging due to limitations in dataset granularity, inconsistent annotations, high computational demands, and difficulties integrating new technologies into workflows. To address these issues, we propose a solution that enhances data quality, model performance, and usability by creating a lightweight, extensible cell segmentation and classification model. 

First, we update data labels through cross-relabeling to refine annotations of PanNuke and MoNuSAC, producing a unified dataset with seven distinct cell types. Second, we leverage the H-Optimus foundation model as a fixed encoder to improve feature representation for simultaneous segmentation and classification tasks. Third, to address foundation models' computational demands, we distill knowledge to reduce model size and complexity while maintaining comparable performance. Finally, we integrate the distilled model into QuPath, a widely used open-source digital pathology platform. 

Results demonstrate improved segmentation and classification performance using the H-Optimus-based model compared to a CNN-based model. Specifically, average $R^2$ improved from 0.575 to 0.871, and average $PQ$ score improved from 0.450 to 0.492, indicating better alignment with actual cell counts and enhanced segmentation quality. The distilled model maintains comparable performance while reducing parameter count by a factor of 48. By reducing computational complexity and integrating into workflows, this approach may significantly impact diagnostics, reduce pathologist workload, and improve outcomes. Although the method shows promise, extensive validation is necessary prior to clinical deployment.
\end{abstract}
\clearpage

\section{Introduction}
In digital pathology, accurate segmentation and classification of cells are crucial for many diagnostic, prognostic, and predictive analyses \cite{Jaber_Beziaeva_etal._2019,Lin_Pan_etal._2022,Park_Ock_etal._2022,Shen_Choi_etal._2024}. Nowadays, developments in computational pathology offer multiple solutions \cite{H._Qu_P._Wu_etal._2020,Javed_Mahmood_etal._2020} to utilize cell-level datasets to train machine learning models that solve these problems. The quality and specificity of training datasets are critical for robust and accurate models. Adhering to the principle of "garbage in, garbage out", it is essential to ensure that these datasets are extensively and accurately labeled with distinct classes that reflect the diverse biological characteristics of different cell types. Unfortunately, the number of open-source datasets comprising such high-quality annotations is limited. Existing cell segmentation datasets \cite{Gamper_Koohbanani_etal._2019,Graham_Vu_etal._2019,Verma_Kumar_etal._2021} may offer extensive annotations for certain cell types while providing more general labels for others. For example, in PanNuke, which is one of the largest open-source datasets comprising labeled cells, various types of morphologically and functionally different inflammatory cells like macrophages and lymphocytes are clustered in a broad "inflammatory" class. Consequently, these classes are frequently omitted from analyses or aggregated into broader meta-classes \cite{Gamper_Koohbanani_etal._2020} and likely interfere with other cell classes included in the dataset. This and similar inconsistencies in annotation granularity limit the ability of machine learning models to learn the comprehensive and nuanced features necessary for accurate cell segmentation and classification. To address these challenges, methods for refining and standardizing dataset annotations are essential to enhance the quality of training data.

A complementary approach to mitigate the absence of high-quality training data is the use of foundation models. Foundation models as encoders are defined as large-scale, versatile networks pre-trained on vast, diverse datasets using self-supervised learning, contrasting with convolutional neural network (CNN) pre-trained encoders that rely on supervised learning with labeled data. In practice, foundation models leverage enormous amounts of weakly or unlabeled data from millions of whole slide images (WSIs) and employ self-attention mechanisms to capture long-range dependencies and global context \cite{Chen_Ding_etal._2024,Saillard_Jenatton_etal._2024,Vorontsov_Bozkurt_etal._2024,Xu_Usuyama_etal._2024}. As a consequence, foundation models are able to produce transferable feature representations across different cell types and tissue environments. The feature representations can be leveraged by decoder networks to produce segmentation masks and pixel-level classifications. Because foundation models have comprehensive feature representations, they can be effectively fine-tuned using much smaller amounts of cell-level data compared to the large datasets needed to train models from scratch. Furthermore, foundation models incorporate adversarial training elements or contrastive learning \cite{Chen_Ding_etal._2024,Xu_Usuyama_etal._2024}, enhancing their resilience and adaptability by exposing them to challenging and varied scenarios during training. This may result in more generalizable models, often making them well-suited for diverse and complex tasks in digital pathology.

Despite the inherent advantages of foundation models, their deployment for practical use faces its own obstacles. In particular, they require substantial computational power, financial investments and rigorous testing to ensure reliability and efficacy for a given task \cite{Akkus_Dangott_etal._2022,Dragomir_Cocuz_etal._2022,Go_2022,Jafri_Farooqui_etal._2024}. Moreover, while foundation models enhance feature representation and performance, they depend on the quality of available annotations for decoder fine-tuning and, like any other model, cannot resolve existing inconsistencies or ambiguities in data labels. Therefore, there remains a critical need for solutions that address both data quality and practical deployment considerations.
Further, integrating new technologies into existing clinical workflows often encounters resistance, as it necessitates adjustments to established diagnostic processes. So, there is a need to develop solutions that could be integrated into current practices, minimizing the burden on medical professionals to adopt new tools \cite{King_Williams_etal._2023}.

Existing solutions \cite{Goldsborough_Philps_etal._2024,Hörst_Rempe_etal._2024}, while addressing some aspects of these challenges, fall short in providing a comprehensive approach. To address the data quality and clinical deployment issues, we propose a multi-faceted solution that encompasses data refinement, model optimization, and integration with existing pathology tools (\hyperref[fig:fig1]{Figure 1}). The outcome is a lightweight cell segmentation and classification model that can be integrated into digital pathology workflows for practical clinical use.

\begin{figure}[h!]
    \centering
    \includegraphics[width=\textwidth, height=0.82\textheight, keepaspectratio]{images/Figure_1.pdf}
    \caption{Overview of the proposed solution, including 1) Data refinement using cross-relabeling, 2) Teacher model development and fine tuning, 3) Student model optimization with knowledge distillation and 4) Student model and QuPath integration}
    \label{fig:fig1}
\end{figure}
\clearpage

Our approach begins with preparing the data for the fine-tuning and training of the machine learning models. We create a refined dataset, acquired via cross-relabeling two cell-level datasets, enhancing annotation specificity and consistency of the labeled data. Subsequently, we create a cell segmentation and classification model based on the foundation model. We leverage the foundation model as a fixed encoder and fine-tune a decoder using the refined dataset to improve generalization across diverse tissue- and cell types.
To ensure that the model remains lightweight and deployable in a possibly resource-constrained environment, we employ knowledge distillation to approximate the functionality of the foundation model. Finally, to facilitate the practical application of our model in digital pathology workflows, we integrate it with the QuPath \cite{Bankhead_Loughrey_etal._2017} application. Each methodological component contributes to the overarching goal of enhancing model performance, generalizability, and usability in clinical settings.

The primary contributions of this paper are:
\begin{enumerate}
    \item \textit{Data labels refinement through cross-relabeling:}
    
    We propose a new method for refining labels of cell-level datasets through cross-relabeling. This method employs classification models to re-label broad and ambiguous instances, resulting in a more diverse dataset. Our evaluation demonstrates that these classification models achieve high accuracy on test subsets, indicating the reliability of the method for label refinement.

    \item \textit{Enhanced model performance via foundation models:}
    
    We employ a foundation model as a feature extractor for the cell segmentation and classification task. In comparison with training a CNN model from scratch, the foundation model backbone only needs fine-tuning, which significantly reduces training time, computational resources and data requirements. We show that using a foundation model encoder leads to better performance in cell segmentation and classification networks than using a CNN-based encoder. This improvement may enable the model to generalize more effectively across various tissue types and imaging methods.
    
    \item \textit{Model optimization through knowledge distillation:}
    
    We show that a smaller student model trained using knowledge distillation on the refined dataset obtained via our cross-relabeling approach from a foundation model achieves comparable performance in cell segmentation and quantification tasks. As a result, this model is more suitable for deployment in environments without high-performance computing resources.
    
    \item \textit{Integration with QuPath:}
    
    We integrate the distilled cell segmentation and classification model into QuPath, a widely used open-source digital pathology platform, to accelerate clinical adaptation by enabling pathologists to more easily incorporate advanced computational tools into their existing workflows.
\end{enumerate}

Through these methodological steps, we aim to bridge the gap between advanced machine learning techniques and practical clinical applications, making accurate and efficient digital pathology accessible in a broader range of healthcare settings.

\section{Refining Existing Datasets Using Cross-Relabeling}
To address the limitations of sparse and ambiguous labeling of cell-level datasets, we propose a generalizable cross-relabeling strategy that can be applied to any dataset containing broadly categorized or imprecisely labeled cell types. This approach involves training and subsequently leveraging classification models to refine broad categories into more specific or biologically relevant classes.
When applied to cell-level data, the methodology includes extracting individual cell images from the dataset patches, preprocessing these images to standardize the size and accommodate partial cells, and then training deep learning classifiers capable of distinguishing between the finer cell subtypes within the coarser categories. 
To illustrate our approach, we focus on the PanNuke \cite{Gamper_Koohbanani_etal._2020, Gamper_Koohbanani_etal._2019} and MoNuSAC \cite{Verma_Kumar_etal._2021} datasets that we have used to train models for cell quantification in our previous works \cite{Shvetsov_Grønnesby_etal._2022,Shvetsov_Sildnes_etal._2024}. We find that for better cell differentiation we have to introduce more granular labels. PanNuke includes a broad classification of "inflammatory" cells, encompassing lymphocytes, macrophages, and neutrophils. Each cell type differs significantly in structure, function, and clinical relevance. Conversely, MoNuSAC uses the label "epithelial" for a class that comprises both benign epithelial cells and malignant neoplastic cells. This practice makes it challenging to differentiate between benign and malignant epithelial cells in the dataset, which is a critical distinction when identifying tumor areas within tissue samples. To address these issues, we implement a cross-relabeling strategy as shown in \hyperref[fig:fig2]{Figure 2}. The key components are two classification models: one is trained on singular cell images from PanNuke data to classify the epithelial meta-class into epithelial and neoplastic classes. The other is trained on MoNuSAC to refine the inflammatory class into lymphocytes, neutrophils, and macrophages.

\begin{figure}[h!]
    \centering
    \includegraphics[width=\textwidth]{images/Figure_2.pdf}
    \caption{Refined dataset generation via cross relabeling}
    \label{fig:fig2}
\end{figure}

The refining approach consists of three consecutive steps. The first is the preprocessing step, in which we extract individual cells from both datasets (\hyperref[fig:fig3]{Figure 3}). The specifics of PanNuke and MoNuSAC patch preparation before cell preprocessing are provided in \hyperref[chap:S1]{Appendix S1}.

\begin{figure}[h!]
    \centering
    \includegraphics[width=\textwidth]{images/Figure_3.pdf}
    \caption{Cell instances preprocessing including (1) cell map extraction, (2) bounding box delineation, (3) adjusting cell boxes and (4) cropping and resizing of cell images}
    \label{fig:fig3}
\end{figure}

During preprocessing, we extract cell type maps from the ground truth label mask and calculate bounding boxes around each cell instance. To accommodate partial cells at patch borders, a common issue in cropped patch images, we employ mirror padding and extend the field of view of the cell label by 15 pixels to capture adjacent cells. We then crop and resize the identified regions to $64 \times 64$ pixels using bicubic interpolation.

The preprocessed PanNuke dataset comprises 68,031 neoplastic and 23,207 epithelial cell images, while MoNuSAC comprises  33,104 lymphocytes, 1,252 neutrophils, and 1,695 macrophages, which we subsequently use in training cell classification models and classifying the cell image data \hyperref[fig:S2]{Appendix Figure S2 (1)}. 

The next step is to train two distinct ResNet50-based classifiers tailored to address the specific labeling challenges inherent in each dataset. We use ResNet50 for classification models due to its proven effectiveness for image classification tasks in histopathology \cite{pan2022reviewmachinelearningapproaches}, and its compatibility with small images. For the PanNuke dataset, we design the classifier, trained on MoNuSAC data, to disaggregate the heterogeneous "inflammatory" cell category into distinct subtypes: lymphocytes, macrophages, and neutrophils. Similarly, for the MoNuSAC dataset, the classifier is trained on PanNuke data and distinguishes between benign and malignant epithelial cells within the overarching "epithelial" label. By applying these targeted classifiers to their respective datasets, we assign more specific labels to individual cell instances, thus enabling us to create a unified dataset.
To ensure a balanced representation of classes, we train both models on datasets that had been equalized to match the size of the least represented class. Thus, we obtain datasets comprising 23,207 samples per class for PanNuke and 1,252 samples per class for MoNuSAC data. Next, we partition both of them into training (70\%), validation (20\%), and testing (10\%) subsets. To mitigate the risk of overfitting, we use a single dropout layer with a rate of p=0.5 in both models and data augmentation using randomized color perturbations, rotation, and horizontal and vertical flipping. We employ AdamW optimizer and the cross-entropy loss function for the training criterion.

To evaluate the two trained models, we measure the classification accuracy on the respective test subsets. The accuracies on the test subset for both classifiers are presented in \hyperref[tab:1]{Table 1}. The PanNuke model achieves an average accuracy of 93.57\%, with higher accuracy for neoplastic cells (96.06\%) compared to epithelial cells (86.26\%). The confusion matrix in Figure A3.1 shows that the model predominantly distinguishes accurately between epithelial and neoplastic tissues, with a substantial number of correct classifications and relatively few misclassifications. The MoNuSAC model demonstrates an average accuracy of 98.92\%, excelling in classifying lymphocytes (99.67\%) and macrophages (94.12\%), with lower performance for neutrophils (85.71\%). The confusion matrix in Figure A3.2 shows that the model identifies lymphocytes and performs reasonably well with macrophages and neutrophils.

\begin{table}[h!]
\renewcommand{\arraystretch}{1.5}
  \centering
  \caption{Cell classification results for PanNuke and MoNuSAC trained models (CI 95\%).}
  \label{tab:1}
  \begin{tabular}{|l|c|c|}
   \hline
   %\rowcolor{gray!30}
    Accuracy               & PanNuke model              & MoNuSAC model              \\
    \hline
    Average      & 0.936 (0.931--0.941)         & 0.989 (0.986--0.993)        \\
    \hline
    Neoplastic   & 0.961 (0.956--0.965)         & -                          \\
    \hline
    Epithelial   & 0.863 (0.849--0.877)         & -                          \\
    \hline
    Lymphocytes  & -                          & 0.997 (0.995--0.999)        \\
    \hline
    Neutrophils  & -                          & 0.857 (0.796--0.918)        \\
    \hline
    Macrophages  & -                          & 0.941 (0.906--0.976)        \\
    \hline
  \end{tabular}
\end{table}

Finally, during the last step, we use the model trained on PanNuke data for epithelial cells in MoNuSAC and the model trained on MoNuSAC for the inflammatory cells class in PanNuke. Specifically, we use classifier models to relabel epithelial cells in MoNuSAC and inflammatory cells in PanNuke data. Then we combine cells with refined labels and the rest of the cells in both datasets to create a refined dataset (\hyperref[fig:S2]{Appendix Figure S2 (2)}). The process of relabeling cells and visualizing them on a patch is shown in \hyperref[fig:fig4]{Figure 4}. The cell counts in the refined dataset are provided in \hyperref[tab:S4]{Appendix Table S4}.

\begin{figure}[h!]
    \centering
    \includegraphics[width=\textwidth, height=0.42\textheight, keepaspectratio]{images/Figure_4.pdf}
    \caption{Cell relabeling procedure for epithelial and inflammatory cell classes}
    \label{fig:fig4}
\end{figure}

%\hfill

Relabeling and combining datasets have been explored in a prior study \cite{Parulekar_Kanwat_etal._2023}, where consecutive fine-tuning on multiple datasets was employed to account for hierarchical class label structures. While the method presented in \cite{Parulekar_Kanwat_etal._2023} is intuitive, it often lacks consistency and requires multiple fine-tuning runs, which can be cumbersome and time-consuming. 
In contrast, cross-relabeling simplifies this process by using specialized classification models tailored to each dataset's specific labeling challenges. This approach provides better transparency and produces a unified dataset encompassing seven distinct cell types across multiple tissue samples, enhancing data diversity for further model training or fine-tuning.

Despite these improvements, cross-relabeling does not entirely resolve issues related to poor labeling quality or the amount of labeled data. Specifically, our results show lower accuracies persist for underrepresented classes, such as macrophages, which may stem from a limited sample availability and intrinsic challenges in distinguishing these cells based solely on H\&E staining. Furthermore, while our method enhances label specificity, it relies on the initial quality of the broad labels; thus, any fundamental inaccuracies in the original annotations can propagate through the relabeling process. Addressing the overall problem of limited data labels may require integrating additional data sources or utilizing complementary immunohistochemical staining methods.
Although the reported performance metrics are obtained from evaluations on the native test sets of each dataset, it is important to note that the primary application of these classifiers is to perform cross-relabeling, where a model trained on one dataset (e.g., PanNuke) is applied to another (e.g., MoNuSAC) and vice versa. We acknowledge that a more systematic evaluation of cross-dataset generalization is needed and could be performed in future work.

Overall, the refined dataset produced by our approach can enhance the supervised training or fine-tuning of cell segmentation and classification models, especially those that utilize pre-trained foundation models to improve feature extraction robustness. In addition, these models can detect nuanced classes that enable researchers to conduct more detailed analyses of biological processes in computational pathology.

\section{Foundation models for robust cell segmentation and classification}

Accurate cell segmentation and classification in digital pathology are hindered by limited labeled data and the fact that conventional CNNs are unable to capture global contextual information due to their local receptive field constraints \cite{Gheflati_Rivaz_2022,Yang_Marcus_etal.}. Traditional approaches in cell quantification have predominantly relied on CNN encoders, such as ResNet50, given their proven effectiveness in semantic segmentation tasks \cite{Deshmane_2023,Graham_Vu_etal._2019,Mukasheva_Koishiyeva_etal._2024,Stringer_Wang_etal._2021}. However, approaches that include fine-tuning of pretrained CNNs, data augmentation, and stain normalization to partially increase data variability and address staining differences often fail to achieve the necessary generalization and robustness across diverse tissue types and staining conditions \cite{G._Wang_W._Li_etal._2018,Gao_Bagci_etal._2018,Karim_El_Khoury_Martin_Fockedey_etal._2021}.

To overcome these challenges, we leverage an encoder-decoder network that uses a foundation model as the encoder and a CNN upsampling decoder (\hyperref[fig:fig5]{Figure 5}) for simultaneous cell segmentation and classification in 2D patches extracted from WSIs. Foundation models with transformer-based architectures are viable alternatives to CNN-based encoders \cite{Shamshad_Khan_etal._2023,Sourget_2023}. They enable the creation of more advanced architectures that can decode or transform learned features more effectively \cite{Chen_Duan_etal._2023,Cheng_Misra_etal._2022,Xie_Wang_etal._2021}.

\begin{figure}[h!]
    \centering
    \includegraphics[width=\textwidth]{images/Figure_5.pdf}
    \caption{UNETR-like model with foundational model as backbone}
    \label{fig:fig5}
\end{figure}

By utilizing a transformer-based encoder, we incorporate global contextual information into the feature extraction process, which is a key advantage of such architectures \cite{Chen_Lu_etal._2021}. This foundation model integration facilitates accurate pixel-wise segmentation and classification without the need for extensive encoder training, thereby potentially improving generalization across varied cellular structures and tissue types.
In our implementation, we employ a modified UNETR \cite{Hatamizadeh_Tang_etal._2021} architecture that combines a vision transformer (ViT) \cite{Dosovitskiy_Beyer_etal._2021} encoder with a CNN-based decoder. The encoder utilizes the pretrained H-Optimus foundation model, which contains 1.1 billion parameters and is trained on over 500,000 H\&E stained WSIs \cite{Saillard_Jenatton_etal._2024}. We extract outputs from four evenly spaced transformer blocks $Z_i$, where $i \in [1, 14, 26, 38]$, to serve as residual connections for the CNN decoder. We select these blocks based on our observation that features from non-adjacent levels of the encoder lead to better overall performance on the test subset.

The CNN decoder upsamples the feature representations, acquired from the transformer blocks, to generate an intermediate vector that is handled by two task-specific layers that generate cell segmentation and classification masks. The first task-specific layer is the ‘Cellpose head’,  which is used to delineate cell instances. The layer generates horizontal and vertical gradient maps to form vector fields that are refined through gradient tracking in a post-processing step using the Cellpose algorithm \cite{Stringer_Wang_etal._2021}, known for its efficacy in cell segmentation tasks and generalizability across multiple domains \cite{Pachitariu_Stringer_2022,Stringer_Pachitariu_2024}. The second task-specific layer is the "Cell type head", which assigns labels to individual pixels. In the post-processing step, we determine the output classification label of each segmented cell instance by majority voting over the labeled pixels that comprise the cell in the segmentation map.

To evaluate model performance and measure the impact of adding a foundation model as backbone, we compare it to a ResNet50-based model. ResNet50 is a widely used solution for encoders in segmentation architectures in the medical domain \cite{Deshmane_2023,Graham_Vu_etal._2019,Mukasheva_Koishiyeva_etal._2024,Stringer_Wang_etal._2021}. For the H-Optimus-based model, we utilize frozen weights for the encoder and only fine-tune the decoder to take advantage of the extensive pre-training of the foundation model. For the ResNet50-based model we start with ImageNet \cite{Deng_Dong_etal.} weights and train both encoder and decoder parts. Hyperparameters for the training step are set to be identical, where possible, for comparable evaluation. 
For this evaluation, we deliberately use the PanNuke dataset to provide a standardized and controlled comparison between the H‑Optimus and ResNet50-based models (\hyperref[fig:S2]{Appendix Figure S2 (3)}). Specifically, we use two of the default PanNuke dataset splits (66\%) for training and validation, and reserve the third split (33\%) for testing.

To address the challenge of cell class imbalance in the PanNuke dataset, which is a common characteristic in most cell-level H\&E patch datasets, both models’ training processes employ a weighted loss function comprising cross-entropy and focal loss \cite{Lin_Goyal_etal._2018}. The focal loss component is adjusted with coefficients derived from each cell class' instance frequency, emphasizing learning from underrepresented classes and enhancing the model's sensitivity to rare but significant cellular patterns. The cross-entropy loss is augmented with spectral decoupling regularization \cite{Pezeshki_Kaba_etal._2021,Pohjonen_Stürenberg_etal._2022} and spatially varying label smoothing \cite{Islam_Glocker_2021}, which potentially stabilizes training and improves generalization in case of complex tissue morphologies. For optimization, we employ the \textit{AdamW} \cite{Loshchilov_Hutter_2019} to counter unbalanced class scenarios, with cosine annealing learning rate scheduler.

We utilize the scikit-learn library \cite{Van_der_Walt_Schönberger_etal._2014} and HoVer-Net \cite{Graham_Vu_etal._2019} implementations of $R^2$ (the coefficient of determination) and $PQ$ (panoptic quality) to evaluate our experiments. Complete mathematical formulations and detailed explanations of these metrics are provided in \hyperref[chap:S5]{Appendix S5}. To compute confidence intervals, we use nonparametric bootstrapping, where after calculating the metric on the full sample, we generated 1000 bootstrap replicates by resampling with replacement and then determined the 95\% confidence intervals as the 2.5th and 97.5th percentiles of the resulting empirical distribution.

%\hfill

The model comparisons are summarized in \hyperref[tab:2]{Table 2}. The H‑Optimus-based model achieves higher $R^2$ across all cell classes compared to the ResNet50-based model, which means that its predictions are more closely aligned with the PanNuke cell counts, indicating a stronger correlation with the observed data. Notably, the improvement of $R^2_{dead}$ may be an indicator of better global contextual representations provided by the foundation model backbone. In terms of segmentation and classification quality combined, measured by the PQ score, the H‑Optimus-based model demonstrates notable improvements across most cell classes. Overall, the average $R^2$ improved from 0.575 to 0.871, while the average $PQ$ score improved from 0.450 to 0.492, demonstrating better performance of the H-Optimus-based model.

\begin{table}[h!]
\renewcommand{\arraystretch}{1.5}
  \centering
  \caption{Cell quantification metrics for baseline and proposed models (CI 95\%).}
  \label{tab:2}
  \begin{tabular}{|l|c|c|}
    \hline
    %\rowcolor{gray!30}
    Metric             & Resnet50-based            & H-optimus-based              \\
    \hline
    $R^2_{neoplastic}$    & 0.681 (0.576--0.769)       & \textbf{0.941 (0.917--0.960)} \\
    \hline
    $R^2_{inflammatory}$  & 0.863 (0.778--0.903)       & \textbf{0.949 (0.918--0.966)} \\
    \hline
    $R^2_{connective}$    & 0.600 (0.488--0.698)       & 0.609 (0.436--0.772)          \\
    \hline
    $R^2_{dead}$          & 0.097 (-11.389--0.669)     & 0.925 (0.404--0.982)          \\
    \hline
    $R^2_{epithelial}$    & 0.635 (0.490--0.747)       & \textbf{0.930 (0.886--0.964)} \\
    \hline
    $PQ_{neoplastic}$       & 0.517 (0.499--0.535)       & \textbf{0.589 (0.575--0.604)} \\
    \hline
    $PQ_{inflammatory}$     & 0.455 (0.429--0.482)       & \textbf{0.528 (0.507--0.549)} \\
    \hline
    $PQ_{connective}$       & 0.416 (0.400--0.431)       & \textbf{0.451 (0.436--0.465)} \\
    \hline
    $PQ_{dead}$             & 0.374 (0.342--0.408)       & 0.292 (0.209--0.365)          \\
    \hline
    $PQ_{epithelial}$       & 0.488 (0.460--0.519)       & \textbf{0.599 (0.579--0.618)} \\
    \hline
  \end{tabular}
\end{table}

Our results  show that integrating the H‑Optimus foundation model within the UNETR architecture enhances the model's ability to segment and classify cells across diverse tissues from PanNuke data. The pretrained transformer encoder provides robust feature representations, resulting in higher average $R^2$ and $PQ$ scores compared to the CNN-based model. This leads to more reliable cell quantification and more accurate downstream analysis. Additionally, the streamlined fine-tuning process reduces computational overhead and training time, making the model more adaptable for new data.

Despite these advancements, the foundation model-based approach does not fully resolve all challenges related to cell segmentation and classification. We observe lower metric scores for underrepresented classes in the training data. Furthermore, foundation models typically encompass billions of parameters, resulting in substantial computational and memory requirements. It therefore poses challenges for deployment in resource-constrained environments, limiting their practical applicability in certain clinical settings.

\section{Model optimization via Knowledge Distillation}

To address the limitations posed by the extensive size of foundation models, we implement knowledge distillation — a model compression technique that leverages the teacher-student paradigm \cite{Hinton_Vinyals_etal._2015}. By training a smaller, more efficient student model to replicate the output of a larger, pre-trained teacher model, we retain performance while significantly reducing the model's complexity and resource requirements (\hyperref[fig:fig6]{Figure 6}).

\begin{figure}[h!]
    \centering
    \includegraphics[width=\textwidth, height=0.45\textheight, keepaspectratio]{images/Figure_6.pdf}
    \caption{Knowledge distillation framework for training a student model using a pre-trained teacher}
    \label{fig:fig6}
\end{figure}

We employ knowledge distillation to compress the H‑Optimus-based teacher model into a more efficient student model. The teacher model is the modified UNETR architecture with the H‑Optimus foundation model described in the previous chapter. The student model is based on a UNet architecture augmented with residual connections and incorporates a smaller ViT encoder with 9 million parameters \cite{Steiner_Kolesnikov_etal._2022,Wightman_2019}. 

First, we fine-tune the teacher model using the refined dataset from the cross-relabeling procedure (Section 2). Initially we train the decoder of the teacher model while keeping the encoder weights frozen. We split the refined dataset into train (70\%), validation (20\%) and test (10\%) subsets (\hyperref[fig:S2]{Appendix Figure S2 (4)}). During fine-tuning, we use the train and validation subsets, while leaving the test subset for model evaluation. We set the training procedure and model hyperparameters to be identical to those that were used to demonstrate the utility of foundation models for the simultaneous cell segmentation and classification task.

Next, we perform knowledge distillation from teacher to student using the refined dataset used to fine-tune the teacher model. The student model is trained to replicate the teacher model's outputs. We utilize a specialized loss function that aligns the student's predicted probability distribution with the teacher's, incorporating the teacher's class probability distribution derived from the output. Following the methodology of Hinton et al. \cite{Hinton_Vinyals_etal._2015}, we experiment with various hyperparameter settings for the temperature ($T$) and the balancing coefficients ($\alpha$ and $\beta$) in the loss function. We vary $T$ from 1 to 20 and adjust $\alpha$ and $\beta$ to balance the distillation and student losses. Through iterative tuning and evaluation, we identify that setting $T=14$, $\alpha=0.3$, and $\beta=0.7$ yields a configuration that converges and closely approximates the teacher model's performance during training.

Finally, we assess the performance of both models using the $R^2$ and $PQ$ (defined in \hyperref[chap:S5]{Appendix S5}) on the test set of the refined dataset (\hyperref[tab:3]{Table 3}). We observe that the 95\% confidence intervals overlap for most cell types, so we cannot claim statistically significant performance differences between the teacher and student models. One exception appears in the neoplastic class. The teacher model produces an $R^2$ of 0.919, while the student model shows an $R^2$ of 0.852. In addition, the student model achieves higher $PQ$ values for the neoplastic and connective classes, though the confidence intervals show overlap.

\begin{table}[h!]
\renewcommand{\arraystretch}{1.5}
  \centering
  \caption{Cell quantification metrics for teacher and distilled student models (CI 95\%).}
  \label{tab:3}
  \begin{tabular}{|l|c|c|}
    \hline
    %\rowcolor{gray!30}
    Metric & Teacher & Student \\
    \hline
    $R^2_{neoplastic}$    & \textbf{0.919} (0.898--0.939) & 0.852 (0.800--0.891) \\
    \hline
    $R^2_{lymphocyte}$    & 0.969 (0.956--0.977)         & 0.969 (0.956--0.978) \\
    \hline
    $R^2_{connective}$    & 0.694 (0.548--0.809)         & 0.618 (0.469--0.741) \\
    \hline
    $R^2_{dead}$          & 0.755 (0.400--0.908)         & 0.424 (0.100--0.731) \\
    \hline
    $R^2_{epithelial}$    & 0.922 (0.870--0.958)         & 0.843 (0.738--0.917) \\
    \hline
    $R^2_{macrophage}$    & 0.384 (-0.369--0.724)        & 0.704 (0.352--0.859) \\
    \hline
    $R^2_{neutrofil}$     & 0.854 (0.578--0.929)         & 0.833 (0.502--0.925) \\
    \hline
    $PQ_{neoplastic}$       & 0.581 (0.569--0.593)         & 0.601 (0.588--0.613) \\
    \hline
    $PQ_{lymphocyte}$       & 0.536 (0.520--0.553)         & 0.563 (0.544--0.579) \\
    \hline
    $PQ_{connective}$       & 0.436 (0.421--0.451)         & 0.457 (0.441--0.474) \\
    \hline
    $PQ_{dead}$             & 0.272 (0.235--0.315)         & 0.279 (0.201--0.369) \\
    \hline
    $PQ_{epithelial}$       & 0.522 (0.500--0.545)         & 0.530 (0.506--0.555) \\
    \hline
    $PQ_{macrophage}$       & 0.524 (0.459--0.588)         & 0.474 (0.405--0.543) \\
    \hline
    $PQ_{neutrofil}$        & 0.541 (0.490--0.592)         & 0.565 (0.522--0.607) \\
    \hline
  \end{tabular}
\end{table}


We further decompose the $PQ$ metric into its $SQ$ and $DQ$ components (\hyperref[tab:S6]{Appendix Table S6}). Both models produce nearly identical $SQ$ values, which indicates that they predict instance boundaries with similar precision. Although the student model shows some improvement in $DQ$ scores for certain classes, the confidence intervals overlap and do not confirm a statistically significant difference.

We observe that the student and teacher models yield comparable detection performance despite the student model using a much smaller and simpler architecture. A model with fewer parameters reduces the risk of overfitting when training data are scarce relative to the model’s complexity \cite{Farias_Ludermir_etal._2022}. The knowledge distillation process also encourages the student model to focus on the most generalizable detection features learned from the teacher. These factors enable the student model to achieve similar detection performance across different cell types.

Additionally, considering the model sizes reported in \hyperref[tab:4]{Table 4}, the distilled model achieves a significant reduction compared to the teacher model, with a 48-fold decrease in parameter count and a 5.5-fold reduction in on-disk size. In inference mode, the teacher model requires 16 GB of VRAM for a batch size of 32, while the distilled model only needs 3 GB of VRAM for the same batch size. These reductions make the distilled model significantly more practical for fine-tuning and deployment in resource-constrained environments.

\begin{table}[h!]
\renewcommand{\arraystretch}{1.5}
  \centering
  \caption{Parameter counts and size of teacher and distilled model}
  \label{tab:4}
  \adjustbox{max width=\textwidth}{%
  \begin{tabular}{|l|c|c|c|}
    \hline
    %\rowcolor{gray!30}
    Metric & H-optimus-based (Teacher) & mobileViT-based (Student) & Magnitude of difference \\
    \hline
    Parameters count       & 1,158,917,906   & \textbf{24,093,393}   & \textbf{48x}  \\
    \hline
    Estimated Total Size (MB) & 87,912       & \textbf{15,935}    & \textbf{5.5x} \\
    \hline
  \end{tabular}%
}
\end{table}

%\hfill

With recent advancements in complex network architectures and the use of pretrained encoders to achieve state-of-the-art performance \cite{Baumann_Dislich_etal._2024,Hörst_Rempe_etal._2024} in cell segmentation and classification tasks, model size, computational complexity, and processing times have increased. This limits the scalability and accessibility of these models. As we demonstrate, this may be mitigated using knowledge distillation. Studies in the field of natural language processing have demonstrated the efficacy of knowledge distillation in retaining the capabilities of the teacher model while achieving significant reductions in size and complexity \cite{Huangpu_Gao_2024,Sun_Yu_etal.}. 

We demonstrate the feasibility of knowledge distillation in digital pathology, specifically for cell segmentation and classification tasks. Moreover, we achieve this performance while also significantly reducing the parameter count. In addressing the challenge of knowledge transfer, we found that distillation from a transformer-based model to a smaller transformer is more straightforward than attempting to map transformer features to CNN blocks. In our experiments, using a CNN-based network as a student results in worse cell quantification performance due to the structural constraints of CNN feature space dimensions. 

Although our primary approach relies on a transformer-based student model that performs well, it can be further optimized to incorporate advantages from CNN architectures. For example, employing alternative techniques such as using ViT adapters \cite{Chen_Duan_etal._2023} or $1 \times 1$ convolutions to adjust feature map sizes may be beneficial for harnessing CNN advantages like enhanced local feature extraction. Moreover, if additional performance improvements are desired, the process can be further enhanced by applying supplementary knowledge distillation techniques, such as self-distillation \cite{Zhang_Song_etal._2019} or online distillation \cite{Houyon_Cioppa_etal._2023}.

Despite these promising results, further validation on independent datasets is necessary to fully understand the model's limitations. Underrepresented classes may pose challenges when addressing complex cases. Pathologists need to validate these models to adopt them in clinical settings. While the distilled models are smaller and more deployable, a technological gap persists because pathologists traditionally rely on established methods for inspecting WSIs and diagnosing diseases. Addressing the complexities involved in deploying models for inference and supporting pathologists in adopting new tools is essential for integrating these models into clinical workflows.

\section{Model integration with QuPath}
Digital pathology tools with graphical user interfaces are essential for visualizing and analyzing WSIs. To make our student model useful in clinical pathology workflows, it needs to be integrated into a tool that enables inspecting regions, creating annotations, and providing quantitative analyses of biomarkers. Therefore, we integrate the trained student model from the previous chapter into the QuPath open‑source platform \cite{Bankhead_Loughrey_etal._2017}. QuPath provides the required annotation, visualization, and analysis tools to interpret complex histological data, including workflows for cell segmentation, classification, and quantification (\hyperref[fig:fig7]{Figure 7}). 

\begin{figure}[h!]
    \centering
    \includegraphics[width=\textwidth]{images/Figure_7.pdf}
    \caption{Visualization of model-generated cell quantification annotations (left) and the corresponding unannotated slide (right) in QuPath}
    \label{fig:fig7}
\end{figure}

To identify the regions in a WSI critical for prognosticating tumor development, such as specific tumor areas or border regions without overlapping healthy tissue, the pathologist uses QuPath to outline these regions. Then, the pathologist initiates a cell segmentation and classification script through the QuPath interface for the selected regions. The resulting annotations and quantified cell information are then directly overlaid onto the WSI in the QuPath interface. Additional design and implementation details are in \hyperref[chap:S7]{Appendix S7}. 

Two common approaches for integrating deep learning models into QuPath are Java‑based native QuPath extensions \cite{Goldsborough_Philps_etal._2024} and the execution of RESTful API requests to a model server coupled with handling the response via an extension, as demonstrated in the application of cell segmentation models applied to immunofluorescence images \cite{Sugawara_2023}. While the community is actively working on these integration strategies, there is currently no universal solution that fully addresses all integration and performance requirements.

Extensions may offer better integration with QuPath, allowing slightly improved performance and more widespread usage of the built-in QuPath models, but they lack the flexibility to customize models and modify their behavior. For example, the newest version of QuPath includes models such as StarDist \cite{Weigert_Schmidt} and InstanSeg \cite{Goldsborough_Philps_etal._2024} that can perform cell segmentation. Both models pose limitations when applied to simultaneous cell segmentation and classification. StarDist performs well only on convex, round shapes by design, whereas some neoplastic, inflammatory, and connective cells exhibit complex and non-convex shapes. InstanSeg provides only semantic segmentation without assigning classes to the segmented cells.

%\hfill

In contrast, our approach offers an alternative integration strategy. It utilizes the paquo library to directly interact with QuPath’s internal application programming interface from within Python. This enables data exchange and processing without the need for intermediate conversion steps and provides greater control over model customization, retraining, and the incorporation of custom processing steps.

The integration of our custom model with QuPath underscores its potential to significantly enhance the diagnostic process by reducing the time burden on pathologists and enabling them to focus on more complex interpretative tasks using familiar software. Leveraging a tool that is already well-established among pathologists increases the likelihood of its adoption into daily clinical workflows. The quantitative data generated through the automated workflow is critical for both clinical decision-making and research, facilitating more accurate biomarker analysis, enabling robust statistical evaluations, and supporting hypothesis generation and testing. Additionally, by streamlining cell segmentation and classification, the tool enhances the scalability and reproducibility of pathological assessments, ultimately contributing to improved diagnostic accuracy and patient outcomes.

\section{Conclusion and future work}

In this study, we address critical challenges in digital pathology and tackle the usability and deployment issues of the developed models in standard computing environments without the need for high-performance computing systems. Our multi-faceted approach encompasses data refinement through cross-relabeling, leveraging foundation models for robust cell segmentation and classification, optimizing model performance via knowledge distillation, and integrating the optimized model into the QuPath software for practical application. This approach is used to construct a capable, versatile, and adjustable model for cell segmentation and classification, with enhanced performance and usability.

\begin{sloppypar}
While our approach shows potential in the field of computational pathology, certain limitations persist. 
For example, our implementation currently exhibits lower performance in detecting macrophages. 
This serves as an instance of the broader challenge of accurately identifying complex cell types. In order to address this issue, extending our approach to incorporate additional data sources, exploring alternative modeling approaches, and integrating other imaging modalities such as immunohistochemical staining may help improve detection accuracy. Moreover, although the distilled model reduces computational demands, integrating advanced deep learning models into clinical practice requires addressing technological gaps and potential resistance to adopting new tools within established diagnostic processes.
\end{sloppypar}

Future work could focus on several key areas to refine the proposed approach and facilitate its adoption in clinical environments. Enhancing the cell-relabeling process with additional datasets \cite{Graham_Jahanifar_etal._2021} could improve the representation of underrepresented cell types and enhance overall model performance. Also, incorporating additional data sources, such as multi-modal imaging or complementary staining methods, may address limitations related to cell type differentiation and class imbalance. Exploring other foundation models \cite{Vorontsov_Bozkurt_etal._2024,Zimmermann_Vorontsov_etal._2024} or introducing additional modalities \cite{Ding_Wagner_etal._2024,Vaidya_Zhang_etal._2025} may provide alternative architectures better suited to specific tasks or offer improved efficiency. Implementing more complex knowledge distillation techniques \cite{Houyon_Cioppa_etal._2023,Zhang_Song_etal._2019} could further optimize the model's performance and adaptability. Additionally, deeper integration with QuPath or other digital pathology software could provide pathologists more control over cell quantification analysis directly within the QuPath interface, thereby increasing accessibility and usability. Such enhancements would not only refine model performance but also ensure greater adaptability and scalability within various clinical environments. Finally, extensive validation of the model by pathologists and benchmarking against independent datasets are essential steps toward establishing the model's reliability and fostering confidence in its clinical utility.

\section*{Acknowledgments} 
This work was funded in part by the Research Council of Norway grant no. 309439 SFI Visual Intelligence, and the North Norwegian Health Authority grant no. HNF1521-20.

\bibliographystyle{IEEEtran}
\begin{sloppypar}
\begin{thebibliography}{99}

\bibitem{chaplot2020neural} Chaplot, Devendra Singh, et al. "Neural topological slam for visual navigation." Proceedings of the IEEE/CVF conference on computer vision and pattern recognition. 2020.

\bibitem{maksymets2021thda} Maksymets, Oleksandr, et al. "Thda: Treasure hunt data augmentation for semantic navigation." Proceedings of the IEEE/CVF International Conference on Computer Vision. 2021.

\bibitem{mezghan2022memory} Mezghan, Lina, et al. "Memory-augmented reinforcement learning for image-goal navigation." 2022 IEEE/RSJ International Conference on Intelligent Robots and Systems (IROS). IEEE, 2022.

\bibitem{al2022zero} Al-Halah, Ziad, Santhosh Kumar Ramakrishnan, and Kristen Grauman. "Zero experience required: Plug \& play modular transfer learning for semantic visual navigation." Proceedings of the IEEE/CVF Conference on Computer Vision and Pattern Recognition. 2022.

\bibitem{ye2021auxiliary} Ye, Joel, et al. "Auxiliary tasks and exploration enable objectgoal navigation." Proceedings of the IEEE/CVF international conference on computer vision. 2021.

\bibitem{chaplot2020object} Chaplot, Devendra Singh, et al. "Object goal navigation using goal-oriented semantic exploration." Advances in Neural Information Processing Systems 33 (2020)

\bibitem{ramakrishnan2022poni} Ramakrishnan, Santhosh Kumar, et al. "Poni: Potential functions for objectgoal navigation with interaction-free learning." Proceedings of the IEEE/CVF Conference on Computer Vision and Pattern Recognition. 2022.

\bibitem{ramrakhya2022habitat} Ramrakhya, Ram, et al. "Habitat-web: Learning embodied object-search strategies from human demonstrations at scale." Proceedings of the IEEE/CVF Conference on Computer Vision and Pattern Recognition. 2022.

\bibitem{mousavian2019visual} Mousavian, Arsalan, et al. "Visual representations for semantic target driven navigation." 2019 International Conference on Robotics and Automation (ICRA). IEEE, 2019.

\bibitem{dhariwal2021diffusion} Dhariwal, Prafulla, and Alexander Nichol. "Diffusion models beat gans on image synthesis." Advances in neural information processing systems 34 (2021)

\bibitem{ho2022classifier} Ho, Jonathan, and Tim Salimans. "Classifier-free diffusion guidance." arXiv preprint arXiv:2207.12598 (2022).

\bibitem{nichol2021glide} Nichol, Alex, et al. "Glide: Towards photorealistic image generation and editing with text-guided diffusion models." arXiv preprint arXiv:2112.10741 (2021)

\bibitem{brooks2023instructpix2pix} Brooks, Tim, Aleksander Holynski, and Alexei A. Efros. "Instructpix2pix: Learning to follow image editing instructions." Proceedings of the IEEE/CVF Conference on Computer Vision and Pattern Recognition. 2023.

\bibitem{fu2023guiding} Fu, Tsu-Jui, et al. "Guiding instruction-based image editing via multimodal large language models." arXiv preprint arXiv:2309.17102 (2023).

\bibitem{geng2024instructdiffusion} Geng, Zigang, et al. "Instructdiffusion: A generalist modeling interface for vision tasks." Proceedings of the IEEE/CVF Conference on Computer Vision and Pattern Recognition. 2024.

\bibitem{zhou2024minedreamer} Zhou, Enshen, et al. "Minedreamer: Learning to follow instructions via chain-of-imagination for simulated-world control." arXiv preprint arXiv:2403.12037 (2024).

\bibitem{zhou2023esc} Zhou, Kaiwen, et al. "Esc: Exploration with soft commonsense constraints for zero-shot object navigation." International Conference on Machine Learning. PMLR, 2023.

\bibitem{yu2023l3mvn} Yu, Bangguo, Hamidreza Kasaei, and Ming Cao. "L3mvn: Leveraging large language models for visual target navigation." 2023 IEEE/RSJ International Conference on Intelligent Robots and Systems (IROS). IEEE, 2023.

\bibitem{gadre2023cows} Gadre, Samir Yitzhak, et al. "Cows on pasture: Baselines and benchmarks for language-driven zero-shot object navigation." Proceedings of the IEEE/CVF Conference on Computer Vision and Pattern Recognition. 2023.

\bibitem{shah2023navigation} Shah, Dhruv, et al. "Navigation with large language models: Semantic guesswork as a heuristic for planning." Conference on Robot Learning. PMLR, 2023.

\bibitem{cai2024bridging} Cai, Wenzhe, et al. "Bridging zero-shot object navigation and foundation models through pixel-guided navigation skill." 2024 IEEE International Conference on Robotics and Automation (ICRA). IEEE, 2024.

\bibitem{yu2023co} Yu, Bangguo, Hamidreza Kasaei, and Ming Cao. "Co-NavGPT: Multi-robot cooperative visual semantic navigation using large language models." arXiv preprint arXiv:2310.07937 (2023).

\bibitem{wu2024voronav} Wu, Pengying, et al. "Voronav: Voronoi-based zero-shot object navigation with large language model." arXiv preprint arXiv:2401.02695 (2024).

\bibitem{qin2023mp5} Qin, Yiran, et al. "Mp5: A multi-modal open-ended embodied system in minecraft via active perception." arXiv preprint arXiv:2312.07472 (2023).

\bibitem{du2024learning} Du, Yilun, et al. "Learning universal policies via text-guided video generation." Advances in Neural Information Processing Systems 36 (2024).

\bibitem{ajay2024compositional} Ajay, Anurag, et al. "Compositional foundation models for hierarchical planning." Advances in Neural Information Processing Systems 36 (2024).

\bibitem{liang2024skilldiffuser} Liang, Zhixuan, et al. "Skilldiffuser: Interpretable hierarchical planning via skill abstractions in diffusion-based task execution." Proceedings of the IEEE/CVF Conference on Computer Vision and Pattern Recognition. 2024.

\bibitem{heusel2017gans} Heusel, Martin, et al. "Gans trained by a two time-scale update rule converge to a local nash equilibrium." Advances in neural information processing systems 30 (2017).

\bibitem{zhang2018unreasonable} Zhang, Richard, et al. "The unreasonable effectiveness of deep features as a perceptual metric." Proceedings of the IEEE conference on computer vision and pattern recognition. 2018.

\bibitem{brown2020language} Brown, Tom B. "Language models are few-shot learners." arXiv preprint arXiv:2005.14165 (2020).

\bibitem{podell2023sdxl} Podell, Dustin, et al. "Sdxl: Improving latent diffusion models for high-resolution image synthesis." arXiv preprint arXiv:2307.01952 (2023).

\bibitem{brohan2022rt} Brohan, Anthony, et al. "Rt-1: Robotics transformer for real-world control at scale." arXiv preprint arXiv:2212.06817 (2022).

\bibitem{brohan2023rt} Brohan, Anthony, et al. "Rt-2: Vision-language-action models transfer web knowledge to robotic control." arXiv preprint arXiv:2307.15818 (2023).

\bibitem{li2024manipllm} Li, Xiaoqi, et al. "Manipllm: Embodied multimodal large language model for object-centric robotic manipulation." Proceedings of the IEEE/CVF Conference on Computer Vision and Pattern Recognition. 2024.

\bibitem{shah2023vint} Shah, Dhruv, et al. "ViNT: A foundation model for visual navigation." arXiv preprint arXiv:2306.14846 (2023).

\bibitem{liu2024visual} Liu, Haotian, et al. "Visual instruction tuning." Advances in neural information processing systems 36 (2024).

\bibitem{hu2021lora} Hu, Edward J., et al. "Lora: Low-rank adaptation of large language models." arXiv preprint arXiv:2106.09685 (2021).

\bibitem{qin2023supfusion} Qin, Yiran, et al. "SupFusion: Supervised LiDAR-camera fusion for 3D object detection." Proceedings of the IEEE/CVF International Conference on Computer Vision. 2023.

\bibitem{qin2024worldsimbench} Qin, Yiran, et al. "Worldsimbench: Towards video generation models as world simulators." arXiv preprint arXiv:2410.18072 (2024).

\bibitem{yu2025gamefactory} Yu, Jiwen, et al. "GameFactory: Creating New Games with Generative Interactive Videos." arXiv preprint arXiv:2501.08325 (2025).

\bibitem{zhou2024code} Zhou, Enshen, et al. "Code-as-Monitor: Constraint-aware Visual Programming for Reactive and Proactive Robotic Failure Detection." arXiv preprint arXiv:2412.04455 (2024).

\bibitem{zhang2024ad} Zhang, Zaibin, et al. "AD-H: Autonomous Driving with Hierarchical Agents." arXiv preprint arXiv:2406.03474 (2024).

\bibitem{wang2024toward} Wang, Chaoqun, et al. "Toward Accurate Camera-based 3D Object Detection via Cascade Depth Estimation and Calibration." arXiv preprint arXiv:2402.04883 (2024).

\bibitem{huang2024story3d} Huang, Yuzhou, et al. "Story3d-agent: Exploring 3d storytelling visualization with large language models." arXiv preprint arXiv:2408.11801 (2024).

\bibitem{savinov2018semi} Savinov, Nikolay, Alexey Dosovitskiy, and Vladlen Koltun. "Semi-parametric topological memory for navigation." arXiv preprint arXiv:1803.00653 (2018).

\bibitem{majumdar2022zson} Majumdar, Arjun, et al. "Zson: Zero-shot object-goal navigation using multimodal goal embeddings." Advances in Neural Information Processing Systems 35 (2022): 32340-32352.

\bibitem{yadav2023offline} Yadav, Karmesh, et al. "Offline visual representation learning for embodied navigation." Workshop on Reincarnating Reinforcement Learning at ICLR 2023. 2023.

\bibitem{yadav2023ovrl} Yadav, Karmesh, et al. "Ovrl-v2: A simple state-of-art baseline for imagenav and objectnav." arXiv preprint arXiv:2303.07798 (2023).

\bibitem{sun2024fgprompt} Sun, Xinyu, et al. "FGPrompt: fine-grained goal prompting for image-goal navigation." Advances in Neural Information Processing Systems 36 (2024).

\bibitem{zhu2017target} Zhu, Yuke, et al. "Target-driven visual navigation in indoor scenes using deep reinforcement learning." 2017 IEEE international conference on robotics and automation (ICRA). IEEE, 2017.

\bibitem{koh2024generating} Koh, Jing Yu, Daniel Fried, and Russ R. Salakhutdinov. "Generating images with multimodal language models." Advances in Neural Information Processing Systems 36 (2024).

\bibitem{krantz2022instance} Krantz, Jacob, et al. "Instance-specific image goal navigation: Training embodied agents to find object instances." arXiv preprint arXiv:2211.15876 (2022).

\bibitem{schulman2017proximal} Schulman, John, et al. "Proximal policy optimization algorithms." arXiv preprint arXiv:1707.06347 (2017).

\bibitem{anderson2018evaluation} Anderson, Peter, et al. "On evaluation of embodied navigation agents." arXiv preprint arXiv:1807.06757 (2018).

\bibitem{lin2024navcot} Lin, Bingqian, et al. "NavCoT: Boosting LLM-Based Vision-and-Language Navigation via Learning Disentangled Reasoning." arXiv preprint arXiv:2403.07376 (2024).

\bibitem{NavGPT} Zhou, Gengze, Yicong Hong, and Qi Wu. "Navgpt: Explicit reasoning in vision-and-language navigation with large language models." Proceedings of the AAAI Conference on Artificial Intelligence.

\bibitem{hahn2021no} Hahn, Meera, et al. "No rl, no simulation: Learning to navigate without navigating." Advances in Neural Information Processing Systems 34 (2021): 26661-26673.

\bibitem{li2025t2isafety} Li, Lijun, et al. "T2ISafety: Benchmark for Assessing Fairness, Toxicity, and Privacy in Image Generation." arXiv preprint arXiv:2501.12612 (2025).

\bibitem{an2024agfsync} An, Jingkun, et al. "AGFSync: Leveraging AI-Generated Feedback for Preference Optimization in Text-to-Image Generation." arXiv preprint arXiv:2403.13352 (2024).


\end{thebibliography}
\end{sloppypar}

\clearpage
\beginsupplement
\section*{Appendix}
\renewcommand{\thesubsection}{S\arabic{subsection}}

\subsection{\label{chap:S1}PanNuke and MoNuSAC preprocessing}
The PanNuke dataset comprises a set of 7,901 RGB patches, each with dimensions of $256 \times 256$ pixels, which we set as the standard patch size for our analysis. In contrast, the MoNuSAC dataset encompasses 294 images of heterogeneous dimensions. To standardize the MoNuSAC images with our experiments, we implement a standardization protocol. Specifically, for images exceeding the dimensions of $256 \times 256$ pixels, we segment them into equal-sized patches and apply mirror padding to the remaining portions to avoid information loss at the peripherals. Patches with dimensions less than $128 \times 128$ pixels are excluded from the dataset due to the insufficient resolution to capture relevant cellular details. For patches where either dimension falls between 128 and 256 pixels, we employ upsampling to achieve the standard patch size. As a result, we obtain a total of 2,823 RGB patches derived from the MoNuSAC dataset for subsequent analysis. For additional details on the MoNuSAC data preparation process, refer to the source code \cite{Shvetsov_2025a}.
\clearpage

\subsection{\label{chap:S2}Data usage for the methodology}

\counterwithin{figure}{subsection}
\renewcommand{\thefigure}{S\arabic{subsection}}

\begin{figure}[h!]
    \centering
    \includegraphics[width=\textwidth, height=0.85\textheight, keepaspectratio]{images/A2.pdf}
    \caption{Overview of the methodology for cross-labeling, dataset refinement, and model comparison. (1) Cross-relabeling - training and testing cell classification models, (2) Cross-relabeling - using cell classification models to create refined dataset, (3) Fine-tuning and training models for comparison, (4) Student knowledge distillation with refined dataset}
    \label{fig:S2}
\end{figure}
\clearpage

\subsection{\label{chap:S3}Confusion matrices for classification models}
\counterwithin{figure}{subsection}
\renewcommand{\thefigure}{S\arabic{subsection}.\arabic{figure}}

\begin{figure}[h!]
    \centering
    \includegraphics[width=\textwidth, height=0.4\textheight, keepaspectratio]{images/A3_1.pdf}
    \caption{Confusion matrix for PanNuke trained model}
    \label{fig:S3.1}
\end{figure}

\begin{figure}[h!]
    \centering
    \includegraphics[width=\textwidth, height=0.4\textheight, keepaspectratio]{images/A3_2.pdf}
    \caption{Confusion matrix for MoNuSAC trained model}
    \label{fig:S3.2}
\end{figure}

\clearpage

\subsection{\label{chap:S4}Datasets cell counts}

\counterwithin{table}{subsection}
\renewcommand{\thetable}{S\arabic{subsection}}

\begin{table}[h!]
\renewcommand{\arraystretch}{2.0}
\centering
\caption{\label{tab:S4}Cell counts for PanNuke, MoNuSAC and refined datasets. Numbers in parentheses indicate preprocessed cell counts for cell classifier models training and testing.}
%\adjustbox{max width=\textwidth}{%
\begin{tabular}{|l|c|c|c|}
\hline
%\rowcolor{gray!30}
Cell type & PanNuke & MoNuSAC & Refined \\
\hline
Neoplastic & 77,403 (68,031) & - & 105,451 \\
\hline
Epithelial & 26,572 (23,207) & - & 29,926 \\
\hline
Epithelial (benign and malignant) & - & 31,402 & - \\
\hline
Inflammatory & 32,276 & - & - \\
\hline
Lymphocytes & - & 37,045 (33,104) & 65,275 \\
\hline
Neutrophils & - & 1,355 (1,252) & 3,833 \\
\hline
Macrophage & - & 1,842 (1,695) & 3,410 \\
\hline
Dead & 2,908 & - & 2,908 \\
\hline
Connective & 50,585 & - & 50,585 \\
\hline
\end{tabular}
%
%}
\end{table}



\clearpage

\subsection{\label{chap:S5}Definition of validation metrics}
\counterwithin{equation}{subsection}
\renewcommand{\theequation}{\arabic{equation}}

\subsubsection{\label{chap:S5.1}R\textsuperscript{2}}
The coefficient of determination, denoted as $R^2$, is a statistical measure that represents the proportion of variance in the dependent variable that is predictable from the independent variables. In the context of cell quantification in pathology, $R^2$ is used to assess how well the predicted quantities of different cell types in a patch align with the actual quantities observed in the ground truth data, with higher values representing more accurate quantification. $R^2$ is defined as
\begin{equation*}
R^2 = 1 - \frac{\sum_{i=1}^n (y_i - \hat{y}_i)^2}{\sum_{i=1}^n (y_i - \bar{y})^2},
\end{equation*}
where $y_i$ represents the actual number of cells of a specific type in the $i$-th image, $\hat{y}_i$ represents the predicted number of cells of that type in the $i$-th image, $\bar{y}$ is the mean of the actual numbers across all images, and $n$ is the total number of images in the dataset.

The $R^2$ metric has a range of $(-\infty, 1]$. An $R^2$ of 1 indicates perfect prediction, where all predicted values exactly match the actual values. An $R^2$ of 0 suggests that the model explains none of the variability of the response data around its mean. If $R^2$ is negative, it indicates that the model performs worse than a model that simply predicts the mean of the actual values for all observations.

\subsubsection{\label{chap:S5.2}PQ}
Panoptic Quality ($PQ$) is a comprehensive metric used to evaluate the performance of segmentation models in tasks that require both instance segmentation and classification. $PQ$ provides a single score that encapsulates both the detection accuracy (i.e., how many objects were correctly identified) and the segmentation quality (i.e., how accurately the objects' boundaries were delineated). This metric is particularly useful in multiclass scenarios where each pixel is classified into distinct categories, such as different cell types in pathology images.

$PQ$ is calculated as the product of two terms: Detection Quality ($DQ$) and Segmentation Quality ($SQ$). It can be expressed as
\begin{equation*}
PQ = DQ \cdot SQ,
\end{equation*}
where
\begin{equation*}
DQ = \frac{TP}{TP + 0.5\, FP + 0.5\, FN},
\end{equation*}
\begin{equation*}
SQ = \frac{\sum_{(p, g) \in \mathcal{M}} IoU(p, g)}{TP}.
\end{equation*}
In these formulas, $TP$ denotes the number of correctly matched instances between ground truth and prediction, $FP$ denotes the predicted instances that have no corresponding ground truth, $FN$ denotes the ground truth instances that were not detected, $IoU(p, g)$ is the Intersection over Union for a pair of matched instances $p$ (prediction) and $g$ (ground truth), and $\mathcal{M}$ is the set of matched pairs.

The $PQ$ metric is calculated for each class and is averaged across classes to provide a global performance measure.

The $PQ$ score has a range of $[0, 1.0]$, where a higher score indicates better performance in both detecting and segmenting the instances correctly. A $PQ$ of 1 signifies perfect identification and segmentation of all instances, whereas a $PQ$ of 0 indicates that no instances were correctly identified and segmented.

\clearpage

\subsection{\label{chap:S6}Segmentation and Detection quality metrics for teacher and student models}

\begin{table}[h!]
\renewcommand{\arraystretch}{2.0}
\centering
\caption{Segmentation and detection quality for student and teacher models (CI 95\%)}
\label{tab:S6}
%\adjustbox{max width=\textwidth}{%
\begin{tabular}{|l|c|c|}
\hline
%\rowcolor{gray!30}
Metric & Teacher & Student \\
\hline
$SQ_{neoplastic}$ & 0.819 (0.815--0.823) & 0.824 (0.819--0.828) \\
\hline
$SQ_{lymphocyte}$ & 0.795 (0.788--0.802) & 0.790 (0.783--0.796) \\
\hline
$SQ_{connective}$ & 0.770 (0.762--0.776) & 0.780 (0.772--0.786) \\
\hline
$SQ_{dead}$ & 0.659 (0.623--0.688) & 0.657 (0.624--0.695) \\
\hline
$SQ_{epithelial}$ & 0.780 (0.770--0.790) & 0.788 (0.779--0.797) \\
\hline
$SQ_{macrophage}$ & 0.788 (0.760--0.810) & 0.757 (0.730--0.783) \\
\hline
$SQ_{neutrofil}$ & 0.782 (0.761--0.801) & 0.775 (0.759--0.792) \\
\hline
$DQ_{neoplastic}$ & 0.706 (0.692--0.719) & 0.727 (0.712--0.741) \\
\hline
$DQ_{lymphocyte}$ & 0.675 (0.656--0.698) & 0.713 (0.691--0.734) \\
\hline
$DQ_{connective}$ & 0.566 (0.546--0.584) & 0.583 (0.565--0.602) \\
\hline
$DQ_{dead}$ & 0.410 (0.361--0.465) & 0.435 (0.306--0.561) \\
\hline
$DQ_{epithelial}$ & 0.668 (0.639--0.694) & 0.673 (0.644--0.702) \\
\hline
$DQ_{macrophage}$ & 0.657 (0.583--0.727) & 0.615 (0.531--0.703) \\
\hline
$DQ_{neutrofil}$ & 0.691 (0.625--0.753) & 0.729 (0.679--0.778) \\
\hline
\end{tabular}
%
%}
\end{table}

\clearpage

\subsection{\label{chap:S7}QuPath integration method}
We adopt an integration strategy leveraging the paquo \cite{Bayer_AG} library, a Python package that enables direct interaction with QuPath’s internal API, thereby facilitating seamless data exchange without intermediate conversion steps. The data processing pipeline (\hyperref[fig:S7]{Appendix Figure S7}) begins with the acquisition of WSIs and their associated annotations from QuPath, which are represented as Shapely \cite{Gillies_Wel_etal._2024} polygons. Utilizing paquo, we directly read, create, and modify these annotations and detections within a QuPath project in the Python environment. Images are then cropped using these polygons and processed by cell segmentation and classification models employing standard vision processing toolkits such as OpenCV, pyvips, and PyTorch. Additionally, QuPath employs Groovy scripts to initiate a Python process that starts the entire pipeline from QuPath graphical interface: fetching polygons, extracting images from them, and running deep learning model inference on the cropped images. 
The results are returned to QuPath, leveraging paquo's Python bindings to manipulate QuPath data while minimizing the computational overhead typically associated with cross-environment communication.

\counterwithin{figure}{subsection}
\renewcommand{\thefigure}{S\arabic{subsection}}

\begin{figure}[h!]
    \centering
    \includegraphics[width=\textwidth]{images/A7.pdf}
    \caption{QuPath integration workflow using Python environment}
    \label{fig:S7}
\end{figure}

Compared to traditional workflows that involve exporting annotations as GeoJSON, classifying them in Python, and reimporting them into QuPath, our approach offers several advantages. We eliminate the need to switch between programming languages, providing a cohesive and streamlined development process entirely within QuPath software and removing the necessity to use other tools. Meanwhile, we avoid storing annotations as intermediate JSON files unless required for external use or archiving. By conducting the entire inference and post-processing workflow within the Python environment, we leverage the power and flexibility of Python libraries for image processing and machine learning. This approach also enables adjustments to any set of labels and models, thereby improving its applicability.

%\hfill

The distilled model and QuPath integration code are packaged into a Docker container, enabling streamlined execution with the Docker engine. Detailed integration code and deployment instructions can be found in the GitHub repository \cite{Shvetsov_2025b}.

Despite these benefits, we acknowledge that the paquo library is a proof‑of‑concept project in its early development stage and has not been tested across all versions of QuPath.

\clearpage

\subsection{\label{chap:S8}Data and code availability statement}
All datasets, models, and code used in this study are publicly available and can be obtained from the repositories listed below. 
The PanNuke \cite{Gamper_Koohbanani_etal._2019} and MoNuSAC \cite{Verma_Kumar_etal._2021} datasets are publicly accessible, and download information along with detailed descriptions can be found in their respective articles. Preprocessing scripts for PanNuke and MoNuSAC data, as well as individual cell extraction scripts, are available on GitHub \cite{Shvetsov_2025a}. The H-Optimus foundation model used in our experiments can be downloaded from the HuggingFace repository \cite{hoptimus2024}, and model information is available on GitHub \cite{Saillard_Jenatton_etal._2024}. In addition, the integration code for QuPath and the distilled model packaged in a Docker container are provided in the repository \cite{Shvetsov_2025b}, and paquo Python library is available from the authors GitHub repository \cite{Bayer_AG}.
\clearpage

\end{document}


\subsection{Experimental Setup}\label{subsec:setup}
\textbf{Datasets.} We evaluate the performance of \ourmethod on two real-world benchmarks, GOOD~\citep{good} and DrugOOD~\citep{drugood}, with various distribution shifts to evaluate our method. Specifically, GOOD is a comprehensive graph OOD benchmark, and we selected three datasets: (1) GOOD-HIV~\citep{wu2018moleculenet}, a molecular graph dataset predicting HIV inhibition; (2) GOOD-CMNIST~\citep{arjovsky2019invariant}, containing graphs transformed from MNIST using superpixel techniques; and (3) GOOD-Motif~\citep{wu2022discovering}, a synthetic dataset where graph motifs determine the label. DrugOOD is designed for AI-driven drug discovery with three types of distribution shifts: scaffold, size, and assay, and applies these to two measurements (IC50 and EC50). Details of datasets are in Appendix \ref{appe:data}.%applied to six molecular datasets for predicting drug-target binding affinity.
%We employ two real-world benchmarks containing various distribution shifts for graph OOD generalization to evaluate the effectiveness of our method.
% \begin{itemize}
%     \item \textbf{GOOD} is a systematic and comprehensive graph OOD benchmark, offering detailed distributions partition across various types of graphs. For the graph classification task, we selected three distinct datasets: (1) GOOD-HIV, a molecular graph dataset with the task of binary classification to predict whether a molecule can inhibit HIV; (2) GOOD-CMNIST, which consists of graphs representing hand-written digits transformed from the MNIST database using superpixel techniques; and (3) GOOD-Motif, a synthetic dataset where each graph is formed by connecting a base graph with a motif, where the motif alone determines the label.
%     \item \textbf{DrugOOD} is an OOD benchmark designed specifically for AI-driven drug discovery, where the data consists of molecular graphs. DrugOOD includes three basis of distribution shift: scaffold, size, and assay, and applies these to two measurements (IC50 and EC50). The benchmark contains six datasets, and all of them are tasked with predicting drug-target binding affinity, framed as a binary classification problem.
% \end{itemize}

\noindent\textbf{Baselines}.  We compare \ourmethod against ERM and two kinds of OOD baselines: (1)~Traditional OOD generalization approaches, including  Coral~\citep{coral}, IRM~\citep{arjovsky2019invariant} and VREx~\citep{krueger2021out}; (2)~graph-specific OOD generalization methods, including environment-based approaches (MoleOOD~\citep{yang2022learning}, CIGA~\citep{chen2022learning}, GIL~\citep{li2022learning}, and GREA~\citep{liu2022graph}, IGM~\citep{jia2024graph}), causal explanation-based approaches (Disc~\citep{fan2022debiasing} and DIR\citep{wu2022discovering}), and advanced architecture-based approaches (CAL~\citep{sui2022causal} and GSAT~\citep{miao2022interpretable}, iMoLD~\citep{zhuang2023learning}), GALA~\citep{Equad}, EQuAD~\citep{gala}. Details of all baselines are in Appendix \ref{appe:baseline}.

\noindent\textbf{Implementation Details}. To ensure fairness, we adopt the same experimental setup as iMold across two benchmarks. For molecular datasets with edge features, we use a three-layer GIN with a hidden dimension of 300, while for non-molecular graphs, we employ a four-layer GIN with a hidden dimension of 128. The projector is a two-layer MLP with a hidden dimension set to half that of the GIN encoder. EMA rate $\alpha$ for prototype updating is fixed at 0.99. Adam optimizer is used for model parameter updates.  All baselines use the optimal parameters from their original papers. Additional hyperparameter details can be found in Appendix~\ref{appe:hyperparam}.

\subsection{Performance Comparison}\label{subsec:results}

In this experiment, we aim to answer
\textbf{Q1: Whether \ourmethod achieves the best performance on OOD generalization benchmarks?} 
The answer is \textbf{YES}, since \ourmethod shows the best results on the majority of datasets. Specifically, we have the following observations.

 \noindent$\rhd$ \textsf{State-of-the-art results.}
According to Table~\ref{tab:main}, \ourmethod achieves state-of-the-art performance on 9 out of 11 datasets, and secures the second place on the remaining dataset. The average improvements against the previous SOTA are $2.17\%$ on GOOD and $1.68\%$ on DrugOOD. Notably, \ourmethod achieves competitive performance across various types of datasets with different data shifts, demonstrating its generalization ability on different data. Moreover, our model achieves the best results in both binary and multi-class tasks, highlighting the effectiveness of the multi-prototype classifier in handling different classification tasks.

 \noindent$\rhd$ \textsf{Sub-optimal performance of environment-based methods.}
Among all baselines, environment-based methods only achieve the best performance on 3 datasets, while architecture-based OOD generalization methods achieve the best results on most datasets. These observations suggest that environment-based methods are limited by the challenge of accurately capturing environmental information in graph data, leading to a discrepancy between theoretical expectations and empirical results. In contrast, the remarkable performance of \ourmethod also proves that graph OOD generalization can still be achieved without specific environmental information.
% According to Table~\ref{tab:main}, \ourmethod shows \textit{state-of-the-art performance on 10 out of 11 datasets} and secures second place on the remaining dataset. Notably, \ourmethod achieves competitive performance across various types of datasets with different data shifts, demonstrating its superior ability to achieve environment invariance. The superior performance of \ourmethod in both binary and multi-class tasks highlights the strength of the multi-prototype-based classification approach. In contrast, we find that the \textit{environment-based methods display sub-optimal performance} in most cases, demonstrating their limitations in capturing the environments. The remarkable performance of \ourmethod also proves that graph OOD generalization can still be achieved without specific environmental information.
\begin{figure*}[!t]
    \centering
    \subfigure[Sensitivity of $k$]{ \label{subfig:paramk}
    \includegraphics[height=0.16\textwidth]{4_exp/param_hiv_k.pdf}
    }
    \hfill
    \subfigure[Sensitivity of $\beta$]{ \label{subfig:paramb}
    \includegraphics[height=0.16\textwidth]{4_exp/param_hiv_beta.pdf}
    }
    \hfill
    \subfigure[Impact of prototype updating mechanisms]{ \label{subfig:update}
    \includegraphics[height=0.16\textwidth]{4_exp/init.pdf}
    }
    \hfill
    \subfigure[Impact of different statistical metrics]{ \label{subfig:metric}
    \includegraphics[height=0.16\textwidth]{4_exp/std.pdf}
    }
    \vspace{-4mm}
    \caption{The two figures on the left present a hyperparameter analysis of the  $K$ and $\beta$, while the two figures on the right illustrate the comparison of different module designs on prototype update and metric used in Eq.~\eqref{classify}. } 
    \vspace{-2mm}
    \label{fig:four_images}
    
\end{figure*}
\subsection{Ablation Study}
We aim to discover 
\textbf{Q2: Does each module in \ourmethod contribute to effective OOD generalization?} The answer is \textbf{YES}, as removing any key component leads to performance degradation, as demonstrated by the results in Table~\ref{tab:ablation}. We have the following discussions.%according to our ablation experiments that verify the effectiveness of the loss constraint (i.e., $\mathcal{L}_{\mathrm{IPM}}$ and $\mathcal{L}_{\mathrm{PS}}$), and the components of our model, including hyperspherical projection, multi-prototype mechanism, weight updating, and weight pruning techniques. The results are shown in Table~\ref{tab:ablation}, with the following discussions.

% as we conduct experiments on three datasets to verify the role of our proposed loss constraint $\mathcal{L}_{\mathrm{IPM}}$, $\mathcal{L}_{\mathrm{PS}}$, and the component of our model: projector and weight pruning technique. The results are shown in Table~\ref{ablation}.
\begin{table*}
  [t]
  \centering
  \resizebox{\textwidth}{!}{%
  \begin{tabular}{cccccccccccc}
    \toprule \multicolumn{2}{c}{Components}                                                             & \multicolumn{5}{c}{Re-executability Rate (\%)} & \multicolumn{5}{c}{Readability (\#)} \\
    \cmidrule(lr){1-2} \cmidrule(lr){3-7} \cmidrule(lr){8-12}        \hspace{8pt}\labelemoji\hspace{8pt}                                                                & \hspace{8pt}\toolemoji\hspace{8pt}                                      & O0                                 & O1             & O2             & O3             & AVG            & O0             & O1             & O2             & O3             & AVG            \\
    \hline
    \rowcolor[rgb]{0.93,0.93,0.93}\multicolumn{12}{c}{\textbf{Initialize with LLM4Decompile-End-6.7B~\citep{llm4decompile}}}   \\
    \xmark                                                                                              & \xmark                                    & 69.51                              & 46.95          & 50.61          & 46.34          & 53.35          & 3.98 & 3.41 & 3.44 & 3.38 & 3.55 \\
    \cmark                                                                                              & \xmark                                    & 75.61                              & 50.61          & 50.00          & 50.00          & 56.55          & 4.01 & 3.44 & 3.39 & \textbf{3.49} & 3.58 \\
    \xmark                                                                                              & \cmark                                    & 83.54                     & \textbf{56.10}          & 51.22          & 50.61 & 60.37 & 4.05 & 3.51 & 3.51 & 3.42 & 3.62 \\
    \cmark                                                                                              & \cmark                                    & \textbf{85.37}                            & \textbf{56.10}                     & \textbf{51.83} & \textbf{52.43}          & \textbf{61.43} & \textbf{4.13} & \textbf{3.60} & \textbf{3.54} & \textbf{3.49} & \textbf{3.69} \\

    \rowcolor[rgb]{0.93,0.93,0.93}\multicolumn{12}{c}{\textbf{Initialize with Deepseek-Coder-6.7B-base~\citep{deepseekcoder}}} \\
    \xmark                                                                                              & \xmark                                    & 59.15                              & 35.98          & 39.02          & 37.80          & 42.99          & 3.71 & 3.05 & 3.16 & 3.05 & 3.24 \\
    \cmark                                                                                              & \xmark                                    & 66.46                              & 41.46          & 38.41          & 36.59          & 45.73          & 3.76 & 3.17 & \textbf{3.21} & 3.08 & 3.31 \\
    \xmark                                                                                              & \cmark                                    & 70.73                              & 39.63          & 39.02          & 40.24          & 47.41          & 3.90 & 3.17 & 3.08 & 3.11 & 3.31 \\
    \cmark                                                                                              & \cmark                                    & \textbf{79.88}                     & \textbf{45.73} & \textbf{43.90} & \textbf{42.68} & \textbf{53.05} & \textbf{3.96} & \textbf{3.21} & 3.18 & \textbf{3.19} & \textbf{3.38} \\
    \bottomrule
  \end{tabular}%
  }
  \caption{The ablation study of different methods across four optimization levels
  (O0, O1, O2, O3), as well as their average scores (AVG). The results in bold represent the optimal performance. The ~\labelemoji~ and ~\toolemoji~ means Relabedling and Function Call. \textbf{Bold} denotes the best performance.}
  \label{tab:ablation}
\end{table*}
\noindent$\rhd$ \textsf{Ablation on $\mathcal{L}_{\mathrm{IPM}}$ and $\mathcal{L}_{\mathrm{PS}}$.}
We remove $\mathcal{L}_{\mathrm{IPM}}$ and $\mathcal{L}_{\mathrm{PS}}$ in the Eq.~\eqref{eq: target2} respectively to explore their impacts on the performance of graph OOD generalization. The experimental results demonstrate a clear fact: merely optimizing for invariance (w/o $\mathcal{L}_{\mathrm{PS}}$) or separability (w/o $\mathcal{L}_{\mathrm{IPM}}$) weakens the OOD generalization ability of our model, especially for the multi-class classification task, as shown in CMNIST-color. This provides strong evidence that ensuring both invariance and separability is a sufficient and necessary condition for effective OOD generalization in graph learning.
\noindent$\rhd$ \textsf{Ablation on the design of \ourmethod.} To verify the effectiveness of each module designed for \ourmethod,
we conducted ablation studies by removing the hyperspherical projection(w/o Project), multi-prototype mechanism (w/o Multi-P), invariant encoder (w/o Inv.Enc), and prototype-related weight calculations (w/o Update) and pruning (w/o Prune). The results confirm their necessity. First, removing the hyperspherical projection significantly drops performance, as optimizing Eq.~(\ref{eq: target2}) requires hyperspherical space. Without it, results are even worse than ERM. Similarly, setting the prototype count to one blurs decision boundaries and affects the loss function $\mathcal{L}_{\mathrm{PS}}$, compromising inter-class separability. Lastly, replacing the invariant encoder $\mathrm{GNN}_{S}$ with $\mathrm{GNN}_{E}$ directly introduces environment-related noise, making it difficult to obtain effective invariant features, thus hindering OOD generalization.
Additionally, the removal of prototype-related weight calculations and weight pruning degraded prototypes into the average of all class samples, resulting in the prototypes degrading into the average representation of all samples in the class, failing to maintain classification performance in OOD scenarios.

\subsection{Visualized Validation}
In this subsection, we aim to investigate \textbf{Q3: Can these key designs (i.e., hyperspherical space and multi-prototype mechanism) tackle two unique challenges in graph OOD generalization tasks?} The answer is \textbf{YES}, we conduct the following visualization experiments to verify this conclusion. %\textbf{Q3: Whether hyperspherical Spaces are more separable than ordinary latent Spaces} Yes, the visualization shows the advantage of hyperspherical space over traditional latent space.

\noindent$\rhd$ \textsf{Hyperspherical representation space.} To validate the advantage of hyperspherical space in enhancing class separability, we compare the 1-order Wasserstein distance~\cite{villani2009optimal} between same-class and different-class samples, as shown in Fig.~\ref{wl_Dis}.
It is evident that \ourmethod produces more separable invariant representations (higher inter-class distance), while also exhibiting tighter clustering for samples of the same class (lower intra-class distance). In contrast, although traditional latent spaces-based SOTA  achieves a certain level of intra-class compactness, its lower separability hinders its overall performance. Additionally, we visualized the sample representations learned by our \ourmethod and SOTA using t-SNE in Fig.~\ref{tsne}, where corresponding phenomenon can be witnessed.
% we visualized the invariant representation 
% $\hat{z}_{inv}$ in both the training set (ID) and the test set (OOD). Using t-SNE, we visualize the representation distributions learned by \ourmethod and the SOTA method, CIGA, as shown in Fig.~\ref{fig:tsne}. 
% \begin{figure*}[t!] \label{v_prototype}
% \centering    
% \includegraphics[scale=0.39]{3_method/sx_prototype.pdf}
% \caption{In the case of the DrugOOD-IC50-assay (binary classification task), we set up three prototypes for each class and visualized the closest example to it. }  
% \label{fig:intro} 
% \end{figure*}





\begin{figure}[t]
\vspace{-3mm}
\centering 
\subfigure[1-order Wasserstein distance]{ \label{wl_Dis}
\includegraphics[width=0.17\textwidth,height=0.13\textwidth]{4_exp/wl_score.pdf}
}
\subfigure[T-SNE visualization, left: \ourmethod, right: SOTA ]{ \label{tsne}
\includegraphics[width=0.13\textwidth,height=0.13\textwidth]{4_exp/my.pdf}
\includegraphics[width=0.13\textwidth,height=0.13\textwidth]{4_exp/imold.pdf}
}
\vspace{-3mm}
\caption{Visualization and quantitative analysis of the separability advantages in hyperspherical space on HIV-scaffold.}
\vspace{-4mm}
\end{figure}

% \begin{figure}[t]
%     \centering
%     % \captionsetup[wragfigure]{ font=footnotesize}

%     \includegraphics[width=0.45\textwidth]{4_exp/wl_score.pdf}

%     \caption{The 1-order WL distance between samples of the same class (D(Y=0), D(Y=1)) and between samples of different classes (inter-distance).}
%     \vspace{-10pt}
%     \label{fig:wl}
% \end{figure}
\noindent$\rhd$ \textsf{Prototypes visualization.} We also reveal the characteristics of prototypes by visualizing samples that exhibit the highest similarity to each prototype. Fig.~\ref{fig:proto} shows that prototypes from different classes capture distinct invariant subgraphs, ensuring a strong correlation with their respective labels. Furthermore, within the same category, different prototypes encapsulate samples with varying environmental subgraphs. This validates that multi-prototype learning can effectively capture label-correlated invariant features without explicit environment definitions, which solve the challenges of out-of-distribution generalization in real-world graph data.
\begin{figure}[t]
    \centering
    % \captionsetup[wragfigure]{ font=footnotesize}

    \includegraphics[width=0.45\textwidth]{4_exp/proto.pdf}

    \caption{Visualizations of prototypes and invariant subgraphs~(highlighted) of IC50-assay dataset.}
    \label{fig:proto}
\vspace{-5mm}
\end{figure}

\subsection{In-Depth Analysis}
In this experiment, we will investigate \textbf{Q4: How do the details (hyperparameter settings and variable designs) in \ourmethod impact performance?} The following experiments are conducted to answer this question and the experimental results are in Fig.~\ref{fig:four_images}.

\noindent$\rhd$ \textsf{Hyperparameter Analysis.}
To investigate the sensitivity of the number of prototypes and the coefficient $\beta$ in $\mathcal{L}_{\mathrm{IPM}}$ on performance, we vary $k$ from 2 to 6 and $\beta$ from $\{0.01, 0.05, 0.1, 0.2, 0.3\}$. Our conclusions are as follows: \ding{192} In Fig.~\ref{subfig:paramk}, the best performance is achieved when the number of prototypes is approximately twice the number of classes. Deviating from this optimal range, either too many or too few prototypes negatively impacts the final performance. \ding{193} According to Fig.~\ref{subfig:paramb}, a smaller $\beta$  hampers the model’s ability to effectively learn invariant features, while selecting a moderate $\beta$ leads to the best performance.
 
% \noindent$\rhd$ \textsf{Numerical quantitative analysis} Analysis of $\beta$. To discover the sensitivity of \ourmethod to coefficient $\beta$ in $\mathcal{L}_{\mathrm{IPM}}$, we search $\beta$ from $\{0.01, 0.05, 0.1, 0.2, 0.3\}$ and present the results in Fig.~\ref{hyper_3} and \ref{hyper_4}. We observe that a small $\beta$ (e.g., 0.01 and 0.05) hampers the model’s ability to effectively learn invariant features, while selecting a moderate $\beta$ (i.e., 0.1) leads to the best performance.

\noindent$\rhd$ \textsf{Module design analysis.}
To investigate the impact of different prototype update mechanisms and statistical metrics in Eq.~\eqref{classify}, we conducted experimental analyses and found that \ding{192} According to Fig.~\ref{subfig:update}, all compared methods lead to performance drops due to their inability to ensure that the updated prototypes possess both intra-class diversity and inter-class separability, which is the key to the success of MPHIL's prototype update method. \ding{193} In Fig.~\ref{subfig:metric}, $\max$ achieves the best performance by selecting the most similar prototype to the sample, helping the classifier converge faster to the correct decision space.
% To discover the sensitivity of \ourmethod to coefficient $\beta$ in $\mathcal{L}_{\mathrm{IPM}}$, we search $\beta$ from $\{0.01, 0.05, 0.1, 0.2, 0.3\}$ and present the results in Fig.~\ref{hyper_3} and \ref{hyper_4}. We observe that a small $\beta$ (e.g., 0.01 and 0.05) hampers the model’s ability to effectively learn invariant features, while selecting a moderate $\beta$ (i.e., 0.1) leads to the best performance.



% \begin{figure*}[h]
% \centering 
% \subfigure[CIGA] { 
% \includegraphics[width=0.4\textwidth]{imold.pdf} \label{t1}
% }
% \subfigure[Ours] { 
% \includegraphics[width=0.4\textwidth]{my(1).pdf} \label{t2}
% }
% \caption{Left, Middle:} 
% \label{t-sne}
% \vspace{-1pt}
% \end{figure*}


\section{Conclusion}
\section*{Conclusion}
This paper aims to enhance our understanding of the computational complexity of computing various Shapley value variants. We found that for various ML models --- including decision trees, regression tree ensembles, weighted automata, and linear regression --- both local and global interventional and baseline SHAP can be computed in polynomial time under HMM modeled distributions. This extends popular algorithms, such as TreeSHAP, beyond their empirical distributional scope. We also establish strict complexity gaps between the various SHAP variants (baseline, interventional, and conditional) and prove the intractability of computing SHAP for tree ensembles and neural networks in simplified scenarios. Overall, we present SHAP as a versatile framework whose complexity depends on four key factors: \begin{inparaenum}[(i)] \item model type, \item SHAP variant, \item distribution modeling approach, \item and local vs. global explanations\end{inparaenum}. We believe this perspective provides deeper insight into the computational complexity of SHAP, paving the way for future work.




%We believe that our framework provides a more intricate understanding of SHAP computation complexity across different models, distributions, and variants, paving the way for further research.

Our work opens promising directions for future research. First, expanding our computational analysis to other SHAP-related metrics, such as asymmetric SHAP~\citep{frye20} and SAGE~\citep{covert2020understanding}, would be valuable. Additionally, we aim to explore more expressive distribution classes and relaxed assumptions beyond those in Section \ref{sec:tractable} while maintaining tractable SHAP computation. Finally, when exact computation is intractable (Section \ref{sec:intractable}), investigating the approximability of SHAP metrics through approximation and parameterized complexity theory~\citep{downey2012parameterized} is an important direction.

%Our work opens several promising avenues for future research on the computational properties of explainable AI methods, with a particular focus on SHAP. First, it would be interesting to broaden the computational analysis conducted in this work to include other popular SHAP-related metrics in the literature, such as asymmetric SHAP \cite{frye20} and SAGE \cite{covert2020understanding}. Also, in the future, we aim to explore more expressive distribution classes and relaxed distributional assumptions—extending beyond those examined in Section \ref{sec:tractable} —that still yield tractable SHAP computation. Finally, when exact computation proves intractable (Section \ref{sec:intractable}), it is worthwhile to theoretically investigate the question of the approximability of computing the SHAP metrics across various configurations, through the lens of approximation and parametrized complexity theory \cite{arora2009computational}.

%This paper aims to deepen our understanding of the computational complexity involved in obtaining different Shapley value variants. We found that for a variety of ML models, including decision trees, tree ensembles for regression, weighted automata, and linear regression models — computing both local and global interventional and baseline SHAP can be done in polynomial time when distributions are modeled by HMMs. This extends the distributional scope of popular algorithms like TreeSHAP, which is limited to empirical distributions. Additionally, we demonstrate a strict complexity gap between SHAP variants, showing that interventional and baseline SHAP can be strictly easier to compute than conditional SHAP. Despite these positive results, we uncovered intractability for various SHAP variants in neural networks and tree ensembles. Finally, we provided generalized complexity relations across SHAP variants. We believe that our framework offers a deeper understanding of the complexity involved in computing SHAP across various variants, models, distributions, as well as in both local and global computations, laying the groundwork for future research.
\bibliography{iclr2025_conference}
\bibliographystyle{iclr2025_conference}

\clearpage
\appendix

% \section{Related Works} \label{appe:rw}
% \section{Related Work}\label{sec:relatedwork}

Internet of Things (IoT) has seen rapid advancements in recent years, becoming an integral part of various domains, such as smart industries and homes, and serving as a key enabler in modern society.
However, despite its growth, IoT continues to face numerous security challenges, prompting significant research efforts aimed at improving IoT security.
With the rise of artificial intelligence (AI), machine learning (ML) and deep learning (DL)-based approaches have become increasingly popular in designing defense mechanisms for IoT devices, including malicious traffic classification~\cite{luo2022transformer,shafiq2020corrauc}, malware detection~\cite{vasan2020mthael,chaganti2022deep,aung2022atlas}, vulnerability discovery~\cite{neshenko2019demystifying}, and others~\cite{al2020survey,otoum2022dl,tambe2019detection}.

More recently, inspired by the success of large language models (LLMs), researchers have begun exploring the potential of LLMs to enhance IoT-related security tasks.
For instance, LLMs have been applied to existing IoT security challenges such as threat detection and fuzzing. Ferrag \etal~\cite{sokiotllm} introduced a BERT-based model, SecurityBERT, to achieve better cyber threat detection accuracy over traditional ML and DL-based methods. 
Similarly, Ma \etal~\cite{ma} and Wang \etal~\cite{llmiotfuz} proposed LLM-assisted fuzzing methods to uncover hidden bugs in IoT devices, enabling the detection of complex vulnerabilities that traditional techniques might miss.
Additionally, Yang \etal~\cite{yang2023iot} combined LLMs with static code analysis using prompt engineering to create a cost-effective solution for IoT vulnerability detection.
\cite{ji2024sevenllm} collected cybersecurity raw texts to train cybersecurity LLM to augment the analysis of cybersecurity events, and \cite{llmtikg} made use of a larger LLM to build knowledge graphs from public threat intelligence and use GPT to create datasets to fine-tune a smaller LLM to extract entities and TTPs from attack description.
Ferraris \etal~\cite{ferraris2024ici} proposed utilizing ChatGPT to enhance IoT trust semantics, aligning with W3C Web of Things (WoT) recommendations\footnote{\scriptsize \url{https://www.w3.org/WoT/}}.
This work extends the TrUStAPIS framework~\cite{ferraris2020trustapis}.

Beyond the above tasks, LLMs have been employed in other IoT challenges.
Meyuhas \etal~\cite{meyuhas2024iotlabel} used LLMs to address the problem of labeling previously unseen IoT devices.
\cite{llmiotcontrol,cui2024llmind} explored leveraging LLMs to control IoT devices and facilitate effective collaboration among them.
Mo \etal~\cite{mo2024iot} collected IoT sensor-natural language paired data and trained IoT-LM to interpret and interact with physical IoT sensors.
Xu \etal~\cite{xu2024penetrative} employed ChatGPT to interpret IoT sensor data and reason over tasks in the physical realm, introducing novel ways of integrating human knowledge into cyber-physical systems. 

Recently, Deldari \etal~\cite{deldari2024auditnet} proposed AuditNet, a conversational AI-based security assistant, which is most similar to \chatiot\ and also augmented by external knowledge.
However, AuditNet focused on standards, policies, and regulations of portable document format (PDF), and aimed to reduce the manual effort of security experts involved in compliance checks of IoT. 
On the other hand, we integrate IoT threat intelligence of various sources into \chatiot\ and can assist multiple kinds of users. Besides, we provide an end-to-end toolkit to process data in various formats, not limited to PDF. 

Together, these studies indicate that LLMs have great potential to improve the security of IoT systems in various domains, from vulnerability discovery to trustworthiness management. 
By integrating LLMs with IoT-specific threat intelligence, these models can be guided to meet the unique challenges posed by the IoT ecosystem.
Moreover, the continuous advancements in the LLM community, combined with increasingly accessible IoT datasets, are likely to further drive the adoption of LLMs in IoT-related research and practical applications.

% You may include other additional sections here.




\section{MPHIL Objective Deductions} \label{appe:proof}
\subsection{Proof of the overall objective}\label{appe:deduct}
In this section, we explain how we derived our goal in Eq.~(\ref{eq: target2}) from Eq.~(\ref{eq:soft}). Let's recall that Eq.~(\ref{eq:soft}) is formulated as:
\begin{equation}
   \underset{f_{c}, g}{\min} -I(y; \mathbf{z}_{inv})+\beta I(\mathbf{z}_{inv};e),
\end{equation}
For the first term $-I(y;\mathbf{z}_{inv})$, since we are mapping invariant features to hyperspherical space, we replace $\mathbf{z}_{inv}$ with $\mathbf{\hat{z}}_{inv}$. Then according to the definition of mutual information:
\begin{equation}\label{eq:muinfor}
-I(y;\mathbf{\hat{z}}_{inv}) = - \mathbb{E}_{y, \mathbf{\hat{z}}_{inv}} \left[ \log \frac{p(y,\mathbf{\hat{z}}_{inv})}{p(y) p(\mathbf{\hat{z}}_{inv})} \right].
\end{equation}

We introduce intermediate variables $\boldsymbol{\mu}^{y}$ to rewrite Eq. (\ref{eq:muinfor}) as:
\begin{equation}
    \resizebox{.9\hsize}{!}{
$-I(y; \mathbf{\hat{z}}_{inv}) = -E_{y, \mathbf{\hat{z}}_{inv}, \boldsymbol{\mu}^{y}} \left[ \log \frac{p(y, \mathbf{\hat{z}}_{inv}, \boldsymbol{\mu}^{y})}{p(\mathbf{\hat{z}}_{inv}, \boldsymbol{\mu}^{y}) p(y)} \right] + E_{y, \mathbf{\hat{z}}_{inv},\boldsymbol{\mu}^{y}} \left[ \log \frac{p(y,\boldsymbol{\mu}^{y}|\mathbf{\hat{z}}_{inv})}{p(y|\mathbf{\hat{z}}_{inv}) p(\boldsymbol{\mu}^{y})|\mathbf{\hat{z}}_{inv}} \right]. $}
\end{equation}




By the definition of Conditional mutual information, we have the following equation:
\begin{equation}
    \begin{aligned} 
    -I(y;\mathbf{\hat{z}}_{inv})  &= -I(y;\mathbf{\hat{z}}_{inv},\boldsymbol{\mu}^{y})+I(y;\boldsymbol{\mu}^{y}|\mathbf{\hat{z}}_{inv}),  \\ 
    -I(y;\boldsymbol{\mu}^{y})  &= -I(y;\mathbf{\hat{z}}_{inv},\boldsymbol{\mu}^{y})+I(y;\mathbf{\hat{z}}_{inv}|\boldsymbol{\mu}^{y}). 
\end{aligned}
\end{equation}
% \section{Related Work}\label{sec:relatedwork}

Internet of Things (IoT) has seen rapid advancements in recent years, becoming an integral part of various domains, such as smart industries and homes, and serving as a key enabler in modern society.
However, despite its growth, IoT continues to face numerous security challenges, prompting significant research efforts aimed at improving IoT security.
With the rise of artificial intelligence (AI), machine learning (ML) and deep learning (DL)-based approaches have become increasingly popular in designing defense mechanisms for IoT devices, including malicious traffic classification~\cite{luo2022transformer,shafiq2020corrauc}, malware detection~\cite{vasan2020mthael,chaganti2022deep,aung2022atlas}, vulnerability discovery~\cite{neshenko2019demystifying}, and others~\cite{al2020survey,otoum2022dl,tambe2019detection}.

More recently, inspired by the success of large language models (LLMs), researchers have begun exploring the potential of LLMs to enhance IoT-related security tasks.
For instance, LLMs have been applied to existing IoT security challenges such as threat detection and fuzzing. Ferrag \etal~\cite{sokiotllm} introduced a BERT-based model, SecurityBERT, to achieve better cyber threat detection accuracy over traditional ML and DL-based methods. 
Similarly, Ma \etal~\cite{ma} and Wang \etal~\cite{llmiotfuz} proposed LLM-assisted fuzzing methods to uncover hidden bugs in IoT devices, enabling the detection of complex vulnerabilities that traditional techniques might miss.
Additionally, Yang \etal~\cite{yang2023iot} combined LLMs with static code analysis using prompt engineering to create a cost-effective solution for IoT vulnerability detection.
\cite{ji2024sevenllm} collected cybersecurity raw texts to train cybersecurity LLM to augment the analysis of cybersecurity events, and \cite{llmtikg} made use of a larger LLM to build knowledge graphs from public threat intelligence and use GPT to create datasets to fine-tune a smaller LLM to extract entities and TTPs from attack description.
Ferraris \etal~\cite{ferraris2024ici} proposed utilizing ChatGPT to enhance IoT trust semantics, aligning with W3C Web of Things (WoT) recommendations\footnote{\scriptsize \url{https://www.w3.org/WoT/}}.
This work extends the TrUStAPIS framework~\cite{ferraris2020trustapis}.

Beyond the above tasks, LLMs have been employed in other IoT challenges.
Meyuhas \etal~\cite{meyuhas2024iotlabel} used LLMs to address the problem of labeling previously unseen IoT devices.
\cite{llmiotcontrol,cui2024llmind} explored leveraging LLMs to control IoT devices and facilitate effective collaboration among them.
Mo \etal~\cite{mo2024iot} collected IoT sensor-natural language paired data and trained IoT-LM to interpret and interact with physical IoT sensors.
Xu \etal~\cite{xu2024penetrative} employed ChatGPT to interpret IoT sensor data and reason over tasks in the physical realm, introducing novel ways of integrating human knowledge into cyber-physical systems. 

Recently, Deldari \etal~\cite{deldari2024auditnet} proposed AuditNet, a conversational AI-based security assistant, which is most similar to \chatiot\ and also augmented by external knowledge.
However, AuditNet focused on standards, policies, and regulations of portable document format (PDF), and aimed to reduce the manual effort of security experts involved in compliance checks of IoT. 
On the other hand, we integrate IoT threat intelligence of various sources into \chatiot\ and can assist multiple kinds of users. Besides, we provide an end-to-end toolkit to process data in various formats, not limited to PDF. 

Together, these studies indicate that LLMs have great potential to improve the security of IoT systems in various domains, from vulnerability discovery to trustworthiness management. 
By integrating LLMs with IoT-specific threat intelligence, these models can be guided to meet the unique challenges posed by the IoT ecosystem.
Moreover, the continuous advancements in the LLM community, combined with increasingly accessible IoT datasets, are likely to further drive the adoption of LLMs in IoT-related research and practical applications.


By merging the same terms, we have:
\begin{equation}
     -I(y;\mathbf{\hat{z}}_{inv})  = -I(y;\boldsymbol{\mu}^{y})+[I(y;\boldsymbol{\mu}^{y}|\mathbf{\hat{z}}_{inv})-I(y;\mathbf{\hat{z}}_{inv}|\boldsymbol{\mu}^{y})]. 
\end{equation}

Since our classification is based on the distance between $\boldsymbol{\mu}^{y}$ and $\mathbf{\hat{z}}_{inv}$, we add $-I(y;\mathbf{\hat{z}}_{inv},\boldsymbol{\mu}^{y})$ back into the above equation and obtain a lower bound:
\begin{equation}
\resizebox{.9\hsize}{!}{
$-I(y;\mathbf{\hat{z}}_{inv}) \geq -I(y;\boldsymbol{\mu}^{y})+[I(y;\boldsymbol{\mu}^{y}|\mathbf{\hat{z}}_{inv})-I(y;\mathbf{\hat{z}}_{inv}|\boldsymbol{\mu}^{y})]-I(y;\mathbf{\hat{z}}_{inv},\boldsymbol{\mu}^{y}).$}
\end{equation}
Since the $\boldsymbol{\mu}^{y}$ are updated by $\mathbf{\hat{z}}_{inv}$ from the same class, we can approximate $I(y;\boldsymbol{\mu}^{y}|\mathbf{\hat{z}}_{inv})$ equal to $I(y;\mathbf{\hat{z}}_{inv}|\boldsymbol{\mu}^{y})$ and obtain the new lower bound:
\begin{equation}
\label{eq: first part}
    -I(y;\mathbf{\hat{z}}_{inv}) \geq -I(y;\boldsymbol{\mu}^{y})-I(y;\mathbf{\hat{z}}_{inv},\boldsymbol{\mu}^{y}).
\end{equation}

For the second term $I(\mathbf{z}_{inv};e)$, we can also rewrite it as:
\begin{equation}
  \begin{aligned}
    I(\mathbf{\hat{z}}_{inv};e)  &= I(\mathbf{\hat{z}}_{inv};e,\boldsymbol{\mu}^{y})-I(\mathbf{\hat{z}}_{inv};\boldsymbol{\mu}^{y}|e).
    \end{aligned}
\end{equation}

Given that the environmental labels  $e$ are unknown, we drop the term $I(\mathbf{\hat{z}}_{inv};e,\boldsymbol{\mu}^{y})$ as it cannot be directly computed. This leads to the following lower bound:
\begin{equation}\label{z_e}
  \begin{aligned}
    I(\mathbf{\hat{z}}_{inv};e)  &\geq-I(\mathbf{\hat{z}}_{inv};\boldsymbol{\mu}^{y}|e).
    \end{aligned}
\end{equation}

We can obtain a achievable target by Eq. (\ref{eq: first part}) and Eq. (\ref{z_e}) as follow:
\begin{equation}\label{uneq_1}
\resizebox{.9\hsize}{!}{$
    -I(y; \mathbf{z}_{inv})+\beta I(\mathbf{z}_{inv};e)\geq-I(y;\boldsymbol{\mu}^{y})-I(y;\mathbf{\hat{z}}_{inv},\boldsymbol{\mu}^{y})-\beta I(\mathbf{\hat{z}}_{inv};\boldsymbol{\mu}^{y}|e).$}
\end{equation}

In fact, $p(\mathbf{\hat{z}}_{inv},\boldsymbol{\mu}^{y}|e) \geq p(\mathbf{\hat{z}}_{inv};\boldsymbol{\mu}^{y})$, Eq. (\ref{uneq_1}) can be achieved by:
\begin{equation}\resizebox{.9\hsize}{!}{$
    -I(y; \mathbf{z}_{inv})+\beta I(\mathbf{z}_{inv};e)\geq-I(y;\boldsymbol{\mu}^{y})-I(y;\mathbf{\hat{z}}_{inv},\boldsymbol{\mu}^{y})-\beta I(\mathbf{\hat{z}}_{inv};\boldsymbol{\mu}^{y}). $}
\end{equation}

Finally, optimizing Eq. (\ref{eq:soft}) can be equivalent to optimizing its lower bound and we can obtain the objective without environment $e$ as shown in Eq. (\ref{eq: target2}):
\begin{equation}
\setlength\abovedisplayskip{9pt}
\setlength\belowdisplayskip{9pt}
   \underset{f_{c}, g}{\min} \underbrace{-I(y;\hat{\mathbf{z}}_{inv},\boldsymbol{\mu}^{(y)}) }_{\mathcal{L}_{\mathrm{C}}}\underbrace{-I(y ; \boldsymbol{\mu}^{(y)}) }_{\mathcal{L}_{\mathrm{PS}}}\underbrace{-\beta I(\hat{\mathbf{z}}_{inv};\boldsymbol{\mu}^{(y)})}_{\mathcal{L}_{\mathrm{IPM}}}.
\end{equation}

\subsection{Proof of $\mathcal{L}_{\mathrm{C}}$ } \label{appe:deduct2}
For the term $I(y;\hat{\mathbf{z}}_{inv},\boldsymbol{\mu}^{(y)})$, it can be written as:
\begin{equation}
    I(y;\hat{\mathbf{z}}_{inv},\boldsymbol{\mu}^{(y)}) = E_{y, \mathbf{\hat{z}}_{inv}, \boldsymbol{\mu}^{y}} \left[ \log \frac{p(y| \mathbf{\hat{z}}_{inv}, \boldsymbol{\mu}^{y})}{p(y)} \right],
\end{equation}
according to ~\citep{graphpro}, we have:
\begin{equation}
    I(y;\hat{\mathbf{z}}_{inv},\boldsymbol{\mu}^{(y)}) \geq E_{y, \mathbf{\hat{z}}_{inv}, \boldsymbol{\mu}^{y}} \left[ \log \frac{q_{\theta}(y| \gamma(\mathbf{\hat{z}}_{inv}, \boldsymbol{\mu}^{y}))}{p(y)} \right],
\end{equation}
where $\gamma(,)$ is the function to calculate the similarity between $\mathbf{\hat{z}}_{inv}$ and $\boldsymbol{\mu}^{y}$. $q_{\theta}(y| \gamma(\mathbf{\hat{z}}_{inv}, \boldsymbol{\mu}^{y}))$ is the variational approximation of the $p(y| \gamma(\mathbf{\hat{z}}_{inv}, \boldsymbol{\mu}^{y}))$. Then we can have:
\begin{align}\nonumber
    I(y;\hat{\mathbf{z}}_{inv},\boldsymbol{\mu}^{(y)}) &\geq E_{y, \mathbf{\hat{z}}_{inv}, \boldsymbol{\mu}^{y}} \left[ \log \frac{q_{\theta}(y| \gamma(\mathbf{\hat{z}}_{inv}, \boldsymbol{\mu}^{y}))}{p(y)} \right] \\ \nonumber
    &\geq E_{y, \mathbf{\hat{z}}_{inv}, \boldsymbol{\mu}^{y}} \left[ \log {q_{\theta}(y| \gamma(\mathbf{\hat{z}}_{inv}, \boldsymbol{\mu}^{y}))} \right]- E_{y}[\log p(y)]\\ \nonumber
    &\geq E_{y, \mathbf{\hat{z}}_{inv}, \boldsymbol{\mu}^{y}} \left[ \log {q_{\theta}(y| \gamma(\mathbf{\hat{z}}_{inv}, \boldsymbol{\mu}^{y}))} \right]\\
    &:= -\mathcal{L}_{\mathrm{C}}.
\end{align}

Finally, we prove that $\min I(y;\hat{\mathbf{z}}_{inv},\boldsymbol{\mu}^{(y)})$ is equivalent to minimizing the classification loss $\mathcal{L}_{\mathrm{C}}$.

\section{Methodology Details}\label{appe:method_detail}

\subsection{Overall Algorithm of \ourmethod} 
The training algorithm of \ourmethod is shown in Algorithm.~\ref{code:train}. After that, we use the well-trained $\mathrm{GNN}_{S}$,$\mathrm{GNN}_{E}$, $\mathrm{Proj}$ and all prototypes $\mathbf{M}^{(c)} = \{ \boldsymbol{\mu}_{k}^{(c)}\}^{K}_{k=1}$ to perform inference on the test set. The pseudo-code for this process is shown in Algorithm.~\ref{code:test}.
\begin{algorithm}[h!]
\renewcommand{\algorithmicrequire}{\textbf{Input:}}
	\renewcommand{\algorithmicensure}{\textbf{Output:}}
  \caption{The training algorithm of \ourmethod.}
  \begin{algorithmic}[1]
    \REQUIRE
      Scoring GNN $\mathrm{GNN}_{S}$;
      Encoding GNN $\mathrm{GNN}_{E}$;
      Projection $\mathrm{Proj}$;
      Number of prototypes for each class $K$;
      The data loader of in-distribution training set $D_{\mathrm{train}}$.
      
    \ENSURE
    Well-trained $\mathrm{GNN}_{S}$, $\mathrm{GNN}_{E}$, $\mathrm{Proj}$
       and all prototypes $\mathbf{M}^{(c)}$.
    \STATE 
For each class $c \in \{1, \cdots, C\}$, assign $K$ prototypes for it which can be denoted by $\mathbf{M}^{(c)} = \{ \boldsymbol{\mu}_{k}^{(c)}\}^{K}_{k=1}$.    
\STATE Initialize each of them by $\boldsymbol{\mu}_{k}^{(c)} \sim \mathcal{N}(\textbf{0}, \textbf{I})$
     \FOR{epoch in epochs}  
    \FOR{each $G_\mathrm{{batch}}$ in $D_{\mathrm{train}}$}
       \STATE Obtain $Z_{inv}$  using $\mathrm{GNN}_{S}$ and $\mathrm{GNN}_{E}$ via Eq. (\ref{eq:score}) and (\ref{eq:readout})
       \STATE Obtain $\hat{Z}_{inv}$ using $\mathrm{Proj}$ via Eq. (\ref{eq: hype_rinv})
       \STATE Compute $W^{(c)}$ using $\boldsymbol{u}^{(c)}$ and $\hat{Z}_{inv}$ via Eq. (\ref{eq: att}).
        \FOR{each prototype $\boldsymbol{u}_{k}^{(c)}$}
        \STATE Update it using $\hat{Z}_{inv}$ and $W^{(c)}$ via Eq. (\ref{eq: update_prototype}).
        
        \ENDFOR
        \STATE Get $p(y = c \mid \hat{\mathbf{z}}_{i}; \{ w^{c},\boldsymbol{u}^{(c)} \}_{c=1}^{(C)})$ using $\hat{Z}_{inv}$, $W^{(c)}$ and $\boldsymbol{\mu}^{(c)}$ via Eq. (\ref{classify})
        \STATE Compute the final loss $\mathcal{L}$ with $\hat{Z}_{inv}$, $\boldsymbol{\mu}^{(c)}$ and $p(y = c \mid \hat{\mathbf{z}}_{i}; \{ w^{c},\boldsymbol{u}^{(c)} \}_{c=1}^{(C)})$ via Eq. (\ref{ipm}), (\ref{ps}) and (\ref{cls})
        % \State Generate $\overline{G}$ and add to $TL$;
        \STATE Update parameters of $\mathrm{GNN}_{S}$, $\mathrm{GNN}_{E}$ and $\mathrm{Proj}$ with the gradient of $\mathcal{L}$.
      \ENDFOR
    \ENDFOR
  \end{algorithmic}\label{code:train}
\end{algorithm}
\vspace{-5mm}
\begin{algorithm}[h!]
\renewcommand{\algorithmicrequire}{\textbf{Input:}}
	\renewcommand{\algorithmicensure}{\textbf{Output:}}
  \caption{The inference algorithm of \ourmethod.}
  \begin{algorithmic}[1]
    \REQUIRE
      Well-trained $\mathrm{GNN}_{S}$,$\mathrm{GNN}_{E}$, $\mathrm{Proj}$ and all prototypes $\mathbf{M}^{(c)} = \{ \boldsymbol{\mu}_{k}^{(c)}\}^{K}_{k=1}$.
      The data loader of Out-of-distribution testing set $D_{\mathrm{test}}$.
      
    \ENSURE
    Classification probability $p(y = c \mid \hat{\mathbf{z}}_{i}; \{ w^{c},\boldsymbol{\mu}^{(c)} \}_{c=1}^{(C)})$ 
    \FOR{each $G_\mathrm{{batch}}$ in $D_{\mathrm{test}}$}
       \STATE Obtain $Z_{inv}$  using $\mathrm{GNN}_{S}$ and $\mathrm{GNN}_{E}$ via Eq. (\ref{eq:score}) and (\ref{eq:readout})
       \STATE Obtain $\hat{Z}_{inv}$ using $\mathrm{Proj}$ via Eq. (\ref{eq: hype_rinv})
       \STATE Compute $W^{(c)}$ using $\boldsymbol{\mu}^{(c)}$ and $\hat{Z}_{inv}$ via Eq. (\ref{eq: att}).
        
        \STATE Get $p(y = c \mid \hat{\mathbf{z}}_{i}; \{ w^{c},\boldsymbol{\mu}^{(c)} \}_{c=1}^{(C)})$ using $\hat{Z}_{inv}$, $W^{(c)}$ and $\boldsymbol{\mu}^{(c)}$ via Eq. (\ref{classify})
        
        % \State Generate $\overline{G}$ and add to $TL$;
        
      \ENDFOR
  \end{algorithmic}\label{code:test}

\end{algorithm}
\vspace{-6mm}

\subsection{Weight Pruning} \label{appe:pruning}

Directly assigning weights to all prototypes within a class can lead to excessive similarity between prototypes, especially for difficult samples. This could blur decision boundaries and reduce the model's ability to correctly classify hard-to-distinguish samples.

To address this, we apply a \textbf{top-$n$ pruning} strategy, which keeps only the most relevant prototypes for each sample. The max weights are retained, and the rest are pruned as follows:
\begin{equation}
    W_{i,k}^{(c)} = {1}[W_{i,k}^{(c)}>\beta]\ast W_{i,k}^{(c)}, 
\end{equation}
where $\beta$ is the threshold corresponding to the top-$n$ weight, and ${1}[W_{i,k}^{(c)}>\beta]$ is an indicator function that retains only the weights for the top-$n$ prototypes. This pruning mechanism ensures that the prototypes remain distinct and that the decision space for each class is well-defined, allowing for improved classification performance. By applying this attention-based weight calculation and top-$n$ pruning, the model ensures a more accurate and robust matching of samples to prototypes, enhancing classification, especially in OOD scenarios.


\subsection{Complexity Analysis}
The time complexity of \ourmethod is $
\mathcal{O}(|E|d+|V|d^{2})$, where $|V|$ denotes the number of nodes and $|E|$ denotes the number of edges, $d$ is the dimension of the final representation. Specifically, for $\mathrm{GNN}_{S}$ and $\mathrm{GNN}_{E}$, their complexity is denoted as $
\mathcal{O}(|E|d+|V|d^{2})$. The complexity of the projector is $
\mathcal{O}(|V|d^{2})$, while the complexities of calculating weights and updating prototypes are $
\mathcal{O}(|V||K|d)$ where $K$ is the number of prototypes. The complexity of computing the final classification probability also is $
\mathcal{O}(|V|Kd)$. Since $K$ is a very small constant, we can ignore $\mathcal{O}(|V|Kd)$, resulting in a final complexity of $
\mathcal{O}(|E|d+|V|d^{2})$. Theoretically, the time complexity of \ourmethod is on par with the existing methods.

% \begin{figure}[htb]
%  \centering
% \subfigure[CIGA] { 
% \includegraphics[width=0.27\textwidth]{4_exp/imold.pdf} \label{t1}
% }
% \subfigure[Ours] { 
% \includegraphics[width=0.27\textwidth]{4_exp/my.pdf} \label{t2}
% }
% \caption{t-SNE visualization on HIV-Scaffold.}
% \label{fig:tsne}
% \end{figure}
\section{Experimental Details}\label{appe:exp}

\subsection{Datasets}\label{appe:data}
\textbf{Overview of the Dataset.} In this work, we use 11 publicly benchmark datasets, 5 of them are from GOOD~\citep{good} benchmark. They are the combination of 3 datasets (GOOD-HIV, GOOD-Motif and GOOD-CMNIST) with different distribution shift (scaffold, size, basis, color). The rest 6 datasets are from DrugOOD~\citep{drugood} benchmark, including IC50-Assay, IC50-Scaffold, IC50-Size, EC50-Assay, EC50-Scaffold, and EC50-Size. The prefix denotes the measurement and the suffix denotes the distribution-splitting strategies. We use the default dataset split proposed in each benchmark.  Statistics of each dataset are in Table \ref{data_st}.

\begin{table}[t]
\renewcommand{\arraystretch}{1.5}
\setlength\tabcolsep{3pt}
\centering
\caption{Dateset statistics.}
\resizebox{\columnwidth}{!}{%
\begin{tabular}{cccccccc}
\toprule
\multicolumn{3}{c|}{Dataset}                                                                            & Task                       & Metric  & Train & Val   & Test  \\ \midrule
\multicolumn{1}{c|}{\multirow{5}{*}{GOOD}}    & \multirow{2}{*}{HIV}   & \multicolumn{1}{c|}{scaffold} & Binary Classification      & ROC-AUC & 24682 & 4133  & 4108  \\
\multicolumn{1}{c|}{}                         &                        & \multicolumn{1}{c|}{size}     & Binary Classification      & ROC-AUC & 26169 & 4112  & 3961  \\ \cline{2-8} 
\multicolumn{1}{c|}{}                         & \multirow{2}{*}{Motif} & \multicolumn{1}{c|}{basis}    & Multi-label Classification & ACC     & 18000 & 3000  & 3000  \\
\multicolumn{1}{c|}{}                         &                        & \multicolumn{1}{c|}{size}     & Multi-label Classification & ACC     & 18000 & 3000  & 3000  \\ \cline{2-8} 
\multicolumn{1}{c|}{}                         & CMNIST                 & \multicolumn{1}{c|}{color}    & Multi-label Classification & ACC     & 42000 & 7000  & 7000  \\ \midrule
\multicolumn{1}{c|}{\multirow{6}{*}{DrugOOD}} & \multirow{3}{*}{IC50}  & \multicolumn{1}{c|}{assay}    & Binary Classification      & ROC-AUC & 34953 & 19475 & 19463 \\
\multicolumn{1}{c|}{}                         &                        & \multicolumn{1}{c|}{scaffold} & Binary Classification      & ROC-AUC & 22025 & 19478 & 19480 \\
\multicolumn{1}{c|}{}                         &                        & \multicolumn{1}{c|}{size}     & Binary Classification      & ROC-AUC & 37497 & 17987 & 16761 \\ \cline{2-8} 
\multicolumn{1}{c|}{}                         & \multirow{3}{*}{EC50}  & \multicolumn{1}{c|}{assay}    & Binary Classification      & ROC-AUC & 4978  & 2761  & 2725  \\
\multicolumn{1}{c|}{}                         &                        & \multicolumn{1}{c|}{scaffold} & Binary Classification      & ROC-AUC & 2743  & 2723  & 2762  \\
\multicolumn{1}{c|}{}                         &                        & \multicolumn{1}{c|}{size}     & Binary Classification      & ROC-AUC & 5189  & 2495  & 2505  \\ \bottomrule
\end{tabular}%
}\label{data_st}
\vspace{-4mm}
\end{table}

\textbf{Distribution split.} In this work, we investigate various types of distribution-splitting strategies for different datasets.
\begin{itemize}
    \item \textbf{Scaffold.}  Molecular scaffold is the core structure of a molecule that supports its overall composition, but it only exhibits specific properties when combined with particular functional groups. 
    \item \textbf{Size.} The size of a graph refers to the total number of nodes, and it is also implicitly related to the graph's structural properties.     
    \item \textbf{Assay.} The assay is an experimental technique used to examine or determine molecular characteristics. Due to differences in assay conditions and targets, activity values measured by different assays can vary significantly. 
    \item \textbf{Basis.} The generation of a motif involves combining a base graph (wheel, tree, ladder, star, and path) with a motif (house, cycle, and crane), but only the motif is directly associated with the label. 
    \item \textbf{Color.} CMNIST is a graph dataset constructed from handwritten digit images. Following previous research, we declare a distribution shift when the color of the handwritten digits changes.
\end{itemize}
\subsection{Baselines}\label{appe:baseline}
In our experiments, the methods we compared can be divided into two categories, one is ERM and traditional OOD generalization methods:
\begin{itemize}
    \item \textbf{ERM} is a standard learning approach that minimizes the average training error, assuming the training and test data come from the same distribution.
    \item \textbf{IRM}~\citep{arjovsky2019invariant} aims to learn representations that remain invariant across different environments, by minimizing the maximum error over all environments.
    \item \textbf{VREx}~\citep{krueger2021out} propose a penalty on the variance of training risks which can providing more robustness to changes in the input distribution.  
    \item \textbf{Coral}~\citep{coral} utilize a nonlinear transformation to align the second-order statistical features of the source and target domain distributions
\end{itemize}
Another class of methods is specifically designed for Graph OOD generalization:
\begin{itemize}
    \item \textbf{MoleOOD}~\citep{yang2022learning} learn the environment invariant molecular substructure by a environment inference model and a molecular decomposing model.
    \item \textbf{CIGA}~\citep{chen2022learning} proposes an optimization objective based on mutual information to ensure the learning of invariant subgraphs that are not affected by the environment.
    \item \textbf{GIL}~\citep{li2022learning} performs environment identification and invariant risk loss optimization by separating the invariant subgraph and the environment subgraph.
    \item \textbf{GERA}~\citep{liu2022graph} performs data augmentation by replacing the input graph with the environment subgraph to improve the generalization ability of the model
    \item \textbf{IGM}~\citep{jia2024graph} performs data augmentation by simultaneously performing a hybrid strategy of invariant subgraphs and environment subgraphs.
    \item \textbf{DIR}~\citep{wu2022discovering} identifies causal relation between input graphs and labels by performing counterfactual interventions.
    \item \textbf{DisC}~\citep{fan2022debiasing} learns causal and bias representations through a causal and disentangling based learning strategy separately.
    \item \textbf{GSAT}~\citep{miao2022interpretable} learns the interpretable label-relevant subgraph through an stochasticity attention mechanism.
    \item \textbf{CAL}~\cite{sui2022causal} proposes a causal attention learning strategy to ensure that GNNs learn effective representations instead of optimizing loss through shortcuts.
    \item  \textbf{iMoLD}~\citep{zhuang2023learning} designs two GNNs to directly extract causal features from the encoded graph representation.
    \item { \textbf{GALA}~\citep{gala} designs designs a new loss function to ensure graph OOD generalization without environmental information as much as possible.}
    \item  {\textbf{EQuAD}~\citep{Equad} learns how to effectively remove spurious features by optimizing the self-supervised informax function.}
\end{itemize}
\subsection{Implementation Details}\label{appe:hyperparam}
\textbf{Baselines.} For all traditional OOD methods, we conduct experiments on different datasets using the code provided by GOOD~\citep{good} and DrugOOD~\citep{drugood} benchmark. For graph OOD generalization methods with public code, we perform experiments in the same environments as our method and employ grid search to select hyper-parameters, ensuring fairness in the results.

\begin{table}[t] 
\renewcommand{\arraystretch}{0.9}
\setlength\tabcolsep{2pt}
\centering
\caption{Hyper-parameter configuration.}
% \vspace{-4mm}
% \resizebox{\columnwidth}{!}{%
\begin{tabular}{cccccccc}
\toprule
                                          & \multicolumn{1}{l}{}       &          & $\mathrm{proj\_dim}$ & $\mathrm{att\_dim}$ & $K$ & $lr$ & $\beta$ \\ \midrule
\multirow{6}{*}{DrugOOD}                  & \multirow{3}{*}{IC50}      & Assay    & 300      & 128     & 3   & 0.001     & 0.1     \\
                                          &                            & Scaffold & 300      & 128     & 3   & 0.001      & 0.1     \\
                                          &                            & Size     & 300      & 128     & 3   & 0.001     & 0.1     \\ \cline{2-8} 
                                          & \multirow{3}{*}{EC50}      & Assay    & 300      & 128     & 3   & 0.001      & 0.1     \\
                                          &                            & Scaffold & 300      & 128     & 3   & 0.001      & 0.1     \\
                                          &                            & Size     & 300      & 128     & 3   & 0.001     & 0.1     \\ \midrule
\multicolumn{1}{c}{\multirow{5}{*}{GOOD}} & \multirow{2}{*}{HIV}       & Scaffold & 300      & 128     & 3   & 0.01     & 0.1     \\
\multicolumn{1}{c}{}                      &                            & Size     & 300      & 128     & 3   & 0.01     & 0.1     \\ \cline{2-8} 
\multicolumn{1}{c}{}                      & \multirow{2}{*}{Motif}     & Basis    & 256     & 128     & 6   & 0.01    & 0.2     \\
\multicolumn{1}{c}{}                      &                            & Size     & 256      & 128     & 6   & 0.01    & 0.2     \\ \cline{2-8} 
\multicolumn{1}{c}{}                      & \multicolumn{1}{l}{CMNIST} & Color    & 256      & 128     & 5   & 0.01    & 0.2     \\ \bottomrule
\end{tabular}\label{hyperpater}
% }
% \vspace{-7mm}
\end{table}
\textbf{Our method.} We implement our proposed \ourmethod under the Pytorch~\citep{pytorch} and PyG~\citep{pyg}. For all datasets containing molecular graphs (all datasets from DrugOOD and GOODHIV), we fix the learning rate to $0.001$ and select the hyper-parameters by ranging the $\mathrm{proj\_dim}$ from $\{100,200,300\}$, $\mathrm{att\_dim}$ from $\{64,128,256\}$, $K$ from $\{2,3,4,5\}$ and $\beta$ from $\{0.01,0.1,0.2\}$. For the other datasets, we fix the learning rate to $0.01$ and select the hyper-parameters by ranging the $\mathrm{proj\_dim}$ from $\{64,128,256\}$, $\mathrm{att\_dim}$ from $\{64,128,256\}$, $K$ from $\{3,4,5,6\}$ and $\beta$ from $\{0.01,0.1,0.2\}$. For the top-$n$ pruning, we force $n$ to be half of $K$. We conduct a grid search to select hyper-parameters and refer to Table \ref{hyperpater} for the detailed configuration. For all experiments, we fix the number of epochs to 200 and run the experiment five times with different seeds, select the model to run on the test set based on its performance on validation, and report the mean and standard deviation. 
% Please add the following required packages to your document preamble:
% \usepackage{multirow}
% \usepackage{graphicx}
% Please add the following required packages to your document preamble:
% \usepackage{multirow}
% \usepackage{graphicx}

% Please add the following required packages to your document preamble:
% \usepackage{graphicx}

\subsection{Supplemental Results}\label{appe:more_results}
We report the complete experimental results with means and standard deviations in Tables \ref{tab:main_good} and \ref{tab:main_drugood}
\begin{table}[h!] 
\caption{Performance comparison in terms of average accuracy (standard deviation) on GOOD benchmark.}
\centering
\begin{tabular}{l|cc|c|cc}
\toprule
\multirow{2}{*}{\textbf{Method}}           & \multicolumn{2}{c|}{\textbf{GOOD-Motif}} & \multicolumn{1}{c|}{\textbf{GOOD-CMNIST}} & \multicolumn{2}{c}{\textbf{GOOD-HIV}} \\
           & \textbf{basis}         & \textbf{size}          & \textbf{color}         & \textbf{scaffold}      & \textbf{size} \\
\midrule
ERM        & 60.93 (2.11)  & 46.63 (7.12)   & 26.64 (2.37)   & 69.55 (2.39)   & 59.19 (2.29)\\
IRM        & 64.94 (4.85)   & {54.52 (3.27)}   & 29.63 (2.06)   & 70.17 (2.78)   & 59.94 (1.59)\\
VREX       & 61.59 (6.58)   & \underline{55.85 (9.42)}   & 27.13 (2.90)   & 69.34 (3.54)   & 58.49 (2.28)\\
Coral      & 61.95 (4.36)  & 55.80 (4.05)   & 29.21 (6.87)   & 70.69 (2.25)   & 59.39 (2.90)\\
\midrule
MoleOOD      & -  & -   & -  & 69.39 (3.43)   & 58.63 (1.78)\\
CIGA       & 67.81 (2.42)   & 51.87 (5.15)   & 25.06 (3.07)   & 69.40 (1.97)   & {61.81 (1.68)}\\
GIL      & 65.30 (3.02)   & 54.65 (2.09)   & 31.82 (4.24)   & 68.59 (2.11)   & 60.97 (2.88)\\
GREA      & 59.91 (2.74)   & 47.36 (3.82)   & 22.12 (5.07)   & {71.98 (2.87)}   & 60.11 (1.07)\\
IGM    & {74.69 (8.51) }  & 52.01 (5.87)   & 33.95 (4.16)   & {71.36 (2.87)}   & 62.54 (2.88)\\
\midrule
DIR        & 64.39 (2.02)   & 43.11 (2.78)   & 22.53 (2.56)   & 68.44 (2.51)   & 57.67 (3.75)\\
DisC       & 65.08 (5.06)  & 42.23 (4.20)   & 23.53 (0.67)   & 58.85 (7.26)  & 49.33 (3.84)         \\ 
\midrule
GSAT       & 62.27 (8.79)   & 50.03 (5.71)   & {35.02 (2.78)}   & 70.07 (1.76)   & 60.73 (2.39)\\
CAL       & {68.01 (3.27)}   & 47.23 (3.01)   & 27.15 (5.66)   & 69.12 (1.10)   & 59.34 (2.14)\\
GALA       & {66.91 (2.77)}   & 45.39 (5.84)   & {38.95 (2.97)}   & 69.12 (1.10)   & 59.34 (2.14)\\
iMoLD      & - & - & -  & \underline{72.05 (2.16)}   & {62.86 (2.34)}\\
\textcolor{blue}{GALA} & \textcolor{blue}{72.97 (4.28)} & \textcolor{blue}{\textbf{60.82 (0.51)}} & \textcolor{blue}{\underline{40.62 (2.11)}} & \textcolor{blue}{71.22 (1.93)}& \textcolor{blue}{\underline{65.29 (0.72)}}\\
\textcolor{blue}{EQuAD}      &\textcolor{blue}{\underline{75.46 (4.35)}}   & \textcolor{blue}{55.10 (2.91)}   & \textcolor{blue}{{40.29 (3.95)}}   & \textcolor{blue}{71.49 (0.67)}    &\textcolor{blue}{{64.09 (1.08)}}     \\
\midrule
\ourmethod     & \textbf{76.23 (4.89)} & {58.43 (3.15)} & \textbf{41.29 (3.85)} & \textbf{73.94 (1.77)} & \textbf{66.84 (1.09)}\\ \bottomrule
\end{tabular}\label{tab:main_good}
\end{table}
\begin{table*}[h!] 
\caption{Performance comparison in terms of average accuracy (standard deviation) on DrugOOD benchmark.}
\centering
\resizebox{1\textwidth}{!}{
\begin{tabular}{l|ccc|ccc}
\toprule
\multirow{2}{*}{\textbf{Method}} & \multicolumn{3}{c|}{\textbf{DrugOOD-IC50}} & \multicolumn{3}{c}{\textbf{DrugOOD-EC50}} \\ 
 & \textbf{assay} & \textbf{scaffold} & \textbf{size} & \textbf{assay} & \textbf{scaffold} & \textbf{size} \\ \midrule
ERM & 70.61 (0.75) & 67.54 (0.42) & 66.10 (0.31) & 65.27 (2.39) & 65.02 (1.10) & 65.17 (0.32) \\ 
IRM & 71.15 (0.57) & 67.22 (0.62) & \underline{67.58 (0.58)} & 67.77 (2.71) & 63.86 (1.36) & 59.19 (0.83) \\ 
VREx & 70.98 (0.77) & 68.02 (0.43) & 65.67 (0.19) & 69.84 (1.88) & 62.31 (0.96) & {65.89 (0.83)} \\ 
Coral & 71.28 (0.91) & 68.36 (0.61) & 67.53 (0.32) & 72.08 (2.80) & 64.83 (1.64) & 58.47 (0.43) \\ \midrule
MoleOOD & 71.62 (0.50) & 68.58 (1.14) & 67.22 (0.96) & 72.69 (4.16) & 65.78 (1.47) & 64.11 (1.04) \\ 
CIGA & \underline{71.86 (1.37)} & \textbf{69.14 (0.70)} & 66.99 (1.40) & 69.15 (5.79) & \underline{67.32 (1.35)} & 65.60 (0.82) \\ 
GIL & 70.66 (1.75) & 67.81 (1.03) & 66.23 (1.98) & 70.25 (5.79) & 63.95 (1.17) & 64.91 (0.76) \\ 
GREA & 70.23 (1.17) & 67.20 (0.77) & 66.09 (0.56) & 74.17 (1.47) & 65.84 (1.35) & 61.11 (0.46) \\ 
IGM & 68.05 (1.84) & 63.16 (3.29) & 63.89 (2.97) & 76.28 (4.43) & 67.57 (0.62) & 60.98 (1.05) \\ \midrule
DIR & 69.84 (1.41) & 66.33 (0.65) & 62.92 (1.89) & 65.81 (2.93) & 63.76 (3.22) & 61.56 (4.23) \\ 
DisC & 61.40 (2.56) & 62.70 (2.11) & 64.43 (0.60) & 63.71 (5.56) & 60.57 (2.27) & 57.38 (2.48) \\ \midrule
GSAT & 70.59 (0.43) & 66.94 (1.43) & 64.53 (0.51) & 73.82 (2.62) & 62.65 (1.79) & 62.65 (1.79) \\ 
CAL & 70.09 (1.03) & 65.90 (1.04) & 64.42 (0.50) & {74.54 (1.48)} & 65.19 (0.87) & 61.21 (1.76) 
\\
iMoLD & 71.77 (0.54) & 67.94 (0.59) & 66.29 (0.74) & {77.23 (1.72)} & 66.95 (1.26) & \underline{67.18 (0.86)} 
\\ 

\textcolor{blue}{GALA} & \textcolor{blue}{70.58 (2.63)} & \textcolor{blue}{66.35 (0.86)} & \textcolor{blue}{66.54 (0.93)}  & \textcolor{blue}{{77.24 (2.17)}} & \textcolor{blue}{66.98 (0.84)} & \textcolor{blue}{63.71 (1.17)}  \\
\textcolor{blue}{EQuAD} &\textcolor{blue}{71.57 (0.95)} & \textcolor{blue}{67.74 (0.57)} & \textcolor{blue}{67.54 (0.27)} &\textcolor{blue}{\underline{77.64 (0.63)}} &\textcolor{blue}{65.73 (0.17)} &\textcolor{blue}{64.39 (0.67)}  \\
\midrule
\ourmethod & \textbf{72.96 (1.21)} & \underline{68.62 (0.78)} & \textbf{68.06 (0.55)} & \textbf{78.08 (0.54)} & \textbf{68.34 (0.61)} & \textbf{68.11 (0.58)} \\ \bottomrule
\end{tabular}
}\label{tab:main_drugood}
\end{table*}.
\end{document}
\endinput
%%