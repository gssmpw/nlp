%%
%% This is file `sample-sigconf-authordraft.tex',
%% generated with the docstrip utility.
%%
%% The original source files were:
%%
%% samples.dtx  (with options: `all,proceedings,bibtex,authordraft')
%% 
%% IMPORTANT NOTICE:
%% 
%% For the copyright see the source file.
%% 
%% Any modified versions of this file must be renamed
%% with new filenames distinct from sample-sigconf-authordraft.tex.
%% 
%% For distribution of the original source see the terms
%% for copying and modification in the file samples.dtx.
%% 
%% This generated file may be distributed as long as the
%% original source files, as listed above, are part of the
%% same distribution. (The sources need not necessarily be
%% in the same archive or directory.)
%%
%%
%% Commands for TeXCount
%TC:macro \cite [option:text,text]
%TC:macro \citep [option:text,text]
%TC:macro \citet [option:text,text]
%TC:envir table 0 1
%TC:envir table* 0 1
%TC:envir tabular [ignore] word
%TC:envir displaymath 0 word
%TC:envir math 0 word
%TC:envir comment 0 0
%%
%% The first command in your LaTeX source must be the \documentclass
%% command.
%%
%% For submission and review of your manuscript please change the
%% command to \documentclass[manuscript, screen, review]{acmart}.
%%
%% When submitting camera ready or to TAPS, please change the command
%% to \documentclass[sigconf]{acmart} or whichever template is required
%% for your publication.
%%
\documentclass[sigconf]{acmart}
% \documentclass[sigconf,review]{acmart}
%% NOTE that a single column version may required for 
%% submission and peer review. This can be done by changing
%% the \doucmentclass[...]{acmart} in this template to 
%% \documentclass[manuscript,screen]{acmart}
%% 
%% To ensure 100% compatibility, please check the white list of
%% approved LaTeX packages to be used with the Master Article Template at
%% https://www.acm.org/publications/taps/whitelist-of-latex-packages 
%% before creating your document. The white list page provides 
%% information on how to submit additional LaTeX packages for 
%% review and adoption.
%% Fonts used in the template cannot be substituted; margin 
%% adjustments are not allowed.

%%
%% \BibTeX command to typeset BibTeX logo in the docs
%% \BibTeX command to typeset BibTeX logo in the docs
\AtBeginDocument{%
  \providecommand\BibTeX{{%
    Bib\TeX}}}

%% Rights management information.  This information is sent to you
%% when you complete the rights form.  These commands have SAMPLE
%% values in them; it is your responsibility as an author to replace
%% the commands and values with those provided to you when you
%% complete the rights form.
\setcopyright{acmlicensed}
\copyrightyear{2018}
\acmYear{2018}
\acmDOI{XXXXXXX.XXXXXXX}
%% These commands are for a PROCEEDINGS abstract or paper.
\acmConference[Conference acronym 'XX]{Make sure to enter the correct
  conference title from your rights confirmation emai}{June 03--05,
  2018}{Woodstock, NY}
%%
%%  Uncomment \acmBooktitle if the title of the proceedings is different
%%  from ``Proceedings of ...''!
%%
%%\acmBooktitle{Woodstock '18: ACM Symposium on Neural Gaze Detection,
%%  June 03--05, 2018, Woodstock, NY}
\acmISBN{978-1-4503-XXXX-X/18/06}


%%
%% Submission ID.
%% Use this when submitting an article to a sponsored event. You'll
%% receive a unique submission ID from the organizers
%% of the event, and this ID should be used as the parameter to this command.
%%\acmSubmissionID{123-A56-BU3}

%%
%% For managing citations, it is recommended to use bibliography
%% files in BibTeX format.
%%
%% You can then either use BibTeX with the ACM-Reference-Format style,
%% or BibLaTeX with the acmnumeric or acmauthoryear sytles, that include
%% support for advanced citation of software artefact from the
%% biblatex-software package, also separately available on CTAN.
%%
%% Look at the sample-*-biblatex.tex files for templates showcasing
%% the biblatex styles.
%%

%%
%% The majority of ACM publications use numbered citations and
%% references.  The command \citestyle{authoryear} switches to the
%% "author year" style.
%%
%% If you are preparing content for an event
%% sponsored by ACM SIGGRAPH, you must use the "author year" style of
%% citations and references.
%% Uncommenting
%% the next command will enable that style.
%%\citestyle{acmauthoryear}

\usepackage{graphicx}
\usepackage{multirow}
\usepackage{hyperref}       % hyperlinks
\usepackage{url}            % simple URL typesetting
\usepackage{booktabs}       % professional-quality tables
\usepackage{amsfonts}       % blackboard math symbols
\usepackage{nicefrac}       % compact symbols for 1/2, etc.
\usepackage{microtype}      % microtypography
\usepackage{xcolor}         % colors
% \usepackage{minipage}

\usepackage{mathrsfs}
\usepackage{amsmath}


\usepackage{bm}
\usepackage{algorithm}
\usepackage{algorithmic}
%\usepackage{algorithmicx}
\usepackage{float}  
\usepackage{lipsum}

\usepackage{xcolor}    % 用于 \colorbox
\usepackage{varwidth}  % 用于 varwidth 环境

\let\Bbbk\relax         %%redefined in newtxmath.sty
\usepackage{amssymb}
\usepackage{pifont}
 \usepackage{enumitem}
\usepackage{multirow}
\usepackage{amsthm}
\theoremstyle{definition}
\newtheorem{definition}{Definition}
\usepackage{wrapfig}

% \documentclass{article}
\usepackage{booktabs}  % 提供更好看的横线
\usepackage{array}     % 提供更多的列格式控制
% \usepackage{booktabs}  % 提供更好看的横线
\usepackage{graphicx}  % 允许图片插入
\usepackage{subfigure}
\usepackage{multirow}  % 允许跨行合并
\usepackage{multicol}  % 允许跨列合并

\usepackage{colortbl}
\usepackage{xcolor}
\usepackage{array}
\usepackage{xspace}
\newcommand{\ourmethod}{\textsc{{MPhil}}\xspace}
%%
%% end of the preamble, start of the body of the document source.
\begin{document}

%%
%% The "title" command has an optional parameter,
%% allowing the author to define a "short title" to be used in page headers.
\title{Raising the Bar in Graph OOD Generalization: \\ Invariant Learning Beyond Explicit Environment Modeling}

%%
%% The "author" command and its associated commands are used to define
%% the authors and their affiliations.
%% Of note is the shared affiliation of the first two authors, and the
%% "authornote" and "authornotemark" commands
%% used to denote shared contribution to the research.
\author{Xu Shen }
\authornote{Both authors contributed equally to this research.}

\affiliation{%
  \institution{Jilin University}
 \streetaddress{}
  \city{Changchun}
  \country{China}}
  \email{shenxu23@mails.jlu.edu.cn}

\author{Yixin Liu }
\authornotemark[1]
\affiliation{%
  \institution{Griffith University}
  \streetaddress{}
  \city{Goldcoast}
  \country{Australia}}
\email{yixin.liu@griffith.edu.au}

\author{Yili Wang }

\affiliation{%
  \institution{Jilin University}
  \streetaddress{}
  \city{Changchun}
  \country{China}}
\email{wangyl21@mails.jlu.edu.cn}

\author{Rui Miao}
\affiliation{%
  \institution{Jilin University}
  \streetaddress{}
  \city{Changchun}
  \country{China}}
\email{ruimiao20@mails.jlu.edu.cn}

\author{Yiwei Dai}
\affiliation{%
  \institution{Jilin University}
  \streetaddress{}
  \city{Changchun}
  \country{China}}
\email{daiyw23@mails.jlu.edu.cn}

\author{Shirui Pan }
\affiliation{%
  \institution{Griffith University}
  \streetaddress{}
  \city{Goldcoast}
  \country{Australia}}
\email{s.pan@griffith.edu.au}

\author{Xin Wang}
\authornote{Corresponding author.}
% \authornotemark[1]
\affiliation{%
  \institution{Jilin University}
  \streetaddress{}
  \city{Changchun}
  \country{China}}
\email{xinwang@jlu.edu.cn}

%%
%% By default, the full list of authors will be used in the page
%% headers. Often, this list is too long, and will overlap
%% other information printed in the page headers. This command allows
%% the author to define a more concise list
%% of authors' names for this purpose.
\renewcommand{\shortauthors}{Xu Shen et al.}

%%
%% The abstract is a short summary of the work to be presented in the
%% article.

%%
%% The "title" command has an optional parameter,
%% allowing the author to define a "short title" to be used in page headers.


%%
%% The "author" command and its associated commands are used to define
%% the authors and their affiliations.
%% Of note is the shared affiliation of the first two authors, and the
%% "authornote" and "authornotemark" commands
%% used to denote shared contribution to the research.

%%
%% The abstract is a short summary of the work to be presented in the
%% article.
\begin{abstract}
Out-of-distribution (OOD) generalization has emerged as a critical challenge in graph learning, as real-world graph data often exhibit diverse and shifting environments that traditional models fail to generalize across. A promising solution to address this issue is graph invariant learning (GIL), which aims to learn invariant representations by disentangling label-correlated invariant subgraphs from environment-specific subgraphs. However, existing GIL methods face two major challenges: (1) the difficulty of \textbf{capturing and modeling diverse environments} in graph data, and (2) the \textbf{semantic cliff}, where invariant subgraphs from different classes are difficult to distinguish, leading to poor class separability and increased misclassifications. 
To tackle these challenges, we propose a novel method termed \textbf{M}ulti-\textbf{P}rototype \textbf{H}yperspherical \textbf{I}nvariant \textbf{L}earning (\ourmethod), which introduces two key innovations: (1) \textit{hyperspherical invariant representation extraction}, enabling robust and highly discriminative hyperspherical invariant feature extraction, and (2) \textit{multi-prototype hyperspherical classification}, which employs class prototypes as intermediate variables to eliminate the need for explicit environment modeling in GIL and mitigate the semantic cliff issue. Derived from the theoretical framework of GIL, we introduce two novel objective functions: the \textit{invariant prototype matching loss} to ensure samples are matched to the correct class prototypes, and the \textit{prototype separation loss} to increase the distinction between prototypes of different classes in the hyperspherical space.
Extensive experiments on 11 OOD generalization benchmark datasets demonstrate that \ourmethod achieves state-of-the-art performance, significantly outperforming existing methods across graph data from various domains and with different distribution shifts. The source code of \ourmethod is available at \href{https://anonymous.4open.science/r/MPHIL-23C0/}{https://anonymous.4open.science/r/MPHIL-23C0/}.
\end{abstract}

%%
%% The code below is generated by the tool at http://dl.acm.org/ccs.cfm.
%% Please copy and paste the code instead of the example below.
%%
\begin{CCSXML}
<ccs2012>
   <concept>
       <concept_id>10002950.10003624.10003633.10010917</concept_id>
       <concept_desc>Mathematics of computing~Graph algorithms</concept_desc>
       <concept_significance>500</concept_significance>
   </concept>
   <concept>
       <concept_id>10010147.10010341.10010366.10010367</concept_id>
       <concept_desc>Computing methodologies~Machine learning</concept_desc>
       <concept_significance>500</concept_significance>
   </concept>
   <concept>
       <concept_id>10010405.10010406.10010430.10010431</concept_id>
       <concept_desc>Applied computing</concept_desc>
       <concept_significance>300</concept_significance>
   </concept>
   <concept>
       <concept_id>10002951.10003227.10003241</concept_id>
       <concept_desc>Information systems~Data mining</concept_desc>
       <concept_significance>300</concept_significance>
   </concept>
</ccs2012>
\end{CCSXML}

\ccsdesc[500]{Mathematics of computing~Graph algorithms}
\ccsdesc[500]{Computing methodologies~Machine learning}
\ccsdesc[300]{Applied computing}
\ccsdesc[300]{Information systems~Data mining}
%%
%% Keywords. The author(s) should pick words that accurately describe
%% the work being presented. Separate the keywords with commas.
\keywords{Graph out-of-distribution generalization, invariant learning, hyperspherical space}
%% A "teaser" image appears between the author and affiliation
%% information and the body of the document, and typically spans the
%% page.


\received{20 February 2007}
\received[revised]{12 March 2009}
\received[accepted]{5 June 2009}

%%
%% This command processes the author and affiliation and title
%% information and builds the first part of the formatted document.

\maketitle

\section{Introduction} \label{sec:intro}
%\vspace{-4mm}
\section{Introduction}

In machine learning, the availability of vast amounts of unlabeled data has created an opportunity to learn meaningful representations without relying on costly labeled datasets \cite{jaiswal2020survey,shurrab2022self,jing2020self}. Self-supervised learning has emerged as a powerful solution to this problem, allowing models to leverage the inherent structure in data to build useful representations. Among self-supervised methods, contrastive learning (CL) is widely adopted for its ability to create robust representations by distinguishing between similar (positive) and dissimilar (negative) data pairs. With success in fields like image and language processing \cite{chen2020simple,radford2021learning}, contrastive learning now also shows promise in domains where cross-modal, noisy, or structurally complex data make labeling especially challenging \cite{liu2021drop,vishnubhotla2024towards,chen2023instance}.


Traditional contrastive learning methods primarily aim to bring positive pairs---often augmentations of the same sample---closer together in representation space. While effective, this approach often struggles with real-world challenges such as noise in views, variations in data quality, or shifts introduced by complex transformations, where positive pairs may not perfectly align. Additionally, in tasks requiring domain generalization, aligning representations across diverse domains (e.g., variations in style or sensor type) is critical but difficult to achieve with standard contrastive learning, which typically lacks mechanisms for incorporating domain-specific relationships. These limitations highlight the need for a more flexible approach that can adapt alignment strategies based on the data structure, allowing for finer control over similarity and dissimilarity among samples. 


To address this challenge, we introduce a novel \emph{generalized contrastive alignment} (GCA) framework, which reinterprets contrastive learning as a distributional alignment problem. Our method allows flexible control over the alignment of samples by defining a target transport plan, \(\mathbf{P}_{tgt}\), that serves as a customizable alignment guide. For example, setting \(\mathbf{P}_{tgt}\) to resemble a diagonal matrix encourages each positive to align primarily with itself or its augmentations, thereby reducing the effect of noise between views. Alternatively, we can incorporate more complex constraints, such as weighting alignments based on view quality or enforcing partial alignment structures where noise or data heterogeneity is prevalent. This flexibility enables GCA to adapt effectively to a wide range of tasks, from simple twin view alignments to scenarios with noisy or variably aligned data.

Our approach also bridges connections between GCA and established methods, such as InfoNCE (INCE) \cite{oord2018representation}, Robust InfoNCE (RINCE) \cite{chuang2022robust}, and BYOL \cite{grill2020bootstrap}, demonstrating that these can be viewed as iterative alignment objectives with Bregman projections \cite{cai2022developments,grathwohl2019your}. This perspective allows us to systematically analyze and improve uniformity within the latent space, a property that enhances representation quality and ultimately boosts downstream classification performance.

We validate our method through extensive experiments on both image classification and noisy data tasks, demonstrating that GCA’s unbalanced OT (UOT) formulations improve classification performance by relaxing our constraints on alignment. Our results show that \ours~offers a robust and versatile framework for contrastive learning, providing flexibility and performance gains over existing methods and presenting a promising approach to addressing different sources of variability in self-supervised learning.


The contributions of this work include:
\begin{itemize}
    \item A new framework called \emph{generalized contrastive alignment} (GCA), which reinterprets standard contrastive learning as a distributional alignment problem, using optimal transport to provide flexible control over alignment objectives. This approach allows us to derive a novel class of contrastive losses and algorithms that adapt effectively to varied data structures and build  customizable transport plans.

    \item We present GCA-UOT, a contrastive learning method that achieves strong performance on standard augmentation regimes and excels in scenarios with more extreme augmentations or data corrupted by transformations. GCA-UOT leverages unbalanced transport to adaptively weight positive alignments, enhancing robustness against view noise and cross-domain variations.

    \item We provide theoretical guarantees for the convergence of our GCA-based methods and show that our alignment objectives improve representation quality by enhancing the uniformity of negatives and strengthening alignment within positive pairs. This leads to more discriminative and resilient representations, even in challenging data conditions.

    \item Empirically, we demonstrate the effectiveness of \ours~in both image classification and domain generalization tasks. Through flexible, unbalanced OT-based losses, \ours~achieves superior classification performance and adapts alignment to include domain-specific information where relevant, without compromising classification accuracy in domain generalization.
\end{itemize}



\section{Related Works} \label{appe:rw}
Our work draws heavily from the literature on semiparametric inference and double machine learning~\citep{robins1994estimation,robins1995semiparametric,tsiatis2006semiparametric,chernozhukov2018double}. In particular, our estimator is an optimal combination of several Augmented Inverse Probability Weighting~(\aipw) estimators, whose outcome regressions are replaced with foundation models. Importantly, the standard $\aipw$ estimator, which relies on an outcome regression estimated using experimental data alone, is also included in the combination. This approach allows \ours~to significantly reduce finite sample (and potentially asymptotic) variance while attaining the semiparametric \emph{efficiency bound}---the smallest asymptotic variance among all consistent and asymptotically normal estimators of the average treatment effect---even when the foundation models are arbitrarily biased.


\paragraph{Integrating foundation models}
Prediction-powered inference~(\ppi)~\citep{angelopoulos2023prediction} is a statistical framework that constructs valid confidence intervals using a small labeled dataset and a large unlabeled dataset imputed by a foundation model. $\ppi$ has been applied in various domains, including generalization of causal inferences~\citep{demirel24prediction}, large language model evaluation~\citep{fisch2024stratified,dorner2024limitsscalableevaluationfrontier}, and improving the efficiency of social science experiments~\citep{broskamixed,egami2024using}. However, unlike our approach, $\ppi$ requires access to an additional unlabeled dataset from the same distribution as the experimental sample, which may be as costly as labeled data. Recent work by \citet{poulet2025prediction} introduces 
Prediction-powered inference for clinical trials ($\ppct$), an adaptation of $\ppi$ to estimate  average treatment effects in randomized experiments without any additional  external data. $\ppct$ combines the difference in means estimator with an 
$\aipw$ estimator that integrates the same foundation model as the outcome regression for both treatment and control groups. However, our work differs in two key aspects:
(i) $\ppct$ integrates a single foundation model, and (ii) $\ppct$ does not include the standard $\aipw$ estimator with the outcome regression estimated from experimental data. As a result, $\ppct$ cannot achieve the efficiency bound unless the foundation model is almost surely equal to the underlying outcome regression. 


 



\paragraph{Integrating observational data} There is growing interest in augmenting randomized experiments with data from observational studies to improve statistical precision. One approach involves first testing whether the observational data is compatible with the experimental data~\citep{dahabreh2024using}---for instance, using a statistical test to assess if the mean of the outcome conditional on the covariates is invariant across studies \cite{luedtke2019omnibus,hussain2023falsification,de2024detecting}—and then combining the datasets to improve precision, if the test does not reject. These tests, however, have low statistical power, especially when the experimental sample size is small, which is precisely when leveraging observational data would be most beneficial. To overcome this, a recent line of work integrates a prognostic score estimated from observational data as a covariate when estimating the outcome regression~\citep{schuler2022increasing,liao2023prognostic}. However, increasing the dimensionality of the problem---by adding an additional covariate---can increase estimation error and inflate the finite sample variance. Finally, the work most closely related to ours is \citet{karlsson2024robust}, that integrates an outcome regression estimated from observational data into the \aipw~estimator. In contrast, our approach is not constrained by the availability of well-structured observational data, since it leverages black-box foundation models trained on external data sources.
\section{Preliminaries and Background}
In this section, we introduce the preliminaries and background of this work, including the formulation of the graph OOD generalization problem, graph invariant learning, and hyperspherical embeddings.

% \subsection{Problem Formulation}
\subsection{Problem Formulation} 
In this paper, we focus on the OOD generalization problem on graph classification tasks~\citep{li2022out,jia2024graph,fan2022debiasing,wu2022discovering}. We denote a graph data sample as $(G,y)$, where $G \in \mathcal{G}$ represents a graph instance and $y \in \mathcal{Y}$ represents its label. The dataset collected from a set of environments $\mathcal{E}$ is denoted as $\mathcal{D} = \{\mathcal{D}^{e}\}_{e \in \mathcal{E}}$, where $\mathcal{D}^{e} =\{({G}^{e}_{i},{y}^{e}_{i})\}^{n^{e}}_{i=1}$ represents the data from environment $e$, and $n^e$ is the number of instances in environment $e$. Each pair $({G}^{e}_{i}, {y}^{e}_{i})$ is sampled independently from the joint distribution $P_{e}(\mathcal{G}, \mathcal{Y}) = P(\mathcal{G}, \mathcal{Y} | e)$. 
In the context of graph OOD generalization, the difficulty arises from the discrepancy between the training data distribution $P_{e_{tr}}(\mathcal{G}, \mathcal{Y})$ from environments $e_{tr} \in \mathcal{E}_{tr}$, and the testing data distribution $P_{e_{te}}(\mathcal{G}, \mathcal{Y})$ from unseen environments $e_{te} \in \mathcal{E}_{test}$, where $\mathcal{E}_{te} \neq \mathcal{E}_{tr}$. The goal of OOD generalization is to learn an optimal predictor $f: \mathcal{G} \rightarrow \mathcal{Y}$ that performs well across both training and unseen environments, $\mathcal{E}_{all} = \mathcal{E}_{tr} \cup \mathcal{E}_{te}$, i.e., 
\begin{equation}
\label{eq: OOD_target} \min_{f \in \mathcal{F}} \max_{e \in \mathcal{E}_{\mathrm{all}}} \mathbb{E}_{(G^{e}, y^{e}) \sim P_{e}}[\ell(f(G^e), y^e)], 
\end{equation}
where $\mathcal{F}$ denotes the hypothesis space, and $\ell(\cdot,\cdot)$ represents the empirical risk function. 

\subsection{Graph Invariant Learning (GIL)}
Invariant learning focuses on capturing representations that preserve consistency across different environments, ensuring that the learned invariant representation $\mathbf{z}_{inv}$ maintains consistency with the label $y$~\citep{mitrovic2020representation,wu2022discovering,chen2022learning}. Specifically, for graph OOD generalization, the objective of GIL is to learn an invariant GNN $f:= f_{c} \circ g$, where $g: \mathcal{G} \rightarrow \mathcal{Z}_{inv}$ is an encoder that extracts the invariant representation from the input graph $G$, and $f_{c}: \mathcal{Z}_{inv} \rightarrow \mathcal{Y}$ is a classifier that predicts the label $y$ based on $\mathbf{z}_{inv}$. From this perspective, the optimization objective of OOD generalization, as stated in Eq.~(\ref{eq: OOD_target}), can be reformulated as: 
\begin{equation}
\label{eq: causal} \max_{f_{c}, g} I(\mathbf{z}_{inv}; y), \text{ s.t. } \mathbf{z}_{inv} \perp e,\forall e \in \mathcal{E}_{tr}, \mathbf{z}_{inv} = g(G),
\end{equation}
{where $I(\mathbf{z}_{inv}; y)$ denotes the mutual information between the invariant representation $\mathbf{z}_{inv}$ and the label $y$.}
This objective ensures that $\mathbf{z}_{inv}$ is independent of the environment $e$, focusing solely on the most relevant information for predicting $y$. 

\subsection{Hyperspherical Embedding} 
Hyperspherical learning enhances the discriminative ability and generalization of deep learning models by mapping feature vectors onto a unit sphere~\citep{liu2017deep}. 
To learn a hyperspherical embedding for the input sample, its representation vector $\mathbf{z}$ is mapped into hyperspherical space with arbitrary linear or non-linear projection functions, followed by normalization to ensure that the projected vector $\hat{\mathbf{z}}$ lies on the unit hypersphere ($\|\hat{\mathbf{z}}\|^{2}=1$). 
To make classification prediction, the hyperspherical embeddings $\hat{\mathbf{z}}$ are modeled using the von Mises-Fisher (vMF) distribution~\citep{ming2022exploit}, with the probability density for a unit vector in class $c$ is given by:
\begin{equation}
\label{eq: vMF} p(\hat{\mathbf{z}}; \boldsymbol{\mu}^{(c)}, \kappa ) = Z(\kappa) \exp(\kappa {\boldsymbol{\mu}^{(c)}}^\top \hat{\mathbf{z}}), 
\end{equation} 
where $\boldsymbol{\mu}^{(c)}$ denotes the prototype vector of class $c$ with the unit norm, serving as the mean direction for class $c$, while $\kappa$ controls the concentration of samples around $\boldsymbol{\mu}_c$.
The term $Z(\kappa)$ serves as the normalization factor for the distribution. Given the probability model in Eq.(\ref{eq: vMF}), the hyperspherical embedding $\hat{\mathbf{z}}$ is assigned to class $c$ with the following probability:
\begin{equation} \label{eq: prototpye_1}
    \begin{aligned}
\mathbb{P}\left(y = c \mid \hat{\mathbf{z}}; \{\kappa, \boldsymbol{\mu}^{(i)}\}_{i = 1}^{C}\right) &= \frac{Z(\kappa) \exp \left(\kappa {\boldsymbol{\mu}^{(c)}}^{\top} \hat{\mathbf{z}}\right)}{\sum_{i = 1}^{C} Z(\kappa) \exp \left(\kappa {\boldsymbol{\mu}^{(i)}}^{\top} \hat{\mathbf{z}}\right)}\\ &= \frac{\exp \left({\boldsymbol{\mu}^{(c)}}^{\top} \hat{\mathbf{z}} / \tau\right)}{\sum_{i = 1}^{C} \exp \left({\boldsymbol{\mu}^{(i)}}^{\top} \hat{\mathbf{z}}/ \tau\right)},
    \end{aligned}
\end{equation}
where $\tau = 1/\kappa$ is a temperature parameter. In this way, the classification problem is transferred to the distance measurement between the graph embedding and the prototype of each class in hyperspherical space, where the class prototype is usually defined as the embedding centroid of each class.

\section{Methodology}
\section{Methodology}


\begin{figure*}[t]
  \centering
\includegraphics[width=0.70\textwidth]{main_figure_v2.png}
  \vspace{-1em}
  \caption{Systematic overview of our Chorus CVR model.}
  \label{choruscvr}
  \vspace{-1em}
\end{figure*}


\subsection{Preliminary}
In the ranking stage of industrial recommendation system, all \textit{exposure} user-item pairs will be collected and formed as a data-streaming for model training, i.e., $\mathcal{D}$.
Specifically, each user-item sample in $\mathcal{D}$ could represent as $(u, i, \{\mathbf{x}_u, \mathbf{x}_i,$ $\mathbf{x}_{ui}\}, o_{ui}, r_{ui}) \in \mathcal{D}$, where $u$/$i$ denotes the user-item pair, $\mathbf{x}_u\in\mathbb{R}^{d_u}, \mathbf{x}_i\in\mathbb{R}^{d_i}, \mathbf{x}_{ui}\in\mathbb{R}^{d_{ui}}$ are the user-side features (e.g., user ID), item-side features (e.g., item ID), and item-aware cross features (e.g., SIM \cite{sim}).
%
The $o_{ui}\in\{0,1\}$ and $r_{ui}\in\{0,1\}$ are user-item ground-truth interacted labels, where $o_{ui}$ denotes whether user $u$ clicked item $i$ and $r_{ui}$ denotes whether user $u$ converted item $i$. 
%
According to the entire \textbf{\textit{exposure} space} $\mathcal{D}$, we could further obtain several subset spaces:
%
\begin{itemize}[leftmargin=*,align=left]
\item \textbf{\textit{Click} space} $\mathcal{O}\in\mathcal{D}$, if click label $o_{ui} = 1$.
\item \textbf{\textit{un-Click} space} $\mathcal{N}=\mathcal{D} - \mathcal{O}$, if click label $o_{ui} = 0$.
\item \textbf{\textit{Conversion} space} $\mathcal{R}\in\mathcal{O}$: if label $o_{ui}=1$ and $r_{ui}=1$.
\item \textbf{\textit{un-Conversion} space} $\mathcal{M}=\mathcal{O}-\mathcal{R}$: if label $o_{ui}=1$ and $r_{ui}=0$.
\end{itemize}
%
Based on them, a simple ranking model can be formed as:
\begin{equation}
% \small
\begin{split}
&\hat{y}^{ctr}_{ui} = \texttt{MLP}^{ctr}(\mathbf{x}_{ui}),\quad \hat{y}^{cvr}_{ui} = \texttt{MLP}^{cvr}(\mathbf{x}_{ui}),\\
&\mathbf{x}_{ui} = \texttt{Multi-Task-Encoder}(\mathbf{x}_u\oplus \mathbf{x}_i\oplus \mathbf{x}_{ui}),\\
\end{split}
\label{base}
\end{equation}
where the $\oplus$ denotes the concatenate operator, $\mathbf{x}\in\mathbb{R}^d$ is the encoded hidden states, and $\texttt{MLP}(\cdot)$ denotes a stacked neural-network. We use a share-bottom based multi-task paradigm to predict CTR and CVR scores, $\hat{y}^{ctr},\hat{y}^{cvr}$.
%
Next, we directly minimize the cross-entropy binary classification loss to train CTR tower and CVR towers with corresponding space samples:
%
%
%
\begin{equation}
% \small
% \footnotesize
\begin{split}
&\mathcal{L}^{ctr} = - \frac{1}{|\mathcal{D}|}\big(\sum_{(u,i)\in\mathcal{D}}\delta(\hat{y}^{ctr}_{ui}, o_{ui})\big),\\
&\mathcal{L}^{cvr} = - \frac{1}{|\mathcal{O}|}\big(\sum_{(u,i)\in\mathcal{O}}\delta(\hat{y}^{cvr}_{ui}, r_{ui})\big).
\end{split}
\label{crossentropy}
\end{equation}
%
% 
%
In inference, given the hundreds item candidates in a certain user request, we could obtain predicted CTCVR by  $\hat{y}^{ctcvr}_{ui} =\hat{y}^{ctr}_{ui} \cdot \hat{y}^{cvr}_{ui}$ for each item, which is used for final ranking. Then top K highest items will be returned and shown to user. The CVR are learned in click space during training but be predicted in an assumed explore space during inference, which brings up the question of sample selection bias problem.
% 





To alleviate sample selection bias,  ESMM \cite{essm} expand the click-space CVR learning task to exposure-space CTCVR learning task, to directly solve the inconsistency between training and inference:
\begin{equation}
% \small
\begin{split}
\mathcal{L}^{ctcvr} = &- \frac{1}{|\mathcal{D}|}\Big(\sum_{(u,i)\in\mathcal{D}}\delta(\hat{y}^{ctr}_{ui}\cdot\hat{y}^{cvr}_{ui}, o_{ui}\cdot r_{ui})\Big)
\end{split}
\label{ctxcvr}
\end{equation}
which treats all un-clicked samples as negative samples of CTCVR task. However those un-clicked samples that would be converted if clicked, which are falsely negative samples,  still leads to missing not at random (MNAR) problem  \cite{multiipw}. To mitigate this problem, inverse propensity weighting (IPW) \cite{multiipw,escm2} based method inversely weight the CVR loss in click space by propensity score of observing  $(u,i)$ in click space $\mathcal{O}$, to eliminate the influence of click event to CVR estimation in entire space $D$
\begin{equation}
\small
\begin{split}
\mathcal{L}^{cvr}_{IPW} =& - \frac{1}{|\mathcal{O}|}\Big(\sum_{(u,i)\in\mathcal{O}}\frac{\delta(\hat{y}^{cvr}_{ui}, r_{ui})}{\hat{y}^{ctr}_{ui}}\Big),
\end{split}
\label{cvripw}
\end{equation}
% 
%
Our method is based on above ESMM with IPW  framework. Although alleviating SSB and MNAR problem, IPW-based methods still lack reasonable labels for \textit{un-clicked} samples, which we solve by generating discriminative and robust soft labels.






\subsection{ChorusCVR}
In this section, we dive into ChorusCVR and explain how we realize entire-space debiased CVR learning by generating discriminative and robust soft CVR labels (as shown in Figure~\ref{choruscvr}).

\subsubsection{Negative sample Discrimination Module (NDM)}

As mentioned before, the soft labels introduced by previous works are suboptimal for lack either discriminability or robustness. As shown in Figure~\ref{intro} (d), an ideal discrimination surface should separate the factual negative samples (clicked but un-converted) from positive samples (clicked \& converted), and factual negative samples from ambiguous negative samples (un-clicked). With this in mind, we find the ideal discrimination surface implies a new task, CTunCVR prediction. We formulate CTunCVR labels as:
\begin{equation}
y^{ctuncvr} = o_{ui} * (1-r_{ui}) = 
\begin{cases} 
1 & o_{ui}=1~\&~r_{ui} = 0, \\
0 &  o_{ui}=0, \\
0 & o_{ui}=1~\&~r_{ui} = 1,
\end{cases}
\end{equation}
where only \textit{clicked but un-converted} samples are positive samples, both \textit{clicked \& converted} and \textit{un-clicked} samples are negative samples. Instead of directly predicting CTunCVR score in exposure space, we follow a typical two-stage prediction paradigm to obtain CTunCVR to 
reduce cumulative error. We firstly introduce an additional unCVR tower to predict unCVR score $\hat{y}^{uncvr}$, then combine it with $\hat{y}^{ctr}$ to form CTunCVR score:
% 
\begin{equation}
% \small
\begin{split}
\hat{y}^{uncvr}_{ui} = \hat{y}^{ctr}_{ui}\cdot\hat{y}^{cvr}_{ui}\quad \quad
\hat{y}^{ctuncvr}_{ui} =\hat{y}^{ctr}_{ui}\cdot\hat{y}^{uncvr}_{ui}
\end{split}
\label{uncvr}
\end{equation}
Then we can naturally optimize CTunCVR objective in exposure space by cross entropy loss: 
\begin{equation}
% \small
\begin{split}
\mathcal{L}^{ctuncvr} = - \frac{1}{|\mathcal{D}|}\Big(\sum_{(u,i)\in\mathcal{D}}\delta(\hat{y}^{ctuncvr}_{ui}, o_{ui} * (1-r_{ui})\Big).
\end{split}
\label{uncvr}
\end{equation}
% With the help of this formulation, we can narrow down the problem to the accurate estimation of unCVR. 
With the help of $\mathcal{L}^{ctuncvr}$ and an extra \textbf{unCVR prediction result} $\hat{y}^{uncvr}_{ui}$, we can narrow down the aforementioned problem to consider \textbf{R1. Discriminability} and \textbf{R2. Robustness} problem at same time.
%
For the $\hat{y}^{uncvr}_{ui}$ generation, we add an mirror unCVR tower which similar with the Eq.(\ref{base}) and (\ref{crossentropy}):
%
%
%
% 
%
%
% $
% 
\begin{equation}
% \smalluncvr
\begin{split}
\hat{y}^{uncvr}_{ui} &= \texttt{MLP}^{uncvr}(\mathbf{x}_{ui}), \\
\mathcal{L}^{uncvr} = - \frac{1}{|\mathcal{O}|}\Big(&\sum_{(u,i)\in\mathcal{O}}\delta\big(\hat{y}^{uncvr}_{ui}, 1-r_{ui})\big)\Big)
\end{split}
\label{uncvr}
\end{equation}
% 
Next, analogously with the Eq.(\ref{cvripw}), we then adopt the predicted click $\hat{y}^{ctr}_{ui}$ to inversely weight the unCVR error, to $\mathcal{L}^{uncvr}$ as:
% 
\begin{equation}
% \small
\begin{split}
\mathcal{L}^{uncvr}_{IPW} = - \frac{1}
{|\mathcal{O}|}\Big(\sum_{(u,i)\in\mathcal{O}}\frac{\delta(\hat{y}^{uncvr}_{ui}, 1-r_{ui})}{\hat{y}^{ctr}_{ui}}\Big)
\end{split}
\label{uncvr}
\end{equation}
In this way, the \textit{click} space tendency can be alleviated that higher/lower $pCTR$ sample will declined/enhanced for a fair training. So far we obtain debiased unCVR soft labels, which we will utilize to help the CTunCVR training and CVR component supervision.



\subsubsection{Soft Alignment Module (SAM)}
Up to now, we fulfill the initial goal of obtain high-quality soft labels in un-clicked space. In this section we present the solution to utilize the unCVR score as soft labels to supervise CVR learning, which we call \emph{soft alignment mechanism}. We first use $1-unCVR$ manner as soft labels for entropy-based CVR learning. In the same time, we also use $1-CVR$ manner to generate soft labels for unCVR learning, in a mutual supervision fashion to align unCVR predictions to CVR. All these objectives are inversely weighted by predicted CTR in a IPW paradigm (see $\mathcal{L}^{align1}_{IPW}$ and $\mathcal{L}^{align2}_{IPW}$ in Figure.~\ref{choruscvr}). To further alleviate SSB for un-click space, we also propose a \textit{un-click space IPW} approach, to inversely weight the un-click samples with $1-pCTR$ for CTR and unCVR alignment objectives (see $\mathcal{L}^{align3}_{IPW}$ and $\mathcal{L}^{align4}_{IPW}$ in Figure.~\ref{choruscvr}). Overall, all alignment objectives are as follows:
\begin{equation}
% \small
\begin{split}
\mathcal{L}^{align}_{IPW} = &- \frac{1}{|\mathcal{O}|}\big(\frac{\delta(\hat{y}^{cvr}_{ui}, 1-\texttt{sg}(\hat{y}^{uncvr}_{ui}))}{\hat{y}^{ctr}_{ui}}\big)
- \frac{1}{|\mathcal{N}|}\big(\frac{\delta(\hat{y}^{cvr}_{ui}, 1-\texttt{sg}(\hat{y}^{uncvr}_{ui}))}{1-\hat{y}^{ctr}_{ui}}\big)\\
&- \frac{1}{|\mathcal{O}|}\big(\frac{\delta(\hat{y}^{uncvr}_{ui}, 1-\texttt{sg}(\hat{y}^{cvr}_{ui}))}{\hat{y}^{ctr}_{ui}}\big)
- \frac{1}{|\mathcal{N}|}\big(\frac{\delta(\hat{y}^{uncvr}_{ui}, 1-\texttt{sg}(\hat{y}^{cvr}_{ui}))}{1-\hat{y}^{ctr}_{ui}}\big)
\end{split}
\label{soft}
\end{equation}

where the $\texttt{sg}(\cdot)$ means the stop gradient function, the $\hat{y}^{ctr}_{ui}, (1 - \hat{y}^{ctr}_{ui})$ denote the click propensity in the \textit{click} and \textit{un-click} space, respectively.
% 
%
All losses of our ChorusCVR are as follows:
\begin{equation}
% \small
\begin{split}
\mathcal{L} = \mathcal{L}^{ctcvr} + \mathcal{L}^{cvr}_{IPW} + \mathcal{L}^{ctuncvr} + \mathcal{L}^{uncvr}_{IPW} + \mathcal{L}^{align}_{IPW}
\end{split}
\label{soft}
\end{equation}
In this way, our ChorusCVR  make CVR and unCVR supervise each other during training, which results in an equilibrium. 

\section{Experiments}

In this section, we present our experimental setup  (Sec.~\ref{subsec:setup}) and showcase the results in (Sec.~\ref{subsec:results}). For each experiment, we first highlight the research question being addressed, followed by a detailed discussion of the findings.


% This must be in the first 5 lines to tell arXiv to use pdfLaTeX, which is strongly recommended.
\pdfoutput=1
% In particular, the hyperref package requires pdfLaTeX in order to break URLs across lines.

\documentclass[11pt]{article}

% Change "review" to "final" to generate the final (sometimes called camera-ready) version.
% Change to "preprint" to generate a non-anonymous version with page numbers.
\usepackage{acl}

% Standard package includes
\usepackage{times}
\usepackage{latexsym}

% Draw tables
\usepackage{booktabs}
\usepackage{multirow}
\usepackage{xcolor}
\usepackage{colortbl}
\usepackage{array} 
\usepackage{amsmath}

\newcolumntype{C}{>{\centering\arraybackslash}p{0.07\textwidth}}
% For proper rendering and hyphenation of words containing Latin characters (including in bib files)
\usepackage[T1]{fontenc}
% For Vietnamese characters
% \usepackage[T5]{fontenc}
% See https://www.latex-project.org/help/documentation/encguide.pdf for other character sets
% This assumes your files are encoded as UTF8
\usepackage[utf8]{inputenc}

% This is not strictly necessary, and may be commented out,
% but it will improve the layout of the manuscript,
% and will typically save some space.
\usepackage{microtype}
\DeclareMathOperator*{\argmax}{arg\,max}
% This is also not strictly necessary, and may be commented out.
% However, it will improve the aesthetics of text in
% the typewriter font.
\usepackage{inconsolata}

%Including images in your LaTeX document requires adding
%additional package(s)
\usepackage{graphicx}
% If the title and author information does not fit in the area allocated, uncomment the following
%
%\setlength\titlebox{<dim>}
%
% and set <dim> to something 5cm or larger.

\title{Wi-Chat: Large Language Model Powered Wi-Fi Sensing}

% Author information can be set in various styles:
% For several authors from the same institution:
% \author{Author 1 \and ... \and Author n \\
%         Address line \\ ... \\ Address line}
% if the names do not fit well on one line use
%         Author 1 \\ {\bf Author 2} \\ ... \\ {\bf Author n} \\
% For authors from different institutions:
% \author{Author 1 \\ Address line \\  ... \\ Address line
%         \And  ... \And
%         Author n \\ Address line \\ ... \\ Address line}
% To start a separate ``row'' of authors use \AND, as in
% \author{Author 1 \\ Address line \\  ... \\ Address line
%         \AND
%         Author 2 \\ Address line \\ ... \\ Address line \And
%         Author 3 \\ Address line \\ ... \\ Address line}

% \author{First Author \\
%   Affiliation / Address line 1 \\
%   Affiliation / Address line 2 \\
%   Affiliation / Address line 3 \\
%   \texttt{email@domain} \\\And
%   Second Author \\
%   Affiliation / Address line 1 \\
%   Affiliation / Address line 2 \\
%   Affiliation / Address line 3 \\
%   \texttt{email@domain} \\}
% \author{Haohan Yuan \qquad Haopeng Zhang\thanks{corresponding author} \\ 
%   ALOHA Lab, University of Hawaii at Manoa \\
%   % Affiliation / Address line 2 \\
%   % Affiliation / Address line 3 \\
%   \texttt{\{haohany,haopengz\}@hawaii.edu}}
  
\author{
{Haopeng Zhang$\dag$\thanks{These authors contributed equally to this work.}, Yili Ren$\ddagger$\footnotemark[1], Haohan Yuan$\dag$, Jingzhe Zhang$\ddagger$, Yitong Shen$\ddagger$} \\
ALOHA Lab, University of Hawaii at Manoa$\dag$, University of South Florida$\ddagger$ \\
\{haopengz, haohany\}@hawaii.edu\\
\{yiliren, jingzhe, shen202\}@usf.edu\\}



  
%\author{
%  \textbf{First Author\textsuperscript{1}},
%  \textbf{Second Author\textsuperscript{1,2}},
%  \textbf{Third T. Author\textsuperscript{1}},
%  \textbf{Fourth Author\textsuperscript{1}},
%\\
%  \textbf{Fifth Author\textsuperscript{1,2}},
%  \textbf{Sixth Author\textsuperscript{1}},
%  \textbf{Seventh Author\textsuperscript{1}},
%  \textbf{Eighth Author \textsuperscript{1,2,3,4}},
%\\
%  \textbf{Ninth Author\textsuperscript{1}},
%  \textbf{Tenth Author\textsuperscript{1}},
%  \textbf{Eleventh E. Author\textsuperscript{1,2,3,4,5}},
%  \textbf{Twelfth Author\textsuperscript{1}},
%\\
%  \textbf{Thirteenth Author\textsuperscript{3}},
%  \textbf{Fourteenth F. Author\textsuperscript{2,4}},
%  \textbf{Fifteenth Author\textsuperscript{1}},
%  \textbf{Sixteenth Author\textsuperscript{1}},
%\\
%  \textbf{Seventeenth S. Author\textsuperscript{4,5}},
%  \textbf{Eighteenth Author\textsuperscript{3,4}},
%  \textbf{Nineteenth N. Author\textsuperscript{2,5}},
%  \textbf{Twentieth Author\textsuperscript{1}}
%\\
%\\
%  \textsuperscript{1}Affiliation 1,
%  \textsuperscript{2}Affiliation 2,
%  \textsuperscript{3}Affiliation 3,
%  \textsuperscript{4}Affiliation 4,
%  \textsuperscript{5}Affiliation 5
%\\
%  \small{
%    \textbf{Correspondence:} \href{mailto:email@domain}{email@domain}
%  }
%}

\begin{document}
\maketitle
\begin{abstract}
Recent advancements in Large Language Models (LLMs) have demonstrated remarkable capabilities across diverse tasks. However, their potential to integrate physical model knowledge for real-world signal interpretation remains largely unexplored. In this work, we introduce Wi-Chat, the first LLM-powered Wi-Fi-based human activity recognition system. We demonstrate that LLMs can process raw Wi-Fi signals and infer human activities by incorporating Wi-Fi sensing principles into prompts. Our approach leverages physical model insights to guide LLMs in interpreting Channel State Information (CSI) data without traditional signal processing techniques. Through experiments on real-world Wi-Fi datasets, we show that LLMs exhibit strong reasoning capabilities, achieving zero-shot activity recognition. These findings highlight a new paradigm for Wi-Fi sensing, expanding LLM applications beyond conventional language tasks and enhancing the accessibility of wireless sensing for real-world deployments.
\end{abstract}

\section{Introduction}

In today’s rapidly evolving digital landscape, the transformative power of web technologies has redefined not only how services are delivered but also how complex tasks are approached. Web-based systems have become increasingly prevalent in risk control across various domains. This widespread adoption is due their accessibility, scalability, and ability to remotely connect various types of users. For example, these systems are used for process safety management in industry~\cite{kannan2016web}, safety risk early warning in urban construction~\cite{ding2013development}, and safe monitoring of infrastructural systems~\cite{repetto2018web}. Within these web-based risk management systems, the source search problem presents a huge challenge. Source search refers to the task of identifying the origin of a risky event, such as a gas leak and the emission point of toxic substances. This source search capability is crucial for effective risk management and decision-making.

Traditional approaches to implementing source search capabilities into the web systems often rely on solely algorithmic solutions~\cite{ristic2016study}. These methods, while relatively straightforward to implement, often struggle to achieve acceptable performances due to algorithmic local optima and complex unknown environments~\cite{zhao2020searching}. More recently, web crowdsourcing has emerged as a promising alternative for tackling the source search problem by incorporating human efforts in these web systems on-the-fly~\cite{zhao2024user}. This approach outsources the task of addressing issues encountered during the source search process to human workers, leveraging their capabilities to enhance system performance.

These solutions often employ a human-AI collaborative way~\cite{zhao2023leveraging} where algorithms handle exploration-exploitation and report the encountered problems while human workers resolve complex decision-making bottlenecks to help the algorithms getting rid of local deadlocks~\cite{zhao2022crowd}. Although effective, this paradigm suffers from two inherent limitations: increased operational costs from continuous human intervention, and slow response times of human workers due to sequential decision-making. These challenges motivate our investigation into developing autonomous systems that preserve human-like reasoning capabilities while reducing dependency on massive crowdsourced labor.

Furthermore, recent advancements in large language models (LLMs)~\cite{chang2024survey} and multi-modal LLMs (MLLMs)~\cite{huang2023chatgpt} have unveiled promising avenues for addressing these challenges. One clear opportunity involves the seamless integration of visual understanding and linguistic reasoning for robust decision-making in search tasks. However, whether large models-assisted source search is really effective and efficient for improving the current source search algorithms~\cite{ji2022source} remains unknown. \textit{To address the research gap, we are particularly interested in answering the following two research questions in this work:}

\textbf{\textit{RQ1: }}How can source search capabilities be integrated into web-based systems to support decision-making in time-sensitive risk management scenarios? 
% \sq{I mention ``time-sensitive'' here because I feel like we shall say something about the response time -- LLM has to be faster than humans}

\textbf{\textit{RQ2: }}How can MLLMs and LLMs enhance the effectiveness and efficiency of existing source search algorithms? 

% \textit{\textbf{RQ2:}} To what extent does the performance of large models-assisted search align with or approach the effectiveness of human-AI collaborative search? 

To answer the research questions, we propose a novel framework called Auto-\
S$^2$earch (\textbf{Auto}nomous \textbf{S}ource \textbf{Search}) and implement a prototype system that leverages advanced web technologies to simulate real-world conditions for zero-shot source search. Unlike traditional methods that rely on pre-defined heuristics or extensive human intervention, AutoS$^2$earch employs a carefully designed prompt that encapsulates human rationales, thereby guiding the MLLM to generate coherent and accurate scene descriptions from visual inputs about four directional choices. Based on these language-based descriptions, the LLM is enabled to determine the optimal directional choice through chain-of-thought (CoT) reasoning. Comprehensive empirical validation demonstrates that AutoS$^2$-\ 
earch achieves a success rate of 95–98\%, closely approaching the performance of human-AI collaborative search across 20 benchmark scenarios~\cite{zhao2023leveraging}. 

Our work indicates that the role of humans in future web crowdsourcing tasks may evolve from executors to validators or supervisors. Furthermore, incorporating explanations of LLM decisions into web-based system interfaces has the potential to help humans enhance task performance in risk control.






\section{Related Work}
\label{sec:relatedworks}

\input{tables/tab-related-full}

\noindent \textbf{Prompting-based LLM Agents.} Due to the lack of agent-specific pre-training corpus, existing LLM agents rely on either prompt engineering~\cite{hsieh2023tool,lu2024chameleon,yao2022react,wang2023voyager} or instruction fine-tuning~\cite{chen2023fireact,zeng2023agenttuning} to understand human instructions, decompose high-level tasks, generate grounded plans, and execute multi-step actions. 
However, prompting-based methods mainly depend on the capabilities of backbone LLMs (usually commercial LLMs), failing to introduce new knowledge and struggling to generalize to unseen tasks~\cite{sun2024adaplanner,zhuang2023toolchain}. 

\noindent \textbf{Instruction Finetuning-based LLM Agents.} Considering the extensive diversity of APIs and the complexity of multi-tool instructions, tool learning inherently presents greater challenges than natural language tasks, such as text generation~\cite{qin2023toolllm}.
Post-training techniques focus more on instruction following and aligning output with specific formats~\cite{patil2023gorilla,hao2024toolkengpt,qin2023toolllm,schick2024toolformer}, rather than fundamentally improving model knowledge or capabilities. 
Moreover, heavy fine-tuning can hinder generalization or even degrade performance in non-agent use cases, potentially suppressing the original base model capabilities~\cite{ghosh2024a}.

\noindent \textbf{Pretraining-based LLM Agents.} While pre-training serves as an essential alternative, prior works~\cite{nijkamp2023codegen,roziere2023code,xu2024lemur,patil2023gorilla} have primarily focused on improving task-specific capabilities (\eg, code generation) instead of general-domain LLM agents, due to single-source, uni-type, small-scale, and poor-quality pre-training data. 
Existing tool documentation data for agent training either lacks diverse real-world APIs~\cite{patil2023gorilla, tang2023toolalpaca} or is constrained to single-tool or single-round tool execution. 
Furthermore, trajectory data mostly imitate expert behavior or follow function-calling rules with inferior planning and reasoning, failing to fully elicit LLMs' capabilities and handle complex instructions~\cite{qin2023toolllm}. 
Given a wide range of candidate API functions, each comprising various function names and parameters available at every planning step, identifying globally optimal solutions and generalizing across tasks remains highly challenging.



\section{Preliminaries}
\label{Preliminaries}
\begin{figure*}[t]
    \centering
    \includegraphics[width=0.95\linewidth]{fig/HealthGPT_Framework.png}
    \caption{The \ourmethod{} architecture integrates hierarchical visual perception and H-LoRA, employing a task-specific hard router to select visual features and H-LoRA plugins, ultimately generating outputs with an autoregressive manner.}
    \label{fig:architecture}
\end{figure*}
\noindent\textbf{Large Vision-Language Models.} 
The input to a LVLM typically consists of an image $x^{\text{img}}$ and a discrete text sequence $x^{\text{txt}}$. The visual encoder $\mathcal{E}^{\text{img}}$ converts the input image $x^{\text{img}}$ into a sequence of visual tokens $\mathcal{V} = [v_i]_{i=1}^{N_v}$, while the text sequence $x^{\text{txt}}$ is mapped into a sequence of text tokens $\mathcal{T} = [t_i]_{i=1}^{N_t}$ using an embedding function $\mathcal{E}^{\text{txt}}$. The LLM $\mathcal{M_\text{LLM}}(\cdot|\theta)$ models the joint probability of the token sequence $\mathcal{U} = \{\mathcal{V},\mathcal{T}\}$, which is expressed as:
\begin{equation}
    P_\theta(R | \mathcal{U}) = \prod_{i=1}^{N_r} P_\theta(r_i | \{\mathcal{U}, r_{<i}\}),
\end{equation}
where $R = [r_i]_{i=1}^{N_r}$ is the text response sequence. The LVLM iteratively generates the next token $r_i$ based on $r_{<i}$. The optimization objective is to minimize the cross-entropy loss of the response $\mathcal{R}$.
% \begin{equation}
%     \mathcal{L}_{\text{VLM}} = \mathbb{E}_{R|\mathcal{U}}\left[-\log P_\theta(R | \mathcal{U})\right]
% \end{equation}
It is worth noting that most LVLMs adopt a design paradigm based on ViT, alignment adapters, and pre-trained LLMs\cite{liu2023llava,liu2024improved}, enabling quick adaptation to downstream tasks.


\noindent\textbf{VQGAN.}
VQGAN~\cite{esser2021taming} employs latent space compression and indexing mechanisms to effectively learn a complete discrete representation of images. VQGAN first maps the input image $x^{\text{img}}$ to a latent representation $z = \mathcal{E}(x)$ through a encoder $\mathcal{E}$. Then, the latent representation is quantized using a codebook $\mathcal{Z} = \{z_k\}_{k=1}^K$, generating a discrete index sequence $\mathcal{I} = [i_m]_{m=1}^N$, where $i_m \in \mathcal{Z}$ represents the quantized code index:
\begin{equation}
    \mathcal{I} = \text{Quantize}(z|\mathcal{Z}) = \arg\min_{z_k \in \mathcal{Z}} \| z - z_k \|_2.
\end{equation}
In our approach, the discrete index sequence $\mathcal{I}$ serves as a supervisory signal for the generation task, enabling the model to predict the index sequence $\hat{\mathcal{I}}$ from input conditions such as text or other modality signals.  
Finally, the predicted index sequence $\hat{\mathcal{I}}$ is upsampled by the VQGAN decoder $G$, generating the high-quality image $\hat{x}^\text{img} = G(\hat{\mathcal{I}})$.



\noindent\textbf{Low Rank Adaptation.} 
LoRA\cite{hu2021lora} effectively captures the characteristics of downstream tasks by introducing low-rank adapters. The core idea is to decompose the bypass weight matrix $\Delta W\in\mathbb{R}^{d^{\text{in}} \times d^{\text{out}}}$ into two low-rank matrices $ \{A \in \mathbb{R}^{d^{\text{in}} \times r}, B \in \mathbb{R}^{r \times d^{\text{out}}} \}$, where $ r \ll \min\{d^{\text{in}}, d^{\text{out}}\} $, significantly reducing learnable parameters. The output with the LoRA adapter for the input $x$ is then given by:
\begin{equation}
    h = x W_0 + \alpha x \Delta W/r = x W_0 + \alpha xAB/r,
\end{equation}
where matrix $ A $ is initialized with a Gaussian distribution, while the matrix $ B $ is initialized as a zero matrix. The scaling factor $ \alpha/r $ controls the impact of $ \Delta W $ on the model.

\section{HealthGPT}
\label{Method}


\subsection{Unified Autoregressive Generation.}  
% As shown in Figure~\ref{fig:architecture}, 
\ourmethod{} (Figure~\ref{fig:architecture}) utilizes a discrete token representation that covers both text and visual outputs, unifying visual comprehension and generation as an autoregressive task. 
For comprehension, $\mathcal{M}_\text{llm}$ receives the input joint sequence $\mathcal{U}$ and outputs a series of text token $\mathcal{R} = [r_1, r_2, \dots, r_{N_r}]$, where $r_i \in \mathcal{V}_{\text{txt}}$, and $\mathcal{V}_{\text{txt}}$ represents the LLM's vocabulary:
\begin{equation}
    P_\theta(\mathcal{R} \mid \mathcal{U}) = \prod_{i=1}^{N_r} P_\theta(r_i \mid \mathcal{U}, r_{<i}).
\end{equation}
For generation, $\mathcal{M}_\text{llm}$ first receives a special start token $\langle \text{START\_IMG} \rangle$, then generates a series of tokens corresponding to the VQGAN indices $\mathcal{I} = [i_1, i_2, \dots, i_{N_i}]$, where $i_j \in \mathcal{V}_{\text{vq}}$, and $\mathcal{V}_{\text{vq}}$ represents the index range of VQGAN. Upon completion of generation, the LLM outputs an end token $\langle \text{END\_IMG} \rangle$:
\begin{equation}
    P_\theta(\mathcal{I} \mid \mathcal{U}) = \prod_{j=1}^{N_i} P_\theta(i_j \mid \mathcal{U}, i_{<j}).
\end{equation}
Finally, the generated index sequence $\mathcal{I}$ is fed into the decoder $G$, which reconstructs the target image $\hat{x}^{\text{img}} = G(\mathcal{I})$.

\subsection{Hierarchical Visual Perception}  
Given the differences in visual perception between comprehension and generation tasks—where the former focuses on abstract semantics and the latter emphasizes complete semantics—we employ ViT to compress the image into discrete visual tokens at multiple hierarchical levels.
Specifically, the image is converted into a series of features $\{f_1, f_2, \dots, f_L\}$ as it passes through $L$ ViT blocks.

To address the needs of various tasks, the hidden states are divided into two types: (i) \textit{Concrete-grained features} $\mathcal{F}^{\text{Con}} = \{f_1, f_2, \dots, f_k\}, k < L$, derived from the shallower layers of ViT, containing sufficient global features, suitable for generation tasks; 
(ii) \textit{Abstract-grained features} $\mathcal{F}^{\text{Abs}} = \{f_{k+1}, f_{k+2}, \dots, f_L\}$, derived from the deeper layers of ViT, which contain abstract semantic information closer to the text space, suitable for comprehension tasks.

The task type $T$ (comprehension or generation) determines which set of features is selected as the input for the downstream large language model:
\begin{equation}
    \mathcal{F}^{\text{img}}_T =
    \begin{cases}
        \mathcal{F}^{\text{Con}}, & \text{if } T = \text{generation task} \\
        \mathcal{F}^{\text{Abs}}, & \text{if } T = \text{comprehension task}
    \end{cases}
\end{equation}
We integrate the image features $\mathcal{F}^{\text{img}}_T$ and text features $\mathcal{T}$ into a joint sequence through simple concatenation, which is then fed into the LLM $\mathcal{M}_{\text{llm}}$ for autoregressive generation.
% :
% \begin{equation}
%     \mathcal{R} = \mathcal{M}_{\text{llm}}(\mathcal{U}|\theta), \quad \mathcal{U} = [\mathcal{F}^{\text{img}}_T; \mathcal{T}]
% \end{equation}
\subsection{Heterogeneous Knowledge Adaptation}
We devise H-LoRA, which stores heterogeneous knowledge from comprehension and generation tasks in separate modules and dynamically routes to extract task-relevant knowledge from these modules. 
At the task level, for each task type $ T $, we dynamically assign a dedicated H-LoRA submodule $ \theta^T $, which is expressed as:
\begin{equation}
    \mathcal{R} = \mathcal{M}_\text{LLM}(\mathcal{U}|\theta, \theta^T), \quad \theta^T = \{A^T, B^T, \mathcal{R}^T_\text{outer}\}.
\end{equation}
At the feature level for a single task, H-LoRA integrates the idea of Mixture of Experts (MoE)~\cite{masoudnia2014mixture} and designs an efficient matrix merging and routing weight allocation mechanism, thus avoiding the significant computational delay introduced by matrix splitting in existing MoELoRA~\cite{luo2024moelora}. Specifically, we first merge the low-rank matrices (rank = r) of $ k $ LoRA experts into a unified matrix:
\begin{equation}
    \mathbf{A}^{\text{merged}}, \mathbf{B}^{\text{merged}} = \text{Concat}(\{A_i\}_1^k), \text{Concat}(\{B_i\}_1^k),
\end{equation}
where $ \mathbf{A}^{\text{merged}} \in \mathbb{R}^{d^\text{in} \times rk} $ and $ \mathbf{B}^{\text{merged}} \in \mathbb{R}^{rk \times d^\text{out}} $. The $k$-dimension routing layer generates expert weights $ \mathcal{W} \in \mathbb{R}^{\text{token\_num} \times k} $ based on the input hidden state $ x $, and these are expanded to $ \mathbb{R}^{\text{token\_num} \times rk} $ as follows:
\begin{equation}
    \mathcal{W}^\text{expanded} = \alpha k \mathcal{W} / r \otimes \mathbf{1}_r,
\end{equation}
where $ \otimes $ denotes the replication operation.
The overall output of H-LoRA is computed as:
\begin{equation}
    \mathcal{O}^\text{H-LoRA} = (x \mathbf{A}^{\text{merged}} \odot \mathcal{W}^\text{expanded}) \mathbf{B}^{\text{merged}},
\end{equation}
where $ \odot $ represents element-wise multiplication. Finally, the output of H-LoRA is added to the frozen pre-trained weights to produce the final output:
\begin{equation}
    \mathcal{O} = x W_0 + \mathcal{O}^\text{H-LoRA}.
\end{equation}
% In summary, H-LoRA is a task-based dynamic PEFT method that achieves high efficiency in single-task fine-tuning.

\subsection{Training Pipeline}

\begin{figure}[t]
    \centering
    \hspace{-4mm}
    \includegraphics[width=0.94\linewidth]{fig/data.pdf}
    \caption{Data statistics of \texttt{VL-Health}. }
    \label{fig:data}
\end{figure}
\noindent \textbf{1st Stage: Multi-modal Alignment.} 
In the first stage, we design separate visual adapters and H-LoRA submodules for medical unified tasks. For the medical comprehension task, we train abstract-grained visual adapters using high-quality image-text pairs to align visual embeddings with textual embeddings, thereby enabling the model to accurately describe medical visual content. During this process, the pre-trained LLM and its corresponding H-LoRA submodules remain frozen. In contrast, the medical generation task requires training concrete-grained adapters and H-LoRA submodules while keeping the LLM frozen. Meanwhile, we extend the textual vocabulary to include multimodal tokens, enabling the support of additional VQGAN vector quantization indices. The model trains on image-VQ pairs, endowing the pre-trained LLM with the capability for image reconstruction. This design ensures pixel-level consistency of pre- and post-LVLM. The processes establish the initial alignment between the LLM’s outputs and the visual inputs.

\noindent \textbf{2nd Stage: Heterogeneous H-LoRA Plugin Adaptation.}  
The submodules of H-LoRA share the word embedding layer and output head but may encounter issues such as bias and scale inconsistencies during training across different tasks. To ensure that the multiple H-LoRA plugins seamlessly interface with the LLMs and form a unified base, we fine-tune the word embedding layer and output head using a small amount of mixed data to maintain consistency in the model weights. Specifically, during this stage, all H-LoRA submodules for different tasks are kept frozen, with only the word embedding layer and output head being optimized. Through this stage, the model accumulates foundational knowledge for unified tasks by adapting H-LoRA plugins.
\input{tab/visual_comprehension_part1}
\input{tab/modality_transfer}

\noindent \textbf{3rd Stage: Visual Instruction Fine-Tuning.}  
In the third stage, we introduce additional task-specific data to further optimize the model and enhance its adaptability to downstream tasks such as medical visual comprehension (e.g., medical QA, medical dialogues, and report generation) or generation tasks (e.g., super-resolution, denoising, and modality conversion). Notably, by this stage, the word embedding layer and output head have been fine-tuned, only the H-LoRA modules and adapter modules need to be trained. This strategy significantly improves the model's adaptability and flexibility across different tasks.


\section{Experiment}
\label{s:experiment}

\subsection{Data Description}
We evaluate our method on FI~\cite{you2016building}, Twitter\_LDL~\cite{yang2017learning} and Artphoto~\cite{machajdik2010affective}.
FI is a public dataset built from Flickr and Instagram, with 23,308 images and eight emotion categories, namely \textit{amusement}, \textit{anger}, \textit{awe},  \textit{contentment}, \textit{disgust}, \textit{excitement},  \textit{fear}, and \textit{sadness}. 
% Since images in FI are all copyrighted by law, some images are corrupted now, so we remove these samples and retain 21,828 images.
% T4SA contains images from Twitter, which are classified into three categories: \textit{positive}, \textit{neutral}, and \textit{negative}. In this paper, we adopt the base version of B-T4SA, which contains 470,586 images and provides text descriptions of the corresponding tweets.
Twitter\_LDL contains 10,045 images from Twitter, with the same eight categories as the FI dataset.
% 。
For these two datasets, they are randomly split into 80\%
training and 20\% testing set.
Artphoto contains 806 artistic photos from the DeviantArt website, which we use to further evaluate the zero-shot capability of our model.
% on the small-scale dataset.
% We construct and publicly release the first image sentiment analysis dataset containing metadata.
% 。

% Based on these datasets, we are the first to construct and publicly release metadata-enhanced image sentiment analysis datasets. These datasets include scenes, tags, descriptions, and corresponding confidence scores, and are available at this link for future research purposes.


% 
\begin{table}[t]
\centering
% \begin{center}
\caption{Overall performance of different models on FI and Twitter\_LDL datasets.}
\label{tab:cap1}
% \resizebox{\linewidth}{!}
{
\begin{tabular}{l|c|c|c|c}
\hline
\multirow{2}{*}{\textbf{Model}} & \multicolumn{2}{c|}{\textbf{FI}}  & \multicolumn{2}{c}{\textbf{Twitter\_LDL}} \\ \cline{2-5} 
  & \textbf{Accuracy} & \textbf{F1} & \textbf{Accuracy} & \textbf{F1}  \\ \hline
% (\rownumber)~AlexNet~\cite{krizhevsky2017imagenet}  & 58.13\% & 56.35\%  & 56.24\%& 55.02\%  \\ 
% (\rownumber)~VGG16~\cite{simonyan2014very}  & 63.75\%& 63.08\%  & 59.34\%& 59.02\%  \\ 
(\rownumber)~ResNet101~\cite{he2016deep} & 66.16\%& 65.56\%  & 62.02\% & 61.34\%  \\ 
(\rownumber)~CDA~\cite{han2023boosting} & 66.71\%& 65.37\%  & 64.14\% & 62.85\%  \\ 
(\rownumber)~CECCN~\cite{ruan2024color} & 67.96\%& 66.74\%  & 64.59\%& 64.72\% \\ 
(\rownumber)~EmoVIT~\cite{xie2024emovit} & 68.09\%& 67.45\%  & 63.12\% & 61.97\%  \\ 
(\rownumber)~ComLDL~\cite{zhang2022compound} & 68.83\%& 67.28\%  & 65.29\% & 63.12\%  \\ 
(\rownumber)~WSDEN~\cite{li2023weakly} & 69.78\%& 69.61\%  & 67.04\% & 65.49\% \\ 
(\rownumber)~ECWA~\cite{deng2021emotion} & 70.87\%& 69.08\%  & 67.81\% & 66.87\%  \\ 
(\rownumber)~EECon~\cite{yang2023exploiting} & 71.13\%& 68.34\%  & 64.27\%& 63.16\%  \\ 
(\rownumber)~MAM~\cite{zhang2024affective} & 71.44\%  & 70.83\% & 67.18\%  & 65.01\%\\ 
(\rownumber)~TGCA-PVT~\cite{chen2024tgca}   & 73.05\%  & 71.46\% & 69.87\%  & 68.32\% \\ 
(\rownumber)~OEAN~\cite{zhang2024object}   & 73.40\%  & 72.63\% & 70.52\%  & 69.47\% \\ \hline
(\rownumber)~\shortname  & \textbf{79.48\%} & \textbf{79.22\%} & \textbf{74.12\%} & \textbf{73.09\%} \\ \hline
\end{tabular}
}
\vspace{-6mm}
% \end{center}
\end{table}
% 

\subsection{Experiment Setting}
% \subsubsection{Model Setting.}
% 
\textbf{Model Setting:}
For feature representation, we set $k=10$ to select object tags, and adopt clip-vit-base-patch32 as the pre-trained model for unified feature representation.
Moreover, we empirically set $(d_e, d_h, d_k, d_s) = (512, 128, 16, 64)$, and set the classification class $L$ to 8.

% 

\textbf{Training Setting:}
To initialize the model, we set all weights such as $\boldsymbol{W}$ following the truncated normal distribution, and use AdamW optimizer with the learning rate of $1 \times 10^{-4}$.
% warmup scheduler of cosine, warmup steps of 2000.
Furthermore, we set the batch size to 32 and the epoch of the training process to 200.
During the implementation, we utilize \textit{PyTorch} to build our entire model.
% , and our project codes are publicly available at https://github.com/zzmyrep/MESN.
% Our project codes as well as data are all publicly available on GitHub\footnote{https://github.com/zzmyrep/KBCEN}.
% Code is available at \href{https://github.com/zzmyrep/KBCEN}{https://github.com/zzmyrep/KBCEN}.

\textbf{Evaluation Metrics:}
Following~\cite{zhang2024affective, chen2024tgca, zhang2024object}, we adopt \textit{accuracy} and \textit{F1} as our evaluation metrics to measure the performance of different methods for image sentiment analysis. 



\subsection{Experiment Result}
% We compare our model against the following baselines: AlexNet~\cite{krizhevsky2017imagenet}, VGG16~\cite{simonyan2014very}, ResNet101~\cite{he2016deep}, CECCN~\cite{ruan2024color}, EmoVIT~\cite{xie2024emovit}, WSCNet~\cite{yang2018weakly}, ECWA~\cite{deng2021emotion}, EECon~\cite{yang2023exploiting}, MAM~\cite{zhang2024affective} and TGCA-PVT~\cite{chen2024tgca}, and the overall results are summarized in Table~\ref{tab:cap1}.
We compare our model against several baselines, and the overall results are summarized in Table~\ref{tab:cap1}.
We observe that our model achieves the best performance in both accuracy and F1 metrics, significantly outperforming the previous models. 
This superior performance is mainly attributed to our effective utilization of metadata to enhance image sentiment analysis, as well as the exceptional capability of the unified sentiment transformer framework we developed. These results strongly demonstrate that our proposed method can bring encouraging performance for image sentiment analysis.

\setcounter{magicrownumbers}{0} 
\begin{table}[t]
\begin{center}
\caption{Ablation study of~\shortname~on FI dataset.} 
% \vspace{1mm}
\label{tab:cap2}
\resizebox{.9\linewidth}{!}
{
\begin{tabular}{lcc}
  \hline
  \textbf{Model} & \textbf{Accuracy} & \textbf{F1} \\
  \hline
  (\rownumber)~Ours (w/o vision) & 65.72\% & 64.54\% \\
  (\rownumber)~Ours (w/o text description) & 74.05\% & 72.58\% \\
  (\rownumber)~Ours (w/o object tag) & 77.45\% & 76.84\% \\
  (\rownumber)~Ours (w/o scene tag) & 78.47\% & 78.21\% \\
  \hline
  (\rownumber)~Ours (w/o unified embedding) & 76.41\% & 76.23\% \\
  (\rownumber)~Ours (w/o adaptive learning) & 76.83\% & 76.56\% \\
  (\rownumber)~Ours (w/o cross-modal fusion) & 76.85\% & 76.49\% \\
  \hline
  (\rownumber)~Ours  & \textbf{79.48\%} & \textbf{79.22\%} \\
  \hline
\end{tabular}
}
\end{center}
\vspace{-5mm}
\end{table}


\begin{figure}[t]
\centering
% \vspace{-2mm}
\includegraphics[width=0.42\textwidth]{fig/2dvisual-linux4-paper2.pdf}
\caption{Visualization of feature distribution on eight categories before (left) and after (right) model processing.}
% 
\label{fig:visualization}
\vspace{-5mm}
\end{figure}

\subsection{Ablation Performance}
In this subsection, we conduct an ablation study to examine which component is really important for performance improvement. The results are reported in Table~\ref{tab:cap2}.

For information utilization, we observe a significant decline in model performance when visual features are removed. Additionally, the performance of \shortname~decreases when different metadata are removed separately, which means that text description, object tag, and scene tag are all critical for image sentiment analysis.
Recalling the model architecture, we separately remove transformer layers of the unified representation module, the adaptive learning module, and the cross-modal fusion module, replacing them with MLPs of the same parameter scale.
In this way, we can observe varying degrees of decline in model performance, indicating that these modules are indispensable for our model to achieve better performance.

\subsection{Visualization}
% 


% % 开始使用minipage进行左右排列
% \begin{minipage}[t]{0.45\textwidth}  % 子图1宽度为45%
%     \centering
%     \includegraphics[width=\textwidth]{2dvisual.pdf}  % 插入图片
%     \captionof{figure}{Visualization of feature distribution.}  % 使用captionof添加图片标题
%     \label{fig:visualization}
% \end{minipage}


% \begin{figure}[t]
% \centering
% \vspace{-2mm}
% \includegraphics[width=0.45\textwidth]{fig/2dvisual.pdf}
% \caption{Visualization of feature distribution.}
% \label{fig:visualization}
% % \vspace{-4mm}
% \end{figure}

% \begin{figure}[t]
% \centering
% \vspace{-2mm}
% \includegraphics[width=0.45\textwidth]{fig/2dvisual-linux3-paper.pdf}
% \caption{Visualization of feature distribution.}
% \label{fig:visualization}
% % \vspace{-4mm}
% \end{figure}



\begin{figure}[tbp]   
\vspace{-4mm}
  \centering            
  \subfloat[Depth of adaptive learning layers]   
  {
    \label{fig:subfig1}\includegraphics[width=0.22\textwidth]{fig/fig_sensitivity-a5}
  }
  \subfloat[Depth of fusion layers]
  {
    % \label{fig:subfig2}\includegraphics[width=0.22\textwidth]{fig/fig_sensitivity-b2}
    \label{fig:subfig2}\includegraphics[width=0.22\textwidth]{fig/fig_sensitivity-b2-num.pdf}
  }
  \caption{Sensitivity study of \shortname~on different depth. }   
  \label{fig:fig_sensitivity}  
\vspace{-2mm}
\end{figure}

% \begin{figure}[htbp]
% \centerline{\includegraphics{2dvisual.pdf}}
% \caption{Visualization of feature distribution.}
% \label{fig:visualization}
% \end{figure}

% In Fig.~\ref{fig:visualization}, we use t-SNE~\cite{van2008visualizing} to reduce the dimension of data features for visualization, Figure in left represents the metadata features before model processing, the features are obtained by embedding through the CLIP model, and figure in right shows the features of the data after model processing, it can be observed that after the model processing, the data with different label categories fall in different regions in the space, therefore, we can conclude that the Therefore, we can conclude that the model can effectively utilize the information contained in the metadata and use it to guide the model for classification.

In Fig.~\ref{fig:visualization}, we use t-SNE~\cite{van2008visualizing} to reduce the dimension of data features for visualization.
The left figure shows metadata features before being processed by our model (\textit{i.e.}, embedded by CLIP), while the right shows the distribution of features after being processed by our model.
We can observe that after the model processing, data with the same label are closer to each other, while others are farther away.
Therefore, it shows that the model can effectively utilize the information contained in the metadata and use it to guide the classification process.

\subsection{Sensitivity Analysis}
% 
In this subsection, we conduct a sensitivity analysis to figure out the effect of different depth settings of adaptive learning layers and fusion layers. 
% In this subsection, we conduct a sensitivity analysis to figure out the effect of different depth settings on the model. 
% Fig.~\ref{fig:fig_sensitivity} presents the effect of different depth settings of adaptive learning layers and fusion layers. 
Taking Fig.~\ref{fig:fig_sensitivity} (a) as an example, the model performance improves with increasing depth, reaching the best performance at a depth of 4.
% Taking Fig.~\ref{fig:fig_sensitivity} (a) as an example, the performance of \shortname~improves with the increase of depth at first, reaching the best performance at a depth of 4.
When the depth continues to increase, the accuracy decreases to varying degrees.
Similar results can be observed in Fig.~\ref{fig:fig_sensitivity} (b).
Therefore, we set their depths to 4 and 6 respectively to achieve the best results.

% Through our experiments, we can observe that the effect of modifying these hyperparameters on the results of the experiments is very weak, and the surface model is not sensitive to the hyperparameters.


\subsection{Zero-shot Capability}
% 

% (1)~GCH~\cite{2010Analyzing} & 21.78\% & (5)~RA-DLNet~\cite{2020A} & 34.01\% \\ \hline
% (2)~WSCNet~\cite{2019WSCNet}  & 30.25\% & (6)~CECCN~\cite{ruan2024color} & 43.83\% \\ \hline
% (3)~PCNN~\cite{2015Robust} & 31.68\%  & (7)~EmoVIT~\cite{xie2024emovit} & 44.90\% \\ \hline
% (4)~AR~\cite{2018Visual} & 32.67\% & (8)~Ours (Zero-shot) & 47.83\% \\ \hline


\begin{table}[t]
\centering
\caption{Zero-shot capability of \shortname.}
\label{tab:cap3}
\resizebox{1\linewidth}{!}
{
\begin{tabular}{lc|lc}
\hline
\textbf{Model} & \textbf{Accuracy} & \textbf{Model} & \textbf{Accuracy} \\ \hline
(1)~WSCNet~\cite{2019WSCNet}  & 30.25\% & (5)~MAM~\cite{zhang2024affective} & 39.56\%  \\ \hline
(2)~AR~\cite{2018Visual} & 32.67\% & (6)~CECCN~\cite{ruan2024color} & 43.83\% \\ \hline
(3)~RA-DLNet~\cite{2020A} & 34.01\%  & (7)~EmoVIT~\cite{xie2024emovit} & 44.90\% \\ \hline
(4)~CDA~\cite{han2023boosting} & 38.64\% & (8)~Ours (Zero-shot) & 47.83\% \\ \hline
\end{tabular}
}
\vspace{-5mm}
\end{table}

% We use the model trained on the FI dataset to test on the artphoto dataset to verify the model's generalization ability as well as robustness to other distributed datasets.
% We can observe that the MESN model shows strong competitiveness in terms of accuracy when compared to other trained models, which suggests that the model has a good generalization ability in the OOD task.

To validate the model's generalization ability and robustness to other distributed datasets, we directly test the model trained on the FI dataset, without training on Artphoto. 
% As observed in Table 3, compared to other models trained on Artphoto, we achieve highly competitive zero-shot performance, indicating that the model has good generalization ability in out-of-distribution tasks.
From Table~\ref{tab:cap3}, we can observe that compared with other models trained on Artphoto, we achieve competitive zero-shot performance, which shows that the model has good generalization ability in out-of-distribution tasks.


%%%%%%%%%%%%
%  E2E     %
%%%%%%%%%%%%


\section{Conclusion}
In this paper, we introduced Wi-Chat, the first LLM-powered Wi-Fi-based human activity recognition system that integrates the reasoning capabilities of large language models with the sensing potential of wireless signals. Our experimental results on a self-collected Wi-Fi CSI dataset demonstrate the promising potential of LLMs in enabling zero-shot Wi-Fi sensing. These findings suggest a new paradigm for human activity recognition that does not rely on extensive labeled data. We hope future research will build upon this direction, further exploring the applications of LLMs in signal processing domains such as IoT, mobile sensing, and radar-based systems.

\section*{Limitations}
While our work represents the first attempt to leverage LLMs for processing Wi-Fi signals, it is a preliminary study focused on a relatively simple task: Wi-Fi-based human activity recognition. This choice allows us to explore the feasibility of LLMs in wireless sensing but also comes with certain limitations.

Our approach primarily evaluates zero-shot performance, which, while promising, may still lag behind traditional supervised learning methods in highly complex or fine-grained recognition tasks. Besides, our study is limited to a controlled environment with a self-collected dataset, and the generalizability of LLMs to diverse real-world scenarios with varying Wi-Fi conditions, environmental interference, and device heterogeneity remains an open question.

Additionally, we have yet to explore the full potential of LLMs in more advanced Wi-Fi sensing applications, such as fine-grained gesture recognition, occupancy detection, and passive health monitoring. Future work should investigate the scalability of LLM-based approaches, their robustness to domain shifts, and their integration with multimodal sensing techniques in broader IoT applications.


% Bibliography entries for the entire Anthology, followed by custom entries
%\bibliography{anthology,custom}
% Custom bibliography entries only
\bibliography{main}
\newpage
\appendix

\section{Experiment prompts}
\label{sec:prompt}
The prompts used in the LLM experiments are shown in the following Table~\ref{tab:prompts}.

\definecolor{titlecolor}{rgb}{0.9, 0.5, 0.1}
\definecolor{anscolor}{rgb}{0.2, 0.5, 0.8}
\definecolor{labelcolor}{HTML}{48a07e}
\begin{table*}[h]
	\centering
	
 % \vspace{-0.2cm}
	
	\begin{center}
		\begin{tikzpicture}[
				chatbox_inner/.style={rectangle, rounded corners, opacity=0, text opacity=1, font=\sffamily\scriptsize, text width=5in, text height=9pt, inner xsep=6pt, inner ysep=6pt},
				chatbox_prompt_inner/.style={chatbox_inner, align=flush left, xshift=0pt, text height=11pt},
				chatbox_user_inner/.style={chatbox_inner, align=flush left, xshift=0pt},
				chatbox_gpt_inner/.style={chatbox_inner, align=flush left, xshift=0pt},
				chatbox/.style={chatbox_inner, draw=black!25, fill=gray!7, opacity=1, text opacity=0},
				chatbox_prompt/.style={chatbox, align=flush left, fill=gray!1.5, draw=black!30, text height=10pt},
				chatbox_user/.style={chatbox, align=flush left},
				chatbox_gpt/.style={chatbox, align=flush left},
				chatbox2/.style={chatbox_gpt, fill=green!25},
				chatbox3/.style={chatbox_gpt, fill=red!20, draw=black!20},
				chatbox4/.style={chatbox_gpt, fill=yellow!30},
				labelbox/.style={rectangle, rounded corners, draw=black!50, font=\sffamily\scriptsize\bfseries, fill=gray!5, inner sep=3pt},
			]
											
			\node[chatbox_user] (q1) {
				\textbf{System prompt}
				\newline
				\newline
				You are a helpful and precise assistant for segmenting and labeling sentences. We would like to request your help on curating a dataset for entity-level hallucination detection.
				\newline \newline
                We will give you a machine generated biography and a list of checked facts about the biography. Each fact consists of a sentence and a label (True/False). Please do the following process. First, breaking down the biography into words. Second, by referring to the provided list of facts, merging some broken down words in the previous step to form meaningful entities. For example, ``strategic thinking'' should be one entity instead of two. Third, according to the labels in the list of facts, labeling each entity as True or False. Specifically, for facts that share a similar sentence structure (\eg, \textit{``He was born on Mach 9, 1941.''} (\texttt{True}) and \textit{``He was born in Ramos Mejia.''} (\texttt{False})), please first assign labels to entities that differ across atomic facts. For example, first labeling ``Mach 9, 1941'' (\texttt{True}) and ``Ramos Mejia'' (\texttt{False}) in the above case. For those entities that are the same across atomic facts (\eg, ``was born'') or are neutral (\eg, ``he,'' ``in,'' and ``on''), please label them as \texttt{True}. For the cases that there is no atomic fact that shares the same sentence structure, please identify the most informative entities in the sentence and label them with the same label as the atomic fact while treating the rest of the entities as \texttt{True}. In the end, output the entities and labels in the following format:
                \begin{itemize}[nosep]
                    \item Entity 1 (Label 1)
                    \item Entity 2 (Label 2)
                    \item ...
                    \item Entity N (Label N)
                \end{itemize}
                % \newline \newline
                Here are two examples:
                \newline\newline
                \textbf{[Example 1]}
                \newline
                [The start of the biography]
                \newline
                \textcolor{titlecolor}{Marianne McAndrew is an American actress and singer, born on November 21, 1942, in Cleveland, Ohio. She began her acting career in the late 1960s, appearing in various television shows and films.}
                \newline
                [The end of the biography]
                \newline \newline
                [The start of the list of checked facts]
                \newline
                \textcolor{anscolor}{[Marianne McAndrew is an American. (False); Marianne McAndrew is an actress. (True); Marianne McAndrew is a singer. (False); Marianne McAndrew was born on November 21, 1942. (False); Marianne McAndrew was born in Cleveland, Ohio. (False); She began her acting career in the late 1960s. (True); She has appeared in various television shows. (True); She has appeared in various films. (True)]}
                \newline
                [The end of the list of checked facts]
                \newline \newline
                [The start of the ideal output]
                \newline
                \textcolor{labelcolor}{[Marianne McAndrew (True); is (True); an (True); American (False); actress (True); and (True); singer (False); , (True); born (True); on (True); November 21, 1942 (False); , (True); in (True); Cleveland, Ohio (False); . (True); She (True); began (True); her (True); acting career (True); in (True); the late 1960s (True); , (True); appearing (True); in (True); various (True); television shows (True); and (True); films (True); . (True)]}
                \newline
                [The end of the ideal output]
				\newline \newline
                \textbf{[Example 2]}
                \newline
                [The start of the biography]
                \newline
                \textcolor{titlecolor}{Doug Sheehan is an American actor who was born on April 27, 1949, in Santa Monica, California. He is best known for his roles in soap operas, including his portrayal of Joe Kelly on ``General Hospital'' and Ben Gibson on ``Knots Landing.''}
                \newline
                [The end of the biography]
                \newline \newline
                [The start of the list of checked facts]
                \newline
                \textcolor{anscolor}{[Doug Sheehan is an American. (True); Doug Sheehan is an actor. (True); Doug Sheehan was born on April 27, 1949. (True); Doug Sheehan was born in Santa Monica, California. (False); He is best known for his roles in soap operas. (True); He portrayed Joe Kelly. (True); Joe Kelly was in General Hospital. (True); General Hospital is a soap opera. (True); He portrayed Ben Gibson. (True); Ben Gibson was in Knots Landing. (True); Knots Landing is a soap opera. (True)]}
                \newline
                [The end of the list of checked facts]
                \newline \newline
                [The start of the ideal output]
                \newline
                \textcolor{labelcolor}{[Doug Sheehan (True); is (True); an (True); American (True); actor (True); who (True); was born (True); on (True); April 27, 1949 (True); in (True); Santa Monica, California (False); . (True); He (True); is (True); best known (True); for (True); his roles in soap operas (True); , (True); including (True); in (True); his portrayal (True); of (True); Joe Kelly (True); on (True); ``General Hospital'' (True); and (True); Ben Gibson (True); on (True); ``Knots Landing.'' (True)]}
                \newline
                [The end of the ideal output]
				\newline \newline
				\textbf{User prompt}
				\newline
				\newline
				[The start of the biography]
				\newline
				\textcolor{magenta}{\texttt{\{BIOGRAPHY\}}}
				\newline
				[The ebd of the biography]
				\newline \newline
				[The start of the list of checked facts]
				\newline
				\textcolor{magenta}{\texttt{\{LIST OF CHECKED FACTS\}}}
				\newline
				[The end of the list of checked facts]
			};
			\node[chatbox_user_inner] (q1_text) at (q1) {
				\textbf{System prompt}
				\newline
				\newline
				You are a helpful and precise assistant for segmenting and labeling sentences. We would like to request your help on curating a dataset for entity-level hallucination detection.
				\newline \newline
                We will give you a machine generated biography and a list of checked facts about the biography. Each fact consists of a sentence and a label (True/False). Please do the following process. First, breaking down the biography into words. Second, by referring to the provided list of facts, merging some broken down words in the previous step to form meaningful entities. For example, ``strategic thinking'' should be one entity instead of two. Third, according to the labels in the list of facts, labeling each entity as True or False. Specifically, for facts that share a similar sentence structure (\eg, \textit{``He was born on Mach 9, 1941.''} (\texttt{True}) and \textit{``He was born in Ramos Mejia.''} (\texttt{False})), please first assign labels to entities that differ across atomic facts. For example, first labeling ``Mach 9, 1941'' (\texttt{True}) and ``Ramos Mejia'' (\texttt{False}) in the above case. For those entities that are the same across atomic facts (\eg, ``was born'') or are neutral (\eg, ``he,'' ``in,'' and ``on''), please label them as \texttt{True}. For the cases that there is no atomic fact that shares the same sentence structure, please identify the most informative entities in the sentence and label them with the same label as the atomic fact while treating the rest of the entities as \texttt{True}. In the end, output the entities and labels in the following format:
                \begin{itemize}[nosep]
                    \item Entity 1 (Label 1)
                    \item Entity 2 (Label 2)
                    \item ...
                    \item Entity N (Label N)
                \end{itemize}
                % \newline \newline
                Here are two examples:
                \newline\newline
                \textbf{[Example 1]}
                \newline
                [The start of the biography]
                \newline
                \textcolor{titlecolor}{Marianne McAndrew is an American actress and singer, born on November 21, 1942, in Cleveland, Ohio. She began her acting career in the late 1960s, appearing in various television shows and films.}
                \newline
                [The end of the biography]
                \newline \newline
                [The start of the list of checked facts]
                \newline
                \textcolor{anscolor}{[Marianne McAndrew is an American. (False); Marianne McAndrew is an actress. (True); Marianne McAndrew is a singer. (False); Marianne McAndrew was born on November 21, 1942. (False); Marianne McAndrew was born in Cleveland, Ohio. (False); She began her acting career in the late 1960s. (True); She has appeared in various television shows. (True); She has appeared in various films. (True)]}
                \newline
                [The end of the list of checked facts]
                \newline \newline
                [The start of the ideal output]
                \newline
                \textcolor{labelcolor}{[Marianne McAndrew (True); is (True); an (True); American (False); actress (True); and (True); singer (False); , (True); born (True); on (True); November 21, 1942 (False); , (True); in (True); Cleveland, Ohio (False); . (True); She (True); began (True); her (True); acting career (True); in (True); the late 1960s (True); , (True); appearing (True); in (True); various (True); television shows (True); and (True); films (True); . (True)]}
                \newline
                [The end of the ideal output]
				\newline \newline
                \textbf{[Example 2]}
                \newline
                [The start of the biography]
                \newline
                \textcolor{titlecolor}{Doug Sheehan is an American actor who was born on April 27, 1949, in Santa Monica, California. He is best known for his roles in soap operas, including his portrayal of Joe Kelly on ``General Hospital'' and Ben Gibson on ``Knots Landing.''}
                \newline
                [The end of the biography]
                \newline \newline
                [The start of the list of checked facts]
                \newline
                \textcolor{anscolor}{[Doug Sheehan is an American. (True); Doug Sheehan is an actor. (True); Doug Sheehan was born on April 27, 1949. (True); Doug Sheehan was born in Santa Monica, California. (False); He is best known for his roles in soap operas. (True); He portrayed Joe Kelly. (True); Joe Kelly was in General Hospital. (True); General Hospital is a soap opera. (True); He portrayed Ben Gibson. (True); Ben Gibson was in Knots Landing. (True); Knots Landing is a soap opera. (True)]}
                \newline
                [The end of the list of checked facts]
                \newline \newline
                [The start of the ideal output]
                \newline
                \textcolor{labelcolor}{[Doug Sheehan (True); is (True); an (True); American (True); actor (True); who (True); was born (True); on (True); April 27, 1949 (True); in (True); Santa Monica, California (False); . (True); He (True); is (True); best known (True); for (True); his roles in soap operas (True); , (True); including (True); in (True); his portrayal (True); of (True); Joe Kelly (True); on (True); ``General Hospital'' (True); and (True); Ben Gibson (True); on (True); ``Knots Landing.'' (True)]}
                \newline
                [The end of the ideal output]
				\newline \newline
				\textbf{User prompt}
				\newline
				\newline
				[The start of the biography]
				\newline
				\textcolor{magenta}{\texttt{\{BIOGRAPHY\}}}
				\newline
				[The ebd of the biography]
				\newline \newline
				[The start of the list of checked facts]
				\newline
				\textcolor{magenta}{\texttt{\{LIST OF CHECKED FACTS\}}}
				\newline
				[The end of the list of checked facts]
			};
		\end{tikzpicture}
        \caption{GPT-4o prompt for labeling hallucinated entities.}\label{tb:gpt-4-prompt}
	\end{center}
\vspace{-0cm}
\end{table*}
% \section{Full Experiment Results}
% \begin{table*}[th]
    \centering
    \small
    \caption{Classification Results}
    \begin{tabular}{lcccc}
        \toprule
        \textbf{Method} & \textbf{Accuracy} & \textbf{Precision} & \textbf{Recall} & \textbf{F1-score} \\
        \midrule
        \multicolumn{5}{c}{\textbf{Zero Shot}} \\
                Zero-shot E-eyes & 0.26 & 0.26 & 0.27 & 0.26 \\
        Zero-shot CARM & 0.24 & 0.24 & 0.24 & 0.24 \\
                Zero-shot SVM & 0.27 & 0.28 & 0.28 & 0.27 \\
        Zero-shot CNN & 0.23 & 0.24 & 0.23 & 0.23 \\
        Zero-shot RNN & 0.26 & 0.26 & 0.26 & 0.26 \\
DeepSeek-0shot & 0.54 & 0.61 & 0.54 & 0.52 \\
DeepSeek-0shot-COT & 0.33 & 0.24 & 0.33 & 0.23 \\
DeepSeek-0shot-Knowledge & 0.45 & 0.46 & 0.45 & 0.44 \\
Gemma2-0shot & 0.35 & 0.22 & 0.38 & 0.27 \\
Gemma2-0shot-COT & 0.36 & 0.22 & 0.36 & 0.27 \\
Gemma2-0shot-Knowledge & 0.32 & 0.18 & 0.34 & 0.20 \\
GPT-4o-mini-0shot & 0.48 & 0.53 & 0.48 & 0.41 \\
GPT-4o-mini-0shot-COT & 0.33 & 0.50 & 0.33 & 0.38 \\
GPT-4o-mini-0shot-Knowledge & 0.49 & 0.31 & 0.49 & 0.36 \\
GPT-4o-0shot & 0.62 & 0.62 & 0.47 & 0.42 \\
GPT-4o-0shot-COT & 0.29 & 0.45 & 0.29 & 0.21 \\
GPT-4o-0shot-Knowledge & 0.44 & 0.52 & 0.44 & 0.39 \\
LLaMA-0shot & 0.32 & 0.25 & 0.32 & 0.24 \\
LLaMA-0shot-COT & 0.12 & 0.25 & 0.12 & 0.09 \\
LLaMA-0shot-Knowledge & 0.32 & 0.25 & 0.32 & 0.28 \\
Mistral-0shot & 0.19 & 0.23 & 0.19 & 0.10 \\
Mistral-0shot-Knowledge & 0.21 & 0.40 & 0.21 & 0.11 \\
        \midrule
        \multicolumn{5}{c}{\textbf{4 Shot}} \\
GPT-4o-mini-4shot & 0.58 & 0.59 & 0.58 & 0.53 \\
GPT-4o-mini-4shot-COT & 0.57 & 0.53 & 0.57 & 0.50 \\
GPT-4o-mini-4shot-Knowledge & 0.56 & 0.51 & 0.56 & 0.47 \\
GPT-4o-4shot & 0.77 & 0.84 & 0.77 & 0.73 \\
GPT-4o-4shot-COT & 0.63 & 0.76 & 0.63 & 0.53 \\
GPT-4o-4shot-Knowledge & 0.72 & 0.82 & 0.71 & 0.66 \\
LLaMA-4shot & 0.29 & 0.24 & 0.29 & 0.21 \\
LLaMA-4shot-COT & 0.20 & 0.30 & 0.20 & 0.13 \\
LLaMA-4shot-Knowledge & 0.15 & 0.23 & 0.13 & 0.13 \\
Mistral-4shot & 0.02 & 0.02 & 0.02 & 0.02 \\
Mistral-4shot-Knowledge & 0.21 & 0.27 & 0.21 & 0.20 \\
        \midrule
        
        \multicolumn{5}{c}{\textbf{Suprevised}} \\
        SVM & 0.94 & 0.92 & 0.91 & 0.91 \\
        CNN & 0.98 & 0.98 & 0.97 & 0.97 \\
        RNN & 0.99 & 0.99 & 0.99 & 0.99 \\
        % \midrule
        % \multicolumn{5}{c}{\textbf{Conventional Wi-Fi-based Human Activity Recognition Systems}} \\
        E-eyes & 1.00 & 1.00 & 1.00 & 1.00 \\
        CARM & 0.98 & 0.98 & 0.98 & 0.98 \\
\midrule
 \multicolumn{5}{c}{\textbf{Vision Models}} \\
           Zero-shot SVM & 0.26 & 0.25 & 0.25 & 0.25 \\
        Zero-shot CNN & 0.26 & 0.25 & 0.26 & 0.26 \\
        Zero-shot RNN & 0.28 & 0.28 & 0.29 & 0.28 \\
        SVM & 0.99 & 0.99 & 0.99 & 0.99 \\
        CNN & 0.98 & 0.99 & 0.98 & 0.98 \\
        RNN & 0.98 & 0.99 & 0.98 & 0.98 \\
GPT-4o-mini-Vision & 0.84 & 0.85 & 0.84 & 0.84 \\
GPT-4o-mini-Vision-COT & 0.90 & 0.91 & 0.90 & 0.90 \\
GPT-4o-Vision & 0.74 & 0.82 & 0.74 & 0.73 \\
GPT-4o-Vision-COT & 0.70 & 0.83 & 0.70 & 0.68 \\
LLaMA-Vision & 0.20 & 0.23 & 0.20 & 0.09 \\
LLaMA-Vision-Knowledge & 0.22 & 0.05 & 0.22 & 0.08 \\

        \bottomrule
    \end{tabular}
    \label{full}
\end{table*}




\end{document}


\subsection{Experimental Setup}\label{subsec:setup}
\textbf{Datasets.} We evaluate the performance of \ourmethod on two real-world benchmarks, GOOD~\citep{good} and DrugOOD~\citep{drugood}, with various distribution shifts to evaluate our method. Specifically, GOOD is a comprehensive graph OOD benchmark, and we selected three datasets: (1) GOOD-HIV~\citep{wu2018moleculenet}, a molecular graph dataset predicting HIV inhibition; (2) GOOD-CMNIST~\citep{arjovsky2019invariant}, containing graphs transformed from MNIST using superpixel techniques; and (3) GOOD-Motif~\citep{wu2022discovering}, a synthetic dataset where graph motifs determine the label. DrugOOD is designed for AI-driven drug discovery with three types of distribution shifts: scaffold, size, and assay, and applies these to two measurements (IC50 and EC50). Details of datasets are in Appendix \ref{appe:data}.%applied to six molecular datasets for predicting drug-target binding affinity.
%We employ two real-world benchmarks containing various distribution shifts for graph OOD generalization to evaluate the effectiveness of our method.
% \begin{itemize}
%     \item \textbf{GOOD} is a systematic and comprehensive graph OOD benchmark, offering detailed distributions partition across various types of graphs. For the graph classification task, we selected three distinct datasets: (1) GOOD-HIV, a molecular graph dataset with the task of binary classification to predict whether a molecule can inhibit HIV; (2) GOOD-CMNIST, which consists of graphs representing hand-written digits transformed from the MNIST database using superpixel techniques; and (3) GOOD-Motif, a synthetic dataset where each graph is formed by connecting a base graph with a motif, where the motif alone determines the label.
%     \item \textbf{DrugOOD} is an OOD benchmark designed specifically for AI-driven drug discovery, where the data consists of molecular graphs. DrugOOD includes three basis of distribution shift: scaffold, size, and assay, and applies these to two measurements (IC50 and EC50). The benchmark contains six datasets, and all of them are tasked with predicting drug-target binding affinity, framed as a binary classification problem.
% \end{itemize}

\noindent\textbf{Baselines}.  We compare \ourmethod against ERM and two kinds of OOD baselines: (1)~Traditional OOD generalization approaches, including  Coral~\citep{coral}, IRM~\citep{arjovsky2019invariant} and VREx~\citep{krueger2021out}; (2)~graph-specific OOD generalization methods, including environment-based approaches (MoleOOD~\citep{yang2022learning}, CIGA~\citep{chen2022learning}, GIL~\citep{li2022learning}, and GREA~\citep{liu2022graph}, IGM~\citep{jia2024graph}), causal explanation-based approaches (Disc~\citep{fan2022debiasing} and DIR\citep{wu2022discovering}), and advanced architecture-based approaches (CAL~\citep{sui2022causal} and GSAT~\citep{miao2022interpretable}, iMoLD~\citep{zhuang2023learning}), GALA~\citep{Equad}, EQuAD~\citep{gala}. Details of all baselines are in Appendix \ref{appe:baseline}.

\noindent\textbf{Implementation Details}. To ensure fairness, we adopt the same experimental setup as iMold across two benchmarks. For molecular datasets with edge features, we use a three-layer GIN with a hidden dimension of 300, while for non-molecular graphs, we employ a four-layer GIN with a hidden dimension of 128. The projector is a two-layer MLP with a hidden dimension set to half that of the GIN encoder. EMA rate $\alpha$ for prototype updating is fixed at 0.99. Adam optimizer is used for model parameter updates.  All baselines use the optimal parameters from their original papers. Additional hyperparameter details can be found in Appendix~\ref{appe:hyperparam}.

\subsection{Performance Comparison}\label{subsec:results}

In this experiment, we aim to answer
\textbf{Q1: Whether \ourmethod achieves the best performance on OOD generalization benchmarks?} 
The answer is \textbf{YES}, since \ourmethod shows the best results on the majority of datasets. Specifically, we have the following observations.

 \noindent$\rhd$ \textsf{State-of-the-art results.}
According to Table~\ref{tab:main}, \ourmethod achieves state-of-the-art performance on 9 out of 11 datasets, and secures the second place on the remaining dataset. The average improvements against the previous SOTA are $2.17\%$ on GOOD and $1.68\%$ on DrugOOD. Notably, \ourmethod achieves competitive performance across various types of datasets with different data shifts, demonstrating its generalization ability on different data. Moreover, our model achieves the best results in both binary and multi-class tasks, highlighting the effectiveness of the multi-prototype classifier in handling different classification tasks.

 \noindent$\rhd$ \textsf{Sub-optimal performance of environment-based methods.}
Among all baselines, environment-based methods only achieve the best performance on 3 datasets, while architecture-based OOD generalization methods achieve the best results on most datasets. These observations suggest that environment-based methods are limited by the challenge of accurately capturing environmental information in graph data, leading to a discrepancy between theoretical expectations and empirical results. In contrast, the remarkable performance of \ourmethod also proves that graph OOD generalization can still be achieved without specific environmental information.
% According to Table~\ref{tab:main}, \ourmethod shows \textit{state-of-the-art performance on 10 out of 11 datasets} and secures second place on the remaining dataset. Notably, \ourmethod achieves competitive performance across various types of datasets with different data shifts, demonstrating its superior ability to achieve environment invariance. The superior performance of \ourmethod in both binary and multi-class tasks highlights the strength of the multi-prototype-based classification approach. In contrast, we find that the \textit{environment-based methods display sub-optimal performance} in most cases, demonstrating their limitations in capturing the environments. The remarkable performance of \ourmethod also proves that graph OOD generalization can still be achieved without specific environmental information.
\begin{figure*}[!t]
    \centering
    \subfigure[Sensitivity of $k$]{ \label{subfig:paramk}
    \includegraphics[height=0.16\textwidth]{4_exp/param_hiv_k.pdf}
    }
    \hfill
    \subfigure[Sensitivity of $\beta$]{ \label{subfig:paramb}
    \includegraphics[height=0.16\textwidth]{4_exp/param_hiv_beta.pdf}
    }
    \hfill
    \subfigure[Impact of prototype updating mechanisms]{ \label{subfig:update}
    \includegraphics[height=0.16\textwidth]{4_exp/init.pdf}
    }
    \hfill
    \subfigure[Impact of different statistical metrics]{ \label{subfig:metric}
    \includegraphics[height=0.16\textwidth]{4_exp/std.pdf}
    }
    \vspace{-4mm}
    \caption{The two figures on the left present a hyperparameter analysis of the  $K$ and $\beta$, while the two figures on the right illustrate the comparison of different module designs on prototype update and metric used in Eq.~\eqref{classify}. } 
    \vspace{-2mm}
    \label{fig:four_images}
    
\end{figure*}
\subsection{Ablation Study}
We aim to discover 
\textbf{Q2: Does each module in \ourmethod contribute to effective OOD generalization?} The answer is \textbf{YES}, as removing any key component leads to performance degradation, as demonstrated by the results in Table~\ref{tab:ablation}. We have the following discussions.%according to our ablation experiments that verify the effectiveness of the loss constraint (i.e., $\mathcal{L}_{\mathrm{IPM}}$ and $\mathcal{L}_{\mathrm{PS}}$), and the components of our model, including hyperspherical projection, multi-prototype mechanism, weight updating, and weight pruning techniques. The results are shown in Table~\ref{tab:ablation}, with the following discussions.

% as we conduct experiments on three datasets to verify the role of our proposed loss constraint $\mathcal{L}_{\mathrm{IPM}}$, $\mathcal{L}_{\mathrm{PS}}$, and the component of our model: projector and weight pruning technique. The results are shown in Table~\ref{ablation}.
% \begin{table}[!t]
% \centering
% \scalebox{0.68}{
%     \begin{tabular}{ll cccc}
%       \toprule
%       & \multicolumn{4}{c}{\textbf{Intellipro Dataset}}\\
%       & \multicolumn{2}{c}{Rank Resume} & \multicolumn{2}{c}{Rank Job} \\
%       \cmidrule(lr){2-3} \cmidrule(lr){4-5} 
%       \textbf{Method}
%       &  Recall@100 & nDCG@100 & Recall@10 & nDCG@10 \\
%       \midrule
%       \confitold{}
%       & 71.28 &34.79 &76.50 &52.57 
%       \\
%       \cmidrule{2-5}
%       \confitsimple{}
%     & 82.53 &48.17
%        & 85.58 &64.91
     
%        \\
%        +\RunnerUpMiningShort{}
%     &85.43 &50.99 &91.38 &71.34 
%       \\
%       +\HyReShort
%         &- & -
%        &-&-\\
       
%       \bottomrule

%     \end{tabular}
%   }
% \caption{Ablation studies using Jina-v2-base as the encoder. ``\confitsimple{}'' refers using a simplified encoder architecture. \framework{} trains \confitsimple{} with \RunnerUpMiningShort{} and \HyReShort{}.}
% \label{tbl:ablation}
% \end{table}
\begin{table*}[!t]
\centering
\scalebox{0.75}{
    \begin{tabular}{l cccc cccc}
      \toprule
      & \multicolumn{4}{c}{\textbf{Recruiting Dataset}}
      & \multicolumn{4}{c}{\textbf{AliYun Dataset}}\\
      & \multicolumn{2}{c}{Rank Resume} & \multicolumn{2}{c}{Rank Job} 
      & \multicolumn{2}{c}{Rank Resume} & \multicolumn{2}{c}{Rank Job}\\
      \cmidrule(lr){2-3} \cmidrule(lr){4-5} 
      \cmidrule(lr){6-7} \cmidrule(lr){8-9} 
      \textbf{Method}
      & Recall@100 & nDCG@100 & Recall@10 & nDCG@10
      & Recall@100 & nDCG@100 & Recall@10 & nDCG@10\\
      \midrule
      \confitold{}
      & 71.28 & 34.79 & 76.50 & 52.57 
      & 87.81 & 65.06 & 72.39 & 56.12
      \\
      \cmidrule{2-9}
      \confitsimple{}
      & 82.53 & 48.17 & 85.58 & 64.91
      & 94.90&78.40 & 78.70& 65.45
       \\
      +\HyReShort{}
       &85.28 & 49.50
       &90.25 & 70.22
       & 96.62&81.99 & \textbf{81.16}& 67.63
       \\
      +\RunnerUpMiningShort{}
       % & 85.14& 49.82
       % &90.75&72.51
       & \textbf{86.13}&\textbf{51.90} & \textbf{94.25}&\textbf{73.32}
       & \textbf{97.07}&\textbf{83.11} & 80.49& \textbf{68.02}
       \\
   %     +\RunnerUpMiningShort{}
   %    & 85.43 & 50.99 & 91.38 & 71.34 
   %    & 96.24 & 82.95 & 80.12 & 66.96
   %    \\
   %    +\HyReShort{} old
   %     &85.28 & 49.50
   %     &90.25 & 70.22
   %     & 96.62&81.99 & 81.16& 67.63
   %     \\
   % +\HyReShort{} 
   %     % & 85.14& 49.82
   %     % &90.75&72.51
   %     & 86.83&51.77 &92.00 &72.04
   %     & 97.07&83.11 & 80.49& 68.02
   %     \\
      \bottomrule

    \end{tabular}
  }
\caption{\framework{} ablation studies. ``\confitsimple{}'' refers using a simplified encoder architecture. \framework{} trains \confitsimple{} with \RunnerUpMiningShort{} and \HyReShort{}. We use Jina-v2-base as the encoder due to its better performance.
}
\label{tbl:ablation}
\end{table*}
\noindent$\rhd$ \textsf{Ablation on $\mathcal{L}_{\mathrm{IPM}}$ and $\mathcal{L}_{\mathrm{PS}}$.}
We remove $\mathcal{L}_{\mathrm{IPM}}$ and $\mathcal{L}_{\mathrm{PS}}$ in the Eq.~\eqref{eq: target2} respectively to explore their impacts on the performance of graph OOD generalization. The experimental results demonstrate a clear fact: merely optimizing for invariance (w/o $\mathcal{L}_{\mathrm{PS}}$) or separability (w/o $\mathcal{L}_{\mathrm{IPM}}$) weakens the OOD generalization ability of our model, especially for the multi-class classification task, as shown in CMNIST-color. This provides strong evidence that ensuring both invariance and separability is a sufficient and necessary condition for effective OOD generalization in graph learning.
\noindent$\rhd$ \textsf{Ablation on the design of \ourmethod.} To verify the effectiveness of each module designed for \ourmethod,
we conducted ablation studies by removing the hyperspherical projection(w/o Project), multi-prototype mechanism (w/o Multi-P), invariant encoder (w/o Inv.Enc), and prototype-related weight calculations (w/o Update) and pruning (w/o Prune). The results confirm their necessity. First, removing the hyperspherical projection significantly drops performance, as optimizing Eq.~(\ref{eq: target2}) requires hyperspherical space. Without it, results are even worse than ERM. Similarly, setting the prototype count to one blurs decision boundaries and affects the loss function $\mathcal{L}_{\mathrm{PS}}$, compromising inter-class separability. Lastly, replacing the invariant encoder $\mathrm{GNN}_{S}$ with $\mathrm{GNN}_{E}$ directly introduces environment-related noise, making it difficult to obtain effective invariant features, thus hindering OOD generalization.
Additionally, the removal of prototype-related weight calculations and weight pruning degraded prototypes into the average of all class samples, resulting in the prototypes degrading into the average representation of all samples in the class, failing to maintain classification performance in OOD scenarios.

\subsection{Visualized Validation}
In this subsection, we aim to investigate \textbf{Q3: Can these key designs (i.e., hyperspherical space and multi-prototype mechanism) tackle two unique challenges in graph OOD generalization tasks?} The answer is \textbf{YES}, we conduct the following visualization experiments to verify this conclusion. %\textbf{Q3: Whether hyperspherical Spaces are more separable than ordinary latent Spaces} Yes, the visualization shows the advantage of hyperspherical space over traditional latent space.

\noindent$\rhd$ \textsf{Hyperspherical representation space.} To validate the advantage of hyperspherical space in enhancing class separability, we compare the 1-order Wasserstein distance~\cite{villani2009optimal} between same-class and different-class samples, as shown in Fig.~\ref{wl_Dis}.
It is evident that \ourmethod produces more separable invariant representations (higher inter-class distance), while also exhibiting tighter clustering for samples of the same class (lower intra-class distance). In contrast, although traditional latent spaces-based SOTA  achieves a certain level of intra-class compactness, its lower separability hinders its overall performance. Additionally, we visualized the sample representations learned by our \ourmethod and SOTA using t-SNE in Fig.~\ref{tsne}, where corresponding phenomenon can be witnessed.
% we visualized the invariant representation 
% $\hat{z}_{inv}$ in both the training set (ID) and the test set (OOD). Using t-SNE, we visualize the representation distributions learned by \ourmethod and the SOTA method, CIGA, as shown in Fig.~\ref{fig:tsne}. 
% \begin{figure*}[t!] \label{v_prototype}
% \centering    
% \includegraphics[scale=0.39]{3_method/sx_prototype.pdf}
% \caption{In the case of the DrugOOD-IC50-assay (binary classification task), we set up three prototypes for each class and visualized the closest example to it. }  
% \label{fig:intro} 
% \end{figure*}





\begin{figure}[t]
\vspace{-3mm}
\centering 
\subfigure[1-order Wasserstein distance]{ \label{wl_Dis}
\includegraphics[width=0.17\textwidth,height=0.13\textwidth]{4_exp/wl_score.pdf}
}
\subfigure[T-SNE visualization, left: \ourmethod, right: SOTA ]{ \label{tsne}
\includegraphics[width=0.13\textwidth,height=0.13\textwidth]{4_exp/my.pdf}
\includegraphics[width=0.13\textwidth,height=0.13\textwidth]{4_exp/imold.pdf}
}
\vspace{-3mm}
\caption{Visualization and quantitative analysis of the separability advantages in hyperspherical space on HIV-scaffold.}
\vspace{-4mm}
\end{figure}

% \begin{figure}[t]
%     \centering
%     % \captionsetup[wragfigure]{ font=footnotesize}

%     \includegraphics[width=0.45\textwidth]{4_exp/wl_score.pdf}

%     \caption{The 1-order WL distance between samples of the same class (D(Y=0), D(Y=1)) and between samples of different classes (inter-distance).}
%     \vspace{-10pt}
%     \label{fig:wl}
% \end{figure}
\noindent$\rhd$ \textsf{Prototypes visualization.} We also reveal the characteristics of prototypes by visualizing samples that exhibit the highest similarity to each prototype. Fig.~\ref{fig:proto} shows that prototypes from different classes capture distinct invariant subgraphs, ensuring a strong correlation with their respective labels. Furthermore, within the same category, different prototypes encapsulate samples with varying environmental subgraphs. This validates that multi-prototype learning can effectively capture label-correlated invariant features without explicit environment definitions, which solve the challenges of out-of-distribution generalization in real-world graph data.
\begin{figure}[t]
    \centering
    % \captionsetup[wragfigure]{ font=footnotesize}

    \includegraphics[width=0.45\textwidth]{4_exp/proto.pdf}

    \caption{Visualizations of prototypes and invariant subgraphs~(highlighted) of IC50-assay dataset.}
    \label{fig:proto}
\vspace{-5mm}
\end{figure}

\subsection{In-Depth Analysis}
In this experiment, we will investigate \textbf{Q4: How do the details (hyperparameter settings and variable designs) in \ourmethod impact performance?} The following experiments are conducted to answer this question and the experimental results are in Fig.~\ref{fig:four_images}.

\noindent$\rhd$ \textsf{Hyperparameter Analysis.}
To investigate the sensitivity of the number of prototypes and the coefficient $\beta$ in $\mathcal{L}_{\mathrm{IPM}}$ on performance, we vary $k$ from 2 to 6 and $\beta$ from $\{0.01, 0.05, 0.1, 0.2, 0.3\}$. Our conclusions are as follows: \ding{192} In Fig.~\ref{subfig:paramk}, the best performance is achieved when the number of prototypes is approximately twice the number of classes. Deviating from this optimal range, either too many or too few prototypes negatively impacts the final performance. \ding{193} According to Fig.~\ref{subfig:paramb}, a smaller $\beta$  hampers the model’s ability to effectively learn invariant features, while selecting a moderate $\beta$ leads to the best performance.
 
% \noindent$\rhd$ \textsf{Numerical quantitative analysis} Analysis of $\beta$. To discover the sensitivity of \ourmethod to coefficient $\beta$ in $\mathcal{L}_{\mathrm{IPM}}$, we search $\beta$ from $\{0.01, 0.05, 0.1, 0.2, 0.3\}$ and present the results in Fig.~\ref{hyper_3} and \ref{hyper_4}. We observe that a small $\beta$ (e.g., 0.01 and 0.05) hampers the model’s ability to effectively learn invariant features, while selecting a moderate $\beta$ (i.e., 0.1) leads to the best performance.

\noindent$\rhd$ \textsf{Module design analysis.}
To investigate the impact of different prototype update mechanisms and statistical metrics in Eq.~\eqref{classify}, we conducted experimental analyses and found that \ding{192} According to Fig.~\ref{subfig:update}, all compared methods lead to performance drops due to their inability to ensure that the updated prototypes possess both intra-class diversity and inter-class separability, which is the key to the success of MPHIL's prototype update method. \ding{193} In Fig.~\ref{subfig:metric}, $\max$ achieves the best performance by selecting the most similar prototype to the sample, helping the classifier converge faster to the correct decision space.
% To discover the sensitivity of \ourmethod to coefficient $\beta$ in $\mathcal{L}_{\mathrm{IPM}}$, we search $\beta$ from $\{0.01, 0.05, 0.1, 0.2, 0.3\}$ and present the results in Fig.~\ref{hyper_3} and \ref{hyper_4}. We observe that a small $\beta$ (e.g., 0.01 and 0.05) hampers the model’s ability to effectively learn invariant features, while selecting a moderate $\beta$ (i.e., 0.1) leads to the best performance.



% \begin{figure*}[h]
% \centering 
% \subfigure[CIGA] { 
% \includegraphics[width=0.4\textwidth]{imold.pdf} \label{t1}
% }
% \subfigure[Ours] { 
% \includegraphics[width=0.4\textwidth]{my(1).pdf} \label{t2}
% }
% \caption{Left, Middle:} 
% \label{t-sne}
% \vspace{-1pt}
% \end{figure*}


\section{Conclusion}
\section{Conclusion}
In this work, we propose a simple yet effective approach, called SMILE, for graph few-shot learning with fewer tasks. Specifically, we introduce a novel dual-level mixup strategy, including within-task and across-task mixup, for enriching the diversity of nodes within each task and the diversity of tasks. Also, we incorporate the degree-based prior information to learn expressive node embeddings. Theoretically, we prove that SMILE effectively enhances the model's generalization performance. Empirically, we conduct extensive experiments on multiple benchmarks and the results suggest that SMILE significantly outperforms other baselines, including both in-domain and cross-domain few-shot settings.
\bibliography{iclr2025_conference}
\bibliographystyle{iclr2025_conference}

\clearpage
\appendix

% \section{Related Works} \label{appe:rw}
% Our work draws heavily from the literature on semiparametric inference and double machine learning~\citep{robins1994estimation,robins1995semiparametric,tsiatis2006semiparametric,chernozhukov2018double}. In particular, our estimator is an optimal combination of several Augmented Inverse Probability Weighting~(\aipw) estimators, whose outcome regressions are replaced with foundation models. Importantly, the standard $\aipw$ estimator, which relies on an outcome regression estimated using experimental data alone, is also included in the combination. This approach allows \ours~to significantly reduce finite sample (and potentially asymptotic) variance while attaining the semiparametric \emph{efficiency bound}---the smallest asymptotic variance among all consistent and asymptotically normal estimators of the average treatment effect---even when the foundation models are arbitrarily biased.


\paragraph{Integrating foundation models}
Prediction-powered inference~(\ppi)~\citep{angelopoulos2023prediction} is a statistical framework that constructs valid confidence intervals using a small labeled dataset and a large unlabeled dataset imputed by a foundation model. $\ppi$ has been applied in various domains, including generalization of causal inferences~\citep{demirel24prediction}, large language model evaluation~\citep{fisch2024stratified,dorner2024limitsscalableevaluationfrontier}, and improving the efficiency of social science experiments~\citep{broskamixed,egami2024using}. However, unlike our approach, $\ppi$ requires access to an additional unlabeled dataset from the same distribution as the experimental sample, which may be as costly as labeled data. Recent work by \citet{poulet2025prediction} introduces 
Prediction-powered inference for clinical trials ($\ppct$), an adaptation of $\ppi$ to estimate  average treatment effects in randomized experiments without any additional  external data. $\ppct$ combines the difference in means estimator with an 
$\aipw$ estimator that integrates the same foundation model as the outcome regression for both treatment and control groups. However, our work differs in two key aspects:
(i) $\ppct$ integrates a single foundation model, and (ii) $\ppct$ does not include the standard $\aipw$ estimator with the outcome regression estimated from experimental data. As a result, $\ppct$ cannot achieve the efficiency bound unless the foundation model is almost surely equal to the underlying outcome regression. 


 



\paragraph{Integrating observational data} There is growing interest in augmenting randomized experiments with data from observational studies to improve statistical precision. One approach involves first testing whether the observational data is compatible with the experimental data~\citep{dahabreh2024using}---for instance, using a statistical test to assess if the mean of the outcome conditional on the covariates is invariant across studies \cite{luedtke2019omnibus,hussain2023falsification,de2024detecting}—and then combining the datasets to improve precision, if the test does not reject. These tests, however, have low statistical power, especially when the experimental sample size is small, which is precisely when leveraging observational data would be most beneficial. To overcome this, a recent line of work integrates a prognostic score estimated from observational data as a covariate when estimating the outcome regression~\citep{schuler2022increasing,liao2023prognostic}. However, increasing the dimensionality of the problem---by adding an additional covariate---can increase estimation error and inflate the finite sample variance. Finally, the work most closely related to ours is \citet{karlsson2024robust}, that integrates an outcome regression estimated from observational data into the \aipw~estimator. In contrast, our approach is not constrained by the availability of well-structured observational data, since it leverages black-box foundation models trained on external data sources.
% You may include other additional sections here.




\section{MPHIL Objective Deductions} \label{appe:proof}
\subsection{Proof of the overall objective}\label{appe:deduct}
In this section, we explain how we derived our goal in Eq.~(\ref{eq: target2}) from Eq.~(\ref{eq:soft}). Let's recall that Eq.~(\ref{eq:soft}) is formulated as:
\begin{equation}
   \underset{f_{c}, g}{\min} -I(y; \mathbf{z}_{inv})+\beta I(\mathbf{z}_{inv};e),
\end{equation}
For the first term $-I(y;\mathbf{z}_{inv})$, since we are mapping invariant features to hyperspherical space, we replace $\mathbf{z}_{inv}$ with $\mathbf{\hat{z}}_{inv}$. Then according to the definition of mutual information:
\begin{equation}\label{eq:muinfor}
-I(y;\mathbf{\hat{z}}_{inv}) = - \mathbb{E}_{y, \mathbf{\hat{z}}_{inv}} \left[ \log \frac{p(y,\mathbf{\hat{z}}_{inv})}{p(y) p(\mathbf{\hat{z}}_{inv})} \right].
\end{equation}

We introduce intermediate variables $\boldsymbol{\mu}^{y}$ to rewrite Eq. (\ref{eq:muinfor}) as:
\begin{equation}
    \resizebox{.9\hsize}{!}{
$-I(y; \mathbf{\hat{z}}_{inv}) = -E_{y, \mathbf{\hat{z}}_{inv}, \boldsymbol{\mu}^{y}} \left[ \log \frac{p(y, \mathbf{\hat{z}}_{inv}, \boldsymbol{\mu}^{y})}{p(\mathbf{\hat{z}}_{inv}, \boldsymbol{\mu}^{y}) p(y)} \right] + E_{y, \mathbf{\hat{z}}_{inv},\boldsymbol{\mu}^{y}} \left[ \log \frac{p(y,\boldsymbol{\mu}^{y}|\mathbf{\hat{z}}_{inv})}{p(y|\mathbf{\hat{z}}_{inv}) p(\boldsymbol{\mu}^{y})|\mathbf{\hat{z}}_{inv}} \right]. $}
\end{equation}




By the definition of Conditional mutual information, we have the following equation:
\begin{equation}
    \begin{aligned} 
    -I(y;\mathbf{\hat{z}}_{inv})  &= -I(y;\mathbf{\hat{z}}_{inv},\boldsymbol{\mu}^{y})+I(y;\boldsymbol{\mu}^{y}|\mathbf{\hat{z}}_{inv}),  \\ 
    -I(y;\boldsymbol{\mu}^{y})  &= -I(y;\mathbf{\hat{z}}_{inv},\boldsymbol{\mu}^{y})+I(y;\mathbf{\hat{z}}_{inv}|\boldsymbol{\mu}^{y}). 
\end{aligned}
\end{equation}
% Our work draws heavily from the literature on semiparametric inference and double machine learning~\citep{robins1994estimation,robins1995semiparametric,tsiatis2006semiparametric,chernozhukov2018double}. In particular, our estimator is an optimal combination of several Augmented Inverse Probability Weighting~(\aipw) estimators, whose outcome regressions are replaced with foundation models. Importantly, the standard $\aipw$ estimator, which relies on an outcome regression estimated using experimental data alone, is also included in the combination. This approach allows \ours~to significantly reduce finite sample (and potentially asymptotic) variance while attaining the semiparametric \emph{efficiency bound}---the smallest asymptotic variance among all consistent and asymptotically normal estimators of the average treatment effect---even when the foundation models are arbitrarily biased.


\paragraph{Integrating foundation models}
Prediction-powered inference~(\ppi)~\citep{angelopoulos2023prediction} is a statistical framework that constructs valid confidence intervals using a small labeled dataset and a large unlabeled dataset imputed by a foundation model. $\ppi$ has been applied in various domains, including generalization of causal inferences~\citep{demirel24prediction}, large language model evaluation~\citep{fisch2024stratified,dorner2024limitsscalableevaluationfrontier}, and improving the efficiency of social science experiments~\citep{broskamixed,egami2024using}. However, unlike our approach, $\ppi$ requires access to an additional unlabeled dataset from the same distribution as the experimental sample, which may be as costly as labeled data. Recent work by \citet{poulet2025prediction} introduces 
Prediction-powered inference for clinical trials ($\ppct$), an adaptation of $\ppi$ to estimate  average treatment effects in randomized experiments without any additional  external data. $\ppct$ combines the difference in means estimator with an 
$\aipw$ estimator that integrates the same foundation model as the outcome regression for both treatment and control groups. However, our work differs in two key aspects:
(i) $\ppct$ integrates a single foundation model, and (ii) $\ppct$ does not include the standard $\aipw$ estimator with the outcome regression estimated from experimental data. As a result, $\ppct$ cannot achieve the efficiency bound unless the foundation model is almost surely equal to the underlying outcome regression. 


 



\paragraph{Integrating observational data} There is growing interest in augmenting randomized experiments with data from observational studies to improve statistical precision. One approach involves first testing whether the observational data is compatible with the experimental data~\citep{dahabreh2024using}---for instance, using a statistical test to assess if the mean of the outcome conditional on the covariates is invariant across studies \cite{luedtke2019omnibus,hussain2023falsification,de2024detecting}—and then combining the datasets to improve precision, if the test does not reject. These tests, however, have low statistical power, especially when the experimental sample size is small, which is precisely when leveraging observational data would be most beneficial. To overcome this, a recent line of work integrates a prognostic score estimated from observational data as a covariate when estimating the outcome regression~\citep{schuler2022increasing,liao2023prognostic}. However, increasing the dimensionality of the problem---by adding an additional covariate---can increase estimation error and inflate the finite sample variance. Finally, the work most closely related to ours is \citet{karlsson2024robust}, that integrates an outcome regression estimated from observational data into the \aipw~estimator. In contrast, our approach is not constrained by the availability of well-structured observational data, since it leverages black-box foundation models trained on external data sources.

By merging the same terms, we have:
\begin{equation}
     -I(y;\mathbf{\hat{z}}_{inv})  = -I(y;\boldsymbol{\mu}^{y})+[I(y;\boldsymbol{\mu}^{y}|\mathbf{\hat{z}}_{inv})-I(y;\mathbf{\hat{z}}_{inv}|\boldsymbol{\mu}^{y})]. 
\end{equation}

Since our classification is based on the distance between $\boldsymbol{\mu}^{y}$ and $\mathbf{\hat{z}}_{inv}$, we add $-I(y;\mathbf{\hat{z}}_{inv},\boldsymbol{\mu}^{y})$ back into the above equation and obtain a lower bound:
\begin{equation}
\resizebox{.9\hsize}{!}{
$-I(y;\mathbf{\hat{z}}_{inv}) \geq -I(y;\boldsymbol{\mu}^{y})+[I(y;\boldsymbol{\mu}^{y}|\mathbf{\hat{z}}_{inv})-I(y;\mathbf{\hat{z}}_{inv}|\boldsymbol{\mu}^{y})]-I(y;\mathbf{\hat{z}}_{inv},\boldsymbol{\mu}^{y}).$}
\end{equation}
Since the $\boldsymbol{\mu}^{y}$ are updated by $\mathbf{\hat{z}}_{inv}$ from the same class, we can approximate $I(y;\boldsymbol{\mu}^{y}|\mathbf{\hat{z}}_{inv})$ equal to $I(y;\mathbf{\hat{z}}_{inv}|\boldsymbol{\mu}^{y})$ and obtain the new lower bound:
\begin{equation}
\label{eq: first part}
    -I(y;\mathbf{\hat{z}}_{inv}) \geq -I(y;\boldsymbol{\mu}^{y})-I(y;\mathbf{\hat{z}}_{inv},\boldsymbol{\mu}^{y}).
\end{equation}

For the second term $I(\mathbf{z}_{inv};e)$, we can also rewrite it as:
\begin{equation}
  \begin{aligned}
    I(\mathbf{\hat{z}}_{inv};e)  &= I(\mathbf{\hat{z}}_{inv};e,\boldsymbol{\mu}^{y})-I(\mathbf{\hat{z}}_{inv};\boldsymbol{\mu}^{y}|e).
    \end{aligned}
\end{equation}

Given that the environmental labels  $e$ are unknown, we drop the term $I(\mathbf{\hat{z}}_{inv};e,\boldsymbol{\mu}^{y})$ as it cannot be directly computed. This leads to the following lower bound:
\begin{equation}\label{z_e}
  \begin{aligned}
    I(\mathbf{\hat{z}}_{inv};e)  &\geq-I(\mathbf{\hat{z}}_{inv};\boldsymbol{\mu}^{y}|e).
    \end{aligned}
\end{equation}

We can obtain a achievable target by Eq. (\ref{eq: first part}) and Eq. (\ref{z_e}) as follow:
\begin{equation}\label{uneq_1}
\resizebox{.9\hsize}{!}{$
    -I(y; \mathbf{z}_{inv})+\beta I(\mathbf{z}_{inv};e)\geq-I(y;\boldsymbol{\mu}^{y})-I(y;\mathbf{\hat{z}}_{inv},\boldsymbol{\mu}^{y})-\beta I(\mathbf{\hat{z}}_{inv};\boldsymbol{\mu}^{y}|e).$}
\end{equation}

In fact, $p(\mathbf{\hat{z}}_{inv},\boldsymbol{\mu}^{y}|e) \geq p(\mathbf{\hat{z}}_{inv};\boldsymbol{\mu}^{y})$, Eq. (\ref{uneq_1}) can be achieved by:
\begin{equation}\resizebox{.9\hsize}{!}{$
    -I(y; \mathbf{z}_{inv})+\beta I(\mathbf{z}_{inv};e)\geq-I(y;\boldsymbol{\mu}^{y})-I(y;\mathbf{\hat{z}}_{inv},\boldsymbol{\mu}^{y})-\beta I(\mathbf{\hat{z}}_{inv};\boldsymbol{\mu}^{y}). $}
\end{equation}

Finally, optimizing Eq. (\ref{eq:soft}) can be equivalent to optimizing its lower bound and we can obtain the objective without environment $e$ as shown in Eq. (\ref{eq: target2}):
\begin{equation}
\setlength\abovedisplayskip{9pt}
\setlength\belowdisplayskip{9pt}
   \underset{f_{c}, g}{\min} \underbrace{-I(y;\hat{\mathbf{z}}_{inv},\boldsymbol{\mu}^{(y)}) }_{\mathcal{L}_{\mathrm{C}}}\underbrace{-I(y ; \boldsymbol{\mu}^{(y)}) }_{\mathcal{L}_{\mathrm{PS}}}\underbrace{-\beta I(\hat{\mathbf{z}}_{inv};\boldsymbol{\mu}^{(y)})}_{\mathcal{L}_{\mathrm{IPM}}}.
\end{equation}

\subsection{Proof of $\mathcal{L}_{\mathrm{C}}$ } \label{appe:deduct2}
For the term $I(y;\hat{\mathbf{z}}_{inv},\boldsymbol{\mu}^{(y)})$, it can be written as:
\begin{equation}
    I(y;\hat{\mathbf{z}}_{inv},\boldsymbol{\mu}^{(y)}) = E_{y, \mathbf{\hat{z}}_{inv}, \boldsymbol{\mu}^{y}} \left[ \log \frac{p(y| \mathbf{\hat{z}}_{inv}, \boldsymbol{\mu}^{y})}{p(y)} \right],
\end{equation}
according to ~\citep{graphpro}, we have:
\begin{equation}
    I(y;\hat{\mathbf{z}}_{inv},\boldsymbol{\mu}^{(y)}) \geq E_{y, \mathbf{\hat{z}}_{inv}, \boldsymbol{\mu}^{y}} \left[ \log \frac{q_{\theta}(y| \gamma(\mathbf{\hat{z}}_{inv}, \boldsymbol{\mu}^{y}))}{p(y)} \right],
\end{equation}
where $\gamma(,)$ is the function to calculate the similarity between $\mathbf{\hat{z}}_{inv}$ and $\boldsymbol{\mu}^{y}$. $q_{\theta}(y| \gamma(\mathbf{\hat{z}}_{inv}, \boldsymbol{\mu}^{y}))$ is the variational approximation of the $p(y| \gamma(\mathbf{\hat{z}}_{inv}, \boldsymbol{\mu}^{y}))$. Then we can have:
\begin{align}\nonumber
    I(y;\hat{\mathbf{z}}_{inv},\boldsymbol{\mu}^{(y)}) &\geq E_{y, \mathbf{\hat{z}}_{inv}, \boldsymbol{\mu}^{y}} \left[ \log \frac{q_{\theta}(y| \gamma(\mathbf{\hat{z}}_{inv}, \boldsymbol{\mu}^{y}))}{p(y)} \right] \\ \nonumber
    &\geq E_{y, \mathbf{\hat{z}}_{inv}, \boldsymbol{\mu}^{y}} \left[ \log {q_{\theta}(y| \gamma(\mathbf{\hat{z}}_{inv}, \boldsymbol{\mu}^{y}))} \right]- E_{y}[\log p(y)]\\ \nonumber
    &\geq E_{y, \mathbf{\hat{z}}_{inv}, \boldsymbol{\mu}^{y}} \left[ \log {q_{\theta}(y| \gamma(\mathbf{\hat{z}}_{inv}, \boldsymbol{\mu}^{y}))} \right]\\
    &:= -\mathcal{L}_{\mathrm{C}}.
\end{align}

Finally, we prove that $\min I(y;\hat{\mathbf{z}}_{inv},\boldsymbol{\mu}^{(y)})$ is equivalent to minimizing the classification loss $\mathcal{L}_{\mathrm{C}}$.

\section{Methodology Details}\label{appe:method_detail}

\subsection{Overall Algorithm of \ourmethod} 
The training algorithm of \ourmethod is shown in Algorithm.~\ref{code:train}. After that, we use the well-trained $\mathrm{GNN}_{S}$,$\mathrm{GNN}_{E}$, $\mathrm{Proj}$ and all prototypes $\mathbf{M}^{(c)} = \{ \boldsymbol{\mu}_{k}^{(c)}\}^{K}_{k=1}$ to perform inference on the test set. The pseudo-code for this process is shown in Algorithm.~\ref{code:test}.
\begin{algorithm}[h!]
\renewcommand{\algorithmicrequire}{\textbf{Input:}}
	\renewcommand{\algorithmicensure}{\textbf{Output:}}
  \caption{The training algorithm of \ourmethod.}
  \begin{algorithmic}[1]
    \REQUIRE
      Scoring GNN $\mathrm{GNN}_{S}$;
      Encoding GNN $\mathrm{GNN}_{E}$;
      Projection $\mathrm{Proj}$;
      Number of prototypes for each class $K$;
      The data loader of in-distribution training set $D_{\mathrm{train}}$.
      
    \ENSURE
    Well-trained $\mathrm{GNN}_{S}$, $\mathrm{GNN}_{E}$, $\mathrm{Proj}$
       and all prototypes $\mathbf{M}^{(c)}$.
    \STATE 
For each class $c \in \{1, \cdots, C\}$, assign $K$ prototypes for it which can be denoted by $\mathbf{M}^{(c)} = \{ \boldsymbol{\mu}_{k}^{(c)}\}^{K}_{k=1}$.    
\STATE Initialize each of them by $\boldsymbol{\mu}_{k}^{(c)} \sim \mathcal{N}(\textbf{0}, \textbf{I})$
     \FOR{epoch in epochs}  
    \FOR{each $G_\mathrm{{batch}}$ in $D_{\mathrm{train}}$}
       \STATE Obtain $Z_{inv}$  using $\mathrm{GNN}_{S}$ and $\mathrm{GNN}_{E}$ via Eq. (\ref{eq:score}) and (\ref{eq:readout})
       \STATE Obtain $\hat{Z}_{inv}$ using $\mathrm{Proj}$ via Eq. (\ref{eq: hype_rinv})
       \STATE Compute $W^{(c)}$ using $\boldsymbol{u}^{(c)}$ and $\hat{Z}_{inv}$ via Eq. (\ref{eq: att}).
        \FOR{each prototype $\boldsymbol{u}_{k}^{(c)}$}
        \STATE Update it using $\hat{Z}_{inv}$ and $W^{(c)}$ via Eq. (\ref{eq: update_prototype}).
        
        \ENDFOR
        \STATE Get $p(y = c \mid \hat{\mathbf{z}}_{i}; \{ w^{c},\boldsymbol{u}^{(c)} \}_{c=1}^{(C)})$ using $\hat{Z}_{inv}$, $W^{(c)}$ and $\boldsymbol{\mu}^{(c)}$ via Eq. (\ref{classify})
        \STATE Compute the final loss $\mathcal{L}$ with $\hat{Z}_{inv}$, $\boldsymbol{\mu}^{(c)}$ and $p(y = c \mid \hat{\mathbf{z}}_{i}; \{ w^{c},\boldsymbol{u}^{(c)} \}_{c=1}^{(C)})$ via Eq. (\ref{ipm}), (\ref{ps}) and (\ref{cls})
        % \State Generate $\overline{G}$ and add to $TL$;
        \STATE Update parameters of $\mathrm{GNN}_{S}$, $\mathrm{GNN}_{E}$ and $\mathrm{Proj}$ with the gradient of $\mathcal{L}$.
      \ENDFOR
    \ENDFOR
  \end{algorithmic}\label{code:train}
\end{algorithm}
\vspace{-5mm}
\begin{algorithm}[h!]
\renewcommand{\algorithmicrequire}{\textbf{Input:}}
	\renewcommand{\algorithmicensure}{\textbf{Output:}}
  \caption{The inference algorithm of \ourmethod.}
  \begin{algorithmic}[1]
    \REQUIRE
      Well-trained $\mathrm{GNN}_{S}$,$\mathrm{GNN}_{E}$, $\mathrm{Proj}$ and all prototypes $\mathbf{M}^{(c)} = \{ \boldsymbol{\mu}_{k}^{(c)}\}^{K}_{k=1}$.
      The data loader of Out-of-distribution testing set $D_{\mathrm{test}}$.
      
    \ENSURE
    Classification probability $p(y = c \mid \hat{\mathbf{z}}_{i}; \{ w^{c},\boldsymbol{\mu}^{(c)} \}_{c=1}^{(C)})$ 
    \FOR{each $G_\mathrm{{batch}}$ in $D_{\mathrm{test}}$}
       \STATE Obtain $Z_{inv}$  using $\mathrm{GNN}_{S}$ and $\mathrm{GNN}_{E}$ via Eq. (\ref{eq:score}) and (\ref{eq:readout})
       \STATE Obtain $\hat{Z}_{inv}$ using $\mathrm{Proj}$ via Eq. (\ref{eq: hype_rinv})
       \STATE Compute $W^{(c)}$ using $\boldsymbol{\mu}^{(c)}$ and $\hat{Z}_{inv}$ via Eq. (\ref{eq: att}).
        
        \STATE Get $p(y = c \mid \hat{\mathbf{z}}_{i}; \{ w^{c},\boldsymbol{\mu}^{(c)} \}_{c=1}^{(C)})$ using $\hat{Z}_{inv}$, $W^{(c)}$ and $\boldsymbol{\mu}^{(c)}$ via Eq. (\ref{classify})
        
        % \State Generate $\overline{G}$ and add to $TL$;
        
      \ENDFOR
  \end{algorithmic}\label{code:test}

\end{algorithm}
\vspace{-6mm}

\subsection{Weight Pruning} \label{appe:pruning}

Directly assigning weights to all prototypes within a class can lead to excessive similarity between prototypes, especially for difficult samples. This could blur decision boundaries and reduce the model's ability to correctly classify hard-to-distinguish samples.

To address this, we apply a \textbf{top-$n$ pruning} strategy, which keeps only the most relevant prototypes for each sample. The max weights are retained, and the rest are pruned as follows:
\begin{equation}
    W_{i,k}^{(c)} = {1}[W_{i,k}^{(c)}>\beta]\ast W_{i,k}^{(c)}, 
\end{equation}
where $\beta$ is the threshold corresponding to the top-$n$ weight, and ${1}[W_{i,k}^{(c)}>\beta]$ is an indicator function that retains only the weights for the top-$n$ prototypes. This pruning mechanism ensures that the prototypes remain distinct and that the decision space for each class is well-defined, allowing for improved classification performance. By applying this attention-based weight calculation and top-$n$ pruning, the model ensures a more accurate and robust matching of samples to prototypes, enhancing classification, especially in OOD scenarios.


\subsection{Complexity Analysis}
The time complexity of \ourmethod is $
\mathcal{O}(|E|d+|V|d^{2})$, where $|V|$ denotes the number of nodes and $|E|$ denotes the number of edges, $d$ is the dimension of the final representation. Specifically, for $\mathrm{GNN}_{S}$ and $\mathrm{GNN}_{E}$, their complexity is denoted as $
\mathcal{O}(|E|d+|V|d^{2})$. The complexity of the projector is $
\mathcal{O}(|V|d^{2})$, while the complexities of calculating weights and updating prototypes are $
\mathcal{O}(|V||K|d)$ where $K$ is the number of prototypes. The complexity of computing the final classification probability also is $
\mathcal{O}(|V|Kd)$. Since $K$ is a very small constant, we can ignore $\mathcal{O}(|V|Kd)$, resulting in a final complexity of $
\mathcal{O}(|E|d+|V|d^{2})$. Theoretically, the time complexity of \ourmethod is on par with the existing methods.

% \begin{figure}[htb]
%  \centering
% \subfigure[CIGA] { 
% \includegraphics[width=0.27\textwidth]{4_exp/imold.pdf} \label{t1}
% }
% \subfigure[Ours] { 
% \includegraphics[width=0.27\textwidth]{4_exp/my.pdf} \label{t2}
% }
% \caption{t-SNE visualization on HIV-Scaffold.}
% \label{fig:tsne}
% \end{figure}
\section{Experimental Details}\label{appe:exp}

\subsection{Datasets}\label{appe:data}
\textbf{Overview of the Dataset.} In this work, we use 11 publicly benchmark datasets, 5 of them are from GOOD~\citep{good} benchmark. They are the combination of 3 datasets (GOOD-HIV, GOOD-Motif and GOOD-CMNIST) with different distribution shift (scaffold, size, basis, color). The rest 6 datasets are from DrugOOD~\citep{drugood} benchmark, including IC50-Assay, IC50-Scaffold, IC50-Size, EC50-Assay, EC50-Scaffold, and EC50-Size. The prefix denotes the measurement and the suffix denotes the distribution-splitting strategies. We use the default dataset split proposed in each benchmark.  Statistics of each dataset are in Table \ref{data_st}.

\begin{table}[t]
\renewcommand{\arraystretch}{1.5}
\setlength\tabcolsep{3pt}
\centering
\caption{Dateset statistics.}
\resizebox{\columnwidth}{!}{%
\begin{tabular}{cccccccc}
\toprule
\multicolumn{3}{c|}{Dataset}                                                                            & Task                       & Metric  & Train & Val   & Test  \\ \midrule
\multicolumn{1}{c|}{\multirow{5}{*}{GOOD}}    & \multirow{2}{*}{HIV}   & \multicolumn{1}{c|}{scaffold} & Binary Classification      & ROC-AUC & 24682 & 4133  & 4108  \\
\multicolumn{1}{c|}{}                         &                        & \multicolumn{1}{c|}{size}     & Binary Classification      & ROC-AUC & 26169 & 4112  & 3961  \\ \cline{2-8} 
\multicolumn{1}{c|}{}                         & \multirow{2}{*}{Motif} & \multicolumn{1}{c|}{basis}    & Multi-label Classification & ACC     & 18000 & 3000  & 3000  \\
\multicolumn{1}{c|}{}                         &                        & \multicolumn{1}{c|}{size}     & Multi-label Classification & ACC     & 18000 & 3000  & 3000  \\ \cline{2-8} 
\multicolumn{1}{c|}{}                         & CMNIST                 & \multicolumn{1}{c|}{color}    & Multi-label Classification & ACC     & 42000 & 7000  & 7000  \\ \midrule
\multicolumn{1}{c|}{\multirow{6}{*}{DrugOOD}} & \multirow{3}{*}{IC50}  & \multicolumn{1}{c|}{assay}    & Binary Classification      & ROC-AUC & 34953 & 19475 & 19463 \\
\multicolumn{1}{c|}{}                         &                        & \multicolumn{1}{c|}{scaffold} & Binary Classification      & ROC-AUC & 22025 & 19478 & 19480 \\
\multicolumn{1}{c|}{}                         &                        & \multicolumn{1}{c|}{size}     & Binary Classification      & ROC-AUC & 37497 & 17987 & 16761 \\ \cline{2-8} 
\multicolumn{1}{c|}{}                         & \multirow{3}{*}{EC50}  & \multicolumn{1}{c|}{assay}    & Binary Classification      & ROC-AUC & 4978  & 2761  & 2725  \\
\multicolumn{1}{c|}{}                         &                        & \multicolumn{1}{c|}{scaffold} & Binary Classification      & ROC-AUC & 2743  & 2723  & 2762  \\
\multicolumn{1}{c|}{}                         &                        & \multicolumn{1}{c|}{size}     & Binary Classification      & ROC-AUC & 5189  & 2495  & 2505  \\ \bottomrule
\end{tabular}%
}\label{data_st}
\vspace{-4mm}
\end{table}

\textbf{Distribution split.} In this work, we investigate various types of distribution-splitting strategies for different datasets.
\begin{itemize}
    \item \textbf{Scaffold.}  Molecular scaffold is the core structure of a molecule that supports its overall composition, but it only exhibits specific properties when combined with particular functional groups. 
    \item \textbf{Size.} The size of a graph refers to the total number of nodes, and it is also implicitly related to the graph's structural properties.     
    \item \textbf{Assay.} The assay is an experimental technique used to examine or determine molecular characteristics. Due to differences in assay conditions and targets, activity values measured by different assays can vary significantly. 
    \item \textbf{Basis.} The generation of a motif involves combining a base graph (wheel, tree, ladder, star, and path) with a motif (house, cycle, and crane), but only the motif is directly associated with the label. 
    \item \textbf{Color.} CMNIST is a graph dataset constructed from handwritten digit images. Following previous research, we declare a distribution shift when the color of the handwritten digits changes.
\end{itemize}
\subsection{Baselines}\label{appe:baseline}
In our experiments, the methods we compared can be divided into two categories, one is ERM and traditional OOD generalization methods:
\begin{itemize}
    \item \textbf{ERM} is a standard learning approach that minimizes the average training error, assuming the training and test data come from the same distribution.
    \item \textbf{IRM}~\citep{arjovsky2019invariant} aims to learn representations that remain invariant across different environments, by minimizing the maximum error over all environments.
    \item \textbf{VREx}~\citep{krueger2021out} propose a penalty on the variance of training risks which can providing more robustness to changes in the input distribution.  
    \item \textbf{Coral}~\citep{coral} utilize a nonlinear transformation to align the second-order statistical features of the source and target domain distributions
\end{itemize}
Another class of methods is specifically designed for Graph OOD generalization:
\begin{itemize}
    \item \textbf{MoleOOD}~\citep{yang2022learning} learn the environment invariant molecular substructure by a environment inference model and a molecular decomposing model.
    \item \textbf{CIGA}~\citep{chen2022learning} proposes an optimization objective based on mutual information to ensure the learning of invariant subgraphs that are not affected by the environment.
    \item \textbf{GIL}~\citep{li2022learning} performs environment identification and invariant risk loss optimization by separating the invariant subgraph and the environment subgraph.
    \item \textbf{GERA}~\citep{liu2022graph} performs data augmentation by replacing the input graph with the environment subgraph to improve the generalization ability of the model
    \item \textbf{IGM}~\citep{jia2024graph} performs data augmentation by simultaneously performing a hybrid strategy of invariant subgraphs and environment subgraphs.
    \item \textbf{DIR}~\citep{wu2022discovering} identifies causal relation between input graphs and labels by performing counterfactual interventions.
    \item \textbf{DisC}~\citep{fan2022debiasing} learns causal and bias representations through a causal and disentangling based learning strategy separately.
    \item \textbf{GSAT}~\citep{miao2022interpretable} learns the interpretable label-relevant subgraph through an stochasticity attention mechanism.
    \item \textbf{CAL}~\cite{sui2022causal} proposes a causal attention learning strategy to ensure that GNNs learn effective representations instead of optimizing loss through shortcuts.
    \item  \textbf{iMoLD}~\citep{zhuang2023learning} designs two GNNs to directly extract causal features from the encoded graph representation.
    \item { \textbf{GALA}~\citep{gala} designs designs a new loss function to ensure graph OOD generalization without environmental information as much as possible.}
    \item  {\textbf{EQuAD}~\citep{Equad} learns how to effectively remove spurious features by optimizing the self-supervised informax function.}
\end{itemize}
\subsection{Implementation Details}\label{appe:hyperparam}
\textbf{Baselines.} For all traditional OOD methods, we conduct experiments on different datasets using the code provided by GOOD~\citep{good} and DrugOOD~\citep{drugood} benchmark. For graph OOD generalization methods with public code, we perform experiments in the same environments as our method and employ grid search to select hyper-parameters, ensuring fairness in the results.

\begin{table}[t] 
\renewcommand{\arraystretch}{0.9}
\setlength\tabcolsep{2pt}
\centering
\caption{Hyper-parameter configuration.}
% \vspace{-4mm}
% \resizebox{\columnwidth}{!}{%
\begin{tabular}{cccccccc}
\toprule
                                          & \multicolumn{1}{l}{}       &          & $\mathrm{proj\_dim}$ & $\mathrm{att\_dim}$ & $K$ & $lr$ & $\beta$ \\ \midrule
\multirow{6}{*}{DrugOOD}                  & \multirow{3}{*}{IC50}      & Assay    & 300      & 128     & 3   & 0.001     & 0.1     \\
                                          &                            & Scaffold & 300      & 128     & 3   & 0.001      & 0.1     \\
                                          &                            & Size     & 300      & 128     & 3   & 0.001     & 0.1     \\ \cline{2-8} 
                                          & \multirow{3}{*}{EC50}      & Assay    & 300      & 128     & 3   & 0.001      & 0.1     \\
                                          &                            & Scaffold & 300      & 128     & 3   & 0.001      & 0.1     \\
                                          &                            & Size     & 300      & 128     & 3   & 0.001     & 0.1     \\ \midrule
\multicolumn{1}{c}{\multirow{5}{*}{GOOD}} & \multirow{2}{*}{HIV}       & Scaffold & 300      & 128     & 3   & 0.01     & 0.1     \\
\multicolumn{1}{c}{}                      &                            & Size     & 300      & 128     & 3   & 0.01     & 0.1     \\ \cline{2-8} 
\multicolumn{1}{c}{}                      & \multirow{2}{*}{Motif}     & Basis    & 256     & 128     & 6   & 0.01    & 0.2     \\
\multicolumn{1}{c}{}                      &                            & Size     & 256      & 128     & 6   & 0.01    & 0.2     \\ \cline{2-8} 
\multicolumn{1}{c}{}                      & \multicolumn{1}{l}{CMNIST} & Color    & 256      & 128     & 5   & 0.01    & 0.2     \\ \bottomrule
\end{tabular}\label{hyperpater}
% }
% \vspace{-7mm}
\end{table}
\textbf{Our method.} We implement our proposed \ourmethod under the Pytorch~\citep{pytorch} and PyG~\citep{pyg}. For all datasets containing molecular graphs (all datasets from DrugOOD and GOODHIV), we fix the learning rate to $0.001$ and select the hyper-parameters by ranging the $\mathrm{proj\_dim}$ from $\{100,200,300\}$, $\mathrm{att\_dim}$ from $\{64,128,256\}$, $K$ from $\{2,3,4,5\}$ and $\beta$ from $\{0.01,0.1,0.2\}$. For the other datasets, we fix the learning rate to $0.01$ and select the hyper-parameters by ranging the $\mathrm{proj\_dim}$ from $\{64,128,256\}$, $\mathrm{att\_dim}$ from $\{64,128,256\}$, $K$ from $\{3,4,5,6\}$ and $\beta$ from $\{0.01,0.1,0.2\}$. For the top-$n$ pruning, we force $n$ to be half of $K$. We conduct a grid search to select hyper-parameters and refer to Table \ref{hyperpater} for the detailed configuration. For all experiments, we fix the number of epochs to 200 and run the experiment five times with different seeds, select the model to run on the test set based on its performance on validation, and report the mean and standard deviation. 
% Please add the following required packages to your document preamble:
% \usepackage{multirow}
% \usepackage{graphicx}
% Please add the following required packages to your document preamble:
% \usepackage{multirow}
% \usepackage{graphicx}

% Please add the following required packages to your document preamble:
% \usepackage{graphicx}

\subsection{Supplemental Results}\label{appe:more_results}
We report the complete experimental results with means and standard deviations in Tables \ref{tab:main_good} and \ref{tab:main_drugood}
\begin{table*}[h!] 

\caption{Performance comparison in terms of average accuracy (standard deviation) on GOOD benchmark.}
\centering
\resizebox{0.7\textwidth}{!}{
\begin{tabular}{l|cc|c|cc}
\toprule
\multirow{2}{*}{\textbf{Method}}           & \multicolumn{2}{c|}{\textbf{GOOD-Motif}} & \multicolumn{1}{c|}{\textbf{GOOD-CMNIST}} & \multicolumn{2}{c}{\textbf{GOOD-HIV}} \\
           & \textbf{basis}         & \textbf{size}          & \textbf{color}         & \textbf{scaffold}      & \textbf{size} \\
\midrule
ERM        & 60.93 (2.11)  & 46.63 (7.12)   & 26.64 (2.37)   & 69.55 (2.39)   & 59.19 (2.29)\\
IRM        & 64.94 (4.85)   & {54.52 (3.27)}   & 29.63 (2.06)   & 70.17 (2.78)   & 59.94 (1.59)\\
VREX       & 61.59 (6.58)   & {55.85 (9.42)}   & 27.13 (2.90)   & 69.34 (3.54)   & 58.49 (2.28)\\
Coral      & 61.95 (4.36)  & 55.80 (4.05)   & 29.21 (6.87)   & 70.69 (2.25)   & 59.39 (2.90)\\
\midrule
MoleOOD      & -  & -   & -  & 69.39 (3.43)   & 58.63 (1.78)\\
CIGA       & 67.81 (2.42)   & 51.87 (5.15)   & 25.06 (3.07)   & 69.40 (1.97)   & {61.81 (1.68)}\\
GIL      & 65.30 (3.02)   & 54.65 (2.09)   & 31.82 (4.24)   & 68.59 (2.11)   & 60.97 (2.88)\\
GREA      & 59.91 (2.74)   & 47.36 (3.82)   & 22.12 (5.07)   & {71.98 (2.87)}   & 60.11 (1.07)\\
IGM    & {74.69 (8.51) }  & 52.01 (5.87)   & 33.95 (4.16)   & {71.36 (2.87)}   & 62.54 (2.88)\\
\midrule
DIR        & 64.39 (2.02)   & 43.11 (2.78)   & 22.53 (2.56)   & 68.44 (2.51)   & 57.67 (3.75)\\
DisC       & 65.08 (5.06)  & 42.23 (4.20)   & 23.53 (0.67)   & 58.85 (7.26)  & 49.33 (3.84)         \\ 
\midrule
GSAT       & 62.27 (8.79)   & 50.03 (5.71)   & {35.02 (2.78)}   & 70.07 (1.76)   & 60.73 (2.39)\\
CAL       & {68.01 (3.27)}   & 47.23 (3.01)   & 27.15 (5.66)   & 69.12 (1.10)   & 59.34 (2.14)\\
GALA       & {66.91 (2.77)}   & 45.39 (5.84)   & {38.95 (2.97)}   & 69.12 (1.10)   & 59.34 (2.14)\\
iMoLD      & - & - & -  & \underline{72.05 (2.16)}   & {62.86 (2.34)}\\
\textcolor{black}{GALA} & \textcolor{black}{72.97 (4.28)} & \textcolor{black}{\textbf{60.82 (0.51)}} & \textcolor{black}{\underline{40.62 (2.11)}} & \textcolor{black}{71.22 (1.93)}& \textcolor{black}{\underline{65.29 (0.72)}}\\
\textcolor{black}{EQuAD}      &\textcolor{black}{\underline{75.46 (4.35)}}   & \textcolor{black}{55.10 (2.91)}   & \textcolor{black}{{40.29 (3.95)}}   & \textcolor{black}{71.49 (0.67)}    &\textcolor{black}{{64.09 (1.08)}}     \\
\midrule
\ourmethod     & \textbf{76.23 (4.89)} & \underline{58.43 (3.15)} & \textbf{41.29 (3.85)} & \textbf{73.94 (1.77)} & \textbf{66.84 (1.09)}\\ \bottomrule
\end{tabular}\label{tab:main_good}}
\end{table*}
\begin{table*}[h!] 
\caption{Performance comparison in terms of average accuracy (standard deviation) on DrugOOD benchmark.}
\centering
\resizebox{1\textwidth}{!}{
\begin{tabular}{l|ccc|ccc}
\toprule
\multirow{2}{*}{\textbf{Method}} & \multicolumn{3}{c|}{\textbf{DrugOOD-IC50}} & \multicolumn{3}{c}{\textbf{DrugOOD-EC50}} \\ 
 & \textbf{assay} & \textbf{scaffold} & \textbf{size} & \textbf{assay} & \textbf{scaffold} & \textbf{size} \\ \midrule
ERM & 70.61 (0.75) & 67.54 (0.42) & 66.10 (0.31) & 65.27 (2.39) & 65.02 (1.10) & 65.17 (0.32) \\ 
IRM & 71.15 (0.57) & 67.22 (0.62) & \underline{67.58 (0.58)} & 67.77 (2.71) & 63.86 (1.36) & 59.19 (0.83) \\ 
VREx & 70.98 (0.77) & 68.02 (0.43) & 65.67 (0.19) & 69.84 (1.88) & 62.31 (0.96) & {65.89 (0.83)} \\ 
Coral & 71.28 (0.91) & 68.36 (0.61) & 67.53 (0.32) & 72.08 (2.80) & 64.83 (1.64) & 58.47 (0.43) \\ \midrule
MoleOOD & 71.62 (0.50) & 68.58 (1.14) & 67.22 (0.96) & 72.69 (4.16) & 65.78 (1.47) & 64.11 (1.04) \\ 
CIGA & \underline{71.86 (1.37)} & \textbf{69.14 (0.70)} & 66.99 (1.40) & 69.15 (5.79) & \underline{67.32 (1.35)} & 65.60 (0.82) \\ 
GIL & 70.66 (1.75) & 67.81 (1.03) & 66.23 (1.98) & 70.25 (5.79) & 63.95 (1.17) & 64.91 (0.76) \\ 
GREA & 70.23 (1.17) & 67.20 (0.77) & 66.09 (0.56) & 74.17 (1.47) & 65.84 (1.35) & 61.11 (0.46) \\ 
IGM & 68.05 (1.84) & 63.16 (3.29) & 63.89 (2.97) & 76.28 (4.43) & 67.57 (0.62) & 60.98 (1.05) \\ \midrule
DIR & 69.84 (1.41) & 66.33 (0.65) & 62.92 (1.89) & 65.81 (2.93) & 63.76 (3.22) & 61.56 (4.23) \\ 
DisC & 61.40 (2.56) & 62.70 (2.11) & 64.43 (0.60) & 63.71 (5.56) & 60.57 (2.27) & 57.38 (2.48) \\ \midrule
GSAT & 70.59 (0.43) & 66.94 (1.43) & 64.53 (0.51) & 73.82 (2.62) & 62.65 (1.79) & 62.65 (1.79) \\ 
CAL & 70.09 (1.03) & 65.90 (1.04) & 64.42 (0.50) & {74.54 (1.48)} & 65.19 (0.87) & 61.21 (1.76) 
\\
iMoLD & 71.77 (0.54) & 67.94 (0.59) & 66.29 (0.74) & {77.23 (1.72)} & 66.95 (1.26) & \underline{67.18 (0.86)} 
\\ 

\textcolor{blue}{GALA} & \textcolor{blue}{70.58 (2.63)} & \textcolor{blue}{66.35 (0.86)} & \textcolor{blue}{66.54 (0.93)}  & \textcolor{blue}{{77.24 (2.17)}} & \textcolor{blue}{66.98 (0.84)} & \textcolor{blue}{63.71 (1.17)}  \\
\textcolor{blue}{EQuAD} &\textcolor{blue}{71.57 (0.95)} & \textcolor{blue}{67.74 (0.57)} & \textcolor{blue}{67.54 (0.27)} &\textcolor{blue}{\underline{77.64 (0.63)}} &\textcolor{blue}{65.73 (0.17)} &\textcolor{blue}{64.39 (0.67)}  \\
\midrule
\ourmethod & \textbf{72.96 (1.21)} & \underline{68.62 (0.78)} & \textbf{68.06 (0.55)} & \textbf{78.08 (0.54)} & \textbf{68.34 (0.61)} & \textbf{68.11 (0.58)} \\ \bottomrule
\end{tabular}
}\label{tab:main_drugood}
\end{table*}.
\end{document}
\endinput
%%