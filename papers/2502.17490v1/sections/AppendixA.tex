\section{Appendix}
%=========================================
\subsection{Broyden's method}
\label{app:broyden}
%
Algorithm~\ref{algo:broyden} presents Broyden's method \cite{broyden1965} to iteratively solve for the inelastic stretch, $\bm{U}_i$, in a derivative-free manner. 
To directly compute the inverse of the Jacobian, $\mathbf{B}$, the Sherman-Morrison formula \cite{sherman1950} is exploited.
%
\begin{algorithm}[h]
\caption{Broyden's method to determine $\bm{U}_i$ in Voigt's notation $(\bullet)'$}
\label{algo:broyden}
\begin{algorithmic}[1]
\State \textbf{Input:} tolerance \( \epsilon \), weights $\mathbf{W}$, biases $\mathbf{b}$, $\Delta t$, $\bm{C}_{n+1}$, $\bm{U}_{i_n}$
\State \textbf{Output:} $\bm{U}_i$
\State Set \( \bm{U}_i' \gets \bm{U}_{i_n}'\)
\State Initialize residual, $\mathbf{r}_0$, according to Equation~\eqref{eq:implicit_integration} in Voigt's notation
\State Set \( \mathbf{B} \gets \mathbf{I}_{6\times 6}\)
\For{$m = 1$ to $50$}
    \State $\mathbf{s} \gets -  \mathbf{B}\, \mathbf{r}_{m-1}$ 
    \State $\bm{U}_{i_m}' \gets \bm{U}_{i_{m-1}}' + \mathbf{s}$
    \State Compute new residual, $\mathbf{r}_m$, with updated inelastic stretch $\bm{U}_{i_m}$
    \State $\mathbf{y} \gets \mathbf{r}_m - \mathbf{r}_{m-1}$
    \State $\mathbf{B} \gets \mathbf{B} + \frac{\mathbf{s} - \mathbf{B}\,\mathbf{y}}{\mathbf{s}^T\mathbf{B}\,\mathbf{y}}\, \mathbf{s}^T\mathbf{B}$
    \If{$|\mathbf{r}_m| < \epsilon$}
        \State \textbf{break}
    \EndIf   
\EndFor
\State \textbf{return} $\bm{U}_i$
\end{algorithmic}
\end{algorithm}
%
\FloatBarrier
%
%=========================================
\subsection{Additional stress invariants}
\label{app:stress_invars}
%
We can find a relation between the additional stress invariants $I_2^{\bar{\bm{\Sigma}}}$ and $I_3^{\bar{\bm{\Sigma}}}$ and our three basic invariants as follows
%
\begin{align}
    I_2^{\bar{\bm{\Sigma}}} &= \frac{\left(I_1^{\bar{\bm{\Sigma}}}\right)^2}{6} + J_2^{\bar{\bm{\Sigma}}} \\
    I_3^{\bar{\bm{\Sigma}}} &= \frac{\left(I_1^{\bar{\bm{\Sigma}}}\right)^3}{27} + \frac{2}{3}\, I_1^{\bar{\bm{\Sigma}}}\, J_2^{\bar{\bm{\Sigma}}} + J_3^{\bar{\bm{\Sigma}}}.
\end{align}
%
While $I_2^{\bar{\bm{\Sigma}}}$ is included in the general architecture, we are unable to obtain $I_3^{\bar{\bm{\Sigma}}}$ since neither the activation function $(\bullet)^3$ nor the multiplication $I_1^{\bar{\bm{\Sigma}}}\, J_2^{\bar{\bm{\Sigma}}}$ is present in the general network architecture.
We can express the dual potential as $\omega^*=\tilde{\omega}^*\left(I_1^{\bar{\bm{\Sigma}}}, \sqrt{J_2^{\bar{\bm{\Sigma}}}}, \sqrt[3]{J_3^{\bar{\bm{\Sigma}}}}, \sqrt{I_2^{\bar{\bm{\Sigma}}}(I_1^{\bar{\bm{\Sigma}}},J_2^{\bar{\bm{\Sigma}}} )}, \sqrt[3]{I_3^{\bar{\bm{\Sigma}}}(I_1^{\bar{\bm{\Sigma}}},J_2^{\bar{\bm{\Sigma}}},J_3^{\bar{\bm{\Sigma}}} )} \right)$.
With these relations at hand, we find
%
\begin{align}
    \frac{\partial I_2^{\bar{\bm{\Sigma}}}}{\partial \bar{\bm{\Sigma}}} : \bar{\bm{\Sigma}} &= 2\,I_2^{\bar{\bm{\Sigma}}} \\
    \frac{\partial I_3^{\bar{\bm{\Sigma}}}}{\partial \bar{\bm{\Sigma}}} : \bar{\bm{\Sigma}} &= 3\,I_3^{\bar{\bm{\Sigma}}}
\end{align}
%
which, in analogy to Equation~\eqref{eq:reduced_dissipation_vector}, proves thermodynamic consistency if the specific network (Figure~\ref{fig:NN_potential}) is convex, zero-valued, and non-negative with respect to its five inputs.