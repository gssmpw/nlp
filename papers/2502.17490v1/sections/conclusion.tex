\section{Conclusion}
\label{sec:conclusion}
%
We took a further step towards the understanding and discovery of general inelasticity at finite strains through a rigorous mathematical formulation that ensures a convex, non-negative, and zero-valued potential. 
This enabled the design of a scalable, interpretable network architecture akin to traditional feed-forward networks, encompassing various classical potentials in continuum mechanics. 
Our thermodynamically consistent approach inherently satisfies the dissipation inequality, ensuring predictions beyond training remain aligned with fundamental physical principles.

We introduced a regularization scheme inspired by Constitutive Artificial Neural Networks (CANNs), employing lasso and ridge techniques for sparse representation. 
This allowed the iCANN to autonomously determine the degree of inelasticity and reducing to a purely elastic CANN when appropriate. 
Gradient clipping prevented excessive weight growth in the potential network, avoiding unbounded stress responses and training instabilities.

Our studies successfully discovered multiple constitutive material models with high accuracy. 
A staggered discovery scheme prevented potential network weights from collapsing to zero when inelastic dissipation was low in the training data. 
The combination of our network architecture and regularization framework enabled stable identification of visco-elastic behavior, even with noisy data.

However, we encountered several limitations. 
The implicit time integration scheme was unstable when using Broyden's method to iteratively solve the evolution equation. 
Critically, while our network accurately captured training data for an experimentally measured polymer, it failed to predict accurately beyond the training regime, despite exploring various data set sizes. 
We attribute this to insufficient data richness, consistent with our findings for artificial data sets.

Balancing network complexity and data richness is a key challenge for meaningful weight discovery. 
The success of neural networks in approximating material behaviors hinges on sufficiently informative experimental data. 
This underscores the need for collaboration between experimental mechanics and computational modeling to design new experimental setups and discovery strategies. 
Future experiments may aim to maximize information content while minimizing test specimens and addressing uncertainties in material properties.

For our approach, this implies extending the network architecture to boundary value problems, leveraging displacement field information to enhance material discovery.