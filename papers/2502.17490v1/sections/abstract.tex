We present a methodology for designing a generalized dual potential, or pseudo potential, for inelastic Constitutive Artificial Neural Networks (iCANNs). 
This potential, expressed in terms of stress invariants, inherently satisfies thermodynamic consistency for large deformations. 
In comparison to our previous work, the new potential captures a broader spectrum of material behaviors, including pressure-sensitive inelasticity.

To this end, we revisit the underlying thermodynamic framework of iCANNs for finite strain inelasticity and derive conditions for constructing a convex, zero-valued, and non-negative dual potential. 
To embed these principles in a neural network, we detail the architecture's design, ensuring a priori compliance with thermodynamics.

To evaluate the proposed architecture, we study its performance and limitations discovering visco-elastic material behavior, though the method is not limited to visco-elasticity.
In this context, we investigate different aspects in the strategy of discovering inelastic materials.
Our results indicate that the novel architecture robustly discovers interpretable models and parameters, while autonomously revealing the degree of inelasticity.

The iCANN framework, implemented in JAX, is publicly accessible at \url{https://doi.org/10.5281/zenodo.14894687}.