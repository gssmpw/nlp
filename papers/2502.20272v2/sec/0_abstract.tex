
\begin{figure*}[htp]
    \centering
    \includegraphics[width=0.96\linewidth]{img/fig1.pdf}
    \vspace{-0.2cm}
    \caption{The top row illustrates the process of transforming images from the sRGB color space, via HSV, to the HVI color space. The bottom row presents the corresponding test results. The sRGB color space is known for its high color sensitivity, often causing color distortions in test images. By decoupling brightness and color to obtain the HSV color space, the illumination enhancement appears normalized. However, this transformation introduces varying levels of red discontinuities and black regions, which subsequently cause artifacts in the enhanced images. Introducing polarization to HSV ensures continuity in the red regions. Introducing a learnable intensity function $\mathbf{C}_k$, on the other hand, helps collapse the black regions, yielding the HVI color space with optimal image enhancement.}
    \vspace{-0.2cm}
    \label{fig:1}
\end{figure*}

\begin{abstract}
Low-Light Image Enhancement (LLIE) is a crucial computer vision task that aims to restore detailed visual information from corrupted low-light images.
Many existing LLIE methods are based on standard RGB (sRGB) space, which often produce color bias and brightness artifacts due to inherent high color sensitivity in sRGB.
While converting the images using Hue, Saturation and Value (HSV) color space helps resolve the brightness issue, it introduces significant red and black noise artifacts.
To address this issue, we propose a new color space for LLIE, namely Horizontal/Vertical-Intensity (\textbf{HVI}), defined by polarized HS maps and learnable intensity. The former enforces small distances for red coordinates to remove the red artifacts, while the latter compresses the low-light regions to remove the black artifacts. To fully leverage the chromatic and intensity information, 
% which not only decouples brightness and color from RGB channels to mitigate the instability during enhancement, but also solve the red discontinuity and black plane noise issues caused by HSV.
% Furthermore, we design
a novel Color and Intensity Decoupling Network (\textbf{CIDNet}) is further introduced 
% optimize HVI, in which two network branches are devised to respectively model decoupled color and brightness information in the polarized HS via a HV branch and an Intensity branch. This helps establish the
to learn accurate photometric mapping function under different lighting conditions in the HVI space. 
% This yields an optimized noise-free HVI color space that largely enhances the brightness of low-light images while preserving their natural colors.
% , thereby effectively removing the red and black noise artifacts.
% with a lightweight, computationally-efficient network.
% with two branches dedicated to processing the decoupled image brightness and color in the HVI space.
% Finally,
Comprehensive results from benchmark and ablation experiments show that the proposed HVI color space with CIDNet outperforms the state-of-the-art methods on 10 datasets. The code is available at \href{https://github.com/Fediory/HVI-CIDNet}{https://github.com/Fediory/HVI-CIDNet}.
\end{abstract}