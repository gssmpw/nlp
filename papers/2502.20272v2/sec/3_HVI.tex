\section{HVI Color Space}
The HVI space is built upon the HSV color space, which is proposed to address the color space noise issues arising from the HSV space. The key intuition in HVI is that the restored images should have good perceptual quality for respective colors, \ie, similar colors have small Euclidean distances. Below we introduce the HVI transformation in detail, where the HSV color space is first applied to decouple the brightness and color information of input images, which could cause color space noise (\eg, red discontinuity and black plane). We then introduce our proposed polarized HS operations and learnable intensity collapse function in HVI to effectively address these issues.

\subsection{Color Space Noises in HSV}

\textbf{Intensity Map.} In the task of LLIE, one crucial aspect is accurately estimating the illumination intensity map of the scene from a sRGB input image.
Previous methods \cite{RetinexNet, PairLIE, KinD} largely rely on the Retinex theory \cite{land1977retinex}, using deep learning to directly generate the corresponding normal-light map. 
While this approach aligns with statistical principles, it often struggles to fit physical laws and human perception \cite{gevers2012color}, resulting in limited generalizability \cite{wang2024zero}. 
Therefore, we instead refer to the Max-RGB theory \cite{land1977retinex} to estimate the intensity map, \textit{a.k.a.}, Value in HSV, rather than using neural networks to generate it.
According to Max-RGB, for each individual pixel $x$, we can estimate the intensity map of an image $\mathbf{I}_{max} \in \mathbb{R}^{\mathrm{H\times W}} $ as follows:
\begin{equation}
\mathbf{I}_{max}(x) =\max_{\mathbf{c}\in \{R,G,B\}} (\mathbf{I_{c}}(x)).
\label{eq:1}
\end{equation}


The intensity map then goes through the sRGB-HSV transformation that can lead to various red and black noises. We introduce these separately as follows.

\textbf{Hue/Saturation Plane.}
Real-world low-light images often contain significant noise, making its identification and removal a key challenge in LLIE.
Recent studies \cite{RetinexNet,yi2023diff} indicate that the noise in low-light images is a primary cause to shifts in Hue and Saturation, \textit{a.k.a.} a general case of Reintex theory \cite{wang2024zero}, while having minimal impact on light intensity.
Therefore, decoupling the sRGB color space, known for its high color sensitivity, can be advantageous for the LLIE task. 
By leveraging pixel-based photometric invariance \cite{gevers2012color} and dichromatic reflection modeling \cite{shafer1985using}, sRGB can be decoupled into illuminance and chromatic components, yielding the HSV (Hue/Saturation-Value) color space. 
In this representation, the Value map ($\mathbf{V}$) component corresponds to light intensity map ($\mathbf{V}=\mathbf{I}_{max}$), while the HS plane forms a chromaticity plane independent of illuminance constraints. Specifically, the transformation of sRGB image to Saturation map ($\mathbf{S}$) is defined as follows
\begin{equation}
\mathbf{s}  = \begin{cases}
0, &  \mathbf{I}_{max}   = \mathbf{0} \\
\frac{\Delta}{\mathbf{I}_{max} }, &  \mathbf{I}_{max}  \ne \mathbf{0} \\
\end{cases} \\
,
\end{equation}
where $\Delta=\mathbf{I}_{max}-min(\mathbf{I}_c)$ and $\mathbf{s}$ is any pixel in $\mathbf{S}$. The Hue map ($\mathbf{H}$) is formulated as
\begin{equation}
\mathbf{h}= \begin{cases}
0, &\text{if } \mathbf{s} = 0 \\
\frac{\mathbf{I_{G}} - \mathbf{I_{B}}}{\Delta} \mod{6},  &\text{if } \mathbf{I}_{max}  = \mathbf{I_{R}} \\
2+\frac{\mathbf{I_{B}} - \mathbf{I_{R}}}{\Delta},  &\text{if } \mathbf{I}_{max}   = \mathbf{I_{G}} \\
4+\frac{\mathbf{I_{R}} - \mathbf{I_{G}}}{\Delta},  &\text{if } \mathbf{I}_{max}   = \mathbf{I_{B}}
\end{cases}
,
\label{eq:hue}
\end{equation}
where $\mathbf{h}$ is any pixel in $\mathbf{H}$. 

\textbf{Color Space Noises.} 
Converting an sRGB image to HSV space effectively decouples brightness from color, enabling more accurate color denoising and more natural illuminance recovery. 
However, this transformation also amplifies noise in the red and dark regions \cite{gevers2012color}, which are critical for LLIE tasks. 
As illustrated in Fig. \ref{fig:1} (b), enhancing the image brightness within the HSV color space yields a more balanced brightness level. 
However, excessive noise in the red discontinuities and the black plane introduces significant artifacts, particularly in red and dark areas of output image, which greatly degrade the perceptual quality. 

To address this issue, we propose the HVI color space as follows, which effectively preserves the decoupling of brightness and color while minimizing these artifacts.

\subsection{Horizontal/Vertical Plane with Polarized HS and Collapsible Intensity}
\label{sec:HVI}
Our primary approach to addressing the color space noise issue is to ensure that more similar colors exhibit smaller Euclidean distances. 
Along the Hue axis, the color red appears identically at both $\mathbf{h}=0$ and $\mathbf{h}=6$, due to the modular arithmetic of Hue-axis, which splits the same color across two ends of the spectrum. 
In particular,
% This facilitates pattern recognition by the neural network, enabling more effective learning.
to address the red discontinuity issue, we apply polarization to the Hue axis ($\mathbf{h}$) at each pixel in $\mathbf{H}$, obtaining orthogonal $h= \cos (\frac{\pi \mathbf{h}}{3})$ and $v= \sin (\frac{\pi \mathbf{h}}{3})$.
% to be orthogonalized. 
When the Hue axis is polarized, it forms an angle within the orthogonalized $h-v$ plane, with $\mathbf{s}$ representing the distance from the origin.

For the black plane noise issue, we aim to collapse regions of low light intensity while preserving those with higher intensity. However, the optimal extent of collapse varies across different datasets and networks. 
Therefore, it is important to make this region adaptively collapsible through a learning process. 
To achieve this, we introduce an adaptive intensity collapse function $\mathbf{C}_k$ as follows
\begin{equation}
\mathbf{C}_k(x)=\sqrt[k]{\sin (\frac{ \pi \mathbf{I}_{max}(x) }{2} )+\mathcal{\varepsilon} },
\label{eq:2}
\end{equation}
where $k\in \mathbb{Q^+}$ is a trainable parameter to control the dark color point density, and a small  $\mathcal{\varepsilon}=1\times10^{-8}$ is used to avoid gradient explosion. 
Essentially, $\mathbf{C}_k$ serves a radius mapping function, with smaller $\mathbf{C}_k$ corresponding to smaller radius or lower intensity values. Thus, black points are clustered together as $\mathbf{C}_k$ decreases. We then formalize the Horizontal ($\mathbf{\hat{H}}$) map and Vertical ($\mathbf{\hat{V}}$) map as
\begin{equation}
\begin{split}
    \mathbf{\hat{H}} &= \mathbf{C}_k \odot  \mathbf{S}  \odot H,\\
    \mathbf{\hat{V}} &= \mathbf{C}_k \odot  \mathbf{S}  \odot V,
\end{split}
\label{eq:7}
\end{equation}
where $h \in H$, $v \in V$, and $\odot$ denotes the element-wise multiplication. $\mathbf{\hat{H}}$, $\mathbf{\hat{V}}$, and $\mathbf{I}_{max}$ can be concatenated to form an HVI image.

% , which can achieve the best result with lowest artifacts by training in such color space, as shown in Fig. \ref{fig:1}. 
% More designs that improve generalization are provided in the supplementary.
Thanks to these operations, the HVI space builds a strong color space that maintains the advantages of HSV while removing the red and black noises. More importantly, since HVI is trainable due to $k$ and $\mathbf{C}_k$, specifically designed neural networks, as discussed in the following, can be created to optimize LLIE upon HVI under various lighting conditions.