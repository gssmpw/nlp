
\section{Conclusion}
% In this paper, we address the red discontinuity and low-light noise issues in HSV through polarization and the $C_k$ formula, while preserving the brightness-color decoupling property, resulting in the HVI color space.
% Furthermore, we present the CIDNet to decouple image brightness and color and adapt to various illumination scales. 
% The dual-branch network, build upon the HVI color space, simultaneously processes brightness and color, aided by the plug-and-play symmetric HVI Transform module. 
% Our CIDNet obtained excellent results on all 10 datasets and validated the HVI color space.

In this work, we introduce the HVI color space and the CIDNet approach to address the color bias and brightness artifact issues that occur to current sRGB-based LLIE approaches. 
By encapsulating polarized HS maps and a learnable intensity component, HVI shows strong robustness to both issues. 
To further enhance LLIE, CIDNet is designed to model decoupled chromatic and intensity information in the HVI space for achieving precise photometric adjustments under varying lighting conditions. 
Experimental results on 10 datasets demonstrate that the HVI color space, combined with CIDNet, outperforms SOTA LLIE methods, establishing it as a robust solution for low-light enhancement.