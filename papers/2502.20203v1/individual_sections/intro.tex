Blockchains, pioneered by Bitcoin, are systems that support digital transactions in a completely decentralized manner. However, most blockchains have poor transaction throughput, a fundamental limitation that stems from their decentralized design (e.g., Bitcoin processes around ten transactions per second \cite{croman2016scaling}). This low throughput results in exorbitant transaction fees and hinders widespread adoption. To be a viable option in practice, blockchain throughput must scale significantly. \textit{Layer-two blockchain mechanisms} are tools that allow many transactions to take place outside of the underlying blockchain system, thereby increasing the system's throughput. See \cite{gudgeon2020sok} for a comprehensive survey of these methods. A payment channel network (PCN) is one such layer-two mechanism that is used in practice. Recent years have seen considerable research interest on PCNs, with a focus on improving their security as well as their efficiency. This paper focuses on their long-term transaction processing efficiency.

As the name suggests, a PCN is a network composed of multiple \textit{payment channels}. A payment channel is a special account that two parties jointly create by depositing some funds. Once a channel is established, the parties transact by exchanging digitally signed messages between themselves, without recording these transactions on the blockchain. Each message is simply a fresh agreement on how the channel's funds are split between the two parties. A PCN consists of many such payment channels operating together, allowing users who do not share a channel to direct their transactions through intermediaries. Thus, a PCN facilitates a much larger volume of transactions for the same amount of escrowed funds than what would be possible through standalone payment channels.

Although a payment channel can support indefinitely many transactions, the transactions are subject to some constraints. Every transaction through a payment channel moves funds from one party to the other. A steady flow of transactions in one direction depletes the funds available at one end of the channel. In the long term, a channel can only sustain a \textit{balanced flow} of transactions; \textit{i.e.}, the flows of money in each direction are equal. These balance constraints apply to each channel in a PCN, potentially limiting the PCN's ability to meet the transaction demands of its users. The central problem we address in this work is how the network can serve the transaction demands it receives to the best extent possible, while maintaining \textit{detailed balanced flows} (balanced flows on each channel). 

In this work, we propose a joint flow-control and routing protocol for PCNs, which we call the \textit{DEBT (DEtailed Balance Transaction) control protocol}. In this protocol, channels quote prices for serving transactions. The prices penalize flows that increase the degree of imbalance through a channel and incentivize flows in the opposite direction. Transacting users calculate the path price (the sum of channel prices along a path) for different paths between them. By selecting the path with the minimum path price to execute the transaction, the users perform \textit{routing}. In addition, they adjust the transaction amount as a function of the minimum price, thereby performing \textit{flow-control}. Channels update their prices over time based on the net flow of transactions through them. Our main contribution is to prove that given an arbitrary demand and an arbitrary PCN topology, the prices and the flows under this protocol converge to a value that is optimal in the following sense. The flows maximize the total utility of all users, subject to the detailed balance condition. Loosely, this translates to the network serving as many transactions as possible over the long run without relying on the main blockchain. 

We begin this paper by introducing the basic terminology for payment channels in Section \ref{sec:payment_channels}. Next, we review the literature on payment channel networks in Section \ref{sec:pcn_review} with a focus on price-based routing protocols. With this background, we describe the main contributions of this paper in Section \ref{sec:contribution}. In Section \ref{sec:PCN_model}, we present a discrete-time model of a payment channel network, specifying the order of events in a time slot, the nature of transaction requests, and the feasibility constraints imposed by the network. Section \ref{sec:debt_control_protocol} is devoted to the design of the DEBT control protocol. The protocol's convergence follows under some additional assumptions; this is presented in Section \ref{sec:convergence}. Simulation results of the protocol on some simple PCNs are shown in Section \ref{sec:pcn_simulations}. These examples illustrate how the protocol performs routing and flow-control by reacting to prices. We conclude this paper in Section \ref{sec:discussion} with a discussion of the implications of the assumptions we have made in our analysis and the potential practical impact of our work.