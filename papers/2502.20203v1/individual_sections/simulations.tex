\section{Simulation Results}\label{sec:pcn_simulations}
In this section, we show simulation results of the DEBT control protocol $\eqref{eq:algorithm}$ on three different instances of a PCN, illustrating how it performs dynamic routing and flow-control to ensure optimal flows in the long run.
\subsection{Dynamic Routing Example}\label{sec:routing_example}
Consider a payment channel network with three nodes, $A, B$, and $C$, and three channels, $A-B$, $B-C$, and $C-A$. Each of the channels have a capacity of $100$ and they are all initially perfectly balanced. The demand is a circulation with $a_{A,B} = a_{B,C} = a_{C,A} = 10$, which can be served entirely without any on-chain rebalancing. Consider, for example, the transactions from $A$ to $B$. There are two possible paths to route such transactions: a short path directly along the channel $A-B$, and a longer path via $C$. Similarly, the other two demands also can be served over two possible paths. Observe that routing transactions along the shortest path at all times is not sustainable; it skews the channel balances. The transactions must alternate between the short path and the long path.

\begin{figure*}[!t]
\centering
\subfloat[]{\includegraphics[width=3.5in]{figures/new_figures/cyclic_example.pdf}%
\label{fig:dynamic_routing_1}}
\hfil
\subfloat[]{\includegraphics[width=3.5in]{figures/new_figures/cyclic_example_smooth.pdf}%
\label{fig:dynamic_routing_2}}
\caption{Dynamic routing ensures perennial operation in a PCN with circulation demand: an illustration of the effect of the DEBT control protocol in the PCN described in Section \ref{sec:routing_example}, (a) without the regularizer and (b) with the regularizer.}
\label{fig:dynamic_routing}
\end{figure*}

\begin{figure*}[!t]
\centering
\subfloat[]{\includegraphics[width=3.5in]{figures/new_figures/deadlock_example.pdf}%
\label{fig:deadlock_1}}
\hfil
\subfloat[]{\includegraphics[width=3.5in]{figures/new_figures/deadlock_example_smooth.pdf}%
\label{fig:deadlock_2}}
\caption{Deadlock prevention via flow-control: an illustration of the effect of the DEBT control protocol in the PCN described in Section \ref{sec:flow_control_example}, (a) without the regularizer and (b) with the regularizer.}
\label{fig:deadlock}
\end{figure*}


Figure \ref{fig:dynamic_routing} illustrates the flows and the prices in this PCN over time under the DEBT Control protocol. 
(We only plot the flows from $A$ to $B$; the other flows are identical.) Figure \ref{fig:dynamic_routing_1} shows the case where the regularizer coefficient $\eta$ is zero, while Figure \ref{fig:dynamic_routing_2} shows the case where $\eta = 0.1$.  The stepsize $\gamma$ is set as $0.01$ in both Figures \ref{fig:dynamic_routing_1} and \ref{fig:dynamic_routing_2}. In Figure \ref{fig:dynamic_routing_1}, nodes route the entire flow along the cheaper path at all times. Ties are broken in favor of the shorter path. A flow in one direction raises prices in that direction, incentivizing nodes to take the other path in the next step. We see that the protocol follows a periodic pattern, choosing the shorter path twice in succession followed by choosing the longer path once. In Figure \ref{fig:dynamic_routing_2}, nodes split each transaction along both the paths, giving more weightage to the path with the lower price. Thus, adding the regularizer smoothens the flows as a function of the path prices. This smooth variation allows both the flows and the prices to converge. Finally, note that in both cases in Figure \ref{fig:dynamic_routing}, the long-term average of the flows satisfies the detailed balance condition.

\begin{figure*}[!t]
\centering
\subfloat[]{\includegraphics[width=3.5in]{figures/new_figures/flows_small_step.pdf}%
\label{fig:five_nodes_flow_small_steps}}
\hfil
\subfloat[]{\includegraphics[width=3.5in]{figures/new_figures/resets_small_step.pdf}%
\label{fig:five_nodes_resets_small_steps}}
\caption{Behavior of the five node PCN with a stepsize of $\gamma = 0.01$; (a) shows flows as a function of time; (b) shows the times at which channel  resets occur.}
\label{fig:five_nodes_small}
\end{figure*}

\begin{figure*}[!t]
\centering
\subfloat[]{\includegraphics[width=3.5in]{figures/new_figures/flows_large_step.pdf}
\label{fig:five_nodes_flow_large_steps}}
\hfil
\subfloat[]{\includegraphics[width=3.5in]{figures/new_figures/resets_large_step.pdf}%
\label{fig:five_nodes_resets_large_steps}}
\caption{Behavior of the five node PCN with a stepsize of $\gamma = 0.1$; (a) shows flows as a function of time; (b) shows the times at which channel  resets occur.}
\label{fig:five_nodes_large}
\end{figure*}

\subsection{Flow-Control Example}\label{sec:flow_control_example}
We now demonstrate the ability of algorithm $\eqref{eq:algorithm}$ to perform flow-control by considering its behavior on a simple PCN prone to deadlocks; the example is taken from \cite{sivaraman2021effect}. The PCN has three nodes, $A, B,$ and $C$, and two channels: $A-B$ and $B-C$, with a capacity of $100$ each (initially balanced). The demands are: $a_{A,C} = a_{C, A} = a_{B,A} = a_{B,C} = 10.$ In this example, the demand from $A$ to $C$ and back form a circulation and can be sustained forever, whereas the demand from $B$ to $A$ and $C$ are DAG demands and therefore cannot be sustained. Moreover, if the network tries to serve the entire demand, eventually the balances in the two channels will get skewed towards $A$ and $C$, with no balance left at $B$. In such a state, all four flows are rendered infeasible (see \eqref{eq:feasibility} for the definition of feasibility), thus creating a deadlock.

Figure \ref{fig:deadlock} shows how algorithm \eqref{eq:algorithm} avoids this deadlock by curbing the flow from $B$ to $A$ and $C$ via channel prices. Due to symmetry, we only show the flows and path prices from $A$ to $C$ and from $B$ to $C$. The corresponding quantities from $C$ to $A$ and from $B$ to $A$ are identical to these. The stepsize $\gamma$ is chosen to be $0.01$. Once again, we show two cases, without the quadratic regularizer ($\eta = 0$) in Figure \ref{fig:deadlock_1} and with the regularizer ($\eta = 0.1$) in Figure \ref{fig:deadlock_2}. Further, we choose the utility function to be $U(\totalflow) = \totalflow$ for all the flows. Adopting this utility function means that for any path, if the price is strictly above $1$, the corresponding flow will be zero. In Figure \ref{fig:deadlock_1}, we see that with every time step, the path prices from $B$ to $A$ and $B$ to $C$ keeps increasing linearly and at some point, they exceed one. At this point, the corresponding flows turn off, which keeps the prices stable. The channel price in the $A\rightarrow B$ direction is negative, and for $B-C$ in the $B\rightarrow C$ direction is positive, which implies the prices from $A \rightarrow C$ and $C \rightarrow A$ add up to zero at all times. Thus, these flows continue unabated, as desired. The effect of the regularizer can be seen by comparing Figures \ref{fig:deadlock_1} and \ref{fig:deadlock_2}. In the absence of the regularizer, the flows exhibit a switching behavior as a function of the price. In contrast, with a regularizer, the flows vary smoothly as a function of the path price. However, the asymptotic behavior of the protocol is the same in both cases. 
% \begin{figure}[htbp]
% \centering
% \begin{subfigure}{0.45\textwidth}
%     \includegraphics[width=\textwidth]{figures/new_figures/deadlock_example.pdf}
%     \caption{Flows and prices without the regularizer.}
%     \label{fig:deadlock_1}
% \end{subfigure}
% \hfill
% \begin{subfigure}{0.45\textwidth}
%     \includegraphics[width=\textwidth]{figures/new_figures/deadlock_example_smooth.pdf}
%     \caption{Flows and prices with the regularizer.}
%     \label{fig:deadlock_2}
% \end{subfigure}   
% \caption{Deadlock prevention through flow-control in a PCN.}
% \label{fig:deadlock}
% \end{figure}



\subsection{Five-Node Network Example}\label{sec:five_node}
{Consider a payment channel network with five nodes ($A, B, C, D, E$) and five channels ($A$-$B$, $B$-$C$, $C$-$D$, $D$-$E$, $E$-$A$), arranged as a ring. Each of the channels has a capacity of a hundred. Because of the topology, each pair of nodes has two possible paths to transact along. The demand matrix is as follows (source along rows, destination along columns):
\begin{align*}
    \begin{array}{cccccc}
        S/D & A & B & C & D & E \\
        A & 0 & 0 & 5 & 10 & 11 \\
        B & 0 & 0 & 0 & 0 & 0 \\
        C & 9 & 0 & 0 & 9 & 0 \\
        D & 0 & 0 & 0 & 0 & 15 \\
        E & 0 & 10 & 11 & 13 & 0 \\ 
    \end{array}
\end{align*}
Note that the demands are much smaller than the channel capacities.
Each of the transacting node pairs has the same utility function $U(f) = 5f$. They also share the same regularizer coefficient $\eta = 1$.}

\begin{figure*}[!t]
\centering
\subfloat[]{\includegraphics[width=3.5in]{figures/new_figures/flows_small_step_random_demand.pdf}
\label{fig:five_nodes_flow_random_demand}}
\hfil
\subfloat[]{\includegraphics[width=3.5in]{figures/new_figures/flows_small_step_sudden_change.pdf}%
\label{fig:five_nodes_flows_sudden_demand}}
\caption{Flows in the five node PCN with time-varying demand; (a) when the demand at each step is independent and Poisson distributed; (b) when the demand is piece-wise constant, with a sudden change in the middle of the time horizon.}
\label{fig:five_nodes_variable_demand}
\end{figure*}

{We simulate the protocol under two different settings: once with a small stepsize of $\gamma = 0.01$ (see Figure \ref{fig:five_nodes_small}) and once with a large stepsize of $\gamma = 0.1$ (see Figure \ref{fig:five_nodes_large}). In each setting, we plot the total flow between each of the transacting node pairs (Figures \ref{fig:five_nodes_flow_small_steps} and \ref{fig:five_nodes_flow_large_steps}) as well as the instances of channel resets (Figures \ref{fig:five_nodes_resets_small_steps} and \ref{fig:five_nodes_resets_large_steps}). The figures illustrate that when the step size is small, the flows converge to a stationary point. This behavior is consistent with Proposition \ref{prop:convergence}. However, the transient period is long ($\approx$ five hundred steps), which results in a fairly large number of channel resets. In contrast, when the step size is large, the flows do not converge to a single point; rather, they oscillate about a particular value. (The channel prices, not shown, also oscillate around a fixed value.) However, the protocol reaches a `steady state' much faster; note the difference in scales of the time axis in the two plots. Consequently, the number of channel resets are also smaller. In summary, varying the stepsize allows us to trade-off the smoothness of flows with the number of channel resets.}

In addition, we also demonstrate the performance of the DEBT control protocol on the same network, but with variable demands in Figure \ref{fig:five_nodes_variable_demand}. First, we consider a setting with stochastic demand. The demand between any two nodes is Poisson distributed with the same mean as in the earlier simulation, and is independent at each time slot. The flows shown in Figure \ref{fig:five_nodes_flow_random_demand} are smoothened with a ten-step moving average filter. In comparison to \ref{fig:five_nodes_flow_small_steps}, the flows have similar, albeit slightly smaller (average) values. Second, we simulate a setting with a piecewise constant demand. The initial demand is the same as that of Figure \ref{fig:five_nodes_small}. Midway through the execution, the demand between every pair of nodes reverses (the source becomes the destination and vice-versa). The resulting flows are shown in Figure \ref{fig:five_nodes_flows_sudden_demand}. The plots illustrate that the protocol can quickly adapt to changes in the demand.
