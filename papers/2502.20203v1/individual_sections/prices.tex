\subsection{Lagrange Multipliers as Prices}\label{sec:prices}
%We now illustrate how algorithm $\eqref{eq:algorithm}$ can be implemented in a distributed way on a payment channel network. We use the model of a PCN presented in Section \ref{sec:PCN_model}. In addition, we view the Lagrange multiplier $\lambda_{u,v}[t]$ as the price quoted by channel $(u,v)$ to route a unit of flow in the direction $u \rightarrow v$ at time $t$. We show that this viewpoint gives a meaningful interpretation of algorithm $\eqref{eq:algorithm}$. For ease of exposition, we split this discussion into two parts: 
% \begin{itemize}
%     \item the flow-control and routing decisions made by the nodes
%     \item the price update performed by the channels.
% \end{itemize}
%To understand how the routing and flow-control decisions are decentralized,
% We now interpret the Lagrange multipliers %, $\lambda_{u,v}[t]$, 
% as prices quoted by channels %$(u,v)$
% to route a unit of flow through them. %in the direction $u \rightarrow v$ at time $t$. 
% With this interpretation, the flows % at time $t$, $f[t]$, 
% can be viewed as a rational response by the transacting node pairs to the prices. 
The first step towards an intuitive interpretation of \eqref{eq:algorithm} is to observe that the Lagrangian is a sum of terms, each concerning one transacting node-pair. Observe that the Lagrangian depends on $\lambda$ only through the term $R^T \lambda$ (see \eqref{eq:lagrangian}). Define $\mu \triangleq R^T\lambda$. Then $\mu$ is a vector indexed by the paths in the PCN, such that
\begin{equation}\label{eq:path_price}
    \mu_{i,j,k} = \sum\nolimits_{u \rightarrow v \in p_{i,j,k}} \lambda_{u,v} - \sum\nolimits_{v \rightarrow u \in p_{i,j,k}} \lambda_{u,v}.
\end{equation}
Similar to the notation $f_{i,j}$, define $\mu_{i,j} \triangleq (\mu_{i,j,1}, \ldots, \mu_{i,j,k})$. Further, define $\Tilde{L}(f, \mu)$ to be 
\begin{align}\label{eq:lagrangian_f_mu}
    &\Tilde{L}(f, \mu) \triangleq \sum\nolimits_{(i,j) \in \mathcal{N}} L_{i,j}(f_{i,j}, \mu_{i,j}), \text{ where } \\
    &L_{i,j}(f_{i,j}, \mu_{i,j}) \triangleq U_{i,j}(\totalflow_{i,j}) - \sum_{k = 1}^ {|P_{i,j}|}\left(f_{i,j,k}\mu_{i,j,k} \,+ \eta_{i,j}(f_{i,j,k})^2\right) \nonumber
\end{align}
%$\Tilde{L}(f, \mu) = L(f, \lambda)$ for any ($\mu$, $\lambda$) satisfying $\mu = R^T\lambda$.

Next, observe that in \eqref{eq:algorithm}, the flows are chosen by solving:
\[ f[t] = \argmax_{f \in A} L(f, \lambda[t])  = \Tilde{L}(f, \mu[t]) \, ; \quad \mu[t] = R^T\lambda[t].\]
% \begin{align}\label{eq:flow_updates}
%     f[t] &\in \arg \max_{f \in A} L(f, \lambda[t])  = L(f, \mu[t]) \, ; \quad \mu[t] = R^T\lambda[t] \nonumber \\
%     &= \arg \max_{f \in A} \sum_{i,j} U_{i,j}(f_{i,j}) - \lambda[t]^TRf  \nonumber \\
%     &= \arg \max_{f \in A} \sum_{i,j} U_{i,j}(f_{i,j}) - \mu[t]^Tf \quad \quad \left(\mu  \triangleq R^T \lambda \right) \nonumber \\
%     &= \arg \max_{f \in A} \sum_{i,j} \left(U_{i,j}(f_{i,j}) - \sum_{k = 1}^{|P_{i,j}|} \mu_{i,j,k}[t]f_{i,j,k}\right) 
% \end{align}
The expression of $\Tilde{L}(f, \mu)$ given in \eqref{eq:lagrangian_f_mu} shows that this optimization problem also separates across node-pairs. Therefore, given $\mu$, the flows between each $(i,j) \in \mathcal{N}$ can be determined independently by solving:
% \begin{align}
% \label{eq:flow_opt1}
%     &\arg \max\{f_{i,j,k} \geq 0 \, \forall \, k, f_{i,j} = \Sigma_k f_{i,j,k} \leq a_{i,j}\}  \nonumber\\  & \quad \quad \quad  U_{i,j}(f_{i,j}) - 
%     \sum_{k = 1}^ {|P_{i,j}|}\left(f_{i,j,k}\mu_{i,j,k} \,+ \eta_{i,j}(f_{i,j,k})^2\right).
% \end{align}
\begin{align}
\label{eq:flow_opt1}
    f_{i,j}[t] = \argmax_{\{f_{i,j}: \, f_{i,j,k} \geq 0 \, \forall \, k, \ q_{i,j} \leq a_{i,j}\}} \  L_{i,j}(f_{i,j}, \mu_{i,j}[t])
\end{align}

The next step towards interpreting algorithm \eqref{eq:algorithm} is to endow the Lagrange multipliers with some meaning. 
Let $\lambda_{u,v}$ be interpreted as the \textit{channel price}, i.e., the cost of routing one unit of flow in the direction $u \rightarrow v$ through the channel $(u,v)$. Let the price for routing flows in the opposite direction of the same channel be $-\lambda_{u,v}$. Then $\mu_{i,j,k}$ can be interpreted as the \textit{path price}, i.e., the cost that the node pair $(i,j)$ needs to pay to send a unit flow along the path $p_{i,j,k}$. The path price is equal to the sum of the channel prices of each of the channels in the path with the appropriate direction (or sign) incorporated, as shown in \eqref{eq:path_price}. Note that the channel prices, and therefore the path prices, may be negative. 

With this interpretation, $f_{i,j,k} \mu_{i,j,k}$ is the cost of sending a flow of amount $f_{i,j,k}$ along the path $p_{i,j,k}$. Thus, $\sum_k f_{i,j,k} \mu_{i,j,k}$ is the total cost incurred by the node-pair $(i,j)$ for splitting the total flow amount $\totalflow_{i,j}$ along different paths. In  $L_{i,j}(f_{i,j}, \mu_{i,j})$, this cost is subtracted from the utility gained by the node-pair $(i,j)$ in executing a transaction of amount $\totalflow_{i,j}$  (see \eqref{eq:lagrangian_f_mu}). The quadratic term $\eta_{i,j} \sum_{k} (f_{i,j,k})^2$, coming from the regularizer, can be interpreted as a penalty for concentrating all flows along a single path. Equivalently, it acts as an incentive to split the total flow along different paths. This is because for any fixed value of $\totalflow_{i,j}$, the sum $ \sum_{k} (f_{i,j,k})^2$ is minimized by splitting $\totalflow_{i,j}$ equally among all $f_{i,j,k}$. Thus, the interpretation of the optimization problem in \eqref{eq:flow_opt1} is that each node-pair tries to maximize its net utility, i.e., the utility of executing a transaction with the cost of execution subtracted from it. This is a rational decision for the nodes.

\subsection{Flow-Control and Routing Decisions} \label{sec:flow_control_routing} The previous section establishes that each node-pair can solve \eqref{eq:flow_opt1} in parallel, independently of each other. We now show how solving \eqref{eq:flow_opt1} can be interpreted as simultaneously making routing and flow-control decisions.
First, consider the case when $\eta_{i,j}$ is zero. In this case, it is optimal to route the transaction only along the path with the minimum price; all other paths from $i$ to $j$ carry zero flow. Let $\mu^*_{i,j}[t]$ denote the minimum path price. The amount of flow carried by this path, $\totalflow_{i,j}[t]$, is given by: 
\begin{equation}\label{eq:scalar_flow}
    \totalflow_{i,j}[t] = \argmax_{\totalflow \in [0, a_{i,j}]}U_{i,j}(\totalflow) - \totalflow \mu^*_{i,j}[t].
\end{equation}
The choice of the total amount of flow is interpreted as a flow-control action and the choice of the path to carry the flow is interpreted as a routing decision. 
% If $\mu[t]$ is such that there are multiple paths with the same minimum price, then \eqref{eq:flow_opt1} has multiple solutions: the total flow $f_{i,j}[t]$ may be split arbitrarily among these paths. Moreover, a small fluctuation in $\lambda[t]$ can lead to the flow switching entirely from one path to another. Therefore, at such values of $\lambda$, the dual function is not differentiable. The quadratic regularizer smoothens the variation of $f[t]$ as a function of $\lambda[t]$. When $\eta_{i,j} > 0$, the total flow amount $f_{i,j}[t]$ continues to be given by \eqref{eq:scalar_flow}, but with $\mu^*_{i,j}[t]$ being equal to:
% \begin{equation}\label{eq:softmin}
%     \min\nolimits_{\Tilde{f}_{i,j,k} \in \mathcal{P}_{|P_{i,j}|}} \sum\nolimits_{k = 1}^ {|P_{i,j}|}\left(\Tilde{f}_{i,j,k}\mu_{i,j,k}[t] + \eta_{i,j}(\Tilde{f}_{i,j,k})^2 \right)
% \end{equation}
% Here, $\mu^*_{i,j}[t]$ will be strictly larger than the minimum path price, but will not exceed this amount by more than $\eta_{i,j}$. The solution to \eqref{eq:softmin} is to choose $\Tilde{f}_{i,j,k} = ((\zeta_{i,j} - \mu_{i,j,k})/2\eta_{i,j})^+$; $\zeta_{i,j}$ is chosen such that the $\Tilde{f}_{i,j,k}$s sum to one. Note that all paths with the minimum path price share the flows equally. Any path whose price exceeds the minimum path price by more than $2\eta_{i,j}$ will always carry zero flow.

%We now turn to the method of calculating the flow amount, i.e., the solution to \eqref{eq:scalar_flow}.  
% Suppose $U_{i,j}(\cdot)$ is continuously differentiable and its derivative, denoted by $U'_{i,j}(\cdot)$, is strictly decreasing throughout the domain $[0, a_{i,j}]$. Then the function $U'^{-1}_{i,j}(\cdot)$ is well defined and the solution of \eqref{eq:scalar_flow} is given by: %The interpretation of \eqref{eq:flow_opt2} is given below.
% \begin{align}\label{eq:flow_opt2}
%     f_{i,j}[t] = 
%     \begin{cases} 
%     a_{i,j} & \text{if } \mu^*_{i,j}[t] \leq U'(a_{i,j}) \\
%     U'^{-1}_{i,j}(\mu^*_{i,j}[t]) & \text{if } \mu^*_{i,j}[t] \in (U'(a_{i,j}), U'(0)] \\
%     0 & \text{if } \mu^*_{i,j}[t] > U'_{i,j}(0) 
%     \end{cases}
% \end{align}
% Note that if all the path prices are very high, the optimal response is to drop the transaction, i.e., set $f_{i,j}[t] = 0$.

% More generally, $U'_{i,j}(\cdot)$ may not be strictly decreasing over some interval over $[0, a_{i,j}]$ (e.g., when $U_{i,j}(\cdot)$ is linear). In this case, we define $U'^{-1}_{i,j}(\cdot)$ to be a multi-valued function and continue to use \eqref{eq:flow_opt2} to decide the flows. Wherever $U'^{-1}_{i,j}(\mu_{i,j}^*[t])$ has multiple values, any one of those values can be chosen for $f_{i,j}[t]$. When $U_{i,j}(\cdot)$ is linear throughout its domain, the optimal solution is to either route the entire requested amount $a_{i,j}$ or none of it. Thus, linear utility functions lead to atomic transactions. It is also possible that $U_{i,j}(\cdot)$ is not differentiable everywhere. However, since $U_{i,j}(\cdot)$ is a concave function, a subdifferential set exists at each point. At points of non-differentiability, the sub-differential is a range of values. Define $U'_{i,j}(\cdot)$ as a multi-valued function on the interval $[0, a_{i,j}]$. Then $U'^{-1}_{i,j}(\cdot)$ can be defined on the entire real line and $f_{i,j}[t]$ can be calculated using \eqref{eq:flow_opt2}.

Now consider the case where the regularizer coefficient $\eta_{i,j}$ is strictly positive. When there is a single path between $i$ and $j$, the quadratic regularizer term can be absorbed into the utility function. Thus, the problem in \eqref{eq:flow_opt1} reduces to \eqref{eq:scalar_flow}. When there are multiple paths from $i$ to $j$, the solution to \eqref{eq:flow_opt1} can be expressed in terms of the classical \textit{waterfilling scheme}. To illustrate this, we invoke the idea of Lagrange multipliers once again to deal with the constraints in \eqref{eq:flow_opt1}. Let $\nu_{i,j}$ denote the Lagrange multiplier for the demand constraint $\totalflow_{i,j} \leq a_{i,j}$. 
By the KKT conditions \cite{bertsekas1999nonlinear}, the optimal solution to \eqref{eq:flow_opt1} must satisfy:
\begin{align}\label{eq:waterfilling}
f_{i,j,k} = \left(\frac{U'_{i,j}(\totalflow_{i,j}) - \nu_{i,j} - \mu_{i,j,k}}{2\eta_{i,j}}\right) ^+ \ \forall \ k.
\end{align}
The total flow $\totalflow_{i,j}$ must also satisfy the demand constraint 
($\totalflow_{i,j} \leq a_{i,j}$) and the  complementary slackness condition: $(\totalflow_{i,j} - a_{i,j})\nu_{i,j} = 0$. Lastly, each $\nu_{i,j}$ must be nonnegative.

Interpreting \eqref{eq:waterfilling} as a waterfilling scheme is easiest when the utility function is linear and the optimal solution satisfies $\totalflow_{i,j} < a_{i,j}$. In this case, $\nu_{i,j}$ is equal to zero and $U'_{i,j}(\totalflow_{i,j})$ is constant; call it $U'_{i,j}$. The term $U'_{i,j}$ acts as a \textit{price ceiling}; any path with a price larger than this price ceiling does not carry any flow. The remaining paths carry a flow that is proportional to the gap between the path price and the price ceiling. Thus, the path with the least price gets allotted the maximum flow and the other paths get progressively smaller flows. Paths with the same price always get allotted the same amount of flow. 



Now consider the general case. %Here, we interpret the term $U'_{i,j}(\totalflow_{i,j}) - \nu_{i,j}$ as the price ceiling. 
The method to solve \eqref{eq:waterfilling} is based on the following observation: increasing either $\totalflow_{i,j}$ or $\nu_{i,j}$ in \eqref{eq:waterfilling} tends to decrease the individual flows $f_{i,j,k}$. %Begin by setting $\nu_{i,j} = 0$ and $q_{i,j} = 0$ in \eqref{eq:waterfilling}.
If all the path prices are bigger than $U'_{i,j}(0)$, then the optimal solution is to set all the flows equal to zero. If not, set $\totalflow_{i,j}$ to $\demand_{i,j}$ and $\nu_{i,j} = 0$ in \eqref{eq:waterfilling} and check whether the corresponding flows, $f_{i,j,k}$, %(from \eqref{eq:waterfilling} with $\totalflow_{i,j} = \demand_{i,j}$) 
add up to a value more than $\demand_{i,j}$. If so, keep $\totalflow_{i,j} = \demand_{i,j}$ and increase $\nu_{i,j}$ to the value such that the corresponding flows add up exactly to $\demand_{i,j}$. If not, keep $\nu_{i,j} = 0$ and find the appropriate value of $\totalflow_{i,j}$ using a few iterations of binary search as follows. For any $\totalflow_{i,j} \in (0, \demand_{i,j})$, check whether the flows given by \eqref{eq:waterfilling} add up to less than $\totalflow_{i,j}$ or not, and adjust $\totalflow_{i,j}$ accordingly. Once $\totalflow_{i,j}$ is fixed, the flows on the individual paths are given by \eqref{eq:waterfilling}.


%Defined this way, $U'_{i,j}(\cdot)$ is guaranteed to be a decreasing function. When $U'_{i,j}(\cdot)$ is a strictly decreasing function, $U'^{-1}_{i,j}(\cdot)$ is well defined. Even if the function is not strictly decreasing (but simply non-increasing), we define $U'^{-1}_{i,j}(\cdot)$ to be a set and set $f^*_{i,j}[t]$ to be the maximum value within that set.