\section{The DEBT (DEtailed Balance Transactions) Control Protocol for PCNs}\label{sec:debt_control_protocol}
\subsection{Overview}\label{sec:protocol_overview}
The model described in the previous section provides a framework for discussing network protocols for PCNs. 
In this section, we present such a protocol which, under any stationary demand, asymptotically maximizes the total utility of all transacting node-pairs while avoiding the expensive operation of periodic on-chain rebalancing.
The protocol guides the transacting node-pairs to make flow-control and routing decisions such that the flows in the network ultimately converge to a suitable stationary flow $f^*$.
On the one hand, $f^*$ should maximize the sum of all node-pairs' utilities; here, we extend the notion of utility from individual transactions to transaction rates. On the other hand, $f^*$ must be a detailed balance flow and should not exceed the requested demand.
We call this protocol the DEBT (DEtailed Balance Transactions) control protocol. 

The key idea of the protocol is to use channel prices as a mechanism to control flows. To elaborate, channels quote prices to nodes for routing flows through them. These channel prices are directional; the price in one direction is always the negative of the other. The price of a path is the sum of the prices of the channels along the path. If multiple paths exist between the same pair of nodes, the protocol recommends choosing a path with the least price or splits the flow along multiple competitive paths. The flow-control decisions are made by comparing the utility gained in having the transaction served to the cost for serving a transaction along a path, where the cost is the product of the price and the transaction volume. A high path price signals the nodes to reduce the flow along the path, while a low path price provides an incentive to increase the corresponding flow. %The channel prices are adjusted in a manner such that asymptotically, only the maximal detailed balance flow circulates through the network, while all other flows are restrained.

The DEBT control protocol is an iterative one. Initially, all channel prices are zero; consequently, the path prices are zero as well. Therefore, all transaction requests are served and routing choices are made arbitrarily. As flows vary over time, channels adjust their prices based on the net flow through them. A channel $(u,v)$ increases the price in the direction $u \rightarrow v$ if there is a net flow in that direction. This change in the price dissuades further flow in that direction and encourages flow in the opposite direction. As the flows converge to a detailed balance flow, the channel prices begin to converge as well. 

Prior to convergence, the flows along each channel may not be balanced; indeed, without an imbalanced flow, prices would not change at all. An imbalanced flow causes the channel balances to change over time. If a channel's capacity is large enough, the cumulative net flow of transactions throughout the transient period never depletes the channel's balances at either end. In this case, the channel will never undergo rebalancing. Else, a channel might occasionally find it infeasible to route the requested flows due to its skewed balances. Whenever such an occasion arises, a channel undergoes on-chain rebalancing and resets its balances. Eventually, no more rebalancing is required as the flows converge to the flow $f^*$ satisfying the detailed balance condition.

The DEBT control protocol is derived in the following steps:
\begin{enumerate}
%    \item We assume that each pair of nodes that wish to transact has some desired flow rate, but is willing to accept a smaller flow rate as well. In other words, the transaction demands are elastic. To incorporate such a behaviour, we assign each transacting node-pair with a utility function. The utility gain by a node-pair is an increasing function of the flow rate obtained.
    \item The objective of the protocol is posed as a \textit{network utility maximization} problem, which we call the primal problem. An optimal set of flows is one which maximizes the sum of all the node-pair's utilities, subject to the constraint that flows satisfy the detailed balance condition and the flow rates served do not exceed the desired flow rates.
    \item The dual problem corresponding to the primal problem is derived by introducing Lagrange multipliers for the detailed balanced constraint. By the theory of strong duality for convex programs, a solution to the dual problem provides a solution to the primal one.
    \item An iterative gradient-descent algorithm is proposed to solve the dual problem. The algorithm updates the Lagrange multipliers (dual variables) in small steps and the flow variables (primal variables) are also updated in response.
    \item The above algorithm is shown to have a decentralized implementation, suitable for implementation on a PCN. The Lagrange multipliers are interpreted as prices quoted by channels. This decentralized implementation constitutes the DEBT control protocol.
%    \item The optimality of the protocol is argued based on the conditions at stationary points of the protocol and the equivalence between the primal, dual and saddle-point formulations.
\end{enumerate}

%Firstly, we assume that each pair of nodes demands a constant transaction rate from the PCN. Put mathematically, for every $(i,j)  \in V \times V$, we assume that user $i$ wants to pay user $j$ at a rate of $a_{i,j}$ per unit time, with $a_{i,j} \in \mathbb{R}_+$. If $a_{i,j} = 0$ for some node-pair $(i,j)$, it means that $i$ does not wish to send any money to $j$. We do not make any other assumptions on the values of $\{a_{i,j}: (i,j) \in V \times V\}$.  %We view the constant demand regime as an approximation of a stationary demand regime (see discussion in Section \ref{sec:transaction_requests}). 

%Secondly, we assume that all demands are elastic. In other words, a slower rate than what they requested is acceptable to the nodes, although a faster rate is preferable. This is a realistic assumption because users have alternate means of transacting, e.g., on the main blockchain. We model elastic demands by means of a utility function. We assume that the node-pair $(i,j)$ gains a utility of $U_{i,j}(f_{i,j})$ upon being served a flow of rate $f_{i,j} \in [0, a_{i,j}]$ by the network. We assume that $U_{i,j}(\cdot)$ is concave and nondecreasing over the interval $[0, a_{i,j}]$. We also assume that $U_{i,j}(0) = 0$ and $U'_{i,j}(0) < \infty$. The utility functions may be different for different node-pairs. 

%Two important special cases are:
% \begin{itemize}
%     \item linear utility functions, \textit{i.e.}, $U_{i,j}(f_{i,j}) = \nu_{i,j} f_{i,j}$, where $\nu_{i,j} > 0$ is a constant, for all $(i,j) \in \mathcal{N}$. 
%     \item strongly concave, twice  differentiable utility functions, \textit{i.e.},  $U''_{i,j}(f) \leq -\alpha < 0 \ \forall \ f \in [0, a_{i,j}]$, for all $(i,j) \in \mathcal{N}$.
% \end{itemize}
