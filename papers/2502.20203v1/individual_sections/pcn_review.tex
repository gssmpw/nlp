\subsection{Routing in Payment Channel Networks: A Review}\label{sec:pcn_review}
Payment channel networks are used as an effective tool to increase a blockchain's transaction processing capability. The most popular such network is the Lightning Network for Bitcoin, proposed by \cite{poon2016bitcoin}. At the time of writing, the network has more than $10, 000$ nodes, $40,000$ channels, and $5,000$ Bitcoin escrowed in the channels (see \url{1ml.com} for live statistics). The Lightning network has played a central role in the growth of small-scale transactions via Bitcoin \cite{river2023lightning}. Alongside the surge in popularity of the Lightning network, there has been a growing academic literature on payment channel networks. Various different aspects of payment channel networks have been investigated, such as their security, privacy, and efficiency \cite{gudgeon2020sok, papadis2020blockchain}. 

In a payment channel network, the channel capacities and the topology of the network are known to all nodes, but the real-time balances are not. While this lack of balance information is beneficial from a privacy perspective, it poses a significant challenge in even the simplest task of finding a \textit{feasible path} for a transaction \cite{tang2020privacy}. (A path connecting two transacting nodes is said to be feasible if each channel along the path has sufficient balance to execute the transaction.) A naive approach for the path discovery problem, used by the Lightning Network, is the following. One attempts to transact along the shortest path in the network (which can be found through the known topology); if the attempt fails, the faulty edge is removed from the view and again a shortest path is found. With deeper research into routing strategies for PCNs, it was soon established that the naive routing strategy used by the Lightning network is suboptimal.

There are many routing protocols that have been proposed in the literature. We summarize some of the important findings of this literature that have a direct bearing on our work.
\begin{itemize}
    \item Smaller transactions are more likely to succeed than larger ones, simply because their feasibility requirements are lower. Therefore, it is worth splitting a large transaction into multiple smaller components and sending these along different paths \cite{stasi2018routing, sivaraman2020high}.
    \item A PCN's throughput can suffer from channel congestion. For example, if all nodes choose to transact along the shortest path between them, channels with high centrality (\textit{i.e.,} with many shortest paths through them) may get congested; the total demand through them could exceed their capacity (see discussion in Section \ref{sec:payment_channels}). These channels will be forced to drop transactions because they do not have the balance to serve them \cite{sivaraman2020high, liu2023balanced}.
    \item Another factor affecting a PCN's throughput is that channels can get imbalanced. As mentioned in Section \ref{sec:payment_channels}, a channel through which there is a net flow of money from one end to another eventually gets imbalanced and is unable to support further flows in that direction. An imbalanced channel could potentially stifle out many end-to-end flows, lowering the PCN's throughput \cite{lin2020funds, van2021merchant}.
    \item Smart routing can improve a PCN's throughput by mitigating both congestion and imbalance. A routing protocol can discover paths that are slightly longer than the shortest path where channels are underutilized (or have higher capacity), thereby reducing congestion \cite{liu2023balanced}. Similarly, a routing protocol can also direct transactions along paths such that the transactions rebalance channels along the path \cite{stasi2018routing, ren2018optimal, lin2020funds, van2021merchant}.
    \item Using channel prices/fees as a means of routing is an economically sound and practically viable approach. Each channel can charge a price that dynamically changes with the extent of imbalance or congestion it faces. In particular, a channel can charge different fees for transactions in the two opposite directions in order to incentivize balanced flows. Nodes (users) are likely to make the rational choice of choosing the cheapest path. This automatically leads to routing decisions that reduce congestion and channel imbalance  \cite{engelmann2017towards, stasi2018routing, ren2018optimal, van2021merchant, varma2021throughput, wang2024fence}.
    \item There is a potential privacy concern that such channel prices can reflect channel balances and thereby leak sensitive transaction information. This concern can be offset by reducing the frequency of price updates or adding some noise to the price (differential privacy style solutions) \cite{tang2020privacy, wang2024fence}.
    \item Choosing the best path by minimizing the transaction fee over all possible paths in the graph is computationally expensive \cite{engelmann2017towards, stasi2018routing, chen2022routing}. Repeating this procedure for every transaction may be infeasible. A practical approach is for nodes to sporadically establish a small set of candidate paths among all possible ones and then optimize the prices only among these paths for every transaction \cite{van2021merchant}.
\end{itemize}

Next, we discuss the few papers that design PCN routing protocols with theoretical guarantees: the Spider protocol \cite{sivaraman2020high}, the protocol of \cite{varma2021throughput}, and Fence \cite{wang2024fence}. These protocols share some common features. First, they aim to reduce congestion and imbalance in PCN channels. Second, they involve channels signaling congestion or imbalance to transacting nodes (via prices or delays), who adjust routing decisions accordingly. Third, their theoretical guarantees stem from connections to network utility maximization problems.

Spider \cite{sivaraman2020high}, inspired by TCP protocols, breaks transactions into packets handled independently and maintains queues at both ends of each channel. A channel queues a packet if it lacks balance to forward it immediately, with large queues signaling congestion/imbalance and leading to delays. Nodes adjust transaction rates based on these delays, reducing rates when delays are high. The efficacy of Spider is demonstrated through extensive simulations. Moreover, \cite{sivaraman2020high} proves that its steady-state flow-rate optimizes a constrained utility maximization problem with logarithmic utilities.  However, this result is shown only for a specific network topology and arbitrarily large demand. Moreover, \cite{sivaraman2020high} does not provide any convergence guarantees.

Inspired by Spider, \cite{varma2021throughput} proposes a protocol that also breaks transactions into packets and maintains queues. However, it replaces delay-based signaling with explicit congestion and imbalance prices. Congestion prices are identical in both directions, while imbalance prices are opposite in sign. The path price is the sum of these two prices over all channels, and each packet takes the lowest-price path. The paper analyzes the protocol under i.i.d. demand and proves that under certain constraints on the mean demand, it stabilizes all queues (executes all transactions) with minimal on-chain rebalancing. The proof technique involves drawing parallels between the protocol and the dual to a constrained throughput maximization problem. However, the results hold only for a circulation demand (see discussion below).

Fence \cite{wang2024fence} also uses prices to manage congestion and imbalance and is based on weighted throughput maximization with constraints. Unlike the long-term flow-based models of \cite{sivaraman2020high} and \cite{varma2021throughput}, Fence adopts a finite-horizon approach, leading to a competitive ratio bound for worst-case demand. However, this result holds only for uni-directional PCNs, and \cite{wang2024fence} shows that no meaningful competitive ratio exists for bidirectional channels. Like \cite{varma2021throughput}, Fence tracks congestion and imbalance but uses an exponential price function instead of a linear one. These steep prices enable the competitive-ratio analysis in \cite{wang2024fence}.

We conclude this section with a short note on the fundamental limitations of any payment channel network in serving demand. Consider a setting where the demand between any two nodes is steady over time and the demand is small enough that capacity constraints are easily met (there is no congestion). The demand is said to be a  \textit{circulation} if each node sends as much money as it receives \cite{sivaraman2020high}. If this is not the case, \textit{i.e.}, some nodes may send more than they receive and vice-versa, the demand is said to be \textit{acyclic} \cite{sivaraman2020high}. Any demand can be split into a circulation component and a leftover acyclic component. In \cite{sivaraman2020high}, it is shown empirically that if the demand is a circulation, a PCN can support the entire demand perpetually without any on-chain rebalancing. Later, \cite{varma2021throughput} analyzes a very similar protocol and theoretically proves this result. In contrast, acyclic demands can be executed in full only with persistent on-chain rebalancing. In summary, a PCN may not always be able to serve the demand in entirety.

The papers \cite{sivaraman2020high} and \cite{sivaraman2021effect} also uncover the phenomenon of \textit{deadlocks}. A deadlock is a scenario where, owing to a few channels being imbalanced, a large number of transactions are rendered infeasible; moreover, they remain infeasible unless the channel undergoes on-chain rebalancing. \cite{sivaraman2021effect} explains that deadlocks may arise when the demand has an acyclic component. In particular, if a PCN tries to serve any transaction request as long as it is feasible, channels can get completely imbalanced, leading to deadlocks.  \cite{sivaraman2021effect} proposes a method to design the topology of the PCN to reduce the prevalence of deadlocks. However, it does not propose a protocol to avoid deadlocks. Preventing deadlocks is an important task for a PCN to operate efficiently in the long run. 
