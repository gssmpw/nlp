\subsection{Our Contribution}\label{sec:contribution}

In this work, we propose a simple routing and flow-control protocol for PCNs, called DEBT control, with provable convergence and optimality guarantees. We provide a brief overview of our protocol's design and analysis here in order to distinguish it from prior work. We assume we are given a payment channel network with an arbitrary topology; this topology is known and fixed. We are also given a set of transacting user-pairs who have some steady, elastic demand between them. Given the network and the demand, we set the objective of the PCN to maximize the total utility of all users, subject to the balance constraint imposed by each channel. The optimal solution to this problem is a set of flows that can be sustained indefinitely by the PCN without any rebalancing. We show that under the DEBT Control protocol, the flows in the network converge to some such optimal solution.

We arrive at the DEBT control protocol using the following methodology, which is inspired by similar works in the domain of communication networks \cite{kelly1998rate, low1999optimization, srikant2013communication, kelly2014stochastic}. We derive the dual of the aforementioned network utility maximization problem and show that the dual is an unconstrained convex optimization problem. We show that the gradient of the dual can be easily calculated, thus making it straightforward to use gradient-descent to solve the dual problem. Next, we show that the gradient-descent method admits a decentralized network implementation; this forms the DEBT control protocol. Endowing the Lagrange multipliers with the interpretation of channel prices gives an intuitive understanding of the protocol. Crucially, we allow channel prices to be negative. Using known results of strong duality (and other standard results in optimization), we show that the protocol converges to an optimal solution for the dual, and consequently, to optimal flows for the original problem. Our theoretical result, as well as the method to obtain it, is novel in the context of payment channel networks. 

Next, we compare our work to the most closely related prior work \cite{sivaraman2020high, varma2021throughput, wang2024fence}. Firstly, our protocol is designed from a long-term, or infinite-horizon point of view. This long-term viewpoint allows us to pose the network utility maximization problem in terms of flows, as done by \cite{sivaraman2020high}, and focus on the asymptotic convergence of the protocol to a steady-state solution. This is in contrast to the work of \cite{wang2024fence}, which takes a short-term, or finite-horizon, viewpoint. This difference is reflected in the flavor of the results: while \cite{wang2024fence} prove competitive ratio results, thereby giving guarantees for the worst-case performance over any possible demand, we prove the convergence of our protocol to the optimum flows under steady demands. Our model and result are also different from \cite{varma2021throughput}; it assumes stochastic demands, under which it proves bounds on the queue lengths, whereas we assume deterministic demands, and prove convergence of flows to the optimum value. Finally, our result is more general than the result of \cite{sivaraman2020high} in two ways. First, we show that our protocol converges to a steady-state solution, whereas \cite{sivaraman2020high} does not show any convergence guarantees. Second, our optimality guarantees hold for any network topology, any demand, and a wide class of utility functions, whereas the guarantees in \cite{sivaraman2020high} hold only for parallel networks, infinite demand, and logarithmic utility functions.

Our work also highlights the efficacy of price-based flow-control in preventing deadlocks. As we illustrate in Section \ref{sec:flow_control_example}, our protocol can selectively curb the acyclic component of the demand while allowing the circulation component of the demand to flow. This is indeed necessary for the flows to converge to an optimal steady-state value. It was observed in \cite{sivaraman2020high} that the Spider protocol is unsuccessful in preventing all deadlocks. Our analysis offers an explanation for this observation; we show that using logarithmic utilities does not allow the protocol to completely stifle deadlock-causing acyclic demands.