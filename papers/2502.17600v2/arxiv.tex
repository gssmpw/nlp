\documentclass[12pt,oneside]{article}
% \documentclass[11pt,a5paper,footinclude=true,headinclude=true]{scrbook} % KOMA-Script book
% \usepackage[linedheaders,parts,pdfspacing]{classicthesis} % ,manychapters
\usepackage[osf]{libertine}

\usepackage[OT1]{fontenc}
\usepackage[utf8]{inputenc}
\usepackage{kpfonts}
\usepackage{graphicx}
\usepackage[dvipsnames]{xcolor}
\usepackage{amsthm}
%\usepackage[margin=1.1in]{geometry}
\usepackage{fullpage}
\usepackage[sf,bf,small,raggedright,compact]{titlesec}
\usepackage{hyperref}
\usepackage{mathrsfs}
\usepackage{amssymb}
%\usepackage{amsmath}
\usepackage{stmaryrd}



\definecolor{linkblue}{named}{MidnightBlue}
\hypersetup{colorlinks=true, linkcolor=linkblue,  anchorcolor=linkblue,
        citecolor=linkblue, filecolor=linkblue, menucolor=linkblue,
        urlcolor=linkblue}
\usepackage[capitalize]{cleveref}
\usepackage{paralist}
\usepackage[longnamesfirst,numbers,sort&compress]{natbib}
% \usepackage[normalem]{ulem}


\usepackage{listings,newtxtt}
\lstset{basicstyle=\ttfamily, keywordstyle=\bfseries}


\newcommand{\R}{\mathbb{R}}
\newcommand{\eps}{\varepsilon}
\newcommand{\CP}{\mathord{\it CP}}



\newcommand{\pref}[1]{(S\ref{#1})}
\setlength{\parskip}{1ex}

\newtheorem{thm}{Theorem}
\newtheorem{obs}{Observation}
\newtheorem{lem}{Lemma}
\newtheorem{cor}{Corollary}
\newtheorem{conj}{Conjecture}
\newtheorem{prb}{Problem}
\newtheorem{rem}{Remark}
\newtheorem{prop}{Proposition}

\usepackage{dsfont}  % for \mathds{A}
\usepackage[noend]{algorithmic}

\usepackage[normalem]{ulem}
\usepackage{cancel}
\usepackage{enumitem}

\usepackage{todonotes}

\usepackage[longnamesfirst,numbers,sort&compress]{natbib}

\newcommand{\Rho}{\mathrm{S}}

% \newcommand{\harpoon}{\overset{\rightharpoonup}}
\newcommand{\qa}{\overset{\rightharpoonup}{\varphi}}
\newcommand{\qb}{\overset{\rightharpoondown}{\varphi}}
\newcommand{\qap}{\overset{\rightharpoonup}{\sigma}}
\newcommand{\qbp}{\overset{\rightharpoondown}{\sigma}}

\usepackage[mathlines]{lineno}
\setlength{\linenumbersep}{2em}
% \linenumbers
% \rightlinenumbers
% \linenumbers
\newcommand*\patchAmsMathEnvironmentForLineno[1]{%
 \expandafter\let\csname old#1\expandafter\endcsname\csname #1\endcsname
 \expandafter\let\csname oldend#1\expandafter\endcsname\csname end#1\endcsname
 \renewenvironment{#1}%
    {\linenomath\csname old#1\endcsname}%
    {\csname oldend#1\endcsname\endlinenomath}}%
\newcommand*\patchBothAmsMathEnvironmentsForLineno[1]{%
 \patchAmsMathEnvironmentForLineno{#1}%
 \patchAmsMathEnvironmentForLineno{#1*}}%
\AtBeginDocument{%
\patchBothAmsMathEnvironmentsForLineno{equation}%
\patchBothAmsMathEnvironmentsForLineno{align}%
\patchBothAmsMathEnvironmentsForLineno{flalign}%
\patchBothAmsMathEnvironmentsForLineno{alignat}%
\patchBothAmsMathEnvironmentsForLineno{gather}%
\patchBothAmsMathEnvironmentsForLineno{multline}%
}


\newcommand{\coloured}[2]{{\color{#1}{#2}}}
\newenvironment{colourblock}[1]{\color{#1}}{}

\newcommand{\condref}[1]{(C\ref{#1})}

% Taken from
% https://tex.stackexchange.com/questions/42726/align-but-show-one-equation-number-at-the-end
\newcommand\numberthis{\addtocounter{equation}{1}\tag{\theequation}}


\setlength{\parskip}{1ex}


\DeclareMathOperator{\diam}{diam}
\newcommand{\DIAM}{\diam}
\DeclareMathOperator{\tw}{tw}
\DeclareMathOperator{\lca}{lca}
\DeclareMathOperator{\polylog}{polylog}

\DeclareMathOperator{\x}{x}
\DeclareMathOperator{\height}{height}
\DeclareMathOperator{\depth}{depth}
\DeclareMathOperator{\dist}{dist}
\DeclareMathOperator{\sh}{cbt}
\DeclareMathOperator{\cbt}{cbt}
\DeclareMathOperator{\sgn}{sgn}
\DeclareMathOperator{\dc}{dc}

\usepackage{skull}
\usepackage{paralist}
\usepackage{graphicx}
\usepackage[noend]{algorithmic}

\usepackage[normalem]{ulem}
\usepackage{cancel}
% \usepackage{enumitem}
\usepackage{comment}


\newcommand{\algo}[1]{\textsc{#1}}
\newcommand{\defin}[1]{\emph{\textcolor{Maroon}{#1}}}

\makeatletter
\renewcommand{\fnum@figure}{Fig. \thefigure}
\makeatother

\definecolor{brew1}{rgb}{0.552941176471, 0.827450980392, 0.780392156863}
\definecolor{brew2}{rgb}{1.0, 1.0, 0.701960784314}
\definecolor{brew3}{rgb}{0.745098039216, 0.729411764706, 0.854901960784}
\definecolor{brew4}{rgb}{0.98431372549, 0.501960784314, 0.447058823529}
\definecolor{brew5}{rgb}{0.501960784314, 0.694117647059, 0.827450980392}
\definecolor{brew6}{rgb}{0.992156862745, 0.705882352941, 0.38431372549}
\definecolor{brew7}{rgb}{0.701960784314, 0.870588235294, 0.411764705882}
\definecolor{brew8}{rgb}{0.988235294118, 0.803921568627, 0.898039215686}


\title{Tight Bounds on the Number of Closest Pairs in Vertical Slabs}

\author{
\normalsize{Ahmad Biniaz}\thanks{School of Computer Science, University of Windsor, Windsor, Canada. Research supported by NSERC.} \and
\normalsize{Prosenjit Bose}\thanks{School of Computer Science, Carleton University, Ottawa, Canada. Research supported by NSERC.} \and
\normalsize{Chaeyoon Chung}\thanks{Department of Computer Science and Engineering, Pohang University of Science and Technology, Pohang, Republic of Korea. Research supported by 
Institute of Information \& Communications
Technology Planning \& Evaluation (IITP) grant funded by the Korea government (MSIT).} \and
\normalsize{Jean-Lou De Carufel}\thanks{School of Electrical Engineering and Computer Science, University of Ottawa, Ottawa, Canada. Research supported by NSERC.} \and 
\normalsize{John Iacono}\thanks{Department of Computer Science, Universit{\'e} libre de Bruxelles, Brussels, Belgium. This work was supported by the Fonds de la Recherche Scientifique-FNRS.} \and
\normalsize{Anil Maheshwari}\footnotemark[2] \and 
\normalsize{Saeed Odak}\footnotemark[4] \and 
\normalsize{Michiel Smid}\footnotemark[2] \and 
\normalsize{Csaba D. T\'{o}th}\thanks{Department of Mathematics, California State University Northridge, Los Angeles, CA, and Department of Computer Science, Tufts University, Medford, MA, USA. Research supported in part by the NSF award DMS-2154347.}
}

\date{}

\bibliographystyle{plainurl}

\begin{document}

\maketitle

%TODO mandatory: add short abstract of the document
\begin{abstract}
Let $S$ be a set of $n$ points in $\R^d$, where $d \geq 2$ is a constant,
and let $H_1,H_2,\ldots,H_{m+1}$ be a sequence of vertical hyperplanes 
that are sorted by their first coordinates, such that exactly 
$n/m$ points of $S$ are between any two successive hyperplanes. 
Let $|A(S,m)|$ be the number of different closest pairs in the 
${{m+1} \choose 2}$ vertical slabs that are bounded by $H_i$ and $H_j$, 
over all $1 \leq i < j \leq m+1$. We prove tight bounds for the largest
possible value of $|A(S,m)|$, over all point sets of size $n$, and for 
all values of $1 \leq m \leq n$. 

As a result of these bounds, we obtain, for any constant $\eps>0$, 
a data structure of size $O(n)$,
such that for any vertical query slab $Q$, the closest pair in the 
set $Q \cap S$ can be reported in $O(n^{1/2+\eps})$ time. Prior to this 
work, no linear space data structure with sublinear query time was 
known.
\end{abstract}

\newpage\setcounter{page}{1}

\section{Introduction}
Throughout this paper, we consider point sets in $\R^d$, where the
dimension $d$ is an integer constant. For any real number $a$, we define
the \emph{vertical hyperplane} $H_{a}$ to be the set
\[ H_{a} = \{ (x_1,x_2,\ldots,x_d) \in \R^d : x_1 = a \} .
\]
Note that this is a hyperplane with normal vector $(1,0,0,\ldots,0)$. 
For any two real numbers $a$ and $b$ with $a<b$, we define the
\emph{vertical slab} $\llbracket H_a, H_b \rrbracket$ to be the set
\[ \llbracket H_a, H_b \rrbracket = 
    \{ (x_1,x_2,\ldots,x_d) \in \R^d : a \leq x_1 \leq b \} . 
\]
Let $S$ be a set of $n$ points in $\R^d$, in which no two points have
the same first coordinate and all ${n} \choose {2}$ pairwise Euclidean
distances are distinct.

For any two real numbers $a$ and $b$ with $a<b$, we define
$\CP(S,H_a,H_b)$ to be the closest-pair among all points in the set
$\llbracket H_a, H_b \rrbracket \cap S$, i.e., all points of $S$
that are in the vertical slab $\llbracket H_a, H_b \rrbracket$.
If $\llbracket H_a, H_b \rrbracket \cap S$ has size at most one, then
$\CP(S,H_a,H_b) = \infty$.

Clearly, there are $\Theta(n^2)$ 
 %many 
combinatorially different sets of
the form $\llbracket H_a, H_b \rrbracket \cap S$.
Sharathkumar and Gupta~\cite{sharathkumar2007range} have shown that, for $d=2$, the size of the set
\[ \{ \CP(S,H_a,H_b) : a < b \} 
\]
is only $O(n \log n)$. That is, even though there are $\Theta(n^2)$
%many 
combinatorially ``different'' vertical slabs with respect to $S$, the number of different closest
pairs in these slabs is only $O(n \log n)$.

In this paper, we generalize this result to the case when the
dimension $d$ can be any constant and the slabs
$\llbracket H_a, H_b \rrbracket$ come from a restricted set.

Let $m$ be an integer with $1 \leq m \leq n$, and let
$a_1 < a_2 < \cdots < a_{m+1}$ be real numbers such that
for each $i = 1,2,\ldots,m$, there are 
exactly\footnote{In order to avoid floors and ceilings, we assume for simplicity that $n$ is a multiple of $m$.}
$n/m$ points of $S$ %are 
in the interior of the vertical slab
$\llbracket H_{a_i}, H_{a_{i+1}} \rrbracket$.
Observe that this implies that all points in $S$ are in the interior of the vertical slab $\llbracket H_{a_1}, H_{a_{m+1}} \rrbracket$.

We define
\[ A(S,m) = \{ \CP(S,H_{a_i},H_{a_j}) : 1 \leq i < j \leq m+1 \} . 
\]
That is, %$A(S,m)$ is the set of all 
 $|A(S,m)|$ is the number of
\emph{different} closest pairs over all
${m+1} \choose 2$ slabs bounded by vertical hyperplanes whose
first coordinates belong to $\{ a_1,a_2,\ldots,a_{m+1} \}$.
Finally, we define
\[ f_d(n,m) = \max \{ |A(S,m)| : |S|=n \} . 
\]
Using this notation, Sharathkumar and Gupta~\cite{sharathkumar2007range} have shown that
$f_2(n,n) = O(n \log n)$.

In dimension $d=1$, it is easy to see that $f_1(n,m) = \Theta(m)$. Our main results are the
following tight bounds on $f_d(n,m)$, for any constant $d \geq 2$ and
any $m$ with $1 \leq m \leq n$:

\begin{thm}
\label{thm-main1}
Let $d \geq 2$ be a constant, and let $m$ and $n$ be integers such that
$1 \leq m \leq n$.
\begin{enumerate}
\item If $m = O(\sqrt{n})$, then $f_d(n,m) = \Theta(m^2)$.
\item If $m = \omega(\sqrt{n})$, then
$f_d(n,m) = \Theta(n \log (m/\sqrt{n}))$.
\item In particular, if $m=n$, then $f_d(n,m) = \Theta(n \log n)$.
\end{enumerate}
\end{thm}

\subsection{Motivation}

In the \emph{range closest pair problem}, we have to store a given set
$S$ of $n$ points in $\R^d$ in a data structure such that queries of
the following type can be answered: Given a query range $R$ in $\R^d$,
report the closest pair among all points in the set $R \cap S$.

Many results are known for different classes of query ranges. We 
mention some of the currently best data structures. 
Xue \emph{et al.}~\cite{DBLP:journals/dcg/XueLRJ22} present data 
structures for the case when $d=2$ and the query ranges are quadrants, 
halfplanes, or axes-parallel rectangles. Again for the case when $d=2$, 
data structures for query regions that are translates of a fixed shape 
are given by Xue \emph{et al.}~\cite{translate}. Some results in any 
constant dimension $d \geq 3$ are given by 
Chan \emph{et al.}~\cite{higherdimension}. References to many other
data structures can be found in 
\cite{higherdimension,translate,DBLP:journals/dcg/XueLRJ22}.

Most of the currently known data structures use super-linear space. 
To the best of our knowledge, linear-sized data structures are known 
only for the following classes of regions, all in dimension $d=2$: 
Quadrants and halfplanes~\cite{DBLP:journals/dcg/XueLRJ22}, and 
translates of a fixed polygon (possibly with holes)~\cite{translate}.
% of constant size~\cite{translate}. 
% \csaba{Any fixed polygon has constant size.}
In all these three cases, the query time is $O(\log n)$. 

If each query range $R$ is a vertical slab
$\llbracket H_a, H_b \rrbracket$, we refer to the problem as the
\emph{vertical slab closest pair problem}. In dimension $d=1$, it is
easy to obtain a data structure of size $O(n)$ such that the closest
pair in any ``vertical slab'' (i.e., interval on the real line) can be
computed in $O(\log n)$ time. In dimension $d=2$, Sharathkumar and Gupta~\cite{sharathkumar2007range} gave a data structure of size $O(n \log^2 n)$ that allows
queries to be answered in $O(\log n)$ time.
Xue \emph{et al.}~\cite{DBLP:journals/dcg/XueLRJ22} improved the space bound to $O(n \log n)$,
while keeping a query time of $O(\log n)$.
Both these results use the fact that $f_2(n,n) = O(n \log n)$. 
In fact, both data structures explicitly store the set 
$\{ \CP(S,H_a,H_b) : a < b \}$, whose size is equal to $f_2(n,n)$ 
in the worst case. 

The starting point of our work was to design a data structure of size
$O(n)$ for vertical slab closest pair queries. This led us to
the problem of determining the asymptotic value of the 
function $f_d(n,m)$. Using our bounds in
Theorem~\ref{thm-main1}, we will obtain the following result.

\begin{thm}
\label{thm-main2}
Let $d \geq 2$ be an integer constant and let $\eps>0$ be a real constant.
For every set $S$ of $n$ points in $\R^d$, there exists a data structure
of size $O(n)$ that allows vertical slab closest pair queries to be
answered in $O(n^{1/2+\eps})$ time.
\end{thm}

Note that, prior to our work, no $O(n)$-space data structure with a 
query time of $o(n)$ was known for $d \geq 2$.

%\subsection{Organization}
\paragraph{Organization.}
In Section~\ref{secupper}, we will present the upper bounds in 
Theorem~\ref{thm-main1} on $f_d(n,m)$. The corresponding lower 
bounds will be given in Section~\ref{seclower}. The data structure 
in Theorem~\ref{thm-main2} will be presented in 
Section~\ref{data-structure}. 
We conclude in Section~\ref{future} with some open problems. 

\paragraph{Notation and Terminology.}
Throughout the rest of this paper, the notions of left and right in $\R^d$ will always refer to the ordering in the first coordinate. That is, if $p=(p_1,p_2,\ldots,p_d)$ and $q=(q_1,q_2,\ldots,q_d)$ are two points in $\R^d$ with $p_1 < q_1$, then we say that $p$ is to the \emph{left} of $q$, and $q$ is to the \emph{right} of $p$. For a vertical hyperplane $H_a$, we say that $p$ is to the \emph{left} of $H_a$ if $p_1 < a$. If $p_1 > a$, then $p$ is to the \emph{right} of $H_a$. 

The Euclidean distance between any two points $p$ and $q$ in $\R^d$ will be denoted by 
$|| p-q ||$. The length, or norm, of any vector $v$ will be denoted by $||v||$. 

Our data structure in Section~\ref{data-structure} will work in the real-RAM model. In this model, the coordinates of the points are arbitrary real numbers, and the standard operations of addition, subtraction, multiplication, and division take $O(1)$ time. 


\section{Upper bounds on $f_d(n, m)$}
\label{secupper}

Let $d \geq 2$ be a constant, let $m$ and $n$ be integers with
$1 \leq m \leq n$, and let $S$ be a set of $n$ points in $\R^d$. Let
$a_1 < a_2 < \cdots < a_{m+1}$ be real numbers such that for each 
$i = 1,2,\ldots,m$, there are exactly $n/m$ points in $S$ %that are 
between the vertical hyperplanes $H_{a_i}$ and $H_{a_{i+1}}$. 

For any $m$, it is clear that $f_d(n,m) = O(m^2)$, because there are  ${m+1}\choose{2}$ vertical slabs of the form 
$\llbracket H_{a_i}, H_{a_j} \rrbracket$. Thus, the upper bound in Theorem~\ref{thm-main1} holds when $m = O(\sqrt{n})$. In the rest of this 
section, we assume that $m = \omega(\sqrt{n})$. 

The following lemma was
proved by Sharathkumar and Gupta~\cite{sharathkumar2007range} 
for the case when $d=2$. This lemma will be the key tool to prove our 
upper bound on $f_d(n,m)$. 

\begin{lem} \label{lem:gupta-higher-diminsions}
    Let $S$ be a set of $n$ points in $\R^d$ and let $\mathcal{C}\mathcal{P}$ be the set of segments corresponding to the elements of $A(S,n)$. That is, for each pair in $A(S,n)$, the set $\mathcal{C}\mathcal{P}$ contains the line segment connecting the two points in this pair. For any vertical hyperplane $H$, the number of elements of $\mathcal{C}\mathcal{P}$ that cross $H$ is $O(n)$.    
\end{lem}

We first present a proof of this lemma for the case when $d=2$. We 
believe that our proof is simpler than the one 
in~\cite{sharathkumar2007range}. Afterwards, we present a proof 
for any dimension $d \geq 2$. 

\noindent 
{\bf Proof of Lemma~\ref{lem:gupta-higher-diminsions} when $d=2$.}
We write $\ell$ for the vertical
line. %hyperplane. 
We define a graph, $G^+$, with vertex set $S$. Each segment of $\mathcal{C}\mathcal{P}$ with a positive slope represents an edge in the graph $G^+$. Let $F$ be the subgraph of $G^+$ induced by the segments of $\mathcal{CP}$ that cross $\ell$. We will show that $F$ does not contain a cycle.

Suppose, to the contrary, that there is a cycle $C$ in $F$. Let $a$ and $b$ be the endpoints of the shortest edge in $C$ such that $a$ is to the left of $\ell$ and $b$ is to the right of $\ell$. Let $ac$ and $bd$ be the other edges of the cycle that are incident to $a$ and $b$, respectively. Since both $ab$
and $ac$ represent pairs in $\mathcal{CP}$ and both have a positive slope, 
we have $a_x < c_x < b_x$ and $a_y < b_y < c_y$. Similarly, we have 
$a_x < d_x < b_x$ and $d_y < a_y < b_y$; see \cref{figs:gupta}.

\begin{figure}
    \begin{center}
        \includegraphics[scale=0.75]{figs/gupta}
    \end{center}
    \caption{The pairs in $A(S,n)$ with positive slope that cross $\ell$  do not contain a cycle.}
    \label{figs:gupta}
\end{figure}

Let $c'$ be the reflection of the point $c$ with respect to the 
horizontal line through $b$. Note that $||b-c'|| = ||b-c|| > ||b-d||$, because  
$bd$ represents a pair in $A(S,n)$ and the vertical slab 
$\llbracket b_x, d_x \rrbracket$ contains the point $c$. Since 
$||b-c'|| > ||b-d||$, we have $c'_y < d_y$. We also have $d_y < a_y$ and $a_x < d_x < c_x = c'_x$. It follows that $||a-d|| < ||a-c'||$.

Consider the bisector of the segment $cc'$ (which is the horizontal line through $b$). Observe that the point $a$ is located on the same side as $c'$ with respect to this bisector. Therefore, 
$||a-c'|| < ||a-c||$. Combined with $||a-d|| < ||a-c'||$, this implies that 
$||a-d|| < ||a-c||$. This contradicts the facts that $ac$ represents a pair in $A(S,n)$ and the point $d$ is  in the slab $\llbracket a_x, c_x \rrbracket$.

A similar argument shows that the segments in $\mathcal{C}\mathcal{P}$ 
that cross $\ell$ and have negative slopes do not contain a cycle. Therefore, the total number of segments in $\mathcal{C}\mathcal{P}$  
that cross the line $\ell$ is $O(n)$.
\qed
\vspace{0.4cm}

% \begin{rem} In the proof of the \cref{lem:gupta}, we only used the fact that the subgraph $H$ does not contain a path of length three such that the middle edge in the path has a shorter length than the other two edges. It is easy to see that the existence of a cycle in $H$ implies the existence of such a path of length 3. \todo{as requested, send this to the main part of the lemma.}
% \end{rem}



To prove Lemma~\ref{lem:gupta-higher-diminsions} for dimensions 
$d \geq 2$, we will use the \emph{Well-Separated Pair Decomposition (WSPD)}, as introduced by Callahan and Kosaraju~\cite{well-separated-2}. 

Let $S$ be a set of $n$ points in $\R^d$ and let $s>1$ be a real number, called the \emph{separation ratio}. A WSPD for $S$ is a set of pairs 
$\{A_i, B_i\}$, for $i=1,2,\ldots,k$, for some positive integer $k$ such that 

\begin{enumerate} 
\item for each $i$, $A_i \subseteq S$ and $B_i \subseteq S$, 
\item for each $i$, there exist two balls $D$ and $D'$ of the same radius, say $\rho$, such that $A_i \subseteq D$, $B_i \subseteq D'$, and the distance between $D$ and $D'$ is at least $s \cdot \rho$, 
i.e., the distance between their centers is at least $(s+2) \cdot \rho$, 
\item for any two distinct points $p$ and $q$ in $S$, there is a unique index $i$ such that $p \in A_i$ and $q \in B_i$ or vice-versa. 
\end{enumerate} 

Consider a pair $\{A_i,B_i\}$ in a WSPD. If $p$ and $p'$ are two points in $A_i$ and $q$ is a point in $B_i$, then it is easy to see that 
\begin{equation}\label{eq:WSPD}
||p-p'|| \leq (2/s) \cdot ||p-q|| . 
\end{equation}
In particular, if $s>2$, then $||p-p'|| < ||p-q||$.

%Callahan and Kosaraju in \cite{well-separated-2} show that a well-separated pair decomposition for a point set $S$, with separation ratio $s$ ($s > 2$ in our application), can be constructed, such that the number of well-separated pairs is $O(|S|)$.

\begin{lem}[Callahan and Kosaraju~\cite{well-separated-2}] \label{callahan}
    Let $S$ be a set of $n$ points in $\R^d$, and let $s > 1$ be a real number. A well-separated pair decomposition for $S$, with separation ratio $s$, consisting of $O(s^d n)$ pairs, can be computed in $O(n\log{n} + s^d n)$ time.
\end{lem}

\noindent 
{\bf Proof of Lemma~\ref{lem:gupta-higher-diminsions} when $d \geq 2$.}
Let $s>2$ be a constant and consider a WSPD $\{A_i,B_i\}$, $i=1,2,\ldots,k$, for the point set $S$ with separation ratio $s$, where $k=O(n)$; see Lemma~\ref{callahan}. We define the following geometric graph $G$ on the point set $S$. For each $i$ with $1 \le i \le k$, let 

\begin{itemize}
    \item $a_i$ be the rightmost point in $A_i$ that is to the left of $H$,
    \item $b_i$ be the leftmost point in $B_i$ that is to the right of $H$,
    \item $a'_i$ be the leftmost point in $A_i$ that is to the right of $H$, and
    \item $b'_i$ be the rightmost point in $B_i$ that is to the left of $H$. 
\end{itemize}
We add the edges $a_i b_i$ and $a'_i b'_i$ to the graph $G$. Note that some of these points may not exist, in which case we ignore the corresponding edge. It is clear that $G$ has $O(n)$ edges. The lemma will follow from the fact that every segment in $\mathcal{C}\mathcal{P}$ that crosses $H$ is an edge in $G$. 

Let $pq$ be a pair in $\mathcal{C}\mathcal{P}$ that crosses $H$, and let $Q$ be a vertical slab such that $pq$ is the closest pair in $Q \cap S$. We may assume, without loss of generality, that $p$ is to the left of $H$ and $q$ is to the right of $H$. Let $i$ be the index such that (i) $p \in A_i$ and $q \in B_i$ or (ii) $p \in B_i$ and $q \in A_i$. We may assume, without loss of generality, that (i) holds. 

We claim that $p = a_i$. To prove this, suppose that $p \neq a_i$. Then, since $p$ is to the left of $a_i$, $a_i$ is in the slab $Q$. Since $s>2$, Equation~\eqref{eq:WSPD} yields 
$||p - a_i|| < ||p-q||$, which is a contradiction. 
By a symmetric argument, we have $q = b_i$. Thus, $pq$ is an edge in $G$.
\qed
\vspace{0.4cm}

%\todo{mention in a paragraph that the upper bound also follows from results by Abam et al.~\ci te{DBLP:journals/comgeo/AbamCFS13}. maybe a remark.}

Lemma~\ref{lem:gupta-higher-diminsions} gives us a 
divide-and-conquer approach to prove an upper bound on $f_d(n,m)$: 

\begin{thm} \label{thm:odak-set}
Let $d \geq 2$ be a constant, and let $m$ and $n$ be integers with 
$m = \omega(\sqrt{n})$ and $m \leq n$. Then 
$f_d(n,m) = O(n\log{(m/\sqrt{n})})$. 
\end{thm}

\begin{proof}
    Let $S$ be a set of $n$ points in $\R^d$ for which 
    $f_d(n,m) = |A(S,m)|$. Let $a_1 < a_2 < \cdots < a_{m+1}$ be real numbers such that for each $i = 1,2,\ldots,m$, there are exactly $n/m$ points in $S$ that are strictly inside the vertical slab $\llbracket H_{a_i}, H_{a_{i+1}} \rrbracket$. 
    
    Let $H = H_{a_{1+m/2}}$. Observe that $n/2$ points of $S$ are to the left of $H$ and $n/2$ points of 
    $S$ are to the right of $H$. Denote these two subsets by $S^-$ 
    and $S^+$, respectively. Each pair in $A(S,m)$ is either a pair in 
    $A(S^-,m/2)$ or a pair in $A(S^+,m/2)$ or it crosses $H$. Using 
    Lemma~\ref{lem:gupta-higher-diminsions}, it follows that 
    \begin{eqnarray*} 
        f_d(n,m) & = & |A(S,m)| \\ 
          & = & |A(S^-,m/2)| + |A(S^+,m/2)| + O(n) \\ 
          & \leq & 2 \cdot f_d(n/2,m/2) + O(n) . 
    \end{eqnarray*} 
    If we apply this recurrence $k$ times, we get 
    \[ f_d(n,m) \leq 2^k \cdot f_d (n/2^k,m/2^k) + O(kn) . 
    \]
    For $k=\log (m^2/n)$, we have $n/2^k = n^2/m^2$ and 
    $m/2^k = n/m$. Thus, 
    \[ f_d(n,m) \leq
         \frac{m^2}{n} \cdot f_d \left( n^2/m^2 , n/m \right) + 
         O( n \log(m^2/n) ) . 
    \]
Since $f_d \left( n^2/m^2 , n/m \right) = O(n^2/m^2)$, we conclude that 
\[ f_d(n,m) = O( n + n \log(m^2/n) ) = O( n \log(m/\sqrt{n})) .
    \qedhere
\]
\end{proof}


\section{Lower bounds on $f_d(n, m)$}
\label{seclower}

In this section, we prove the lower bounds on $f_d(n,m)$ as stated in 
Theorem~\ref{thm-main1}. We will prove these lower bounds for the case 
when $d=2$. It is clear that this will imply the same lower bound for 
any constant dimension $d \geq 2$. 

\begin{thm} \label{thm:smid-set}
    Let $n$ and $m$ be positive integers with $n \geq m(m+1)$. 
    Then $f_2(n, m) = {{m+1} \choose 2}$.
\end{thm}

\begin{proof}
It is clear that $f_2(n, m) \leq {{m+1} \choose 2}$. To prove the lower
bound, we will construct a set $S$ of $n$ points in $\R^2$ such that 
%each of the ${m+1} \choose 2$ vertical slabs contains a unique closest pair. 
the ${m+1} \choose 2$ vertical slabs have distinct closest pairs. 

For $i=1,2,\ldots,m+1$, we take $a_i = i$ and consider the corresponding 
hyperplane $H_i$. Let 
$\mathcal{Q} = \{ \llbracket H_i, H_j \rrbracket : 1 \leq i < j \leq m+1 \}$ be the set of all possible vertical slabs. We define the \emph{size} 
of a slab $\llbracket H_i, H_j \rrbracket$ to be the difference $j-i$ 
of their indices. 

We start by constructing a set $P$ of $m(m+1)$ points such that 
%each slab in $\mathcal{Q}$ contains a unique closest pair in $P$, 
the slabs in $\mathcal{Q}$ contain distinct closest pairs in $P$, 
and for each $i=1,2,\ldots,m$, the slab $\llbracket H_i,H_{i+1} \rrbracket$
contains exactly $m+1$ points of $P$. 

Note that the slab $\llbracket H_1 , H_{m+1} \rrbracket$ has the largest
size. Let $p$ be an arbitrary point in $\llbracket H_1,H_2 \rrbracket$
and let $q$ be an arbitrary point in $\llbracket H_m,H_{m+1} \rrbracket$. 
We initialize $P = \{p,q\}$, $D = || p-q ||$, and delete the slab 
$\llbracket H_1 , H_{m+1} \rrbracket$ from $\mathcal{Q}$. 

As long as $\mathcal{Q}$ is non-empty, we do the following: 

\begin{itemize}
\item Take a slab $\llbracket H_i,H_j \rrbracket$ of largest size 
in $\mathcal{Q}$. 
\item Let $p$ be an arbitrary point in 
$\llbracket H_i,H_{i+1} \rrbracket$ such that $p$ is above the bounding 
box of $P$, and the distance between $p$ and any point in $P$ is more than $D+2$. 
\item Let $q$ be an arbitrary point in 
$\llbracket H_{j-1},H_j \rrbracket$ such that $q$ is above the bounding 
box of $P$, the distance between $q$ 
and any point in $P$ is more than $D+2$, and $|| p-q || = D+1$. 
\item Add $p$ and $q$ to $P$.
\item Set $D = || p-q ||$. 
\item Delete the slab $\llbracket H_i , H_j \rrbracket$ from $\mathcal{Q}$.
\end{itemize}

It is not difficult to see that the final point set $P$ has the 
properties stated above. 

To obtain the final point set $S$, of size $n$, we define a set $P'$ of 
$n-m(m+1)$ points, such that each point in $P'$ has distance more than 
$D$ to all points of $P$, the closest pair distance in $P'$ is more 
than $D$, and for each $i=1,2,\ldots,m$, the slab 
$\llbracket H_i , H_{i+1} \rrbracket$ contains $n/m - (m-1)$ points 
of $P'$. The point set $S = P \cup P'$ has the property that 
$|A(S,m)| = {{m+1} \choose 2}$. 
\end{proof}

\begin{cor}
    Let $n$ and $m$ be sufficiently large positive integers with $n < m(m+1)$ and 
    $m \leq 3 \sqrt{n}$. Then $f_2(n,m) = \Omega(m^2)$. 
\end{cor}
\begin{proof}
For $i=1,2,\ldots,m+1$, we take $a_i = i$ and consider the 
corresponding hyperplane $H_i$. 

Let $m' = \sqrt{n}/4$ and $n' = m' (m' + 1)$. We apply 
Theorem~\ref{thm:smid-set}, where we replace $n$ by $n'$ 
and $m$ by $m'$. This gives us a set $S'$ of $n'$ points
with $|A(S',m')| = f_2(n',m')$. The points of $S'$ 
are between the hyperplanes $H_1$ and $H_{m'+1}$; for each 
$i =1,2,\ldots,m'$, the vertical slab
$\llbracket H_i , H_{i+1} \rrbracket$ contains $n'/m'$ 
points of $S'$. Note that 
\[ |A(S',m')| = {{m'+1} \choose 2} = \Omega\left( (m')^2 \right) .
\]
Let $D$ be the diameter of $S'$. 
Let $S$ be the union of $S'$ and a set of $n-n'$ 
additional points that have pairwise distances more than 
$D$, whose distances to the points in $S'$ are more 
than $D$, and such that for each $i=1,2,\ldots,m$, the 
vertical slab $\llbracket H_i , H_{i+1} \rrbracket$ 
contains $n/m$ points of $S$. It is clear that 
\[ f_2(n,m) \geq |A(S',m')|  = \Omega((m')^2) . 
\]
Note that this construction is possible, because 
(i) $n' < n$, (ii) $m' < m$, and (iii) $n'/m' < n/m$; 
these inequalities follow by straightforward algebraic 
manipulations, using the assumptions on $n$ and $m$ in the 
statement of the corollary. Finally, these assumptions 
imply that $m' \geq m/12$. We conclude that 
$f_2(n,m) = \Omega (m^2)$. 
\end{proof}

Before we prove the lower bound for the remaining case, i.e., $m > 3 \sqrt{n}$, we 
consider the case when $m=n$, which will serve as a warm up. 


\begin{thm} \label{thm:odak-set-lower}
    We have $f_2(n,n) = \Omega(n\log{n})$.
\end{thm}

% \begin{figure}
%     \begin{center}
%         \includegraphics[scale=0.6]{figs/cp}
%     \end{center}
%     \caption{The point set $S$ corresponding to a 3-dimensional hypercube, $Q_3$ (the dotted parts of the line segments represent longer distances). \textcolor{Green}{is this caption good enough?}}
%     \label{figs:cp}
% \end{figure}

\begin{proof}
We assume for simplicity that $n$ is a sufficiently large power of two. 
    We will construct a point set $S$ of size $n$ for which 
    $|A(S,n)| = \Omega(n \log n)$. 

    Let $k = \log n$. For $i=0,1,\ldots,k-1$, let $x_i = 2^i$ and let
    $v_i = (x_i,y_i)$ be a vector, where the value of $y_i$ is inductively defined as follows: We set $y_{k-1} = 0$. Assuming that 
    $y_{k-1},y_{k-2},\ldots,y_{i+1}$ have been defined, we set $y_i$ to 
    an integer such that 
    \begin{equation} 
    \label{blacklozenge}
      ||v_i|| > 2\sum_{j=i+1}^{k-1} ||v_j|| . 
    \end{equation} 
    We define 
    \[ S = \left\{ \sum_{i=0}^{k-1} \beta_i v_i : 
        (\beta_0,\beta_1, \cdots, \beta_{k-1}) \in \{0,1\}^k
           \right\} .
    \] 
    Note that each binary sequence of length $k$ represents a unique point 
    in $S$. Using this representation, each point of $S$ corresponds 
    to a vertex of a $k$-dimensional hypercube $Q_k$. 
    We will prove below that each edge of $Q_k$ corresponds to a closest
    pair in a unique vertical slab. Since $Q_k$ has 
    $k \cdot 2^{k-1} = \Omega(n \log n)$ edges, this will complete the 
    proof. 

    Consider an arbitrary edge of $Q_k$. The two vertices of this edge 
    are binary sequences of length $k$ that have Hamming distance one, 
    i.e., they differ in exactly one bit. Let $t$ be the position at 
    which they differ. Observe that $0 \leq t \leq k-1$. Let $r$ and 
    $s$ be the points of $S$ that correspond to the two vertices of 
    this edge. Then $v_t$ is either $r-s$ or $s-r$. We will prove that 
    $r$ and $s$ form the closest pair in the vertical slab 
    $\llbracket H_{r_1} , H_{s_1} \rrbracket$, where $r_1$ and $s_1$ are the first 
    coordinates of $r$ and $s$, respectively (assuming that 
    $r_1 < s_1$). Note that $r_1$ and $s_1$ are integers. 

    Let $p$ and $q$ be two points in 
    $\llbracket H_{ r_1 } , H_{ s_1 }\rrbracket \cap S$ such that 
    $\{ p,q \} \neq \{ r,s \}$. We have to show that 
    $|| r-s || < || p-q ||$. Since $p$ and $q$ are points in $S$, 
    we can write them as 
    \[ p = \sum_{i=0}^{k-1} \beta_{p,i} v_i
    \]
    and 
    \[ q = \sum_{i=0}^{k-1} \beta_{q,i} v_i . 
    \]
    Let $\ell$ be the smallest index for which 
    $\beta_{p,\ell} \neq \beta_{q,\ell}$. Then 
    \[ || p-q || = \left| \left| \sum_{i=\ell}^{k-1} 
               ( \beta_{p,i} - \beta_{q,i} ) v_i \right| \right| . 
    \]
    By the triangle inequality, we have $||a+b|| \leq ||a||+||b||$ and 
    $||a+b|| \geq ||a||-||b||$ for any two vectors $a$ and $b$. These two inequalities, 
    together with (\ref{blacklozenge}), imply that 
    \[ || p-q || \geq || v_{\ell} || - 
            \sum_{i=\ell+1}^{k-1} || v_i || > 
            \frac{1}{2} \cdot || v_{\ell} || . 
    \]
    Thus, it suffices to show that 
    \[ || v_{\ell} || \geq 2 \cdot || r-s || = 2 \cdot || v_t || .
    \]
    If we can show that $\ell < t$, then, using (\ref{blacklozenge}), 
    \[ || v_{\ell} || > 2 \sum_{i=\ell+1}^{k-1} || v_i || 
         \geq 2 \cdot || v_t || 
    \]
    and the proof is complete. 

    The horizontal distance between $p$ and $q$ (i.e., $| p_1-q_1 |$) 
    is smaller than the horizontal distance between $r$ and $s$. 
    Recall that $v_t$ is either $r-s$ or $s-r$. Thus, the horizontal 
    distance between $r$ and $s$ is equal to the first coordinate of 
    the vector $v_t$, which is $x_t$. 

    Let $\ell'$ be the largest index for which 
    $\beta_{p,\ell'} \neq \beta_{q,\ell'}$. 
    If $\ell = \ell'$, then 
    \[ | p_1 - q_1 | = | \beta_{p,\ell} - \beta_{q,\ell} | x_{\ell} 
               = x_{\ell} .
    \]
    Assume that $\ell < \ell'$. We may assume, without loss
    of generality, that 
    $\beta_{p,\ell'} = 1$ and $\beta_{q,\ell'} = 0$. 
    Then (recall that $x_i = 2^i$), 
    \begin{eqnarray*} 
      | p_1 - q_1 | & = & \left| x_{\ell'} + \sum_{i=\ell}^{\ell'-1} 
               ( \beta_{p,i} - \beta_{q,i} ) x_i \right| \\ 
               & = & \left| 2^{\ell'} + \sum_{i=\ell}^{\ell'-1} 
               ( \beta_{p,i} - \beta_{q,i} ) 2^i \right| \\ 
         & \geq & 2^{\ell'} - \sum_{i=\ell}^{\ell'-1} 2^i \\ 
         & = & 2^{\ell} \\ 
         & = & x_{\ell}. 
    \end{eqnarray*}
    We conclude that 
    \[ x_{\ell} \leq | p_1 - q_1 | < x_t , 
    \]
    which is equivalent to $\ell < t$. 
\end{proof}

\begin{thm} \label{prop:toth-set}
    Let $n$ and $m$ be positive integers with $3 \sqrt{n} < m \leq n$.
    Then $f_2(n,m) = \Omega(n\log{(m/\sqrt{n})})$.
\end{thm}

\begin{proof}
Our approach will be to use multiple scaled and shifted copies of the construction in Theorem~\ref{thm:odak-set-lower} to define a set $S$ of 
$n$ points in $\R^2$ for which $|A(S,m)| = \Omega(n\log{(m/\sqrt{n})})$.

For $i=1,2,\ldots,m+1$, we take $a_i = i$ and consider the corresponding 
vertical hyperplane $H_i$. For each $i=1,2,\ldots,m$, the point set $S$ will 
contain exactly $n/m$ points in the vertical slab
$\llbracket H_i , H_{i+1} \rrbracket$. 

Throughout the proof, we will use the following notation. 
Let $v_0, v_1, \ldots, v_{k-1}$ be a sequence of pairwise 
distinct vectors in 
the plane. The \emph{hypercube-set} defined by these vectors is the 
point set 
\[ Q(v_0,v_1,\ldots,v_{k-1}) = 
    \left\{\sum_{i=0}^{k-1} \beta_i v_i 
        : (\beta_0, \beta_1, \cdots, \beta_{k-1}) \in \{0,1\}^k \right\}.
\]

For each $g = 1,2,\ldots,n/m$, we define a hypercube-set $Q_g$: 
\begin{itemize}
    \item Let $k_g = \lfloor \log (m/(2g-1)) \rfloor$.
    \item For each $i=0,1,\ldots,k_g-1$, let 
          \[ x_{g,i} = (2g-1) \cdot 2^i 
          \]
          and let 
          \[ v_{g,i} = ( x_{g,i} , y_{g,i} ) 
          \] 
          be a vector, whose second coordinate $y_{g,i}$ will be 
          defined later. 
    \item Let 
          \[ Q_g = Q( v_{g,0} , v_{g,1} , \ldots, v_{g,k_g-1}) . 
          \]
\end{itemize}
Since, for integers $g$, $g'$, $i$, and $i'$, 
$(2g-1) \cdot 2^i = (2g'-1) \cdot 2^{i'}$ if and only if 
$g=g'$ and $i = i'$, then all values $x_{g,i}$ are pairwise distinct. 

To define the values $y_{g,i}$, we sort all vectors $v_{g,i}$ by their 
first coordinates. We go through the sorted sequence in decreasing 
order:
\begin{itemize}
    \item For the vector with the largest first coordinate, we set its 
    $y$-value to zero. 
    \item For each subsequent vector $v_{g,i}$, we set $y_{g,i}$ to be an 
    integer such that 
    \begin{equation} \label{eq2} 
      || v_{g,i} || > 2 \sum_{g',i'} || v_{g',i'} || ,        
    \end{equation}
    where the summation is over all pairs $g',i'$ for which $y_{g',i'}$ 
    has already been defined (i.e., $x_{g,i} < x_{g',i'}$). 
\end{itemize}

We choose pairwise distinct real numbers $0 < \eps_g < 1$, for 
$g = 1,2,\ldots,n/m$, and set 
\[ \Delta = 1 + \max \{ \DIAM (Q_g) : 1 \leq g \leq n/m \},
\]
where $\DIAM(Q_g)$ denotes the diameter of the point set $Q_g$. 

For each $g=1,2,\ldots,n/m$ and $i=1,2,\ldots,2g-1$, let 
\[ S_{g,i} = Q_g + 
    ( i + \eps_g , 2 ( (g-1)^2 + i-1 ) \Delta ) ,\]
that is, $S_{g,i}$ is the translate of $Q_g$ by the vector $( i + \eps_g , 2 ( (g-1)^2 + i-1 ) \Delta )$.
We define $S'$ to be the union of all these sets $S_{g,i}$, i.e., 
\[ S' = \bigcup_{g=1}^{n/m} \bigcup_{i=1}^{2g-1} S_{g,i} . 
\]
Note that the sets $S_{g,i}$ are pairwise  disjoint: Indeed if $g\neq g'$ or $i\neq i'$, then the $y$-projections of $S_{g,i}$ and $S_{g',i'}$  (i.e., the sets of second coordinates) are disjoint by construction. 
Consequently, the size of the union of these point sets satisfies 
\[  |S'| 
= \sum_{g=1}^{n/m} \sum_{i=1}^{2g-1} |S_{g,i}|
= \sum_{g=1}^{n/m} \sum_{i=1}^{2g-1} 2^{k_g} 
      \leq \sum_{g=1}^{n/m} (2g-1) \cdot \frac{m}{2g-1} 
     =  n .
\]
For each $1 \le g \le n/m$, by construction of $Q_g$ and the fact that the sets $S_{g,i}$ are disjoint translations of $Q_g$, each slab $\llbracket H_j , H_{j+1} \rrbracket$ contains at most one point of $\bigcup_{i=1}^{2g-1} S_{g,i}$. Therefore, each slab $\llbracket H_j , H_{j+1} \rrbracket$ contains at most $n/m$ points of $S'$.

To obtain the final point set $S$ of size $n$, we add $n-|S'|$ points to $S'$ 
such that 
each slab $\llbracket H_j , H_{j+1} \rrbracket$ contains exactly 
$n/m$ points of $S$, and the added points are sufficiently far from each 
other and from all points of $S'$.

In the rest of this proof, we will prove the following claim: For each 
$g=1,2,\ldots,n/m$, consider two binary strings of length $k_g$ that 
differ in exactly one position (recall that the number of such pairs of 
strings is equal to $k_g \cdot 2^{k_g-1}$). These strings correspond to
two points of the hypercube-set $Q_g$. Thus, for any 
$i=1,2,\ldots,2g-1$, they
correspond to two points, say $r$ and $s$, in the set $S_{g,i}$. 
We claim that $r$ and $s$ form the closest pair in the set 
$\llbracket H_{\lfloor r_1 \rfloor} , H_{\lceil s_1 \rceil} \rrbracket \cap S$, where $r_1$ and $s_1$ are
the first coordinates of $r$ and $s$, respectively (assuming that 
$r_1 < s_1$). Note that we take the floor and the ceiling, because $r_1$ and $s_1$ are not integers. This claim will imply that 
\[ f_2(n,m) \geq |A(S,m)| \geq 
  \sum_{g=1}^{n/m} \sum_{i=1}^{2g-1} k_g \cdot 2^{k_g-1} .
\]
Since 
\[ k_g > \log \left( \frac{m}{2g-1} \right) - 1 ,
\]
we get 
\begin{eqnarray*}
    f_2(n,m) & \geq & 
    \sum_{g=1}^{n/m} (2g-1) 
       \left( \log \left( \frac{m}{2g-1} \right) - 1 \right) \cdot 
          \frac{m}{4(2g-1)} \\ 
    & = & \sum_{g=1}^{n/m} \frac{m}{4} 
                    \log \left( \frac{m}{2g-1} \right) - 
                    \sum_{g=1}^{n/m} \frac{m}{4} . 
\end{eqnarray*}
Since each term in the first summation is larger than the last term, 
which is larger than $(m/4) \log ( m^2 / (2n) )$, we get 
\begin{eqnarray*} 
 f_2(n,m) & \geq & \frac{n}{m} \cdot \frac{m}{4} \log ( m^2 / (2n) ) 
          - \frac{n}{4} \\ 
          & = & \frac{n}{4} 
          \left( \log (m^2/(2n)) - 1 \right) . 
\end{eqnarray*}  
Since $m > 3 \sqrt{n}$, 
\[ \log (m^2/(2n)) - 1 = \Omega (\log (m^2/(2n))) . 
\]
We conclude that 
\[ f_2(n,m) = \Omega (n \log (m^2/n)) = 
    \Omega( n \log ( m / \sqrt{n} )) . 
\]

It remains to prove the above claim. 
Let $g$ and $i$ be integers with $1 \leq g \leq n/m$ and 
$1 \leq i \leq 2g-1$. Consider two binary strings of length $k_g$ that 
differ in exactly one position; denote this position by $t$. 
Let $r$ and $s$ be the two corresponding points of $S_{g,i}$. 
Note that the vector $v_{g,t}$ is equal to either $r-s$ or $s-r$. 

We may assume, without loss of generality, that $r_1 < s_1$.  
To prove that $r$ and $s$ form the closest pair in the set 
$\llbracket H_{\lfloor r_1 \rfloor} , H_{\lceil s_1 \rceil} \rrbracket \cap S$, we consider an 
arbitrary pair $p$ and $q$ of points in 
$\llbracket H_{\lfloor r_1 \rfloor} , H_{\lceil s_1 \rceil} \rrbracket \cap S$ such that 
$\{p,q\} \neq \{r,s\}$. We will show that $|| r-s || < || p-q ||$.

Let $g'$, $g''$, $i'$, and $i''$ be such that 
$p \in S_{g',i'}$ and $q \in S_{g'',i''}$. If $g' \neq g''$ or 
$i' \neq i''$, then 
\[ \| p-q \| \geq | p_2 - q_2 | \geq \Delta > 
     \DIAM(Q_g) \geq \| r-s \| . 
\]
In the rest of the proof, we assume that $g' = g''$ and $i' = i''$. 
Since both $p$ and $q$ are in $S_{g',i'}$, we can write them as 
\[ p = \sum_{j=0}^{k_{g'}-1} \beta_{p,j} v_{g',j} + 
    ( i' + \eps_{g'} , 2 ( (g'-1)^2 + i'-1 ) \Delta ) 
\]
and 
\[ q = \sum_{j=0}^{k_{g'}-1} \beta_{q,j} v_{g',j} + 
    ( i' + \eps_{g'} , 2 ( (g'-1)^2 + i'-1 ) \Delta ) . 
\]
Let $\ell$ be the smallest index such that $0 \leq \ell \leq k_{g'}-1$ 
and $\beta_{p,\ell} \neq \beta_{q,\ell}$. Using the triangle 
inequality and (\ref{eq2}), we have 
\begin{eqnarray*} 
 || p-q || & = & 
 \left| \left| \sum_{j=\ell}^{k_{g'}-1} ( \beta_{p,j} - \beta_{q,j} ) 
                            v_{g',j} 
 \right| \right| \\ 
 & \geq & || v_{g',\ell} || - 
          \sum_{j=\ell+1}^{k_{g'}-1} || v_{g',j}|| \\ 
    & > & \frac{1}{2} \cdot || v_{g',\ell} || . 
\end{eqnarray*} 
Thus, it suffices to show that 
\[ || v_{g',\ell} || \geq 2 \cdot || r-s || = 2 \cdot || v_{g,t} || . 
\]
 Since both $p$ and $q$ are in $S_{g',i'}$, we have $|p_1 - q_1| \le |r_1-s_1|$ (where $r$ and $s$ are each at distance $\eps_g$ from the left boundary of their corresponding slabs, and $p$ and $q$ are at distance $\eps_{g'}$ from the left boundary of their corresponding slabs). Let $\ell'$ be the largest index such that $0 \leq \ell' \leq k_{g'}-1$ 
and $\beta_{p,\ell'} \neq \beta_{q,\ell'}$. We have 
\begin{eqnarray*}
    x_{g,t} & = & | r_1 - s_1| \\
    &  \ge & | p_1 - q_1 | \\
    & = & \left| \sum_{j=\ell}^{\ell'} ( \beta_{p,j} - \beta_{q,j} ) 
                       x_{g',j} 
           \right| \\ 
    & \geq & x_{g',\ell'} - \sum_{j=\ell}^{\ell'-1} x_{g',j} \\ 
    & = & ( 2g'-1 ) 
    \left( 2^{\ell'} - \sum_{j=\ell}^{\ell'-1} 2^j
    \right) \\ 
    & = & ( 2g'-1 ) \cdot 2^{\ell} \\ 
    & = & x_{g',\ell} . 
\end{eqnarray*}
Since $\{p,q\} \neq \{r,s\}$, then $x_{g,t}$ cannot be equal to $x_{g',\ell}$. Therefore, $x_{g,t} > x_{g',\ell}$. Using (\ref{eq2}), it then follows that 
$|| v_{g',\ell} || \geq 2 \cdot || v_{g,t} ||$. 
\end{proof}

\section{The Data Structure} \label{data-structure}

In this section, we will present a data structure for vertical closest
pair queries. Our data structure will use the results in the following three lemmas. 

\begin{lem} 
\label{lemshortest}
Let $S$ be a set of $n$ points in $\R^d$ and let $L$ be a set of $k$ 
line segments such that the endpoints of each segment belongs to $S$. 
There exists a data structure of size $O(n+k)$, such that for any two 
real numbers $a$ and $b$ with $a<b$, the shortest segment in $L$ that 
is completely inside the vertical slab $\llbracket H_a,H_b \rrbracket$
can be reported in $O(\log n)$ time. 
\end{lem}
\begin{proof}
Xue \emph{et al.}~\cite[Section 3]{DBLP:journals/dcg/XueLRJ22} proved the
claim in the case where $d=2$. A careful analysis of their construction
shows that the claim in fact holds for any constant dimension $d \geq 2$. 
\end{proof}

The next lemma is due to Mehlhorn~\cite[page~44]{m-mdscg-84}; see Smid~\cite{smid} for a complete analysis of this data structure.

\begin{lem} \label{lem:range-report}
Let $S$ be a set of $n$ points in $\R^d$ and let $\eps>0$ be a real constant. There exists a data structure of size $O(n)$, such that for any axis-parallel rectangular box $B$, all points in $B \cap S$ can be reported in $O(n^\eps + |B \cap S|)$ time. 
\end{lem}

The last tool that we need is a standard sparsity property. 

\begin{lem} \label{obs:constant-points}
    Let $r>0$ be a real number, and let $X$ be a set of points in $\R^d$ that are contained in an $r \times 2r \times 2r \times \cdots \times 2r$ rectangular box $B$. If the distance of the closest pair of points in $X$ is at least $r$, then $|X| \le 2^{d-1} \cdot c^d$, where $c = 1 + \lceil \sqrt{d} \rceil$.
\end{lem}
\begin{proof}
Partition $B$ into $2^{d-1} \cdot c^d$ hypercubes, each one being an
%$r/c \times r/c \times \cdots \times r/c$ 
$\frac{r}{c} \times\frac{r}{c}  \times \cdots \times \frac{r}{c}$ 
box. Since the diameter of any such box is $\sqrt{d} \cdot r/c$, which is less than $r$, it contains at most one point of $X$.
\end{proof}

In the rest of this section, we will prove the following result.

\begin{thm} \label{thm:main2}
    Let $S$ be a set of $n$ points in $\R^d$, let $m$ be an integer 
    with $1 \leq m \leq n$, and let $\eps>0$ be a real constant. 
    There exists a data structure of size $O(n + f_d(n,m))$ such that 
    for any two real numbers $a$ and $b$ with $a<b$, the closest pair 
    in the vertical slab $\llbracket H_a,H_b \rrbracket$ can be 
    reported in $O(n^{1+\eps} /m)$ time.
\end{thm}

\begin{proof}
Let $a_1 < a_2 < \cdots < a_{m+1}$ be real numbers such that for each 
$i = 1,2,\ldots,m$, the vertical slab 
$\llbracket H_{a_i} , H_{a_{i+1}} \rrbracket$ contains $n/m$ points 
of $S$. Let $k = |A(S,m)|$. Note that $k \leq f_d(n,m)$. Our data structure 
consists of the following components: 

\begin{itemize} 
\item An array storing the numbers $a_1,a_2,\ldots,a_{m+1}$. For each 
$i=1,2,\ldots,m$, the $i$-th entry stores, besides the number $a_i$, 
a list of all points in 
$\llbracket H_{a_i} , H_{a_{i+1}} \rrbracket \cap S$. 
\item The data structure of Lemma~\ref{lemshortest}, where 
$L$ is the set of line segments corresponding to the pairs in 
$\{ \CP(S,H_{a_i},H_{a_j}) : 1 \leq i < j \leq m+1 \}$. 
\item The data structure of Lemma~\ref{lem:range-report}.
\end{itemize} 
The size of the entire data structure is 
$O(m+n+k)$, which is $O(n + f_d(n,m))$. 

We next describe the query algorithm. Let $a$ and $b$ be real numbers 
with $a<b$. Using binary search, we compute, in 
$O(\log m) = O(n^{1+\eps} / m)$ time, 
the indices $i$ and $j$ such that $H_a$ is in the slab 
$\llbracket H_{a_{i-1}} , H_{a_i} \rrbracket$ 
and $H_b$ is in the slab 
$\llbracket H_{a_j} , H_{a_{j+1}} \rrbracket$. 

If $i=j$, then the slab $\llbracket H_a,H_b \rrbracket$ contains 
$O(n/m)$ points of $S$. In this case, we use the  
algorithm of Bentley and Shamos~\cite{bs-dcms-76} to compute the 
closest pair in $\llbracket H_a,H_b \rrbracket$ in 
$O((n/m) \log (n/m)) = O(n^{1+\eps} / m)$ time. 


\begin{figure}
    \begin{center}
        \includegraphics[scale=0.9]{figs/regions.pdf}
    \end{center}
    \caption{(A), (B), and (C) are the three regions created by a query $\llbracket H_a,H_b \rrbracket$. The rectangle $R_p$ is the range query for the point $p$ with respect to the query $\llbracket H_a,H_b \rrbracket$.}
    \label{figs:regions}
\end{figure}

Assume that $i < j$. The two hyperplanes $H_{a_i}$ and $H_{a_j}$ split the query slab $\llbracket H_a, H_b \rrbracket$ into three parts $(A)$, $(B)$, and $(C)$, where 
$(A)$ is the slab $\llbracket H_a , H_{a_i} \rrbracket$, 
$(B)$ is the slab $\llbracket H_{a_i} , H_{a_j} \rrbracket$, and 
$(C)$ is the slab $\llbracket H_{a_j} , H_b \rrbracket$; refer to 
Figure~\ref{figs:regions}. 

Let $S_{AC}$ be the set of points in $S$ that are in the union of $(A)$ 
and $(C)$, and let $S_B$ be the set of points in $S$ that are in $(B)$. 
There are three possibilities for the closest pair in 
$\llbracket H_a,H_b \rrbracket$: Both endpoints are in $S_{AC}$, both 
endpoints are in $S_B$, or one endpoint is in $S_{AC}$ and the other 
endpoint is in $S_B$. 

Using the algorithm of Bentley and Shamos~\cite{bs-dcms-76}, we 
compute the closest pair distance $\delta_1$ in $S_{AC}$, in 
$O((n/m) \log (n/m)) = O(n^{1+\eps} / m)$ time. Using the data 
structure of Lemma~\ref{lemshortest}, we compute the closest pair 
distance $\delta_2$ in $S_B$ in 
$O(\log n) = O(n^{1+\eps} / m)$ time. 

Let $\delta = \min(\delta_1,\delta_2)$. In the final part of the query 
algorithm, we use the data structure of Lemma~\ref{lem:range-report}:

\begin{itemize} 
\item For each point $p$ in the region $(A)$, we compute the set of 
all points in $S$ that are in the part, say $P_p$, of the axes-parallel box 
\[ [ p_1,p_1+\delta ] \times [ p_2-\delta, p_2+\delta]
    \times \cdots \times [ p_d-\delta,p_d+\delta] 
\]
that is to the left of $H_{a_j}$. Then we compute $\delta_p$, which is 
the minimum distance between $p$ and any point inside $S \cap P_p$. 
\item For each point $p$ in the region $(C)$, we compute the set of 
all points in $S$ that are in the part, say $P'_p$, of the axes-parallel box 
\[ [ p_1-\delta,p_1 ] \times [ p_2-\delta, p_2+\delta]
    \times \cdots \times [ p_d-\delta,p_d+\delta] 
\]
that is to the right of $H_{a_i}$. Then we compute $\delta_p$, which is
the minimum distance between $p$ and any point inside $S \cap P'_p$. 
\item At the end , we return the minimum of $\delta$ and 
$\min \{ \delta_p : p \in S_{AC} \}$. 
\end{itemize}
By Lemma~\ref{obs:constant-points}, the boxes $P_p$ and $P'_p$ each contain
$O(1)$ points of $S$. In total, there are $O(n/m)$ queries to the 
data structure of Lemma~\ref{lem:range-report}, and each one 
takes $O(n^{\eps})$ time. Thus, this final part of the query 
algorithm takes $O((n/m) \cdot n^{\eps}) = O(n^{1+\eps} / m)$ time. 
\end{proof}

The proof of Theorem~\ref{thm-main2} follows by taking $m = \sqrt{n}$ 
in Theorem~\ref{thm:main2} and using Theorem~\ref{thm-main1}.


% \todo[inline, color=pink, bordercolor=purple] {This text is useless now. but we should write something here.

% In order to design the linear space data structure, using \cref{thm:main2}, we partition the points in the set $S$ using $O(\sqrt{n})$ vertical hyperplanes into blocks of size $\sqrt{n}$. Naturally, by considering more lines to partition, one might want to use a technique as in $O(n\log{n})$ space data structure \cite{DBLP:journals/dcg/XueLRJ22}. Assume we partition the point set $S$, using $m = \Theta(n^{2/3})$ lines, into blocks of size at most $n^{1/3}$. How many candidate pairs are there with respect to these $m$ lines? Can it be linear in $n$? These questions motivate our definition of the combinatorial parameter $f_d(n,m)$ for the point sets of size $n$.}

\section{Future Work}
\label{future} 

The point sets that we constructed for the lower bounds on 
$f_d(n,m)$ have coordinates that are at least 
exponential 
in the number of points. Recall that the \emph{spread} 
(also known as \emph{aspect ratio})
of a point set is the ratio of the diameter and the closest pair distance. 
It is well-known that the spread of any set of $n$ points in $\R^d$ is 
$\Omega(n^{1-1/d})$. It is natural to define 
$f_d(n,m,\Phi)$ as the quantity analogous to $f_d(n,m)$, where 
we only consider sets of $n$ points in $\R^d$ having spread at most 
$\Phi$. 

\begin{prb}
    Determine the value of $f_d(n,m,\Phi)$.
\end{prb}

For any set $S$ of $n$ points in $\R^d$, where $d=2$, 
Xue \emph{et al.}~\cite{DBLP:journals/dcg/XueLRJ22} have presented a 
data structure of size $O(n \log n)$ that can be used to answer 
vertical slab closest pair queries in $O(\log n)$ time. Our data 
structure uses only $O(n)$ space and works in any constant dimension 
$d \geq 2$. However, its query time is $O(n^{1/2+\eps})$. 

\begin{prb}
    Is there a linear space data structure that supports vertical slab closest pair queries in $o(\sqrt{n})$ time, or even 
    in $O(\polylog{(n)})$ time?
\end{prb}
 

%As mentioned at the beginning of the \cref{data-structure}, Xue et al. \cite{DBLP:journals/dcg/XueLRJ22}, in their data structure, find all $O(n \log{n})$ closest pairs in advance, storing them in a data structure, after which the query is reduced to just point location in the plane subdivisions. However, in our case, we only store $O(n)$ pairs, leaving $\Omega(n \log{n})$ pairs undiscovered in the worst case. Consequently, our query would involve much more than just point location queries. The computation of the remaining pairs needs to be handled during the query, which could easily increase the query time.

Another interesting research direction is to design linear space data 
structures for closest pair queries with other types of ranges, such 
as axes-parallel 3-sided ranges and general axes-parallel 
rectangular ranges. 

%%
%% Csaba: Hide acknowledgement for the anonymous submission
%%

\section*{Acknowledgement}

This research was partly conducted at the Eleventh Annual Workshop on Geometry and Graphs, held at the Bellairs Research Institute in Barbados, March 8--15, 2024. The authors are grateful to the organizers and to the participants of this workshop.


%%
%% Bibliography
%%

%% Please use bibtex, 

\bibliography{cpq}

%\appendix

\end{document}