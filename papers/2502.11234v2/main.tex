% CVPR 2025 Paper Template; see https://github.com/cvpr-org/author-kit

\documentclass[10pt,twocolumn,letterpaper]{article}

%%%%%%%%% PAPER TYPE  - PLEASE UPDATE FOR FINAL VERSION
%\usepackage{cvpr}              % To produce the CAMERA-READY version
%\usepackage[review]{iccv}      % To produce the REVIEW version
\usepackage[pagenumbers]{cvpr} % To force page numbers, e.g. for an arXiv version

\usepackage{lineno}
% Import additional packages in the preamble file, before hyperref
%
% --- inline annotations
%
\newcommand{\red}[1]{{\color{red}#1}}
\newcommand{\todo}[1]{{\color{red}#1}}
\newcommand{\TODO}[1]{\textbf{\color{red}[TODO: #1]}}
% --- disable by uncommenting  
% \renewcommand{\TODO}[1]{}
% \renewcommand{\todo}[1]{#1}



\newcommand{\VLM}{LVLM\xspace} 
\newcommand{\ours}{PeKit\xspace}
\newcommand{\yollava}{Yo’LLaVA\xspace}

\newcommand{\thisismy}{This-Is-My-Img\xspace}
\newcommand{\myparagraph}[1]{\noindent\textbf{#1}}
\newcommand{\vdoro}[1]{{\color[rgb]{0.4, 0.18, 0.78} {[V] #1}}}
% --- disable by uncommenting  
% \renewcommand{\TODO}[1]{}
% \renewcommand{\todo}[1]{#1}
\usepackage{slashbox}
% Vectors
\newcommand{\bB}{\mathcal{B}}
\newcommand{\bw}{\mathbf{w}}
\newcommand{\bs}{\mathbf{s}}
\newcommand{\bo}{\mathbf{o}}
\newcommand{\bn}{\mathbf{n}}
\newcommand{\bc}{\mathbf{c}}
\newcommand{\bp}{\mathbf{p}}
\newcommand{\bS}{\mathbf{S}}
\newcommand{\bk}{\mathbf{k}}
\newcommand{\bmu}{\boldsymbol{\mu}}
\newcommand{\bx}{\mathbf{x}}
\newcommand{\bg}{\mathbf{g}}
\newcommand{\be}{\mathbf{e}}
\newcommand{\bX}{\mathbf{X}}
\newcommand{\by}{\mathbf{y}}
\newcommand{\bv}{\mathbf{v}}
\newcommand{\bz}{\mathbf{z}}
\newcommand{\bq}{\mathbf{q}}
\newcommand{\bff}{\mathbf{f}}
\newcommand{\bu}{\mathbf{u}}
\newcommand{\bh}{\mathbf{h}}
\newcommand{\bb}{\mathbf{b}}

\newcommand{\rone}{\textcolor{green}{R1}}
\newcommand{\rtwo}{\textcolor{orange}{R2}}
\newcommand{\rthree}{\textcolor{red}{R3}}
\usepackage{amsmath}
%\usepackage{arydshln}
\DeclareMathOperator{\similarity}{sim}
\DeclareMathOperator{\AvgPool}{AvgPool}

\newcommand{\argmax}{\mathop{\mathrm{argmax}}}     


\usepackage{graphicx} % Add this to your preamble if not already included

% It is strongly recommended to use hyperref, especially for the review version.
% hyperref with option pagebackref eases the reviewers' job.
% Please disable hyperref *only* if you encounter grave issues, 
% e.g. with the file validation for the camera-ready version.
%
% If you comment hyperref and then uncomment it, you should delete *.aux before re-running LaTeX.
% (Or just hit 'q' on the first LaTeX run, let it finish, and you should be clear).
\definecolor{cvprblue}{rgb}{0.21,0.49,0.74}
\usepackage[pagebackref,breaklinks,colorlinks,allcolors=cvprblue]{hyperref}

%%%%%%%%% PAPER ID  - PLEASE UPDATE
\def\paperID{33} % *** Enter the Paper ID here
\def\confName{ICCV}
\def\confYear{2025}

%%%%%% Tao Hu
%%%%% NEW MATH DEFINITIONS %%%%%

\usepackage{amsmath,amsfonts,bm}
\usepackage{derivative}
% Mark sections of captions for referring to divisions of figures
\newcommand{\figleft}{{\em (Left)}}
\newcommand{\figcenter}{{\em (Center)}}
\newcommand{\figright}{{\em (Right)}}
\newcommand{\figtop}{{\em (Top)}}
\newcommand{\figbottom}{{\em (Bottom)}}
\newcommand{\captiona}{{\em (a)}}
\newcommand{\captionb}{{\em (b)}}
\newcommand{\captionc}{{\em (c)}}
\newcommand{\captiond}{{\em (d)}}

% Highlight a newly defined term
\newcommand{\newterm}[1]{{\bf #1}}

% Derivative d 
\newcommand{\deriv}{{\mathrm{d}}}

% Figure reference, lower-case.
\def\figref#1{figure~\ref{#1}}
% Figure reference, capital. For start of sentence
\def\Figref#1{Figure~\ref{#1}}
\def\twofigref#1#2{figures \ref{#1} and \ref{#2}}
\def\quadfigref#1#2#3#4{figures \ref{#1}, \ref{#2}, \ref{#3} and \ref{#4}}
% Section reference, lower-case.
\def\secref#1{section~\ref{#1}}
% Section reference, capital.
\def\Secref#1{Section~\ref{#1}}
% Reference to two sections.
\def\twosecrefs#1#2{sections \ref{#1} and \ref{#2}}
% Reference to three sections.
\def\secrefs#1#2#3{sections \ref{#1}, \ref{#2} and \ref{#3}}
% Reference to an equation, lower-case.
\def\eqref#1{equation~\ref{#1}}
% Reference to an equation, upper case
\def\Eqref#1{Equation~\ref{#1}}
% A raw reference to an equation---avoid using if possible
\def\plaineqref#1{\ref{#1}}
% Reference to a chapter, lower-case.
\def\chapref#1{chapter~\ref{#1}}
% Reference to an equation, upper case.
\def\Chapref#1{Chapter~\ref{#1}}
% Reference to a range of chapters
\def\rangechapref#1#2{chapters\ref{#1}--\ref{#2}}
% Reference to an algorithm, lower-case.
\def\algref#1{algorithm~\ref{#1}}
% Reference to an algorithm, upper case.
\def\Algref#1{Algorithm~\ref{#1}}
\def\twoalgref#1#2{algorithms \ref{#1} and \ref{#2}}
\def\Twoalgref#1#2{Algorithms \ref{#1} and \ref{#2}}
% Reference to a part, lower case
\def\partref#1{part~\ref{#1}}
% Reference to a part, upper case
\def\Partref#1{Part~\ref{#1}}
\def\twopartref#1#2{parts \ref{#1} and \ref{#2}}

\def\ceil#1{\lceil #1 \rceil}
\def\floor#1{\lfloor #1 \rfloor}
\def\1{\bm{1}}
\newcommand{\train}{\mathcal{D}}
\newcommand{\valid}{\mathcal{D_{\mathrm{valid}}}}
\newcommand{\test}{\mathcal{D_{\mathrm{test}}}}

\def\eps{{\epsilon}}


% Random variables
\def\reta{{\textnormal{$\eta$}}}
\def\ra{{\textnormal{a}}}
\def\rb{{\textnormal{b}}}
\def\rc{{\textnormal{c}}}
\def\rd{{\textnormal{d}}}
\def\re{{\textnormal{e}}}
\def\rf{{\textnormal{f}}}
\def\rg{{\textnormal{g}}}
\def\rh{{\textnormal{h}}}
\def\ri{{\textnormal{i}}}
\def\rj{{\textnormal{j}}}
\def\rk{{\textnormal{k}}}
\def\rl{{\textnormal{l}}}
% rm is already a command, just don't name any random variables m
\def\rn{{\textnormal{n}}}
\def\ro{{\textnormal{o}}}
\def\rp{{\textnormal{p}}}
\def\rq{{\textnormal{q}}}
\def\rr{{\textnormal{r}}}
\def\rs{{\textnormal{s}}}
\def\rt{{\textnormal{t}}}
\def\ru{{\textnormal{u}}}
\def\rv{{\textnormal{v}}}
\def\rw{{\textnormal{w}}}
\def\rx{{\textnormal{x}}}
\def\ry{{\textnormal{y}}}
\def\rz{{\textnormal{z}}}

% Random vectors
\def\rvepsilon{{\mathbf{\epsilon}}}
\def\rvphi{{\mathbf{\phi}}}
\def\rvtheta{{\mathbf{\theta}}}
\def\rva{{\mathbf{a}}}
\def\rvb{{\mathbf{b}}}
\def\rvc{{\mathbf{c}}}
\def\rvd{{\mathbf{d}}}
\def\rve{{\mathbf{e}}}
\def\rvf{{\mathbf{f}}}
\def\rvg{{\mathbf{g}}}
\def\rvh{{\mathbf{h}}}
\def\rvu{{\mathbf{i}}}
\def\rvj{{\mathbf{j}}}
\def\rvk{{\mathbf{k}}}
\def\rvl{{\mathbf{l}}}
\def\rvm{{\mathbf{m}}}
\def\rvn{{\mathbf{n}}}
\def\rvo{{\mathbf{o}}}
\def\rvp{{\mathbf{p}}}
\def\rvq{{\mathbf{q}}}
\def\rvr{{\mathbf{r}}}
\def\rvs{{\mathbf{s}}}
\def\rvt{{\mathbf{t}}}
\def\rvu{{\mathbf{u}}}
\def\rvv{{\mathbf{v}}}
\def\rvw{{\mathbf{w}}}
\def\rvx{{\mathbf{x}}}
\def\rvy{{\mathbf{y}}}
\def\rvz{{\mathbf{z}}}

% Elements of random vectors
\def\erva{{\textnormal{a}}}
\def\ervb{{\textnormal{b}}}
\def\ervc{{\textnormal{c}}}
\def\ervd{{\textnormal{d}}}
\def\erve{{\textnormal{e}}}
\def\ervf{{\textnormal{f}}}
\def\ervg{{\textnormal{g}}}
\def\ervh{{\textnormal{h}}}
\def\ervi{{\textnormal{i}}}
\def\ervj{{\textnormal{j}}}
\def\ervk{{\textnormal{k}}}
\def\ervl{{\textnormal{l}}}
\def\ervm{{\textnormal{m}}}
\def\ervn{{\textnormal{n}}}
\def\ervo{{\textnormal{o}}}
\def\ervp{{\textnormal{p}}}
\def\ervq{{\textnormal{q}}}
\def\ervr{{\textnormal{r}}}
\def\ervs{{\textnormal{s}}}
\def\ervt{{\textnormal{t}}}
\def\ervu{{\textnormal{u}}}
\def\ervv{{\textnormal{v}}}
\def\ervw{{\textnormal{w}}}
\def\ervx{{\textnormal{x}}}
\def\ervy{{\textnormal{y}}}
\def\ervz{{\textnormal{z}}}

% Random matrices
\def\rmA{{\mathbf{A}}}
\def\rmB{{\mathbf{B}}}
\def\rmC{{\mathbf{C}}}
\def\rmD{{\mathbf{D}}}
\def\rmE{{\mathbf{E}}}
\def\rmF{{\mathbf{F}}}
\def\rmG{{\mathbf{G}}}
\def\rmH{{\mathbf{H}}}
\def\rmI{{\mathbf{I}}}
\def\rmJ{{\mathbf{J}}}
\def\rmK{{\mathbf{K}}}
\def\rmL{{\mathbf{L}}}
\def\rmM{{\mathbf{M}}}
\def\rmN{{\mathbf{N}}}
\def\rmO{{\mathbf{O}}}
\def\rmP{{\mathbf{P}}}
\def\rmQ{{\mathbf{Q}}}
\def\rmR{{\mathbf{R}}}
\def\rmS{{\mathbf{S}}}
\def\rmT{{\mathbf{T}}}
\def\rmU{{\mathbf{U}}}
\def\rmV{{\mathbf{V}}}
\def\rmW{{\mathbf{W}}}
\def\rmX{{\mathbf{X}}}
\def\rmY{{\mathbf{Y}}}
\def\rmZ{{\mathbf{Z}}}

% Elements of random matrices
\def\ermA{{\textnormal{A}}}
\def\ermB{{\textnormal{B}}}
\def\ermC{{\textnormal{C}}}
\def\ermD{{\textnormal{D}}}
\def\ermE{{\textnormal{E}}}
\def\ermF{{\textnormal{F}}}
\def\ermG{{\textnormal{G}}}
\def\ermH{{\textnormal{H}}}
\def\ermI{{\textnormal{I}}}
\def\ermJ{{\textnormal{J}}}
\def\ermK{{\textnormal{K}}}
\def\ermL{{\textnormal{L}}}
\def\ermM{{\textnormal{M}}}
\def\ermN{{\textnormal{N}}}
\def\ermO{{\textnormal{O}}}
\def\ermP{{\textnormal{P}}}
\def\ermQ{{\textnormal{Q}}}
\def\ermR{{\textnormal{R}}}
\def\ermS{{\textnormal{S}}}
\def\ermT{{\textnormal{T}}}
\def\ermU{{\textnormal{U}}}
\def\ermV{{\textnormal{V}}}
\def\ermW{{\textnormal{W}}}
\def\ermX{{\textnormal{X}}}
\def\ermY{{\textnormal{Y}}}
\def\ermZ{{\textnormal{Z}}}

% Vectors
\def\vzero{{\bm{0}}}
\def\vone{{\bm{1}}}
\def\vmu{{\bm{\mu}}}
\def\vtheta{{\bm{\theta}}}
\def\vphi{{\bm{\phi}}}
\def\va{{\bm{a}}}
\def\vb{{\bm{b}}}
\def\vc{{\bm{c}}}
\def\vd{{\bm{d}}}
\def\ve{{\bm{e}}}
\def\vf{{\bm{f}}}
\def\vg{{\bm{g}}}
\def\vh{{\bm{h}}}
\def\vi{{\bm{i}}}
\def\vj{{\bm{j}}}
\def\vk{{\bm{k}}}
\def\vl{{\bm{l}}}
\def\vm{{\bm{m}}}
\def\vn{{\bm{n}}}
\def\vo{{\bm{o}}}
\def\vp{{\bm{p}}}
\def\vq{{\bm{q}}}
\def\vr{{\bm{r}}}
\def\vs{{\bm{s}}}
\def\vt{{\bm{t}}}
\def\vu{{\bm{u}}}
\def\vv{{\bm{v}}}
\def\vw{{\bm{w}}}
\def\vx{{\bm{x}}}
\def\vy{{\bm{y}}}
\def\vz{{\bm{z}}}

% Elements of vectors
\def\evalpha{{\alpha}}
\def\evbeta{{\beta}}
\def\evepsilon{{\epsilon}}
\def\evlambda{{\lambda}}
\def\evomega{{\omega}}
\def\evmu{{\mu}}
\def\evpsi{{\psi}}
\def\evsigma{{\sigma}}
\def\evtheta{{\theta}}
\def\eva{{a}}
\def\evb{{b}}
\def\evc{{c}}
\def\evd{{d}}
\def\eve{{e}}
\def\evf{{f}}
\def\evg{{g}}
\def\evh{{h}}
\def\evi{{i}}
\def\evj{{j}}
\def\evk{{k}}
\def\evl{{l}}
\def\evm{{m}}
\def\evn{{n}}
\def\evo{{o}}
\def\evp{{p}}
\def\evq{{q}}
\def\evr{{r}}
\def\evs{{s}}
\def\evt{{t}}
\def\evu{{u}}
\def\evv{{v}}
\def\evw{{w}}
\def\evx{{x}}
\def\evy{{y}}
\def\evz{{z}}

% Matrix
\def\mA{{\bm{A}}}
\def\mB{{\bm{B}}}
\def\mC{{\bm{C}}}
\def\mD{{\bm{D}}}
\def\mE{{\bm{E}}}
\def\mF{{\bm{F}}}
\def\mG{{\bm{G}}}
\def\mH{{\bm{H}}}
\def\mI{{\bm{I}}}
\def\mJ{{\bm{J}}}
\def\mK{{\bm{K}}}
\def\mL{{\bm{L}}}
\def\mM{{\bm{M}}}
\def\mN{{\bm{N}}}
\def\mO{{\bm{O}}}
\def\mP{{\bm{P}}}
\def\mQ{{\bm{Q}}}
\def\mR{{\bm{R}}}
\def\mS{{\bm{S}}}
\def\mT{{\bm{T}}}
\def\mU{{\bm{U}}}
\def\mV{{\bm{V}}}
\def\mW{{\bm{W}}}
\def\mX{{\bm{X}}}
\def\mY{{\bm{Y}}}
\def\mZ{{\bm{Z}}}
\def\mBeta{{\bm{\beta}}}
\def\mPhi{{\bm{\Phi}}}
\def\mLambda{{\bm{\Lambda}}}
\def\mSigma{{\bm{\Sigma}}}

% Tensor
\DeclareMathAlphabet{\mathsfit}{\encodingdefault}{\sfdefault}{m}{sl}
\SetMathAlphabet{\mathsfit}{bold}{\encodingdefault}{\sfdefault}{bx}{n}
\newcommand{\tens}[1]{\bm{\mathsfit{#1}}}
\def\tA{{\tens{A}}}
\def\tB{{\tens{B}}}
\def\tC{{\tens{C}}}
\def\tD{{\tens{D}}}
\def\tE{{\tens{E}}}
\def\tF{{\tens{F}}}
\def\tG{{\tens{G}}}
\def\tH{{\tens{H}}}
\def\tI{{\tens{I}}}
\def\tJ{{\tens{J}}}
\def\tK{{\tens{K}}}
\def\tL{{\tens{L}}}
\def\tM{{\tens{M}}}
\def\tN{{\tens{N}}}
\def\tO{{\tens{O}}}
\def\tP{{\tens{P}}}
\def\tQ{{\tens{Q}}}
\def\tR{{\tens{R}}}
\def\tS{{\tens{S}}}
\def\tT{{\tens{T}}}
\def\tU{{\tens{U}}}
\def\tV{{\tens{V}}}
\def\tW{{\tens{W}}}
\def\tX{{\tens{X}}}
\def\tY{{\tens{Y}}}
\def\tZ{{\tens{Z}}}


% Graph
\def\gA{{\mathcal{A}}}
\def\gB{{\mathcal{B}}}
\def\gC{{\mathcal{C}}}
\def\gD{{\mathcal{D}}}
\def\gE{{\mathcal{E}}}
\def\gF{{\mathcal{F}}}
\def\gG{{\mathcal{G}}}
\def\gH{{\mathcal{H}}}
\def\gI{{\mathcal{I}}}
\def\gJ{{\mathcal{J}}}
\def\gK{{\mathcal{K}}}
\def\gL{{\mathcal{L}}}
\def\gM{{\mathcal{M}}}
\def\gN{{\mathcal{N}}}
\def\gO{{\mathcal{O}}}
\def\gP{{\mathcal{P}}}
\def\gQ{{\mathcal{Q}}}
\def\gR{{\mathcal{R}}}
\def\gS{{\mathcal{S}}}
\def\gT{{\mathcal{T}}}
\def\gU{{\mathcal{U}}}
\def\gV{{\mathcal{V}}}
\def\gW{{\mathcal{W}}}
\def\gX{{\mathcal{X}}}
\def\gY{{\mathcal{Y}}}
\def\gZ{{\mathcal{Z}}}

% Sets
\def\sA{{\mathbb{A}}}
\def\sB{{\mathbb{B}}}
\def\sC{{\mathbb{C}}}
\def\sD{{\mathbb{D}}}
% Don't use a set called E, because this would be the same as our symbol
% for expectation.
\def\sF{{\mathbb{F}}}
\def\sG{{\mathbb{G}}}
\def\sH{{\mathbb{H}}}
\def\sI{{\mathbb{I}}}
\def\sJ{{\mathbb{J}}}
\def\sK{{\mathbb{K}}}
\def\sL{{\mathbb{L}}}
\def\sM{{\mathbb{M}}}
\def\sN{{\mathbb{N}}}
\def\sO{{\mathbb{O}}}
\def\sP{{\mathbb{P}}}
\def\sQ{{\mathbb{Q}}}
\def\sR{{\mathbb{R}}}
\def\sS{{\mathbb{S}}}
\def\sT{{\mathbb{T}}}
\def\sU{{\mathbb{U}}}
\def\sV{{\mathbb{V}}}
\def\sW{{\mathbb{W}}}
\def\sX{{\mathbb{X}}}
\def\sY{{\mathbb{Y}}}
\def\sZ{{\mathbb{Z}}}

% Entries of a matrix
\def\emLambda{{\Lambda}}
\def\emA{{A}}
\def\emB{{B}}
\def\emC{{C}}
\def\emD{{D}}
\def\emE{{E}}
\def\emF{{F}}
\def\emG{{G}}
\def\emH{{H}}
\def\emI{{I}}
\def\emJ{{J}}
\def\emK{{K}}
\def\emL{{L}}
\def\emM{{M}}
\def\emN{{N}}
\def\emO{{O}}
\def\emP{{P}}
\def\emQ{{Q}}
\def\emR{{R}}
\def\emS{{S}}
\def\emT{{T}}
\def\emU{{U}}
\def\emV{{V}}
\def\emW{{W}}
\def\emX{{X}}
\def\emY{{Y}}
\def\emZ{{Z}}
\def\emSigma{{\Sigma}}

% entries of a tensor
% Same font as tensor, without \bm wrapper
\newcommand{\etens}[1]{\mathsfit{#1}}
\def\etLambda{{\etens{\Lambda}}}
\def\etA{{\etens{A}}}
\def\etB{{\etens{B}}}
\def\etC{{\etens{C}}}
\def\etD{{\etens{D}}}
\def\etE{{\etens{E}}}
\def\etF{{\etens{F}}}
\def\etG{{\etens{G}}}
\def\etH{{\etens{H}}}
\def\etI{{\etens{I}}}
\def\etJ{{\etens{J}}}
\def\etK{{\etens{K}}}
\def\etL{{\etens{L}}}
\def\etM{{\etens{M}}}
\def\etN{{\etens{N}}}
\def\etO{{\etens{O}}}
\def\etP{{\etens{P}}}
\def\etQ{{\etens{Q}}}
\def\etR{{\etens{R}}}
\def\etS{{\etens{S}}}
\def\etT{{\etens{T}}}
\def\etU{{\etens{U}}}
\def\etV{{\etens{V}}}
\def\etW{{\etens{W}}}
\def\etX{{\etens{X}}}
\def\etY{{\etens{Y}}}
\def\etZ{{\etens{Z}}}

% The true underlying data generating distribution
\newcommand{\pdata}{p_{\rm{data}}}
\newcommand{\ptarget}{p_{\rm{target}}}
\newcommand{\pprior}{p_{\rm{prior}}}
\newcommand{\pbase}{p_{\rm{base}}}
\newcommand{\pref}{p_{\rm{ref}}}

% The empirical distribution defined by the training set
\newcommand{\ptrain}{\hat{p}_{\rm{data}}}
\newcommand{\Ptrain}{\hat{P}_{\rm{data}}}
% The model distribution
\newcommand{\pmodel}{p_{\rm{model}}}
\newcommand{\Pmodel}{P_{\rm{model}}}
\newcommand{\ptildemodel}{\tilde{p}_{\rm{model}}}
% Stochastic autoencoder distributions
\newcommand{\pencode}{p_{\rm{encoder}}}
\newcommand{\pdecode}{p_{\rm{decoder}}}
\newcommand{\precons}{p_{\rm{reconstruct}}}

\newcommand{\laplace}{\mathrm{Laplace}} % Laplace distribution

\newcommand{\E}{\mathbb{E}}
\newcommand{\Ls}{\mathcal{L}}
\newcommand{\R}{\mathbb{R}}
\newcommand{\emp}{\tilde{p}}
\newcommand{\lr}{\alpha}
\newcommand{\reg}{\lambda}
\newcommand{\rect}{\mathrm{rectifier}}
\newcommand{\softmax}{\mathrm{softmax}}
\newcommand{\sigmoid}{\sigma}
\newcommand{\softplus}{\zeta}
\newcommand{\KL}{D_{\mathrm{KL}}}
\newcommand{\Var}{\mathrm{Var}}
\newcommand{\standarderror}{\mathrm{SE}}
\newcommand{\Cov}{\mathrm{Cov}}
% Wolfram Mathworld says $L^2$ is for function spaces and $\ell^2$ is for vectors
% But then they seem to use $L^2$ for vectors throughout the site, and so does
% wikipedia.
\newcommand{\normlzero}{L^0}
\newcommand{\normlone}{L^1}
\newcommand{\normltwo}{L^2}
\newcommand{\normlp}{L^p}
\newcommand{\normmax}{L^\infty}

\newcommand{\parents}{Pa} % See usage in notation.tex. Chosen to match Daphne's book.

\DeclareMathOperator*{\argmax}{arg\,max}
\DeclareMathOperator*{\argmin}{arg\,min}

\DeclareMathOperator{\sign}{sign}
\DeclareMathOperator{\Tr}{Tr}
\let\ab\allowbreak


\newcommand{\mmm}{\texttt{[MASK]}}


\usepackage{graphicx}
\usepackage{amsmath}
\usepackage{amssymb}
\usepackage{booktabs}
\usepackage{times}
\usepackage{microtype}
\usepackage{epsfig}
%\usepackage[table,xcdraw]{xcolor}
\usepackage{caption}
\usepackage{xcolor}
\usepackage{float}
\usepackage{placeins}
\usepackage{color, colortbl}
\usepackage{stfloats}
\usepackage{enumitem}
\usepackage{tabularx}
\usepackage{xstring}
\usepackage{multirow}
\usepackage{xspace}
\usepackage{url}
\usepackage{subcaption}
\usepackage{arydshln}
\usepackage{xcolor}
\usepackage{bbm}
\usepackage[hang,flushmargin]{footmisc}
\usepackage{pifont}%
\newcommand{\cmark}{\ding{51}}%
\newcommand{\xmark}{\ding{55}}%
\usepackage{hyperref}
\usepackage{wrapfig}
\usepackage{lipsum}
\usepackage{comment}
\usepackage{algorithm}
\usepackage{algorithmic}
\usepackage{listings}
\usepackage{makecell} 
\usepackage{xcolor}
\definecolor{lightblue}{RGB}{185, 200, 240}
\usepackage{pifont}
\usepackage{tikz}
%\usepackage[commentColor=black,beginLComment=/*~, endLComment=~*/]{algpseudocodex}
%\newcommand{\LFM}[0]{\gL_{\rm FM}}
%\newcommand{\LCFM}[0]{\gL_{\rm CFM}}
%\newcommand{\LECFM}[0]{\gL_{\rm ECFM}}
%\newcommand{\theHalgorithm}{\arabic{algorithm}}
\usepackage{makecell}



\newcommand{\et}[2]{${#1}^{\pm{#2}}$}
\newcommand{\etb}[2]{$\mathbf{{#1}}^{\pm{#2}}$}
\newcommand{\etr}[2]{$\textcolor{red}{{#1}}^{\pm{#2}}$}
\newcommand{\etbb}[2]{$\textcolor{blue}{{#1}}^{\pm{#2}}$}
\newcommand{\ets}[2]{$\underline{{#1}}^{\pm{#2}}$}


%\usepackage[pagebackref,breaklinks,colorlinks]{hyperref}



\usepackage[capitalize]{cleveref}
\crefname{section}{Sec.}{Secs.}
\Crefname{section}{Section}{Sections}
\Crefname{table}{Table}{Tables}
\crefname{table}{Tab.}{Tabs.}

\usepackage{wrapfig}




\newcommand{\tao}[1]{\textcolor{red}{Tao: #1}}
\newcommand{\michael}[1]{\textcolor{blue}{M: #1}}
\newcommand{\timy}[1]{\textcolor{green}{Timy: #1}}

\newcommand{\tabref}[1]{Table~\ref{#1}}

\newcommand{\bff}[1]{\textcolor{yellow}{BF: #1}}
\newcommand{\cs}[1]{\textcolor{orange}{CS: #1}}
\newcommand{\psmm}[1]{\textcolor{blue}{PM: #1}}
\newcommand{\yc}[1]{\textcolor{olive}{YC: #1}}


\newcommand{\noisemarg}{p_{t|1}}
\newcommand{\denoise}{p_{1|t}}
\newcommand{\x}{x}
\newcommand{\y}{y}
%\newcommand{\pdata}{p_{\mathrm{data}}}
\newcommand{\statespace}{S}
\newcommand{\SO}{\mathrm{SO}(3)}
\newcommand{\SE}{\mathrm{SE}(3)}
\newcommand{\Dt}{\Delta t}
\newcommand{\Dtilde}{\Delta \tilde{t}}
\newcommand{\kdelta}[2]{\delta \left\{ #1, #2 \right\} }


\newcommand{\ourmethod}{MDFM}
\newcommand{\ourmethodlong}{Revist Discrete Visual Generation}

%%%%% Tao Hu

%%%%%%%%% TITLE - PLEASE UPDATE
\title{
MaskFlow: Discrete Flows For Flexible and Efficient Long Video Generation
}

%%%%%%%%% AUTHORS - PLEASE UPDATE
\begin{comment}
\author{Vincent Tao Hu\\
Institution1\\
Institution1 address\\
{\tt\small firstauthor@i1.org}
% For a paper whose authors are all at the same institution,
% omit the following lines up until the closing ``}''.
% Additional authors and addresses can be added with ``\and'',
% just like the second author.
% To save space, use either the email address or home page, not both
\and
Björn Ommer\\
Institution2\\
First line of institution2 address\\
{\tt\small secondauthor@i2.org}
}
\end{comment}




\author{Michael Fuest, Vincent Tao Hu\thanks{ Project Leader}, Björn Ommer \\
%\footnote{\tt\small{\textsuperscript{*}Project Leader}}\\
%Institution1\\
%Institution1 address\\
%{\tt\small firstauthor@i1.org}
% For a paper whose authors are all at the same institution,
% omit the following lines up until the closing ``}''.
% Additional authors and addresses can be added with ``\and'',
% just like the second author.
% To save space, use either the email address or home page, not both
%\and
%Björn Ommer\\
%Institution2\\
%First line of institution2 address\\
%{\tt\small secondauthor@i2.org}
%{\tt\small secondauthor@i2.org}
{ CompVis @ LMU Munich, MCML}\\
%\email{lncs@springer.com}
\\
\url{https://compvis.github.io/maskflow/}
}
\vspace{-10pt}

\setlength{\parskip}{0pt}

\begin{document}
\maketitle

\begin{abstract}
    \begin{abstract}  
Test time scaling is currently one of the most active research areas that shows promise after training time scaling has reached its limits.
Deep-thinking (DT) models are a class of recurrent models that can perform easy-to-hard generalization by assigning more compute to harder test samples.
However, due to their inability to determine the complexity of a test sample, DT models have to use a large amount of computation for both easy and hard test samples.
Excessive test time computation is wasteful and can cause the ``overthinking'' problem where more test time computation leads to worse results.
In this paper, we introduce a test time training method for determining the optimal amount of computation needed for each sample during test time.
We also propose Conv-LiGRU, a novel recurrent architecture for efficient and robust visual reasoning. 
Extensive experiments demonstrate that Conv-LiGRU is more stable than DT, effectively mitigates the ``overthinking'' phenomenon, and achieves superior accuracy.
\end{abstract}  
\end{abstract}

\begin{figure}
    \centering
    \begin{tikzpicture}[font=\footnotesize]
        \node (img) {\includegraphics[width=0.7\columnwidth]{figpaper/nfe_vs_fvd_vs_ep_ffs_teaser.pdf}};
            \node[anchor=north west, xshift=25pt, yshift=-5pt] at (img.north west) {
                \begin{tabular}{ll}
                \scriptsize
                    \textcolor[HTML]{A0A0A0}{\rule{6pt}{6pt}} &Rolling Diffusion \cite{ruhe2024rollingdiffusionmodels} \\
                    \textcolor[HTML]{e9cbc4}{\rule{6pt}{6pt}} &Diffusion Forcing \cite{chen2024diffusionforcing} \\
                    \textcolor[HTML]{F4A700}{\rule{6pt}{6pt}} &MaskFlow (\textit{Ours})
                \end{tabular}
            };
    \end{tikzpicture}
    \vspace{-7pt}
    \caption{\textbf{Our method (MaskFlow) improves video quality compared to baselines while simultaneously requiring fewer function evaluations (NFE)} when generating videos $2\times$, $5\times$, and $10\times$ longer than the training window.
}
    \label{fig:teaser}
    \vspace{-10pt}
\end{figure}

\section{Introduction}

Due to the high computational demands of both training and sampling processes, long video generation remains a challenging task in computer vision. Many recent state-of-the-art video generation approaches train on fixed sequence lengths \cite{blattmann2023stable,blattmann2023align_videoldm,ho2022video} and thus struggle to scale to longer sampling horizons. Many use cases not only require long video generation, but also require the ability to generate videos with varying length. A common way to address this is by adopting an autoregressive diffusion approach similar to LLMs \cite{gao2024vid}, where videos are generated frame by frame. This has other downsides, since it requires traversing the entire denoising chain for every frame individually, which is computationally expensive. Since autoregressive models condition the generative process recursively on previously generated frames, error accumulation, specifically when rolling out to videos longer than the training videos, is another challenge.
\par
Several recent works \cite{ruhe2024rollingdiffusionmodels, chen2024diffusionforcing} have attempted to unify the flexibility of autoregressive generation approaches with the advantages of full sequence generation. These approaches are built on the intuition that the data corruption process in diffusion models can serve as an intermediary for injecting temporal inductive bias. Progressively increasing noise schedules \cite{xie2024progressive,ruhe2024rollingdiffusionmodels} are an example of a sampling schedule enabled by this paradigm. These works impose monotonically increasing noise schedules w.r.t. frame position in the window during training, limiting their flexibility in interpolating between fully autoregressive, frame-by-frame generation and full-sequence generation. This is alleviated in \cite{chen2024diffusionforcing}, where independent, uniformly sampled noise levels are applied to frames during training, and the diffusion model is trained to denoise arbitrary sequences of noisy frames. All of these works use continuous representations.
\par
We transfer this idea to a discrete token space for two main reasons: First, it allows us to use a masking-based data corruption process, which enables confidence-based heuristic sampling that drastically speeds up the generative process. This becomes especially relevant when considering frame-by-frame autoregressive generation. Second, it allows us to use discrete flow matching dynamics, which provide a more flexible design space and the ability to further increase our sampling speed. Specifically, we adopt a \emph{frame-level masking} scheme in training (versus a \emph{constant-level masking} baseline, see Figure~\ref{fig:training}), which allows us to condition on an arbitrary number of previously generated frames while still being consistent with the training task. This makes our method inherently versatile, allowing us to generate videos using both full-sequence and autoregressive frame-by-frame generation, and use different sampling modes. We show that confidence-based masked generative model (MGM) style sampling is uniquely suited to this setting, generating high-quality results with a low number of function evaluations (NFE), and does not degrade quality compared to diffusion-like flow matching (FM)-style sampling that uses larger NFE. 
Combining frame-level masking during training with MGM-style sampling enables highly efficient long-horizon rollouts of our video generation models beyond $10 \times$ training frame lengths without degradation. We also demonstrate that this sampling method can be applied in a timestep-\emph{independent} setting that omits explicit timestep conditioning, even when models were trained in a timestep-dependent manner, which further underlines the flexibility of our approach. In summary, our contributions are the following:

\begin{itemize}
    \item To the best of our knowledge, we are the first to unify the paradigms of discrete representations in video flow matching with rolling out generative models to generate arbitrary-length videos. 
    \item We introduce MaskFlow, a frame-level masking approach that supports highly flexible sampling methods in a single unified model architecture.
    \item We demonstrate that MaskFlow with MGM-style sampling generates long videos faster while simultaneously preserving high visual quality (as shown in Figure~\ref{fig:teaser}).
    \item Additionally, we demonstrate an additional increase in quality when using full autoregressive generation or partial context guidance combined with MaskFlow for very long sampling horizons.
    \item We show that we can apply MaskFlow to both timestep-dependent and timestep-independent model backbones without re-training.
\end{itemize}

\begin{figure}
    \centering
    \includegraphics[width=0.75\linewidth]{figpaper/training.pdf}
    \caption{\textbf{MaskFlow Training:} For each video, Baseline training applies a single masking ratios to all frames, whereas our method samples masking ratios independently for each frame.}
    \vspace{-10pt}
    \label{fig:training}
\end{figure}


















\section{Background and Related Work}
% This paper is related to previous work on dense and MoE model merging.

\subsection{Dense Model Merging}

% Model merging methods combine multiple domain experts into a single model to obtain diverse capabilities \cite{wortsman2022model, ilharco2022editing, goddard2024arcee, jin2022dataless, matena2022merging, yadav2024ties, yu2024language, roberts2024pretrained}. Most merging methods focus on homogeneous dense expert models into another dense model of the same architecture: average merging \cite{wortsman2022model} applies averaging over the model parameters; task vector merging \cite{ilharco2022editing} first extracts the task vector (the difference of parameters between the base and the experts) and adds the unweighted sum of the task vectors back to the dense model with a scaling term. Other works explore how to determine the weights of the task vector instead of an unweighted sum \cite{jin2022dataless, matena2022merging}. Dare and Ties \cite{yadav2024ties, yu2024language}, two SoTA merging methods, choose to trim the task vector to resolve parameter interference: in Dare, the task vector is randomly trimmed and rescaled by a fixed amount;
% in the Ties method, on the other hand, the authors
% set the vector parameter to 0 by magnitude and adjust the sign of each parameter to reduce the sign conflict.

Dense merging methods combine multiple dense models into one to achieve diverse capabilities \cite{wortsman2022model, ilharco2022editing, goddard2024arcee, jin2022dataless, matena2022merging, roberts2024pretrained}. Most approaches focus on merging homogeneous dense models into another dense model. For example, average merging~\cite{wortsman2022model} averages model parameters, while task vector merging~\cite{ilharco2022editing} adds the unweighted sum of task vectors (the difference between base and expert parameters) back to the dense model with scaling. Other work determines task vector weights instead of using an unweighted sum \cite{jin2022dataless, matena2022merging}. SoTA methods like Dare and Ties \cite{yadav2024ties, yu2024language} trim the task vector to resolve parameter interference: Dare trims the task vector randomly and rescales, while Ties sets vector parameters to zero by magnitude and adjusts signs to reduce conflicts.

% In addition to homogeneous model merging, \citet{roberts2024pretrained} proposes an approach to merge heterogeneous models to a dense model by incorporating the projectors. \citet{wan2024knowledge} applies the knowledge distillation to fuse heterogeneous models.
% Compared with previous merging works for dense models, our paper focuses on merging experts to an MoE model. For homogeneous experts, we explore a more efficient merging method and merging without fine-tuning. We are the first work to propose a pipeline for merging heterogeneous models into an MoE.

In addition to homogeneous model merging, \citet{roberts2024pretrained} propose merging heterogeneous models into a dense model using projectors, while \citet{wan2024knowledge} apply knowledge distillation to fuse heterogeneous models. 
In this work, we introduce a more efficient method for merging experts with limited or no further fine-tuning and, unlike previous work focusing on dense models, we explore merging homogeneous and heterogeneous experts into an MoE model. 
%Additionally, we propose 
%For homogeneous experts, we propose a more efficient merging method without fine-tuning. We are also the first to introduce a pipeline for merging heterogeneous models into an MoE.

\subsection{MoE Training and Merging}

% MoE architectures allows for faster training given a fixed computational budget, and faster inference given a fixed number of parameters.  They do so by introducing Sparse MoE layers, where a router mechanism dispatches tokens to the top-K expert FFNs in parallel (where k is usually 1 or 2)  \cite{fedus2022switch, shazeer2017outrageously, zhang2022mixtureattentionheadsselecting}. In terms of MoE training, most work chooses to train the whole MoE on all domain data to obtain capabilities across multiple tasks \cite{komatsuzaki2022sparse, jiang2024mixtral, dou2024loramoe, dai2024deepseekmoe}, but full communication during training on different GPUs is costly \cite{sukhbaatar2024branchtrainmixmixingexpertllms}. To address this challenge, inspired by the Branch-Train-Merge (BTM) method \cite{gururangan2023scaling, li2022branch}, where they average the final model output distributions from different experts, the Branch-Train-Mix (BTX) method \cite{sukhbaatar2024branchtrainmixmixingexpertllms} first branches the base model to multiple copies, trains the base model in different domains, and merges the experts into a unified MoE. Another recent work, Self-MoE \cite{kang2024self}, uses the LoRa style \cite{hu2021lora} to fine-tune each expert on self-generated data and then combine all trained adapters into a base model for a LoRa MoE. 

MoE architectures enable quicker inference with a certain parameter count by introducing Sparse MoE layers, where a router mechanism assigns tokens to the top-$K$ expert FFNs (usually 1 or 2) in parallel \cite{fedus2022switch, shazeer2017outrageously, zhang2022mixtureattentionheadsselecting}. Most MoE training approaches, known as upcycling, train the entire model from scratch to handle multiple tasks \cite{komatsuzaki2022sparse, jiang2024mixtral, dou2024loramoe, dai2024deepseekmoe}. These methods first initialize the MoE model from a pretrained base model and then train it on the entire dataset. However, due to the costly communication between GPUs, the upcycling method introduces significant computational overhead \cite{sukhbaatar2024branchtrainmixmixingexpertllms, li-etal-2024-pedants}. To address this, methods like Branch-Train-Merge (BTM) \cite{gururangan2023scaling, li2022branch} average model outputs from different experts, while Branch-Train-Mix (BTX) \cite{sukhbaatar2024branchtrainmixmixingexpertllms} branches the base model, trains each on different domains, and merges them into a unified MoE. 
BTX is shown to be more effective than BTM as well as dense CPT and MoE upcycling baselines.
%VS: do we need to explain upcycling?
Another recent approach, Self-MoE \cite{kang2024self}, uses low-rank adaptation (LoRA) \cite{hu2021lora} to fine-tune experts on generated synthetic data \cite{liu2024csrec} and combines trained adapters into an MoE.
To our knowledge, we are the first to introduce a framework for merging heterogeneous models into an MoE.



\section{Method}

\begin{figure*}[ht!]
    \centering
    \includegraphics[width=0.9\linewidth]{figpaper/maskflow_sampling.pdf}
    \vspace{-7pt}
    \caption{\textbf{MaskFlow Sampling:} Given $m=2$ \textcolor{red}{context frames} used to initialize generation, we unmask the current window and use \textcolor{color_generated_frame}{newly generated frames} as new context frames in the next chunk of size $k=5$, using stride $s=3$. (\textit{Tokenization omitted here to simplify understanding}) .
    }
    \label{fig:sampling}
    \vspace{-10pt}
\end{figure*}

\subsection{Task formulation: Long video generation}
\label{subsec:long_vid}

 There are, generally, three distinct approaches to long video generation. The first is the naive approach of training on long video sequences. This is challenging due to the quadratic complexity in attention mechanisms with respect to token numbers. Although works like\cite{tan2024video,harvey2022flexible} address this by distributing the generative process or by generating every \(n\)-th frame and subsequently infilling the remaining frames, the approach remains fundamentally resource-intensive. The second approach is a \emph{rolling} (or ``sliding-window'') approach, which applies monotonically increasing noise dependent on a frame's position in the sliding window. This process can be rolled out indefinitely, removing frames from the window when they are fully denoised and appending random noise frames at the end of the window. Works such as \cite{ruhe2024rollingdiffusionmodels, wu2023ar, xie2024progressive} belong to this paradigm. The third approach is \emph{chunkwise-autoregression}, also referred to as blockwise-autoregression \cite{ruhe2024rollingdiffusionmodels}. Here, the video of length $L$ is divided into overlapping \emph{chunks} of length $k \ll L$, where each chunk overlaps by $m$ frames, which we refer to as context frames. Concretely, we define a video and its frames as

\begin{equation}
   \mathbf{v} = (v^1, v^2, \dots, v^L)
\end{equation}

which we divide into overlapping chunks of length $k$. Let $\ell\;=\;\left\lceil \frac{L - k}{s} \right\rceil + 1$ denote the number of chunks needed to cover the video of length $L$, and we further define each chunk $\mathbf{v}^{(i)}$ as

\begin{equation}
   \mathbf{v}^{(i)} = \bigl(v^{(i-1)\,s + 1}, \dots, v^{(i-1)\,s + k}\bigr),
\end{equation}

where $s \le k$ is the sampling window stride, i.e., how far the context start shifts at each step. Often, one sets $s = k - m$, but this is not strictly required. The video distribution then factorizes as

\begin{equation}
\label{eq:markov}
   p(\mathbf{v};{\theta})
   \;=\;
   p(\mathbf{v}^{(1)};\theta)
   \prod_{i=2}^{\ell}
      p
         \bigl(
            \mathbf{v}^{(i)} 
            \;\bigm|\;
            \mathbf{v}^{(i-1)};\theta
         \bigr).
\end{equation}

Because each $\mathbf{v}^{(i)}$ overlaps the previous chunk by $m$ frames, the context frames feed into the next chunk's generation, ensuring smooth transitions and continuity between chunks. To enable such Markovian temporal dependencies during sampling, it is crucial to train a flexible backbone model \(p(\mathbf{v};\theta)\) that can generalize across different sampling schemes, such as the one defined in Equation~\eqref{eq:markov}.


\subsection{Preliminary: Flow Matching for Videos}
\label{subsec:flow}

Our masking flow matching approach, named \emph{MaskFlow}, draws inspiration from previous works that apply individual noise levels to individual frames in a sequence \cite{chen2024diffusionforcing, ruhe2024rollingdiffusionmodels}. These works operate in a continuous space, and use diffusion processes to corrupt data. MaskFlow operates in a discrete token space and uses \emph{masking} to corrupt data. We seek to learn a continuous transition process in ``time'' \(t\) that moves from a purely masked sequence at \(t=0\) to the unmasked token sequence at \(t=1\). In our method, the timestep $t$ corresponds to the masking ratio, and represents the frame-level probability of a token being masked. Consider a video consisting of \(L\) frames, where each frame is mapped to a discrete latent space using a vector-quantized (VQ) tokenizer \cite{esser2021taming_vqgan}. This tokenizer encodes each frame in the video $\mathbf{v}$ to a set of discrete latent indices $\mathbf{x}_{\text{latent}} \in [K]^N$, which consists of $N$ tokens drawn from the tokenizer vocabulary of size $K$. Let \(\mathcal{F}\) denote the VQ encoder-decoder, i.e., the function that maps a video in pixel space to its tokenized representation. Then, we have

\begin{equation}
    \mathbf{x} = \mathcal{F}(\mathbf{v}) \in [K]^{L \times N},
\end{equation}

where \([K] = \{1, 2, \ldots, K\}\) is the set of all possible token indices which includes a special ``mask token" $M \in [K]$. The choice of tokenization is essential here, since it compresses spatial dimensions of $\mathbf{x}$ compared to $\mathbf{v}$ and allows us to employ discrete flow matching, which we outline in further detail in the following section.

\begin{algorithm}[!ht]
\caption{\textbf{Training with Frame-level Masking}}
\label{alg:training}
\begin{algorithmic}[1]
\REQUIRE 
  Dataset of tokenized video clips $\mathcal{D}$, 
  network $p(\mathbf{x}_1 \mid \mathbf{x}_t, \mathbf{t};\theta)$, 
  chunk size $k$

\WHILE{not converged}
    \STATE \textbf{Sample} a chunk of $k$ frames  from $\mathcal{D}$, denoted 
      $\mathbf{x}_1 = (x_1^1, x_1^2, \dots, x_1^k)$
    %\STATE \textbf{Initialize} fully-masked chunk: 
      %$\mathbf{x}_0 = (\, [M], [M], \dots, [M] \,)$
    \FOR{$f = 1, \dots, k$}
        \STATE $t_f \sim \mathcal{U}(0,1)$
        \STATE 
           $x_{t^f} \;\sim\; p_{t^f \mid 0,1}\bigl(\,\cdot \mid x_0^f,\,x_1^f\bigr)$,
        where $p_{t^f \mid 0,1}$ follows 
        $
           (1 - t^f)\,\delta_{x_0^f} 
           \;+\; 
           t^f\,\delta_{x_1^f}.
        $
    \ENDFOR
    \STATE $\mathbf{x}_t = (x_{t^1}^1,\, x_{t^2}^2,\, \dots,\, x_{t^k}^k)$
    \STATE 
    $
       \hat{\mathbf{x}}_1 \;=\; p\bigl(\mathbf{x}_1 \mid \mathbf{x}_t,\, \mathbf{t};\theta\bigr),
    $
    where $\mathbf{t} = (t^1,\ldots,t^k)$
    \STATE \textbf{Backpropagate} $\mathcal{L}_\theta(\mathbf{x}_1, \hat{\mathbf{x}}_1)$ and \textbf{update} $\theta$.
\ENDWHILE
\end{algorithmic}
\end{algorithm}
\vspace{-10pt}

\paragraph{Discrete Flow Matching.}  
Discrete flow matching \cite{gat2024discrete} defines a vector field \(u_t\) in a discrete space that can be traversed to yield a smooth probability transition between our source distribution of fully masked frame sequences $p(\mathbf{x}_0)$ and the distribution of unmasked sequences $p(\mathbf{x}_1)$. This vector field defines an optimal transport path between the two distributions. Concretely, we construct the conditional probability path:
\begin{equation}
    p_{t \,\vert\, 0,1}\bigl(\mathbf{x} \,\vert\, \mathbf{x}_0, \mathbf{x}_1\bigr)
    \;=\; (1-t)\,\delta_{\mathbf{x}_0}(\mathbf{x})
    \;+\; t\,\delta_{\mathbf{x}_1}(\mathbf{x}),
\end{equation}

where \(\delta_{\mathbf{x}_0}(\mathbf{x})\) and \(\delta_{\mathbf{x}_1}(\mathbf{x})\) are Dirac delta functions (analogous to one hot encodings) in the discrete space that allocate all probability mass to the fully masked and fully unmasked sequences at $t=0$ and $t=1$, respectively. For any intermediate value \(t \in (0,1)\), the interpolation governed by the weights \((1-t)\) and \(t\) yields a new video sequence \(\mathbf{x}_t\) that represents a partially corrupted sequence. This is achieved by sampling each token from a mixture distribution where $1-t$ represents the probability of a token being masked.
\vspace{-10pt}
\paragraph{Kolmogorov Equation in Discrete State Spaces.}
In continuous-state models, one leverages the Continuity Equation \cite{song2021scorebased_sde} to ensure that a vector field \(u(\mathbf{x}_t, t)\) induces the desired probability transition between \(p(\mathbf{x}_0)\) and \(p(\mathbf{x}_1)\). The discrete counterpart is given by the Kolmogorov Equation \cite{campbell2024generative}, which similarly characterizes how a probability distribution evolves in time over discrete states. To achieve a transition between the fully masked and fully unmasked video distributions, we define the vector field:

\begin{equation}
    u_t(\mathbf{x}_t) 
    \;=\; 
    \frac{t}{\,1 - t} 
    \Bigl[
    p_{1 \,\vert\, t}(\mathbf{x}_1 \mid \mathbf{x}_t, t; \theta) 
    \;-\; 
    \delta_{\mathbf{x}_t}(x)
    \Bigr],
\end{equation}

where \(p_{1|t}(\mathbf{x}_1 \mid \mathbf{x}_t, t; \theta)\) is the model-predicted distribution of clean tokens \(\mathbf{x}_1\) given a partially corrupted sequence \(\mathbf{x}_t\) at time \(t\). Here, \(\delta_{\mathbf{x}_t}(x)\) represents the discrete Dirac delta centered at \(\mathbf{x}_t\). By following \(u_t\) through time, we recover a path that transforms \(p(\mathbf{x}_0)\) into \(p(\mathbf{x}_1)\).

\subsection{Training with Frame-Level Masking}
\label{subsec:training}

The flow matching formulation introduced in Sections~\ref{subsec:long_vid} and \ref{subsec:flow} employs a single scalar timestep $t$ to interpolate between the fully masked and fully unmasked video distributions. Our training procedure uses a reparametrization of this timestep. In our method, videos are generated in chunks, and only a subset of the frames (the non-context frames) are sampled from a fully masked initial state. To better simulate this process during training, we reparametrize the global timestep $t$ into a per-frame timestep vector $\mathbf{t}=(t^1,\dots, t^k)$ where each timestep $t^f$ specifies the masking ratio applied to frame $f$. In our setup, the context frames are assigned $t^f = 1$ (i.e. fully unmasked) while the new frames receive a masking level sampled from $\mathcal{U}(0,1)$.
By training the model to unmask frames with varying masking ratios per frame, we ensure that the network can effectively handle unmasked context frames while still learning a continuous transition from $p(\mathbf{x}_0)$ to $p(\mathbf{x}_1)$. To emphasize the reconstruction of masked tokens, we follow \cite{hu2024maskneed} in applying a masking operation on the cross-entropy loss. This results in the following objective:
\begin{multline}
\label{eq:ce_loss_masked}
    \mathcal{L}_{\theta} \;=\; 
    \E_{p(\mathbf{x}_1)\,p(\mathbf{x}_0)\,\mathcal{U}(\mathbf{t};0,1)\,p_{t|0,1}(\mathbf{x}_t \,\vert\, \mathbf{x}_0,\mathbf{x}_1)} \\
    \Bigl[\;\underbrace{\delta_{[M]}(\mathbf{x}_t)\,( \mathbf{x}_1 )^\top}_{\text{Loss Masking}}
    \;\log \denoise(\mathbf{x}_1 \,\vert\, \mathbf{x}_t,\mathbf{t};\theta) 
    \Bigr],
\end{multline}

where $\delta_{[M]}(\mathbf{x}_t)$ indicates that only masked tokens are used in the cross-entropy computation. The choice of frame-level masking is essential because it aligns the task of generating chunks of size $k$ conditioned on $m$ clean context frames with our training task. In both scenarios, our models are tasked with unmasking frame sequences with varying masking levels across frames. We show that compared to a constant masking level baseline, this training choice enables chunkwise autoregressive rollout to long sequence lengths. Our training algorithm is shown in detail in Algorithm~\ref{alg:training}.

\subsection{Chunkwise Autoregression for Long Videos}
\label{sec:flex_long}

 To generate a coherent video of length \(L \gg k\), we employ the chunkwise autoregressive approach as described previously. Let \(m\) be the number of context frames provided to the model (drawn initially from ground-truth, later from previous generated frames). In each iteration, we pass \(k\) frames to the model, where the first \(m\) of these frames are context and the remaining \((k - m)\) frames are fully masked. The model unmasks these frames. Afterwards, we shift the context window forward by \(s\) and repeat this process, until we have generated \(L\) total frames. Figure~\ref{fig:sampling} illustrates this pipeline. Note that we dynamically increase the number of context frames $m$ in the final chunk in case there are less than $s$ frames left to generate. In those cases we set $m = k - R$ where $R$ is the remaining number of frames, giving the final chunk a larger context. We do this to avoid generating video lengths beyond $L$ which would result in either discarding generated frames or generating videos longer than $L$. This is shown in detail in Algorithm~\ref{alg:chunkwise}.

\paragraph{Autoregressive \textit{v.s.} Full-Sequence Generation.}
By varying the stride \(s\), we can interpolate between (i) a fully autoregressive mode (\(s = 1\)) with $m = k - s$, where we generate a single new frame per chunk, and (ii) a full-sequence mode (\(s = k - m\)), where we generate $k - m$ new frames simultaneously in each chunk. Smaller \(s\) increases compute cost but may yield higher frame quality, whereas larger \(s\) is more efficient, but may result in a drop in frame quality. Our experimental results shown in Table~\ref{tab:autoregression} support this intuition.

\paragraph{FM-Style \textit{v.s.} MGM-Style Sampling.}
MaskFlow supports two distinct sampling modes. In FM-style sampling, we gradually traverse the probability path from the fully masked sequence $\mathbf{x}_0$ to the final unmasked sequence $\mathbf{x}_1$. A smaller step size yields smoother transitions at the cost of more denoising steps. Alternatively, in MGM-style sampling, we apply confidence-based heuristic sampling similar to \citet{chang2022maskgit}. In each sampling step, the model computes token-wise confidence scores for each predicted token and selects a fraction of the most confident tokens to unmask. This sampling process allows us to generate video chunks efficiently in much fewer sampling steps.
\vspace{-10pt}

\paragraph{Timestep-dependent models and timestep-independent sampling.}

By default, our model backbones are timstep-dependent, meaning each forward pass receives a timestep vector $\mathbf{t}\in[0,1]^k$ that indicates the masking ratio of each frame. Internally, we embed $\mathbf{t}$ through a learnable mapping to produce conditioning vectors that modulate various layers (e.g., via layer norm shifts/scales). Interestingly, we can still sample these models timstep-independently. Concretely, when using MGM-style sampling, we iteratively unmask a chunk of tokens while simply passing $\mathbf{t}=\mathbf{0}$ at each iteration, effectively treating our timestep-dependent model as if it were timestep-independent:

\begin{equation}
    p(\mathbf{x}_1|\mathbf{x}_t;\theta) \approx p(\mathbf{x}_1|\mathbf{x}_{t}, \mathbf{t}=\mathbf{0};\theta).
\end{equation}

This works, since the learned network can infer the corruption state (mask ratio) from the input tokens alone. Thus, in practice, \emph{a single} trained model can serve both as a standard time-dependent (flow-matching) generator \emph{and} as a time-independent (MGM-style) sampler, providing greater flexibility at inference time.

\begin{algorithm}[ht!]
\caption{Chunkwise Autoregression for Long Videos}
\label{alg:chunkwise}
\begin{algorithmic}[1]
\REQUIRE Video length \(L\), context frames \(\mathbf{x}^{1:m} = (x^1,\dots,x^m)\), chunk size \(k\), stride \(s\), fully masked frame \([M]\), network \(p(\mathbf{x}_1 \mid \mathbf{x}_t,\mathbf{t};\theta)\)
\STATE \textbf{Initialize:} \(\hat{\mathbf{x}}_{1} \leftarrow (x^1,\dots,x^m)\); \(c \leftarrow m\) \COMMENT{current frame}
\WHILE{\(c < L\)}
    \STATE \(R \leftarrow L - c\) \COMMENT{remaining frames}
    \STATE \(h \leftarrow \min(R,\, s)\) \COMMENT{frames to generate this chunk}
    \IF{\(R \le s\)}
        \STATE \(m \leftarrow k - R\)
    \ENDIF
    \STATE \(\mathbf{x}_{\mathrm{context}} \leftarrow (x^{\,c-m+1}, \dots, x^c)\)
    \STATE \(\mathbf{x}_{\mathrm{mask}} \leftarrow (\underbrace{[M], \dots, [M]}_{h\text{ times}})\)
    \STATE \(\mathbf{x}_{\mathrm{out}} \sim p\Bigl(\mathbf{x}_1 \mid (\mathbf{x}_{\mathrm{context}},\, \mathbf{x}_{\mathrm{mask}}), \mathbf{t};\theta\Bigr)\)
    \STATE \(\mathbf{x}_{\mathrm{new}} \leftarrow (x_{\mathrm{out}}^{\,m+1}, \dots, x_{\mathrm{out}}^{\,m+h})\)
    \STATE \(\hat{\mathbf{x}}_1 \leftarrow (\hat{\mathbf{x}}_1,\, \mathbf{x}_{\mathrm{new}})\)
    \STATE \(c \leftarrow c + h\)
\ENDWHILE
\RETURN \(\hat{\mathbf{x}}_1\)
\end{algorithmic}
\end{algorithm}


\section{Experiment}

\subsection{Dataset}
We construct a multilingual dataset consisting of copyrighted song lyrics in four languages: English, Chinese, French, and Korean. Lyrics represent a distinct form of copyrighted content, differing notably from other text sources such as book chapters. They exhibit rhyming and repetitive patterns, which can influence model memorization and reproduction \cite{doi:10.1021/acsenergylett.2c02758}. Moreover, song lyrics are widely shared, discussed, and searched for on social media, forums, and dedicated lyrics websites, increasing their likelihood of being incorporated into the training data of language models.
Dataset details are presented in Appendix \ref{appendixdata}.

\subsection{Language Models}
For open-source models, we evaluate Meta’s Llama-3-70B \cite{meta2024llama3}, Mistral AI’s Mistral-7B \cite{jiang2023mistral} and Mixtral-8x7B \cite{jiang2024mixtral}. For API-based models, we test OpenAI’s GPT-3.5-Turbo \cite{openai2024chatgptapis} and GPT-4o \cite{openai2024gpt4o}, Anthropic’s Claude-3.5-Haiku \cite{anthropic2024claude3}, as well as Google’s Gemini-2.0 \cite{google2024gemini2}.
For prompt, we adopt direct probing using the format of "\textit{What are the lyrics of the song [TITLE] by [SINGER]?}" and instruct the LLM to respond in the language of the song. More details can be found in Appendix \ref{appendixa}.

\subsection{Evaluation Metrics}
Following previous works \cite{karamolegkou2023copyright, liu-etal-2024-shield}, we primarily use the Longest Common Substring (LCS) and ROUGE-L scores to measure the volume of verbatim reproduction. To assess the model’s ability to decline requests for copyrighted content, we adopt the Refusal Rate. Additionally, for models with low refusal rate, we leverage GPT-4o to further assess the Hallucination Rate, which quantifies the proportion of fabricated lyric in the lyric generated, providing a more comprehensive evaluation of potential copyright infringement. Details of the metrics can be found in Appendix \ref{appendixmetric}.



%\vspace{-10pt}
\section{Conclusion}


We have presented a discrete flow matching framework for flexible long video generation, leveraging frame-level masking during training to enable flexible, efficient sampling. Our experiments demonstrate that this approach can generate high-quality videos beyond 10$\times$ the training window length, while substantially reducing sampling cost through MGM-style unmasking. Notably, our models can seamlessly switch between timestep-dependent (flow matching) and timestep-independent (MGM) sampling modes without additional training, offering a unified solution that supports both full-sequence rollout and fully autoregressive generation. We believe discrete tokens have great potential for scalable visual generation. 

\section{Acknowledgements}

Special thanks go out to Timy Phan for proof-reading and
providing helpful comments.
%
This project has been supported by the German Federal Ministry for Economic Affairs and Climate Action within the project “NXT GEN AI METHODS – Generative Methoden für Perzeption, Prädiktion und Planung”, the bidt project KLIMA-MEMES, Bayer AG, and the German Research Foundation (DFG) project 421703927. The authors gratefully acknowledge the Gauss Center for Supercomputing for providing compute through the NIC on JUWELS at JSC and the HPC resources supplied by the Erlangen National High Performance Computing Center (NHR@FAU funded by DFG).

\newpage


{
    \small
    \bibliographystyle{ieeenat_fullname}
    \bibliography{main}
}

% WARNING: do not forget to delete the supplementary pages from your submission 
% \clearpage
\pagenumbering{gobble}
\maketitlesupplementary

\section{Additional Results on Embodied Tasks}

To evaluate the broader applicability of our EgoAgent's learned representation beyond video-conditioned 3D human motion prediction, we test its ability to improve visual policy learning for embodiments other than the human skeleton.
Following the methodology in~\cite{majumdar2023we}, we conduct experiments on the TriFinger benchmark~\cite{wuthrich2020trifinger}, which involves a three-finger robot performing two tasks: reach cube and move cube. 
We freeze the pretrained representations and use a 3-layer MLP as the policy network, training each task with 100 demonstrations.

\begin{table}[h]
\centering
\caption{Success rate (\%) on the TriFinger benchmark, where each model's pretrained representation is fixed, and additional linear layers are trained as the policy network.}
\label{tab:trifinger}
\resizebox{\linewidth}{!}{%
\begin{tabular}{llcc}
\toprule
Methods       & Training Dataset & Reach Cube & Move Cube \\
\midrule
DINO~\cite{caron2021emerging}         & WT Venice        & 78.03     & 47.42     \\
DoRA~\cite{venkataramanan2023imagenet}          & WT Venice        & 81.62     & 53.76     \\
DoRA~\cite{venkataramanan2023imagenet}          & WT All           & 82.40     & 48.13     \\
\midrule
EgoAgent-300M & WT+Ego-Exo4D      & 82.61    & 54.21      \\
EgoAgent-1B   & WT+Ego-Exo4D      & \textbf{85.72}      & \textbf{57.66}   \\
\bottomrule
\end{tabular}%
}
\end{table}

As shown in Table~\ref{tab:trifinger}, EgoAgent achieves the highest success rates on both tasks, outperforming the best models from DoRA~\cite{venkataramanan2023imagenet} with increases of +3.32\% and +3.9\% respectively.
This result shows that by incorporating human action prediction into the learning process, EgoAgent demonstrates the ability to learn more effective representations that benefit both image classification and embodied manipulation tasks.
This highlights the potential of leveraging human-centric motion data to bridge the gap between visual understanding and actionable policy learning.



\section{Additional Results on Egocentric Future State Prediction}

In this section, we provide additional qualitative results on the egocentric future state prediction task. Additionally, we describe our approach to finetune video diffusion model on the Ego-Exo4D dataset~\cite{grauman2024ego} and generate future video frames conditioned on initial frames as shown in Figure~\ref{fig:opensora_finetune}.

\begin{figure}[b]
    \centering
    \includegraphics[width=\linewidth]{figures/opensora_finetune.pdf}
    \caption{Comparison of OpenSora V1.1 first-frame-conditioned video generation results before and after finetuning on Ego-Exo4D. Fine-tuning enhances temporal consistency, but the predicted pixel-space future states still exhibit errors, such as inaccuracies in the basketball's trajectory.}
    \label{fig:opensora_finetune}
\end{figure}

\subsection{Visualizations and Comparisons}

More visualizations of our method, DoRA, and OpenSora in different scenes (as shown in Figure~\ref{fig:supp pred}). For OpenSora, when predicting the states of $t_k$, we use all the ground truth frames from $t_{0}$ to $t_{k-1}$ as conditions. As OpenSora takes only past observations as input and neglects human motion, it performs well only when the human has relatively small motions (see top cases in Figure~\ref{fig:supp pred}), but can not adjust to large movements of the human body or quick viewpoint changes (see bottom cases in Figure~\ref{fig:supp pred}).

\begin{figure*}
    \centering
    \includegraphics[width=\linewidth]{figures/supp_pred.pdf}
    \caption{Retrieval and generation results for egocentric future state prediction. Correct and wrong retrieval images are marked with green and red boundaries, respectively.}
    \label{fig:supp pred}
\end{figure*}

\begin{figure*}[t]
    \centering
    \includegraphics[width=0.9\linewidth]{figures/motion_prediction.pdf}
    \vspace{-0.5mm}
    \caption{Motion prediction results in scenes with minor changes in observation.}
    \vspace{-1.5mm}
    \label{fig:motion_prediction}
\end{figure*}

\subsection{Finetuning OpenSora on Ego-Exo4D}

OpenSora V1.1~\cite{opensora}, initially trained on internet videos and images, produces severely inconsistent results when directly applied to infer future videos on the Ego-Exo4D dataset, as illustrated in Figure~\ref{fig:opensora_finetune}.
To address the gap between general internet content and egocentric video data, we fine-tune the official checkpoint on the Ego-Exo4D training set for 50 epochs.
OpenSora V1.1 proposed a random mask strategy during training to enable video generation by image and video conditioning. We adopted the default masking rate, which applies: 75\% with no masking, 2.5\% with random masking of 1 frame to 1/4 of the total frames, 2.5\% with masking at either the beginning or the end for 1 frame to 1/4 of the total frames, and 5\% with random masking spanning 1 frame to 1/4 of the total frames at both the beginning and the end.

As shown in Fig.~\ref{fig:opensora_finetune}, despite being trained on a large dataset, OpenSora struggles to generalize to the Ego-Exo4D dataset, producing future video frames with minimal consistency relative to the conditioning frame. While fine-tuning improves temporal consistency, the moving trajectories of objects like the basketball and soccer ball still deviate from realistic physical laws. Compared with our feature space prediction results, this suggests that training world models in a reconstructive latent space is more challenging than training them in a feature space.


\section{Additional Results on 3D Human Motion Prediction}

We present additional qualitative results for the 3D human motion prediction task, highlighting a particularly challenging scenario where egocentric observations exhibit minimal variation. This scenario poses significant difficulties for video-conditioned motion prediction, as the model must effectively capture and interpret subtle changes. As demonstrated in Fig.~\ref{fig:motion_prediction}, EgoAgent successfully generates accurate predictions that closely align with the ground truth motion, showcasing its ability to handle fine-grained temporal dynamics and nuanced contextual cues.

\section{OpenSora for Image Classification}

In this section, we detail the process of extracting features from OpenSora V1.1~\cite{opensora} (without fine-tuning) for an image classification task. Following the approach of~\cite{xiang2023denoising}, we leverage the insight that diffusion models can be interpreted as multi-level denoising autoencoders. These models inherently learn linearly separable representations within their intermediate layers, without relying on auxiliary encoders. The quality of the extracted features depends on both the layer depth and the noise level applied during extraction.


\begin{table}[h]
\centering
\caption{$k$-NN evaluation results of OpenSora V1.1 features from different layer depths and noising scales on ImageNet-100. Top1 and Top5 accuracy (\%) are reported.}
\label{tab:opensora-knn}
\resizebox{0.95\linewidth}{!}{%
\begin{tabular}{lcccccc}
\toprule
\multirow{2}{*}{Timesteps} & \multicolumn{2}{c}{First Layer} & \multicolumn{2}{c}{Middle Layer} & \multicolumn{2}{c}{Last Layer} \\
\cmidrule(r){2-3}   \cmidrule(r){4-5}  \cmidrule(r){6-7}  & Top1           & Top5           & Top1            & Top5           & Top1           & Top5          \\
\midrule
32        &  6.10           & 18.20             & 34.04               & 59.50             & 30.40             & 55.74             \\
64        & 6.12              & 18.48              & 36.04               & 61.84              & 31.80         & 57.06         \\
128       & 5.84             & 18.14             & 38.08               & 64.16              & 33.44       & 58.42 \\
256       & 5.60             & 16.58              & 30.34               & 56.38              &28.14          & 52.32        \\
512       & 3.66              & 11.70            & 6.24              & 17.62              & 7.24              & 19.44  \\ 
\bottomrule
\end{tabular}%
}
\end{table}

As shown in Table~\ref{tab:opensora-knn}, we first evaluate $k$-NN classification performance on the ImageNet-100 dataset using three intermediate layers and five different noise scales. We find that a noise timestep of 128 yields the best results, with the middle and last layers performing significantly better than the first layer.
We then test this optimal configuration on ImageNet-1K and find that the last layer with 128 noising timesteps achieves the best classification accuracy.

\section{Data Preprocess}
For egocentric video sequences, we utilize videos from the Ego-Exo4D~\cite{grauman2024ego} and WT~\cite{venkataramanan2023imagenet} datasets.
The original resolution of Ego-Exo4D videos is 1408×1408, captured at 30 fps. We sample one frame every five frames and use the original resolution to crop local views (224×224) for computing the self-supervised representation loss. For computing the prediction and action loss, the videos are downsampled to 224×224 resolution.
WT primarily consists of 4K videos (3840×2160) recorded at 60 or 30 fps. Similar to Ego-Exo4D, we use the original resolution and downsample the frame rate to 6 fps for representation loss computation.
As Ego-Exo4D employs fisheye cameras, we undistort the images to a pinhole camera model using the official Project Aria Tools to align them with the WT videos.

For motion sequences, the Ego-Exo4D dataset provides synchronized 3D motion annotations and camera extrinsic parameters for various tasks and scenes. While some annotations are manually labeled, others are automatically generated using 3D motion estimation algorithms from multiple exocentric views. To maximize data utility and maintain high-quality annotations, manual labels are prioritized wherever available, and automated annotations are used only when manual labels are absent.
Each pose is converted into the egocentric camera's coordinate system using transformation matrices derived from the camera extrinsics. These matrices also enable the computation of trajectory vectors for each frame in a sequence. Beyond the x, y, z coordinates, a visibility dimension is appended to account for keypoints invisible to all exocentric views. Finally, a sliding window approach segments sequences into fixed-size windows to serve as input for the model. Note that we do not downsample the frame rate of 3D motions.

\section{Training Details}
\subsection{Architecture Configurations}
In Table~\ref{tab:arch}, we provide detailed architecture configurations for EgoAgent following the scaling-up strategy of InternLM~\cite{team2023internlm}. To ensure the generalization, we do not modify the internal modules in InternML, \emph{i.e.}, we adopt the RMSNorm and 1D RoPE. We show that, without specific modules designed for vision tasks, EgoAgent can perform well on vision and action tasks.

\begin{table}[ht]
  \centering
  \caption{Architecture configurations of EgoAgent.}
  \resizebox{0.8\linewidth}{!}{%
    \begin{tabular}{lcc}
    \toprule
          & EgoAgent-300M & EgoAgent-1B \\
          \midrule
    Depth & 22    & 22 \\
    Embedding dim & 1024  & 2048 \\
    Number of heads & 8     & 16 \\
    MLP ratio &    8/3   & 8/3 \\
    $\#$param.  & 284M & 1.13B \\
    \bottomrule
    \end{tabular}%
    }
  \label{tab:arch}%
\end{table}%

Table~\ref{tab:io_structure} presents the detailed configuration of the embedding and prediction modules in EgoAgent, including the image projector ($\text{Proj}_i$), representation head/state prediction head ($\text{MLP}_i$), action projector ($\text{Proj}_a$) and action prediction head ($\text{MLP}_a$).
Note that the representation head and the state prediction head share the same architecture but have distinct weights.

\begin{table}[t]
\centering
\caption{Architecture of the embedding ($\text{Proj}_i$, $\text{Proj}_a$) and prediction ($\text{MLP}_i$, $\text{MLP}_a$) modules in EgoAgent. For details on module connections and functions, please refer to Fig.~2 in the main paper.}
\label{tab:io_structure}
\resizebox{\linewidth}{!}{%
\begin{tabular}{lcl}
\toprule
       & \multicolumn{1}{c}{Norm \& Activation} & \multicolumn{1}{c}{Output Shape}  \\
\midrule
\multicolumn{3}{l}{$\text{Proj}_i$ (\textit{Image projector})} \\
\midrule
Input image  & -          & 3$\times$224$\times$224 \\
Conv 2D (16$\times$16) & -       & Embedding dim$\times$14$\times$14    \\
\midrule
\multicolumn{3}{l}{$\text{MLP}_i$ (\textit{State prediction head} \& \textit{Representation head)}} \\
\midrule
Input embedding  & -          & Embedding dim \\
Linear & GELU       & 2048          \\
Linear & GELU       & 2048          \\
Linear & -          & 256           \\
Linear & -          & 65536     \\
\midrule
\multicolumn{3}{l}{$\text{Proj}_a$ (\textit{Action projector})} \\
\midrule
Input pose sequence  & -          & 4$\times$5$\times$17 \\
Conv 2D (5$\times$17) & LN, GELU   & Embedding dim$\times$1$\times$1    \\
\midrule
\multicolumn{3}{l}{$\text{MLP}_a$ (\textit{Action prediction head})} \\
\midrule
Input embedding  & -          & Embedding dim$\times$1$\times$1 \\
Linear & -          & 4$\times$5$\times$17     \\
\bottomrule
\end{tabular}%
}
\end{table}


\subsection{Training Configurations}
In Table~\ref{tab:training hyper}, we provide the detailed training hyper-parameters for experiments in the main manuscripts.

\begin{table}[ht]
  \centering
  \caption{Hyper-parameters for training EgoAgent.}
  \resizebox{0.86\linewidth}{!}{%
    \begin{tabular}{lc}
    \toprule
    Training Configuration & EgoAgent-300M/1B \\
    \midrule
    Training recipe: &  \\
    optimizer & AdamW~\cite{loshchilov2017decoupled} \\
    optimizer momentum & $\beta_1=0.9, \beta_2=0.999$ \\
    \midrule
    Learning hyper-parameters: &  \\
    base learning rate & 6.0E-04 \\
    learning rate schedule & cosine \\
    base weight decay & 0.04 \\
    end weight decay & 0.4 \\
    batch size & 1920 \\
    training iters & 72,000 \\
    lr warmup iters & 1,800 \\
    warmup schedule & linear \\
    gradient clip & 1.0 \\
    data type & float16 \\
    norm epsilon & 1.0E-06 \\
    \midrule
    EMA hyper-parameters: &  \\
    momentum & 0.996 \\
    \bottomrule
    \end{tabular}%
    }
  \label{tab:training hyper}%
\end{table}%

\clearpage


\section{Appendix}
\subsection{Algorithm for Semantic Tokenization}\label{sec:semantic_token}
As shown in Algorithm~\ref{alg:rq}, we present RQ-VAE for semantic tokenization.
\begin{figure}[!htb]
\vspace{-1em}
\centering
\small
\begin{algorithm}[H]
\caption{RQ-VAE for Semantic Tokenization}\label{alg:rq}
\textbf{Input:} Sentence embedding $\mathcal{X}_{u} = (\boldsymbol{x}_{i_{1}}, \boldsymbol{x}_{i_{2}}, \ldots, \boldsymbol{x}_{i_{T}})$ of user $u$\\
\textbf{Output:} Semantic representation $\hat{\mathcal{Z}}_{u} = (\hat{\boldsymbol{z}}_{i_{1}}, \hat{\boldsymbol{z}}_{i_{2}}, \ldots, \hat{\boldsymbol{z}}_{i_{T}})$ of user $u$\\
\begin{algorithmic}[1]
\FOR{$t = 1 \rightarrow T$ in parallel} 
 \STATE $\boldsymbol{z}_{i_t} = \textbf{Encoder} ({\boldsymbol{x}}_{i_t})$  \# encode the text embedding
\STATE $\boldsymbol{r}_1 = \boldsymbol{z}_{i_t}$, $\hat{\boldsymbol{{z}}}_{i_t} = 0$
    \FOR{$l = 1 \rightarrow L$}
            \STATE $\left\{\boldsymbol{e}^c_{k}\right\}_{k=1}^K, \boldsymbol{e}^c_{k} \in \mathbb{R}^{1 \times D'}$ \# codebook embedding of each layer 
        \STATE $k=\arg \min_k\left\|\boldsymbol{r}_{l}-\boldsymbol{e}^c_{k}\right\|$ \# search the index of closest codebook
        \STATE $\boldsymbol{r}_{l + 1} = \boldsymbol{r}_l-\boldsymbol{e}^c_{k}$ 
 \STATE $\hat{\boldsymbol{{z}}}_{i_t} += \boldsymbol{e}^c_{k}$ \# accumulate the quantized embedding
 \STATE $\mathcal{L}_{\text {rqvae }} += \left\|\operatorname{sg}\left[\boldsymbol{r}_l\right]-\boldsymbol{e}^c_{k}\right\|^2+\beta\left\|\boldsymbol{r}_l-\operatorname{sg}\left[\boldsymbol{e}^c_{k}\right]\right\|^2$ \# $\operatorname{sg}$ means stop gradient
    \ENDFOR
 \STATE $\hat{\boldsymbol{x}}_{i_t} = \textbf{Decoder}(\hat{\boldsymbol{z}}_{i_t})$  \# decode the quantized semantic embedding
  \STATE $\mathcal{L}_{\text {recon}} += \left\|\boldsymbol{x}_{i_t} - \hat{\boldsymbol{x}}_{i_t}\right\|^2$ \# reconstruction loss
    \ENDFOR
    \STATE \textbf{return} $\hat{\mathcal{Z}}_{u}$
\end{algorithmic}
\end{algorithm}
\vspace{-1em}
\end{figure}
\subsection{Implementation Details}\label{appendix:implementation}
Following TIGER~\citep{rajput2024recommender}, to obtain the semantic tokens, we utilize the pre-trained Sentence-T5~\citep{ni2021sentence}. Specifically, we construct item's sentence description using its content features, including title, brand, category and price. This constructed sentence is then fed into Sentence-T5, which outputs a 768-dimensional text embedding for each item as the input in our task. Besides, the RQ-VAE model includes a DNN encoder, a residual quantizer, and a DNN decoder. The DNN encoder takes the input text embedding and transforms the dimension to be aligned with codebook embedding. This encoder is activated by ReLU with layer sizes 512, 256, and 128, which ultimately produces a 64-dimensional latent representation. With the 64-dimensional latent representation from encoder, the residual quantizer then performs three levels of residual quantization. At each level, a codebook with size $K$ is used, where each token within the codebook has a dimension of 64. The output semantic token quantized by residual quantizer is then fed into the DNN decoder, which decodes it back to the original text embedding space. Note different from TIGER, we set the dimension of semantic token as 64 for alignment with ID token in our sequential recommendation setting. 

As for the implementation of sequential recommendation, we directly use the framework of $\text{S}^3\text{-Rec}$~\citep{zhou2020s3}. But as we train the model in an end-to-end manner, we just use the fine-tuning setting and do not use the pre-training setting of their framework. In our setting, we employ the Adam optimizer~\citep{kingma2014adam} with a learning rate of 0.001 and the batch size is set as 256.

\subsection{Baselines}\label{appendix:baseline}
In this section, we provide a brief overview of the baseline models employed for comparison:
\begin{itemize}[leftmargin=*]
\item \textbf{FM}~\citep{fm}: The Factorization Machine (FM) model characterizes pairwise interactions among variables through a factorized representation.
    \item \textbf{GRU4Rec}~\citep{gru4rec}: This model represents the pioneering application of recurrent neural networks (RNNs) for sequential recommendation, specifically utilizing a customized Gated Recurrent Unit (GRU).
    \item \textbf{Caser}~\citep{caser}: Caser introduces a convolution neural network (CNN) architecture designed to capture high-order Markov Chains. It achieves this through the implementation of both horizontal and vertical convolution operations tailored for sequential recommendation.
\item \textbf{HGN}~\citep{hgn}: The Hierarchical Gating Network (HGN) effectively models long-short-term user preference through an innovative gating mechanism.
\item \textbf{SASRec}~\citep{sasrec}: Self-Attentive Sequential Recommendation (SASRec) employs a causal masked self attention to model user’s historical behavior sequence.
\item \textbf{BERT4Rec}~\citep{bert4rec}: This model applies the bi-directional Transformer BERT for enhanced sequential recommender.
\end{itemize}

\subsection{Data Description}\label{appendix:data}
\begin{table}[htb!]
% \vspace{-1.6cm}
\centering
\caption{Data statistics for benchmark datasets after 5-core filtering. Here Sports and Toys are the `Sports and Outdoors' and `Toys and Games', respectively, from Amazon review datasets.}
\label{tab:data}
\begin{tabular}{ccccc}
\toprule
Dataset & \# Users &  \# Items & {Average Len.}\\
\midrule
Beauty & 22,363 & 12,101 & 8.87 \\
Sports & 35,598 & 18,357 & 8.32\\
Toys & 19,412 & 11,924 & 8.63 \\
\bottomrule
\end{tabular}
\end{table}
We utilize three real-world benchmark datasets derived from the Amazon Product Reviews dataset~\citep{he2016ups}, which includes user reviews and item metadata spanning from May 1996 to July 2014. In our task, we focus on three specific categories within this dataset: "Beauty," "Sports and Outdoors," and "Toys and Games." Table~\ref{tab:data} presents a summary of the statistics associated with these datasets, where "Average Len." represents the average length of all users' item sequences. To construct item sequences, we organize users' review histories chronologically by timestamp, ensuring that only users with a minimum of five reviews are retained in our analysis.


\subsection{Codebook Size Study}\label{appendix:codebooksize}

\begin{table*}[!htb]
\centering
\caption{Increasing codebook size does not improve the performance too much on Sports dataset.}
\label{tab:codebook_size}
\begin{tabular}{cccccc}
\hline
Codebook   Size & HR@5            & NDCG@5          & HR@10           & NDCG@10         & MRR             \\ \hline
64              & 0.3792          & 0.2675          & 0.5138          & 0.3109          & 0.2675          \\ \hline
128             & \textbf{0.3849} & \textbf{0.2717} & \textbf{0.5247} & \textbf{0.3168} & \textbf{0.2722} \\ \hline
256             & 0.3786          & 0.2672          & 0.5184          & 0.3123          & 0.2688          \\ \hline
521             & 0.3842          & 0.2719          & 0.5218          & 0.3163          & 0.2720          \\ \hline
1024            & 0.3809          & 0.2691          & 0.5202          & 0.3140          & 0.2696          \\ \hline
\end{tabular}
\end{table*}

% \begin{table}[!htb]
% \centering
% \caption{Increasing codebook size does not improve the performance too much on Sports dataset.}
% \label{tab:codebook_size}
% \begin{tabular}{cccccc}
% \hline
% Codebook   Size & HR@5            & NDCG@5          & HR@10           & NDCG@10         & MRR            \\ \hline
% 256             & 0.3786          & 0.2672          & 0.5184          & 0.3123          & 0.2688         \\ \hline
% 512             & \textbf{0.3842} & \textbf{0.2719} & \textbf{0.5218} & \textbf{0.3163} & \textbf{0.272} \\ \hline
% 1024            & 0.3809          & 0.2691          & 0.5202          & 0.314           & 0.2696         \\ \hline
% \end{tabular}
% \end{table}
As the first and third codebook in Amazon Sports dataset degenerate in Figure~\ref{fig:vis_sport}, we want to study whether the size of codebook $K$ has significant impact on this degeneration problem. Thus we vary the codebook size $K$ from 64 to 1024 as Table~\ref{tab:codebook_size}, and have the following discovery.
\begin{itemize}[leftmargin=*]
\item \textbf{Increasing codebook size does not improve the performance too much.} The performance reaches peak when codebook size is 128, but the performance fluctuates when codebook size grows to 256 and over.
\end{itemize}


\begin{figure*}[htb!]
		\centering
		\begin{tabular}{cccc}
\includegraphics[width=0.22\linewidth]{fig/first_layer64Sports_and_Outdoors.png} &
       \includegraphics[width=0.22\linewidth]{fig/second_layer64Sports_and_Outdoors.png}  & \includegraphics[width=0.22\linewidth]{fig/third_layer64Sports_and_Outdoors.png}  &
       \includegraphics[width=0.22\linewidth]{fig/unique64Sports_and_Outdoors.png}
	     \\ First Codebook & Second Codebook & Third Codebook & Unique Tokens
		\end{tabular}
	\caption{The patterns of codebooks are various across different layers but kind of sparse on Sports dataset with codebook size 64.}	\label{fig:vis_sports_64}
\end{figure*} 



\begin{figure*}[htb!]
		\centering
		\begin{tabular}{cccc}
\includegraphics[width=0.22\linewidth]{fig/first_layer128Sports_and_Outdoors.png} &
       \includegraphics[width=0.22\linewidth]{fig/second_layer128Sports_and_Outdoors.png}  & \includegraphics[width=0.22\linewidth]{fig/third_layer128Sports_and_Outdoors.png}  &
       \includegraphics[width=0.22\linewidth]{fig/unique128Sports_and_Outdoors.png}
	     \\ First Codebook & Second Codebook & Third Codebook & Unique Tokens
		\end{tabular}
	\caption{The patterns of codebooks are various across different layers on Sports dataset with codebook size 128.}	\label{fig:vis_sports_128}
\end{figure*} 

\begin{figure*}[htb!]
		\centering
		\begin{tabular}{cccc}
\includegraphics[width=0.22\linewidth]{fig/first_layerSports_and_Outdoors.png} &
       \includegraphics[width=0.22\linewidth]{fig/second_layerSports_and_Outdoors.png}  & \includegraphics[width=0.22\linewidth]{fig/third_layerSports_and_Outdoors.png}  &
       \includegraphics[width=0.22\linewidth]{fig/uniqueSports_and_Outdoors.png}
	     \\ First Codebook & Second Codebook & Third Codebook & Unique Tokens
      \end{tabular}
	\caption{The first and third codebooks start to degenerate on Sports dataset with codebook size 256.}	\label{fig:vis_sport_256}
\end{figure*} 


\begin{figure*}[htb!]
		\centering
		\begin{tabular}{cccc}
\includegraphics[width=0.22\linewidth]{fig/first_layer512Sports_and_Outdoors.png} &
       \includegraphics[width=0.22\linewidth]{fig/second_layer512Sports_and_Outdoors.png}  & \includegraphics[width=0.22\linewidth]{fig/third_layer512Sports_and_Outdoors.png}  &
       \includegraphics[width=0.22\linewidth]{fig/unique512Sports_and_Outdoors.png}
	     \\ First Codebook & Second Codebook & Third Codebook & Unique Tokens
		\end{tabular}
	\caption{The first and third codebooks still degenerate on Sports dataset with codebook size 512. And the second codebook also begin to degenerate.}	\label{fig:vis_sports_512}
\end{figure*} 

\begin{figure*}[htb!]
		\centering
		\begin{tabular}{cccc}
\includegraphics[width=0.22\linewidth]{fig/first_layer1024Sports_and_Outdoors.png} &
       \includegraphics[width=0.22\linewidth]{fig/second_layer1024Sports_and_Outdoors.png}  & \includegraphics[width=0.22\linewidth]{fig/third_layer1024Sports_and_Outdoors.png}  &
       \includegraphics[width=0.22\linewidth]{fig/unique1024Sports_and_Outdoors.png}
	     \\ First Codebook & Second Codebook & Third Codebook & Unique Tokens
		\end{tabular}
	\caption{Almost all codebooks degenerate on Sports dataset with codebook size 1024. In particular, the first and second codebooks degenerate extremely.}	\label{fig:vis_sports_1024}
\end{figure*} 

Besides, we also visualize the token distribution when codebook sizes are 64, 256, 512 and 1024 as Figure~\ref{fig:vis_sports_64} to \ref{fig:vis_sports_1024}. From the figure we can discover that:
\begin{itemize}[leftmargin=*]
\item \textbf{The codebooks begin to degenerate and be redundant when codebook size is greater than 256.} The first layer and second layer of codebooks begin to degenerate when codebook size is 256. With the increase of codebook size, the degeneration problem becomes more serious.
\item \textbf{The unique tokens are not influenced by codebook size too much.} With the growth of codebook size, the distribution of unqiue tokens almost keep unchange.

\end{itemize}

\begin{figure*}[htb!]
		\centering
		\begin{tabular}{cccc}
\includegraphics[width=0.22\linewidth]{fig/first_layer128Sports_and_Outdoors.png} &
       \includegraphics[width=0.22\linewidth]{fig/second_layer128Sports_and_Outdoors.png}  & \includegraphics[width=0.22\linewidth]{fig/third_layer128Sports_and_Outdoors.png}  &
       \includegraphics[width=0.22\linewidth]{fig/unique128Sports_and_Outdoors.png}
	     \\ First Codebook & Second Codebook & Third Codebook & Unique Tokens
      \end{tabular}
	\caption{The patterns of codebooks are various across different layers and unique tokens are uniform for different items on Sports dataset.}	\label{fig:vis_sport}
\end{figure*} 

\begin{figure*}[htb!]
		\centering
		\begin{tabular}{cccc}
\includegraphics[width=0.22\linewidth]{fig/first_layerToys_and_Games.png} &
       \includegraphics[width=0.22\linewidth]{fig/second_layerToys_and_Games.png}  & \includegraphics[width=0.22\linewidth]{fig/third_layerToys_and_Games.png}  &
       \includegraphics[width=0.22\linewidth]{fig/uniqueToys_and_Games.png}
	     \\ First Codebook & Second Codebook & Third Codebook & Unique Tokens
		\end{tabular}
	\caption{The patterns of codebooks are various across different layers and unique tokens are uniform for different items on Toys dataset.}	\label{fig:vis_toys}
\end{figure*} 

\subsection{Token Visualization on More Datasets}\label{sec:visual_token}
As shown in Figure~\ref{fig:vis_sport} and \ref{fig:vis_toys}, we visualize the patterns of codebooks on Sport and Toys datasets.













\end{document}
