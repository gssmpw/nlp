% CVPR 2025 Paper Template; see https://github.com/cvpr-org/author-kit

\documentclass[10pt,twocolumn,letterpaper]{article}

%%%%%%%%% PAPER TYPE  - PLEASE UPDATE FOR FINAL VERSION
%\usepackage{cvpr}              % To produce the CAMERA-READY version
%\usepackage[review]{iccv}      % To produce the REVIEW version
\usepackage[pagenumbers]{cvpr} % To force page numbers, e.g. for an arXiv version

\usepackage{lineno}
% Import additional packages in the preamble file, before hyperref
\newcommand{\CG}{\mathcal{G}\xspace}
\newcommand{\CV}{\mathcal{V}\xspace}
\newcommand{\CE}{\mathcal{E}\xspace}
\newcommand{\CA}{\mathcal{A}\xspace}
\newcommand{\CF}{\mathcal{F}\xspace}
\newcommand{\CR}{\mathcal{R}\xspace}
\newcommand{\CB}{\mathcal{B}\xspace}
\newcommand{\CX}{\mathcal{X}\xspace}
\newcommand{\CK}{\mathcal{K}\xspace}
\newcommand{\CM}{\mathcal{M}\xspace}
\newcommand{\CC}{\mathcal{C}\xspace}
\newcommand{\CL}{\mathcal{L}\xspace}
\newcommand{\CI}{\mathcal{I}\xspace}
\newcommand{\CQ}{\mathcal{Q}\xspace}
\newcommand{\CO}{\mathcal{O}\xspace}
\newcommand{\CP}{\mathcal{P}\xspace}
\newcommand{\CS}{\mathcal{S}\xspace}
\newcommand{\CT}{\mathcal{T}\xspace}
\newcommand{\CJ}{\mathcal{J}\xspace}
\usepackage[para]{footmisc}
\usepackage{subfig}
% \usepackage{subcaption}
% \usepackage{array}
% \usepackage{colortbl}


\usepackage{graphicx} % Add this to your preamble if not already included

% It is strongly recommended to use hyperref, especially for the review version.
% hyperref with option pagebackref eases the reviewers' job.
% Please disable hyperref *only* if you encounter grave issues, 
% e.g. with the file validation for the camera-ready version.
%
% If you comment hyperref and then uncomment it, you should delete *.aux before re-running LaTeX.
% (Or just hit 'q' on the first LaTeX run, let it finish, and you should be clear).
\definecolor{cvprblue}{rgb}{0.21,0.49,0.74}
\usepackage[pagebackref,breaklinks,colorlinks,allcolors=cvprblue]{hyperref}

%%%%%%%%% PAPER ID  - PLEASE UPDATE
\def\paperID{33} % *** Enter the Paper ID here
\def\confName{ICCV}
\def\confYear{2025}

%%%%%% Tao Hu
%%%%% NEW MATH DEFINITIONS %%%%%

% \usepackage{amsmath,amsfonts,bm}
\usepackage{amsmath,amsfonts}

\usepackage{pifont}


\newcommand{\R}{\mathbb{R}}


\def\va{{\mathbf{a}}}
\def\vg{{\mathbf{g}}}

% Sets
\def\sR{\mathbb{R}}
\def\sC{\mathbb{C}}
\def\sZ{\mathbb{Z}}
\def\sN{\mathbb{N}}
\def\sQ{\mathbb{Q}}

\def\sS{\mathcal{S}}



% Vectors
\def\vzero{{\mathbf{0}}}
\def\vone{{\mathbf{1}}}
\def\vmu{{\mathbf{\mu}}}
\def\vtheta{{\mathbf{\theta}}}
\def\va{{\mathbf{a}}}
\def\vb{{\mathbf{b}}}
\def\vc{{\mathbf{c}}}
\def\vd{{\mathbf{d}}}
\def\ve{{\mathbf{e}}}
\def\vf{{\mathbf{f}}}
\def\vg{{\mathbf{g}}}
\def\vh{{\mathbf{h}}}
\def\vi{{\mathbf{i}}}
\def\vj{{\mathbf{j}}}
\def\vk{{\mathbf{k}}}
\def\vl{{\mathbf{l}}}
\def\vm{{\mathbf{m}}}
\def\vn{{\mathbf{n}}}
\def\vo{{\mathbf{o}}}
\def\vp{{\mathbf{p}}}
\def\vq{{\mathbf{q}}}
\def\vr{{\mathbf{r}}}
\def\vs{{\mathbf{s}}}
\def\vt{{\mathbf{t}}}
\def\vu{{\mathbf{u}}}
\def\vv{{\mathbf{v}}}
\def\vw{{\mathbf{w}}}
\def\vx{{\mathbf{x}}}
\def\vy{{\mathbf{y}}}
\def\vz{{\mathbf{z}}}
\def\vzeta{{\mathbf{\zeta}}}

% Matrix
\def\mA{{\mathbf{A}}}
\def\mB{{\mathbf{B}}}
\def\mC{{\mathbf{C}}}
\def\mD{{\mathbf{D}}}
\def\mE{{\mathbf{E}}}
\def\mF{{\mathbf{F}}}
\def\mG{{\mathbf{G}}}
\def\mH{{\mathbf{H}}}
\def\mI{{\mathbf{I}}}
\def\mJ{{\mathbf{J}}}
\def\mK{{\mathbf{K}}}
\def\mL{{\mathbf{L}}}
\def\mM{{\mathbf{M}}}
\def\mN{{\mathbf{N}}}
\def\mO{{\mathbf{O}}}
\def\mP{{\mathbf{P}}}
\def\mQ{{\mathbf{Q}}}
\def\mR{{\mathbf{R}}}
\def\mS{{\mathbf{S}}}
\def\mT{{\mathbf{T}}}
\def\mU{{\mathbf{U}}}
\def\mV{{\mathbf{V}}}
\def\mW{{\mathbf{W}}}
\def\mX{{\mathbf{X}}}
\def\mY{{\mathbf{Y}}}
\def\mZ{{\mathbf{Z}}}
\def\mBeta{{\mathbf{\beta}}}
\def\mPhi{{\mathbf{\Phi}}}
\def\mLambda{{\mathbf{\Lambda}}}
\def\mSigma{{\mathbf{\Sigma}}}


% Expectation
% \def\eE{\mathop{\mathbb{E}}\limits}
\def\eE{\mathbb{E}}

% Probability
\def\pP{\mathbb{P}}

% Tilde
\def\tf{\tilde{f}}
\def\tS{\tilde{S}}
\def\wtF{\widetilde{\mathcal{F}}}
\def\whR{\widehat{R}}
\def\tvx{\tilde{\mathbf{x}}}
\def\ty{\tilde{y}}


\def\defeq{\overset{\textup{def}}{=}}
% \def\defeq{\overset{.}{=}}
\def\defone{\overset{\text{\ding{172}}}{=}}
\def\deftwo{\overset{\text{\ding{173}}}{=}}
\def\leqone{\overset{\text{\ding{172}}}{\leq}}
\def\leqtwo{\overset{\text{\ding{173}}}{\leq}}
\def\leqthree{\overset{\text{\ding{174}}}{\leq}}
\def\leqfour{\overset{\text{\ding{175}}}{\leq}}
\def\eqone{\overset{\text{\ding{172}}}{=}}
\def\eqtwo{\overset{\text{\ding{173}}}{=}}
\def\eqthree{\overset{\text{\ding{174}}}{=}}
\def\eqfour{\overset{\text{\ding{175}}}{=}}
\def\geqfive{\overset{\text{\ding{176}}}{\geq}}

\newcommand{\mmm}{\texttt{[MASK]}}


\usepackage{graphicx}
\usepackage{amsmath}
\usepackage{amssymb}
\usepackage{booktabs}
\usepackage{times}
\usepackage{microtype}
\usepackage{epsfig}
%\usepackage[table,xcdraw]{xcolor}
\usepackage{caption}
\usepackage{xcolor}
\usepackage{float}
\usepackage{placeins}
\usepackage{color, colortbl}
\usepackage{stfloats}
\usepackage{enumitem}
\usepackage{tabularx}
\usepackage{xstring}
\usepackage{multirow}
\usepackage{xspace}
\usepackage{url}
\usepackage{subcaption}
\usepackage{arydshln}
\usepackage{xcolor}
\usepackage{bbm}
\usepackage[hang,flushmargin]{footmisc}
\usepackage{pifont}%
\newcommand{\cmark}{\ding{51}}%
\newcommand{\xmark}{\ding{55}}%
\usepackage{hyperref}
\usepackage{wrapfig}
\usepackage{lipsum}
\usepackage{comment}
\usepackage{algorithm}
\usepackage{algorithmic}
\usepackage{listings}
\usepackage{makecell} 
\usepackage{xcolor}
\definecolor{lightblue}{RGB}{185, 200, 240}
\usepackage{pifont}
\usepackage{tikz}
%\usepackage[commentColor=black,beginLComment=/*~, endLComment=~*/]{algpseudocodex}
%\newcommand{\LFM}[0]{\gL_{\rm FM}}
%\newcommand{\LCFM}[0]{\gL_{\rm CFM}}
%\newcommand{\LECFM}[0]{\gL_{\rm ECFM}}
%\newcommand{\theHalgorithm}{\arabic{algorithm}}
\usepackage{makecell}



\newcommand{\et}[2]{${#1}^{\pm{#2}}$}
\newcommand{\etb}[2]{$\mathbf{{#1}}^{\pm{#2}}$}
\newcommand{\etr}[2]{$\textcolor{red}{{#1}}^{\pm{#2}}$}
\newcommand{\etbb}[2]{$\textcolor{blue}{{#1}}^{\pm{#2}}$}
\newcommand{\ets}[2]{$\underline{{#1}}^{\pm{#2}}$}


%\usepackage[pagebackref,breaklinks,colorlinks]{hyperref}



\usepackage[capitalize]{cleveref}
\crefname{section}{Sec.}{Secs.}
\Crefname{section}{Section}{Sections}
\Crefname{table}{Table}{Tables}
\crefname{table}{Tab.}{Tabs.}

\usepackage{wrapfig}




\newcommand{\tao}[1]{\textcolor{red}{Tao: #1}}
\newcommand{\michael}[1]{\textcolor{blue}{M: #1}}
\newcommand{\timy}[1]{\textcolor{green}{Timy: #1}}

\newcommand{\tabref}[1]{Table~\ref{#1}}

\newcommand{\bff}[1]{\textcolor{yellow}{BF: #1}}
\newcommand{\cs}[1]{\textcolor{orange}{CS: #1}}
\newcommand{\psmm}[1]{\textcolor{blue}{PM: #1}}
\newcommand{\yc}[1]{\textcolor{olive}{YC: #1}}


\newcommand{\noisemarg}{p_{t|1}}
\newcommand{\denoise}{p_{1|t}}
\newcommand{\x}{x}
\newcommand{\y}{y}
%\newcommand{\pdata}{p_{\mathrm{data}}}
\newcommand{\statespace}{S}
\newcommand{\SO}{\mathrm{SO}(3)}
\newcommand{\SE}{\mathrm{SE}(3)}
\newcommand{\Dt}{\Delta t}
\newcommand{\Dtilde}{\Delta \tilde{t}}
\newcommand{\kdelta}[2]{\delta \left\{ #1, #2 \right\} }


\newcommand{\ourmethod}{MDFM}
\newcommand{\ourmethodlong}{Revist Discrete Visual Generation}

%%%%% Tao Hu

%%%%%%%%% TITLE - PLEASE UPDATE
\title{
MaskFlow: Discrete Flows For Flexible and Efficient Long Video Generation
}

%%%%%%%%% AUTHORS - PLEASE UPDATE
\begin{comment}
\author{Vincent Tao Hu\\
Institution1\\
Institution1 address\\
{\tt\small firstauthor@i1.org}
% For a paper whose authors are all at the same institution,
% omit the following lines up until the closing ``}''.
% Additional authors and addresses can be added with ``\and'',
% just like the second author.
% To save space, use either the email address or home page, not both
\and
Björn Ommer\\
Institution2\\
First line of institution2 address\\
{\tt\small secondauthor@i2.org}
}
\end{comment}




\author{Michael Fuest, Vincent Tao Hu\thanks{ Project Leader}, Björn Ommer \\
%\footnote{\tt\small{\textsuperscript{*}Project Leader}}\\
%Institution1\\
%Institution1 address\\
%{\tt\small firstauthor@i1.org}
% For a paper whose authors are all at the same institution,
% omit the following lines up until the closing ``}''.
% Additional authors and addresses can be added with ``\and'',
% just like the second author.
% To save space, use either the email address or home page, not both
%\and
%Björn Ommer\\
%Institution2\\
%First line of institution2 address\\
%{\tt\small secondauthor@i2.org}
%{\tt\small secondauthor@i2.org}
{ CompVis @ LMU Munich, MCML}\\
%\email{lncs@springer.com}
\\
\url{https://compvis.github.io/maskflow/}
}
\vspace{-10pt}

\setlength{\parskip}{0pt}

\begin{document}
\maketitle

\begin{abstract}
    \begin{abstract}
Retrieval-Augmented Generation (RAG) is often used with Large Language Models (LLMs) to infuse domain knowledge or user-specific information. In RAG, given a user query, a retriever extracts chunks of relevant text from a knowledge base. These chunks are sent to an LLM as part of the input prompt. Typically, any given chunk is repeatedly retrieved across user questions. However, currently, for every question, attention-layers in LLMs fully compute the key values (KVs) repeatedly for the input chunks, as state-of-the-art methods cannot reuse KV-caches when chunks appear at arbitrary locations with arbitrary contexts. Naive reuse leads to output quality degradation.  This leads to potentially redundant computations on expensive GPUs and increases latency. In this work, we propose \sys, a system for managing and reusing precomputed KVs corresponding to the text chunks (we call \textit{chunk-caches}) in RAG-based systems. We present how to identify \hl{\textit{chunk-caches} that are reusable}, how to efficiently perform a small fraction of recomputation to \textit{fix} the cache to maintain output quality, and how to efficiently store and evict \textit{chunk-caches} in the hardware for maximizing reuse while masking any overheads. With real production workloads as well as synthetic datasets, we show that \sys reduces redundant computation by \textbf{51\%} over SOTA prefix-caching and \textbf{75\%} over full recomputation.
\hl{Additionally, with continuous batching on a real production workload, we get a \textbf{1.6$\times$} speedup in throughput and a \textbf{2$\times$} reduction in end-to-end response latency over prefix-caching while maintaining quality, for both the \llama-3-8B and \llama-3-70B models. 
}
\end{abstract}





\end{abstract}

\section{Introduction}
% 
Motion planning is a key ingredient in autonomous robotic systems, whose aim is computing collision-free trajectories for a robot operating in environments cluttered with obstacles~\cite{lavalle2006planning}. 
Over the years, various approaches have been developed for tackling the problem, including potential fields~\cite{luo2024potential}, geometric methods~\cite{halperin2017algorithmic}, and optimization-based approaches~\cite{SchulmanDHLABPPGA14,MalyutaEtAl2022,MarcucciEA23}. %, and sampling-based planners~\cite{}. 
In this work, we focus on sampling-based planners (SBPs), which aim to capture the structure of the robot's free space through graph approximations that result from configuration sampling (typically in a random fashion) and connecting nearby samples. 
SBPs have enjoyed popularity in recent years due to their relative scalability, in terms of the number of robot degrees of freedom (DoFs), and the ease of their implementation~\cite{OrtheyCK24}. 

\begin{figure*}[h!]
  \centering
  \subfloat[$\X_{\dZ_2}^{\delta,\epsilon}$ sample set.]{
    \includegraphics[width=0.27\textwidth, trim={2.2cm 1.7cm 0.9cm 1.0cm},clip]{Images/ZN_2D.png}
    %\label{fig:2d_lattices:z}
    }
  \hfil
  \subfloat[$\X_{D_2^*}^{\delta,\epsilon}$ sample set.]{
    \includegraphics[width=0.27\textwidth, trim={2.2cm 1.8cm 0.9cm 1.0cm},clip]{Images/DN_2D.png}
    %\label{fig:2d_lattices:d}
    }
  \hfil
  \subfloat[$\X_{A_2^*}^{\delta,\epsilon}$  sample set.]{
    \includegraphics[width=0.27\textwidth, trim={2.4cm 1.7cm 0.9cm 1.0cm},clip]{Images/AN_2D.png}
    %\label{fig:2d_lattices:a}
    }
  \caption{Sample sets within a fixed disc in $\dR^2$, derived from the lattices $\dZ^2, D_2^*$ and $A^*_2$, which yield \decomp guarantees for the same values of $\delta$ and $\eps$. The set $\X_{\dZ_2}^{\delta,\epsilon}$ can be viewed as a tessellation of space using cubes. The set $\X_{D_2^*}^{\delta,\epsilon}$ is obtained by placing a (rescaled) standard grid, and then placing another point in the middle of each cube. The set $\X_{A_2^*}^{\delta,\epsilon}$ can be viewed as a rescaled hexagonal grid as each point is surrounded by a hexagon whose vertices are points in the set. Note that the density of $\X_{\dZ^2}^{\delta,\eps}$ and $\X_{D^*_2}^{\delta,\eps}$ is the same, and higher than the density of $\X_{A^*_2}^{\delta,\eps}$.}
  \label{fig:2d_lattices}
\end{figure*}

Another key benefit is the ability of SBPs to escape local minima (unlike potential fields) and global solution guarantees (in contrast, optimization-based approaches~\cite{SchulmanDHLABPPGA14}, which typically provide only local guarantees). Earlier work on the theoretical foundations of SBPs has focused on deriving probabilistic completeness (PC) guarantees for methods such as PRM~\cite{kavraki1996probabilistic} or RRT~\cite{LaVKuf01,KunzS14,Kleinbort.Solovey.ea.19}. PC implies that the probability of a given planner finding a solution (if one exists) converges to one as the number of samples tends to infinity. The work of~\citet{karaman2011sampling} initiated studying the quality of the solution returned by SBPs. Specifically, they introduced the planners PRM* and RRT*, and proved that the solution length of those planners converges to the optimum as the number of samples tends to infinity---a property called asymptotic optimality (AO). Subsequent work has introduced even more powerful AO planners for geometric~\cite{JSCP15,GammellBS20} and dynamical~\cite{HauserZ16,LiETAL16} systems.

Unfortunately, the practical relevance of the aforementioned theoretical findings remains limited due to the lack of meaningful finite-time implications. Specifically, when a solution is obtained using a finite number of samples, it is unclear to what extent its quality can be improved with additional computation time. Moreover, in cases where no solution is returned, it is uncertain whether a solution does not exist or if the algorithm simply failed to find one. Developing finite-time bounds through randomized sampling continues to be a significant challenge~\cite{DobsonMB15,shaw2024towards}.

Deterministic sampling methods such as grid sampling or Halton sequences~\cite{lavalle2006planning}, where samples are generated according to a geometric principle, can improve the performance of SBPs in practice and simplify the algorithm analysis. Specifically, some deterministic sampling procedures have a significantly lower dispersion than uniform random sampling, which implies that the former requires fewer samples to cover the search space to a desired resolution~\cite{janson2018deterministic}. 
Recently, Tsao et al.~\cite{tsao2020sample} have leveraged deterministic sampling to disrupt the asymptotic analysis paradigm by introducing a significantly stronger notion than AO, called \decomps, that yields finite-time guarantees for PRM-based algorithms such as PRM*~\cite{karaman2011sampling}, FMT*~\cite{JSCP15}, BIT*~\cite{GammellBS20}, and GLS~\cite{MandalikaCSS19}. Informally, a \emph{finite} sample set is \decomp for a given approximation factor $\eps>0$ and clearance parameter $\delta>0$, if the corresponding planner returns a solution whose length is at most $(1+\eps)$ times the length of the shortest $\delta$-clear solution. If no solution is found using a \decomp sample set then no solution of clearance $\delta$ exists. 

The work of~\citet{tsao2020sample} derived a relation between \decomps and geometric space coverage to obtain lower bounds on the number of samples necessary to achieve \decomps, as well as upper bounds accompanied with explicit (deterministic) sampling distributions. A follow-up work by~\citet{dayan2023near} has introduced an even more compact \decomp sample distribution that is more efficient than the one proposed in~\cite{tsao2020sample} or rectangular grid sampling. In particular, the staggered grid~\cite{dayan2023near} consists of two shifted and rescaled copies of the rectangular grid (see Figure~\ref{fig:2d_lattices} and Figure~\ref{fig:3d_lattices}). 

However, the work~\cite{dayan2023near} still leaves a significant gap between the lower bound in~\cite{tsao2020sample} and the upper bound obtained with the staggered grid. In practice, this gap limits the applicability of the \decomps theory to relatively low dimensions (up to dimension 6) due to the large number of samples currently needed to satisfy this property, which can lead to excessive running times. 

\vspace{5pt}
\noindent \textbf{Contribution.} In this work, we develop a theoretical framework for obtaining highly-efficient \decomp sample sets by leveraging the foundational theory of lattices\footnote{Lattices are point sets exhibiting a regular geometric structure, which are obtained by transforming the integer lattice $\dZ^d$. For instance, the aforementioned rectangular grid and the staggered grid can be viewed as lattices.}~\cite{conway2013sphere}, which has been instrumental in diverse areas from number theory~\cite{siegel_geometry_numbers}, coding theory~\cite{ebeling2013lattices}, and crystallography~\cite{sands1994introduction}. Specifically, we show that lattices can be transformed to obtain \decomp sample sets (Theorem~\ref{thm:decomp_lattices}) and develop tight theoretical bounds on their size (Theorem~\ref{thm:general_sample_complexity}), which allows to compare between different sample sets qualitatively. 
Using this machinery, we not only refine and generalize previous results on the staggered grid~\cite{dayan2023near} but also introduce a new highly efficient \decomp sample set that is based on the $\AN$ lattice, which is famous for its minimalist coverage properties~\cite{conway2013sphere}. We also initiate the study of a new property, which estimates the computational cost resulting from using a given sample set in a more informative manner than sample complexity. In particular, the property called collision-check complexity captures the amount of collision checks, which is typically a computational bottleneck.

From a practical perspective, when solving motion-planning problems using lattice-based sample sets, we show that our $\AN$-based sample sets can result in at least order-of-magnitude improvement in terms of running time over staggered-grid samples and two orders of magnitude improvements over rectangular grids. Moreover, $\AN$-based sample sets are vastly superior in practice to the widely-used uniform random sampling, which is evident in improved running times, success rates, and solution quality.

\vspace{5pt}
\noindent \textbf{Organization.} In Section~\ref{sec:preliminaries} we review basic definitions on motion planning and \decomps, and formally define our objectives. In Section~\ref{sec:lattices}, we develop a general tool for transforming lattices into \decomp sample sets. We obtain sample-complexity bounds for lattice-based sample sets in Section~\ref{sec:sample_complexity}, and generalize those bounds to collision-check complexity in Section~\ref{sec:collision_complexity}. We evaluate the practical implications of our theory in Section~\ref{sec:experiments}, and conclude with a discussion of limitations and future directions in Section~\ref{sec:future}.


% \itai{Added the intro. used some from the thesis-proposal, added different stuff at the end}
%  the field of autonomous robots, the problem of getting a robot “from point A to point B” can be divided into three general stages: estimating the robot's position, planning the robot's path and controlling the robot. We use estimation methods (like the Kalman Filters) to understand where we are in the world, we use planning methods to figure out how to reach the goal, and we use control methods (like PID) to follow the planned path during execution.


% Focusing on the planning part of the problem, instead of using the \emph{workspace} of the robot it is convenient to use a representation of it called a \emph{configuration space}---a parameterization of the robot’s position in space, which turns the set of points defined as a \emph{robot} to a single-point robot. A quick example would be thinking of a polygon in the workspace as three parameters: $(x,y)\in \mathbb{R}^2$ for its location, and $\theta\in[0,2\pi]$ for its rotation, which means the configuration space is $\mathbb{R}^2\times S^1$. Furthermore, we use the term \emph{free space} in both contexts to describe the area of the space with no obstacles. 


% Even though using configuration spaces is much more convenient in terms of the robot being a single point, it quickly becomes apparent that even simple configuration spaces of dimensions $d\geq 3$ can be challenging to properly describe (due to the need to describe the obstacles in the new space, among other reasons). Thus, instead of explicitly representing the whole configuration space, methods were developed to sample the space: sampling-based approaches aim to approximate the space via a graph structure that is induced by sampled configurations. This can drastically reduce the computational effort of path planning.


% One of the most widely used sampling-based algorithms is the \emph{probabilistic roadmap method} (PRM)~\cite{kavraki1996probabilistic}. This approach generates (typically random) samples across the space, and connects nearby samples while checking for collisions with obstacles, which gives rise to a graph data structure---a path between two nodes in the graph yields a collision-free path for the robot connecting between the two configurations corresponding to the end-point nodes. PRM has the theoretical guarantee to return a path with a probability tending to 1 if enough samples are generated~\cite{laddgeneralizing}. 


% Another well known sampling-based planner is the \emph{rapidly-exploring random trees} (RRT)~\cite{lavalle1998rapidly}: It randomly expands towards nearby samples in space, creating in the process a “tree” structure that eventually finds a path to the goal~\cite{kleinbort2018probabilistic}. Later, a notion of \emph{asymptotically optimal} (AO) algorithms was introduced: with infinite samples, the algorithm can converge to an \textbf{optimal} path. Both PRM and RRT  were expanded to AO versions (PRM*, RRT*) in a paper by Karaman et al.~\cite{karaman2011sampling}. RRT itself had seen many expansions, including dRRT/dRRT* to apply to multiple robots~\cite{solovey2015finding, dobson2017scalable}.

% Approximately optimal methods were demonstrated using deterministic sample sets, achieving good results in finite time, in Dayan et al.'s paper~\cite{dayan2023near}, demonstrating superior results over random sets---although those improvements diminish as the desired approximation factor of the optimal path lowers.

% Still, all these methods have a main limitation: the number of points required to guarantee finding a path rises exponentially with the robot's degrees of freedom~\cite{tsao2020sample}.


% Dayan et al.~\cite{dayan2023near}, using a "staggered grid" structure (recognized in this paper as the $\DN$ set), gave guarantess for an approximately-optimal solution in finite time, which outperformed random sets in certain situations. For this, they introduced the concept of a \decomp set, a set that generates such approximate solutions. Still, as we seek better and better approximations for the optimal path, the staggered grid in the paper falls off against random sets. The question that stands, then, is what other sample sets can be used to provide better results?


% In this paper, we would like to utilize \decomp sets and investigate a specific series of deterministic sample sets, using lattices---a generalization of the regular grid structure using a general set of mutually-independent base vectors (not necessarily the usual $(0,\dots,1,\dots,0)$ vectors). We first familiarize the reader with three different lattices \Lattices we intend on investigating, and then move on to using Dayan et al.'s~\cite{dayan2023near} definition of a \decomp set to define lattice sample sets as such sets. This definition tells us that these sets can give us a good approximation for the optimal solution at a finite time. 


% After that, we use our new lattice sample sets to investigate the upper bounds on the number of sample points, and on the sum of edge length in a typical PRM vertices-connecting $r$-Ball---something we use as a measure point to the algorithm's complexity, as it is known that collision checks along the edges are the bottleneck in today's PRM algorithms.


% We will end up demonstrating, theoretically and practically, that one lattice, $\AN$, stands out as performing much better than the regular grid often used in many MP algorithms.

\section{Background and Related Work}
% This paper is related to previous work on dense and MoE model merging.

\subsection{Dense Model Merging}

% Model merging methods combine multiple domain experts into a single model to obtain diverse capabilities \cite{wortsman2022model, ilharco2022editing, goddard2024arcee, jin2022dataless, matena2022merging, yadav2024ties, yu2024language, roberts2024pretrained}. Most merging methods focus on homogeneous dense expert models into another dense model of the same architecture: average merging \cite{wortsman2022model} applies averaging over the model parameters; task vector merging \cite{ilharco2022editing} first extracts the task vector (the difference of parameters between the base and the experts) and adds the unweighted sum of the task vectors back to the dense model with a scaling term. Other works explore how to determine the weights of the task vector instead of an unweighted sum \cite{jin2022dataless, matena2022merging}. Dare and Ties \cite{yadav2024ties, yu2024language}, two SoTA merging methods, choose to trim the task vector to resolve parameter interference: in Dare, the task vector is randomly trimmed and rescaled by a fixed amount;
% in the Ties method, on the other hand, the authors
% set the vector parameter to 0 by magnitude and adjust the sign of each parameter to reduce the sign conflict.

Dense merging methods combine multiple dense models into one to achieve diverse capabilities \cite{wortsman2022model, ilharco2022editing, goddard2024arcee, jin2022dataless, matena2022merging, roberts2024pretrained}. Most approaches focus on merging homogeneous dense models into another dense model. For example, average merging~\cite{wortsman2022model} averages model parameters, while task vector merging~\cite{ilharco2022editing} adds the unweighted sum of task vectors (the difference between base and expert parameters) back to the dense model with scaling. Other work determines task vector weights instead of using an unweighted sum \cite{jin2022dataless, matena2022merging}. SoTA methods like Dare and Ties \cite{yadav2024ties, yu2024language} trim the task vector to resolve parameter interference: Dare trims the task vector randomly and rescales, while Ties sets vector parameters to zero by magnitude and adjusts signs to reduce conflicts.

% In addition to homogeneous model merging, \citet{roberts2024pretrained} proposes an approach to merge heterogeneous models to a dense model by incorporating the projectors. \citet{wan2024knowledge} applies the knowledge distillation to fuse heterogeneous models.
% Compared with previous merging works for dense models, our paper focuses on merging experts to an MoE model. For homogeneous experts, we explore a more efficient merging method and merging without fine-tuning. We are the first work to propose a pipeline for merging heterogeneous models into an MoE.

In addition to homogeneous model merging, \citet{roberts2024pretrained} propose merging heterogeneous models into a dense model using projectors, while \citet{wan2024knowledge} apply knowledge distillation to fuse heterogeneous models. 
In this work, we introduce a more efficient method for merging experts with limited or no further fine-tuning and, unlike previous work focusing on dense models, we explore merging homogeneous and heterogeneous experts into an MoE model. 
%Additionally, we propose 
%For homogeneous experts, we propose a more efficient merging method without fine-tuning. We are also the first to introduce a pipeline for merging heterogeneous models into an MoE.

\subsection{MoE Training and Merging}

% MoE architectures allows for faster training given a fixed computational budget, and faster inference given a fixed number of parameters.  They do so by introducing Sparse MoE layers, where a router mechanism dispatches tokens to the top-K expert FFNs in parallel (where k is usually 1 or 2)  \cite{fedus2022switch, shazeer2017outrageously, zhang2022mixtureattentionheadsselecting}. In terms of MoE training, most work chooses to train the whole MoE on all domain data to obtain capabilities across multiple tasks \cite{komatsuzaki2022sparse, jiang2024mixtral, dou2024loramoe, dai2024deepseekmoe}, but full communication during training on different GPUs is costly \cite{sukhbaatar2024branchtrainmixmixingexpertllms}. To address this challenge, inspired by the Branch-Train-Merge (BTM) method \cite{gururangan2023scaling, li2022branch}, where they average the final model output distributions from different experts, the Branch-Train-Mix (BTX) method \cite{sukhbaatar2024branchtrainmixmixingexpertllms} first branches the base model to multiple copies, trains the base model in different domains, and merges the experts into a unified MoE. Another recent work, Self-MoE \cite{kang2024self}, uses the LoRa style \cite{hu2021lora} to fine-tune each expert on self-generated data and then combine all trained adapters into a base model for a LoRa MoE. 

MoE architectures enable quicker inference with a certain parameter count by introducing Sparse MoE layers, where a router mechanism assigns tokens to the top-$K$ expert FFNs (usually 1 or 2) in parallel \cite{fedus2022switch, shazeer2017outrageously, zhang2022mixtureattentionheadsselecting}. Most MoE training approaches, known as upcycling, train the entire model from scratch to handle multiple tasks \cite{komatsuzaki2022sparse, jiang2024mixtral, dou2024loramoe, dai2024deepseekmoe}. These methods first initialize the MoE model from a pretrained base model and then train it on the entire dataset. However, due to the costly communication between GPUs, the upcycling method introduces significant computational overhead \cite{sukhbaatar2024branchtrainmixmixingexpertllms, li-etal-2024-pedants}. To address this, methods like Branch-Train-Merge (BTM) \cite{gururangan2023scaling, li2022branch} average model outputs from different experts, while Branch-Train-Mix (BTX) \cite{sukhbaatar2024branchtrainmixmixingexpertllms} branches the base model, trains each on different domains, and merges them into a unified MoE. 
BTX is shown to be more effective than BTM as well as dense CPT and MoE upcycling baselines.
%VS: do we need to explain upcycling?
Another recent approach, Self-MoE \cite{kang2024self}, uses low-rank adaptation (LoRA) \cite{hu2021lora} to fine-tune experts on generated synthetic data \cite{liu2024csrec} and combines trained adapters into an MoE.
To our knowledge, we are the first to introduce a framework for merging heterogeneous models into an MoE.



\section{Method}

\begin{figure*}[ht!]
    \centering
    \includegraphics[width=0.9\linewidth]{figpaper/maskflow_sampling.pdf}
    \vspace{-7pt}
    \caption{\textbf{MaskFlow Sampling:} Given $m=2$ \textcolor{red}{context frames} used to initialize generation, we unmask the current window and use \textcolor{color_generated_frame}{newly generated frames} as new context frames in the next chunk of size $k=5$, using stride $s=3$. (\textit{Tokenization omitted here to simplify understanding}) .
    }
    \label{fig:sampling}
    \vspace{-10pt}
\end{figure*}

\subsection{Task formulation: Long video generation}
\label{subsec:long_vid}

 There are, generally, three distinct approaches to long video generation. The first is the naive approach of training on long video sequences. This is challenging due to the quadratic complexity in attention mechanisms with respect to token numbers. Although works like\cite{tan2024video,harvey2022flexible} address this by distributing the generative process or by generating every \(n\)-th frame and subsequently infilling the remaining frames, the approach remains fundamentally resource-intensive. The second approach is a \emph{rolling} (or ``sliding-window'') approach, which applies monotonically increasing noise dependent on a frame's position in the sliding window. This process can be rolled out indefinitely, removing frames from the window when they are fully denoised and appending random noise frames at the end of the window. Works such as \cite{ruhe2024rollingdiffusionmodels, wu2023ar, xie2024progressive} belong to this paradigm. The third approach is \emph{chunkwise-autoregression}, also referred to as blockwise-autoregression \cite{ruhe2024rollingdiffusionmodels}. Here, the video of length $L$ is divided into overlapping \emph{chunks} of length $k \ll L$, where each chunk overlaps by $m$ frames, which we refer to as context frames. Concretely, we define a video and its frames as

\begin{equation}
   \mathbf{v} = (v^1, v^2, \dots, v^L)
\end{equation}

which we divide into overlapping chunks of length $k$. Let $\ell\;=\;\left\lceil \frac{L - k}{s} \right\rceil + 1$ denote the number of chunks needed to cover the video of length $L$, and we further define each chunk $\mathbf{v}^{(i)}$ as

\begin{equation}
   \mathbf{v}^{(i)} = \bigl(v^{(i-1)\,s + 1}, \dots, v^{(i-1)\,s + k}\bigr),
\end{equation}

where $s \le k$ is the sampling window stride, i.e., how far the context start shifts at each step. Often, one sets $s = k - m$, but this is not strictly required. The video distribution then factorizes as

\begin{equation}
\label{eq:markov}
   p(\mathbf{v};{\theta})
   \;=\;
   p(\mathbf{v}^{(1)};\theta)
   \prod_{i=2}^{\ell}
      p
         \bigl(
            \mathbf{v}^{(i)} 
            \;\bigm|\;
            \mathbf{v}^{(i-1)};\theta
         \bigr).
\end{equation}

Because each $\mathbf{v}^{(i)}$ overlaps the previous chunk by $m$ frames, the context frames feed into the next chunk's generation, ensuring smooth transitions and continuity between chunks. To enable such Markovian temporal dependencies during sampling, it is crucial to train a flexible backbone model \(p(\mathbf{v};\theta)\) that can generalize across different sampling schemes, such as the one defined in Equation~\eqref{eq:markov}.


\subsection{Preliminary: Flow Matching for Videos}
\label{subsec:flow}

Our masking flow matching approach, named \emph{MaskFlow}, draws inspiration from previous works that apply individual noise levels to individual frames in a sequence \cite{chen2024diffusionforcing, ruhe2024rollingdiffusionmodels}. These works operate in a continuous space, and use diffusion processes to corrupt data. MaskFlow operates in a discrete token space and uses \emph{masking} to corrupt data. We seek to learn a continuous transition process in ``time'' \(t\) that moves from a purely masked sequence at \(t=0\) to the unmasked token sequence at \(t=1\). In our method, the timestep $t$ corresponds to the masking ratio, and represents the frame-level probability of a token being masked. Consider a video consisting of \(L\) frames, where each frame is mapped to a discrete latent space using a vector-quantized (VQ) tokenizer \cite{esser2021taming_vqgan}. This tokenizer encodes each frame in the video $\mathbf{v}$ to a set of discrete latent indices $\mathbf{x}_{\text{latent}} \in [K]^N$, which consists of $N$ tokens drawn from the tokenizer vocabulary of size $K$. Let \(\mathcal{F}\) denote the VQ encoder-decoder, i.e., the function that maps a video in pixel space to its tokenized representation. Then, we have

\begin{equation}
    \mathbf{x} = \mathcal{F}(\mathbf{v}) \in [K]^{L \times N},
\end{equation}

where \([K] = \{1, 2, \ldots, K\}\) is the set of all possible token indices which includes a special ``mask token" $M \in [K]$. The choice of tokenization is essential here, since it compresses spatial dimensions of $\mathbf{x}$ compared to $\mathbf{v}$ and allows us to employ discrete flow matching, which we outline in further detail in the following section.

\begin{algorithm}[!ht]
\caption{\textbf{Training with Frame-level Masking}}
\label{alg:training}
\begin{algorithmic}[1]
\REQUIRE 
  Dataset of tokenized video clips $\mathcal{D}$, 
  network $p(\mathbf{x}_1 \mid \mathbf{x}_t, \mathbf{t};\theta)$, 
  chunk size $k$

\WHILE{not converged}
    \STATE \textbf{Sample} a chunk of $k$ frames  from $\mathcal{D}$, denoted 
      $\mathbf{x}_1 = (x_1^1, x_1^2, \dots, x_1^k)$
    %\STATE \textbf{Initialize} fully-masked chunk: 
      %$\mathbf{x}_0 = (\, [M], [M], \dots, [M] \,)$
    \FOR{$f = 1, \dots, k$}
        \STATE $t_f \sim \mathcal{U}(0,1)$
        \STATE 
           $x_{t^f} \;\sim\; p_{t^f \mid 0,1}\bigl(\,\cdot \mid x_0^f,\,x_1^f\bigr)$,
        where $p_{t^f \mid 0,1}$ follows 
        $
           (1 - t^f)\,\delta_{x_0^f} 
           \;+\; 
           t^f\,\delta_{x_1^f}.
        $
    \ENDFOR
    \STATE $\mathbf{x}_t = (x_{t^1}^1,\, x_{t^2}^2,\, \dots,\, x_{t^k}^k)$
    \STATE 
    $
       \hat{\mathbf{x}}_1 \;=\; p\bigl(\mathbf{x}_1 \mid \mathbf{x}_t,\, \mathbf{t};\theta\bigr),
    $
    where $\mathbf{t} = (t^1,\ldots,t^k)$
    \STATE \textbf{Backpropagate} $\mathcal{L}_\theta(\mathbf{x}_1, \hat{\mathbf{x}}_1)$ and \textbf{update} $\theta$.
\ENDWHILE
\end{algorithmic}
\end{algorithm}
\vspace{-10pt}

\paragraph{Discrete Flow Matching.}  
Discrete flow matching \cite{gat2024discrete} defines a vector field \(u_t\) in a discrete space that can be traversed to yield a smooth probability transition between our source distribution of fully masked frame sequences $p(\mathbf{x}_0)$ and the distribution of unmasked sequences $p(\mathbf{x}_1)$. This vector field defines an optimal transport path between the two distributions. Concretely, we construct the conditional probability path:
\begin{equation}
    p_{t \,\vert\, 0,1}\bigl(\mathbf{x} \,\vert\, \mathbf{x}_0, \mathbf{x}_1\bigr)
    \;=\; (1-t)\,\delta_{\mathbf{x}_0}(\mathbf{x})
    \;+\; t\,\delta_{\mathbf{x}_1}(\mathbf{x}),
\end{equation}

where \(\delta_{\mathbf{x}_0}(\mathbf{x})\) and \(\delta_{\mathbf{x}_1}(\mathbf{x})\) are Dirac delta functions (analogous to one hot encodings) in the discrete space that allocate all probability mass to the fully masked and fully unmasked sequences at $t=0$ and $t=1$, respectively. For any intermediate value \(t \in (0,1)\), the interpolation governed by the weights \((1-t)\) and \(t\) yields a new video sequence \(\mathbf{x}_t\) that represents a partially corrupted sequence. This is achieved by sampling each token from a mixture distribution where $1-t$ represents the probability of a token being masked.
\vspace{-10pt}
\paragraph{Kolmogorov Equation in Discrete State Spaces.}
In continuous-state models, one leverages the Continuity Equation \cite{song2021scorebased_sde} to ensure that a vector field \(u(\mathbf{x}_t, t)\) induces the desired probability transition between \(p(\mathbf{x}_0)\) and \(p(\mathbf{x}_1)\). The discrete counterpart is given by the Kolmogorov Equation \cite{campbell2024generative}, which similarly characterizes how a probability distribution evolves in time over discrete states. To achieve a transition between the fully masked and fully unmasked video distributions, we define the vector field:

\begin{equation}
    u_t(\mathbf{x}_t) 
    \;=\; 
    \frac{t}{\,1 - t} 
    \Bigl[
    p_{1 \,\vert\, t}(\mathbf{x}_1 \mid \mathbf{x}_t, t; \theta) 
    \;-\; 
    \delta_{\mathbf{x}_t}(x)
    \Bigr],
\end{equation}

where \(p_{1|t}(\mathbf{x}_1 \mid \mathbf{x}_t, t; \theta)\) is the model-predicted distribution of clean tokens \(\mathbf{x}_1\) given a partially corrupted sequence \(\mathbf{x}_t\) at time \(t\). Here, \(\delta_{\mathbf{x}_t}(x)\) represents the discrete Dirac delta centered at \(\mathbf{x}_t\). By following \(u_t\) through time, we recover a path that transforms \(p(\mathbf{x}_0)\) into \(p(\mathbf{x}_1)\).

\subsection{Training with Frame-Level Masking}
\label{subsec:training}

The flow matching formulation introduced in Sections~\ref{subsec:long_vid} and \ref{subsec:flow} employs a single scalar timestep $t$ to interpolate between the fully masked and fully unmasked video distributions. Our training procedure uses a reparametrization of this timestep. In our method, videos are generated in chunks, and only a subset of the frames (the non-context frames) are sampled from a fully masked initial state. To better simulate this process during training, we reparametrize the global timestep $t$ into a per-frame timestep vector $\mathbf{t}=(t^1,\dots, t^k)$ where each timestep $t^f$ specifies the masking ratio applied to frame $f$. In our setup, the context frames are assigned $t^f = 1$ (i.e. fully unmasked) while the new frames receive a masking level sampled from $\mathcal{U}(0,1)$.
By training the model to unmask frames with varying masking ratios per frame, we ensure that the network can effectively handle unmasked context frames while still learning a continuous transition from $p(\mathbf{x}_0)$ to $p(\mathbf{x}_1)$. To emphasize the reconstruction of masked tokens, we follow \cite{hu2024maskneed} in applying a masking operation on the cross-entropy loss. This results in the following objective:
\begin{multline}
\label{eq:ce_loss_masked}
    \mathcal{L}_{\theta} \;=\; 
    \E_{p(\mathbf{x}_1)\,p(\mathbf{x}_0)\,\mathcal{U}(\mathbf{t};0,1)\,p_{t|0,1}(\mathbf{x}_t \,\vert\, \mathbf{x}_0,\mathbf{x}_1)} \\
    \Bigl[\;\underbrace{\delta_{[M]}(\mathbf{x}_t)\,( \mathbf{x}_1 )^\top}_{\text{Loss Masking}}
    \;\log \denoise(\mathbf{x}_1 \,\vert\, \mathbf{x}_t,\mathbf{t};\theta) 
    \Bigr],
\end{multline}

where $\delta_{[M]}(\mathbf{x}_t)$ indicates that only masked tokens are used in the cross-entropy computation. The choice of frame-level masking is essential because it aligns the task of generating chunks of size $k$ conditioned on $m$ clean context frames with our training task. In both scenarios, our models are tasked with unmasking frame sequences with varying masking levels across frames. We show that compared to a constant masking level baseline, this training choice enables chunkwise autoregressive rollout to long sequence lengths. Our training algorithm is shown in detail in Algorithm~\ref{alg:training}.

\subsection{Chunkwise Autoregression for Long Videos}
\label{sec:flex_long}

 To generate a coherent video of length \(L \gg k\), we employ the chunkwise autoregressive approach as described previously. Let \(m\) be the number of context frames provided to the model (drawn initially from ground-truth, later from previous generated frames). In each iteration, we pass \(k\) frames to the model, where the first \(m\) of these frames are context and the remaining \((k - m)\) frames are fully masked. The model unmasks these frames. Afterwards, we shift the context window forward by \(s\) and repeat this process, until we have generated \(L\) total frames. Figure~\ref{fig:sampling} illustrates this pipeline. Note that we dynamically increase the number of context frames $m$ in the final chunk in case there are less than $s$ frames left to generate. In those cases we set $m = k - R$ where $R$ is the remaining number of frames, giving the final chunk a larger context. We do this to avoid generating video lengths beyond $L$ which would result in either discarding generated frames or generating videos longer than $L$. This is shown in detail in Algorithm~\ref{alg:chunkwise}.

\paragraph{Autoregressive \textit{v.s.} Full-Sequence Generation.}
By varying the stride \(s\), we can interpolate between (i) a fully autoregressive mode (\(s = 1\)) with $m = k - s$, where we generate a single new frame per chunk, and (ii) a full-sequence mode (\(s = k - m\)), where we generate $k - m$ new frames simultaneously in each chunk. Smaller \(s\) increases compute cost but may yield higher frame quality, whereas larger \(s\) is more efficient, but may result in a drop in frame quality. Our experimental results shown in Table~\ref{tab:autoregression} support this intuition.

\paragraph{FM-Style \textit{v.s.} MGM-Style Sampling.}
MaskFlow supports two distinct sampling modes. In FM-style sampling, we gradually traverse the probability path from the fully masked sequence $\mathbf{x}_0$ to the final unmasked sequence $\mathbf{x}_1$. A smaller step size yields smoother transitions at the cost of more denoising steps. Alternatively, in MGM-style sampling, we apply confidence-based heuristic sampling similar to \citet{chang2022maskgit}. In each sampling step, the model computes token-wise confidence scores for each predicted token and selects a fraction of the most confident tokens to unmask. This sampling process allows us to generate video chunks efficiently in much fewer sampling steps.
\vspace{-10pt}

\paragraph{Timestep-dependent models and timestep-independent sampling.}

By default, our model backbones are timstep-dependent, meaning each forward pass receives a timestep vector $\mathbf{t}\in[0,1]^k$ that indicates the masking ratio of each frame. Internally, we embed $\mathbf{t}$ through a learnable mapping to produce conditioning vectors that modulate various layers (e.g., via layer norm shifts/scales). Interestingly, we can still sample these models timstep-independently. Concretely, when using MGM-style sampling, we iteratively unmask a chunk of tokens while simply passing $\mathbf{t}=\mathbf{0}$ at each iteration, effectively treating our timestep-dependent model as if it were timestep-independent:

\begin{equation}
    p(\mathbf{x}_1|\mathbf{x}_t;\theta) \approx p(\mathbf{x}_1|\mathbf{x}_{t}, \mathbf{t}=\mathbf{0};\theta).
\end{equation}

This works, since the learned network can infer the corruption state (mask ratio) from the input tokens alone. Thus, in practice, \emph{a single} trained model can serve both as a standard time-dependent (flow-matching) generator \emph{and} as a time-independent (MGM-style) sampler, providing greater flexibility at inference time.

\begin{algorithm}[ht!]
\caption{Chunkwise Autoregression for Long Videos}
\label{alg:chunkwise}
\begin{algorithmic}[1]
\REQUIRE Video length \(L\), context frames \(\mathbf{x}^{1:m} = (x^1,\dots,x^m)\), chunk size \(k\), stride \(s\), fully masked frame \([M]\), network \(p(\mathbf{x}_1 \mid \mathbf{x}_t,\mathbf{t};\theta)\)
\STATE \textbf{Initialize:} \(\hat{\mathbf{x}}_{1} \leftarrow (x^1,\dots,x^m)\); \(c \leftarrow m\) \COMMENT{current frame}
\WHILE{\(c < L\)}
    \STATE \(R \leftarrow L - c\) \COMMENT{remaining frames}
    \STATE \(h \leftarrow \min(R,\, s)\) \COMMENT{frames to generate this chunk}
    \IF{\(R \le s\)}
        \STATE \(m \leftarrow k - R\)
    \ENDIF
    \STATE \(\mathbf{x}_{\mathrm{context}} \leftarrow (x^{\,c-m+1}, \dots, x^c)\)
    \STATE \(\mathbf{x}_{\mathrm{mask}} \leftarrow (\underbrace{[M], \dots, [M]}_{h\text{ times}})\)
    \STATE \(\mathbf{x}_{\mathrm{out}} \sim p\Bigl(\mathbf{x}_1 \mid (\mathbf{x}_{\mathrm{context}},\, \mathbf{x}_{\mathrm{mask}}), \mathbf{t};\theta\Bigr)\)
    \STATE \(\mathbf{x}_{\mathrm{new}} \leftarrow (x_{\mathrm{out}}^{\,m+1}, \dots, x_{\mathrm{out}}^{\,m+h})\)
    \STATE \(\hat{\mathbf{x}}_1 \leftarrow (\hat{\mathbf{x}}_1,\, \mathbf{x}_{\mathrm{new}})\)
    \STATE \(c \leftarrow c + h\)
\ENDWHILE
\RETURN \(\hat{\mathbf{x}}_1\)
\end{algorithmic}
\end{algorithm}


\section{Experiment}

\subsection{Dataset}
We construct a multilingual dataset consisting of copyrighted song lyrics in four languages: English, Chinese, French, and Korean. Lyrics represent a distinct form of copyrighted content, differing notably from other text sources such as book chapters. They exhibit rhyming and repetitive patterns, which can influence model memorization and reproduction \cite{doi:10.1021/acsenergylett.2c02758}. Moreover, song lyrics are widely shared, discussed, and searched for on social media, forums, and dedicated lyrics websites, increasing their likelihood of being incorporated into the training data of language models.
Dataset details are presented in Appendix \ref{appendixdata}.

\subsection{Language Models}
For open-source models, we evaluate Meta’s Llama-3-70B \cite{meta2024llama3}, Mistral AI’s Mistral-7B \cite{jiang2023mistral} and Mixtral-8x7B \cite{jiang2024mixtral}. For API-based models, we test OpenAI’s GPT-3.5-Turbo \cite{openai2024chatgptapis} and GPT-4o \cite{openai2024gpt4o}, Anthropic’s Claude-3.5-Haiku \cite{anthropic2024claude3}, as well as Google’s Gemini-2.0 \cite{google2024gemini2}.
For prompt, we adopt direct probing using the format of "\textit{What are the lyrics of the song [TITLE] by [SINGER]?}" and instruct the LLM to respond in the language of the song. More details can be found in Appendix \ref{appendixa}.

\subsection{Evaluation Metrics}
Following previous works \cite{karamolegkou2023copyright, liu-etal-2024-shield}, we primarily use the Longest Common Substring (LCS) and ROUGE-L scores to measure the volume of verbatim reproduction. To assess the model’s ability to decline requests for copyrighted content, we adopt the Refusal Rate. Additionally, for models with low refusal rate, we leverage GPT-4o to further assess the Hallucination Rate, which quantifies the proportion of fabricated lyric in the lyric generated, providing a more comprehensive evaluation of potential copyright infringement. Details of the metrics can be found in Appendix \ref{appendixmetric}.



%\vspace{-10pt}
\section{Conclusion}


We have presented a discrete flow matching framework for flexible long video generation, leveraging frame-level masking during training to enable flexible, efficient sampling. Our experiments demonstrate that this approach can generate high-quality videos beyond 10$\times$ the training window length, while substantially reducing sampling cost through MGM-style unmasking. Notably, our models can seamlessly switch between timestep-dependent (flow matching) and timestep-independent (MGM) sampling modes without additional training, offering a unified solution that supports both full-sequence rollout and fully autoregressive generation. We believe discrete tokens have great potential for scalable visual generation. 

\section{Acknowledgements}

Special thanks go out to Timy Phan for proof-reading and
providing helpful comments.
%
This project has been supported by the German Federal Ministry for Economic Affairs and Climate Action within the project “NXT GEN AI METHODS – Generative Methoden für Perzeption, Prädiktion und Planung”, the bidt project KLIMA-MEMES, Bayer AG, and the German Research Foundation (DFG) project 421703927. The authors gratefully acknowledge the Gauss Center for Supercomputing for providing compute through the NIC on JUWELS at JSC and the HPC resources supplied by the Erlangen National High Performance Computing Center (NHR@FAU funded by DFG).

\newpage


{
    \small
    \bibliographystyle{ieeenat_fullname}
    \bibliography{main}
}

% WARNING: do not forget to delete the supplementary pages from your submission 
% 
\clearpage
% \setcounter{page}{1}
% \maketitlesupplementary
\begin{center}
Supplementary Material
\end{center}

% {
%     \onecolumn
%     \centering
%     \Large
%     \textbf{\thetitle}\\
%     \vspace{0.5em}Supplementary Material \\
%     \vspace{1.0em}
% }

\section{Proof of \cref{theorem:dr}}
We require some additional regularity assumptions:
\begin{assumption} 1) The number of classes $C$ is bounded w.r.t the number of samples $N$, 2) the missingness mechanism $P(A=1|Y,\theta)$, as well as its estimated counterpart $P(A=1|Y,\theta)$, are bounded below by some constant $\epsilon > 0$, 3) the quantities $P(Y|X,\theta)$ and $P(A|Y,\theta)$ are estimated using auxiliary samples independent of samples used for the sample averaging.
\label{assumption:extra}
\end{assumption}
Assumptions 1 and 2 are natural. For the missingness mechanism, the ground truth being bounded means that there is a non-vanishing proportion of samples for every class. The boundedness of the estimate can be enforced by clipping the estimate. Assumption 3 is called sample splitting in \cite{kennedy-dr}.

For convenience we use operator $\E_N$ to denote the average of $N$ samples i.e. $\frac{1}{N}\sum_{i=1}^N$. Note that this is by itself a random variable, in contrast to $\E$ which is a fixed number.

\begin{proof}[Proof of \cref{theorem:dr}] Because $C$ is bounded (assumption \ref{assumption:extra}), we can fix a class $c$ and prove the theorem.
Let us define the influence function $\phi$, parameterized by $\theta$, as
\begin{equation}
\phi(O | \theta)(c) = P(Y=c|X,\theta) + \frac{\one(A=1)}{P(A=1|Y,\theta)} (\one(Y=c) - P(Y=c|X,\theta)) - P(Y=c)
\end{equation}
As we have done in the main text, we use $\phi(O)$ to denote the same function but all estimated quantities are replaced with their truths. In other words, we use $\phi(O)$ for $\phi(O|\theta_0)$ where $\theta_0$ is the truth, given that our model contains $\theta_0$ e.g. when the model is consistent.

Recall that:
\begin{equation}
\begin{aligned}
\Psi_{dr}(\theta)(c) &= \frac{1}{N}\sum_{i=1}^N \left\{P(Y=c|X,\theta) + \frac{\one(A=1)}{P(A=1|Y,\theta)} (\one(Y=c) - P(Y=c|X,\theta))\right\}\\
&= \E_N [\phi(O|\theta)(c)] + P(Y=c)
\end{aligned}
\end{equation}

We will show that:
\begin{equation}
\Psi_{dr}(\theta)(c) - P(Y=c) = (\E_N - \E)[\phi(O)(c)] + o_P(N^{-1/2})
\label{eq:proof-linearity}
\end{equation}
To do that, we use the following decomposition
\begin{equation}
\begin{aligned}
\Psi_{dr}(\theta)(c) - P(Y=c) &= \E_N [\phi(O|\theta)(c)] \\
&= (\E_N - \E)[\phi(O)(c)] + (\E_N - \E)[\phi(O|\theta)(c) - \phi(O)(c)] + \E[\phi(O|\theta)(c)]
% &+ (\E_n - \E)[\phi(O;\theta) - \phi(O)]\\
% &+ \E[P(Y=c|X,\theta)] - \E[P(Y=c|X)] + \E[\phi(O,\theta)]
\end{aligned}
\end{equation}
and analyze the second and third term. The third term is:
\begin{equation}
\begin{aligned}
\E[\phi(O|\theta)(c)] &= \E[P(Y=c|X,\theta)] + \E\left[\frac{\one(A=1)}{P(A=1|Y,\theta)}(\one(Y=c) - P(Y=c|X,\theta))\right]- P(Y=c) \\
&= \E\left[P(Y=c|X,\theta) + \frac{P(A=1|Y)}{P(A=1|Y,\theta)}(P(Y=c|X) - P(Y=c|X,\theta))\right] - \E[P(Y=c|X)]\\
&= \E\left[(P(Y=c|X,\theta) - P(Y=c|X)) (P(A=1|Y,\theta) -P(A=1|Y)) \frac{1}{P(A=1|Y,\theta)}\right]\\
\end{aligned}
\end{equation}
by Cauchy-Schwarz inequality:
\begin{equation}
\begin{aligned}
\E[\phi(O|\theta)(c)] &\le \frac{1}{\epsilon} \|P(A=1|Y,\theta) - P(A=1|Y)\|_2 \|P(Y=c|X,\theta) - P(Y=c|X)\|_{L_2(P)}\\
&= \frac{1}{\epsilon} o_P(N^{-1/4} N^{-1/4}) = o_P(N^{-1/2})
\end{aligned}
\end{equation}
by assumption \ref{assumption:4th-root-n} and that $P(A=1|Y,\theta) > \epsilon$ (assumption \ref{assumption:extra}). The second term can be bounded by Chebyshev inequality
% \begin{equation}
% \begin{aligned}
% \E[\E_N[\phi(O|\theta)(c) - \phi(O)(c)]] &= \E[\phi(O|\theta)(c) - \phi(O)(c)]\\
% \var[\E_N[\phi(O|\theta)(c) - \phi(O)(c)]] &= \frac{1}{N}\var[\phi(O|\theta)(c) - \phi(O)(c)] \le 
% \end{aligned}
% \end{equation}
\begin{equation}
P(|(\E_N - \E)[\phi(O|\theta)(c) - \phi(O)(c)]| \ge t) \le \frac{\var[\E_N[\phi(O|\theta)(c) - \phi(O)(c)]]}{t^2} = \frac{\var[\phi(O|\theta)(c) - \phi(O)(c)]}{Nt^2}
\end{equation}
note here that $\theta$ is independent of the samples used for $\E_N$ by assumption \ref{assumption:extra}. For any $\varepsilon > 0$, by picking $t = \frac{1}{\sqrt{N\varepsilon}}$ we get
\begin{equation}
P\left(\left|\frac{(\E_N - \E)[\phi(O|\theta)(c) - \phi(O)(c)]}{N^{-1/2}}\right| \ge \frac{1}{\sqrt{\varepsilon}}\right) \le \varepsilon \var[\phi(O|\theta)(c) - \phi(O)(c)]
\end{equation}
by the definition of $O_P$, we then get
\begin{equation}
(\E_N - \E)[\phi(O|\theta)(c) - \phi(O)(c)] = O_P(N^{-1/2}\var[\phi(O|\theta)(c) - \phi(O)(c)])
\end{equation}
Because $\phi$ is a continuous function of $P(Y|X,\theta)$ and $P(A|Y,\theta)$ (given $P(A|Y,\theta) > \epsilon$, assumption \ref{assumption:extra}), by the continuous mapping theorem and the fact that $P(Y|X,\theta)$ and $P(A|Y,\theta)$ are convergent in probability (assumption \ref{assumption:4th-root-n}), we get $\var[\phi(O|\theta)(c) - \phi(O)(c)] = o_P(1)$. This gives
\begin{equation}
(\E_N - \E)[\phi(O|\theta)(c) - \phi(O)(c)] = o_P(N^{-1/2})
\end{equation}
Therefore, we have shown that the second and third term are both $o_P(N^{-1/2})$, proving \cref{eq:proof-linearity}. As the final step, multiply both sides of this equation by $\sqrt{N}$ we get:
\begin{equation}
\sqrt{N}(\Psi_{dr}(\theta)(c) - P(Y=c)) = \sqrt{N} (\E_N - \E)[\phi(O)(c)] + o_P(1) \rightsquigarrow \mathcal{N}(0, \var[\phi(O)(c)])
\end{equation}
by the central limit theorem, and $\var[\phi(O)(c)] = \E[\phi(O)(c)^2]$ because $\E[\phi(O)(c)] = 0$.
\end{proof}

While we started with the definition of $\phi$, \cref{eq:proof-linearity} shows that $\phi$ is indeed an influence function. Now we show that $\phi$ is also the efficient influence function, by using the characterization of the model's tangent space \cite{tsiatis-missingdata}. Note that the joint probability factorizes as $P(X,A,Y) = P(X)P(Y|X)P(A|Y)$, therefore the tangent space $\mathcal{T}$ factorizes as $\mathcal{T} = \mathcal{T}_{X} \oplus \mathcal{T}_{Y|X} \oplus \mathcal{T}_{A|Y}$ where $\mathcal{T}_X = \{h(X): \E[h] = 0\}$, $\mathcal{T}_{Y|X} = \{h(X,Y): \E[h|X] = 0\}$, $\mathcal{T}_{A|Y} = \{h(A,Y): \E[h|Y] = 0\}$, and the 3 subspaces are pairwise orthogonal. All influence functions are orthogonal to the tangent space, but the influence function that is also in the tangent space has the smallest variance and is called the efficient influence function. As $\phi$ is already an influence function, we need only show that $\phi$ is in $\mathcal{T}$. We write $\phi$ as
\begin{equation}
\phi(O)(c) = (P(Y=c|X) - P(Y=c)) + \left[\frac{\one(A=1)}{P(A=1|Y)} - 1\right](\one(Y=c) - P(Y=c|X)) + (\one(Y=c) - P(Y=c|X))
\end{equation}
and note that the first, second and third term are in $\mathcal{T}_X$, $\mathcal{T}_{A|Y}$ and $\mathcal{T}_{Y|X}$ respectively. Therefore, $\phi$ is indeed in $\mathcal{T}$. The efficient influence function has the smallest variance of all influence function, and therefore our estimator being asymptotically linear in $\phi$ (\cref{eq:proof-linearity}) has the smallest mean squared error in a local asymptotic minimax sense \cite{kennedy-dr, asymptoticstatistics}

\section{Further background and related work}
\paragraph{Discussion on semi-supervised EM.}
It appears that semi-supervised EM was first used for parameter estimation when the missingness mechanism is non-ignorable in \cite{ibrahim1996parameter}, but has not been used for label shift estimation.
Perhaps this is because the semi-supervised situation where additional unlabeled data is available during training is rarer than the test-time adaptation case. EM is well suited to take advantage of the extra unlabeled data to improve the classifier under very scarce and long-tailed labeled data. While the connection between pseudo-labeling and EM has been explored before \cite{entropyminimization}, the situation with label shift has not until recently \cite{simpro}. Here the application of EM is much more interesting, because other than simply giving pseudo-labeling a rigorous formulation, EM also estimates the missingness mechanism (equivalently the label distribution shift), which is important for shift correction and thus high-quality pseudo-labels \cite{acr}. The application of confidence thresholding can be seen as a sparse variant of EM \cite{neal1998view}.

\paragraph{The doubly-robust risk.} 
\label{subsec:dr-risk}
A technique that also derives from the theory of semi-parametric efficiency is orthogonal statistical learning \citep{foster2023orthogonal}. The idea is to minimize the doubly-robust risk:
\label{subsec:method-dr-risk}
\begin{equation}
\label{eq:dr-risk}
\mathcal{R}(\theta_2) = \frac{1}{N} \sum_{i=1}^N \Bigg[ l(x_i, \hat y_i|\theta_2) + \frac{\one(a_i=1)}{P(A=a_i|Y=y_i, \theta_1)} (l(x_i, y_i | \theta_2) - l(x_i, \hat y_i | \theta_2))\Bigg]
\end{equation}
where $l(x,y|\theta) = -\sum_{c=1}^C [y]_c \log P(Y=c|X=x,\theta)$ is the negative cross-entropy. 
The notation $[y]_c$ means that we are using the $c$-entry in a C-dimension probability vector $y$. 
Thus, $y_i$ denotes the one-hot label of observation $i$, while $\hat y_i$ denotes the pseudo-label, which can be one-hot or all-zero. 
Finally, we use $\theta_1$ to denote that $P(a|y,\theta_1)$ is an estimation from a previous stage, but it can be estimated with $\theta_2$ as well. 
The risk $\mathcal{R}(\theta_2)$ can be used as a training loss in a straightforward fashion. 
Similar to the doubly robust estimation of $P(Y)$, the doubly robust risk provides approximately unbiased estimation of the risk. 
This property has been used in \citep{arelabelsinformative, onnonrandommissinglabels, drst} also in the semi-supervised learning setting.
More broadly, it is at the heart of one of the core techniques in heterogenous treatment effect estimation in causal estimation \cite{kennedy2023towards, foster2023orthogonal, wager2018estimation}. 
The focus here is not the estimation of $\mathcal{R}(\theta_2)$ per se, but the quality of the learned model \cite{foster2023orthogonal}.
By using the doubly-robust risk, we can achieve an optimality result similar in spirit to our theorem \cref{theorem:dr}, but for the generalization error.
While this is appealing, in practice there are 2 problems with this approach. First, the inverse probability weight $P(A=a_i|Y=y_i,\theta_1)$ can be very large if the class ratio is highly unlabeled, making training unstable \cite{kallus2020deepmatch, pham2023stable}. 
This problem exists for our estimation as well. However, it is much easier to control for estimation than for training because of the iterative nature of model update. Secondly, we can further write $\mathcal{R}$ as:
\begin{equation}
\mathcal{R}(\theta_2) = \frac{1}{N}\sum_{i=1}^N l\left(x_i, \hat y_i + \frac{\one(a_i=1)}{P(A=a_i|Y=y_i,\theta_1)} (y_i - \hat y_i)\Bigg\vert\theta_2\right)
\end{equation}
which is a cross-entropy loss with new meta-pseudo-labels. However, these labels are not meant to be learned exactly, and furthermore they can be negative. Thus, theoretical works have to put stringent assumptions on the models. In \cref{subsec:ablation-1}, we show that experimentally that the instability problem makes doubly-robust risk performance worse than our 2-stage approach.

\section{Training and hyperparameter settings.}
\label{subsec:training-setting}
For neural network training, we follow the implementation and hyperparameter settings of \cite{simpro}. In particular, we adapt the core code of SimPro for Supervised, MLE and EM. For MLE, we update $P(A|Y)$ using the Adam optimizer with learning rate 1e-3, while for EM we use a momentum update similar to SimPro's update of $P(Y|A)$ because it has a a closed-form solution at each mini-batch. We use Wide ResNet-28-2 on all methods and all datasets in this section, including Imagenet-127, because we are motivated by the fact that stage-1's goal is not classification accuracy but the estimation of a finite-dimensional parameter. When using Wide ResNet-28-2 for Imagenet-127, we use the hyperparameters of CIFAR-100, except we lower the batch size of unlabeled data to 2 times that of labeled data instead of 8 for memory reason. We do not perform additional hyperparameter tuning. All experiments can be performed on 1 A6000 RTX GPU, and are run 3 times. We report the total variation distance between the estimated and the ground truth unlabeled class distribution, similar to its usage in Theorem 3.1 of \cite{lsc}, and the top-1 classification accuracy.

In the second stage of our algorithm, we freeze our estimation and plug it in SimPro and BOAT.
We keep exactly the same hyperparameter settings that SimPro and BOAT use. In particular, for Imagenet-127, we now use ResNet-50 and run each experiment once.
In SimPro, we set the unlabeled class distribution $P(Y|A=0)$ at the E-step;  however, we still keep a running estimate of the class distribution $P(Y)$ in the logit adjustment loss \cref{eq:simpro-la-loss}. While it is possible to use the first stage estimate in the logit adjustment loss, we observe that doing so results in lower accuracy than using the the running average. This is conceptually consistent with the role of the running average - serving not as an accurate estimate of $P(Y)$ but to make the classifier's class distribution uniform through the logit adjustment loss, which is good for the test set. Similarly, in BOAT, we only replace $\Delta_c = \log P(Y|A=1) - \log P(Y|A=0)$ in equation (4) of \cite{boat}, which is adjusting a classifier's predictions from the labeled to the unlabeled class distribution, with our SimPro + DR estimate instead of their on-the-fly estimate. 


% \section{Additional experiments}
% % \begin{table*}[t]
\centering
\caption{Total Variation Distance on CIFAR-10-LT ($N_l = 500$, $M_l = 4000$) with different class imbalance ratios $\gamma_l$ and $\gamma_u$ under five different unlabeled class distributions.}
\label{tab:cifar10-tv}
\resizebox{\textwidth}{!}{
\begin{tabular}{lccccccccccc}
\toprule
& & \multicolumn{2}{c}{consistent} & \multicolumn{2}{c}{uniform} & \multicolumn{2}{c}{reversed} & \multicolumn{2}{c}{middle} & \multicolumn{2}{c}{head-tail} \\
\cmidrule(lr){3-4} \cmidrule(lr){5-6} \cmidrule(lr){7-8} \cmidrule(lr){9-10} \cmidrule(lr){11-12}
& & $\gamma_l = 150$ & $\gamma_l = 100$ & $\gamma_l = 150$ & $\gamma_l = 100$ & $\gamma_l = 150$ & $\gamma_l = 100$ & $\gamma_l = 150$ & $\gamma_l = 100$ & $\gamma_l = 150$ & $\gamma_l = 100$ \\
Model & Estimator & $\gamma_u = 150$ & $\gamma_u = 100$ & $\gamma_u = 1$ & $\gamma_u = 1$ & $\gamma_u = 1/150$ & $\gamma_u = 1/100$ & $\gamma_u = 150$ & $\gamma_u = 100$ & $\gamma_u = 150$ & $\gamma_u = 100$ \\
\midrule
Supervised & MLLS & 0.269 ± 0.252 & 0.038 ± 0.006 & 0.251 ± 0.046 & 0.255 ± 0.060 & 0.429 ± 0.028 & 0.493 ± 0.050 & 0.333 ± 0.042 & 0.320 ± 0.009 & 0.457 ± 0.034 & 0.444 ± 0.043 \\
Supervised & RLLS & 0.043 ± 0.001 & 0.044 ± 0.010 & 0.348 ± 0.034 & 0.305 ± 0.068 & 0.769 ± 0.016 & 0.678 ± 0.028 & 0.430 ± 0.008 & 0.368 ± 0.013 & 0.539 ± 0.018 & 0.503 ± 0.020 \\
\midrule
MLE & IPW & 0.027 ± 0.001 & 0.027 ± 0.000 & 0.319 ± 0.072 & 0.243 ± 0.010 & 0.674 ± 0.020 & 0.646 ± 0.041 & 0.438 ± 0.020 & 0.454 ± 0.026 & 0.547 ± 0.049 & 0.491 ± 0.059 \\
MLE & OR & 0.045 ± 0.004 & 0.042 ± 0.000 & 0.215 ± 0.026 & 0.203 ± 0.032 & 0.433 ± 0.017 & 0.395 ± 0.033 & 0.193 ± 0.006 & 0.209 ± 0.037 & 0.307 ± 0.147 & 0.249 ± 0.130 \\
MLE & DR & 0.090 ± 0.002 & 0.079 ± 0.000 & 0.407 ± 0.027 & 0.360 ± 0.007 & 0.425 ± 0.007 & 0.421 ± 0.029 & 0.256 ± 0.001 & 0.286 ± 0.031 & 0.435 ± 0.136 & 0.362 ± 0.122 \\
\midrule
EM & IPW & 0.035 ± 0.002 & 0.040 ± 0.001 & 0.021 ± 0.001 & 0.029 ± 0.015 & 0.303 ± 0.187 & 0.091 ± 0.010 & 0.119 ± 0.011 & 0.105 ± 0.022 & 0.104 ± 0.026 & 0.104 ± 0.051 \\
EM & OR & 0.037 ± 0.003 & 0.042 ± 0.002 & 0.016 ± 0.001 & 0.024 ± 0.012 & 0.269 ± 0.183 & 0.090 ± 0.008 & 0.122 ± 0.012 & 0.103 ± 0.022 & 0.072 ± 0.012 & 0.073 ± 0.024 \\
EM & DR & 0.034 ± 0.004 & 0.037 ± 0.001 & 0.014 ± 0.001 & 0.027 ± 0.020 & 0.264 ± 0.191 & 0.092 ± 0.005 & 0.111 ± 0.019 & 0.097 ± 0.026 & 0.077 ± 0.016 & 0.073 ± 0.028 \\
\midrule
SimPro & IPW & 0.070 ± 0.011 & 0.058 ± 0.000 & 0.046 ± 0.001 & 0.049 ± 0.005 & 0.254 ± 0.074 & 0.223 ± 0.098 & 0.097 ± 0.025 & 0.067 ± 0.002 & 0.105 ± 0.066 & 0.110 ± 0.079 \\
SimPro & OR & 0.071 ± 0.012 & 0.058 ± 0.000 & 0.045 ± 0.001 & 0.049 ± 0.006 & 0.040 ± 0.003 & 0.059 ± 0.017 & 0.074 ± 0.006 & 0.075 ± 0.002 & 0.033 ± 0.003 & 0.033 ± 0.003 \\
SimPro & DR & 0.017 ± 0.004 & 0.026 ± 0.001 & 0.019 ± 0.002 & 0.018 ± 0.003 & 0.039 ± 0.003 & 0.058 ± 0.025 & 0.091 ± 0.007 & 0.031 ± 0.001 & 0.015 ± 0.003 & 0.019 ± 0.007 \\
\bottomrule
\end{tabular}
}
\end{table*}
% 

\begin{table*}[t]
\centering
\caption{Total Variation Distance on CIFAR-100-LT ($N_l = 50$, $M_l = 400$) with different class imbalance ratios $\gamma_l$ and $\gamma_u$ under five different unlabeled class distributions.}
\label{tab:cifar100-tv}
\resizebox{\textwidth}{!}{
\begin{tabular}{lccccccccccc}
\toprule
& & \multicolumn{2}{c}{consistent} & \multicolumn{2}{c}{uniform} & \multicolumn{2}{c}{reversed} & \multicolumn{2}{c}{middle} & \multicolumn{2}{c}{head-tail} \\
\cmidrule(lr){3-4} \cmidrule(lr){5-6} \cmidrule(lr){7-8} \cmidrule(lr){9-10} \cmidrule(lr){11-12}
& & $\gamma_l = 20$ & $\gamma_l = 10$ & $\gamma_l = 20$ & $\gamma_l = 10$ & $\gamma_l = 20$ & $\gamma_l = 10$ & $\gamma_l = 20$ & $\gamma_l = 10$ & $\gamma_l = 20$ & $\gamma_l = 10$ \\
Model & Estimator & $\gamma_u = 20$ & $\gamma_u = 10$ & $\gamma_u = 1$ & $\gamma_u = 1$ & $\gamma_u = 1/20$ & $\gamma_u = 1/10$ & $\gamma_u = 20$ & $\gamma_u = 10$ & $\gamma_u = 20$ & $\gamma_u = 10$ \\
\midrule
Supervised & MLLS & 0.707 ± 0.016 & 0.313 ± 0.100 & 0.445 ± 0.172 & 0.309 ± 0.119 & 0.383 ± 0.075 & 0.397 ± 0.006 & 0.570 ± 0.001 & 0.373 ± 0.107 & 0.543 ± 0.009 & 0.231 ± 0.057 \\
Supervised & RLLS & 0.520 ± 0.007 & 0.133 ± 0.003 & 0.337 ± 0.125 & 0.253 ± 0.082 & 0.424 ± 0.060 & 0.463 ± 0.003 & 0.454 ± 0.021 & 0.306 ± 0.074 & 0.460 ± 0.028 & 0.241 ± 0.040 \\
\midrule
MLE & IPW & 0.075 ± 0.000 & 0.071 ± 0.001 & 0.229 ± 0.001 & 0.167 ± 0.002 & 0.565 ± 0.005 & 0.443 ± 0.007 & 0.415 ± 0.000 & 0.311 ± 0.005 & 0.343 ± 0.000 & 0.280 ± 0.001 \\
MLE & OR & 0.065 ± 0.002 & 0.061 ± 0.001 & 0.200 ± 0.007 & 0.143 ± 0.001 & 0.526 ± 0.011 & 0.399 ± 0.023 & 0.360 ± 0.003 & 0.256 ± 0.012 & 0.328 ± 0.003 & 0.266 ± 0.005 \\
MLE & DR & 0.149 ± 0.019 & 0.145 ± 0.010 & 0.243 ± 0.004 & 0.214 ± 0.019 & 0.568 ± 0.005 & 0.464 ± 0.014 & 0.403 ± 0.014 & 0.309 ± 0.012 & 0.365 ± 0.007 & 0.320 ± 0.004 \\
\midrule
EM & IPW & 0.097 ± 0.008 & 0.092 ± 0.004 & 0.239 ± 0.007 & 0.179 ± 0.003 & 0.478 ± 0.012 & 0.329 ± 0.020 & 0.262 ± 0.016 & 0.202 ± 0.003 & 0.312 ± 0.002 & 0.227 ± 0.001 \\
EM & OR & 0.121 ± 0.007 & 0.108 ± 0.005 & 0.261 ± 0.007 & 0.189 ± 0.004 & 0.489 ± 0.013 & 0.335 ± 0.020 & 0.274 ± 0.016 & 0.211 ± 0.004 & 0.336 ± 0.003 & 0.235 ± 0.001 \\
EM & DR & 0.125 ± 0.005 & 0.111 ± 0.004 & 0.269 ± 0.007 & 0.194 ± 0.005 & 0.497 ± 0.010 & 0.336 ± 0.024 & 0.281 ± 0.019 & 0.219 ± 0.008 & 0.336 ± 0.007 & 0.233 ± 0.004 \\
\midrule
SimPro & IPW & 0.125 ± 0.001 & 0.100 ± 0.005 & 0.166 ± 0.007 & 0.141 ± 0.009 & 0.353 ± 0.023 & 0.261 ± 0.008 & 0.202 ± 0.003 & 0.158 ± 0.005 & 0.277 ± 0.009 & 0.197 ± 0.003 \\
SimPro & OR & 0.133 ± 0.005 & 0.100 ± 0.004 & 0.160 ± 0.007 & 0.138 ± 0.010 & 0.322 ± 0.014 & 0.253 ± 0.008 & 0.202 ± 0.003 & 0.156 ± 0.005 & 0.269 ± 0.006 & 0.191 ± 0.004 \\
SimPro & DR & 0.122 ± 0.003 & 0.106 ± 0.006 & 0.188 ± 0.009 & 0.149 ± 0.006 & 0.343 ± 0.023 & 0.257 ± 0.007 & 0.219 ± 0.010 & 0.172 ± 0.002 & 0.279 ± 0.007 & 0.198 ± 0.004 \\
\bottomrule
\end{tabular}
}
\end{table*}

\clearpage
\setcounter{page}{1}
\maketitlesupplementary


\renewcommand{\thetable}{S\arabic{table}}
\renewcommand{\thefigure}{S\arabic{figure}}

\clearpage
\appendix

\onecolumn 

\section{Appendix}
\tableofcontents 
\clearpage


\begin{table*}[t]
\centering
\small
\renewcommand{\arraystretch}{1.2}
\setlength{\tabcolsep}{6pt}
\begin{tabular*}{\textwidth}{@{\extracolsep{\fill}} 
  >{\centering\arraybackslash}m{3cm}  % Sampling Mode
  >{\centering\arraybackslash}m{1.5cm} % Stride
  >{\centering\arraybackslash}m{3cm}   % Extrapolation Factor
  >{\centering\arraybackslash}m{2cm}   % Total NFE
  >{\centering\arraybackslash}m{3cm}   % Sampling Time [s]
  >{\centering\arraybackslash}m{1cm}   % FVD subcolumn 1
  >{\centering\arraybackslash}m{1cm}}  % FVD subcolumn 2
\toprule
\makecell{\textbf{Sampling}\\\textbf{Mode}} & 
\makecell{\textbf{Stride}} & 
\makecell{\textbf{Extrapolation}\\\textbf{Factor}} & 
\makecell{\textbf{Total}\\\textbf{NFE}} & 
\makecell{\textbf{Sampling}\\\textbf{Time [s]}} & 
\multicolumn{2}{c}{\textbf{FVD$\downarrow$}} \\
\cmidrule(rr){6-7}
 & & & & & \textbf{DMLab} & \textbf{FFS} \\
 \midrule
\rowcolor{gray!8}Diffusion Forcing~\cite{chen2024diffusionforcing} & $s=k-m$ & $1\times$ & $286 / 266$ & $45.32$ / $52.26$ & $60.30$ & $51.90$ \\
Rolling Diffusion~\cite{ruhe2024rollingdiffusionmodels} & $s=k-m$ & $1\times$ & $500$ / $500$ & $79.24$ / $98.23$ & \textbf{52.43} & \textbf{45.51} \\
\rowcolor{gray!8}\textit{MaskFlow} (MGM-Style) & $s=k-m$ & $1\times$ & \textbf{20} / \textbf{20} & \textbf{3.17 / 3.93} & $53.17$ & $45.92$ \\
\midrule
Diffusion Forcing~\cite{chen2024diffusionforcing} & $s=k-m$ & $2\times$ & $858$ / $798$ & $135.97$ / $156.78$ & $175.01$ & $144.43$ \\
\rowcolor{gray!8}Rolling Diffusion~\cite{ruhe2024rollingdiffusionmodels} & $s=k-m$ & $2\times$ & $896$ / $788$ & $141.99 / 154.81$ & 201.70 & 72.49 \\
\textit{MaskFlow} (MGM-Style) & $s=k-m$ & $2\times$ & \textbf{60} / \textbf{60} & \textbf{9.51} / \textbf{9.30} & $188.02$ &  $59.93$ \\
\rowcolor{gray!8}\textit{MaskFlow} (MGM-Style) & $s=1$ & $2\times$ & $740$ / $340$ & 117.27 / 66.80 & \textbf{50.87} & \textbf{30.43} \\
\midrule
Diffusion Forcing~\cite{chen2024diffusionforcing} & $s=k-m$ & $5\times$ & $2{,}002$ / $1{,}596$ & $317.27$ / $313.56$ & $232.89$ & $272.14$ \\
\rowcolor{gray!8}Rolling Diffusion~\cite{ruhe2024rollingdiffusionmodels} & $s=k-m$ & $5\times$ & $2{,}084$ / $1{,}652$ & $330.27$ / $324.56$ & $338.34$ & $248.13$ \\
\textit{MaskFlow} (MGM-Style) & $s=k-m$ & $5\times$ & \textbf{140} / \textbf{120} & \textbf{22.19 / 23.58} & $334.15$ & $108.74$ \\
\rowcolor{gray!8}\textit{MaskFlow} (MGM-Style) & $s=1$ & $5\times$ & $2{,}900$ / $1{,}300$ & 100.09/379.91 & \textbf{181.11} & \textbf{103.69} \\
\bottomrule
\end{tabular*}
\caption{\textbf{MGM Style sampling is much faster without sacrificing quality.} We report the total number of function evaluations (NFE), sampling time (in seconds), and FVD for various sampling methods and extrapolation factors across both datasets.}
\label{tab:speed_comparison}
\end{table*}



\subsection{Additional Related Work}

\paragraph{Masked Diffusion Models.} 
Limitations of autoregressive models for probabilistic language modeling have recently sparked increasing interest in masked diffusion models. Recent works like \cite{shi2024simplifiedgeneralizedmaskeddiffusion} and \cite{sahoo2024simpleeffectivemaskeddiffusion} have aligned masked generative models with the design space of diffusion models by formulating continuous-time forward and sampling processes. Works like \cite{nie2024scalingmaskeddiffusionmodels} and \cite{gong2024scalingdiffusionlanguagemodels} also demonstrate the significant scaling potential of MDM for language tasks, indicating that this masked modeling paradigm can rival autoregressive approaches for modalities beyond language such as protein co-design \cite{campbell2024generative} and vision.

\subsection{Computation of NFE for Different Sampling Methods}

Our sampling speed evaluations are determined by computing the required number of chunks 
\[
\ell = \left\lceil \frac{L - k}{s} \right\rceil + 1,
\]
to generate a video of total length \(L\), where \(k\) is the chunk size and \(s\) is the stride with which the chunk start is shifted. The overall number of function evaluations (NFEs) is then obtained by multiplying \(\ell\) with the number of sampling steps required to generate one chunk. We apply this methodology for all chunkwise-autoregressive approaches.

\begin{itemize}
    \item \textbf{MGM-Style Sampling:} In this method each chunk is generated in $20$ forward passes, so that the total NFE is
    \[
    \text{NFE}_{\mathrm{MGM}} = \ell \times 20.
    \]

    \item \textbf{FM-Style Sampling:} Here we generate each chunk in $250$ forward passes:
    \[
    \text{NFE}_{\mathrm{FM}} = \ell \times 250.
    \]
    
    \item \textbf{Diffusion Forcing with Pyramid Scheduling:} Here, we apply $250$ sampling timesteps per frame but begin unmasking earlier frames as the denoising process proceeds. For a chunk of \(k\) frames, we generate a scheduling matrix with 
    \[
    H = 250 + (k-1) + 1 = k + 250
    \]
    rows and \(k\) columns. Each entry in the scheduling matrix is computed as
    \[
    \text{scheduling\_matrix}[i,j] = 250 + j - i,\quad \text{for } i=0,\ldots,H-1 \text{ and } j=0,\ldots,k-1,
    \]
    and then clipped to the interval \([0,249]\). Since we iterate through each of the $H$ rows of the denoising matrix in each chunk we effectively compute
    \[
    \text{NFE}_{\text{DiffusionForcing}} = k + 250.
    \] 
    
    \item \textbf{RDM Sampling:} This approach proceeds in three stages:
    \begin{enumerate}
        \item \textit{Initialization (Init-Schedule):} The initial window of \(k\) frames is processed using a fixed schedule that applies $T=250$ forward passes to bring the window to its rolling state.
        
        \item \textit{Sliding Window Handling:} After initialization, the window is shifted by one frame at a time. For each shift, an inner loop is executed that updates the denoising levels until the first non-context frame (i.e., the frame immediately following the \(m\) context frames) is fully denoised (i.e., reaches a value of 1). This inner loop requires $\left\lceil \frac{T}{k-m} \right\rceil$ forward passes per window shift. As the window is shifted \((L - k)\) times, this stage contributes roughly \((L - k) \times \left\lceil \frac{T}{k-m} \right\rceil\) forward passes.
        
        \item \textit{Final Window Processing:} Once the sliding window stage is complete, the final (partial) window is further refined until all frames are fully denoised. This final stage requires additional $250$ forward passes.
    \end{enumerate}
    
    Thus, the total NFE for RDM is given by
    \[
    \text{NFE}_{\mathrm{Rolling}} = 250 \; (\text{init-schedule}) + (L - k) \times \left\lceil \frac{T}{k-m} \right\rceil\ \; (\text{sliding}) + 250 \; (\text{final window}).
    \]
\end{itemize}




\subsection{Training \& Implementation Details}

All FFS models were trained on 4 H100 GPUs with a local batch size of $4$. We run training for a total of $200{,}000$ steps and use a sigmoid scheduler that determines the per-frame masking ratio for a sampled masking level $t^k$. We use an AdamW optimizer with a learning rate of $1e-4$ and $\beta_1 = 0.9$ and $\beta_2 = 0.999$. We additionally incorporate a frame-level loss weighting mechanism based that is also based on \(t^k\). We adopt \emph{fused}-SNR loss weighting from \cite{hang2023efficient,chen2024diffusionforcing} and derive it for discrete flow matching. Let

\[
\text{SNR}(t) \;=\; \frac{\kappa(t)^2}{\,1 - \kappa(t)^2\,},
\]

where \(\kappa(t)\) is the masking schedule. The \emph{fused}-SNR mechanism smoothes SNR values across time steps in a video by computing an exponentially decaying SNR from previous frames (or tokens). We refer the reader to~\cite{chen2024diffusionforcing} for full details.


\begin{algorithm}[!ht]
\caption{\textbf{FM-Style Sampling with Context Frames for a Single Chunk}}
\label{alg:fmsampling}
\begin{algorithmic}[1]
\REQUIRE 
   $p(\mathbf{x}_1 | \mathbf{x}_t, \mathbf{t};\theta)$, 
   $t$, 
   context frames $\mathbf{c} = (c^1,\dots,c^m)$, 
   fully masked frame \([M]\) (i.e., a frame where every token equals the mask token \(M\)),
   $t \in [0,1]$, 
   $\Delta t$

\STATE $\mathbf{x}_t \,\gets\, (\,c^1,\dots,c^m,\,[M],\dots,[M])$
\STATE $t \,\gets\, 0$
\STATE $\mathbf{t} \gets (1,\dots,1,0,\dots0)$

\WHILE{$t \,\le\, 1 - \Delta t$}
    \STATE $u_t(\mathbf{x}_t) 
        \;=\; 
        \frac{t}{1-t}
        \Bigl[
          p_\theta(\mathbf{x}_1 \mid \mathbf{x}_t,\,\mathbf{t}) 
          \;-\; 
          \delta_{\mathbf{x}_t}
        \Bigr]$
    \STATE $p_\theta\!\bigl(\mathbf{x}_1 \mid \mathbf{x}_{t+\Delta t},\,\mathbf{t}+\Delta t\bigr)
        \;=\;
        \mathrm{Cat}\!\Bigl[\,
          \delta_{\mathbf{x}_t}
          \;+\; 
          u_t(\mathbf{x}_t)\,\Delta t
        \Bigr]$
    \STATE \textbf{For each token} $n$ in $\mathbf{x}_t$: 
    \STATE \quad 
    $
       x_{t+\Delta t}^{n} \gets
       \begin{cases}
          x_t^{n}, & \text{if } x_t^{n} \neq M,\\
          p(\cdot | \mathbf{x}_{t+\Delta t},\,\mathbf{t}+\Delta t; \theta), & \text{if } x_t^{n} = M.
       \end{cases}
    $
\STATE $t \gets t + \Delta t$
\STATE $\mathbf{t} \gets \mathbf{t} + \Delta t$

\ENDWHILE
\STATE \textbf{return} $\mathbf{x}_t$
\end{algorithmic}
\end{algorithm}

\begin{algorithm}[ht]
\caption{\textbf{MGM-Style Sampling for a Single Chunk}}
\label{alg:mgm_chunk_unmasking_revised}
\begin{algorithmic}[1]
\REQUIRE 
  Network $p(\mathbf{x}_1 \mid \mathbf{x}_t, \mathbf{t}; \theta)$,  
  context frames $\mathbf{c} = (c^1,\dots,c^m)$,  
  masked frame $[M]$ (i.e., every token equals $M$),    
  total unmasking steps $T$
\STATE \textbf{Initialize:}\\
$\mathbf{x}_t \;\leftarrow\; (\mathbf{c},\, [M],\dots,[M])$\\
$\mathbf{t} \;\leftarrow\; (\underbrace{1,\dots,1}_{m},\, \underbrace{0,\dots,0}_{k-m})$
\STATE Define the set of masked token indices in $\mathbf{x}_t$:\\
$\mathcal{M} \;\triangleq\; \{\, n \mid x_t^n = M \,\}.$
\FOR{$i=1$ \textbf{to} $T$}
    \STATE Compute token-wise logits:\\
    $\boldsymbol{\lambda} \;\leftarrow\; p(\mathbf{x}_1 \mid \mathbf{x}_t, \mathbf{t}; \theta).$
    \STATE \textbf{For each token} $n \in \mathcal{M}$: \\
    sample $\hat{x}_t^n \sim \mathrm{Cat}\Bigl(\mathrm{Softmax}\bigl(\boldsymbol{\lambda}^n\bigr)\Bigr)$ \\
    and compute the confidence score 
    $C_n \;=\; \mathrm{Softmax}\bigl(\boldsymbol{\lambda}^n\bigr)_{\hat{x}_t^n}.$ \\
    \STATE \textbf{Define the confidence threshold:}\\
    Let $\alpha$ denote the desired fraction of masked tokens to update in each iteration (e.g. $\alpha = 1/T$). \\
    
    Then set 
    $\tau_c \;=\; \min\Bigl\{ c \in [0,1] \;\Bigm|\; \Bigl|\{ j \in \mathcal{M} \mid C_j \ge c \}\Bigr| \ge \Bigl\lceil \alpha\,|\mathcal{M}| \Bigr\rceil \Bigr\}.$ \\
    
    (That is, $\tau_c$ is chosen as the minimum confidence such that at least $\lceil \alpha\,|\mathcal{M}| \rceil$ tokens have confidence scores at or above $\tau_c$, thereby selecting the top $\lceil \alpha\,|\mathcal{M}| \rceil$ tokens.)
    \STATE \textbf{For each token} $n \in \mathcal{M}$ with $C_n \ge \tau_c$, update:\\
    $x_t^n \;\leftarrow\; \hat{x}_t^n.$
    \STATE Update the set of masked indices:\\
    $\mathcal{M} \;\leftarrow\; \{\, n \mid x_t^n = M \,\}.$
    \IF{$\mathcal{M} = \varnothing$}
         \STATE \textbf{break}
    \ENDIF
\ENDFOR
\STATE \textbf{return} $\mathbf{x}_t$.
\end{algorithmic}
\end{algorithm}


\subsection{Baseline Details}

The two most comparable works to our method are \citet{chen2024diffusionforcing} and \citet{ruhe2024rollingdiffusionmodels}. Both of these techniques propose novel sampling methods that can be rolled out to long video lengths, and also apply frame-specific noise levels. Both of these approaches are diffusion-based and operate on continuous representations, whereas we operate on discrete tokens and use masking. We re-implement both the pyramid sampling scheme proposed in Diffusion Forcing and the Rolling Diffusion sampling method in our discrete setting. This allows us to compare the baseline sampling methods to MaskFlow on the same model backbones. 
%
To isolate the effect of our chunkwise autoregressive sampling methodology on performance from the effects of tokenization, we reimplement both the pyramid sampling scheme proposed in Diffusion Forcing and the Rolling Diffusion sampling method for our discrete setting. This allows us to compare the baseline sampling methods on the same timestep-dependent model backbone. 
%
Although it is conceivable that Rolling Diffusion sampling may perform better when applied to a model explicitly trained using the progressive noise schedule suggested in \citet{ruhe2024rollingdiffusionmodels}, we believe this comparison is still fair. Our training methodology does not inject any inductive bias by way of the masking level into the model, so there is no obvious advantage that our sampling should have over other methods. 
We provide a comprehensive evaluation of performance and sampling efficiency across both datasets and different sampling modes.


\subsection{Dataset Details}

\paragraph{Deepmind Lab.} The Deepmind Lab (DMLab) navigation dataset contains $64 \times 64$ resolution videos of random walks in a 3D maze environment. We use the total 625 videos with frame length 300 frames, and randomly sample sequences of 36 consecutive frames from each video during training. We upscale video frames to a resolution of $256 \times 256$ before tokenizing them similar to our approach for FaceForensics. We disregard the provided actions, focusing on action-unconditional video generation. We use $m=12$ and $s=24$ for the DMLab full sequence generation experiments unless stated otherwise.

\paragraph{FaceForensics.} FaceForensics (FFS) is a dataset that contains $150\times150$ images of deepfake faces, totaling 704 videos with varying number of frames at 8 frames-per-second. We upsample the resolution to $256 \times 256$, before encoding individual frames using the image-based tokenizer SD-VQGAN \cite{rombach2022high_latentdiffusion_ldm}. While image-based tokenizers have shown to lead to flickering issues, we observe high-reconstruction quality (reconstruction FVD $\approx 8$ on FFS) on our datasets and thus leave work on video tokenization to other works. After tokenization, we train on encoded frame sequences of 16 frames, each consisting of token grids with dimensionality $32 \times 32$. We generally use $m=2$ ground-truth context frames for conditioning, and $s=14$.

\subsection{Further Quantitative Results}

\paragraph{Our chunkwise autoregressive MGM-style sampling is preferable to full sequence training in settings with limited hardware.} To evaluate our method for long video generation against a longer training window baseline, we compare the performance of a frame-level masking model trained on $16$ frames with full sequence generation of a constant-masking level model trained on $32$ frames with similar batch size and on similar hardware. In Table ~\ref{tab:longer_train_window_baseline} we show that iterative rollout of our MGM-style sampling outperforms full sequence generation even when the full sequence model is trained on a longer window.

\begin{table}[ht]
    \centering
    \normalsize
    \resizebox{0.48\textwidth}{!}{%
    \begin{tabular}{l|cccc}
    \toprule
    \makecell{\textbf{Sampling} \\ \textbf{Mode}} 
    & \makecell{\textbf{Training} \\ \textbf{Window}} 
    & \makecell{\textbf{Sampling} \\ \textbf{Window}} 
    & \makecell{\textbf{Total} \\ \textbf{NFE}}
    & \makecell{\textbf{FVD} $\downarrow$} \\
    \midrule
    FM-Style (bs=2) & 32 & 32 & 250 & 253.08 \\
    \midrule
    \textit{MaskFlow} (MGM-Style) (bs=2) & 16 & 32 & 60 & 192.76 \\
    \rowcolor{gray!8}\textit{MaskFlow} (MGM-Style) (bs=4) & 16 & 32 & 60 & \textbf{59.93} \\
    \bottomrule
    \end{tabular}
    }
    \caption{\textbf{Our MGM-style sampling is more efficient and generates better results over baseline for larger training windows}. We train a constant masking ratio model on larger window sizes with similar batch size on similar hardware, and compare full sequence generation to generating the same length using our chunkwise MGM-style sampling.}
    \label{tab:longer_train_window_baseline}
\end{table}

\begin{table}[ht]
    \centering
    \normalsize
    \resizebox{0.48\textwidth}{!}{%
    \begin{tabular}{l|ccrr}
        \toprule
        & \makecell{\textbf{Extrapolation} \\ \textbf{Factor}}
        & \makecell{\textbf{Sampling} \\ \textbf{Stride}}
        & \makecell{\textbf{Total} \\ \textbf{NFE}}
        & \makecell{\textbf{FVD} $\downarrow$} \\
        \midrule
        FaceForensics   & $2\times$  & $s=14$ (\textit{full sequence}) & \textbf{60} & 59.93 \\
        \rowcolor{gray!8}FaceForensics   & $2\times$  & $s=1$ (\textit{autoregressive})  & 340 & \textbf{30.43} \\
        \midrule
        FaceForensics   & $5\times$  & $s=14$ (\textit{full sequence}) & \textbf{120} & 108.74 \\
        \rowcolor{gray!8}FaceForensics & $5\times$  & $s=1$ (\textit{autoregressive})  & 1,300 & \textbf{103.69} \\
        \midrule
        FaceForensics   & $10\times$ & $s=14$ (\textit{full sequence}) & \textbf{240} & 214.39 \\
       \rowcolor{gray!8} FaceForensics   & $10\times$ & $s=1$ (\textit{autoregressive})  & 2,900 & \textbf{165.02} \\
        \midrule
        \midrule
        DMLab & $2\times$  & $s=24$ (\textit{full sequence}) & \textbf{60} & 188.22 \\
        \rowcolor{gray!8}DMLab & $2\times$  & $s=1$ (\textit{autoregressive})  & 740  & \textbf{50.87} \\
        \midrule
        DMLab & $5\times$  & $s=24$ (\textit{full sequence}) & \textbf{140}  & 334.15 \\
       \rowcolor{gray!8} DMLab & $5\times$  & $s=1$ (\textit{autoregressive})  & 2,900 & \textbf{181.11} \\
        \bottomrule
    \end{tabular}
    }
    \caption{\textbf{Autoregressive sampling outperforms full sequence sampling on timestep-dependent models at the cost of higher NFE.}}
    \label{tab:autoregression_dependent}
\end{table}


\begin{figure*}[ht!]
    \centering
    \includegraphics[width=0.7\textwidth]{figpaper/realestate.pdf}\hfill
    \caption{\textbf{Further visualizations on the Realestate10K \cite{zhou2018stereo} dataset.} Models trained on chunk size $k = 16$ with $4$ H100 GPUs. Due to computational limitations, we cannot  provide further analyses on this larger, more compute intensive dataset.}
    \label{fig:faces_comparison}
\end{figure*}




\begin{table}[ht]
    \centering
    \normalsize
    \resizebox{0.58\textwidth}{!}{%
    \begin{tabular}{l|ccrr}
        \toprule
        & \makecell{\textbf{Extrapolation} \\ \textbf{Factor}}
        & \makecell{\textbf{Sampling} \\ \textbf{Stride}}
        & \makecell{\textbf{Total} \\ \textbf{NFE}}
        & \makecell{\textbf{FVD} $\downarrow$} \\
        \midrule
        FaceForensics   & $2\times$  & $s=14$ (\textit{full sequence}) & \textbf{60} & 109.96 \\
        FaceForensics   & $2\times$  & $s=1$ (\textit{autoregressive}) & 340 & \textbf{43.91} \\
        \midrule
        FaceForensics   & $5\times$  & $s=14$ (\textit{full sequence}) & \textbf{120} & \textbf{137.66} \\
        FaceForensics & $5\times$  & $s=1$ (\textit{autoregressive})  & 1,300 & 193.90 \\
        \midrule
        FaceForensics   & $10\times$ & $s=14$ (\textit{full sequence}) & \textbf{240} & \textbf{174.92} \\
        FaceForensics   & $10\times$ & $s=1$ (\textit{autoregressive})  & 2,900 & 293.16 \\
        \midrule
        \midrule
        DMLab & $2\times$  & $s=24$ (\textit{full sequence}) & \textbf{60} & 219.33 \\
        DMLab & $2\times$  & $s=1$ (\textit{autoregressive})  & 740  & \textbf{42.53} \\
        \midrule
        DMLab & $5\times$  & $s=24$ (\textit{full sequence}) & \textbf{140}  & 402.73 \\
        DMLab & $5\times$  & $s=1$ (\textit{autoregressive})  & 2,900 & \textbf{80.56} \\
        \bottomrule
    \end{tabular}
    }
    \caption{\textbf{Autoregressive sampling outperforms full sequence sampling on timestep-independent models at the cost of higher NFE.} Performance improvement on DMLab is substantial.}
    \label{tab:autoregression_independent}
\end{table}

\newpage

\subsection{Further Qualitative Results}

\begin{figure*}[ht!]
    \centering
    \includegraphics[width=0.48\textwidth]{figpaper/faces1.pdf}\hfill
    \includegraphics[width=0.48\textwidth]{figpaper/faces2.pdf}
    \caption{\textbf{Visualizations of FaceForensics generation results with different context frames.}}
    \label{fig:faces_comparison}
\end{figure*}















\input{6_toremove}

\end{document}
