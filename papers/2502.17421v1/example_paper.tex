%%%%%%%% ICML 2025 EXAMPLE LATEX SUBMISSION FILE %%%%%%%%%%%%%%%%%

\documentclass{article}

% Recommended, but optional, packages for figures and better typesetting:
\usepackage{comment}
\usepackage{microtype}
\usepackage{graphicx}
% \usepackage{subfigure}
\usepackage{multirow} 
\usepackage{booktabs} % for professional tables
\usepackage{adjustbox}
\usepackage{verbatim}
\usepackage{graphicx}
\usepackage{hyperref}
\usepackage{url}
\usepackage{makecell}
\usepackage{array}
% \usepackage{subfigure}
\usepackage{wrapfig}
\usepackage{enumerate}
\usepackage{enumitem}
\usepackage{soul}
\usepackage{subfig}
\usepackage[table]{xcolor}
% \usepackage{inconsolata}
\usepackage[figuresright]{rotating} 
% \usepackage[svgnames]{xcolor}
\usepackage{tcolorbox}
\definecolor{myblue}{HTML}{F0F8FF} 
\definecolor{mypurple}{HTML}{E6E6FA} 
\definecolor{mygreen}{HTML}{F0FFF0} 
\definecolor{mypink}{HTML}{FFF0F5} 
\newcommand{\fengzhuo}[1]{{\color{red}  [\mathrm{Fengzhuo:} #1]}}
\newcommand{\cunxiao}[1]{{\color{magenta}  [\mathrm{Cunxiao:} #1]}}
\newcommand{\penghui}[1]{{\color{blue}  [\mathrm{Penghui:} #1]}}

\newcommand{\att}{\mathsf{attn}}
\newcommand{\mha}{\mathsf{mha}}
\newcommand{\pe}{\mathsf{RoPE}}
\newcommand{\tilmha}{\widetilde{\mathsf{mha}}}
\newcommand{\ff}{\mathsf{ffn}}
\newcommand{\sm}{\mathsf{softmax}}
\newcommand{\lnor}{\mathsf{LN}}
\newcommand{\llm}{\mathsf{transformer}}
\newcommand{\pt}{\texttt{prompt}}
\newcommand{\sink}{\mathsf{lazy}}
\newcommand{\id}{\mathsf{Id}}
\newcommand{\unemb}{\mathsf{unemb}}
\newcommand{\lip}{{\mathsf{lip}}}
\newcommand{\test}{\texttt{test}}
\newcommand{\argmin}{\text{argmin}}
\usepackage{acronym}
\acrodef{mha}[MHA]{Multi-Head Attention}
\acrodef{ff}[FF]{Feed-Forward}
\acrodef{mle}[MLE]{Maximum Likelihood Estimate}
% hyperref makes hyperlinks in the resulting PDF.
% If your build breaks (sometimes temporarily if a hyperlink spans a page)
% please comment out the following usepackage line and replace
% \usepackage{icml2025} with \usepackage[nohyperref]{icml2025} above.
\usepackage{hyperref}


% Attempt to make hyperref and algorithmic work together better:
\newcommand{\theHalgorithm}{\arabic{algorithm}}

% Use the following line for the initial blind version submitted for review:
% \usepackage{icml2025}

% If accepted, instead use the following line for the camera-ready submission:
\usepackage[accepted]{icml2025}

% For theorems and such
\usepackage{amsmath}
\usepackage{amssymb}
\usepackage{mathtools}
\usepackage{amsthm}
%
% --- inline annotations
%
\newcommand{\red}[1]{{\color{red}#1}}
\newcommand{\todo}[1]{{\color{red}#1}}
\newcommand{\TODO}[1]{\textbf{\color{red}[TODO: #1]}}
% --- disable by uncommenting  
% \renewcommand{\TODO}[1]{}
% \renewcommand{\todo}[1]{#1}



\newcommand{\VLM}{LVLM\xspace} 
\newcommand{\ours}{PeKit\xspace}
\newcommand{\yollava}{Yo’LLaVA\xspace}

\newcommand{\thisismy}{This-Is-My-Img\xspace}
\newcommand{\myparagraph}[1]{\noindent\textbf{#1}}
\newcommand{\vdoro}[1]{{\color[rgb]{0.4, 0.18, 0.78} {[V] #1}}}
% --- disable by uncommenting  
% \renewcommand{\TODO}[1]{}
% \renewcommand{\todo}[1]{#1}
\usepackage{slashbox}
% Vectors
\newcommand{\bB}{\mathcal{B}}
\newcommand{\bw}{\mathbf{w}}
\newcommand{\bs}{\mathbf{s}}
\newcommand{\bo}{\mathbf{o}}
\newcommand{\bn}{\mathbf{n}}
\newcommand{\bc}{\mathbf{c}}
\newcommand{\bp}{\mathbf{p}}
\newcommand{\bS}{\mathbf{S}}
\newcommand{\bk}{\mathbf{k}}
\newcommand{\bmu}{\boldsymbol{\mu}}
\newcommand{\bx}{\mathbf{x}}
\newcommand{\bg}{\mathbf{g}}
\newcommand{\be}{\mathbf{e}}
\newcommand{\bX}{\mathbf{X}}
\newcommand{\by}{\mathbf{y}}
\newcommand{\bv}{\mathbf{v}}
\newcommand{\bz}{\mathbf{z}}
\newcommand{\bq}{\mathbf{q}}
\newcommand{\bff}{\mathbf{f}}
\newcommand{\bu}{\mathbf{u}}
\newcommand{\bh}{\mathbf{h}}
\newcommand{\bb}{\mathbf{b}}

\newcommand{\rone}{\textcolor{green}{R1}}
\newcommand{\rtwo}{\textcolor{orange}{R2}}
\newcommand{\rthree}{\textcolor{red}{R3}}
\usepackage{amsmath}
%\usepackage{arydshln}
\DeclareMathOperator{\similarity}{sim}
\DeclareMathOperator{\AvgPool}{AvgPool}

\newcommand{\argmax}{\mathop{\mathrm{argmax}}}     



% if you use cleveref..
\usepackage[capitalize,noabbrev]{cleveref}

%%%%%%%%%%%%%%%%%%%%%%%%%%%%%%%%
% THEOREMS
%%%%%%%%%%%%%%%%%%%%%%%%%%%%%%%%
\DeclareRobustCommand{\mylogo}{\adjustbox{valign=c}{\includegraphics[width=1.2cm]{Figure/logo.pdf}}}

\theoremstyle{plain}
\newtheorem{theorem}{Theorem}[section]
\newtheorem{proposition}[theorem]{Proposition}
\newtheorem{lemma}[theorem]{Lemma}
\newtheorem{corollary}[theorem]{Corollary}
\theoremstyle{definition}
\newtheorem{definition}[theorem]{Definition}
\newtheorem{assumption}[theorem]{Assumption}
\theoremstyle{remark}
\newtheorem{remark}[theorem]{Remark}

% Todonotes is useful during development; simply uncomment the next line
%    and comment out the line below the next line to turn off comments
%\usepackage[disable,textsize=tiny]{todonotes}
\usepackage[textsize=tiny]{todonotes}


% The \icmltitle you define below is probably too long as a header.
% Therefore, a short form for the running title is supplied here:
\icmltitlerunning{LongSpec: Long-Context Speculative Decoding with Efficient Drafting and Verification}

\begin{document}

\twocolumn[
% \icmltitle{LongSpec: Memory-efficient Draft Model with Parallel Tree Verification \\
%            for Lossless Long-Context Speculative Decoding}
% \icmltitle{LongSpec: Memory-Efficient Long-Context Speculative Decoding \\
%            with Sink Position ID and Aggregated Attention}
% \icmltitle{\textsc{LongSpec}: Long-Context Speculative Decoding \\
%            with Efficient Drafting and Verification}

\icmltitle{
  \texorpdfstring{
    \mylogo~~\LARGE \textsc{LongSpec}:~~~~~~~~~~~~~~~\\ \Large Long-Context Speculative Decoding with Efficient Drafting and Verification
  }{
    LongSpec: Long-Context Speculative Decoding with Efficient Drafting and Verification
  }
}

% It is OKAY to include author information, even for blind
% submissions: the style file will automatically remove it for you
% unless you've provided the [accepted] option to the icml2025
% package.

% List of affiliations: The first argument should be a (short)
% identifier you will use later to specify author affiliations
% Academic affiliations should list Department, University, City, Region, Country
% Industry affiliations should list Company, City, Region, Country

% You can specify symbols, otherwise they are numbered in order.
% Ideally, you should not use this facility. Affiliations will be numbered
% in order of appearance and this is the preferred way.
\icmlsetsymbol{equal}{*}

\begin{icmlauthorlist}
\icmlauthor{Penghui Yang}{equal,2}
\icmlauthor{Cunxiao Du}{equal,1}
\icmlauthor{Fengzhuo Zhang}{3}
\icmlauthor{Haonan Wang}{3}
\icmlauthor{Tianyu Pang}{1}
\icmlauthor{Chao Du}{1}
\icmlauthor{Bo An}{2}
\end{icmlauthorlist}

\icmlaffiliation{1}{Sea AI Lab, Singapore}
\icmlaffiliation{2}{Nanyang Technological University}
\icmlaffiliation{3}{National University of Singapore}

\icmlcorrespondingauthor{Penghui Yang}{phyang.cs@gmail.com}
\icmlcorrespondingauthor{Cunxiao Du}{ducx@sea.com}
\icmlcorrespondingauthor{Fengzhuo Zhang}{fzzhang@u.nus.edu}

% You may provide any keywords that you
% find helpful for describing your paper; these are used to populate
% the "keywords" metadata in the PDF but will not be shown in the document
\icmlkeywords{Machine Learning, ICML}

\vskip 0.3in]

% this must go after the closing bracket ] following \twocolumn[ ...

% This command actually creates the footnote in the first column
% listing the affiliations and the copyright notice.
% The command takes one argument, which is text to display at the start of the footnote.
% The \icmlEqualContribution command is standard text for equal contribution.
% Remove it (just {}) if you do not need this facility.

% \printAffiliationsAndNotice{}  % leave blank if no need to mention equal contribution
% \printAffiliationsAndNotice{\icmlEqualContribution} % otherwise use the standard text.
\printAffiliationsAndNoticeArxiv{\icmlEqualContribution}  % leave blank if no need to mention equal contribution
% \printAffiliationsAndNotice{\icmlEqualContribution} 
\begin{abstract}
Speculative decoding has become a promising technique to mitigate the high inference latency of autoregressive decoding in Large Language Models (LLMs). Despite its promise, the effective application of speculative decoding in LLMs still confronts three key challenges: the increasing memory demands of the draft model, the distribution shift between the short-training corpora and long-context inference, and inefficiencies in attention implementation. 
In this work, we enhance the performance of speculative decoding in long-context settings by addressing these challenges. First, we propose a memory-efficient draft model with a constant-sized Key-Value (KV) cache. Second, we introduce novel position indices for short-training data, enabling seamless adaptation from short-context training to long-context inference. 
Finally, we present an innovative attention aggregation method that combines fast implementations for prefix computation with standard attention for tree mask handling, effectively resolving the latency and memory inefficiencies of tree decoding.
Our approach achieves strong results on various long-context tasks, including repository-level code completion, long-context summarization, and o1-like long reasoning tasks, demonstrating significant improvements in latency reduction.
The code is available at \url{https://github.com/sail-sg/LongSpec}.
\end{abstract}

\section{Introduction}

Video generation has garnered significant attention owing to its transformative potential across a wide range of applications, such media content creation~\citep{polyak2024movie}, advertising~\citep{zhang2024virbo,bacher2021advert}, video games~\citep{yang2024playable,valevski2024diffusion, oasis2024}, and world model simulators~\citep{ha2018world, videoworldsimulators2024, agarwal2025cosmos}. Benefiting from advanced generative algorithms~\citep{goodfellow2014generative, ho2020denoising, liu2023flow, lipman2023flow}, scalable model architectures~\citep{vaswani2017attention, peebles2023scalable}, vast amounts of internet-sourced data~\citep{chen2024panda, nan2024openvid, ju2024miradata}, and ongoing expansion of computing capabilities~\citep{nvidia2022h100, nvidia2023dgxgh200, nvidia2024h200nvl}, remarkable advancements have been achieved in the field of video generation~\citep{ho2022video, ho2022imagen, singer2023makeavideo, blattmann2023align, videoworldsimulators2024, kuaishou2024klingai, yang2024cogvideox, jin2024pyramidal, polyak2024movie, kong2024hunyuanvideo, ji2024prompt}.


In this work, we present \textbf{\ours}, a family of rectified flow~\citep{lipman2023flow, liu2023flow} transformer models designed for joint image and video generation, establishing a pathway toward industry-grade performance. This report centers on four key components: data curation, model architecture design, flow formulation, and training infrastructure optimization—each rigorously refined to meet the demands of high-quality, large-scale video generation.


\begin{figure}[ht]
    \centering
    \begin{subfigure}[b]{0.82\linewidth}
        \centering
        \includegraphics[width=\linewidth]{figures/t2i_1024.pdf}
        \caption{Text-to-Image Samples}\label{fig:main-demo-t2i}
    \end{subfigure}
    \vfill
    \begin{subfigure}[b]{0.82\linewidth}
        \centering
        \includegraphics[width=\linewidth]{figures/t2v_samples.pdf}
        \caption{Text-to-Video Samples}\label{fig:main-demo-t2v}
    \end{subfigure}
\caption{\textbf{Generated samples from \ours.} Key components are highlighted in \textcolor{red}{\textbf{RED}}.}\label{fig:main-demo}
\end{figure}


First, we present a comprehensive data processing pipeline designed to construct large-scale, high-quality image and video-text datasets. The pipeline integrates multiple advanced techniques, including video and image filtering based on aesthetic scores, OCR-driven content analysis, and subjective evaluations, to ensure exceptional visual and contextual quality. Furthermore, we employ multimodal large language models~(MLLMs)~\citep{yuan2025tarsier2} to generate dense and contextually aligned captions, which are subsequently refined using an additional large language model~(LLM)~\citep{yang2024qwen2} to enhance their accuracy, fluency, and descriptive richness. As a result, we have curated a robust training dataset comprising approximately 36M video-text pairs and 160M image-text pairs, which are proven sufficient for training industry-level generative models.

Secondly, we take a pioneering step by applying rectified flow formulation~\citep{lipman2023flow} for joint image and video generation, implemented through the \ours model family, which comprises Transformer architectures with 2B and 8B parameters. At its core, the \ours framework employs a 3D joint image-video variational autoencoder (VAE) to compress image and video inputs into a shared latent space, facilitating unified representation. This shared latent space is coupled with a full-attention~\citep{vaswani2017attention} mechanism, enabling seamless joint training of image and video. This architecture delivers high-quality, coherent outputs across both images and videos, establishing a unified framework for visual generation tasks.


Furthermore, to support the training of \ours at scale, we have developed a robust infrastructure tailored for large-scale model training. Our approach incorporates advanced parallelism strategies~\citep{jacobs2023deepspeed, pytorch_fsdp} to manage memory efficiently during long-context training. Additionally, we employ ByteCheckpoint~\citep{wan2024bytecheckpoint} for high-performance checkpointing and integrate fault-tolerant mechanisms from MegaScale~\citep{jiang2024megascale} to ensure stability and scalability across large GPU clusters. These optimizations enable \ours to handle the computational and data challenges of generative modeling with exceptional efficiency and reliability.


We evaluate \ours on both text-to-image and text-to-video benchmarks to highlight its competitive advantages. For text-to-image generation, \ours-T2I demonstrates strong performance across multiple benchmarks, including T2I-CompBench~\citep{huang2023t2i-compbench}, GenEval~\citep{ghosh2024geneval}, and DPG-Bench~\citep{hu2024ella_dbgbench}, excelling in both visual quality and text-image alignment. In text-to-video benchmarks, \ours-T2V achieves state-of-the-art performance on the UCF-101~\citep{ucf101} zero-shot generation task. Additionally, \ours-T2V attains an impressive score of \textbf{84.85} on VBench~\citep{huang2024vbench}, securing the top position on the leaderboard (as of 2025-01-25) and surpassing several leading commercial text-to-video models. Qualitative results, illustrated in \Cref{fig:main-demo}, further demonstrate the superior quality of the generated media samples. These findings underscore \ours's effectiveness in multi-modal generation and its potential as a high-performing solution for both research and commercial applications.

\section{Related Work}

\subsection{Large 3D Reconstruction Models}
Recently, generalized feed-forward models for 3D reconstruction from sparse input views have garnered considerable attention due to their applicability in heavily under-constrained scenarios. The Large Reconstruction Model (LRM)~\cite{hong2023lrm} uses a transformer-based encoder-decoder pipeline to infer a NeRF reconstruction from just a single image. Newer iterations have shifted the focus towards generating 3D Gaussian representations from four input images~\cite{tang2025lgm, xu2024grm, zhang2025gslrm, charatan2024pixelsplat, chen2025mvsplat, liu2025mvsgaussian}, showing remarkable novel view synthesis results. The paradigm of transformer-based sparse 3D reconstruction has also successfully been applied to lifting monocular videos to 4D~\cite{ren2024l4gm}. \\
Yet, none of the existing works in the domain have studied the use-case of inferring \textit{animatable} 3D representations from sparse input images, which is the focus of our work. To this end, we build on top of the Large Gaussian Reconstruction Model (GRM)~\cite{xu2024grm}.

\subsection{3D-aware Portrait Animation}
A different line of work focuses on animating portraits in a 3D-aware manner.
MegaPortraits~\cite{drobyshev2022megaportraits} builds a 3D Volume given a source and driving image, and renders the animated source actor via orthographic projection with subsequent 2D neural rendering.
3D morphable models (3DMMs)~\cite{blanz19993dmm} are extensively used to obtain more interpretable control over the portrait animation. For example, StyleRig~\cite{tewari2020stylerig} demonstrates how a 3DMM can be used to control the data generated from a pre-trained StyleGAN~\cite{karras2019stylegan} network. ROME~\cite{khakhulin2022rome} predicts vertex offsets and texture of a FLAME~\cite{li2017flame} mesh from the input image.
A TriPlane representation is inferred and animated via FLAME~\cite{li2017flame} in multiple methods like Portrait4D~\cite{deng2024portrait4d}, Portrait4D-v2~\cite{deng2024portrait4dv2}, and GPAvatar~\cite{chu2024gpavatar}.
Others, such as VOODOO 3D~\cite{tran2024voodoo3d} and VOODOO XP~\cite{tran2024voodooxp}, learn their own expression encoder to drive the source person in a more detailed manner. \\
All of the aforementioned methods require nothing more than a single image of a person to animate it. This allows them to train on large monocular video datasets to infer a very generic motion prior that even translates to paintings or cartoon characters. However, due to their task formulation, these methods mostly focus on image synthesis from a frontal camera, often trading 3D consistency for better image quality by using 2D screen-space neural renderers. In contrast, our work aims to produce a truthful and complete 3D avatar representation from the input images that can be viewed from any angle.  

\subsection{Photo-realistic 3D Face Models}
The increasing availability of large-scale multi-view face datasets~\cite{kirschstein2023nersemble, ava256, pan2024renderme360, yang2020facescape} has enabled building photo-realistic 3D face models that learn a detailed prior over both geometry and appearance of human faces. HeadNeRF~\cite{hong2022headnerf} conditions a Neural Radiance Field (NeRF)~\cite{mildenhall2021nerf} on identity, expression, albedo, and illumination codes. VRMM~\cite{yang2024vrmm} builds a high-quality and relightable 3D face model using volumetric primitives~\cite{lombardi2021mvp}. One2Avatar~\cite{yu2024one2avatar} extends a 3DMM by anchoring a radiance field to its surface. More recently, GPHM~\cite{xu2025gphm} and HeadGAP~\cite{zheng2024headgap} have adopted 3D Gaussians to build a photo-realistic 3D face model. \\
Photo-realistic 3D face models learn a powerful prior over human facial appearance and geometry, which can be fitted to a single or multiple images of a person, effectively inferring a 3D head avatar. However, the fitting procedure itself is non-trivial and often requires expensive test-time optimization, impeding casual use-cases on consumer-grade devices. While this limitation may be circumvented by learning a generalized encoder that maps images into the 3D face model's latent space, another fundamental limitation remains. Even with more multi-view face datasets being published, the number of available training subjects rarely exceeds the thousands, making it hard to truly learn the full distibution of human facial appearance. Instead, our approach avoids generalizing over the identity axis by conditioning on some images of a person, and only generalizes over the expression axis for which plenty of data is available. 

A similar motivation has inspired recent work on codec avatars where a generalized network infers an animatable 3D representation given a registered mesh of a person~\cite{cao2022authentic, li2024uravatar}.
The resulting avatars exhibit excellent quality at the cost of several minutes of video capture per subject and expensive test-time optimization.
For example, URAvatar~\cite{li2024uravatar} finetunes their network on the given video recording for 3 hours on 8 A100 GPUs, making inference on consumer-grade devices impossible. In contrast, our approach directly regresses the final 3D head avatar from just four input images without the need for expensive test-time fine-tuning.



% For an integer $n$, let $[n]$ denote the set $\{1,2, \cdots, n\}$. For a finite set $A$, let $|A|$ be the number of elements in $A$, and $\Delta_{A}$ denote the set of all distributions on $A$. 

\paragraph{Proper Scoring Rule} Given a finite set $\Sigset$, a scoring rule $\ps: \Sigset\times \Delta_{\Sigset} \to \mathbb{R}$ maps an element $\sigi\in\Sigset$ and a distribution $\vpr$ on $\Sigset$ to a score. A scoring rule $PS$ is {\em proper} if for any distributions $\vpr_1$ and $\vpr_2$, $\Ex_{\sigi \sim \vpr_1}[\ps(\sigi, \vpr_1)] \ge \Ex_{\sigi \sim \vpr_1}[\ps(\sigi, \vpr_2)]$ and {\em strictly proper} if the equality holds only at $\vpr_1 = \vpr_2$. 
\begin{example}
    Given a distribution $\prQ$ on a finite set $\Sigset$, let $\pr(s)$ be the probability of $s\in \Sigset$ in $\prQ$. The log score rule $\ps_L(s, \prQ) = \log (\pr(s))$. The Brier/quadratic scoring rule $\ps_B(s, \prQ) = 2\cdot \pr(s) - \prQ\cdot \prQ$. Both the log scoring rule and the Brier scoring rule are strictly proper. 
\end{example}

\subsection{Bayesian Game Model}
A Bayesian game $\inst = ([n], (\rpset_i)_{i \in [n]}, (\Sigset_i)_{i\in[n]}, (\vt_i)_{i\in [n]}, \prQ)$ is defined by the following components. 
\begin{itemize}
    \item The set of agents $[n]$. 
    \item For each agent $i$, $\rpset_i$ is the set of available actions of $i$. The action profile $\rpp = (\rp_1, \rp_2, \cdots, \rp_\ag)$ is the vector of actions of all the agents. 
    \item For each agent $i$, $\Sigset_i$ is the set of possible types of agent $i$. The type characterizes the private information agent $i$ holds, and the agent can only observe his/her type in the game. The type vector $\sigp = (\sigi_1, \sigi_2, \cdots, \sigi_\ag)$ is the vector of types of all agents. 
    \item For each agent $i$, $\vt_i: \Sigset_i \times \rpset_1\times \cdots \times \rpset_n \to \mathbb{R}$ is $i$'s utility function that maps $i$'s type and the action of all the agents to $i$'s utility. 
    \item A {\em common prior} that the types of the agents follow is a joint distribution $\prQ$. For a signal $\sigi_i$ of agent $i$, we use $\pr(\sigi_i)$ to denote the marginal prior probability that $i$'s signal is $\sigi_i$. We assume that $\pr(\sigi_i) > 0$ for any $i$ and any $\sigi_i \in \Sigset_i$. 
\end{itemize}

For each agent $i$, a (mixed) strategy $\stg_i: \Sigset_i \to \Delta_{\rpset_i}$ maps $i$ private signal to a distribution on his/her actions. A strategy profile $\stgp = (\stg_i)_{i \in [n]}$ is a vector of the strategies of all the agents. 

Given a strategy profile $\stgp$, the {\em ex-ante} expected utility of agent $i$ is 
\begin{equation*}
    \ut_i(\stgp) = \Ex_{S \sim \prQ}\ \Ex_{A}[\vt_i(\sigi_i, \rp_1, \cdots, \rp_n)\mid \stgp].
\end{equation*}

Similarly, given a strategy profile $\stgp$ and a type $\sigi_i$, the {\em \qi{}} expected utility of agent $i$ conditioned on his/her type being $\sigi_i$ is 
\begin{equation*}
    \ut_i(\stgp \mid \sigi_i) = \Ex_{S_{-i} \sim \prQ_{-i\mid \sigi_i}}\ \Ex_{A}[\vt_i(\sigi_i, \rp_1, \cdots, \rp_n)\mid \stgp],
\end{equation*}
where $\sigp_{-i}$ is the type vector of all agents except for agent $i$, and $\prQ_{-i\mid \sigi_i}$ is the joint distribution on $\sigp_{-i}$ conditioned on agent $i$'s signal being $\sigi_{i}$. 

\subsection{(Ex-ante) Bayesian \textit{k}-Strong Equilibrium}
In this paper, we focus on agents that coordinate for strategic behaviors before they know their types. This assumption relates to various constraints in real-world scenarios that prevent agents from discussions after knowing their types. 
\begin{example}
    Consider the online crowdsourcing group in Example~\ref{ex:motive}. The website requires workers to make an immediate report after seeing the task so that workers cannot communicate with each other after they know their types. (For example, workers have to submit the report in 30 seconds to reflect their intuition.) However, workers may collude on the same report before seeing the task.
\end{example}
Both equilibria share the same high-level form: there does not exist a group of $\kd$ agents and a deviating strategy such that all the deviators' expected utility in the deviation is as good as the equilibrium strategy profile and at least one deviator's expected utility strictly increases. The difference lies in the expected utility. Ex-ante Bayesian $\kd$-strong equilibrium adopts ex-ante expected utility, while Bayesian $\kd$-strong equilibrium adopts interim expected utility on every type. 

\begin{definition}[ex-ante Bayesian $\kd$-strong equilibrium]

\label{def:ex_ante}
    Given an integer $\kd \ge 1$, a strategy profile $\stgp$ is an ex-ante Bayesian $k$-strong equilibrium ($\kd$-EBSE) if there does not exist a group of agent $D$ with $|D| \le \kd$ and a different strategy profile $\stgp' = (\stg'_{\sag})$ such that 
    \begin{enumerate}
    \item for all agent $i \not \in D$, $\stg'_{\sag} = \stg_{\sag}$; 
    \item for all $\sag\in D$, $ \ut_i(\stgp') \ge  \ut_i(\stgp)$;
    \item there exists an $\sag\in D$ such that $\ut_i(\stgp') > \ut_i(\stgp)$. 
\end{enumerate}
\end{definition}

\begin{definition}[Bayesian $\kd$-strong equilibrium]
\label{def:qi}
    Given an integer $\kd \ge 1$, a strategy profile $\stgp$ is a Bayesian $\kd$-strong equilibrium ($\kd$-BSE) if there does not exist a group of agent $D$ with $|D| \le k$ and a different strategy profile $\stgp' = (\stg'_{\sag})$ such that 
    \begin{enumerate}
    \item for all agent $i \not \in D$, $\stg'_i = \stg_i$; 
    \item for every $\sag\in D$ and every $\sigi_i \in \Sigset_i$, $ \ut_i(\stgp'\mid \sigi_i) \ge  \ut_i(\stgp\mid \sigi_i)$;
    \item there exist an $i\in D$ and an $\sigi_i \in \Sigset_i$ such that $\ut_i(\stgp'\mid \sigi_i) > \ut_i(\stgp\mid \sigi_i)$. 
\end{enumerate}
\end{definition}

% \begin{definition}[$\varepsilon$-approximation $\kd$-strong \textbf{interim} Bayesian-Nash Equilibrium]
%     Given an integer $\kd \ge 1$ and a constant $\varepsilon 
%     \ge 0$, a strategy profile $\stgp$ is an $\varepsilon$-approximation $\kd$-strong interim Bayesian-Nash Equilibrium ($\varepsilon$-apx-$\kd$-strong IBNE) if there does not exist a group of agent $D$ with $|D| \le k$ and a different strategy profile $\stgp' = (\stg'_{\sag})$ such that 
%     \begin{enumerate}
%     \item for all agent $i \not \in D$, $\stg'_i = \stg_i$; 
%     \item For every $\sag\in D$, \textbf{there exists a }$\sigi_i \in \Sigset_i$ such that $ \ut_i(\stgp'\mid \sigi_i) \ge  \ut_i(\stgp\mid \sigi_i)$.
%     \item There exists a $i\in D$ and a $\sigi_i \in \Sigset_i$ such that $\ut_i(\stgp'\mid \sigi_i) > \ut_i(\stgp\mid \sigi_i) + \varepsilon$. 
% \end{enumerate}
% \end{definition}

In both solution concepts, if such a deviating group $D$ and a strategy profile $\stgp'$ exist, we say that the deviation succeeds.

When $\kd = 1$, both ex-ante Bayesian $1$-strong equilibrium and Bayesian $1$-strong equilibrium are equivalent to the Bayesian Nash equilibrium~\citep{Harsanyi67}. (See Appendix~\ref{apx:equiv}.) However, the two solution concepts are not equivalent for larger $\kd$. Example~\ref{ex:difference} illustrates a scenario in the peer prediction mechanism where the same deviation succeeds under the ex-ante Bayesian $\kd$-strong equilibrium but fails under the Bayesian $\kd$-strong equilibrium. 

We interpret the difference between the two solution concepts as different attitudes of agents towards deviations. Agents are assumed to be more conservative, i.e., unwilling to suffer loss, towards deviations under Bayesian $k$-strong equilibrium, as they will deviate only when the deviation brings them higher interim expected utility conditioned on every type. On the other hand, agents under the ex-ante Bayesian $k$-strong equilibrium will deviate once their ex-ante expected utility increases.  Proposition~\ref{prop:etoq} supports our interpretation by revealing that an ex-ante Bayesian $\kd$-strong equilibrium implies a Bayesian $\kd$-strong equilibrium. 

\begin{prop}
\label{prop:etoq}
    For every strategy profile $\stgp$ and every $1\le \kd \le \ag$, if $\stgp$ is an ex-ante Bayesian $\kd$-strong equilibrium, then $\stgp$ is a Bayesian $\kd$-strong equilibrium. 
\end{prop}
\begin{proof}
    Suppose $\stgp'$ is an arbitrary deviating profile from $\stgp$ with no more than $\kd$ deviators, and $i$ is an arbitrary deviator in $\stgp'$. 
    Since $\stgp$ is an ex-ante Bayesian $\kd$-strong equilibrium, then $\ut_i (\stgp') \le \ut_i (\stgp) $. By the law of total probability,  
    $\ut_i(\stgp) = \sum_{\sigi_i \in \Sigset_i} \prQ(\sigi_i)\cdot \ut_i(\stgp \mid \sigi_i)$. 
    Therefore, one of the following must hold: (1) for all $\sigi\in \Sigset_i$, $\ut_i (\stgp' \mid \sigi_i) = \ut_i (\stgp \mid \sigi_i)$, or (2) there exists a $\sigi\in\Sigset_i$, $\ut_i (\stgp' \mid \sigi_i) < \ut_i (\stgp \mid \sigi_i)$. In either case, the deviation fails. Therefore, $\stgp$ is a Bayesian $\kd$-strong equilibrium. 
\end{proof}

\subsection{Peer Prediction Mechanism}
In a peer prediction mechanism, each agent receives a private signal in $\Sigset = \{\ell, h\}$ and reports it to the mechanism. All the agents share the same type set $\Sigset_i = \Sigset$ and action set $\rpset_i = \Sigset$. 

$\prQ$ is the common prior joint distribution of the signals. Let $\Sigrv_{\sag}$ denote the random variable of agent $i$'s private signal.
% Formally, the probability that agent 1 has signal $\sigi_1$, agent 2 has signal $\sigi_2$, $\cdots$, and agent $\ag$ has signal $\sigi_\ag$ is $\pr(\Sigrv_1=\sigi_1, \Sigrv_2 = \sigi_2,\cdots, \Sigrv_\ag = \sigi_\ag)$. 
We assume that the common prior $\prQ$ is symmetric --- for any permutation $\pi$ on $[\ag]$, $\prQ(\Sigrv_1=\sigi_1, \Sigrv_2 = \sigi_2,\cdots, \Sigrv_\ag = \sigi_\ag)=\pr(\Sigrv_1=\sigi_{\pi(1)}, \Sigrv_2 = \sigi_{\pi(2)},\cdots, \Sigrv_\ag = \sigi_{\pi(\ag)})$. 

$\pr(\sigi)$ is the prior marginal belief that an agent has signal $\sigi$, and $\pr(\sigi \mid \sigi')$ be the posterior belief of an agent with private signal $\sigi'$ on another agent having signal $\sigi$. We also define $\vpr_{\sigi} = \pr(\cdot \mid \sigi)$ be the marginal distribution on $\Sigset$ conditioned on $\sigi$. We assume that an agent with $h$ signal has a higher estimation than an agent with $\ell$ signal on the probability that another agent has $h$ signal, i.e., $\pr(h\mid h) > \pr(h \mid \ell)$. We also assume that any pair of signals is not fully correlated, which is $\pr(h\mid \ell) > 0$ and $\pr(\ell \mid h) > 0$. 

We adopt a modified version of the peer prediction mechanism~\citep{Miller05:Eliciting} characterized by a (strictly) proper scoring rule $\ps$. The mechanism compares the report of agent $i$, denoted by $\rp_i$, with the reports of all other agents. For each agent $j$ with report $\rp_j$, the reward $i$ gains from comparison with $j$'s report is $\rwd_i(\rp_j) = \ps(\rp_j, \vpr_{\rp_i}).$
The utility of agent $i$ is the average reward from each $j$.
\begin{equation*}
    \vt_i(\sigi_i, \rpp) = \frac{1}{\ag-1}\sum_{j\in[n], j\neq i} \rwd_i(\rp_j). 
\end{equation*}
\begin{remark}
    In the original mechanism in~\citep{Miller05:Eliciting}, the reward of an agent $i$ is $\rwd_i(\rp_j)$, where $j$ is chosen uniformly at random from all other agents. We derandomize the mechanism so that it fits better into the Bayesian game framework while the expected utility of an agent is unchanged. 
\end{remark}

\begin{example}
    \label{ex:setting}
    Suppose $n = 100$. For the common prior, the prior belief $\pr(h) = 2/3$, and $\pr(\ell) = 1/3$. The posterior belief $\pr(h \mid h) = 0.8$ and $\pr(\ell \mid \ell) = 0.6$. Suppose the Brier scoring rule is applied to the peer prediction mechanism. Consider an agent $i$ with report $\rp_i = h$. Then, $i$'s reward from a peer $j$ with report $\rp_j = h$ is $\rwd_i(\rp_j) = \ps_B(h, \prQ_h) = 2\cdot \pr(h \mid h) - \pr(h\mid h)^2 - \pr(\ell \mid h)^2 = 0.92$. Similarly, $i$' reward from another peer $j'$ with report $\rp_{j'} = \ell$ is $\ps_B(\ell, \prQ_h) = -0.28$. 
\end{example}

A (mixed) strategy $\stg: \Sigset_i \to \Delta_{\rpset_i}$ maps an agent's type to a distribution on his/her action. A strategy profile $\stgp = (\stg_i)_{i \in [n]}$ is a vector of the strategies of all the agents. An agent is {\em truthful} if he/she always truthfully reports his/her private signal. Let $\stg^*$ be the truthful strategy and $\stgp^*$ be the strategy profile where all agents are truthful. 
We also represent a strategy in the form $\stg = (\bpl, \bph) \in [0, 1]^2$, where $\bpl$ and $\bph$ are the probability that an agent playing $\stg$ reports $h$ conditioned on his/her signal begin $\ell$  and $h$, respectively. The truthful strategy $\stg^* = (0, 1)$.

Given the strategy profile $\stgp$, the ex-ante expected utility of an agent $i$ is
\begin{equation*}
    \ut_i(\stgp) = \frac{1}{n-1}\sum_{j\in [n], j\neq i} \Ex_{\sigi_i \sim \prQ
    , \rp_i \sim \stg_i(\sigi_i)} \Ex_{\sigi_j \sim \vpr_{\sigi_i}, \rp_j \sim \stg_j(\sigi_j)} \rwd_i(\rp_j). 
\end{equation*}

Given a strategy profile $\stgp$ and a type $\sigi_i$, the \qi{} expected utility of an agent $i$ conditioned on his/her type being $\sigi_i$ is 
\begin{equation*}
    \ut_i(\stgp\mid \sigi_i) = \frac{1}{n-1}\sum_{j\in [n], j\neq i} \Ex_{\rp_i \sim \stg_i(\sigi_i)} \Ex_{\sigi_j \sim \vpr_{\sigi_i}, \rp_j \sim \stg_j(\sigi_j)} \rwd_i(\rp_j). 
\end{equation*}

\begin{example} 
\label{ex:difference}
    We follow the setting in example~\ref{ex:setting}. Let $\stgp^*$ be the profile where all agents report truthfully. Let $D$ be a group containing $\kd = 40$ agents and $\stgp'$ be the profile where all deviators report $h$. 

    For truthful reporting, consider an agent $i$ and his/her peer $j$. The probability that both $i$ and $j$ receive (and report) signal $h$ is $\pr(h)\cdot \pr(h \mid h) = 2/3 * 0.8 = 0.533$, and $i$ will be rewarded $\ps(h, \vpr_h) = 0.92$. Other probabilities can be calculated similarly. Adding on the expectation of different pairs of signals, we can calculate the ex-ante expected utility of $i$ in truthful reporting: $\ut_i(\stgp^*) = \sum_{\sigi_i, \sigi_j \in \{\ell, h\}} \pr(\sigi_i) \cdot \pr(\sigi_j\mid \sigi_i)\cdot \ps(\sigi_j, \vpr_{\sigi_i}) = 0.627$. 

    Now we consider the expected utility of a deviator $i$ deviating profile $\stgp'$. Since all the deviators always report $h$, the expected reward $i$ gets from a deviator is $\ps(h, \vpr_h) = 0.92$. For the rewards from a truthful reporter, $i$'s expected reward is $\pr(h)\cdot \ps(h, \vpr_h) + \pr(\ell)\cdot \ps(\ell, \vpr_h) = 0.52$. Among all the other agents, $\kd - 1 = 39$ agents are deviators, and $\ag - \kd = 60$ agents are truthful reporters. Therefore, $i$'s expected utility on $\stgp'$ is $\ut_i(\stgp') = 0.682 > \ut_i(\stgp^*)$. Therefore, the deviation succeeds under the ex-ante Bayesian $\kd$-strong equilibrium. 

    However, the deviation fails under the Bayesian $\kd$-strong equilibrium. The truthful expected utility conditioned on $i$'s signal is $\ell$ is $\ut_i(\stgp^* \mid \ell) = \sum_{\sigi_j \in \{\ell, h\}} \pr(\sigi_j\mid \ell)\cdot \ps(\sigi_j, \vpr_{\ell}) = 0.52$. On the other hand, when agents deviate to $\stgp'$, $i$'s reward from a truthful agents becomes $\sum_{\sigi_j \in \{\ell, h\}} \pr(\sigi_j\mid \ell)\cdot \ps(\sigi_j, \vpr_{h}) = 0.2$. Therefore, $i$'s interim expected utility $\ut_i(\stgp' \mid \ell) = 0.484 < \ut_i(\stgp^* \mid \ell).$ 
\end{example}
% \section{Preliminary}
% \subsection{Problem Formulation}
% \subsection{Motivation}

% The ability to handle long-context inputs has become increasingly critical because of applications such as long-document summarization and repository-level code completion. Speculative decoding, as an important technique for accelerating LLM inference, exhibits great potential for addressing the computational challenges in long-context scenarios. However, current speculative decoding methods in these scenarios only use draft models that do not require additional training. While this approach provides simplicity, it inherently limits their effectiveness and adaptability to diverse tasks.

% Despite their success, state-of-the-art speculative decoding methods designed for short-context scenarios are faced with several critical limitations when applied to long-context scenarios.

% \begin{itemize}

% \item \textbf{ Growth in Memory Requirements.} As the decoding length increases, current methods require progressively larger KV caches. This introduces significant storage overhead, especially in long-context settings where memory efficiency is of great importance. Addressing this issue demands an innovative approach to ensure constant memory usage, irrespective of context length.

% \item \textbf{Dependence on Predefined Model Structures.} The draft models used by these methods rely heavily on information from the target model, including its rotary position embeddings (RoPE). Since the target model's RoPE base is fixed, the draft model cannot be directly trained on short texts with a small RoPE base, as is common in long-context scenarios. This limitation hinders the draft model's ability to generalize from short-context training to long-context applications. A novel mechanism is needed to bridge this gap without altering the RoPE base.

% \item \textbf{Incompatibility with Flash Attention.} Current SOTA techniques often employ tree decoding, a method incompatible with Flash Attention. Flash Attention is a widely adopted acceleration mechanism critical for reducing latency and memory usage during inference. The inability to integrate tree decoding with Flash Attention exacerbates latency and memory challenges, particularly in long-context scenarios.

% \end{itemize}

\section{Study Design}
% robot: aliengo 
% We used the Unitree AlienGo quadruped robot. 
% See Appendix 1 in AlienGo Software Guide PDF
% Weight = 25kg, size (L,W,H) = (0.55, 0.35, 06) m when standing, (0.55, 0.35, 0.31) m when walking
% Handle is 0.4 m or 0.5 m. I'll need to check it to see which type it is.
We gathered input from primary stakeholders of the robot dog guide, divided into three subgroups: BVI individuals who have owned a dog guide, BVI individuals who were not dog guide owners, and sighted individuals with generally low degrees of familiarity with dog guides. While the main focus of this study was on the BVI participants, we elected to include survey responses from sighted participants given the importance of social acceptance of the robot by the general public, which could reflect upon the BVI users themselves and affect their interactions with the general population \cite{kayukawa2022perceive}. 

The need-finding processes consisted of two stages. During Stage 1, we conducted in-depth interviews with BVI participants, querying their experiences in using conventional assistive technologies and dog guides. During Stage 2, a large-scale survey was distributed to both BVI and sighted participants. 

This study was approved by the University’s Institutional Review Board (IRB), and all processes were conducted after obtaining the participants' consent.

\subsection{Stage 1: Interviews}
We recruited nine BVI participants (\textbf{Table}~\ref{tab:bvi-info}) for in-depth interviews, which lasted 45-90 minutes for current or former dog guide owners (DO) and 30-60 minutes for participants without dog guides (NDO). Group DO consisted of five participants, while Group NDO consisted of four participants.
% The interview participants were divided into two groups. Group DO (Dog guide Owner) consisted of five participants who were current or former dog guide owners and Group NDO (Non Dog guide Owner) consisted of three participants who were not dog guide owners. 
All participants were familiar with using white canes as a mobility aid. 

We recruited participants in both groups, DO and NDO, to gather data from those with substantial experience with dog guides, offering potentially more practical insights, and from those without prior experience, providing a perspective that may be less constrained and more open to novel approaches. 

We asked about the participants' overall impressions of a robot dog guide, expectations regarding its potential benefits and challenges compared to a conventional dog guide, their desired methods of giving commands and communicating with the robot dog guide, essential functionalities that the robot dog guide should offer, and their preferences for various aspects of the robot dog guide's form factors. 
For Group DO, we also included questions that asked about the participants' experiences with conventional dog guides. 

% We obtained permission to record the conversations for our records while simultaneously taking notes during the interviews. The interviews lasted 30-60 minutes for NDO participants and 45-90 minutes for DO participants. 

\subsection{Stage 2: Large-Scale Surveys} 
After gathering sufficient initial results from the interviews, we created an online survey for distributing to a larger pool of participants. The survey platform used was Qualtrics. 

\subsubsection{Survey Participants}
The survey had 100 participants divided into two primary groups. Group BVI consisted of 42 blind or visually impaired participants, and Group ST consisted of 58 sighted participants. \textbf{Table}~\ref{tab:survey-demographics} shows the demographic information of the survey participants. 

\subsubsection{Question Differentiation} 
Based on their responses to initial qualifying questions, survey participants were sorted into three subgroups: DO, NDO, and ST. Each participant was assigned one of three different versions of the survey. The surveys for BVI participants mirrored the interview categories (overall impressions, communication methods, functionalities, and form factors), but with a more quantitative approach rather than the open-ended questions used in interviews. The DO version included additional questions pertaining to their prior experience with dog guides. The ST version revolved around the participants' prior interactions with and feelings toward dog guides and dogs in general, their thoughts on a robot dog guide, and broad opinions on the aesthetic component of the robot's design. 


% \begin{table*}[t]
% \centering
% \resizebox{\textwidth}{!}{%
% \begin{tabular}{l|l|l|l|l|l|l|l|l|l|l|l}
% \hline
% Model          & Setting       & \multicolumn{2}{c|}{GovReport} & \multicolumn{2}{c|}{QMSum} & \multicolumn{2}{c|}{Multi-News} & \multicolumn{2}{c|}{LCC} & \multicolumn{2}{c}{RepoBench-P} \\ \hline
%                &               & $\tau$ & Tokens/s & $\tau$ & Tokens/s & $\tau$ & Tokens/s & $\tau$ & Tokens/s & $\tau$ & Tokens/s \\ \hline
% Vicuna-7B      & Vanilla Torch & 1          & 25.25    & 1          & 18.12    & 1          & 27.29    & 1          & 25.25    & 1          & 19.18    \\
%                & Vanilla       & 1          & 45.76    & 1          & 43.68    & 1          & 55.99    & 1          & 54.07    & 1          & 46.61    \\
%                & Seq           & 2.30       & 96.94    & 2.05       & 86.01    & 2.27       & 98.25    & 2.48       & 105.80   & 2.54       & 108.30   \\
%                & Tree          & 3.57       & 102.23   & 3.14       & 88.87    & 3.51       & 100.55   & 3.73       & 107.30   & 3.86       & 110.76   \\ \hline
% Vicuna-13B     & Vanilla Torch & 1          & 17.25    & 1          & 11.86    & 1          & 18.81    & 1          & 17.25    & 1          & 13.44    \\
%                & Vanilla       & 1          & 28.52    & 1          & 27.43    & 1          & 35.01    & 1          & 33.87    & 1          & 29.14    \\
%                & Seq           & 2.15       & 61.79    & 1.76       & 49.62    & 2.23       & 69.21    & 2.34       & 71.22    & 2.30       & 66.33    \\
%                & Tree          & 3.31       & 71.08    & 2.76       & 57.15    & 3.44       & 78.20    & 3.57       & 81.00    & 3.59       & 77.22    \\ \hline
% LongChat-7B    & Vanilla Torch & 1          & 25.27    & 1          & 14.11    & 1          & 27.66    & 1          & 25.27    & 1          & 17.02    \\
%                & Vanilla       & 1          & 42.14    & 1          & 36.87    & 1          & 50.19    & 1          & 54.17    & 1          & 42.69    \\
%                & Seq           & 2.30       & 94.27    & 2.01       & 79.07    & 2.21       & 91.61    & 2.78       & 119.37   & 2.66       & 108.80   \\
%                & Tree          & 3.59       & 101.43   & 3.06       & 85.23    & 3.41       & 97.93    & 4.21       & 122.30   & 4.03       & 115.27   \\ \hline
% LongChat-13B   & Vanilla Torch & 1          & 17.72    & 1          & 12.08    & 1          & 18.74    & 1          & 17.72    & 1          & 13.85    \\
%                & Vanilla       & 1          & 28.56    & 1          & 27.18    & 1          & 35.37    & 1          & 34.58    & 1          & 29.74    \\
%                & Seq           & 2.28       & 65.55    & 2.02       & 56.57    & 2.29       & 71.32    & 2.62       & 79.87    & 2.89       & 83.43    \\
%                & Tree          & 3.58       & 76.26    & 3.15       & 64.41    & 3.50       & 80.48    & 4.01       & 90.92    & 4.46       & 96.96    \\ \hline
% LLaMA3-8B      & Vanilla Torch & 1          & 21.59    & 1          & 18.67    & 1          & 29.91    & 1          & 29.48    & 1          & 22.77    \\
%                & Vanilla       & 1          & 53.14    & 1          & 51.22    & 1          & 56.94    & 1          & 56.73    & 1          & 54.08    \\
%                & Seq           & 2.21       & 84.39    & 1.96       & 73.19    & 2.23       & 88.92    & 2.14       & 85.74    & 2.15       & 82.60    \\
%                & Tree          & 3.25       & 84.57    & 2.99       & 75.68    & 3.36       & 91.11    & 3.28       & 89.33    & 3.39       & 91.28    \\ \hline
% \end{tabular}
% }
% \caption{Speedup ratio and average acceptance length $\tau$ on GovReport, QMSum, Multi-News, LCC, and RepoBench-P when temperature $T=0$.}
% \label{tab:main_t=0}
% \end{table*}

% \begin{table*}[t]
% \centering
% \vspace{.2em}
% \scalebox{0.92}{\begin{tabular}{lccccccccccc}
% \toprule
% \textbf{Setting} & 
% \multicolumn{2}{c}{\textbf{GovReport}} & \multicolumn{2}{c}{\textbf{QMSum}} & 
% \multicolumn{2}{c}{\textbf{Multi-News}} & \multicolumn{2}{c}{\textbf{LCC}} & 
% \multicolumn{2}{c}{\textbf{RepoBench-P}} \\
% \cmidrule(lr){2-3} \cmidrule(lr){4-5} \cmidrule(lr){6-7} \cmidrule(lr){8-9} \cmidrule(lr){10-11}
% & $\tau$ & Tokens/s & $\tau$ & Tokens/s & $\tau$ & Tokens/s & $\tau$ & Tokens/s & $\tau$ & Tokens/s \\
% \midrule

% \multicolumn{11}{l}{\textbf{Vicuna-7B}} \\ \cmidrule(lr){1-1}
% \rowcolor{blue!8} Vanilla HF    & 1.00 & 25.25 & 1.00 & 18.12 & 1.00 & 27.29 & 1.00 & 25.25 & 1.00 & 19.18 \\
% \rowcolor{blue!8} Vanilla FA    & 1.00 & 45.76 & 1.00 & 43.68 & 1.00 & 55.99 & 1.00 & 54.07 & 1.00 & 46.61 \\
% \rowcolor{blue!8} MagicDec      & 2.23 & 41.68 & 2.29 & 42.91 & 2.31 & 44.82 & 2.52 & 46.96 & 2.57 & 48.75 \\
% \rowcolor{blue!8} Tree          & \textbf{3.57} & \textbf{102.23} & \textbf{3.14} & \textbf{88.87} & \textbf{3.51} & \textbf{100.55} & \textbf{3.73} & \textbf{107.30} & \textbf{3.86} & \textbf{110.76} \\

% \multicolumn{11}{l}{\textbf{Vicuna-13B}} \\ \cmidrule(lr){1-1}
% \rowcolor{purple!8} Vanilla HF  & 1.00 & 17.25 & 1.00 & 11.86 & 1.00 & 18.81 & 1.00 & 17.25 & 1.00 & 13.44 \\
% \rowcolor{purple!8} Vanilla FA  & 1.00 & 28.52 & 1.00 & 27.43 & 1.00 & 35.01 & 1.00 & 33.87 & 1.00 & 29.14 \\
% \rowcolor{purple!8} MagicDec    & 2.95 & 38.24 & 2.87 & 37.15 & 2.97 & 39.47 & 2.96 & 38.40 & 2.94 & 36.66 \\
% \rowcolor{purple!8} Tree        & \textbf{3.31} & \textbf{71.08} & \textbf{2.76} & \textbf{57.15} & \textbf{3.44} & \textbf{78.20} & \textbf{3.57} & \textbf{81.00} & \textbf{3.59} & \textbf{77.22} \\

% \multicolumn{11}{l}{\textbf{LongChat-7B}} \\ \cmidrule(lr){1-1}
% \rowcolor{green!8} Vanilla HF   & 1.00 & 25.27 & 1.00 & 14.11 & 1.00 & 27.66 & 1.00 & 25.27 & 1.00 & 17.02 \\
% \rowcolor{green!8} Vanilla FA   & 1.00 & 42.14 & 1.00 & 36.87 & 1.00 & 50.19 & 1.00 & 54.17 & 1.00 & 42.69 \\
% \rowcolor{green!8} MagicDec     & 2.26 & 41.90 & 2.20 & 40.82 & 2.32 & 43.94 & 2.77 & 51.73 & 2.57 & 44.13 \\
% \rowcolor{green!8} Tree         & \textbf{3.59} & \textbf{101.43} & \textbf{3.06} & \textbf{85.23} & \textbf{3.41} & \textbf{97.93} & \textbf{4.21} & \textbf{122.30} & \textbf{4.03} & \textbf{115.27} \\

% \multicolumn{11}{l}{\textbf{LongChat-13B}} \\ \cmidrule(lr){1-1}
% \rowcolor{yellow!8} Vanilla HF  & 1.00 & 17.72 & 1.00 & 12.08 & 1.00 & 18.74 & 1.00 & 17.72 & 1.00 & 13.85 \\
% \rowcolor{yellow!8} Vanilla FA  & 1.00 & 28.56 & 1.00 & 27.18 & 1.00 & 35.37 & 1.00 & 34.58 & 1.00 & 29.74 \\
% \rowcolor{yellow!8} MagicDec    & 2.40 & 31.37 & 2.38 & 30.84 & 2.43 & 32.58 & 2.68 & 35.77 & 2.85 & 35.67 \\
% \rowcolor{yellow!8} Tree        & \textbf{3.58} & \textbf{76.26} & \textbf{3.15} & \textbf{64.41} & \textbf{3.50} & \textbf{80.48} & \textbf{4.01} & \textbf{90.92} & \textbf{4.46} & \textbf{96.96} \\

% \multicolumn{11}{l}{\textbf{LLaMA3-8B}} \\ \cmidrule(lr){1-1}
% \rowcolor{orange!8} Vanilla HF  & 1.00 & 21.59 & 1.00 & 18.67 & 1.00 & 29.91 & 1.00 & 29.48 & 1.00 & 22.77 \\
% \rowcolor{orange!8} Vanilla FA  & 1.00 & 53.14 & 1.00 & 51.22 & 1.00 & 56.94 & 1.00 & 56.73 & 1.00 & 54.08 \\
% \rowcolor{orange!8} MagicDec    & 2.04 & 36.14 & 2.00 & 35.78 & 2.33 & 39.57 & 2.65 & 46.95 & 2.61 & 44.39 \\
% \rowcolor{orange!8} Tree        & \textbf{3.25} & \textbf{84.57} & \textbf{2.99} & \textbf{75.68} & \textbf{3.36} & \textbf{91.11} & \textbf{3.28} & \textbf{89.33} & \textbf{3.39} & \textbf{91.28} \\

% \bottomrule
% \end{tabular}}
% \caption{Speedup ratio ($\tau$) and decoding speed (Tokens/s) across different models and settings. All results are computed at $T=0$.}
% \label{tab:final_table}
% \end{table*}

% \begin{table*}[t]
% \centering
% \vspace{.2em}
% \scalebox{0.7}{
% \begin{tabular}{ccccccccccccccccc}
% \toprule
% \multirow{2}{*}{\textbf{Setting}} &
% \multicolumn{3}{c}{\textbf{GovReport}} &
% \multicolumn{3}{c}{\textbf{QMSum}} &
% \multicolumn{3}{c}{\textbf{Multi-News}} &
% \multicolumn{3}{c}{\textbf{LCC}} &
% \multicolumn{3}{c}{\textbf{Repo-P}} \\
% \cmidrule(lr){2-4} \cmidrule(lr){5-7} \cmidrule(lr){8-10} \cmidrule(lr){11-13} \cmidrule(lr){14-16}
% & $\tau$ & Tokens/s & Speedup
% & $\tau$ & Tokens/s & Speedup
% & $\tau$ & Tokens/s & Speedup
% & $\tau$ & Tokens/s & Speedup
% & $\tau$ & Tokens/s & Speedup \\
% \midrule

% \multicolumn{16}{l}{\textbf{Vicuna-7B}} \\ \cmidrule(lr){1-1}
% % \rowcolor{blue!8}
% Vanilla HF
% & 1.00 & 25.25 & -
% & 1.00 & 18.12 & -
% & 1.00 & 27.29 & -
% & 1.00 & 25.25 & -
% & 1.00 & 19.18 & - \\
% % \rowcolor{blue!8}
% Vanilla FA
% & 1.00 & 45.76 & 1.00
% & 1.00 & 43.68 & 1.00
% & 1.00 & 55.99 & 1.00
% & 1.00 & 54.07 & 1.00
% & 1.00 & 46.61 & 1.00 \\
% % \rowcolor{blue!8}
% MagicDec
% & 2.23 & 41.68 & 0.91
% & 2.29 & 42.91 & 0.98
% & 2.31 & 44.82 & 0.80
% & 2.52 & 46.96 & 0.87
% & 2.57 & 48.75 & 1.05 \\
% % \rowcolor{blue!8}
% LongSpec
% & \textbf{3.57} & \textbf{102.23} & \textbf{2.23}
% & \textbf{3.14} & \textbf{88.87} & \textbf{2.04}
% & \textbf{3.51} & \textbf{100.55} & \textbf{1.80}
% & \textbf{3.73} & \textbf{107.30} & \textbf{1.99}
% & \textbf{3.86} & \textbf{110.76} & \textbf{2.38} \\
% \midrule

% \multicolumn{16}{l}{\textbf{Vicuna-13B}} \\ \cmidrule(lr){1-1}
% % \rowcolor{purple!8}
% Vanilla HF
% & 1.00 & 17.25 & -
% & 1.00 & 11.86 & -
% & 1.00 & 18.81 & -
% & 1.00 & 17.25 & -
% & 1.00 & 13.44 & - \\
% % \rowcolor{purple!8}
% Vanilla FA
% & 1.00 & 28.52 & 1.00
% & 1.00 & 27.43 & 1.00
% & 1.00 & 35.01 & 1.00
% & 1.00 & 33.87 & 1.00
% & 1.00 & 29.14 & 1.00 \\
% % \rowcolor{purple!8}
% MagicDec
% & 2.95 & 38.24 & 1.34
% & 2.87 & 37.15 & 1.35
% & 2.97 & 39.47 & 1.13
% & 2.96 & 38.40 & 1.13
% & 2.94 & 36.66 & 1.26 \\
% % \rowcolor{purple!8}
% LongSpec
% & \textbf{3.31} & \textbf{71.08} & \textbf{2.49}
% & \textbf{2.76} & \textbf{57.15} & \textbf{2.08}
% & \textbf{3.44} & \textbf{78.20} & \textbf{2.23}
% & \textbf{3.57} & \textbf{81.00} & \textbf{2.39}
% & \textbf{3.59} & \textbf{77.22} & \textbf{2.65} \\
% \midrule

% \multicolumn{16}{l}{\textbf{LongChat-7B}} \\ \cmidrule(lr){1-1}
% % \rowcolor{green!8}
% Vanilla HF
% & 1.00 & 25.27 & -
% & 1.00 & 14.11 & -
% & 1.00 & 27.66 & -
% & 1.00 & 25.27 & -
% & 1.00 & 17.02 & - \\
% % \rowcolor{green!8}
% Vanilla FA
% & 1.00 & 42.14 & 1.00
% & 1.00 & 36.87 & 1.00
% & 1.00 & 50.19 & 1.00
% & 1.00 & 54.17 & 1.00
% & 1.00 & 42.69 & 1.00 \\
% % \rowcolor{green!8}
% MagicDec
% & 2.26 & 41.90 & 0.99
% & 2.20 & 40.82 & 1.11
% & 2.32 & 43.94 & 0.88
% & 2.77 & 51.73 & 0.96
% & 2.57 & 44.13 & 1.03 \\
% % \rowcolor{green!8}
% LongSpec
% & \textbf{3.59} & \textbf{101.43} & \textbf{2.41}
% & \textbf{3.06} & \textbf{85.23} & \textbf{2.31}
% & \textbf{3.41} & \textbf{97.93} & \textbf{1.95}
% & \textbf{4.21} & \textbf{122.30} & \textbf{2.26}
% & \textbf{4.03} & \textbf{115.27} & \textbf{2.70} \\
% \midrule

% \multicolumn{16}{l}{\textbf{LongChat-13B}} \\ \cmidrule(lr){1-1}
% % \rowcolor{yellow!8}
% Vanilla HF
% & 1.00 & 17.72 & -
% & 1.00 & 12.08 & -
% & 1.00 & 18.74 & -
% & 1.00 & 17.72 & -
% & 1.00 & 13.85 & - \\
% % \rowcolor{yellow!8}
% Vanilla FA
% & 1.00 & 28.56 & 1.00
% & 1.00 & 27.18 & 1.00
% & 1.00 & 35.37 & 1.00
% & 1.00 & 34.58 & 1.00
% & 1.00 & 29.74 & 1.00 \\
% % \rowcolor{yellow!8}
% MagicDec
% & 2.40 & 31.37 & 1.10
% & 2.38 & 30.84 & 1.13
% & 2.43 & 32.58 & 0.92
% & 2.68 & 35.77 & 1.03
% & 2.85 & 35.67 & 1.20 \\
% % \rowcolor{yellow!8}
% LongSpec
% & \textbf{3.58} & \textbf{76.26} & \textbf{2.67}
% & \textbf{3.15} & \textbf{64.41} & \textbf{2.37}
% & \textbf{3.50} & \textbf{80.48} & \textbf{2.28}
% & \textbf{4.01} & \textbf{90.92} & \textbf{2.63}
% & \textbf{4.46} & \textbf{96.96} & \textbf{3.26} \\
% \midrule

% \multicolumn{16}{l}{\textbf{LLaMA3-8B}} \\ \cmidrule(lr){1-1}
% % \rowcolor{orange!8}
% Vanilla HF
% & 1.00 & 21.59 & -
% & 1.00 & 18.67 & -
% & 1.00 & 29.91 & -
% & 1.00 & 29.48 & -
% & 1.00 & 22.77 & - \\
% % \rowcolor{orange!8}
% Vanilla FA
% & 1.00 & 53.14 & 1.00
% & 1.00 & 51.22 & 1.00
% & 1.00 & 56.94 & 1.00
% & 1.00 & 56.73 & 1.00
% & 1.00 & 54.08 & 1.00 \\
% % \rowcolor{orange!8}
% MagicDec
% & 2.04 & 36.14 & 0.68
% & 2.00 & 35.78 & 0.70
% & 2.33 & 39.57 & 0.70
% & 2.65 & 46.95 & 0.83
% & 2.61 & 44.39 & 0.82 \\
% % \rowcolor{orange!8}
% LongSpec
% & \textbf{3.25} & \textbf{84.57} & \textbf{1.59}
% & \textbf{2.99} & \textbf{75.68} & \textbf{1.48}
% & \textbf{3.36} & \textbf{91.11} & \textbf{1.60}
% & \textbf{3.28} & \textbf{89.33} & \textbf{1.57}
% & \textbf{3.39} & \textbf{91.28} & \textbf{1.69} \\

% \bottomrule
% \end{tabular}
% }
% \caption{Speedup ratio ($\tau$) and decoding speed (Tokens/s) across different models and settings. All results are computed at $T=0$.}
% \label{tab:final_table}
% \end{table*}

\begin{table*}[t]
\centering
\caption{Mean accepted length ($\tau$), decoding speed (tokens/s), and speedups across different models and settings. Specifically, ``Vanilla HF'' refers to HuggingFace’s PyTorch-based attention implementation, while ``Vanilla FA'' employs \texttt{Flash\_Decoding}. The speedup statistic calculates the acceleration ratio relative to the Vanilla HF method. All results are computed at $T=0$.}
\label{tab:final_table}
\vspace{.2em}
\scalebox{0.7}{
\begin{tabular}{c c c c c c c c c c c c c c c c c}
\toprule
 & \multirow{2}{*}{\textbf{Setting}} 
& \multicolumn{3}{c}{\textbf{GovReport}} 
& \multicolumn{3}{c}{\textbf{QMSum}} 
& \multicolumn{3}{c}{\textbf{Multi-News}} 
& \multicolumn{3}{c}{\textbf{LCC}} 
& \multicolumn{3}{c}{\textbf{RepoBench-P}} \\
\cmidrule(lr){3-5} \cmidrule(lr){6-8} \cmidrule(lr){9-11} \cmidrule(lr){12-14} \cmidrule(lr){15-17}
& 
& $\tau$ & Tokens/s & Speedup
& $\tau$ & Tokens/s & Speedup
& $\tau$ & Tokens/s & Speedup
& $\tau$ & Tokens/s & Speedup
& $\tau$ & Tokens/s & Speedup \\
\midrule

% Vicuna-7B
\multirow{4}{*}{\rotatebox{90}{V-7B}}
& Vanilla HF  
& 1.00 & 25.25 & -
& 1.00 & 18.12 & -
& 1.00 & 27.29 & -
& 1.00 & 25.25 & -
& 1.00 & 19.18 & - \\

& Vanilla FA  
& 1.00 & 45.76 & 1.00$\times$
& 1.00 & 43.68 & 1.00$\times$
& 1.00 & 55.99 & 1.00$\times$
& 1.00 & 54.07 & 1.00$\times$
& 1.00 & 46.61 & 1.00$\times$ \\

& MagicDec    
& 2.23 & 41.68 & 0.91$\times$
& 2.29 & 42.91 & 0.98$\times$
& 2.31 & 44.82 & 0.80$\times$
& 2.52 & 46.96 & 0.87$\times$
& 2.57 & 48.75 & 1.05$\times$ \\

& \textbf{LongSpec}  
& \textbf{3.57} & \textbf{102.23} & \textbf{2.23}$\times$
& \textbf{3.14} & \textbf{88.87}  & \textbf{2.04}$\times$
& \textbf{3.51} & \textbf{100.55} & \textbf{1.80}$\times$
& \textbf{3.73} & \textbf{107.30} & \textbf{1.99}$\times$
& \textbf{3.86} & \textbf{110.76} & \textbf{2.38}$\times$ \\
\midrule

% Vicuna-13B
\multirow{4}{*}{\rotatebox{90}{V-13B}}
& Vanilla HF  
& 1.00 & 17.25 & -
& 1.00 & 11.86 & -
& 1.00 & 18.81 & -
& 1.00 & 17.25 & -
& 1.00 & 13.44 & - \\

& Vanilla FA  
& 1.00 & 28.52 & 1.00$\times$
& 1.00 & 27.43 & 1.00$\times$
& 1.00 & 35.01 & 1.00$\times$
& 1.00 & 33.87 & 1.00$\times$
& 1.00 & 29.14 & 1.00$\times$ \\

& MagicDec    
& 2.95 & 38.24 & 1.34$\times$
& 2.87 & 37.15 & 1.35$\times$
& 2.97 & 39.47 & 1.13$\times$
& 2.96 & 38.40 & 1.13$\times$
& 2.94 & 36.66 & 1.26$\times$ \\

& \textbf{LongSpec}  
& \textbf{3.31} & \textbf{71.08} & \textbf{2.49}$\times$
& \textbf{2.76} & \textbf{57.15} & \textbf{2.08}$\times$
& \textbf{3.44} & \textbf{78.20} & \textbf{2.23}$\times$
& \textbf{3.57} & \textbf{81.00} & \textbf{2.39}$\times$
& \textbf{3.59} & \textbf{77.22} & \textbf{2.65}$\times$ \\
\midrule

% LongChat-7B
\multirow{4}{*}{\rotatebox{90}{LC-7B}}
& Vanilla HF  
& 1.00 & 25.27 & -
& 1.00 & 14.11 & -
& 1.00 & 27.66 & -
& 1.00 & 25.27 & -
& 1.00 & 17.02 & - \\

& Vanilla FA  
& 1.00 & 42.14 & 1.00$\times$
& 1.00 & 36.87 & 1.00$\times$
& 1.00 & 50.19 & 1.00$\times$
& 1.00 & 54.17 & 1.00$\times$
& 1.00 & 42.69 & 1.00$\times$ \\

& MagicDec    
& 2.26 & 41.90 & 0.99$\times$
& 2.20 & 40.82 & 1.11$\times$
& 2.32 & 43.94 & 0.88$\times$
& 2.77 & 51.73 & 0.96$\times$
& 2.57 & 44.13 & 1.03$\times$ \\

& \textbf{LongSpec}  
& \textbf{3.59} & \textbf{101.43} & \textbf{2.41}$\times$
& \textbf{3.06} & \textbf{85.23} & \textbf{2.31}$\times$
& \textbf{3.41} & \textbf{97.93} & \textbf{1.95}$\times$
& \textbf{4.21} & \textbf{122.30} & \textbf{2.26}$\times$
& \textbf{4.03} & \textbf{115.27} & \textbf{2.70}$\times$ \\
\midrule

% LongChat-13B
\multirow{4}{*}{\rotatebox{90}{LC-13B}}
& Vanilla HF  
& 1.00 & 17.72 & -
& 1.00 & 12.08 & -
& 1.00 & 18.74 & -
& 1.00 & 17.72 & -
& 1.00 & 13.85 & - \\

& Vanilla FA  
& 1.00 & 28.56 & 1.00$\times$
& 1.00 & 27.18 & 1.00$\times$
& 1.00 & 35.37 & 1.00$\times$
& 1.00 & 34.58 & 1.00$\times$
& 1.00 & 29.74 & 1.00$\times$ \\

& MagicDec    
& 2.40 & 31.37 & 1.10$\times$
& 2.38 & 30.84 & 1.13$\times$
& 2.43 & 32.58 & 0.92$\times$
& 2.68 & 35.77 & 1.03$\times$
& 2.85 & 35.67 & 1.20$\times$ \\

& \textbf{LongSpec}  
& \textbf{3.58} & \textbf{76.26} & \textbf{2.67}$\times$
& \textbf{3.15} & \textbf{64.41} & \textbf{2.37}$\times$
& \textbf{3.50} & \textbf{80.48} & \textbf{2.28}$\times$
& \textbf{4.01} & \textbf{90.92} & \textbf{2.63}$\times$
& \textbf{4.46} & \textbf{96.96} & \textbf{3.26}$\times$ \\
\midrule

% LLaMA3-8B
\multirow{4}{*}{\rotatebox{90}{L-8B}}
& Vanilla HF  
& 1.00 & 21.59 & -
& 1.00 & 18.67 & -
& 1.00 & 29.91 & -
& 1.00 & 29.48 & -
& 1.00 & 22.77 & - \\

& Vanilla FA  
& 1.00 & 53.14 & 1.00$\times$
& 1.00 & 51.22 & 1.00$\times$
& 1.00 & 56.94 & 1.00$\times$
& 1.00 & 56.73 & 1.00$\times$
& 1.00 & 54.08 & 1.00$\times$ \\

& MagicDec    
& 2.04 & 36.14 & 0.68$\times$
& 2.00 & 35.78 & 0.70$\times$
& 2.33 & 39.57 & 0.70$\times$
& 2.65 & 46.95 & 0.83$\times$
& 2.61 & 44.39 & 0.82$\times$ \\

& \textbf{LongSpec}  
& \textbf{3.25} & \textbf{84.57} & \textbf{1.59}$\times$
& \textbf{2.99} & \textbf{75.68} & \textbf{1.48}$\times$
& \textbf{3.36} & \textbf{91.11} & \textbf{1.60}$\times$
& \textbf{3.28} & \textbf{89.33} & \textbf{1.57}$\times$
& \textbf{3.39} & \textbf{91.28} & \textbf{1.69}$\times$ \\

\bottomrule
\end{tabular}
}
\vspace{-.3cm}
\end{table*}

\begin{figure*}
    \centering
    \includegraphics[width=1\linewidth]{Figure/T1.pdf}
    \vspace{-.7cm}
    \caption{Decoding speed (tokens/s) across different models and settings. All results are computed at $T=1$. The letters G, Q, M, L, and R on the horizontal axis represent the dataset GovReport, QMSum, Multi-News, LCC, and RepoBench-P respectively.}
    \label{fig:final_fig}
    \vspace{-.3cm}
\end{figure*}



\section{Experiments}

\subsection{Settings}

\textbf{Target and draft models.} 
We select four widely-used long-context LLMs, Vicuna~(including 7B and 13B)~\cite{chiang2023vicuna}, LongChat~(including 7B and 13B)~\cite{li2023longchat}, LLaMA-3.1-8B-Instruct~\cite{dubey2024llama}, and QwQ-32B~\cite{qwen2024qwq}, as target models.
In order to make the draft model and target model more compatible, our draft model is consistent with the target model in various parameters such as the number of KV heads. 

\textbf{Training Process.} 
We first train our draft model with Achor-Offest Indices on the SlimPajama-6B pretraining dataset~\cite{cerebras2023slimpajama}. 
The random offset is set as a random integer from 0 to 15k for Vicuna models and LongChat-7B, and 0 to 30k for the other three models because they have longer maximum context length.
Then we train our model on a small subset of the Prolong-64k long-context dataset~\cite{gao2024train} in order to gain the ability to handle long texts. 
Finally, we finetune our model on a self-built long-context supervised-finetuning~(SFT) dataset to further improve the model performance.
The position index of the last two stages is the vanilla indexing policy because the training data is sufficiently long.
We apply flash noisy training during all three stages to mitigate the training and inference inconsistency, the extra overhead of flash noisy training is negligible.
Standard cross-entropy is used to optimize the draft model while the parameters of the target model are kept frozen. To mitigate the VRAM peak caused by the computation of the logits, we use a fused-linear-and-cross-entropy loss implemented by the Liger Kernel~\cite{hsu2024liger}, which computes the LM head and the softmax function together and can greatly alleviate this problem. More details on model training can be found in Appendix~\ref{appendix:training_details}.



\textbf{Test Benchmarks.}
We select tasks from the LongBench benchmark~\cite{bai2024longbench} that involve generating longer outputs, because tasks with shorter outputs, such as document-QA, make it challenging to measure the speedup ratio fairly with speculative decoding. 
Specifically, we focus on long-document summarization and code completion tasks and conduct tests on five datasets: GovReport~\cite{huang2021efficient}, QMSum~\cite{zhong2021qmsum}, Multi-News~\cite{fabbri2019multi}, LCC~\cite{guo2023longcoder}, and RepoBench-P~\cite{liu2024repobench}. We test QwQ-32B on the famous reasoning dataset AIME24~\cite{numina2024aime}.

We compare our method with the original target model and MagicDec, a simple prototype of TriForce. 
To highlight the significance of \texttt{Flash\_Decoding} in long-context scenarios, we also present the performance of the original target model using both eager attention implemented by Huggingface and \texttt{Flash\_Decoding} for comparison.
To make a fair comparison, we also use \texttt{Flash\_Decoding} for baseline MagicDec.
The most important metric for speculative decoding is the \emph{walltime speedup ratio}, which is the actual test speedup ratio relative to vanilla autoregressive decoding. 
We also test the \emph{average acceptance length} $\tau$, \emph{i.e.}, the average number of tokens accepted per forward pass of the target LLM. 
% \emph{Acceptance rate} $\alpha$, the ratio of accepted to generated tokens during drafting, is taken into consideration as well. Following EAGLE~\cite{li2024eagle}, when measuring this metric, we utilize chain drafts without tree attention to assess the acceptance rate per location more precisely. Specifically, for the $n$-th token in the draft tokens, the acceptance rate is denoted as $n\text{-}\alpha$.

\subsection{Main Results}

Table~\ref{tab:final_table} and Figure~\ref{fig:final_fig} show the decoding speeds and mean accept lengths across the five evaluated datasets at $T=0$ and $T=1$ respectively. 
Our proposed method significantly outperforms all other approaches on both summarization tasks and code completion tasks. When $T=0$, on summarization tasks, our method can achieve a mean accepted length of around 3.5 and a speedup of up to 2.67$\times$; and on code completion tasks, our method can achieve a mean accepted length of around 4 and a speedup of up to 3.26$\times$. This highlights the robustness and generalizability of our speculative decoding approach, particularly in long-text generation tasks. At $T=1$, our method's performance achieves around 2.5$\times$ speedup, maintaining a substantial lead over MagicDec. This indicates that our approach is robust across different temperature settings, further validating its soundness and efficiency.

While MagicDec demonstrates competitive acceptance rates with LongSpec, its speedup is noticeably lower in our experiments. This is because MagicDec is primarily designed for scenarios with large batch sizes and tensor parallelism. 
In low-batch-size settings, its draft model leverages all parameters of the target model with sparse KV Cache becomes excessively heavy. 
This design choice leads to inefficiencies, as the draft model's computational overhead outweighs its speculative benefits. 
Our results reveal that MagicDec only achieves acceleration ratios~$>\!1$ on partial datasets when using a guess length $\gamma \!=\! 2$ and consistently exhibits negative acceleration around 0.7$\times$ when $\gamma\!\geq\!3$, further underscoring the limitations of this method in such configurations.
% \looseness=-1

Lastly, we can find attention implementation plays a critical role in long-context speculative decoding performance. In our experiments, ``Vanilla HF'' refers to HuggingFace’s attention implementation, while ``Vanilla FA'' employs \texttt{Flash\_Decoding}. The latter demonstrates nearly a $2\times$ speedup over the former, even as a standalone component, and our method can achieve up to $6\times$ speedup over HF Attention on code completion datasets. 
This result underscores the necessity for speculative decoding methods to be compatible with optimized attention mechanisms like \texttt{Flash\_Decoding}, especially in long-text settings. Our hybrid tree attention approach achieves this compatibility, allowing us to fully leverage the advantages of \texttt{Flash\_Decoding} and further speedup.

\subsection{Ablation Studies}

\textbf{Anchor-Offset Indices.} 
The experimental results demonstrate the significant benefits of incorporating the Anchor-Offset Indices. Figure~\ref{fig:ablation_1} shows that pretrained with Anchor-Offset Indices achieve a lower initial loss and final loss compared to those trained without it when training over the real long-context dataset. 
Notably, the initalization with Anchor-Offset Indices reaches the same loss level $3.93\times$ faster than its counterpart. 
Table~\ref{tab:ablation_1} further highlights the performance improvements across two datasets, a summary dataset Multi-News, and a code completion dataset RepoBench-P. 
Models with Anchor-Offset Indices exhibit faster output speed and larger average acceptance length $\tau$. These results underscore the effectiveness of Anchor-Offset Indices in enhancing both training efficiency and model performance.

\begin{table}[t]
\vspace{-0.cm}
    \centering
    \caption{Performance comparison with and without Anchor-Offset Indices on the Multi-News and RepoBench-P datasets. Models with Anchor-Offset Indices achieve higher output speed and larger accept length, highlighting its efficiency and effectiveness.}
    \label{tab:ablation_1}
    \vspace{.1cm}
    \scalebox{0.9}{
    \begin{tabular}{c c c c c}
        \toprule
        & \multicolumn{2}{c}{\textbf{Multi-News}} 
        & \multicolumn{2}{c}{\textbf{RepoBench-P}} \\
        \cmidrule(lr){2-3} \cmidrule(lr){4-5}
        & $\tau$ & Tokens/s
        & $\tau$ & Tokens/s \\
        \midrule
        w/o Anchor-Offset & 3.20 & 85.98 & 3.26 & 85.21\\
        w/ Anchor-Offset  & 3.36 & 91.11 & 3.39 & 91.28\\
        \bottomrule
    \end{tabular}
    }
\end{table}

\begin{figure}[t]
    \vspace{-.2cm}
    \centering
    \includegraphics[width=1\linewidth]{Figure/ablation_1.pdf}
    \vspace{-.7cm}
    \caption{Training loss curves on long-context data. 
     Pretrained models with Anchor-Offset Indices exhibit lower initial and final loss, and reach the same loss level 3.93$\times$ faster compared to models without Anchor-Offset Indices.}
    \label{fig:ablation_1}
    \vspace{-.1cm}
\end{figure}


\textbf{Hybrid Tree Attention.}
The results presented in Figure \ref{fig:ablation_2} highlight the effectiveness of the proposed Hybrid Tree Attention, which combines \texttt{Flash\_Decoding} with the Triton kernel \texttt{fused\_mask\_attn}. 
While the time spent on the draft model forward pass and the target model FFN computations remain comparable across the two methods, the hybrid approach exhibits a significant reduction in latency for the target model's attention layer (the yellow part). 
Specifically, the attention computation latency decreases from 49.92 ms in the HF implementation to 12.54 ms in the hybrid approach, resulting in an approximately 75\% improvement. 
The verification step time difference is minimal, further solidifying the conclusion that the primary performance gains stem from optimizing the attention mechanism.

\begin{figure}[t]
    \centering
    \includegraphics[width=1.0\linewidth]{Figure/runtime.pdf}
    \vspace{-.8cm}
    \caption{Latency breakdown for a single speculative decoding loop comparing the EAGLE implementation and the proposed Hybrid Tree Attention. Significant latency reduction is observed in the target model's attention layer (the yellow part) using our approach.}
    \label{fig:ablation_2}
    \vspace{-.cm}
\end{figure}


\begin{figure}[t]
    \centering
    \includegraphics[width=1\linewidth]{Figure/qwq.pdf}
    \vspace{-.5cm}
    \caption{Performance of our method on the QwQ-32B model with the AIME24 dataset, using a maximum output length of 32k tokens. The left plot shows the tokens generated per second, where our approach achieves 2.25$\times$ higher speed compared to the baseline. 
    The right plot shows the mean number of accepted tokens, where our method achieves an average of 3.82 mean accepted tokens.}
    \label{fig:qwq}
    \vspace{.1cm}
\end{figure}

\subsection{Long CoT Acceleration}

Long Chain-of-Thought (LongCoT) tasks have gained significant attention recently due to their ability to enable models to perform complex reasoning and problem-solving over extended outputs~\cite{qwen2024qwq, openai2024o1}. In these tasks, while the prefix input is often relatively short, the generated output can be extremely long, posing unique challenges in terms of efficiency and token acceptance. Our method is particularly well-suited for addressing these challenges, effectively handling scenarios with long outputs. It is worth mentioning that MagicDec is not suitable for such long-output scenarios because the initial inference stage of the LongCoT task is not the same as the traditional long-context task. In LongCoT tasks, where the prefix is relatively short, the draft model in MagicDec will completely degrade into the target model, failing to achieve acceleration.
\looseness=-1

We evaluate our method on the QwQ-32B model using the widely-used benchmark AIME24 dataset, with a maximum output length set to 32k tokens. 
The results, illustrated in Figure~\ref{fig:qwq}, demonstrate a significant improvement in both generation speed and mean accepted tokens. 
Specifically, our method achieved a generation rate of 42.63 tokens/s, 2.25$\times$ higher than the baseline's 18.92 tokens/s, and an average of 3.82 mean accepted tokens.
Notably, QwQ-32B with \textsc{LongSpec} achieves even lower latency than the vanilla 7B model with \texttt{Flash\_Decoding}, demonstrating that our method effectively accelerates the LongCoT model.
These findings not only highlight the effectiveness of our method in the LongCoT task but also provide new insights into lossless inference acceleration for the o1-like model.
We believe that speculative decoding will play a crucial role in accelerating this type of model in the future.


\subsection{Throughput}

The throughput results of Vicuna-7B on the RepoBench-P dataset show that \textsc{LongSpec} consistently outperforms both Vanilla and MagicDec across all batch sizes. At a batch size of 8, \textsc{LongSpec} achieves a throughput of 561.32 tokens/s, approximately 1.8$\times$ higher than MagicDec (310.58 tokens/s) and nearly 2$\times$ higher than Vanilla (286.96 tokens/s). MagicDec, designed with throughput optimization in mind, surpasses Vanilla as the batch size increases, reflecting its targeted improvements. However, \textsc{LongSpec} still sustains its advantage, maintaining superior throughput across all tested batch sizes.

\begin{figure}
    \centering
    \includegraphics[width=0.85\linewidth]{Figure/throughput.pdf}
    \vspace{-.2cm}
    \caption{Throughput comparison of Vanilla, MagicDec, and \textsc{LongSpec} on RepoBench-P using Vicuna-7B across different batch sizes. \textsc{LongSpec} shows superior throughput and scalability, outperforming both Vanilla and MagicDec in all batch sizes.}
    \label{fig:throughput}
    \vspace{-.3cm}
\end{figure}


\section{Conclusion}

In this paper, we propose \textsc{LongSpec}, a novel framework designed to enhance speculative decoding for long-context scenarios. Unlike previous speculative decoding methods that primarily focus on short-context settings, \textsc{LongSpec} directly addresses three key challenges: excessive memory overhead, inadequate training for large position indices, and inefficient tree attention computation. To mitigate memory constraints, we introduce an efficient draft model architecture that maintains a constant memory footprint by leveraging a combination of sliding window self-attention and cache-free cross-attention. To resolve the training limitations associated with short context data, we propose the Anchor-Offset Indices, ensuring that large positional indices are sufficiently trained even within short-sequence datasets. Finally, we introduce Hybrid Tree Attention, 
which efficiently integrates tree-based speculative decoding with \texttt{Flash\_Decoding}. Extensive experiments demonstrate the effectiveness of \textsc{LongSpec} in long-context understanding tasks and real-world long reasoning tasks. Our findings highlight the importance of designing speculative decoding methods specifically tailored for long-context settings and pave the way for future research in efficient large-scale language model inference.

\section*{Impact Statements}
This paper presents work whose goal is to advance the field of Machine Learning. There are many potential societal consequences of our work, none which we feel must be specifically highlighted here.

\bibliography{example_paper}
\bibliographystyle{icml2025}


%%%%%%%%%%%%%%%%%%%%%%%%%%%%%%%%%%%%%%%%%%%%%%%%%%%%%%%%%%%%%%%%%%%%%%%%%%%%%%%
%%%%%%%%%%%%%%%%%%%%%%%%%%%%%%%%%%%%%%%%%%%%%%%%%%%%%%%%%%%%%%%%%%%%%%%%%%%%%%%
% APPENDIX
%%%%%%%%%%%%%%%%%%%%%%%%%%%%%%%%%%%%%%%%%%%%%%%%%%%%%%%%%%%%%%%%%%%%%%%%%%%%%%%
%%%%%%%%%%%%%%%%%%%%%%%%%%%%%%%%%%%%%%%%%%%%%%%%%%%%%%%%%%%%%%%%%%%%%%%%%%%%%%%
\newpage
\appendix
\onecolumn

\section{Correctness for Attention Aggregation}
\label{appendix:attn_aggr}

Because the query matrix $Q$ can be decomposed into several rows, each representing a separate query $q$, we can only consider the output of each row's $q$ after calculating attention with KV. In this way, we can assume that the KV involved in the calculation has undergone the tree mask, which can simplify our proof. We only need to prove that the output $o$ obtained from each individual $q$ meets the requirements, which can indicate that the overall output $O$ of the entire matrix $Q$ also meets the requirements.

\begin{proposition}
    Denote the log-sum-exp of the merged attention as follows:
    \begin{equation*}
        \mathrm{LSE}_{\mathrm{merge}} = \log\Bigl(\exp\bigl(\mathrm{LSE}_{\mathrm{cache}}\bigr) \;+\; \exp\bigl(\mathrm{LSE}_{\mathrm{specs}}\bigr)\Bigr),
    \end{equation*}
    Then we can write the merged attention output in the following way:
    \begin{equation*}
    o_{\mathrm{merge}} = o_{\mathrm{cache}} \cdot \exp\bigl(\mathrm{LSE}_{\mathrm{cache}} - \mathrm{LSE}_{\mathrm{merge}}\bigr) + o_{\mathrm{specs}} \cdot\exp\bigl(\mathrm{LSE}_{\mathrm{specs}} - \mathrm{LSE}_{\mathrm{merge}}\bigr).
    \end{equation*}
\end{proposition}

\begin{proof}
    
A standard scaled dot-product attention for $ q $ (of size $d_{qk}$) attending to $ K_{\mathrm{merge}} $ and $ V_{\mathrm{merge}} $ (together of size $(M+N) \times d_{qk}$ and $(M+N) \times d_v$ respectively) can be written as:

\begin{equation*}
    o_{\mathrm{merge}} = \mha\left(q, K_{\mathrm{merge}}, V_{\mathrm{merge}}\right) =
    \sm\left(
      qK_{\mathrm{merge}}^\top/\sqrt{d_{qk}}
    \right) V_{\mathrm{merge}}.
\end{equation*}

Because $ K $ and $ V $ are formed by stacking $\left(K_{\mathrm{specs}}, K_{\mathrm{cache}}\right)$ and $\left(V_{\mathrm{specs}}, V_{\mathrm{cache}}\right)$, we split the logit matrix accordingly:

\begin{equation*}
    q K_{\mathrm{merge}}^\top / \sqrt{d_{qk}} = 
    \texttt{concat}\Bigl(
    \underbrace{
      q \, K_{\mathrm{cache}}^\top / \sqrt{d_{qk}}
    }_{\mathrm{sub-logits for history}}
    \;, \;
    \underbrace{
      q \, K_{\mathrm{specs}}^\top / \sqrt{d_{qk}}
    }_{\mathrm{sub-logits for new}}
    \Bigr).
\end{equation*}

Denote these sub-logit matrices as:
\[
Z_{\mathrm{cache}} \;=\; q \, K_{\mathrm{cache}}^\top / \sqrt{d_{qk}}
,\;
Z_{\mathrm{specs}} \;=\; q \, K_{\mathrm{specs}}^\top / \sqrt{d_{qk}}.
\]

Each row $i$ of $Z_{\mathrm{specs}}$ corresponds to the dot products between the $i$-th query in $q$ and all rows in $K_{\mathrm{specs}}$, while rows of $Z_{\mathrm{cache}}$ correspond to the same query but with $K_{\mathrm{cache}}$.

In order to combine partial attentions, we keep track of the log of the sum of exponentials of each sub-logit set. Concretely, define:

\begin{equation}
    \mathrm{LSE}_{\mathrm{cache}} = \log\left(\sum\nolimits_{j=1}^{N} \exp\left(Z_{\mathrm{cache}}^{(j)}\right)\right),
    \;
    \mathrm{LSE}_{\mathrm{specs}} = \log\left(\sum\nolimits_{j=1}^{M} \exp\left(Z_{\mathrm{specs}}^{(j)}\right)\right),
    \,
\end{equation}

where $Z_{\mathrm{specs}}^{(j)}$ denotes the logit for the $j$-th element, and similarly for $Z_{\mathrm{cache}}^{(j)}$.

% Then the standard scaled dot-product attention for $ q $ (of size $d_{qk}$) attending to $ K_{\mathrm{cache}} $ and $ V_{\mathrm{cache}} $ (together of size $N \times d_{qk}$ and $N \times d_v$ respectively) can be written as:

% \begin{equation}
%     \label{equ:history_attn}
%     o_{\mathrm{cache}} = \mha\left(q, K_{\mathrm{cache}}, V_{\mathrm{cache}}\right) =
%     \frac{\sum_{j=1}^{N} \exp\left(Z_{\mathrm{cache}}^{(j)}\right) V_{\mathrm{cache}}^{(j)}}{\exp\left(\mathrm{LSE}_{\mathrm{cache}}\right)}.
% \end{equation}

% Similarly, the standard scaled dot-product attention for $ q $ (of size $M \times d_{qk}$) attending to $ K_{\mathrm{specs}} $ and $ V_{\mathrm{specs}} $ (together of size $M \times d_{qk}$ and $M \times d_v$ respectively) can be written as:

% \begin{equation}
%     \label{equ:new_attn}
%     o_{\mathrm{specs}} = \mha\left(q, K_{\mathrm{specs}}, V_{\mathrm{specs}}\right) =
%     \frac{\sum_{j=1}^{M} \exp\left(Z_{\mathrm{specs}}^{(j)}\right) V_{\mathrm{specs}}^{(j)}}{\exp\left(\mathrm{LSE}_{\mathrm{specs}}\right)}.
% \end{equation}

Then $o_{\mathrm{cache}}$ and $o_{\mathrm{specs}}$ can be written as:
\begin{equation}
    \label{equ:split_attn}
    o_{\mathrm{cache}} = \frac{\sum_{j=1}^{N} \exp\left(Z_{\mathrm{cache}}^{(j)}\right) V_{\mathrm{cache}}^{(j)}}{\exp\left(\mathrm{LSE}_{\mathrm{cache}}\right)}, \;
    o_{\mathrm{specs}} = \frac{\sum_{j=1}^{M} \exp\left(Z_{\mathrm{specs}}^{(j)}\right) V_{\mathrm{specs}}^{(j)}}{\exp\left(\mathrm{LSE}_{\mathrm{specs}}\right)}.
\end{equation}

And the whole attention score can be written as:

\begin{equation}
    \label{equ:all_attn}
    o_{\mathrm{merge}} =
    \frac{\sum_{j=1}^{N} \exp\left(Z_{\mathrm{cache}}^{(j)}\right) V_{\mathrm{cache}}^{(j)} + \sum_{j=1}^{M} \exp\left(Z_{\mathrm{specs}}^{(j)}\right) V_{\mathrm{specs}}^{(j)}}{\exp\left(\mathrm{LSE}_{\mathrm{cache}}\right) + \exp\left(\mathrm{LSE}_{\mathrm{specs}}\right)}.
\end{equation}

By aggregating Equation \ref{equ:split_attn} into Equation \ref{equ:all_attn}, we can get the following equation:

\begin{equation}
    o_{\mathrm{merge}} = o_{\mathrm{cache}} \cdot\exp\bigl(\mathrm{LSE}_{\mathrm{cache}} - \mathrm{LSE}_{\mathrm{merge}}\bigr)
     + o_{\mathrm{specs}} \cdot \exp\bigl(\mathrm{LSE}_{\mathrm{specs}} - \mathrm{LSE}_{\mathrm{merge}}\bigr).
\end{equation}

\end{proof}

\section{Experiments Details}
\label{appendix:training_details}

All models are trained using eight A100 80GB GPUs. For the 7B, 8B, and 13B target models trained on short-context data, we employ \textsc{LongSpec} with ZeRO-1~\cite{rasley2020deepspeed}. For the 7B, 8B, and 13B models trained on long-context data, as well as for all settings of the 33B target models, we utilize ZeRO-3.  

For the SlimPajama-6B dataset, we configure the batch size (including accumulation) to 2048, set the maximum learning rate to 5e-4 with a cosine learning rate schedule~\cite{loshchilov2017sgdr}, and optimize the draft model using AdamW~\cite{kingma2015adam}. When training on long-context datasets, we adopt a batch size of 256 and a maximum learning rate of 5e-6. The draft model is trained for only one epoch on all datasets.

It is important to note that the primary computational cost arises from forwarding the target model to obtain the KV cache. Recently, some companies have introduced a service known as context caching~\cite{deepseek2024contextcaching, gemini2024contextcaching}, which involves storing large volumes of KV cache. Consequently, in real-world deployment, these pre-stored KV caches can be directly utilized as training data, significantly accelerating the training process.  

For the tree decoding of \textsc{LongSpec}, we employ dynamic beam search to construct the tree. Previous studies have shown that beam search, while achieving high acceptance rates, suffers from slow processing speed in speculative decoding~\cite{du2024glide}. Our research identifies that this slowdown is primarily caused by KV cache movement. In traditional beam search, nodes that do not fall within the top-$k$ likelihood are discarded, a step that necessitates KV cache movement. However, in speculative decoding, discarding these nodes is unnecessary, as draft sequences are not required to maintain uniform lengths. Instead, we can simply halt the computation of descendant nodes for low-likelihood branches without removing them entirely. By adopting this approach, beam search attains strong performance without excessive computational overhead. In our experiments, the beam width is set to $[4, 16, 16, 16, 16]$ for each speculation step. All inference experiments in this study are conducted using float16 precision on a single A100 80GB GPU.  

\section{Experiments}
\label{sec:Experiments} 

We conduct several experiments across different problem settings to assess the efficiency of our proposed method. Detailed descriptions of the experimental settings are provided in \cref{sec:apendix_experiments}.
%We conduct experiments on optimizing PINNs for convection, wave PDEs, and a reaction ODE. 
%These equations have been studied in previous works investigating difficulties in training PINNs; we use the formulations in \citet{krishnapriyan2021characterizing, wang2022when} for our experiments. 
%The coefficient settings we use for these equations are considered challenging in the literature \cite{krishnapriyan2021characterizing, wang2022when}.
%\cref{sec:problem_setup_additional} contains additional details.

%We compare the performance of Adam, \lbfgs{}, and \al{} on training PINNs for all three classes of PDEs. 
%For Adam, we tune the learning rate by a grid search on $\{10^{-5}, 10^{-4}, 10^{-3}, 10^{-2}, 10^{-1}\}$.
%For \lbfgs, we use the default learning rate $1.0$, memory size $100$, and strong Wolfe line search.
%For \al, we tune the learning rate for Adam as before, and also vary the switch from Adam to \lbfgs{} (after 1000, 11000, 31000 iterations).
%These correspond to \al{} (1k), \al{} (11k), and \al{} (31k) in our figures.
%All three methods are run for a total of 41000 iterations.

%We use multilayer perceptrons (MLPs) with tanh activations and three hidden layers. These MLPs have widths 50, 100, 200, or 400.
%We initialize these networks with the Xavier normal initialization \cite{glorot2010understanding} and all biases equal to zero.
%Each combination of PDE, optimizer, and MLP architecture is run with 5 random seeds.

%We use 10000 residual points randomly sampled from a $255 \times 100$ grid on the interior of the problem domain. 
%We use 257 equally spaced points for the initial conditions and 101 equally spaced points for each boundary condition.

%We assess the discrepancy between the PINN solution and the ground truth using $\ell_2$ relative error (L2RE), a standard metric in the PINN literature. Let $y = (y_i)_{i = 1}^n$ be the PINN prediction and $y' = (y'_i)_{i = 1}^n$ the ground truth. Define
%\begin{align*}
%    \mathrm{L2RE} = \sqrt{\frac{\sum_{i = 1}^n (y_i - y'_i)^2}{\sum_{i = 1}^n y'^2_i}} = \sqrt{\frac{\|y - y'\|_2^2}{\|y'\|_2^2}}.
%\end{align*}
%We compute the L2RE using all points in the $255 \times 100$ grid on the interior of the problem domain, along with the 257 and 101 points used for the initial and boundary conditions.

%We develop our experiments in PyTorch 2.0.0 \cite{paszke2019pytorch} with Python 3.10.12.
%Each experiment is run on a single NVIDIA Titan V GPU using CUDA 11.8.
%The code for our experiments is available at \href{https://github.com/pratikrathore8/opt_for_pinns}{https://github.com/pratikrathore8/opt\_for\_pinns}.


\subsection{2D Allen Cahn Equation}
\begin{figure*}[t]
    \centering
    \includegraphics[scale=0.38]{figs/Burgers_operator.pdf}
    \caption{1D Burgers' Equation (Operator Learning): Steady-state solutions for different initializations $u_0$ under varying viscosity $\varepsilon$: (a) $\varepsilon = 0.5$, (b) $\varepsilon = 0.1$, (c) $\varepsilon = 0.05$. The results demonstrate that all final test solutions converge to the correct steady-state solution. (d) Illustration of the evolution of a test initialization $u_0$ following homotopy dynamics. The number of residual points is $\nres = 128$.}
    \label{fig:Burgers_result}
\end{figure*}
First, we consider the following time-dependent problem:
\begin{align}
& u_t = \varepsilon^2 \Delta u - u(u^2 - 1), \quad (x, y) \in [-1, 1] \times [-1, 1] \nonumber \\
& u(x, y, 0) = - \sin(\pi x) \sin(\pi y) \label{eq.hom_2D_AC}\\
& u(-1, y, t) = u(1, y, t) = u(x, -1, t) = u(x, 1, t) = 0. \nonumber
\end{align}
We aim to find the steady-state solution for this equation with $\varepsilon = 0.05$ and define the homotopy as:
\begin{equation}
    H(u, s, \varepsilon) = (1 - s)\left(\varepsilon(s)^2 \Delta u - u(u^2 - 1)\right) + s(u - u_0),\nonumber
\end{equation}
where $s \in [0, 1]$. Specifically, when $s = 1$, the initial condition $u_0$ is automatically satisfied, and when $s = 0$, it recovers the steady-state problem. The function $\varepsilon(s)$ is given by
\begin{equation}
\varepsilon(s) = 
\left\{\begin{array}{l}
s, \quad s \in [0.05, 1], \\
0.05, \quad s \in [0, 0.05].
\end{array}\right.\label{eq:epsilon_t}
\end{equation}

Here, $\varepsilon(s)$ varies with $s$ during the first half of the evolution. Once $\varepsilon(s)$ reaches $0.05$, it remains fixed, and only $s$ continues to evolve toward $0$. As shown in \cref{fig:2D_Allen_Cahn_Equation}, the relative $L_2$ error by homotopy dynamics is $8.78 \times 10^{-3}$, compared with the result obtained by PINN, which has a $L_2$ error of $9.56 \times 10^{-1}$. This clearly demonstrates that the homotopy dynamics-based approach significantly improves accuracy.

\subsection{High Frequency Function Approximation }
We aim to approximate the following function:
$u=    \sin(50\pi x), \quad x \in [0,1].$
The homotopy is defined as $H(u,\varepsilon) = u - \sin(\frac{1}{\varepsilon}\pi x), $
where $\varepsilon \in [\frac{1}{50},\frac{1}{15}]$.

\begin{table}[htbp!]
    \caption{Comparison of the lowest loss achieved by the classical training and homotopy dynamics for different values of $\varepsilon$ in approximating $\sin\left(\frac{1}{\varepsilon} \pi x\right)$
    }
    \vskip 0.15in
    \centering
    \tiny
    \begin{tabular}{|c|c|c|c|c|} 
    \hline 
    $ $ & $\varepsilon = 1/15$ & $\varepsilon = 1/35$ & $\varepsilon = 1/50$ \\ \hline 
    Classical Loss                & 4.91e-6     & 7.21e-2     & 3.29e-1       \\ \hline 
    Homotopy Loss $L_H$                      & 1.73e-6     & 1.91e-6     & \textbf{2.82e-5}       \\ \hline
    \end{tabular}
    % On convection, \al{} provides 14.2$\times$ and 1.97$\times$ improvement over Adam or \lbfgs{} on L2RE. 
    % On reaction, \al{} provides 1.10$\times$ and 1.99$\times$ improvement over Adam or \lbfgs{} on L2RE.
    % On wave, \al{} provides 6.32$\times$ and 6.07$\times$ improvement over Adam or \lbfgs{} on L2RE.}
    \label{tab:loss_approximate}
\end{table}

As shown in \cref{fig:high_frequency_result}, due to the F-principle \cite{xu2024overview}, training is particularly challenging when approximating high-frequency functions like $\sin(50\pi x)$. The loss decreases slowly, resulting in poor approximation performance. However, training based on homotopy dynamics significantly reduces the loss, leading to a better approximation of high-frequency functions. This demonstrates that homotopy dynamics-based training can effectively facilitate convergence when approximating high-frequency data. Additionally, we compare the loss for approximating functions with different frequencies $1/\varepsilon$ using both methods. The results, presented in \cref{tab:loss_approximate}, show that the homotopy dynamics training method consistently performs well for high-frequency functions.





\subsection{Burgers Equation}
In this example, we adopt the operator learning framework to solve for the steady-state solution of the Burgers equation, given by:
\begin{align}
& u_t+\left(\frac{u^2}{2}\right)_x - \varepsilon u_{xx}=\pi \sin (\pi x) \cos (\pi x), \quad x \in[0, 1]\nonumber\\
& u(x, 0)=u_0(x),\label{eq:1D_Burgers} \\
& u(0, t)=u(1, t)=0, \nonumber 
\end{align}
with Dirichlet boundary conditions, where $u_0 \in L_{0}^2((0, 1); \mathbb{R})$ is the initial condition and $\varepsilon \in \mathbb{R}$ is the viscosity coefficient. We aim to learn the operator mapping the initial condition to the steady-state solution, $G^{\dagger}: L_{0}^2((0, 1); \mathbb{R}) \rightarrow H_{0}^r((0, 1); \mathbb{R})$, defined by $u_0 \mapsto u_{\infty}$ for any $r > 0$. As shown in Theorem 2.2 of \cite{KREISS1986161} and Theorems 2.5 and 2.7 of \cite{hao2019convergence}, for any $\varepsilon > 0$, the steady-state solution is independent of the initial condition, with a single shock occurring at $x_s = 0.5$. Here, we use DeepONet~\cite{lu2021deeponet} as the network architecture. 
The homotopy definition, similar to ~\cref{eq.hom_2D_AC}, can be found in \cref{Ap:operator}. The results can be found in \cref{fig:Burgers_result} and \cref{tab:loss_burgers}. Experimental results show that the homotopy dynamics strategy performs well in the operator learning setting as well.


\begin{table}[htbp!]
    \caption{Comparison of loss between classical training and homotopy dynamics for different values of $\varepsilon$ in the Burgers equation, along with the MSE distance to the ground truth shock location, $x_s$.}
    \vskip 0.15in
    \centering
    \tiny
    \begin{tabular}{|c|c|c|c|c|} 
    \hline  
    $ $ & $\varepsilon = 0.5$ & $\varepsilon = 0.1$ & $\varepsilon = 0.05$ \\ \hline 
    Homotopy Loss $L_H$                &  7.55e-7     & 3.40e-7     & 7.77e-7       \\ \hline 
    L2RE                      & 1.50e-3     & 7.00e-4     & 2.52e-2       \\ \hline
        MSE Distance $x_s$                      & 1.75e-8     & 9.14e-8      & 1.2e-3      \\ \hline
    \end{tabular}
    % On convection, \al{} provides 14.2$\times$ and 1.97$\times$ improvement over Adam or \lbfgs{} on L2RE. 
    % On reaction, \al{} provides 1.10$\times$ and 1.99$\times$ improvement over Adam or \lbfgs{} on L2RE.
    % On wave, \al{} provides 6.32$\times$ and 6.07$\times$ improvement over Adam or \lbfgs{} on L2RE.}
    \label{tab:loss_burgers}
\end{table}



% \begin{itemize}
%     \item Relate the curvature in the problem to the differential operator. Use this to demonstrate why the problem is ill-conditioned
%     \item Give an argument for why using Adam + L-BFGS is better than just using L-BFGS outright. The idea is that Adam lowers the errors to the point where the rest of the optimization becomes convex \ldots
%     \item Show why we need second-order methods. We would like to prove that once we are close to the optimum, second-order methods will give condition-number free linear convergence. Specialize this to the Gauss-Newton setting, with the randomized low-rank approximation.
%     % \item Show that it is not possible to get superlinear convergence under the interpolation assumption with an overparameterized neural network. This should be true b/c the Hessian at the optimum will have rank $\min(n, d)$, and when $d > n$, this guarantees that we cannot have strong convexity.
% \end{itemize}

\section{Case Study}
\label{appendix:visualization}
Here we display some illustrative cases from GovReport, where tokens marked in blue indicate draft tokens accepted by the target model. 

\begin{tcolorbox}
    \textcolor{blue}{The} \textcolor{blue}{Cong}\textcolor{black}{r}\textcolor{blue}{essional} \textcolor{blue}{Gold} \textcolor{blue}{Medal} \textcolor{blue}{is} \textcolor{blue}{a} \textcolor{black}{pr}\textcolor{blue}{estig}\textcolor{blue}{ious} \textcolor{blue}{award} \textcolor{blue}{given} \textcolor{black}{by} \textcolor{blue}{the} \textcolor{blue}{United} \textcolor{blue}{States} \textcolor{blue}{Congress} \textcolor{black}{to} \textcolor{black}{individuals} \textcolor{blue}{and} \textcolor{blue}{groups} \textcolor{blue}{in} \textcolor{blue}{recognition} \textcolor{blue}{of} \textcolor{black}{their} \textcolor{blue}{distinguished} \textcolor{blue}{contributions}\textcolor{black}{,} \textcolor{blue}{achiev}\textcolor{blue}{ements}\textcolor{blue}{,} \textcolor{blue}{and} \textcolor{black}{services} \textcolor{blue}{to} \textcolor{blue}{the} \textcolor{black}{country}\textcolor{blue}{.} \textcolor{blue}{The} \textcolor{black}{tradition} \textcolor{blue}{of} \textcolor{blue}{award}\textcolor{blue}{ing} \textcolor{black}{gold} \textcolor{blue}{med}\textcolor{blue}{als} \textcolor{black}{dates} \textcolor{blue}{back} \textcolor{blue}{to} \textcolor{blue}{the} \textcolor{blue}{late} \textcolor{blue}{}\textcolor{black}{1}\textcolor{blue}{8}\textcolor{blue}{th} \textcolor{blue}{century}\textcolor{blue}{,} \textcolor{black}{and} \textcolor{black}{it} \textcolor{blue}{has} \textcolor{blue}{been} \textcolor{black}{used} \textcolor{blue}{to} \textcolor{blue}{honor} \textcolor{black}{a} \textcolor{blue}{wide} \textcolor{blue}{range} \textcolor{blue}{of} \textcolor{blue}{individuals}\textcolor{blue}{,} \textcolor{black}{including} \textcolor{blue}{military} \textcolor{blue}{leaders}\textcolor{blue}{,} \textcolor{black}{scient}\textcolor{blue}{ists}\textcolor{blue}{,} \textcolor{blue}{artists}\textcolor{blue}{,} \textcolor{blue}{and} \textcolor{black}{human}\textcolor{black}{it}\textcolor{blue}{ari}\textcolor{blue}{ans}\textcolor{blue}{.}
    
    \textcolor{blue}{The} \textcolor{blue}{first} \textcolor{black}{Cong}\textcolor{blue}{r}\textcolor{blue}{essional} \textcolor{blue}{Gold} \textcolor{blue}{Med}\textcolor{blue}{als} \textcolor{black}{were} \textcolor{blue}{issued} \textcolor{blue}{by} \textcolor{blue}{the} \textcolor{black}{Cont}\textcolor{blue}{inental} \textcolor{blue}{Congress} \textcolor{blue}{in} \textcolor{blue}{the} \textcolor{blue}{late} \textcolor{black}{}\textcolor{blue}{1}\textcolor{blue}{7}\textcolor{blue}{0}\textcolor{blue}{0}\textcolor{blue}{s}\textcolor{black}{,} \textcolor{blue}{and} \textcolor{black}{since} \textcolor{blue}{then}\textcolor{blue}{,} \textcolor{blue}{Congress} \textcolor{blue}{has} \textcolor{black}{awarded} \textcolor{black}{over} \textcolor{blue}{}\textcolor{blue}{2}\textcolor{blue}{,}\textcolor{blue}{0}\textcolor{blue}{0}\textcolor{black}{0} \textcolor{blue}{med}\textcolor{blue}{als} \textcolor{blue}{to} \textcolor{black}{various} \textcolor{blue}{individuals} \textcolor{blue}{and} \textcolor{blue}{groups}\textcolor{blue}{.} \textcolor{blue}{The} \textcolor{black}{award}\textcolor{black}{ing} \textcolor{blue}{of} \textcolor{blue}{the} \textcolor{black}{Cong}\textcolor{blue}{r}\textcolor{blue}{essional} \textcolor{blue}{Gold} \textcolor{blue}{Medal} \textcolor{blue}{is} \textcolor{black}{not} \textcolor{blue}{a} \textcolor{black}{permanent} \textcolor{blue}{stat}\textcolor{blue}{ut}\textcolor{blue}{ory} \textcolor{black}{provision}\textcolor{blue}{,} \textcolor{blue}{and} \textcolor{blue}{it} \textcolor{blue}{is} \textcolor{black}{typically} \textcolor{blue}{done} \textcolor{blue}{through} \textcolor{black}{special} \textcolor{blue}{legisl}\textcolor{blue}{ation}\textcolor{blue}{.}
    
    \textcolor{blue}{The} \textcolor{black}{process} \textcolor{blue}{of} \textcolor{blue}{award}\textcolor{blue}{ing} \textcolor{blue}{the} \textcolor{black}{Cong}\textcolor{blue}{r}\textcolor{blue}{essional} \textcolor{blue}{Gold} \textcolor{blue}{Medal} \textcolor{black}{involves} \textcolor{black}{several} \textcolor{blue}{steps}\textcolor{blue}{,} \textcolor{blue}{including} \textcolor{black}{the} \textcolor{blue}{introduction} \textcolor{blue}{of} \textcolor{blue}{legisl}\textcolor{blue}{ation}\textcolor{black}{,} \textcolor{blue}{the} \textcolor{black}{consideration} \textcolor{blue}{of} \textcolor{blue}{the} \textcolor{blue}{legisl}\textcolor{blue}{ation} \textcolor{black}{by} \textcolor{blue}{the} \textcolor{black}{relevant} \textcolor{blue}{comm}\textcolor{blue}{itte}\textcolor{blue}{es}\textcolor{blue}{,} \textcolor{blue}{and} \textcolor{black}{the} \textcolor{blue}{appro}\textcolor{blue}{val} \textcolor{blue}{of} \textcolor{blue}{the} \textcolor{blue}{legisl}\textcolor{black}{ation} \textcolor{blue}{by} \textcolor{blue}{both} \textcolor{blue}{the} \textcolor{black}{House} \textcolor{blue}{of} \textcolor{blue}{Representatives} \textcolor{blue}{and} \textcolor{black}{the} \textcolor{blue}{Senate}\textcolor{blue}{.} \textcolor{blue}{Once} \textcolor{black}{the} \textcolor{blue}{legisl}\textcolor{blue}{ation} \textcolor{blue}{is} \textcolor{blue}{approved}\textcolor{blue}{,} \textcolor{black}{the} \textcolor{blue}{Secretary} \textcolor{blue}{of} \textcolor{blue}{the} \textcolor{blue}{Tre}\textcolor{blue}{as}\textcolor{black}{ury} \textcolor{black}{is} \textcolor{blue}{responsible} \textcolor{blue}{for} \textcolor{black}{striking} \textcolor{blue}{the} \textcolor{blue}{medal}\textcolor{blue}{,} \textcolor{black}{which} \textcolor{blue}{is} \textcolor{blue}{then} \textcolor{black}{presented}
    
    \textcolor{blue}{The} \textcolor{black}{design} \textcolor{blue}{of} \textcolor{blue}{the} \textcolor{blue}{Cong}\textcolor{blue}{r}\textcolor{blue}{essional} \textcolor{black}{Gold} \textcolor{blue}{Medal} \textcolor{blue}{is} \textcolor{blue}{typically} \textcolor{black}{determined} \textcolor{blue}{by} \textcolor{blue}{the} \textcolor{blue}{Secretary} \textcolor{blue}{of} \textcolor{blue}{the} \textcolor{black}{Tre}\textcolor{blue}{as}\textcolor{blue}{ury}\textcolor{blue}{,} \textcolor{black}{in} \textcolor{blue}{consult}\textcolor{blue}{ation} \textcolor{blue}{with} \textcolor{blue}{the} \textcolor{black}{Cit}\textcolor{blue}{iz}\textcolor{blue}{ens} \textcolor{black}{Co}\textcolor{blue}{in}\textcolor{black}{age} \textcolor{blue}{Ad}\textcolor{blue}{vis}\textcolor{blue}{ory} \textcolor{blue}{Committee} \textcolor{blue}{and} \textcolor{black}{the} \textcolor{black}{Commission} \textcolor{black}{of} \textcolor{blue}{Fine} \textcolor{blue}{Arts}\textcolor{blue}{.} \textcolor{blue}{The} \textcolor{blue}{medal} \textcolor{black}{typically} \textcolor{blue}{features} \textcolor{blue}{a} \textcolor{black}{portrait} \textcolor{blue}{of} \textcolor{blue}{the} \textcolor{blue}{recip}\textcolor{blue}{ient}\textcolor{black}{,} \textcolor{black}{as} \textcolor{blue}{well} \textcolor{blue}{as} \textcolor{blue}{ins}\textcolor{black}{cri}\textcolor{blue}{ptions} \textcolor{blue}{and} \textcolor{black}{symbols} \textcolor{blue}{that} \textcolor{blue}{reflect} \textcolor{blue}{the} \textcolor{black}{recip}\textcolor{blue}{ient}\textcolor{blue}{'}\textcolor{blue}{s} \textcolor{blue}{achiev}\textcolor{blue}{ements} \textcolor{black}{and} \textcolor{blue}{contributions}\textcolor{blue}{.}
    
    \textcolor{black}{The} \textcolor{blue}{Cong}\textcolor{blue}{r}\textcolor{blue}{essional} \textcolor{blue}{Gold} \textcolor{blue}{Medal} \textcolor{black}{is} \textcolor{blue}{considered} \textcolor{blue}{one} \textcolor{blue}{of} \textcolor{blue}{the} \textcolor{black}{highest} \textcolor{blue}{civil}\textcolor{blue}{ian} \textcolor{blue}{hon}\textcolor{blue}{ors} \textcolor{blue}{in} \textcolor{black}{the} \textcolor{blue}{United} \textcolor{blue}{States}\textcolor{blue}{,} \textcolor{blue}{and} \textcolor{blue}{it} \textcolor{black}{is} \textcolor{blue}{often} \textcolor{black}{given} \textcolor{blue}{to} \textcolor{blue}{individuals} \textcolor{blue}{who} \textcolor{blue}{have} \textcolor{black}{made} \textcolor{blue}{significant} \textcolor{blue}{contributions} \textcolor{blue}{to} \textcolor{black}{their} \textcolor{black}{field} \textcolor{blue}{or} \textcolor{blue}{to} \textcolor{black}{the} \textcolor{blue}{country} \textcolor{black}{as} \textcolor{blue}{a} \textcolor{blue}{whole}\textcolor{blue}{.} \textcolor{blue}{The} \textcolor{blue}{award} \textcolor{black}{has} \textcolor{blue}{been} \textcolor{blue}{given} \textcolor{blue}{to} \textcolor{black}{a} \textcolor{blue}{wide} \textcolor{blue}{range} \textcolor{blue}{of} \textcolor{blue}{individuals}\textcolor{blue}{,} \textcolor{black}{including} \textcolor{black}{military} \textcolor{blue}{hero}\textcolor{blue}{es}\textcolor{blue}{,} \textcolor{black}{civil} \textcolor{blue}{rights} \textcolor{blue}{leaders}\textcolor{blue}{,} \textcolor{blue}{and} \textcolor{black}{artists}\textcolor{blue}{.}
    
    \textcolor{black}{In} \textcolor{blue}{recent} \textcolor{blue}{years}\textcolor{blue}{,} \textcolor{blue}{the} \textcolor{black}{number} \textcolor{blue}{of} \textcolor{blue}{Cong}\textcolor{blue}{r}\textcolor{blue}{essional} \textcolor{blue}{Gold} \textcolor{black}{Med}\textcolor{blue}{als} \textcolor{blue}{awarded} \textcolor{black}{has} \textcolor{blue}{increased}\textcolor{blue}{,} \textcolor{black}{with} \textcolor{black}{over} \textcolor{blue}{}\textcolor{blue}{5}\textcolor{blue}{0} \textcolor{black}{b}\textcolor{blue}{ills} \textcolor{blue}{introduced} \textcolor{black}{in} \textcolor{blue}{the} \textcolor{blue}{}\textcolor{blue}{1}\textcolor{blue}{1}\textcolor{black}{3}\textcolor{blue}{th} \textcolor{blue}{Congress} \textcolor{black}{alone}\textcolor{blue}{.} \textcolor{blue}{The} \textcolor{blue}{award} \textcolor{blue}{has} \textcolor{black}{also} \textcolor{blue}{been} \textcolor{blue}{given} \textcolor{blue}{to} \textcolor{black}{a} \textcolor{black}{growing} \textcolor{blue}{number} \textcolor{blue}{of} \textcolor{blue}{groups}\textcolor{blue}{,} \textcolor{blue}{including} \textcolor{black}{military} \textcolor{black}{units}\textcolor{blue}{,} \textcolor{blue}{organizations}\textcolor{blue}{,} \textcolor{blue}{and} \textcolor{black}{even} \textcolor{black}{entire} \textcolor{blue}{cities}\textcolor{blue}{.}
    
    \textcolor{blue}{Over}\textcolor{blue}{all}\textcolor{black}{,} \textcolor{blue}{the} \textcolor{blue}{Cong}\textcolor{blue}{r}\textcolor{blue}{essional} \textcolor{blue}{Gold} \textcolor{black}{Medal} \textcolor{blue}{is} \textcolor{blue}{a} \textcolor{blue}{pr}\textcolor{blue}{estig}\textcolor{blue}{ious} \textcolor{black}{award} \textcolor{blue}{that} \textcolor{blue}{recogn}\textcolor{blue}{izes} \textcolor{blue}{the} \textcolor{black}{achiev}\textcolor{blue}{ements} \textcolor{blue}{and} \textcolor{blue}{contributions} \textcolor{blue}{of} \textcolor{blue}{individuals} \textcolor{black}{and} \textcolor{blue}{groups} \textcolor{blue}{to} \textcolor{blue}{the} \textcolor{black}{United} \textcolor{blue}{States}\textcolor{blue}{.} \textcolor{blue}{The} \textcolor{blue}{award} \textcolor{blue}{is} \textcolor{black}{given} \textcolor{black}{through} \textcolor{black}{special} \textcolor{blue}{legisl}\textcolor{blue}{ation} \textcolor{blue}{and} \textcolor{black}{involves} \textcolor{black}{several} \textcolor{blue}{steps}\textcolor{blue}{,} \textcolor{blue}{including} \textcolor{black}{the} \textcolor{blue}{introduction} \textcolor{blue}{of} \textcolor{blue}{legisl}\textcolor{blue}{ation}\textcolor{blue}{,} \textcolor{black}{the} \textcolor{blue}{consideration} \textcolor{blue}{of} \textcolor{blue}{the} \textcolor{black}{legisl}\textcolor{blue}{ation} \textcolor{blue}{by} \textcolor{blue}{the} \textcolor{black}{relevant} \textcolor{blue}{comm}\textcolor{blue}{itte}\textcolor{blue}{es}\textcolor{blue}{,} \textcolor{blue}{and} \textcolor{black}{the} \textcolor{blue}{appro}\textcolor{blue}{val} \textcolor{blue}{of} \textcolor{blue}{the} \textcolor{blue}{legisl}\textcolor{black}{ation} \textcolor{blue}{by} \textcolor{blue}{both} \textcolor{black}{the} \textcolor{blue}{House} \textcolor{blue}{of} \textcolor{blue}{Representatives} \textcolor{black}{and} \textcolor{blue}{the} \textcolor{blue}{Senate}\textcolor{blue}{.} \textcolor{blue}{The} \textcolor{black}{medal} \textcolor{blue}{is} \textcolor{blue}{typically} \textcolor{black}{designed} \textcolor{blue}{by} \textcolor{blue}{the} \textcolor{black}{Secretary} \textcolor{blue}{of} \textcolor{blue}{the} \textcolor{blue}{Tre}\textcolor{blue}{as}\textcolor{blue}{ury}\textcolor{black}{,} \textcolor{blue}{in} \textcolor{blue}{consult}\textcolor{blue}{ation} \textcolor{blue}{with} \textcolor{blue}{the} \textcolor{black}{Cit}\textcolor{blue}{iz}\textcolor{blue}{ens} \textcolor{blue}{Co}\textcolor{blue}{in}\textcolor{black}{age} \textcolor{blue}{Ad}\textcolor{blue}{vis}\textcolor{blue}{ory} \textcolor{blue}{Committee} \textcolor{blue}{and} \textcolor{black}{the} \textcolor{blue}{Commission} \textcolor{blue}{of} \textcolor{blue}{Fine} \textcolor{blue}{Arts}\textcolor{black}{,} \textcolor{blue}{and} \textcolor{blue}{it} \textcolor{black}{features} \textcolor{blue}{a} \textcolor{black}{portrait} \textcolor{blue}{of} \textcolor{blue}{the} \textcolor{blue}{recip}\textcolor{blue}{ient}\textcolor{blue}{,} \textcolor{black}{as} \textcolor{blue}{well} \textcolor{blue}{as} \textcolor{blue}{ins}\textcolor{blue}{cri}\textcolor{blue}{ptions} \textcolor{black}{and} \textcolor{blue}{symbols} \textcolor{blue}{that} \textcolor{blue}{reflect} \textcolor{blue}{the} \textcolor{black}{recip}\textcolor{blue}{ient}\textcolor{blue}{'}\textcolor{blue}{s} \textcolor{blue}{achiev}\textcolor{blue}{ements} \textcolor{black}{and} \textcolor{blue}{contributions}\textcolor{blue}{.}
\end{tcolorbox}
\newpage

\begin{tcolorbox}
  \textcolor{blue}{The} \textcolor{black}{report} \textcolor{black}{discuss}\textcolor{blue}{es} \textcolor{blue}{the} \textcolor{blue}{use} \textcolor{blue}{of} \textcolor{blue}{op}\textcolor{black}{io}\textcolor{blue}{id} \textcolor{blue}{medic}\textcolor{blue}{ations} \textcolor{blue}{in} \textcolor{black}{health}\textcolor{blue}{care} \textcolor{blue}{and} \textcolor{black}{the} \textcolor{blue}{potential} \textcolor{blue}{ris}\textcolor{blue}{ks} \textcolor{blue}{associated} \textcolor{blue}{with} \textcolor{black}{their} \textcolor{black}{mis}\textcolor{blue}{use}\textcolor{blue}{.} \textcolor{blue}{Op}\textcolor{blue}{io}\textcolor{blue}{id} \textcolor{black}{medic}\textcolor{blue}{ations} \textcolor{blue}{are} \textcolor{blue}{used} \textcolor{blue}{to} \textcolor{blue}{treat} \textcolor{black}{pain} \textcolor{blue}{and} \textcolor{black}{can} \textcolor{blue}{also} \textcolor{blue}{be} \textcolor{blue}{used} \textcolor{blue}{to} \textcolor{blue}{treat} \textcolor{black}{other} \textcolor{black}{health} \textcolor{blue}{problems}\textcolor{blue}{,} \textcolor{blue}{such} \textcolor{blue}{as} \textcolor{black}{severe} \textcolor{blue}{c}\textcolor{blue}{ough}\textcolor{blue}{ing}\textcolor{blue}{.} \textcolor{black}{There} \textcolor{blue}{are} \textcolor{blue}{three} \textcolor{blue}{types} \textcolor{blue}{of} \textcolor{blue}{op}\textcolor{black}{io}\textcolor{blue}{id} \textcolor{blue}{medic}\textcolor{blue}{ations} \textcolor{blue}{that} \textcolor{blue}{are} \textcolor{black}{approved} \textcolor{blue}{for} \textcolor{blue}{use} \textcolor{blue}{in} \textcolor{blue}{the} \textcolor{blue}{treatment} \textcolor{black}{of} \textcolor{blue}{op}\textcolor{blue}{io}\textcolor{blue}{id} \textcolor{blue}{use} \textcolor{blue}{dis}\textcolor{black}{orders}\textcolor{blue}{:} \textcolor{black}{m}\textcolor{blue}{eth}\textcolor{blue}{ad}\textcolor{blue}{one}\textcolor{blue}{,} \textcolor{black}{bu}\textcolor{blue}{pr}\textcolor{blue}{en}\textcolor{blue}{orph}\textcolor{blue}{ine}\textcolor{blue}{,} \textcolor{black}{and} \textcolor{blue}{n}\textcolor{blue}{alt}\textcolor{blue}{re}\textcolor{blue}{x}\textcolor{blue}{one}\textcolor{black}{.} \textcolor{blue}{M}\textcolor{blue}{eth}\textcolor{blue}{ad}\textcolor{blue}{one} \textcolor{blue}{is} \textcolor{black}{a} \textcolor{blue}{full} \textcolor{black}{op}\textcolor{blue}{io}\textcolor{blue}{id} \textcolor{blue}{ag}\textcolor{blue}{on}\textcolor{blue}{ist}\textcolor{black}{,} \textcolor{blue}{meaning} \textcolor{blue}{it} \textcolor{black}{bind}\textcolor{blue}{s} \textcolor{blue}{to} \textcolor{black}{and} \textcolor{blue}{activ}\textcolor{blue}{ates} \textcolor{blue}{op}\textcolor{blue}{io}\textcolor{blue}{id} \textcolor{black}{re}\textcolor{blue}{cept}\textcolor{blue}{ors} \textcolor{blue}{in} \textcolor{blue}{the} \textcolor{blue}{body}\textcolor{black}{.} \textcolor{black}{Bu}\textcolor{blue}{pr}\textcolor{blue}{en}\textcolor{blue}{orph}\textcolor{blue}{ine} \textcolor{blue}{is} \textcolor{black}{a} \textcolor{blue}{partial} \textcolor{blue}{op}\textcolor{blue}{io}\textcolor{blue}{id} \textcolor{black}{ag}\textcolor{blue}{on}\textcolor{blue}{ist}\textcolor{blue}{,} \textcolor{blue}{meaning} \textcolor{blue}{it} \textcolor{black}{also} \textcolor{blue}{bind}\textcolor{blue}{s} \textcolor{blue}{to} \textcolor{black}{and} \textcolor{black}{activ}\textcolor{blue}{ates} \textcolor{blue}{op}\textcolor{blue}{io}\textcolor{blue}{id} \textcolor{blue}{re}\textcolor{black}{cept}\textcolor{blue}{ors}\textcolor{blue}{,} \textcolor{blue}{but} \textcolor{black}{to} \textcolor{blue}{a} \textcolor{blue}{less}\textcolor{blue}{er} \textcolor{blue}{extent} \textcolor{blue}{than} \textcolor{black}{m}\textcolor{blue}{eth}\textcolor{blue}{ad}\textcolor{blue}{one}\textcolor{blue}{.} \textcolor{black}{N}\textcolor{blue}{alt}\textcolor{blue}{re}\textcolor{blue}{x}\textcolor{blue}{one} \textcolor{blue}{is} \textcolor{black}{an} \textcolor{blue}{op}\textcolor{blue}{io}\textcolor{blue}{id} \textcolor{blue}{ant}\textcolor{blue}{agon}\textcolor{black}{ist}\textcolor{blue}{,} \textcolor{blue}{meaning} \textcolor{blue}{it} \textcolor{black}{bind}\textcolor{blue}{s} \textcolor{blue}{to} \textcolor{black}{and} \textcolor{black}{blocks} \textcolor{blue}{the} \textcolor{black}{effects} \textcolor{blue}{of} \textcolor{blue}{op}\textcolor{blue}{io}\textcolor{blue}{id} \textcolor{blue}{re}\textcolor{black}{cept}\textcolor{blue}{ors}\textcolor{blue}{.}
  
  \textcolor{black}{The} \textcolor{black}{report} \textcolor{black}{also} \textcolor{blue}{discuss}\textcolor{blue}{es} \textcolor{blue}{the} \textcolor{blue}{potential} \textcolor{blue}{ris}\textcolor{black}{ks} \textcolor{blue}{associated} \textcolor{blue}{with} \textcolor{blue}{the} \textcolor{blue}{use} \textcolor{black}{of} \textcolor{blue}{op}\textcolor{blue}{io}\textcolor{blue}{id} \textcolor{blue}{medic}\textcolor{blue}{ations}\textcolor{black}{,} \textcolor{blue}{including} \textcolor{blue}{the} \textcolor{black}{risk} \textcolor{blue}{of} \textcolor{blue}{add}\textcolor{blue}{iction} \textcolor{blue}{and} \textcolor{black}{the} \textcolor{blue}{risk} \textcolor{blue}{of} \textcolor{blue}{over}\textcolor{blue}{d}\textcolor{blue}{ose}\textcolor{black}{.} \textcolor{blue}{The} \textcolor{blue}{use} \textcolor{blue}{of} \textcolor{blue}{op}\textcolor{blue}{io}\textcolor{black}{id} \textcolor{blue}{medic}\textcolor{blue}{ations} \textcolor{blue}{can} \textcolor{blue}{lead} \textcolor{blue}{to} \textcolor{black}{physical} \textcolor{blue}{dependence} \textcolor{blue}{and} \textcolor{black}{toler}\textcolor{blue}{ance}\textcolor{blue}{,} \textcolor{blue}{which} \textcolor{blue}{can} \textcolor{black}{make} \textcolor{blue}{it} \textcolor{blue}{difficult} \textcolor{blue}{to} \textcolor{black}{stop} \textcolor{blue}{using} \textcolor{blue}{the} \textcolor{blue}{medic}\textcolor{blue}{ation}\textcolor{blue}{.} \textcolor{black}{Additionally}\textcolor{blue}{,} \textcolor{blue}{the} \textcolor{black}{mis}\textcolor{blue}{use} \textcolor{blue}{of} \textcolor{blue}{op}\textcolor{blue}{io}\textcolor{blue}{id} \textcolor{black}{medic}\textcolor{blue}{ations} \textcolor{blue}{can} \textcolor{blue}{lead} \textcolor{blue}{to} \textcolor{black}{add}\textcolor{blue}{iction}\textcolor{blue}{,} \textcolor{blue}{which} \textcolor{blue}{can} \textcolor{black}{have} \textcolor{blue}{serious} \textcolor{blue}{consequences} \textcolor{blue}{for} \textcolor{blue}{the} \textcolor{black}{individual} \textcolor{blue}{and} \textcolor{blue}{their} \textcolor{black}{loved} \textcolor{blue}{ones}\textcolor{blue}{.}
  
  \textcolor{blue}{The} \textcolor{black}{report} \textcolor{blue}{also} \textcolor{blue}{discuss}\textcolor{blue}{es} \textcolor{blue}{the} \textcolor{blue}{potential} \textcolor{black}{ris}\textcolor{blue}{ks} \textcolor{blue}{associated} \textcolor{blue}{with} \textcolor{blue}{the} \textcolor{black}{di}\textcolor{blue}{version} \textcolor{blue}{of} \textcolor{blue}{op}\textcolor{blue}{io}\textcolor{blue}{id} \textcolor{black}{medic}\textcolor{blue}{ations}\textcolor{blue}{,} \textcolor{blue}{which} \textcolor{black}{is} \textcolor{blue}{the} \textcolor{black}{illegal} \textcolor{blue}{use} \textcolor{blue}{of} \textcolor{blue}{pres}\textcolor{blue}{cription} \textcolor{blue}{op}\textcolor{black}{io}\textcolor{blue}{ids} \textcolor{blue}{for} \textcolor{black}{non}\textcolor{blue}{-}\textcolor{blue}{med}\textcolor{blue}{ical} \textcolor{blue}{purposes}\textcolor{blue}{.} \textcolor{black}{D}\textcolor{blue}{ivers}\textcolor{blue}{ion} \textcolor{black}{can} \textcolor{blue}{lead} \textcolor{blue}{to} \textcolor{blue}{increased} \textcolor{black}{rates} \textcolor{blue}{of} \textcolor{blue}{add}\textcolor{blue}{iction}\textcolor{blue}{,} \textcolor{black}{over}\textcolor{blue}{d}\textcolor{blue}{ose}\textcolor{blue}{,} \textcolor{blue}{and} \textcolor{blue}{death}\textcolor{black}{.}
  
  \textcolor{blue}{The} \textcolor{blue}{report} \textcolor{black}{con}\textcolor{blue}{cludes} \textcolor{blue}{by} \textcolor{black}{discuss}\textcolor{blue}{ing} \textcolor{blue}{the} \textcolor{blue}{importance} \textcolor{blue}{of} \textcolor{black}{proper} \textcolor{blue}{use} \textcolor{blue}{and} \textcolor{black}{monitoring} \textcolor{blue}{of} \textcolor{blue}{op}\textcolor{blue}{io}\textcolor{blue}{id} \textcolor{blue}{medic}\textcolor{black}{ations}\textcolor{blue}{,} \textcolor{blue}{as} \textcolor{blue}{well} \textcolor{blue}{as} \textcolor{blue}{the} \textcolor{black}{need} \textcolor{blue}{for} \textcolor{black}{continued} \textcolor{blue}{research} \textcolor{blue}{and} \textcolor{black}{development} \textcolor{blue}{of} \textcolor{blue}{new} \textcolor{blue}{treat}\textcolor{blue}{ments} \textcolor{black}{for} \textcolor{blue}{op}\textcolor{blue}{io}\textcolor{blue}{id} \textcolor{blue}{use} \textcolor{black}{dis}\textcolor{blue}{orders}\textcolor{blue}{.}
\end{tcolorbox}

\begin{tcolorbox}
    \textcolor{blue}{The} \textcolor{blue}{Rail}\textcolor{black}{road} \textcolor{blue}{Ret}\textcolor{blue}{irement} \textcolor{blue}{Board} \textcolor{black}{(}\textcolor{blue}{R}\textcolor{blue}{R}\textcolor{blue}{B}\textcolor{blue}{)} \textcolor{blue}{is} \textcolor{black}{an} \textcolor{blue}{independent} \textcolor{blue}{federal} \textcolor{blue}{ag}\textcolor{blue}{ency} \textcolor{blue}{that} \textcolor{black}{admin}\textcolor{blue}{ister}\textcolor{blue}{s} \textcolor{black}{ret}\textcolor{blue}{irement}\textcolor{blue}{,} \textcolor{black}{surv}\textcolor{blue}{iv}\textcolor{blue}{or}\textcolor{blue}{,} \textcolor{black}{dis}\textcolor{blue}{ability}\textcolor{blue}{,} \textcolor{blue}{un}\textcolor{black}{emp}\textcolor{blue}{loyment}\textcolor{blue}{,} \textcolor{blue}{and} \textcolor{blue}{sick}\textcolor{blue}{ness} \textcolor{black}{ins}\textcolor{blue}{urance} \textcolor{blue}{for} \textcolor{black}{rail}\textcolor{blue}{road} \textcolor{blue}{workers} \textcolor{blue}{and} \textcolor{blue}{their} \textcolor{blue}{families}\textcolor{black}{.} \textcolor{blue}{The} \textcolor{blue}{R}\textcolor{blue}{R}\textcolor{blue}{B} \textcolor{black}{covers} \textcolor{black}{workers} \textcolor{blue}{who} \textcolor{blue}{are} \textcolor{blue}{employed} \textcolor{blue}{by} \textcolor{blue}{rail}\textcolor{black}{ro}\textcolor{blue}{ads} \textcolor{black}{engaged} \textcolor{blue}{in} \textcolor{black}{inter}\textcolor{blue}{state} \textcolor{blue}{commerce} \textcolor{blue}{and} \textcolor{black}{related} \textcolor{black}{subs}\textcolor{blue}{idi}\textcolor{blue}{aries}\textcolor{blue}{,} \textcolor{black}{rail}\textcolor{blue}{road} \textcolor{blue}{associations}\textcolor{blue}{,} \textcolor{blue}{and} \textcolor{blue}{rail}\textcolor{black}{road} \textcolor{blue}{labor} \textcolor{blue}{organizations}\textcolor{blue}{.}
    
    \textcolor{blue}{The} \textcolor{black}{R}\textcolor{blue}{R}\textcolor{blue}{B} \textcolor{blue}{has} \textcolor{blue}{two} \textcolor{blue}{main} \textcolor{black}{programs}\textcolor{blue}{:} \textcolor{blue}{the} \textcolor{blue}{Rail}\textcolor{blue}{road} \textcolor{blue}{Ret}\textcolor{black}{irement} \textcolor{black}{Act} \textcolor{blue}{(}\textcolor{blue}{R}\textcolor{blue}{RA}\textcolor{blue}{)} \textcolor{black}{and} \textcolor{blue}{the} \textcolor{blue}{Rail}\textcolor{blue}{road} \textcolor{black}{Un}\textcolor{blue}{emp}\textcolor{blue}{loyment} \textcolor{blue}{In}\textcolor{blue}{sur}\textcolor{blue}{ance} \textcolor{black}{Act} \textcolor{blue}{(}\textcolor{blue}{R}\textcolor{black}{UI}\textcolor{blue}{A}\textcolor{blue}{).} \textcolor{blue}{The} \textcolor{blue}{R}\textcolor{blue}{RA} \textcolor{black}{author}\textcolor{blue}{izes} \textcolor{blue}{ret}\textcolor{blue}{irement}\textcolor{black}{,} \textcolor{black}{surv}\textcolor{blue}{iv}\textcolor{blue}{or}\textcolor{blue}{,} \textcolor{blue}{and} \textcolor{blue}{dis}\textcolor{black}{ability} \textcolor{blue}{benefits} \textcolor{blue}{for} \textcolor{blue}{rail}\textcolor{blue}{road} \textcolor{blue}{workers} \textcolor{black}{and} \textcolor{blue}{their} \textcolor{blue}{families}\textcolor{blue}{.} \textcolor{blue}{The} \textcolor{blue}{R}\textcolor{black}{UI}\textcolor{blue}{A} \textcolor{black}{provides} \textcolor{black}{un}\textcolor{blue}{emp}\textcolor{blue}{loyment} \textcolor{blue}{and} \textcolor{black}{sick}\textcolor{blue}{ness} \textcolor{blue}{benefits} \textcolor{blue}{for} \textcolor{blue}{rail}\textcolor{blue}{road} \textcolor{black}{workers}\textcolor{blue}{.}
    
    \textcolor{blue}{The} \textcolor{black}{number} \textcolor{blue}{of} \textcolor{blue}{rail}\textcolor{blue}{road} \textcolor{blue}{workers} \textcolor{black}{has} \textcolor{black}{been} \textcolor{blue}{decl}\textcolor{blue}{ining} \textcolor{blue}{since} \textcolor{blue}{the} \textcolor{blue}{}\textcolor{black}{1}\textcolor{blue}{9}\textcolor{blue}{5}\textcolor{blue}{0}\textcolor{blue}{s}\textcolor{blue}{,} \textcolor{black}{although} \textcolor{blue}{the} \textcolor{blue}{rate} \textcolor{blue}{of} \textcolor{black}{decl}\textcolor{blue}{ine} \textcolor{blue}{has} \textcolor{blue}{been} \textcolor{black}{irregular}\textcolor{blue}{.} \textcolor{blue}{In} \textcolor{blue}{recent} \textcolor{blue}{years}\textcolor{blue}{,} \textcolor{black}{rail}\textcolor{blue}{road} \textcolor{blue}{employ}\textcolor{blue}{ment} \textcolor{blue}{has} \textcolor{blue}{increased} \textcolor{black}{after} \textcolor{black}{reaching} \textcolor{blue}{an} \textcolor{blue}{all}\textcolor{blue}{-}\textcolor{blue}{time} \textcolor{blue}{low} \textcolor{black}{of} \textcolor{blue}{}\textcolor{blue}{2}\textcolor{black}{1}\textcolor{blue}{5}\textcolor{blue}{,}\textcolor{blue}{0}\textcolor{blue}{0}\textcolor{blue}{0} \textcolor{black}{workers} \textcolor{blue}{in} \textcolor{blue}{January} \textcolor{blue}{}\textcolor{blue}{2}\textcolor{blue}{0}\textcolor{black}{1}\textcolor{blue}{0}\textcolor{blue}{.} \textcolor{blue}{In} \textcolor{black}{April} \textcolor{blue}{}\textcolor{blue}{2}\textcolor{blue}{0}\textcolor{blue}{1}\textcolor{blue}{5}\textcolor{black}{,} \textcolor{blue}{rail}\textcolor{blue}{road} \textcolor{blue}{employ}\textcolor{blue}{ment} \textcolor{black}{pe}\textcolor{blue}{aked} \textcolor{blue}{at} \textcolor{blue}{}\textcolor{blue}{2}\textcolor{black}{5}\textcolor{black}{3}\textcolor{blue}{,}\textcolor{blue}{0}\textcolor{blue}{0}\textcolor{blue}{0} \textcolor{blue}{workers}\textcolor{black}{,} \textcolor{blue}{the} \textcolor{blue}{highest} \textcolor{blue}{level} \textcolor{blue}{since} \textcolor{black}{November} \textcolor{blue}{}\textcolor{blue}{1}\textcolor{blue}{9}\textcolor{blue}{9}\textcolor{blue}{9}\textcolor{black}{,} \textcolor{blue}{and} \textcolor{blue}{then} \textcolor{black}{decl}\textcolor{blue}{ined} \textcolor{black}{through} \textcolor{black}{F}\textcolor{blue}{Y}\textcolor{blue}{2}\textcolor{blue}{0}\textcolor{blue}{1}\textcolor{blue}{7}\textcolor{black}{,} \textcolor{blue}{falling} \textcolor{blue}{to} \textcolor{blue}{}\textcolor{blue}{2}\textcolor{blue}{2}\textcolor{black}{1}\textcolor{blue}{,}\textcolor{blue}{0}\textcolor{blue}{0}\textcolor{blue}{0} \textcolor{black}{workers}\textcolor{blue}{.}
    
    %\textcolor{blue}{The} \textcolor{blue}{total} \textcolor{black}{number} \textcolor{blue}{of} \textcolor{blue}{benef}\textcolor{blue}{ici}\textcolor{blue}{aries} \textcolor{black}{under} \textcolor{blue}{the} \textcolor{blue}{R}\textcolor{blue}{RA} \textcolor{black}{and} \textcolor{blue}{R}\textcolor{blue}{UI}\textcolor{black}{A} \textcolor{blue}{decre}\textcolor{blue}{ased} \textcolor{blue}{from} \textcolor{blue}{}\textcolor{black}{6}\textcolor{blue}{2}\textcolor{black}{3}\textcolor{blue}{,}\textcolor{blue}{0}\textcolor{blue}{0}\textcolor{blue}{0} \textcolor{blue}{in} \textcolor{black}{F}\textcolor{blue}{Y}\textcolor{blue}{2}\textcolor{blue}{0}\textcolor{blue}{0}\textcolor{black}{8} \textcolor{blue}{to} \textcolor{blue}{}\textcolor{blue}{5}\textcolor{black}{7}\textcolor{black}{4}\textcolor{blue}{,}\textcolor{blue}{0}\textcolor{blue}{0}\textcolor{blue}{0} \textcolor{blue}{in} \textcolor{black}{F}\textcolor{blue}{Y}\textcolor{blue}{2}\textcolor{blue}{0}\textcolor{blue}{1}\textcolor{blue}{7}\textcolor{black}{.} \textcolor{blue}{Total} \textcolor{black}{benefit} \textcolor{blue}{pay}\textcolor{blue}{ments} \textcolor{black}{increased} \textcolor{blue}{from} \textcolor{blue}{\$}\textcolor{blue}{1}\textcolor{black}{0}\textcolor{blue}{.}\textcolor{blue}{1} \textcolor{black}{billion} \textcolor{blue}{to} \textcolor{blue}{\$}\textcolor{blue}{1}\textcolor{blue}{2}\textcolor{blue}{.}\textcolor{black}{6} \textcolor{blue}{billion} \textcolor{blue}{during} \textcolor{blue}{the} \textcolor{blue}{same} \textcolor{blue}{period}\textcolor{black}{.} \textcolor{blue}{In} \textcolor{blue}{F}\textcolor{blue}{Y}\textcolor{blue}{2}\textcolor{blue}{0}\textcolor{black}{1}\textcolor{blue}{7}\textcolor{blue}{,} \textcolor{blue}{the} \textcolor{blue}{R}\textcolor{blue}{R}\textcolor{black}{B} \textcolor{blue}{paid} \textcolor{blue}{nearly} \textcolor{blue}{\$}\textcolor{blue}{1}\textcolor{blue}{2}\textcolor{black}{.}\textcolor{blue}{5} \textcolor{blue}{billion} \textcolor{blue}{in} \textcolor{blue}{ret}\textcolor{blue}{irement}\textcolor{black}{,} \textcolor{blue}{dis}\textcolor{blue}{ability}\textcolor{blue}{,} \textcolor{blue}{and} \textcolor{black}{surv}\textcolor{blue}{iv}\textcolor{blue}{or} \textcolor{blue}{benefits} \textcolor{blue}{to} \textcolor{black}{approximately} \textcolor{blue}{}\textcolor{blue}{5}\textcolor{blue}{4}\textcolor{black}{8}\textcolor{blue}{,}\textcolor{blue}{0}\textcolor{blue}{0}\textcolor{blue}{0} \textcolor{blue}{benef}\textcolor{black}{ici}\textcolor{blue}{aries}\textcolor{blue}{.} \textcolor{black}{Al}\textcolor{blue}{most} \textcolor{blue}{\$}\textcolor{blue}{1}\textcolor{black}{0}\textcolor{blue}{5}\textcolor{blue}{.}\textcolor{black}{4} \textcolor{blue}{million} \textcolor{blue}{in} \textcolor{blue}{un}\textcolor{blue}{emp}\textcolor{blue}{loyment} \textcolor{black}{and} \textcolor{blue}{sick}\textcolor{blue}{ness} \textcolor{blue}{benefits} \textcolor{blue}{were} \textcolor{blue}{paid} \textcolor{black}{to} \textcolor{black}{approximately} \textcolor{blue}{}\textcolor{blue}{2}\textcolor{black}{8}\textcolor{blue}{,}\textcolor{blue}{0}\textcolor{blue}{0}\textcolor{blue}{0} \textcolor{black}{claim}\textcolor{blue}{ants}\textcolor{blue}{.}
    
    %\textcolor{blue}{The} \textcolor{black}{R}\textcolor{blue}{RA} \textcolor{blue}{and} \textcolor{black}{R}\textcolor{black}{UI}\textcolor{blue}{A} \textcolor{blue}{are} \textcolor{blue}{fund}\textcolor{blue}{ed} \textcolor{blue}{by} \textcolor{black}{pay}\textcolor{blue}{roll} \textcolor{blue}{tax}\textcolor{blue}{es}\textcolor{blue}{,} \textcolor{black}{financial} \textcolor{black}{inter}\textcolor{blue}{changes} \textcolor{black}{from} \textcolor{black}{Social} \textcolor{blue}{Security}\textcolor{blue}{,} \textcolor{blue}{and} \textcolor{black}{trans}\textcolor{blue}{fers} \textcolor{blue}{from} \textcolor{blue}{the} \textcolor{blue}{National} \textcolor{blue}{Rail}\textcolor{black}{road} \textcolor{blue}{Ret}\textcolor{blue}{irement} \textcolor{blue}{In}\textcolor{black}{vest}\textcolor{blue}{ment} \textcolor{blue}{Trust} \textcolor{blue}{(}\textcolor{blue}{NR}\textcolor{blue}{R}\textcolor{black}{IT}\textcolor{blue}{).} \textcolor{blue}{The} \textcolor{blue}{R}\textcolor{blue}{R}\textcolor{blue}{B}\textcolor{black}{'}\textcolor{blue}{s} \textcolor{blue}{main} \textcolor{blue}{source} \textcolor{blue}{of} \textcolor{black}{fund}\textcolor{blue}{ing} \textcolor{blue}{is} \textcolor{blue}{the} \textcolor{black}{pay}\textcolor{blue}{roll} \textcolor{blue}{tax}\textcolor{blue}{es} \textcolor{blue}{paid} \textcolor{blue}{by} \textcolor{black}{rail}\textcolor{blue}{road} \textcolor{blue}{employ}\textcolor{blue}{ers} \textcolor{black}{and} \textcolor{blue}{employees}\textcolor{blue}{.} \textcolor{blue}{The} \textcolor{black}{Tier} \textcolor{blue}{I} \textcolor{black}{tax} \textcolor{blue}{is} \textcolor{blue}{the} \textcolor{black}{same} \textcolor{blue}{as} \textcolor{blue}{the} \textcolor{blue}{Social} \textcolor{blue}{Security} \textcolor{black}{pay}\textcolor{blue}{roll} \textcolor{blue}{tax}\textcolor{blue}{,} \textcolor{blue}{while} \textcolor{blue}{the} \textcolor{black}{Tier} \textcolor{blue}{II} \textcolor{blue}{tax} \textcolor{black}{is} \textcolor{blue}{set} \textcolor{black}{each} \textcolor{blue}{year} \textcolor{blue}{based} \textcolor{blue}{on} \textcolor{blue}{the} \textcolor{black}{rail}\textcolor{blue}{road} \textcolor{blue}{ret}\textcolor{blue}{irement} \textcolor{black}{system}\textcolor{blue}{'}\textcolor{blue}{s} \textcolor{blue}{asset} \textcolor{black}{bal}\textcolor{blue}{ances}\textcolor{blue}{,} \textcolor{black}{benefit} \textcolor{blue}{pay}\textcolor{blue}{ments}\textcolor{blue}{,} \textcolor{blue}{and} \textcolor{black}{administrative} \textcolor{blue}{costs}\textcolor{blue}{.}
    
    \textcolor{blue}{The} \textcolor{black}{R}\textcolor{blue}{R}\textcolor{blue}{B}\textcolor{blue}{'}\textcolor{blue}{s} \textcolor{black}{programs} \textcolor{blue}{are} \textcolor{blue}{designed} \textcolor{blue}{to} \textcolor{blue}{provide} \textcolor{black}{compreh}\textcolor{blue}{ensive} \textcolor{blue}{benefits} \textcolor{blue}{to} \textcolor{blue}{rail}\textcolor{blue}{road} \textcolor{black}{workers} \textcolor{blue}{and} \textcolor{blue}{their} \textcolor{blue}{families}\textcolor{blue}{.} \textcolor{blue}{The} \textcolor{black}{R}\textcolor{blue}{RA} \textcolor{blue}{and} \textcolor{black}{R}\textcolor{blue}{UI}\textcolor{blue}{A} \textcolor{black}{are} \textcolor{black}{important} \textcolor{blue}{components} \textcolor{blue}{of} \textcolor{blue}{the} \textcolor{blue}{rail}\textcolor{blue}{road} \textcolor{black}{industry}\textcolor{blue}{'}\textcolor{blue}{s} \textcolor{blue}{ret}\textcolor{blue}{irement} \textcolor{blue}{and} \textcolor{black}{benefits} \textcolor{blue}{system}\textcolor{blue}{.} \textcolor{blue}{The} \textcolor{blue}{R}\textcolor{blue}{R}\textcolor{black}{B}\textcolor{blue}{'}\textcolor{blue}{s} \textcolor{black}{efforts} \textcolor{blue}{to} \textcolor{blue}{maintain} \textcolor{black}{and} \textcolor{blue}{improve} \textcolor{blue}{these} \textcolor{black}{programs} \textcolor{blue}{are} \textcolor{black}{cru}\textcolor{blue}{cial} \textcolor{blue}{for} \textcolor{blue}{the} \textcolor{black}{well}\textcolor{blue}{-}\textcolor{blue}{be}\textcolor{blue}{ing} \textcolor{blue}{of} \textcolor{blue}{rail}\textcolor{black}{road} \textcolor{blue}{workers} \textcolor{blue}{and} \textcolor{blue}{their} \textcolor{blue}{families}\textcolor{blue}{.}
\end{tcolorbox}

\begin{tcolorbox}
    \textcolor{blue}{The} \textcolor{blue}{report} \textcolor{black}{provides} \textcolor{blue}{an} \textcolor{blue}{over}\textcolor{blue}{view} \textcolor{blue}{of} \textcolor{blue}{the} \textcolor{black}{annual} \textcolor{blue}{appropri}\textcolor{blue}{ations} \textcolor{blue}{for} \textcolor{blue}{the} \textcolor{blue}{Department} \textcolor{black}{of} \textcolor{blue}{Hom}\textcolor{blue}{eland} \textcolor{blue}{Security} \textcolor{blue}{(}\textcolor{blue}{D}\textcolor{black}{HS}\textcolor{blue}{)} \textcolor{blue}{for} \textcolor{blue}{F}\textcolor{blue}{Y}\textcolor{blue}{2}\textcolor{black}{0}\textcolor{blue}{1}\textcolor{blue}{9}\textcolor{blue}{.} \textcolor{blue}{It} \textcolor{black}{comp}\textcolor{blue}{ares} \textcolor{blue}{the} \textcolor{black}{en}\textcolor{blue}{act}\textcolor{blue}{ed} \textcolor{blue}{F}\textcolor{blue}{Y}\textcolor{blue}{2}\textcolor{black}{0}\textcolor{blue}{1}\textcolor{blue}{8} \textcolor{blue}{appropri}\textcolor{blue}{ations} \textcolor{blue}{for} \textcolor{black}{D}\textcolor{blue}{HS}\textcolor{blue}{,} \textcolor{blue}{the} \textcolor{black}{Trump} \textcolor{blue}{Administration}\textcolor{blue}{'}\textcolor{blue}{s} \textcolor{blue}{F}\textcolor{black}{Y}\textcolor{blue}{2}\textcolor{blue}{0}\textcolor{blue}{1}\textcolor{blue}{9} \textcolor{blue}{budget} \textcolor{black}{request}\textcolor{blue}{,} \textcolor{blue}{and} \textcolor{blue}{the} \textcolor{black}{appropri}\textcolor{blue}{ations} \textcolor{blue}{measures} \textcolor{black}{developed} \textcolor{blue}{and} \textcolor{black}{considered} \textcolor{blue}{by} \textcolor{blue}{Congress} \textcolor{blue}{in} \textcolor{black}{response} \textcolor{blue}{to} \textcolor{blue}{the} \textcolor{black}{request}\textcolor{blue}{.} \textcolor{blue}{The} \textcolor{blue}{report} \textcolor{black}{ident}\textcolor{blue}{ifies} \textcolor{blue}{additional} \textcolor{black}{inform}\textcolor{blue}{ational} \textcolor{blue}{resources}\textcolor{blue}{,} \textcolor{black}{reports}\textcolor{blue}{,} \textcolor{blue}{and} \textcolor{black}{policy} \textcolor{blue}{exper}\textcolor{blue}{ts} \textcolor{blue}{that} \textcolor{black}{can} \textcolor{blue}{provide} \textcolor{black}{further} \textcolor{blue}{information} \textcolor{blue}{on} \textcolor{blue}{D}\textcolor{blue}{HS} \textcolor{black}{appropri}\textcolor{blue}{ations}\textcolor{blue}{.}
    
    \textcolor{blue}{The} \textcolor{blue}{report} \textcolor{black}{explains} \textcolor{black}{several} \textcolor{black}{special}\textcolor{blue}{ized} \textcolor{blue}{budget}\textcolor{blue}{ary} \textcolor{black}{concepts}\textcolor{blue}{,} \textcolor{blue}{including} \textcolor{blue}{budget} \textcolor{black}{authority}\textcolor{blue}{,} \textcolor{black}{oblig}\textcolor{blue}{ations}\textcolor{blue}{,} \textcolor{blue}{out}\textcolor{black}{l}\textcolor{blue}{ays}\textcolor{blue}{,} \textcolor{blue}{dis}\textcolor{blue}{cret}\textcolor{blue}{ion}\textcolor{black}{ary} \textcolor{black}{and} \textcolor{black}{mand}\textcolor{blue}{atory} \textcolor{blue}{sp}\textcolor{blue}{ending}\textcolor{blue}{,} \textcolor{black}{offset}\textcolor{blue}{ting} \textcolor{black}{collections}\textcolor{blue}{,} \textcolor{blue}{alloc}\textcolor{black}{ations}\textcolor{blue}{,} \textcolor{blue}{and} \textcolor{black}{adjust}\textcolor{blue}{ments} \textcolor{blue}{to} \textcolor{blue}{the} \textcolor{black}{dis}\textcolor{blue}{cret}\textcolor{blue}{ion}\textcolor{blue}{ary} \textcolor{blue}{sp}\textcolor{blue}{ending} \textcolor{black}{caps} \textcolor{black}{under} \textcolor{blue}{the} \textcolor{black}{Bud}\textcolor{blue}{get} \textcolor{blue}{Control} \textcolor{blue}{Act} \textcolor{blue}{(}\textcolor{black}{BC}\textcolor{blue}{A}\textcolor{blue}{).} \textcolor{blue}{It} \textcolor{blue}{also} \textcolor{black}{provides} \textcolor{blue}{a} \textcolor{blue}{detailed} \textcolor{blue}{analysis} \textcolor{blue}{of} \textcolor{blue}{the} \textcolor{black}{appropri}\textcolor{blue}{ations} \textcolor{blue}{process} \textcolor{blue}{for} \textcolor{blue}{D}\textcolor{blue}{HS}\textcolor{black}{,} \textcolor{blue}{including} \textcolor{blue}{the} \textcolor{black}{various} \textcolor{blue}{comm}\textcolor{blue}{itte}\textcolor{blue}{es} \textcolor{blue}{and} \textcolor{black}{sub}\textcolor{blue}{comm}\textcolor{blue}{itte}\textcolor{blue}{es} \textcolor{blue}{involved}\textcolor{blue}{,} \textcolor{black}{and} \textcolor{blue}{the} \textcolor{blue}{role} \textcolor{blue}{of} \textcolor{blue}{the} \textcolor{black}{Cong}\textcolor{blue}{r}\textcolor{blue}{essional} \textcolor{blue}{Bud}\textcolor{blue}{get} \textcolor{blue}{Office} \textcolor{black}{(}\textcolor{blue}{C}\textcolor{blue}{BO}\textcolor{blue}{)} \textcolor{blue}{and} \textcolor{blue}{the} \textcolor{black}{Government} \textcolor{blue}{Account}\textcolor{blue}{ability} \textcolor{blue}{Office} \textcolor{blue}{(}\textcolor{blue}{GA}\textcolor{black}{O}\textcolor{blue}{).}
    
    \textcolor{black}{The} \textcolor{blue}{report} \textcolor{black}{highlight}\textcolor{blue}{s} \textcolor{blue}{the} \textcolor{black}{key} \textcolor{blue}{issues} \textcolor{black}{and} \textcolor{black}{deb}\textcolor{blue}{ates} \textcolor{blue}{surrounding} \textcolor{black}{D}\textcolor{blue}{HS} \textcolor{blue}{appropri}\textcolor{blue}{ations}\textcolor{blue}{,} \textcolor{blue}{including} \textcolor{black}{fund}\textcolor{blue}{ing} \textcolor{blue}{for} \textcolor{blue}{border} \textcolor{blue}{security}\textcolor{blue}{,} \textcolor{black}{imm}\textcolor{blue}{igration} \textcolor{blue}{enfor}\textcolor{blue}{cement}\textcolor{blue}{,} \textcolor{black}{cy}\textcolor{blue}{ber}\textcolor{blue}{security}\textcolor{blue}{,} \textcolor{blue}{and} \textcolor{blue}{dis}\textcolor{black}{aster} \textcolor{blue}{response}\textcolor{blue}{.} \textcolor{blue}{It} \textcolor{blue}{also} \textcolor{black}{discuss}\textcolor{blue}{es} \textcolor{blue}{the} \textcolor{blue}{impact} \textcolor{blue}{of} \textcolor{blue}{the} \textcolor{black}{B}\textcolor{blue}{CA} \textcolor{blue}{on} \textcolor{black}{D}\textcolor{blue}{HS} \textcolor{blue}{appropri}\textcolor{black}{ations} \textcolor{blue}{and} \textcolor{blue}{the} \textcolor{black}{potential} \textcolor{blue}{for} \textcolor{blue}{future} \textcolor{black}{changes} \textcolor{blue}{to} \textcolor{blue}{the} \textcolor{black}{sp}\textcolor{blue}{ending} \textcolor{blue}{caps}\textcolor{blue}{.}
    
    \textcolor{blue}{Over}\textcolor{black}{all}\textcolor{blue}{,} \textcolor{blue}{the} \textcolor{blue}{report} \textcolor{blue}{provides} \textcolor{blue}{a} \textcolor{black}{compreh}\textcolor{blue}{ensive} \textcolor{blue}{analysis} \textcolor{blue}{of} \textcolor{blue}{the} \textcolor{black}{annual} \textcolor{blue}{appropri}\textcolor{blue}{ations} \textcolor{blue}{for} \textcolor{blue}{D}\textcolor{blue}{HS} \textcolor{black}{and} \textcolor{blue}{the} \textcolor{black}{factors} \textcolor{blue}{that} \textcolor{blue}{influence} \textcolor{blue}{the} \textcolor{black}{allocation} \textcolor{blue}{of} \textcolor{blue}{fund}\textcolor{blue}{ing}\textcolor{blue}{.} \textcolor{blue}{It} \textcolor{black}{is} \textcolor{blue}{a} \textcolor{blue}{valuable} \textcolor{blue}{resource} \textcolor{blue}{for} \textcolor{blue}{polic}\textcolor{black}{ym}\textcolor{blue}{akers}\textcolor{blue}{,} \textcolor{black}{anal}\textcolor{blue}{yst}\textcolor{blue}{s}\textcolor{blue}{,} \textcolor{blue}{and} \textcolor{blue}{st}\textcolor{black}{ake}\textcolor{blue}{hold}\textcolor{blue}{ers} \textcolor{blue}{interested} \textcolor{blue}{in} \textcolor{blue}{understanding} \textcolor{black}{the} \textcolor{blue}{complex}\textcolor{blue}{ities} \textcolor{blue}{of} \textcolor{blue}{D}\textcolor{blue}{HS} \textcolor{black}{appropri}\textcolor{blue}{ations} \textcolor{blue}{and} \textcolor{blue}{the} \textcolor{black}{challeng}\textcolor{blue}{es} \textcolor{blue}{facing} \textcolor{blue}{the} \textcolor{black}{department} \textcolor{blue}{in} \textcolor{blue}{the} \textcolor{black}{coming} \textcolor{blue}{years}\textcolor{blue}{.}

\end{tcolorbox}

\end{document}

