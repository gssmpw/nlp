%%%%%%%% ICML 2025 EXAMPLE LATEX SUBMISSION FILE %%%%%%%%%%%%%%%%%

\documentclass{article}

% Recommended, but optional, packages for figures and better typesetting:
\usepackage{comment}
\usepackage{microtype}
\usepackage{graphicx}
% \usepackage{subfigure}
\usepackage{multirow} 
\usepackage{booktabs} % for professional tables
\usepackage{adjustbox}
\usepackage{verbatim}
\usepackage{graphicx}
\usepackage{hyperref}
\usepackage{url}
\usepackage{makecell}
\usepackage{array}
% \usepackage{subfigure}
\usepackage{wrapfig}
\usepackage{enumerate}
\usepackage{enumitem}
\usepackage{soul}
\usepackage{subfig}
\usepackage[table]{xcolor}
% \usepackage{inconsolata}
\usepackage[figuresright]{rotating} 
% \usepackage[svgnames]{xcolor}
\usepackage{tcolorbox}
\definecolor{myblue}{HTML}{F0F8FF} 
\definecolor{mypurple}{HTML}{E6E6FA} 
\definecolor{mygreen}{HTML}{F0FFF0} 
\definecolor{mypink}{HTML}{FFF0F5} 
\newcommand{\fengzhuo}[1]{{\color{red}  [\mathrm{Fengzhuo:} #1]}}
\newcommand{\cunxiao}[1]{{\color{magenta}  [\mathrm{Cunxiao:} #1]}}
\newcommand{\penghui}[1]{{\color{blue}  [\mathrm{Penghui:} #1]}}

\newcommand{\att}{\mathsf{attn}}
\newcommand{\mha}{\mathsf{mha}}
\newcommand{\pe}{\mathsf{RoPE}}
\newcommand{\tilmha}{\widetilde{\mathsf{mha}}}
\newcommand{\ff}{\mathsf{ffn}}
\newcommand{\sm}{\mathsf{softmax}}
\newcommand{\lnor}{\mathsf{LN}}
\newcommand{\llm}{\mathsf{transformer}}
\newcommand{\pt}{\texttt{prompt}}
\newcommand{\sink}{\mathsf{lazy}}
\newcommand{\id}{\mathsf{Id}}
\newcommand{\unemb}{\mathsf{unemb}}
\newcommand{\lip}{{\mathsf{lip}}}
\newcommand{\test}{\texttt{test}}
\newcommand{\argmin}{\text{argmin}}
\usepackage{acronym}
\acrodef{mha}[MHA]{Multi-Head Attention}
\acrodef{ff}[FF]{Feed-Forward}
\acrodef{mle}[MLE]{Maximum Likelihood Estimate}
% hyperref makes hyperlinks in the resulting PDF.
% If your build breaks (sometimes temporarily if a hyperlink spans a page)
% please comment out the following usepackage line and replace
% \usepackage{icml2025} with \usepackage[nohyperref]{icml2025} above.
\usepackage{hyperref}


% Attempt to make hyperref and algorithmic work together better:
\newcommand{\theHalgorithm}{\arabic{algorithm}}

% Use the following line for the initial blind version submitted for review:
% \usepackage{icml2025}

% If accepted, instead use the following line for the camera-ready submission:
\usepackage[accepted]{icml2025}

% For theorems and such
\usepackage{amsmath}
\usepackage{amssymb}
\usepackage{mathtools}
\usepackage{amsthm}
\newcommand{\CG}{\mathcal{G}\xspace}
\newcommand{\CV}{\mathcal{V}\xspace}
\newcommand{\CE}{\mathcal{E}\xspace}
\newcommand{\CA}{\mathcal{A}\xspace}
\newcommand{\CF}{\mathcal{F}\xspace}
\newcommand{\CR}{\mathcal{R}\xspace}
\newcommand{\CB}{\mathcal{B}\xspace}
\newcommand{\CX}{\mathcal{X}\xspace}
\newcommand{\CK}{\mathcal{K}\xspace}
\newcommand{\CM}{\mathcal{M}\xspace}
\newcommand{\CC}{\mathcal{C}\xspace}
\newcommand{\CL}{\mathcal{L}\xspace}
\newcommand{\CI}{\mathcal{I}\xspace}
\newcommand{\CQ}{\mathcal{Q}\xspace}
\newcommand{\CO}{\mathcal{O}\xspace}
\newcommand{\CP}{\mathcal{P}\xspace}
\newcommand{\CS}{\mathcal{S}\xspace}
\newcommand{\CT}{\mathcal{T}\xspace}
\newcommand{\CJ}{\mathcal{J}\xspace}
\usepackage[para]{footmisc}
\usepackage{subfig}
% \usepackage{subcaption}
% \usepackage{array}
% \usepackage{colortbl}



% if you use cleveref..
\usepackage[capitalize,noabbrev]{cleveref}

%%%%%%%%%%%%%%%%%%%%%%%%%%%%%%%%
% THEOREMS
%%%%%%%%%%%%%%%%%%%%%%%%%%%%%%%%
\DeclareRobustCommand{\mylogo}{\adjustbox{valign=c}{\includegraphics[width=1.2cm]{Figure/logo.pdf}}}

\theoremstyle{plain}
\newtheorem{theorem}{Theorem}[section]
\newtheorem{proposition}[theorem]{Proposition}
\newtheorem{lemma}[theorem]{Lemma}
\newtheorem{corollary}[theorem]{Corollary}
\theoremstyle{definition}
\newtheorem{definition}[theorem]{Definition}
\newtheorem{assumption}[theorem]{Assumption}
\theoremstyle{remark}
\newtheorem{remark}[theorem]{Remark}

% Todonotes is useful during development; simply uncomment the next line
%    and comment out the line below the next line to turn off comments
%\usepackage[disable,textsize=tiny]{todonotes}
\usepackage[textsize=tiny]{todonotes}


% The \icmltitle you define below is probably too long as a header.
% Therefore, a short form for the running title is supplied here:
\icmltitlerunning{LongSpec: Long-Context Speculative Decoding with Efficient Drafting and Verification}

\begin{document}

\twocolumn[
% \icmltitle{LongSpec: Memory-efficient Draft Model with Parallel Tree Verification \\
%            for Lossless Long-Context Speculative Decoding}
% \icmltitle{LongSpec: Memory-Efficient Long-Context Speculative Decoding \\
%            with Sink Position ID and Aggregated Attention}
% \icmltitle{\textsc{LongSpec}: Long-Context Speculative Decoding \\
%            with Efficient Drafting and Verification}

\icmltitle{
  \texorpdfstring{
    \mylogo~~\LARGE \textsc{LongSpec}:~~~~~~~~~~~~~~~\\ \Large Long-Context Speculative Decoding with Efficient Drafting and Verification
  }{
    LongSpec: Long-Context Speculative Decoding with Efficient Drafting and Verification
  }
}

% It is OKAY to include author information, even for blind
% submissions: the style file will automatically remove it for you
% unless you've provided the [accepted] option to the icml2025
% package.

% List of affiliations: The first argument should be a (short)
% identifier you will use later to specify author affiliations
% Academic affiliations should list Department, University, City, Region, Country
% Industry affiliations should list Company, City, Region, Country

% You can specify symbols, otherwise they are numbered in order.
% Ideally, you should not use this facility. Affiliations will be numbered
% in order of appearance and this is the preferred way.
\icmlsetsymbol{equal}{*}

\begin{icmlauthorlist}
\icmlauthor{Penghui Yang}{equal,2}
\icmlauthor{Cunxiao Du}{equal,1}
\icmlauthor{Fengzhuo Zhang}{3}
\icmlauthor{Haonan Wang}{3}
\icmlauthor{Tianyu Pang}{1}
\icmlauthor{Chao Du}{1}
\icmlauthor{Bo An}{2}
\end{icmlauthorlist}

\icmlaffiliation{1}{Sea AI Lab, Singapore}
\icmlaffiliation{2}{Nanyang Technological University}
\icmlaffiliation{3}{National University of Singapore}

\icmlcorrespondingauthor{Penghui Yang}{phyang.cs@gmail.com}
\icmlcorrespondingauthor{Cunxiao Du}{ducx@sea.com}
\icmlcorrespondingauthor{Fengzhuo Zhang}{fzzhang@u.nus.edu}

% You may provide any keywords that you
% find helpful for describing your paper; these are used to populate
% the "keywords" metadata in the PDF but will not be shown in the document
\icmlkeywords{Machine Learning, ICML}

\vskip 0.3in]

% this must go after the closing bracket ] following \twocolumn[ ...

% This command actually creates the footnote in the first column
% listing the affiliations and the copyright notice.
% The command takes one argument, which is text to display at the start of the footnote.
% The \icmlEqualContribution command is standard text for equal contribution.
% Remove it (just {}) if you do not need this facility.

% \printAffiliationsAndNotice{}  % leave blank if no need to mention equal contribution
% \printAffiliationsAndNotice{\icmlEqualContribution} % otherwise use the standard text.
\printAffiliationsAndNoticeArxiv{\icmlEqualContribution}  % leave blank if no need to mention equal contribution
% \printAffiliationsAndNotice{\icmlEqualContribution} 
\begin{abstract}
Speculative decoding has become a promising technique to mitigate the high inference latency of autoregressive decoding in Large Language Models (LLMs). Despite its promise, the effective application of speculative decoding in LLMs still confronts three key challenges: the increasing memory demands of the draft model, the distribution shift between the short-training corpora and long-context inference, and inefficiencies in attention implementation. 
In this work, we enhance the performance of speculative decoding in long-context settings by addressing these challenges. First, we propose a memory-efficient draft model with a constant-sized Key-Value (KV) cache. Second, we introduce novel position indices for short-training data, enabling seamless adaptation from short-context training to long-context inference. 
Finally, we present an innovative attention aggregation method that combines fast implementations for prefix computation with standard attention for tree mask handling, effectively resolving the latency and memory inefficiencies of tree decoding.
Our approach achieves strong results on various long-context tasks, including repository-level code completion, long-context summarization, and o1-like long reasoning tasks, demonstrating significant improvements in latency reduction.
The code is available at \url{https://github.com/sail-sg/LongSpec}.
\end{abstract}

\section{Introduction}\label{sec:intro}

In computational finance, Monte Carlo simulations are used extensively to estimate the expected value of financial payoffs based on the solution of stochastic differential equations (SDEs) which model the evolution of stock prices, interest rates, exchange rates and other quantities \cite{glasserman04}.  Monte Carlo methods are very general and flexible, but for high accuracy it requires generating a large number of costly SDE path approximations, which has motivated research into a number of variance reduction or, equivalently, cost reduction techniques. One such method is
Multilevel Monte Carlo (MLMC), which was proposed in \cite{GILES2008} and was adapted for various applications that are summarised in \cite{Giles_overview17} and successfully combined with other methods such as quasi-Monte Carlo methods. The main idea of MLMC is to approximate the payoff using different time stepping resolutions when numerically solving the underlying SDE and to generate an optimal number of samples on each level, such that the overall computational cost is minimised subject to the desired bound on the variance. %, such that the total computational cost is minimised. 
The computational savings come from the fact that most samples are computed on the coarser levels and hence are less expensive while only a few samples from the finest levels are required \cite{GILES2008}.


Among the directions in which the computational cost 
of MLMC methods could further be reduced, an important avenue is the use of lower precision calculations, especially for the first Monte Carlo levels where the targeted accuracy is relatively low. 
 An overview of the research on mixed precision for the standard Monte Carlo (MC) framework is provided in \cite{ChowMixedPrecisionStandardMC} but only a few references study the potential of low precision computation in the MLMC framework \cite{Rounding_error_oliver}. To the best of our knowledge, the only MLMC framework with customised precision in the literature is \cite{brugger2014mixed}, but they use a uniform precision for all operations on each Monte Carlo level instead of optimising 
 the precision of each intermediary variable to reduce as much as possible the cost of path generation.
 
An important motivation for an MLMC framework with variable precision would be performing the low precision computations on reconfigurable hardware devices such as Field Programmable Gate Arrays (FPGAs). FPGAs contain customizable logic blocks and connectors that make it easy to adapt the digital circuit architecture for a specific application, leading to a highly parallel and optimised implementation. Therefore they are successfully exploited in applications that require high speed and have high computational workload, such as signal processing \cite{woods2008fpga}, and real time applications like high frequency trading \cite{HFT1,HFT2}. That is why a number of previous works in hardware architecture design implemented the MLMC algorithm to price financial options using FPGAs as accelerators, which resulted in improved speed and power efficiency compared to full CPU architectures \cite{Schryver2013AMM}. The paper \cite{lindsey2016domain} also proposed 
a Domain Specific Language to automate the configuration of FPGAs for this specific application. However, only \cite{brugger2014mixed} proposed a heuristic to reduce the precision in calculations.

In addition, all aforementioned works considered that the random number generation (RNG) is performed in single or double precision. Yet in most cases an important portion of the workload in the overall MLMC simulation comes from the RNG and in \cite{brugger2014mixed} this limited the total computational savings.
To reduce the cost of MLMC simulations in particular those based on the Geometric Brownian Motion (GBM), \cite{approximateICDF_Oliver, NestedOliver} have proposed to use approximate random numbers that are generated by applying an approximation of the inverse CDF to uniform random numbers. In \cite{NestedOliver}, the authors proposed a way to integrate these lower precision random variables into a \textit{nested} MLMC framework and completed a numerical analysis to bound the resulting error at each MC level by a product of the time step and the error in the random number approximation. The same authors show in \cite{approximateICDF_Oliver} that using approximate random variables reduces the cost of path generation by a factor 7.


In this paper we propose a nested MLMC framework that combines the use of approximate random normal variables and lower precision calculations to reduce the computational cost of MLMC even further than \cite{brugger2014mixed,NestedOliver}. We illustrate the efficiency of our framework in Matlab, after making several assumptions on the cost of operations and size of the errors that we carefully justify. We focus on the case of GBM and use the approximate RNG methods presented in \cite{approximateICDF_Oliver} as well as a new slightly modified method that combines CDF inversion and the central limit theorem. To choose the precision of the variables in the low precision path generation, we introduce a novel method to optimise the bit-widths. This optimisation is performed before the main path generation loop is executed and is based on a linear model of the payoff error  
due to rounding when computing in low precision. The error model relies on algorithmic differentiation in a similar manner to \cite{unifying-bwoptim,bitwidth-AD,ADAPT}. The bit-width optimisation procedure can be performed off-line, so this stage can be excluded from the on-line time complexity of our framework. The user specified desired accuracy is then enforced by calculating on-line the number of samples that need to be generated.

In terms of hardware design, we suggest implementing the low precision path generation on FPGAs and the full-precision ones on a CPU or GPU. 
The FPGA offers enough flexibility to define a separate bit-width for every variable in the low precision path generation, and can be reconfigured periodically to update the bit-widths when the market parameters have changed considerably. 


The paper is organized as follows : \Cref{sec:MLMC} introduces MLMC and nested MLMC to make clear the estimator that is implemented in our framework. Then in \Cref{sec:RNG} we detail the methods that could be used to obtain approximate random normally distributed numbers very cheaply for the low precision path generation. In \Cref{sec:error_model} and \Cref{sec:costModel} we propose an error model and a cost model (resp.) that we then use to formulate the optimisation problem that is solved to obtain the optimal bit-widths of fixed point variables in \Cref{sec:optimisation}. Finally we summarise our results and future directions in \Cref{sec:conclusion}.




\section{Related Work}
\label{sec:related_work}

The original investigation \cite{gibson1979ecological} on the relationship between visual perception and human action defines \emph{affordance} as the opportunities for interaction with the surrounding environment. Behavioral studies on regular and cognitively impaired persons have shown evidence that perception results in both visual and motor signals in the human brain. An extended study \cite{anderson2002attentional} shows that visual attention to the spatial characteristics of the perceived objects initiates automatic motor signals for different actions. In computer vision, human affordance learning involves novel pose prediction such that the estimated pose represents a valid human action within the scene context. The task is fundamental to many problems requiring robust semantic reasoning about the environment, such as human motion synthesis \cite{wang2021scene} and scene-aware human pose generation \cite{wang2017binge, roy2016multi, zhang2022inpaint, yao2023scene}.

Earlier methods of affordance learning have explored knowledge mining \cite{zhu2014reasoning} and multimodal feature cues \cite{roy2016multi} to address the problem. In \cite{zhu2014reasoning}, the authors use a Markov Logic Network for constructing a knowledge base by extracting several object attributes from different image and metadata sources, which can perform various downstream visual inference tasks without any additional classifier, including zero-shot affordance prediction. In \cite{roy2016multi}, the authors use depth map, surface normals, and segmentation map as multimodal cues to train a multi-scale convolutional neural network (CNN) for scene-level semantic label assignment associated with specific human actions. In \cite{do2018affordancenet}, the authors design a multi-branch end-to-end CNN with two separate pathways for object detection and affordance label assignment to achieve high real-time inference throughput. Researchers \cite{chuang2018learning} have also explored socially imposed constraints for affordance learning. In \cite{chuang2018learning}, the authors propose a graph neural network (GNN) to propagate contextual scene information from egocentric views for action-object affordance reasoning.

Probabilistic modeling of scene-aware human motion generation also involves semantic reasoning of human interaction with the environment. Initial works on human motion synthesis have taken different architectural approaches, such as sequence-to-sequence models \cite{barsoum2018hp}, generative adversarial networks (GAN) \cite{barsoum2018hp, cai2018deep, yang2018pose}, graph convolutional networks (GCN) \cite{yan2019convolutional}, and variational autoencoders (VAE) \cite{guo2020action2motion}. However, these methods have mostly ignored the role of environmental semantics. Due to potential uncertainty in human motion, in a recent approach \cite{wang2021scene}, the authors address such motion synthesis with a GAN conditioned on scene attributes and motion trajectory to predict probable body pose dynamics.

One key challenge of human affordance generation in 2D scenes is the lack of large-scale datasets with rich pose annotations. In \cite{wang2017binge}, the authors compile the only public dataset of annotated human body poses in complex 2D indoor scenes by extracting frames from sitcom videos. Aiming to generate a contextually valid human affordance at a user-defined location, the authors propose sampling the scale and deformation parameters for an existing human pose template using a VAE conditioned on the localized image patches as scene context. In \cite{zhang2022inpaint}, the authors introduce a two-stage GAN architecture for achieving a similar goal by estimating the affine bounding box parameters to localize a probable human in the scene and then generating a potential body pose at that location. The method uses the input scene, corresponding depth, and segmentation maps as semantic guidance. In \cite{yao2023scene}, the authors propose a transformer-based approach with knowledge distillation for generating human affordances in 2D indoor scenes.



% \section{Preliminaries}
\label{sec:prelim}

\subsection{Large Vision-Language Models}




Large Vision-Language Models (LVLM) are generative models that are typically composed of a visual model $h(\cdot)$, a language model parameterized by $\vtheta$, and a fusion model $g(\cdot)$.
The most popular implementations of LVLMs, such as Llava~\cite{liu2024visual}, combine a pre-trained visual encoder (e.g., CLIP~\cite{radford2021learning}) and a pre-trained Large Language Model (LLM) (e.g., Vicuna~\cite{chiang2023vicuna}) by training a projection network as the fusion model to convert extracted visual features into the LLM's embedding space in a process known as visual instruction tuning.
During inference, an LVLM takes an input image $\mI$ and a text prompt $\mX=[\rx_1, ..., \rx_l]$, and outputs a text response $\mY=[\ry_1, ..., \ry_m]$, where $\rx_i$ and $\ry_j$ are individual tokens. This is achieved by first converting the image into a sequence of visual tokens using the visual model $\mV=[\rv_1, ..., \rv_k]=g\circ h(\mI)$ and then sampling the response from the conditional distribution in an autoregressive manner:
$p_\vtheta(\mY|\mX,\mV) = \prod_{j=1}^m p_\vtheta(\ry_j|\mX,\mV,\mY_{<j})$.

\paragraph{Hallucination of LVLMs.}
The problem of hallucination originates from the space of language models, where the generated text response is either non-factual (conflicts with verifiable facts) or unfaithful (does not follow the user's instructions). In the context of LVLMs, hallucination refers to the phenomenon where the generated text response deviates from the provided visual content. Common types of LVLM hallucinations include \textit{object} hallucination (e.g., falsely identifying non-existent objects), \textit{attribute} hallucination (e.g., wrong color, shape, or material),
and \textit{relation} hallucination (e.g., human-object interaction, relative position)~\cite{bai2024hallucination}.


\subsection{Split Conformal Prediction}



Split conformal prediction (SCP)~\cite{vovk2005algorithmic,shafer2008tutorial} is a distribution-free method for quantifying the uncertainty of black-box prediction algorithms by constructing prediction sets with finite-sample coverage properties. 

\paragraph{Coverage Guarantee.}
For a black-box prediction function $f: \gX \rightarrow \gY$, let $\{(X_i, Y_i)\}_{i=1}^{n+1}$ be an exchangeable set of feature and label pairs sampled from the joint distribution on $\gX\times\gY$. The goal of split conformal prediction is to use the calibration data $\{(X_i, Y_i)\}_{i=1}^{n}$ and $f$ to construct a prediction set $\hat{C}: \gX \rightarrow 2^\gY$ for the new data point such that it achieves valid \textit{coverage}, i.e., containing the true label with high probability
$\prob\big(Y_{n+1}\in \hat{C}(X_{n+1})\big) \geq 1-\alpha$
for any user-specified error rate $\alpha \in (0, 1)$.

\paragraph{Conformal Calibration.}
Suppose there is a \textit{conformity score} function $S(X, Y)\in \sR$ that measures how well a given sample \textit{conforms} to the observed data.
The split conformal procedure uses the calibration data set $\{(X_i, Y_i)\}_{i=1}^{n}$ to derive \textit{conformity} scores $\{S(X_i, Y_i) \}_{i=1}^n$, where a larger value indicates the model is more confident about the prediction being true. To calibrate the prediction set to the desired level of coverage, we then compute a threshold $\hat{\tau}$
that is approximately the $1-\alpha$ quantile of the conformity scores.
At the time of inference, given a new data point $X_{n+1}$, we construct the prediction set as $\hat{C}(X_{n+1}) = \{y\in\gY: S(X, y) \geq \hat{\tau} \}$. If the data are exchangeable, then this prediction set will satisfy the desired coverage property.

% \section{Preliminary}
% \subsection{Problem Formulation}
% \subsection{Motivation}

% The ability to handle long-context inputs has become increasingly critical because of applications such as long-document summarization and repository-level code completion. Speculative decoding, as an important technique for accelerating LLM inference, exhibits great potential for addressing the computational challenges in long-context scenarios. However, current speculative decoding methods in these scenarios only use draft models that do not require additional training. While this approach provides simplicity, it inherently limits their effectiveness and adaptability to diverse tasks.

% Despite their success, state-of-the-art speculative decoding methods designed for short-context scenarios are faced with several critical limitations when applied to long-context scenarios.

% \begin{itemize}

% \item \textbf{ Growth in Memory Requirements.} As the decoding length increases, current methods require progressively larger KV caches. This introduces significant storage overhead, especially in long-context settings where memory efficiency is of great importance. Addressing this issue demands an innovative approach to ensure constant memory usage, irrespective of context length.

% \item \textbf{Dependence on Predefined Model Structures.} The draft models used by these methods rely heavily on information from the target model, including its rotary position embeddings (RoPE). Since the target model's RoPE base is fixed, the draft model cannot be directly trained on short texts with a small RoPE base, as is common in long-context scenarios. This limitation hinders the draft model's ability to generalize from short-context training to long-context applications. A novel mechanism is needed to bridge this gap without altering the RoPE base.

% \item \textbf{Incompatibility with Flash Attention.} Current SOTA techniques often employ tree decoding, a method incompatible with Flash Attention. Flash Attention is a widely adopted acceleration mechanism critical for reducing latency and memory usage during inference. The inability to integrate tree decoding with Flash Attention exacerbates latency and memory challenges, particularly in long-context scenarios.

% \end{itemize}



\section{Methodology}
\paragraph{Preliminaries.}
We primarily focus on the homologous model merging, in which $\boldsymbol{\theta}_i$ all come from the same base model $\boldsymbol{\theta}_{\rm{base}}$. Given $K$ tasks $\{T_1,T_2,\cdots,T_K\}$ and $K$ corresponding fine-tuned models with parameters $\{\boldsymbol{\theta}_1,\boldsymbol{\theta}_2,\cdots,\boldsymbol{\theta}_K\}$, model merging aims to combine $K$ fine-tuned models into one single model simultaneously performing on $\{T_1,T_2,\cdots,T_K\}$ without post-training~\cite{method_p1_1,method_p1_2}.
Task vector~\cite{ilharco2023editing,yang2024adamerging} is a key element in merging method which could enhances the base model‘s ability or enable the model to handle other tasks. Specifically, for task $T_i$, the task vector $\boldsymbol\tau_i\in \mathbb{R}^D$ is defined as the vector obtained by subtracting the SFT weights $\boldsymbol{\theta}_i$ from the base model weight
$\boldsymbol{\theta}_{\rm{base}}$, \emph{i.e.}, $\boldsymbol\tau_i=\boldsymbol{\theta}_i-\boldsymbol{\theta}_{\rm{base}}$. The merged model could be denoted as $\boldsymbol{\theta}_m=\boldsymbol{\theta}_{\rm{base}}+\sum_i \lambda_i\boldsymbol{\tau}_i$, which $\lambda_i$ is the scaling factor measuring the importance of task vector. For clarification, we also denote the neuron set in $\boldsymbol{\theta}_i$ as $\mathcal{N}_i$, the neuron set in $\boldsymbol{\tau}_i$ as $\mathcal{T}_i$.



\begin{algorithm}[!ht]
    \caption{LED-Merging}
    \label{alg1}
    \begin{algorithmic}[1]
        \REQUIRE  base model $\boldsymbol{\theta}_{\rm{base}}$, SFT models $\{\boldsymbol{\theta}_{i}\mid i\in [K]\}$, mask ratios \{$r_{i} \mid i\in [K]\}$, scaling factors $\{\lambda_i\mid i\in[K]\}$, location datasets $\{\mathcal{X}_{i}\mid i\in[K]\}$
        \ENSURE merged parameter $\boldsymbol{\theta}_{m}$
        \STATE $\mathcal{M}\leftarrow\phi$
        \STATE $\boldsymbol{\theta}_{m}\leftarrow \boldsymbol{\theta}_{\rm{base}}$
        \FOR{$i\in [K]$}
        \STATE $I(\boldsymbol{\theta}_i)=\mathbb{E}_{x\sim \mathcal{X}_i}|\boldsymbol{\theta}_{i}\odot \nabla_{\boldsymbol{\theta}_i}\mathcal{L}(x)|$
        \STATE $I(\boldsymbol{\theta}_{\rm{base}})=\mathbb{E}_{x\sim \mathcal{X}_i}|\boldsymbol{\theta}_{\rm{base}}\odot \nabla_{\boldsymbol{\theta}_{\rm{base}}}\mathcal{L}(x)|$
        
        \STATE calculate $\mathcal{T}^{r_i}_{i}$ following Equation \ref{vote}
        \STATE  $\mathcal{M}\leftarrow \mathcal{M}\cup\{\mathcal{T}^{r_i}_i\}$
       
        
   
        
        
        \ENDFOR  
        \FOR{$i\in [K]$}
        
        \STATE calculate $\text{Disjoint}(\mathcal{T}_i^{r_i})$ use Equation~\ref{disjoint_safety}
        \STATE $\boldsymbol{m}_i \leftarrow \boldsymbol{0}$
        \FOR{$d\in \mathcal{T}_i^{r_i}$}
        \STATE $\boldsymbol{m}_{i,d}=1$
        \ENDFOR
        \STATE $\boldsymbol{\theta}_{m}\leftarrow \boldsymbol{\theta}_{m}+\lambda_i \boldsymbol{\tau}_i\odot \boldsymbol{m}_{i}$
        \ENDFOR
    \end{algorithmic}
\end{algorithm}
    %\vspace{-5pt}
\begin{figure*}[h!]
    \centering
    \includegraphics[width=\linewidth]{figs/pipeline_v2.pdf}
    \vspace{-40mm}
    \caption{Overview of our two-stage training pipeline {\ours}.}
    \label{fig:pipeline}
\end{figure*}


\paragraph{LED-Merging: Location, Election, and Disjoint Merging}
To address the neuron misidentification and interference issues in existing model merging methods, we propose LED-Merging (Location, Election, and Disjoint Merging). Specifically, previous studies \cite{modelstock, ilharco2023editing, tiesmerging} fail to accurately identify safety-related neurons in task vectors with a single magnitude score, namely \textit{neuron misidentification}. Meanwhile, there exists an interference between safety-related and utility-related task vector neurons during the merging process, namely \textit{neuron interference}. To address neuron misidentification, we first locate important neurons both in the base and fine-tuned models and then elect neurons from the task vector considering these two scores together. Subsequently, to mitigate the interference, we introduce a disjoint step, isolating these important neurons so that they influence different base neurons. The whole process is illustrated in Figure~\ref{fig:method}. 




In the location and election step, we consider the importance score from base and fine-tuned models simultaneously to locate task-specific neurons. In this way, it is more accurate than relying on the magnitude score alone because task-specific neurons with high importance score in the fine-tuned model may not necessarily score high in the base model, and vice versa.

{\textbf{Location}}.  We first calculate importance scores for each neuron in a base/fine-tuned model. Given a location dataset $\mathcal{X}_i=\{(x,y)_k\}$, where $x$ is the question and $y$ is the answer, we calculate the importance scores for the weight $\boldsymbol{\theta}_i\in\mathbb{R}^D$ in any  layer as follows~\cite{snip,spareseGPT,sun2024a}:
\begin{equation}
    I(\boldsymbol{\theta}_i)=\mathbb{E}_{x\sim \mathcal{X}_i}[\boldsymbol{\theta}_i\odot \nabla _{\boldsymbol{\theta}_i}\mathcal{L}(x)],
    \label{location}
\end{equation}
which $\mathcal{L}(x)=-\log p(y\mid x)$ is the conditional negative log-likelihood loss. We choose the SNIP score~\cite{snip} because it balances computational efficiency and performance~\cite{cq}. Please refer to Sec.~\ref{sec:ablation} for the comparison between different location methods. After computing importance scores, we choose top-$r_i$ neurons as the important neuron subset $\mathcal{N}_{i}^{r_i}$ from $I(\boldsymbol{\theta}_i)$.
 
 % After computing locating scores, we select the neurons scoring both high in base and fine-tuned models as important neurons in task vectors. Then in the disjoint step,  with preventing  polysemantic neurons  from receiving gradient updates towards different directions,
 % we use set difference to isolate the safety   and utility-related neurons  and construct corresponding masks for merging process,

{\textbf{Election}}. A natural question is how to select important neurons in the task vector $\boldsymbol{\tau}_i$ based on $I(\boldsymbol{\theta}_{\rm{base}})$ and $I(\boldsymbol{\theta}_{i})$. The important neurons in the base model may be different from neurons in the fine-tuned model. Therefore, we introduce the following election strategy to select neurons with high scores in both base and fine-tuned models:
\begin{equation}
    \mathcal{T}_i^{r_i}=\mathcal{N}_i^{r_i}\cap \mathcal{N}_{\rm{base}}^{r_i}.
    \label{vote}
\end{equation}
\emph{Remark}. We compare different choosing methods, including scoring low or high in base or fine-tuned model in Section~\ref{sec:ablation} and find that Equation \ref{vote} achieves the best performance.





{\textbf{Disjoint}}. As important neurons from different task vectors may conflict with each other at the same position, we use the set difference to disjoint the neurons from others to prevent interference:
\begin{equation}
    \text{Disjoint}(\mathcal{T}^{r_i}_{i})=\mathcal{T}^{r_i}_{i}-\mathop{\cup}\limits_{{J}\subsetneqq [K],|J|\geq 2}\mathop{\cap}\limits_{j\in {J}}\mathcal{T}^{r_j}_{j}.
    \label{disjoint_safety}
\end{equation}

Next, we construct a mask $\boldsymbol{m}_i\in\mathbb{R}^D$ to implement disjoint in the merging process. Specifically, this mask $\boldsymbol{m}_i$ is used to select neurons from $\mathcal{T}_i$. The mask ratio is $r_i$, where $r\in(0,1]$. The mask $\boldsymbol{m}_i$ can be derived from:
\begin{equation}
    \boldsymbol{m}_{i,d}=\begin{aligned} &\left\{ \begin{array}{ll} 1, & \text{if } d\in \text{Disjoint}(\mathcal{T}_{i}^{r_i}), \\ 0, & \text{otherwise}. \end{array} \right. \end{aligned}
    \label{mask_safety}
\end{equation}


% \subsection{Merging Models with Masks}
{\textbf{Merging}}. The final
merged task vector $\boldsymbol{\tau}_m$ is as follows:
\begin{equation}
    \boldsymbol{\tau}_m= \sum_i \lambda_i\boldsymbol{\tau}_{i}\odot\boldsymbol{m}_i.
    \label{merged_task_vector}
\end{equation}
We summarize the workflow in Algorithm \ref{alg1}.




% \begin{table*}[t]
% \centering
% \resizebox{\textwidth}{!}{%
% \begin{tabular}{l|l|l|l|l|l|l|l|l|l|l|l}
% \hline
% Model          & Setting       & \multicolumn{2}{c|}{GovReport} & \multicolumn{2}{c|}{QMSum} & \multicolumn{2}{c|}{Multi-News} & \multicolumn{2}{c|}{LCC} & \multicolumn{2}{c}{RepoBench-P} \\ \hline
%                &               & $\tau$ & Tokens/s & $\tau$ & Tokens/s & $\tau$ & Tokens/s & $\tau$ & Tokens/s & $\tau$ & Tokens/s \\ \hline
% Vicuna-7B      & Vanilla Torch & 1          & 25.25    & 1          & 18.12    & 1          & 27.29    & 1          & 25.25    & 1          & 19.18    \\
%                & Vanilla       & 1          & 45.76    & 1          & 43.68    & 1          & 55.99    & 1          & 54.07    & 1          & 46.61    \\
%                & Seq           & 2.30       & 96.94    & 2.05       & 86.01    & 2.27       & 98.25    & 2.48       & 105.80   & 2.54       & 108.30   \\
%                & Tree          & 3.57       & 102.23   & 3.14       & 88.87    & 3.51       & 100.55   & 3.73       & 107.30   & 3.86       & 110.76   \\ \hline
% Vicuna-13B     & Vanilla Torch & 1          & 17.25    & 1          & 11.86    & 1          & 18.81    & 1          & 17.25    & 1          & 13.44    \\
%                & Vanilla       & 1          & 28.52    & 1          & 27.43    & 1          & 35.01    & 1          & 33.87    & 1          & 29.14    \\
%                & Seq           & 2.15       & 61.79    & 1.76       & 49.62    & 2.23       & 69.21    & 2.34       & 71.22    & 2.30       & 66.33    \\
%                & Tree          & 3.31       & 71.08    & 2.76       & 57.15    & 3.44       & 78.20    & 3.57       & 81.00    & 3.59       & 77.22    \\ \hline
% LongChat-7B    & Vanilla Torch & 1          & 25.27    & 1          & 14.11    & 1          & 27.66    & 1          & 25.27    & 1          & 17.02    \\
%                & Vanilla       & 1          & 42.14    & 1          & 36.87    & 1          & 50.19    & 1          & 54.17    & 1          & 42.69    \\
%                & Seq           & 2.30       & 94.27    & 2.01       & 79.07    & 2.21       & 91.61    & 2.78       & 119.37   & 2.66       & 108.80   \\
%                & Tree          & 3.59       & 101.43   & 3.06       & 85.23    & 3.41       & 97.93    & 4.21       & 122.30   & 4.03       & 115.27   \\ \hline
% LongChat-13B   & Vanilla Torch & 1          & 17.72    & 1          & 12.08    & 1          & 18.74    & 1          & 17.72    & 1          & 13.85    \\
%                & Vanilla       & 1          & 28.56    & 1          & 27.18    & 1          & 35.37    & 1          & 34.58    & 1          & 29.74    \\
%                & Seq           & 2.28       & 65.55    & 2.02       & 56.57    & 2.29       & 71.32    & 2.62       & 79.87    & 2.89       & 83.43    \\
%                & Tree          & 3.58       & 76.26    & 3.15       & 64.41    & 3.50       & 80.48    & 4.01       & 90.92    & 4.46       & 96.96    \\ \hline
% LLaMA3-8B      & Vanilla Torch & 1          & 21.59    & 1          & 18.67    & 1          & 29.91    & 1          & 29.48    & 1          & 22.77    \\
%                & Vanilla       & 1          & 53.14    & 1          & 51.22    & 1          & 56.94    & 1          & 56.73    & 1          & 54.08    \\
%                & Seq           & 2.21       & 84.39    & 1.96       & 73.19    & 2.23       & 88.92    & 2.14       & 85.74    & 2.15       & 82.60    \\
%                & Tree          & 3.25       & 84.57    & 2.99       & 75.68    & 3.36       & 91.11    & 3.28       & 89.33    & 3.39       & 91.28    \\ \hline
% \end{tabular}
% }
% \caption{Speedup ratio and average acceptance length $\tau$ on GovReport, QMSum, Multi-News, LCC, and RepoBench-P when temperature $T=0$.}
% \label{tab:main_t=0}
% \end{table*}

% \begin{table*}[t]
% \centering
% \vspace{.2em}
% \scalebox{0.92}{\begin{tabular}{lccccccccccc}
% \toprule
% \textbf{Setting} & 
% \multicolumn{2}{c}{\textbf{GovReport}} & \multicolumn{2}{c}{\textbf{QMSum}} & 
% \multicolumn{2}{c}{\textbf{Multi-News}} & \multicolumn{2}{c}{\textbf{LCC}} & 
% \multicolumn{2}{c}{\textbf{RepoBench-P}} \\
% \cmidrule(lr){2-3} \cmidrule(lr){4-5} \cmidrule(lr){6-7} \cmidrule(lr){8-9} \cmidrule(lr){10-11}
% & $\tau$ & Tokens/s & $\tau$ & Tokens/s & $\tau$ & Tokens/s & $\tau$ & Tokens/s & $\tau$ & Tokens/s \\
% \midrule

% \multicolumn{11}{l}{\textbf{Vicuna-7B}} \\ \cmidrule(lr){1-1}
% \rowcolor{blue!8} Vanilla HF    & 1.00 & 25.25 & 1.00 & 18.12 & 1.00 & 27.29 & 1.00 & 25.25 & 1.00 & 19.18 \\
% \rowcolor{blue!8} Vanilla FA    & 1.00 & 45.76 & 1.00 & 43.68 & 1.00 & 55.99 & 1.00 & 54.07 & 1.00 & 46.61 \\
% \rowcolor{blue!8} MagicDec      & 2.23 & 41.68 & 2.29 & 42.91 & 2.31 & 44.82 & 2.52 & 46.96 & 2.57 & 48.75 \\
% \rowcolor{blue!8} Tree          & \textbf{3.57} & \textbf{102.23} & \textbf{3.14} & \textbf{88.87} & \textbf{3.51} & \textbf{100.55} & \textbf{3.73} & \textbf{107.30} & \textbf{3.86} & \textbf{110.76} \\

% \multicolumn{11}{l}{\textbf{Vicuna-13B}} \\ \cmidrule(lr){1-1}
% \rowcolor{purple!8} Vanilla HF  & 1.00 & 17.25 & 1.00 & 11.86 & 1.00 & 18.81 & 1.00 & 17.25 & 1.00 & 13.44 \\
% \rowcolor{purple!8} Vanilla FA  & 1.00 & 28.52 & 1.00 & 27.43 & 1.00 & 35.01 & 1.00 & 33.87 & 1.00 & 29.14 \\
% \rowcolor{purple!8} MagicDec    & 2.95 & 38.24 & 2.87 & 37.15 & 2.97 & 39.47 & 2.96 & 38.40 & 2.94 & 36.66 \\
% \rowcolor{purple!8} Tree        & \textbf{3.31} & \textbf{71.08} & \textbf{2.76} & \textbf{57.15} & \textbf{3.44} & \textbf{78.20} & \textbf{3.57} & \textbf{81.00} & \textbf{3.59} & \textbf{77.22} \\

% \multicolumn{11}{l}{\textbf{LongChat-7B}} \\ \cmidrule(lr){1-1}
% \rowcolor{green!8} Vanilla HF   & 1.00 & 25.27 & 1.00 & 14.11 & 1.00 & 27.66 & 1.00 & 25.27 & 1.00 & 17.02 \\
% \rowcolor{green!8} Vanilla FA   & 1.00 & 42.14 & 1.00 & 36.87 & 1.00 & 50.19 & 1.00 & 54.17 & 1.00 & 42.69 \\
% \rowcolor{green!8} MagicDec     & 2.26 & 41.90 & 2.20 & 40.82 & 2.32 & 43.94 & 2.77 & 51.73 & 2.57 & 44.13 \\
% \rowcolor{green!8} Tree         & \textbf{3.59} & \textbf{101.43} & \textbf{3.06} & \textbf{85.23} & \textbf{3.41} & \textbf{97.93} & \textbf{4.21} & \textbf{122.30} & \textbf{4.03} & \textbf{115.27} \\

% \multicolumn{11}{l}{\textbf{LongChat-13B}} \\ \cmidrule(lr){1-1}
% \rowcolor{yellow!8} Vanilla HF  & 1.00 & 17.72 & 1.00 & 12.08 & 1.00 & 18.74 & 1.00 & 17.72 & 1.00 & 13.85 \\
% \rowcolor{yellow!8} Vanilla FA  & 1.00 & 28.56 & 1.00 & 27.18 & 1.00 & 35.37 & 1.00 & 34.58 & 1.00 & 29.74 \\
% \rowcolor{yellow!8} MagicDec    & 2.40 & 31.37 & 2.38 & 30.84 & 2.43 & 32.58 & 2.68 & 35.77 & 2.85 & 35.67 \\
% \rowcolor{yellow!8} Tree        & \textbf{3.58} & \textbf{76.26} & \textbf{3.15} & \textbf{64.41} & \textbf{3.50} & \textbf{80.48} & \textbf{4.01} & \textbf{90.92} & \textbf{4.46} & \textbf{96.96} \\

% \multicolumn{11}{l}{\textbf{LLaMA3-8B}} \\ \cmidrule(lr){1-1}
% \rowcolor{orange!8} Vanilla HF  & 1.00 & 21.59 & 1.00 & 18.67 & 1.00 & 29.91 & 1.00 & 29.48 & 1.00 & 22.77 \\
% \rowcolor{orange!8} Vanilla FA  & 1.00 & 53.14 & 1.00 & 51.22 & 1.00 & 56.94 & 1.00 & 56.73 & 1.00 & 54.08 \\
% \rowcolor{orange!8} MagicDec    & 2.04 & 36.14 & 2.00 & 35.78 & 2.33 & 39.57 & 2.65 & 46.95 & 2.61 & 44.39 \\
% \rowcolor{orange!8} Tree        & \textbf{3.25} & \textbf{84.57} & \textbf{2.99} & \textbf{75.68} & \textbf{3.36} & \textbf{91.11} & \textbf{3.28} & \textbf{89.33} & \textbf{3.39} & \textbf{91.28} \\

% \bottomrule
% \end{tabular}}
% \caption{Speedup ratio ($\tau$) and decoding speed (Tokens/s) across different models and settings. All results are computed at $T=0$.}
% \label{tab:final_table}
% \end{table*}

% \begin{table*}[t]
% \centering
% \vspace{.2em}
% \scalebox{0.7}{
% \begin{tabular}{ccccccccccccccccc}
% \toprule
% \multirow{2}{*}{\textbf{Setting}} &
% \multicolumn{3}{c}{\textbf{GovReport}} &
% \multicolumn{3}{c}{\textbf{QMSum}} &
% \multicolumn{3}{c}{\textbf{Multi-News}} &
% \multicolumn{3}{c}{\textbf{LCC}} &
% \multicolumn{3}{c}{\textbf{Repo-P}} \\
% \cmidrule(lr){2-4} \cmidrule(lr){5-7} \cmidrule(lr){8-10} \cmidrule(lr){11-13} \cmidrule(lr){14-16}
% & $\tau$ & Tokens/s & Speedup
% & $\tau$ & Tokens/s & Speedup
% & $\tau$ & Tokens/s & Speedup
% & $\tau$ & Tokens/s & Speedup
% & $\tau$ & Tokens/s & Speedup \\
% \midrule

% \multicolumn{16}{l}{\textbf{Vicuna-7B}} \\ \cmidrule(lr){1-1}
% % \rowcolor{blue!8}
% Vanilla HF
% & 1.00 & 25.25 & -
% & 1.00 & 18.12 & -
% & 1.00 & 27.29 & -
% & 1.00 & 25.25 & -
% & 1.00 & 19.18 & - \\
% % \rowcolor{blue!8}
% Vanilla FA
% & 1.00 & 45.76 & 1.00
% & 1.00 & 43.68 & 1.00
% & 1.00 & 55.99 & 1.00
% & 1.00 & 54.07 & 1.00
% & 1.00 & 46.61 & 1.00 \\
% % \rowcolor{blue!8}
% MagicDec
% & 2.23 & 41.68 & 0.91
% & 2.29 & 42.91 & 0.98
% & 2.31 & 44.82 & 0.80
% & 2.52 & 46.96 & 0.87
% & 2.57 & 48.75 & 1.05 \\
% % \rowcolor{blue!8}
% LongSpec
% & \textbf{3.57} & \textbf{102.23} & \textbf{2.23}
% & \textbf{3.14} & \textbf{88.87} & \textbf{2.04}
% & \textbf{3.51} & \textbf{100.55} & \textbf{1.80}
% & \textbf{3.73} & \textbf{107.30} & \textbf{1.99}
% & \textbf{3.86} & \textbf{110.76} & \textbf{2.38} \\
% \midrule

% \multicolumn{16}{l}{\textbf{Vicuna-13B}} \\ \cmidrule(lr){1-1}
% % \rowcolor{purple!8}
% Vanilla HF
% & 1.00 & 17.25 & -
% & 1.00 & 11.86 & -
% & 1.00 & 18.81 & -
% & 1.00 & 17.25 & -
% & 1.00 & 13.44 & - \\
% % \rowcolor{purple!8}
% Vanilla FA
% & 1.00 & 28.52 & 1.00
% & 1.00 & 27.43 & 1.00
% & 1.00 & 35.01 & 1.00
% & 1.00 & 33.87 & 1.00
% & 1.00 & 29.14 & 1.00 \\
% % \rowcolor{purple!8}
% MagicDec
% & 2.95 & 38.24 & 1.34
% & 2.87 & 37.15 & 1.35
% & 2.97 & 39.47 & 1.13
% & 2.96 & 38.40 & 1.13
% & 2.94 & 36.66 & 1.26 \\
% % \rowcolor{purple!8}
% LongSpec
% & \textbf{3.31} & \textbf{71.08} & \textbf{2.49}
% & \textbf{2.76} & \textbf{57.15} & \textbf{2.08}
% & \textbf{3.44} & \textbf{78.20} & \textbf{2.23}
% & \textbf{3.57} & \textbf{81.00} & \textbf{2.39}
% & \textbf{3.59} & \textbf{77.22} & \textbf{2.65} \\
% \midrule

% \multicolumn{16}{l}{\textbf{LongChat-7B}} \\ \cmidrule(lr){1-1}
% % \rowcolor{green!8}
% Vanilla HF
% & 1.00 & 25.27 & -
% & 1.00 & 14.11 & -
% & 1.00 & 27.66 & -
% & 1.00 & 25.27 & -
% & 1.00 & 17.02 & - \\
% % \rowcolor{green!8}
% Vanilla FA
% & 1.00 & 42.14 & 1.00
% & 1.00 & 36.87 & 1.00
% & 1.00 & 50.19 & 1.00
% & 1.00 & 54.17 & 1.00
% & 1.00 & 42.69 & 1.00 \\
% % \rowcolor{green!8}
% MagicDec
% & 2.26 & 41.90 & 0.99
% & 2.20 & 40.82 & 1.11
% & 2.32 & 43.94 & 0.88
% & 2.77 & 51.73 & 0.96
% & 2.57 & 44.13 & 1.03 \\
% % \rowcolor{green!8}
% LongSpec
% & \textbf{3.59} & \textbf{101.43} & \textbf{2.41}
% & \textbf{3.06} & \textbf{85.23} & \textbf{2.31}
% & \textbf{3.41} & \textbf{97.93} & \textbf{1.95}
% & \textbf{4.21} & \textbf{122.30} & \textbf{2.26}
% & \textbf{4.03} & \textbf{115.27} & \textbf{2.70} \\
% \midrule

% \multicolumn{16}{l}{\textbf{LongChat-13B}} \\ \cmidrule(lr){1-1}
% % \rowcolor{yellow!8}
% Vanilla HF
% & 1.00 & 17.72 & -
% & 1.00 & 12.08 & -
% & 1.00 & 18.74 & -
% & 1.00 & 17.72 & -
% & 1.00 & 13.85 & - \\
% % \rowcolor{yellow!8}
% Vanilla FA
% & 1.00 & 28.56 & 1.00
% & 1.00 & 27.18 & 1.00
% & 1.00 & 35.37 & 1.00
% & 1.00 & 34.58 & 1.00
% & 1.00 & 29.74 & 1.00 \\
% % \rowcolor{yellow!8}
% MagicDec
% & 2.40 & 31.37 & 1.10
% & 2.38 & 30.84 & 1.13
% & 2.43 & 32.58 & 0.92
% & 2.68 & 35.77 & 1.03
% & 2.85 & 35.67 & 1.20 \\
% % \rowcolor{yellow!8}
% LongSpec
% & \textbf{3.58} & \textbf{76.26} & \textbf{2.67}
% & \textbf{3.15} & \textbf{64.41} & \textbf{2.37}
% & \textbf{3.50} & \textbf{80.48} & \textbf{2.28}
% & \textbf{4.01} & \textbf{90.92} & \textbf{2.63}
% & \textbf{4.46} & \textbf{96.96} & \textbf{3.26} \\
% \midrule

% \multicolumn{16}{l}{\textbf{LLaMA3-8B}} \\ \cmidrule(lr){1-1}
% % \rowcolor{orange!8}
% Vanilla HF
% & 1.00 & 21.59 & -
% & 1.00 & 18.67 & -
% & 1.00 & 29.91 & -
% & 1.00 & 29.48 & -
% & 1.00 & 22.77 & - \\
% % \rowcolor{orange!8}
% Vanilla FA
% & 1.00 & 53.14 & 1.00
% & 1.00 & 51.22 & 1.00
% & 1.00 & 56.94 & 1.00
% & 1.00 & 56.73 & 1.00
% & 1.00 & 54.08 & 1.00 \\
% % \rowcolor{orange!8}
% MagicDec
% & 2.04 & 36.14 & 0.68
% & 2.00 & 35.78 & 0.70
% & 2.33 & 39.57 & 0.70
% & 2.65 & 46.95 & 0.83
% & 2.61 & 44.39 & 0.82 \\
% % \rowcolor{orange!8}
% LongSpec
% & \textbf{3.25} & \textbf{84.57} & \textbf{1.59}
% & \textbf{2.99} & \textbf{75.68} & \textbf{1.48}
% & \textbf{3.36} & \textbf{91.11} & \textbf{1.60}
% & \textbf{3.28} & \textbf{89.33} & \textbf{1.57}
% & \textbf{3.39} & \textbf{91.28} & \textbf{1.69} \\

% \bottomrule
% \end{tabular}
% }
% \caption{Speedup ratio ($\tau$) and decoding speed (Tokens/s) across different models and settings. All results are computed at $T=0$.}
% \label{tab:final_table}
% \end{table*}

\begin{table*}[t]
\centering
\caption{Mean accepted length ($\tau$), decoding speed (tokens/s), and speedups across different models and settings. Specifically, ``Vanilla HF'' refers to HuggingFace’s PyTorch-based attention implementation, while ``Vanilla FA'' employs \texttt{Flash\_Decoding}. The speedup statistic calculates the acceleration ratio relative to the Vanilla HF method. All results are computed at $T=0$.}
\label{tab:final_table}
\vspace{.2em}
\scalebox{0.7}{
\begin{tabular}{c c c c c c c c c c c c c c c c c}
\toprule
 & \multirow{2}{*}{\textbf{Setting}} 
& \multicolumn{3}{c}{\textbf{GovReport}} 
& \multicolumn{3}{c}{\textbf{QMSum}} 
& \multicolumn{3}{c}{\textbf{Multi-News}} 
& \multicolumn{3}{c}{\textbf{LCC}} 
& \multicolumn{3}{c}{\textbf{RepoBench-P}} \\
\cmidrule(lr){3-5} \cmidrule(lr){6-8} \cmidrule(lr){9-11} \cmidrule(lr){12-14} \cmidrule(lr){15-17}
& 
& $\tau$ & Tokens/s & Speedup
& $\tau$ & Tokens/s & Speedup
& $\tau$ & Tokens/s & Speedup
& $\tau$ & Tokens/s & Speedup
& $\tau$ & Tokens/s & Speedup \\
\midrule

% Vicuna-7B
\multirow{4}{*}{\rotatebox{90}{V-7B}}
& Vanilla HF  
& 1.00 & 25.25 & -
& 1.00 & 18.12 & -
& 1.00 & 27.29 & -
& 1.00 & 25.25 & -
& 1.00 & 19.18 & - \\

& Vanilla FA  
& 1.00 & 45.76 & 1.00$\times$
& 1.00 & 43.68 & 1.00$\times$
& 1.00 & 55.99 & 1.00$\times$
& 1.00 & 54.07 & 1.00$\times$
& 1.00 & 46.61 & 1.00$\times$ \\

& MagicDec    
& 2.23 & 41.68 & 0.91$\times$
& 2.29 & 42.91 & 0.98$\times$
& 2.31 & 44.82 & 0.80$\times$
& 2.52 & 46.96 & 0.87$\times$
& 2.57 & 48.75 & 1.05$\times$ \\

& \textbf{LongSpec}  
& \textbf{3.57} & \textbf{102.23} & \textbf{2.23}$\times$
& \textbf{3.14} & \textbf{88.87}  & \textbf{2.04}$\times$
& \textbf{3.51} & \textbf{100.55} & \textbf{1.80}$\times$
& \textbf{3.73} & \textbf{107.30} & \textbf{1.99}$\times$
& \textbf{3.86} & \textbf{110.76} & \textbf{2.38}$\times$ \\
\midrule

% Vicuna-13B
\multirow{4}{*}{\rotatebox{90}{V-13B}}
& Vanilla HF  
& 1.00 & 17.25 & -
& 1.00 & 11.86 & -
& 1.00 & 18.81 & -
& 1.00 & 17.25 & -
& 1.00 & 13.44 & - \\

& Vanilla FA  
& 1.00 & 28.52 & 1.00$\times$
& 1.00 & 27.43 & 1.00$\times$
& 1.00 & 35.01 & 1.00$\times$
& 1.00 & 33.87 & 1.00$\times$
& 1.00 & 29.14 & 1.00$\times$ \\

& MagicDec    
& 2.95 & 38.24 & 1.34$\times$
& 2.87 & 37.15 & 1.35$\times$
& 2.97 & 39.47 & 1.13$\times$
& 2.96 & 38.40 & 1.13$\times$
& 2.94 & 36.66 & 1.26$\times$ \\

& \textbf{LongSpec}  
& \textbf{3.31} & \textbf{71.08} & \textbf{2.49}$\times$
& \textbf{2.76} & \textbf{57.15} & \textbf{2.08}$\times$
& \textbf{3.44} & \textbf{78.20} & \textbf{2.23}$\times$
& \textbf{3.57} & \textbf{81.00} & \textbf{2.39}$\times$
& \textbf{3.59} & \textbf{77.22} & \textbf{2.65}$\times$ \\
\midrule

% LongChat-7B
\multirow{4}{*}{\rotatebox{90}{LC-7B}}
& Vanilla HF  
& 1.00 & 25.27 & -
& 1.00 & 14.11 & -
& 1.00 & 27.66 & -
& 1.00 & 25.27 & -
& 1.00 & 17.02 & - \\

& Vanilla FA  
& 1.00 & 42.14 & 1.00$\times$
& 1.00 & 36.87 & 1.00$\times$
& 1.00 & 50.19 & 1.00$\times$
& 1.00 & 54.17 & 1.00$\times$
& 1.00 & 42.69 & 1.00$\times$ \\

& MagicDec    
& 2.26 & 41.90 & 0.99$\times$
& 2.20 & 40.82 & 1.11$\times$
& 2.32 & 43.94 & 0.88$\times$
& 2.77 & 51.73 & 0.96$\times$
& 2.57 & 44.13 & 1.03$\times$ \\

& \textbf{LongSpec}  
& \textbf{3.59} & \textbf{101.43} & \textbf{2.41}$\times$
& \textbf{3.06} & \textbf{85.23} & \textbf{2.31}$\times$
& \textbf{3.41} & \textbf{97.93} & \textbf{1.95}$\times$
& \textbf{4.21} & \textbf{122.30} & \textbf{2.26}$\times$
& \textbf{4.03} & \textbf{115.27} & \textbf{2.70}$\times$ \\
\midrule

% LongChat-13B
\multirow{4}{*}{\rotatebox{90}{LC-13B}}
& Vanilla HF  
& 1.00 & 17.72 & -
& 1.00 & 12.08 & -
& 1.00 & 18.74 & -
& 1.00 & 17.72 & -
& 1.00 & 13.85 & - \\

& Vanilla FA  
& 1.00 & 28.56 & 1.00$\times$
& 1.00 & 27.18 & 1.00$\times$
& 1.00 & 35.37 & 1.00$\times$
& 1.00 & 34.58 & 1.00$\times$
& 1.00 & 29.74 & 1.00$\times$ \\

& MagicDec    
& 2.40 & 31.37 & 1.10$\times$
& 2.38 & 30.84 & 1.13$\times$
& 2.43 & 32.58 & 0.92$\times$
& 2.68 & 35.77 & 1.03$\times$
& 2.85 & 35.67 & 1.20$\times$ \\

& \textbf{LongSpec}  
& \textbf{3.58} & \textbf{76.26} & \textbf{2.67}$\times$
& \textbf{3.15} & \textbf{64.41} & \textbf{2.37}$\times$
& \textbf{3.50} & \textbf{80.48} & \textbf{2.28}$\times$
& \textbf{4.01} & \textbf{90.92} & \textbf{2.63}$\times$
& \textbf{4.46} & \textbf{96.96} & \textbf{3.26}$\times$ \\
\midrule

% LLaMA3-8B
\multirow{4}{*}{\rotatebox{90}{L-8B}}
& Vanilla HF  
& 1.00 & 21.59 & -
& 1.00 & 18.67 & -
& 1.00 & 29.91 & -
& 1.00 & 29.48 & -
& 1.00 & 22.77 & - \\

& Vanilla FA  
& 1.00 & 53.14 & 1.00$\times$
& 1.00 & 51.22 & 1.00$\times$
& 1.00 & 56.94 & 1.00$\times$
& 1.00 & 56.73 & 1.00$\times$
& 1.00 & 54.08 & 1.00$\times$ \\

& MagicDec    
& 2.04 & 36.14 & 0.68$\times$
& 2.00 & 35.78 & 0.70$\times$
& 2.33 & 39.57 & 0.70$\times$
& 2.65 & 46.95 & 0.83$\times$
& 2.61 & 44.39 & 0.82$\times$ \\

& \textbf{LongSpec}  
& \textbf{3.25} & \textbf{84.57} & \textbf{1.59}$\times$
& \textbf{2.99} & \textbf{75.68} & \textbf{1.48}$\times$
& \textbf{3.36} & \textbf{91.11} & \textbf{1.60}$\times$
& \textbf{3.28} & \textbf{89.33} & \textbf{1.57}$\times$
& \textbf{3.39} & \textbf{91.28} & \textbf{1.69}$\times$ \\

\bottomrule
\end{tabular}
}
\vspace{-.3cm}
\end{table*}

\begin{figure*}
    \centering
    \includegraphics[width=1\linewidth]{Figure/T1.pdf}
    \vspace{-.7cm}
    \caption{Decoding speed (tokens/s) across different models and settings. All results are computed at $T=1$. The letters G, Q, M, L, and R on the horizontal axis represent the dataset GovReport, QMSum, Multi-News, LCC, and RepoBench-P respectively.}
    \label{fig:final_fig}
    \vspace{-.3cm}
\end{figure*}



\section{Experiments}

\subsection{Settings}

\textbf{Target and draft models.} 
We select four widely-used long-context LLMs, Vicuna~(including 7B and 13B)~\cite{chiang2023vicuna}, LongChat~(including 7B and 13B)~\cite{li2023longchat}, LLaMA-3.1-8B-Instruct~\cite{dubey2024llama}, and QwQ-32B~\cite{qwen2024qwq}, as target models.
In order to make the draft model and target model more compatible, our draft model is consistent with the target model in various parameters such as the number of KV heads. 

\textbf{Training Process.} 
We first train our draft model with Achor-Offest Indices on the SlimPajama-6B pretraining dataset~\cite{cerebras2023slimpajama}. 
The random offset is set as a random integer from 0 to 15k for Vicuna models and LongChat-7B, and 0 to 30k for the other three models because they have longer maximum context length.
Then we train our model on a small subset of the Prolong-64k long-context dataset~\cite{gao2024train} in order to gain the ability to handle long texts. 
Finally, we finetune our model on a self-built long-context supervised-finetuning~(SFT) dataset to further improve the model performance.
The position index of the last two stages is the vanilla indexing policy because the training data is sufficiently long.
We apply flash noisy training during all three stages to mitigate the training and inference inconsistency, the extra overhead of flash noisy training is negligible.
Standard cross-entropy is used to optimize the draft model while the parameters of the target model are kept frozen. To mitigate the VRAM peak caused by the computation of the logits, we use a fused-linear-and-cross-entropy loss implemented by the Liger Kernel~\cite{hsu2024liger}, which computes the LM head and the softmax function together and can greatly alleviate this problem. More details on model training can be found in Appendix~\ref{appendix:training_details}.



\textbf{Test Benchmarks.}
We select tasks from the LongBench benchmark~\cite{bai2024longbench} that involve generating longer outputs, because tasks with shorter outputs, such as document-QA, make it challenging to measure the speedup ratio fairly with speculative decoding. 
Specifically, we focus on long-document summarization and code completion tasks and conduct tests on five datasets: GovReport~\cite{huang2021efficient}, QMSum~\cite{zhong2021qmsum}, Multi-News~\cite{fabbri2019multi}, LCC~\cite{guo2023longcoder}, and RepoBench-P~\cite{liu2024repobench}. We test QwQ-32B on the famous reasoning dataset AIME24~\cite{numina2024aime}.

We compare our method with the original target model and MagicDec, a simple prototype of TriForce. 
To highlight the significance of \texttt{Flash\_Decoding} in long-context scenarios, we also present the performance of the original target model using both eager attention implemented by Huggingface and \texttt{Flash\_Decoding} for comparison.
To make a fair comparison, we also use \texttt{Flash\_Decoding} for baseline MagicDec.
The most important metric for speculative decoding is the \emph{walltime speedup ratio}, which is the actual test speedup ratio relative to vanilla autoregressive decoding. 
We also test the \emph{average acceptance length} $\tau$, \emph{i.e.}, the average number of tokens accepted per forward pass of the target LLM. 
% \emph{Acceptance rate} $\alpha$, the ratio of accepted to generated tokens during drafting, is taken into consideration as well. Following EAGLE~\cite{li2024eagle}, when measuring this metric, we utilize chain drafts without tree attention to assess the acceptance rate per location more precisely. Specifically, for the $n$-th token in the draft tokens, the acceptance rate is denoted as $n\text{-}\alpha$.

\subsection{Main Results}

Table~\ref{tab:final_table} and Figure~\ref{fig:final_fig} show the decoding speeds and mean accept lengths across the five evaluated datasets at $T=0$ and $T=1$ respectively. 
Our proposed method significantly outperforms all other approaches on both summarization tasks and code completion tasks. When $T=0$, on summarization tasks, our method can achieve a mean accepted length of around 3.5 and a speedup of up to 2.67$\times$; and on code completion tasks, our method can achieve a mean accepted length of around 4 and a speedup of up to 3.26$\times$. This highlights the robustness and generalizability of our speculative decoding approach, particularly in long-text generation tasks. At $T=1$, our method's performance achieves around 2.5$\times$ speedup, maintaining a substantial lead over MagicDec. This indicates that our approach is robust across different temperature settings, further validating its soundness and efficiency.

While MagicDec demonstrates competitive acceptance rates with LongSpec, its speedup is noticeably lower in our experiments. This is because MagicDec is primarily designed for scenarios with large batch sizes and tensor parallelism. 
In low-batch-size settings, its draft model leverages all parameters of the target model with sparse KV Cache becomes excessively heavy. 
This design choice leads to inefficiencies, as the draft model's computational overhead outweighs its speculative benefits. 
Our results reveal that MagicDec only achieves acceleration ratios~$>\!1$ on partial datasets when using a guess length $\gamma \!=\! 2$ and consistently exhibits negative acceleration around 0.7$\times$ when $\gamma\!\geq\!3$, further underscoring the limitations of this method in such configurations.
% \looseness=-1

Lastly, we can find attention implementation plays a critical role in long-context speculative decoding performance. In our experiments, ``Vanilla HF'' refers to HuggingFace’s attention implementation, while ``Vanilla FA'' employs \texttt{Flash\_Decoding}. The latter demonstrates nearly a $2\times$ speedup over the former, even as a standalone component, and our method can achieve up to $6\times$ speedup over HF Attention on code completion datasets. 
This result underscores the necessity for speculative decoding methods to be compatible with optimized attention mechanisms like \texttt{Flash\_Decoding}, especially in long-text settings. Our hybrid tree attention approach achieves this compatibility, allowing us to fully leverage the advantages of \texttt{Flash\_Decoding} and further speedup.

\subsection{Ablation Studies}

\textbf{Anchor-Offset Indices.} 
The experimental results demonstrate the significant benefits of incorporating the Anchor-Offset Indices. Figure~\ref{fig:ablation_1} shows that pretrained with Anchor-Offset Indices achieve a lower initial loss and final loss compared to those trained without it when training over the real long-context dataset. 
Notably, the initalization with Anchor-Offset Indices reaches the same loss level $3.93\times$ faster than its counterpart. 
Table~\ref{tab:ablation_1} further highlights the performance improvements across two datasets, a summary dataset Multi-News, and a code completion dataset RepoBench-P. 
Models with Anchor-Offset Indices exhibit faster output speed and larger average acceptance length $\tau$. These results underscore the effectiveness of Anchor-Offset Indices in enhancing both training efficiency and model performance.

\begin{table}[t]
\vspace{-0.cm}
    \centering
    \caption{Performance comparison with and without Anchor-Offset Indices on the Multi-News and RepoBench-P datasets. Models with Anchor-Offset Indices achieve higher output speed and larger accept length, highlighting its efficiency and effectiveness.}
    \label{tab:ablation_1}
    \vspace{.1cm}
    \scalebox{0.9}{
    \begin{tabular}{c c c c c}
        \toprule
        & \multicolumn{2}{c}{\textbf{Multi-News}} 
        & \multicolumn{2}{c}{\textbf{RepoBench-P}} \\
        \cmidrule(lr){2-3} \cmidrule(lr){4-5}
        & $\tau$ & Tokens/s
        & $\tau$ & Tokens/s \\
        \midrule
        w/o Anchor-Offset & 3.20 & 85.98 & 3.26 & 85.21\\
        w/ Anchor-Offset  & 3.36 & 91.11 & 3.39 & 91.28\\
        \bottomrule
    \end{tabular}
    }
\end{table}

\begin{figure}[t]
    \vspace{-.2cm}
    \centering
    \includegraphics[width=1\linewidth]{Figure/ablation_1.pdf}
    \vspace{-.7cm}
    \caption{Training loss curves on long-context data. 
     Pretrained models with Anchor-Offset Indices exhibit lower initial and final loss, and reach the same loss level 3.93$\times$ faster compared to models without Anchor-Offset Indices.}
    \label{fig:ablation_1}
    \vspace{-.1cm}
\end{figure}


\textbf{Hybrid Tree Attention.}
The results presented in Figure \ref{fig:ablation_2} highlight the effectiveness of the proposed Hybrid Tree Attention, which combines \texttt{Flash\_Decoding} with the Triton kernel \texttt{fused\_mask\_attn}. 
While the time spent on the draft model forward pass and the target model FFN computations remain comparable across the two methods, the hybrid approach exhibits a significant reduction in latency for the target model's attention layer (the yellow part). 
Specifically, the attention computation latency decreases from 49.92 ms in the HF implementation to 12.54 ms in the hybrid approach, resulting in an approximately 75\% improvement. 
The verification step time difference is minimal, further solidifying the conclusion that the primary performance gains stem from optimizing the attention mechanism.

\begin{figure}[t]
    \centering
    \includegraphics[width=1.0\linewidth]{Figure/runtime.pdf}
    \vspace{-.8cm}
    \caption{Latency breakdown for a single speculative decoding loop comparing the EAGLE implementation and the proposed Hybrid Tree Attention. Significant latency reduction is observed in the target model's attention layer (the yellow part) using our approach.}
    \label{fig:ablation_2}
    \vspace{-.cm}
\end{figure}


\begin{figure}[t]
    \centering
    \includegraphics[width=1\linewidth]{Figure/qwq.pdf}
    \vspace{-.5cm}
    \caption{Performance of our method on the QwQ-32B model with the AIME24 dataset, using a maximum output length of 32k tokens. The left plot shows the tokens generated per second, where our approach achieves 2.25$\times$ higher speed compared to the baseline. 
    The right plot shows the mean number of accepted tokens, where our method achieves an average of 3.82 mean accepted tokens.}
    \label{fig:qwq}
    \vspace{.1cm}
\end{figure}

\subsection{Long CoT Acceleration}

Long Chain-of-Thought (LongCoT) tasks have gained significant attention recently due to their ability to enable models to perform complex reasoning and problem-solving over extended outputs~\cite{qwen2024qwq, openai2024o1}. In these tasks, while the prefix input is often relatively short, the generated output can be extremely long, posing unique challenges in terms of efficiency and token acceptance. Our method is particularly well-suited for addressing these challenges, effectively handling scenarios with long outputs. It is worth mentioning that MagicDec is not suitable for such long-output scenarios because the initial inference stage of the LongCoT task is not the same as the traditional long-context task. In LongCoT tasks, where the prefix is relatively short, the draft model in MagicDec will completely degrade into the target model, failing to achieve acceleration.
\looseness=-1

We evaluate our method on the QwQ-32B model using the widely-used benchmark AIME24 dataset, with a maximum output length set to 32k tokens. 
The results, illustrated in Figure~\ref{fig:qwq}, demonstrate a significant improvement in both generation speed and mean accepted tokens. 
Specifically, our method achieved a generation rate of 42.63 tokens/s, 2.25$\times$ higher than the baseline's 18.92 tokens/s, and an average of 3.82 mean accepted tokens.
Notably, QwQ-32B with \textsc{LongSpec} achieves even lower latency than the vanilla 7B model with \texttt{Flash\_Decoding}, demonstrating that our method effectively accelerates the LongCoT model.
These findings not only highlight the effectiveness of our method in the LongCoT task but also provide new insights into lossless inference acceleration for the o1-like model.
We believe that speculative decoding will play a crucial role in accelerating this type of model in the future.


\subsection{Throughput}

The throughput results of Vicuna-7B on the RepoBench-P dataset show that \textsc{LongSpec} consistently outperforms both Vanilla and MagicDec across all batch sizes. At a batch size of 8, \textsc{LongSpec} achieves a throughput of 561.32 tokens/s, approximately 1.8$\times$ higher than MagicDec (310.58 tokens/s) and nearly 2$\times$ higher than Vanilla (286.96 tokens/s). MagicDec, designed with throughput optimization in mind, surpasses Vanilla as the batch size increases, reflecting its targeted improvements. However, \textsc{LongSpec} still sustains its advantage, maintaining superior throughput across all tested batch sizes.

\begin{figure}
    \centering
    \includegraphics[width=0.85\linewidth]{Figure/throughput.pdf}
    \vspace{-.2cm}
    \caption{Throughput comparison of Vanilla, MagicDec, and \textsc{LongSpec} on RepoBench-P using Vicuna-7B across different batch sizes. \textsc{LongSpec} shows superior throughput and scalability, outperforming both Vanilla and MagicDec in all batch sizes.}
    \label{fig:throughput}
    \vspace{-.3cm}
\end{figure}


\section{Conclusion}

In this paper, we propose \textsc{LongSpec}, a novel framework designed to enhance speculative decoding for long-context scenarios. Unlike previous speculative decoding methods that primarily focus on short-context settings, \textsc{LongSpec} directly addresses three key challenges: excessive memory overhead, inadequate training for large position indices, and inefficient tree attention computation. To mitigate memory constraints, we introduce an efficient draft model architecture that maintains a constant memory footprint by leveraging a combination of sliding window self-attention and cache-free cross-attention. To resolve the training limitations associated with short context data, we propose the Anchor-Offset Indices, ensuring that large positional indices are sufficiently trained even within short-sequence datasets. Finally, we introduce Hybrid Tree Attention, 
which efficiently integrates tree-based speculative decoding with \texttt{Flash\_Decoding}. Extensive experiments demonstrate the effectiveness of \textsc{LongSpec} in long-context understanding tasks and real-world long reasoning tasks. Our findings highlight the importance of designing speculative decoding methods specifically tailored for long-context settings and pave the way for future research in efficient large-scale language model inference.

\section*{Impact Statements}
This paper presents work whose goal is to advance the field of Machine Learning. There are many potential societal consequences of our work, none which we feel must be specifically highlighted here.

\bibliography{example_paper}
\bibliographystyle{icml2025}


%%%%%%%%%%%%%%%%%%%%%%%%%%%%%%%%%%%%%%%%%%%%%%%%%%%%%%%%%%%%%%%%%%%%%%%%%%%%%%%
%%%%%%%%%%%%%%%%%%%%%%%%%%%%%%%%%%%%%%%%%%%%%%%%%%%%%%%%%%%%%%%%%%%%%%%%%%%%%%%
% APPENDIX
%%%%%%%%%%%%%%%%%%%%%%%%%%%%%%%%%%%%%%%%%%%%%%%%%%%%%%%%%%%%%%%%%%%%%%%%%%%%%%%
%%%%%%%%%%%%%%%%%%%%%%%%%%%%%%%%%%%%%%%%%%%%%%%%%%%%%%%%%%%%%%%%%%%%%%%%%%%%%%%
\newpage
\appendix
\onecolumn

\section{Correctness for Attention Aggregation}
\label{appendix:attn_aggr}

Because the query matrix $Q$ can be decomposed into several rows, each representing a separate query $q$, we can only consider the output of each row's $q$ after calculating attention with KV. In this way, we can assume that the KV involved in the calculation has undergone the tree mask, which can simplify our proof. We only need to prove that the output $o$ obtained from each individual $q$ meets the requirements, which can indicate that the overall output $O$ of the entire matrix $Q$ also meets the requirements.

\begin{proposition}
    Denote the log-sum-exp of the merged attention as follows:
    \begin{equation*}
        \mathrm{LSE}_{\mathrm{merge}} = \log\Bigl(\exp\bigl(\mathrm{LSE}_{\mathrm{cache}}\bigr) \;+\; \exp\bigl(\mathrm{LSE}_{\mathrm{specs}}\bigr)\Bigr),
    \end{equation*}
    Then we can write the merged attention output in the following way:
    \begin{equation*}
    o_{\mathrm{merge}} = o_{\mathrm{cache}} \cdot \exp\bigl(\mathrm{LSE}_{\mathrm{cache}} - \mathrm{LSE}_{\mathrm{merge}}\bigr) + o_{\mathrm{specs}} \cdot\exp\bigl(\mathrm{LSE}_{\mathrm{specs}} - \mathrm{LSE}_{\mathrm{merge}}\bigr).
    \end{equation*}
\end{proposition}

\begin{proof}
    
A standard scaled dot-product attention for $ q $ (of size $d_{qk}$) attending to $ K_{\mathrm{merge}} $ and $ V_{\mathrm{merge}} $ (together of size $(M+N) \times d_{qk}$ and $(M+N) \times d_v$ respectively) can be written as:

\begin{equation*}
    o_{\mathrm{merge}} = \mha\left(q, K_{\mathrm{merge}}, V_{\mathrm{merge}}\right) =
    \sm\left(
      qK_{\mathrm{merge}}^\top/\sqrt{d_{qk}}
    \right) V_{\mathrm{merge}}.
\end{equation*}

Because $ K $ and $ V $ are formed by stacking $\left(K_{\mathrm{specs}}, K_{\mathrm{cache}}\right)$ and $\left(V_{\mathrm{specs}}, V_{\mathrm{cache}}\right)$, we split the logit matrix accordingly:

\begin{equation*}
    q K_{\mathrm{merge}}^\top / \sqrt{d_{qk}} = 
    \texttt{concat}\Bigl(
    \underbrace{
      q \, K_{\mathrm{cache}}^\top / \sqrt{d_{qk}}
    }_{\mathrm{sub-logits for history}}
    \;, \;
    \underbrace{
      q \, K_{\mathrm{specs}}^\top / \sqrt{d_{qk}}
    }_{\mathrm{sub-logits for new}}
    \Bigr).
\end{equation*}

Denote these sub-logit matrices as:
\[
Z_{\mathrm{cache}} \;=\; q \, K_{\mathrm{cache}}^\top / \sqrt{d_{qk}}
,\;
Z_{\mathrm{specs}} \;=\; q \, K_{\mathrm{specs}}^\top / \sqrt{d_{qk}}.
\]

Each row $i$ of $Z_{\mathrm{specs}}$ corresponds to the dot products between the $i$-th query in $q$ and all rows in $K_{\mathrm{specs}}$, while rows of $Z_{\mathrm{cache}}$ correspond to the same query but with $K_{\mathrm{cache}}$.

In order to combine partial attentions, we keep track of the log of the sum of exponentials of each sub-logit set. Concretely, define:

\begin{equation}
    \mathrm{LSE}_{\mathrm{cache}} = \log\left(\sum\nolimits_{j=1}^{N} \exp\left(Z_{\mathrm{cache}}^{(j)}\right)\right),
    \;
    \mathrm{LSE}_{\mathrm{specs}} = \log\left(\sum\nolimits_{j=1}^{M} \exp\left(Z_{\mathrm{specs}}^{(j)}\right)\right),
    \,
\end{equation}

where $Z_{\mathrm{specs}}^{(j)}$ denotes the logit for the $j$-th element, and similarly for $Z_{\mathrm{cache}}^{(j)}$.

% Then the standard scaled dot-product attention for $ q $ (of size $d_{qk}$) attending to $ K_{\mathrm{cache}} $ and $ V_{\mathrm{cache}} $ (together of size $N \times d_{qk}$ and $N \times d_v$ respectively) can be written as:

% \begin{equation}
%     \label{equ:history_attn}
%     o_{\mathrm{cache}} = \mha\left(q, K_{\mathrm{cache}}, V_{\mathrm{cache}}\right) =
%     \frac{\sum_{j=1}^{N} \exp\left(Z_{\mathrm{cache}}^{(j)}\right) V_{\mathrm{cache}}^{(j)}}{\exp\left(\mathrm{LSE}_{\mathrm{cache}}\right)}.
% \end{equation}

% Similarly, the standard scaled dot-product attention for $ q $ (of size $M \times d_{qk}$) attending to $ K_{\mathrm{specs}} $ and $ V_{\mathrm{specs}} $ (together of size $M \times d_{qk}$ and $M \times d_v$ respectively) can be written as:

% \begin{equation}
%     \label{equ:new_attn}
%     o_{\mathrm{specs}} = \mha\left(q, K_{\mathrm{specs}}, V_{\mathrm{specs}}\right) =
%     \frac{\sum_{j=1}^{M} \exp\left(Z_{\mathrm{specs}}^{(j)}\right) V_{\mathrm{specs}}^{(j)}}{\exp\left(\mathrm{LSE}_{\mathrm{specs}}\right)}.
% \end{equation}

Then $o_{\mathrm{cache}}$ and $o_{\mathrm{specs}}$ can be written as:
\begin{equation}
    \label{equ:split_attn}
    o_{\mathrm{cache}} = \frac{\sum_{j=1}^{N} \exp\left(Z_{\mathrm{cache}}^{(j)}\right) V_{\mathrm{cache}}^{(j)}}{\exp\left(\mathrm{LSE}_{\mathrm{cache}}\right)}, \;
    o_{\mathrm{specs}} = \frac{\sum_{j=1}^{M} \exp\left(Z_{\mathrm{specs}}^{(j)}\right) V_{\mathrm{specs}}^{(j)}}{\exp\left(\mathrm{LSE}_{\mathrm{specs}}\right)}.
\end{equation}

And the whole attention score can be written as:

\begin{equation}
    \label{equ:all_attn}
    o_{\mathrm{merge}} =
    \frac{\sum_{j=1}^{N} \exp\left(Z_{\mathrm{cache}}^{(j)}\right) V_{\mathrm{cache}}^{(j)} + \sum_{j=1}^{M} \exp\left(Z_{\mathrm{specs}}^{(j)}\right) V_{\mathrm{specs}}^{(j)}}{\exp\left(\mathrm{LSE}_{\mathrm{cache}}\right) + \exp\left(\mathrm{LSE}_{\mathrm{specs}}\right)}.
\end{equation}

By aggregating Equation \ref{equ:split_attn} into Equation \ref{equ:all_attn}, we can get the following equation:

\begin{equation}
    o_{\mathrm{merge}} = o_{\mathrm{cache}} \cdot\exp\bigl(\mathrm{LSE}_{\mathrm{cache}} - \mathrm{LSE}_{\mathrm{merge}}\bigr)
     + o_{\mathrm{specs}} \cdot \exp\bigl(\mathrm{LSE}_{\mathrm{specs}} - \mathrm{LSE}_{\mathrm{merge}}\bigr).
\end{equation}

\end{proof}

\section{Experiments Details}
\label{appendix:training_details}

All models are trained using eight A100 80GB GPUs. For the 7B, 8B, and 13B target models trained on short-context data, we employ \textsc{LongSpec} with ZeRO-1~\cite{rasley2020deepspeed}. For the 7B, 8B, and 13B models trained on long-context data, as well as for all settings of the 33B target models, we utilize ZeRO-3.  

For the SlimPajama-6B dataset, we configure the batch size (including accumulation) to 2048, set the maximum learning rate to 5e-4 with a cosine learning rate schedule~\cite{loshchilov2017sgdr}, and optimize the draft model using AdamW~\cite{kingma2015adam}. When training on long-context datasets, we adopt a batch size of 256 and a maximum learning rate of 5e-6. The draft model is trained for only one epoch on all datasets.

It is important to note that the primary computational cost arises from forwarding the target model to obtain the KV cache. Recently, some companies have introduced a service known as context caching~\cite{deepseek2024contextcaching, gemini2024contextcaching}, which involves storing large volumes of KV cache. Consequently, in real-world deployment, these pre-stored KV caches can be directly utilized as training data, significantly accelerating the training process.  

For the tree decoding of \textsc{LongSpec}, we employ dynamic beam search to construct the tree. Previous studies have shown that beam search, while achieving high acceptance rates, suffers from slow processing speed in speculative decoding~\cite{du2024glide}. Our research identifies that this slowdown is primarily caused by KV cache movement. In traditional beam search, nodes that do not fall within the top-$k$ likelihood are discarded, a step that necessitates KV cache movement. However, in speculative decoding, discarding these nodes is unnecessary, as draft sequences are not required to maintain uniform lengths. Instead, we can simply halt the computation of descendant nodes for low-likelihood branches without removing them entirely. By adopting this approach, beam search attains strong performance without excessive computational overhead. In our experiments, the beam width is set to $[4, 16, 16, 16, 16]$ for each speculation step. All inference experiments in this study are conducted using float16 precision on a single A100 80GB GPU.  

\section{Experiments}
\label{sec:Experiments} 

We conduct several experiments across different problem settings to assess the efficiency of our proposed method. Detailed descriptions of the experimental settings are provided in \cref{sec:apendix_experiments}.
%We conduct experiments on optimizing PINNs for convection, wave PDEs, and a reaction ODE. 
%These equations have been studied in previous works investigating difficulties in training PINNs; we use the formulations in \citet{krishnapriyan2021characterizing, wang2022when} for our experiments. 
%The coefficient settings we use for these equations are considered challenging in the literature \cite{krishnapriyan2021characterizing, wang2022when}.
%\cref{sec:problem_setup_additional} contains additional details.

%We compare the performance of Adam, \lbfgs{}, and \al{} on training PINNs for all three classes of PDEs. 
%For Adam, we tune the learning rate by a grid search on $\{10^{-5}, 10^{-4}, 10^{-3}, 10^{-2}, 10^{-1}\}$.
%For \lbfgs, we use the default learning rate $1.0$, memory size $100$, and strong Wolfe line search.
%For \al, we tune the learning rate for Adam as before, and also vary the switch from Adam to \lbfgs{} (after 1000, 11000, 31000 iterations).
%These correspond to \al{} (1k), \al{} (11k), and \al{} (31k) in our figures.
%All three methods are run for a total of 41000 iterations.

%We use multilayer perceptrons (MLPs) with tanh activations and three hidden layers. These MLPs have widths 50, 100, 200, or 400.
%We initialize these networks with the Xavier normal initialization \cite{glorot2010understanding} and all biases equal to zero.
%Each combination of PDE, optimizer, and MLP architecture is run with 5 random seeds.

%We use 10000 residual points randomly sampled from a $255 \times 100$ grid on the interior of the problem domain. 
%We use 257 equally spaced points for the initial conditions and 101 equally spaced points for each boundary condition.

%We assess the discrepancy between the PINN solution and the ground truth using $\ell_2$ relative error (L2RE), a standard metric in the PINN literature. Let $y = (y_i)_{i = 1}^n$ be the PINN prediction and $y' = (y'_i)_{i = 1}^n$ the ground truth. Define
%\begin{align*}
%    \mathrm{L2RE} = \sqrt{\frac{\sum_{i = 1}^n (y_i - y'_i)^2}{\sum_{i = 1}^n y'^2_i}} = \sqrt{\frac{\|y - y'\|_2^2}{\|y'\|_2^2}}.
%\end{align*}
%We compute the L2RE using all points in the $255 \times 100$ grid on the interior of the problem domain, along with the 257 and 101 points used for the initial and boundary conditions.

%We develop our experiments in PyTorch 2.0.0 \cite{paszke2019pytorch} with Python 3.10.12.
%Each experiment is run on a single NVIDIA Titan V GPU using CUDA 11.8.
%The code for our experiments is available at \href{https://github.com/pratikrathore8/opt_for_pinns}{https://github.com/pratikrathore8/opt\_for\_pinns}.


\subsection{2D Allen Cahn Equation}
\begin{figure*}[t]
    \centering
    \includegraphics[scale=0.38]{figs/Burgers_operator.pdf}
    \caption{1D Burgers' Equation (Operator Learning): Steady-state solutions for different initializations $u_0$ under varying viscosity $\varepsilon$: (a) $\varepsilon = 0.5$, (b) $\varepsilon = 0.1$, (c) $\varepsilon = 0.05$. The results demonstrate that all final test solutions converge to the correct steady-state solution. (d) Illustration of the evolution of a test initialization $u_0$ following homotopy dynamics. The number of residual points is $\nres = 128$.}
    \label{fig:Burgers_result}
\end{figure*}
First, we consider the following time-dependent problem:
\begin{align}
& u_t = \varepsilon^2 \Delta u - u(u^2 - 1), \quad (x, y) \in [-1, 1] \times [-1, 1] \nonumber \\
& u(x, y, 0) = - \sin(\pi x) \sin(\pi y) \label{eq.hom_2D_AC}\\
& u(-1, y, t) = u(1, y, t) = u(x, -1, t) = u(x, 1, t) = 0. \nonumber
\end{align}
We aim to find the steady-state solution for this equation with $\varepsilon = 0.05$ and define the homotopy as:
\begin{equation}
    H(u, s, \varepsilon) = (1 - s)\left(\varepsilon(s)^2 \Delta u - u(u^2 - 1)\right) + s(u - u_0),\nonumber
\end{equation}
where $s \in [0, 1]$. Specifically, when $s = 1$, the initial condition $u_0$ is automatically satisfied, and when $s = 0$, it recovers the steady-state problem. The function $\varepsilon(s)$ is given by
\begin{equation}
\varepsilon(s) = 
\left\{\begin{array}{l}
s, \quad s \in [0.05, 1], \\
0.05, \quad s \in [0, 0.05].
\end{array}\right.\label{eq:epsilon_t}
\end{equation}

Here, $\varepsilon(s)$ varies with $s$ during the first half of the evolution. Once $\varepsilon(s)$ reaches $0.05$, it remains fixed, and only $s$ continues to evolve toward $0$. As shown in \cref{fig:2D_Allen_Cahn_Equation}, the relative $L_2$ error by homotopy dynamics is $8.78 \times 10^{-3}$, compared with the result obtained by PINN, which has a $L_2$ error of $9.56 \times 10^{-1}$. This clearly demonstrates that the homotopy dynamics-based approach significantly improves accuracy.

\subsection{High Frequency Function Approximation }
We aim to approximate the following function:
$u=    \sin(50\pi x), \quad x \in [0,1].$
The homotopy is defined as $H(u,\varepsilon) = u - \sin(\frac{1}{\varepsilon}\pi x), $
where $\varepsilon \in [\frac{1}{50},\frac{1}{15}]$.

\begin{table}[htbp!]
    \caption{Comparison of the lowest loss achieved by the classical training and homotopy dynamics for different values of $\varepsilon$ in approximating $\sin\left(\frac{1}{\varepsilon} \pi x\right)$
    }
    \vskip 0.15in
    \centering
    \tiny
    \begin{tabular}{|c|c|c|c|c|} 
    \hline 
    $ $ & $\varepsilon = 1/15$ & $\varepsilon = 1/35$ & $\varepsilon = 1/50$ \\ \hline 
    Classical Loss                & 4.91e-6     & 7.21e-2     & 3.29e-1       \\ \hline 
    Homotopy Loss $L_H$                      & 1.73e-6     & 1.91e-6     & \textbf{2.82e-5}       \\ \hline
    \end{tabular}
    % On convection, \al{} provides 14.2$\times$ and 1.97$\times$ improvement over Adam or \lbfgs{} on L2RE. 
    % On reaction, \al{} provides 1.10$\times$ and 1.99$\times$ improvement over Adam or \lbfgs{} on L2RE.
    % On wave, \al{} provides 6.32$\times$ and 6.07$\times$ improvement over Adam or \lbfgs{} on L2RE.}
    \label{tab:loss_approximate}
\end{table}

As shown in \cref{fig:high_frequency_result}, due to the F-principle \cite{xu2024overview}, training is particularly challenging when approximating high-frequency functions like $\sin(50\pi x)$. The loss decreases slowly, resulting in poor approximation performance. However, training based on homotopy dynamics significantly reduces the loss, leading to a better approximation of high-frequency functions. This demonstrates that homotopy dynamics-based training can effectively facilitate convergence when approximating high-frequency data. Additionally, we compare the loss for approximating functions with different frequencies $1/\varepsilon$ using both methods. The results, presented in \cref{tab:loss_approximate}, show that the homotopy dynamics training method consistently performs well for high-frequency functions.





\subsection{Burgers Equation}
In this example, we adopt the operator learning framework to solve for the steady-state solution of the Burgers equation, given by:
\begin{align}
& u_t+\left(\frac{u^2}{2}\right)_x - \varepsilon u_{xx}=\pi \sin (\pi x) \cos (\pi x), \quad x \in[0, 1]\nonumber\\
& u(x, 0)=u_0(x),\label{eq:1D_Burgers} \\
& u(0, t)=u(1, t)=0, \nonumber 
\end{align}
with Dirichlet boundary conditions, where $u_0 \in L_{0}^2((0, 1); \mathbb{R})$ is the initial condition and $\varepsilon \in \mathbb{R}$ is the viscosity coefficient. We aim to learn the operator mapping the initial condition to the steady-state solution, $G^{\dagger}: L_{0}^2((0, 1); \mathbb{R}) \rightarrow H_{0}^r((0, 1); \mathbb{R})$, defined by $u_0 \mapsto u_{\infty}$ for any $r > 0$. As shown in Theorem 2.2 of \cite{KREISS1986161} and Theorems 2.5 and 2.7 of \cite{hao2019convergence}, for any $\varepsilon > 0$, the steady-state solution is independent of the initial condition, with a single shock occurring at $x_s = 0.5$. Here, we use DeepONet~\cite{lu2021deeponet} as the network architecture. 
The homotopy definition, similar to ~\cref{eq.hom_2D_AC}, can be found in \cref{Ap:operator}. The results can be found in \cref{fig:Burgers_result} and \cref{tab:loss_burgers}. Experimental results show that the homotopy dynamics strategy performs well in the operator learning setting as well.


\begin{table}[htbp!]
    \caption{Comparison of loss between classical training and homotopy dynamics for different values of $\varepsilon$ in the Burgers equation, along with the MSE distance to the ground truth shock location, $x_s$.}
    \vskip 0.15in
    \centering
    \tiny
    \begin{tabular}{|c|c|c|c|c|} 
    \hline  
    $ $ & $\varepsilon = 0.5$ & $\varepsilon = 0.1$ & $\varepsilon = 0.05$ \\ \hline 
    Homotopy Loss $L_H$                &  7.55e-7     & 3.40e-7     & 7.77e-7       \\ \hline 
    L2RE                      & 1.50e-3     & 7.00e-4     & 2.52e-2       \\ \hline
        MSE Distance $x_s$                      & 1.75e-8     & 9.14e-8      & 1.2e-3      \\ \hline
    \end{tabular}
    % On convection, \al{} provides 14.2$\times$ and 1.97$\times$ improvement over Adam or \lbfgs{} on L2RE. 
    % On reaction, \al{} provides 1.10$\times$ and 1.99$\times$ improvement over Adam or \lbfgs{} on L2RE.
    % On wave, \al{} provides 6.32$\times$ and 6.07$\times$ improvement over Adam or \lbfgs{} on L2RE.}
    \label{tab:loss_burgers}
\end{table}



% \begin{itemize}
%     \item Relate the curvature in the problem to the differential operator. Use this to demonstrate why the problem is ill-conditioned
%     \item Give an argument for why using Adam + L-BFGS is better than just using L-BFGS outright. The idea is that Adam lowers the errors to the point where the rest of the optimization becomes convex \ldots
%     \item Show why we need second-order methods. We would like to prove that once we are close to the optimum, second-order methods will give condition-number free linear convergence. Specialize this to the Gauss-Newton setting, with the randomized low-rank approximation.
%     % \item Show that it is not possible to get superlinear convergence under the interpolation assumption with an overparameterized neural network. This should be true b/c the Hessian at the optimum will have rank $\min(n, d)$, and when $d > n$, this guarantees that we cannot have strong convexity.
% \end{itemize}

\section{Case Study}
\label{appendix:visualization}
Here we display some illustrative cases from GovReport, where tokens marked in blue indicate draft tokens accepted by the target model. 

\begin{tcolorbox}
    \textcolor{blue}{The} \textcolor{blue}{Cong}\textcolor{black}{r}\textcolor{blue}{essional} \textcolor{blue}{Gold} \textcolor{blue}{Medal} \textcolor{blue}{is} \textcolor{blue}{a} \textcolor{black}{pr}\textcolor{blue}{estig}\textcolor{blue}{ious} \textcolor{blue}{award} \textcolor{blue}{given} \textcolor{black}{by} \textcolor{blue}{the} \textcolor{blue}{United} \textcolor{blue}{States} \textcolor{blue}{Congress} \textcolor{black}{to} \textcolor{black}{individuals} \textcolor{blue}{and} \textcolor{blue}{groups} \textcolor{blue}{in} \textcolor{blue}{recognition} \textcolor{blue}{of} \textcolor{black}{their} \textcolor{blue}{distinguished} \textcolor{blue}{contributions}\textcolor{black}{,} \textcolor{blue}{achiev}\textcolor{blue}{ements}\textcolor{blue}{,} \textcolor{blue}{and} \textcolor{black}{services} \textcolor{blue}{to} \textcolor{blue}{the} \textcolor{black}{country}\textcolor{blue}{.} \textcolor{blue}{The} \textcolor{black}{tradition} \textcolor{blue}{of} \textcolor{blue}{award}\textcolor{blue}{ing} \textcolor{black}{gold} \textcolor{blue}{med}\textcolor{blue}{als} \textcolor{black}{dates} \textcolor{blue}{back} \textcolor{blue}{to} \textcolor{blue}{the} \textcolor{blue}{late} \textcolor{blue}{}\textcolor{black}{1}\textcolor{blue}{8}\textcolor{blue}{th} \textcolor{blue}{century}\textcolor{blue}{,} \textcolor{black}{and} \textcolor{black}{it} \textcolor{blue}{has} \textcolor{blue}{been} \textcolor{black}{used} \textcolor{blue}{to} \textcolor{blue}{honor} \textcolor{black}{a} \textcolor{blue}{wide} \textcolor{blue}{range} \textcolor{blue}{of} \textcolor{blue}{individuals}\textcolor{blue}{,} \textcolor{black}{including} \textcolor{blue}{military} \textcolor{blue}{leaders}\textcolor{blue}{,} \textcolor{black}{scient}\textcolor{blue}{ists}\textcolor{blue}{,} \textcolor{blue}{artists}\textcolor{blue}{,} \textcolor{blue}{and} \textcolor{black}{human}\textcolor{black}{it}\textcolor{blue}{ari}\textcolor{blue}{ans}\textcolor{blue}{.}
    
    \textcolor{blue}{The} \textcolor{blue}{first} \textcolor{black}{Cong}\textcolor{blue}{r}\textcolor{blue}{essional} \textcolor{blue}{Gold} \textcolor{blue}{Med}\textcolor{blue}{als} \textcolor{black}{were} \textcolor{blue}{issued} \textcolor{blue}{by} \textcolor{blue}{the} \textcolor{black}{Cont}\textcolor{blue}{inental} \textcolor{blue}{Congress} \textcolor{blue}{in} \textcolor{blue}{the} \textcolor{blue}{late} \textcolor{black}{}\textcolor{blue}{1}\textcolor{blue}{7}\textcolor{blue}{0}\textcolor{blue}{0}\textcolor{blue}{s}\textcolor{black}{,} \textcolor{blue}{and} \textcolor{black}{since} \textcolor{blue}{then}\textcolor{blue}{,} \textcolor{blue}{Congress} \textcolor{blue}{has} \textcolor{black}{awarded} \textcolor{black}{over} \textcolor{blue}{}\textcolor{blue}{2}\textcolor{blue}{,}\textcolor{blue}{0}\textcolor{blue}{0}\textcolor{black}{0} \textcolor{blue}{med}\textcolor{blue}{als} \textcolor{blue}{to} \textcolor{black}{various} \textcolor{blue}{individuals} \textcolor{blue}{and} \textcolor{blue}{groups}\textcolor{blue}{.} \textcolor{blue}{The} \textcolor{black}{award}\textcolor{black}{ing} \textcolor{blue}{of} \textcolor{blue}{the} \textcolor{black}{Cong}\textcolor{blue}{r}\textcolor{blue}{essional} \textcolor{blue}{Gold} \textcolor{blue}{Medal} \textcolor{blue}{is} \textcolor{black}{not} \textcolor{blue}{a} \textcolor{black}{permanent} \textcolor{blue}{stat}\textcolor{blue}{ut}\textcolor{blue}{ory} \textcolor{black}{provision}\textcolor{blue}{,} \textcolor{blue}{and} \textcolor{blue}{it} \textcolor{blue}{is} \textcolor{black}{typically} \textcolor{blue}{done} \textcolor{blue}{through} \textcolor{black}{special} \textcolor{blue}{legisl}\textcolor{blue}{ation}\textcolor{blue}{.}
    
    \textcolor{blue}{The} \textcolor{black}{process} \textcolor{blue}{of} \textcolor{blue}{award}\textcolor{blue}{ing} \textcolor{blue}{the} \textcolor{black}{Cong}\textcolor{blue}{r}\textcolor{blue}{essional} \textcolor{blue}{Gold} \textcolor{blue}{Medal} \textcolor{black}{involves} \textcolor{black}{several} \textcolor{blue}{steps}\textcolor{blue}{,} \textcolor{blue}{including} \textcolor{black}{the} \textcolor{blue}{introduction} \textcolor{blue}{of} \textcolor{blue}{legisl}\textcolor{blue}{ation}\textcolor{black}{,} \textcolor{blue}{the} \textcolor{black}{consideration} \textcolor{blue}{of} \textcolor{blue}{the} \textcolor{blue}{legisl}\textcolor{blue}{ation} \textcolor{black}{by} \textcolor{blue}{the} \textcolor{black}{relevant} \textcolor{blue}{comm}\textcolor{blue}{itte}\textcolor{blue}{es}\textcolor{blue}{,} \textcolor{blue}{and} \textcolor{black}{the} \textcolor{blue}{appro}\textcolor{blue}{val} \textcolor{blue}{of} \textcolor{blue}{the} \textcolor{blue}{legisl}\textcolor{black}{ation} \textcolor{blue}{by} \textcolor{blue}{both} \textcolor{blue}{the} \textcolor{black}{House} \textcolor{blue}{of} \textcolor{blue}{Representatives} \textcolor{blue}{and} \textcolor{black}{the} \textcolor{blue}{Senate}\textcolor{blue}{.} \textcolor{blue}{Once} \textcolor{black}{the} \textcolor{blue}{legisl}\textcolor{blue}{ation} \textcolor{blue}{is} \textcolor{blue}{approved}\textcolor{blue}{,} \textcolor{black}{the} \textcolor{blue}{Secretary} \textcolor{blue}{of} \textcolor{blue}{the} \textcolor{blue}{Tre}\textcolor{blue}{as}\textcolor{black}{ury} \textcolor{black}{is} \textcolor{blue}{responsible} \textcolor{blue}{for} \textcolor{black}{striking} \textcolor{blue}{the} \textcolor{blue}{medal}\textcolor{blue}{,} \textcolor{black}{which} \textcolor{blue}{is} \textcolor{blue}{then} \textcolor{black}{presented}
    
    \textcolor{blue}{The} \textcolor{black}{design} \textcolor{blue}{of} \textcolor{blue}{the} \textcolor{blue}{Cong}\textcolor{blue}{r}\textcolor{blue}{essional} \textcolor{black}{Gold} \textcolor{blue}{Medal} \textcolor{blue}{is} \textcolor{blue}{typically} \textcolor{black}{determined} \textcolor{blue}{by} \textcolor{blue}{the} \textcolor{blue}{Secretary} \textcolor{blue}{of} \textcolor{blue}{the} \textcolor{black}{Tre}\textcolor{blue}{as}\textcolor{blue}{ury}\textcolor{blue}{,} \textcolor{black}{in} \textcolor{blue}{consult}\textcolor{blue}{ation} \textcolor{blue}{with} \textcolor{blue}{the} \textcolor{black}{Cit}\textcolor{blue}{iz}\textcolor{blue}{ens} \textcolor{black}{Co}\textcolor{blue}{in}\textcolor{black}{age} \textcolor{blue}{Ad}\textcolor{blue}{vis}\textcolor{blue}{ory} \textcolor{blue}{Committee} \textcolor{blue}{and} \textcolor{black}{the} \textcolor{black}{Commission} \textcolor{black}{of} \textcolor{blue}{Fine} \textcolor{blue}{Arts}\textcolor{blue}{.} \textcolor{blue}{The} \textcolor{blue}{medal} \textcolor{black}{typically} \textcolor{blue}{features} \textcolor{blue}{a} \textcolor{black}{portrait} \textcolor{blue}{of} \textcolor{blue}{the} \textcolor{blue}{recip}\textcolor{blue}{ient}\textcolor{black}{,} \textcolor{black}{as} \textcolor{blue}{well} \textcolor{blue}{as} \textcolor{blue}{ins}\textcolor{black}{cri}\textcolor{blue}{ptions} \textcolor{blue}{and} \textcolor{black}{symbols} \textcolor{blue}{that} \textcolor{blue}{reflect} \textcolor{blue}{the} \textcolor{black}{recip}\textcolor{blue}{ient}\textcolor{blue}{'}\textcolor{blue}{s} \textcolor{blue}{achiev}\textcolor{blue}{ements} \textcolor{black}{and} \textcolor{blue}{contributions}\textcolor{blue}{.}
    
    \textcolor{black}{The} \textcolor{blue}{Cong}\textcolor{blue}{r}\textcolor{blue}{essional} \textcolor{blue}{Gold} \textcolor{blue}{Medal} \textcolor{black}{is} \textcolor{blue}{considered} \textcolor{blue}{one} \textcolor{blue}{of} \textcolor{blue}{the} \textcolor{black}{highest} \textcolor{blue}{civil}\textcolor{blue}{ian} \textcolor{blue}{hon}\textcolor{blue}{ors} \textcolor{blue}{in} \textcolor{black}{the} \textcolor{blue}{United} \textcolor{blue}{States}\textcolor{blue}{,} \textcolor{blue}{and} \textcolor{blue}{it} \textcolor{black}{is} \textcolor{blue}{often} \textcolor{black}{given} \textcolor{blue}{to} \textcolor{blue}{individuals} \textcolor{blue}{who} \textcolor{blue}{have} \textcolor{black}{made} \textcolor{blue}{significant} \textcolor{blue}{contributions} \textcolor{blue}{to} \textcolor{black}{their} \textcolor{black}{field} \textcolor{blue}{or} \textcolor{blue}{to} \textcolor{black}{the} \textcolor{blue}{country} \textcolor{black}{as} \textcolor{blue}{a} \textcolor{blue}{whole}\textcolor{blue}{.} \textcolor{blue}{The} \textcolor{blue}{award} \textcolor{black}{has} \textcolor{blue}{been} \textcolor{blue}{given} \textcolor{blue}{to} \textcolor{black}{a} \textcolor{blue}{wide} \textcolor{blue}{range} \textcolor{blue}{of} \textcolor{blue}{individuals}\textcolor{blue}{,} \textcolor{black}{including} \textcolor{black}{military} \textcolor{blue}{hero}\textcolor{blue}{es}\textcolor{blue}{,} \textcolor{black}{civil} \textcolor{blue}{rights} \textcolor{blue}{leaders}\textcolor{blue}{,} \textcolor{blue}{and} \textcolor{black}{artists}\textcolor{blue}{.}
    
    \textcolor{black}{In} \textcolor{blue}{recent} \textcolor{blue}{years}\textcolor{blue}{,} \textcolor{blue}{the} \textcolor{black}{number} \textcolor{blue}{of} \textcolor{blue}{Cong}\textcolor{blue}{r}\textcolor{blue}{essional} \textcolor{blue}{Gold} \textcolor{black}{Med}\textcolor{blue}{als} \textcolor{blue}{awarded} \textcolor{black}{has} \textcolor{blue}{increased}\textcolor{blue}{,} \textcolor{black}{with} \textcolor{black}{over} \textcolor{blue}{}\textcolor{blue}{5}\textcolor{blue}{0} \textcolor{black}{b}\textcolor{blue}{ills} \textcolor{blue}{introduced} \textcolor{black}{in} \textcolor{blue}{the} \textcolor{blue}{}\textcolor{blue}{1}\textcolor{blue}{1}\textcolor{black}{3}\textcolor{blue}{th} \textcolor{blue}{Congress} \textcolor{black}{alone}\textcolor{blue}{.} \textcolor{blue}{The} \textcolor{blue}{award} \textcolor{blue}{has} \textcolor{black}{also} \textcolor{blue}{been} \textcolor{blue}{given} \textcolor{blue}{to} \textcolor{black}{a} \textcolor{black}{growing} \textcolor{blue}{number} \textcolor{blue}{of} \textcolor{blue}{groups}\textcolor{blue}{,} \textcolor{blue}{including} \textcolor{black}{military} \textcolor{black}{units}\textcolor{blue}{,} \textcolor{blue}{organizations}\textcolor{blue}{,} \textcolor{blue}{and} \textcolor{black}{even} \textcolor{black}{entire} \textcolor{blue}{cities}\textcolor{blue}{.}
    
    \textcolor{blue}{Over}\textcolor{blue}{all}\textcolor{black}{,} \textcolor{blue}{the} \textcolor{blue}{Cong}\textcolor{blue}{r}\textcolor{blue}{essional} \textcolor{blue}{Gold} \textcolor{black}{Medal} \textcolor{blue}{is} \textcolor{blue}{a} \textcolor{blue}{pr}\textcolor{blue}{estig}\textcolor{blue}{ious} \textcolor{black}{award} \textcolor{blue}{that} \textcolor{blue}{recogn}\textcolor{blue}{izes} \textcolor{blue}{the} \textcolor{black}{achiev}\textcolor{blue}{ements} \textcolor{blue}{and} \textcolor{blue}{contributions} \textcolor{blue}{of} \textcolor{blue}{individuals} \textcolor{black}{and} \textcolor{blue}{groups} \textcolor{blue}{to} \textcolor{blue}{the} \textcolor{black}{United} \textcolor{blue}{States}\textcolor{blue}{.} \textcolor{blue}{The} \textcolor{blue}{award} \textcolor{blue}{is} \textcolor{black}{given} \textcolor{black}{through} \textcolor{black}{special} \textcolor{blue}{legisl}\textcolor{blue}{ation} \textcolor{blue}{and} \textcolor{black}{involves} \textcolor{black}{several} \textcolor{blue}{steps}\textcolor{blue}{,} \textcolor{blue}{including} \textcolor{black}{the} \textcolor{blue}{introduction} \textcolor{blue}{of} \textcolor{blue}{legisl}\textcolor{blue}{ation}\textcolor{blue}{,} \textcolor{black}{the} \textcolor{blue}{consideration} \textcolor{blue}{of} \textcolor{blue}{the} \textcolor{black}{legisl}\textcolor{blue}{ation} \textcolor{blue}{by} \textcolor{blue}{the} \textcolor{black}{relevant} \textcolor{blue}{comm}\textcolor{blue}{itte}\textcolor{blue}{es}\textcolor{blue}{,} \textcolor{blue}{and} \textcolor{black}{the} \textcolor{blue}{appro}\textcolor{blue}{val} \textcolor{blue}{of} \textcolor{blue}{the} \textcolor{blue}{legisl}\textcolor{black}{ation} \textcolor{blue}{by} \textcolor{blue}{both} \textcolor{black}{the} \textcolor{blue}{House} \textcolor{blue}{of} \textcolor{blue}{Representatives} \textcolor{black}{and} \textcolor{blue}{the} \textcolor{blue}{Senate}\textcolor{blue}{.} \textcolor{blue}{The} \textcolor{black}{medal} \textcolor{blue}{is} \textcolor{blue}{typically} \textcolor{black}{designed} \textcolor{blue}{by} \textcolor{blue}{the} \textcolor{black}{Secretary} \textcolor{blue}{of} \textcolor{blue}{the} \textcolor{blue}{Tre}\textcolor{blue}{as}\textcolor{blue}{ury}\textcolor{black}{,} \textcolor{blue}{in} \textcolor{blue}{consult}\textcolor{blue}{ation} \textcolor{blue}{with} \textcolor{blue}{the} \textcolor{black}{Cit}\textcolor{blue}{iz}\textcolor{blue}{ens} \textcolor{blue}{Co}\textcolor{blue}{in}\textcolor{black}{age} \textcolor{blue}{Ad}\textcolor{blue}{vis}\textcolor{blue}{ory} \textcolor{blue}{Committee} \textcolor{blue}{and} \textcolor{black}{the} \textcolor{blue}{Commission} \textcolor{blue}{of} \textcolor{blue}{Fine} \textcolor{blue}{Arts}\textcolor{black}{,} \textcolor{blue}{and} \textcolor{blue}{it} \textcolor{black}{features} \textcolor{blue}{a} \textcolor{black}{portrait} \textcolor{blue}{of} \textcolor{blue}{the} \textcolor{blue}{recip}\textcolor{blue}{ient}\textcolor{blue}{,} \textcolor{black}{as} \textcolor{blue}{well} \textcolor{blue}{as} \textcolor{blue}{ins}\textcolor{blue}{cri}\textcolor{blue}{ptions} \textcolor{black}{and} \textcolor{blue}{symbols} \textcolor{blue}{that} \textcolor{blue}{reflect} \textcolor{blue}{the} \textcolor{black}{recip}\textcolor{blue}{ient}\textcolor{blue}{'}\textcolor{blue}{s} \textcolor{blue}{achiev}\textcolor{blue}{ements} \textcolor{black}{and} \textcolor{blue}{contributions}\textcolor{blue}{.}
\end{tcolorbox}
\newpage

\begin{tcolorbox}
  \textcolor{blue}{The} \textcolor{black}{report} \textcolor{black}{discuss}\textcolor{blue}{es} \textcolor{blue}{the} \textcolor{blue}{use} \textcolor{blue}{of} \textcolor{blue}{op}\textcolor{black}{io}\textcolor{blue}{id} \textcolor{blue}{medic}\textcolor{blue}{ations} \textcolor{blue}{in} \textcolor{black}{health}\textcolor{blue}{care} \textcolor{blue}{and} \textcolor{black}{the} \textcolor{blue}{potential} \textcolor{blue}{ris}\textcolor{blue}{ks} \textcolor{blue}{associated} \textcolor{blue}{with} \textcolor{black}{their} \textcolor{black}{mis}\textcolor{blue}{use}\textcolor{blue}{.} \textcolor{blue}{Op}\textcolor{blue}{io}\textcolor{blue}{id} \textcolor{black}{medic}\textcolor{blue}{ations} \textcolor{blue}{are} \textcolor{blue}{used} \textcolor{blue}{to} \textcolor{blue}{treat} \textcolor{black}{pain} \textcolor{blue}{and} \textcolor{black}{can} \textcolor{blue}{also} \textcolor{blue}{be} \textcolor{blue}{used} \textcolor{blue}{to} \textcolor{blue}{treat} \textcolor{black}{other} \textcolor{black}{health} \textcolor{blue}{problems}\textcolor{blue}{,} \textcolor{blue}{such} \textcolor{blue}{as} \textcolor{black}{severe} \textcolor{blue}{c}\textcolor{blue}{ough}\textcolor{blue}{ing}\textcolor{blue}{.} \textcolor{black}{There} \textcolor{blue}{are} \textcolor{blue}{three} \textcolor{blue}{types} \textcolor{blue}{of} \textcolor{blue}{op}\textcolor{black}{io}\textcolor{blue}{id} \textcolor{blue}{medic}\textcolor{blue}{ations} \textcolor{blue}{that} \textcolor{blue}{are} \textcolor{black}{approved} \textcolor{blue}{for} \textcolor{blue}{use} \textcolor{blue}{in} \textcolor{blue}{the} \textcolor{blue}{treatment} \textcolor{black}{of} \textcolor{blue}{op}\textcolor{blue}{io}\textcolor{blue}{id} \textcolor{blue}{use} \textcolor{blue}{dis}\textcolor{black}{orders}\textcolor{blue}{:} \textcolor{black}{m}\textcolor{blue}{eth}\textcolor{blue}{ad}\textcolor{blue}{one}\textcolor{blue}{,} \textcolor{black}{bu}\textcolor{blue}{pr}\textcolor{blue}{en}\textcolor{blue}{orph}\textcolor{blue}{ine}\textcolor{blue}{,} \textcolor{black}{and} \textcolor{blue}{n}\textcolor{blue}{alt}\textcolor{blue}{re}\textcolor{blue}{x}\textcolor{blue}{one}\textcolor{black}{.} \textcolor{blue}{M}\textcolor{blue}{eth}\textcolor{blue}{ad}\textcolor{blue}{one} \textcolor{blue}{is} \textcolor{black}{a} \textcolor{blue}{full} \textcolor{black}{op}\textcolor{blue}{io}\textcolor{blue}{id} \textcolor{blue}{ag}\textcolor{blue}{on}\textcolor{blue}{ist}\textcolor{black}{,} \textcolor{blue}{meaning} \textcolor{blue}{it} \textcolor{black}{bind}\textcolor{blue}{s} \textcolor{blue}{to} \textcolor{black}{and} \textcolor{blue}{activ}\textcolor{blue}{ates} \textcolor{blue}{op}\textcolor{blue}{io}\textcolor{blue}{id} \textcolor{black}{re}\textcolor{blue}{cept}\textcolor{blue}{ors} \textcolor{blue}{in} \textcolor{blue}{the} \textcolor{blue}{body}\textcolor{black}{.} \textcolor{black}{Bu}\textcolor{blue}{pr}\textcolor{blue}{en}\textcolor{blue}{orph}\textcolor{blue}{ine} \textcolor{blue}{is} \textcolor{black}{a} \textcolor{blue}{partial} \textcolor{blue}{op}\textcolor{blue}{io}\textcolor{blue}{id} \textcolor{black}{ag}\textcolor{blue}{on}\textcolor{blue}{ist}\textcolor{blue}{,} \textcolor{blue}{meaning} \textcolor{blue}{it} \textcolor{black}{also} \textcolor{blue}{bind}\textcolor{blue}{s} \textcolor{blue}{to} \textcolor{black}{and} \textcolor{black}{activ}\textcolor{blue}{ates} \textcolor{blue}{op}\textcolor{blue}{io}\textcolor{blue}{id} \textcolor{blue}{re}\textcolor{black}{cept}\textcolor{blue}{ors}\textcolor{blue}{,} \textcolor{blue}{but} \textcolor{black}{to} \textcolor{blue}{a} \textcolor{blue}{less}\textcolor{blue}{er} \textcolor{blue}{extent} \textcolor{blue}{than} \textcolor{black}{m}\textcolor{blue}{eth}\textcolor{blue}{ad}\textcolor{blue}{one}\textcolor{blue}{.} \textcolor{black}{N}\textcolor{blue}{alt}\textcolor{blue}{re}\textcolor{blue}{x}\textcolor{blue}{one} \textcolor{blue}{is} \textcolor{black}{an} \textcolor{blue}{op}\textcolor{blue}{io}\textcolor{blue}{id} \textcolor{blue}{ant}\textcolor{blue}{agon}\textcolor{black}{ist}\textcolor{blue}{,} \textcolor{blue}{meaning} \textcolor{blue}{it} \textcolor{black}{bind}\textcolor{blue}{s} \textcolor{blue}{to} \textcolor{black}{and} \textcolor{black}{blocks} \textcolor{blue}{the} \textcolor{black}{effects} \textcolor{blue}{of} \textcolor{blue}{op}\textcolor{blue}{io}\textcolor{blue}{id} \textcolor{blue}{re}\textcolor{black}{cept}\textcolor{blue}{ors}\textcolor{blue}{.}
  
  \textcolor{black}{The} \textcolor{black}{report} \textcolor{black}{also} \textcolor{blue}{discuss}\textcolor{blue}{es} \textcolor{blue}{the} \textcolor{blue}{potential} \textcolor{blue}{ris}\textcolor{black}{ks} \textcolor{blue}{associated} \textcolor{blue}{with} \textcolor{blue}{the} \textcolor{blue}{use} \textcolor{black}{of} \textcolor{blue}{op}\textcolor{blue}{io}\textcolor{blue}{id} \textcolor{blue}{medic}\textcolor{blue}{ations}\textcolor{black}{,} \textcolor{blue}{including} \textcolor{blue}{the} \textcolor{black}{risk} \textcolor{blue}{of} \textcolor{blue}{add}\textcolor{blue}{iction} \textcolor{blue}{and} \textcolor{black}{the} \textcolor{blue}{risk} \textcolor{blue}{of} \textcolor{blue}{over}\textcolor{blue}{d}\textcolor{blue}{ose}\textcolor{black}{.} \textcolor{blue}{The} \textcolor{blue}{use} \textcolor{blue}{of} \textcolor{blue}{op}\textcolor{blue}{io}\textcolor{black}{id} \textcolor{blue}{medic}\textcolor{blue}{ations} \textcolor{blue}{can} \textcolor{blue}{lead} \textcolor{blue}{to} \textcolor{black}{physical} \textcolor{blue}{dependence} \textcolor{blue}{and} \textcolor{black}{toler}\textcolor{blue}{ance}\textcolor{blue}{,} \textcolor{blue}{which} \textcolor{blue}{can} \textcolor{black}{make} \textcolor{blue}{it} \textcolor{blue}{difficult} \textcolor{blue}{to} \textcolor{black}{stop} \textcolor{blue}{using} \textcolor{blue}{the} \textcolor{blue}{medic}\textcolor{blue}{ation}\textcolor{blue}{.} \textcolor{black}{Additionally}\textcolor{blue}{,} \textcolor{blue}{the} \textcolor{black}{mis}\textcolor{blue}{use} \textcolor{blue}{of} \textcolor{blue}{op}\textcolor{blue}{io}\textcolor{blue}{id} \textcolor{black}{medic}\textcolor{blue}{ations} \textcolor{blue}{can} \textcolor{blue}{lead} \textcolor{blue}{to} \textcolor{black}{add}\textcolor{blue}{iction}\textcolor{blue}{,} \textcolor{blue}{which} \textcolor{blue}{can} \textcolor{black}{have} \textcolor{blue}{serious} \textcolor{blue}{consequences} \textcolor{blue}{for} \textcolor{blue}{the} \textcolor{black}{individual} \textcolor{blue}{and} \textcolor{blue}{their} \textcolor{black}{loved} \textcolor{blue}{ones}\textcolor{blue}{.}
  
  \textcolor{blue}{The} \textcolor{black}{report} \textcolor{blue}{also} \textcolor{blue}{discuss}\textcolor{blue}{es} \textcolor{blue}{the} \textcolor{blue}{potential} \textcolor{black}{ris}\textcolor{blue}{ks} \textcolor{blue}{associated} \textcolor{blue}{with} \textcolor{blue}{the} \textcolor{black}{di}\textcolor{blue}{version} \textcolor{blue}{of} \textcolor{blue}{op}\textcolor{blue}{io}\textcolor{blue}{id} \textcolor{black}{medic}\textcolor{blue}{ations}\textcolor{blue}{,} \textcolor{blue}{which} \textcolor{black}{is} \textcolor{blue}{the} \textcolor{black}{illegal} \textcolor{blue}{use} \textcolor{blue}{of} \textcolor{blue}{pres}\textcolor{blue}{cription} \textcolor{blue}{op}\textcolor{black}{io}\textcolor{blue}{ids} \textcolor{blue}{for} \textcolor{black}{non}\textcolor{blue}{-}\textcolor{blue}{med}\textcolor{blue}{ical} \textcolor{blue}{purposes}\textcolor{blue}{.} \textcolor{black}{D}\textcolor{blue}{ivers}\textcolor{blue}{ion} \textcolor{black}{can} \textcolor{blue}{lead} \textcolor{blue}{to} \textcolor{blue}{increased} \textcolor{black}{rates} \textcolor{blue}{of} \textcolor{blue}{add}\textcolor{blue}{iction}\textcolor{blue}{,} \textcolor{black}{over}\textcolor{blue}{d}\textcolor{blue}{ose}\textcolor{blue}{,} \textcolor{blue}{and} \textcolor{blue}{death}\textcolor{black}{.}
  
  \textcolor{blue}{The} \textcolor{blue}{report} \textcolor{black}{con}\textcolor{blue}{cludes} \textcolor{blue}{by} \textcolor{black}{discuss}\textcolor{blue}{ing} \textcolor{blue}{the} \textcolor{blue}{importance} \textcolor{blue}{of} \textcolor{black}{proper} \textcolor{blue}{use} \textcolor{blue}{and} \textcolor{black}{monitoring} \textcolor{blue}{of} \textcolor{blue}{op}\textcolor{blue}{io}\textcolor{blue}{id} \textcolor{blue}{medic}\textcolor{black}{ations}\textcolor{blue}{,} \textcolor{blue}{as} \textcolor{blue}{well} \textcolor{blue}{as} \textcolor{blue}{the} \textcolor{black}{need} \textcolor{blue}{for} \textcolor{black}{continued} \textcolor{blue}{research} \textcolor{blue}{and} \textcolor{black}{development} \textcolor{blue}{of} \textcolor{blue}{new} \textcolor{blue}{treat}\textcolor{blue}{ments} \textcolor{black}{for} \textcolor{blue}{op}\textcolor{blue}{io}\textcolor{blue}{id} \textcolor{blue}{use} \textcolor{black}{dis}\textcolor{blue}{orders}\textcolor{blue}{.}
\end{tcolorbox}

\begin{tcolorbox}
    \textcolor{blue}{The} \textcolor{blue}{Rail}\textcolor{black}{road} \textcolor{blue}{Ret}\textcolor{blue}{irement} \textcolor{blue}{Board} \textcolor{black}{(}\textcolor{blue}{R}\textcolor{blue}{R}\textcolor{blue}{B}\textcolor{blue}{)} \textcolor{blue}{is} \textcolor{black}{an} \textcolor{blue}{independent} \textcolor{blue}{federal} \textcolor{blue}{ag}\textcolor{blue}{ency} \textcolor{blue}{that} \textcolor{black}{admin}\textcolor{blue}{ister}\textcolor{blue}{s} \textcolor{black}{ret}\textcolor{blue}{irement}\textcolor{blue}{,} \textcolor{black}{surv}\textcolor{blue}{iv}\textcolor{blue}{or}\textcolor{blue}{,} \textcolor{black}{dis}\textcolor{blue}{ability}\textcolor{blue}{,} \textcolor{blue}{un}\textcolor{black}{emp}\textcolor{blue}{loyment}\textcolor{blue}{,} \textcolor{blue}{and} \textcolor{blue}{sick}\textcolor{blue}{ness} \textcolor{black}{ins}\textcolor{blue}{urance} \textcolor{blue}{for} \textcolor{black}{rail}\textcolor{blue}{road} \textcolor{blue}{workers} \textcolor{blue}{and} \textcolor{blue}{their} \textcolor{blue}{families}\textcolor{black}{.} \textcolor{blue}{The} \textcolor{blue}{R}\textcolor{blue}{R}\textcolor{blue}{B} \textcolor{black}{covers} \textcolor{black}{workers} \textcolor{blue}{who} \textcolor{blue}{are} \textcolor{blue}{employed} \textcolor{blue}{by} \textcolor{blue}{rail}\textcolor{black}{ro}\textcolor{blue}{ads} \textcolor{black}{engaged} \textcolor{blue}{in} \textcolor{black}{inter}\textcolor{blue}{state} \textcolor{blue}{commerce} \textcolor{blue}{and} \textcolor{black}{related} \textcolor{black}{subs}\textcolor{blue}{idi}\textcolor{blue}{aries}\textcolor{blue}{,} \textcolor{black}{rail}\textcolor{blue}{road} \textcolor{blue}{associations}\textcolor{blue}{,} \textcolor{blue}{and} \textcolor{blue}{rail}\textcolor{black}{road} \textcolor{blue}{labor} \textcolor{blue}{organizations}\textcolor{blue}{.}
    
    \textcolor{blue}{The} \textcolor{black}{R}\textcolor{blue}{R}\textcolor{blue}{B} \textcolor{blue}{has} \textcolor{blue}{two} \textcolor{blue}{main} \textcolor{black}{programs}\textcolor{blue}{:} \textcolor{blue}{the} \textcolor{blue}{Rail}\textcolor{blue}{road} \textcolor{blue}{Ret}\textcolor{black}{irement} \textcolor{black}{Act} \textcolor{blue}{(}\textcolor{blue}{R}\textcolor{blue}{RA}\textcolor{blue}{)} \textcolor{black}{and} \textcolor{blue}{the} \textcolor{blue}{Rail}\textcolor{blue}{road} \textcolor{black}{Un}\textcolor{blue}{emp}\textcolor{blue}{loyment} \textcolor{blue}{In}\textcolor{blue}{sur}\textcolor{blue}{ance} \textcolor{black}{Act} \textcolor{blue}{(}\textcolor{blue}{R}\textcolor{black}{UI}\textcolor{blue}{A}\textcolor{blue}{).} \textcolor{blue}{The} \textcolor{blue}{R}\textcolor{blue}{RA} \textcolor{black}{author}\textcolor{blue}{izes} \textcolor{blue}{ret}\textcolor{blue}{irement}\textcolor{black}{,} \textcolor{black}{surv}\textcolor{blue}{iv}\textcolor{blue}{or}\textcolor{blue}{,} \textcolor{blue}{and} \textcolor{blue}{dis}\textcolor{black}{ability} \textcolor{blue}{benefits} \textcolor{blue}{for} \textcolor{blue}{rail}\textcolor{blue}{road} \textcolor{blue}{workers} \textcolor{black}{and} \textcolor{blue}{their} \textcolor{blue}{families}\textcolor{blue}{.} \textcolor{blue}{The} \textcolor{blue}{R}\textcolor{black}{UI}\textcolor{blue}{A} \textcolor{black}{provides} \textcolor{black}{un}\textcolor{blue}{emp}\textcolor{blue}{loyment} \textcolor{blue}{and} \textcolor{black}{sick}\textcolor{blue}{ness} \textcolor{blue}{benefits} \textcolor{blue}{for} \textcolor{blue}{rail}\textcolor{blue}{road} \textcolor{black}{workers}\textcolor{blue}{.}
    
    \textcolor{blue}{The} \textcolor{black}{number} \textcolor{blue}{of} \textcolor{blue}{rail}\textcolor{blue}{road} \textcolor{blue}{workers} \textcolor{black}{has} \textcolor{black}{been} \textcolor{blue}{decl}\textcolor{blue}{ining} \textcolor{blue}{since} \textcolor{blue}{the} \textcolor{blue}{}\textcolor{black}{1}\textcolor{blue}{9}\textcolor{blue}{5}\textcolor{blue}{0}\textcolor{blue}{s}\textcolor{blue}{,} \textcolor{black}{although} \textcolor{blue}{the} \textcolor{blue}{rate} \textcolor{blue}{of} \textcolor{black}{decl}\textcolor{blue}{ine} \textcolor{blue}{has} \textcolor{blue}{been} \textcolor{black}{irregular}\textcolor{blue}{.} \textcolor{blue}{In} \textcolor{blue}{recent} \textcolor{blue}{years}\textcolor{blue}{,} \textcolor{black}{rail}\textcolor{blue}{road} \textcolor{blue}{employ}\textcolor{blue}{ment} \textcolor{blue}{has} \textcolor{blue}{increased} \textcolor{black}{after} \textcolor{black}{reaching} \textcolor{blue}{an} \textcolor{blue}{all}\textcolor{blue}{-}\textcolor{blue}{time} \textcolor{blue}{low} \textcolor{black}{of} \textcolor{blue}{}\textcolor{blue}{2}\textcolor{black}{1}\textcolor{blue}{5}\textcolor{blue}{,}\textcolor{blue}{0}\textcolor{blue}{0}\textcolor{blue}{0} \textcolor{black}{workers} \textcolor{blue}{in} \textcolor{blue}{January} \textcolor{blue}{}\textcolor{blue}{2}\textcolor{blue}{0}\textcolor{black}{1}\textcolor{blue}{0}\textcolor{blue}{.} \textcolor{blue}{In} \textcolor{black}{April} \textcolor{blue}{}\textcolor{blue}{2}\textcolor{blue}{0}\textcolor{blue}{1}\textcolor{blue}{5}\textcolor{black}{,} \textcolor{blue}{rail}\textcolor{blue}{road} \textcolor{blue}{employ}\textcolor{blue}{ment} \textcolor{black}{pe}\textcolor{blue}{aked} \textcolor{blue}{at} \textcolor{blue}{}\textcolor{blue}{2}\textcolor{black}{5}\textcolor{black}{3}\textcolor{blue}{,}\textcolor{blue}{0}\textcolor{blue}{0}\textcolor{blue}{0} \textcolor{blue}{workers}\textcolor{black}{,} \textcolor{blue}{the} \textcolor{blue}{highest} \textcolor{blue}{level} \textcolor{blue}{since} \textcolor{black}{November} \textcolor{blue}{}\textcolor{blue}{1}\textcolor{blue}{9}\textcolor{blue}{9}\textcolor{blue}{9}\textcolor{black}{,} \textcolor{blue}{and} \textcolor{blue}{then} \textcolor{black}{decl}\textcolor{blue}{ined} \textcolor{black}{through} \textcolor{black}{F}\textcolor{blue}{Y}\textcolor{blue}{2}\textcolor{blue}{0}\textcolor{blue}{1}\textcolor{blue}{7}\textcolor{black}{,} \textcolor{blue}{falling} \textcolor{blue}{to} \textcolor{blue}{}\textcolor{blue}{2}\textcolor{blue}{2}\textcolor{black}{1}\textcolor{blue}{,}\textcolor{blue}{0}\textcolor{blue}{0}\textcolor{blue}{0} \textcolor{black}{workers}\textcolor{blue}{.}
    
    %\textcolor{blue}{The} \textcolor{blue}{total} \textcolor{black}{number} \textcolor{blue}{of} \textcolor{blue}{benef}\textcolor{blue}{ici}\textcolor{blue}{aries} \textcolor{black}{under} \textcolor{blue}{the} \textcolor{blue}{R}\textcolor{blue}{RA} \textcolor{black}{and} \textcolor{blue}{R}\textcolor{blue}{UI}\textcolor{black}{A} \textcolor{blue}{decre}\textcolor{blue}{ased} \textcolor{blue}{from} \textcolor{blue}{}\textcolor{black}{6}\textcolor{blue}{2}\textcolor{black}{3}\textcolor{blue}{,}\textcolor{blue}{0}\textcolor{blue}{0}\textcolor{blue}{0} \textcolor{blue}{in} \textcolor{black}{F}\textcolor{blue}{Y}\textcolor{blue}{2}\textcolor{blue}{0}\textcolor{blue}{0}\textcolor{black}{8} \textcolor{blue}{to} \textcolor{blue}{}\textcolor{blue}{5}\textcolor{black}{7}\textcolor{black}{4}\textcolor{blue}{,}\textcolor{blue}{0}\textcolor{blue}{0}\textcolor{blue}{0} \textcolor{blue}{in} \textcolor{black}{F}\textcolor{blue}{Y}\textcolor{blue}{2}\textcolor{blue}{0}\textcolor{blue}{1}\textcolor{blue}{7}\textcolor{black}{.} \textcolor{blue}{Total} \textcolor{black}{benefit} \textcolor{blue}{pay}\textcolor{blue}{ments} \textcolor{black}{increased} \textcolor{blue}{from} \textcolor{blue}{\$}\textcolor{blue}{1}\textcolor{black}{0}\textcolor{blue}{.}\textcolor{blue}{1} \textcolor{black}{billion} \textcolor{blue}{to} \textcolor{blue}{\$}\textcolor{blue}{1}\textcolor{blue}{2}\textcolor{blue}{.}\textcolor{black}{6} \textcolor{blue}{billion} \textcolor{blue}{during} \textcolor{blue}{the} \textcolor{blue}{same} \textcolor{blue}{period}\textcolor{black}{.} \textcolor{blue}{In} \textcolor{blue}{F}\textcolor{blue}{Y}\textcolor{blue}{2}\textcolor{blue}{0}\textcolor{black}{1}\textcolor{blue}{7}\textcolor{blue}{,} \textcolor{blue}{the} \textcolor{blue}{R}\textcolor{blue}{R}\textcolor{black}{B} \textcolor{blue}{paid} \textcolor{blue}{nearly} \textcolor{blue}{\$}\textcolor{blue}{1}\textcolor{blue}{2}\textcolor{black}{.}\textcolor{blue}{5} \textcolor{blue}{billion} \textcolor{blue}{in} \textcolor{blue}{ret}\textcolor{blue}{irement}\textcolor{black}{,} \textcolor{blue}{dis}\textcolor{blue}{ability}\textcolor{blue}{,} \textcolor{blue}{and} \textcolor{black}{surv}\textcolor{blue}{iv}\textcolor{blue}{or} \textcolor{blue}{benefits} \textcolor{blue}{to} \textcolor{black}{approximately} \textcolor{blue}{}\textcolor{blue}{5}\textcolor{blue}{4}\textcolor{black}{8}\textcolor{blue}{,}\textcolor{blue}{0}\textcolor{blue}{0}\textcolor{blue}{0} \textcolor{blue}{benef}\textcolor{black}{ici}\textcolor{blue}{aries}\textcolor{blue}{.} \textcolor{black}{Al}\textcolor{blue}{most} \textcolor{blue}{\$}\textcolor{blue}{1}\textcolor{black}{0}\textcolor{blue}{5}\textcolor{blue}{.}\textcolor{black}{4} \textcolor{blue}{million} \textcolor{blue}{in} \textcolor{blue}{un}\textcolor{blue}{emp}\textcolor{blue}{loyment} \textcolor{black}{and} \textcolor{blue}{sick}\textcolor{blue}{ness} \textcolor{blue}{benefits} \textcolor{blue}{were} \textcolor{blue}{paid} \textcolor{black}{to} \textcolor{black}{approximately} \textcolor{blue}{}\textcolor{blue}{2}\textcolor{black}{8}\textcolor{blue}{,}\textcolor{blue}{0}\textcolor{blue}{0}\textcolor{blue}{0} \textcolor{black}{claim}\textcolor{blue}{ants}\textcolor{blue}{.}
    
    %\textcolor{blue}{The} \textcolor{black}{R}\textcolor{blue}{RA} \textcolor{blue}{and} \textcolor{black}{R}\textcolor{black}{UI}\textcolor{blue}{A} \textcolor{blue}{are} \textcolor{blue}{fund}\textcolor{blue}{ed} \textcolor{blue}{by} \textcolor{black}{pay}\textcolor{blue}{roll} \textcolor{blue}{tax}\textcolor{blue}{es}\textcolor{blue}{,} \textcolor{black}{financial} \textcolor{black}{inter}\textcolor{blue}{changes} \textcolor{black}{from} \textcolor{black}{Social} \textcolor{blue}{Security}\textcolor{blue}{,} \textcolor{blue}{and} \textcolor{black}{trans}\textcolor{blue}{fers} \textcolor{blue}{from} \textcolor{blue}{the} \textcolor{blue}{National} \textcolor{blue}{Rail}\textcolor{black}{road} \textcolor{blue}{Ret}\textcolor{blue}{irement} \textcolor{blue}{In}\textcolor{black}{vest}\textcolor{blue}{ment} \textcolor{blue}{Trust} \textcolor{blue}{(}\textcolor{blue}{NR}\textcolor{blue}{R}\textcolor{black}{IT}\textcolor{blue}{).} \textcolor{blue}{The} \textcolor{blue}{R}\textcolor{blue}{R}\textcolor{blue}{B}\textcolor{black}{'}\textcolor{blue}{s} \textcolor{blue}{main} \textcolor{blue}{source} \textcolor{blue}{of} \textcolor{black}{fund}\textcolor{blue}{ing} \textcolor{blue}{is} \textcolor{blue}{the} \textcolor{black}{pay}\textcolor{blue}{roll} \textcolor{blue}{tax}\textcolor{blue}{es} \textcolor{blue}{paid} \textcolor{blue}{by} \textcolor{black}{rail}\textcolor{blue}{road} \textcolor{blue}{employ}\textcolor{blue}{ers} \textcolor{black}{and} \textcolor{blue}{employees}\textcolor{blue}{.} \textcolor{blue}{The} \textcolor{black}{Tier} \textcolor{blue}{I} \textcolor{black}{tax} \textcolor{blue}{is} \textcolor{blue}{the} \textcolor{black}{same} \textcolor{blue}{as} \textcolor{blue}{the} \textcolor{blue}{Social} \textcolor{blue}{Security} \textcolor{black}{pay}\textcolor{blue}{roll} \textcolor{blue}{tax}\textcolor{blue}{,} \textcolor{blue}{while} \textcolor{blue}{the} \textcolor{black}{Tier} \textcolor{blue}{II} \textcolor{blue}{tax} \textcolor{black}{is} \textcolor{blue}{set} \textcolor{black}{each} \textcolor{blue}{year} \textcolor{blue}{based} \textcolor{blue}{on} \textcolor{blue}{the} \textcolor{black}{rail}\textcolor{blue}{road} \textcolor{blue}{ret}\textcolor{blue}{irement} \textcolor{black}{system}\textcolor{blue}{'}\textcolor{blue}{s} \textcolor{blue}{asset} \textcolor{black}{bal}\textcolor{blue}{ances}\textcolor{blue}{,} \textcolor{black}{benefit} \textcolor{blue}{pay}\textcolor{blue}{ments}\textcolor{blue}{,} \textcolor{blue}{and} \textcolor{black}{administrative} \textcolor{blue}{costs}\textcolor{blue}{.}
    
    \textcolor{blue}{The} \textcolor{black}{R}\textcolor{blue}{R}\textcolor{blue}{B}\textcolor{blue}{'}\textcolor{blue}{s} \textcolor{black}{programs} \textcolor{blue}{are} \textcolor{blue}{designed} \textcolor{blue}{to} \textcolor{blue}{provide} \textcolor{black}{compreh}\textcolor{blue}{ensive} \textcolor{blue}{benefits} \textcolor{blue}{to} \textcolor{blue}{rail}\textcolor{blue}{road} \textcolor{black}{workers} \textcolor{blue}{and} \textcolor{blue}{their} \textcolor{blue}{families}\textcolor{blue}{.} \textcolor{blue}{The} \textcolor{black}{R}\textcolor{blue}{RA} \textcolor{blue}{and} \textcolor{black}{R}\textcolor{blue}{UI}\textcolor{blue}{A} \textcolor{black}{are} \textcolor{black}{important} \textcolor{blue}{components} \textcolor{blue}{of} \textcolor{blue}{the} \textcolor{blue}{rail}\textcolor{blue}{road} \textcolor{black}{industry}\textcolor{blue}{'}\textcolor{blue}{s} \textcolor{blue}{ret}\textcolor{blue}{irement} \textcolor{blue}{and} \textcolor{black}{benefits} \textcolor{blue}{system}\textcolor{blue}{.} \textcolor{blue}{The} \textcolor{blue}{R}\textcolor{blue}{R}\textcolor{black}{B}\textcolor{blue}{'}\textcolor{blue}{s} \textcolor{black}{efforts} \textcolor{blue}{to} \textcolor{blue}{maintain} \textcolor{black}{and} \textcolor{blue}{improve} \textcolor{blue}{these} \textcolor{black}{programs} \textcolor{blue}{are} \textcolor{black}{cru}\textcolor{blue}{cial} \textcolor{blue}{for} \textcolor{blue}{the} \textcolor{black}{well}\textcolor{blue}{-}\textcolor{blue}{be}\textcolor{blue}{ing} \textcolor{blue}{of} \textcolor{blue}{rail}\textcolor{black}{road} \textcolor{blue}{workers} \textcolor{blue}{and} \textcolor{blue}{their} \textcolor{blue}{families}\textcolor{blue}{.}
\end{tcolorbox}

\begin{tcolorbox}
    \textcolor{blue}{The} \textcolor{blue}{report} \textcolor{black}{provides} \textcolor{blue}{an} \textcolor{blue}{over}\textcolor{blue}{view} \textcolor{blue}{of} \textcolor{blue}{the} \textcolor{black}{annual} \textcolor{blue}{appropri}\textcolor{blue}{ations} \textcolor{blue}{for} \textcolor{blue}{the} \textcolor{blue}{Department} \textcolor{black}{of} \textcolor{blue}{Hom}\textcolor{blue}{eland} \textcolor{blue}{Security} \textcolor{blue}{(}\textcolor{blue}{D}\textcolor{black}{HS}\textcolor{blue}{)} \textcolor{blue}{for} \textcolor{blue}{F}\textcolor{blue}{Y}\textcolor{blue}{2}\textcolor{black}{0}\textcolor{blue}{1}\textcolor{blue}{9}\textcolor{blue}{.} \textcolor{blue}{It} \textcolor{black}{comp}\textcolor{blue}{ares} \textcolor{blue}{the} \textcolor{black}{en}\textcolor{blue}{act}\textcolor{blue}{ed} \textcolor{blue}{F}\textcolor{blue}{Y}\textcolor{blue}{2}\textcolor{black}{0}\textcolor{blue}{1}\textcolor{blue}{8} \textcolor{blue}{appropri}\textcolor{blue}{ations} \textcolor{blue}{for} \textcolor{black}{D}\textcolor{blue}{HS}\textcolor{blue}{,} \textcolor{blue}{the} \textcolor{black}{Trump} \textcolor{blue}{Administration}\textcolor{blue}{'}\textcolor{blue}{s} \textcolor{blue}{F}\textcolor{black}{Y}\textcolor{blue}{2}\textcolor{blue}{0}\textcolor{blue}{1}\textcolor{blue}{9} \textcolor{blue}{budget} \textcolor{black}{request}\textcolor{blue}{,} \textcolor{blue}{and} \textcolor{blue}{the} \textcolor{black}{appropri}\textcolor{blue}{ations} \textcolor{blue}{measures} \textcolor{black}{developed} \textcolor{blue}{and} \textcolor{black}{considered} \textcolor{blue}{by} \textcolor{blue}{Congress} \textcolor{blue}{in} \textcolor{black}{response} \textcolor{blue}{to} \textcolor{blue}{the} \textcolor{black}{request}\textcolor{blue}{.} \textcolor{blue}{The} \textcolor{blue}{report} \textcolor{black}{ident}\textcolor{blue}{ifies} \textcolor{blue}{additional} \textcolor{black}{inform}\textcolor{blue}{ational} \textcolor{blue}{resources}\textcolor{blue}{,} \textcolor{black}{reports}\textcolor{blue}{,} \textcolor{blue}{and} \textcolor{black}{policy} \textcolor{blue}{exper}\textcolor{blue}{ts} \textcolor{blue}{that} \textcolor{black}{can} \textcolor{blue}{provide} \textcolor{black}{further} \textcolor{blue}{information} \textcolor{blue}{on} \textcolor{blue}{D}\textcolor{blue}{HS} \textcolor{black}{appropri}\textcolor{blue}{ations}\textcolor{blue}{.}
    
    \textcolor{blue}{The} \textcolor{blue}{report} \textcolor{black}{explains} \textcolor{black}{several} \textcolor{black}{special}\textcolor{blue}{ized} \textcolor{blue}{budget}\textcolor{blue}{ary} \textcolor{black}{concepts}\textcolor{blue}{,} \textcolor{blue}{including} \textcolor{blue}{budget} \textcolor{black}{authority}\textcolor{blue}{,} \textcolor{black}{oblig}\textcolor{blue}{ations}\textcolor{blue}{,} \textcolor{blue}{out}\textcolor{black}{l}\textcolor{blue}{ays}\textcolor{blue}{,} \textcolor{blue}{dis}\textcolor{blue}{cret}\textcolor{blue}{ion}\textcolor{black}{ary} \textcolor{black}{and} \textcolor{black}{mand}\textcolor{blue}{atory} \textcolor{blue}{sp}\textcolor{blue}{ending}\textcolor{blue}{,} \textcolor{black}{offset}\textcolor{blue}{ting} \textcolor{black}{collections}\textcolor{blue}{,} \textcolor{blue}{alloc}\textcolor{black}{ations}\textcolor{blue}{,} \textcolor{blue}{and} \textcolor{black}{adjust}\textcolor{blue}{ments} \textcolor{blue}{to} \textcolor{blue}{the} \textcolor{black}{dis}\textcolor{blue}{cret}\textcolor{blue}{ion}\textcolor{blue}{ary} \textcolor{blue}{sp}\textcolor{blue}{ending} \textcolor{black}{caps} \textcolor{black}{under} \textcolor{blue}{the} \textcolor{black}{Bud}\textcolor{blue}{get} \textcolor{blue}{Control} \textcolor{blue}{Act} \textcolor{blue}{(}\textcolor{black}{BC}\textcolor{blue}{A}\textcolor{blue}{).} \textcolor{blue}{It} \textcolor{blue}{also} \textcolor{black}{provides} \textcolor{blue}{a} \textcolor{blue}{detailed} \textcolor{blue}{analysis} \textcolor{blue}{of} \textcolor{blue}{the} \textcolor{black}{appropri}\textcolor{blue}{ations} \textcolor{blue}{process} \textcolor{blue}{for} \textcolor{blue}{D}\textcolor{blue}{HS}\textcolor{black}{,} \textcolor{blue}{including} \textcolor{blue}{the} \textcolor{black}{various} \textcolor{blue}{comm}\textcolor{blue}{itte}\textcolor{blue}{es} \textcolor{blue}{and} \textcolor{black}{sub}\textcolor{blue}{comm}\textcolor{blue}{itte}\textcolor{blue}{es} \textcolor{blue}{involved}\textcolor{blue}{,} \textcolor{black}{and} \textcolor{blue}{the} \textcolor{blue}{role} \textcolor{blue}{of} \textcolor{blue}{the} \textcolor{black}{Cong}\textcolor{blue}{r}\textcolor{blue}{essional} \textcolor{blue}{Bud}\textcolor{blue}{get} \textcolor{blue}{Office} \textcolor{black}{(}\textcolor{blue}{C}\textcolor{blue}{BO}\textcolor{blue}{)} \textcolor{blue}{and} \textcolor{blue}{the} \textcolor{black}{Government} \textcolor{blue}{Account}\textcolor{blue}{ability} \textcolor{blue}{Office} \textcolor{blue}{(}\textcolor{blue}{GA}\textcolor{black}{O}\textcolor{blue}{).}
    
    \textcolor{black}{The} \textcolor{blue}{report} \textcolor{black}{highlight}\textcolor{blue}{s} \textcolor{blue}{the} \textcolor{black}{key} \textcolor{blue}{issues} \textcolor{black}{and} \textcolor{black}{deb}\textcolor{blue}{ates} \textcolor{blue}{surrounding} \textcolor{black}{D}\textcolor{blue}{HS} \textcolor{blue}{appropri}\textcolor{blue}{ations}\textcolor{blue}{,} \textcolor{blue}{including} \textcolor{black}{fund}\textcolor{blue}{ing} \textcolor{blue}{for} \textcolor{blue}{border} \textcolor{blue}{security}\textcolor{blue}{,} \textcolor{black}{imm}\textcolor{blue}{igration} \textcolor{blue}{enfor}\textcolor{blue}{cement}\textcolor{blue}{,} \textcolor{black}{cy}\textcolor{blue}{ber}\textcolor{blue}{security}\textcolor{blue}{,} \textcolor{blue}{and} \textcolor{blue}{dis}\textcolor{black}{aster} \textcolor{blue}{response}\textcolor{blue}{.} \textcolor{blue}{It} \textcolor{blue}{also} \textcolor{black}{discuss}\textcolor{blue}{es} \textcolor{blue}{the} \textcolor{blue}{impact} \textcolor{blue}{of} \textcolor{blue}{the} \textcolor{black}{B}\textcolor{blue}{CA} \textcolor{blue}{on} \textcolor{black}{D}\textcolor{blue}{HS} \textcolor{blue}{appropri}\textcolor{black}{ations} \textcolor{blue}{and} \textcolor{blue}{the} \textcolor{black}{potential} \textcolor{blue}{for} \textcolor{blue}{future} \textcolor{black}{changes} \textcolor{blue}{to} \textcolor{blue}{the} \textcolor{black}{sp}\textcolor{blue}{ending} \textcolor{blue}{caps}\textcolor{blue}{.}
    
    \textcolor{blue}{Over}\textcolor{black}{all}\textcolor{blue}{,} \textcolor{blue}{the} \textcolor{blue}{report} \textcolor{blue}{provides} \textcolor{blue}{a} \textcolor{black}{compreh}\textcolor{blue}{ensive} \textcolor{blue}{analysis} \textcolor{blue}{of} \textcolor{blue}{the} \textcolor{black}{annual} \textcolor{blue}{appropri}\textcolor{blue}{ations} \textcolor{blue}{for} \textcolor{blue}{D}\textcolor{blue}{HS} \textcolor{black}{and} \textcolor{blue}{the} \textcolor{black}{factors} \textcolor{blue}{that} \textcolor{blue}{influence} \textcolor{blue}{the} \textcolor{black}{allocation} \textcolor{blue}{of} \textcolor{blue}{fund}\textcolor{blue}{ing}\textcolor{blue}{.} \textcolor{blue}{It} \textcolor{black}{is} \textcolor{blue}{a} \textcolor{blue}{valuable} \textcolor{blue}{resource} \textcolor{blue}{for} \textcolor{blue}{polic}\textcolor{black}{ym}\textcolor{blue}{akers}\textcolor{blue}{,} \textcolor{black}{anal}\textcolor{blue}{yst}\textcolor{blue}{s}\textcolor{blue}{,} \textcolor{blue}{and} \textcolor{blue}{st}\textcolor{black}{ake}\textcolor{blue}{hold}\textcolor{blue}{ers} \textcolor{blue}{interested} \textcolor{blue}{in} \textcolor{blue}{understanding} \textcolor{black}{the} \textcolor{blue}{complex}\textcolor{blue}{ities} \textcolor{blue}{of} \textcolor{blue}{D}\textcolor{blue}{HS} \textcolor{black}{appropri}\textcolor{blue}{ations} \textcolor{blue}{and} \textcolor{blue}{the} \textcolor{black}{challeng}\textcolor{blue}{es} \textcolor{blue}{facing} \textcolor{blue}{the} \textcolor{black}{department} \textcolor{blue}{in} \textcolor{blue}{the} \textcolor{black}{coming} \textcolor{blue}{years}\textcolor{blue}{.}

\end{tcolorbox}

\end{document}

