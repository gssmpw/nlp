\section{Related Work}
\paragraph{\textbf{\textit{Human Value Alignment for LLMs.}}}
% The rapid development of LLMs has revolutionized various domains ____. While LLMs demonstrate remarkable capabilities in handling complex tasks, aligning LLMs with human values remains a challenge due to biases in training data and trade-offs between usefulness and safety ____.
Aligning LLMs with human values remains a challenge due to biases in training data and trade-offs between usefulness and safety ____. 
Approaches such as Reinforcement Learning from Human Feedback (RLHF) ____ have been proposed to improve fairness ____, safety ____ and eliminate hallucinations ____. 
% To further promote the alignment of LLM and human values, we explore the vulnerability in LLMs' imagination in our study.


\paragraph{\textbf{\textit{Jailbreak Attacks on LLMs.}}}
Jailbreak attacks threaten the safety alignment mechanisms of LLMs, potentially leading to the generation of harmful content ____. 
% These attacks can be broadly categorized into white-box and black-box approaches. White-box attacks leverage detailed knowledge of model architectures and parameters to bypass safety controls ____, while black-box attacks rely on crafted inputs to exploit vulnerabilities in alignment mechanisms ____. 
Our study is inspired by two key methods in black-box attacks: prompt nesting and multi-turn dialogue attacks.
1) {{{Prompt Nesting Attack.}}}
Prompt nesting bypasses security features by nesting malicious intents in normal prompts, altering LLMs’ context. 
% Techniques like 
DeepInception ____ exploit nested scenarios, while ReNeLLM ____ rewrites prompt to jailbreak based on code completion, text continuation, or form-filling tasks. MJP ____ uses multi-step approaches with contextual contamination to reduce moral constraints, prompting malicious responses.
2) {{Multi-turn Dialogue Attack.}}
LLMs that are safe in isolated, single-round interactions can be gradually manipulated into generating harmful outputs through multiple rounds of interaction ____. Multi-turn dialogue attack leverages the multi-turn nature of conversational interactions to gradually erode an LLM's content restrictions. 
% Crescendo ____ exploits seemingly benign exchanges to prompt malicious tasks. Chain-of-Attack ____ uses iterative prompting to gradually increase the relevance of responses to the harmful objective while avoiding explicit safety triggers.


%  {{Multi-turn Dialogue Attack.}}
% LLMs that are safe in isolated, single-round interactions can be gradually manipulated into generating harmful outputs through multiple rounds of interaction ____. Multi-turn dialogue attack leverages the multi-turn nature of conversational interactions to gradually erode an LLM's content restrictions. Crescendo ____ exploits seemingly benign exchanges to prompt malicious tasks. Chain-of-Attack ____ uses iterative prompting to gradually increase the relevance of responses to the harmful objective while avoiding explicit safety triggers.
% % % To defend against these attacks, ____ evaluates methods like XML tagging, structured outputs, and query-rewriting.



% {{Prompt Nesting Attack.}}
% Prompt nesting conceals malicious intents within seemingly benign prompts, altering the LLM’s context to bypass security features ____.

% {{Multi-turn Dialogue Attack.}}
% Multi-turn dialogue attacks exploit the iterative nature of conversations to gradually erode content restrictions across multiple rounds of interaction ____.