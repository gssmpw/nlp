Abstract:
{
  Artificial intelligence (AI) is quickly becoming a cornerstone of modern radiology, accelerating triage,or offering second opinions to ensure correct medical image interpretation.
  However, the performance of AI models can degrade during clinical deployment due to population shifts, there is a need for population-specific clinical trials to ensure high performance for each adopting institution.
  However, deployment-time validation on collected imaging data is not practically feasible.
  Here we propose using conditional image synthesis as a flexible, scalable alternative to real data collection that can be incorporated much earlier in virtual clinical trials to anticipate changes in model performance on diverse target populations.
  We demonstrate for the first time that a generative model is capable of realistically synthesizing full-body, 3D computed tomography images, conditioned on patient attributes such as sex, age, height, and weight.
  Furthermore, we show in a controlled experiment that these images are of sufficient quality to replicate hidden biases in AI models that are otherwise present in real data.
  This method offers developers and regulatory bodies a scalable tool for identifying and correcting potential weaknesses in AI systems, ensuring their robustness across diverse populations over time.
  By enabling earlier detection of performance issues, our approach represents a significant advancement in ensuring the safety and efficacy of AI in healthcare.
}

Precision medicine aims to maximize healthcare quality by tailoring interventions to individuals.\cite{kosorok2019precision} % using data-driven decision-making
Central to this vision are algorithms developed using artificial intelligence and machine learning (AI/ML) to analyze medical images and support personalized, data-driven treatment plans. AI algorithms for automated CT analysis, for instance, have been used to assist in body composition measurement,\cite{zopfs2020evaluating} disease detection\cite{pickhardt2013opportunistic, jang2019opportunistic,eng2021automated, oh2024evaluation} and clinical decision-making.\cite{ozsahin2020review, murugesan2022a, mohammadi2024deep}
However, the performance of machine learning models may degrade during deployment due to population shift.\cite{oakden2020hidden, compton2023when, william2022external}
Anticipating performance degradations is paramount because an erroneous output can adversely affect patient outcomes and erode trust in AI systems.\cite{topol2019high}
Despite this, there is currently no way to ensure that the performance measured during pre-market development translates to post-market deployment.
While traditional clinical trials are appropriate for validating a model's initial performance through large scale cohort selection and data collection (see Fig.~\ref{fig:overview}a), they are not a practical solution for deployment-time validation due to cost and privacy considerations \cite{rajpurkar2022ai, menard2024artificial}.
% A single vendor may deploy an algorithm across thousands of client institutions.
At the same time, health care institutions are ill-equipped to validate AI systems on their own, due to lack of domain knowledge, computational resources, or sufficient access to proprietary models.
Thus, there is a need to replicate the gold standard or real clinical trials virtually, to evaluate model performance more easily and well in advance of deployments.

\begin{figure}[t]
    \centering
    \includegraphics[width=\linewidth]{images/overview.png}
    \caption{AI-based medical image analysis algorithms are susceptible to drops in performance when deployed on new populations. (a) The approval pipeline for medical image AI necessitates large cohort selection and costly data collection processes so as to ensure good performance across the given population. performance may still decline when deployed on new populations.\cite{oakden2020hidden}
      (b) We propose a novel framework for medical image AI validation, where a conditional generative model provides full-body images with the same distribution of attributes, \ie. demographics or other characteristics, as the target population. This enables \emph{in silico} clinical trials much earlier in the development pipeline, ensuring high performance on desired populations before real clinical trials.
      }
    \label{fig:overview}
\end{figure}

Central to this vision is the ability to synthesize medical images based on population-level attributes.
Current approaches relying on computational human phantoms\cite{abadi2020virtual, badano2023stochastic} are fundamentally limited in the types of variation they can express. These phantoms use boundary representations to capture variations in anatomical structures, making them well-suited for many types of virtual imaging trials\cite{segars2008realistic} but ultimately limited to shape-based variations.
It is not clear how to extend these phantoms to support population-level attributes.
Generative models using machine learning, on the other hand, are able to consume and produce practically unlimited data.\cite{ibrahim2024generative}
They have been used to improve model performance by augmenting training data with synthetic images.\cite{khader2023denoising}
In principle, generative models are highly flexible in the kind of conditioning parameters they can support, but there are significant challenges in developing a model with the capabilities necessary for conducting VCTs in precision medicine.

% ,\cite{bahrami2021comparison} anatomy,\cite{txurio2023diffusion, pan20232d} or conditioning parameters.\cite{guo2024maisi}

We demonstrate for the first time that a generative model is capable of realistically synthesizing full-body, 3D medical images. Furthermore, we show these images are of sufficient quality to replicate hidden biases in downstream models that are otherwise present in real data.
Rather than modeling the conditional distribution $p(\mathbf{X}|\mathbf{Y})$ of the 3D CT image $\mathbf{X}$ given an organ segmentation $\mathbf{Y}$,\cite{guo2024maisi} our approach ensures anatomical consistency by learning the joint distribution $p(\mathbf{X}, \mathbf{Y})$. This results in images that are visually and anatomically accurate, even compared to methods that focus on smaller regions and have access to detailed segmentations.
% as evaluated by quantitative similary measures and third-party segmentation algorithms.\cite{wasserthal2023totalsegmentator}
Our generative model achieves this consistency in a memory-efficient manner by operating on latent representations of the full-body image and segmentation.\cite{esser2020taming, rombach2022high}
We further model the distribution $p(\mathbf{X},\mathbf{Y}|\mathbf{a})$ conditioned on patient attributes $\mathbf{a}$, namely patient sex, age, height, and weight, to allow for sampling of virtual patient cohorts using a distribution of attributes.
These can be obtained from real health records of the deployment population or from statistical models, \eg, based on census data (see Fig.~\ref{fig:overview}b).
In controlled experiments, we show how virtual clinical trials (VCTs) using these cohorts can replicate model performance for multiple tasks in precision medicine, compared to performance on real data from the corresponding attribute distribution which would be otherwise inaccessible.
Whereas downstream models reliably estimate quantities like overall body fat and muscle mass percentage for data from the same attribute distribution as the training and validation data, those same models fail to generalize across a population shift because of hidden biases in the training data.
Those same degradations are observed on images generated from the synthetic patient cohort with the same attributes, which would enable a vendor to anticipate deployment time changes in performance and prevent adverse effects on patient care.




% Old paragraph 2, 2024-12-02
Degradations between controlled trial and deployment scenarios may result from differences, including the demographic makeup of the population, medical history, image collection protocols, and device specifications.
While clinical trials may detect such degradations through large scale cohort selection and data collection (see Fig.~\ref{fig:overview}a), they are not a practical solution for post-market deployment-time validation after significant investment in controlled trials \cite{rajpurkar2022ai, menard2024artificial}.
At the same time, health care institutions are ill-equipped to validate AI systems on their own, due to lack of domain knowledge, computational resources, or sufficient access to proprietary models.
Virtual Clinical Trials (VCTs), on the other hand, use \emph{in silico} simulation of medical images to replicate real world data collection processes.
\todo{Just want to say there are approaches for conducting virtual clinical trials, but there is no clear path to scaling these models.}
\todo{Current approaches relying on computational human phantoms\cite{abadi2020virtual, badano2023stochastic} are capable of representing differencing in imaging protocols and device specifications, but they are fundamentally limited in the types of variation they can express. Computational human phantoms use boundary representations to capture variations in anatomical structures, making them well-suited for many types of virtual imaging trials\cite{segars2008realistic} but ultimately limited to shape-based variations.
It is not clear how to extend these phantoms to examine a major likely cause for performance degradations between controlled trials and deployment populations, namely population-level patient attributes.}
Generative models using machine learning, on the other hand, are able to consume and produce practically unlimited data.\cite{ibrahim2024generative}
They have been used to improve model performance by augmenting training data with synthetic images.\cite{khader2023denoising}
In principle, generative models are highly flexible in the kind of conditioning parameters they can support, but there are significant challenges in developing a model with the capabilities necessary for conducting VCTs in precision medicine.
