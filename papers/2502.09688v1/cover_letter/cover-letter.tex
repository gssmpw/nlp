
\documentclass[10pt]{JHUletter}
\usepackage{fontspec}
%\setmainfont{Tahoma}
\usepackage[T1]{fontenc}
\fontfamily{georgia}\selectfont
\usepackage{tikz}
\usepackage{xcolor}
	% Set up custom JHU color
	\definecolor{jhublue}{rgb}{0,0.17647,0.44705}
\usepackage{lipsum}
\usepackage{fancyhdr}
\usepackage{lastpage}
\usepackage{hyperref}
%
% This section is just a bunch of busywork so that the second and following pages read ``Page X of Y''
\pagestyle{fancy}
\fancyhf{}
\renewcommand{\headrulewidth}{0pt}
\renewcommand{\footrulewidth}{0pt}
\rhead{Page \thepage \hspace{1pt} of \pageref{LastPage}}
%
%

\newcommand{\todo}[1]{{\color{red} [#1]}}

%
% The material below is a whole big dang thing whose purpose is just to set up a fixed coordinate system for \tikz so that you can put the Department or School address in the upper right-hand side without it moving all around every time you change something in the page.  I think it works.
% Defining a new coordinate system for the page:
%
% --------------------------
% |(-1,1)    (0,1)    (1,1)|
% |                        |
% |(-1,0)    (0,0)    (1,0)|
% |                        |
% |(-1,-1)   (0,-1)  (1,-1)|
% --------------------------
\makeatletter
\def\parsecomma#1,#2\endparsecomma{\def\page@x{#1}\def\page@y{#2}}
\tikzdeclarecoordinatesystem{page}{
    \parsecomma#1\endparsecomma
    \pgfpointanchor{current page}{north east}
    % Save the upper right corner
    \pgf@xc=\pgf@x%
    \pgf@yc=\pgf@y%
    % save the lower left corner
    \pgfpointanchor{current page}{south west}
    \pgf@xb=\pgf@x%
    \pgf@yb=\pgf@y%
    % Transform to the correct placement
    \pgfmathparse{(\pgf@xc-\pgf@xb)/2.*\page@x+(\pgf@xc+\pgf@xb)/2.}
    \expandafter\pgf@x\expandafter=\pgfmathresult pt
    \pgfmathparse{(\pgf@yc-\pgf@yb)/2.*\page@y+(\pgf@yc+\pgf@yb)/2.}
    \expandafter\pgf@y\expandafter=\pgfmathresult pt
}
\makeatother
%
%
%%%%%%%%%%% Put Personal Information Here %%%%%%%%%%%
%
\def\name{Benjamin D.~Killeen}
%
% List your degree(s), licences, etc. here if you like.
%\def\What{, Your degrees, etc.}
%
% Set the name of your Department or School here
% I honestly don't know why the negative spacing is necessary to get the alignment of the first line correct.  This must be a ``\tikz'' thing.
%%%%%%%%%%%%%%%%%%  School or Department %%%%%%%%%%%%%%%
\def\Where{\hspace{-1.2mm}\textbf{\color{jhublue}
Whiting School of Engineering\\
Department of Computer Science
}}
%%%%%%%%%%%%  Additional Contact Information %%%%%%%%%%%
%
% Set your preferred primary contact address here.
\def\Address{3400 North Charles Street\\
Malone Hall 160}
%
\def\CityZip{Baltimore, MD~21218, USA}
%
% Set your OSU e-mail here
\def\Email{\makebox[1.0cm][l]{{\color{jhublue}Mail}:} killeen@jhu.edu}
%
% Set your preferred OSU contact number here
\def\TEL{\makebox[1.0cm][l]{{\color{jhublue}Cell}:} +1 (314) 651-6809}
%
% Set your department or personal website here
\def\URL{\makebox[1.0cm][l]{{\color{jhublue}Web}:} benjamindkilleen.com}
%
%%%%%%%%%%%%%%%%  Footer information  %%%%%%%%%%%%%%%%%%
%
%  The next line is for your college, used as a footer.  If you prefer not to have this, just comment out these lines in favor of the line labeled "[[Alternate]]" below
%\def\school{\small{
%  Oregon State University $\cdot$
%     ~College of Engineering $\cdot$
%     ~101 Covell Hall $\cdot$
%     ~Corvallis, OR 97331-2409} }
% \def\school{~}  % [[Alternate]]
%
%%%%%%%%%%%%%%%%%%%%%  Signature line  %%%%%%%%%%%%%%%%%%%%%
%
% Set your signature line here.
% One can add a signature image in a PDF file using the following code; this requires a file called "signature_block.pdf" to be installed in the same folder as the .tex file.  The vertical spacing (\vspace) and the scaling will have to be adjusted to get things to look correct for your particular signature image. Alternatively, comment out the following line in favor of the one labeled "[[Alternate]]" if you want to sign a paper copy of the letter.
%
\signature{
\vspace{-10mm}\includegraphics[scale=0.1]{signature}\\\vspace{-7mm}
\name}
%\signature{\name}  % [[Alternate]]

% This block sets up the address on the right-hand side of the header.
%
% The following lines just compile the information you set up into the LaTex letter variable "address" for later use.
%
%The following command "clears out" the default address so that it can be better set using \tikz
\address{}

\def\newaddress{
\Where\\
\Address\\
\CityZip\\
\vspace{10pt}
{\color{jhublue} \textbf{Benjamin D.~Killeen}}\\
Ph.D. Candidate\\
\TEL\\
\Email\\
\URL
}
%
%%%%%%%%%%%  DATE  %%%%%%%%%%%%%%%%%%%%%%%%%
% If you want a date different from the current date, comment out the next line in favor of the line labeled "[[Alternate]]".  The ``\vspace{10mm}'' just moves the date down a tiny bit so it doesn't interfere with the header.  This can be adjusted to your preference.
%
% \date{\vspace{10mm} \today}
%\date{\vspace{10mm} 20 September 2020}  %[[Alternate]]
\date{\vspace{5pt}~}
%
%%%%%%%%%%% Set the subject here if there is one  %%%%
% \subject{} % optional subject line

% From https://www.nature.com/srep/contact#:~:text=Contact%20us,.collections%40nature.com.
\def\toadd{
  % \today
  Editorial Office \hfill \today\\
  % The Campus\\
  4 Crinan Street\\
  London, N1 9XW, United Kingdom
}

\begin{document}
%
%
%%%%%%%%  The "To" address goes here.
%

\begin{letter}{\toadd}
% \begin{letter}{\vspace{-20mm}\toadd}

% This line sets up the return address to the right-side of the logo.  The location is set with absolute node addresses using ``\tikz''.  It can still be a bit fussy, and you may need to alter this a little to get things to look right.  The bit that changes the position are the numbers in parentheses ``at (14.2,2.7)''
%
\begin{tikzpicture}[remember picture,overlay,,every node/.style={anchor=center}]
\node[text width=7cm] at (page cs:0.5,0.73){\small \newaddress};
\end{tikzpicture}


\opening{\vspace{-4mm}Dear Magdalena Skipper,}

% In addition, a cover letter needs to be written with the
% following:
% \begin{enumerate}
%  \item A 100 word or less summary indicating on scientific grounds
% why the paper should be considered for a wide-ranging journal like
% \textsl{Nature} instead of a more narrowly focused journal.
%  \item A 100 word or less summary aimed at a non-scientific audience,
% written at the level of a national newspaper.  It may be used for
% \textsl{Nature}'s press release or other general publicity.
%  \item The cover letter should state clearly what is included as the
% submission, including number of figures, supporting manuscripts
% and any Supplementary Information (specifying number of items and
% format).
%  \item The cover letter should also state the number of
% words of text in the paper; the number of figures and parts of
% figures (for example, 4 figures, comprising 16 separate panels in
% total); a rough estimate of the desired final size of figures in
% terms of number of pages; and a full current postal address,
% telephone and fax numbers, and current e-mail address.
% \end{enumerate}

% See \textsl{Nature}'s website
% (\texttt{http://www.nature.com/nature/submit/gta/index.html}) for
% complete submission guidelines.

I am pleased to submit our research article, ``Towards Virtual Clinical Trials of Radiology AI Through Conditional Generative Modeling'' for consideration in \emph{Nature}. It is undeniable that artificial intelligence and machine learning (AI/ML) is on the verge of transforming medicine. However, substantial evidence points toward the brittleness of AI/ML models when deployed outside controlled environments, a limitation that can lead to significant performance degradation, and in the healthcare context, adverse patient outcomes. This work addresses a pressing challenge in the adoption of AI/ML in healthcare: the ability to automatically anticipate performance degradations on unseen patient populations.

Our study introduces a first generative AI model for full-body CT image synthesis, designed to support virtual clinical trials (VCTs) in radiology-based precision medicine. By conditioning on patient attributes such as sex, age, height, and weight, this model for the first time enables the creation of realistic, anatomically accurate synthetic patient cohorts. These synthetic cohorts serve as a scalable solution for assessing model robustness and quantifying bias with respect to certain attributes, effectively replicating real-world degradation patterns observed in clinical deployments. Key contributions of our manuscript include:
\begin{enumerate}
  \item A conditional generative AI model that synthesizes volumetric full-body CT images with high anatomical fidelity, incorporating latent diffusion techniques for memory-efficient representation.
  \item Controlled experiments demonstrating the model’s ability to detect and quantify AI/ML performance degradation on a par with real data tests, highlighting its utility in safeguarding patient care and ensuring equitable outcomes.
\end{enumerate}

We believe this work aligns with \emph{Nature}'s commitment to publish transformative research with broad impact. In recent years, generative AI has made enormous strides, with increasingly capable models for image generation, video editing, 3D modeling, and data augmentation. Our work adds a new category to this list: the ability to derive actionable information about the real-world performance of other AI models for precision medicine.

We confirm that this manuscript has not been published elsewhere and is not under consideration by another journal. All authors have approved the manuscript and agree with its submission to \emph{Nature}.
Please do not hesitate to contact me with any questions or comments. I look forward to hearing from you.

\closing{Sincerely,}

\vspace{10mm}

\end{letter}

\end{document}

%%% Local Variables:
%%% coding: utf-8
%%% mode: latex
%%% TeX-engine: xetex
%%% End:
