\section{Related Work}
\label{sec:related_works}

\paragraph{Fairness \& Bias in LLM-Based Recommendations.}
LLMs increasingly serve as \emph{zero-shot recommenders}~\citep{hou2024large,zhang2023chatgpt}, generating item suggestions without explicit fine-tuning. Despite their versatility, large-scale pre-training can encode biases that exacerbate demographic disparities \citep{bender2021dangers}. For example, small changes in sensitive attributes (for example, sex or age) can produce disproportionately different results \citep{zhang2023chatgpt}. Recent efforts employ \emph{post hoc} techniques such as semantic checks in the embedding space \citep{lucy2021gender} and prompt-level interventions~\citep{che2023federated}, yet deciding a fair threshold for “excessive” disparity remains challenging. Conformal or otherwise \emph{statistical} methods thus offer a data-driven way to calibrate acceptable variations, providing principled fairness guarantees beyond subjective judgments.

\vspace{-1em}
\paragraph{Instruction Tuning \& RLHF.}
Instruction tuning and RLHF~\citep{ouyang2022training,bai2022training} aim to mitigate harmful behaviors by incorporating human-generated feedback signals (rewards) into training. Although these methods can reduce overt toxicity or explicit discrimination, they may not fully address subtler biases manifested in personalized recommendations \citep{sharma2023framework}. Additionally, many industrial deployments cannot easily retrain large models, making parameter-free or black-box mitigation techniques essential.

\vspace{-1em}
\paragraph{Fairness in Recommendation.}
Earlier work in fairness-aware recommendation \citep{greenwood2024user} focuses on balancing exposure and relevance across demographic groups. More recent approaches adopt foundation-model architectures—e.g., UP5~\citep{hua2023up5}—that incorporate fairness directly into large-scale ranking systems. Nonetheless, empirical evaluations have found that LLM-based recommendation can inadvertently amplify group-level biases \citep{hou2024large,zhang2023chatgpt}. This underscores the need for robust monitoring and adaptive calibration beyond a single pre-trained checkpoint.

\vspace{-1em}
\paragraph{Embedding-Based Post Hoc Mitigation.}
Post hoc bias detection via embeddings is attractive in black-box LLM deployments because it does not require modifying model weights \citep{borkan2019nuanced,lucy2021gender}. By examining how generated outputs diverge when protected attributes change, one can identify concerning patterns and then apply \emph{prompt-level} corrections \citep{zhang2023chatgpt}. However, standard practice often lacks a principled mechanism for deciding when to label a particular semantic difference as unacceptable.

\vspace{-1em}
\paragraph{Conformal Prediction for LLM Fairness.}
Conformal prediction~\citep{shafer2008tutorial} provides statistical coverage guarantees, using a calibration set to define non-conformity scores that bound future predictions. In fairness contexts, it can systematically control the violation rate by explicitly incorporating sensitive attributes in the scoring scheme \citep{dwork2012fairness}. While most conformal methods target classification tasks or simple regression, extending them to LLM-based recommendations involves defining semantic non-conformity measures that capture large textual or item-level disparities across protected groups. By coupling these measures with prompt updates (rather than retraining model parameters), we achieve an iterative, \emph{black-box-friendly} approach to fairness calibration. Our framework, \textbf{FACTER}, operationalizes this idea by adaptively lowering a threshold whenever a recommendation violates local fairness constraints. Section~\ref{sec:method} details the methodology and threshold adaptation, while our experiments (\S\ref{sec:experiments}) demonstrate significant bias reduction with minimal accuracy trade-offs.