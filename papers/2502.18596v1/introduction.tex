High-performance computing (HPC) environments are increasingly heterogeneous, comprising various types of computing resources such as CPUs, GPUs, and specialized accelerators spread across multiple facilities. Efficiently managing and utilizing these diverse resources is crucial for maximizing computational throughput and minimizing operational costs. Traditional resource management frameworks often struggle to adapt to the complexities and dynamic nature of modern HPC environments, particularly when dealing with user privilege restrictions and varying resource availability.

The JLab Integrated Research Infrastructure Across Facilities (JIRIAF) framework addresses these challenges by providing a scalable and flexible resource management solution designed to seamlessly integrate and manage heterogeneous HPC resources. Central to JIRIAF is the JIRIAF Resource Manager (JRM), which plays a pivotal role in this framework. The JRM leverages the Kubernetes framework with Virtual Kubelet to enable resource management in environments where traditional kubelet installation is impractical due to user privilege restrictions. By operating in userspace, the Virtual Kubelet implementation allows JIRIAF to execute user applications as containers across different computing sites, providing unified control and monitoring capabilities.

This paper presents the design and implementation of the JIRIAF framework, with a particular emphasis on the JIRIAF Resource Manager (JRM). We demonstrate the framework's effectiveness through a proof-of-concept deployment on the Perlmutter system at the National Energy Research Scientific Computing Center (NERSC). The deployment involves data-stream processing pipelines for the CLAS12 experiment, showcasing JIRIAF's ability to manage large-scale HPC applications efficiently. Additionally, we discuss the integration of a digital twin model for a simulated queue system related to a streaming system. The digital twin, implemented using a Dynamic Bayesian Network (DBN), enhances real-time monitoring and control, offering insights into system performance and optimization strategies.
