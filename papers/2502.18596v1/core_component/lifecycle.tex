\subsection{Lifecycle of Containers and Pods}
\label{lifecycle}

The lifecycle of containers and pods defined by VK involves various states and transitions, crucial for monitoring and managing workload execution effectively. This section describes the container states and the methods used to create and monitor them.

\subsubsection{Description of Container States}

Each field in the container state tables provides specific information about the container's status and transitions, as described in Table \ref{table:field_descriptions}. Container states represent the stages a container goes through during its lifecycle. These states are essential for understanding the behavior and status of containers as they are created, run, and terminated. Notice that these methods can be found in \texttt{internal/provider/mock/mock.go} in the source code of GitHub \cite{virtual-kubelet-cmd}.

\paragraph{Container State Transitions in \texttt{CreatePod} Method}

The \texttt{CreatePod} method transitions the container through several states. Table \ref{table:uid_index_createpod} lists the UID Index for these states.

\begin{table}[h!]
\centering
\caption{UID Index for \texttt{CreatePod} method}
\label{table:uid_index_createpod}
\begin{tabular}{|l|l|}
\hline
\textbf{UID} & \textbf{UID Index} \\
\hline
create-cont-readDefaultVolDirError & 0 \\
\hline
create-cont-copyFileError & 1 \\
\hline
create-cont-cmdStartError & 2 \\
\hline
create-cont-getPgidError & 3 \\
\hline
create-cont-createStdoutFileError & 4 \\
\hline
create-cont-createStderrFileError & 5 \\
\hline
create-cont-cmdWaitError & 6 \\
\hline
create-cont-writePgidError & 7 \\
\hline
create-cont-containerStarted & 8 \\
\hline
\end{tabular}
\end{table}

\paragraph{Container State Transitions in \texttt{GetPods} Method}

The \texttt{GetPods} method periodically monitors the container's status. Table \ref{table:uid_index_getpods} lists the UID Index for the \texttt{GetPods} method.

\begin{table}[h!]
\centering
\caption{UID Index for \texttt{GetPods} method}
\label{table:uid_index_getpods}
\begin{tabular}{|l|l|}
\hline
\textbf{UID} & \textbf{UID Index} \\
\hline
get-cont-create & 0 \\
\hline
get-cont-getPidsError & 1 \\
\hline
get-cont-getStderrFileInfoError & 2 \\
\hline
get-cont-stderrNotEmpty & 3 \\
\hline
get-cont-completed & 4 \\
\hline
get-cont-running & 5 \\
\hline
\end{tabular}
\end{table}

\subsubsection{The Flowchart for Creating and Monitoring Lifecycle of the Containers in a Pod}

The flowchart in Figure \ref{fig:flowchart} illustrates the process of creating and monitoring containers and pods in VK.

\begin{figure}[htbp]
\centering
\includegraphics[width=\linewidth]{core_component/fig/flowchart.pdf}
\caption{Flowchart illustrating the process of creating and monitoring the lifecycle of containers within a pod created by VK. The flowchart includes the initial creation of container states, updating pod status based on container states, and handling errors during the lifecycle. Key blocks indicate looping over containers, creating container states, updating pod status, and redirecting flows based on conditions.}
\label{fig:flowchart}
\end{figure}

\subsubsection{Implementation of Container Lifecycle Management}

To ensure proper lifecycle management, the following Go code initializes and updates pod states based on various transition conditions:

\begin{verbatim}
pod.Status.Conditions = []v1.PodCondition{
  {
    Type:               v1.PodScheduled,
    Status:             v1.ConditionTrue,
    LastTransitionTime: startTime,
  },
  {
    Type:               v1.PodReady,
    Status:             podReady,
    LastTransitionTime: startTime,
  },
  {
    Type:               v1.PodInitialized,
    Status:             v1.ConditionTrue,
    LastTransitionTime: startTime,
  },
}
\end{verbatim}

Similarly, pod readiness is assessed in the \texttt{GetPods} function:

\begin{verbatim}
Conditions: []v1.PodCondition{
  {
    Type:   v1.PodScheduled,
    Status: v1.ConditionTrue,
    LastTransitionTime: *prevPodStartTime,
  },
  {
    Type:   v1.PodInitialized,
    Status: v1.ConditionTrue,
    LastTransitionTime: *prevPodStartTime,
  },
  {
    Type:   v1.PodReady,
    Status: podReady,
    LastTransitionTime: prevContainerStartTime[firstContainerName],
  },
}
\end{verbatim}

\paragraph{Walltime Adjustment}
The system sets the walltime for the container lifecycle as follows:

\begin{verbatim}
sleep $JIRIAF_WALLTIME
echo "Walltime $JIRIAF_WALLTIME has ended. Terminating the processes."
pkill -f "./start.sh"
\end{verbatim}
