\subsection{Deploying and Managing Pods on VK Nodes}
\label{deploying-pods}

Pods and their containers can be deployed on VK nodes. Table \ref{table:vk_vs_kubelet} provides a comparative analysis of VK nodes and standard kubelets:

\begin{table}[h!]
\centering
\caption{Comparison of VK Nodes and Regular Kubelets}
\begin{tabular}{|l|l|l|}
\hline
\textbf{Property} & \textbf{Virtual-Kubelet-CMD} & \textbf{Regular Kubelet} \\
\hline
Container Execution & Executes as Linux processes & Runs as a Docker container \\
\hline
Image Definition & Defined as a BASH script & Defined as a Docker container image \\
\hline
\end{tabular}
\label{table:vk_vs_kubelet}
\end{table}

\subsubsection{Managing Container Processes Using Process Group ID}

The \texttt{pgid} file records the process group ID (PGID) of the processes initiated by the script. Each container has a unique \texttt{pgid} file, located at \texttt{\$HOME/\$podName/containers/\$containerName/pgid}. This file is essential for controlling and managing the process group, ensuring that all processes within the group are appropriately handled.

\subsubsection{Procedure to Deploy a Pod Executing a BASH Script}

To deploy a pod that runs a script, follow these steps:

\begin{itemize}
    \item Store the script \texttt{stress.sh} in a \texttt{ConfigMap}.
    \item Mount the script into the container using \texttt{volumeMounts}. The \texttt{stress.sh} script will be located at \texttt{\$HOME/\$podName/containers/direct-stress/stress/} when the pod starts running.
    \item Specify the script execution using the \texttt{command} and \texttt{args} fields.
\end{itemize}

An example of a pod running a shell script is provided below:

\begin{verbatim}
kind: ConfigMap
apiVersion: v1
metadata:
  name: direct-stress
data:
  stress.sh: |
    #!/bin/bash
    stress --timeout $1 --cpu $2
---
apiVersion: v1
kind: Pod
metadata:
  name: direct-stress
spec:
  containers:
    - name: direct-stress
      image: direct-stress
      command: ["bash", "/stress/stress.sh"]
      args: ["300", "2"]
      volumeMounts:
        - name: direct-stress
          mountPath: /stress
  volumes:
    - name: direct-stress
      configMap:
        name: direct-stress
  nodeSelector:
    kubernetes.io/role: agent
  tolerations:
    - key: "virtual-kubelet.io/provider"
      value: "mock"
      effect: "NoSchedule"
\end{verbatim}

Important parameters in the YAML file are outlined in Table \ref{table:script_storage_execution}.

\begin{table}[h!]
\centering
\caption{Features for Script Management and Execution in Pods}
\begin{tabular}{|l|p{10cm}|}
\hline
\textbf{Feature} & \textbf{Description} \\
\hline
\texttt{ConfigMap} & Volume type for storing the script \\
\hline
\texttt{volumes} & Manages the use of \texttt{ConfigMap} \\
\hline
\texttt{volumeMounts} & Relocates the script to the specified path as \texttt{\$HOME/\$podName/containers/\$containerName/\$mountPath/} \\
\hline
\texttt{command} and \texttt{args} & Specifies the script to be executed and its arguments \\
\hline
\texttt{env} & Passes environment variables to the script \\
\hline
\texttt{image} & Name of the container image, same as the volume name for the script \\
\hline
\end{tabular}
\label{table:script_storage_execution}
\end{table}

\subsubsection{Setting Affinity for Pods on VK Nodes}

Pod affinity for VK nodes is determined by three labels: \texttt{jiriaf.nodetype}, \texttt{jiriaf.site}, and \texttt{jiriaf.alivetime}. These labels correspond to the variables \texttt{JIRIAF\_NODETYPE}, \texttt{JIRIAF\_SITE}, and \texttt{JIRIAF\_WALLTIME} in the \texttt{start.sh} script.

\begin{itemize}
    \item If \texttt{JIRIAF\_WALLTIME} is set to \texttt{0}, the \texttt{jiriaf.alivetime} label is not defined, and thus, affinity is not applied.
    \item VK status changes from \texttt{Ready} to \texttt{NotReady} once \texttt{jiriaf.alivetime} reaches zero. However, the VK process is not terminated.
    \item To add additional labels to VK nodes, modify the \texttt{ConfigureNode} function in \texttt{internal/provider/mock/mock.go} in the source code of GitHub \cite{virtual-kubelet-cmd}.
\end{itemize}

An example of setting affinity is provided below:

\begin{verbatim}
affinity:
  nodeAffinity:
    requiredDuringSchedulingIgnoredDuringExecution:
      nodeSelectorTerms:
      - matchExpressions:
        - key: jiriaf.nodetype
          operator: In
          values:
          - "cpu"
        - key: jiriaf.site
          operator: In
          values:
          - "nersc"
        - key: jiriaf.alivetime
          operator: Gt
          values:
          - "10"
\end{verbatim}
