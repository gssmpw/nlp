\subsection{Starting Virtual-Kubelet-Cmd}
Virtual-Kubelet-Cmd (VK) is a critical component that must be properly initialized to function within a Kubernetes environment. This section delineates the procedures for initializing VK using either the VK binary or a Docker image. The source code is accessible in the GitHub repository \cite{virtual-kubelet-cmd}.

\subsubsection{Using VK Binary to Start Virtual-Kubelet-Cmd}

The following script initializes Virtual-Kubelet-Cmd (VK) with the environment variables listed in Table \ref{table:environment_variables}:

\begin{verbatim}
#!/bin/bash
export MAIN="/workspaces/virtual-kubelet-cmd"
export VK_PATH="$MAIN/test-run/apiserver"
export VK_BIN="$MAIN/bin"
export APISERVER_CERT_LOCATION="$VK_PATH/client.crt"
export APISERVER_KEY_LOCATION="$VK_PATH/client.key"
export KUBECONFIG="$HOME/.kube/config"
export NODENAME="vk"
export VKUBELET_POD_IP="172.17.0.1"
export KUBELET_PORT="10255"
export JIRIAF_WALLTIME="60"
export JIRIAF_NODETYPE="cpu"
export JIRIAF_SITE="Local"

"$VK_BIN/virtual-kubelet" --nodename $NODENAME --provider mock --klog.v 3 > ./$NODENAME.log 2>&1
\end{verbatim}

\begin{table}[h!]
\centering
\caption{Environment Variables used to start VK}
\begin{tabular}{|l|p{10cm}|}
\hline
\textbf{Environment Variable} & \textbf{Description} \\
\hline
\texttt{MAIN} & Main workspace directory \\
\hline
\texttt{VK\_PATH} & Path to the directory containing the apiserver files \\
\hline
\texttt{VK\_BIN} & Path to the binary files \\
\hline
\texttt{APISERVER\_CERT\_LOCATION} & Location of the apiserver certificate \\
\hline
\texttt{APISERVER\_KEY\_LOCATION} & Location of the apiserver key \\
\hline
\texttt{KUBECONFIG} & Kubernetes configuration file location (default: \texttt{\$HOME/.kube/config}) \\
\hline
\texttt{NODENAME} & Name of the node in the Kubernetes cluster \\
\hline
\texttt{VKUBELET\_POD\_IP} & IP address of Pod IP \\
\hline
\texttt{KUBELET\_PORT} & Kubelet HTTP server port (default: 10250) \\
\hline
\texttt{JIRIAF\_WALLTIME} & Node runtime limit in seconds (0 for no limit) \\
\hline
\texttt{JIRIAF\_NODETYPE} & Node type for labeling purposes \\
\hline
\texttt{JIRIAF\_SITE} & Site for labeling purposes \\
\hline
\end{tabular}
\label{table:environment_variables}
\end{table}

\subsubsection{Using Docker Image for Virtual-Kubelet-Cmd}

The Docker image for \texttt{virtual-kubelet-cmd} is available on Docker Hub \cite{docker-vk-cmd}. This image includes the \texttt{virtual-kubelet-cmd} binary and necessary files, organized into a \texttt{vk-cmd} directory within the container. The containerization of VK is inspired by the KinD project \cite{KinD}, and the source code can be found on GitHub \cite{vk-cmd-build-img}.

The following script demonstrates how to run VK using the Docker image:

\begin{verbatim}
#!/bin/bash
export NODENAME="vk"
export KUBECONFIG="$HOME/.kube/config"
export VKUBELET_POD_IP="123.123.123.123"
export KUBELET_PORT="10260"

export JIRIAF_WALLTIME="0"
export JIRIAF_NODETYPE="cpu"
export JIRIAF_SITE="mylin"

export VK_CMD_IMAGE="jlabtsai/vk-cmd:main"
docker pull $VK_CMD_IMAGE

container_id=$(docker run -itd --rm --name vk-cmd $VK_CMD_IMAGE)
docker cp ${container_id}:/vk-cmd `pwd` && docker stop ${container_id}
cd `pwd`/vk-cmd

./start.sh $KUBECONFIG $NODENAME $VKUBELET_POD_IP $KUBELET_PORT $JIRIAF_WALLTIME \
    $JIRIAF_NODETYPE $JIRIAF_SITE
\end{verbatim}

When running VK on a remote machine, it is necessary to establish SSH tunnels to the control plane. Two tunnels are required: one for the API server and one for the metrics server. The following commands establish these tunnels on the worker node:

\begin{table}[h!]
\centering
\caption{SSH Tunnel for VK to API server}
\begin{tabular}{|l|l|}
\hline
\textbf{Connection} & API server listens to VK \\
\hline
\textbf{Environment Variables} & \texttt{APISERVER\_PORT}, \texttt{CONTROL\_PLANE\_IP} \\
\hline
\textbf{SSH Command} & \texttt{ssh -NfL \$APISERVER\_PORT:localhost:\$APISERVER\_PORT \$CONTROL\_PLANE\_IP} \\
\hline
\end{tabular}
\label{table:ssh_tunnel_vk_to_api}
\end{table}

\begin{table}[h!]
\centering
\caption{SSH Tunnel for Metrics server to VK}
\begin{tabular}{|l|l|}
\hline
\textbf{Connection} & VK listens to metrics server \\
\hline
\textbf{Environment Variables} & \texttt{KUBELET\_PORT}, \texttt{CONTROL\_PLANE\_IP} \\
\hline
\textbf{SSH Command} & \texttt{ssh -NfR *:\$KUBELET\_PORT:localhost:\$KUBELET\_PORT \$CONTROL\_PLANE\_IP} \\
\hline
\end{tabular}
\label{table:ssh_tunnel_metrics_to_vk}
\end{table}

\textbf{Note:} The \texttt{*} symbol in the SSH command for the metrics server to VK is used to bind the port to all network interfaces on the control plane.
