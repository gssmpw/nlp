In the broader context of our study on optimizing computational resource allocation in high-throughput systems, we integrated a digital twin model to enhance real-time monitoring and control capabilities. The digital twin component leverages a Dynamic Bayesian Network (DBN) to simulate the behavior of a queue system, providing valuable insights into system dynamics and aiding in decision-making processes. We utilized the code from \cite{kapteyn2021probabilistic} to build our DBN model, demonstrating the practical application of their proposed framework.

\subsection{Digital Twin Model and Methodology}

The digital twin model was developed to mirror the state and behavior of a physical queue system, comprising a stream sender and receiver with a FIFO queue. The DBN framework was employed to capture dependencies among system variables, offering a probabilistic approach to real-time data assimilation and state estimation. 

Our experimental setup involved adjusting the event sending rates (\(\lambda\)) and measuring the resulting processing rates (\(\mu\)) and observed queue lengths (Obs. \(L_q\)) under different computational capacities (16 and 32 threads). The theoretical queue length (Calc. \(L_q\)) was calculated using the M/M/1 queue theory equation:

\begin{align}
L_q = \frac{\lambda^2}{\mu (\mu - \lambda)}
\label{eq:Lq_sec}
\end{align}

The data collected from these experiments were used to construct and validate the DBN model, enabling it to make accurate state predictions and recommend optimal control actions.

The DBN structure is depicted in Figure~\ref{fig:bayesian_network_sec}, illustrating the relationships between the digital twin state (\(D(t)\)), control (\(U(t)\)), and observation (\(O(t)\)) nodes.

\begin{figure}[htbp]
\centering
\includegraphics[width=3.5in]{dt/fig/graph.pdf}
\caption{Dynamic Bayesian Network representation of the digital twin. It consists of the nodes \(D(t)\), \(O(t)\), and \(U(t)\).}
\label{fig:bayesian_network_sec}
\end{figure}

\subsection{Ground Truth and Experimental Data}

To evaluate the digital twin's performance, we defined the ground truth behavior of the queue system over time using a piecewise function. This artificial ground truth allowed us to control the state changes in a predictable manner:

\begin{itemize}
    \item If timestep is less than 10, state increases by 0.4.
    \item If timestep is between 20 and 30, state decreases by 0.4.
    \item If timestep is between 40 and 50, state increases by 0.4.
    \item If timestep is between 60 and 70, state decreases by 0.4.
    \item No changes in state during unspecified periods.
\end{itemize}

The observation data \(o_t\) as Obs. \(L_q\) at timestep \(t\) were constructed by interpolating data from Tables~\ref{tab:16_threads_sec} and \ref{tab:32_threads_sec}, which highlight the experimental setup.

\begin{table}[htbp]
\caption{System Metrics for 16 Threads}
\centering
\small
\begin{tabular}{|c|c|c|c|c|c|}
\hline
State & \(\lambda\) (Hz) & \(\mu\) (Hz) & Proc. Units & Obs. \(L_q\) & Calc. \(L_q\) \\
\hline
0 & 162 & 167 & 16 & 32 & 33.74 \\
1 & 163 & 167 & 16 & 41 & 43.48 \\
2 & 164 & 167 & 16 & 58 & 60.52 \\
3 & 165 & 167 & 16 & 97 & 98.01 \\
4 & 166 & 167 & 16 & 241 & 248.00 \\
\hline
\end{tabular}
\label{tab:16_threads_sec}
\end{table}

\begin{table}[htbp]
\caption{System Metrics for 32 Threads}
\centering
\small
\begin{tabular}{|c|c|c|c|c|c|}
\hline
State & \(\lambda\) (Hz) & \(\mu\) (Hz) & Proc. Units & Obs. \(L_q\) & Calc. \(L_q\) \\
\hline
0 & 162 & 222 & 32 & 1.56 & 1.96 \\
1 & 163 & 222 & 32 & 2.5 & 2.02 \\
2 & 164 & 222 & 32 & 2.56 & 2.08 \\
3 & 165 & 222 & 32 & 3.5 & 2.14 \\
4 & 166 & 222 & 32 & 3.56 & 2.21 \\
\hline
\end{tabular}
\label{tab:32_threads_sec}
\end{table}

\subsection{Control History and Observations}

As shown in Figure~\ref{fig:queue_length_sec}, the queue system initially operates with 16 threads, which results in longer queue lengths. As pressure increases, the digital twin recommends switching to 32 threads to reduce system congestion.

\begin{figure}[htbp]
\centering
\includegraphics[width=3.5in]{dt/fig/sensor_plot.pdf}
\caption{Observed Queue Length Over Time with Control Regions (Log Scale). Black dots represent observed queue length, while blue and red regions indicate different control actions.}
\label{fig:queue_length_sec}
\end{figure}

As demonstrated in Figure~\ref{fig:control_history_sec}, the estimated control actions closely followed the predicted control recommendations, demonstrating the model's capability to adaptively manage system resources.

\begin{figure}[htbp]
\centering
\includegraphics[width=3.5in]{dt/fig/control_plot.pdf}
\caption{Control History Over Time. The red dashed line shows predicted control actions, while the blue line represents estimated control actions.}
\label{fig:control_history_sec}
\end{figure}

\subsection{Challenges and Future Directions}

While the digital twin model demonstrated robust performance, certain challenges were identified. The model exhibited delays in state tracking during periods of decreasing queue lengths, which may be attributed to indistinguishable calculated queue lengths and inaccuracies in the Conditional Probability Tables (CPTs). Addressing these challenges is crucial for enhancing predictive accuracy.

Future work will focus on optimizing the CPTs using historical data and refining the DBN model to handle more complex scenarios. Additionally, we plan to integrate the digital twin with other resource allocation frameworks to provide a holistic optimization solution.

\subsection{Conclusion}

The digital twin model, as part of our broader study, has proven to be a powerful tool for real-time monitoring and control of queue systems. By integrating the DBN framework, we have achieved significant improvements in system performance and efficiency. The insights gained from this integration will inform future developments in digital twin technology for high-throughput computational environments.
