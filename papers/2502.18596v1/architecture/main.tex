The JIRIAF architecture is meticulously designed to enable seamless integration and efficient resource management across diverse computing facilities. At the heart of this system is the JFM (JIRIAF Facility Manager), responsible for maintaining a dynamic resource pool by periodically scraping data from each computing facility to ensure an up-to-date inventory of available resources. The JCS (JIRIAF Central Service) functions as the central command, initiating pilot jobs through the JRM (JIRIAF Resource Manager). The JRM, which can operate in userspace to accommodate heterogeneous HPC setups, leases resources reported by the JFM, awaiting utilization. The JMS (JIRIAF Matching Service Algorithm) then steps in to update the available resource database, aligning resources with user requests. The JFE (JIRIAF Front End) finalizes the process by managing user requests and populating the user workflow request table. This comprehensive architecture is depicted in Figure~\ref{arch}, illustrating JIRIAF's commitment to providing a seamless and efficient computing environment.

\begin{figure}[htbp]
\centerline{\includegraphics[width=\linewidth]{architecture/fig/arch.pdf}}
\caption{JIRIAF System Architecture and Workflow: This figure visually represents the sophisticated architecture of JIRIAF, emphasizing the roles and interconnectivity of its key components — the JFM, JCS, JRM, JMS, and JFE. By illustrating the flow of data and control across these components, the diagram elucidates the dynamic and efficient resource management system designed to optimize high-performance computing across heterogeneous environments.}
\label{arch}
\end{figure}
\newpage

