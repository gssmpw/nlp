\section{Related Work}
We review the work that is closely related to ours. Readers are referred to~\citep{tam2012registration,deng2022survey} for more comprehensive studies.


\paragraph{Non-rigid point cloud registration.} In contrast to shape matching, which aims to establish correct correspondences, non-rigid point cloud registration focuses on optimizing the deformation field. Amberg \etal~\citep{amberg2007optimal} extended the rigid \emph{Iterative Closest Point} (ICP) algorithm~\citep{besl1992method} to non-rigid registration (NICP) by introducing stiffness parameters to control the deformation. \emph{Coherent Point Drift} (CPD) represented point clouds using Gaussian mixture models (GMMs) and formulated the registration problem as a probabilistic density estimation process, while~\citep{jian2010robust} directly minimized the Euclidean distance between two GMMs for registration. More recently, Hirose~\citep{hirose2020bayesian,hirose2022geodesic} reformulated CPD in a Bayesian setting, improving CPD in scalability and robustness. \cite{zhao2024correspondence} leveraged unsupervised clustering analysis to deal with non-rigid registration under large deformations. To enhance efficiency, deformation graph-based approaches~\citep{sumner2007embedded,bozic2021neural,zhao2022graphreg,qin2023deep} represent shapes using sparsely sub-sampled node graphs, where deformation is associated with each graph node and applied to nearby geometry. 


With the advent of deep learning, neural network-based approaches have also been developed for non-rigid point cloud registration. 
However, many of these methods rely on neural networks to extract features for point correspondences and subsequently employ classical registration methods like NICP for deformation optimization. For instance,   SyNoRiM~\citep{huang2022multiway} employed 3D CNN networks to learn non-orthogonal functional bases for shape matching without relying on the Laplace-Beltrami operators~\citep{ovsjanikov2012functional}. Lepard~\citep{li2022lepard} disentangled point cloud representations into feature and position spaces and developed a Transformer-based method for point-wise matching. DFR~\citep{sun2024non} conducted registration by aligning the source mesh towards a target point cloud using correspondences induced by deep functional maps~\citep{litany2017deep}. \cite{cao2023self} presented a self-supervised network for multimodal shape matching. While it shows promising performance, it does not address the deformations of occluded parts. \cite{sundararaman2022implicit} utilized an auto-decoder structure to implicitly align two volumes, which requires surface normals for training. Additionally, it employs a bi-directional Chamfer Distance for inferences, a metric that may be susceptible to occlusion. Unlike these approaches that primarily focus on shape matching and heavily depend on data annotations, our method is unsupervised, enabling us to achieve faithful registration that is more generalizable to unknown categories. 






\paragraph{Neural deformation representation.}  The representation of deformation fields is a core aspect in non-rigid point cloud registration. Traditional methods often rely on manually defined deformation functions, such as the thin-plate spline~\citep{bookstein1989principal} and the radial basis function~\citep{yuille1989mathematical}. However, the emergence of implicit neural representations has introduced a new paradigm to describe deformations. Specifically, the coordinate-based \emph{Multi-Layer Perception} (MLP) architecture utilizes an MLP to map the input coordinates to deformation fields, avoiding the need for explicit deformation definitions. For instance, NSFP~\citep{li2021neural} developed a coordinate-based MLP that implicitly regularized non-rigid deformations in scene flow estimation. This method directly minimizes the Chamfer distance at runtime without relying on extensive labeled data to capture prior statistics. Building upon NSFP, NDP~\citep{li2022non} extended the approach to a hierarchical motion decomposition using a pyramid structure. This enables controlling of the coarse-to-fine motion (from rigid to non-rigid) across low to high-frequency signals. More recently, DPF~\citep{10377509} introduced a method for modeling non-rigid surfaces also based on the Chamfer distance, where the deformation field was further regularized by the well-established as-isometric-as-possible (AIAP)~\citep{kilian2007geometric} constraint. %also characterized by an implicit neural representation. 
We also utilize a coordinate-based MLP to represent the continuous deformation field, which allows us to model and parameterize the deformations in an unsupervised and flexible manner. However, unlike previous approaches that often face challenges in occlusion scenarios, we thoroughly analyze and address these issues, leading to higher-quality non-rigid point cloud registration results.