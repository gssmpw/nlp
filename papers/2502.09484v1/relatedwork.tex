\section{Related Work}
\label{Related work}

The intersection of AI and cybersecurity is a highly active area of research, with studies ranging
from AI's role in detecting intrusions to aiding in offensive security including ethical hacking. The rise of sophisticated language models like GPT-3, introduced by Brown et al.\ \cite{brown2020language}, has expanded research possibilities by enabling strong performance on various tasks, including of course cybersecurity as we show in this paper.  %even without further fine-tuning. 
Handa et al.\ \cite{Handa2018machine} review the application of machine learning in cybersecurity, emphasising its role in areas like zero-day malware detection and anomaly-based intrusion detection, while also addressing the challenge of adversarial attacks on these algorithms.  
%Other  studies, including that by Gupta et al.\ \cite{gupta2023chatgpt}, dualise AI's role, showing how it could be used both for cyberattacks and for cyber defence. 
Other  studies, including that by Gupta et al.\ \cite{gupta2023chatgpt}, examine the dual role of GenAI models like ChatGPT in cybersecurity and privacy, highlighting both their potential for malicious use in attacks such as social engineering and automated hacking, and their application in enhancing cyber defense  measures.

Moreover,  LLMs, a form of GenAI, are being applied across various domains, including cybersecurity. For example, they are used to fix vulnerable code~\cite{pearce2023examining} and identify the root causes of incidents in cloud environments~\cite{ahmed2023recommending}.  In addition, various LLM-based tools have been recently developed, such as Code Insight\footnote{\url{https://blog.virustotal.com/2023/04/introducing-virustotal-code-insight.html}} by VirusTotal, which analyses and explains the functionality of malware written in PowerShell. %Furthermore, tools for vulnerability scanning\footnote{\url{https://github.com/aress31/burpgpt}} and penetration testing\footnote{\url{https://github.com/GreyDGL/PentestGPT}}~\cite{deng2023pentestgpt}    have also  emerged  lately.

 

 A recent practical study by Harrison et
al.\ \cite{DBLP:conf/eurosp/HarrisonTM23} shows how advances in AI's deep learning algorithms can
be used to enhance acoustic side-channel attacks against keyboards, achieving impressive keystroke
classification accuracy via common devices like smartphones and Zoom. This development poses a
significant threat, potentially enabling the theft of sensitive information such as passwords and
PINs from devices without needing physical access to the victim's machine. A recent panel
discussion, \cite{bertino2021ai}, also highlighted the dual role of AI in enhancing cybersecurity
while addressing the rising threat of adversarial attacks that exploit AI system vulnerabilities.

Recent research has also identified new vulnerabilities in the security  mechanisms of  LLMs. Jiang et al.\ \cite{jiang2024artprompt} introduced `ArtPrompt', an innovative ASCII
art-based jailbreak attack that exploits the inability of LLMs to recognise prompts encoded in
ASCII art. This work underscores the need for further research into the robustness of AI models,
particularly as these vulnerabilities can bypass safety measures and induce undesired behaviors in
state-of-the-art LLMs such as GPT-4 and Claude.

Park et al.\ \cite{SecAI24_SystematicBugReproductionWithLargeLanguageModel} introduce a technique for automating the reproduction of 1-day vulnerabilities using LLMs. Their approach involves a three-stage prompting system, guiding LLMs through vulnerability analysis, identifying relevant input fields, and generating bug-triggering inputs for use in directed fuzzing. The method, tested on real-world programs, showed some improvements in fuzzing performance compared to traditional methods. This research demonstrates the potential of LLMs to enhance cybersecurity processes, particularly in automating complex tasks such as vulnerability reproduction.


Fujii and Yamagishi\ \cite{SecAI24_FeasibilityStudyforSupportingStaticMalwareAnalysisUsingLLM_2024} explore the use of LLMs  to support static malware analysis, demonstrating that LLMs can achieve practical accuracy. A user study was conducted to assess their utility and identify areas for future improvement.


Our earlier research has  contributed to this field by investigating the role of GenAI across various phases of ethical hacking. For example, in \cite{STM24_UnleashingAIinEthicalHacking}, we proposed a conceptual framework for integrating GenAI capabilities into ethical hacking workflows, encompassing all five standard stages of penetration testing. Subsequent studies extended this work through experimental evaluations of GenAI tools like ChatGPT in controlled environments, focusing on both Windows-based systems~\cite{TechReportUnAIInEH_HC_2024} and Linux-based virtual machines~\cite{TechReport_AI-EnhancedEthicalHackingALinux-FocusedExperiment_HC_2024}~\cite{Paper_AdvancingEthicalHackingWithAI:ALinux-BasedExperimentalStudy_HC_2024}.  More recently, we examined GenAI's application within manual exploitation and privilege escalation tasks in Linux penetration testing~\cite{TechReport_APracticalExaminationOfManualExploitationAndPrivilegeEscalationInLinuxEnvironments_HC_2024}, focusing on its potential to assist in these critical yet challenging activities. Collectively, these studies highlighted GenAI's ability to enhance efficiency, support decision-making, and streamline ethical hacking processes across key stages, from reconnaissance to reporting.

Building on this foundation, PenTest++ introduces a user-centric, AI-powered automation framework designed to streamline ethical hacking workflows while preserving human oversight. By leveraging GenAI capabilities, PenTest++ automates repetitive tasks, facilitates dynamic user interaction, and supports informed decision-making, enabling ethical hackers to adapt strategies in real-time. This approach addresses critical challenges such as maintaining control over AI-driven processes, ensuring transparency, and enhancing flexibility, making it a robust and flexible toolset for ethical hackers. 

Finally, this study advances current discussions by systematically investigating the integration of GenAI-powered automation  into each phase of the ethical hacking process, an area that has received limited attention in existing literature. Through controlled lab-based experiments, PenTest++ demonstrates how automation and GenAI can enhance ethical hacking workflows, offering practical applications in real-world cybersecurity contexts. The study focuses particularly on PenTest++'s ability to provide a structured and adaptive framework for AI-assisted ethical hacking, moving beyond traditional automation to incorporate real-time AI insights.