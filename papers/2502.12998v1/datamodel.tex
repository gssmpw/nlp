\section{Data Model \& Problem Definition}\label{sec:dm}
%We describe our data model and formalize our problem.

\subsection{Data Model}
\noindent \textbf{Input Query:} A user writes a query $q$ with an input parameter $k$, with the goal of obtaining the top-$k$ set of entities to $q$.

In our running example, $q$ = "Hotels with Unique Decor in Manhattan" and we need to find the top-$3$ hotels with respect to $q$.  

\noindent \textbf{Database.} The query comes to a database $D$ containing data of different modalities, such as, pictures, text, audio, etc. Item $i$ represents a physical entity $e$ (such as a hotel). There may be multiple items associated with an entity. Let $E$ represent the set of unique entities, such that, $|E|=n$.

\autoref{fig:motivating_scenario} contains a snapshot of our example with $5$ entities or hotels. Each entity has two associated items. For instance, the  hotel "HNY", has \{ i1, i3 \}, an audio and an image respectively. \\

\noindent \textbf{Scoring function and characteristics.} A user provides a set-based scoring function $\mathcal{F}$ that admits a query  $q$ and an integer $k$, and scores any subset $s \subseteq E$ entities, such that $|s|=k$, i.e., $\mathcal{F}(s,q) \rightarrow R$. 

A scoring function needs to satisfy two conditions: (a) it is set-based, meaning it computes the score of a set of size $k$ and need not provide the individual order of  entities in the set, (b) it is decomposable into $\ell$ constructs each of which could be sent to an oracle (in our case LLM) to obtain its predicted value. 

Using our running example, $\ell=2$ and contains two constructs: a unary construct $Rel(q,e)$, and a binary construct $Rel(e_i,e_j)$.

\noindent \textbf{Top-$k$ set.} Given a query $q$ and an integer $k$, a set of entities $E$, and a set-based scoring function $\mathcal{F}$, $s_q$ is a top-$k$ set, if $s_q \subseteq E$, $|s_q|=k$, and 
$s_q = argmax_{s \subseteq E: |s|=k} \mathcal{F}(s,q)$.

Given our running example, 
 $c_1 = \{ HNY, MLN, HYN \}$ is a candidate set of top-$3$ hotels. Determining the top-$k$ set requires more insights on the unknown values in \autoref{tab:ny_hotels_relevance} and \autoref{tab:ny_hotels_diversity}.

\noindent \textbf{Candidate.} A candidate $c$ is a set of $k$ entities with score $\mathcal{F}(c,q)$. When the score of $c$ is partially computed, it's between a range, (LB\_c, UB\_c). Score of $c$ is then treated as a uniform distribution with $m$ discrete values between $(LB\_c, UB\_c)$.

Using our example, the set of hotels $c_1=\{ HNY, MLN, HYN \}$ is a candidate, and its score is a uniform probability distribution function (PDF) within the range $(0, 2)$ with 5 discrete values as $0, 0.5, ..., 2.0$.  

\noindent \textbf{Candidate set.} At a given point in time, the process keeps track of $C=\{c_1, c_2, \ldots, c_M\}$ candidates, where $M \leq \binom{n}{k}$. 

Given our running example, $M=3$ and there are three possible candidates, \[
    c_1 = \{\text{HNY, MLN, HYN}\} \]
 \[
    c_2 = \{\text{HNY, MLN, WLD}\},c_3 = \{\text{HNY, HYN, SHN}\}
\]

\noindent \textbf{Questions.} Given a user-defined scoring function $\mathcal{F}$ with $\ell$ constructs, a question $\mathcal{Q}$ asks the predicted score over one of the $\ell$ constructs appropriate inputs.

For our example function, $\mathcal{Q}(e_i,q,Rel)$ asks for the relevance score between the query $q$ and an entity $e_i$, whereas, $\mathcal{Q}(e_i,e_j,Div)$ asks for the diversity score between entities $e_i$ and $e_j$.
 
 %   \item $\mathcal{Q}_2$, which is a question of the form $\mathcal{Q}(e_i, e_j, Div)$.
%\end{itemize}
   %where $e_i,e_j$ are the $i$-th, $j$-th entities, $q$ is the query, and $F_{\ell}$ is the $\ell$-th component of the scoring function. Depending on the nature of the scoring function, some constructs involve returning a score between a query and one of the two items, e.g., to compute relevance, which requires questions of the form $Q_1$. Some other construct may admit two entities, to compute diversity, which requires questions of the form $Q_2$.

Considering our running example, one can propose different questions about the hotels mentioned in \autoref{tab:ny_hotels_relevance}. For instance, $\mathcal{Q}(HNY,q,Rel)$ asks for the relevance between  hotel "HNY" and $q$, and $\mathcal{Q}(WLD,SHN,Div)$ refers to a question about the diversity between hotels "WLD" and "SHN".

% NOTE; figure 1 change sec column to hotel names
\noindent \textbf{Response from oracle (LLM).} Given a question \( \mathcal{Q} \), and a pre-specified range of values \([MIN,MAX]\) which the oracle is instructed to respond within, its response \( \mathcal{R} \) is defined as one of:
\begin{itemize}
    \item \( \mathcal{R}_1 = r \in [MIN,MAX] \), where \( r \) is a discrete value within the specified range \([MIN, MAX]\).
    \item \( \mathcal{R}_2 = (l,u) \), where \( MIN \leq l < u \leq MAX \) is a continuous range of values that the oracle can provide as a response.
\end{itemize}

As an example, the last column of \autoref{tab:ny_hotels_relevance} is an oracle response of type $\mathcal{R}_1$, and each cell in \autoref{tab:ny_hotels_diversity} is an oracle response of type $\mathcal{R}_1$. For the question $\mathcal{Q}(HYN,q,Rel)$, the oracle returns a discrete value $Rel(HNY,q) = 0.5$ for which the $[MIN,MAX]$ are assumed to be $0$ and $1$ respectively. At a given point in time, some responses might be unknown and are denoted by $U$ in the tables. For a question $\mathcal{Q}(WLD,SHN,Div)$, we could have $Div(WLD,SHN) = [0.3,0.7]$ as the response which is a continuous range. 

We study the former kind in depth in this work, and explain how the latter could be adapted in Section~\ref{sec:ext}.

%\textcolor{orange}{we need a small table of notations - for m, M, n, e, c, C, $Q_u$..put it next to table 1 without incurring any more space}

\subsection{Formalism}
Given a scoring function $\mathcal{F}$, we define the set of all possible questions, denoted as \( Q_U \), as the union of $\ell$ decomposable constructs instantiated with input data, as appropriate.

In our example,  \( Q_U \) contains relevance on each of the $5$ hotels in $D$, plus a total of $\binom{5}{2}$ pairwise diversity questions. At a given point in time, the response to some of these questions could be known. 

\textbf{Known information.} Let \( Q_K \) be the set of questions whose responses are known. Thus, we have:
\[
Q_K = \{ q \in Q_{ALL} \mid \text{response to } q \text{ is known} \}
\]

\textbf{Unknown information.} Conversely, we can define the unknown information as the set of questions for which the responses have not yet been obtained. This set, denoted as \( Q_{ALL} \), is given by:
\[
Q_U = Q_{ALL} \setminus Q_K
\]

Using our running example, the response to each relevance and diversity question has been mapped to a cell in \autoref{tab:ny_hotels_relevance} and \autoref{tab:ny_hotels_diversity}, respectively. Cells marked as $U$ indicate an unknown value for that specific question. Hence, based on these two tables, we have:

\[
Q_K = \left\{
\begin{array}{lll}
Rel(\text{MLN}, q) & Rel(\text{WLD}, q) & Rel(\text{HNY}, q) \\
Div(\text{HNY}, \text{MLN}) & Div(\text{MLN}, \text{HNY}) & Div(\text{SHN}, \text{WLD}) \\
Div(\text{HYN}, \text{SHN}) & &
\end{array}
\right.
\]


\[
Q_U = \left\{
\begin{array}{lll}
Rel(\text{HYN}, q) & Rel(\text{SHN}, q) & Div(\text{HYN}, \text{HNY}) \\
Div(\text{HYN}, \text{MLN}) & Div(\text{HYN}, \text{WLD}) & Div(\text{HNY}, \text{HYN}) \\
Div(\text{MLN}, \text{HYN}) & Div(\text{SHN}, \text{HYN}) &
\end{array}
\right.
\]

Given the running example with $C=\{c_1,c_2,c_3\}$, we aim to find the next best question, that is the most useful for finding the query answer. 

\begin{table}[h!]
\centering
\begin{minipage}{0.45\textwidth}
\centering
\begin{tabular}{|c|l|}
\hline
\textbf{Notation} & \textbf{Definition} \\ \hline
$m$ & \# discrete score values \\ \hline
$M$ & \# candidates \\ \hline
$n$ & \# entities \\ \hline
$e$ & Entity \\ \hline
$c$ & Candidate \\ \hline
$C$ & Candidates set \\ \hline
$Q_u$ & Set of unknown questions \\ \hline
\end{tabular}
\caption{\small Notations}
\end{minipage}%
\hspace{1cm}  % Adjust space between the tables
\begin{minipage}{0.45\textwidth}
\centering
\begin{tabular}{|c|l|c|}
\hline
\textbf{Abbreviation} & \textbf{Hotel Name} & \textbf{Relevance} \\ \hline
HNY & Hilton NY      & U \\ \hline
MLN & Marriott LN    & 1.0 \\ \hline
HYN & Hyatt NY       & 1.0 \\ \hline
SHN & Sheraton NY    & 0.0 \\ \hline
WLD & Waldorf NY     & 0.5 \\ \hline
\end{tabular}
\caption{\small Relevance scores of the hotels}
\label{tab:ny_hotels_relevance}
\end{minipage}
\end{table}






\begin{table}[h!]
\centering
\begin{tabular}{|c|c|c|c|c|c|}
\hline
       & HNY  & MLN  & HYN  & SHN  & WLD  \\ \hline
HNY    & -   & 1.0  & 0.5    & 0.5  & 0.5  \\ \hline
MLN    & 1.0  & -    & U    & U  & U  \\ \hline
HYN    & 0.5    & U    & -    & U  & 0.5    \\ \hline
SHN    & 0.5  & U  & U  & -    & U  \\ \hline
WLD    & 0.5  & U  & 0.5   & U  & -    \\ \hline
\end{tabular}
\caption{\small Diversity scores between pair of hotels}
\label{tab:ny_hotels_diversity}
\end{table}


\textbf{Problem Definition:} Given \( n \) entities, an integer $k$, an input query \( q \), a user-defined set-based scoring function \( \mathcal{F} \) with $\ell$ constructs, and already known information \( Q_K \), identify the next question to be asked to the oracle (LLM) that has the highest likelihood of identifying the true top-$k$ set.


We aim to find the next best question to ask, i.e., the one that is most useful in finding the the top-$k$ set with the highest score.
    
Potentially, the next question could be any of the cells marked as U (Unknown) in \autoref{tab:ny_hotels_relevance} or \autoref{tab:ny_hotels_diversity}. As an example, consider two possible questions to ask, and compare how useful they can be:
    
    \begin{itemize}
        \item \( Q_1 \): \(Div(MLN, HYN)\) - the diversity score between "MLN" and "HYN"
        \item \( Q_2 \): \(Div(HYN, SHN)\) - the diversity score between "HYN" and "SHN"
    \end{itemize}

Before asking these questions, by using the known values have:
\[
\begin{aligned}
    \mathcal{F}(c_1, q) = 
    & Rel(\text{HNY},q) + Rel(\text{MLN},q) + Rel(\text{HYN},q) + Div(\text{HNY, MLN}) \\
    & + Div(\text{HNY, HYN}) + Div(\text{MLN, HYN}) \\
    = & 3.5 + Rel(\text{HNY},q) + Div(\text{MLN, HYN}) \\
\end{aligned}
\]

\[
\begin{aligned}
    \mathcal{F}(c_2, q) =  3.0 + Rel(\text{HNY},q) + Div(\text{MLN, WLD}) 
\end{aligned}
\]

\[
\begin{aligned}
    \mathcal{F}(c_3, q) =  2.0 + Rel(\text{HNY},q) + Div(\text{HYN, SHN}) 
\end{aligned}
\]

Suppose we choose to ask \( Q_1 \), revealing that \(Div(MLN, HYN) = 1.0 \). Knowing this additional information, we update the overall score of \( c_1 \) as follows:
\[
\begin{aligned}
    \mathcal{F}(c_1, q) = 
     & 4.5 + Rel(\text{HNY},q) \\
\end{aligned}
\]

Since all the known/unknown values are within a pre-specified range, in this example [0,1], regardless of the values of remaining unknown variables in $\mathcal{F}(c_2, q)$, $\mathcal{F}(c_1, q)$, and $\mathcal{F}(c_1, q)$, we will have:
    \[
\begin{aligned}
    \mathcal{F}(c_1, q) \geq \mathcal{F}(c_2, q) \quad \text{AND} \quad
    \mathcal{F}(c_1, q) \geq \mathcal{F}(c_3, q)
\end{aligned}
\]

    Hence, in this case, we can conclude that $c_1$ is the query answer. By asking \(Div(MLN, HYN)\) from the LLM, we could potentially get a score different from $1.0$ as well. However, in this example, if \(Div(MLN, HYN) \geq 0.5 \), we could still come up with the same conclusion as we did. Therefore, asking \(Div(MLN, HYN)\) can likely result in finding the top-$3$ set.  

 On the other hand, suppose we choose to ask \( Q_2 \), resulting in \(Div(MLN, HYN) = 1.0 \). Knowing this additional information, we update the overall scores of \( c_3 \) as follows:  
    \[
    \begin{aligned}
        \mathcal{F}(c_3, q) = 
         & 3.0 + Rel(\text{HNY},q) \\
    \end{aligned}
    \]

In this case, the top-$3$ set is not final yet. In more detail, based on different possible values of $Div(MLN, HYN)$ and $Div(HYN, SHN)$ corresponding to $c_1$ and $c_2$ respectively, any of them could have a higher final score. Hence, asking $Q_2$ will determine the top-$k$ set. 

    Generally, at a given point in time, there might be more than one question that could likely determine the top-$k$ set. Alternatively, there might be no question that could determine the top-$k$ set at a specific time, but choosing one question over another might result in a fewer number of required questions to ask in the future to determine the top-$k$ set. 
        
    Hence, it is crucial that at each step we ask the question that is most likely going to reduce the uncertainty in finding the top-$k$ set and guide us toward identifying the query answer more efficiently.



%In Section ~\ref{sec:section2.3}, we introduce each component in our proposed framework to solve this problem.



