\section{Experimental Evaluation}\label{sec:exp}
We present the effectiveness of our framework using two key metrics: 
\textbf{M1} Quality of the solution, which assesses the framework's ability to reduce costs (i.e., number of LLM calls) compared to alternative approaches; and 
\textbf{M2} Scalability of each component, which analyzes the time trends as the candidate space grows for each component. 

\subsection{Experimental Setup}\label{sec:exp_setup}

\noindent \textbf{Experiment Settings.} 
Algorithms are implemented in Python 3.11.1, utilizing GPT-4o mini as the oracle/expert in our framework. Experiments are conducted on Wulver (NJIT's HPC cluster using 6 nodes) with a 2.45 GHz AMD EPYC 7753 processor and 512 GB RAM (per node). The code and data are publicly available.\footnote{\it {\it \href{https://github.com/sohrabnamazinia/Personalized-Top-K-Set-Queries}{https://github.com/sohrabnamazinia/Personalized-Top-K-Set-Queries}}} Results are averaged over 10 runs.

\subsubsection{Applications}\label{sec:exp_apps}

We have three use cases: hotels, movies, and Yelp businesses. In all cases, the input comprises a user's query, \( k \), and personalized definitions for each scoring function construct (relevance and diversity). Each LLM prompt requests either the relevance (or diversity) score of an entity (or pair of entities). We use six scoring functions (two per use case) with distinct relevance and diversity definitions as shown in Table \ref{tab:Scoring_functions}.  Prompts are carefully designed to include all necessary unstructured data, i.e., the user's query, relevance and diversity definitions, and unstructured data associated with the related entity (or entities). We use the LangChain framework~\footnote{\url{https://www.langchain.com/}} to ensure that the LLM returns scores in our desired normalized floating-point format. 

\noindent \textbf{1) Top-k Hotels.} 
This is a large dataset \cite{arnab_das_2024} of hotels, containing detailed descriptions of 719,218 hotels sourced from websites, travel agencies, and review platforms. These descriptions serve as the unstructured data associated with hotel entities that we use in our prompts. Two scoring functions with distinct relevance and diversity definitions are used, \(\mathcal{F}_1\) and \(\mathcal{F}_2\) defined in Table \ref{tab:Scoring_functions}.

\begin{comment}
  \begin{itemize}
  \item \(\mathcal{F}_1\): 
    \begin{itemize}
      \item Relevance definition: "Hotel rating"
      \item Diversity definition: "Physical distance between hotels"
    \end{itemize}
  \item \(\mathcal{F}_2\):
    \begin{itemize}
      \item Relevance definition: "Proximity to city center"
      \item Diversity definition: "Star rating"
    \end{itemize}
\end{itemize}  
\end{comment}


\noindent \textbf{2) Top-k Movies.} 
This is the Wikipedia Movie Plots\footnote{\it {\it \href{https://www.kaggle.com/datasets/jrobischon/wikipedia-movie-plots}{https://www.kaggle.com/datasets/jrobischon/wikipedia-movie-plots}}}, which includes 33,869 movies with  plot descriptions from Wikipedia pages. These plots form the unstructured data associated with movie entities in our prompts. Two scoring functions with distinct relevance and diversity definitions are used, \(\mathcal{F}_3\) and \(\mathcal{F}_4\) defined in Table \ref{tab:Scoring_functions}.

\begin{comment}
    \begin{itemize}
  \item \(\mathcal{F}_3\): 
    \begin{itemize}
      \item Relevance definition: "Brief plot"
      \item Diversity definition: "Production year"
    \end{itemize}
  \item \(\mathcal{F}_4\):
    \begin{itemize}
      \item Relevance definition: "Popularity"
      \item Diversity definition: "Genres \& movie eras"
    \end{itemize}
\end{itemize}
\end{comment}


\noindent \textbf{3) Top-k businesses in Yelp dataset.} This is the Yelp dataset, which contains various attributes for 150,346 businesses (e.g., restaurants), including reviews and images. We used a bundle of reviews and images as the associated unstructured data for each business entity in our prompts. LangChain is used for image processing with LLMs. Two scoring functions with distinct relevance and diversity definitions are used, \(\mathcal{F}_5\) and \(\mathcal{F}_6\) defined in Table \ref{tab:Scoring_functions}.

\begin{comment}
   \begin{itemize}
  \item \(\mathcal{F}_5\): 
    \begin{itemize}
      \item Relevance definition: "Location near New York"
      \item Diversity definition: "Cost variety"
    \end{itemize}
  \item \(\mathcal{F}_6\):
    \begin{itemize}
      \item Relevance definition: "Cuisine type"
      \item Diversity definition: "Varied operating hours"
    \end{itemize}
\end{itemize} 
\end{comment}


\subsubsection{Implemented Algorithms}\label{sec:exp_algs}
\ \\
There does not exist any related work that studies the problem that we do. However, we still design baselines that are appropriate.\\
\noindent {\bf ***} \texttt{Baseline.} This approach makes all LLM calls to calculate the exact score for all candidates without maintaining bounds. It then selects the highest scoring candidate.\\
\noindent {\bf ***} \texttt{EntrRed} using {\tt ProbDep.} The original version of our  framework, which does not assume independence among candidates and accounts for their dependencies in the computation.\\
\noindent {\bf ***} {\tt EntrRed} using {\tt ProbInd.} A time-efficient variation of our  framework that assumes independence among candidates, enabling more efficient computation of the probability distribution function (pdf).\\
\noindent {\bf ***} {\tt Random}. This approach maintains score bounds. However, instead of computing a pdf and determining the next question based on that, it selects the next best question randomly.


\subsubsection{Measures}\label{sec:exp_measures}
\ \\ 
%We evaluated the effectiveness and efficiency of our  framework using the following metrics:
\noindent \textbf{Number of LLM Calls.} The primary objective of our framework is to identify the top-k set with the highest score while minimizing the cost of using LLMs as oracles/experts. The number of LLM calls serves as a key indicator of cost. %, particularly in large-scale data management applications where LLM calls can be expensive. 
We hence compared the total number of LLM calls for {\tt Random} vs. {\tt EntrRed} using {\tt ProbInd}. We also compared {\tt EntrRed} using {\tt ProbDep} vs. {\tt EntrRed} using {\tt ProbInd} to investigate if computing probabilities considering dependence using {\tt EntrRed} using {\tt ProbDep} can result in reducing cost more effectively compared to {\tt EntrRed} using {\tt ProbInd}. 

\noindent \textbf{Recall.} We evaluate recall of the top-$k$ produced by our designed solutions wrt baseline. 

\noindent \textbf{Time Taken for Each Component.} We measured the total time taken by the different algorithms and the time taken for each of the four components of our framework. 

\subsection{Results}\label{sec:results}
The top-$k$ set produced by our proposed algorithms achieve 100\% recall always, as expected.  
\subsubsection{\noindent \textbf{Cost Experiments.}}
Given a fixed number of candidates $n$, we vary \( k \) and report the cost (\# LLM calls) of finding the top-k set. Since Baseline performs all  calls, it is omitted from this experiment. 

\begin{figure*}[!htbp]
    \centering
    \subfigure[Hotels - Scoring function \( \mathcal{F}_1 \)]{
        \includegraphics[width=0.32\textwidth,height=0.18\textheight]{Charts/Quality_Ind_Random/Results_Hotels_REL_Rating_of_the_hotel_DIV_Physical_distance_of_the_hotels.jpg}
    }\hfill
    \subfigure[Hotels - Scoring function \( \mathcal{F}_2 \)]{
        \includegraphics[width=0.32\textwidth,height=0.18\textheight]{Charts/Quality_Ind_Random/Results_Hotels_REL_Distance_from_city_center_DIV_Star_rating.jpg}
    }\hfill
    \subfigure[Movies - Scoring function \( \mathcal{F}_3 \)]{\includegraphics[width=0.32\textwidth,height=0.18\textheight]{Charts/Quality_Ind_Random/Results_Movies_REL_Brief_plot_DIV_Different_years.jpg}
    }\vfill
    \subfigure[Movies - Scoring function \( \mathcal{F}_4 \)]{
        \includegraphics[width=0.32\textwidth,height=0.18\textheight]{Charts/Quality_Ind_Random/Results_Movies_REL_Popularity_DIV_Genre_and_movie_periods.jpg}
    }\hfill
    \subfigure[Businesses - Scoring function \( \mathcal{F}_5 \)]{
        \includegraphics[width=0.32\textwidth,height=0.18\textheight]{Charts/Quality_Ind_Random/Results_Businesses_REL_Location_Around_New_York_DIV_Cost.jpg}
    }\hfill
    \subfigure[Businesses - Scoring function \( \mathcal{F}_6 \)]{
        \includegraphics[width=0.32\textwidth,height=0.18\textheight]{Charts/Quality_Ind_Random/Results_Businesses_REL_Type_of_food_DIV_Open_hours.jpg}
    }
    \caption{\small \#LLM calls varying \( k \) - {\tt EntrRed} using {\tt ProbInd} vs. {\tt Random}}
    \label{fig:quality_ind_random}
\vspace{-0.1in}
\end{figure*}
%\vspace{-0.05in}
1) {\tt EntrRed} using {\tt ProbInd} vs. {\tt Random}:
Figure \ref{fig:quality_ind_random} compares the cost between our  framework with {\tt Random} across the 6 scoring functions defined over the 3 datasets, as detailed in the previous subsection \ref{sec:exp_setup}. These figures demonstrate two key observations:
\begin{itemize}
    \item Our method for determining the next question significantly reduces  the cost (by an order of magnitude) compared to the {\tt Random} algorithm. 
    \item As expected, the cost increases with \( k \) for both the solutions, because a larger \( k \) involves more questions contributing to each candidate's score. However, the cost increases sublinearly for ours with increasing $k$.
\end{itemize}

2) {\tt EntrRed} using {\tt ProbDep} vs. {\tt EntrRed} using {\tt ProbInd}: Here, we focus on comparing the two variants of our  proposed probabilistic models discussed in Section~\ref{winning_probability}. Figure \ref{fig:quality_ind_dep} compares the costs of {\tt EntrRed} using {\tt ProbInd} versus {\tt EntrRed} using {\tt ProbDep}. For brevity, we only present a subset of results that are representative (additional results could be found in our technical report~\cite{tr}).
\begin{itemize}
    \item  {\tt ProbDep} outperforms {\tt ProbInd} in terms of cost for {\tt EntrRed}. This is expected, since unlike in {\tt ProbInd},  {\tt ProbDep} factors in dependence among candidates in calculating the winning probability of each candidate, requiring smaller number of oracle calls in the end.
    \item However, the difference in cost between these two variants is minor, validating that either variant is highly effective for cost reduction compared to {\tt Random}.
\end{itemize}

\subsubsection{\noindent \textbf{Scalability Experiments.}} We present the scalability of our framework by measuring running time of the designed algorithms of each of the four tasks, as well as computing the total running time needed for solving all four tasks. Since {\tt Random} is very poor qualitatively, it is omitted from these experiments.

1) \textbf{Total time taken}: In this exp, we vary the size of the candidate set and measure overall running time of the framework. Figure \ref{fig:scalibility_total_time} shows the total time taken by our framework with {\tt EntrRed} using {\tt ProbDep} vs. {\tt EntrRed} using {\tt ProbInd}. The independence assumption in {\tt EntrRed} using {\tt ProbInd} makes the solution significantly lean in computation time compared to {\tt EntrRed} using {\tt ProbDep} - as expected, the former exhibits an order of magnitude speed up compared to the latter. 

\begin{comment}
2) \textbf{Identifying candidates}: Figures \ref{fig:scalibility_init_candidates_results} and \ref{fig:scalibility_init_candidates_last_two} depict the total time taken for identifying the candidates set. This time is negligible and scales well with an increasing number of candidates.    
\end{comment}

2) \textbf{Computing bounds}: Figure \ref{fig:scalibility_computing_bounds} shows the time for computing bounds in our framework (additional results could be found in our tech report~\cite{tr}). These results corroborate our theoretical analysis - computing bounds is a lightweight task and scales very well. The time takes in negligible compared to the total time.

3) \textbf{Probabilistic model}: This is the most computationally expensive component of the framework. Figure \ref{fig:scalibility_compute_pdf} illustrates (refer to~\cite{tr} for additional results) the time for computing the probabilistic models. Additional results could be found in our technical report~\cite{tr}. Computing probabilities using {\tt ProbDep} is significantly more expensive compared to {\tt ProbInd}, where we assume candidates are independent. This results in a challenge for scalability when the number of candidates grows, whereas {\tt ProbInd} remains highly scalable. Indeed, computing the probabilistic model using {\tt ProbInd} is an order of magnitude faster than its counterpart.

4) \textbf{Determining the next question}: Figure \ref{fig:scalibility_determine_next_q} shows the time for the next best question of {\tt EntrRed} using {\tt ProbInd}. Additional results could be found in our technical report~\cite{tr}. As expected, it is highly efficient and scales well when increasing the number of candidates.

5) \textbf{LLM response}: Figure \ref{fig:scalibility_llm_response} depicts the time taken by the LLM to respond to a question by our proposed framework. {\tt Baseline} makes considerably more LLM calls, leading to a significant increase in processing time. These results reinforce our motivation - calling an LLM is expensive and hence minimizing that cost is important.

\subsection{Summary of Results}
Our experimental evaluation reveals two key findings: 1. As expected, our proposed framework consistently returns the exact top-$k$ results. Additionally, it is highly effective in minimizing the number of LLM calls, achieving a reduction by an order of magnitude compared to baseline methods. The framework demonstrates exceptional generalizability, performing well across various large-scale applications that involve multimodal data. 2. The scalability investigations conducted in this study prove effective. As theoretically analyzed, computing the probabilistic model is the most time-consuming component of the framework. The improvements we suggested in our scalable solution, {\b EntrRed} using {\tt ProbInd}, are both effective and efficient.


\begin{comment}
    In a nutshell, Here is the order of time taken for each component of EntrRedInd and EntrRedDep:
\[
\text{Computing PDF} \gg \text{LLM Response} > \text{Determine next question}
\]
\[
> \text{Update bounds} > \text{Init candidates}.
\]
\end{comment}

\begin{figure*}[!htbp]
    \centering
    \subfigure[Hotels - Scoring function \( \mathcal{F}_1 \)]{
        \includegraphics[width=0.32\textwidth,height=0.18\textheight]{Charts/Quality_Ind_Dep/Results_Hotels_REL_Rating_of_the_hotel_DIV_Physical_distance_of_the_hotels.jpg}
    }\hfill
    \subfigure[Movies - Scoring function \( \mathcal{F}_3 \)]{
        \includegraphics[width=0.32\textwidth,height=0.18\textheight]{Charts/Quality_Ind_Dep/Results_Movies_REL_Brief_plot_DIV_Different_years.jpg}
        }\hfill
    \subfigure[Businesses - Scoring function \( \mathcal{F}_5 \)]{
        \includegraphics[width=0.32\textwidth,height=0.18\textheight]{Charts/Quality_Ind_Dep/Results_Businesses_REL_Location_Around_New_York_DIV_Cost.jpg}
    }
    \caption{\small \#LLM calls varying \( k \) - {\tt EntrRed} using {\tt ProbDep} vs. {\tt EntrRed} using {\tt ProbInd}}
    \label{fig:quality_ind_dep}
     \vspace{-0.1in}
\end{figure*}

\begin{table}[!htbp]
\centering
\begin{tabular}{|c|c|c|}
\hline
\textbf{Scoring} & \textbf{Relevance} & \textbf{Diversity} \\ 
\textbf{Function} & \textbf{Construct} & \textbf{Construct} \\ \hline
\(\mathcal{F}_1\) & Hotel rating & Physical distance \\ \hline
\(\mathcal{F}_2\) & Proximity to city center & Star rating \\ \hline
\(\mathcal{F}_3\) & Brief plot & Production year \\ \hline
\(\mathcal{F}_4\) & Popularity & Genres \& movie eras \\ \hline
\(\mathcal{F}_5\) & Location near New York & Cost variety \\ \hline
\(\mathcal{F}_6\) & Cuisine type & Varied operating hours \\ \hline
\end{tabular}
\caption{\small Personalized scoring functions used in experiments}\label{tab:Scoring_functions}
\end{table}
\vspace{-0.3in}



\begin{figure*}[!htbp]
    \centering
    \subfigure[Hotels - Scoring function \( \mathcal{F}_1 \)]{
        \includegraphics[width=0.32\textwidth,height=0.18\textheight]{Charts/Scalibility/Results_Hotels_REL_Rating_of_the_hotel_DIV_Physical_distance_of_the_hotels_Total_time.jpg}
    }\hfill
    \subfigure[Hotels - Scoring function \( \mathcal{F}_2 \)]{
        \includegraphics[width=0.32\textwidth,height=0.18\textheight]{Charts/Scalibility/Results_Hotels_REL_Distance_from_city_center_DIV_Star_rating_Total_time.jpg}
    }\hfill
    \subfigure[Movies - Scoring function \( \mathcal{F}_3 \)]{
        \includegraphics[width=0.32\textwidth,height=0.18\textheight]{Charts/Scalibility/Results_Movies_REL_Brief_plot_DIV_Different_years_Total_time.jpg}
    }\vfill
    \subfigure[Movies - Scoring function \( \mathcal{F}_4 \)]{
        \includegraphics[width=0.32\textwidth,height=0.18\textheight]{Charts/Scalibility/Results_Movies_REL_Popularity_DIV_Genre_and_movie_periods_Total_time.jpg}
    }\hfill
    \subfigure[Businesses - Scoring function \( \mathcal{F}_5 \)]{
        \includegraphics[width=0.32\textwidth,height=0.18\textheight]{Charts/Scalibility/Results_Businesses_REL_Location_Around_New_York_DIV_Cost_Total_time.jpg}
    }\hfill
    \subfigure[Businesses - Scoring function \( \mathcal{F}_6 \)]{
        \includegraphics[width=0.32\textwidth,height=0.18\textheight]{Charts/Scalibility/Results_Businesses_REL_Type_of_food_DIV_Open_hours_Total_time.jpg}
        }
    \caption{\small Total time taken for {\tt EntrRed} using {\tt ProbDep} vs. {\tt EntrRed} using {\tt ProbInd} varying \#candidates.}
    \label{fig:scalibility_total_time}
    \vspace{-0.1in}
\end{figure*}

\begin{comment}
\begin{figure*}[!htbp]
    \centering
    \subfigure[Hotels - Scoring function \( \mathcal{F}_1 \)]{
        \includegraphics[width=0.32\textwidth,height=0.18\textheight]{Charts/Scalibility/Results_Hotels_REL_Rating_of_the_hotel_DIV_Physical_distance_of_the_hotels_total_time_init_candidates_set.jpg}
    }\hfill
    \subfigure[Businesses - Scoring function \( \mathcal{F}_2 \)]{
        \includegraphics[width=0.32\textwidth,height=0.18\textheight]{Charts/Scalibility/Results_Businesses_REL_Type_of_food_DIV_Open_hours_total_time_init_candidates_set.jpg}
    }\hfill
    \subfigure[Movies - Scoring function \( \mathcal{F}_3 \)]{
        \includegraphics[width=0.32\textwidth,height=0.18\textheight]{Charts/Scalibility/Results_Movies_REL_Brief_plot_DIV_Different_years_total_time_init_candidates_set.jpg}
    }\vfill
    \subfigure[Movies - Scoring function \( \mathcal{F}_4 \)]{
        \includegraphics[width=0.32\textwidth,height=0.18\textheight]{Charts/Scalibility/Results_Movies_REL_Popularity_DIV_Genre_and_movie_periods_total_time_init_candidates_set.jpg}
    }\hfill
    \subfigure[Businesses - Scoring function \( \mathcal{F}_5 \)]{
        \includegraphics[width=0.32\textwidth,height=0.18\textheight]{Charts/Scalibility/Results_Businesses_REL_Location_Around_New_York_DIV_Cost_total_time_init_candidates_set.jpg}
    }\hfill
    \subfigure[Hotels - Scoring function \( \mathcal{F}_6 \)]{
        \includegraphics[width=0.32\textwidth,height=0.18\textheight]{Charts/Scalibility/Results_Hotels_REL_Distance_from_city_center_DIV_Star_rating_total_time_init_candidates_set.jpg}
    }
    \caption{\small Total time taken for identifying candidates set with increasing number of candidates.}
    \label{fig:scalibility_total_time_results}
\end{figure*}    
\end{comment}




\begin{figure*}[!htbp]
    \centering
    \subfigure[Hotels - Scoring function \( \mathcal{F}_1 \)]{
        \includegraphics[width=0.32\textwidth,height=0.18\textheight]{Charts/Scalibility/Results_Hotels_REL_Rating_of_the_hotel_DIV_Physical_distance_of_the_hotels_total_time_update_bounds.jpg}
    }\hfill
    \subfigure[Movies - Scoring function \( \mathcal{F}_3 \)]{
        \includegraphics[width=0.32\textwidth,height=0.18\textheight]{Charts/Scalibility/Results_Movies_REL_Brief_plot_DIV_Different_years_total_time_update_bounds.jpg}
    }\hfill
    \subfigure[Businesses - Scoring function \( \mathcal{F}_5 \)]{
        \includegraphics[width=0.32\textwidth,height=0.18\textheight]{Charts/Scalibility/Results_Businesses_REL_Location_Around_New_York_DIV_Cost_total_time_update_bounds.jpg}
    }
    \caption{\small Time taken for computing bounds varying \#candidates.}
    \label{fig:scalibility_computing_bounds}
     \vspace{-0.1in}
\end{figure*}


\begin{figure*}[!htbp]
    \centering
    \subfigure[Hotels - Scoring function \(\mathcal{F}_1\)]{
        \includegraphics[width=0.32\textwidth,height=0.18\textheight]{Charts/Scalibility/Results_Hotels_REL_Rating_of_the_hotel_DIV_Physical_distance_of_the_hotels_total_time_compute_pdf.jpg}
    }\hfill
    \subfigure[Businesses - Scoring function \(\mathcal{F}_3\)]{
        \includegraphics[width=0.32\textwidth,height=0.18\textheight]{Charts/Scalibility/Results_Movies_REL_Brief_plot_DIV_Different_years_total_time_compute_pdf.jpg}
    }\hfill
    \subfigure[Businesses - Scoring function \(\mathcal{F}_5\)]{
        \includegraphics[width=0.32\textwidth,height=0.18\textheight]{Charts/Scalibility/Results_Businesses_REL_Location_Around_New_York_DIV_Cost_total_time_compute_pdf.jpg}
    }
    \caption{\small Time taken for computing probabilistic model {\tt ProbInd} vs. {\tt ProbDep} varying \#candidates.}
    \label{fig:scalibility_compute_pdf}
    \vspace{-0.1in}
\end{figure*}


\begin{figure*}[!htbp]
    \centering
    \subfigure[Hotels - Scoring function \(\mathcal{F}_1\)]{
        \includegraphics[width=0.32\textwidth,height=0.18\textheight]{Charts/Scalibility/Results_Hotels_REL_Rating_of_the_hotel_DIV_Physical_distance_of_the_hotels_total_time_determine_next_question.jpg}
    }\hfill
    \subfigure[Movies - Scoring function \(\mathcal{F}_3\)]{
        \includegraphics[width=0.32\textwidth,height=0.18\textheight]{Charts/Scalibility/Results_Movies_REL_Brief_plot_DIV_Different_years_total_time_determine_next_question.jpg}
    }\hfill
    \subfigure[Businesses - Scoring function \(\mathcal{F}_5\)]{
        \includegraphics[width=0.32\textwidth,height=0.18\textheight]{Charts/Scalibility/Results_Businesses_REL_Location_Around_New_York_DIV_Cost_total_time_determine_next_question.jpg}
    }
    \caption{\small Time taken for {\tt EntrRed} using {\tt ProbInd} to determine the next question varying \#candidates.}
    \label{fig:scalibility_determine_next_q}
     \vspace{-0.1in}
\end{figure*}


\begin{figure*}[!htbp]
    \centering
    \subfigure[Hotels - Scoring function \(\mathcal{F}_1\)]{
        \includegraphics[width=0.32\textwidth,height=0.18\textheight]{Charts/Scalibility/Results_Hotels_REL_Rating_of_the_hotel_DIV_Physical_distance_of_the_hotels_total_time_llm_response.jpg}
    }\hfill
    \subfigure[Movies - Scoring function \(\mathcal{F}_4\)]{
        \includegraphics[width=0.32\textwidth,height=0.18\textheight]{Charts/Scalibility/Results_Movies_REL_Popularity_DIV_Genre_and_movie_periods_total_time_llm_response.jpg}
    }\hfill
    \subfigure[Businesses - Scoring function \(\mathcal{F}_5\)]{
        \includegraphics[width=0.32\textwidth,height=0.18\textheight]{Charts/Scalibility/Results_Businesses_REL_Location_Around_New_York_DIV_Cost_total_time_llm_response.jpg}
    }\hfill
    \caption{\small Time taken for LLM response varying \#candidates.}
    \label{fig:scalibility_llm_response}
     \vspace{-0.1in}
\end{figure*}
