\begin{definition}
    \textbf{Response Processing. }Let $\mathcal{Q}$ be a given question, and let $R = \{R_1, R_2, \ldots, R_n\}$ represent the set of responses from $n$ different experts. Suppose there is a set of desired possible responses $\mathcal{S} = \{r_1, r_2, \ldots, r_l\}$, where each $r_i$ represents a potential valid answer to be evaluated.
   \end{definition}

Response processing can be defined as a mapping from the set of experts' answers $R$ to a probability density function (PDF) over the set of desired responses $\mathcal{S}$. This process generates a PDF $f_{\mathcal{S}}(r)$, representing the probability density over $\mathcal{S}$, indicating the likelihood of each desired response $r_i \in \mathcal{S}$, given the set of experts' answers $R$. 

Hence, we can formulate this task as follows:

\[
\text{Response Processing}: R \mapsto f_{\mathcal{S}}(r), \quad r \in \mathcal{S}.
\]

The definition can be further elaborated based on the number of experts (single vs. multiple) and the nature of responses (single  vs. a range of responses). This leads to four possible cases:
\textbf{Single Expert, Single Response:} Let a single expert provide one single response $R_1$. The  processing maps this response to a corresponding desired response $r^* \in \mathcal{S}$, where this mapping can be written as:
\begin{align*}
    R_1 \mapsto r^*, \quad r^* \in \mathcal{S}
\end{align*}


The probability density function $f_{\mathcal{S}}(r)$ over the set of desired responses $\mathcal{S}$ is then given by:

\begin{align*}
f_{\mathcal{S}}(r) =
\begin{cases}
1 & \text{if } r = r^* \\
0 & \text{if } r \neq r^*
\end{cases}
\end{align*}


This mapping assigns the probability 1 to the specific response $r^*$ that is derived from the expert's response $R_1$, and 0 to all other responses in $\mathcal{S}$.

\textbf{Single Expert, Range of Responses:} Let a single expert provide a range of responses denoted by $(R_{1,i}, R_{1,j})$, where $R_{1,i}$ and $R_{1,j}$ represent the lower and upper bounds of the response range, respectively. The response processing maps this range to a corresponding subset of desired responses $\mathcal{S}^* \subseteq \mathcal{S}$. In other words:

\begin{align*}
(R_{1,i}, R_{1,j}) & \mapsto \mathcal{S}^* = \{r_1, r_2, \ldots, r_k\}, \\
& \quad \mathcal{S}^* \subseteq \mathcal{S}, \quad k \geq 1.
\end{align*}


The probability density function $f_{\mathcal{S}}(r)$ over the set of desired responses $\mathcal{S}$ is then defined as:

\begin{align*}
f_{\mathcal{S}}(r) = 
\begin{cases}
\frac{1}{k}, & \text{if } r \in \mathcal{S}^*, \\
0, & \text{if } r \notin \mathcal{S}^*.
\end{cases}
\end{align*}


This mapping assigns an equal probability of $\frac{1}{k}$ to each response in the subset $\mathcal{S}^*$, derived from the range $(R_{1,i}, R_{1,j})$, and a probability of 0 to all other responses in $\mathcal{S}$.


{\bf Multiple Experts, Single Response Each:} Let multiple experts provide single responses, denoted by the set $R = \{R_1, R_2, \ldots, R_n\}$, where $n$ is the number of experts. Each expert response $R_i$ is mapped to an element in the set of desired responses $\mathcal{S}$. In other words:

\begin{align*}
R_i \mapsto r_i, \quad & r_i \in \mathcal{S}, \quad i = 1, 2, \ldots, n.
\end{align*}


The list of aggregated mapped responses is denoted as $\mathcal{S}^* = [r_1, r_2, \ldots, r_n]$, where each $r_i$ is derived from $R_i$.

The probability density function $f_{\mathcal{S}}(r)$ over the set of desired responses $\mathcal{S}$ is then defined based on the frequency of each response in the list $\mathcal{S}^*$:

\begin{align*}
f_{\mathcal{S}}(r) &= \frac{\text{count}(r)}{n}, \quad \forall r \in \mathcal{S}.
\end{align*}


where $\text{count}(r)$ denotes the number of times the response $r$ appears in the list $\mathcal{S}^*$. 



\textbf{Multiple Experts, Range of Responses Each:} Let multiple experts provide a range of responses each, denoted by $(R_{i,1}, R_{i,2})$ for the $i$-th expert, where $R_{i,1}$ and $R_{i,2}$ represent the lower and upper bounds of the response range for expert $i$. The set of experts' responses is given by $R = \{(R_{1,1}, R_{1,2}), (R_{2,1}, R_{2,2}), \ldots, (R_{n,1}, R_{n,2})\}$ for $n$ experts.

Each expert's range of responses $(R_{i,1}, R_{i,2})$ is mapped to a subset of desired responses $\mathcal{S}_i^* \subseteq \mathcal{S}$:

\begin{align*}
(R_{i,1}, R_{i,2}) & \mapsto \mathcal{S}_i^* = \{r_{i,1}, r_{i,2}, \ldots, r_{i,k_i}\}, \\
& \quad \mathcal{S}_i^* \subseteq \mathcal{S}, \quad k_i \geq 1.
\end{align*}


The overall mapping results in a list of desired responses denoted as $\mathcal{S}^*$, which is an aggregation of each $\mathcal{S}_i^*$. In other words:

\begin{align*}
\mathcal{S}^* = \left[ r_{1,1}, r_{1,2}, \ldots, r_{1,k_1}, \; 
                     r_{2,1}, r_{2,2}, \ldots, r_{2,k_2}, \; 
                     \ldots, \; 
                     r_{n,1}, r_{n,2}, \ldots, r_{n,k_n} \right]
\end{align*}


The probability density function $f_{\mathcal{S}}(r)$ over the set of desired responses $\mathcal{S}$ is then determined by the frequency of each response $r$ in the list $\mathcal{S}^*$, divided by the total number of elements in $\mathcal{S}^*$, denoted as $n^*$:

\[
f_{\mathcal{S}}(r) = \frac{\text{count}(r)}{n^*}, \quad \forall r \in \mathcal{S}.
\]


where $\text{count}(r)$ denotes the number of times the response $r$ appears in the list $\mathcal{S}^*$. 




Imagine we have two experts ($e_1$ and $e_2$) each of which returns a range of response (case d).
For simplicity, imagine the set of desired responses S is defined as follows:
        \begin{align*}
            S = \{0.0, 0.5, 1.0\}
        \end{align*}

        Given the running example, let's assume we ask  $Q_1 = D(MLN, HYN)$ as the next question, and we get the following responses from the two experts:

        \[
R_{e1} = (0.4, 1.0), \quad R_{e2} = (0.3, 0.6)
\]

Next, we map the above responses to elements of \( \mathcal{S} \) as follows:

\begin{align*}
    e_1: \quad (R_{e1, 1}, R_{e1, 2}) & = (0.4, 1.0) \mapsto \mathcal{S}_1^* = \{0.5, 1.0\} \\
    e_2: \quad (R_{e2, 1}, R_{e2, 2}) & = (0.3, 0.6) \mapsto \mathcal{S}_2^* = \{0.5\}
\end{align*}


Then, we aggregate the two sets together:

\begin{align*}
    \mathcal{S}^* & = [0.5, 1.0, 0.5]
\end{align*}


Then, we can compute the probability density function over the set of desired responses \( \mathcal{S} \) as follows:

\begin{align*}
    f_{\mathcal{S}}(r) & = \frac{\text{count}(r)}{n^*} \\
                       & = \frac{\text{count}(r)}{3}, \quad \forall r \in S = \{0.0, 0.5, 1.0\}
\end{align*}


where \( \text{count}(r) \) denotes the number of times the response \( r \) appears in the list \( \mathcal{S}^* \), and $n^*$ is the total number of elements in $S^*$. Finally, we will have:

\begin{align*}
    f_{\mathcal{S}}(r) =
    \begin{cases}
        \frac{0}{3} & r = 0.0 \\
        \frac{2}{3} & r = 0.5 \\
        \frac{1}{3} & r = 1.0 \\
    \end{cases}
\end{align*}



\autoref{fig:algorithm_framework} illustrates the key components of the computational framework. In \autoref{section3}, we provide a detailed description of our algorithm for each component.
% NOTE: Response preparation - also adds a task 5
% NOTE: Number them
% NOTE: Input box has some inputs
\begin{figure}[ht]
    \centering
    \includegraphics[width=0.5\textwidth]{figures/Algorithmic_Framework.jpg}
    \caption{The five different components of the probability framework that guide the process of identifying the final answer.}
    \label{fig:algorithm_framework}
\end{figure}


Similarly, we update the bounds for \( c_{5} \), \( c_{6} \), \( c_{8} \), \( c_{9} \), and \( c_{10} \). The new candidate set bounds are as follows:

\[
\text{candidates\_set\_bounds} = \{ 
\begin{aligned}
c_1 &: (0, 6), & c_2 &: (0, 6), \\
c_3 &: (0.5, 5.5), & c_4 &: (0, 6), \\
c_5 &: (0.5, 5.5), & c_6 &: (0.5, 5.5), \\
c_7 &: (0, 6), & c_8 &: (0.5, 5.5), \\
c_9 &: (0.5, 5.5), & c_{10} &: (0.5, 5.5) 
\end{aligned}
\}
\]


This candidates set corresponds to a specific time \( t = 1 \), which only \( R(WLD, q) \) has been asked. When the next question is asked (\( t = 2 \)), the candidates set will be updated again.

Now, at (\( t = 2 \)), let's assume we choose a diversity question like \( D(HNY, MLN) \) to ask. \autoref{tab:ny_hotels_diversity} represents the diversity scores for each pair of hotels. Hence, the expert response is \(1.0\) as \( D(HNY, MLN) = 1.0\) in this table. Now, we need to update candidates including both HNY and MLN, which are \( c_{1} \), \( c_{2} \), and \( c_{3} \). For instance, for \( c_{3} \) we will have:

\begin{align*}
\mathcal{F}(c_3, q) & = R(HNY, q) + R(MLN, q) + 0.5 + D(HNY, MLN) \\
          & \quad + D(HNY, WLD) + D(MLN, WLD) \\
          & = R(HNY, q) + R(MLN, q) + 0.5 + 1.0 \\
          & \quad + D(HNY, WLD) + D(MLN, WLD) \\
          & \implies 1.5 \leq \mathcal{F}(c_3, q) \leq 5.5
\end{align*}

Similarly, we update the bounds for \( c_{1} \) and \( c_{2} \). The new candidates set bounds are as follows:

\[
\text{candidates\_set\_bounds} = \{ 
\begin{aligned}
c_1 &: (1.0, 6), & c_2 &: (1.0, 6), \\
c_3 &: (1.5, 5.5), & c_4 &: (0.0, 6.0), \\
c_5 &: (0.5, 5.5), & c_6 &: (0.5, 5.5), \\
c_7 &: (0.0, 6.0), & c_8 &: (0.5, 5.5), \\
c_9 &: (0.5, 5.5), & c_{10} &: (0.5, 5.5)
\end{aligned}
\}
\]

Finally, let's assume we are at the current point in time, when we have asked all the questions which have known values in \autoref{tab:ny_hotels_relevance} and \autoref{tab:ny_hotels_diversity}, resulting in those two tables. If we update the bounds accordingly based on all the known values, we will have up-to-date bounds as follows:

\[
\text{candidates\_set\_bounds} = \{ 
\begin{aligned}
c_1 &: (3.5, 5.5), & c_2 &: (2.5, 4.5), \\
c_3 &: (3.0, 5.0), & c_4 &: (2.0, 4.0), \\
c_5 &: (3.0, 4.0), & c_6 &: (1.5, 3.5), \\
c_7 &: (2.0, 5.0), & c_8 &: (3.0, 5.0), \\
c_9 &: (1.5, 4.5), & c_{10} &: (2.0, 4.0)
\end{aligned}
\}
\]
