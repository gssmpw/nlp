
\subsection{High-Level Taks in Computational Framework}\label{sec:section2.3}
Our proposed framework consists of five essential tasks in the overall top-k selection process:

\begin{enumerate}
    \item \textbf{Task 1: Identifying the Candidates' Set}: 
    %\begin{definition}
    %{\bf Candidate.} A candidate c is a set of $k$-entities that could potentially be produced as the answer to the query.
    %\end{definition}

    %\begin{definition}
     %   {\bf Candidates set.} A set $\mathcal{C}$ containing any possible candidate c which is likely to be the top-k set at a given point in time. 
    %\end{definition}

    Depending on the underlying scoring function and algorithm, the framework identifies one, several, or all candidate top-$k$ sets as answers to the query.
    
    \begin{example}
        Using the running example, two possible candidates could be: 
        \[
        c_1 = \{\text{HNY, MLN, HYN}\}, \quad c_2 = \{\text{MLN, HYN, WLD}\}
        \]
        Hence, we can define the candidates' set C as follows:
        \[
        C = \{c1, c2\}
        \]
        
    \end{example}
    
    \item \textbf{Task 2: Computing bounds of Candidates}: 
         \begin{definition}
     {\bf Lower and upper bounds of a candidate.} Given a scoring function $\mathcal{F}$, and the known information $Q_K$, the lower bound (resp. upper bound) of score of c is the smallest (resp. largest) score that c can have.
    \end{definition}

    \begin{example}
        Considering our running example, let us assume we are interested in finding the bounds for c1:
        \[
        c_1 = \{\text{HNY, MLN, HYN}\}
        \]
        Given $\mathcal{F}_{\text{rel+div}}$ as the scoring function, we can compute the score of c1 as follows:
        
        
        \begin{align*}
            \mathcal{F}(c_1, q) = 
            &R(\text{HNY}) + R(\text{MLN}) + R(\text{HYN}) + D(\text{HNY, MLN}) \\
            &+ D(\text{HNY, HYN}) + D(\text{MLN, HYN})
        \end{align*}

    
        To find the lowest and highest possible scores of $c_1$, we should replace known values from \autoref{tab:ny_hotels_relevance} and \autoref{tab:ny_hotels_diversity} in the above formula and substitute unknown values in the formula with minimum and maximum possible response values which are assumed to be 0 and 1 respectively. Therefore, we can compute the minimum and maximum overall scores of $c_1$:
\[
\mathcal{F}_{\text{min}}(c_1, q) = R(\text{HNY}) + R(\text{MLN}) + 0 + D(\text{HNY, MLN}) + 0 + 0
\]
\[
\mathcal{F}_{\text{min}}(c_1, q) = 0.5 + 1.0 + 0 + 0.5 + 0 + 0 = 2.0
\]

\[
\mathcal{F}_{\text{max}}(c_1, q) = R(\text{HNY}) + R(\text{MLN}) + 1 + D(\text{HNY, MLN}) + 1 + 1
\]
\[
\mathcal{F}_{\text{max}}(c_1, q) = 0.5 + 1.0 + 1 + 0.5 + 1 + 1 = 5.0
\]

Therefore, the lower bound (LB) and upper bound (UB) of \( c_1 \) are:
\begin{align*}
    (\text{LB, UB})_{c_1} &= (2.0, 5.0)
\end{align*}
    
    \end{example}

   
    \item \textbf{Task 3: Probabilistic Model for finding the answer}: 
Given the space of candidates set, query answer is modeled as a discrete random variable $\mathcal{A}$ that takes one candidate $c \in C$, with the probability associated with $c$ to be the answer.
  \begin{definition}
    {\bf Probability of a candidate being the answer.} Given a set $\mathcal{C} = 
    \{c_1, c_2, \ldots, c_m\}$, the probability that $c_i$ is the answer of the query can be formally written as: 
    
    \begin{align*}
    P(c_i = \text{winner}) &= P\left(\bigcap_{c_j \in \mathcal{C}} \mathcal{F}(c_i, q) \geq \mathcal{F}(c_j, q)\right)
\end{align*}

\end{definition}

    \begin{example}
    Let's consider the scenario described in Example \ref{example2.3}, having 3 possible candidates $c_1, c_2$, and $c_3$. Imagine we want to evaluate the chance of $c_1$ being the query answer. For each candidate we have:

    \[
\begin{aligned}
    \mathcal{F}(c_1, q) 
    = & 3.5 + R(\text{HNY}) + D(\text{MLN, HYN}) \\
    \mathcal{F}(c_2, q) 
    = & 3.0 + R(\text{HNY}) + D(\text{MLN, WLD}) \\
    \mathcal{F}(c_3, q) 
    = & 2.0 + R(\text{HNY}) + D(\text{HYN, SHN}) 
\end{aligned}
\]  

Hence, it is trivial that: 

\begin{align*}
    0 < P(c_3 = \text{winner}) < P(c_2 = \text{winner}) < P(c_1 = \text{winner}) < 1
\end{align*}

Following the same scenario as described in Example \ref{example2.3}, if we ask $Q_1 = D(MLN, HYN)$ and get \(D(MLN, HYN) = 1.0 \), regardless of the unknown values, $c_1$ will have higher overall score compared to others. Therefore, in this case, we will have:
\begin{align*}
   P(c_1 = \text{winner}) = 1 \\
   P(c_2 = \text{winner}) = 0 \\
   P(c_3 = \text{winner}) = 0 \\
\end{align*}
It can be observed how asking a question can change the probability of a candidate to be the query answer. Further detail on how to compute the exact winning probability $P(c = \text{winner})$ is mentioned in \autoref{winning_probability}. 

    
    \end{example}
    
    \item \textbf{Task 4: Determining the Next Question}: The next task involves determining the next question to ask. 

 \begin{definition}
     {\bf Uncertainty of $\mathcal{A}$.} {Given $\mathcal{A}$ as the query answer, the uncertainty of finding the top-k set can be formulated as the entropy of $\mathcal{A}$, and it can be written as follows: }
     
    \begin{align*}
    H(\mathcal{A}) &= - \sum_{c \in \mathcal{C}} p_c \log(p_c)
\end{align*}


    Where $p_c$ is the probability of candidate c to be the query answer. 
    \end{definition}

    \begin{example}
        Let's consider the scenario described in Example \ref{example2.3}, having 3 possible candidates $c_1, c_2$, and $c_3$. The uncertainty of the query answer $\mathcal{A}$ can be formulated as:
        \begin{align*}
    H(\mathcal{A}) &= - (p_{c_1} \log(p_{c_1}) + p_{c_2} \log(p_{c_2}) + p_{c_3} \log(p_{c_3}))
    \end{align*}

    Following the same scenario as described in Example \ref{example2.3}, we showed that if we ask $Q_1 = D(MLN, HYN)$ and get \(D(MLN, HYN) = 1.0 \), $c_1$ will be the winner candidate regardless of the unknown values. In other words:
    \begin{align*}
   P(c_1 = \text{winner}) = 1 \\
   P(c_2 = \text{winner}) = 0 \\
   P(c_3 = \text{winner}) = 0 \\
    \end{align*}
    By replacing that in the entropy formula, we will have:
    \begin{align*}
    H(\mathcal{A}) &= - (1 \log(1)) = 0
    \end{align*}

    Hence, after asking \(Q_1\), the query answer is definitively \(c_1\), with no remaining uncertainty. In general, our goal is to identify the next best question that is most likely to maximize the reduction in the entropy of the query answer.
        
    \end{example}
    

   \item \textbf{Task 5: Response Processing}: Instead of relying on a single expert for a question, we could potentially ask it from multiple experts. The goal of a response processing task is to process experts' responses to be used in the bound computation task for updating the bounds.   

\begin{definition}
    \textbf{Response Processing. }Let $\mathcal{Q}$ be a given question, and let $R = \{R_1, R_2, \ldots, R_n\}$ represent the set of responses from $n$ different experts. Suppose there is a set of desired possible responses $\mathcal{S} = \{r_1, r_2, \ldots, r_l\}$, where each $r_i$ represents a potential valid answer to be evaluated.

Response processing can be defined as a mapping from the set of experts' answers $R$ to a probability density function (PDF) over the set of desired responses $\mathcal{S}$. This process generates a PDF $f_{\mathcal{S}}(r)$, representing the probability density over $\mathcal{S}$, indicating the likelihood of each desired response $r_i \in \mathcal{S}$, given the set of experts' answers $R$. 

Hence, we can formulate this task as follows:

\[
\text{Response Processing}: R \mapsto f_{\mathcal{S}}(r), \quad r \in \mathcal{S}.
\]

The definition can be further elaborated based on the number of experts (single vs. multiple) and the nature of responses (single  vs. a range of responses). This leads to four possible cases:

\begin{enumerate}
    \item \textbf{Single Expert, Single Response:} Let a single expert provide one single response $R_1$. The  processing maps this response to a corresponding desired response $r^* \in \mathcal{S}$, where this mapping can be written as:
\begin{align*}
    R_1 \mapsto r^*, \quad r^* \in \mathcal{S}
\end{align*}


The probability density function $f_{\mathcal{S}}(r)$ over the set of desired responses $\mathcal{S}$ is then given by:

\begin{align*}
f_{\mathcal{S}}(r) =
\begin{cases}
1 & \text{if } r = r^* \\
0 & \text{if } r \neq r^*
\end{cases}
\end{align*}


This mapping assigns the probability 1 to the specific response $r^*$ that is derived from the expert's response $R_1$, and 0 to all other responses in $\mathcal{S}$.


    \item \textbf{Single Expert, Range of Responses:} Let a single expert provide a range of responses denoted by $(R_{1,i}, R_{1,j})$, where $R_{1,i}$ and $R_{1,j}$ represent the lower and upper bounds of the response range, respectively. The response processing maps this range to a corresponding subset of desired responses $\mathcal{S}^* \subseteq \mathcal{S}$. In other words:

\begin{align*}
(R_{1,i}, R_{1,j}) & \mapsto \mathcal{S}^* = \{r_1, r_2, \ldots, r_k\}, \\
& \quad \mathcal{S}^* \subseteq \mathcal{S}, \quad k \geq 1.
\end{align*}


The probability density function $f_{\mathcal{S}}(r)$ over the set of desired responses $\mathcal{S}$ is then defined as:

\begin{align*}
f_{\mathcal{S}}(r) = 
\begin{cases}
\frac{1}{k}, & \text{if } r \in \mathcal{S}^*, \\
0, & \text{if } r \notin \mathcal{S}^*.
\end{cases}
\end{align*}


This mapping assigns an equal probability of $\frac{1}{k}$ to each response in the subset $\mathcal{S}^*$, derived from the range $(R_{1,i}, R_{1,j})$, and a probability of 0 to all other responses in $\mathcal{S}$.


    \item \textbf{Multiple Experts, Single Response Each:} Let multiple experts provide single responses, denoted by the set $R = \{R_1, R_2, \ldots, R_n\}$, where $n$ is the number of experts. Each expert response $R_i$ is mapped to an element in the set of desired responses $\mathcal{S}$. In other words:

\begin{align*}
R_i \mapsto r_i, \quad & r_i \in \mathcal{S}, \quad i = 1, 2, \ldots, n.
\end{align*}


The list of aggregated mapped responses is denoted as $\mathcal{S}^* = [r_1, r_2, \ldots, r_n]$, where each $r_i$ is derived from $R_i$.

The probability density function $f_{\mathcal{S}}(r)$ over the set of desired responses $\mathcal{S}$ is then defined based on the frequency of each response in the list $\mathcal{S}^*$:

\begin{align*}
f_{\mathcal{S}}(r) &= \frac{\text{count}(r)}{n}, \quad \forall r \in \mathcal{S}.
\end{align*}


where $\text{count}(r)$ denotes the number of times the response $r$ appears in the list $\mathcal{S}^*$. 



\item \textbf{Multiple Experts, Range of Responses Each:} Let multiple experts provide a range of responses each, denoted by $(R_{i,1}, R_{i,2})$ for the $i$-th expert, where $R_{i,1}$ and $R_{i,2}$ represent the lower and upper bounds of the response range for expert $i$. The set of experts' responses is given by $R = \{(R_{1,1}, R_{1,2}), (R_{2,1}, R_{2,2}), \ldots, (R_{n,1}, R_{n,2})\}$ for $n$ experts.

Each expert's range of responses $(R_{i,1}, R_{i,2})$ is mapped to a subset of desired responses $\mathcal{S}_i^* \subseteq \mathcal{S}$:

\begin{align*}
(R_{i,1}, R_{i,2}) & \mapsto \mathcal{S}_i^* = \{r_{i,1}, r_{i,2}, \ldots, r_{i,k_i}\}, \\
& \quad \mathcal{S}_i^* \subseteq \mathcal{S}, \quad k_i \geq 1.
\end{align*}


The overall mapping results in a list of desired responses denoted as $\mathcal{S}^*$, which is an aggregation of each $\mathcal{S}_i^*$. In other words:

\begin{align*}
\mathcal{S}^* = \left[ r_{1,1}, r_{1,2}, \ldots, r_{1,k_1}, \; 
                     r_{2,1}, r_{2,2}, \ldots, r_{2,k_2}, \; 
                     \ldots, \; 
                     r_{n,1}, r_{n,2}, \ldots, r_{n,k_n} \right]
\end{align*}




The probability density function $f_{\mathcal{S}}(r)$ over the set of desired responses $\mathcal{S}$ is then determined by the frequency of each response $r$ in the list $\mathcal{S}^*$, divided by the total number of elements in $\mathcal{S}^*$, denoted as $n^*$:

\[
f_{\mathcal{S}}(r) = \frac{\text{count}(r)}{n^*}, \quad \forall r \in \mathcal{S}.
\]


where $\text{count}(r)$ denotes the number of times the response $r$ appears in the list $\mathcal{S}^*$. 


\end{enumerate}
\end{definition}


    \begin{example}
        Imagine we have two experts ($e_1$ and $e_2$) each of which returns a range of response (case d).
        
        For simplicity, imagine the set of desired responses S is defined as follows:
        \begin{align*}
            S = \{0.0, 0.5, 1.0\}
        \end{align*}

        Given the running example, let's assume we ask  $Q_1 = D(MLN, HYN)$ as the next question, and we get the following responses from the two experts:

        \[
R_{e1} = (0.4, 1.0), \quad R_{e2} = (0.3, 0.6)
\]

Next, we map the above responses to elements of \( \mathcal{S} \) as follows:

\begin{align*}
    e_1: \quad (R_{e1, 1}, R_{e1, 2}) & = (0.4, 1.0) \mapsto \mathcal{S}_1^* = \{0.5, 1.0\} \\
    e_2: \quad (R_{e2, 1}, R_{e2, 2}) & = (0.3, 0.6) \mapsto \mathcal{S}_2^* = \{0.5\}
\end{align*}


Then, we aggregate the two sets together:

\begin{align*}
    \mathcal{S}^* & = [0.5, 1.0, 0.5]
\end{align*}


Then, we can compute the probability density function over the set of desired responses \( \mathcal{S} \) as follows:

\begin{align*}
    f_{\mathcal{S}}(r) & = \frac{\text{count}(r)}{n^*} \\
                       & = \frac{\text{count}(r)}{3}, \quad \forall r \in S = \{0.0, 0.5, 1.0\}
\end{align*}


where \( \text{count}(r) \) denotes the number of times the response \( r \) appears in the list \( \mathcal{S}^* \), and $n^*$ is the total number of elements in $S^*$. Finally, we will have:

\begin{align*}
    f_{\mathcal{S}}(r) =
    \begin{cases}
        \frac{0}{3} & r = 0.0 \\
        \frac{2}{3} & r = 0.5 \\
        \frac{1}{3} & r = 1.0 \\
    \end{cases}
\end{align*}

    \end{example}
\end{enumerate}

\autoref{fig:algorithm_framework} illustrates the key components of the computational framework. In \autoref{section3}, we provide a detailed description of our algorithm for each component.
% NOTE: Response preparation - also adds a task 5
% NOTE: Number them
% NOTE: Input box has some inputs
\begin{figure}[ht]
    \centering
    \includegraphics[width=0.5\textwidth]{figures/Algorithmic_Framework.jpg}
    \caption{The five different components of the probability framework that guide the process of identifying the final answer.}
    \label{fig:algorithm_framework}
\end{figure}

