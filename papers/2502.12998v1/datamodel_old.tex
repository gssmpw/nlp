\section{Data Model, Problem Definition, Proposed Framework}
We describe our data model, specify our problem definition, and present our propose framework.

\subsection{Data Model}
\textbf{Input Query:} A user writes a query $q$ with an input parameter $k$, with the goal of obtaining the top-$k$ set of entities to $q$.

\begin{example}
    The input query can be $q$ = "Affordable hotels in Manhattan" and $k = 3$. It means the goal is to find the top-3 hotels with respect to q.  
\end{example}

\textbf{Database.} The query comes to a database $D$ containing data of different modalities, such as, pictures, text, audio, etc. Item $i$ represents a physical entity $e$ (such as a hotel). There may be multiple items associated with an entity. Let $n$ represent the total number of unique entities.

The toy example \autoref{fig:motivating_scenario} represents the underlying data with $5$ entities or hotels. Each entity has two associated items with it. For instance, the  hotel "HNY", has \{ i1, i3 \} associated with it, an audio and an image file respectively.

\textbf{Scoring function and characteristics.} A user provides a set based scoring function $\mathcal{F}$ that admits a query  $q$ and an integer $k$, and scores any subset $s \subseteq n$ entities, such that $|s|=k$, i.e., $\mathcal{F}(s,q) \rightarrow R$. 

A scoring function needs to satisfy two conditions: (a) it is set based - meaning it computes the score of a set of size $k$ overall and do not provide the individual order of the entities in the set, (b) It is decomposable into a set of constructs, each of these constructs could be sent to an oracle to obtain its predicted value.

\begin{comment}
    \begin{example}
Assume that the scoring function used to evaluate sets is defined as:

\[
\mathcal{F}_{\text{rel+div}}(c, q) = F_{\text{relevance}}(c, q) + F_{\text{diversity}}(c)
\]

Where $F_{\text{relevance}}(c, q)$ is the sum of relevance between query and each entity in that candidate. In other words:

\[
F_{\text{relevance}}(c, q) = \sum_{\text{e} \in c} \text{R}(\text{e}, q)
\]

Similarly, $F_{\text{diversity}}(c)$ is defined as the sum of diversity between any two entities in c:

\[
F_{\text{diversity}}(c) = \sum_{\text{(e1, e2)} \in c}\text{D(e1, e2)}
\]  

Hence, $\mathcal{F}_{\text{rel+div}}(c, q)$ can be used to generate a score for any set of $k$ entities.
\end{example}
\end{comment}





%Now, we can expand the summations and write the overall score formula for any candidate. For instance:

%\[
%\begin{aligned}
%\mathcal{F}(c, q) = & \, F_{\text{relevance}}(c, q) + F_{\text{diversity}}(c) \\
%= & \, R(\text{HNY}) + R(\text{MLN}) + R(\text{HYN}) \\
%  & \, + D(\text{HNY}, \text{MLN}) + D(\text{HNY}, \text{HYN}) + D(\text{MLN}, \text{HYN})
%\end{aligned}
%\]

%similarly, we can expand the formula for computing the score of any other top-3 candidates. Initially, all the candidates are equally probable to be the top-3 set. As we ask the LLM any \( q \in Q \), we gain information that changes the likelihood of each candidate being better than another.

\begin{comment}
    \textbf{Components of a Scoring Function.} Given a scoring function $\mathcal{F}$, the set of components of $\mathcal{F}$ can be denoted by:
\[
\mathcal{S} = \{\mathcal{F}_1, \mathcal{F}_2, \ldots, \mathcal{F}_m\}
\]

where $\mathcal{F}_{\ell}$ denotes the $\ell$-th component. Each component $\mathcal{F}_{\ell}$ may involve individual entities, pairs of entities, or more complex combinations, depending on the nature of the component. Given the required number of entities as input to a component, the score for that set of entities with respect to the component can be computed.
\end{comment}

Imagine a user defined scoring function with the purpose to compute the top-$k$ set that maximizes the sum of relevance and diversity $\mathcal{F}(s, q)= \Sigma_{e \in s} Rel(q,e)+ \Sigma_{e_i,e_j \in s} Div(e_i,e_j)$. The scoring function identifies the top-$k$ set and not the respective order of the entities in the set, and its two constructs can be written as, $Rel(q, e)$ or $Rel(e_i, e_j)$.

\textbf{Top-$k$ set.} Given  a query  $q$ and an integer $k$, a set of entities $E$, and a set based scoring function $\mathcal{F}$, $s_q$ is a top-$k$ set, if $s_q \subseteq E$, $|s_q|=k$, and 
$s_q = argmax_{s \subseteq E: |s|=k} \mathcal{F}(s, q)$.

Given the running example, the following set of 3 hotels denoted as 
 $c_1 = \{ HNY, MLN, HYN \}$ could be the potential set of top-$3$ hotels. However, certainly determining the top-k set requires more insights regarding the unknown values in \autoref{tab:ny_hotels_relevance} and \autoref{tab:ny_hotels_diversity}.
   

\textbf{Oracle calls/Questions.} A question $\mathcal{Q}$ and a user defined scoring function $\mathcal{F}$ (or a call to a DNN, a human expert, or an LLM) is a triple of one of the following two forms: 
\begin{itemize}
    \item $\mathcal{Q}_1$, which is a question of the form $\mathcal{Q}(e_i, q, F_{\ell})$.
    \item $\mathcal{Q}_2$, which is a question of the form $\mathcal{Q}(e_i, e_j, F_{\ell})$.
\end{itemize}
   where $e_i,e_j$ are the $i$-th, $j$-th entities, $q$ is the query, and $F_{\ell}$ is the $\ell$-th component of the scoring function. Depending on the nature of the scoring function, some constructs involve returning a score between a query and one of the two items, e.g., to compute relevance, which requires questions of the form $Q_1$. Some other construct may admit two entities, to compute diversity, which requires questions of the form $Q_2$.

Considering our running example (example query and scoring function), one can propose different questions about the hotels mentioned in \autoref{tab:ny_hotels_relevance}. For instance, $\mathcal{Q}(HNY, q, relevance)$ asks for the relevance between the hotel HNY and the query q, while $\mathcal{Q}(WLD, SHN, diversity)$ represents a question about the diversity between hotels "WLD" and "SHN".

   
% NOTE; figure 1 change sec column to hotel names
\textbf{Response from oracle.} Given a question \( \mathcal{Q} \), and a pre-specified range of values \([MIN, MAX]\) which the oracle is instructed to respond within, the oracle response \( \mathcal{R} \) is defined to have one of the following two forms:

\begin{itemize}
    \item \( \mathcal{R}_1 = r \in [MIN, MAX] \), where \( r \) is a discrete value within the specified range \([MIN, MAX]\).
    
    \item \( \mathcal{R}_2 = (l, u) \), where \( MIN \leq l < u \leq MAX \) represents a continuous range of values that the oracle can provide as a response.
\end{itemize}

Our framework supports all of these possibilities.


Using the running example, the last column of \autoref{tab:ny_hotels_relevance} and each cell in \autoref{tab:ny_hotels_diversity} represents an oracle response of type $\mathcal{R}_1$. 
    
As an example, for the question $\mathcal{Q}(HYN, q, relevance)$, the oracle returns a discrete value $Relevance(HNY, q) = 0.5$. The [MIN, MAX] response values are assumed to be 0 and 1 respectively. At a given point in time, some responses might be unknown and are denoted by $U$ in these tables.
    
Alternatively, let us assume we wish to have responses of type $\mathcal{R}_2$. In this case, as an instance, for a question like $\mathcal{Q}(WLD, SHN, diversity)$, we could have $Diversity(WLD, SHN) = (0.3, 0.7)$ as the response which is a continuous range rather than a discrete value. 



\subsection{Problem Definition}
Given a scoring function $\mathcal{F}$, we define the set of all possible questions, denoted as \( Q_U \), as the union of decomposable constructs.

Using the running example,  \( Q_U \) contains relevance on each of the $5$ hotels in the data, plus a total of $\binom{5}{2}$ diversity questions on each pairs of hotels. At a given point in time, the response to some of these questions could be known. 

\textbf{Known Information.} Let \( Q_K \) be the set of questions whose  responses are known. Thus, we have:

\[
Q_K = \{ q \in Q_{ALL} \mid \text{response to } q \text{ is known} \}
\]

\textbf{Unknown Information.} Conversely, we can define the unknown information as the set of questions for which the responses have not yet been obtained. This set, denoted as \( Q_{ALL} \), is given by:

\[
Q_U = Q_{ALL} \setminus Q_K
\]

\begin{example}
    Let us assume there are five hotels as mentioned in \autoref{tab:ny_hotels_relevance}, and the scoring function is $\mathcal{F}_{\text{rel+div}}$. The response to each relevance and diversity question has been mapped to a cell in \autoref{tab:ny_hotels_relevance} and \autoref{tab:ny_hotels_diversity}, respectively. Cells marked as $U$ indicate an unknown value for that specific question. Hence, based on these two tables, we have:

\[
Q_K = \left\{
\begin{array}{lll}
R(\text{MLN}, q) & R(\text{WLD}, q) & R(\text{HNY}, q) \\
D(\text{HNY}, \text{MLN}) & D(\text{MLN}, \text{HNY}) & D(\text{SHN}, \text{WLD}) \\
D(\text{HYN}, \text{SHN}) & &
\end{array}
\right.
\]


\[
Q_U = \left\{
\begin{array}{lll}
R(\text{HYN}, q) & R(\text{SHN}, q) & D(\text{HYN}, \text{HNY}) \\
D(\text{HYN}, \text{MLN}) & D(\text{HYN}, \text{WLD}) & D(\text{HNY}, \text{HYN}) \\
D(\text{MLN}, \text{HYN}) & D(\text{SHN}, \text{HYN}) &
\end{array}
\right.
\]

We aim to find out what is the next best question to know which is going to be most useful in finding the query answer. 

\end{example}



\begin{table}[h!]
\centering
\begin{tabular}{|c|l|c|}
\hline
\textbf{Abbreviation} & \textbf{Hotel Name} & \textbf{Relevance} \\ \hline
HNY & Hilton NY      & U \\ \hline
MLN & Marriott LN    & 1.0 \\ \hline
HYN & Hyatt NY       & 1.0 \\ \hline
SHN & Sheraton NY    & 0.0 \\ \hline
WLD & Waldorf NY     & 0.5 \\ \hline
\end{tabular}
\caption{Hotels and their relevance to the query "Affordable hotels in Manhattan"}
\label{tab:ny_hotels_relevance}
\end{table}



\begin{table}[h!]
\centering
\begin{tabular}{|c|c|c|c|c|c|}
\hline
       & HNY  & MLN  & HYN  & SHN  & WLD  \\ \hline
HNY    & -   & 1.0  & 0.5    & 0.5  & 0.5  \\ \hline
MLN    & 1.0  & -    & U    & U  & U  \\ \hline
HYN    & 0.5    & U    & -    & U  & 0.5    \\ \hline
SHN    & 0.5  & U  & U  & -    & U  \\ \hline
WLD    & 0.5  & U  & 0.5   & U  & -    \\ \hline
\end{tabular}
\caption{Diversity scores between pairs of hotels. Some values are unknown.}
\label{tab:ny_hotels_diversity}
\end{table}





\textbf{Problem Definition:} Given \( n \) entities, k, an input query \( q \), and a user defined set based scoring function \( \mathcal{F} \), already known information \( Q_K \), identify the next question to be asked to the oracle that has the highest likelihood of identifying the true top-$k$ set.

\begin{example}
\label{example2.3}
    Given the running example, Assume we want to find the top-3 hotels with respect to the query "affordable hotels in Manhattan". Assume that at a given point in time, some information is already known which is shown in \autoref{tab:ny_hotels_relevance} and \autoref{tab:ny_hotels_diversity}. For simplicity, let's imagine the ultimate query answer is going to be chosen among either of the following three sets, meaning that we disregard any other possible sets rather than these:
    
    \[
    c_1 = \{\text{HNY, MLN, HYN}\}, 
\]
\[
    c_2 = \{\text{HNY, MLN, WLD}\},
\]
\[
    c_3 = \{\text{HNY, HYN, SHN}\}
\]

    Hence, we aim to find the next best question to ask which is most useful in finding the query answer, which is the top-k set with the highest score.
    
    Potentially, the next question could be any of the cells marked as U (Unknown) in \autoref{tab:ny_hotels_relevance} or \autoref{tab:ny_hotels_diversity}. For simplicity, let's consider two possible questions to ask, and compare how useful they can be:
    
    \begin{itemize}
        \item \( Q_1 \): \(D(MLN, HYN)\) - the diversity score between MLN and HYN
        \item \( Q_2 \): \(D(HYN, SHN)\) - the diversity score between HYN and SHN
    \end{itemize}

    Before asking any of these two questions, by using the known values have:

\[
\begin{aligned}
    \mathcal{F}(c_1, q) = 
    & R(\text{HNY}) + R(\text{MLN}) + R(\text{HYN}) + D(\text{HNY, MLN}) \\
    & + D(\text{HNY, HYN}) + D(\text{MLN, HYN}) \\
    = & 3.5 + R(\text{HNY}) + D(\text{MLN, HYN}) \\
\end{aligned}
\]

\[
\begin{aligned}
    \mathcal{F}(c_2, q) = 
    & R(\text{HNY}) + R(\text{MLN}) + R(\text{WLD}) + D(\text{HNY, MLN}) \\
    & + D(\text{HNY, WLD}) + D(\text{MLN, WLD}) \\
    = & 3.0 + R(\text{HNY}) + D(\text{MLN, WLD}) 
\end{aligned}
\]


\[
\begin{aligned}
    \mathcal{F}(c_3, q) = 
    & R(\text{HNY}) + R(\text{HYN}) + R(\text{SHN}) + D(\text{HNY, HYN}) \\
    & + D(\text{HNY, SHN}) + D(\text{HYN, SHN}) \\
    = & 2.0 + R(\text{HNY}) + D(\text{HYN, SHN}) 
\end{aligned}
\]



    
    Suppose we choose to ask \( Q_1 \), revealing that \(D(MLN, HYN) = 1.0 \). Knowing this additional information, we update the overall scores of \( c_1 \) as follows:
    
\[
\begin{aligned}
    \mathcal{F}(c_1, q) = 
     & 4.5 + R(\text{HNY}) \\
\end{aligned}
\]

    Since all the known/unknown values are within a pre-specified range (In this example: (0, 1)), regardless of the values of remaining unknown variables in $\mathcal{F}(c_2, q)$, $\mathcal{F}(c_1, q)$, and $\mathcal{F}(c_1, q)$, we will have:

    \[
\begin{aligned}
    \mathcal{F}(c_1, q) \geq \mathcal{F}(c_2, q) \quad \text{AND} \quad
    \mathcal{F}(c_1, q) \geq \mathcal{F}(c_3, q)
\end{aligned}
\]

    Hence, in this case, we can conclude that $c_1$ is the query answer. By asking \(D(MLN, HYN)\) from expert, we could potentially get some other answer rather than $1.0$ as well. However, in this example, if \(D(MLN, HYN) \geq 0.5 \), we could still come up with the same conclusion as we did. Therefore, asking \(D(MLN, HYN)\) can likely result in finding the top-3 set.  

    On the other hand, Suppose we choose to ask \( Q_2 \), resulting in \(D(MLN, HYN) = 1.0 \). Knowing this additional information, we update the overall scores of \( c_3 \) as follows:
    
    \[
    \begin{aligned}
        \mathcal{F}(c_3, q) = 
         & 3.0 + R(\text{HNY}) \\
    \end{aligned}
    \]

    In this case, the top-3 set is not clear yet. In more detail, based on different possible values of $D(MLN, HYN)$ and $D(HYN, SHN)$ corresponding to $c_1$ and $c_2$ respectively, any of them could have higher final score than the other. Hence, asking $Q_2$ will determine the top-k set. 

    Generally, At a given point in time, there might be more than one question that could likely determine the top-k set. On the other hand, there might be no question that could determine the top-k set at a specific time, but choosing one question over another might result in fewer number of required questions to ask in the future for finding out the top-k set. 
        
    Hence, it is crucial at each point in time to ask the question that is most likely going to reduce the uncertainty in finding the top-k set and guide us toward identifying the query answer.
\end{example}


%In Section ~\ref{sec:section2.3}, we introduce each component in our proposed framework to solve this problem.



