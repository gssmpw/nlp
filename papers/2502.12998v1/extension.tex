\section{Extensions}\label{sec:ext}
This work opens up several intriguing avenues for future exploration, some of which we discuss below. 

\noindent {\bf Asking Multiple Questions.}
Our approach focused on selecting the next best question. However, an alternative approach would involve determining the next best set of $k'$ questions. This would necessitate adapting the second and third tasks in our framework. We are actively pursuing this extension.

\noindent {\bf Response Processing.}
We assumed a single discrete response from one oracle. However, we recognize that other possibilities exist, such as:
a. An oracle providing a range of values rather than a discrete response,
b. Multiple oracles each offering a discrete response,
c. Multiple oracles each providing a range of responses.

For case (a), this could be treated as a uniform probability distribution, and we could adapt the bound computation task to handle such distributions. The rest of the framework would not require significant changes. Handling (b) and (c) presents additional challenges. As an example, they could be treated as a problem of combining multiple probability distributions, and classical techniques, such as, convolution of probability distributions~\cite{olds1952note} could be used. Alternatively, machine learning methods such as Reinforcement Learning with Human Feedback (RLHF)~\cite{rlhf1, rlhf2, rlhf3, rlhf4, rlhf5} could also be applied. We plan to explore these possibilities in future work.

\noindent {\bf Multiple Noisy Oracles.}
An interesting extension involves the scenario where multiple noisy oracles are present, each with different costs associated with querying. This would require a substantial modification of our framework. Instead of minimizing the number of questions asked, the focus would shift to minimizing the total cost of querying while still maximizing the likelihood of correctly identifying the top-$k$ entities.

\noindent {\bf Rank-based Querying.}
Our framework is designed to identify the top-$k$ entities based on a user-defined scoring function. A natural next step is to extend this to determine the top-$k$ entities in a ranked order. Adapting our framework to handle ranked queries would involve modifications to the first three tasks. We hope to explore this extension as part of our future work as well.
