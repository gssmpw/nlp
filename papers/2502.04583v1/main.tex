%%%%%%%% ICML 2025 EXAMPLE LATEX SUBMISSION FILE %%%%%%%%%%%%%%%%%

\documentclass{article}

% Recommended, but optional, packages for figures and better typesetting:
\usepackage{microtype}
\usepackage{graphicx}
\usepackage{subfigure}
\usepackage{subcaption}
\usepackage{booktabs} % for professional tables

% hyperref makes hyperlinks in the resulting PDF.
% If your build breaks (sometimes temporarily if a hyperlink spans a page)
% please comment out the following usepackage line and replace
% \usepackage{icml2025} with \usepackage[nohyperref]{icml2025} above.
\usepackage{hyperref}


% Attempt to make hyperref and algorithmic work together better:
\newcommand{\theHalgorithm}{\arabic{algorithm}}

% Use the following line for the initial blind version submitted for review:
% \usepackage{icml2025}

% If accepted, instead use the following line for the camera-ready submission:
\usepackage[accepted]{icml2025}

% For theorems and such
\usepackage{amsmath}
\usepackage{amssymb}
\usepackage{mathtools}
\usepackage{amsthm}

% if you use cleveref..
\usepackage[capitalize,noabbrev]{cleveref}

%%%%%%%%%%%%%%%%%%%%%%%%%%%%%%%%
% THEOREMS
%%%%%%%%%%%%%%%%%%%%%%%%%%%%%%%%
\theoremstyle{plain}
\newtheorem{theorem}{Theorem}[section]
\newtheorem{proposition}[theorem]{Proposition}
\newtheorem{lemma}[theorem]{Lemma}
\newtheorem{corollary}[theorem]{Corollary}
\theoremstyle{definition}
\newtheorem{definition}[theorem]{Definition}
\newtheorem{assumption}[theorem]{Assumption}
\theoremstyle{remark}
\newtheorem{remark}[theorem]{Remark}

% Todonotes is useful during development; simply uncomment the next line
%    and comment out the line below the next line to turn off comments
%\usepackage[disable,textsize=tiny]{todonotes}
\usepackage[textsize=tiny]{todonotes}


%%%%%%%%%%%%%%%%%%%%%%%%%%%%%%%%
% Custom Package
%%%%%%%%%%%%%%%%%%%%%%%%%%%%%%%%
\newcommand{\mathbold}[1]{\ensuremath{\boldsymbol{\mathbf{#1}}}}

% # PROBABILITY
\newcommand{\g}{\,|\,}
% \renewcommand{\gg}{\,\|\,}
\renewcommand{\d}[1]{\ensuremath{\operatorname{d}\!{#1}}}
\newcommand{\nestedmathbold}[1]{{\mathbold{#1}}}

% # BOLD MATHEMATICS

\newcommand{\mba}{\nestedmathbold{a}}
\newcommand{\mbb}{\nestedmathbold{b}}
\newcommand{\mbc}{\nestedmathbold{c}}
\newcommand{\mbd}{\nestedmathbold{d}}
\newcommand{\mbe}{\nestedmathbold{e}}
\newcommand{\mbf}{\nestedmathbold{f}}
\newcommand{\mbg}{\nestedmathbold{g}}
\newcommand{\mbh}{\nestedmathbold{h}}
\newcommand{\mbi}{\nestedmathbold{i}}
\newcommand{\mbj}{\nestedmathbold{j}}
\newcommand{\mbk}{\nestedmathbold{k}}
\newcommand{\mbl}{\nestedmathbold{l}}
\newcommand{\mbm}{\nestedmathbold{m}}
\newcommand{\mbn}{\nestedmathbold{n}}
\newcommand{\mbo}{\nestedmathbold{o}}
\newcommand{\mbp}{\nestedmathbold{p}}
\newcommand{\mbq}{\nestedmathbold{q}}
\newcommand{\mbr}{\nestedmathbold{r}}
\newcommand{\mbs}{\nestedmathbold{s}}
\newcommand{\mbt}{\nestedmathbold{t}}
\newcommand{\mbu}{\nestedmathbold{u}}
\newcommand{\mbv}{\nestedmathbold{v}}
\newcommand{\mbw}{\nestedmathbold{w}}
\newcommand{\mbx}{\nestedmathbold{x}}
\newcommand{\mby}{\nestedmathbold{y}}
\newcommand{\mbz}{\nestedmathbold{z}}

\newcommand{\mbA}{\nestedmathbold{A}}
\newcommand{\mbB}{\nestedmathbold{B}}
\newcommand{\mbC}{\nestedmathbold{C}}
\newcommand{\mbD}{\nestedmathbold{D}}
\newcommand{\mbE}{\nestedmathbold{E}}
\newcommand{\mbF}{\nestedmathbold{F}}
\newcommand{\mbG}{\nestedmathbold{G}}
\newcommand{\mbH}{\nestedmathbold{H}}
\newcommand{\mbI}{\nestedmathbold{I}}
\newcommand{\mbJ}{\nestedmathbold{J}}
\newcommand{\mbK}{\nestedmathbold{K}}
\newcommand{\mbL}{\nestedmathbold{L}}
\newcommand{\mbM}{\nestedmathbold{M}}
\newcommand{\mbN}{\nestedmathbold{N}}
\newcommand{\mbO}{\nestedmathbold{O}}
\newcommand{\mbP}{\nestedmathbold{P}}
\newcommand{\mbQ}{\nestedmathbold{Q}}
\newcommand{\mbR}{\nestedmathbold{R}}
\newcommand{\mbS}{\nestedmathbold{S}}
\newcommand{\mbT}{\nestedmathbold{T}}
\newcommand{\mbU}{\nestedmathbold{U}}
\newcommand{\mbV}{\nestedmathbold{V}}
\newcommand{\mbW}{\nestedmathbold{W}}
\newcommand{\mbX}{\nestedmathbold{X}}
\newcommand{\mbY}{\nestedmathbold{Y}}
\newcommand{\mbZ}{\nestedmathbold{Z}}

\newcommand{\mbalpha}{\nestedmathbold{\alpha}}
\newcommand{\mbbeta}{\nestedmathbold{\beta}}
\newcommand{\mbdelta}{\nestedmathbold{\delta}}
\newcommand{\mbepsilon}{\nestedmathbold{\epsilon}}
\newcommand{\mbchi}{\nestedmathbold{\chi}}
\newcommand{\mbeta}{\nestedmathbold{\eta}}
\newcommand{\mbgamma}{\nestedmathbold{\gamma}}
\newcommand{\mbiota}{\nestedmathbold{\iota}}
\newcommand{\mbkappa}{\nestedmathbold{\kappa}}
\newcommand{\mblambda}{\nestedmathbold{\lambda}}
\newcommand{\mbmu}{\nestedmathbold{\mu}}
\newcommand{\mbnu}{\nestedmathbold{\nu}}
\newcommand{\mbomega}{\nestedmathbold{\omega}}
\newcommand{\mbphi}{\nestedmathbold{\phi}}
\newcommand{\mbpi}{\nestedmathbold{\pi}}
\newcommand{\mbpsi}{\nestedmathbold{\psi}}
\newcommand{\mbrho}{\nestedmathbold{\rho}}
\newcommand{\mbsigma}{\nestedmathbold{\sigma}}
\newcommand{\mbtau}{\nestedmathbold{\tau}}
\newcommand{\mbtheta}{\nestedmathbold{\theta}}
\newcommand{\mbupsilon}{\nestedmathbold{\upsilon}}
\newcommand{\mbvarepsilon}{\nestedmathbold{\varepsilon}}
\newcommand{\mbvarphi}{\nestedmathbold{\varphi}}
\newcommand{\mbvartheta}{\nestedmathbold{\vartheta}}
\newcommand{\mbvarrho}{\nestedmathbold{\varrho}}
\newcommand{\mbxi}{\nestedmathbold{\xi}}
\newcommand{\mbzeta}{\nestedmathbold{\zeta}}

\newcommand{\mbDelta}{\nestedmathbold{\Delta}}
\newcommand{\mbGamma}{\nestedmathbold{\Gamma}}
\newcommand{\mbLambda}{\nestedmathbold{\Lambda}}
\newcommand{\mbOmega}{\nestedmathbold{\Omega}}
\newcommand{\mbPhi}{\nestedmathbold{\Phi}}
\newcommand{\mbPi}{\nestedmathbold{\Pi}}
\newcommand{\mbPsi}{\nestedmathbold{\Psi}}
\newcommand{\mbSigma}{\nestedmathbold{\Sigma}}
\newcommand{\mbTheta}{\nestedmathbold{\Theta}}
\newcommand{\mbUpsilon}{\nestedmathbold{\Upsilon}}
\newcommand{\mbXi}{\nestedmathbold{\Xi}}

\newcommand{\mbzero}{\nestedmathbold{0}}
\newcommand{\mbone}{\nestedmathbold{1}}
\newcommand{\mbtwo}{\nestedmathbold{2}}
\newcommand{\mbthree}{\nestedmathbold{3}}
\newcommand{\mbfour}{\nestedmathbold{4}}
\newcommand{\mbfive}{\nestedmathbold{5}}
\newcommand{\mbsix}{\nestedmathbold{6}}
\newcommand{\mbseven}{\nestedmathbold{7}}
\newcommand{\mbeight}{\nestedmathbold{8}}
\newcommand{\mbnine}{\nestedmathbold{9}}

% # MISCELLANEOUS

\newcommand{\ELBO}{\textsc{elbo}}
\newcommand{\GELBO}{\textsc{gelbo}}
\newcommand{\scH}{\textsc{h}}
\DeclareRobustCommand{\KL}[2]{\ensuremath{\textsc{kl}\left[#1\;\|\;#2\right]}}
\DeclareRobustCommand{\DV}[2]{\ensuremath{\textsc{dv}\left[#1\;\|\;#2\right]}}
\DeclareRobustCommand{\Df}[2]{\ensuremath{\mathcal{D}_f\left[#1\;\|\;#2\right]}}

\newcommand{\diag}{\textrm{diag}}
\newcommand{\supp}{\textrm{supp}}
\DeclareMathOperator*{\argmax}{arg\,max}
\DeclareMathOperator*{\argmin}{arg\,min}
\newcommand\indep{\protect\mathpalette{\protect\independenT}{\perp}}
\def\independenT#1#2{\mathrel{\rlap{$#1#2$}\mkern2mu{#1#2}}}

\newcommand{\cD}{\mathcal{D}}
\newcommand{\cL}{\mathcal{L}}
\newcommand{\cN}{\mathcal{N}}
\newcommand{\cP}{\mathcal{P}}
\newcommand{\cQ}{\mathcal{Q}}
\newcommand{\cR}{\mathcal{R}}
\newcommand{\cF}{\mathcal{F}}
\newcommand{\cI}{\mathcal{I}}
\newcommand{\cT}{\mathcal{T}}
\newcommand{\cV}{\mathcal{V}}
\newcommand{\cE}{\mathcal{E}}
\newcommand{\cG}{\mathcal{G}}
\newcommand{\cH}{\mathcal{H}}
\newcommand{\cY}{\mathcal{Y}}

\newcommand{\E}{\mathbb{E}}
\newcommand{\bbH}{\mathbb{H}}
\newcommand{\bbR}{\mathbb{R}}
\newcommand{\bbC}{\mathbb{C}}
\newcommand{\bbV}{\mathbb{V}}
\newcommand{\bbG}{\mathbb{G}}

 % GP stuff
\newcommand{\bigO}{\mathcal{O}}
\newcommand{\GP}{{\mathcal{GP}}}

% Distributions
\newcommand{\Poisson}{\text{Poisson}}
\usepackage{xcolor}         % colors
% \usepackage{wrapfig}
% \usepackage{mathtools}
% \usepackage{amsthm}
\usepackage{enumitem}
% \usepackage{subcaption}
% \usepackage{natbib}
% %\usepackage{cleveref}
% \usepackage[capitalize,noabbrev]{cleveref}
% \usepackage{algorithm}
% % \usepackage{algpseudocode}
% \usepackage{multirow}
% \usepackage{soul}    % 취소선을 넣기위해 사용되는 일시적인 패키지, st{text}를 하면, text에 취소선을 그을 수 있음. 
% \usepackage{thmtools} 
% \usepackage{thm-restate}
\usepackage{kotex}

\providecommand{\grad}{\operatorname{grad}}

\providecommand{\blue}{\textcolor{blue}}
%\providecommand{\red}{\textcolor{black}}
\providecommand{\red}{\textcolor{red}}
\providecommand{\brown}{\textcolor{brown}}
\providecommand{\purple}{\textcolor{purple}}

\usepackage{cancel}
\newcommand{\jm}[1]{{\scriptsize\textbf{\color{magenta} JM: #1}}}
\newcommand{\jw}[1]{{\scriptsize\textbf{\color{red} JW: #1}}}
\newcommand{\dk}[1]{{\scriptsize\textbf{\color{purple} DK: #1}}}


% The \icmltitle you define below is probably too long as a header.
% Therefore, a short form for the running title is supplied here:
\icmltitlerunning{Overcoming Fake Solutions in Semi-Dual Neural Optimal Transport}

\begin{document}

\twocolumn[
\icmltitle{
%Guaranteeing the Learning of Optimal Transport Maps in Semi-Dual Neural Optimal Transport
Overcoming Fake Solutions in Semi-Dual Neural Optimal Transport: \\
A Smoothing Approach for Learning the Optimal Transport Plan
}

% It is OKAY to include author information, even for blind
% submissions: the style file will automatically remove it for you
% unless you've provided the [accepted] option to the icml2025
% package.

% List of affiliations: The first argument should be a (short)
% identifier you will use later to specify author affiliations
% Academic affiliations should list Department, University, City, Region, Country
% Industry affiliations should list Company, City, Region, Country

% You can specify symbols, otherwise they are numbered in order.
% Ideally, you should not use this facility. Affiliations will be numbered
% in order of appearance and this is the preferred way.

\icmlsetsymbol{equal}{*}

\begin{icmlauthorlist}
\icmlauthor{Jaemoo Choi}{gatech}
\icmlauthor{Jaewoong Choi}{equal,skku}
\icmlauthor{Dohyun Kwon}{equal,uos,kias}
%\icmlauthor{}{sch}
% \icmlauthor{Firstname8 Lastname8}{sch}
% \icmlauthor{Firstname8 Lastname8}{yyy,comp}
%\icmlauthor{}{sch}
%\icmlauthor{}{sch}
\end{icmlauthorlist}

\icmlaffiliation{gatech}{Georgia Institute of Technology}
\icmlaffiliation{skku}{Sungkyunkwan University}
\icmlaffiliation{uos}{University of Seoul}
\icmlaffiliation{kias}{Korea Institute for Advanced Study}

\icmlcorrespondingauthor{Dohyun Kwon}{dh.dohyun.kwon@gmail.com}
\icmlcorrespondingauthor{Jaewoong Choi}{jaewoongchoi@skku.edu}

% You may provide any keywords that you
% find helpful for describing your paper; these are used to populate
% the "keywords" metadata in the PDF but will not be shown in the document
\icmlkeywords{Machine Learning, ICML}

\vskip 0.3in
]

% this must go after the closing bracket ] following \twocolumn[ ...

% This command actually creates the footnote in the first column
% listing the affiliations and the copyright notice.
% The command takes one argument, which is text to display at the start of the footnote.
% The \icmlEqualContribution command is standard text for equal contribution.
% Remove it (just {}) if you do not need this facility.

% \printAffiliationsAndNotice{}  % leave blank if no need to mention equal contribution

\printAffiliationsAndNotice{\icmlEqualContribution} % otherwise use the standard text.

\begin{abstract}
We address the convergence problem in learning the Optimal Transport (OT) map, where the OT Map refers to a map from one distribution to another while minimizing the transport cost. Semi-dual Neural OT, a widely used approach for learning OT Maps with neural networks, often generates fake solutions that fail to transfer one distribution to another accurately. We identify a sufficient condition under which the max-min solution of Semi-dual Neural OT recovers the true OT Map. Moreover, to address cases when this sufficient condition is not satisfied, we propose a novel method, OTP, which learns both the OT Map and the Optimal Transport Plan, representing the optimal coupling between two distributions. Under sharp assumptions on the distributions, we prove that our model eliminates the fake solution issue and correctly solves the OT problem. Our experiments show that the OTP model recovers the optimal transport map where existing methods fail and outperforms current OT-based models in image-to-image translation tasks. Notably, the OTP model can learn stochastic transport maps when deterministic OT Maps do not exist, such as one-to-many tasks like colorization.
\end{abstract}

In this section, we introduce DAB: \textbf{D}iscrete \textbf{A}uto-regressive \textbf{B}iasing. 
First, we present the formulation of the target distribution as a joint distribution and explain the motivation behind this approach. Next, we describe how our algorithm samples from the joint distribution by alternating between biased auto-regressive generation and discrete gradient-based sampling. We finally demonstrate that gradient-based discrete sampling enables our algorithm to have more thorough, stable, and efficient sampling when compared to continuous methods. 
\subsection{Formulation}
\section{Auxiliary-Variable Adaptive Control Barrier Functions}
\label{sec:AVBCBF}

In this section, we introduce Auxiliary-Variable Adaptive Control Barrier Functions (AVCBFs) for safety-critical control.
We start with a simple example to motivate the need for AVCBFs.

\subsection{Motivation Example: Simplified Adaptive Cruise Control}
\label{subsec:SACC-problem}

Consider a Simplified Adaptive Cruise Control (SACC) problem with the dynamics of ego vehicle expressed as 
\begin{small}
\begin{equation}
\label{eq:SACC-dynamics}
\underbrace{\begin{bmatrix}
\dot{z}(t) \\
\dot{v}(t) 
\end{bmatrix}}_{\dot{\boldsymbol{x}}(t)}  
=\underbrace{\begin{bmatrix}
 v_{p}-v(t) \\
 0
\end{bmatrix}}_{f(\boldsymbol{x}(t))} 
+ \underbrace{\begin{bmatrix}
  0 \\
  1 
\end{bmatrix}}_{g(\boldsymbol{x}(t))}u(t),
\end{equation}
\end{small}
where $v_{p}>0, v(t)>0$ denote the velocity of the lead vehicle (constant velocity) and ego vehicle, respectively, $z(t)$ denotes the distance between the lead and ego vehicle and $u(t)$ denotes the acceleration (control) of ego vehicle, subject to the control constraints
\begin{equation}
\label{eq:simple-control-constraint}
u_{min}\le u(t) \le u_{max}, \forall t \ge0,
\end{equation}
where $u_{min}<0$ and $u_{max}>0$ are the minimum and maximum control input, respectively.

 For safety, we require that $z(t)$ always be greater than or equal to the safety distance denoted by $l_{p}>0,$ i.e., $z(t)\ge l_{p}, \forall t \ge 0.$ Based on Def. \ref{def:HOCBF}, let $\psi_{0}(\boldsymbol{x})\coloneqq b(\boldsymbol{x})=z(t)-l_{p}.$ From \eqref{eq:sequence-f1} and \eqref{eq:sequence-set1}, since the relative degree of $b(\boldsymbol{x})$ is 2, we have
\begin{equation}
\label{eq:SACC-HOCBF-sequence}
\begin{split}
&\psi_{1}(\boldsymbol{x})\coloneqq v_{p}-v(t)+k_{1}\psi_{0}(\boldsymbol{x})\ge 0
,\\
&\psi_{2}(\boldsymbol{x})\coloneqq -u(t)+k_{1}(v_{p}-v(t))+k_{2}\psi_{1}(\boldsymbol{x})\ge 0,
\end{split}
\end{equation}
where $\alpha_{1}(\psi_{0}(\boldsymbol{x}))\coloneqq k_{1}\psi_{0}(\boldsymbol{x}), \alpha_{2}(\psi_{1}(\boldsymbol{x}))\coloneqq k_{2}\psi_{1}(\boldsymbol{x}), k_{1}>0, k_{2}>0.$ The constant class $\kappa$ coefficients $k_{1},k_{2}$ are always chosen small to equip ego vehicle with a conservative control strategy to keep it safe, i.e., smaller $k_{1},k_{2}$ make ego vehicle brake earlier (see \cite{xiao2021high}). Suppose we wish to minimize the energy cost as $\int_{0}^{T} u^{2}(t)dt.$ We can then formulate the cost in the QP with constraint $\psi_{2}(\boldsymbol{x})\ge0$ and control input constraint \eqref{eq:simple-control-constraint} to get the optimal controller for the SACC problem. However, the feasible set of input can easily become empty if $u(t)\le k_{1}(v_{p}-v(t))+k_{2}\psi_{1}(\boldsymbol{x})<u_{min}$,  which causes infeasibility of the optimization. In \cite{xiao2021adaptive}, the authors introduced penalty variables in front of class $\kappa$ functions to enhance the feasibility. This approach defines $\psi_{0}(\boldsymbol{x})\coloneqq b(\boldsymbol{x})=z(t)-l_{p}$ as PACBF and other constraints can be further defined as
\begin{equation}
\label{eq:SACC-PACBF-sequence}
\begin{split}
\psi_{1}(\boldsymbol{x},p_{1}(t))&\coloneqq v_{p}-v(t)+p_{1}(t)k_{1}\psi_{0}(\boldsymbol{x})\ge 0,\\
\psi_{2}(\boldsymbol{x},p_{1}(t),&\boldsymbol{\nu})\coloneqq \nu_{1}(t)k_{1}\psi_{0}(\boldsymbol{x})+p_{1}(t)k_{1}(v_{p}\\
-v(t))&-u(t)+\nu_{2}(t)k_{2}\psi_{1}(\boldsymbol{x},p_{1}(t))\ge 0,
\end{split}
\end{equation}
where $\nu_{1}(t)=\dot{p}_{1}(t),\nu_{2}(t)=p_{2}(t), p_{1}(t)\ge0,p_{2}(t)\ge0,\boldsymbol{\nu}=(\nu_{1}(t),\nu_{2}(t)).$ $p_{1}(t),p_{2}(t)$ are time-varying penalty variables, which alleviate the conservativeness of the control strategy and $\nu_{1}(t),\nu_{2}(t)$ are auxiliary inputs, which relax the constraints for $u(t)$ in $\psi_{2}(\boldsymbol{x},p_{1}(t),\boldsymbol{\nu})\ge0$ and \eqref{eq:simple-control-constraint}. However, in practice, we need to define several additional constraints to make PACBF valid as shown in Eqs. (24)-(27) in \cite{xiao2021adaptive}. First, we need to define HOCBFs ($b_{1}(p_{1}(t))=p_{1}(t),b_{2}(p_{2}(t))=p_{2}(t))$ based on Def. \ref{def:HOCBF} to ensure $p_{1}(t)\ge0,p_{2}(t)\ge0.$ Next we need to define HOCBF ($b_{3}(p_{1}(t))=p_{1,max}-p_{1}(t)$) to confine the value of $p_{1}(t)$ in the range $[0,p_{1,max}].$ We also need to define CLF ($V(p_{1}(t))=(p_{1}(t)-p_{1}^{\ast})^{2}$) based on Def. \ref{def:control-l-f} to keep $p_{1}(t)$ close to a small value $p_{1}^{\ast}.$ $b_{3}(p_{1}(t)), V(p_{1}(t))$ are necessary since $\psi_{0}(\boldsymbol{x},p_{1}(t))\coloneqq p_{1}(t)k_{1}\psi_{0}(\boldsymbol{x})$ in first constraint in \eqref{eq:SACC-PACBF-sequence} is not a class $\kappa$ function with respect to $\psi_{0}(\boldsymbol{x}),$ i.e., $p_{1}(t)k_{1}\psi_{0}(\boldsymbol{x})$ is not guaranteed to strictly increase since $\psi_{0}(\boldsymbol{x},p_{1}(t))$ is in fact a class $\kappa$ function with respect to $p_{1}(t)\psi_{0}(\boldsymbol{x})$, which is against Thm. \ref{thm:safety-guarantee}, therefore $\psi_{1}(\boldsymbol{x},p_{1}(t))\ge 0$ in \eqref{eq:SACC-PACBF-sequence} may not guarantee $\psi_{0}(\boldsymbol{x})\ge 0.$ This illustrates why we have to limit the growth of $p_{1}(t)$ by defining $b_{3}(p_{1}(t)),V(p_{1}(t)).$ However, the way to choose appropriate values for $p_{1,max},p_{1}^{\ast}$ is not straightforward. We can imagine as the relative degree of $b(\boldsymbol{x})$ gets higher, the number of additional constraints we should define also gets larger, which results in complicated parameter-tuning process. To address this issue, we introduce $a_{1}(t),a_{2}(t)$ in the form
\begin{small}
\begin{equation}
\label{eq:SACC-AVBCBF-sequence}
\begin{split}
\psi_{1}(\boldsymbol{x},\boldsymbol{a},\dot{a}_{1}(t))\coloneqq a_{2}(t)(\dot{\psi}_{0}(\boldsymbol{x},a_{1}(t))
+k_{1}\psi_{0}(\boldsymbol{x},a_{1}(t)))\ge 0,\\
\psi_{2}(\boldsymbol{x},\boldsymbol{a},\dot{a}_{1}(t),\boldsymbol{\nu})\coloneqq \nu_{2}(t)\frac{\psi_{1}(\boldsymbol{x},\boldsymbol{a},\dot{a}_{1}(t))}{a_{2}(t)} +a_{2}(t)(\nu_{1}(t)(z(t)\\
-l_{p})+2\dot{a}_{1}(t)(v_{p}-v(t))-a_{1}(t)u(t)+k_{1}\dot{\psi}_{0}(\boldsymbol{x},a_{1}(t)))\\
+k_{2}\psi_{1}(\boldsymbol{x},\boldsymbol{a},\dot{a}_{1}(t))\ge 0, 
\end{split}
\end{equation}
\end{small}
where $\psi_{0}(\boldsymbol{x},a_{1}(t))\coloneqq a_{1}(t)b (\boldsymbol{x})=a_{1}(t)(z(t)-l_{p}),\boldsymbol{\nu}=[\nu_{1}(t),\nu_{2}(t)]^{T}=[\ddot{a}_{1}(t),\dot{a}_{2}(t)]^{T},\boldsymbol{a}=[a_{1}(t),a_{2}(t)]^{T},$ $a_{1}(t),a_{2}(t)$ are time-varying auxiliary variables. Since $\psi_{0}(\boldsymbol{x},a_{1}(t))\ge0,\psi_{1}(\boldsymbol{x},\boldsymbol{a},\dot{a}_{1}(t))\ge 0$ will not be against $b(\boldsymbol{x})\ge 0,\dot{\psi}_{0}(\boldsymbol{x},a_{1}(t))
+k_{1}\psi_{0}(\boldsymbol{x},a_{1}(t))\ge 0$ iff $a_{1}(t)>0,a_{2}(t)>0,$ we need to define HOCBFs for auxiliary variables to make $a_{1}(t)>0,a_{2}(t)>0,$ which will be illustrated in Sec. \ref{sec:AVCBFs}.  $\nu_{1}(t),\nu_{2}(t)$ are auxiliary inputs which are used to alleviate the restriction of constraints for $u(t)$ in $\psi_{2}(\boldsymbol{x},\boldsymbol{a},\dot{a}_{1}(t),\boldsymbol{\nu})\ge0$ and \eqref{eq:simple-control-constraint}. Different from the first constraint in \eqref{eq:SACC-PACBF-sequence}, $k_{1}\psi_{0}(\boldsymbol{x},a_{1}(t))$ is still a class $\kappa$ function with respect to $\psi_{0}(\boldsymbol{x},a_{1}(t)),$ therefore we do not need to define additional HOCBFs and CLFs like $b_{3}(p_{1}(t)),V(p_{1}(t))$ to limit the growth of $a_{1}(t).$
We can rewrite $\psi_{1} (\boldsymbol{x},\boldsymbol{a},\dot{a}_{1}(t))$ in \eqref{eq:SACC-AVBCBF-sequence} as
\begin{equation}
\label{eq:SACC-AVBCBF-sequence-rewrite}
\begin{split}
\psi_{1}(\boldsymbol{x},\boldsymbol{a},\dot{a}_{1}(t))\coloneqq a_{2}(t)a_{1}(t)(v_{p}-v(t)\\
+k_{1}(1+\frac{\dot{a}_{1}(t)}{k_{1}a_{1}(t)})b(\boldsymbol{x}))\ge 0.
\end{split}
\end{equation}
Compared to the first constraint in \eqref{eq:SACC-HOCBF-sequence}, $\frac{\dot{a}_{1}(t)}{a_{1}(t)}$ is a time-varying auxiliary term to alleviate the conservativeness of control that small $k_{1}$ originally has, which shows the adaptivity of auxiliary terms to constant class $\kappa$ coefficients. 

% There is another type of adaptive CBFs called Relaxation-Adaptive Control Barrier Functions (RACBFs) in \cite{xiao2021adaptive}. The RACBF $b(\boldsymbol{x})$ is in the form:
% \begin{equation}
% \label{eq:RACBF}
% \psi_{0}(\boldsymbol{x},r(t))\coloneqq b(\boldsymbol{x})-r(t),
% \end{equation}
% where $r(t)\ge0$ is a relaxation that plays the similar role as Backup policy introduced in \cite{chen2021backup} {\color{red} How a relaxation is related to the backup policy?}. However, it is difficult for us to find the appropriate backup policy for controller of complicated dynamic system. Two main drawbacks affect the performance of RACBFs. {\color{red}wording} In the first place, $r(t)$ contracts the coverage of feasible space of states defined by $b(\boldsymbol{x})\ge0$, i.e., the distance $z(t)$ allowable for two vehicles is even smaller {\color{red}This should be larger} by $z(t)-l_{p}-r(t)\ge0$ because of the existence of non-negative $r(t)$. Secondly, the feasibility of solving QP with RACBF constraints is limited by the existence of upper bound of auxiliary input $\nu_{r}(t)$ related to $r(t)$ defined in Eq. (29) in \cite{xiao2021adaptive} {\color{red}What is $\nu_r$? you should make it self-contained.}. We can define the highest order {\color{red}what is this?} of $r(t)$ to be 2, then from \eqref{eq:SACC-HOCBF-sequence} normally we have
% \begin{equation}
% \label{eq:highest-order-RACBF}
% \begin{split}
% \psi_{2}(\boldsymbol{x},r(t),\dot{r}(t),\nu_{r}(t))\coloneqq -u(t)-\nu_{r}(t)\\
% +k_{1}(v_{p}-v(t)-\dot{r}(t))+k_{2}(v_{p}-v(t)-\dot{r}(t)\\
% +k_{1}(z(t)-l_{p}-r(t))\ge0, \nu_{r}(t)=\ddot{r}(t),
% \end{split}
% \end{equation}
% which sets the upper bound {\color{red}This is not clear} for $\nu_{r}(t)$ and there will easily be empty feasible set for $\nu_{r}(t)$ if the lower bound of $\nu_{r}(t)$ defined by constraint (31) in \cite{xiao2021adaptive} is too large. Compared to RACBFs, AVCBFs will neither contract the feasible space of states, nor set the upper bound for $\boldsymbol{\nu}$ (at least no upper bound for $\nu_{1}(t))$ as shown in the proof of Thm. \ref{thm:feasibility-guarantee} in Sec. \ref{subsec: optimal-control}, which shows the great benefits of AVCBFs in terms of safety and feasibility. 

% \subsection{HOCBFs for Auxiliary Coefficients}
\subsection{Adaptive HOCBFs for Safety:\ AVCBFs}
\label{sec:AVCBFs}

Motivated by the SACC example in Sec. \ref{subsec:SACC-problem}, given a function $b:\mathbb{R}^{n}\to\mathbb{R}$ with relative degree $m$ for system \eqref{eq:affine-control-system} and a time-varying vector $\boldsymbol{a}(t)\coloneqq [a_{1}(t),\dots,a_{m}(t)]^{T}$ with positive components called auxiliary variables, the key idea in converting a regular HOCBF into an adaptive
one without defining excessive constraints is to place one auxiliary variable in front of each function in \eqref{eq:sequence-f1} similar to \eqref{eq:SACC-AVBCBF-sequence}. 
As described in Sec. \ref{subsec:SACC-problem}, we only need to define HOCBFs for auxiliary variables to ensure each $a_{i}(t)>0, i \in \{1,...,m\}.$ To realize this, we need to define auxiliary systems that contain auxiliary states $\boldsymbol{\pi}_{i}(t)$ and inputs $\nu_{i}(t)$, through which systems we can extend each HOCBF to desired relative degree, just like $b(\boldsymbol{x})$ has relative degree $m$
with respect to the dynamics \eqref{eq:affine-control-system}. Consider $m$ auxiliary systems in the form 
\begin{equation}
\label{eq:virtual-system}
\dot{\boldsymbol{\pi}}_{i}=F_{i}(\boldsymbol{\pi}_{i})+G_{i}(\boldsymbol{\pi}_{i})\nu_{i}, i \in \{1,...,m\},
\end{equation}
where $\boldsymbol{\pi}_{i}(t)\coloneqq [\pi_{i,1}(t),\dots,\pi_{i,m+1-i}(t)]^{T}\in \mathbb{R}^{m+1-i}$ denotes an auxiliary state with $\pi_{i,j}(t)\in \mathbb{R}, j \in \{1,...,m+1-i\}.$ $\nu_{i}\in \mathbb{R}$ denotes an auxiliary input for \eqref{eq:virtual-system}, $F_{i}:\mathbb{R}^{m+1-i}\to\mathbb{R}^{m+1-i}$ and $G_{i}:\mathbb{R}^{m+1-i}\to\mathbb{R}^{m+1-i}$ are locally Lipschitz. For simplicity, we just build up the connection between an auxiliary variable and the system as $a_{i}(t)=\pi_{i,1}(t), \dot{\pi}_{i,1}(t)=\pi_{i,2}(t),\dots,\dot{\pi}_{i,m-i}(t)=\pi_{i,m+1-i}(t)$ and make $\dot{\pi}_{i,m+1-i}(t)=\nu_{i},$ then we can define many specific HOCBFs $h_{i}$ to enable $a_{i}(t)$ to be positive. 

Given a function $h_{i}:\mathbb{R}^{m+1-i}\to\mathbb{R},$ we can define a sequence of functions $\varphi_{i,j}:\mathbb{R}^{m+1-i}\to\mathbb{R}, i \in\{1,...,m\}, j \in\{1,...,m+1-i\}:$
\begin{equation}
\label{eq:virtual-HOCBFs}
\varphi_{i,j}(\boldsymbol{\pi}_{i})\coloneqq\dot{\varphi}_{i,j-1}(\boldsymbol{\pi}_{i})+\alpha_{i,j}(\varphi_{i,j-1}(\boldsymbol{\pi}_{i})),
\end{equation}
where $\varphi_{i,0}(\boldsymbol{\pi}_{i})\coloneqq h_{i}(\boldsymbol{\pi}_{i}),$ $\alpha_{i,j}(\cdot)$ are $(m+1-i-j)^{th}$ order differentiable class $\kappa$ functions. Sets $\mathcal{B}_{i,j}$ are defined as
\begin{equation}
\label{eq:virtual-sets}
\mathcal B_{i,j}\coloneqq \{\boldsymbol{\pi}_{i}\in\mathbb{R}^{m+1-i}:\varphi_{i,j}(\boldsymbol{\pi}_{i})>0\}, \ j\in \{0,...,m-i\}. 
\end{equation}
Let $\varphi_{i,j}(\boldsymbol{\pi}_{i}),\ j\in \{1,...,m+1-i\}$ and $\mathcal B_{i,j},\ j\in \{0,...,m-i\}$ be defined by \eqref{eq:virtual-HOCBFs} and \eqref{eq:virtual-sets} respectively. By Def. \ref{def:HOCBF}, a function $h_{i}:\mathbb{R}^{m+1-i}\to\mathbb{R}$ is a HOCBF with relative degree $m+1-i$ for system \eqref{eq:virtual-system} if there exist class $\kappa$ functions $\alpha_{i,j},\ j\in \{1,...,m+1-i\}$ as in \eqref{eq:virtual-HOCBFs} such that
\begin{small}
\begin{equation}
\label{eq:highest-SHOCBF}
\begin{split}
\sup_{\nu_{i}\in \mathbb{R}}[L_{F_{i}}^{m+1-i}h_{i}(\boldsymbol{\pi}_{i})+L_{G_{i}}L_{F_{i}}^{m-i}h_{i}(\boldsymbol{\pi}_{i})\nu_{i}+O_{i}(h_{i}(\boldsymbol{\pi}_{i}))\\
+ \alpha_{i,m+1-i}(\varphi_{i,m-i}(\boldsymbol{\pi}_{i}))] \ge \epsilon,
\end{split}
\end{equation}
\end{small}
$\forall\boldsymbol{\pi}_{i}\in \mathcal B_{i,0}\cap,...,\cap \mathcal B_{i,m-i}$. $O_{i}(\cdot)=\sum_{j=1}^{m-i}L_{F_{i}}^{j}(\alpha_{i,m-i}\circ\varphi_{i,m-1-i})(\boldsymbol{\pi}_{i}) $ where $\circ$ denotes the composition of functions. $\epsilon$ is a positive constant which can be infinitely small. 

\begin{remark}
\label{rem:safety-guarantee-2}
If $h_{i}(\boldsymbol{\pi}_{i})$ is a HOCBF illustrated above and $\boldsymbol{\pi}_{i}(0) \in \mathcal {B}_{i,0}\cap \dots \cap \mathcal {B}_{i,m-i},$ then satisfying constraint in \eqref{eq:highest-SHOCBF} is equivalent to making $\varphi_{i,m+1-i}(\boldsymbol{\pi}_{i}(t))\ge \epsilon>0, \forall t\ge 0.$ Based on
\eqref{eq:virtual-HOCBFs}, since $\boldsymbol{\pi}_{i}(0) \in \mathcal {B}_{i,m-i}$ (i.e., $\varphi_{i,m-i}(\boldsymbol{\pi}_{i}(0))>0),$ then we have $\varphi_{i,m-i}(\boldsymbol{\pi}_{i}(t))>0$ (If there exists a $t_{1}\in (0,t_{2}]$, which makes $\varphi_{i,m-i}(\boldsymbol{\pi}_{i}(t_{1}))=0,$ then we have $\dot{\varphi}_{i,m-i}((\boldsymbol{\pi}_{i}(t_{1}))>0\Leftrightarrow \varphi_{i,m-i}(\boldsymbol{\pi}_{i}(t_{1}^{-}))\varphi_{i,m-i}(\boldsymbol{\pi}_{i}(t_{1}^{+}))<0,$ which is against the definition of $\alpha_{i,m+1-i}(\cdot),$ therefore $\forall t_{1}>0, \varphi_{i,m-i}(\boldsymbol{\pi}_{i}(t_{1}))>0,$ note that $t_{1}^{-},t_{1}^{+}$ denote the left and right limit). Based on \eqref{eq:virtual-HOCBFs}, since $\boldsymbol{\pi}_{i}(0) \in \mathcal {B}_{i,m-1-i},$ then similarly we have $\varphi_{i,m-1-i}(\boldsymbol{\pi}_{i}(t))>0,\forall t\ge 0.$ Repeatedly, we have $\varphi_{i,0}(\boldsymbol{\pi}_{i}(t))>0,\forall t\ge 0,$ therefore the sets $\mathcal {B}_{i,0},\dots,\mathcal {B}_{i,m-i}$ are forward invariant.
\end{remark}

For simplicity, we can make $h_{i}(\boldsymbol{\pi}_{i})=\pi_{i,1}(t)=a_{i}(t).$ Based on Rem. \ref{rem:safety-guarantee-2}, each $a_{i}(t)$ will be positive.

The remaining question is how to define an adaptive HOCBF to guarantee $b(\boldsymbol{x})\ge0$ with the assistance of auxiliary variables. Let $\boldsymbol{\Pi}(t)\coloneqq [\boldsymbol{\pi}_{1}(t),\dots,\boldsymbol{\pi}_{m}(t)]^{T}$ and $\boldsymbol{\nu}\coloneqq [\nu_{1},\dots,\nu_{m}]^{T}$ denote the auxiliary states and control inputs of system \eqref{eq:virtual-system}. We can define a sequence of functions 
\begin{small}
\begin{equation}
\label{eq:AVBCBF-sequence}
\begin{split}
&\psi_{0}(\boldsymbol{x},\boldsymbol{\Pi}(t))\coloneqq a_{1}(t)b(\boldsymbol{x}),\\
&\psi_{i}(\boldsymbol{x},\boldsymbol{\Pi}(t))\coloneqq a_{i+1}(t)(\dot{\psi}_{i-1}(\boldsymbol{x},\boldsymbol{\Pi}(t))+\alpha_{i}(\psi_{i-1}(\boldsymbol{x},\boldsymbol{\Pi}(t)))),
\end{split}
\end{equation}
\end{small}
where $i \in \{1,...,m-1\}, \psi_{m}(\boldsymbol{x},\boldsymbol{\Pi}(t))\coloneqq \dot{\psi}_{m-1}(\boldsymbol{x},\boldsymbol{\Pi}(t))+\alpha_{m}(\psi_{m-1}(\boldsymbol{x},\boldsymbol{\Pi}(t))).$ We further define a sequence of sets $\mathcal{C}_{i}$ associated with \eqref{eq:AVBCBF-sequence} in the form 
\begin{equation}
\label{eq:AVBCBF-set}
\begin{split}
\mathcal C_{i}\coloneqq \{(\boldsymbol{x},\boldsymbol{\Pi}(t)) \in \mathbb{R}^{n} \times \mathbb{R}^{m}:\psi_{i}(\boldsymbol{x},\boldsymbol{\Pi}(t))\ge 0\}, 
\end{split}
\end{equation}
where $i \in \{0,...,m-1\}.$
Since $a_{i}(t)$ is a HOCBF with relative degree $m+1-i$ for \eqref{eq:virtual-system}, based on \eqref{eq:highest-SHOCBF}, we define a constraint set $\mathcal{U}_{\boldsymbol{a}}$ for $\boldsymbol{\nu}$ as 
\begin{small}
\begin{equation}
\label{eq:constraint-up}
\begin{split}
\mathcal{U}_{\boldsymbol{a}}(\boldsymbol{\Pi})\coloneqq \{\boldsymbol{\nu}\in\mathbb{R}^{m}:   L_{F_{i}}^{m+1-i}a_{i}+[L_{G_{i}}L_{F_{i}}^{m-i}a_{i}]\nu_{i}\\
+O_{i}(a_{i})+ \alpha_{i,m+1-i}(\varphi_{i,m-i}(a_{i})) \ge \epsilon, i\in \{1,\dots,m\}\},
\end{split}
\end{equation}
\end{small}
where $\varphi_{i,m-i}(\cdot)$ is defined similar to \eqref{eq:virtual-HOCBFs} and $a_{i}(t)$ is ensured positive. $\epsilon$ is a positive constant which can be infinitely small. 

\begin{definition}[AVCBF]
\label{def:AVBCBF}
Let $\psi_{i}(\boldsymbol{x},\boldsymbol{\Pi}(t)),\ i\in \{1,...,m\}$ be defined by \eqref{eq:AVBCBF-sequence} and $\mathcal C_{i},\ i\in \{0,...,m-1\}$ be defined by \eqref{eq:AVBCBF-set}. A function $b(\boldsymbol{x}):\mathbb{R}^{n}\to\mathbb{R}$ is an Auxiliary-Variable Adaptive Control Barrier Function (AVCBF) with relative degree $m$ for system \eqref{eq:affine-control-system} if every $a_{i}(t),i\in \{1,...,m\}$ is a
HOCBF with relative degree $m+1-i$ for the auxiliary system
\eqref{eq:virtual-system}, and there exist $(m-j)^{th}$ order differentiable class $\kappa$ functions $\alpha_{j},j\in \{1,...,m-1\}$
and a class $\kappa$ functions $\alpha_{m}$ s.t.
\begin{small}
\begin{equation}
\label{eq:highest-AVBCBF}
\begin{split}
\sup_{\boldsymbol{u}\in \mathcal{U},\boldsymbol{\nu}\in \mathcal{U}_{\boldsymbol{a}}}[\sum_{j=2}^{m-1}[(\prod_{k=j+1}^{m}a_{k})\frac{\psi_{j-1}}{a_{j}}\nu_{j}] + \frac{\psi_{m-1}}{a_{m}}\nu_{m} \\ +(\prod_{i=2}^{m}a_{i})b(\boldsymbol{x})\nu_{1} +(\prod_{i=1}^{m}a_{i})(L_{f}^{m}b(\boldsymbol{x})+L_{g}L_{f}^{m-1}b(\boldsymbol{x})\boldsymbol{u})\\+R(b(\boldsymbol{x}),\boldsymbol{\Pi})
+ \alpha_{m}(\psi_{m-1})] \ge 0,
\end{split}
\end{equation}
\end{small}
$\forall (\boldsymbol{x},\boldsymbol{\Pi})\in \mathcal C_{0}\cap,...,\cap \mathcal C_{m-1}$ and each $a_{i}>0, i\in\{1,\dots,m\}.$ In \eqref{eq:highest-AVBCBF}, $R(b(\boldsymbol{x}),\boldsymbol{\Pi})$ denotes the remaining Lie derivative terms of $b(\boldsymbol{x})$ (or $\boldsymbol{\Pi}$) along $f$ (or $F_{i},i\in\{1,\dots,m\}$) with degree less than $m$ (or $m+1-i$), which is similar to the form of $O(\cdot )$ in \eqref{eq:highest-HOCBF}.
\end{definition}

\begin{theorem}
\label{thm:safety-guarantee-3}
Given an AVCBF $b(\boldsymbol{x})$ from Def. \ref{def:AVBCBF} with corresponding sets $\mathcal{C}_{0}, \dots,\mathcal {C}_{m-1}$ defined by \eqref{eq:AVBCBF-set}, if $(\boldsymbol{x}(0),\boldsymbol{\Pi}(0)) \in \mathcal {C}_{0}\cap \dots \cap \mathcal {C}_{m-1},$ then if there exists solution of Lipschitz controller $(\boldsymbol{u},\boldsymbol{\nu})$ that satisfies the constraint in \eqref{eq:highest-AVBCBF} and also ensures $(\boldsymbol{x},\boldsymbol{\Pi})\in \mathcal {C}_{m-1}$ for all $t\ge 0,$ then $\mathcal {C}_{0}\cap \dots \cap \mathcal {C}_{m-1}$ will be rendered forward invariant for system \eqref{eq:affine-control-system}, $i.e., (\boldsymbol{x},\boldsymbol{\Pi}) \in \mathcal {C}_{0}\cap \dots \cap \mathcal {C}_{m-1}, \forall t\ge 0.$ Moreover, $b(\boldsymbol{x})\ge 0$ is ensured for all $t\ge 0.$
\end{theorem}

\begin{proof}
If $b(\boldsymbol{x})$ is an AVCBF that is $m^{th}$ order differentiable, then satisfying constraint in \eqref{eq:highest-AVBCBF} while ensuring $(\boldsymbol{x},\boldsymbol{\Pi})\in \mathcal {C}_{m-1}$ for all $t\ge 0$ is equivalent to make $\psi_{m-1}(\boldsymbol{x},\boldsymbol{\Pi})\ge 0, \forall t\ge 0.$ Since $a_{m}(t)>0$, we have $\frac{\psi_{m-1}(\boldsymbol{x},\boldsymbol{\Pi})}{a_{m}(t)}\ge 0.$ Based on
\eqref{eq:AVBCBF-sequence}, since $(\boldsymbol{x}(0),\boldsymbol{\Pi}(0)) \in \mathcal {C}_{m-2}$ (i.e., $\frac{\psi_{m-2}(\boldsymbol{x}(0),\boldsymbol{\Pi}(0))}{a_{m-1}(0)}\ge 0),a_{m-1}(t)>0,$ then we have $\psi_{m-2}(\boldsymbol{x},\boldsymbol{\Pi})\ge 0$ (The proof of this is similar to the proof in Rem. \ref{rem:safety-guarantee-2}), and also $\frac{\psi_{m-2}(\boldsymbol{x},\boldsymbol{\Pi})}{a_{m-1}(t)}\ge 0.$ Based on \eqref{eq:AVBCBF-sequence}, since $(\boldsymbol{x}(0),\boldsymbol{\Pi}(0)) \in \mathcal {C}_{m-3},a_{m-2}(t)>0$ then similarly we have $\psi_{m-3}(\boldsymbol{x},\boldsymbol{\Pi})\ge 0$ and $\frac{\psi_{m-3}(\boldsymbol{x},\boldsymbol{\Pi})}{a_{m-2}(t)}\ge 0,\forall t\ge 0.$ Repeatedly, we have $\psi_{0}(\boldsymbol{x},\boldsymbol{\Pi})\ge 0$ and $\frac{\psi_{0}(\boldsymbol{x},\boldsymbol{\Pi})}{a_{1}(t)}\ge 0,\forall t\ge 0.$ Therefore the sets $\mathcal {C}_{0},\dots,\mathcal {C}_{m-1}$ are forward invariant and $b(\boldsymbol{x})=\frac{\psi_{0}(\boldsymbol{x},\boldsymbol{\Pi})}{a_{1}(t)}\ge 0$ is ensured for all $t\ge 0$.
\end{proof}
Based on Thm. \ref{thm:safety-guarantee-3}, the safety regarding $b(\boldsymbol{x})=\frac{\psi_{0}(\boldsymbol{x},\boldsymbol{\Pi})}{a_{1}(t)}\ge 0$ is guaranteed.

\begin{remark}[Limitation of Approaches with Auxiliary Inputs]
\label{rem: PACBF-AVBCBF} 
Ensuring the satisfaction of the $i^{th}$ order AVCBF constraint as shown in \eqref{eq:AVBCBF-set} when $i\in\{1,\dots,m-1\},$ i.e., $\psi_{i}(\boldsymbol{x},\boldsymbol{\Pi})\ge 0$ will guarantee $\psi_{i-1}(\boldsymbol{x},\boldsymbol{\Pi})\ge 0$ based on the proof of Thm. \ref{thm:safety-guarantee-3}, which theoretically outperforms PACBF. However, both approaches can not ensure satisfying $\psi_{m}(\boldsymbol{x},\boldsymbol{\Pi})\ge 0$ will guarantee $\psi_{m-1}(\boldsymbol{x},\boldsymbol{\Pi})\ge 0$ since the growth of $\boldsymbol{\nu}_{i}$ is unbounded. Therefore in Thm. \ref{thm:safety-guarantee-3}, $(\boldsymbol{x},\boldsymbol{\Pi})\in \mathcal {C}_{m-1}$ for all $t\ge 0$ also needs to be satisfied to guarantee the forward invariance of the intersection of sets. 
\end{remark}

\subsection{Optimal Control with AVCBFs}
\label{subsec: optimal-control}
Consider an optimal control problem as
\begin{small}
\begin{equation}
\label{eq:cost-function-1}
\begin{split}
 \min_{\boldsymbol{u}} \int_{0}^{T} 
 D(\left \| \boldsymbol{u} \right \| )dt,
\end{split}
\end{equation}
\end{small}
where $\left \| \cdot \right \|$ denotes the 2-norm of a vector, $D(\cdot)$ is a strictly increasing function of its argument and $T>0$ denotes the ending time. Since we need to introduce auxiliary inputs $v_{i}$ to enhance the feasibility of optimization, we should reformulate the cost in \eqref{eq:cost-function-1} as
\begin{small}
\begin{equation}
\label{eq:cost-function-2}
\begin{split}
 \min_{\boldsymbol{u},\boldsymbol{\nu}} \int_{0}^{T} 
 [D(\left \| \boldsymbol{u} \right \| )+\sum_{i=1}^{m}W_{i}(\nu_{i}-a_{i,w})^{2}]dt.
\end{split}
\end{equation}
\end{small}
In \eqref{eq:cost-function-2}, $W_{i}$ is a positive scalar and $a_{i,w}\in \mathbb{R}$ is the scalar to which we hope each auxiliary input $\nu_{i}$ converges. Both are chosen to tune the performance of the controller. We can formulate the CLFs, HOCBFs and AVCBFs introduced in Def. \ref{def:control-l-f}, Sec. \ref{sec:AVCBFs} and Def. \ref{def:AVBCBF} as constraints of the QP with cost function \eqref{eq:cost-function-2} to realize safety-critical control. Next we will show AVCBFs can be used to enhance the feasibility of solving QP compared with classical HOCBFs in Def. \ref{def:HOCBF}.

In auxiliary system \eqref{eq:virtual-system}, if we define $a_{i}(t)=\pi_{i,1}(t)=1, \dot{\pi}_{i,1}(t)=\dot{\pi}_{i,2}(t)=\cdots=\dot{\pi}_{i,m+1-i}(t)=0,$ then the way we construct functions and sets in \eqref{eq:virtual-HOCBFs} and \eqref{eq:virtual-sets} are exactly the same as \eqref{eq:sequence-f1} and \eqref{eq:sequence-set1}, which means classical HOCBF is in fact one specific case of AVCBF. Assume that the highest order HOCBF constraint \eqref{eq:highest-HOCBF} conflicts with control input constraints \eqref{eq:control-constraint} at $t=t_{b},$ i.e., we can not find a feasible controller $u(t_{b})$ to satisfy \eqref{eq:highest-HOCBF} and \eqref{eq:control-constraint}. Instead, starting from a time slot $t=t_{a}$ which is just before $t=t_{b}$ ($t_{b}-t_{a}=\varepsilon$ where $\varepsilon$ is an infinitely small positive value), we exchange the control framework of classical HOCBF into AVCBF instantly. Suppose we can find appropriate hyperparameters to ensure two constraints in \eqref{eq:constraint-up} and \eqref{eq:highest-AVBCBF}
% \begin{small}
% \begin{equation}
% \label{eq:constraint-fea-12}
% \begin{split}
%  \nu_{i}
%   > \frac{-L_{F_{i}}^{m+1-i}a_{i}-O_{i}(a_{i})-\alpha_{i,m+1-i}(\varphi_{i,m-i}(a_{i}))}{L_{G_{i}}L_{F_{i}}^{m-i}a_{i}},\\
%   \sum_{j=2}^{m-1}[(\prod_{k=j+1}^{m}a_{k})\frac{\psi_{j-1}}{a_{j}}\nu_{j}] + \frac{\psi_{m-1}}{a_{m}}\nu_{m} +(\prod_{i=2}^{m}a_{i})b(\boldsymbol{x})\nu_{1} \\ \ge -(\prod_{i=1}^{m}a_{i})(L_{f}^{m}b(\boldsymbol{x})+L_{g}L_{f}^{m-1}b(\boldsymbol{x})\boldsymbol{u})-R(b(\boldsymbol{x}),\boldsymbol{\Pi}) \\
% - \alpha_{m}(\psi_{m-1}),  i\in \{1,\dots,m\}
% \end{split}
% \end{equation}
% \end{small}
are satisfied given $\boldsymbol{u}$ constrained by \eqref{eq:control-constraint} at $t_{b},$ then there exists solution $\boldsymbol{u}(t_{b})$ for the optimal control problem and the feasibility of solving QP is enhanced. Relying on AVCBF, We can discretize the whole time period $[0,T]$ into several small time intervals like $[t_{a},t_{b}]$ to maximize the feasibility of solving QP under safety constraints, which calls for the development of automatic parameter-tuning techniques in future.
% \begin{theorem}
% \label{thm:feasibility-guarantee}
% Given an AVCBF $b(\boldsymbol{x})$ from Def. \ref{def:AVBCBF} with corresponding sets $\mathcal{C}_{0}, \dots,\mathcal {C}_{m-1}$ defined by \eqref{eq:AVBCBF-set}, if $(\boldsymbol{x}(0),\boldsymbol{\Pi}(0)) \in \mathcal {C}_{0}\cap \dots \cap \mathcal {C}_{m-1}$ and $L_{G_{i}}L_{F_{i}}^{m-i}a_{i}>0,a_{i}(t)>0, i\in\{1,\dots,m\}$ in \eqref{eq:constraint-up}, then if there exists solution of Lipschitz controller $(\boldsymbol{u},\boldsymbol{\nu})$ that satisfies the constraint in \eqref{eq:highest-AVBCBF} and also ensures $\psi_{0}>0,\dots,\psi_{s}>0,s\in \{0,\dots,m-1\}$ in \eqref{eq:AVBCBF-set}, then the QP with cost function \eqref{eq:cost-function-2} and constraints \eqref{eq:control-constraint},\eqref{eq:AVBCBF-set}-\eqref{eq:highest-AVBCBF} is guranteed to be feasible.
% \end{theorem}

% \begin{proof}
% Rewrite the constraint \eqref{eq:constraint-up} as 
% \begin{equation}
% \label{eq:constraint-fea-1}
% \begin{split}
%  \nu_{i}
%   > \frac{-L_{F_{i}}^{m+1-i}a_{i}-O_{i}(a_{i})-\alpha_{i,m+1-i}(\varphi_{i,m-i}(a_{i}))}{L_{G_{i}}L_{F_{i}}^{m-i}a_{i}},
% \end{split}
% \end{equation}
% where $i\in \{1,\dots,m\}.$ Rewrite the constraint \eqref{eq:highest-AVBCBF} as
% \begin{equation}
% \label{eq:constraint-fea-2}
% \begin{split}
% \sum_{j=2}^{m-1}[(\prod_{k=j+1}^{m}a_{k})\frac{\psi_{j-1}}{a_{j}}\nu_{j}] + \frac{\psi_{m-1}}{a_{m}}\nu_{m} +(\prod_{i=2}^{m}a_{i})b(\boldsymbol{x})\nu_{1} \\ \ge -(\prod_{i=1}^{m}a_{i})(L_{f}^{m}b(\boldsymbol{x})+L_{g}L_{f}^{m-1}b(\boldsymbol{x})\boldsymbol{u})-R(b(\boldsymbol{x}),\boldsymbol{\Pi}) \\
% - \alpha_{m}(\psi_{m-1}),  i\in \{1,\dots,m\}.
% \end{split}
% \end{equation}
% Since $L_{G_{i}}L_{F_{i}}^{m-i}a_{i}>0$ in \eqref{eq:constraint-fea-1}, $\psi_{0}>0,\dots,\psi_{s}>0,s\in \{0,\dots,m-1\}$ in \eqref{eq:constraint-fea-2} and $a_{1}>0,\dots,a_{m}>0,$ we have $(\prod_{i=2}^{m}a_{i})b(\boldsymbol{x})>0,(\prod_{k=j+1}^{m}a_{k})\frac{\psi_{j-1}}{a_{j}}\nu_{j}>0,j\in \{2,\dots,s\}$ are always positive, 
% then there always exist large enough $\nu_{1},\dots,\nu_{s}$ satisfying constraints above {\color{red}you are assuming a very specific (13).} (the upper bounds of $\nu_{1},\dots,\nu_{s}$ are unlimited), hence the feasibility of QP with cost function \eqref{eq:cost-function-2} and constraints \eqref{eq:control-constraint},\eqref{eq:AVBCBF-set}-\eqref{eq:highest-AVBCBF} is guaranteed.  {\color{red}Control limitations (2) are the most critical factor in the feasibility. You completely ignore this. The proof is very sloppy.}
% \end{proof}

Besides safety and feasibility, another benefit of using AVCBFs is that the conservativeness of the control strategy can also be ameliorated. For example, from \eqref{eq:AVBCBF-sequence}, we can rewrite $\psi_{i}(\boldsymbol{x},\boldsymbol{\Pi})\ge 0$ as
\begin{equation}
\label{eq:AVCBF-rewrite}
\begin{split}
\dot{\phi}_{i-1}(\boldsymbol{x},\boldsymbol{\Pi})+k_{i}(1+\frac{\dot{a}_{i}(t)}{k_{i}a_{i}(t)}) \phi_{i-1}(\boldsymbol{x},\boldsymbol{\Pi})\ge0,
\end{split}
\end{equation}
where $\phi_{i-1}(\boldsymbol{x},\boldsymbol{\Pi})=\frac{\psi_{i-1}(\boldsymbol{x},\boldsymbol{\Pi})}{a_{i}(t)},\alpha_{i}(\psi_{i-1}(\boldsymbol{x},\boldsymbol{\Pi}))=k_{i}a_{i}(t)\phi_{i-1}(\boldsymbol{x},\boldsymbol{\Pi}), k_{i}>0, i\in \{1,\dots,m\}.$ Similar to PACBFs, we require $1+\frac{\dot{a}_{i}(t)}{k_{i}a_{i}(t)}\ge0,$ which gives us $\dot{a}_{i}(t)+k_{i}a_{i}(t)\ge0.$
The term $\frac{\dot{a}_{i}(t)}{a_{i}(t)}$ can be adjusted adaptable  to ameliorate the conservativeness of control strategy that $k_{i}\phi_{i-1}(\boldsymbol{x},\boldsymbol{\Pi})$ may have, i.e., the ego vehicle can brake earlier or later given time-varying control constraint $\boldsymbol{u}_{min}(t)\le \boldsymbol{u} \le\boldsymbol{u}_{max}(t),$ which confirms the adaptivity of AVCBFs to control constraint and conservativeness of control strategy. 

\begin{remark}[Parameter-Tuning for AVCBFs]
\label{rem: parameter-tuning}
Based on the analysis of \eqref{eq:AVCBF-rewrite}, we require $\dot{a}_{i}(t)+k_{i}a_{i}(t)\ge0.$ If we define first order HOCBF constraint for $a_{i}(t)>0$ as $\dot{a}_{i}(t)+l_{i}a_{i}(t)\ge0,$ we should choose hyperparameter $l_{i}\le k_{i}$ to guarantee $\dot{a}_{i}(t)+k_{i}a_{i}(t)\ge\dot{a}_{i}(t)+l_{i}a_{i}(t)\ge 0.$ For simplicity, we can use $l_{i}=k_{i}.$ In cost function \eqref{eq:cost-function-2}, we can tune hyperparameters $W_{i}$ and $a_{i,w}$ to adjust the corresponding ratio $\frac{\dot{a}_{i}(t)}{a_{i}(t)}$ to change the performance of the optimal controller.
\end{remark}

\begin{remark}
\label{rem: sufficient-con}
Note that the satisfaction of the constraint in \eqref{eq:highest-AVBCBF} is a sufficient condition for the satisfaction of the original constraint $\psi_{0}(\boldsymbol{x},\boldsymbol{\Pi})>0,$ it is not necessary to introduce auxiliary variables as many as from $a_{1}(t)$ to $a_{m}(t),$ which allows us to choose an appropriate
number of auxiliary variables for the AVCBF constraints to reduce the complexity. In other words, the number of auxiliary variables can be less than or equal to the relative degree $m$.
\end{remark}
\subsection{Sampling Algorithm} 
Sampling directly from the joint distributions of $Y, B$ is challenging. We propose to use Gibbs sampling to sample from the target distribution by alternatively sampling from $P(Y | X, B)$ and $P(B | X, Y)$. First, we describe how our proposed algorithm samples from each conditional distribution. We then introduce the complete sampling algorithm along with some intuition as to how it ensures satisfactory and fluent generations. 

\paragraph{Sampling from $P(B | X, Y)$} 
The conditional distribution for $P(B | X, Y)$ includes the term $P^{LM}(Y | X, B)$. This is to ensure that the sampled $B$ results in output sequences that are high in likelihood under the base model distribution. 
However, directly computing \( P^{LM}(Y | X, B) \) for all possible values of \( B \) is intractable. 
By noting that this term encourages the selection of \( B \) that is consistent with the observed \( Y \), we can infer this property is naturally satisfied when $B$ is close to $Y$. 
Thus, we approximate \( P^{LM}(Y | X, B) \) by performing a single MCMC step with the initial state set to $Y$. 

In order to sample this discrete sequence of tokens, we apply the Discrete Langevin Proposal (DLP) introduced in \citep{zhang2022langevinlike}. 
After initializing $B$ as $B = Y$, and representing the sequence as a sequence of one-hot vectors $\hat{B} = \{\hat{b}_1, \hat{b_2} \dots \hat{b}_n\}$, we execute a single step of DLP with the target distribution being $\exp(f(B | X))$. Below we include the proposal distribution for position $i$:
\begin{align}
\label{eq:dlp_prop}
    b'_i \sim \categorical\left(\underset{j \in |V|}{\softmax} \left( \frac{1}{\tau} (\nabla f(\hat{B} | X))_{i,j} (1 - \hat{b}_{i,j}) \right) \right)
\end{align}
Here, $\tau$ is a temperature hyper-parameter that controls the sharpness of the proposal distribution, $(\nabla f(\hat{B} | X))_{i,j}$ is the $j$th component of the $i$th gradient vector, $\hat{b}_{i,j}$ represents the $j$th component of the one-hot vector $\hat{b}_i$, and $b'_i$ is the token we sample from the distribution over $V$. 
For more details on the application of DLP and the gradient compution, see Appendix \ref{appndx:dlp_proposal}.
Unlike the algorithm presented in \citet{liu2023bolt}, we do not require the use of straight through estimation (STE) \citep{bengio2013estimating} as we differentiate directly with respect to $Y$.
Note that this proposal function can be computed for all sequence positions in parallel. 
We refer to this proposal function as $q_\tau(\cdot | B)$. 

\paragraph{Sampling from $P(Y | X, B)$}
Our goal is to sample from $P(Y | X, B)$ using biased auto-regressive generation, similar to \eqref{eq:bolt-auto-reg}. 
In order to do so, we must map the sequence of bias \textit{tokens} $B$ to a sequence of bias \textit{vectors} $\tilde{B}$.  
The ideal bias vector should reflect the difference in meaning between each token in the vocabulary space $V$ and the sampled token.
To accomplish this, we penalize each token based on the distance to the sampled bias token within the embedding space, as static embeddings reflect semantic meaning \citep{mikolov2013efficient, pennington2014glove, mikolov2013distributed}. Given a bias token $b_i$, embedding table $M$, we define the $j$th coordinate value corresponding to token $v_j$ as follows: 
\begin{align}
\label{eq:bias-vec-def}
    \tilde{b}_{i,j} = || Mb_i - Mv_j||^2_2 
\end{align}
This yields a $|V|$ dimensional vector that can be added to the auto-regressive logits $\tilde{y_i}$. 
When adding the bias term to $\tilde{y_i}$, we also incorporate both a weight term $w_i$ and a normalizing factor $r_i$. While $w_i$ is a hyper-parameter, $r_i$ normalizes the bias vector at the $i$ position to have the same norm as $\tilde{y_i}$. We define the normalizing factor as follows: 
\begin{align}
    \label{eq:dab-normalize}
    r_i = \frac{|| \tilde{y_i} ||_2}{|| \tilde{b}_i ||_2}
\end{align}
We note that while this normalizing factor can also be applied to BOLT and may improve its results, the modified BOLT still underperforms compared to our method. 

We formally define our biased auto-regressive generation as follows: 
\begin{align}
    \label{eq:dab-auto-reg}
    y_i = \argmax_{j \in |V|} \left( \tilde{y}_{i, j} - w_i \cdot r_i \cdot \tilde{b}_{i, j}\right). 
\end{align}
Intuitively, this returns the token corresponding to the maximal coordinate of the biased distribution. Repeating this $n$ times results in the updated response sequence $Y$. 
\begin{figure}[t]
    \centering
    \resizebox{12 cm}{!}{
        \input{Main_Body/diagram}
        }
    \caption{Visualization of the proposed decoding algorithm, DAB. DAB alternates between sampling the response $Y$ and the bias $B$. To sample $B$ given $Y$, we use gradient-based discrete sampling on the constraint function $f$. To sample $Y$ given $B$, we compute a bias vector that penalizes words based on their distance to $B$ and then use this bias to guide the auto-regressive generation.
}
\label{fig:diagram}
\end{figure}

\paragraph{Text Generation Algorithm} We provide a visualization of our algorithm in Figure \ref{fig:diagram}, and include the full algorithm in Appendix \ref{appndx:algrthm-details}. Given some prompt, we first generate some initial auto-regressive generation $Y_1$, with the initial bias vector set to $\Vec{0}$. After obtaining $Y$, we sample from the conditional distribution over $B$ to obtain a sequence of bias tokens. We then use \eqref{eq:bias-vec-def} to compute the new bias vector to use for biased auto-regressive generation. 
We repeat this alternative sampling process for several iterations, returning the sample that best satisfies the constraint at the end as commonly done in the literature \citep{kumar2022gradient,liu2023bolt}. 
For a discussion on the hyper-parameters of our algorithm, see Appendix \ref{appndx:ablation}. 
% \begin{algorithm}
    \caption{Discrete Autoregressive Biasing}
    \begin{algorithmic}[1]
    \REQUIRE Constraint function $f$, $P^{LM}$, prompt $X$, number steps $s$, sequence length $n$, embedding table $M$
    \STATE $\tilde{B} \gets \vec{0}, f_\text{min} \gets -\infty$, $Y_\text{best} \gets \{\}$ \LineComment{Initialize constraint violation as being maximal and current best generation as empty}
    \FOR{step $s$}
        \FOR{position $i$ in range($n$)} 
            \STATE $\tilde{y_i} \gets \log P^{LM} (\cdot | y_{<i}, X)$ \LineComment{Initial auto-regressive distribution over $V$}
            \STATE Calculate normalizing factor $r_i$ if $s > 1$, else $r_i \gets 1$
            \STATE $y_i \gets \text{argmax}_{j \in |V|} \left(\tilde{y}_{i, j} - w_i \cdot r_i \cdot \tilde{b}_{i, j} \right)$ \LineComment{Sample from $P(Y | X, B)$}
        \ENDFOR
        \STATE $B \gets Y$ \LineComment{Initialize $B$ as $Y$}
        \STATE Evaluate $f(B | X)$, update $f_\text{min}$, $Y_\text{best}$
        \STATE $B' \sim q_\tau(\cdot | B)$ as in \eqref{eq:dlp_prop} \LineComment{Approximately sample from $P(B | X, Y)$}
        \STATE Compute $\tilde{B}$ as in \eqref{eq:bias-vec-def}
    \ENDFOR
    \STATE return $Y_\text{best}$
    \end{algorithmic}
\label{alg:text-gen}
\end{algorithm}
\subsection{Advantages of Biasing in Discrete Spaces}
Here we discuss various advantages of discrete sampling in the context of auto-regressive biasing. First, we demonstrate that discrete sampling enables a quicker and more thorough exploration of potential output sequences $Y$. We then describe how discrete sampling solves the stability issue discussed in \citet{liu2023bolt}. Finally, we show that discrete sampling makes use of simpler gradient computations, resulting in a more efficient decoding algorithm.

\paragraph{Exploration of State Space}
Discrete sampling enables DAB to explore the output space more effectively than continuous methods.
We hypothesize that discrete sampling enables more directional and substantial changes to the bias vector, resulting in more token changes in the output sequence across sampling steps. 
We compare with BOLT, a continuous auto-regressive biasing algorithm \citep{liu2023bolt}. 
We examine the performance of BOLT both with and without the normalizing factor defined in \eqref{eq:dab-normalize}. 
We include the comparison of hops across 50 steps in Figure \ref{fig:samplespace_explore}a.
These results show that our method updates substantially more sequence positions across all sampling steps than either variant of BOLT. 

Next, we measure how comprehensively each method explores the sample space of potential sequences. 
For each sequence position, we maintain a record of tokens encountered throughout the sampling process and compute the number of unique tokens within this set. 
Figure \ref{fig:samplespace_explore}b shows the average unique tokens per sequence position for all three algorithms. 
These results indicate that our method samples more unique tokens for each sequence position than either variant of BOLT, demonstrating more comprehensive exploration. 
Collectively, these findings confirm that discrete sampling enables faster, more thorough, and thus more effective exploration of the sample space of potential sequences. 
\paragraph{Sampling Stability}
Discrete sampling allows DAB to have superior stability across sampling steps when compared to continuous methods. 
We show this in Figure \ref{fig:samplespace_explore}c, where we track the average perplexity of the batch at each time step. 
While BOLT faces deteriorating perplexity, DAB remains stable throughout the sampling process. 

We attribute this instability to the difficulty of applying continuous sampling techniques to a discrete domain as discussed in \citet{grathwohl2021gwg}. 
As a result of BOLT's misalignment between the sampling domain and target domain, the energy landscape is too complex to navigate with gradient information.
This results in the divergence seen in Figure \ref{fig:samplespace_explore}c. 

\begin{figure}
    \centering
    \begin{subfigure}[t]{0.3\textwidth}
        \includegraphics[width=.95\linewidth]{Images/token_update_over_step.pdf}
        \caption{Hops per Sample Step}
    \end{subfigure}
    \begin{subfigure}[t]{0.3\textwidth}
        \includegraphics[width=.95\linewidth]{Images/total_unique_token.pdf}
        \caption{Avg. Unique Tokens}
    \end{subfigure}
        \begin{subfigure}[t]{0.3\textwidth}
        \includegraphics[width=.95\linewidth]{Images/mixing_fig.pdf}
        \caption{Perplexity per Sample Step}
    \end{subfigure}
    \caption{(a) Average hops, or token updates per sequence, against sampling steps. Both versions of BOLT suffer from decreasing hops while DAB remains stable.
    (b) Average number of the unique tokens sampled for each sequence position throughout the entire sampling process. DAB discovers many more unique tokens for each position than either variant of BOLT. 
    (c) Comparison of fluency with respect to sampling steps. Dab exhibits stable fluency over sampling steps in comparison to BOLT. 
    % While BOLT suffers from degrading fluency even after incorporating a normalizing factor, DAB exhibits stable behavior. 
    }
    \label{fig:samplespace_explore}
\end{figure}

Our algorithm avoids this entirely as we perform direct sampling on the discrete token space. 
Since we define the sampling domain and target domain to be the same, our algorithm enjoys superior stability throughout all sampling steps. 
This improvement in algorithmic stability removes the need for implementing early-stopping and to carefully tune the number of sampling steps. 
\vspace{-1em}
\begin{wraptable}[7]{l}{.5\textwidth}
\caption{Efficiency comparison between BOLT and DAB in terms of tokens per second.
}
\vspace{-0.5em}
\label{table:speed-table}
\centering
\resizebox{.4\textwidth}{!}{\begin{tabular}{lcccl}\toprule
      & BOLT & DAB (Ours) \\\midrule
\makecell{Tokens per \\ Second}  & $9.495 \pm .095$  & $\mathbf{23.213 \pm 0.304}$ \\
\bottomrule
 \end{tabular}}
\end{wraptable}
 
\vspace{-0.8em}
\paragraph{Improved Efficiency} Discrete sampling enables our algorithm to use simpler gradient computations that provide a computational advantage over continuous sampling methods. 
To evaluate our algorithm's efficiency, we compare the tokens per second of our method to BOLT. 
We also measure the time-cost for computing the bias term for each method. 
We put the results in Table \ref{table:speed-table}, where we observe that our method has over \textbf{2x} the tokens per second output when compared against BOLT. 
Our algorithm achieves this computational advantage as a result of computing the gradient with respect to $\hat{B} = \hat{Y}$, which removes the need to backpropagate through auto-regressive generation. Computing the gradient of $f$ with respect to a continuous bias term $\tilde{B}$ requires first computing $\pderiv{f}{\hat{Y}}$ and then $\pderiv{\hat{Y}}{\tilde{B}}$.
Since each one-hot vector in $\hat{Y}$ is influenced by previous bias terms, the latter term requires backpropagation through auto-regressive generation. 
Simply initializing $\tilde{B}=\hat{Y}$ will not work in continuous sampling because the incremental updates will keep $\tilde{B}$ close to the original $\hat{Y}$. 
In contrast, our method uses gradients to identify which tokens will increase constraint satisfaction and directly samples them, enabling substantial change from the original sequence while incorporating information from the external constraint. 
While continuous sampling cannot exploit this computational shortcut and maintain constraint satisfaction, gradient-based discrete sampling achieves both simultaneously. 


\section*{Acknowledgements}
JW was partially supported by the National Research Foundation of Korea(NRF) grant funded by the Korea government(MSIT) [RS-2024-00349646]. DK was partially supported by the National Research Foundation of Korea (NRF) grant funded by the Korea government (MSIT) (No. RS-2023-00252516 and No. RS-2024-00408003), the POSCO Science Fellowship of POSCO TJ Park Foundation, and the Korea Institute for Advanced Study.

\section*{Impact Statement}
This paper presents work whose goal is to advance the field of Machine Learning. There are many potential societal consequences of our work, none of which we feel must be specifically highlighted here.

\nocite{langley00}

\bibliography{mybib}
\bibliographystyle{icml2025}


%%%%%%%%%%%%%%%%%%%%%%%%%%%%%%%%%%%%%%%%%%%%%%%%%%%%%%%%%%%%%%%%%%%%%%%%%%%%%%%
%%%%%%%%%%%%%%%%%%%%%%%%%%%%%%%%%%%%%%%%%%%%%%%%%%%%%%%%%%%%%%%%%%%%%%%%%%%%%%%
% APPENDIX
%%%%%%%%%%%%%%%%%%%%%%%%%%%%%%%%%%%%%%%%%%%%%%%%%%%%%%%%%%%%%%%%%%%%%%%%%%%%%%%
%%%%%%%%%%%%%%%%%%%%%%%%%%%%%%%%%%%%%%%%%%%%%%%%%%%%%%%%%%%%%%%%%%%%%%%%%%%%%%%
\newpage
\appendix
\onecolumn
\subsection{Lloyd-Max Algorithm}
\label{subsec:Lloyd-Max}
For a given quantization bitwidth $B$ and an operand $\bm{X}$, the Lloyd-Max algorithm finds $2^B$ quantization levels $\{\hat{x}_i\}_{i=1}^{2^B}$ such that quantizing $\bm{X}$ by rounding each scalar in $\bm{X}$ to the nearest quantization level minimizes the quantization MSE. 

The algorithm starts with an initial guess of quantization levels and then iteratively computes quantization thresholds $\{\tau_i\}_{i=1}^{2^B-1}$ and updates quantization levels $\{\hat{x}_i\}_{i=1}^{2^B}$. Specifically, at iteration $n$, thresholds are set to the midpoints of the previous iteration's levels:
\begin{align*}
    \tau_i^{(n)}=\frac{\hat{x}_i^{(n-1)}+\hat{x}_{i+1}^{(n-1)}}2 \text{ for } i=1\ldots 2^B-1
\end{align*}
Subsequently, the quantization levels are re-computed as conditional means of the data regions defined by the new thresholds:
\begin{align*}
    \hat{x}_i^{(n)}=\mathbb{E}\left[ \bm{X} \big| \bm{X}\in [\tau_{i-1}^{(n)},\tau_i^{(n)}] \right] \text{ for } i=1\ldots 2^B
\end{align*}
where to satisfy boundary conditions we have $\tau_0=-\infty$ and $\tau_{2^B}=\infty$. The algorithm iterates the above steps until convergence.

Figure \ref{fig:lm_quant} compares the quantization levels of a $7$-bit floating point (E3M3) quantizer (left) to a $7$-bit Lloyd-Max quantizer (right) when quantizing a layer of weights from the GPT3-126M model at a per-tensor granularity. As shown, the Lloyd-Max quantizer achieves substantially lower quantization MSE. Further, Table \ref{tab:FP7_vs_LM7} shows the superior perplexity achieved by Lloyd-Max quantizers for bitwidths of $7$, $6$ and $5$. The difference between the quantizers is clear at 5 bits, where per-tensor FP quantization incurs a drastic and unacceptable increase in perplexity, while Lloyd-Max quantization incurs a much smaller increase. Nevertheless, we note that even the optimal Lloyd-Max quantizer incurs a notable ($\sim 1.5$) increase in perplexity due to the coarse granularity of quantization. 

\begin{figure}[h]
  \centering
  \includegraphics[width=0.7\linewidth]{sections/figures/LM7_FP7.pdf}
  \caption{\small Quantization levels and the corresponding quantization MSE of Floating Point (left) vs Lloyd-Max (right) Quantizers for a layer of weights in the GPT3-126M model.}
  \label{fig:lm_quant}
\end{figure}

\begin{table}[h]\scriptsize
\begin{center}
\caption{\label{tab:FP7_vs_LM7} \small Comparing perplexity (lower is better) achieved by floating point quantizers and Lloyd-Max quantizers on a GPT3-126M model for the Wikitext-103 dataset.}
\begin{tabular}{c|cc|c}
\hline
 \multirow{2}{*}{\textbf{Bitwidth}} & \multicolumn{2}{|c|}{\textbf{Floating-Point Quantizer}} & \textbf{Lloyd-Max Quantizer} \\
 & Best Format & Wikitext-103 Perplexity & Wikitext-103 Perplexity \\
\hline
7 & E3M3 & 18.32 & 18.27 \\
6 & E3M2 & 19.07 & 18.51 \\
5 & E4M0 & 43.89 & 19.71 \\
\hline
\end{tabular}
\end{center}
\end{table}

\subsection{Proof of Local Optimality of LO-BCQ}
\label{subsec:lobcq_opt_proof}
For a given block $\bm{b}_j$, the quantization MSE during LO-BCQ can be empirically evaluated as $\frac{1}{L_b}\lVert \bm{b}_j- \bm{\hat{b}}_j\rVert^2_2$ where $\bm{\hat{b}}_j$ is computed from equation (\ref{eq:clustered_quantization_definition}) as $C_{f(\bm{b}_j)}(\bm{b}_j)$. Further, for a given block cluster $\mathcal{B}_i$, we compute the quantization MSE as $\frac{1}{|\mathcal{B}_{i}|}\sum_{\bm{b} \in \mathcal{B}_{i}} \frac{1}{L_b}\lVert \bm{b}- C_i^{(n)}(\bm{b})\rVert^2_2$. Therefore, at the end of iteration $n$, we evaluate the overall quantization MSE $J^{(n)}$ for a given operand $\bm{X}$ composed of $N_c$ block clusters as:
\begin{align*}
    \label{eq:mse_iter_n}
    J^{(n)} = \frac{1}{N_c} \sum_{i=1}^{N_c} \frac{1}{|\mathcal{B}_{i}^{(n)}|}\sum_{\bm{v} \in \mathcal{B}_{i}^{(n)}} \frac{1}{L_b}\lVert \bm{b}- B_i^{(n)}(\bm{b})\rVert^2_2
\end{align*}

At the end of iteration $n$, the codebooks are updated from $\mathcal{C}^{(n-1)}$ to $\mathcal{C}^{(n)}$. However, the mapping of a given vector $\bm{b}_j$ to quantizers $\mathcal{C}^{(n)}$ remains as  $f^{(n)}(\bm{b}_j)$. At the next iteration, during the vector clustering step, $f^{(n+1)}(\bm{b}_j)$ finds new mapping of $\bm{b}_j$ to updated codebooks $\mathcal{C}^{(n)}$ such that the quantization MSE over the candidate codebooks is minimized. Therefore, we obtain the following result for $\bm{b}_j$:
\begin{align*}
\frac{1}{L_b}\lVert \bm{b}_j - C_{f^{(n+1)}(\bm{b}_j)}^{(n)}(\bm{b}_j)\rVert^2_2 \le \frac{1}{L_b}\lVert \bm{b}_j - C_{f^{(n)}(\bm{b}_j)}^{(n)}(\bm{b}_j)\rVert^2_2
\end{align*}

That is, quantizing $\bm{b}_j$ at the end of the block clustering step of iteration $n+1$ results in lower quantization MSE compared to quantizing at the end of iteration $n$. Since this is true for all $\bm{b} \in \bm{X}$, we assert the following:
\begin{equation}
\begin{split}
\label{eq:mse_ineq_1}
    \tilde{J}^{(n+1)} &= \frac{1}{N_c} \sum_{i=1}^{N_c} \frac{1}{|\mathcal{B}_{i}^{(n+1)}|}\sum_{\bm{b} \in \mathcal{B}_{i}^{(n+1)}} \frac{1}{L_b}\lVert \bm{b} - C_i^{(n)}(b)\rVert^2_2 \le J^{(n)}
\end{split}
\end{equation}
where $\tilde{J}^{(n+1)}$ is the the quantization MSE after the vector clustering step at iteration $n+1$.

Next, during the codebook update step (\ref{eq:quantizers_update}) at iteration $n+1$, the per-cluster codebooks $\mathcal{C}^{(n)}$ are updated to $\mathcal{C}^{(n+1)}$ by invoking the Lloyd-Max algorithm \citep{Lloyd}. We know that for any given value distribution, the Lloyd-Max algorithm minimizes the quantization MSE. Therefore, for a given vector cluster $\mathcal{B}_i$ we obtain the following result:

\begin{equation}
    \frac{1}{|\mathcal{B}_{i}^{(n+1)}|}\sum_{\bm{b} \in \mathcal{B}_{i}^{(n+1)}} \frac{1}{L_b}\lVert \bm{b}- C_i^{(n+1)}(\bm{b})\rVert^2_2 \le \frac{1}{|\mathcal{B}_{i}^{(n+1)}|}\sum_{\bm{b} \in \mathcal{B}_{i}^{(n+1)}} \frac{1}{L_b}\lVert \bm{b}- C_i^{(n)}(\bm{b})\rVert^2_2
\end{equation}

The above equation states that quantizing the given block cluster $\mathcal{B}_i$ after updating the associated codebook from $C_i^{(n)}$ to $C_i^{(n+1)}$ results in lower quantization MSE. Since this is true for all the block clusters, we derive the following result: 
\begin{equation}
\begin{split}
\label{eq:mse_ineq_2}
     J^{(n+1)} &= \frac{1}{N_c} \sum_{i=1}^{N_c} \frac{1}{|\mathcal{B}_{i}^{(n+1)}|}\sum_{\bm{b} \in \mathcal{B}_{i}^{(n+1)}} \frac{1}{L_b}\lVert \bm{b}- C_i^{(n+1)}(\bm{b})\rVert^2_2  \le \tilde{J}^{(n+1)}   
\end{split}
\end{equation}

Following (\ref{eq:mse_ineq_1}) and (\ref{eq:mse_ineq_2}), we find that the quantization MSE is non-increasing for each iteration, that is, $J^{(1)} \ge J^{(2)} \ge J^{(3)} \ge \ldots \ge J^{(M)}$ where $M$ is the maximum number of iterations. 
%Therefore, we can say that if the algorithm converges, then it must be that it has converged to a local minimum. 
\hfill $\blacksquare$


\begin{figure}
    \begin{center}
    \includegraphics[width=0.5\textwidth]{sections//figures/mse_vs_iter.pdf}
    \end{center}
    \caption{\small NMSE vs iterations during LO-BCQ compared to other block quantization proposals}
    \label{fig:nmse_vs_iter}
\end{figure}

Figure \ref{fig:nmse_vs_iter} shows the empirical convergence of LO-BCQ across several block lengths and number of codebooks. Also, the MSE achieved by LO-BCQ is compared to baselines such as MXFP and VSQ. As shown, LO-BCQ converges to a lower MSE than the baselines. Further, we achieve better convergence for larger number of codebooks ($N_c$) and for a smaller block length ($L_b$), both of which increase the bitwidth of BCQ (see Eq \ref{eq:bitwidth_bcq}).


\subsection{Additional Accuracy Results}
%Table \ref{tab:lobcq_config} lists the various LOBCQ configurations and their corresponding bitwidths.
\begin{table}
\setlength{\tabcolsep}{4.75pt}
\begin{center}
\caption{\label{tab:lobcq_config} Various LO-BCQ configurations and their bitwidths.}
\begin{tabular}{|c||c|c|c|c||c|c||c|} 
\hline
 & \multicolumn{4}{|c||}{$L_b=8$} & \multicolumn{2}{|c||}{$L_b=4$} & $L_b=2$ \\
 \hline
 \backslashbox{$L_A$\kern-1em}{\kern-1em$N_c$} & 2 & 4 & 8 & 16 & 2 & 4 & 2 \\
 \hline
 64 & 4.25 & 4.375 & 4.5 & 4.625 & 4.375 & 4.625 & 4.625\\
 \hline
 32 & 4.375 & 4.5 & 4.625& 4.75 & 4.5 & 4.75 & 4.75 \\
 \hline
 16 & 4.625 & 4.75& 4.875 & 5 & 4.75 & 5 & 5 \\
 \hline
\end{tabular}
\end{center}
\end{table}

%\subsection{Perplexity achieved by various LO-BCQ configurations on Wikitext-103 dataset}

\begin{table} \centering
\begin{tabular}{|c||c|c|c|c||c|c||c|} 
\hline
 $L_b \rightarrow$& \multicolumn{4}{c||}{8} & \multicolumn{2}{c||}{4} & 2\\
 \hline
 \backslashbox{$L_A$\kern-1em}{\kern-1em$N_c$} & 2 & 4 & 8 & 16 & 2 & 4 & 2  \\
 %$N_c \rightarrow$ & 2 & 4 & 8 & 16 & 2 & 4 & 2 \\
 \hline
 \hline
 \multicolumn{8}{c}{GPT3-1.3B (FP32 PPL = 9.98)} \\ 
 \hline
 \hline
 64 & 10.40 & 10.23 & 10.17 & 10.15 &  10.28 & 10.18 & 10.19 \\
 \hline
 32 & 10.25 & 10.20 & 10.15 & 10.12 &  10.23 & 10.17 & 10.17 \\
 \hline
 16 & 10.22 & 10.16 & 10.10 & 10.09 &  10.21 & 10.14 & 10.16 \\
 \hline
  \hline
 \multicolumn{8}{c}{GPT3-8B (FP32 PPL = 7.38)} \\ 
 \hline
 \hline
 64 & 7.61 & 7.52 & 7.48 &  7.47 &  7.55 &  7.49 & 7.50 \\
 \hline
 32 & 7.52 & 7.50 & 7.46 &  7.45 &  7.52 &  7.48 & 7.48  \\
 \hline
 16 & 7.51 & 7.48 & 7.44 &  7.44 &  7.51 &  7.49 & 7.47  \\
 \hline
\end{tabular}
\caption{\label{tab:ppl_gpt3_abalation} Wikitext-103 perplexity across GPT3-1.3B and 8B models.}
\end{table}

\begin{table} \centering
\begin{tabular}{|c||c|c|c|c||} 
\hline
 $L_b \rightarrow$& \multicolumn{4}{c||}{8}\\
 \hline
 \backslashbox{$L_A$\kern-1em}{\kern-1em$N_c$} & 2 & 4 & 8 & 16 \\
 %$N_c \rightarrow$ & 2 & 4 & 8 & 16 & 2 & 4 & 2 \\
 \hline
 \hline
 \multicolumn{5}{|c|}{Llama2-7B (FP32 PPL = 5.06)} \\ 
 \hline
 \hline
 64 & 5.31 & 5.26 & 5.19 & 5.18  \\
 \hline
 32 & 5.23 & 5.25 & 5.18 & 5.15  \\
 \hline
 16 & 5.23 & 5.19 & 5.16 & 5.14  \\
 \hline
 \multicolumn{5}{|c|}{Nemotron4-15B (FP32 PPL = 5.87)} \\ 
 \hline
 \hline
 64  & 6.3 & 6.20 & 6.13 & 6.08  \\
 \hline
 32  & 6.24 & 6.12 & 6.07 & 6.03  \\
 \hline
 16  & 6.12 & 6.14 & 6.04 & 6.02  \\
 \hline
 \multicolumn{5}{|c|}{Nemotron4-340B (FP32 PPL = 3.48)} \\ 
 \hline
 \hline
 64 & 3.67 & 3.62 & 3.60 & 3.59 \\
 \hline
 32 & 3.63 & 3.61 & 3.59 & 3.56 \\
 \hline
 16 & 3.61 & 3.58 & 3.57 & 3.55 \\
 \hline
\end{tabular}
\caption{\label{tab:ppl_llama7B_nemo15B} Wikitext-103 perplexity compared to FP32 baseline in Llama2-7B and Nemotron4-15B, 340B models}
\end{table}

%\subsection{Perplexity achieved by various LO-BCQ configurations on MMLU dataset}


\begin{table} \centering
\begin{tabular}{|c||c|c|c|c||c|c|c|c|} 
\hline
 $L_b \rightarrow$& \multicolumn{4}{c||}{8} & \multicolumn{4}{c||}{8}\\
 \hline
 \backslashbox{$L_A$\kern-1em}{\kern-1em$N_c$} & 2 & 4 & 8 & 16 & 2 & 4 & 8 & 16  \\
 %$N_c \rightarrow$ & 2 & 4 & 8 & 16 & 2 & 4 & 2 \\
 \hline
 \hline
 \multicolumn{5}{|c|}{Llama2-7B (FP32 Accuracy = 45.8\%)} & \multicolumn{4}{|c|}{Llama2-70B (FP32 Accuracy = 69.12\%)} \\ 
 \hline
 \hline
 64 & 43.9 & 43.4 & 43.9 & 44.9 & 68.07 & 68.27 & 68.17 & 68.75 \\
 \hline
 32 & 44.5 & 43.8 & 44.9 & 44.5 & 68.37 & 68.51 & 68.35 & 68.27  \\
 \hline
 16 & 43.9 & 42.7 & 44.9 & 45 & 68.12 & 68.77 & 68.31 & 68.59  \\
 \hline
 \hline
 \multicolumn{5}{|c|}{GPT3-22B (FP32 Accuracy = 38.75\%)} & \multicolumn{4}{|c|}{Nemotron4-15B (FP32 Accuracy = 64.3\%)} \\ 
 \hline
 \hline
 64 & 36.71 & 38.85 & 38.13 & 38.92 & 63.17 & 62.36 & 63.72 & 64.09 \\
 \hline
 32 & 37.95 & 38.69 & 39.45 & 38.34 & 64.05 & 62.30 & 63.8 & 64.33  \\
 \hline
 16 & 38.88 & 38.80 & 38.31 & 38.92 & 63.22 & 63.51 & 63.93 & 64.43  \\
 \hline
\end{tabular}
\caption{\label{tab:mmlu_abalation} Accuracy on MMLU dataset across GPT3-22B, Llama2-7B, 70B and Nemotron4-15B models.}
\end{table}


%\subsection{Perplexity achieved by various LO-BCQ configurations on LM evaluation harness}

\begin{table} \centering
\begin{tabular}{|c||c|c|c|c||c|c|c|c|} 
\hline
 $L_b \rightarrow$& \multicolumn{4}{c||}{8} & \multicolumn{4}{c||}{8}\\
 \hline
 \backslashbox{$L_A$\kern-1em}{\kern-1em$N_c$} & 2 & 4 & 8 & 16 & 2 & 4 & 8 & 16  \\
 %$N_c \rightarrow$ & 2 & 4 & 8 & 16 & 2 & 4 & 2 \\
 \hline
 \hline
 \multicolumn{5}{|c|}{Race (FP32 Accuracy = 37.51\%)} & \multicolumn{4}{|c|}{Boolq (FP32 Accuracy = 64.62\%)} \\ 
 \hline
 \hline
 64 & 36.94 & 37.13 & 36.27 & 37.13 & 63.73 & 62.26 & 63.49 & 63.36 \\
 \hline
 32 & 37.03 & 36.36 & 36.08 & 37.03 & 62.54 & 63.51 & 63.49 & 63.55  \\
 \hline
 16 & 37.03 & 37.03 & 36.46 & 37.03 & 61.1 & 63.79 & 63.58 & 63.33  \\
 \hline
 \hline
 \multicolumn{5}{|c|}{Winogrande (FP32 Accuracy = 58.01\%)} & \multicolumn{4}{|c|}{Piqa (FP32 Accuracy = 74.21\%)} \\ 
 \hline
 \hline
 64 & 58.17 & 57.22 & 57.85 & 58.33 & 73.01 & 73.07 & 73.07 & 72.80 \\
 \hline
 32 & 59.12 & 58.09 & 57.85 & 58.41 & 73.01 & 73.94 & 72.74 & 73.18  \\
 \hline
 16 & 57.93 & 58.88 & 57.93 & 58.56 & 73.94 & 72.80 & 73.01 & 73.94  \\
 \hline
\end{tabular}
\caption{\label{tab:mmlu_abalation} Accuracy on LM evaluation harness tasks on GPT3-1.3B model.}
\end{table}

\begin{table} \centering
\begin{tabular}{|c||c|c|c|c||c|c|c|c|} 
\hline
 $L_b \rightarrow$& \multicolumn{4}{c||}{8} & \multicolumn{4}{c||}{8}\\
 \hline
 \backslashbox{$L_A$\kern-1em}{\kern-1em$N_c$} & 2 & 4 & 8 & 16 & 2 & 4 & 8 & 16  \\
 %$N_c \rightarrow$ & 2 & 4 & 8 & 16 & 2 & 4 & 2 \\
 \hline
 \hline
 \multicolumn{5}{|c|}{Race (FP32 Accuracy = 41.34\%)} & \multicolumn{4}{|c|}{Boolq (FP32 Accuracy = 68.32\%)} \\ 
 \hline
 \hline
 64 & 40.48 & 40.10 & 39.43 & 39.90 & 69.20 & 68.41 & 69.45 & 68.56 \\
 \hline
 32 & 39.52 & 39.52 & 40.77 & 39.62 & 68.32 & 67.43 & 68.17 & 69.30  \\
 \hline
 16 & 39.81 & 39.71 & 39.90 & 40.38 & 68.10 & 66.33 & 69.51 & 69.42  \\
 \hline
 \hline
 \multicolumn{5}{|c|}{Winogrande (FP32 Accuracy = 67.88\%)} & \multicolumn{4}{|c|}{Piqa (FP32 Accuracy = 78.78\%)} \\ 
 \hline
 \hline
 64 & 66.85 & 66.61 & 67.72 & 67.88 & 77.31 & 77.42 & 77.75 & 77.64 \\
 \hline
 32 & 67.25 & 67.72 & 67.72 & 67.00 & 77.31 & 77.04 & 77.80 & 77.37  \\
 \hline
 16 & 68.11 & 68.90 & 67.88 & 67.48 & 77.37 & 78.13 & 78.13 & 77.69  \\
 \hline
\end{tabular}
\caption{\label{tab:mmlu_abalation} Accuracy on LM evaluation harness tasks on GPT3-8B model.}
\end{table}

\begin{table} \centering
\begin{tabular}{|c||c|c|c|c||c|c|c|c|} 
\hline
 $L_b \rightarrow$& \multicolumn{4}{c||}{8} & \multicolumn{4}{c||}{8}\\
 \hline
 \backslashbox{$L_A$\kern-1em}{\kern-1em$N_c$} & 2 & 4 & 8 & 16 & 2 & 4 & 8 & 16  \\
 %$N_c \rightarrow$ & 2 & 4 & 8 & 16 & 2 & 4 & 2 \\
 \hline
 \hline
 \multicolumn{5}{|c|}{Race (FP32 Accuracy = 40.67\%)} & \multicolumn{4}{|c|}{Boolq (FP32 Accuracy = 76.54\%)} \\ 
 \hline
 \hline
 64 & 40.48 & 40.10 & 39.43 & 39.90 & 75.41 & 75.11 & 77.09 & 75.66 \\
 \hline
 32 & 39.52 & 39.52 & 40.77 & 39.62 & 76.02 & 76.02 & 75.96 & 75.35  \\
 \hline
 16 & 39.81 & 39.71 & 39.90 & 40.38 & 75.05 & 73.82 & 75.72 & 76.09  \\
 \hline
 \hline
 \multicolumn{5}{|c|}{Winogrande (FP32 Accuracy = 70.64\%)} & \multicolumn{4}{|c|}{Piqa (FP32 Accuracy = 79.16\%)} \\ 
 \hline
 \hline
 64 & 69.14 & 70.17 & 70.17 & 70.56 & 78.24 & 79.00 & 78.62 & 78.73 \\
 \hline
 32 & 70.96 & 69.69 & 71.27 & 69.30 & 78.56 & 79.49 & 79.16 & 78.89  \\
 \hline
 16 & 71.03 & 69.53 & 69.69 & 70.40 & 78.13 & 79.16 & 79.00 & 79.00  \\
 \hline
\end{tabular}
\caption{\label{tab:mmlu_abalation} Accuracy on LM evaluation harness tasks on GPT3-22B model.}
\end{table}

\begin{table} \centering
\begin{tabular}{|c||c|c|c|c||c|c|c|c|} 
\hline
 $L_b \rightarrow$& \multicolumn{4}{c||}{8} & \multicolumn{4}{c||}{8}\\
 \hline
 \backslashbox{$L_A$\kern-1em}{\kern-1em$N_c$} & 2 & 4 & 8 & 16 & 2 & 4 & 8 & 16  \\
 %$N_c \rightarrow$ & 2 & 4 & 8 & 16 & 2 & 4 & 2 \\
 \hline
 \hline
 \multicolumn{5}{|c|}{Race (FP32 Accuracy = 44.4\%)} & \multicolumn{4}{|c|}{Boolq (FP32 Accuracy = 79.29\%)} \\ 
 \hline
 \hline
 64 & 42.49 & 42.51 & 42.58 & 43.45 & 77.58 & 77.37 & 77.43 & 78.1 \\
 \hline
 32 & 43.35 & 42.49 & 43.64 & 43.73 & 77.86 & 75.32 & 77.28 & 77.86  \\
 \hline
 16 & 44.21 & 44.21 & 43.64 & 42.97 & 78.65 & 77 & 76.94 & 77.98  \\
 \hline
 \hline
 \multicolumn{5}{|c|}{Winogrande (FP32 Accuracy = 69.38\%)} & \multicolumn{4}{|c|}{Piqa (FP32 Accuracy = 78.07\%)} \\ 
 \hline
 \hline
 64 & 68.9 & 68.43 & 69.77 & 68.19 & 77.09 & 76.82 & 77.09 & 77.86 \\
 \hline
 32 & 69.38 & 68.51 & 68.82 & 68.90 & 78.07 & 76.71 & 78.07 & 77.86  \\
 \hline
 16 & 69.53 & 67.09 & 69.38 & 68.90 & 77.37 & 77.8 & 77.91 & 77.69  \\
 \hline
\end{tabular}
\caption{\label{tab:mmlu_abalation} Accuracy on LM evaluation harness tasks on Llama2-7B model.}
\end{table}

\begin{table} \centering
\begin{tabular}{|c||c|c|c|c||c|c|c|c|} 
\hline
 $L_b \rightarrow$& \multicolumn{4}{c||}{8} & \multicolumn{4}{c||}{8}\\
 \hline
 \backslashbox{$L_A$\kern-1em}{\kern-1em$N_c$} & 2 & 4 & 8 & 16 & 2 & 4 & 8 & 16  \\
 %$N_c \rightarrow$ & 2 & 4 & 8 & 16 & 2 & 4 & 2 \\
 \hline
 \hline
 \multicolumn{5}{|c|}{Race (FP32 Accuracy = 48.8\%)} & \multicolumn{4}{|c|}{Boolq (FP32 Accuracy = 85.23\%)} \\ 
 \hline
 \hline
 64 & 49.00 & 49.00 & 49.28 & 48.71 & 82.82 & 84.28 & 84.03 & 84.25 \\
 \hline
 32 & 49.57 & 48.52 & 48.33 & 49.28 & 83.85 & 84.46 & 84.31 & 84.93  \\
 \hline
 16 & 49.85 & 49.09 & 49.28 & 48.99 & 85.11 & 84.46 & 84.61 & 83.94  \\
 \hline
 \hline
 \multicolumn{5}{|c|}{Winogrande (FP32 Accuracy = 79.95\%)} & \multicolumn{4}{|c|}{Piqa (FP32 Accuracy = 81.56\%)} \\ 
 \hline
 \hline
 64 & 78.77 & 78.45 & 78.37 & 79.16 & 81.45 & 80.69 & 81.45 & 81.5 \\
 \hline
 32 & 78.45 & 79.01 & 78.69 & 80.66 & 81.56 & 80.58 & 81.18 & 81.34  \\
 \hline
 16 & 79.95 & 79.56 & 79.79 & 79.72 & 81.28 & 81.66 & 81.28 & 80.96  \\
 \hline
\end{tabular}
\caption{\label{tab:mmlu_abalation} Accuracy on LM evaluation harness tasks on Llama2-70B model.}
\end{table}

%\section{MSE Studies}
%\textcolor{red}{TODO}


\subsection{Number Formats and Quantization Method}
\label{subsec:numFormats_quantMethod}
\subsubsection{Integer Format}
An $n$-bit signed integer (INT) is typically represented with a 2s-complement format \citep{yao2022zeroquant,xiao2023smoothquant,dai2021vsq}, where the most significant bit denotes the sign.

\subsubsection{Floating Point Format}
An $n$-bit signed floating point (FP) number $x$ comprises of a 1-bit sign ($x_{\mathrm{sign}}$), $B_m$-bit mantissa ($x_{\mathrm{mant}}$) and $B_e$-bit exponent ($x_{\mathrm{exp}}$) such that $B_m+B_e=n-1$. The associated constant exponent bias ($E_{\mathrm{bias}}$) is computed as $(2^{{B_e}-1}-1)$. We denote this format as $E_{B_e}M_{B_m}$.  

\subsubsection{Quantization Scheme}
\label{subsec:quant_method}
A quantization scheme dictates how a given unquantized tensor is converted to its quantized representation. We consider FP formats for the purpose of illustration. Given an unquantized tensor $\bm{X}$ and an FP format $E_{B_e}M_{B_m}$, we first, we compute the quantization scale factor $s_X$ that maps the maximum absolute value of $\bm{X}$ to the maximum quantization level of the $E_{B_e}M_{B_m}$ format as follows:
\begin{align}
\label{eq:sf}
    s_X = \frac{\mathrm{max}(|\bm{X}|)}{\mathrm{max}(E_{B_e}M_{B_m})}
\end{align}
In the above equation, $|\cdot|$ denotes the absolute value function.

Next, we scale $\bm{X}$ by $s_X$ and quantize it to $\hat{\bm{X}}$ by rounding it to the nearest quantization level of $E_{B_e}M_{B_m}$ as:

\begin{align}
\label{eq:tensor_quant}
    \hat{\bm{X}} = \text{round-to-nearest}\left(\frac{\bm{X}}{s_X}, E_{B_e}M_{B_m}\right)
\end{align}

We perform dynamic max-scaled quantization \citep{wu2020integer}, where the scale factor $s$ for activations is dynamically computed during runtime.

\subsection{Vector Scaled Quantization}
\begin{wrapfigure}{r}{0.35\linewidth}
  \centering
  \includegraphics[width=\linewidth]{sections/figures/vsquant.jpg}
  \caption{\small Vectorwise decomposition for per-vector scaled quantization (VSQ \citep{dai2021vsq}).}
  \label{fig:vsquant}
\end{wrapfigure}
During VSQ \citep{dai2021vsq}, the operand tensors are decomposed into 1D vectors in a hardware friendly manner as shown in Figure \ref{fig:vsquant}. Since the decomposed tensors are used as operands in matrix multiplications during inference, it is beneficial to perform this decomposition along the reduction dimension of the multiplication. The vectorwise quantization is performed similar to tensorwise quantization described in Equations \ref{eq:sf} and \ref{eq:tensor_quant}, where a scale factor $s_v$ is required for each vector $\bm{v}$ that maps the maximum absolute value of that vector to the maximum quantization level. While smaller vector lengths can lead to larger accuracy gains, the associated memory and computational overheads due to the per-vector scale factors increases. To alleviate these overheads, VSQ \citep{dai2021vsq} proposed a second level quantization of the per-vector scale factors to unsigned integers, while MX \citep{rouhani2023shared} quantizes them to integer powers of 2 (denoted as $2^{INT}$).

\subsubsection{MX Format}
The MX format proposed in \citep{rouhani2023microscaling} introduces the concept of sub-block shifting. For every two scalar elements of $b$-bits each, there is a shared exponent bit. The value of this exponent bit is determined through an empirical analysis that targets minimizing quantization MSE. We note that the FP format $E_{1}M_{b}$ is strictly better than MX from an accuracy perspective since it allocates a dedicated exponent bit to each scalar as opposed to sharing it across two scalars. Therefore, we conservatively bound the accuracy of a $b+2$-bit signed MX format with that of a $E_{1}M_{b}$ format in our comparisons. For instance, we use E1M2 format as a proxy for MX4.

\begin{figure}
    \centering
    \includegraphics[width=1\linewidth]{sections//figures/BlockFormats.pdf}
    \caption{\small Comparing LO-BCQ to MX format.}
    \label{fig:block_formats}
\end{figure}

Figure \ref{fig:block_formats} compares our $4$-bit LO-BCQ block format to MX \citep{rouhani2023microscaling}. As shown, both LO-BCQ and MX decompose a given operand tensor into block arrays and each block array into blocks. Similar to MX, we find that per-block quantization ($L_b < L_A$) leads to better accuracy due to increased flexibility. While MX achieves this through per-block $1$-bit micro-scales, we associate a dedicated codebook to each block through a per-block codebook selector. Further, MX quantizes the per-block array scale-factor to E8M0 format without per-tensor scaling. In contrast during LO-BCQ, we find that per-tensor scaling combined with quantization of per-block array scale-factor to E4M3 format results in superior inference accuracy across models. 



\end{document}


