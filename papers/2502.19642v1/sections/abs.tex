\begin{abstract}

Learning representations that are useful for unknown downstream tasks is a fundamental challenge in representation learning. Prominent approaches in this domain include contrastive learning, self-supervised masking, and denoising auto-encoders. In this paper, we introduce a novel method, termed contrastive Mutual Information Machine (cMIM), which aims to enhance the utility of learned representations for downstream tasks. cMIM integrates a new contrastive learning loss with the Mutual Information Machine (MIM) learning framework, a probabilistic auto-encoder that maximizes the mutual information between inputs and latent representations while clustering the latent codes. Despite MIM's potential, initial experiments indicated that the representations learned by MIM were less effective for discriminative downstream tasks compared to state-of-the-art (SOTA) models. The proposed cMIM method directly addresses this limitation.

The main contributions of this work are twofold: (1) We propose a novel contrastive extension to MIM for learning discriminative representations which eliminates the need for data augmentation and is robust to variations in the number of negative examples (\ie, batch size). (2) We introduce a generic method for extracting \textit{informative embeddings} from encoder-decoder models, which significantly improves performance in discriminative downstream tasks without requiring additional training. This method is applicable to any pre-trained encoder-decoder model.

By presenting cMIM, we aim to offer a unified generative model that is effective for both generative and discriminative tasks. Our results demonstrate that the learned representations are valuable for downstream tasks while maintaining the generative capabilities of MIM.

\end{abstract}
