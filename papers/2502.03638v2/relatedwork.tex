\section{Related Work}
There has been a growing body of work in developing machine-learning methods for crystal structure modeling, including the development of datasets and benchmarks \citep{jain2013commentary,saal2013materials,chanussot2021open, miret2023the, lee2023matsciml,choudhary2024jarvis}. Recent work has also focused on developing
architectures that are equivariant to various symmetries \citet{duval2023hitchhiker} or are specifically designed to include inductive biases useful for crystal structures \citep{xie2018crystal,kaba2022equivariant, goodall2022,yan2022periodic,yan2024space}.

In addition to structure-based modeling, prior work has also generated full-atom crystal structures, in which all atoms of the three-dimensional structure are generated.
A range of generation methods including VAEs \citep{noh2019inverse,xie2022crystal}, GANs \citep{nouira2018crystalgan,kim2020generative}, reinforcement learning \citep{govindarajan2023learning}, diffusion models \citep{zeni2023mattergen, yang2023scalable,jiao2023crystal,klipfel2024vector}, flow-matching models \citep{millerflowmm}, and active learning based discovery \citep{merchant2023scaling} have been used.
 These follow similar works in 3D molecule generation \citep{hoogeboom2022equivariant, garcia2021n,  xu2022geodiff}, but extend them by incorporating crystal periodicity.
In addition to full-atom crystal generation, prior work has also applied text-based methods to understand and generate crystals using language models \citep{flam2023language,gruverfine, alampara2024mattext}.

Other works have pointed out the importance of symmetry of the generated structures. DiffCSP++ \citep{jiao2024space}, does so by using predefined structural templates from the training data and learning atomic types and coordinates compatible with the templates. While this is an interesting solution, we show that predefining the templates in this way severely limits the diversity and novelty of the generated samples. CrystalGFN \citep{ai4science2023crystal} incorporates constraints on the lattice parameters and composition based on space groups but does not guarantee that the atomic positions respect the desired symmetry. Finally, the concurrent works CrystalFormer \citep{cao2024space} and WyCryst 
\citep{zhu2024wycryst}  generate symmetric crystals by predicting atom symmetries. However, they directly use the labels of Wyckoff positions to encode symmetries, which does not enable generalization across groups. The methods are therefore limited to generating from space groups that are common in the dataset. By contrast, our method generalizes across groups and can generate valid crystals even from groups that are rare in the dataset.