\section{Conclusion}
\label{Conclusion}
This study presents a knowledge-augmented code generation framework that integrates domain expertise, software engineering principles, and Generative AI to automate GIS-based web application development. Leveraging Context-Aware Visual Prompting and RAG, it transforms user-defined UI wireframes into scalable, maintainable front-end code for environmental and energy data visualization dashboards. The framework bridges the gap between domain scientists and software engineering, enabling users with minimal web development experience to generate functional GIS applications. Case studies demonstrate its effectiveness in generating two interactive dashboards: a meteorological data dashboard for visualizing time-series and spatial datasets from tower sensors, and a wind turbine and land use dashboard overlaying wind farm locations with land cover change data. These AI-generated dashboards support real-time exploration and analysis, ensuring data integrity and usability for scientific research and policy-making. By adopting a modular React-based architecture and integrating software engineering best practices such as MVVM and Separation of Concerns (SoC), the framework enhances scalability, maintainability, and reusability. It highlights the potential of AI-driven UI generation to reduce development time, improve consistency, and streamline GIS visualization tool creation.

Despite these advancements, the framework currently focuses on front-end code generation, requiring manual backend integration. Human expert review remains necessary for validating AI-generated code, optimizing performance, and ensuring compliance with project requirements. Future research will explore automated validation, backend integration, and support for additional front-end frameworks like Angular and Vue.js. Comparative evaluations of LLMs will assess their effectiveness in scientific software development. This work advances AI-assisted software development by pioneering a structured, knowledge-driven approach to LLM-powered code generation. By enhancing accessibility and automation in GIS web development, it lays the foundation for AI-driven environmental monitoring, digital twins, and smart city analytics.