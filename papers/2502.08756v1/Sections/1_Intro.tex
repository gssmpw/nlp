\section{Introduction}
\label{Introduction}
Over the past decades, scientific web applications (Web Apps), such as CyberGIS systems, visual analytics dashboards, digital twin platforms, and online decision support systems, have become indispensable tools for both the scientific community and the public to discover, query, visualize, and download vast urban and environmental data sets for smart city research \citep{ferre2022adoption, dembski2020urban}. Aligned with the NSF's cyberinfrastructure (CI) initiative, academia and government agencies have increasingly adopted these applications as part of interdisciplinary informatics projects, providing effective, user-friendly tools for data dissemination \citep{yu2021coevolution}. With advances in internet and communication technologies, computing hardware, and artificial intelligence, these tools are transforming urban and environmental research by enabling data- and simulation-driven insights for decision support \citep{kadupitige2022enhancing}, fostering collaborative research through data and simulation integration \citep{parashar2019virtual} and enhancing education and public engagement in citizen science and voluntary data collection \citep{skarlatidou2019volunteers}. Key application areas include water resource management \citep{souffront2018cyberinfrastructure, xu2022overview}, hazard mitigation \citep{mandal2024prime, xu2020web, garg2018cloud}, intelligent transportation systems \citep{xu2023smart, xu2022interactive, ghosh2017intelligent}, connected and automated vehicles \citep{xu2023mobile, kampmann2019dynamic}, built-environment and building energy management \citep{jia2019adopting, kim2022design, xu2022geo}, pandemic management \citep{xu2021episemblevis, li2021emerging, thakur2020covid}, and urban planning and design \citep{alatalo2017two}. Numerous interdisciplinary studies highlight the transformative potential of these web applications in advancing environmental and urban research, as well as smart city management. Their ongoing evolution is driven by the integration of emerging technologies like artificial intelligence, the Internet of Things (IoT), edge computing, and cyber-physical systems.

Despite advancements in scientific web applications, developing customized tools like cyberGIS and digital twin platforms for integrating and visualizing diverse environmental or urban data (e.g., hydrological, traffic flow, meteorological data, or simulations) remains highly demanding and resource-intensive \citep{shanjun2024design, siddiqui2024digital, lei2023challenges}. These efforts require expertise in software and data engineering, as well as time-consuming tasks like client-server development, database management, real-time analytics, machine learning, and simulations \citep{ikegwu2022big}. Consequently, researchers often shift from their core scientific work to learn complex web programming, UI/UX design, and database technologies \citep{li2022bibliometric}. Although modern software practices like design patterns aim to streamline development, their effective use demands specialized software engineering knowledge \citep{fayad2015software}. Researchers with data analytics expertise often lack formal software development experience, facing challenges even when skilled engineers are involved \citep{kim2017data}. Designing, deploying, and maintaining large-scale web apps remains labor-intensive, limiting scalability and adaptability in platforms like cyberGIS or digital twins \citep{shah2024optimizing, mcbreen2002software, liu2015cybergis}. Emerging Generative AI (GenAI) technologies offer potential to automate web development for environmental and urban research. While studies have shown the feasibility of training large language models (LLMs) like GPT for automating data analytics and development tasks \citep{liang2024can, liukko2024chatgpt}, challenges remain for complex scientific web apps due to inefficiencies in prompting methods, limitations in domain knowledge, and LLMs' attention mechanisms, which are trained on generalized text data.

\begin{figure*}[htb]
 \centering
\includegraphics[width=\textwidth]{Figures/Figure_Concept.pdf}
 \caption{From annotated wireframe to code generation, a knowledge driven framework for automated software development of cyberGIS platform for visualizing time-series and spatial data. }
 \label{fig:concepts}
\end{figure*}

This paper introduces a knowledge-augmented code generation framework that integrates software engineering best practices, domain expertise, and advanced software stacks to enhance Generative Pre-trained Transformers (GPT) for front-end development. The framework automates the generation of GIS-based web applications, including visualization dashboards and analytical interfaces, directly from user-defined UI wireframes created in tools like PowerPoint or Adobe Illustrator. It leverages industry-standard web frameworks and software engineering best practices to ensure scalability, maintainability, and efficiency. We propose a novel context-Aware visual prompting method, implemented in Python, that interprets wireframes to extract layout structures and interface elements, enabling LLMs to generate front-end code while incorporating both software engineering principles and domain knowledge through Chain-of-Thought (CoT) and Retrieval-Augmented Generation (RAG) paradigms. A case study demonstrates the framework’s capability to autonomously generate a modular and maintainable web platform hosting multiple dashboards for visualizing and interacting with environmental and energy data, including time-series data sets, GIS shapefiles, and raster imagery, based on user-sketched wireframes. By leveraging a knowledge-driven approach, the framework ensures that the generated front-end code adheres to industry-standard software design and architectural patterns, including the React framework and Model-View-ViewModel (MVVM) architecture, facilitating long-term scalability and maintainability. This approach significantly reduces manual effort in UI/UX design, coding, and maintenance, pioneering a scalable, autonomous, and robust solution for developing web applications that support smart city advancements.

 