\begin{abstract} \label{sec:abstract}
Developing web-based GIS applications, commonly known as CyberGIS dashboards, for querying and visualizing GIS data in environmental research often demands repetitive and resource-intensive efforts. While Generative AI offers automation potential for code generation, it struggles with complex scientific applications due to challenges in integrating domain knowledge, software engineering principles, and UI design best practices.
This paper introduces a knowledge-augmented code generation framework that retrieves software engineering best practices, domain expertise, and advanced technology stacks from a specialized knowledge base to enhance Generative Pre-trained Transformers (GPT) for front-end development. The framework automates the creation of GIS-based web applications (e.g., dashboards, interfaces) from user-defined UI wireframes sketched in tools like PowerPoint or Adobe Illustrator. A novel Context-Aware Visual Prompting method, implemented in Python, extracts layouts and interface features from these wireframes to guide code generation.
Our approach leverages Large Language Models (LLMs) to generate front-end code by integrating structured reasoning, software engineering principles, and domain knowledge, drawing inspiration from Chain-of-Thought (CoT) prompting and Retrieval-Augmented Generation (RAG). A case study demonstrates the framework’s capability to generate a modular, maintainable web platform hosting multiple dashboards for visualizing environmental and energy data (e.g., time-series, shapefiles, rasters) from user-sketched wireframes.
By employing a knowledge-driven approach, the framework produces scalable, industry-standard front-end code using design patterns such as Model-View-ViewModel (MVVM) and frameworks like React. This significantly reduces manual effort in design and coding, pioneering an automated and efficient method for developing smart city software.







%marks a significant reduction in manual software design and coding efforts. Our research presents a pioneering path towards next-generation AI-powered urban digital twins that are self-generating, self-organizing, and self-executing, demonstrating the transformative potential of LLMs in urban technology development.



 
\end{abstract} \label{sec:abstract}
\begin{keyword}
Generative AI \sep Large Language Model \sep GIS \sep Visual Dashboard \sep Ontology and Knowledge Graph 
\sep Software Design Patterns
\end{keyword}
 \maketitle