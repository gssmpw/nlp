\section{Result Demonstration}
\label{subsec:Data}
Using our prototyping framework, we automated the generation of front-end code for two web-based GIS dashboards addressing distinct use cases involving environmental and energy infrastructure data. To showcase the advantages of the proposed framework, which leverages knowledge-augmented code generation guided by software engineering best practices and industry standards, we developed both dashboards as single-page applications using the React framework, integrating them within the same React project. 

\subsection{Case Study I - Meteorological Data Dashboard}
Access to continuous and high-quality meteorological data is essential for understanding regional climatology and atmospheric processes. Such data plays a crucial role in research efforts focused on assessing local climate patterns, modeling atmospheric dispersion, evaluating emissions, and ensuring environmental and operational safety. Research institutions like Oak Ridge National Laboratory (ORNL) require reliable meteorological measurements to support site operation, emergency preparedness, and environmental monitoring. However, meteorological data collection is often subject to various challenges, including sensor degradation, power fluctuations, lightning strikes, and instrument failures, all of which can introduce uncertainties and affect data reliability \citep{steckler_2025}. 

To address these challenges, this study aims to leverage the prototyping framework's ability to generate the code base for a robust visual dashboard to enhances the quality and usability of meteorological data collected at ORNL by placing domain experts in the loop to supervise the data collocation and quality control processes. Specifically, a comprehensive quality assessment was conducted using a statistical framework to process of five years of meteorological data, ensuring data integrity and continuity. The visual dashboard is developed to assist the visual exploration and supervision of the data outputs from the statistical framework 
The primary objective is to produce a high-quality, gap-filled dataset that supports accurate atmospheric dispersion modeling, which is crucial for understanding pollutant dispersion and regional air quality dynamics.

The study focuses on meteorological data collected from the Oak Ridge Reservation (ORR), located in East Tennessee. ORR is characterized by a complex ridge-and-valley topography, which significantly influences local wind patterns and atmospheric dispersion processes. The region’s terrain-driven microclimate poses challenges for meteorological modeling, making it essential to have high-quality, site-specific meteorological data. ORNL operates on-site meteorological towers designed to capture critical weather variables, providing a valuable resource for climatological and atmospheric studies.


\begin{figure*}[htbp]
 \centering
\includegraphics[width=\textwidth]{Figures/Figure_Demo1.pdf}
 \caption{The UI wireframe for the homepage of the web-based application includes thumbnails that serve as navigation links, directing users to the dashboard generated for Use Case I. }
 \label{fig:demo-1}
\end{figure*}


\begin{figure*}[htbp]
 \centering
\includegraphics[width=\textwidth]{Figures/Figure_Demo2.pdf}
 \caption{The UI wireframe for the visual dashboard of Use Case I is designed for visualizing meteorological data, incorporating time-series data from tower sensors and shapefiles representing site locations. }
 \label{fig:demo-2}
\end{figure*}

\subsection{Case Study II - Wind Turbine and Landuse Data}
The expansion of renewable energy infrastructure, such as wind farms, has raised concerns about its potential ecological impacts on bird habitats. Previous studies have assessed bird habitats using bird-watching surveys and remote sensing data on natural vegetation cover, offering valuable insights into avian ecology. Building on these methods, this case study investigates the hypothesis that noise and land cover changes resulting from wind turbine operations may displace grassland- and forest-dwelling birds. To explore this, we conducted a preliminary study using data from the United States Geological Survey (USGS)’s Wind Turbine Database (USWTDB) and correlated it with a 20-year time series of land cover changes from the Multi-Resolution Land Characteristics (MRLC)’s National Land Cover Database (NLCD).

This case study focuses on wind farm sites across multiple states in the United States. The USWTDB provides detailed GIS data on wind turbine locations, construction years, and operational specifications, which are linked to land cover changes documented by the NLCD. The NLCD dataset includes high-resolution raster-based land cover data, capturing variations in vegetation and natural land cover over two decades. By comparing land cover data before and after the establishment of wind farms, the study identifies patterns of vegetation loss and fragmentation caused by infrastructure development, including roads, facilities, and pavements.

To validate these findings, this case study aims to develop a web-based GIS dashboard that integrates the time series of land cover changes from the NLCD dataset with wind turbine locations retrieved from the USWTDB. The dashboard visually overlays land cover rasters with wind turbine locations, enabling users to assess potential land cover changes caused by wind farm operations. This tool highlights areas where wildlife conservation strategies may be needed. By providing insights into the ecological impacts of wind farms, this approach establishes a practical framework for mitigating habitat loss and protecting avian species affected by renewable energy development.

\begin{figure*}[htbp]
 \centering
\includegraphics[width=\textwidth]{Figures/Figure_Demo3.pdf}
 \caption{A demonstration of the dashboard's capability to visualize meteorological data (e.g., shapfiles and time-series) for Use Case I. }
 \label{fig:demo-3}
\end{figure*}

\begin{figure*}[htbp]
 \centering
\includegraphics[width=\textwidth]{Figures/Figure_Demo4.pdf}
 \caption{A demonstration of the dashboard's capability to visualize land use and land cover raster data for Use Case II.}
 \label{fig:demo-4}
\end{figure*}

\subsection{Context-aware Visual Prompting}
We present the UIs of the web-based application, which are generated from user-defined wireframes for the homepage (as depicted in Figure \ref{fig:demo-1}) and the visual dashboard for Use Case I (as illustrated in Figure \ref{fig:demo-2}). These UIs are integrated as distinct routes within a single React project, enhancing scalability and extensibility to accommodate additional GIS dashboards under the same project while maintaining a consistent UI style through code reuse. This approach improves code maintainability by adhering to the singleton software engineering principle, implemented using the React framework.

The context-aware visual prompting technique is designed exclusively for generating structured instructions and prompts to enable the LLM to produce front-end code. In our case studies, we developed the back-end application separately to provide web-based API endpoints for querying and retrieving data from the database using the Python FastAPI framework. However, the development of the back-end application is not within the scope of our proposed framework.

\subsection{AI-generated Dashboard Demonstration}
For the meteorological data dashboard, the AI-generated interface provides an interactive platform for exploring time-series meteorological data collected from the ORR. The dashboard enables real-time visualization of meteorological variables captured by tower sensors, including temperature, wind speed, humidity, and atmospheric pressure (as shown in Figure \ref{fig:demo-3}). The AI-driven dashboard generation process ensures that data integrity is preserved by integrating statistical quality assessments to identify missing or inconsistent measurements.

Users can interact with the dashboard to:
\begin{enumerate}
\item Query sensor measurements at different sites, with their locations visualized on the map.
\item Visualize time-series meteorological trends for a large number of parameters over different time periods.
\item Display statistical summaries of the selected time-series data.
\end{enumerate}
The AI-generated dashboard effectively places domain experts in the loop, allowing them to supervise data quality and validate automated statistical assessments, thereby improving the usability of long-term meteorological data sets for atmospheric modeling and environmental monitoring.


For Use Case II, the AI-generated dashboard integrates GIS-based land cover data with wind turbine locations, providing a comprehensive platform for analyzing the ecological impacts of wind energy infrastructure (as depicted in Figure \ref{fig:demo-4}). By utilizing AI-assisted dashboard generation, the system automatically organizes spatial raster data sets and overlays them with wind turbine distributions to reveal patterns of land cover transformation.
The dashboard enables users to perform the following functionalities:
\begin{enumerate}
    \item Examine land cover changes before and after wind farm construction using a 20-year historical data set. Compare the extent of vegetation loss and landscape fragmentation near wind turbine sites.
    \item Assess potential ecological risks associated with wind farm expansion by correlating turbine operations with habitat shifts.
\end{enumerate}
This AI-enhanced approach to dashboard generation significantly reduces development time while improving the consistency and maintainability of GIS visualization tools. By combining AI-assisted UI design with data-driven analysis, this study demonstrates the potential of AI in advancing interactive environmental monitoring systems.
 

\subsection{Limitation and Future Work}
As our study primarily focuses on prototyping a knowledge-driven framework to demonstrate the feasibility of guiding LLMs for adaptive, pattern-driven code generation in creating robust web-based GIS applications using software engineering best practices and industry-grade web frameworks, we do not delve into theoretical or algorithmic research for evaluating the performance of various LLMs and generative AI technologies. Instead, the primary emphasis is on the application and use cases presented in the study. However, our existing framework has several limitations, as listed below: 

\begin{description}
    \item[Lack of Comparative Analysis across Different LLMs] The framework's performance and effectiveness were evaluated using a single large language model (LLM) to demonstrate feasibility. Future work should include comparative studies across various LLMs to assess their suitability for different domains and coding scenarios.
    
    \item[Requires Human Expert Review of AI-Generated Code] While the framework automates front-end code generation, the resulting code still requires human experts to review for correctness, optimization, and adherence to specific project requirements. Future advancements could integrate automatic validation tools or explainable AI mechanisms to reduce dependency on manual reviews.

    \item[Customized Functions Require Manual Programming Efforts] Although the framework automates standard functionalities, developing highly customized features still requires manual programming, which limits full automation. Future iterations of the framework could incorporate a mechanism to better support user-defined customizations through enhanced prompt engineering or plug-and-play modular components.

    \item[Limited Support for Backend Integration] The current framework focuses on front-end code generation with minimal backend integration capabilities. Extending the framework to support full-stack development workflows, including database and API integration, would be a valuable addition.

    \item[Scalability to Larger Projects] The prototyping framework has been demonstrated on two dashboards within a single React project. Future work could explore its scalability to larger and more complex multi-application systems, addressing performance and maintainability challenges.

    \item[Generalizability to Non-React Frameworks] The framework is currently optimized for React-based projects. Future research should evaluate its adaptability to other popular front-end frameworks, such as Angular or Vue.js, to enhance its applicability across diverse development environments.
\end{description}

It is anticipated that the next stage efforts will include conducting a comparative analysis of multiple LLMs to evaluate their performance and adaptability in generating front-end code for GIS web applications. Automated validation mechanisms should also be developed to ensure correctness, optimization, and adherence to best practices, reducing reliance on human review. Additionally, enhancing the framework's support for user-defined customizations would streamline the development of complex features without significant manual effort. Expanding the framework to include backend development capabilities, such as API and database integration, would enable full-stack automation. To broaden applicability, the framework should be generalized to support multiple front-end frameworks, including Angular and Vue.js. Real-time collaboration features can further improve the framework by enabling teams to seamlessly work on AI-generated code. Lastly, domain-specific enhancements tailored to areas like environmental monitoring and urban planning could optimize the framework for specialized use cases.