This research presents a system-based approach to automate the software development of sophisticated GIS-based web applications to facilitate the urban and environmental management in smart city research, using ontology-driven prompts and ChatGPT, a widely used large language model (LLM). Our approach aims to leverage the AI model's generative, reasoning, and learning capability to create a primitive autonomous agent that helps academia researchers and would-be developers lower the technical barrier and cost to developing cyberinfrastructure and scientific software tools. At the technical level, we implemented a prototyping framework that employ a Knowledge Graph (KG) as the digital representation of software engineering practices, popular application stacks, and domain science knowledge to generate, tune, and enhance the engineered prompts through iterative procedures, creating accurate instructions to guide ChatGPT 4.0 API and LangChain to automate the development of a web-based geo-visual dashboards using widely-used frameworks and technologies favored by the industry. We demonstrate the performance and feasibility of our framework through a case of an AI-generated web applications. Through this study, we explore the LLM's potential as a full-stack web developer and GIS analyst trained by scientific knowledge graph to utilize advanced software design and architecture patterns (e.g., MVC and MVVM) and complex scientific data () to produce robust web app on both the server- and client-side. 