referring to different sections of a React JSX script as "code blocks" is understandable, but in React and software engineering terminology, these sections are typically called "code sections", "declarations", or more specifically by their respective roles. Here are the appropriate terms for each part:

Import Statements – This section at the top of the file imports necessary dependencies, such as React, third-party libraries, or custom components.
State and Variable Declarations – These include React state variables (e.g., useState), refs (useRef), and context variables (useContext) that manage data within the component.
Event Handler Functions – Functions defined within the component that handle events such as user interactions (e.g., onClick, onChange).
Effect Hooks and Lifecycle Methods – Side effects (e.g., API calls, subscriptions) are managed within useEffect in functional components or componentDidMount, componentDidUpdate, and componentWillUnmount in class components.
JSX Return Block (Render Logic) – The rendered UI structure, typically within the return() statement in functional components or the render() method in class components.
Custom Hooks or Utility Functions – If the component includes custom logic for reusability, it may contain custom hooks or helper functions.
Software Engineering Term
The most precise term to describe these functionally distinct sections within a React JSX file is "logical sections" or "concerns". A more structured term based on Separation of Concerns (SoC) would be "functional concerns" in a React component.

However, when discussing different code sections in a single file, a widely accepted term is "component structure" or "React component composition", referring to the way code is organized within a React component.

Instead of "code blocks", you may refer to them as:

Logical sections of a React component
Functional concerns in a React file
Component structure in JSX
React composition pattern
Would you like me to refine this further for a specific use case?