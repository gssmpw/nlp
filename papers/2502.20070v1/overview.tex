
%--------------------------------------------------------
%--------------------------------------------------------
\section{Overview}
\label{sec:overview}

In this section, we recall a standard method for dynamic deadlock
detection~\cite{10.5555/645880.672085,DBLP:conf/spin/Harrow00}, point out its shortcomings, and explain our approach to
addressing these shortcomings.

Our discussion relies on traces like $T_1$ shown in \cref{fig:ex0}.
The left part of the diagram represents a program run by a trace of events.
We briefly explain the notation; \cref{sec:prelim} includes formal definitions and details.
The diagram visualizes the interleaved execution of the program's events using
a tabular notation with a separate column for each thread and one
event per row.
The order from top to bottom reflects the observed temporal order of events.

Each event takes place in a specific thread and
represents an operation.
% Operation $\forkE{t}$ creates a new thread $t$ and
% operation $\joinE{t}$ synchronizes with the termination of thread $t$.
We write $\LKA, \LKB, \ldots$ to denote locks and  $\VA, \VB, \ldots$ to denote shared variables.
Operations $\lockE{\LKA}$ and~$\unlockE{\LKA}$ acquire and release lock~$\LKA$, respectively.
Operations $\readE{\VA}$ and $\writeE{\VA}$ are read and write operations, respectively, on shared variable~$\VA$.
The same operation may appear multiple times in a trace, so
we use indices $e_i$ to uniquely identify events in the trace.
Traces are formally expressed as lists, e.g.,
$[e_1,e_2, e_3, e_4, e_5, e_6, e_7,e_8]$ for $T_1$.

\subsection{Lock Dependencies}

\begin{figure}[t]

  \begin{minipage}[b]{.44\textwidth}
    \bda{@{}lcl}
      % ex0
      \begin{array}{|l|l|l||l|}
        \hline
        T_1 & \thread{1} & \thread{2} & \mbox{Lock deps} \\
        \hline
        \eventE{1} & \lockE{\LKA} &&  \\
        \eventE{2} & \HIGHLIGHT{\lockE{\LKB}} && \LD{\thread{1}}{\LKB}{\{\LKA\}} \\
        \eventE{3} & \unlockE{\LKB} && \\
        \eventE{4} & \unlockE{\LKA} && \\
        \eventE{5} && \lockE{\LKB} & \\
        \eventE{6} && \HIGHLIGHT{\lockE{\LKA}} & \LD{\thread{2}}{\LKA}{\{\LKB\}} \\
        \eventE{7} && \unlockE{\LKA} & \\
        \eventE{8} && \unlockE{\LKB} & \\
        \hline
      \end{array}

    \eda
    \subcaption{Trace with lock dependencies.}
    \label{fig:ex0}
  \end{minipage}%
  \hfill%
  \begin{minipage}[b]{.55\textwidth}
    \bda{c}


%%latexTrace $ addLoc ex_3
\ba{|l|l|l|l|}
\hline T_2  & \thread{1} & \thread{2} & \mbox{Lock deps} \\ \hline
\eventE{1}  & \lockE{\LKA} && \\
\eventE{2}  & \lockE{\LKB}&& \LD{\thread{1}}{\LKB}{\{\LKA\}}\\
\eventE{3}  & \unlockE{\LKB}&&\\
\eventE{4}  & \unlockE{\LKA}&&\\
\eventE{5}  & \writeE{\VA}&&\\
\eventE{6}  & &\readE{\VA} &\\
\eventE{7}  & &\lockE{\LKB} & \\
\eventE{8}  & &\lockE{\LKA} & \LD{\thread{2}}{\LKA}{\{\LKB\}}\\
\eventE{9}  & &\unlockE{\LKA} &\\
\eventE{10}  & &\unlockE{\LKB} &\\

 \hline \ea{}

    \eda
    \subcaption{False positive due to last-write dependency.
    }\label{fig:ex_3}
  \end{minipage}
  \caption{Traces with potential deadlocks.}
  \label{f:firstExamples}
\end{figure}

In trace $T_1$ in \cref{fig:ex0}, all events in thread $\thread1$ take place before the
events in thread $\thread2$. A different schedule for the same program
might give rise to a \emph{reordered trace prefix} $[e_1,e_5]$, which
indicates that $T_1$ has the potential to deadlock: the highlighted acquire events $e_2$ and $e_6$
are \emph{enabled} to be scheduled next, but either
would break the lock semantics by acquiring a lock already acquired but not yet released by the other tread.


The  standard approach
predicts such situations by constructing a \emph{lock dependency} of
the form $\LD{t}{\LK{}}{L_t}$ for every acquire event~\cite{10.1007/11678779_15,10.1145/1542476.1542489}.
Here, $t$ is the thread that acquires lock $\LK{}$ and $L_t$ is the
set of locks (aka \emph{lockset}) held (acquired but not yet released) by this thread at the point of acquiring lock~$\LK{}$.
The right part of \cref{fig:ex0} shows the lock dependencies for $T_1$.

The two lock dependencies
$
\LD{\thread{1}}{\LKB}{\{\LKA\}}$ and $ \LD{\thread{2}}{\LKA}{\{\LKB\}}
$ obtained from $T_1$ indicate a potential deadlock, because they
exhibit a \emph{cyclic  chain} of lock dependencies according to the following two conditions:
\begin{description}
\item[\cond{DP-Cycle}] The acquired lock $\LKB$ of the first dependency is in the lockset $\{ \LKB \}$
of the second dependency, and the acquired lock $\LKA$ of the second dependency
is in the lockset $\{ \LKA \}$ of the first dependency.
\item[\cond{DP-Guard}] The underlying locksets $\{\LKA \}$ and $\{ \LKB \}$ are disjoint.
\end{description}
%
Condition~\cond{DP-Guard} ensures that the deadlocked situation shown by
the reordered prefix~$[e_1,e_5]$ can be reached without violating lock semantics by acquiring a lock already held by another tread.
% Any thread-local extension of this trace must either comprise of $e_3$ or $e_7$.
% We say that $e_3$ and $e_7$ are \emph{enabled} in $[e_1,e_2,e_6]$.
Condition~\cond{DP-Cycle} characterizes the deadlocked situation, caused by a cycle of lock dependencies.
% Extention of race $[e_1,e_2,e_6]$ with $e_3$ violates the lock semantics.
% The same observation applies to $e_7$.

Instead of writing out the entire cyclic lock-dependency chain,
we use \emph{deadlock patterns}, i.e.,
the sequence of acquire events that constitute the cycle.
For example, the cyclic lock-dependency chain
$\LD{\thread{1}}{\LKB}{\{\LKA\}}$ and $ \LD{\thread{2}}{\LKA}{\{\LKB\}}$
corresponds to the deadlock pattern~$\DP{e_2,e_6}$.


\subsection{False Positives}

The deadlock-pattern approach suffers from false positives.
% Consider trace $T_2$ in \cref{fig:ex1_b} (on the left).
% $T_2$ is a rearrangement of trace $T_1$ where the fork of thread $t_2$
% takes place as the last event in thread $t_1$.
% All other events remain in the same relative position.
% For $T_2$ we obtain the same lock dependencies as for $T_1$, which
% indicates a potential deadlock.
% However, there is no deadlock in $T_2$  because thread $t_2$ can only start
% once thread $t_1$ is finished.
% We call this indication a \emph{false positive}
%
% Besides forking a thread, there can be other inter-thread dependencies
% that lead to false positives:
Consider trace $T_2$ in \cref{fig:ex_3}.
% (on the right)
This trace gives rise
to the same lock dependencies as $T_1$ in \cref{fig:ex0}. % and $T_2$.
Alas, the resulting deadlock pattern is %again
a \emph{false positive},
because it does not correspond to a true deadlock: there is no (correct) reordering of $T_2$ that exhibits the deadlock.
For example, the reordered prefix $[e_1,e_6,e_7]$ is deadlocked on the enabled events $e_2$ and $e_8$ enabled, but it is incorrect:
the read event~$e_6$ no longer observes the same write event~$e_5$ it does in $T_2$ (i.e., the \emph{last write} of $e_6$ is different).
Hence, $e_6$ may read a different value than before, possibly affecting the program flow such that the events $e_7$--$e_{10}$ may never happen.

A similar observation applies to lockset-based data-race prediction methods
where Condition~\cond{DP-Guard} is used to check if two conflicting
memory operations are in a data race.
Two operations are \emph{conflicting} if they refer to the same memory address and at least one of them
is a write.

We conclude that Conditions~\cond{DP-Cycle} and~\cond{DP-Guard} are not sufficient to guarantee a deadlock: additional conditions are necessary to rule out false positives.
A popular method in the area of data-race prediction is to derive a partial order on
events from the program trace.
Ideally, this partial order captures the inter-event dependencies imposed be the trace and its semantics, making it possible to reason about correctly reordered traces without exploring all possible interleavings.
This way, e.g., a partial order may order a read after its last write, or order the events within the same thread.
Unordered events are then considered concurrent: they may appear in different orders among correct reorderings.
Hence, a data race is signaled if two conflicting memory operations are concurrent.
A sufficiently strong partial order then eliminates falsely signaled data races.


\begin{figure}[t]

  \begin{minipage}[b]{.3\textwidth}
    \bda{@{}lcl}
%%latexTrace $ addLoc ex_3b
\ba{|l|l|l|l|}
\hline T_3  & \thread{1} & \thread{2} & \thread{3}\\ \hline
\eventE{1}  & \lockE{\LKA}&&\\
\eventE{2}  & \lockE{\LKB}&&\\
\eventE{3}  & \unlockE{\LKB}&&\\
\eventE{4}  & \unlockE{\LKA}&&\\
\eventE{5}  & &\lockE{\LKB}&\\
\eventE{6}  & &\writeE{\VA}&\\
\eventE{7}  & &\unlockE{\LKB}&\\
\eventE{8}  & &&\lockE{\LKB}\\
\eventE{9}  & &&\writeE{\VA}\\
\eventE{10}  & &&\lockE{\LKA}\\
\eventE{11}  & &&\unlockE{\LKA}\\
\eventE{12}  & &&\unlockE{\LKB}\\

 \hline \ea{}

 \eda
     \subcaption{WCP false negative.} %% SDP
    \label{fig:ex3b}
  \end{minipage}
  \hfill%
  \begin{minipage}[b]{.55\textwidth}
    \bda{@{}lcl}

    % dl_11
    \begin{array}{|l|l|l||l|l|}
      \hline
      T_4 & \thread{1} & \thread{2} & \multicolumn{2}{|l|}{\mbox{Lock deps}} \\
      \hline
      \eventE{1} & \HIGHLIGHT{\lockE{\LKA}} &&& \\
      \eventE{2} & \lockE{\LKB} && D_1 & \LD{\thread{1}}{\LKB}{\{\HIGHLIGHT{\LKA}\}} \\
      \eventE{3} & \unlockE{\LKB} &&& \\
      \eventE{4} & \lockE{\LKC} &&& \LD{\thread{1}}{\LKC}{\{\LKA\}} \\
      \eventE{5} & \lockE{\LKD} && D_2 & \LD{\thread{1}}{\LKD}{\{\HIGHLIGHT{\LKA},l_3\}} \\
      \eventE{6} & \unlockE{\LKD} &&& \\
      \eventE{7} & \unlockE{\LKC} &&& \\
      \eventE{8} & \unlockE{\LKA} &&& \\
      \eventE{9} && \HIGHLIGHTB{\lockE{\LKB}} && \\
      \eventE{10} && \lockE{\LKA} & D_3 & \LD{\thread{2}}{\LKA}{\{\HIGHLIGHTB{\LKB}\}} \\
      \eventE{11} && \unlockE{\LKA} && \\
      \eventE{12} && \lockE{\LKD} && \LD{\thread{2}}{\LKD}{\{\LKB\}} \\
      \eventE{13} && \lockE{\LKC} & D_4 & \LD{\thread{2}}{\LKC}{\{\HIGHLIGHTB{\LKB},l_4\}} \\
      \eventE{14} && \unlockE{\LKC} && \\
      \eventE{15} && \unlockE{\LKD} && \\
      \eventE{16} && \unlockE{\LKB} && \\
      \hline
    \end{array}

    \eda
    \subcaption{Trace where only the first deadlock pattern is feasible.}
    \label{fig:dp-order}
  \end{minipage}%
  \caption{Partial orders for deadlock prediction.}
  \label{f:traceBlock}
\end{figure}


\subsection{Partial-order Methods for Deadlock Prediction
            %% to Eliminate False Positives
            }

We apply the partial-order idea to the deadlock-prediction setting and refine the definition of a deadlock pattern
$\DP{e_1,\ldots,e_n}$ with the following additional condition:
\begin{description}
  \item[\cond{DP-P}] Events $e_i$ are pairwise concurrent under partial order~P.
\end{description}
%
Which partial order P to use?
Answering this question is a non-trivial task, shown by
review of a number of existing partial orders that have been applied
in the data-race setting.

\paragraph{Happens-Before is too strict.}

We first consider the Happens-Before relation (HB)~\cite{lamport1978time}.
Under HB, lock releases and acquisitions on the same lock
are ordered as they appear in the trace.
For trace $T_2$ in \cref{fig:ex0}, we
then find that $e_4 \HBLt e_5$.
Since HB is closed under trace order within the same thread (program order) and transitivity,
we find that $e_2 \HBLt e_{6}$: these lock acquires are not concurrent.
Thus, Condition~\cond{DP-HB} eliminates the (false-positive) deadlock pattern $\DP{e_2,e_{6}}$.
However, Condition~\cond{DP-HB} does not hold \emph{for any} deadlock pattern,
because
of the strict textual order among lock acquisitions and releases (critical sections).
We conclude that Condition~\cond{DP-HB} reports no false positives but too many false negatives.
Clearly, this means that DP is not useful in practice: we need a weaker partial order
that does not strictly order critical sections.

\paragraph{Weak-Causally-Precedes is not strict enough.}

The Weak-Causally-Precedes relation (WCP)~\cite{conf/pldi/KiniMV17}
considers two critical sections as unordered, unless they contain conflicting memory operations.
In case of trace $T_1$ in \cref{fig:ex0}, we find that $e_2$ and $e_{6}$ are not ordered.
Thus, Condition~\cond{DP-WCP} (correctly) predicts that $\DP{e_2,e_{6}}$ codifies a deadlock.
Unfortunately, we now encounter false negatives
and still some false positives, as shown next.

Trace $T_2$ in \cref{fig:ex_3} shows that we encounter false positives.
Events $e_5$ and $e_6$ refer to conflicting memory operations, but they are not part of a critical section, so there is no order among the events in threads $\thread{1}$ and $\thread{2}$.
Hence, Condition~\cond{DP-WCP} applies to deadlock pattern $\DP{e_2,e_8}$.
This is a false positive, because any reordering that exhibits the deadlock
will violate the last-write dependency among $e_5$ and $e_6$.

The case of false negative is explained by trace $T_3$ in \cref{fig:ex3b}.
The critical sections for lock $\LKB$ in threads $\thread{2}$ and $\thread{3}$
contain some conflicting write operations on $\VA$, and therefore we find that $e_7 \WCPLt e_9$.
WCP composes (to the left and right) with HB.
Hence, $e_2 \WCPLt e_{10}$ by composition of $e_7 \WCPLt e_9$
with $e_2 \HBLt e_7$ and $e_9 \HBLt e_{10}$.
But then Condition~\cond{DP-WCP} (wrongly) rules out $\DP{e_2,e_{10}}$: a false negative.

We conclude that neither HB nor WCP are suitable for the purpose of deadlock prediction.
Similar observations apply to other partial orders such as SDP~\cite{10.1145/3360605}
and DC~\cite{Roemer:2018:HUS:3296979.3192385-obsolete}.


\paragraph{Total-Read-Write order and deadlock-pattern order for soundness.}

Our idea is to adapt WCP to a new Total-Read-Write partial order (TRW), in the following non-trivial way.
Under TRW, if two critical sections contain TRW-ordered events,
the release of the critical section that appears earlier in the trace is ordered before
the TRW-ordered event in the later critical section.
Though similar, there are two main differences between TRW and WCP.
First, TRW does not compose with HB.
This eliminates false negatives, as seen in trace $T_3$ in \cref{fig:ex3b}.
Second, under TRW, any pair of conflicting memory operations is ordered based on their order in the trace.
This eliminates false positives, as seen in trace $T_4$ in \cref{fig:ex3}.


We can show that the additional Condition~\cond{DP-TRW} guarantees
%% MS, see intro, guess it's fine to have some repitition
soundness (no false positives)
if there is no \emph{earlier} deadlock reported.
To explain this issue, consider trace~$T_4$ in \cref{fig:dp-order}.
The deadlock pattern $\DP{e_5,e_{13}}$ does not codify a true deadlock, because there is
an earlier deadlock pattern $\DP{e_2, e_{10}}$.
Hence, there is no correct reordering of trace $T_3$
where events $e_5$ and $e_{13}$ are enabled.

To understand the issue in more detail,
we examine the corresponding cyclic lock-dependency chains in \cref{fig:dp-order}.
The deadlock pattern $\DP{e_2, e_{10}}$ is supported by dependencies~$D_1$ and $D_3$.
The deadlock pattern~$\DP{e_5,e_{13}}$ is supported by $D_2$ and $D_4$.
The lockset of dependency $D_1$ is a subset of the lockset of $D_3$
and the lockset of dependency~$D_2$ is a subset of the lockset of $D_4$.
In both cases, the locks held are acquired by the \emph{same event} as indicated
by the (dark and light gray) highlighting.

Hence, the cyclic lock-dependency chain between $D_2$ and $D_4$ is not reachable,
as it is blocked by the earlier cyclic lock-dependency chain between
$D_1$ and $D_3$.
To rule out such cases, we impose the following condition
that imposes an order among deadlock patterns:
%% In \cref{sec:deadlock}, we condense the intuition gained from this
%% example into the formal definition of a partial order on deadlock
%% patterns, such that $\DP{e_2,e_{10}} < \DP{e_5,e_{13}}$.
%% This ordering can be computed efficiently and gives rise to the fourth condition.
%
\begin{description}
  \item[\cond{DP-Block}] A deadlock pattern is not ordered after any other deadlock patterns.
\end{description}

In \cref{s:sound}, we show that conditions \cond{DP-Guard/-Cycle/-TRW/-Block}
are sufficient to establish soundness (no false positives).


\begin{figure}[t]

  \begin{minipage}[b]{.3\textwidth}
    \bda{@{}lcl}
%%latexTrace $ addLoc ex_3c
\ba{|l|l|l|}
\hline  T_5 & \thread{1} & \thread{2}\\ \hline
\eventE{1}  & \lockE{\LKA}&\\
\eventE{2}  & \lockE{\LKB}&\\
\eventE{3}  & \writeE{\VA}&\\
\eventE{4}  & \unlockE{\LKB}&\\
\eventE{5}  & \unlockE{\LKA}&\\
\eventE{6}  & &\lockE{\LKB}\\
\eventE{7}  & &\writeE{\VA}\\
\eventE{8}  & &\lockE{\LKA}\\
\eventE{9}  & &\unlockE{\LKA}\\
\eventE{10}  & &\unlockE{\LKB}\\

 \hline \ea{}

 \eda
 \subcaption{%%WCP and
             TRW false negative.}
    \label{fig:ex3}
  \end{minipage}%
  \hfill%
  \begin{minipage}[b]{.3\textwidth}
    \bda{@{}lcl}
%%latexTrace $ addLoc ex_4
\ba{|l|l|l|}
\hline  T_6 & \thread{1} & \thread{2}\\ \hline
\eventE{1}  & \lockE{\LKA}&\\
\eventE{2}  & \lockE{\LKB}&\\
\eventE{3}  & \unlockE{\LKB}&\\
\eventE{4}  & \unlockE{\LKA}&\\
\eventE{5}  & &\lockE{\LKA}\\
\eventE{6}  & &\unlockE{\LKA}\\
\eventE{7}  & &\lockE{\LKB}\\
\eventE{8}  & &\lockE{\LKA}\\
\eventE{9}  & &\unlockE{\LKA}\\
\eventE{10}  & &\unlockE{\LKB}\\

 \hline \ea{}
 \eda
 \subcaption{\SPDOffline false negative.}
    \label{fig:ex_4}
  \end{minipage}%
  \hfill%
  \begin{minipage}[b]{.3\textwidth}
    \bda{@{}lcl}
%%latexTrace $ addLoc ex_5
\ba{|l|l|l|}
\hline  T_7 & \thread{1} & \thread{2}\\ \hline
\eventE{1}  & &\writeE{\VA}\\
\eventE{2}  & \lockE{\LKB}&\\
\eventE{3}  & \lockE{\LKC}&\\
\eventE{4}  & \readE{\VA}&\\
\eventE{5}  & \lockE{\LKA}&\\
\eventE{6}  & \unlockE{\LKA}&\\
\eventE{7}  & \unlockE{\LKC}&\\
\eventE{8}  & \unlockE{\LKB}&\\
\eventE{9}  & &\lockE{\LKC}\\
\eventE{10}  & &\writeE{\VA}\\
\eventE{11}  & &\unlockE{\LKC}\\
\eventE{12}  & &\lockE{\LKA}\\
\eventE{13}  & &\lockE{\LKB}\\
\eventE{14}  & &\unlockE{\LKB}\\
\eventE{15}  & &\unlockE{\LKA}\\

 \hline \ea{}
 \eda
 \subcaption{PWR false positive.}
    \label{fig:ex_5}
  \end{minipage}%
  \caption{TRW versus \SPDOffline versus PWR.}
  \label{f:TRWvsSPDvsPWR}
\end{figure}



\paragraph{Comparison against the state of the art.}

TRW and the ordering among deadlock patterns can be computed efficiently, as our experiments confirm.

We also compare experimentally against
\SPDOffline~\cite{conf/pldi/TuncMPV23}, which represents the state of the art when it comes to efficient and sound
deadlock prediction methods.
For an extensive benchmark suite, we show that
our method covers the same deadlocks as \SPDOffline.

The similar performance and precision (absence of false positives) between \SPDOffline and our method does \emph{not} imply that they are equivalent.
We propose an entirely new direction
to achieving efficient and sound deadlock prediction.
Generally speaking, \SPDOffline and our approach are incomparable when it comes
to precision and underlying methods, as we explain
via the following examples.

Consider $T_5$ in \cref{fig:ex3}.
Deadlock pattern $\DP{e_2,e_8}$ is a true positive and reported by \SPDOffline.
However, under \cond{DP-TRW} deadlock pattern $\DP{e_2,e_8}$ is ruled out
for the following reason.
We find that $e_3 \TRWLt e_7$, because these events refer to conflicting memory operations.
Both events are protected by a common lock and therefore ${e_5 \TRWLt e_7}$.
By program order and transitivity, we conclude that $e_2 \TRWLt e_8$.

On the other hand,
\SPDOffline fails to report the deadlock pattern $\DP{e_2,e_7}$ in \cref{fig:ex_4},
whereas our method reports this true positive:
$e_2$ and $e_7$ are not ordered under TRW.
To understand these differences, we take a closer look at \SPDOffline.

\SPDOffline only considers \emph{sync-preserving} deadlocks.
A deadlock is sync preserving, if the reordered trace
that exhibits the deadlock does not reorder acquires on the same lock.
In case of \cref{fig:ex_4},
any reordering of $T_6$ that exhibits the deadlock represented by~$\DP{e_2,e_7}$ will have to reorder
the two acquires on $\LKB$ (events $e_1$ and $e_5$).
Hence, \SPDOffline rejects the deadlock pattern.

\SPDOffline discovers this non-sync-preserving reordering
via a closure construction that starts with
a deadlock pattern satisfying Conditions~\cond{DP-Cycle/-Guard}.
Initially, the closure starts with~$\DP{e_2,e_7}$.
\SPDOffline adds event $e_1$, because $e_1$ \emph{must happen before}~$e_2$ codified in a Must-Happen-Before partial order (MHB).
Based on MHB,
we also add events $e_5$ and $e_6$ to the closure.
Once \SPDOffline encounters $e_1$ and $e_5$ (two acquires on the same lock),
event $e_4$ (the matching release of~$e_1$) is added, ensuring
that the closure is sync preserving.
Via $e_4$, we reach our starting point~$e_2$.
\SPDOffline reports a cycle in the closure construction
and therefore rejects deadlock pattern $\DP{e_2,e_7}$.

Hence, there are connections between \SPDOffline and our method:
both make use of information from partial orders.
The major difference is that we
use TRW as part of
Condition~\cond{DP-TRW} to instantly reject deadlock
patterns.
In contrast, \SPDOffline
employs MHB as part of the closure construction.
This then leads to incomparable precision results,
as shown by the traces in \cref{fig:ex3,fig:ex_4}.
A similar observation applies in
comparison against SeqCheck~\cite{conf/fse/CaiYWQP21}.

\paragraph{Experimenting with other partial orders.}

TRW is carefully crafted to ensure that all false positives
are eliminated but some false negatives remain.
What if we weaken TRW, by only
ordering conflicting memory operations that are part
of a last-write dependency?
The resulting partial order is known as PWR~\cite{conf/mplr/SulzmannS20}.
PWR catches the false positive in \cref{fig:ex_3},
and correctly identifies the deadlocks
in \cref{fig:ex3,fig:ex_4}.
In \cref{s:complete}, we show that
Conditions~\cond{DP-Guard/-Cycle/-PWR/-Block}
are sufficient to establish completeness (no false negatives).

In general, PWR has false positives,
as shown by trace $T_7$ in \cref{fig:ex_5}.
Deadlock pattern $\DP{e_5,e_{13}}$ is valid under Condition~\cond{DP-PWR}.
But any reordering that exhibits the deadlock
would need to put events $[e_9,e_{10},e_{11}]$ before
event $e_3$.
This will violate the last-write dependency
between $e_1$ and~$e_4$.
For comparison, under TRW we find that $e_5 \TRWLt e_{13}$,
because all conflicting events are ordered as they appear
in the trace.

We do not encounter any such false positives in our extensive benchmark suite.
In fact, TRW and PWR report the exact same deadlocks.
This shows that TWR and PWR are excellent candidates
when it comes to sound and complete deadlock prediction.


%% \begin{verbatim}
%%
%%  MHB = LW
%%
%%  MHB subsetof TRW
%%
%% What if we use TRW for closure construction?
%% Too strict.
%% Encounter more false negatives.
%% Why would we want this anyway?
%% MHB is a fairly "weak" must happens-before order.
%%
%% It would be safe to use PWR instead!
%% Via PWR we could be the closure "faster".
%% There are some contrived example where this is benificiary.
%% Of course, there's also the trade-off that computing PWR
%% takes more time compared to MHB.
%%
%% What if we use MHB in DP-P?
%% Not strict enough.
%% Encounter more false positives.
%% However, seems to be rarely the case in practice.
%% PWR performans in particular very well.
%%
%%
%% \end{verbatim}









%%% Local Variables:
%%% mode: latex
%%% TeX-master: "main"
%%% End:
