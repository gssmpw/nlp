
%--------------------------------------------------------
%--------------------------------------------------------
\section{Preliminaries}
\label{sec:prelim}

We introduce executions of concurrent programs with shared variables and locks, where operations are recorded as \emph{traces} of concurrency \emph{events} (\cref{s:prelim:trace}).
Then, we define the semantics of traces in terms of \emph{well formedness}, and how traces may be \emph{reordered} without affecting the semantics of the source trace (\cref{s:prelim:wfCrp}).
Finally, we define \emph{deadlock patterns} and what it means for a trace to contain \emph{resource deadlocks} (\cref{s:prelim:deadlock})

\subsection{Events and traces}
\label{s:prelim:trace}

We use $l_1,l_2,\ldots$ and $x,y,z$ for lock and shared variables, respectively.

\begin{definition}[Events and Traces]
  \label{def:run-time-traces-events}
  \bda{@{}l@{}c@{}ll@{\hskip3em}l@{}c@{}llr@{}}
    \alpha, \beta,\delta & {} ::= {} & 1 \mid 2 \mid \ldots & \mbox{(unique event ids)}
    & e & {} ::= {} & (\alpha, t,op) & \mbox{(events)}
    \\ t,s,u & {} ::= {} & \THD1 \mid \THD2 \mid \ldots & \mbox{(thread ids)}
    & T & {} ::= {} & [] \mid e : T   & \mbox{(traces)}
    \\ op & {} ::= {} & \readE{x} \mid \writeE{x}
    \mid \reqE{l} \mid \acqE{l} \mid \relE{l}
    \span\span\span\span
    %  \mid \forkE{t}
    %  \mid \joinE{t}
    & \mbox{(operations)}
  \eda
\end{definition}

\noindent
A trace $T$ is a list of events reflecting a single execution of a
concurrent program under the sequential consistency memory
model~\cite{Adve:1996:SMC:619013.620590}.
We write $[o_1,\dots,o_n]$
for a list of objects  % as a shorthand of $o_1:\dots:o_n:[]$
and use the operator ``$\conc$'' for list concatenation.

An event $e$ is represented by a triple $(\alpha, t,op)$
where
$\alpha$ is a unique event identifier,
$op$ is an operation, and
$t$ is the thread id in which the operation took place.
The main thread has thread id~$\THD1$.
The unique event identifier unambiguously identifies events under trace reordering (cf.\ \cref{s:prelim:wfCrp}).

The operations $\readE{x}$ and $\writeE{x}$ denote reading of and writing to a shared variable~$x$, respectively.
Operations $\acqE{l}$ and $\relE{l}$ denote acquiring and releasing a lock $l$, respectively.
We also include lock requests $\reqE{l}$ to denote the (possibly unfulfillable) attempt at acquiring a lock $l$; this allows for smoother definitions and is crucial for completeness (cf.\ \cref{s:complete}).
% We write $\forkE{t}$ for the creation of a new thread with thread id~$t$.
% We write $\joinE{t}$ for a join with a thread with thread id~$t$.
Operations to create threads (fork) and synchronize with termination of threads (join) can be modeled using shared variables and locks.

We often omit identifier and/or thread when denoting events, and omit parentheses if only the operation remains, writing $e = (t,op)$ or $e = op$ instead of $e = (\alpha,t,op)$.
Notation $\thd(e)$ extracts the thread id from an event.

We write $e \in T$ to indicate that $T = [e_1, \dots, e_n]$ and $e = e_k$ for $1 \le k \le n$, defining also $\posP{T}{e} = k$.
The set of events in a trace is then $\evts(T) = \{ e \mid e \in T \}$, and the set of thread ids in a trace is $\thds(T) = \{ \thd(e) \mid e \in T \}$.
For trace $T$ and events $e, f \in \evts(T)$, we define trace order: $e \TrLt[T] f$ if $\posP{T}{e} < \posP{T}{f}$.
We then define program order $\POLt[T]$ as the restriction of $\TrLt[T]$ to events with the same thread id.

We often express traces in tabular notation.
Here, traces have one column per thread.
Each event is placed in row of its trace position, and in the column of its thread.
\Cref{f:firstExamples,f:traceBlock,f:TRWvsSPDvsPWR} contain numerous examples.

\subsection{Well Formedness and Correct Reorderings}
\label{s:prelim:wfCrp}

Traces must be well formed, following the standard sequential consistency conditions for concurrent objects \cite{Adve:1996:SMC:619013.620590,Said:2011:GDR:1986308.1986334,Huang:2014:MSP:2666356.2594315}.

\begin{definition}[Well Formedness]
\label{d:WF}
  A trace $T$ is \emph{well formed} if all the following conditions are satisfied:
  \begin{description}
    \item[\cond{WF-Acq}]
      For every $\evtAcc = (t,\acqE{l}), \evtAccA = (s,\acqE{l}) \in T$ where $\evtAcc \TrLt[T] \evtAccA$, there exists $\evtRel = (t,\relE{l}) \in T$ such that $\evtAcc \TrLt[T] \evtAccA$.
      We say that $a$ and $r$ \emph{match}.

   \item[\cond{WF-Rel}]
      For every $\evtRel = (t,\relE{l}) \in T$, there exists $\evtAcc = (t,\acqE{l}) \in T$ such that $a \TrLt[T] r$ and there is no $\evtRelA = (s,\relE{l}) \in T$ with $\evtAcc \TrLt[T] \evtRelA \TrLt[T] \evtRel$.
      We say that $a$ and $r$ \emph{match}.

    \item[\cond{WF-Req}]\label{cond:lock-3}
      For every $\ACQ{} = (t,\acqE{l}) \in T$, there exists $\REQ{} = (t,\reqE{l}) \in T$ such that $\REQ{} \TrLt[T] \ACQ{}$, and for every $\REQ{} = (t,\reqE{l}) \in T$ and $e = (t,op) \in T$ such that $\REQ{} \TrLt[T] e$, if there is no $f = (t,op')$ such that $\REQ{} \TrLt[T] f \TrLt[T] e$ then $e = (t,\acqE{l})$.
      We say that $\REQ{}$ \emph{requests} $e$.

  %  \item[Fork-1:]
  %    For each $t \ne t_1$ there exists at most one $(\dontCare,\forkE{t}) \in T$ and $(\dontCare,\forkE{t_1}) \notin T$.

  %  \item[Fork-2:]
  %    For each $\evtAA = (t,op) \in T$ where $t \ne t_1$ there exists $\evtBB = (s,\forkE{t}) \in T$ where $\evtBB \TrLt \evtAA$.

  %  \item[Join:]
  %    For each $\evtAA = (s,\joinE{t}) \in T$ we have $s \ne t$ and for all $\evtBB = (t,op) \in T$ we have $\evtBB \TrLt \evtAA$.

    % \item[WN]
    %   For every $a = (t,acq(l)),r = (t,rel(l)),a' = (t,acq(l')),r' = (t,rel(l')) \in T$ such that $a$ matches $r$ and $a'$ matches $r'$, $a \traceLt{T} a' \traceLt{T} r$ if and only if $a \traceLt{T} r' \traceLt{T} r$.
  \end{description}
\end{definition}

\noindent
Condition~\cond{WF-Acq} states that a previously acquired lock can only be acquired after it has been released.
Similarly, Condition~\cond{WF-Rel} states that a lock can only be released after is has been acquired but not yet released.
Note that these conditions require matching acquires and releases to occur in the same thread.
Condition~\cond{WF-Req} states that all lock acquires must have been requested in the same thread immediately before.
We often omit requests from example traces, but we assume that they precede each acquire implicitly.
At the end of the trace, acquired locks do not have to be released and requests do not have to be fulfilled.
% Condition \textbf{WN} asserts well-nestedness: a critical section can only be released once all contained critical sections have been released.
All traces in \cref{f:firstExamples,f:traceBlock,f:TRWvsSPDvsPWR} are well formed, with implicit requests; we discuss an example with explicit requests in the next subsection.

% Condition \textbf{Fork-1} states that a thread can be created at most once.
% Condition \textbf{Fork-2} states that each thread except the main thread is preceded by a fork event.
% Both conditions imply that for each event ${(\beta, t,\forkE{s})}$ we have $t \ne s$.

% Condition \textbf{Join} states that all events from a joined thread appear before the join event.
% There can be several join events $\joinE{t}$ for the same thread $t$.
% A join operation $\joinE{t}$ does not necessarily have to appear in the thread that forked thread $t$.

A trace represents only one possible interleaving of events: there can be as many interleavings as there are permutations of the original trace.
However, not all such reorderings are feasible in the sense of being reproducible by executing the program with a different schedule.
In addition to well formedness, a reordering must guarantee that (a)~program order and (b)~last writes are maintained.
A reordering maintains program order if, in any thread, the order of events is unchanged.
For guarantee~(b), note that every read observes some write on the same variable: the last preceding such write.
Last writes are then maintained if the write observed by any read is unchanged.
Guarantee~(b) is particularly important to ensure that the control flow of the traced program is unaffected by the reordering, e.g., when the read is used in a conditional statement.

Reorderings do not have to run to completion, so we consider reordered \emph{prefixes} of traces.

\begin{definition}[Correctly Reordered Prefix]
  \label{def:crp}
  The \emph{projection} of $T$ onto thread $t$, denoted $\proj(t,T)$, restricts $T$ to events $e$ with $\thd(e) = t$.
  That is,
  $\evtAA \in proj(t,T)$ if and only if $e \in T$ and $\thd(e) = t$, and
  $\evtAA \TrLt[proj(t,T)] \evtBB$ implies $\evtAA \TrLt[T] \evtBB$.

  Take $\evtAA = \readE{x},\evtBB = \writeE{x} \in T$.
  We say that $\evtBB$ is the \emph{last write} of $\evtAA$ w.r.t.\ $T$ if $\evtBB$ appears before $\evtAA$ with no other write on $x$ in between.
  That is,
  $\evtBB \TrLt[T] \evtAA$, and
  there is no $\evtCC = \writeE{x} \in T$ such that $\evtBB \TrLt[T] \evtCC \TrLt[T] \evtAA$.

  Trace $T'$ is a \emph{correctly reordered prefix} of $T$ if the following
  conditions are satisfied:
  \begin{description}
    \item[\cond{CRP-WF}] $T'$ is well formed and $\evts(T') \subseteq \evts(T)$.
    \item[\cond{CRP-PO}] For every $t \in \thds(T')$, $\proj(t,T')$ prefixes $\proj(t,T)$.
    \item[\cond{CRP-LW}] For every $\readE{x} \in T'$ with last write $f$ w.r.t.\ $T$, it must be $f$ is also the last write for $e$ w.r.t.\ $T'$.\footnote{Unique event identifiers are crucial for this condition.}
  \end{description}
  We write $\crp(T)$ to denote the set of correctly reordered prefixes of $T$.
\end{definition}

\subsection{Resource Deadlocks}
\label{s:prelim:deadlock}

We consider a standard definition of (resource) deadlock, where multiple threads are simultaneously requesting to acquire a lock, but every such lock is held by another thread: none of the requests can be fulfilled, and so the trace is stuck.
Our goal is to predict that a trace can be rescheduled to deadlock, and so we identify traces with \emph{predictable} deadlocks in terms of correctly reordered prefixes with deadlocks.
We first identify these deadlocks using \emph{deadlock patterns}.

Deadlock patterns codify cycles of requests that cannot be fulfilled because the requested lock is held by the next request.
Hence, we first define what it means for an event to hold a lock, in terms of \emph{critical sections} of events bordered by lock acquires and releases.

\begin{definition}[Critical Sections]
  \label{d:CS}
  Given $a = (t,\acqE{l}) \in T$, we define the \emph{critical section acquired by $a$} as $\CS[T](a) \subseteq \evts(T)$, where $e \in \CS[T](a)$ if all the following conditions hold:
  \begin{description}
    \item[\cond{CS-PO}]
      $e = a$ or $a \POLt[T] e$.
    \item[\cond{CS-Held}]
      There is no $r = \relE{l} \in T$ that matches $a$ with $a \POLt[T] r \POLt[T] e$.
  \end{description}

  The set of \emph{acquires held} by $e \in T$ is defined as $\AH[T](e) = \{ a \in T \mid e \in \CS[T](a) \}$.
  Accordingly, the set of \emph{locks held} is defined as $\LH[T](e) = \{ l \mid \exists \acqE{l} \in \AH[T](e) \}$.
\end{definition}

\noindent
Thus, an event is in a critical section if it is or appears after the acquire in the same thread (Condition~\cond{CS-PO}), as long as the critical section has not yet been closed (Condition~\cond{CS-Held}).
Note that Condition~\cond{CS-Held} considers a lock held if it is never released at the end of the trace.

\begin{definition}[Deadlock Patterns (DPs)]
  \label{d:dp}
  Given trace $T$, let $A = \DP{(\ACQ1,\REQ1),\ldots,(\ACQ{n},\REQ{n})} \subseteq \evts(T) \times \evts(T)$ for $n \ge 2$, where every $(\ACQ{i},\REQ{i}) \in A$ is an acquire-request tuple, and $\thd(\REQ{i}) \ne \thd(\REQ{j})$ for every $1 \le i < j \le n$.
  We define the following properties for $A$:
  \begin{description}
    \item[\cond{DP-Cycle}]
      For every $1 \le i \le n$, $q_i = \reqE{l} \in \CS[T](a_i)$ and $a_{(i \mod n) + 1} = \acqE{l}$.
    \item[\cond{DP-Guard}]
      $\LH[T](\ACQ{i}) \cap \LH[T](\ACQ{j}) = \emptyset$ for $1 \leq i < j \leq n$.
  \end{description}
  Condition~\cond{DP-Cycle} identifies $A$ as a \emph{cycle in~$T$}.
  If $A$ is a cycle in~$T$ that satisfies Condition~\cond{DP-Guard}, then $A$ is a \emph{deadlock pattern (DP)} in~$T$.
\end{definition}

\noindent
Thus, we consider sets of acquire-request pairs, where the request holds the acquire.
Condition~\cond{DP-Cycle} ensures that the set forms a cycle of requests for locks held by other requests.
DPs satisfy an additional condition to eliminate false alarms: Condition~\cond{DP-Guard} ensures that no two requests hold a common lock, which would make it impossible to schedule the deadlock.

Finally, we define predictable deadlocks in terms of DPs.

\begin{definition}[Predictable Deadlock]
  \label{d:deadlock}
  Let $A$ be a DP in well-formed trace $T$.
  We say $A$ is a \emph{predictable deadlock} if there exists $T' \in \crp(T)$ such that, for every $(\ACQ{},\REQ{}) \in A$, $\REQ{} \in T'$ and there is no $e \in T'$ with $\REQ{} \POLt[T'] f$.
\end{definition}

\begin{figure}[t]
  \begin{minipage}[b]{.32\textwidth}
    \bda{@{}c@{}}
      % ex0
      \begin{array}[b]{|l|l|l|}
        \hline
        T'_1 & \thread{1} & \thread{2} \\
        \hline
        \eventE{1} & \lockE{\LKA} & \\
        \eventE{2} & \reqE{\LKB} & \\
        \eventE{3} & \lockE{\LKB} & \\
        \eventE{4} & \unlockE{\LKB} & \\
        \eventE{5} & \unlockE{\LKA} & \\
        \eventE{6} && \lockE{\LKB} \\
        \eventE{7} && \reqE{\LKA} \\
        \eventE{8} && \lockE{\LKA} \\
        \eventE{9} && \unlockE{\LKA} \\
        \eventE{10} && \unlockE{\LKB} \\
        \hline
      \end{array}
      \\ \\
      % ex0
      \begin{array}[b]{|l|l|l|}
        \hline
        T''_1 & \thread{1} & \thread{2} \\
        \hline
        \eventE{1} & \lockE{\LKA} & \\
        \eventE{6} && \lockE{\LKB} \\
        \eventE{2} & \reqE{\LKB} & \\
        \eventE{7} && \reqE{\LKA} \\
        \hline
      \end{array}
    \eda
  \end{minipage}%
  \hfill%
  \begin{minipage}[b]{.32\textwidth}
    \bda{@{}c@{}}
      %%latexTrace $ addLoc ex_3
      \ba{|l|l|l|}
        \hline T'_2  & \thread{1} & \thread{2} \\ \hline
        \eventE{1}  & \lockE{\LKA} & \\
        \eventE{2}  & \reqE{\LKB} & \\
        \eventE{3}  & \lockE{\LKB} & \\
        \eventE{4}  & \unlockE{\LKB} & \\
        \eventE{5}  & \unlockE{\LKA} & \\
        \eventE{6}  & \writeE{\VA} & \\
        \eventE{7}  & &\readE{\VA} \\
        \eventE{8}  & &\lockE{\LKB} \\
        \eventE{9}  & &\reqE{\LKA} \\
        \eventE{10}  & &\lockE{\LKA} \\
        \eventE{11}  & &\unlockE{\LKA} \\
        \eventE{12}  & &\unlockE{\LKB} \\
        \hline
      \ea{}
      \\ \\
      %% minor modification of the above
      \ba{|l|l|l|}
        \hline T''_2  & \thread{1} & \thread{2} \\ \hline
        \eventE{1}  & \lockE{\LKA} & \\
        \eventE{2}  & \writeE{\VA} & \\
        \eventE{3}  & \reqE{\LKB} & \\
        \eventE{4}  & \lockE{\LKB} & \\
        \eventE{5}  & \unlockE{\LKB} & \\
        \eventE{6}  & \unlockE{\LKA} & \\
        \eventE{7}  & &\lockE{\LKB} \\
        \eventE{8}  & &\reqE{\LKA} \\
        \eventE{9}  & &\lockE{\LKA} \\
        \eventE{10}  & &\readE{\VA} \\
        \eventE{11}  & &\unlockE{\LKA} \\
        \eventE{12}  & &\unlockE{\LKB} \\
        \hline
      \ea{}
    \eda
  \end{minipage}%
  \hfill%
  \begin{minipage}[b]{.32\textwidth}
    \bda{@{}c@{}}
      % dl_11
      \begin{array}{|l|l|l|}
        \hline
        T'_4 & \thread{1} & \thread{2} \\
        \hline
        \eventE{1} & \lockE{\LKA} & \\
        \eventE{2} & \reqE{\LKB} & \\
        \eventE{3} & \lockE{\LKB} & \\
        \eventE{4} & \unlockE{\LKB} & \\
        \eventE{5} & \lockE{\LKC} & \\
        \eventE{6} & \reqE{\LKD} & \\
        \eventE{7} & \lockE{\LKD} & \\
        \eventE{8} & \unlockE{\LKD} & \\
        \eventE{9} & \unlockE{\LKC} & \\
        \eventE{10} & \unlockE{\LKA} & \\
        \eventE{11} && \lockE{\LKB} \\
        \eventE{12} && \reqE{\LKA} \\
        \eventE{13} && \lockE{\LKA} \\
        \eventE{14} && \unlockE{\LKA} \\
        \eventE{15} && \lockE{\LKD} \\
        \eventE{16} && \reqE{\LKC} \\
        \eventE{17} && \lockE{\LKC} \\
        \eventE{18} && \unlockE{\LKC} \\
        \eventE{19} && \unlockE{\LKD} \\
        \eventE{20} && \unlockE{\LKB} \\
        \hline
      \end{array}
    \eda
  \end{minipage}%
  \caption{Example traces, with selected requests included (cf.\ \cref{f:firstExamples,f:traceBlock}).}
  \label{f:firstWithReqs}
\end{figure}

\noindent
Note that, for simplicity, \cref{sec:overview} denoted DPs as only the set of acquires requested by the unfulfillable requests.
Compare \cref{f:firstExamples,f:firstWithReqs}, where the latter explicitly includes selected requests.
DP $\{(e_1,e_2),(e_7,e_6)\}$ in trace $T'_1$ is a predictable deadlock, witnessed by $T''_1 \in \crp(T'_1)$.
On the other hand, DP $\{(e_2,e_1),(e_9,e_8)\}$ in trace $T'_2$ is not a predictable deadlock, because all reorderings in which $e_2$ is the last event in $\thread1$ omit $e_6$, which is the last write of $e_7$ and thus violates Condition~\cond{CRP-LW}.


%%% Local Variables:
%%% mode: latex
%%% TeX-master: "main.tex"
%%% End:
