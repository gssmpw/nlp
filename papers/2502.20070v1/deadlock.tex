\section{Soundness: Partial Orders to Eliminate False Positives}
\label{s:sound}

As discussed in \cref{sec:overview}, DPs as defined in \cref{s:prelim:deadlock} alone are imprecise: they often signal false alarms.
To be precise, we identified two situations in which DPs falsely identify deadlocks.
To rule out such situations, we introduce two new conditions on DPs that use partial orders, eliminating all false positives and allowing us to prove soundness (\cref{thm:TRWSound}).

\subsection{Partially-ordered Requests}
\label{s:sound:trw}

A DP is a predictable deadlock in $T$ when there is a $T' \in \crp(T)$ in which its requests are the last events in their respective threads.
Because two requests never directly depend on each other, this implies that the requests can occur in $T'$ in any order, i.e., they are concurrent.
Hence, the first situation occurs when some requests are not concurrent, making it impossible to find such a $T'$.
We rule out this situation by giving a partial order that is appropriate for determining the concurrency of the requests in a DP.

It turns out that our novel Total Read-Write order (TRW), as defined below, is up to the task (cf.\ \cref{s:sound:sound}).
TRW is derived from PWR~\cite{conf/mplr/SulzmannS20} by imposing that all reads and writes on the same variable are ordered as in the source trace.
Moreover, TRW is similar to WCP~\cite{conf/pldi/KiniMV17}, which soundly predicts deadlocks in absence of data races but is not sound in general (cf.\ \cref{sec:related} for details).

\begin{definition}[Total Read-Write Order (TRW)]
  \label{d:trw}
  We say events $e$ and $f$ are \emph{conflicting}, denoted~$e \conf f$, if $e$ and $f$ are reads/writes on the same variable and at least one is a write.

  Given trace $T$, we define the \emph{Total-Read-Write order (TRW)} as a relation $\TRWLt[T]$ on $\evts(T)$, where $e \TRWLt[T] f$ if either of the following conditions holds:
  \begin{description}
    \item[\cond{TRW-PO}]
      $e \POLt[T] f$.
    \item[\cond{TRW-Conf}]
      $e \conf f$ and $e \TrLt[T] f$.
    \item[\cond{TRW-Rel}]
      There are acquires $a_1,a_2 \in T$ on the same lock such that (i)~$e$ is the release matching $a_1$, (ii)~$a_1 \TrLt[T] a_2$, (iii)~$f \in \CS[T](a_2)$, and (iv)~$a_1 \TRWLt[T] f$.
    \item[\cond{TRW-Tr}]
      There exists $g \in T$ such that $e \TRWLt[T] g \TRWLt[T] f$.
  \end{description}
  We define $\TRWcrp(T) = \{ T' \in crp(T) \mid \forall f \in T' , e \in T \colon e \TRWLt[T] f \implies e \TrLt[T'] f \}$.

  A DP $A$ in well-formed trace $T$ satisfies Condition~\cond{DP-TRW} if neither $\REQ{} \TRWLt[T] \REQ{}'$ nor $\REQ{}' \TRWLt[T] \REQ{}$ (denoted $\REQ{} \TRWConc[T] \REQ{}'$) for every distinct $(\ACQ{},\REQ{}),(\ACQ{}',\REQ{}') \in A$.
\end{definition}

\noindent
Crucially, Condition~\cond{TRW-Conf} orders all reads/writes on the same variable by source-trace order;
note that the reads/writes have to be \emph{conflicting}, so one event must be a write meaning that reads are not ordered directly.
Condition~\cond{TRW-Rel} orders (the events within) critical sections on the same lock, if they contain TRW-ordered events;
note that events that precede the later TRW-ordered event~$f$ are not affected, such that $f$ correctly does not affect ordering in correctly ordered prefixes from which $f$ is omitted (we revisit this point in context of completeness in \cref{s:complete}).
Conditions~\cond{TRW-PO/-Tr} close TRW under program order and transitivity, respectively.

To illustrate, we reconsider trace~$T_2$ in \cref{f:firstExamples}, discussed in \cref{sec:overview} to contain a false DP because its requests are ordered.
We consider variant~$T'_2$ in \cref{f:firstWithReqs}, which explicitly includes these requests.
This trace contains DP $\{(e_1,e_2),(e_8,e_9)\}$, but ${e_2 \not\TRWConc[T'_2] e_9}$: we have $e_2 \TRWLt[T'_2] e_6$ (\cond{TRW-PO/-Tr}), $e_6 \TRWLt[T'_2] e_7$ (\cond{TRW-Conf}), and $e_7 \TRWLt[T'_2] e_9$ (\cond{TRW-PO/-Tr}), so $e_2 \TRWLt[T'_2] e_9$ (\cond{TRW-Tr}).

On the other hand, in trace~$T''_2$ in \cref{f:firstWithReqs}, a slight modification of $T'_2$ by moving the write and read events, the DP $\{(e_1,e_3),(e_7,e_8)\}$ does satisfy Condition~\cond{DP-TRW}: $e_2 \TRWLt[T''_2] e_{10}$ (\cond{TRW-Conf}) and hence $e_6 \TRWLt[T''_2] e_{11}$ (\cond{TRW-Rel}), but although $e_3 \TRWLt[T''_2] e_6$ and $e_8 \TRWLt[T''_2] e_{11}$ we have $e_3 \TRWConc[T''_2] e_8$ because their order does not matter in correctly reordered prefixes from which $e_{10}$ is omitted.

Note that the experiments in \cref{sec:experiments} show that soundness under Condition~\cond{DP-TRW} is not trivial, as it does not eliminate any of the predictable deadlocks that occur in the benchmarks.

\subsection{Partially-ordered Deadlock Patterns}

\begin{comment}
\begin{figure}[t]
  \bda{@{}ll@{}}
    % dl_23c
    \begin{array}[t]{|l|l|l|}
      \hline
      T_6 & \thread{1} & \thread{2} \\
      \hline
      \eventE{1} & \acqE{l_1} & \hphantom{\acqE{l_5}} \\
      \eventE{2} & \acqE{l_5} & \\
      \eventE{3} & \reqE{l_2} & \\
      \eventE{4} & \acqE{l_2} & \\
      \eventE{5} & \relE{l_2} & \\
      \eventE{6} & \relE{l_5} & \\
      \eventE{7} & \acqE{l_3} & \\
      \eventE{8} & \reqE{l_4} & \\
      \eventE{9} & \acqE{l_4} & \\
      \eventE{10} & \relE{l_4} & \\
      \eventE{11} & \relE{l_3} & \\
      \eventE{12} & \relE{l_1} & \\
      : && \\
      \hline
    \end{array}
    &
    \begin{array}[t]{|l|l|l|}
      \hline
      & \thread{1} & \thread{2} \\
      \hline
      : && \\
      \eventE{13} & \hphantom{\acqE{l_5}} & \acqE{l_2} \\
      \eventE{14} && \reqE{l_1} \\
      \eventE{15} && \acqE{l_1} \\
      \eventE{16} && \relE{l_1} \\
      \eventE{17} && \acqE{l_4} \\
      \eventE{18} && \reqE{l_3} \\
      \eventE{19} && \acqE{l_3} \\
      \eventE{20} && \relE{l_3} \\
      \eventE{21} && \relE{l_4} \\
      \eventE{22} && \relE{l_2} \\
      \hline
    \end{array}
  \eda
  \caption{Trace with two DPs, where the earlier DP blocks the later.}
  \label{fig:dp-order-reqs}
\end{figure}
\end{comment}

In the second situation, the candidate deadlock cannot be reached because of an ``earlier'' deadlock.
We make this formal by means of a partial order on cycles (sets of acquire-request pairs that satisfy Condition~\cond{DP-Cycle}), and rule out false positives by reporting only the earliest DPs.

\begin{definition}
  \label{def:ltDP}
  Given trace $T$, we define \emph{cycle order} as a relation $\DPLt[T]$ on cycles in $T$, where $A \DPLt[T] B$ if, for every $(\ACQ{},\REQ{}) \in A$, there exists $(\ACQ{}',\REQ{}') \in B$ with $q \TrLt[T] q'$ and $a \in \AH[T](\REQ{}')$.

  A DP $A$ in well-formed trace $T$ satisfies Condition~\cond{DP-Block} if there is no cycle $B$ in $T$ such that $B \DPLt[T] A$.
\end{definition}

\noindent
The ordering crucially requires that every acquire of a request in the earlier cycle is held by a request in the later cycle.
This way, we ensure that the earlier cycle is unavoidable when trying to achieve the later.

To illustrate, we reconsider trace~$T_4$ in \cref{f:traceBlock}, discussed in \cref{sec:overview} to contain a false DP because of an earlier deadlock.
We consider variant $T'_4$ in \cref{f:firstWithReqs}, which explicitly includes the relevant requests.
Let $A = \{(e_1,e_2),(e_{11},e_{12})\}$ and $B = \{(e_5,e_6),(e_{15},e_{16})\}$.
Both $A$ and $B$ are DPs satisfying Condition~\cond{DP-TRW}, but $B$ is a false positive.
Indeed, $e_1 \in \AH[T'_4](e_6)$ and $e_2 \TrLt[T'_4] e_6$, and $e_{11} \in \AH[T'_4] e_{16}$ and $e_{12} \TrLt[T'_4] e_{16}$.
Hence, $A \DPLt[T'_4] B$, so $B$ does not satisfy Condition~\cond{DP-Block}.

Note that deadlocks involving $n$ threads can only be blocked by prior deadlocks involving $m \leq n$ threads; if the prior deadlock involves $m' > n$ threads, at least one of the later requests would have to be in two of the earlier critical sections, which is impossible wy well formedness.
This is reflected by the forall-exists structure of the above ordering on DPs.

\subsection{Soundness}
\label{s:sound:sound}

\begin{figure}[t]
  \begin{minipage}[b]{.41\textwidth}
    \bda{c}
      % guard1
      \begin{array}[t]{|l|l|l|l|}
        \hline
        T_8 & \thread{1} & \thread{2} & \thread{3} \\
        \hline
        \eventE{1} & \acqE{l_3} && \\
        \eventE{2} & \writeE{x} && \\
        \eventE{3} && \readE{x} & \\
        \eventE{4} && \acqE{l_1} & \\
        \eventE{5} && \reqE{l_2} & \\
        \eventE{6} && \acqE{l_2} & \\
        \eventE{7} && \relE{l_2} & \\
        \eventE{8} && \relE{l_1} & \\
        \eventE{9} && \writeE{x} & \\
        \eventE{10} & \readE{x} && \\
        \eventE{11} & \relE{l_3} && \\
        \eventE{12} &&& \acqE{l_3} \\
        \eventE{13} &&& \acqE{l_2} \\
        \eventE{14} &&& \reqE{l_1} \\
        \eventE{15} &&& \acqE{l_1} \\
        \eventE{16} &&& \relE{l_1} \\
        \eventE{17} &&& \relE{l_2} \\
        \eventE{18} &&& \relE{l_3} \\
        \hline
      \end{array}
    \eda
    \subcaption{Trace with unbounded critical section.}
    \label{fig:guarded}
  \end{minipage}%
  \hfill%
  \begin{minipage}[b]{.56\textwidth}
    \bda{c}
      \ba{@{}ll@{}}
        % dl_18
        \begin{array}[t]{|l|l|l|}
          \hline
          T_9 & \thread{1} & \thread{2} \\
          \hline
          \eventE{1} & \acqE{l_1} & \\
          \eventE{2} & \writeE{x} & \\
          \eventE{3} & \reqE{l_2} & \\
          \eventE{4} & \acqE{l_2} & \\
          \eventE{5} & \relE{l_2} & \\
          \eventE{6} & \relE{l_1} & \\
          \eventE{7} && \acqE{l_2} \\
          \eventE{8} && \readE{x} \\
          \eventE{9} && \reqE{l_1} \\
          \eventE{10} && \acqE{l_1} \\
          \eventE{11} && \relE{l_1} \\
          \eventE{12} && \relE{l_2} \\
          \hline
        \end{array}
        &
        % reorder dl_18
        \begin{array}[t]{|l|l|l|}
          \hline
          T'_9 & \thread{1} & \thread{2} \\
          \hline
          \eventE{1} & \acqE{l_1} & \\
          \eventE{2} & \writeE{x} & \\
          \eventE{7} && \acqE{l_2} \\
          \eventE{8} && \readE{x} \\
          \eventE{9} && \reqE{l_1} \\
          \eventE{3} & \reqE{l_2} & \\
          \hline
        \end{array}
      \ea
    \eda

    \subcaption{
      Trace with predictable deadlock.
    }\label{fig:requests}
  \end{minipage}%
  \caption{Traces highlighting technical aspects of soundness and completeness (\cref{s:sound,s:complete}).}
\end{figure}

We prove soundness of DPs that satisfy Conditions~\cond{DP-TRW/-Block}, i.e., those conditions rule out all false positives.
In doing so, we make two technical assumptions:
\begin{itemize}
  \item
    We only consider traces in which locks only protect requests in the same thread.
    That is, we assume that trace $T$ is \emph{TRW bounded}:
    for every request $q \in T$ and acquire $a \in T$ with matching release $r$, if $a \TRWLt[T] q$ and $r \in T$ implies $q \TRWLt[T] r$, then $\thd(q) = \thd(a)$.
    For example, trace~$T_8$ in \cref{fig:guarded} is not TRW bounded, reporting false positive $\{(e_4,e_5),(e_{14},e_{15})\}$.
  \item
    We only consider traces that are \emph{well nested}, meaning that any thread can only release an acquired lock once all further acquired locks have been released.
\end{itemize}
These are common assumptions (cf., e.g., \cite{conf/pldi/KiniMV17}).
It may be possible to lift these assumptions by additional conditions on DPs, but this is well outside the scope of this paper.

\begin{theorem}[Soundness]
  \label{thm:TRWSound}
  If well-formed trace $T$ is TRW~bounded and well nested, and DP $A$ in $T$ satisfies Conditions~\cond{DP-TRW/-Block}, then $A$ is a predictable deadlock.
\end{theorem}

\begin{proofsketch}
  We prove the theorem by showing that there is $T' \in \TRWcrp(T)$ such that every request in $A$ is the last event in its thread.
  Towards contradiction, assume there is not such witness $T'$.
  If there are witnesses in $\crp(T) \setminus \TRWcrp(T)$, we derive that some requests are TRW ordered: Condition~\cond{DP-TRW} is contradicted.
  Otherwise, there are no witnesses at all, and we derive that there must be a cycle $B$ such that $B \DPLt[T] A$: Condition~\cond{DP-Block} is contradiction.
\end{proofsketch}

\section{Completeness: A Partial Order to Eliminate False Negatives}
\label{s:complete}

In the previous section, we were able to leverage the Total Read-Write order (TRW) to non-trivially eliminate all DPs that do not correspond to predictable deadlocks.
On the other hand, TRW is slightly too strong to catch all predictable deadlocks: some deadlocks require reordering some reads and writes, which TRW does not permit.
In other words, a DP that does not satisfy Condition~\cond{DP-TRW} can still be a predictable deadlock.
Although such false negatives rarely occur in practice (cf.\ \cref{sec:experiments}), there is no theoretical guarantee that all predictable deadlocks are caught.
The question is then whether there is a variant of Condition~\cond{DP-TRW} that rules out enough false positives to be practically relevant, while guaranteeing that no predictable deadlock is missed.

\subsection{PWR: Weakening TRW}

Recall that TRW is derived from Program-Write-Release order (PWR)~\cite{conf/mplr/SulzmannS20}.
PWR is subtly weaker than TRW, meaning that it orders events less strictly: TRW orders all conflicting reads and writes, whereas PWR only orders reads and their last writes.
Interestingly, this subtle difference is enough for PWR itself to be a suitable candidate for a complete characterization of predictable deadlocks.
We define PWR and, accordingly, Condition~\cond{DP-PWR}.

\begin{definition}[Program-Write-Release Order (PWR)]
  \label{d:pwr}
  Given trace $T$, we define the Program-Write-Release order (PWR) as a relation $\PWRLt[T]$ on $\evts(T)$, where $e \PWRLt[T] f$ if either of the following conditions hold:
  % (analogous to TRW in \cref{d:trw}, with a slightly weaker second condition):
  \begin{description}
    \item[\cond{PWR-PO}]
      $e \POLt[T] f$.
    \item[\cond{PWR-LW}]
      $e$ is the last write of read $f$ w.r.t.\ $T$.
    \item[\cond{PWR-Rel}]
      There are acquires $a_1,a_2 \in T$ on the same lock such that (i)~$e$ is the release matching $a_1$, (ii)~$a_1 \TrLt[T] a_2$, (iii)~$f \in \CS[T](a_2)$, and (iv)~$a_1 \PWRLt[T] f$.
    \item[\cond{PWR-Tr}]
      There exists $g \in T$ such that $e \PWRLt[T] g \PWRLt[T] f$.
  \end{description}
  A DP $A$ in well-formed trace $T$ satisfies Condition~\cond{DP-PWR} if neither $q \PWRLt[T] q'$ nor $q' \PWRLt[T] q$ (denoted $q \PWRConc[T] q'$) for every distinct $(a,q),(a',q') \in A$.
\end{definition}

\noindent
Thus, PWR is defined exactly as TRW (\cref{d:trw}), except for Condition~\cond{PWR-LW}.

As mentioned before in \cref{s:sound:trw}, the design of Condition~\cond{TRW-/PWR-Rel} is essential for completeness.
Recall that in trace~$T''_2$ in \cref{f:firstWithReqs}, DP $\{(e_1,e_3),(e_7,e_8)\}$ is a predictable deadlock.
Indeed, Condition~\cond{DP-PWR} is satisfied: $e_3 \PWRConc[T''_2] e_8$.
On the other hand, suppose we simplified Condition~\cond{PWR-Rel} to order $e \PWRLt a_2$ instead of only $e \PWRLt f$.
Then $e_3 \PWRLt[T''_2] e_5 \PWRLt[T''_2] e_7 \PWRLt[T''_2] e_8$, leading to an unnecessary false negative.

\subsection{Lock Requests}

As hinted at in \cref{sec:prelim}, using lock requests not only leads to an elegant description of predictable deadlock and deadlock pattern: it is essential for completeness.

To illustrate, suppose we were to define deadlock patterns in terms of lock acquires, and predictable deadlocks by correctly reordered prefixes in which the acquires of a DP are the next event in their thread.
Consider trace~$T_9$ in \cref{fig:requests}.
As witnessed by trace~${T'_9 \in \crp(T_9)}$, DP $\{(e_1,e_4),(e_7,e_{10})\}$ is a predictable deadlock.
However, we have
(i)~$e_2 \PWRLt[T_9] e_8$ (\cond{PWR-LW}),
(ii)~$e_1 \PWRLt[T_9] e_{10}$ ((i), \cond{PWR-PO/-Tr}),
(iii)~$e_6 \PWRLt[T_9] e_{10}$ ((ii), $e_{10} \in \CS[T_9](e_{10})$, \cond{PWR-Rel/-Tr}),
and (iv)~$e_4 \PWRLt[T_9] e_{10}$ ((iii), \cond{PWR-PO/-Tr}):
Condition~\cond{DP-PWR} does not hold, so the DP is falsely rejected.
On the other hand, using requests, we have $e_3 \PWRConc[T_9] e_9$, correctly satisfying Condition~\cond{DP-PWR}.

\subsection{Completeness}

Note that our completeness result does not require a trace to be well nested or bounded, as our soundness result does.

\begin{theorem}[Completeness]
  \label{t:PWRcomplete}
  If DP $A$ in well-formed trace $T$ is a predictable deadlock, then $A$ satisfies Conditions~\cond{DP-PWR/-Block}.
\end{theorem}

\begin{proofsketch}
  By assuming that $A$ is a predictable deadlock, there is $T' \in \crp(T)$ such that every $\REQ{i}$ in $A$ is the last event in its thread.
  We prove both conditions by contradiction.
  Condition~\cond{DP-PWR} is straightforward: requests at the end of threads can be trivially reordered, contradicting any PWR ordering among them.
  Condition~\cond{DP-Block} follows by showing that a preceding cycle would lead to a cyclic trace order in~$T'$.
\end{proofsketch}

Our experiments (\cref{sec:experiments}) show that the result above is not trivial, as PWR does not report any false positives, even though it is theoretically unsound.
As such, our methodology allows a choice between near-sound completeness (PWR) and near-complete soundness~(TRW).
Ideally, we will develop a partial order that is both sound and complete, lying somewhere between TRW and PWR, but this seems to be a quest for a ``holy grail''.


%%% Local Variables:
%%% mode: latex
%%% TeX-master: "main-icse25.tex"
%%% End:
