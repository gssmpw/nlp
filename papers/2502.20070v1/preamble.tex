\usepackage{amsfonts}

\RequirePackage{amsthm,thmtools}
\newtheorem{definition}{Definition}
\newtheorem{example}{Example}
\newtheorem{lemma}{Lemma}
\newtheorem{theorem}{Theorem}
\newtheorem{corollary}{Corollary}

\usepackage{xparse}

\usepackage{listings}
\lstset{
  basicstyle=\small\ttfamily,
  columns=flexible,
  breaklines=true,
  prebreak=\raisebox{0ex}[0ex][0ex]{\ensuremath{\hookleftarrow}},
  postbreak=\raisebox{0ex}[0ex][0ex]{\ensuremath{\hookrightarrow\space}},
  keepspaces=true
}
\usepackage{comment}

\usepackage{wrapfig}

\usepackage{multirow}

\usepackage{pgf}
\usepackage{tikz}
\usepackage{subcaption}

\usepackage{adjustbox}
\usepackage{color}
\definecolor{GrayBgColor}{rgb}{0.9, 0.9, 0.9}
\definecolor{GrayBgBColor}{rgb}{0.7, 0.7, 0.7}

%%%%%%%%%%%%%
%% MACROS
%%%%%%%%%%%%%

\newcommand*{\backref}[1]{}
\newcommand*{\backrefalt}[4]{%
    \ifcase #1%
          \or Cited on page~#2.%
          \else Cited on pages~#2.%
    \fi%
    }

\makeatletter
\newcommand{\xRightarrow}[2][]{\ext@arrow 0359\Rightarrowfill@{#1}{#2}}
\makeatother

\newcommand{\cbox}[1]{\tikz \node[draw,circle, inner sep=0pt, minimum size=4mm]{#1};}

% arrays

\newcommand{\bi}{\begin{array}[t]{@{}l@{}}}
\newcommand{\ei}{\end{array}}
\newcommand{\ba}{\begin{array}}
\newcommand{\ea}{\end{array}}
\newcommand{\bda}[1]{\begin{displaymath}\ba{#1}}
\newcommand{\eda}{\ea\end{displaymath}}
\newcommand{\bp}{\begin{quote}\tt\begin{tabbing}}
\newcommand{\ep}{\end{tabbing}\end{quote}}
\newcommand{\set}[1]{\left\{
    \begin{array}{l}#1
    \end{array}
  \right\}}
\newcommand{\sset}[2]{\left\{~#1 \left|
      \begin{array}{l}#2\end{array}
    \right.     \right\}}


\newcommand{\ignore}[1]{}


\newcommand{\ms}[1]{{\bf MS: #1}}
\newcommand\pt[1]{{\bf PT: #1}}
\newcommand\bh[1]{{\bf BH: #1}}

\newcommand{\mathem}{\sf}

% rules and figures

\newcommand{\fig}[3]
        {\begin{figure*}[t]#3\
            \caption{\label{#1}#2} \end{figure*}}

\newcommand{\figurebox}[1]
        {\fbox{\begin{minipage}{\textwidth} #1 \end{minipage}}}
\newcommand{\boxfig}[3]
        {\begin{figure*}\figurebox{#3\caption{\label{#1}#2}}\end{figure*}}

\def\ruleform#1{{\setlength{\fboxrule}{1pt}\fbox{\normalsize $#1$}}}

\newcommand{\myirule}[2]{{\renewcommand{\arraystretch}{1.2}\ba{c} #1
                      \\ \hline #2 \ea}}

\newcommand{\rlabel}[1]{\mbox{(#1)}}
\newcommand{\turns}{\, \vdash \,}

%% \newcommand{\match}[3]{#1 \stackrel{#2}{\Rightarrow} #3}
\newcommand{\match}[3]{#1 \xRightarrow{#2} #3}

\newcommand{\decU}[1]{#1 \downarrow}

%% operators
\newcommand\Angle[1]{\langle#1\rangle}
\newcommand\Config[2]{(#1, #2)}
\newcommand\Override{\lhd}
\newcommand{\conc}{\cdot}%%{.}

\newcommand{\equivAcqs}[2]{#1 \sim #2}

\newcommand{\coreFunc}{\mbox{\textsc{core}}}

\newcommand{\thread}[1]{\ensuremath{\tau_{#1}}}

\newcommand{\epoch}[2]{#1 \sharp #2} %% epoch, thread id # time stamp

\newcommand{\Vacq}{V_{acq}}
\newcommand{\Vrel}{V_{rel}}

%% events
\newcommand{\evt}[2]{#1_{#2}}

\newcommand{\eventE}[1]{\ensuremath{e_{#1}}}

\newcommand{\mycolorbox}[2]{\adjustbox{margin=.6\fboxsep,bgcolor=#1,margin=-.6\fboxsep}{#2}}

\newcommand{\BLOCKED}[1]{\mycolorbox{GrayBgColor}{\ensuremath{#1}}}

\newcommand{\HIGHLIGHT}[1]{\mycolorbox{GrayBgColor}{\ensuremath{#1}}}

\newcommand{\HIGHLIGHTB}[1]{\mycolorbox{GrayBgBColor}{\ensuremath{#1}}}

\newcommand{\RACE}[1]{\mycolorbox{GrayBgColor}{\ensuremath{#1}}}

%% narrow dont care
\newcommand{\uline}[1]{\rule[0pt]{#1}{0.4pt}}% Fill this blank
\newcommand{\dontCare}{\uline{.15cm}}

%% some events

\newcommand\evtAA{e}
\newcommand\evtBB{f}
\newcommand\evtCC{g}

\newcommand{\evtA}{\evt{e}{\alpha}}
\newcommand{\evtB}{\evt{e}{\beta}}
\newcommand{\evtC}{\evt{e}{\gamma}}
\newcommand{\evtD}{\evt{e}{\delta}}
\newcommand{\evtE}{\evt{e}{\kappa}} %% \epsilon looks funny

\newcommand{\matchRel}[1]{\mathit{match(#1)}}
\newcommand{\matchAcq}[1]{\mathit{match(#1)}}


\newcommand\evtSubject{e}
\newcommand\evtAcc{a}
\newcommand\evtAccB{b}
\newcommand\evtAccC{c}
\newcommand\evtAccA{\evtAcc'}
\newcommand\evtRel{r}
\newcommand\evtRelA{\evtRel'}

%% thread ids
\newcommand{\threadMain}{\THD1}
\newcommand{\threadB}{\THD2}
\newcommand{\threadC}{\THD3}


\newcommand{\clock}[2]{#1^{#2}}
\newcommand{\ppclock}[3]{^{#2}#1^{#3}}
\newcommand{\ploc}[2]{(#1)_{#2}}

\let\threadID\thd
\newcommand{\lockVar}[1]{\textit{lock}(#1)}

\newcommand{\varState}[3]{#1^{(#2,#3)}}

%% shorthands for epochs
%% \newcommand{\e}{\alpha}
%% \newcommand{\f}{\beta}
%% \newcommand{\g}{\gamma}

\newcommand{\initVC}{\overline{0}}

\newcommand{\incC}[2]{{\mathem inc}(#1,#2)}
\newcommand{\maxC}[2]{{\mathem max}(#1,#2)}
\newcommand{\maxN}[2]{{\mathem max}(#1,#2)}

\newcommand{\dom}[1]{\textit{dom}(#1)}

\newcommand{\sep}[2]{(#1 \mid #2)}

\newcommand{\sepT}[3]{(#1 \mid #2 \mid #3)}

\newcommand{\close}[1]{\textit{close}(#1)}
\newcommand{\select}{\textit{select}}

\newcommand{\first}{{\mathem fst}}
\newcommand{\second}{{\mathem snd}}

\newcommand{\head}{{\mathem head}}
\newcommand{\last}{{\mathem last}}
\newcommand{\pp}{\ensuremath{\mathbin{\texttt{++}}}}
\newcommand\ii{\ensuremath{\mathbin{\texttt{!!}}}}

%% Concurrent Go


\newcommand{\INT}{\mbox{\mathem int}}
\newcommand{\BOOL}{\mbox{\mathem bool}}
\newcommand{\STRUCT}[1]{\mbox{\mathem{struct}\{#1\}}}
\newcommand{\COMPOS}[1]{\{#1\}}
%% \newcommand{\GO}{\mbox{\mathem go}}
\newcommand{\GO}{\mbox{\mathem spawn}}
\newcommand{\SELECT}{\mbox{\mathem select}}
\newcommand{\MAKEBUFFERED}[2]{\mbox{\mathem make}({\mathem chan} \ #1, #2)}
\newcommand{\MAKECHAN}[1]{\mbox{\mathem make}({\mathem chan} \ #1)}
\newcommand{\SYNCMAKECHAN}{\MAKECHAN{0}} %%{\mbox{\mathem makeChan}}
\newcommand{\SEND}[2]{#1 \leftarrow #2}
\newcommand{\RCV}[1]{\leftarrow #1}
\newcommand{\CLOSE}[1]{\mbox{\mathem close}(#1)}
\newcommand{\PRE}[3]{\mbox{\mathem{PRE}}  (#1, #2, #3)}
\newcommand{\PREOP}{\mbox{\mathem{PRE}}}
\newcommand{\POSTSND}[2]{\mbox{\mathem{POST}}  (#1, 0, #2)}
\newcommand{\POSTRCV}[3]{\mbox{\mathem{POST}}  (#1, 1, #2, #3)}
\newcommand{\POST}[4]{\mbox{\mathem{POST}} (#1,#2,#3,#4)}
\newcommand{\POSTOP}{\mbox{\mathem{POST}}}
\newcommand{\GID}{\mbox{\mathem{GID()}}}
\newcommand{\LineNum}{\mbox{\mathem{LineNum()}}}
\newcommand{\CHAN}[1]{\mbox{\mathem{chan}} \ #1}

\newcommand{\LOCK}{\mbox{\mathem{lock}}}
\newcommand{\UNLOCK}{\mbox{\mathem{unlock}}}

\newcommand{\Chan}[2]{\textit{Chan}^{#1}(#2)}
\newcommand{\SyncChan}{\textit{Chan}}


\newcommand{\tid}{\mbox{\mathem{tid}}}
\newcommand{\newTID}{\mbox{\mathem{tidB}}}
\newcommand{\isBuffered}[1]{\mathem{isBuffered}(#1)}


\newcommand{\pre}[1]{\textit{pre}(#1)}
\newcommand{\post}[1]{\textit{post}(#1)}
\newcommand{\postRead}[1]{\textit{read}(#1)}
\newcommand{\postWrite}[1]{\textit{write}(#1)}
\newcommand{\initVar}[1]{\textit{init}(#1)}
\newcommand{\postClose}[1]{\textit{close}(#1)}
\newcommand{\postDefault}{\textit{default}}
\newcommand{\postAsync}[2]{\textit{postB}(#1,#2)}

\newcommand{\preRead}[1]{\textit{read}(#1)}
\newcommand{\preWrite}[1]{\textit{write}(#1)}


\newcommand{\signalTrace}[1]{\textit{signal}(#1)}
\newcommand{\waitTrace}[1]{\textit{wait}(#1)}

\newcommand{\hash}[1]{{\mathem hash}(#1)}
\newcommand{\instr}[2]{\textit{instr}(#1) = #2} %%{\turns #1 \leadsto #2}
\newcommand{\instrt}[1]{\textit{instr}(#1)}
\newcommand{\retrieve}[1]{\textit{retr}(#1)}

\newcommand{\semB}[4]{(#1, #2) \turns #3 \Downarrow #4}
\newcommand{\semP}[3]{#1 \xRightarrow{#2} #3}

\newcommand{\postProc}[2]{#1 \Rightarrow #2}

\newcommand{\assign}{:=}

\newcommand{\sndEvt}[2]{#1 \sharp #2!}
\newcommand{\rcvEvt}[3]{#1 \leftarrow #2 \sharp #3?}
\newcommand{\readEvt}[2]{#1 \sharp \textit{read}(#2)}
\newcommand{\writeEvt}[2]{#1 \sharp \textit{write}(#2)}
\newcommand{\closeEvt}[2]{#1 \sharp \textit{close}(#2)}
\newcommand{\defaultEvt}[1]{#1 \sharp \textit{default}}
\newcommand{\initEvt}[2]{#1 \sharp \textit{init}(#2)}

\newcommand{\snd}[1]{#1 !}
\newcommand{\rcv}[1]{#1 ?}
\newcommand{\readR}[1]{\textit{read}(#1)}
\newcommand{\writeW}[1]{\textit{write}(#1)}

\newcommand{\dummyTID}{\infty}


\newcommand{\LKA}{\LK1}
\newcommand{\LKB}{\LK2}
\newcommand{\LKC}{\LK3}
\newcommand{\LKD}{\LK4}
\newcommand{\LKE}{\LK5}

\newcommand{\VA}{x}
\newcommand{\VB}{y}
\newcommand{\VC}{z}


\newcommand{\lockE}[1]{\textit{acq}(#1)} %%{\mathit{acq}(#1)}
\let\acqE\lockE
\newcommand{\reqE}[1]{\textit{req}(#1)}
\newcommand{\reqLockE}[1]{\reqE{#1}} %%{\textit{req}(#1)}
\newcommand{\unlockE}[1]{\textit{rel}(#1)} %%{\mathit{rel}(#1)}
\let\relE\unlockE
\newcommand{\readE}[1]{rd(#1)}
\newcommand{\writeE}[1]{wr(#1)}
\newcommand{\forkE}[1]{\textit{fork}(#1)}
\newcommand{\joinE}[1]{\textit{join}(#1)}

% \newcommand{\readEE}[2]{r(#1)_{#2}}
% \newcommand{\writeEE}[2]{w(#1)_{#2}}
% \newcommand{\reqLockEE}[2]{req(#1)_{#2}}  %%  request
% \newcommand{\lockEE}[2]{acq(#1)_{#2}}  %% acquire
% \newcommand{\unlockEE}[2]{rel(#1)_{#2}} %% release
% \newcommand{\forkEE}[2]{\textit{fork}(#1)_{#2}}
% \newcommand{\joinEE}[2]{\textit{join}(#1)_{#2}}

%% Trace ordering.
%% USE e \XLt[T] f, or e \XLt f.
%% Similarly for \XConc.

%% Order as appearing in the trace.
\newcommand\TrSymbol{\mbox{\tiny Tr}}
\newcommand{\TrLt}[1][]{\mathrel{<_{\TrSymbol}^{#1}}}
% \newcommand{\ltTrace}[3]{#2 \TrLt[#1] #3}
% \newcommand{\ltTraceT}[4]{\ltTrace{#1}{\ltTrace{#1}{#2}{#3}}{#4}}
% \newcommand{\ltTr}[2]{\ltTrace{}{#1}{#2}}
% \newcommand{\ltTrr}[3]{\ltTraceT{}{#1}{#2}{#3}}
\newcommand\POSymbol{\mbox{\tiny PO}}
\newcommand\POLt[1][]{\mathrel{<_{\POSymbol}^{#1}}}
\newcommand\POLeq[1][]{\mathrel{\leq_{\POSymbol}^{#1}}}

% \newcommand{\lteqTr}[2]{#1 \leq_{\textit{tr}} #2}
% \newcommand{\lteqTrace}[3]{#2 \leq_{\textit{tr}}^{#1} #3}

%% partial orders

\newcommand{\hbSymbol}{\mbox{\tiny HB}}
\newcommand{\cpSymbol}{\mbox{\tiny CP}}
\newcommand{\wcpSymbol}{\mbox{\tiny WCP}}
\newcommand{\sdpSymbol}{\mbox{\tiny SDP}}

% \newcommand{\ltHBNoT}[2]{#1 <_{\hbSymbol} #2}
\newcommand{\HBLt}[1][]{\mathrel{<_{\hbSymbol}^{#1}}}
% \newcommand{\ltWCPNoT}[2]{#1 <_{\wcpSymbol} #2}
\newcommand{\WCPLt}[1][]{\mathrel{<_{\wcpSymbol}^{#1}}}


%% Must happens-before order.

\newcommand{\mhbSymbol}{\mbox{\tiny MHB}}
\newcommand{\cmhbNoSymbol}{\mbox{\tiny NoCMHB}}
\newcommand{\cmhbFullSymbol}{\mbox{\tiny FullCMHB}}
\newcommand{\dpSymbol}{\mbox{\tiny DP}}

\newcommand{\negltMustHB}[3]{#2 \not<_{\mhbSymbol}^{#1} #3}
% \newcommand{\ltMustHB}[3]{#2 <_{\mhbSymbol}^{#1} #3}
% \newcommand{\ltNoCMHB}[3]{#2 <_{\cmhbNoSymbol}^{#1} #3}
\newcommand{\NoCMHBLt}[1][]{\mathrel{<_{\cmhbNoSymbol}^{#1}}}
% \newcommand{\ltFullCMHB}[3]{#2 <_{\cmhbFullSymbol}^{#1} #3}
\newcommand{\FullCMHBLt}[1][]{\mathrel{<_{\cmhbFullSymbol}^{#1}}}
% \newcommand{\ltMustHBB}[4]{#2 <_{\mhbSymbol}^{#1} #3 <_{\mhbSymbol}^{#1} #4}
% \newcommand{\ltStrongAndNormalMustHBB}[4]{#2 <_{\smhbSymbol}^{#1} #3 <_{\mhbSymbol}^{#1} #4}
% \newcommand{\negltMustHBB}[4]{#2 \not<_{\mhbSymbol}^{#1} #3 <_{\mhbSymbol}^{#1} #4}
% \newcommand{\ltMHB}[2]{#1 <_{\mhbSymbol} #2}
% \newcommand{\ltFullCMHBNoT}[2]{#1 <_{\cmhbFullSymbol} #2}
% \newcommand{\ltNoCMHBNoT}[2]{#1 <_{\cmhbNoSymbol} #2}

% \newcommand{\ltCMustHB}[4]{\ltMustHB{#1(#2)}{#3}{#4}}
% \newcommand{\ltCMHB}[3]{\ltCMustHB{}{#1}{#2}{#3}}
% \newcommand{\ltCMHBNoT}[2]{\ltCMustHB{}{}{#1}{#2}}
\newcommand{\CMHBLt}[2][]{\mathrel{<_{\mhbSymbol}^{#1(#2)}}}
% \newcommand{\ltDlP}[3]{#2 <_{\dpSymbol}^{#1} #3}
% \newcommand{\ltDP}[2]{\ltDlP{}{#1}{#2}}
\newcommand{\DPLt}[1][]{\mathrel{<_{\dpSymbol}^{#1}}}

% \newcommand{\ltMHBB}[3]{#1 <_{\mhbSymbol} #2 <_{\mhbSymbol} #3}
% \newcommand{\ltMHBNoT}[2]{#1 <_{\mhbSymbol} #2}
% \newcommand{\ltMHBNoTSym}{<_{\mhbSymbol}}

% \newcommand{\concMustHB}[3]{#2 \mathrel{{||}_{\mhbSymbol}^{#1}} #3}
% \newcommand{\concFullCMHB}[3]{#2 \mathrel{{||}_{\cmhbFullSymbol}^{#1}} #3}
% \newcommand{\concNoCMHB}[3]{#2 \mathrel{{||}_{\cmhbNoSymbol}^{#1}} #3}
\newcommand{\NoCMHBConc}[1][]{\mathrel{{||}_{\cmhbNoSymbol}^{#1}}}
% \newcommand{\concMHB}[2]{#1 \mathrel{{||}_{\mhbSymbol}} #2}
% \newcommand{\concFullCMHBNoT}[2]{#1 {||}_{\cmhbFullSymbol} #2}
\newcommand{\FullCMHBConc}[1][]{\mathrel{{||}_{\cmhbFullSymbol}^{#1}}}
% \newcommand{\concPWR}[3]{#2 {\|}_{PWR}^{#1} #3}

% \newcommand{\concCMustHB}[4]{\concMustHB{#1(#2)}{#3}{#4}}
% \newcommand{\concCMHB}[3]{\concMustHB{(#1)}{#2}{#3}}
\newcommand{\CMHBConc}[2][]{\mathrel{{||}_{\mhbSymbol}^{#1}}}

\RequirePackage{amsmath}
\DeclareMathOperator\crp{crp}
\DeclareMathOperator\TRWcrp{TRWcrp}

%% shorthand for acquire event
\newcommand{\ACQ}[1]{\evtAcc_{#1}}% {\mathit{acq}_{{\alpha}_{#1}}}
\newcommand{\ACQA}[1]{\evtAccA_{#1}}% {\mathit{acq}_{{\alpha}_{#1}}}
\newcommand{\ACQB}[1]{\evtAccB_{#1}}% {\mathit{acq}_{{\alpha}_{#1}}}
\newcommand{\REL}[1]{\evtRel_{#1}}
\newcommand{\ReqSymb}{q}
\newcommand{\REQ}[1]{\ReqSymb_{#1}}
\newcommand{\REQA}[1]{\ReqSymb_{#1}}
\newcommand{\REQB}[1]{\ReqSymb'_{#1}}
\newcommand{\REQC}[1]{\ReqSymb''_{#1}}
\newcommand{\LK}[1]{l_{#1}}%{x_{{\alpha}_{#1}}}
\let\THD\thread
% \newcommand{\THD}[1]{\tau_{#1}}% {t_{{\alpha}_{#1}}}

\newcommand{\ACQX}[2]{\mathit{acq(#1)}_{#2}}

%% predictable deadlock configuration
\newcommand{\PRD}[2]{\bot^{#1}\{ #2\}}  %% or \lightning

%% cyclic lock dependency chain
\newcommand{\CLDC}[3]{\circlearrowright^{#1}_{#2} #3}%{\bigcirc^{#1}_{#2} #3}

%% all predictable deadlocks
\newcommand{\TrueDL}[1][{}]{G^{#1}_{DL}}

%% lockset resource deadlock predictor
\newcommand{\PredictDL}[2][{}]{P^{#1}_{DL}(#2)}



%% false positives/negatives
\newcommand{\FalsePDL}[2]{{\mathit FP}^{#1}_{DL} (#2)}
\newcommand{\FalseNDL}[2]{{\mathit FN}^{#1}_{DL} (#2)}

\newcommand{\FPDL}[1]{{\mathit FP}_{DL} (#1)}
\newcommand{\FNDL}[1]{{\mathit FN}_{DL} (#1)}

%% true positives
\newcommand{\TruePDL}[2]{{\mathit TP}^{#1}_{DL} (#2)}
\newcommand{\TPDL}[1]{{\mathit TP}(#1)_{DL}}

\newcommand{\nf}[1]{\mbox{\normalfont{#1}}}

\newcommand{\extract}[2]{#1 \downarrow #2}
\newcommand{\pos}[1]{\textit{pos}(#1)}
\newcommand{\posP}[2]{\textit{pos}_{{\scriptstyle #1}}(#2)}
\newcommand{\compTID}[1]{\textit{thread}(#1)}
\newcommand{\compTIDP}[2]{\textit{thread}_{{\scriptstyle #1}}(#2)}
\newcommand{\evtID}[1]{\textit{ident}(#1)}
\newcommand{\len}[1]{\textit{len}(#1)}
\newcommand{\perm}[1]{\textit{perm}(#1)}
\newcommand\Thds{\Theta}


\newcommand{\supVC}[2]{#1 \sqcup #2}
\newcommand{\threadVC}[1]{\textit{Th}(#1)}
\newcommand{\lockVC}[1]{\textit{Rel}(#1)}
\newcommand{\lastWriteVC}[1]{\ensuremath{L_W}(#1)}
\newcommand{\lastWriteVCt}[1]{\ensuremath{L_{W_t}}(#1)}
\newcommand{\lastWriteVCL}[1]{\ensuremath{L_{W_L}}(#1)}
\newcommand{\lastReadVC}[1]{\ensuremath{L_R}(#1)}
\newcommand{\concEvt}[1]{\textit{conc}(#1)}
\newcommand{\accVC}[2]{#1[ #2 ]}
\newcommand{\rwVC}[1]{\textit{RW}(#1)}

\newcommand{\vcEvt}{\textit{evt}}
\newcommand{\edges}[1]{\textit{edges}(#1)}

\newcommand{\gtEdge}{\prec}

\newcommand{\accTr}[2]{#1[ #2 ]}

%% \newcommand{\cs}[2]{\thread{#1}{\Angle{#2}}}

% \newcommand\conflict[2]{\ensuremath{#1\##2}}%{\ensuremath{#1\mathbin{\,\not\!\leftrightarrow}#2}}
\newcommand\conf{\mathrel{\bowtie}}

%% false positives/negatives

\newcommand{\FalseP}[2]{\textit{FP}^{#1} (#2)}
\newcommand{\FalseN}[2]{\textit{FN}^{#1} (#2)}

\newcommand{\FP}[1]{\textit{FP}(#1)}
\newcommand{\FN}[1]{\textit{FN}(#1)}


%% data race

\newcommand\PredictDRP[2][{}]{\ensuremath{{P'}_{DR}^{#1} (#2)}}
\newcommand\PredictDR[2][{}]{\ensuremath{P_{DR}^{#1} (#2)}}
\newcommand\TrueDR[1][{}]{\ensuremath{G_{DR}^{#1}}}

\newcommand{\negDataRace}[3]{#2 \not\bowtie^{#1} #3}
\newcommand{\DataRace}[3]{#2 \bowtie^{#1} #3}
\newcommand{\DR}[2]{#1 \bowtie #2}

\newcommand\indexedcap{\mathbin{\cap'}}

%% lockdependency

\newcommand{\LD}[3]{{\langle #1, #2, #3 \rangle}}
\newcommand{\LDFour}[4]{{\langle #1, #2, #3,#4 \rangle}}
\newcommand{\LDAcqWithT}[2]{\ensuremath{D_{#1}(#2)}}
\newcommand{\LDAcq}[1]{\ensuremath{D(#1)}}
\newcommand\LDS[4][{}]{\ensuremath{D^{#1}\langle #2, #3, \{#4\}
    \rangle}}
\newcommand\LDInduced[2]{\ensuremath{D^{#1}_{#2}}}

\newcommand{\LDSet}{{\mathcal D}}


\newcommand{\LDt}{\ensuremath{\mathcal{L}_{\mathit D}}} %% temporaries
\newcommand{\LDp}{\ensuremath{\mathcal{D}}} %% permanent

\newcommand{\LDMapSym}{\ensuremath{\mathcal{M}}}
\newcommand{\LDMap}[3]{\LDMapSym{\langle #1, #2, #3 \rangle}}

\newcommand{\GCMapSym}{\ensuremath{\mathcal{G}}}
\newcommand{\GlobalLS}{{\mathit  All_{lh}}}

\newcommand{\Ald}{F} %% abstract lock dependency stmbol consistent with SPD paper
\newcommand{\AldE}{E} %% specific entry

\newcommand{\DDs}{{\mathcal{D}}}
\newcommand{\AAs}{{\mathcal{A}}}
\newcommand{\BBs}{{\mathcal{B}}}

\newcommand{\standard}{S}
\newcommand{\intrathread}{C}
\newcommand{\intrathreadthread}{CT}
\newcommand{\refinedintrathread}{R}

\newcommand{\pwrsymbol}{\mbox{\tiny PWR}}%{PWR}

% \newcommand{\ltPWR}[2]{#1 <_{\pwrsymbol} #2}
% \newcommand{\ltPWRT}[3]{#2 <_{\pwrsymbol}^{#1} #3}
\newcommand{\PWRLt}[1][]{\mathrel{<_{\pwrsymbol}^{#1}}}
\newcommand{\PWRConc}[1][]{\mathrel{{||}_{\pwrsymbol}^{#1}}}


%% program order + fj
\newcommand{\fjsymbol}{\mbox{\tiny FJ}}
% \newcommand{\ltFJ}[2]{#1 <_{\fjsymbol} #2}
% \newcommand{\ltFJT}[3]{#2 <_{\fjsymbol}^{#1} #3}

%% program order + fj + last write
\newcommand{\lwSymbol}{\mbox{\tiny LW}}
% \newcommand{\ltLW}[2]{#1 <_{\lwSymbol} #2}
% \newcommand{\ltLWT}[3]{#2 <_{\lwSymbol}^{#1} #3}
\newcommand{\LWLt}[1][]{\mathrel{<_{\lwSymbol}^{#1}}}

%% total order among r/w
\newcommand{\trwSymbol}{\mbox{\tiny TRW}}
% \newcommand{\ltTRW}[2]{#1 <_{\trwSymbol} #2}
% \newcommand{\ltTRWT}[3]{#2 <_{\trwSymbol}^{#1} #3}
\newcommand{\TRWLt}[1][]{\mathrel{<_{\trwSymbol}^{#1}}}
\newcommand{\TRWConc}[1][]{\mathrel{{||}_{\trwSymbol}^{#1}}}

%% vector clocks
\newcommand{\VCConc}{\mathrel{||}}

%% critical sections
% \newcommand{\CSect}[3]{CS_{#1}(#3)^{#2}}
% \newcommand{\CSec}[2]{CS(#2)^{#1}}    %% drop trace
% \newcommand{\CS}[1]{CS(#1)}           %% drop trace and guard
% \newcommand{\CSct}[2]{CS_{#1}(#2)}
% \newcommand{\CSc}[1]{CS(#1)}
\makeatletter
\DeclareMathOperator\@CS{CS}
\newcommand\CS[1][]{\@CS_{#1}}
\makeatother


%% acquires held
% \newcommand{\AHWithT}[2]{{AH_{#1}(#2)}}
% \newcommand{\AH}[1]{{AH(#1)}}
\makeatletter
\DeclareMathOperator\@AH{AH}
\newcommand\AH[1][]{\@AH_{#1}}
\makeatother

%% locks held
% \newcommand{\LHWithT}[2]{LH_{#1}(#2)}
% \newcommand{\LH}[1]{LH(#1)}
\makeatletter
\DeclareMathOperator\@LH{LH}
\newcommand\LH[1][]{\@LH_{#1}}
\makeatother

%%% three variants
\newcommand{\csAR}[1]{CS_{C}(#1)}    %% \newcommand{\csAR}[1]{CS^{A}_{R}(#1)}
\newcommand{\csR}[1]{CS_{R}(#1)}
\newcommand{\csA}[1]{CS^{A}(#1)}

\newcommand{\AcqRelPair}[2]{\Angle{#1,#2}}




\newcommand{\StdCSect}[3]{\CSect{\standard}{#1}{#2}{#3}}
\newcommand{\StdCSec}[2]{\CSec{\standard}{#1}{#2}}
\newcommand{\StdCS}[1]{\CS{\standard}{#1}}

\newcommand{\IntraCSect}[3]{\CSect{\intrathread}{#1}{#2}{#3}}
\newcommand{\IntraCSec}[2]{\CSec{\intrathread}{#1}{#2}}
\newcommand{\IntraCS}[1]{\CS{\intrathread}{#1}}

\newcommand{\RefIntraCSect}[3]{\CSect{\refinedintrathread}{#1}{#2}{#3}}
\newcommand{\RefIntraCSec}[2]{\CSect{\refinedintrathread}{#1}{#2}}
\newcommand{\RefIntraCS}[1]{\CS{\refinedintrathread}{#1}}





\newcommand{\LHSym}[1]{{LH_{#1}}}           %% symbol
\newcommand{\LocksSym}[1]{\ensuremath{\mathcal{L}_{#1}}}           %% symbol

\newcommand{\IntraLHeld}[2]{\LHeld{\intrathread}{#1}{#2}}
\newcommand{\IntraLH}[1]{\LH(\intrathread){#1}}
\newcommand{\IntraLHSym}{\LHSym{\intrathread}}
\newcommand{\IntraLocksSym}{\LocksSym{\intrathread}}

\newcommand{\CTTLHeld}[2]{\LHeld{\intrathreadthread}{#1}{#2}}
\newcommand{\CTTLH}[1]{\LH(\intrathreadthread){#1}}
\newcommand{\CTTLHSym}{\LHSym{\intrathreadthread}}
\newcommand{\CTTLocksSym}{\LocksSym{\intrathreadthread}}

\newcommand{\RefIntraLHeld}[2]{\LHeld{\refinedintrathread}{#1}{#2}}
\newcommand{\RefIntraLH}[1]{\LH(\refinedintrathread){#1}}
\newcommand{\RefIntraLHSym}{\LHSym{\refinedintrathread}}
\newcommand{\RefIntraLocksSym}{\LocksSym{\refinedintrathread}}

\newcommand{\StdLHeld}[2]{\LHeld{\standard}{#1}{#2}}
\newcommand{\StdLH}[1]{\LH(\standard){#1}}
\newcommand{\StdLHSym}{\LHSym{\standard}}
\newcommand{\StdLocksSym}{\LocksSym{\standard}}

\newcommand{\AcqHeld}{\mathcal{A}_{H}}

\newcommand{\pushAcq}{\mathit{push}}
\newcommand{\popAcq}{\mathit{pop}}
\newcommand{\lockSet}{\mathit{lockset}}

\newcommand{\PWRLH}[1]{\LH(\pwrsymbol){#1}}

\newcommand{\Locks}{L} %% denotes a set of locks

\newcommand{\Hist}[1]{\ensuremath{\mathcal{H}(#1)}} %% lock history used by PWR

%% next event
\newcommand{\NEXT}[3]{{\textit{next}(#2,#3)_{#1}}}

%% successor event
\newcommand{\SUCC}[2]{{\textit{succ}(#2)_{#1}}}

%% predecessor event
\newcommand{\PRED}[2]{{\textit{pred}(#2)_{#1}}}

\newcommand{\acq}[1]{acq(#1)}
\newcommand{\rel}[1]{rel(#1)}

\newcommand{\LS}[1]{LS(#1)}               %% lockset for each event
\newcommand{\LSSym}{LS}
\newcommand{\LockSet}{\mathem Lockset}

\newcommand{\LSt}[1]{LS_t(#1)}   %% lockset for each thread
\newcommand{\LStSym}{LS_t}
\newcommand{\Acq}[1]{Acq(#1)}

\newcommand{\DP}[1]{\{ #1 \}}  %% deadlock pattern

\newcommand{\replay}[3]{\semP{#1}{#2}{#3}}
\newcommand{\sepThree}[3]{(#1 \mid #2 \mid #3)}
\newcommand{\acqLast}{\mathem A}
\newcommand{\writeLast}{\mathem W}
\newcommand{\writeLastMap}{\mathem M}

%% subset of set of traces
\newcommand{\TS}{{\mathcal T}}

\newcommand{\rwTT}[2]{{#1}^{\textit{rw}}_{#2}}
\newcommand{\rwT}[1]{T^{\textit{rw}}_{#1}}
\newcommand{\rwTa}{\rwT{a}}

\newcommand{\rwTr}{T^{\textit{rw}}}
\newcommand{\relT}{T^{\textit{rel}}}
\newcommand{\relTy}{\relT_y}

% \newcommand{\trw}[2]{#1 <^{\scriptscriptstyle TRW} #2}

% \newcommand{\hb}[2]{#1 <^{\scriptscriptstyle HB} #2}
\newcommand{\shb}[2]{#1 <^{\scriptscriptstyle SHB} #2}
% \newcommand{\wcp}[2]{#1 <^{\scriptscriptstyle WCP} #2}
\newcommand{\www}[2]{#1 <^{\scriptscriptstyle W3} #2}
\newcommand{\wwws}[2]{#1 <^{\scriptscriptstyle W3S} #2}
\newcommand{\wwwa}[2]{#1 <^{\scriptscriptstyle W3A} #2}   %% weak wcp with acquire
\newcommand{\wwwr}[2]{#1 <^{\scriptscriptstyle W3R} #2}   %% weak wcp for read

\newcommand{\powrdSym}{<^{\scriptscriptstyle PO+WRD}}
\newcommand{\shbSym}{<^{\scriptscriptstyle SHB}}
% \newcommand{\hbSym}{<^{\scriptscriptstyle HB}}
% \newcommand{\wcpSym}{<^{\scriptscriptstyle WCP}}
\newcommand{\wwwSym}{<^{\scriptscriptstyle W3}}
\newcommand{\wwwsSym}{<^{\scriptscriptstyle W3S}}
\newcommand{\wwwaSym}{<^{\scriptscriptstyle W3A}}
\newcommand{\wwwrSym}{<^{\scriptscriptstyle W3R}}
\newcommand{\wwwhbcSym}{<^{\scriptstyle W3HBC}}

% \newcommand{\wcpRelAcq}[2]{#1 <^{\scriptscriptstyle WCP_1} #2}
% \newcommand{\wcpRelEvt}[2]{#1 <^{\scriptscriptstyle WCP_2} #2}


\newcommand{\allConcsP}[1]{{\mathcal C}^{#1}}
\newcommand{\accConcs}[1]{{PC}(#1)}

\newcommand{\allPredRacesP}[1]{{\mathcal P}^{#1}}
%% \newcommand{\allPredRaces}{{\mathcal P}}
\newcommand{\PotentialP}[2]{{\mathcal R}^{#1}_{\not #2}}
%% \newcommand{\Potential}[1]{{\mathcal R}_{#1}}
\newcommand{\PotentialWWW}[1]{{\mathcal R}^{#1}_{\wwwSym}}

\newcommand{\schedSpecificPredRacesP}[1]{{\mathcal S}^{#1}}
%% \newcommand{\schedSpecificPredRaces}{{\mathcal S}}

\newcommand{\prefixOf}[2]{#1 \rhd #2}
\newcommand{\dataRace}[4]{#3 \stackrel{\prefixOf{#1}{#2}}{\asymp} #4}

\newcommand{\SHBEE}{\mbox{SHB$^{\scriptstyle E+E}$}}
\newcommand{\SHBEELimit}{\mbox{SHB$_{\scriptstyle L}^{\scriptstyle E+E}$}}
\newcommand{\WWWPOEE}{\mbox{W3PO$^{\scriptstyle E+E}$}}
\newcommand{\WWWPOEELimit}{\mbox{W3PO$_{\scriptstyle L}^{\scriptstyle E+E}$}}
\newcommand{\WWWPOWRD}{\mbox{W3PO$^{\scriptstyle WRD}$}}
\newcommand{\WWWPO}{\mbox{W3PO}}
\newcommand{\WWWPOZero}{\mbox{W3PO}$_{\scriptstyle L}$}
\newcommand{\TSANWRD}{\mbox{TSanWRD}}


\newcommand{\NoFP}{\mbox{FP$_{\not\exists}$}}
\newcommand{\OnlyFP}{\mbox{FP$_{\forall}$}}
\newcommand{\NoFN}{\mbox{FN$_{\not\exists}$}}
\newcommand{\OnlyFN}{\mbox{FN$_{\forall}$}}


\RequirePackage{xspace}

\def\Tunc{Tun\c{c}\xspace}

\newcommand{\SPDOfflineUD}{\mbox{SPDOffline$^{*}$}\xspace}    %% our variant
\newcommand{\SPDOffline}{\mbox{SPDOffline}\xspace}            %% theirs

% \AtBeginEnvironment{definition}{\pushQED{%
%   \ifmmode \lhd
%   \else
%     \leavevmode\unskip\penalty9999 \hbox{}\nobreak\hfill
%   %    \end{macrocode}
%   %    The hbox is to prevent a line break within the \cn{qedsymbol} if it
%   %    is defined to be something composite--- e.g., things like
%   %    \verb"(Corollary 1.2) \openbox" as are occasionally done.
%   %    \begin{macrocode}
%     \quad\hbox{$\lhd$}%
%   \fi
% }}
% \AtEndEnvironment{definition}{\popQED}

% Bar chart stuff
\makeatletter
\newdimen\legendxshift
\newdimen\legendyshift
\newcount\legendlines
% distance of frame to legend lines
\newcommand{\bclldist}{1mm}
\newcommand{\bclegend}[3][10mm]{%
	% initialize
	\legendxshift=0pt\relax
	\legendyshift=0pt\relax
	\xdef\legendnodes{}%
	% get width of longest text and number of lines
	\foreach \lcolor/\ltext [count=\ll from 1] in {#3}%
	{\global\legendlines\ll\pgftext{\setbox0\hbox{\bcfontstyle\ltext}\ifdim\wd0>\legendxshift\global\legendxshift\wd0\fi}}%
	% calculate xshift for legend; \bcwidth: from bchart package; \bclldist: from node frame, inner sep=\bclldist (see below)
	% \@tempdima: half width of bar; 0.72em: inner sep from text nodes with some manual adjustment
	\@tempdima#1\@tempdima0.5\@tempdima
	\pgftext{\bcfontstyle\global\legendxshift\dimexpr\bcwidth-\legendxshift-\bclldist-\@tempdima-0.72em}
	% calculate yshift; 5mm: heigt of bar
	\legendyshift\dimexpr5mm+#2\relax
	\legendyshift\legendlines\legendyshift
	% \bcpos-2.5mm: from bchart package; \bclldist: from node frame, inner sep=\bclldist (see below)
	\global\legendyshift\dimexpr\bcpos-2.5mm+\bclldist+\legendyshift
	% draw the legend
	\begin{scope}[shift={(\legendxshift,\legendyshift)}]
		\coordinate (lp) at (0,0);
		\foreach \lcolor/\ltext [count=\ll from 1] in {#3}%
		{
			\node[anchor=north, minimum width=#1, minimum height=5mm,fill=\lcolor] (lb\ll) at (lp) {};
			\node[anchor=west] (l\ll) at (lb\ll.east) {\bcfontstyle\ltext};
			\coordinate (lp) at ($(lp)-(0,5mm+#2)$);
			\xdef\legendnodes{\legendnodes (lb\ll)(l\ll)}
		}
		% draw the frame
		\node[draw, inner sep=\bclldist,fit=\legendnodes] (frame) {};
	\end{scope}
}
\makeatother

\newcommand\cycleLock[3]{CL^{#1}_{#2}(#3)}

\RequirePackage{algorithm}
\RequirePackage{algorithmicx}
\PassOptionsToPackage{noend}{algpseudocode}
\RequirePackage{algpseudocode}
\algnewcommand{\IfThen}[2]{\State \algorithmicif\ #1\ \algorithmicthen\ #2}
\algnewcommand{\ForDo}[2]{\State \algorithmicfor\ #1\ \algorithmicdo\ #2}

\NewDocumentEnvironment{proofsketch}{}{\begin{proof}[Proof (sketch)]}{\end{proof}}

\newcommand\cond[1]{\textsf{[#1]}}
\DeclareMathOperator\evts{evts}
\DeclareMathOperator\thd{thd}
\DeclareMathOperator\var{var}
\DeclareMathOperator\thds{thds}
\DeclareMathOperator\proj{proj}

\makeatletter
\DeclareMathOperator\@mod{mod}
\renewcommand\mod{\mathbin{\%}}
\makeatother

\RequirePackage{xparse}
\let\oldfigure\figure
\let\endoldfigure\endfigure
\RenewDocumentEnvironment{figure}{O{}}{\oldfigure[#1]\small}{\endoldfigure}

\def\ih#1{IH\textsubscript{#1}\xspace}

\RequirePackage{hyperref}
\usepackage[capitalise,noabbrev]{cleveref}
\crefname{line}{line}{lines}
