
%--------------------------------------------------------
%--------------------------------------------------------
\section{Introduction}
\label{sec:intro}

To fully leverage today’s multicore architectures, programs must be
designed for concurrency.
Yet, concurrent programming introduces
significant challenges, as developers must use synchronization
primitives, such as locks, to ensure correct behavior.
Incorrect
use of locks can lead to critical issues, including resource
deadlocks.
Resource deadlock occurs when two or more threads are
stuck waiting for each other’s locks, resulting in a standstill that
halts program execution.
These deadlocks are particularly difficult to
diagnose, as they may only appear intermittently, often after hundreds
of successful runs.

Program analysis can help  programmers detect potential resource
deadlocks.
In particular, dynamic analysis aims at predicting a program's behavior
under different schedules by analyzing a trace of events (including
acquires and releases of locks)
generated during a
%% single, successful
program run.
% Interesting events are acquiring/releasing a lock and reading/writing
% a global variable.
A common approach to predicting deadlocks is to search a trace for \emph{deadlock patterns},
that involve series of lock acquire events with a cyclic dependency.
Deadlock patterns are
not sufficient to characterize resource deadlocks, meaning that they may lead
to false positives: predicted deadlocks that cannot manifest under any schedule.

% The literature distinguishes between resource deadlocks and communication deadlocks.
% A communication deadlock happens when threads are stuck waiting on messages from other threads.
% A resource deadlock happens when when we find a set of $n>1$ threads
% where each thread tries to acquire a lock that is held by some other thread in the set.
% We consider here concurrent programs with shared variables and locks but without any
% message-passing primitives.
% Hence,  we only examine resource deadlocks where
% our focus is on dynamic deadlock prediction methods.

% This
% discussion is limited to concurrent programs that use shared variables
% and locks, without incorporating message-passing primitives. Our
% primary focus is on methods for dynamically predicting resource
% deadlocks.

\paragraph{Prior work.}

To eliminate false positives,
prior work~\cite{kalhauge2018sound,10.1007/978-3-319-23404-5_13,10.1145/1542476.1542489,Samak:2014:TDD:2692916.2555262,10.1007/978-3-319-23404-5_13}
attempts to find a schedule which exposes the deadlock, either symbolically via
SMT solving or by program re-execution.
Such methods can be very costly, as we might need to exhaustively explore
all alternative schedulings of the execution.
To improve efficiency, recent works~\cite{conf/fse/CaiYWQP21,conf/pldi/TuncMPV23}
employ alternative solving methods to remove false positives.
However, these methods may still be costly (the time complexity of \cite{conf/fse/CaiYWQP21} is quadratic in trace length) and leave deadlocks unpredicted (\cite{conf/pldi/TuncMPV23} reports only so-called ``sync-preserving'' deadlocks).

\paragraph{Our novel approach.}

We introduce a new approach that eliminates false-positive deadlock patterns, inspired by partial-order methods known from data-race prediction.
Particularly, our work is novel in that it requires only an analysis of the ordering of trace events, in contrast to prior works that incorporate partial orders (such as~\cite{kalhauge2018sound,conf/fse/CaiYWQP21,conf/pldi/TuncMPV23}) but need further steps akin to trace exploration.

To be precise, our approach refines the notion of deadlock pattern from the
literature by introducing the following two conditions:

\begin{description}

  \item[Partially-ordered acquires]
    A deadlock pattern is a false positive if some of its acquires are ordered.
    Hence, we only consider deadlock patterns where all acquires are concurrent (i.e., pairwise unordered).

  \item[Partially-ordered deadlock patterns]
    A deadlock pattern is a false positive if it is blocked by an earlier deadlock.
    We identify such situations by defining a partial order on deadlock
    patterns, and only consider the ``earliest'' deadlock patterns.
    %
    %% MS: note, DP-Block order can only be computed once we identified all deadlock pattern
    %% The partial order information is computed while processing the trace and
    %% Via the partial order information among acquires and deadlock patterns,

\end{description}

\noindent
This way, given an appropriate partial order, our method instantly removes false positives without requiring extra steps or limiting the search to a subclass of deadlocks.

\paragraph{A new partial order for sound and efficient deadlock prediction.}

Thus, the challenge lies in finding an appropriate partial order among acquires
that fits the purpose of deadlock prediction (the partial order among deadlock patterns is fairly straightforward).
Existing partial orders employed in the context
of data race prediction~\cite{lamport1978time,Smaragdakis:2012:SPR:2103621.2103702,conf/pldi/KiniMV17,10.1145/3360605}
are not suitable, because when applied to deadlock detection we may end up with false positives \emph{and} false negatives.
%% MS: For HB/SHB we would have no false positives because no deadlock will ever be reported.
%% Discussed in the overview section.

In this paper, we introduce the novel TRW partial order.
We are able to show that under TRW all deadlock patterns correspond to true deadlocks.
However, we may encounter false negatives.
Earlier complexity results~\cite{conf/pldi/TuncMPV23} suggest that it is impossible to find
an \emph{efficient} deadlock-prediction method that is \emph{sound} (no false positives) and \emph{complete} (no false negatives).
Hence, we also consider PWR~\cite{conf/mplr/SulzmannS20},
and show that under PWR deadlock patterns are complete, although we may face some false positives.
Our experiments show that PWR and TRW are efficient and report the same set of deadlock patterns
for an extensive benchmark suite.


%% Conclusion later:
%% PWR and TRW, ``good'' candidates,
%% opens up the research for further variants ...

\paragraph{Contributions and outline.}

In summary, our contributions are:
\begin{itemize}

  \item
    We define precisely our refined deadlock patterns
    using partial orders
    among events and among deadlock patterns.
    For the TRW partial order we establish soundness (\cref{{s:sound}}), and
    for the PWR partial order we establish completeness
    (%
    %subject to a mild technical condition;
    \cref{s:complete}).

  \item We present an implementation of our approach as an offline version of
    the UNDEAD deadlock predictor~\cite{conf/ase/ZhouSLCL17}, with versions based
    on PWR and TRW
    (\cref{sec:implementation}).

  \item We study the impact on performance and precision
    of deadlock prediction under PWR and TRW
    (\cref{sec:experiments}).
    We also compare in detail against the sync-preserving deadlock
    predictor \SPDOffline~\cite{conf/pldi/TuncMPV23}.

\end{itemize}
%
\Cref{sec:overview} gives an overview of our work, and \cref{sec:prelim} introduces basic definitions and notations.
\Cref{sec:related,sec:conclusions} discuss related work and draw some conclusions, respectively.
The supplement can be safely ignored; it contains detailed proofs, additional examples and experimentation results, and preliminary access to our implementation (to be submitted as an artifact).


%%% Local Variables:
%%% mode: latex
%%% TeX-master: "main"
%%% End:
