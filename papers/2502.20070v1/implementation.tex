
\newcommand{\setalglineno}[1]{%
  \setcounter{ALG@line}{\numexpr#1-1}}


\newcommand{\all}{{\mathit all}}

\newcommand{\AllLocksSym}{\LocksSym{\all}}
\newcommand{\acqVC}[1]{{\mathit Acq(#1)}}

\newcommand{\RW}[1]{\mathit{RC}(#1)}

\newcommand{\pwrVC}[1]{\mathit{PWR}(#1)}

%% Candidates

\newcommand{\pwrCand}{PWR}
\newcommand{\pwrCandGuarded}{PWR$^{cs}$}
\newcommand{\pwrCandGuardedRef}{PWR$^{rcs}$}
\newcommand{\undeadCand}{UD}
\newcommand{\undeadCandGuardedExtra}{UD$^{cs}$}
\newcommand{\undeadCandGuardedExtraRef}{UD$^{rcs}$}


\newcommand{\OK}{\mathit{ok}}
\newcommand{\FAIL}{\mathit{fail}}

%--------------------------------------------------------
%--------------------------------------------------------
\section{Implementation}
\label{sec:implementation}


We show how we implemented the deadlock-prediction methods
from \cref{s:sound,s:complete} as an algorithm.
Our algorithm takes as input a trace and yields a set of deadlock patterns, in two phases:
(1)~computation of the partial ordering of events and lock dependencies (\cref{sec:vcAndLd}),
followed by
(2)~detection of cyclic lock-dependency chains representing deadlock patterns (\cref{sec:impl:verify}).
The partial ordering is implemented using vector clocks.
Our implementation of Phase~(2)
closely follows the original UNDEAD depth-first algorithm
to compute cyclic lock-dependency chains,
with additional checks for Conditions~\cond{DP-P/-Block}
(where P refers to either TRW or PWR).

In our algorithms, we often write ``$\dontCare$'' in pattern matching to denote arbitrary values.


\subsection{Computation of TRW %%Vector Clocks
            and Abstract Lock Dependencies}
\label{sec:vcAndLd}


%% Specific variant.
%% \begin{algorithm*}
%%   \caption{UNDEAD enhanced with PWR vector clocks for computing lock dependencies}
%%   \label{alg:lock-deps}
%%
%% \begin{tabular}{cc}
%%
%% \begin{minipage}{.6\textwidth}
%%
%%   \bda{ll}
%% %% MS: main text
%% %%       V & ::= [i_1,\ldots,i_n]  \quad \mbox{Vector clock}
%% %%     \\
%% %%     V < V'  & \mbox{if}\ \forall k. V_k \leq V'_k \wedge \exists k. V_k < V'_k
%% %%     \\
%% %%     V \sqcup V' & \mbox{equals $[\maxN{i_1}{j_1},\dots,\maxN{i_n}{j_n}]$ where}
%% %%         \\ &      \mbox{$V=[i_1,\dots, i_n]$ and $V'= [j_1,\dots,j_n]$}
%% %%     \\
%%   \threadVC{t} & \mbox{Thread $t$'s vector clock}
%%   \\
%%   \lastWriteVC{x} & \mbox{Vector clock of most recent $\writeE{x}$}
%%   \\
%%   \StdLocksSym(t) & \mbox{Thread $t$'s (standard) lock set}
%%   \\
%%   \acqVC{l}&  \mbox{Vector clock of most recent $\acqE{l}$}
%%   \\
%%   \Hist{l} & \mbox{History of acquire-release pairs $(\Vacq,\Vrel)$ for lock $l$}
%%   \\
%%   \LDt & \mbox{Global set of lock dependencies}
%%   \eda
%%
%% \end{minipage}
%%
%% &
%%
%% \begin{minipage}{.4\textwidth}
%%
%%
%% \begin{algorithmic}[1]
%%   \Function{syncCS}{$V, \Locks$}
%%     \For {$l \in L$}
%%     \For {$(\Vacq, \Vrel) \in \Hist{l}$}
%%      \If {$\Vacq < V$}
%%      \State $V = V \sqcup \Vrel$
%%      \EndIf
%%      \EndFor
%%      \EndFor
%%
%%     \Return V
%%     \EndFunction
%% \end{algorithmic}
%%
%% \end{minipage}
%% \end{tabular}
%%
%% \begin{tabular}{ccc}
%%
%% \begin{minipage}{.36\textwidth}
%%
%% \begin{algorithmic}[1]
%%   \Procedure{acquire}{$t,l$}
%% \State $\LDt = \LDt \cup \{ \LD{t}{(l,\threadVC{t})}{\StdLocksSym(t)} \}$
%% \State $\StdLocksSym(t) = \StdLocksSym(t) \cup \{l\}$
%% \State $\acqVC{l} = \threadVC{t}$
%% \State $\incC{\threadVC{t}}{t}$
%% \EndProcedure
%% \end{algorithmic}
%%
%%
%% \begin{algorithmic}[1]
%%   \Procedure{release}{$t,l$}
%%   \State $\StdLocksSym(t) = \StdLocksSym(t) \setminus \{l\}$
%%   \State $\Hist{l} = \Hist{l} \cup \{(\Acq{l}, \threadVC{t})\}$
%% \State $\incC{\threadVC{t}}{t}$
%% \EndProcedure
%% \end{algorithmic}
%%
%%
%% \end{minipage}
%%
%% &
%%
%% \begin{minipage}{.33\textwidth}
%%
%% \begin{algorithmic}[1]
%%   \Procedure{write}{$t,x$}
%% \State $\lastWriteVC{x} = \threadVC{t}$
%% \State $\incC{\threadVC{t}}{t}$
%% \EndProcedure
%% \end{algorithmic}
%%
%% \begin{algorithmic}[1]
%%   \Procedure{read}{$t,x$}
%% \State $\threadVC{t} = \threadVC{t} \sqcup \lastWriteVC{x}$
%% \State $\threadVC{t} = \Call{syncCS}{\threadVC{t},\StdLocksSym(t)}$
%% \State $\incC{\threadVC{t}}{t}$
%% \EndProcedure
%% \end{algorithmic}
%%
%%
%% \end{minipage}
%%
%% &
%%
%% \begin{minipage}{.33\textwidth}
%%
%%
%%
%% \begin{algorithmic}[1]
%%   \Procedure{fork}{$t,s$}
%% \State $\threadVC{s} = \threadVC{t}$
%% \State $\incC{\threadVC{t}}{t}$
%% \EndProcedure
%% \end{algorithmic}
%%
%%
%% \begin{algorithmic}[1]
%%   \Procedure{join}{$t,s$}
%%   \State $\threadVC{t} = \threadVC{s} \sqcup \threadVC{t}$
%% \State $\threadVC{t} = \Call{syncCS}{\threadVC{t},\StdLocksSym(t)}$
%% \State $\incC{\threadVC{t}}{t}$
%% \EndProcedure
%% \end{algorithmic}
%%
%% \end{minipage}
%% \end{tabular}
%%
%% \end{algorithm*}

\begin{figure}[t]
  %%latexTrace $ addLoc evict_3
  \bda{|l|l|l||l|l|}
    \hline T_{10} & \thread{1} & \thread{2} & \mbox{Standard lock dependencies} & \mbox{Abstract lock dependencies}\\ \hline
    \eventE{1}  & \lockE{\LKA}&&&\\
    \eventE{2}  & \lockE{\LKB}&& \LD{\thread{1}}{\LKB}{\{\LKA\}} & \LDMap{\thread{1}}{\LKB}{\{\LKA\}} = [(2,V_2,\{e_1\})] \\
    \eventE{3}  & \writeE{\VA}&&&\\
    \eventE{4}  & \unlockE{\LKB}&&&\\
    \eventE{5}  & \unlockE{\LKA}&&&\\
    \eventE{6}  & &\lockE{\LKB}&&\\
    \eventE{7}  & &\readE{\VA}&&\\
    \eventE{8}  & &\lockE{\LKA}& \LD{\thread{2}}{\LKA}{\{\LKB\}} & \LDMap{\thread{2}}{\LKA}{\{\LKB\}} = [(8,V_8,\{e_7\})] \\
    \eventE{9}  & &\unlockE{\LKA} &&\\
    \eventE{10}  & &\unlockE{\LKB}&&\\
    \eventE{11}  & \lockE{\LKA}&&&\\
    \eventE{12}  & \lockE{\LKB}&& \LD{\thread{1}}{\LKB}{\{\LKA\}} & \LDMap{\thread{1}}{\LKB}{\{\LKA\}} = [\smash{\begin{array}[t]{@{}l@{}} % \smash ignores the height of the array.
      (2,V_2,\{e_1\}),
      \\
      (12,V_{12},\{e_{11}\})]
    \end{array}} \\
    \eventE{13}  & \unlockE{\LKB}&&&\\
    \eventE{14}  & \unlockE{\LKA}&&&\\
    \hline
  \eda{}

  \caption{Standard versus abstract lock dependencies.}
  \label{fig:ld-abstract}
\end{figure}

\begin{algorithm*}[t!]
  \caption{Computation of TRW vector clocks and abstract lock dependencies.}
  \label{alg:lock-deps}

  {\small
    \begin{algorithmic}[1]
      \Function{computeTRWLockDeps}{$T$}
      \label{ln:cLDs}
        \State $\forall t \colon \threadVC{t} = [\bar{0}]; \incC{\threadVC{t}}{t}$
        \Comment{Vector clock $\threadVC{t}$ of thread $t$}
        \label{ln:thvc}
        \State $\forall x \colon \lastWriteVC{x} = [\bar{0}]; \lastReadVC{x} = [\bar{0}]$
        \Comment{Vector clocks $\lastWriteVC{x},\lastReadVC{x}$ of most recent $\writeE{x},\readE{x}$}
        \label{ln:lwlr}
        \State $\forall l \colon \acqVC{l} = [\bar{0}]$
        \Comment{Vector clock $\acqVC{l}$ of most recent $\acqE{l}$}
        \label{ln:acqv}
        \State $\forall l \colon \Hist{l} = []$
        \Comment{History $\Hist{l}$ of acquire-release pairs $(\Vacq,\Vrel)$ for lock $l$}
        \label{ln:hist}
        \State $\forall t \colon \AcqHeld(t) = []$
        \Comment{Sequence $\AcqHeld(t)$ of acquires held by thread $t$}
        \label{ln:acqhd}
        \State $\LDMapSym = \emptyset$
        \Comment{Map  with keys $(t,l,ls)$,
                 list values with elements $(i,V,\{a_1,\ldots,a_n\})$}
        \label{ln:ld}
        \ForDo {$e$ in $T$} {\Call{process}{$e$}}
        \State \Return \LDt
      \EndFunction
      \algstore{cld}
    \end{algorithmic}

    \begin{minipage}[t]{.52\textwidth}
      \begin{algorithmic}[1]
        \algrestore{cld}
        \Procedure{process}{$(\alpha,t,acq(l))$}
          \label{ln:procAcq}
          \If {$\AcqHeld(t) \not = []$}
          \label{ln:acq-non-empty}
          \State $ls = \{ l' \mid \acqE{l'} \in \AcqHeld(t) \}$
          \label{ln:ld-ls}
          \State $\LDMap{t}{l}{ls}.pushBack(\alpha,\threadVC{t},\AcqHeld(t))$
          %% \State $\LDt = \LDt \cup \{ \LD{t}{l,\threadVC{t}}{\AcqHeld(t)} \}$
            \label{ln:ld-add}
          \EndIf
          \State $\AcqHeld(t) = \AcqHeld(t) \cup \{ (\alpha,t,acq(l)) \}$
          \label{ln:acq-push}
          %% The above represents the ``pre'' vector clock = request
          \State $\threadVC{t} = \Call{syncCS}{\threadVC{t},\AcqHeld(t)}$
          \label{ln:acq-sync}
          \State $\acqVC{l} = \threadVC{t}$
          \label{ln:acq-store}
          \State $\incC{\threadVC{t}}{t}$
          \label{ln:acq-vc-inc}
        \EndProcedure
        \algstore{eacq}
      \end{algorithmic}


      \begin{algorithmic}[1]
        \algrestore{eacq}
        \Procedure{process}{$(\dontCare,t,rel(l))$}
          \label{ln:procRel}
          \State $\AcqHeld(t) = \{\dontCare,\dontCare,acq(l')) \in \AcqHeld(t) \mid l' \not= l \}$
          \label{ln:acq-pop}
          \State $\Hist{l} = \Hist{l} \cup \{(\Acq{l}, \threadVC{t})\}$
          \label{ln:hist-add}
          \State $\incC{\threadVC{t}}{t}$
          \label{ln:rel-vc-inc}
        \EndProcedure
        \algstore{erel}
      \end{algorithmic}

      %% MS: omit for brevity
      %% \begin{algorithmic}[1]
      %%   \Procedure{\mbox{$e$}@fork}{$t,s$}
      %% \State $\threadVC{s} = \threadVC{t}$
      %% \State $\incC{\threadVC{t}}{t}$
      %% \EndProcedure
      %% \end{algorithmic}

      %% MS: omit for brevity
      %% \begin{algorithmic}[1]
      %%   \Procedure{\mbox{$e$}@join}{$t,s$}
      %%   \State $\threadVC{t} = \threadVC{s} \sqcup \threadVC{t}$
      %%   \If {RO}
      %%   \State $\threadVC{t} = \Call{syncCS}{\threadVC{t},\StdLocksSym(t)}$
      %%   \EndIf
      %% \State $\incC{\threadVC{t}}{t}$
      %% \EndProcedure
      %% \end{algorithmic}
    \end{minipage}%
    %
    \hfill%
    %
    \begin{minipage}[t]{.46\textwidth}
      \begin{algorithmic}[1]
        \algrestore{erel}
        \Procedure{process}{$(\dontCare,t,wr(x))$}
          %% TWR-only
          \label{ln:procWr}
          \State $\threadVC{t} = \threadVC{t} \sqcup \lastWriteVC{x}$
          \label{ln:ww-sync}
          \State $\threadVC{t} = \threadVC{t} \sqcup \lastReadVC{x}$
          \label{ln:rw-sync}
          \State $\threadVC{t} = \Call{syncCS}{\threadVC{t},\AcqHeld(t)}$
          \label{ln:csw-sync}
          %% TWR + PWR
          \State $\lastWriteVC{x} = \threadVC{t}$
          \label{ln:lw-store}
          \State $\incC{\threadVC{t}}{t}$
          \label{ln:wr-vc-inc}
        \EndProcedure
        \algstore{ewr}
      \end{algorithmic}


      \begin{algorithmic}[1]
        \algrestore{ewr}
        \Procedure{process}{$(\dontCare,t,rd(x))$}
          \label{ln:procRd}
          \State $\threadVC{t} = \threadVC{t} \sqcup \lastWriteVC{x}$
          \label{ln:wr-sync}
          \State $\threadVC{t} = \Call{syncCS}{\threadVC{t},\AcqHeld(t)}$
          \label{ln:csr-sync}
            \State $\lastReadVC{x} = \threadVC{t}$ %% TWR-only
          \label{ln:lr-store}
          \State $\incC{\threadVC{t}}{t}$
          \label{ln:rd-vc-inc}
        \EndProcedure
        \algstore{erd}
      \end{algorithmic}

      \begin{algorithmic}[1]
        \algrestore{erd}
        \Function{syncCS}{$V, A$}
          \label{ln:syncCS}
          \For {$\acqE{l} \in A , (\Vacq, \Vrel) \in \Hist{l}$}
          \label{ln:sync-hist-entry}
            \IfThen {$\Vacq < V$} {$V = V \sqcup \Vrel$}
            \label{ln:ro-sync}
          \EndFor
        \State \Return V
        \EndFunction
        % \algstore{sync}
      \end{algorithmic}
    \end{minipage}
  }
\end{algorithm*}


%--------------------------------------------------------

Phase~(1) computes vector clocks and lock dependencies (\cref{alg:lock-deps}).
Although our implementation supports various partial orders, our presentation details TRW (\cref{d:trw}).
Function \textsc{computeTRWLockDeps} (\cref{ln:cLDs}) takes a trace $T$, returning a set of \emph{abstract} lock dependencies
represented by the map $\LDMapSym$.

To understand the difference between standard and abstract lock dependencies,
consider trace~$T_{10}$ in \cref{fig:ld-abstract}.
For brevity, we omit explicit request events,
assuming that acquires implicitly directly preceded by a request.
Events $e_2$ and $e_{12}$ both acquire lock $\LKB$ while holding lock $\LKA$.
The resulting (standard) lock dependency
$\LD{\thread{1}}{\LKA}{\{\LKB\}}$ is sufficient to check for Conditions~\cond{DP-Cycle/-Guard}.
For the additional Conditions~\cond{DP-TRW/-Block},
we require some extra information specific to~$e_2$ and~$e_{12}$.
For Condition~\cond{DP-TRW} we need their vector clocks,
and
for Condition~\cond{DP-Block} we need their trace positions
and, for each lock held, the corresponding acquire event.
We collect this extra information as triples:
for~$e_2$ we have $(2,V_2,\{e_1\})$, and for~$e_{12}$ we have $(12,V_{12},\{e_{11}\})$.
Each triple represents a \emph{concrete} lock dependency.
Concrete lock dependencies that share the same thread, lock and locks held
are stored in an \emph{abstract} lock dependency.
An abstract lock dependency is represented as a (list) value in a map~$\LDMapSym$
with \emph{standard} lock dependencies as keys.
Initially, all lists are empty.
When processing $e_2$ we add $(2,V_{2},\{e_{1}\})$
to the empty list $\LDMap{\thread{1}}{\LKB}{\{\LKA\}}$.
After processing~$e_{12}$, we insert the associated triple, resulting in
$[(2,V_2,\{e_1\}), (12,V_{12},\{e_{11}\})]$.
New triples are added to the back of the list, to maintain the processing
order among elements.

Before detailing \cref{alg:lock-deps},
we repeat some standard definitions for vector clocks.

\begin{definition}[Vector Clocks]
  A \emph{vector clock} $V$ is a list of time stamps of the form
  % \[
  $
    % V ::=
    [i_1,\ldots,i_n]
  $.
  % \]
  We assume vector clocks are of a fixed size $n$.
  \emph{Time stamps} are natural numbers, and a time stamp at position $j$ corresponds to the thread with identifier $\thread{j}$.

  We define
  % \[
  $
    [i_1,\ldots, i_n] \sqcup [j_1,\ldots,j_n] = [\maxN{i_1}{j_1},\ldots,\maxN{i_n}{j_n}]
  $
  % \]
  to \emph{synchronize} two vector clocks, taking pointwise maximal time stamps.
  %
  We write $\incC{V}{j}$ to denote incrementing the vector clock $V$ at position $j$ by one,
  and $\accVC{V}{j}$ to retrieve the time stamp at position $j$.

  We write $V < V'$ if $\forall k \colon \accVC{V}{k} \leq \accVC{V'}{k} \wedge \exists k \colon \accVC{V}{k} < \accVC{V'}{k}$,
  and $V \VCConc V'$ if $V \not< V'$ and $V' \not< V$.
\end{definition}

\Cref{alg:lock-deps} processes events in trace order.
For every event $e$, we call the procedure \textsc{process}, defined differently for each event operation (\cref{ln:procAcq,ln:procRel,ln:procWr,ln:procRd}).
For brevity, we omit the standard treatment of fork/join events.

For the computation of TRW vector clocks, we maintain several state variables that
are either indexed by some thread, lock, or shared variable.
For every thread $t$, we have a vector clock~$\threadVC{t}$.
Vector clocks $\lastWriteVC{x}$ and $\lastReadVC{x}$ represent the last read and write
event on variable~$x$.
The vector clock $\acqVC{l}$ records the last acquire event on lock~$l$.
Initially, all time stamps are set to zero (\cref{ln:thvc,ln:lwlr,ln:acqv}; $\bar{0}$ denotes a sequence of zeros), and $\threadVC{t}$ is set to 1 at position $t$ (\cref{ln:thvc}).

Unlike well-known vector-clock algorithms such as
FastTrack~\cite{flanagan2010fasttrack} and
SHB~\cite{Mathur:2018:HFR:3288538:3276515},
we do not order critical sections by source trace order.
Rather, TRW only orders critical sections that contain conflicting writes and reads.
To check if critical sections are conflicting, we maintain
a critical-section history $\Hist{l}$ to track the vector clocks $\Vacq$ of acquires of lock~$l$ and $\Vrel$ of their matching releases.
In our actual implementation we use thread-local histories for efficiency reasons.
We omit details, as our use of thread-local histories
is along the lines of existing vector-clock algorithms
for WCP~\cite{DBLP:journals/corr/KiniM017} and PWR~\cite{conf/mplr/SulzmannS20}.

Additionally,
we need to know the acquire events connected to locks held.
Hence, each thread~$t$ maintains a sequence $\AcqHeld(t)$ of acquires held (\cref{ln:acqhd}).
As discussed above, the map~$\LDMapSym$~(\cref{ln:ld})
holds the set of abstract lock dependencies.

The \textsc{process} procedures update the above state variables differently for each operation, but all increment the time stamp of thread $t$ in $\threadVC{t}$ as a final step (\cref{ln:acq-vc-inc,ln:rel-vc-inc,ln:wr-vc-inc,ln:rd-vc-inc}).
We use pattern matching to distinguish operations.

For acquire event $(\alpha,t,acq(l))$ (\cref{ln:procAcq}), a new entry $(\alpha,\threadVC{t},\AcqHeld(t))$
is added to the abstract lock dependency $\LDMap{t}{l}{ls}$ (\cref{ln:ld-add}),
assuming that the sequence $\AcqHeld(t)$ of acquires held is not empty (\cref{ln:acq-non-empty}).
From $\AcqHeld(t)$ we derive the lockset~$ls$ (\cref{ln:ld-ls}).
Event id~$\alpha$ refers to the trace position of the acquire event.
Each entry uses the vector clock \threadVC{t} before synchronization with other critical sections.
Effectively, this is the vector clock of the request event that precedes the acquire,
allowing us to avoid explicitly handling requests.
%% MS: we want to deal with ill-nested traces as well
$\AcqHeld(t)$ is extended with the current acquire~(\cref{ln:acq-push}).
Function call $\Call{syncCS}{\threadVC{t},\AcqHeld(t)}$ (\cref{ln:acq-sync}; discussed shortly)
enforces Condition~\cond{TRW-Rel}.
% \ms{likely there's an earlier example we can refer to?}
Finally, we record the vector clock of the acquire event (\cref{ln:acq-store}).

For release events $(\dontCare,t,rel(l))$ (\cref{ln:procRel}),
we remove the matching acquire from $\AcqHeld(t)$~(\cref{ln:acq-pop})
and add a new acquire-release pair to $\Hist{l}$ (\cref{ln:hist-add}).

For write events $(\dontCare,t,wr(x))$ (\cref{ln:procWr}), we synchronize $\threadVC{t}$ with the vector clock of the last write and read on $x$ (\cref{ln:ww-sync,ln:rw-sync}).
For read events $(\dontCare,t,rd(x))$ (\cref{ln:procRd}), we also synchronize $\threadVC{t}$ with the vector clock of the last write on $x$ (\cref{ln:wr-sync}).
In both cases, this enforces Condition~\cond{TRW-Conf}.
Conflicting memory operations with a common lock enforce Condition~\cond{TRW-Rel} using $\Call{syncCS}{\threadVC{t},\AcqHeld(t)}$ (\cref{ln:csw-sync,ln:csr-sync}).

Function \textsc{syncCS} (\cref{ln:syncCS}) synchronizes critical sections on the same lock.
The function cycles through every $\acqE{l} \in A$.
For each $(\Vacq,\Vrel)$ in the history $\Hist{l}$ (\cref{ln:sync-hist-entry}),
we check whether $\Vacq < V$~(\cref{ln:ro-sync}), where $V$ is the vector clock of some acquire/read/write.
If so, we synchronize $V$ and $\Vrel$, thus enforcing Conditions~\cond{TRW-Conf/-Rel}.

\paragraph{PWR vector clocks.}

To obtain PWR vector clocks, we simply drop \cref{ln:ww-sync,ln:rw-sync,ln:csw-sync,ln:lr-store} in \cref{alg:lock-deps};
all other parts remain the same.

\algdef{SE}[DOWHILE]{Do}{DoWhile}{\algorithmicdo}[1]{\algorithmicwhile\ #1}%

\begin{algorithm*}[t]
  \caption{Checking Conditions \cond{DP-Cycle/-Guard/TRW/-Block}.}
  \label{alg:dp-trw-block}

  {\small

    \begin{algorithmic}[1]
      \Function{undeadTRWBlock}{$T$}
      \label{ln:undeadTRWBlock}
      \State $\LDMapSym = \Call{computeTRWLockDeps}{T}$
      \label{ln:call-computeTRWLockDeps}
      \State $\DDs = \{ (\LDMap{\thread{1}}{l_1}{ls_1}, \ldots, \LDMap{\thread{n}}{l_n}{ls_n}) \mid \forall i \ne j \colon l_i \in ls_{(i \mod n) + 1} \wedge ls_i \cap ls_j = \emptyset \}$
      \label{ln:cyclic-chain}
      \State $\AAs = \emptyset$
      \For{$(\Ald_1,\ldots,\Ald_n) \in \DDs$}
      \label{ln:dp-trw-start}
      \If{$\Call{checkTRW}{\Ald_1,\ldots,\Ald_n} = (\OK,(\AldE_1,\ldots,\AldE_n))$}
      \label{ln:call-checkTRW}
      \State $\AAs = \AAs \cup \{ \{ \AldE_1,\ldots,\AldE_n \} \}$
      \EndIf
      \EndFor
      \label{ln:dp-trw-end}
      \State $\BBs = \emptyset$
      \For{$A \in \AAs$}
      \label{ln:dp-block-start}
      \If{$\neg \exists B \in \AAs. \Call{lessThan}{B,A}$}
      \label{ln:dp-block-check}
      \State $\BBs = \BBs \cup \{ A \}$
      \EndIf
      \EndFor
      \label{ln:dp-block-end}
      \State \Return $\BBs$
      \EndFunction
      \algstore{undeadTRWBlock}
    \end{algorithmic}

    \begin{minipage}[t]{.45\textwidth}
      \begin{algorithmic}[1]
        \algrestore{undeadTRWBlock}
        \Function{checkTRW}{$\Ald_1,\ldots,\Ald_n$}
        \label{ln:check-dp-trw}
        \State $k_i = 0$ for $i=1,\ldots,n$
        \label{ln:index-init}
        \While{$\bigwedge_{i=1}^n k_i < \Ald_i.size$}
        \label{ln:iterate-through-all}
        \State $(\dontCare,V_i,\dontCare) = \Ald_i[k_i]$ for $i=1,\ldots,n$
        \label{ln:index-access-vc}
        \If{$V_i \VCConc V_j$ for $i \not= j$}
        \label{ln:dp-trw-okay}
        \State \Return $(\OK,(\Ald_1[k_1],\ldots,\Ald_n[k_n]))$
        \EndIf
        \State $V = V_1 \sqcup \ldots \sqcup V_n$
        \label{ln:sync-candidates}
        \State $k_i = \Call{next}{V,\Ald_i,k_i}$ for $i=1,\ldots,n$
        \label{ln:next-candidates}
        \EndWhile
        \EndFunction
        \State \Return $(\FAIL,\dontCare)$
        \algstore{checkTRW}
      \end{algorithmic}
    \end{minipage}%
    \hfill%
    \begin{minipage}[t]{.50\textwidth}
      \vspace{-5.5em}
      \begin{algorithmic}[1]
        \algrestore{checkTRW}
        \Function{next}{$V,\Ald,i$}
        \Do
        \State $(\dontCare,V',\dontCare) = \Ald[i]$
        \If{$\neg (V' < V)$}
        \label{ln:next-entry}
        \State \Return $i$
        \EndIf
        \State $i = i + 1$
        \DoWhile{$i < \Ald.size$}
        \State \Return $i$
        \EndFunction
        \algstore{next}
      \end{algorithmic}

      \begin{algorithmic}[1]
        \algrestore{next}
        \Function{lessThan}{$A,B$}
        \For{$(i,\dontCare,as) \in A$}
        \If{$\neg \exists (j,\dontCare,bs) \in B \colon i < j \wedge \Call{cycLk}{as} \in bs$}
        \State \Return false
        \EndIf
        \EndFor
        \State \Return true
        \EndFunction
      \end{algorithmic}
    \end{minipage}
  }
\end{algorithm*}




%--------------------------------------------------------
\subsection{Computing and Verifying Deadlock Patterns}
\label{sec:impl:verify}

Function \textsc{undeadTRWBlock} (\cref{ln:undeadTRWBlock})
in \cref{alg:dp-trw-block}
is the main entry point of our algoritm.
We first call \textsc{computeTRWLockDeps} (in \cref{alg:lock-deps}) to compute
the set of abstract lock dependencies~(\cref{ln:call-computeTRWLockDeps} discussed in \cref{sec:vcAndLd}).

Next, we enumerate in $\DDs$ all cyclic chains of abstract lock dependencies (referred to in~\cite{conf/pldi/TuncMPV23} as ``abstract lock dependencies'') that
satisfy Conditions~\cond{DP-Cycle/-Guard} (\cref{ln:cyclic-chain}).
We refer to~\cite{conf/ase/ZhouSLCL17} for details on the computation of~$\DDs$.
Elements in $\DDs$ are sequences of the form $(\Ald_1,\ldots,\Ald_n)$,
where each $\Ald_i$ is an abstract lock dependency, i.e., a list of triples $\AldE_j = (j,V,\{a_1,\ldots,a_k\})$.
For every such~$(\Ald_1,\ldots,\Ald_n)$, we then look for a concrete
instance~$(\AldE_1,\ldots,\AldE_n)$, where every $\AldE_i \in \Ald_i$,
that satisfies Conditions~\cond{DP-TRW/-Block}.

Function \textsc{checkTRW} looks for an instance
that satisfies Condition~\cond{DP-TRW}~(\cref{ln:check-dp-trw}).
We systematically enumerate instances by starting
with the first element in $\Ald_i$~(\cref{ln:index-init}).
We use array access notation to obtain the $k_i$th element
combined with pattern matching to retrieve the vector clock~$V_i$
of that element (\cref{ln:index-access-vc}).
If the vector clocks are pairwise incomparable,
we have found an instance that satisfies~Condition~\cond{DP-TRW}~(\cref{ln:dp-trw-okay}).
Otherwise, we need to select a new instance.

To avoid naive enumeration of all instances, we exploit the property that for
${\Ald = [ \AldE_1,\ldots,\AldE_k]}$ we find that
$V_j < V_{j+1}$ where $\AldE_j = (\dontCare,V_j,\dontCare)$ for every $\AldE_j \in \Ald$.
This follows from the fact that the elements of $\Ald$ are stored in
the order in which they are generated
(cf.\ \cref{ln:ld-add} in \cref{alg:lock-deps}).
Hence, if we find that $V_i < V_j$,
we can immediately conclude that, for any $V'$
where $(\dontCare,V',\dontCare) = \Ald_i[k]$ for~$k<k_i$,
we have $V' < V_j$.
It follows that it suffices to check for ``later'' candidates $V'$
for which~$V' < V_j$ does \emph{not} hold.
This is achieved by synchronizing all candidates (\cref{ln:sync-candidates})
and calling function \textsc{next}~(\cref{ln:next-candidates}).

The resulting instances, that satisfy Condition~\cond{DP-TRW}, are stored in $\AAs$.
What remains is to select only those instances that satisfy Condition~\cond{DP-Block}.
Function~\textsc{lessThan} checks whether two instances are ordered
following the description in \cref{def:ltDP}.
For convenience, we omit the helper function~\textsc{cycLk}
that retrieves the acquire held that is part
of the cyclic dependency, as stated in Condition~\cond{DP-Cycle}.
The resulting instances stored in $\BBs$ represent
deadlocks patterns satisfying Conditions~\cond{DP-Cycle/-Guard/TRW/-Block}.

For example, for trace~$T_{10}$ in \cref{fig:ld-abstract},
we find
%
\[
\DDs = \{ ([(2,V_2,\{e_1\}), (12,V_{12},\{e_{11}\})],[(8,V_8,\{e_7\})]) \}
\]
%
This leads to the call
$
\textsc{checkTRW}(([(2,V_2,\{e_1\}), (12,V_{12},\{e_{11}\})],[(8,V_8,\{e_7\})]))
$.
The first instance to be checked
is the pair of elements $(2,V_2,\{e_1\})$ and $(8,V_8,\{e_7\})$, but it is rejected because $V_2 < V_8$.
The next pair of elements considered is
$(12,V_{12},\{e_{11}\})$ and $(8,V_8,\{e_7\})$.
This pair is successful: we find that $V_{12} \VCConc V_8$.
The resulting instance is $\{ (12,V_{12},\{e_{11}\}), (8,V_8,\{e_7\}) \}$,
which is the only deadlock pattern reported for this example.


%--------------------------------------------------------
\subsection{Time Complexity}
\label{sec:complexity}

\newcommand{\TRACELEN}{\mathcal{N}}
\newcommand{\THREADNO}{\mathcal{T}}
\newcommand{\CSNO}{\mathcal{C}}
\newcommand{\CYCLES}{\mathit{Cyc}}

We first consider the time complexity of \cref{alg:lock-deps}.
Let $\THREADNO$ be the number of threads and
$\CSNO$ be the number of critical sections (acquire-release pairs).
Then, each call to function~\textsc{syncCS} takes time $O(\THREADNO \cdot \CSNO)$,
because the number of acquires held is treated as a constant,
there are $O(\CSNO)$ entries to consider, and
for each entry the vector clock operations involved take time~$O(\THREADNO)$.
Hence, function~\textsc{computeTRWLockDeps} takes time~$O(\TRACELEN \cdot \THREADNO \cdot \CSNO)$ where $\TRACELEN$ is the length of the trace,
assuming that access and manipulation of state variables ($\threadVC{t},...$)
takes time $O(1)$ and ignoring initialization of state variables.
Instead of a global history~$\Hist{l}$ of size $O(\CSNO)$,
our actual implementation follows WCP~\cite{DBLP:journals/corr/KiniM017}
and PWR~\cite{conf/mplr/SulzmannS20} in using thread-local histories.
This way, for realistic examples
the number of entries in thread-local histories can be treated like a constant.
We therefore argue that the time complexity of \textsc{syncCS} can be simplified to $O(\THREADNO)$, and hence that \cref{alg:lock-deps} takes time $O(\TRACELEN \cdot \THREADNO)$.

What remains is to consider the time complexity of~\cref{alg:dp-trw-block}.
As shown by \Tunc et al.~\cite{conf/pldi/TuncMPV23},
the number of cyclic chains of abstract lock dependencies
can be exponential in terms of the number of threads and acquires.
We follow~\cite{conf/pldi/TuncMPV23} in writing $O(\CYCLES)$ to represent this number
as well as the time complexity of computing set $\DDs$ (\cref{ln:cyclic-chain}).
Next, we consider the complexity of~\textsc{checkTRW} (\cref{ln:call-checkTRW}).
We treat the length of the sequence $(\Ald_1,\ldots,\Ald_n)$ as a constant and
always strictly move forward through the list of candidates in $\Ald_i$.
In each step, vector clock operations take time $O(\THREADNO)$.
Hence, function~\textsc{checkTRW} takes time~$O(\TRACELEN \cdot \THREADNO)$.
This shows that \cref{ln:dp-trw-start,ln:call-checkTRW,ln:dp-trw-end}
take time $O(\CYCLES \cdot \TRACELEN \cdot \THREADNO)$.
Function~\textsc{lessThan} takes time $O(\CYCLES)$.
Thus, we find that \cref{alg:dp-trw-block} takes time
$O(\CYCLES + \CYCLES \cdot \TRACELEN \cdot \THREADNO + \CYCLES^2)$.

Overall, our method takes time $O(\TRACELEN \cdot \THREADNO + \CYCLES + \CYCLES \cdot \TRACELEN \cdot \THREADNO + \CYCLES^2)$.
In practice, $O(\CYCLES)$ is small and $O(\THREADNO)$ can be interpreted as a constant, such that the complexity reduces to~$O(\TRACELEN)$.
This is confirmed by our experiments (\cref{sec:experiments}).


%%% Local Variables:
%%% mode: latex
%%% TeX-master: "main"
%%% End:
