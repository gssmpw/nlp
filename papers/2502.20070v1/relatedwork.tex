\section{Related Work}
\label{sec:related}

\paragraph{Lockset-based dynamic resource-deadlock analysis.}

The idea of identifying deadlocks via circularity in the lock-order dependency
of threads dates back to the work by %
Havelund~\cite{10.5555/645880.672085} and
Harrow~\cite{DBLP:conf/spin/Harrow00}.
In subsequent work,
Bensalem and Havelund~\cite{10.1007/11678779_15} introduce lock-order graphs
that capture the lock-order dependency for threads,
where deadlock analysis reduces to checking for cycles.
Extensions of lock-order graphs to avoid false positives if a circular lock-order
dependency takes place within a single thread or is protected by a common lock
are discussed by
Agarwal et al.~\cite{5571951}.

Instead of lock-order graphs,
Joshi et al.~\cite{10.1145/1542476.1542489}
introduce lock dependencies on a per-thread basis.
%% MS: shorter
%% of the form $\LD{t}{l}{ls}$ of the form
%% we also consider,
%% where $t$ is the thread of the acquire operation for lock $l$
%% and $ls$ is the lock set of thread $t$ at this point.
%% Lock dependencies form a cycle if all thread ids are distinct,
%% all lock sets are disjoint, and lock dependencies can be ordered such that
%% each acquired lock $l$ is in the lock set of the following lock dependency.
The advantage compared to lock-order graphs is that common locks or singular threads
can be easily detected.
Several works~\cite{Samak:2014:TDD:2692916.2555262,conf/ase/ZhouSLCL17,6718069}
improve on this idea, e.g., by using an efficient representation
for lock dependencies and/or ignoring impossible cyclic chains
due to fork-join dependencies.
Our experiments show that considering fork-join dependencies
only avoids some false positives, but a significant number remains.

\paragraph{Sound dynamic resource-deadlock analysis.}

Lockset-based analysis methods are prone to false positives.
One way to eliminate false positives is to re-run the trace
(and/or the program)
to verify that a deadlock exists (e.g.~\cite{10.1007/978-3-319-23404-5_13,10.1145/1542476.1542489,Samak:2014:TDD:2692916.2555262}).


%% PickLock by Serrentino~\cite{10.1007/978-3-319-23404-5_13}.
%%
%% Does some rescheduling derived from
%% "PENELOPE: weaving threads to expose atomicity violations."
%%
%% Apparently includes happens-before information to eliminate false positives.
%% Then, the formalization does not cover fork/join dependencies, nor reads/writes.
%% Seems to use the standard lockset construction.


Kalhauge and Palsberg~\cite{kalhauge2018sound} rely on an SMT-solver for exhaustive trace
exploration to eliminate false positives among deadlock patterns, but
the use of SMT-solving may severely impact performance~\cite{conf/pldi/TuncMPV23}.
Also, false positives may arise if
a request is guarded by a lock in another thread (cf., e.g., the example
in \cref{fig:guarded}, adapted from \Tunc et al.~\cite{conf/pldi/TuncMPV23}).
Our soundness result (\cref{thm:TRWSound}) explicitly excludes such traces.

%% \ms{just focus on performance issues}
%% \texttt{Dirk} is claimed to be sound.
%% The work by~\citet{conf/pldi/TuncMPV23} disputes the soundness claim
%% and gives a program for which \texttt{Dirk} yields a false positive.
%% The trace resulting from this program effectively corresponds to ... %% \cref{fig:ex2}.
%% The formalization on which \texttt{Dirk} is based
%% employs the standard lock set construction and does not consider guard locks.
%% \ms{should somewhere mention guard locks, maybe in conclusion, maybe mention the Mathur ``guard'' lock example, excluded by our theory, doesn't seem to arise in practice ...}
%% Hence, the false positive reported for \texttt{Dirk} is not due to an
%% implementation error but because the underlying theory does not cover
%% all possible lock acquisition scenarios.



\Cref{sec:experiments} contains an extensive discussion of the relation
to \SPDOffline~\cite{conf/pldi/TuncMPV23}.

SeqCheck~\cite{conf/fse/CaiYWQP21}
is similar to \SPDOffline, also employing a closure construction
that relies on a partial order to eliminate infeasible deadlock patterns.
The SeqCheck partial order does not impose last-write dependencies
but a weaker form of observation order,
so further checks are required during the closure construction.
According to \Tunc et al.~\cite{conf/pldi/TuncMPV23},
SeqCheck infers almost the same deadlocks as \SPDOffline,
but the running time is significantly higher.
We were not able to include SeqCheck in our own measurements, because our requests to access the artifact were not answered.
%% MS: I wish we could say this
%% A full reimplementation turned out to be difficult
%% as the technical description~\cite{conf/fse/CaiYWQP21}
%% appears to be very dense and details are missing.


Ang and Mathur~\cite{DBLP:conf/cav/AngM24} discuss a novel approach to predictive
monitoring and its application to race and deadlock
detection, based on Mazurkiewicz's trace equivalence rather than
classical partial-order techniques.
The idea is to match critical
patterns like conflicting events or deadlock patterns against a trace
while assuming commutativity of all events except those that are
marked dependent (like those in the same thread, operating on the same
lock, etcetera).
As standard trace equivalence is unsuitable for predicting
deadlocks, they introduce two refinements:
strong trace prefixes and
strong reads-from prefixes.
With these refinements, they obtain
algorithms for detecting sync-preserving data races and
deadlocks.
Their experimental results for deadlocks match those of
previous work~\cite{conf/pldi/TuncMPV23,10.1145/3434317}, but with
substantial slowdown~\cite[Table~3]{DBLP:conf/cav/AngM24}.


\paragraph{Partial-order methods for dynamic data race prediction.}

Starting with Lamport's Happens-Before relation~\cite{lamport1978time},
the literature offers a wide range of partial orders~\cite{Smaragdakis:2012:SPR:2103621.2103702,conf/pldi/KiniMV17,Mathur:2018:HFR:3288538:3276515,10.1145/3360605}.
As discussed in detail in \cref{sec:overview},
none of these partial orders is suitable for sound deadlock prediction,
motivating the introduction of the new TRW partial order.
Like in the case of data-race prediction, establishing soundness is a non-trivial task.
Our soundness proof shares  some similarities with the WCP soundness proof
in that we argue that either (a)~requests in a deadlock pattern can be placed
next to each other, or (b)~there must be a blocking cycle.
The details differ and are specific to TRW and the deadlock-prediction setting.

% \ms{maybe say something about PWR here}

%% MS: omit, don't know much about this work
%% \citet{7423814}
%% More on capture and replay.
%% Use the should happen-before relations to enforce a certain schedule.


%% MS: omit
%%
%%
%%
%%
%% Airlock by Cai, Meng and Palsberg~\cite{10.1145/3377811.3380367}.
%% Uses lock graphs. Online cycle search.
%% Maintains ``small predictive reachability graph''.
%% \ms{note} seems to be a data race in their algo.
%%
%%
%% \ms{note}
%% \cite{kalhauge2018sound} uses ``request'' events (like we do in PPDP'18 where it's called pre events).
%% Don't think this adds much to their work.
%% One benefit is that they explicitly reason about a blocked execution paths (threads end up with
%% a request event). It seems that's what they advertise in their paper.
%%
%% Another benefit (not mentioned) is that we can uncover additional lock dependencies.
%% Consider the following example.
%%
%% \begin{verbatim}
%%     T1              T2
%%
%%    acq(x)
%%    pre_acq(y)
%%                    acq(y)
%%                    pre_acq(x)
%% \end{verbatim}
%%
%% Other than that there seems not much point for ``request'' events.


%%% Local Variables:
%%% mode: latex
%%% TeX-master: "main"
%%% End: