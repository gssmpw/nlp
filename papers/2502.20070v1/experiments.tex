%--------------------------------------------------------
%--------------------------------------------------------
\section{Experiments}
\label{sec:experiments}

\newcommand{\UNDEAD}{\mbox{UNDEAD}\xspace}

%% eviction variants
\newcommand{\UDPWRLimitH}{\mbox{UD-PWR2}\xspace}  %% limit 100

\newcommand{\UDPWRRandom}{\mbox{UD-PWR3}\xspace}  %% limit 5, evict ``random'', latest

\newcommand{\UDPWRRandomLimitH}{\mbox{UD-PWR4}\xspace}  %% limit 100, evict ``random'', latest

\newcommand{\Benchmark}{\mbox{\bf Benchmark}}
\newcommand{\TT}{\mbox{${\mathcal T}$}}
\newcommand{\EE}{\mbox{${\mathcal E}$}}
\newcommand{\MM}{\mbox{${\mathcal M}$}}
\newcommand{\LL}{\mbox{${\mathcal L}$}}
\newcommand{\Sum}{\mbox{$\sum$}}

\newcommand{\Cycles}{\mbox{Dlk}}      %% Deadlock = Dlk = cycle
\newcommand{\Dependencies}{\mbox{Deps}}
\newcommand{\DepsGuarded}{\mbox{Grds}}
\newcommand{\DepsExtras}{\mbox{E}}
\newcommand{\Time}{\mbox{Time}}
\newcommand{\PhaseOne}{\mbox{P1}}
\newcommand{\PhaseTwo}{\mbox{P2}}
\newcommand{\Races}{\mbox{Races}}
\newcommand{\Guards}{\mbox{Guards}}

\newcommand{\bm}[1]{\texttt{#1}}


%% candidates

\newcommand{\udsymbol}{UD}
\newcommand{\UD}{\mbox{\udsymbol}\xspace}
\newcommand{\UDFJ}{\mbox{\udsymbol\textsubscript{FJ}}\xspace}
\newcommand{\UDFJWRD}{\mbox{\udsymbol\textsubscript{LW}}\xspace}
\newcommand{\UDPWR}{\mbox{\udsymbol\textsubscript{PWR}}\xspace}
\newcommand{\UDTRW}{\mbox{\udsymbol\textsubscript{TRW}}\xspace}
\newcommand{\UDTRWEvict}{\mbox{\udsymbol\textsubscript{TRW$\_$R}}\xspace}
\newcommand{\UDPWRSyncP}{\mbox{\udsymbol\textsubscript{PWR}\textsubscript{+SPD}}\xspace}

\newcommand{\UDTRWGuardCheck}{\mbox{\udsymbol\textsubscript{TRW$\_${GC}}}\xspace}

\newcommand{\UDFJWRDVariant}{\mbox{\udsymbol\textsubscript{LW}$^*$}\xspace}

%% playing

\newcommand{\UDPWROpt}{\mbox{UPWROpt}\xspace}
\newcommand{\UDPWRTO}{\mbox{UNDEAD\textsubscript{TO}}\xspace}
\newcommand{\UDSyncP}{\mbox{SPDOffline}\xspace} %% uses FJWR
\newcommand{\UDPWRDropSyncP}{\mbox{SPD\textsubscript{PWR}\textsuperscript{Drop}}\xspace}
\newcommand{\UDPWRTOHACK}{\mbox{UNDEAD\textsubscript{TOH}}\xspace}

%% The overview uses the term ``MHB'' so we stick to the name
%% (but ``LW'' might be more appropriate).
%%
%% Fun fact.
%% SPD just uses PWR-PO+PWR-LW but any partial order could be used that is
%% weaker than the CMHB partial order.
%% If we use for example PWR the closure
\newcommand{\SPDVectorClocks}{LW\xspace}

We evaluated our approach in an offline setting using a large dataset
of pre-recorded program traces from prior work.
These traces include fork and join events, whose straightforward treatment we do not discuss in this paper due to space limitations.

\paragraph{Test candidates.}

For experimentation, we considered the following four test candidates.
All our candidates are implemented in C++.%
\footnote{Available at \url{https://osf.io/ku9fx/files/osfstorage?view_only=b7f53d3110894fe39ad1520ed0fed4ec} (anonymous).}
\begin{description}
  \item[\UD] is the original UNDEAD implementation~\cite{conf/ase/ZhouSLCL17} adapted to work in an
    offline setting.
  \item[\UDTRW] implements TRW according to \cref{alg:lock-deps,alg:dp-trw-block}.
  \item[\UDPWR] adapts \cref{alg:lock-deps,alg:dp-trw-block} to PWR.
  \item[\SPDOfflineUD] is our C++ reimplementation of \SPDOffline~\cite{conf/pldi/TuncMPV23}.\footnote{\SPDOffline is written in Java. For a fairer comparison, we therefore use our version \SPDOfflineUD.}
\end{description}
%

\SPDOfflineUD employs two phases that roughly correspond to \cref{alg:lock-deps} and \cref{alg:dp-trw-block}.
In Phase~(1), \SPDOfflineUD computes \SPDVectorClocks vector clocks, which is a simplification of PWR that only satisfies Conditions~\cond{PWR-PO/-LW} (cf.\ \cref{d:pwr}).
In Phase~(2), \SPDOfflineUD makes use of \SPDVectorClocks vector clocks to check
if there is a sync-preserving instance of
a cyclic chain $(\Ald_1,\ldots,\Ald_n)$ of abstract lock dependencies (recall that sync-preserving means that the order of critical sections on the same lock is preserved).
This check is carried out by a call to function~\textsc{CompSPClosure} from~\cite[Algorithm 1]{conf/pldi/TuncMPV23}, replacing \cref{ln:dp-trw-okay} in \cref{alg:dp-trw-block} (the check for Condition \cond{DP-TRW}).
As \SPDOfflineUD does not need to check for Condition~\cond{DP-Block}, \cref{ln:dp-block-start,ln:dp-block-check,ln:dp-block-end} are dropped.


As discussed in \cref{s:sound,s:complete}, in general
\UDTRW is sound but incomplete whereas
\UDPWR is complete but unsound.
\SPDOfflineUD is sound
but only covers sync-preserving deadlocks~\cite{conf/pldi/TuncMPV23}.


\paragraph{Benchmarks and system setup.}

Our experiments are based on a large set of benchmark traces from
prior work~\cite{conf/pldi/TuncMPV23,10.1145/3503222.3507734}.%
%% MS: skip, traced are obtained from two sources.
%% \footnote{
%%   The traces from~\cite{conf/pldi/TuncMPV23} are part of
%%   an artifact, whereas those from~\cite{10.1145/3503222.3507734}
%%   were obtained separately.
%%   \bh{The latter needs clarification.}
%% }
We excluded five benchmarks (``RayTracer'', ``jigsaw'', ``Sor'', ``Swing'', ``eclipse''),
because we noticed that their traces are not well formed.
For example, locks are acquired by distinct threads with no release in between.
We suspect that inaccurate trace recording is at fault.
The issue has been confirmed by \Tunc et al.~\cite{conf/pldi/TuncMPV23}.

We conducted our experiments on an Apple M1 max CPU with 32GB of RAM
running macOS Monterey (Version 12.1).


\paragraph{Evaluation.}

%% table_benchmarks_time_details4_time_float_relative
\begin{table*}[p]
  \caption{
    \small
    \textbf{Deadlock warnings and running times}.
    Columns~2--5 contain the number of events, of threads,
    of memory locations, and of locks, respectively.
    Columns~6--13 show the number of deadlocks reported and running time for each candidate.
    Times include the time to both compute lock dependencies and identify deadlocks (i.e., Phases~(1) and~(2)).
    For \UD, times are in seconds, and rounded to the nearest hundreth.
    For \UDPWR, \UDTRW and \SPDOfflineUD, times are factors compared to \UD.
  }
  \label{tbl:benchmarks}
  {
  \small
  \setlength{\tabcolsep}{2.9pt} % Compress horizontally.
  \renewcommand{\arraystretch}{0.87} % Compress vertically.
\begin{tabular}{|r|r|r|r|r||r|r||r|r||r|r||r|r|}
 \hline
1 & 2 & 3 & 4 & 5 & 6 & 7 & 8 & 9 & 10 & 11 & 12 & 13
 \\
 \hline
\multirow{2}{*}{\Benchmark} &
\multirow{2}{*}{\EE} &
\multirow{2}{*}{\TT} &
\multirow{2}{*}{\MM} &
\multirow{2}{*}{\LL} &
\multicolumn{2}{c||}{\UD} &
\multicolumn{2}{c||}{\UDPWR} &
\multicolumn{2}{c||}{\UDTRW} &
\multicolumn{2}{c|}{\SPDOfflineUD} \\ \cline{6-7} \cline{8-9} \cline{10-11} \cline{12-13}
 & & & &
& \Cycles &  \Time
& \Cycles & \Time
& \Cycles & \Time
& \Cycles & \Time
 \\
 \hline
Deadlock & 28 & 3 & 3 & 2 & 1 & 0.00 & 0 & 1x & 0 & 1x & 0 & 1x  \\   \hline
NotADeadlock & 42 & 3 & 3 & 4 & 1 & 0.00 & 0 & 1x & 0 & 1x & 0 & 1x  \\   \hline
Picklock & 46 & 3 & 5 & 5 & 2 & 0.00 & 1 & 1x & 1 & 1x & 1 & 1x  \\   \hline
Bensalem & 45 & 4 & 4 & 4 & 2 & 0.00 & 1 & 1x & 1 & 1x & 1 & 1x  \\   \hline
Transfer & 56 & 3 & 10 & 3 & 1 & 0.00 & 0 & 1x & 0 & 1x & 0 & 1x  \\   \hline
Test-Dimminux & 50 & 3 & 8 & 6 & 2 & 0.00 & 2 & 1x & 2 & 1x & 2 & 1x  \\   \hline
StringBuffer & 57 & 3 & 13 & 3 & 1 & 0.00 & 1 & 1x & 1 & 1x & 1 & 1x  \\   \hline
Test-Calfuzzer & 126 & 5 & 15 & 5 & 1 & 0.00 & 1 & 1x & 1 & 1x & 1 & 1x  \\   \hline
DiningPhil & 210 & 6 & 20 & 5 & 1 & 0.00 & 1 & 1x & 1 & 1x & 1 & 1x  \\   \hline
HashTable & 222 & 3 & 4 & 2 & 0 & 0.00 & 0 & 1x & 0 & 1x & 0 & 1x  \\   \hline
Account & 617 & 6 & 46 & 6 & 3 & 0.00 & 0 & 1x & 0 & 1x & 0 & 1x  \\   \hline
Log4j2 & 1\mbox{K} & 4 & 333 & 10 & 0 & 0.00 & 0 & 1x & 0 & 1x & 0 & 1x  \\   \hline
Dbcp1 & 2\mbox{K} & 3 & 767 & 4 & 2 & 0.00 & 1 & 1x & 1 & 1x & 1 & 1x  \\   \hline
Dbcp2 & 2\mbox{K} & 3 & 591 & 9 & 1 & 0.01 & 0 & 1x & 0 & 1x & 0 & 1x  \\   \hline
Derby2 & 3\mbox{K} & 3 & 1\mbox{K} & 3 & 0 & 0.00 & 0 & 1x & 0 & 1x & 0 & 1x  \\   \hline
elevator & 222\mbox{K} & 5 & 726 & 51 & 0 & 0.50 & 0 & 1x & 0 & 1x & 0 & 1x  \\   \hline
hedc & 410\mbox{K} & 7 & 109\mbox{K} & 7 & 0 & 0.97 & 0 & 1x & 0 & 1x & 0 & 1x  \\   \hline
JDBCMySQL-1 & 436\mbox{K} & 3 & 73\mbox{K} & 10 & 2 & 1.02 & 0 & 1x & 0 & 1x & 0 & 1x  \\   \hline
JDBCMySQL-2 & 436\mbox{K} & 3 & 73\mbox{K} & 10 & 0 & 1.00 & 0 & 1x & 0 & 1x & 0 & 1x  \\   \hline
JDBCMySQL-3 & 436\mbox{K} & 3 & 73\mbox{K} & 12 & 8 & 1.02 & 1 & 1x & 1 & 1x & 1 & 1x  \\   \hline
JDBCMySQL-4 & 437\mbox{K} & 3 & 73\mbox{K} & 13 & 6 & 1.00 & 1 & 1x & 1 & 1x & 1 & 1x  \\   \hline
cache4j & 758\mbox{K} & 2 & 46\mbox{K} & 19 & 0 & 1.77 & 0 & 1x & 0 & 1x & 0 & 1x  \\   \hline
ArrayList & 3\mbox{M} & 801 & 121\mbox{K} & 801 & 4 & 5.94 & 4 & 4x & 4 & 4x & 4 & 1x  \\   \hline
IdentityHashMap & 3\mbox{M} & 801 & 496\mbox{K} & 801 & 1 & 6.12 & 1 & 4x & 1 & 4x & 1 & 1x  \\   \hline
Stack & 3\mbox{M} & 801 & 118\mbox{K} & 2\mbox{K} & 3 & 7.66 & 3 & 5x & 3 & 5x & 3 & 1x  \\   \hline
LinkedList & 3\mbox{M} & 801 & 290\mbox{K} & 801 & 4 & 7.85 & 4 & 3x & 4 & 3x & 4 & 1x  \\   \hline
HashMap & 3\mbox{M} & 801 & 555\mbox{K} & 801 & 1 & 7.85 & 1 & 3x & 1 & 3x & 1 & 1x  \\   \hline
WeakHashMap & 3\mbox{M} & 801 & 540\mbox{K} & 801 & 1 & 7.97 & 1 & 3x & 1 & 3x & 1 & 1x  \\   \hline
Vector & 3\mbox{M} & 3 & 14 & 3 & 1 & 8.78 & 1 & 1x & 1 & 1x & 1 & 1x  \\   \hline
LinkedHashMap & 4\mbox{M} & 801 & 617\mbox{K} & 801 & 1 & 9.62 & 1 & 2x & 1 & 3x & 1 & 1x  \\   \hline
montecarlo & 8\mbox{M} & 3 & 850\mbox{K} & 2 & 0 & 18.63 & 0 & 1x & 0 & 1x & 0 & 1x  \\   \hline
TreeMap & 9\mbox{M} & 801 & 493\mbox{K} & 801 & 1 & 20.73 & 1 & 1x & 1 & 1x & 1 & 1x  \\   \hline
hsqldb & 20\mbox{M} & 46 & 945\mbox{K} & 402 & 0 & 49.82 & 0 & 1x & 0 & 1x & 0 & 1x  \\   \hline
sunflow & 21\mbox{M} & 15 & 2\mbox{M} & 11 & 0 & 53.26 & 0 & 1x & 0 & 1x & 0 & 1x  \\   \hline
jspider & 22\mbox{M} & 11 & 5\mbox{M} & 14 & 0 & 56.28 & 0 & 1x & 0 & 1x & 0 & 1x  \\   \hline
tradesoap & 42\mbox{M} & 236 & 3\mbox{M} & 6\mbox{K} & 2 & 114.17 & 0 & 1x & 0 & 1x & 0 & 1x  \\   \hline
tradebeans & 42\mbox{M} & 236 & 3\mbox{M} & 6\mbox{K} & 2 & 114.26 & 0 & 1x & 0 & 1x & 0 & 1x  \\   \hline
TestPerf & 80\mbox{M} & 50 & 598 & 8 & 0 & 173.71 & 0 & 1x & 0 & 1x & 0 & 1x  \\   \hline
Groovy2 & 120\mbox{M} & 13 & 13\mbox{M} & 10\mbox{K} & 0 & 308.10 & 0 & 1x & 0 & 1x & 0 & 1x  \\   \hline
tsp & 307\mbox{M} & 10 & 181\mbox{K} & 2 & 0 & 876.75 & 0 & 1x & 0 & 1x & 0 & 1x  \\   \hline
lusearch & 217\mbox{M} & 10 & 5\mbox{M} & 118 & 0 & 597.37 & 0 & 1x & 0 & 1x & 0 & 1x  \\   \hline
biojava & 221\mbox{M} & 6 & 121\mbox{K} & 78 & 0 & 595.36 & 0 & 1x & 0 & 1x & 0 & 1x  \\   \hline
graphchi & 216\mbox{M} & 20 & 25\mbox{M} & 60 & 0 & 615.90 & 0 & 1x & 0 & 1x & 0 & 1x  \\   \hline
 \hline
\Sum & 1354\mbox{M} & 7\mbox{K} & 61\mbox{M} & 30\mbox{K} & 55 & 3663.45 & 27 & 1x & 27 & 1x & 27 & 1x
 \\  \hline   \end{tabular}
}
\end{table*}




\Cref{tbl:benchmarks} contains all benchmark results.
It reports the running time and the number of deadlocks
reported for each candidate.

\paragraph{Precision.}

As expected, the number of deadlocks reported decreases
when comparing \UD against \UDPWR and \UDTRW.
\UD reports 55 deadlocks overall,
whereas \UDPWR, \UDTRW  and \SPDOfflineUD report 27 deadlocks overall.
\UDPWR and \UDTRW report the exact same deadlocks.
That is, each deadlock reported is the same instance resulting
from a cyclic chain of abstract lock dependencies.
Thus, we can conclude:
\begin{itemize}
   \item All deadlocks reported by \UDPWR\ are true positives, because \UDTRW is sound.
   \item \UDTRW has no false negatives, because \UDPWR is complete.
\end{itemize}

Benchmark ``Groovey2'' contains TRW-unbounded critical sections
and the benchmark ``hsqldb'' is ill nested (cf.\ \cref{s:sound:sound}).
It is straightforward to modify Phase~(1) to check for these conditions; for brevity, we omit details.
As there are no deadlocks reported for both these benchmarks, soundness of the results of \UDTRW is not affected.

For almost all benchmarks, Conditions~\cond{DP-PWR} and~\cond{DP-TRW}
are crucial for eliminating false positives.
Benchmark ``Picklock'' is the only exception:
Condition~\cond{DP-Block} was needed to eliminate a false positive.

\SPDOfflineUD and \UDTRW report similar deadlocks.
The reported deadlocks result from the same cyclic chains of abstract lock dependencies,
but some of the reports by \UDTRW are not sync-preserving.
However, this should not be viewed as evidence that \emph{most} deadlocks
in practice are sync-preserving, as the benchmark traces do not necessarily represent the full spectrum of concurrency patterns in modern programming (in fact, the benchmark traces were obtained
from programs that are several years old).

\paragraph{Performance.}

The overall running times of \UDPWR, \UDTRW and \SPDOfflineUD
are in the same range as \UD.
For seven benchmarks (``ArrayList''--``WeakHashMap'', ``LinkedHashMap'')
we encounter an increase by a factor of two to five
for \UDPWR and \UDTRW compared to \UD and \SPDOfflineUD.
This increase is due to the large number of 801~threads.
The computation of PWR and TRW vector clocks requires tracking conflicts
among critical sections in different threads.
The more threads the more conflict management takes place, leading to
some overhead in Phase~(1).
Despite a similarly large number of threads (256--801),
there is no increase for benchmarks ``TreeMap'', ``tradesoap'', and ``tradebeans''.
This is due to fewer conflicts among critical sections.

The running times of \UDPWR\ and \UDTRW are effectively the same,
except for benchmark~``LinkedHashMap''.
The difference of a factor of two (\UDPWR) versus a factor of three~(\UDTRW)
is due to the fact that \UDTRW has to additionally deal with write-write and read-write conflicts (Condition~\cond{PWR-LW} versus Condition~\cond{TRW-Conf}).
As ``LinkedHashMap'' has a large number of memory locations (617K),
the running time of \UDTRW is slightly higher due to more management and
synchronization of conflicting memory operations.

Overall, \SPDOfflineUD runs a bit faster than \UDPWR and \UDTRW.
The main reason is that in Phase~(1), \SPDOfflineUD computes
\SPDVectorClocks vector clocks that are not affected by large numbers of threads.
On the other hand, Phase~(2) of \SPDOfflineUD builds
sync-preserving closures, whereas \UDTRW only compares
vector clocks (see \cref{ln:dp-trw-okay} in \cref{alg:dp-trw-block}).
This difference entails a complexity difference of $O(\THREADNO)$ for \UDTRW versus $O(\TRACELEN)$ for \SPDOfflineUD, due to which we may expect \UDTRW to run faster than \SPDOfflineUD (usually, $\TRACELEN$ is much bigger than $\THREADNO$).
However, table 2, which splits running times into Phase~(1) and Phase~(2), show that this does not affect the benchmark traces: the running times of Phase~(2) are neglible compared to Phase~(1).


%% MS: fun fact,
%% For Grovvey2, detection of cycles is faster for PWR,
%% cause thanks to DP-3, we can early abandon some infeasible cycle.
%% To be discussed somewhere else.
%


%%table_benchmarks_time_details_time_vcd_no_evt_stats
\begin{table*}[t]
  \caption{
    \textbf{Further details: running times of Phases~(1) and~(2) and numbers of concrete lock dependencies.}
      Columns~2---9 contain the number of deadlocks reported, of acquires that
      lead to a lock dependency (the same for \UDTRW and \SPDOfflineUD), and running time for each candidate.
      Times are rounded in seconds reported for Phases~(1) and~(2) separately.
      %% MS: not true, there are some differences but overall the running times are pretty much the same.
      %% Differences in Phase~(1) running times between \UDTRW and \UDTRWEvict are circumstantial (there is no difference in implementation) and should be disregarded.
  }%
  \label{tbl:details}%
{%
  \small%
  \setlength{\tabcolsep}{3.2pt}% % Compress horizontally.
  % \renewcommand{\arraystretch}{0.91}% % Compress vertically.
\begin{tabular}{|r||r|r|r||r|r||r|r|r|}
 \hline
1 & 2 & 3 & 4 & 5 & 6 & 7 & 8 & 9
 \\
 \hline
\multirow{2}{*}{\Benchmark} &
\multicolumn{3}{c||}{\UDTRWEvict} &
\multicolumn{2}{c||}{\UDTRW} &
\multicolumn{3}{c|}{\SPDOfflineUD} \\ \cline{2-4} \cline{5-6} \cline{7-9}
& \Cycles &  \Dependencies & \Time \ (\PhaseOne+\PhaseTwo)
& \Cycles & \Time \ (\PhaseOne+\PhaseTwo)
& \Cycles & \Dependencies & \Time \ (\PhaseOne+\PhaseTwo)
 \\
 \hline
\ldots & \ldots & \ldots & \ldots  & \ldots & \ldots & \ldots & \ldots & \ldots  \\   \hline
Vector & 1 & 3 & 11 (11+0) & 1 & 10 (10+0) & 1 & 200\mbox{K} & 9 (9+0)  \\   \hline
%% jspider & 0 & 158 & 72 (72+0) & 0 & 73 (73+0) & 0 & 2\mbox{K} & 72 (72+0)  \\   \hline
tradesoap & 0 & 9\mbox{K} & 173 (166+6) & 0 & 174 (168+6) & 0 & 40\mbox{K} & 163 (157+6)  \\   \hline
%% TestPerf & 0 & 0 & 196 (196+0) & 0 & 199 (199+0) & 0 & 0 & 192 (192+0)  \\   \hline
Groovy2 & 0 & 11\mbox{K} & 380 (379+1) & 0 & 386 (385+1) & 0 & 29\mbox{K} & 372 (371+1)  \\   \hline
%% tsp & 0 & 0 & 992 (992+0) & 0 & 989 (989+0) & 0 & 0 & 997 (997+0)  \\   \hline
%% lusearch & 0 & 87 & 719 (719+0) & 0 & 723 (723+0) & 0 & 41\mbox{K} & 719 (719+0)  \\   \hline
%%biojava & 0 & 89 & 656 (656+0) & 0 & 656 (656+0) & 0 & 545 & 661 (661+0)  \\   \hline
%% \hline
\Sum & 27 & 33\mbox{K} & 4533 (4518+14) & 27 & 4546 (4531+15) & 27 & 626\mbox{K} & 4348 (4333+15)
\\  \hline   \end{tabular}%
}%
\end{table*}




\paragraph{Reducing the number of concrete lock dependencies.}

As also observed in \cite{conf/pldi/TuncMPV23},
the number of concrete lock dependencies can be huge while
the number of abstract lock dependencies remains small.
Recall trace~$T_{10}$ in \cref{fig:ld-abstract}, where
subtrace $[e_{11},...,e_{14}]$ might result from the body of a loop.
In further loop iterations, further entries
will be added to the abstract lock dependency~$\LDMap{\thread{1}}{\LKB}{\{\LKA\}}$.
The number of entries (i.e., concrete lock dependency) increases,
whereas the number of abstract lock dependencies remains the same.

In our approach, we can aggressively remove `duplicates' of concrete lock dependencies.
We consider two entries as duplicates if no inter-thread synchronization took place between
processing the respective acquires.
Such an optimization is not possible for \SPDOfflineUD, because the trace order
of concrete lock dependencies matters (for sync-preservation):
removing a concrete lock dependency may cause a deadlock pattern to no longer be sync-preserving.

We implemented a variant of \UDTRW, referred to as \UDTRWEvict, where duplicates are removed.
\Cref{tbl:details} compares these variants to each other and to \SPDOfflineUD by including separately the running time of Phases~(1) and~(2);
for brevity, we only detail a few selected cases but include the totals over all benchmark traces.
The table shows that the number of concrete lock dependencies
can be reduced substantially.
For example, in case of benchmark ``Vector'' the number of concrete lock dependencies
is reduced from 200K to three.
We might expect that fewer concrete lock dependencies causes Phase~(2) of \UDTRWEvict to
run faster, because the loop in
function~\textsc{checkTRW}~(\cref{alg:dp-trw-block}) needs to consider fewer candidates.
However, the measurements in \cref{tbl:details} show that this is not the case in practice:
\UDTRW and \UDTRWEvict run equally fast.
We believe that this is due to the few and rather simple deadlocks in our benchmark suite.



\paragraph{Deadlocks reported in \cite{conf/pldi/TuncMPV23} and trace anomalies.}

The number of deadlocks reported in
\cref{tbl:benchmarks}
differ to that in~\cite[Table~1]{conf/pldi/TuncMPV23} for \SPDOffline.
For example, for ``ArrayList'',
\SPDOfflineUD reports four deadlocks while
\SPDOffline only reports three.
We believe that this is caused by a difference in handling lock-request versus lock-acquire events:

\begin{enumerate}
  \item\label{it:UD}
    Our implementations (derived from UNDEAD) treats the vector clock of a thread just before processing an acquire event to correspond to the corresponding request event, effectively allowing us to
    ignore all request events.
  \item\label{it:SPD}
    \SPDOffline assumes that every acquire event is preceded by a request
    event; based on our own knowledge and correspondence with the authors of~\cite{conf/pldi/TuncMPV23}, \SPDOffline uses these request events explicitly.
\end{enumerate}

Both approaches have advantages and disadvantages.
Some (prematurely-ended) traces may end in request events not followed by a corresponding acquire event.
Hence, Approach~\labelcref{it:SPD} may identify more cycles than Approach~\labelcref{it:UD};
this seems to be the case for ``JDBCMySQL-2'' (based on a comparison of \cref{tbl:benchmarks} and \cite[Table~1]{conf/pldi/TuncMPV23}).
However, not every acquire event is preceded by a request event, so Approach~\labelcref{it:UD} may report more cycles;
this applies to trace~``ArrayList''.

In fact, we even encountered a trace (``Log4j2'')
where there is a matching request-release pair of events without an acquire event between.
Moreover, we encountered some ill-formed traces, where a lock has been acquired by two distinct threads
without being released in between.
This explains the difference in number of deadlocks reported between \cref{tbl:benchmarks} and \cite[Table~1]{conf/pldi/TuncMPV23}.





%%% Local Variables:
%%% mode: latex
%%% TeX-master: "main"
%%% End:
