\section{Preliminaries} \label{sec:background}
This section provides some basics about POM, including how price oracle manipulation occurs, the typical POMs and the representative causes of POM vulnerabilities.
We assume some familiarity with basic concepts such as blockchain, Ethereum, and smart contracts, and refer readers to~\cite{wood2014ethereum} for details. 

\subsection{Types of DeFi Applications}
DeFi applications aim to provide financial services without traditional intermediaries, leveraging blockchain technology and smart contracts.
The main types of DeFi applications include decentralized exchanges (DEXs), stablecoins, lending and borrowing platforms, yield farming and liquidity mining, and decentralized autonomous organizations (DAOs) etc.
This work further classifies DeFi applications into two categories based on their role in the ecosystem:
\paragraph{Price Provider Applications} These applications provide price data to other DeFi applications. Examples include Chainlink oracle contracts, decentralized exchanges (DEXs), and other protocols that define their own logic for determining asset prices.
\paragraph{Price Consumer Applications} Applications falling under this category rely on accurate price data for their operational efficacy. Examples include lending and borrowing platforms, which use price data to determine the valuation of collaterals, and DAOs, which use price data to determine voting weights and other governance parameters.

With this classification, we can accurately analyze the root causes of the price oracle vulnerabilities and design efficient prompts to help LLMs detect such kinds of vulnerabilities.

\subsection{Types of Price Oracles}
Price oracles can be generally categorized into three classes: on-chain oracles, off-chain oracles, and hybrid price oracles. 
    \paragraph{On-Chain Oracles} On-chain oracles, like those used by Uniswap, derive price data directly from on-chain activities such as trading within liquidity pools. These oracles use mechanisms like Constant Product Formula (CPF)~\ref{uniswapcpf} to calculate asset prices based on real-time transactions occurring on the blockchain. Despite their susceptibility to manipulation in low-liquidity scenarios, they offer several advantages:
    \begin{itemize}
        \item Native to Blockchain: Since on-chain oracles operate entirely within the blockchain environment, they provide seamless integration with decentralized applications (DApps) and smart contracts without the need for external dependencies.
        \item Real-time Pricing: Prices reflect current market conditions as they are derived directly from ongoing transactions on the blockchain.
        \item Decentralization: Since these oracles are based on decentralized mechanisms (e.g., Uniswap’s liquidity pools), there is no central authority controlling the price feed, reducing single points of failure.
        \item Full Transparency: Anyone can verify the price data on-chain, ensuring the data's integrity and preventing manipulation by a central entity.
    \end{itemize}


    \paragraph{Off-Chain Oracles} Off-chain oracles, exemplified by Chainlink~\footnote{https://chain.link/}, gather data from external sources and bring it onto the blockchain through a decentralized network of node operators. Their advantages can be summarized as follows:
    
    \begin{itemize}
        \item Access to Diverse Data: Off-chain oracles can pull price information from a wide range of external sources, including traditional financial markets, making them suitable for use cases requiring data beyond the blockchain ecosystem.
        \item Robustness to Manipulation: Since off-chain oracles aggregate data from multiple independent nodes or sources, they are generally more resistant to manipulation or data skewing compared to purely on-chain systems.
        \item Scalability: They are often more scalable than on-chain oracles since they are not dependent on the blockchain’s transaction throughput and can aggregate large volumes of data from diverse sources without congesting the network.
    \end{itemize}
    
    \paragraph{Hybrid Oracles} Hybrid oracles, such as those used by Extra Finace\footnote{https://docs.extrafi.io/extra\_finance/leverage-farming/price-feed} combine features of both on-chain and off-chain oracles to enhance price stability and security. Typically, the prices are derived directly from on-chain data provided by decentralized exchanges (DEXs). 
    However, to mitigate the risk of abnormal price fluctuations, Chainlink price feeds are employed as a safeguard. 
    % The liquidation mechanism is activated only when the price variance between the Time-Weighted Average Price (TWAP), the current on-chain price, and the Chainlink feed remains within a predefined threshold. If the price deviation exceeds this threshold, the transaction is reverted to prevent potential price manipulation attacks. 
    This hybrid approach offers several advantages:

    \begin{itemize}
        \item Comprehensive Data Validation: Hybrid oracles cross-reference off-chain and on-chain data to ensure both accuracy and consistency, reducing the risk of manipulation.
        \item Real-Time Responsiveness: The integration of on-chain mechanisms ensures timely updates to price data, even during volatile market conditions.
        \item Resilience to Attacks: The use of diverse data sources creates redundancy, making hybrid oracles more robust against single-point failures or targeted attacks.
    \end{itemize}

    Despite the popularity of \textit{Off-Chain Oracles} like Chainlink, \textit{On-Chain Oracles} remain a viable option for many in the blockchain community. 
    According to an oracle dashboards \footnote{https://defillama.com/oracles} \footnote{https://defillama.com/oracles/TWAP}, as many as 90 projects opted for On-Chain Oracles.
    This preference is largely due to their inherent consistency with the decentralization philosophy of blockchain systems, as they operate entirely within the blockchain environment, ensuring trustlessness and minimizing reliance on external entities.


\subsection{Price Oracles Manipulation (POM)}
POM can stem from various sources based on above types of oracles. The primary complications arise from on-chain and off-chain oracles.
\paragraph{On-Chain Price Oracle Manipulation}
On-chain oracles can be easily manipulated due to their reliance on spot prices from a single source. For instance, an attacker can use a flash loan to temporarily drain liquidity from a pool, causing the price to be artificially inflated or deflated. This manipulation allows the attacker to exploit the manipulated price, leading to significant financial gains, as demonstrated in the PancakeBunny attack\footnote{https://medium.com/amber-group/bsc-flash-loan-attack-pancakebunny-3361b6d814fd}.
A more detailed example is illustrated in the Appendix~\ref{app:oraclemanip}.

\paragraph{Off-Chain Price Oracle Manipulation}
Off-chain oracles face different challenges. Centralized off-chain oracles depend on a single trusted entity, making them vulnerable to malicious data submission by authorized users for personal gain. Additionally, the compromise of private keys can pose significant risks. Decentralized off-chain oracles mitigate some of these risks by aggregating data from multiple sources, but they are not immune to issues like freeloading or Sybil attacks among data collectors. Further, off-chain infrastructure vulnerabilities—including those in access control, cryptographic implementations, transport, and database security—add layers of complexity in preventing manipulation~\cite{offchainoracle}.

While these issues are broad and affect the overall security of price oracles, this paper focuses specifically on vulnerabilities that adversaries can exploit, particularly through specialized inputs to on-chain contracts. This includes manipulations involving on-chain price oracles and the on-chain components of off-chain price oracles, which can lead to significant financial losses or unfair advantages for attackers.


\subsection{Causes of Price Oracle Manipulation}
Price oracle manipulation arises from various factors that exploit weaknesses in both the underlying mechanisms and the broader DeFi ecosystem. Below are some key causes:

\paragraph{Smart Contract Vulnerabilities} Careless bugs or flawed logic while development in the smart contracts governing liquidity pools or price feed mechanisms can lead to incorrect pricing, enabling attackers to manipulate asset values and potentially causing significant financial losses for users and protocols~\cite{gao2024unveiling}.

\paragraph{Flash Loan Attacks} Flash loans allow users to borrow large amounts of capital without collateral, provided the loan is repaid within the same transaction. Attackers exploit this feature by executing large trades to temporarily inflate or deflate the price of assets in on-chain liquidity pools. This manipulated price can then be leveraged in other DeFi protocols that depend on the oracle, leading to cascading financial consequences~\cite{zhang2023demystifying}.


\paragraph{Front-Running Attacks} Front-running attacks, enabled by the transparency of blockchain transactions, also contribute to price oracle manipulation. Malicious actors monitor pending transactions and strategically place their trades just before large transactions. By doing so, they can profit from the resulting price changes while distorting the price data in liquidity pools~\cite{zhang2023demystifying}.

\paragraph{Impermanent Loss Impact} Liquidity providers may suffer from impermanent loss, where the value of their deposited assets changes due to price fluctuations within the pool. If a DeFi application relies on the pool's price without accounting for these fluctuations, it might overestimate or underestimate the true value of assets~\cite{labadie2022impermanent}.

\paragraph{Slippage} The difference between the expected and actual executed price of a trade presents another avenue for manipulation. In low-liquidity pools, attackers can exploit slippage by executing large trades that cause significant price deviations. These deviations can propagate through dependent DeFi applications, leading to inaccurate price feeds and destabilizing the broader ecosystem~\cite{labadie2022impermanent}.

These factors are inherent features of blockchain and DeFi systems, not deficiencies. While they do not inherently lead to attacks, they can introduce vulnerabilities under certain conditions. The goal is not to eliminate these features but to identify potential weaknesses and mitigate their adverse effects, thereby maximizing their benefits.

