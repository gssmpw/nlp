\section{Related Work}\label{sec:related}

Smart contract vulnerability detection has advanced significantly, with numerous tools and techniques proposed to address security issues. Despite these efforts, detecting and mitigating manipulative behaviors in price oracles remains a persistent challenge. Existing approaches to this problem can be broadly classified into static analysis, dynamic analysis, machine learning-based methods, and emerging techniques leveraging large language models (LLMs).

\subsection{Static Analysis} \label{Static Analysis}
Static analysis techniques examine the source code or bytecode of smart contracts without execution, employing methods such as symbolic execution, formal verification, and pattern matching to identify vulnerabilities. Several tools have been developed for various types of vulnerabilities. For instance, Oyente\cite{luu2016oyente} uses symbolic execution to detect issues such as reentrancy, transaction order dependency, suicidal contracts, and integer overflows. SmartCheck\cite{tikhomirov2018smartcheck} applies rule-based techniques to identify vulnerabilities and bad practices in Solidity contracts, while Slither~\cite{feist2019slither} combines dataflow analysis, taint analysis, and pattern matching to detect a wide range of vulnerabilities efficiently.
Formal verification methods further enhance static analysis by modeling smart contract behavior using formal languages and verifying properties with SMT solvers or theorem provers. Examples include VeriSmart\cite{so2020verismart} and sVerify\cite{gao2021sverify}, which are tailored for smart contract verification against predefined specifications.

However, few works focus explicitly on price oracle manipulation vulnerabilities. Recent research has started addressing this gap:
Foray~\cite{wen2024foray} is an attack synthesis framework for DeFi protocols that uses a domain-specific language to convert smart contracts into token flow graphs. While it identifies strategic paths and synthesizes attacks via symbolic compilation, its focus is limited to four specific types of logical flaws, which partially overlap but do not fully align with our target vulnerabilities.
OVer~\cite{deng2024safeguarding} employs symbolic analysis to model DeFi protocol behavior under skewed oracle inputs, identifying secure parameters and generating guard statements to mitigate manipulation attacks. While effective, its focus on optimizing parameters for specific protocols limits its generalizability to broader applications.
VeriOracle~\cite{mo2023toward} introduces a formal verification framework that deploys a semantic model on the blockchain to monitor smart contract states and detect problematic price feed transactions in real time.
DeFiTainter~\cite{kong2023defitainter} leverages innovative mechanisms to construct call graphs and semantically track inter-contract taint data for detecting price manipulation vulnerabilities. 
However, they both require extensive on-chain transaction data, demanding significant resources and differing from our approach.


\subsection{Dynamic Analysis} \label{Dynamic Analysis}
Dynamic analysis techniques execute smart contract code and monitor its runtime behavior to identify vulnerabilities. These methods often employ fuzzing, symbolic execution, and runtime monitoring to detect issues such as assertion failures, overflows, and frozen ether. Tools like Mythril\cite{mueller2024mythril}, Manticore\cite{mossberg2019manticore}, sFuzz\cite{nguyen2020sfuzz}, and ContractFuzzer\cite{jiang2018contractfuzzer} have been widely used for identifying common vulnerabilities.

Despite their success, traditional dynamic analysis tools have rarely addressed price oracle manipulation vulnerabilities. Only a few works have specifically targeted this challenge.
DeFiRanger~\cite{wu2021defiranger} recovers high-level DeFi semantics from raw Ethereum transactions and identifies price oracle manipulation attacks through pattern matching. However, its approach is post-mortem, as it can only detect observed attack transactions, limiting its usefulness for proactive vulnerability detection.
ProMutator~\cite{wang2021promutator} models typical DeFi usage patterns by analyzing existing transactions and simulates potential price manipulation attacks through mutated transactions. This approach effectively identifies weak points in oracle systems before exploitation. However, accurately modeling DeFi transaction patterns is challenging, especially for novel attack vectors, and its simulation-based method requires significant computational resources, impacting scalability and real-time applicability.
DeFiPoser~\cite{zhou2021defiposer} employs a dual approach: DEFIPOSER-ARB for identifying arbitrage opportunities and SMT solvers to create logical models for detecting complex profitable transactions. While it can uncover new vulnerabilities in real time, the system relies on manual and costly modeling of DeFi protocols, making it resource-intensive. Furthermore, its effectiveness may be limited by the rapid evolution of DeFi protocols, requiring frequent updates to maintain accuracy and relevance.

These limitations highlight the advantages of our AI-driven framework, which eliminates reliance on expert knowledge and enhances scalability, effectively 
overcoming the inefficiencies and adaptability challenges in existing 
methods.

\subsection{Machine Learning-based Methods} \label{Machine Learning-based Methods}
Machine learning-based methods have gained traction in recent years for smart contract vulnerability detection. These approaches typically involve extracting features from the contract's source code or bytecode and training classifiers or deep learning models to predict the presence of vulnerabilities. Some notable works in this domain include ContractWard~\cite{wang2020contractward}, which trains a classifier based on features extracted from the contract's bytecode, and the hybrid approach proposed by Liu et al.~\cite{liu2021combining}, which combines pure neural networks with interpretable graph features and expert patterns.
Graph neural networks have also been explored for smart contract vulnerability detection. These approaches represent the contract's control flow graph or data dependency graph as a graph-structured data and apply graph neural networks to learn vulnerability patterns. EtherGIS~\cite{zeng2022ethergis} is an example of a vulnerability detection framework that utilizes graph learning features to detect vulnerabilities in Ethereum smart contracts.

While these methods achieve high accuracy for various vulnerabilities, they rely heavily on large labeled datasets and often struggle with novel or unseen patterns. Moreover, limited work has specifically addressed price oracle manipulation vulnerabilities, leaving a gap that requires innovative solutions.

\subsection{Large Language Model-based Methods} \label{Large Language Model-based Methods}
Recent advancements in large language models (LLMs) have opened up new possibilities for smart contract vulnerability detection. LLMs, such as GPT, have demonstrated remarkable capabilities in understanding and generating human-like text, and researchers have begun exploring their application in smart contract analysis.
Gao et al.\cite{gao2024unveiling} explored LLMs for detecting complex bugs, including price oracle manipulation, using diverse prompts. However, this early work demonstrated limited performance, making it unsuitable for practical use. Similarly, Issac et al.\cite{david2023you} evaluated ChatGPT-4 and Claude for smart contract audits, identifying logic flaws and coding errors but reporting an unacceptably high false positive rate (95\%), which hinders real-world adoption. GPTLens~\cite{hu2023large} proposed an adversarial framework leveraging LLMs in dual roles to enhance detection accuracy, but its effectiveness on price oracle manipulation vulnerabilities remains limited.
GPTScan~\cite{sun2023gptscan} combines GPT with program analysis techniques to identify logic vulnerabilities in smart contracts. By leveraging GPT's code understanding and static confirmation, GPTScan reduces false positives and achieves high precision and recall in terms of vulnerability type detection across diverse datasets. 

In contrast, our work develops a fully LLM-driven approach focused on prompt engineering for detecting POM vulnerabilities. 
By utilizing domain-specific knowledge extraction and context-aware prompt generation, we enable LLMs to automatically identify manipulation patterns. Our method is user-friendly, generalizable, and provides actionable feedback by leveraging LLMs' capacity to understand the contextual 
nuances of price oracle manipulations.








