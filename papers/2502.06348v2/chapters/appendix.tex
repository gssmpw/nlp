\appendix


\section{Appendix: Prompts of \tool} \label{app:componentprompts}
\subsection{Prompt Generated by Prompt Generator}
\begin{tcolorbox}[colback=gray!20, colframe=gray!50, title={Prompt Generated by Prompt Generator}, label=cotprompt]
description: Detect price oracle manipulation vulnerabilities in a smart contract,  steps:
step: 1, description: Identify the price oracle used by the smart contract. Determine if the price oracle is a decentralized exchange (DEX) or another external price feed.,      
    questions: What is the price oracle used by the smart contract?       
    Is the price oracle a decentralized exchange (DEX) or another external price feed?
step: 2, description: Analyze the smart contract's reliance on the price oracle. Determine if the contract uses the price provided by the oracle directly, without any additional validation or cross-checking.,     
    questions: Does the smart contract rely solely on the price provided by the price oracle, without any additional validation?
    Are there any mechanisms in place to detect and mitigate price manipulation in the price oracle?  
step: 3, description: Evaluate the potential impact of a price oracle manipulation attack on the smart contract. Determine if the attack could lead to financial losses or unfair advantages for the attacker.,      
    questions: What are the potential financial consequences if the price oracle is manipulated?       
    Could the price oracle manipulation lead to unfair advantages for the attacker? 
step: 4, description: Provide recommendations to mitigate the price oracle manipulation vulnerability, such as using multiple price oracles, implementing price validation mechanisms, or using more robust price feeds.,      
    questions: What are the recommended mitigation strategies to address the price oracle manipulation vulnerability?
    How can the smart contract be modified to reduce the risk of price oracle manipulation attacks?
\end{tcolorbox}

\section{Appendix: Knowledge}


\subsection{Human-curated Domain Knowledge} \label{app:humanknowledge}
\begin{tcolorbox}[colback=gray!20!white, colframe=gray!75!black, boxsep=5pt, arc=4pt, boxrule=1pt, left=0pt, right=0pt, title={Human-curated Domain Knowledge}, label=humanknowledge]
Liquidity pools provide the underlying liquidity for decentralized exchanges (DEXs) like Uniswap by holding token reserves. These pools can also act as on-chain price oracles, offering real-time price data to other applications. When discussing smart contracts vulnerable to price oracle manipulation, we focus on two main categories of vulnerabilities:

1. Vulnerabilities in DEXes and Liquidity Pools/reserves: a) Lack of Slippage Tolerance: Users may suffer unexpected losses due to significant price fluctuations during swaps or trades if the smart contracts do not have adequate slippage tolerance settings. b) Susceptibility to Front-Running or Sandwich Attacks: Smart contracts that do not mitigate front-running or sandwich attacks can expose users to losses when malicious actors manipulate transaction order and timing.

2. Vulnerabilities in Price-Dependent Applications: a) Price Oracle Manipulation: Smart contracts that rely on external price oracles can be manipulated through tampering with the price feed, leading to incorrect asset valuations and financial loss. b) Unfair asset valuation: Poorly designed smart contracts may allow malicious users to manipulate asset valuations, even if the oracle price is accurate. Inadequate safeguards can enable unfair trading practices, disadvantaging other users.
\end{tcolorbox}

\section{Appendix: Illustration of POM} \label{app:oraclemanip}
As discussed in Section~\ref{sec:introduction}, price oracles are integral to the functionality of DeFi applications but are also vulnerable to manipulation. One example of POM is depicted in Figure~\ref{fig:priceoracle}, which is excerpted from project \textit{Behodler}\footnote{\url{https://code4rena.com/reports/2022-01-behodler}}.
% The Behodler ecosystem is an efficient, low cost, single sided Automated Market Maker (AMM) with a lower impermanent loss than traditional AMMs.


\begin{figure}[h]
\centering
\begin{lstlisting}[language=Solidity]
function burnAsset(address asset, uint256 amount) public isLive incrementFate {
    require(assetApproved[asset], "LimboDAO: illegal asset");
    address sender = _msgSender();
    require(ERC677(asset).transferFrom(sender, address(this), amount), "LimboDAO: transferFailed");
    uint256 fateCreated = fateState[_msgSender()].fateBalance;
    uint256 actualEyeBalance = IERC20(domainConfig.eye).balanceOf(asset);
    require(actualEyeBalance > 0, "LimboDAO: No EYE");
    uint256 totalSupply = IERC20(asset).totalSupply();
    uint256 eyePerUnit = (actualEyeBalance * ONE) / totalSupply;
    uint256 impliedEye = (eyePerUnit * amount) / ONE;
    fateCreated = impliedEye * 20;
    fateState[_msgSender()].fateBalance += fateCreated;
    emit assetBurnt(_msgSender(), asset, fateCreated);
}
\end{lstlisting}
\caption{Example function \texttt{burnAsset} with potential flash loan attack vulnerability.}
\label{fig:priceoracle}
\end{figure}

The \textbf{burnAsset} function is designed to remove tokens from circulation by burning assets and crediting \textit{Fate} tokens to users. \textit{Fate} tokens serve as a governance currency within the ecosystem, granting holders voting power. This function interacts with EYE-based asset tokens, but the asset pricing formula is vulnerable to flash loan manipulation.

Consider a scenario where there are 1000 EYE and 1000 LINK tokens in a UniswapV2 LINK-EYE pool. The pool’s total supply is 1000, and the attacker holds 100 LP tokens. If the attacker calls the \textbf{burnAsset} function to burn their 100 LP tokens, with the formula in line 9-11, he can earn $1000 \times 100/1000 \times 20 = 2000$ amount of \textit{Fate}.
Here, $1000$ is the \textbf{actualEyeBalance} and $1000$ is the pool's total LP supply. Thus, the attacker rightfully receives 2000 \textit{Fate} tokens.

However, the attacker can exploit the system by swapping in 1000 EYE and receiving 500 LINK from the pool (according to $x \times y = k$, ignoring fees for simplicity). The pool then contains 2000 EYE and 500 LINK tokens. The \textbf{actualEyeBalance} becomes 2000, while the pool's total LP supply and the attacker's LP tokens remain at 1000 and 100, respectively. After this manipulation, the attacker can call the \textbf{burnAsset} function to burn their LP tokens and receive $2000 \times 100/1000 \times 20 = 4000$ amount of \textit{Fate} tokens.Subsequently, the attacker can swap 500 LINK back into the pool to retrieve their 1000 EYE.
Ultimately, the attacker incurs only the transaction fee, yet they gain double the \textit{Fate} tokens (4000) compared to the legitimate amount (2000). With this increased \textit{Fate}, the attacker gains more voting power to influence the system’s decisions or can convert \textit{Fate} to other tokens for direct profit.

This example illustrates how the ratio of pool tokens can be manipulated through flash loans to exploit price oracles, leading to significant imbalances and vulnerabilities in DeFi applications.





% \section{Appendix: Tables}
% \subsection{Average Performance of Varying Prompt Generator and Auditor on DeFiHacks} \label{app:aigenprompt}

% \begin{table}[hbp]
% \caption{Average Performance of Varying Prompt Generator and Auditor on DeFiHacks}
% \label{table:aigenprompt}
% \begin{adjustbox}{width=\textwidth}
% \begin{tabular}{@{}l|cccc|cccc|cccc|cccc@{}}
% \toprule
% Prompt Gen & 4o    & Sonnet & 4o-mini & Haiku & 4o     & Sonnet & 4o-mini & Haiku  & 4o      & Sonnet  & 4o-mini & Haiku   & 4o    & Sonnet & 4o-mini & Haiku \\ 
% Auditor          & 4o    & 4o     & 4o      & 4o    & Sonnet & Sonnet & Sonnet  & Sonnet & 4o-mini & 4o-mini & 4o-mini & 4o-mini & Haiku & Haiku  & Haiku   & Haiku \\ \midrule
% FN               & 23.33 & 23.00  & 25.33   & 24.67 & 30.33  & 28.67  & 30.67   & 27.67  & 18.00   & 12.33   & 12.67   & 13.20   & 22.33 & 26.67  & 24.00   & 25.67 \\
% TP               & 12.67 & 13.00  & 10.67   & 11.33 & 5.67   & 7.33   & 5.33    & 8.33   & 18.00   & 23.67   & 23.33   & 22.80   & 13.67 & 9.33   & 12.00   & 10.33 \\
% FP               & 16.67 & 15.67  & 15.33   & 11.67 & 23.33  & 22.00  & 23.33   & 21.67  & 51.00   & 67.00   & 60.00   & 48.80   & 15.33 & 20.00  & 16.67   & 18.67 \\ \midrule
% Recall           & 0.352 & 0.361  & 0.296   & 0.315 & 0.157  & 0.204  & 0.148   & 0.231  & 0.500   & 0.657   & 0.648   & 0.633   & 0.380 & 0.259  & 0.333   & 0.287 \\
% Precision        & 0.432 & 0.453  & 0.410   & 0.493 & 0.195  & 0.250  & 0.186   & 0.278  & 0.261   & 0.261   & 0.280   & 0.318   & 0.471 & 0.318  & 0.419   & 0.356 \\
% F1               & 0.388 & 0.402  & 0.344   & 0.384 & 0.174  & 0.224  & 0.165   & 0.253  & 0.343   & 0.374   & 0.391   & {\color[HTML]{C834FC} 0.424}   & 0.421 & 0.286  & 0.371   & 0.318 \\ \bottomrule
% \end{tabular}
% \end{adjustbox}
% \end{table}

\section{Appendix: Formulas} \label{app:formulas}
\[
\text{Precision} = \frac{\text{True Positives (TP)}}{\text{True Positives (TP)} + \text{False Positives (FP)}}
\]

\[
\text{Recall} = \frac{\text{True Positives (TP)}}{\text{True Positives (TP)} + \text{False Negatives (FN)}}
\]

\[
F_1\text{-score} = 2 \cdot \frac{\text{Precision} \cdot \text{Recall}}{\text{Precision} + \text{Recall}}
\]