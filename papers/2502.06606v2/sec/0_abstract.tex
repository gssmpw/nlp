\begin{abstract}
% The ABSTRACT is to be in fully justified italicized text, at the top of the left-hand column, below the author and affiliation information.
% Use the word ``Abstract'' as the title, in 12-point Times, boldface type, centered relative to the column, initially capitalized.
% The abstract is to be in 10-point, single-spaced type.
% Leave two blank lines after the Abstract, then begin the main text.
% Look at previous \confName abstracts to get a feel for style and length.

Manipulating the material appearance of objects in images is critical for applications like augmented reality, virtual prototyping, and digital content creation. We present \textbf{MaterialFusion}, a novel framework for high-quality material transfer that allows users to adjust the degree of material application, achieving an optimal balance between new material properties and the object's original features. MaterialFusion seamlessly integrates the modified object into the scene by maintaining background consistency and mitigating boundary artifacts. To thoroughly evaluate our approach, we have compiled a dataset of real-world material transfer examples and conducted complex comparative analyses. Through comprehensive quantitative evaluations and user studies, we demonstrate that MaterialFusion significantly outperforms existing methods in terms of quality, user control, and background preservation. Code is available at \url{https://github.com/ControlGenAI/MaterialFusion}.  


\end{abstract}