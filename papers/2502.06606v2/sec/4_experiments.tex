\section{Experiments}
To compare MaterialFusion with other methods, we created our own dataset of free stock images. Our dataset comprises 15 material images and 25 object-oriented photographs of various objects. Detailed dataset description can be found in Appendix \ref{appendix:dataset} of the supplement. 

We compared our method against the following approaches: Guide-and-Rescale, IP-Adapter with masking, and ZeST. We utilized the authors' original code with the default parameters specified in each method's description.  Detailed configurations of our method and the baselines can be found in the Appendix \ref{appendix:configs}.

Our quantitative analysis involved an assessment of the following aspects:

Firstly, we focus on the preservation of the background of the original images, the geometry of the objects, and the details they contain. To evaluate this, we calculated the Learned Perceptual Image Patch Similarity (LPIPS) \cite{zhang2018unreasonable} between the original object images and those obtained through various material transfer methods.

Secondly, we aim to assess how effectively material can be transferred. To accomplish this, we developed the following scheme: we extracted crops of two sizes, 64x64 and 128x128 pixels, from the resulting images using the object mask, ensuring that only the transferred texture crops were included—without any background. Similarly, we generated crops from the example material images. 
Subsequently, we computed pairwise CLIP similarity scores between these crops to determine the degree of similarity between the textures and then we calculated the average of these scores. 
For a more comprehensive description of the metrics, please refer to Appendix \ref{appendix:metrics}."


\subsection{Qualitative Comparison}
Fig. \ref{fig:qual} presents examples of material transfers utilizing various methods: ZeST, GaR, IP-Adapter with masking, and our proposed approach. The images clearly demonstrate that GaR results in minimal material transfer. While the IP-Adapter successfully captures the texture of the material, it completely fails to preserve details. ZeST consistently performs well in terms of material transfer but struggles to maintain object details. In contrast, our method exhibits robust performance in both material transfer and detail preservation. Additional visual comparisons can be found in Appendix \ref{appendix:qual_anal} of the supplement.


\subsection{Quantitative Comparison}

Fig. \ref{fig:quant_anal} displays all the results of our quantitative analysis. The optimal region on the graph is located in the lower right corner. This area signifies that a high CLIP similarity score indicates effective material transfer to the object, while a low LPIPS value reflects good preservation of the object's details and image's background. 

Upon examining the generated images, we observe that when the CLIP similarity score is below $0.82$, the material does not transfer to the object as effectively as desired. Conversely, a CLIP similarity score greater than $0.84$ indicates successful material transfer. Furthermore, we noted that when the LPIPS value exceeds $0.21$, the material starts to lose its details significantly. Consequently, we have outlined the approximate region of effective material transfer combined with satisfactory preservation of the object in green on the graph.

As illustrated, only two points fall within this favorable zone: MaterialFusion with material transfer strengths of $0.5$ and $0.8$. The results of GaR fall into the region indicating good detail preservation but with low material transfer effectiveness. In contrast, the ZeST performs well in transferring material but fails to preserve the object's details.

This analysis underscores the trade-offs between material transfer efficacy and detail preservation across different methods. For a more comprehensive view, refer to Appendix \ref{appendix:quant_anal} of the supplement, where Fig. \ref{fig:quant_anal} is expanded to include additional methods: our approach without masking and the IP-Adapter with masking.


\begin{figure}[t]
  \centering
  %\fbox{\rule{0pt}{2in} \rule{0.9\linewidth}{0pt}}
   
   \vspace{-15pt}
   \includegraphics[width=0.96\linewidth]{images/quantitative_analysis_14_11.png}

   \caption{Quantitative analysis of material transfer and object preservation. The lower right region represents optimal results with high CLIP scores (effective material transfer) and low LPIPS values (good detail preservation). Our method (\textbf{MaterialFusion}) achieves the best balance, with results in the optimal zone, while GaR and ZeST show trade-offs between transfer efficiency and detail preservation.}
   \label{fig:quant_anal}

   \vspace{-5pt}
\end{figure}

\subsection{User Study}

To evaluate the effectiveness of our method, we conducted a user study comparing our approach with ZeST, the current state-of-the-art method for material transfer. By presenting the results of both our method and ZeST, we asked participants three questions: the first question (Q1) assessed user preferences regarding Overall Preference, the second question (Q2) focused on Material Fidelity, and the third question (Q3) evaluated Detail Preservation of the image results produced by both methods. Details of the user study and all questions used can be found in Appendix \ref{appendix:user_study} of the supplement.

We conducted the study with a total of 3 respondents, each of whom compared results from our method to those from ZeST across 365 pairs of images. In total, this yielded 1,125 responses for each question.



The results of the user study are presented in Table \ref{tab:user_study}. Each value indicates the percentage of users who preferred our method compared to ZeST. According to the respondents, our method produces more realistic images and better preserves the details of the original object compared to ZeST by a wide margin. However, the results of the user study indicate that we transferred the material less effectively than ZeST (41\%/59\%). There is a very simple and logical explanation for this.

When using the simplest approach of cutting the material using a mask and pasting it onto the original image, the material transfer appears perfect; however, this method sacrifices any preservation of the original object. ZeST lacks control over the material transfer force, which can result in outputs that resemble the simplistic cut-and-paste technique (see Fig.\ref{fig:user_study}). 

Therefore, the outcome of 41\%/59\% is quite commendable, as it reflects a balance between maintaining the integrity of the original object and achieving material transfer. Our method may not have transferred the material as effectively as ZeST, but it provided a more realistic and coherent integration of materials and original details, which is a significant achievement in its own right.

\begin{figure}[t]
  \centering
  %\fbox{\rule{0pt}{2in} \rule{0.9\linewidth}{0pt}}
   \vspace{-15pt}
   \includegraphics[width=0.8\linewidth]{images/new_user.png}

   \caption{Examples of comparisons where the vote for Question Q2 (Material Fidelity) was given to ZeST. While ZeST achieves high material transfer, it often overpowers the original object's features, resulting in a "cut-and-paste" effect. Our approach, by contrast, balances material integration with preservation of the object’s details, offering a more coherent and realistic result.}
   \label{fig:user_study}

   \vspace{-5pt}
\end{figure}


\begin{table}
    \vspace{-2pt}
  \caption{User study results comparing our method with ZeST. Our method was preferred overall and rated highly for detail preservation, while ZeST scored better for material fidelity. This balance between material transfer and object fidelity makes our method more effective in delivering coherent and lifelike results.}
  \label{tab:user_study}
  \centering
  \begin{tabular}{@{}lc@{}}
    \toprule
    Questions & Results \\
    \midrule
    Overall Preference (Q1) &  88\% \\
    Material Fidelity (Q2) & 41\% \\
    Detail Preservation (Q3) & 73\% \\
    \bottomrule
  \end{tabular}

  \vspace{-7pt}

\end{table}