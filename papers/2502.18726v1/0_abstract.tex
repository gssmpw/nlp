Deep learning (DL) has revolutionized 
%various fields, enabling significant advancements in 
areas such as computer vision, natural language processing, and more. However, developing DL systems is challenging due to
%many DL systems are developed by domain experts rather than software developers, creating coding challenges due to 
the complexity of DL workflows. Large Language Models (LLMs), such as GPT, Claude, Llama, Mistral, etc., have emerged as promising tools to assist in DL code generation, offering potential solutions to these challenges. Despite this, existing benchmarks such as DS-1000 are limited, as they primarily focus on small DL code snippets related to pre/post-processing tasks and lack a comprehensive coverage of the full DL pipeline, including different DL phases and input data types.

To address this, we introduce \tool, a novel benchmark dataset designed for function-level DL code generation. \tool categorizes DL problems based on three key aspects: phases such as pre-processing, model construction, and training; tasks, including classification, regression, and recommendation; and input data types such as tabular, image, and text. 

GPT-4o---the state-of-the-art LLM---achieved 31\% accuracy on \tool, significantly lower than its 60\% on DS-1000. We observed similar difficulty for other LLMs (e.g., 28\% vs. 54\% for Claude, 21\% vs. 41\% for LLaMA, and 15\% vs. 20\% for Mistral). This result underscores \tool's greater complexity. We also construct a taxonomy of issues and bugs found in LLM-generated DL code, which highlights the distinct challenges that LLMs face when generating DL code compared to general code.

Furthermore, our analysis also reveals substantial performance variations across categories, with differences of up to 7\% among phases and 37\% among tasks. These disparities suggest that \tool offers valuable insights into the LLMs' performance and areas for potential improvement in the DL domain. 

% Overall, our empirical results demonstrate the utility of \tool as a comprehensive benchmark while offering insights for future improvements across diverse functional categories.