\section{Discussion}\label{sec:discussion}
The Berkeley admissions case is a canonical example in the causal fairness literature. \citet{BickelHO75} reached the conclusion of rejecting unfairness while making unrealistic assumptions of no unobserved confounding. We take the next step of analyzing the Berkeley dataset while allowing for unobserved confounding which results in a different conclusion; since there is very strong evidence that the data satisfies the IV inequalities, the conclusion regarding fairness, from the available data, is that it is undecidable.

While our analysis was centered around the Berkeley case, there are multiple aspects that generalize---a) \ifdefined \SINGLE The family of causal models we consider \else $\modelsedgerelax$ \fi can be thought of as a mediator with a confounder between mediator and outcome, which is common in mediation analysis. b) The approach of fairness notions being causal hypotheses, with respect to the class of models defined by modeling assumptions, that need to be translated into statistical tests to be useful in practice. c) The observation that for the case of inequality constraints on observational data, a straightforward Bayesian testing procedure is available.

\textbf{Selection Bias: } \texttt{UCBAdmissions} dataset only has data from the $6$ largest departments as opposed to $85$ in \cite{BickelHO75}. Also, the fraction of female students is significantly smaller than the fraction of male students. There could possibly be other sources of latent selection that alter the causal model resulting in violating the assumptions of, for instance, absence of bidirected edge in the causal graph between $\sex$ and $\outcome$. Since allowing for selection bias enlarges the model class $\modelsedge$, given that the data satisfies IV inequalities, we conclude that allowing for selection bias will not change our conclusion. We leave a deeper analysis that takes selection bias into account as future work.

% \paragraph{Faithfulness Violations and Statistical Testing: }
% We proved equivalence of tests by equating the sets of observational distributions corresponding to the null hypotheses of the fairness notions at different rungs of the causal hierarchy. Given that the set difference of the fairness notions are models that violate faithfulness, the equivalence in the distribution space   
