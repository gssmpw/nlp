% \section{Fairness Notions}\label{sec:notions}\todo{not mentioned counterfactual notions. Redo this section after tying in the counterfactual fairness and path-dependent counterfactual fairness notions from Kusner et al's paper.}
% \todo{Readability suffers from that you're telling two stories at once, the one with-cf and the one without-cf. Didactically it seems better to first tell the no-cf story (esentially reproducing Bickel et al's story) and then the with-cf story.}
%We define a fairness notion to be a certain condition that is required to be satisfied by a causal model to be deemed fair. These conditions can take the form of observational, interventional, counterfactual or graphical queries on the SCMs in the families of causal models defined by modeling assumptions in Section~\ref{sec:modeling}. 


% \todo{Should we explicitly also discuss ``observational'' fairness notions like demographic parity? Most of the fairness literature looks at those notions... Perhaps all readers will know that the notion of demographic parity may fall prone to Simpson's paradox, but perhaps we need to add that discussion for completeness? Anyhow, this is something that can be added later on, let's focus on the causal fairness notions first.}
% Given the SCM in the previous section, there are multiple fairness notions that we could consider. In this section, we motivate a few fairness notions to consider. \todo{Is this the rationale for us to consider all the notions that we do? - Natural?} We leave the issue of coming up with statistical tests to test each of the notions for Section~\ref{sec:tests}.
\section{Berkeley Case: With Latent Confounding Between Department And Outcome}\label{sec:confounder}
\begin{figure}[!th]
    \centering
    \begin{tikzpicture}
\tikzstyle{vertex}=[circle,fill=none,draw=black,minimum size=17pt,inner sep=0pt]
\node[vertex] (S) at (0,0) {$S$};
\node[vertex] (A) at (2,0) {$A$};
\node[vertex] (D) at (1,1) {$D$};
%\node[vertex] (S') at (1,-0.5) {$S'$};
\path (S) edge[bend left=10] (D);
\path (D) edge[bend left=10] (S);
\path (D) edge[bend left=10] (A);
\path (A) edge[bend left=10] (D);
\path[bidirected] (D) edge[bend left=60] (A);
\path[bidirected] (S) edge[bend left=60] (D);
\path[bidirected] (S) edge[bend right=60] (A);
\path (S) edge[bend left=10] (A);
\path (A) edge[bend left=10] (S);
    \end{tikzpicture}
    \caption{Causal graph of a model without assumptions}
    \label{fig:causal_modeling}
\end{figure}

We now take a more careful causal modeling approach. Instead of starting from variables and reasoning about structural equations that we allow, we start with assuming that all structural equations exist.\footnote{Since this allows for causal cycles, this would require using the framework of simple SCMs \citep{BongersFPM21}.} For the Berkeley example, Figure~\ref{fig:causal_modeling} shows a causal graph of an SCM that we start with. \ifdefined\SINGLE We now provide rationale for ruling out few structural equations. Based on chronology of events, we rule out those where $D$ directly affects $S$, where $A$ directly affects $D$ and where $A$ directly affects $S$. \else Based on chronology of events, we rule out structural equations where $D$ directly affects $S$, where $A$ directly affects $D$ and where $A$ directly affects $S$.\fi We rule out unobserved common causes of $S$ and $D$, and $S$ and $A$ since we model $S$ to be sex at birth. While latent selection bias might introduce bidirected edges \citep{ChenZM24} that are incident on $S$, we assume for now that there is no selection bias in the dataset. The resulting class of SCMs has structural equations of the form
\ifdefined\SINGLE
\begin{align}\label{eq:cf-edge}
    \sex &= f_{\sex}(U_{\sex}) \nonumber, \\
    \dept &= f_{\dept}(\sex,U, U_{\dept}), \\
    \outcome &= f_{\outcome}(\sex,\dept,U,U_{\outcome}), \nonumber
\end{align}
\else 
$\sex = f_{\sex}(U_{\sex}), \dept = f_{\dept}(\sex,U, U_{\dept}), \outcome = f_{\outcome}(\sex,\dept,U,U_{\outcome})$
\fi 
where $U,U_{\sex},U_{\dept}$ and $U_{\outcome}$ denote independent exogenous random variables. 
We define $\modelsedgerelax$ to be the family of models parameterized by the above structural equations and the exogenous distribution. Further, we define $\modelsedge = \left\{ \model \in \modelsedgerelax : \forall s, P_{\model}(S=s) > 0 \right\}$. For $\model \in \modelsedgerelax$ (and $\modelsedge$), the causal graph is a subgraph of the one shown in Figure~\ref{fig:cf-edge}. 

Although we arrived at allowing confounding between department and outcome through a careful causal modeling approach, this is not a novel consideration. In particular, Kruskal \citep[Pg 128-129]{FairleyMosteller77} demonstrated an example where the existence of a latent confounder, such as state of residence, can render \citet{BickelHO75}'s analysis incorrect. Other natural latent confounders include, for example, level of department-specific technical skills that influence both the department choice of an applicant and the admissions outcome. 

%We denote all such latent confounders by a single exogenous random variable, $U$ which is independent of other exogenous random variables. We modify the causal mechanisms of $\dept, \outcome$ in \eqref{eq:no-cf-edge} to include $U$ as input.

%In Section~\ref{app:modeling}, we show how we end up with $\modelsedge$ using a more careful causal modeling approach. 

% From the causal graphs of $\modelsunconfedge$ and $
% \modelsedge$ in Figures~\ref{fig:no-cf-edge} and \ref{fig:cf-edge}, a natural subset of fair causal models are those where the red edge does not exist in the causal graph.\todo{It is important to mention that this coincides with the notion that Pearl proposes in his 2009 book (no direct effect of $\sex$ on $\outcome$ w.r.t.\ $\{\sex,\dept,\outcome\}$).}

% Our first notion of fairness stems from the natural question, "Did the admissions committee consider $\formsex$ as a deciding factor while deciding the outcome?". In the SCM framework, this can be formulated as whether the structural equation for $\outcome$ includes $\formsex$ or equivalently, does the edge $\formsex \rightarrow \outcome$ exist in the causal graph of $\model$, $\cG(\model)$? 

% \noindent\textbf{Graphical Notion}: If $\sex \rightarrow \outcome \notin \cG(\model)$, then $M$ is fair. 

% Denote the class of models that are fair according to the 

% \begin{definition}[Graphical Notion of Fairness]\label{def:graph_fairness}
%      $\model \in \modelsunconfedge$ is fair according to the graphical notion of fairness if it belongs to $\nullgraphunconf \triangleq \left \lbrace \model \in \modelsunconfedge : \sex \rightarrow \outcome \notin \cg{\model} \right \rbrace$. $\model \in \modelsedge$ is fair according to the graphical notion of fairness if it belongs to $\nullgraph \triangleq \left \lbrace \model \in \modelsedge : \formsex \rightarrow \outcome \notin \cg{\model} \right \rbrace$.
% \end{definition}
% \todo{I got a bit lost in redundancies in the notation. $H^0_{no-cf-graph} = \model_{no-cf-no-edge}$ and $H^0_{graph}=\model_{no-edge}$ that were already introduced before.}

Since our modeling assumptions expand the family of SCMs under consideration to $\modelsedge$, the fairness notions that we discussed in the previous section are modified accordingly to obtain null hypothesis sets $ \nullgraph, \nullinter$ and $\nullctrf$. 
\ifdefined\SINGLE
\begin{definition}[Fairness Notions with Confounding]\label{def:notions-cf}
For $\model \in \modelsedge$ the null hypothesis set corresponding to the interventional, counterfactual and graphical notion of fairness are 
\begin{align*}\label{eq:cf-def}
    \nullinter &\triangleq \left \lbrace \model \in \modelsedge: \forall \ldept,\lsex, P_{\model}\Paren{\outcome=1\mid\doop{\sex=\lsex},\doop{\dept=\ldept}} = P_{\model}\Paren{\outcome=1\mid\doop{\dept = \ldept}} \right \rbrace, \\
     \nullctrf &\triangleq \left \lbrace \model \in \modelsedge: \forall \ldept, \lsex, P_M(A^{\doop{S=s,D=d}} = A^{\doop{D=d}}) = 1 \right \rbrace, \\
     \nullgraph&\triangleq \left \lbrace \model \in \modelsedge : \sex \rightarrow \outcome \notin \cg{\model} \right \rbrace.
\end{align*}
\end{definition}
\else
\begin{definition}[Fairness Notions with Confounding]\label{def:notions-cf}
For $\model \in \modelsedge$ the null hypothesis set corresponding to the interventional, counterfactual and graphical notion of fairness are 
\begin{align*}\label{eq:cf-def}
    \nullinter &\triangleq \left \lbrace \model \in \modelsedge: \forall \ldept,\lsex, \right.\\
    &\left. P_{\model}\Paren{\outcome=1\mid\doop{\sex=\lsex},\doop{\dept=\ldept}} \right.\\
    &\left. = P_{\model}\Paren{\outcome=1\mid\doop{\dept = \ldept}} \right \rbrace, \\
     \nullctrf &\triangleq \left \lbrace \model \in \modelsedge: \forall \ldept, \lsex, \right.\\
     &\left. P_M(A^{\doop{S=s,D=d}} = A^{\doop{D=d}}) = 1 \right \rbrace, \\
     \nullgraph &\triangleq \left \lbrace \model \in \modelsedge : \sex \rightarrow \outcome \notin \cg{\model} \right \rbrace.
\end{align*}
\end{definition}
\fi
% \nullobs &\triangleq \left \lbrace \model \in \modelsedge : \forall \ldept, P_{\model}\Paren{\outcome=1 \mid \sex =0, \dept = \ldept} = P_{\model}\Paren{\outcome=1 \mid \sex=1, \dept = \ldept} \right \rbrace, \\
% \nullobs &\triangleq \left \lbrace \model \in \modelsedge : \forall \ldept, P_{\model}\Paren{\outcome=1 \mid \sex =0, \dept = \ldept} \right. \\
%     &\left. = P_{\model}\Paren{\outcome=1 \mid \sex=1,\dept = \ldept} \right \rbrace, \\
While the above notions generalize straightforwardly from the no-confounder setting, this is no longer the case for the observational notion. In addition, while the statistical tests for the no-confounder model are straightforward, this is no longer the case for the aforementioned null hypotheses since $\outcome \not\!\perp\!\!\!\perp \sex \mid \dept$ in general. We first consider the graphical notion of fairness and develop a corresponding statistical test.

\subsection{Graphical Notion and the Instrumental Variable (IV) Inequalities}\label{subsec:graph_iv}

In the presence of latent confounding, graphical queries, such as absence of edges, impose equality or inequality constraints \citep{Evans16, WolfeSF19} in addition to conditional independence constraints which are the only constraints imposed by a DAG. For the Berkeley case with confounding, since the path $\sex \rightarrow \dept \leftrightarrow \outcome$ is open when conditioned on $\dept$, we have $\sex \not\!\perp\!\!\!\perp \outcome \mid \dept$ in general. 
%For example, for $\model \in \nullgraphunconf$, by the global Markov property, the graph implies the conditional independence $\sex \indep \outcome \mid \dept$ for any distribution that is Markov with respect to $\model$ \todo{in particular, for the observational distribution of $\model$ (the other distributions are not relevant to consider here)}.
%However, this no longer holds for the case with confounding since the path $\sex \rightarrow \dept \leftrightarrow \outcome$ is open when conditioned on $\dept$. 
Our test for the graphical notion of fairness for $\modelsedge$ stems from the observation that a model $\model \in \nullgraph$, lies in the instrumental variable (IV) model class $\modeliv$ where $\sex$ is considered the instrument, $\dept$ the treatment, and $\outcome$ the effect. If all modeled endogenous variables are discrete-valued, a necessary condition for the observational distribution\footnote{While we express the IV inequalities as a condition satisfied by the observational distribution, in Section~\ref{app:iv} we reason that they are more appropriately expressed as conditions in terms of $P_{\model}(X,Y \mid \doop{Z})$.} resulting from $\model \in \modeliv$ is to satisfy the IV inequalities \citep{Pearl95}, which in the context of Figure~\ref{fig:iv} are given by  
\begin{equation}\label{eq:iv}
    \max_{x} \sum_{y} \max_{z} P_{\model}\Paren{X=x,Y=y\mid Z=z} \leq 1. 
\end{equation}
% \st{This implies that a violation of the instrumental variable inequalities is evidence of unfairness.} \todo{Not necessarily, violations of the IV model class need not imply that there is an edge from $\sex \rightarrow \outcome$. Can we then not conclude anything from the IV inequality violation? }
Since the IV inequalities are only necessary conditions, an arbitrary distribution on $X,Y,Z$ that satisfies the IV inequality does not necessarily imply that it is an entailed distribution of a model from the IV model class. \citet{Bonet01} showed that for the binary instrument, treatment and effect case, the IV inequalities are also sufficient conditions. In Theorem~\ref{thm:iv_tight}, we show that for the case where the instrument and outcome are binary and the treatment is discrete-valued with finite support, any distribution that satisfies the IV inequality is also entailed by some causal model from the IV model class. To the best of our knowledge, Theorem~\ref{thm:iv_tight} is a novel result. We defer the proof to Section~\ref{app:ivsharp}.
\ifdefined\SINGLE
\begin{restatable}{theorem}{ivtight}\label{thm:iv_tight}
Let $X,Y,Z$ be discrete random variables defined on $\cX,\cY,\cZ$ respectively, with $|\cX| = n\geq 2, |\cY|=2, |\cZ| =2$. Let the set of joint distributions that satisfy the IV inequalities be defined as $\distiv \triangleq \left \lbrace P(X,Y,Z) : P(X,Y \mid Z)\text{ satisfies }\eqref{eq:iv} \text{ and } \forall z, P(Z=z)>0 \right \rbrace$. Define $\distmodeliv \triangleq \left \lbrace P_{\model}(X,Y,Z) : \model \in \modeliv \right \rbrace.$ Then $\distiv = \distmodeliv.$
\end{restatable}
\else
\begin{restatable}{theorem}{ivtight}\label{thm:iv_tight}
Let $X,Y,Z$ be discrete random variables defined on $\cX,\cY,\cZ$ respectively, with $|\cX| = n\geq 2, |\cY|=2, |\cZ| =2$. Define $\distmodeliv \triangleq \left \lbrace P_{\model}(X,Y,Z) : \model \in \modeliv \right \rbrace.$ and the set of joint distributions that satisfy the IV inequalities as \begin{align*}\distiv &\triangleq \left \lbrace P(X,Y,Z) : P(X,Y \mid Z)\text{ satisfies }\eqref{eq:iv} \right.\\
&\left.\text{ and } \forall z, P(Z=z)>0 \right \rbrace.\end{align*}  Then $\distiv = \distmodeliv.$
\end{restatable}
\fi
% We make a few observations that help us understand the implications of Theorem~\ref{thm:iv_tight} for concluding finally concluding about  
For the Berkeley admissions case, assuming for now that the true observational distribution over $\sex,\dept,\outcome$ is known, the observational distribution satisfying the IV inequalities implies that there exists a causal explanation (model) where the directed edge $\sex \rightarrow \outcome$ is absent, i.e., given that $P(\outcome,\dept, \sex) \in \distiv$, there exists $\model \in \modeliv$ such that $P_{\model}(\outcome,\dept,\sex) = P(\outcome,\dept,\sex)$. On the other hand, the observational distribution violating the IV inequalities does not necessarily imply that the edge $\sex \rightarrow \outcome$ is present since the IV model class, $\modeliv$, is only a subset of all the models that do not contain the edge $\sex \rightarrow \outcome$ in the causal graph. For example, the existence of latent confounding between $\sex$ and $\outcome$ in a model $\model$ may result in $\model \notin \modeliv$, even though $\cg{\model}$ does not necessarily contain the directed edge $\sex \rightarrow \outcome$. However, the causal modeling assumption that defined $\modelsedge$ rules out latent confounding between $\sex$ and $\outcome$. Therefore, given our modeling assumptions, $\nullgraph = \modeliv$, and in turn, we conclude that violating the IV inequalities implies that $\model \in \modelsedge \backslash \nullgraph$. As in the previous section, it is possible that causal models that lie outside $\nullgraph$  (``unfair'' models) induce observational distributions that lie in $\distiv$, i.e., satisfy the IV inequalities. Therefore, satisfying the IV inequalities is not conclusive evidence that the data-generating mechanism is fair, i.e., our conclusion should be that fairness is undecidable. In Section~\ref{sec:bayesiantest} we introduce a Bayesian test for the IV inequalities.

\subsection{Bounds on Interventional Notion of Fairness}\label{subsec:bounds}
For $\model \in \modelsedge$, the interventional notion of fairness is the CDE, which is not identifiable in our case.
By a response-function parameterization \citep{Balke95, BalkePearl97} of $\model \in \modelsedge$, we can express the interventional distributions in Definition~\ref{def:notions-cf} as a linear function of response variables. Further, the observational distribution is also expressed as a linear function of the response variables. Using the symbolic linear programming approach of \citet{Balke95}, we obtain upper and lower bounds in terms of the observational distribution, specifically, $P_{\model}\Paren{\outcome,\dept \mid \sex}$. Indeed, \citet{CaiKPT08} express the same bounds which we reproduce below. The CDE given by
\ifdefined\SINGLE
\begin{equation*}P_{\model}\Paren{\outcome=1\mid \doop{\sex=1}, \doop{\dept=\ldept}} - P_{\model}\Paren{\outcome=1\mid \doop{\sex=0}, \doop{\dept=\ldept}},
\end{equation*}
\else
\begin{align*}
&P_{\model}\Paren{\outcome=1\mid \doop{\sex=1}, \doop{\dept=\ldept}}\\
&- P_{\model}\Paren{\outcome=1\mid \doop{\sex=0}, \doop{\dept=\ldept}},
\end{align*}
\fi
lies in the interval
\ifdefined\SINGLE
\begin{align*}
    &\left[ \Pr\left(\outcome=1,\dept=\ldept\mid\sex=1\right) + \Pr\left(\outcome=0,\dept=\ldept\mid \sex=0\right) - 1,\right.\\
    & \left.1 - \Pr\left(\outcome=0,\dept=\ldept\mid\sex=1\right) - \Pr\left(\outcome=1,\dept=\ldept\mid\sex=0\right)\right].
\end{align*}
\else
% \begin{align*}
%     &\left[ \Pr\left(\outcome=1,\dept=\ldept\mid \sex=1\right) \right. \\
%     &\left.+ \Pr\left(\outcome=0,\dept=\ldept\mid\sex=0\right) - 1,\right.\\
%     & \left.1 - \Pr\left(\outcome=0,\dept=\ldept\mid\sex=1\right) \right. \\
%     &\left. -\Pr\left(\outcome=1,\dept=\ldept\mid\sex=0\right)\right].
% \end{align*}
\begin{align*}
    &\left[ \Pr\left(\outcome=1,\ldept\mid \sex=1\right) + \Pr\left(\outcome=0,\ldept\mid\sex=0\right) - 1,\right.\\
    & \left.1 - \Pr\left(\outcome=0,\ldept\mid\sex=1\right) -\Pr\left(\outcome=1,\ldept\mid\sex=0\right)\right].
\end{align*}
\fi 
For the interventional notion of fairness, the CDE must be $0$ for all $\ldept$. By setting the lower bound to be at most $0$ and the upper bound to be at least $0$, we recover the IV inequalities in \eqref{eq:iv}. While \citet{CaiKPT08} do not point out the connection to the IV inequalities, they find it ``remarkable that we [they] get such a simple formula, consisting of only one additive expression in the lower bound and one additive expression in the upper bound". In the next subsection, we show that the connection to the IV inequalities is not a coincidence.   

\subsection{A Family of Equivalent Tests}\label{subsec:equivalence}

% \todo{I'm not sure if this is the best way to present things... you spend 1.5 page on deriving bounds for the interventional H0. But then later on you see that you could have skipped that... Does the reader need to go through the same chronological order of understanding? Or is it better to take a different perspective (that would have saved us work if we would have had it initially)?}

The graphical and interventional fairness notions end up imposing identical constraints on the observational distribution.
%\todo{Is it clear they are actually different? Perhaps people would believe them to be equivalent!}
However, note that $\nullinter \supseteq \nullgraph$. In fact, we prove in Section~\ref{app:nested-cf} that $ \nullgraph = \nullctrf \subset \nullinter$. Models in $\nullinter \backslash \nullgraph$ (Example~\ref{ex:unconfexample}) are such that the edge $\sex \rightarrow \outcome$ exists in the causal graph and yet, the interventional queries in Definition~\ref{def:notions-cf} are equal.
% Therefore, models in $\nullinter \backslash \nullgraph$ violate certain faithfulness assumptions. 
Given that the null hypothesis of the interventional fairness notion is a strict superset of that of the graphical fairness notion, we might expect the same relation to hold in the resulting set of observational distributions for these hypotheses, thus giving us potentially different tests. In contrast, like in Section~\ref{sec:modeling}, we show that the corresponding sets of observational distributions resulting from models in $\nullinter$, $\nullctrf$, $\nullgraph$ are identical. Section~\ref{app:equiv-cf} contains the proof.

\begin{restatable}{theorem}{confequiv}\label{thm:equivalence}
Let 
\begin{align*}
\distgraph &\triangleq \left \lbrace P_{\model}\Paren{\dept,\outcome, \sex} : \model \in \nullgraph \right \rbrace, \\
\distinter &\triangleq \left \lbrace P_{\model}\Paren{\dept,\outcome, \sex} : \model \in \nullinter \right \rbrace, \\
\distctrf &\triangleq \left \lbrace P_{\model}\Paren{\dept,\outcome, \sex} : \model \in \nullctrf \right \rbrace.
\end{align*}
Then $\distinter = \distctrf = \distgraph = \distiv,$ where $\distiv$ is defined in Theorem~\ref{thm:iv_tight}.
\end{restatable}

In summary, testing for the graphical, interventional and counterfactual notions of fairness, with confounding, all boil down to testing the IV inequalities.

\subsection{Comparison With Existing Fairness Notions}

The utility of considering statistical tests is that we can now compare different fairness notions for a particular case with respect to the same causal modeling assumptions. In this section, we consider the three existing counterfactual fairness notions, namely the NDE \citep{NabiShpitser18, Chiappa19}, and the counterfactual and path-dependent counterfactual fairness notions in \citet{KusnerLRS17}. 

For NDE, \citet{KaufmanKMGP05} obtain bounds for the all-binary setting. Using these bounds, we obtain a strictly weaker test than the IV inequalities.\footnote{Intuitively, the reason is the same as in Section 3; the NDE averages over departments, and a positive bias in one department may cancel out against a similarly strong negative bias in another department. Hence, vanishing NDE does not imply that each department takes fair decisions.} We show in Section~\ref{app:ctrfkusner-cf} that the counterfactual notion of fairness of \citet{KusnerLRS17} implies demographic parity even when confounding is allowed. In Section~\ref{app:pathwise-cf} we show that testing the path-dependent counterfactual fairness notion of \citet{KusnerLRS17} is equivalent to testing the IV inequalities.

% and only provide a proof sketch here. We first show that $\nullinter \supseteq \nullctrf \superseteq \nullgraph$ which implies the same relations for $\distinter, \distctrf, \distgraph$. We express the observational distributions in $\distinter$ in terms of the response-function parameterization and show that 

% \begin{equation}\Pr\left(\outcome=1,\dept=\ldept|\sex=1\right) + \Pr\left(\outcome=0,\dept=\ldept|\sex=0\right) - 1 \leq 0,$$

% $$1 - \Pr\left(\outcome=0,\dept=\ldept|\sex=1\right) - \Pr\left(\outcome=1,\dept=\ldept|\sex=0\right) \geq 0.
% $$

% The above conditions, for all $\ldept$, are the same as the IV inequalities in \eqref{eq:iv}. As we shall see in the next subsection, this is not coincidence. 



% \begin{equation}\label{eq:inter_resp}
%         \Pr\Paren{\outcome=\loutcome\mid \doop{\formsex=\lsex'}, \doop{\dept=\ldept}} = \sum\limits_{\Paren{\respfunc_1,\respfunc_2}  \in \cX_{\response}}\bm{1}\Brack{\respfunc_2\Paren{\lsex',\ldept}=\loutcome}\tilde{P}\Paren{\respfunc_1,\respfunc_2}.
%    \Pr\Paren{\outcome=\loutcome\mid \dept=\ldept, \sex = \lsex, \doop{\formsex=\lsex'}} &= \sum\limits_{\Paren{\respfunc_1,\respfunc_2}  \in \cX_{\response}}\bm{1}\Brack{\respfunc_2\Paren{\lsex',\ldept}=\loutcome, \respfunc_1\Paren{\lsex} = \ldept}\tilde{P}\Paren{\respfunc_1,\respfunc_2}.\label{eq:cond_resp}
%\end{equation}

% Likewise, the conditional distribution $P_{\model}\Paren{\outcome,\dept|\sex}$ can also be expressed as a linear function of $\tilde{P}$, 

% \begin{equation}\label{eq:obs_resp}
%     P_{\model}\Paren{\outcome=\loutcome,\dept=\ldept|\sex=\lsex} = \sum\limits_{\Paren{\respfunc_1,\respfunc_2}  \in \cX_{\response}}\bm{1}\Brack{\respfunc_2\Paren{\lsex,\ldept}=\loutcome, \respfunc_1\Paren{\lsex} = \ldept}\tilde{P}\Paren{\respfunc_1,\respfunc_2}.
% \end{equation}

% Similar to \cite{Balke95}, we setup a symbolic linear program that derives upper and lower bounds in terms of the observational distribution $P_{\model}\Paren{\outcome,\dept \mid \sex}$ by maximizing and minimizing \eqref{eq:inter_resp} subject to symbolic constraints obtained from \eqref{eq:obs_resp}. These bounds can be used to bound the difference between the interventional queries in \eqref{eq:interfair}. The code to obtain these bounds is in the supplementary material.

% For the conditional notion of fairness, we first rewrite $\Pr\Paren{\outcome|\dept, \sex, \doop{\formsex}} = \frac{\Pr\Paren{\outcome, \dept| \sex, \doop{\formsex}}}{\Pr\Paren{\dept| \sex, \doop{\formsex}}}$. Note that the denominator, $\Pr\Paren{\dept| \sex, \doop{\formsex}}$ is identifiable and equal to $\Pr\Paren{\dept| \sex}$. The numerator can be expressed in terms of a linear function of $\tilde{P}$, 

% \begin{equation}\label{eq:cond_resp}
%        \Pr\Paren{\outcome=\loutcome, \dept=\ldept \mid \sex = \lsex, \doop{\formsex=\lsex'}} = \sum\limits_{\Paren{\respfunc_1,\respfunc_2}  \in \cX_{\response}}\bm{1}\Brack{\respfunc_2\Paren{\lsex',\ldept}=\loutcome, \respfunc_1\Paren{\lsex} = \ldept}\tilde{P}\Paren{\respfunc_1,\respfunc_2}.
% \end{equation}

% With a similar linear program setup, we obtain bounds on the difference of the interventional queries in \eqref{eq:cndfair}. In the next section we see how the IV inequalities and the bounds obtained in this subsection can be turned into a statistical test.

% For the interventional notion of fairness we express the bounds in closed form below. Due to space constraints, for the conditional notion of fairness the bounds are in the supplementary material. The bounds on $\Pr\Paren{\outcome=\loutcome\mid \doop{\formsex=\lsex'}, \doop{\dept=\ldept}}$ are 






% As we shall see in Section~\ref{sec:tests}, testing the graphical notion of fairness poses several challenges. These stem from the difficulty of obtaining constraints that a graphical query imposes on the set of observational distributions since a particular observational distribution can arise from multiple SCMs where the form of

% Since, it is easier to test fairness notions based on observational, interventional or counterfactual queries, we define fairness notions based on these probabilistic queries below. Another possible advantage of defining fairness notions based on probabilistic queries is the ability to analyze them under selection bias. 

% \begin{definition}[Interventional Notion of Fairness]
%     $\model \in \modelsunconfedge$ is fair according to the interventional notion of fairness if it belongs to 
%      \begin{equation}\label{eq:interfairunconf}
%     \nullinterunconf \triangleq \left \lbrace \model \in \modelsunconfedge: \forall \ldept, \Pr\Paren{\outcome=1|\doop{\sex=1},\doop{\dept=\ldept}} = \Pr\Paren{\outcome=1|\doop{\sex=0},\doop{\dept = \ldept}} \right \rbrace.
%     \end{equation}
%     $\model \in \modelsedge$ is fair according to the interventional notion of fairness if it belongs to
%     \begin{equation}\label{eq:interfair}
%     \nullinter \triangleq \left \lbrace \model \in \modelsedge : \forall \ldept, \Pr\Paren{\outcome=1|\doop{\formsex=1},\doop{\dept=\ldept}} = \Pr\Paren{\outcome=1|\doop{\formsex=0},\doop{\dept = \ldept}} \right \rbrace.
%     \end{equation}
%\end{definition}
% \todo{Notation: Pr() or P()? The latter was introduced earlier on as referring to the observational distribution of the SCM, so better to stick to that. And personally I usually write $P_M$ to emphasize the parameter(SCM)-dependence.}
% \todo{I suggest strengthening these to $P(A|do(S),do(D)) = P(A|do(D))$ (and $S'$ instead of $S$, respectively).}

% The criteria for the interventional notion of fairness for $\modelsunconfedge$ is that the controlled direct effect (CDE) \cite{?} of the ``treatment", $\sex$, on the ``outcome", $\outcome$, is $0$ for every value of the mediator, i.e., every department choice $\ldept$. We compare this notion with the counterfactual notion of natural direct effect (NDE) \cite{Pearl01} in the appendix. Although the NDE is identifiable for models in $\modelsunconfedge$, we show that since the expression of the NDE is a weighted average over the department choice, it is possible for the NDE to be $0$ despite there being discrimination \todo{'discrimination' is yet undefined, replace by specific fairness notion}.  

% If confounding between $\dept$ and $\outcome$ is allowed, the CDE and NDE are both unidentifiable. While bounds on the CDE and NDE are known \cite{KaufmanKMGP05, CaiKPT08}, \todo{sentence stops prematurely?} For $\modelsedge$, we now see the utility of modeling the reported sex on the application form separately. While interventions on $\sex$ are hypothetical and come with issues of interpretation \cite{HuKohlerHausmann20}, interventions on the reported sex, $\formsex$, are practically conceivable. Further, by disentangling the reported sex from the birth sex, we no longer require relying on counterfactual queries such as NDE.   

% In \cite{PearlMackenzie18}, Pearl mentions a disadvantage of CDE, namely intervening on department choice. An applicant's preparation might be targeted to a particular department of their choosing and this would influence the admissions committee's decision largely.\todo{I don't see how this would be problematic?} To circumvent this issue, we define a conditional notion of fairness where we condition on $\sex, \dept$ in addition to intervening on $\formsex$.

% \begin{definition}[Conditional Notion of Fairness]
%     $\model \in \modelsedge$ is fair according to the conditional notion of fairness if it belongs to 
%     \begin{equation}\label{eq:cndfair}
%     \nullcond \triangleq \left \lbrace \model \in \modelsedge : \forall \ldept, \lsex, \Pr\Paren{\outcome=1|\dept = \ldept, \sex = \lsex, \doop{\formsex=1}} = \Pr\Paren{\outcome=1|\dept = \ldept, \sex = \lsex, \doop{\formsex=0}} \right \rbrace.
%     \end{equation}
% \end{definition}

% The conditional notion of fairness is interpreted as mandating that for every sex-department combination, the applicants of a particular sex who apply to a certain department of their choice would have the same probability of admission if their sex on the application form were changed. Notice that despite being an interventional query, $\Pr\Paren{\outcome=1|\dept = \ldept, \sex = \lsex, \doop{\formsex\neq \lsex}}$ has a counterfactual `flavor'.  Further, despite conditioning on $\dept$, we escape the pitfalls that come along with it. We can allow for latent confounding, too, since by also conditioning on $\sex$, there are no longer any open path from $\formsex$ to $\outcome$ with $\dept$ as the collider. Note that this is made possible because of the node-splitting modeling trick.\todo{From my perspective, the conditional version was only relevant when dealing when selection bias, so can be left out of this story. Instead, we should define the counterfactual notion in order to relate to Kushner et al.}


