\documentclass[11pt]{article}
\def\withcolors{1}
\def\withnotes{1}
\def\withbiblioconversion{1} % long names for the journals
\usepackage[round]{natbib}
\renewcommand{\bibname}{References}
\renewcommand{\bibsection}{\subsubsection*{\bibname}}
\usepackage{amsmath,amsthm,enumitem,nicefrac,bbm,verbatim,amssymb,amsfonts,amscd, graphicx,algpseudocode,hyperref,float,mathtools,bm,xcolor,appendix,subcaption}
\usepackage{tikz}
\usepackage{tikz-3dplot}
\usetikzlibrary{positioning, decorations.pathreplacing, calc, intersections, pgfplots.groupplots}
\usetikzlibrary{positioning, decorations.markings}
\usepackage{pgfplots}
\usepackage{tikz-network}
\usetikzlibrary{shapes,decorations,arrows,calc,arrows.meta,fit,positioning}
\usepackage{thmtools}
\usepackage{thm-restate}
\pgfplotsset{compat=1.5}
\pgfplotsset{scaled x ticks=false}
\tikzset{
	-Latex,auto,node distance =1 cm and 1 cm,semithick,
	state/.style ={ellipse, draw, minimum width = 0.7 cm},
	point/.style = {circle, draw, inner sep=0.04cm,fill,node contents={}},
	bidirected/.style={Latex-Latex},
	el/.style = {inner sep=2pt, align=left, sloped}
}
\usepackage[utf8]{inputenc}
\usepackage[T1]{fontenc}
\usepackage[ruled, lined, linesnumbered, commentsnumbered, longend]{algorithm2e}

% Theorem-like environments

\newtheorem{Theorem}{Theorem}
\newtheorem{Theorem*}{Theorem}

\newtheorem{Claim}[Theorem]{Claim}
\newtheorem{Claim*}[Theorem]{Claim}
\newtheorem{Corollary}[Theorem]{Corollary}
\newtheorem{Conjecture}[Theorem]{Conjecture}
\newtheorem{CounterExample*}{$\overline{\hbox{\bf Example}}$}
\newtheorem{Definition}[Theorem]{Definition}
\newtheorem{Example}[Theorem]{Example}
\newtheorem{Example*}[Theorem]{Example}
\newtheorem{Exercise}[Theorem]{Exercise}
\newtheorem{Intuition*}[Theorem]{Intuition}
\newtheorem{Joke*}[Theorem]{Joke}
\newtheorem{Lemma}[Theorem]{Lemma}
\newtheorem{Lemma*}[Theorem]{Lemma}
\newtheorem{Open problem}[Theorem]{Open problem}
\newtheorem{Proposition}[Theorem]{Proposition}
\newtheorem{Property}[Theorem]{Property}
\newtheorem{Question}[Theorem]{Question}
\newtheorem{Question*}[Theorem]{Question}
\newtheorem{Remark}[Theorem]{Remark}
\newtheorem{Result}[Theorem]{Result}
\newtheorem{Fact}[Theorem]{Fact}
\newtheorem{Condition}[Theorem]{Condition}

\usepackage{fullpage}

\newtheorem{theorem}{Theorem}
\newtheorem{proposition}[theorem]{Proposition}
\newtheorem{corollary}[theorem]{Corollary}
\newtheorem{assumption}[theorem]{Assumption}
\newtheorem{lemma}[theorem]{Lemma}
\newtheorem{conjecture}[theorem]{Conjecture}
\newtheorem{example}[theorem]{Example}
\newtheorem{definition}[theorem]{Definition}


\newcommand{\prob}{{\rm Pr}}

% Begin / End Theorems

\def \bthm#1{\begin{#1}\upshape\quad}
\def \ethm#1{\rqed\end{#1}}
\def \ethmp#1{\end{#1}}     % No box, when ends in displayed equation

\def \bClaim     {\bthm{Claim}}
\def \eClaim     {\ethm{Claim}}
\def \eClaimp    {\ethmp{Claim}}
\def \bConjecture {\bthm{Conjecture}}
\def \eConjecture {\ethm{Conjecture}}
\def \eConjecturep{\ethmp{Conjecture}}
\def \bCorollary {\bthm{Corollary}}
\def \eCorollary {\ethm{Corollary}}
\def \eCorollaryp{\ethmp{Corollary}}
\def \bCounter   {\bthm{CounterExample*}}
\def \eCounter   {\ethm{CounterExample*}}
\def \eCounterp  {\ethmp{CounterExample}}
\def \bDefinition{\bthm{Definition}}
\def \eDefinition{\ethm{Definition}}
\def \eDefinitionp{\ethmp{Definition}}
\def \bExample   {\bthm{Example}}
\def \eExample   {\ethm{Example}}
\def \eExamplep  {\ethmp{Example}}
\def \bExercise  {\bthm{Exercise}}
\def \eExercise  {\ethm{Exercise}}
\def \eExercisep {\ethmp{Exercise}}
\def \bIntuition {\bthm{Intuition}}
\def \eIntuition {\ethmp{Intuition}}
\def \bJoke      {\bthm{Joke}}
\def \eJoke      {\ethm{Joke}}
\def \bLemma     {\bthm{Lemma}}
\def \eLemma     {\ethm{Lemma}}
\def \eLemmap    {\ethmp{Lemma}}
\def \bOpen      {\bthm{Open problem}}
\def \eOpen      {\ethm{Open problem}}
\def \eOpenp     {\ethmp{Open problem}}
\def \bProperty  {\bthm{Property}}
\def \eProperty  {\ethm{Property}}
\def \ePropertyp {\ethmp{Property}}
\def \bQuestion  {\bthm{Question}}
\def \eQuestion  {\ethm{Question}}
%\def \bRemark    {\bthm{Remark}}
%\def \eRemark    {\ethm{Remark}}
\def \bResult    {\bthm{Result}}
\def \eResult    {\ethm{Result}}
\def \bTheorem   {\bthm{Theorem}}
\def \eTheorem   {\ethm{Theorem}}
\def \eTheoremp  {\ethmp{Theorem}}

\def \bSubexa    {\begin{subexa}}
\def \eSubexa    {\ethm{subexa}}
\def \eSubexap   {\ethmp{subexa}}

% Environments

\usepackage[colorinlistoftodos,textsize=scriptsize]{todonotes}


\newenvironment{Problem}{\textbf{Problem}\\ \begin{enumerate}}{\end{enumerate}}
\newenvironment{Problems}{\textbf{Problems}\\ \begin{enumerate}}{\end{enumerate}}

%\newenvironment{Problems}{\begin{trivlist}\item[]{\textbf{Problems}}{\end{trivlist}}}

% Headers

\def \skpbld#1{\par\noindent\textbf{#1}\quad}
\def \skpblds#1{\skpbld{#1}\par\noindent}

\def \Answer   {\skpbld{Answer}}
\def \Answers  {\skpblds{Answers}}
\def \Basis    {\skpbld{Basis}}
\def \Check    {\skpbld{Check}}
%\def \Example  {\skpbld{Example}}
\def \Examples {\skpbld{Examples}}
\def \Intuition{\skpbld{Intuition}}
\def \Method   {\skpbld{Method}}
\def \Methods  {\skpblds{Methods}}
\def \Outline  {\skpbld{Outline}}
\def \Proof    {\skpbld{Proof}}
\def \Proofs   {\skpblds{Proofs}}
%\def \Problem  {\skpbld{Problem}}
%\def \Problems {\skpblds{Problems}}
\def \Proline  {\skpbld{Proof Outline}}
%\def \Remark   {\skpbld{Remark}}
\def \Remarks  {\skpblds{Remarks}}
\def \Solution {\skpbld{Solution}}
\def \Solutions{\skpblds{Solutions}}
\def \Verify   {\skpbld{Verify}}
\def \Step     {\skpbld{Step}}

% Ignores

\newcommand{\ignore}[1]{}

% Problems & solutions

\newcommand{\problem}[1]{#1}
\newcommand{\solution}[1]{{\\ \bf Solution}\quad #1}
%\newcommand{\solution}[1]{\mbox{}\\ \medskip\noindent{\bf Solution\medskip}#1}
%\newcommand{\solution}[1]{}

\newcommand{\source}[1]{(Taken from~\cite{#1})}
%\newcommand{\source}[1]{}

\newcommand{\modifiedfrom}[1]{(Modified from~\cite{#1})}
%\newcommand{\modifiedfrom}[1]{}

\newcommand{\takenfrom}[1]{(Taken from~\cite{#1})}
%\newcommand{\takenfrom}[1]{}

\newcommand{\reportedby}[1]{Reported by #1.}
%\newcommand{\reportedby}[1]{}

\newcommand{\hint}[1]{Hint: #1}
%\newcommand{\hint}[1]{}

\newcommand{\joke}[1]{\footnote{#1}}
\newcommand{\trivia}[1]{\footnote{#1}}

% Equation formatting

\newcommand{\spreqn}[1]{{\qquad\text{#1}\qquad}}

% Blackboard fonts
\newcommand{\II}{\mathbb{I}}
\newcommand{\EE}{\mathbb{E}}
\newcommand{\CC}{\mathbb{C}}
\newcommand{\NN}{\mathbb{N}}
\newcommand{\QQ}{\mathbb{Q}}
\newcommand{\RR}{\mathbb{R}}
\newcommand{\ZZ}{\mathbb{Z}}
\newcommand{\PP}{\mathbb{P}}
\newcommand{\MM}{\mathbb{M}}


% Number sets

\newcommand{\complex}{\CC}
\newcommand{\integers}{\ZZ}
\newcommand{\naturals}{\NN}
\newcommand{\positives}{\PP}
\newcommand{\rationals}{\QQ}
\newcommand{\reals}{\RR}

\newcommand{\realsge}{{\reals_{\ge}}}
\newcommand{\realsp}{\reals^+}
\newcommand{\integersp}{\integers^+}
\newcommand{\integerss}[1]{\integers_{\ge{#1}}}

% boldface

\def \ba     {{\bf a}}
\def \bx     {{\bf x}}
\def \by     {{\bf y}}

\def \bA     {{\bf A}}
\def \bB     {{\bf B}}
\def \bC     {{\bf C}}
\def \bD     {{\bf D}}
\def \bF     {{\bf F}}
\def \bG     {{\bf G}}
\def \bL     {{\bf L}}
\def \bQ     {{\bf Q}}
\def \bR     {{\bf R}}
\def \bS     {{\bf S}}
\def \bT     {{\bf T}}
\def \bX     {{\bf X}}
\def \bY     {{\bf Y}}
\def \bZ     {{\bf Z}}

% caligraphics

\def \cA     {{\cal A}}
\def \cB     {{\cal B}}
\def \cC     {{\cal C}}
\def \cD     {{\cal D}}
\def \cE     {{\cal E}}
\def \cF     {{\cal F}}
\def \cG     {{\cal G}}
\def \cH     {{\cal H}}
\def \cI     {{\cal I}}
\def \cK     {{\cal K}}
\def \cL     {{\cal L}}
\def \cM     {{\cal M}}
\def \cN     {{\cal N}}
\def \cO     {{\cal O}}
\def \cP     {{\cal P}}
\def \cQ     {{\cal Q}}
\def \cR     {{\cal R}}
\def \cS     {{\cal S}}
\def \cT     {{\cal T}}
\def \cU     {{\cal U}}
\def \cV     {{\cal V}}
\def \cW     {{\cal W}}
\def \cX     {{\cal X}}
\def \cY     {{\cal Y}}
\def \cZ     {{\cal Z}}

% vectors

\def \vct#1{{\overline{#1}}}

\def \vcta  {{\vct a}}
\def \vctb  {{\vct b}}
\def \vctq  {{\vct q}}
\def \vcts  {{\vct s}}
\def \vctu  {{\vct u}}
\def \vctv  {{\vct v}}
\def \vctx  {{\vct x}}
\def \vcty  {{\vct y}}
\def \vctz  {{\vct z}}
\def \vctp  {{\vct p}}

\def \vctV  {{\vct V}}
\def \vctX  {{\vct X}}
\def \vctY  {{\vct Y}}
\def \vctZ  {{\vct Z}}

\def \vctbeta  {{\vct\beta}}

% Random variables

\newcommand{\rnd}[1]{{\boldsymbol #1}}

\newcommand{\rndd}{{\rnd d}}
\newcommand{\rndl}{{\rnd l}}
\newcommand{\rndp}{{\rnd p}}
\newcommand{\rnds}{{\rnd s}}
\newcommand{\rndx}{{\rnd x}}
\newcommand{\rndy}{{\rnd y}}
\newcommand{\rndz}{{\rnd z}}

%\newcommand{\poi}{{\rm poi}}
\newcommand{\Var}{{\rm Var}}

% following should not exist (indicate random sets)

\newcommand{\rndR}{{\rnd R}}
\newcommand{\rndT}{{\rnd T}}

% arrow vectors

\newcommand{\rvc}{{\overrightarrow c}}

% Abbreviations - \xspace puts space iff there is space after command

\newcommand{\eg}{\textit{e.g.,}\xspace}
\newcommand{\ie}{\textit{i.e.,}\xspace}  % note that overridden in spanish
%\newcommand{\iid}{\textit{i.i.d.}\xspace} 
\newcommand{\iid}{\textit{i.i.d.}} % Edit by Theertha, \xspace not compiling

\newcommand{\etc}{etc.\@\xspace}

% Colors

%\definecolor{light}{gray}{.75}


% marginal notes

\newcommand{\mrgs}[2]{\# #1 \#{\marginpar{#2}}}

\newcommand{\mrgcgd}[1]{\mrgs{#1}{CHANGED}}
\newcommand{\mrgchk}[1]{\mrgs{#1}{CHECK}}
\newcommand{\mrgfix}[1]{\mrgs{#1}{FIX}}
\newcommand{\mrgnew}[1]{\mrgs{#1}{NEW}}

% qed's --  Also consider \qedhere

\def \eqed    {\eqno{\qed}}
\def \rqed    {\hbox{}~\hfill~$\qed$}

% sequences

\def \upto  {{,}\ldots{,}}

\def \zn    {0\upto n}
\def \znmo  {0\upto n-1}
\def \znpo  {0\upto n+1}
\def \ztnmo {0\upto 2^n-1}
\def \ok    {1\upto k}
\def \on    {1\upto n}
\def \onmo  {1\upto n-1}
\def \onpo  {1\upto n+1}

% sets

\def \sets#1{{\{#1\}}}
\def \Sets#1{{\left\{#1\right\}}}

\def \set#1#2{{\sets{{#1}\upto{#2}}}}

\def \setpmo   {\sets{\pm 1}}
\def \setmpo   {\{-1{,}1\}}
\def \setzo    {\{0{,}1\}}
\def \setzn    {\{\zn\}}
\def \setznmo  {\{\znmo\}}
\def \setztnmo {\{\ztnmo\}}
\def \setok    {\{\ok\}}
\def \seton    {\{\on\}}
\def \setonmo  {\{\onmo\}}
\def \setzon   {\setzo^n}
\def \setzos   {\setzo^*}

\def \inseg#1{{[#1]}} % use \intsgm instead
\newcommand{\intsgm}[1]{{[#1]}}

% functions

\def \ord    {\#}

\def \suml   {\sum\limits}
\def \prodl  {\prod\limits}

% Set operations




\newcommand{\unionl}{\union\limits}
\newcommand{\Unionl}{\Union\limits}
 

\newcommand{\interl}{\inter\limits}
\newcommand{\Interl}{\Inter\limits}

% Floors, Ceilings, Absolute value

\def \ceil#1{{\lceil{#1}\rceil}}
\def \Ceil#1{{\left\lceil{#1}\right\rceil}}
\def \floor#1{{\lfloor{#1}\rfloor}}
\def \Floor#1{{\left\lfloor{#1}\right\rfloor}}
\def \absvlu#1{{|#1|}}
\def \Absvlu#1{{\left|{#1}\right|}}

% Parentheses, brackets

\def \paren#1{{({#1})}}
\def \Paren#1{{\left({#1}\right)}}
\def \brack#1{{[{#1}]}}
\def \Brack#1{{\left[{#1}\right]}}

%\def \frac#1#2{{{#1}\over{#2}}}
\def \frc#1#2{{\frac{#1}{#2}}}

\def \binomial#1#2{{{#1}\choose{#2}}}

\def \gcd#1#2{{{\rm gcd}\paren{{#1},{#2}}}}
\def \Gcd#1#2{{{\rm gcd}\Paren{{#1},{#2}}}}

\def \lcm#1#2{{{\rm lcm}\paren{{#1},{#2}}}}
\def \Lcm#1#2{{{\rm lcm}\Paren{{#1},{#2}}}}

% number theory

\newcommand{\base}[2]{{[#1]_{#2}}}

% equalities

\newcommand{\ed}{\stackrel{\mathrm{def}}{=}}

\newcommand{\cnvprb}{\stackrel{\mathrm{p}}{\to}}
\newcommand{\cnvdst}{\stackrel{\mathrm{d}}{\to}}
\newcommand{\cnvas}{\stackrel{\mathrm{a.s.}}{\to}}

\def \gap    {\ \hbox{\raisebox{-.6ex}{$\stackrel{\textstyle>}{\sim}$}}\ }

\newcommand{\al}[1]{\stackrel{\mathit{{#1}}}{<}}
\newcommand{\ag}[1]{\stackrel{\mathit{{#1}}}{>}}
\newcommand{\ale}[1]{\stackrel{\mathit{{#1}}}{\le}}
\newcommand{\age}[1]{\stackrel{\mathit{{#1}}}{\ge}}
\newcommand{\aeq}[1]{\stackrel{\mathit{{#1}}}{=}}

%\newcommand{\eae}{\approx} % exponentially asymptotically eq - replace by \ere
%\newcommand{\eal}{\stackrel<\approx} % replace by \erl

\newcommand{\re}{\sim}
\newcommand{\rle}{\stackrel<\sim} % roughly (asymptotically taken) <=
%\newcommand{\rl}{\stackrel<\sim}
\newcommand{\rge}{\stackrel>\sim}
%\newcommand{\rg}{\stackrel>\sim}

\newcommand{\ere}{\approx} % exponentially roughly equal
\newcommand{\erle}{\stackrel<\approx}
%\newcommand{\erl}{\stackrel<\approx}
\newcommand{\erge}{\stackrel>\approx}
%\newcommand{\erg}{\stackrel>\approx}

% Relations

\newcommand{\independent}{\perp\!\!\!\perp}

% values

\def \half    {{\frac12}}
\def \quarter {{\frac14}}
\def \oo#1{{\frac1{#1}}}

% notation

\def \iff    {{\it iff }}
\def \th     {{\rm th }}

% ignore

\def\ignore#1{}

% Logic

\newcommand{\ra}{\rightarrow}
\def \contra {{\leftrightarrows}}
\newcommand{\ol}[1]{{\overline{#1}}}
%\def \ob {\overline}

\def \ve {{\lor}} % needed?
\def \eq {{\equiv}} % needed?

% spaces

\newcommand{\spcin}{\hspace{1.0in}}
\newcommand{\spchin}{\hspace{.5in}}

% Default text appears in regular print in both book and class versions. 
% There are two types of text that need to be highlighted in class:
% clson - not mentioned in book version (eg jokes)
% clsbk - regular text in book (the parts that need be said)

%For book:
%\newcommand{\clson}[1]{}
%\newcommand{\clsbk}[1]{#1}
%For class:
\newcommand{\clson}[1]{\colorbox{light}{{#1}}}
\newcommand{\clsbk}[1]{\colorbox{light}{{#1}}}

%\newcommand{\bi}{\begin{aopl}}
%\newcommand{\ei}{\end{aopl}}

\newcommand{\bi}{\begin{itemize}}
\newcommand{\ei}{\end{itemize}}
%\newcommand{\bq}{\begin{quote}}
%\newcommand{\eq}{\end{quote}}

\newenvironment{aopl}
  {\begin{list}{}{\setlength{\itemsep}{4pt plus 2pt minus 2pt}}}
  {\end{list}}

% operators

\newcommand{\flnpwr}[2]{{#1^{\underline{#2}}}} % Falling power

\def\orpro{\mathop{\mathchoice
   {\vee\kern-.49em\raise.7ex\hbox{$\cdot$}\kern.4em}
   {\vee\kern-.45em\raise.63ex\hbox{$\cdot$}\kern.2em}
   {\vee\kern-.4em\raise.3ex\hbox{$\cdot$}\kern.1em}
   {\vee\kern-.35em\raise2.2ex\hbox{$\cdot$}\kern.1em}}\limits}

\def\andpro{\mathop{\mathchoice
 {\wedge\kern-.46em\lower.69ex\hbox{$\cdot$}\kern.3em}
 {\wedge\kern-.46em\lower.58ex\hbox{$\cdot$}\kern.25em}
 {\wedge\kern-.38em\lower.5ex\hbox{$\cdot$}\kern.1em}
 {\wedge\kern-.3em\lower.5ex\hbox{$\cdot$}\kern.1em}}\limits}

\def\inter{\mathop{\mathchoice
   {\cap}
   {\cap}
   {\cap}
   {\cap}}}
\def\interl {\inter\limits}

\def\carpro{\mathop{\mathchoice
   {\times}
   {\times}
   {\times}
   {\times}}\limits}

\def\simge{\mathrel{%
   \rlap{\raise 0.511ex \hbox{$>$}}{\lower 0.511ex \hbox{$\sim$}}}}

\def\simle{\mathrel{
   \rlap{\raise 0.511ex \hbox{$<$}}{\lower 0.511ex \hbox{$\sim$}}}}

\newcommand{\pfx}{{\prec}}
\newcommand{\pfxeq}{{\preceq}}

\newcommand{\sfx}{{\succ}}
\newcommand{\sfxeq}{{\succeq}}

% for old documents

\newcommand{\twlrm}{\fontsize{12}{14pt}\normalfont\rmfamily}
\newcommand{\tenrm}{\fontsize{10}{12pt}\normalfont\rmfamily}

% picture macros

\newcommand{\grid}[2]{				% grid{width}{height}
  \multiput(0,0)(1,0){#1}{\line(0,1){#2}}
  \put(#1,0){\line(0,1){#2}}
  \multiput(0,0)(0,1){#2}{\line(1,0){#1}}
  \put(0,#2){\line(1,0){#1}}
}

%% \email{} command
\providecommand{\email}[1]{\href{mailto:#1}{\nolinkurl{#1}\xspace}}

%probabilities
\newcommand{\pr}{{p}} % phase out in favor of \prb
\newcommand{\prb}{{p}} 

\newcommand{\prbs}[1]{{\prb^{(#1)}}}

\newcommand{\Prb}{\text{Pr}}
%\newcommand{\Pr}{{\text{Pr}}} % phase out in favor of \Prb
\newcommand{\Prn}{\text{Pr}}  % phase out in favor of \Prb

\newcommand{\Prs}[1]{\Pr\Paren{#1}} % phase out in favor of \Prbs
\newcommand{\Prbs}[1]{{\Prb^{(#1)}}} % Check if used - phase out
\newcommand{\probof}[1]{\Pr\Paren{#1}}


% probability
\newcommand{\expectation}[1]{\EE\left[#1\right]}
\newcommand{\variance}[1]{Var\left(#1\right)}

%\newcommand{\qedsymbol}{\ensuremath{\blacksquare}}

%\newcommand{\qed}{\hfill\ensuremath{\blacksquare}}%


% inductiion macros

%\newcommand{\induction}[3]{
%Inductive statement: {#1}\\
%We prove: {#2}\\
%Basis: {#3}\\
%Step: {#4}\\
%Proof of step: {#5}
%}

% Frequently used symbols
\newcommand{\perturb}{\gamma}
\newcommand{\dims}{d}
\newcommand{\zdims}{k}
\newcommand{\nsamps}{m}
\newcommand{\nb}{t}
\newcommand{\nspu}{m}
\newcommand{\nin}{r}
\newcommand{\nms}{{N_S}}
\newcommand{\nbb}{{t'}}
\newcommand{\nbbb}{{t''}}
\newcommand{\gbit}{s}
\newcommand{\unif}{{\mathbf{u}}}
\newcommand{\ngr}{{n_0}}
\newcommand{\DiffS}{{\Delta_S}}
\newcommand{\epsnew}{{\eps_0}}
\newcommand{\alphanew}{{\alpha_0}}
\newcommand{\rank}{{\text{rank}}}

\newcommand{\Proj}{{\Pi}}
\newcommand{\povmset}{{\mathfrak{M}}}
\newcommand{\nqubits}{{N}}
\newcommand{\pauliI}{{\sigma_I}}
\newcommand{\pauliX}{{\sigma_X}}
\newcommand{\pauliY}{{\sigma_Y}}
\newcommand{\pauliZ}{{\sigma_Z}}
\newcommand{\pauliObsSet}{{\mathcal{P}}}
\newcommand{\prPauli}[2]{{\p_{#2}(#1)}}

\newcommand{\opnorm}[1]{{\left\|#1\right\|}_{\text{op}}}
\newcommand{\tracenorm}[1]{{\left\|#1\right\|}_{1}}
\newcommand{\hsnorm}[1]{{\left\|#1\right\|}_{\text{HS}}}
\newcommand{\barDelta}{{\overline{\Delta}}}
\newcommand{\ptb}{{z}}
\newcommand{\ptbDistr}{{\mathcal{D}_{\ell,\cd}}}

\newcommand{\supparen}[1]{^{(#1)}}
\newcommand{\subparen}[1]{_{(#1)}}

% Constants
\newcommand{\cd}{{c}}
\newcommand{\cop}{\kappa}


\newcommand{\isthestate}{\texttt{YES}}
\newcommand{\notthestate}{\texttt{NO}}


\newcommand{\cA}{\mathcal{A}}

% Font
\newcommand{\tst}{t}  % Time: to be used when not for a sum index (e.g., 
%"at time $tst$")
\newcommand{\ts}{r} % Time step: index. To be used  for a sum index (e.g., 
%"$\sum_{\ts=1}^\ts")

\newcommand{\etaa}{\gamma}
%\newcommand{\etab}{\eta}
\newcommand{\signA}{{\bf{}1}^\star}
\newcommand{\signB}{{\bf{}0}^\star}
\newcommand{\setS}{S}
\newcommand{\vecu}{u}
\newcommand{\zero}{\mathbf{0}}

\newcommand{\chd}[1]{\Delta_{#1}^y}
\newcommand{\ratioparam}[1]{\kappa_{\scalebox{0.5}{\ensuremath{#1}}}}%
\newcommand{\indbig}[1]{\one\left\{#1\right\}}
\newcommand{\bfP}{\mathbf{P}}
\newcommand{\bfQ}{\mathbf{Q}}
\newcommand{\hbP}{\hat{\bfP}}
\newcommand{\trans}{\mathcal{T}}
\newcommand{\odiag}{E}
\newcommand{\x}{\mathbf{x}}
\newcommand{\out}{{x}}


\newcommand{\risk}{\mathcal{R}}
\def\multiset#1#2{\ensuremath{\left(\kern-.3em\left(\genfrac{}{}{0pt}{}{#1}{#2}\right)\kern-.3em\right)}}

\newcommand{\hp}{\widehat{\p}}
 
\newcommand{\ham}[2]{\operatorname{d}_{\rm Ham}(#1,#2)}
\newcommand{\variance}[2]{\var_{#1}{\mleft[#2\mright]}}

% quantum
\newcommand{\bm}{{\mathbf{m}}}
\newcommand{\qs}{\rho}
\newcommand{\qmm}{{\rho_{\text{mm}}}}
\newcommand{\qkn}{{\rho_0}}
\newcommand{\blambda}{{\boldsymbol{\lambda}}}
\newcommand{\SW}{\textbf{SW}}
\newcommand{\ngroups}{N}
\newcommand{\pnew}{\mathbf{p}}


\newcommand{\HH}{\mathbb{H}}
\newcommand{\Herm}[1]{{\HH_{#1}}}
\newcommand{\F}{\mathbb{F}}
\newcommand{\Pow}{\mathbb{P}}
\newcommand{\mD}{\mathcal{D}}
\newcommand{\OPT}{\text{OPT}}

\newcommand{\qbit}[1]{|{#1}\rangle}
\newcommand{\qadjoint}[1]{\langle{#1}|}
\newcommand{\qproj}[1]{\qbit{#1}\qadjoint{#1}}
\newcommand{\qoutprod}[2]{\qbit{#1}\qadjoint{#2}}

\newcommand{\qdotprod}[2]{\langle#1|#2\rangle}
\newcommand{\hdotprod}[2]{\left\langle#1,#2\right\rangle}
\newcommand{\matdotprod}[3]{\langle#1|#2|#3\rangle}
\newcommand{\supop}[2]{\mathcal{N}_{{#1}\rightarrow{#2} }}
\newcommand{\bA}{\mathbf{A}}
\newcommand{\bB}{\mathbf{B}}
\newcommand{\bD}{\mathbf{D}}
\newcommand{\bC}{\mathbf{C}}
\newcommand{\bG}{\mathbf{G}}
\newcommand{\bH}{\mathbf{H}}
\newcommand{\bM}{\mathbf{M}}
\newcommand{\bS}{\mathbf{S}}
\newcommand{\bT}{\mathbf{T}}
\newcommand{\bU}{\mathbf{U}}
\newcommand{\bV}{\mathbf{V}}
\newcommand{\bW}{\mathbf{W}}
\newcommand{\bX}{\mathbf{X}}
\newcommand{\bY}{\mathbf{Y}}
\newcommand{\EE}{\mathbb{E}}
\newcommand{\Var}{\text{Var}}
\newcommand{\eye}{\mathbb{I}}
\newcommand{\Real}{\text{Re}}
\newcommand{\Img}{\text{Im}}
\newcommand{\id}{\text{id}}
\newcommand{\img}{\text{i}}

\newcommand{\rk}{{r}}
\newcommand{\VecOp}{\text{vec}}
\newcommand{\vvec}[1]{|#1\rangle\rangle}
\newcommand{\vadj}[1]{\langle\langle#1|}
\newcommand{\vvdotprod}[2]{\left\langle\left\langle#1|#2\right\rangle\right\rangle}

\newcommand{\bv}{\mathbf{v}}
\newcommand{\bx}{\mathbf{x}}
\newcommand{\outset}{{\mathcal{X}}}

\newcommand{\spec}{\text{spec}}
\newcommand{\vsigma}{\vec{\sigma}}
\newcommand{\Luders}{\mathcal{H}}
\newcommand{\avgLuders}{{\overline{\Luders}}}
\newcommand{\Choi}{{\mathcal{C}}}
\newcommand{\avgChoi}{{\overline{\Choi}}}
\newcommand{\hbasis}{{\mathcal{V}}}
\newcommand{\qest}{{\hat{\rho}}}

\DeclareMathOperator{\diam}{diam}
\DeclareMathOperator{\diag}{diag}
\DeclareMathOperator{\iSWAP}{iSWAP}
\DeclareMathOperator{\Span}{span}

% Representation theory
\newcommand{\PiRank}{r}
\newcommand{\Wg}{\text{Wg}}
\newcommand{\U}{\mathbb{U}}
\newcommand{\bi}{\mathbf{i}}
\newcommand{\bj}{\mathbf{j}}
\newcommand{\Sim}{\mathcal{S}}
\newcommand{\Mob}{\text{Mob}}
\newcommand{\Cat}{\text{Cat}}
\newcommand{\Haar}[1]{{\mathcal{U}_{#1}}}
\newcommand{\POVM}{\mathcal{M}}
\newcommand{\permProd}[2]{{\left\langle#1\right\rangle_{#2}}}
\newcommand{\cycle}{\mathcal{C}}
\newcommand{\ObsPOVM}{\mathcal{N}}

\newcommand{\Xset}{\mathcal{I}_X}
\newcommand{\Yset}{\mathcal{I}_Y}
\newcommand{\zest}{\hat{\ptb}}
\usepackage{utopia}
\usepackage{xifthen}
\usepackage{soul}
\usepackage{hyperref}
\usepackage{authblk}
\newcommand*{\SINGLE}{}
\title{Revisiting the Berkeley Admissions data: Statistical Tests for Causal Hypotheses}
% \author{
%  Sourbh Bhadane\\
%  Korteweg-de Vries Institute for Mathematics, University of Amsterdam, The Netherlands\\
%  \tt{s.n.bhadane@uva.nl}\\
%  \and
%  Joris M. Mooij \\
%  Korteweg-de Vries Institute for Mathematics, University of Amsterdam, The Netherlands \\
%  \tt{j.m.mooij@uva.nl}\\
%  \and
%  Onno Zoeter\\
%  Booking.com, The Netherlands \\
%  \tt{onno.zoeter@booking.com}
% }

\author[1]{Sourbh Bhadane} 
\author[1]{Joris M. Mooij}
\author[1]{Philip Boeken}
\author[2]{Onno Zoeter}
\affil[1]{Korteweg-de Vries Institute for Mathematics, University of Amsterdam}
\affil[2]{Booking.com, Amsterdam, The Netherlands}


% \affil{Korteweg-de Vries Institute for Mathematics, University of Amsterdam \and Booking.com, Amsterdam, The Netherlands}
% \address{Korteweg-de Vries Institute for Mathematics, University of Amsterdam \and Booking.com, Amsterdam, The Netherlands}

\begin{document}
\maketitle
\begin{abstract}
Reasoning about fairness through correlation-based notions is rife with pitfalls. The 1973 University of California, Berkeley graduate school admissions case from \citet{BickelHO75} is a classic example of one such pitfall, namely Simpson’s paradox. The discrepancy in admission rates among male and female applicants, in the aggregate data over all departments, vanishes when admission rates per department are examined. We reason about the Berkeley graduate school admissions case through a causal lens. In the process, we introduce a statistical test for causal hypothesis testing based on Pearl's instrumental-variable inequalities \citep{Pearl95}. We compare different causal notions of fairness that are based on graphical, counterfactual and interventional queries on the causal model, and develop statistical tests for these notions that use only observational data. We study the logical relations between notions, and show that while notions may not be equivalent, their corresponding statistical tests coincide for the case at hand. We believe that a thorough case-based causal analysis helps develop a more principled understanding of both causal hypothesis testing and fairness.
\end{abstract}

\section{Introduction}


\begin{figure}[t]
\centering
\includegraphics[width=0.6\columnwidth]{figures/evaluation_desiderata_V5.pdf}
\vspace{-0.5cm}
\caption{\systemName is a platform for conducting realistic evaluations of code LLMs, collecting human preferences of coding models with real users, real tasks, and in realistic environments, aimed at addressing the limitations of existing evaluations.
}
\label{fig:motivation}
\end{figure}

\begin{figure*}[t]
\centering
\includegraphics[width=\textwidth]{figures/system_design_v2.png}
\caption{We introduce \systemName, a VSCode extension to collect human preferences of code directly in a developer's IDE. \systemName enables developers to use code completions from various models. The system comprises a) the interface in the user's IDE which presents paired completions to users (left), b) a sampling strategy that picks model pairs to reduce latency (right, top), and c) a prompting scheme that allows diverse LLMs to perform code completions with high fidelity.
Users can select between the top completion (green box) using \texttt{tab} or the bottom completion (blue box) using \texttt{shift+tab}.}
\label{fig:overview}
\end{figure*}

As model capabilities improve, large language models (LLMs) are increasingly integrated into user environments and workflows.
For example, software developers code with AI in integrated developer environments (IDEs)~\citep{peng2023impact}, doctors rely on notes generated through ambient listening~\citep{oberst2024science}, and lawyers consider case evidence identified by electronic discovery systems~\citep{yang2024beyond}.
Increasing deployment of models in productivity tools demands evaluation that more closely reflects real-world circumstances~\citep{hutchinson2022evaluation, saxon2024benchmarks, kapoor2024ai}.
While newer benchmarks and live platforms incorporate human feedback to capture real-world usage, they almost exclusively focus on evaluating LLMs in chat conversations~\citep{zheng2023judging,dubois2023alpacafarm,chiang2024chatbot, kirk2024the}.
Model evaluation must move beyond chat-based interactions and into specialized user environments.



 

In this work, we focus on evaluating LLM-based coding assistants. 
Despite the popularity of these tools---millions of developers use Github Copilot~\citep{Copilot}---existing
evaluations of the coding capabilities of new models exhibit multiple limitations (Figure~\ref{fig:motivation}, bottom).
Traditional ML benchmarks evaluate LLM capabilities by measuring how well a model can complete static, interview-style coding tasks~\citep{chen2021evaluating,austin2021program,jain2024livecodebench, white2024livebench} and lack \emph{real users}. 
User studies recruit real users to evaluate the effectiveness of LLMs as coding assistants, but are often limited to simple programming tasks as opposed to \emph{real tasks}~\citep{vaithilingam2022expectation,ross2023programmer, mozannar2024realhumaneval}.
Recent efforts to collect human feedback such as Chatbot Arena~\citep{chiang2024chatbot} are still removed from a \emph{realistic environment}, resulting in users and data that deviate from typical software development processes.
We introduce \systemName to address these limitations (Figure~\ref{fig:motivation}, top), and we describe our three main contributions below.


\textbf{We deploy \systemName in-the-wild to collect human preferences on code.} 
\systemName is a Visual Studio Code extension, collecting preferences directly in a developer's IDE within their actual workflow (Figure~\ref{fig:overview}).
\systemName provides developers with code completions, akin to the type of support provided by Github Copilot~\citep{Copilot}. 
Over the past 3 months, \systemName has served over~\completions suggestions from 10 state-of-the-art LLMs, 
gathering \sampleCount~votes from \userCount~users.
To collect user preferences,
\systemName presents a novel interface that shows users paired code completions from two different LLMs, which are determined based on a sampling strategy that aims to 
mitigate latency while preserving coverage across model comparisons.
Additionally, we devise a prompting scheme that allows a diverse set of models to perform code completions with high fidelity.
See Section~\ref{sec:system} and Section~\ref{sec:deployment} for details about system design and deployment respectively.



\textbf{We construct a leaderboard of user preferences and find notable differences from existing static benchmarks and human preference leaderboards.}
In general, we observe that smaller models seem to overperform in static benchmarks compared to our leaderboard, while performance among larger models is mixed (Section~\ref{sec:leaderboard_calculation}).
We attribute these differences to the fact that \systemName is exposed to users and tasks that differ drastically from code evaluations in the past. 
Our data spans 103 programming languages and 24 natural languages as well as a variety of real-world applications and code structures, while static benchmarks tend to focus on a specific programming and natural language and task (e.g. coding competition problems).
Additionally, while all of \systemName interactions contain code contexts and the majority involve infilling tasks, a much smaller fraction of Chatbot Arena's coding tasks contain code context, with infilling tasks appearing even more rarely. 
We analyze our data in depth in Section~\ref{subsec:comparison}.



\textbf{We derive new insights into user preferences of code by analyzing \systemName's diverse and distinct data distribution.}
We compare user preferences across different stratifications of input data (e.g., common versus rare languages) and observe which affect observed preferences most (Section~\ref{sec:analysis}).
For example, while user preferences stay relatively consistent across various programming languages, they differ drastically between different task categories (e.g. frontend/backend versus algorithm design).
We also observe variations in user preference due to different features related to code structure 
(e.g., context length and completion patterns).
We open-source \systemName and release a curated subset of code contexts.
Altogether, our results highlight the necessity of model evaluation in realistic and domain-specific settings.





\section{Preliminaries}
\label{sec:prelim}
\label{sec:term}
We define the key terminologies used, primarily focusing on the hidden states (or activations) during the forward pass. 

\paragraph{Components in an attention layer.} We denote $\Res$ as the residual stream. We denote $\Val$ as Value (states), $\Qry$ as Query (states), and $\Key$ as Key (states) in one attention head. The \attlogit~represents the value before the softmax operation and can be understood as the inner product between  $\Qry$  and  $\Key$. We use \Attn~to denote the attention weights of applying the SoftMax function to \attlogit, and ``attention map'' to describe the visualization of the heat map of the attention weights. When referring to the \attlogit~from ``$\tokenB$'' to  ``$\tokenA$'', we indicate the inner product  $\langle\Qry(\tokenB), \Key(\tokenA)\rangle$, specifically the entry in the ``$\tokenB$'' row and ``$\tokenA$'' column of the attention map.

\paragraph{Logit lens.} We use the method of ``Logit Lens'' to interpret the hidden states and value states \citep{belrose2023eliciting}. We use \logit~to denote pre-SoftMax values of the next-token prediction for LLMs. Denote \readout~as the linear operator after the last layer of transformers that maps the hidden states to the \logit. 
The logit lens is defined as applying the readout matrix to residual or value states in middle layers. Through the logit lens, the transformed hidden states can be interpreted as their direct effect on the logits for next-token prediction. 

\paragraph{Terminologies in two-hop reasoning.} We refer to an input like “\Src$\to$\brga, \brgb$\to$\Ed” as a two-hop reasoning chain, or simply a chain. The source entity $\Src$ serves as the starting point or origin of the reasoning. The end entity $\Ed$ represents the endpoint or destination of the reasoning chain. The bridge entity $\Brg$ connects the source and end entities within the reasoning chain. We distinguish between two occurrences of $\Brg$: the bridge in the first premise is called $\brga$, while the bridge in the second premise that connects to $\Ed$ is called $\brgc$. Additionally, for any premise ``$\tokenA \to \tokenB$'', we define $\tokenA$ as the parent node and $\tokenB$ as the child node. Furthermore, if at the end of the sequence, the query token is ``$\tokenA$'', we define the chain ``$\tokenA \to \tokenB$, $\tokenB \to \tokenC$'' as the Target Chain, while all other chains present in the context are referred to as distraction chains. Figure~\ref{fig:data_illustration} provides an illustration of the terminologies.

\paragraph{Input format.}
Motivated by two-hop reasoning in real contexts, we consider input in the format $\bos, \text{context information}, \query, \answer$. A transformer model is trained to predict the correct $\answer$ given the query $\query$ and the context information. The context compromises of $K=5$ disjoint two-hop chains, each appearing once and containing two premises. Within the same chain, the relative order of two premises is fixed so that \Src$\to$\brga~always precedes \brgb$\to$\Ed. The orders of chains are randomly generated, and chains may interleave with each other. The labels for the entities are re-shuffled for every sequence, choosing from a vocabulary size $V=30$. Given the $\bos$ token, $K=5$ two-hop chains, \query, and the \answer~tokens, the total context length is $N=23$. Figure~\ref{fig:data_illustration} also illustrates the data format. 

\paragraph{Model structure and training.} We pre-train a three-layer transformer with a single head per layer. Unless otherwise specified, the model is trained using Adam for $10,000$ steps, achieving near-optimal prediction accuracy. Details are relegated to Appendix~\ref{app:sec_add_training_detail}.


% \RZ{Do we use source entity, target entity, and mediator entity? Or do we use original token, bridge token, end token?}





% \paragraph{Basic notations.} We use ... We use $\ve_i$ to denote one-hot vectors of which only the $i$-th entry equals one, and all other entries are zero. The dimension of $\ve_i$ are usually omitted and can be inferred from contexts. We use $\indicator\{\cdot\}$ to denote the indicator function.

% Let $V > 0$ be a fixed positive integer, and let $\vocab = [V] \defeq \{1, 2, \ldots, V\}$ be the vocabulary. A token $v \in \vocab$ is an integer in $[V]$ and the input studied in this paper is a sequence of tokens $s_{1:T} \defeq (s_1, s_2, \ldots, s_T) \in \vocab^T$ of length $T$. For any set $\mathcal{S}$, we use $\Delta(\mathcal{S})$ to denote the set of distributions over $\mathcal{S}$.

% % to a sequence of vectors $z_1, z_2, \ldots, z_T \in \real^{\dout}$ of dimension $\dout$ and length $T$.

% Let $\mU = [\vu_1, \vu_2, \ldots, \vu_V]^\transpose \in \real^{V\times d}$ denote the token embedding matrix, where the $i$-th row $\vu_i \in \real^d$ represents the $d$-dimensional embedding of token $i \in [V]$. Similarly, let $\mP = [\vp_1, \vp_2, \ldots, \vp_T]^\transpose \in \real^{T\times d}$ denote the positional embedding matrix, where the $i$-th row $\vp_i \in \real^d$ represents the $d$-dimensional embedding of position $i \in [T]$. Both $\mU$ and $\mP$ can be fixed or learnable.

% After receiving an input sequence of tokens $s_{1:T}$, a transformer will first process it using embedding matrices $\mU$ and $\mP$ to obtain a sequence of vectors $\mH = [\vh_1, \vh_2, \ldots, \vh_T] \in \real^{d\times T}$, where 
% \[
% \vh_i = \mU^\transpose\ve_{s_i} + \mP^\transpose\ve_{i} = \vu_{s_i} + \vp_i.
% \]

% We make the following definitions of basic operations in a transformer.

% \begin{definition}[Basic operations in transformers] 
% \label{defn:operators}
% Define the softmax function $\softmax(\cdot): \real^d \to \real^d$ over a vector $\vv \in \real^d$ as
% \[\softmax(\vv)_i = \frac{\exp(\vv_i)}{\sum_{j=1}^d \exp(\vv_j)} \]
% and define the softmax function $\softmax(\cdot): \real^{m\times n} \to \real^{m \times n}$ over a matrix $\mV \in \real^{m\times n}$ as a column-wise softmax operator. For a squared matrix $\mM \in \real^{m\times m}$, the causal mask operator $\mask(\cdot): \real^{m\times m} \to \real^{m\times m}$  is defined as $\mask(\mM)_{ij} = \mM_{ij}$ if $i \leq j$ and  $\mask(\mM)_{ij} = -\infty$ otherwise. For a vector $\vv \in \real^n$ where $n$ is the number of hidden neurons in a layer, we use $\layernorm(\cdot): \real^n \to \real^n$ to denote the layer normalization operator where
% \[
% \layernorm(\vv)_i = \frac{\vv_i-\mu}{\sigma}, \mu = \frac{1}{n}\sum_{j=1}^n \vv_j, \sigma = \sqrt{\frac{1}{n}\sum_{j=1}^n (\vv_j-\mu)^2}
% \]
% and use $\layernorm(\cdot): \real^{n\times m} \to \real^{n\times m}$ to denote the column-wise layer normalization on a matrix.
% We also use $\nonlin(\cdot)$ to denote element-wise nonlinearity such as $\relu(\cdot)$.
% \end{definition}

% The main components of a transformer are causal self-attention heads and MLP layers, which are defined as follows.

% \begin{definition}[Attentions and MLPs]
% \label{defn:attn_mlp} 
% A single-head causal self-attention $\attn(\mH;\mQ,\mK,\mV,\mO)$ parameterized by $\mQ,\mK,\mV \in \real^{{\dqkv\times \din}}$ and $\mO \in \real^{\dout\times\dqkv}$ maps an input matrix $\mH \in \real^{\din\times T}$ to
% \begin{align*}
% &\attn(\mH;\mQ,\mK,\mV,\mO) \\
% =&\mO\mV\layernorm(\mH)\softmax(\mask(\layernorm(\mH)^\transpose\mK^\transpose\mQ\layernorm(\mH))).
% \end{align*}
% Furthermore, a multi-head attention with $M$ heads parameterized by $\{(\mQ_m,\mK_m,\mV_m,\mO_m) \}_{m=1}^M$ is defined as 
% \begin{align*}
%     &\Attn(\mH; \{(\mQ_m,\mK_m,\mV_m,\mO_m) \}_{m\in[M]}) \\ =& \sum_{m=1}^M \attn(\mH;\mQ_m,\mK_m,\mV_m,\mO_m) \in \real^{\dout \times T}.
% \end{align*}
% An MLP layer $\mlp(\mH;\mW_1,\mW_2)$ parameterized by $\mW_1 \in \real^{\dhidden\times \din}$ and $\mW_2 \in \real^{\dout \times \dhidden}$ maps an input matrix $\mH = [\vh_1, \ldots, \vh_T] \in \real^{\din \times T}$ to
% \begin{align*}
%     &\mlp(\mH;\mW_1,\mW_2) = [\vy_1, \ldots, \vy_T], \\ \text{where } &\vy_i = \mW_2\nonlin(\mW_1\layernorm(\vh_i)), \forall i \in [T].
% \end{align*}

% \end{definition}

% In this paper, we assume $\din=\dout=d$ for all attention heads and MLPs to facilitate residual stream unless otherwise specified. Given \Cref{defn:operators,defn:attn_mlp}, we are now able to define a multi-layer transformer.

% \begin{definition}[Multi-layer transformers]
% \label{defn:transformer}
%     An $L$-layer transformer $\transformer(\cdot): \vocab^T \to \Delta(\vocab)$ parameterized by $\mP$, $\mU$, $\{(\mQ_m^{(l)},\mK_m^{(l)},\mV_m^{(l)},\mO_m^{(l)})\}_{m\in[M],l\in[L]}$,  $\{(\mW_1^{(l)},\mW_2^{(l)})\}_{l\in[L]}$ and $\Wreadout \in \real^{V \times d}$ receives a sequence of tokens $s_{1:T}$ as input and predict the next token by outputting a distribution over the vocabulary. The input is first mapped to embeddings $\mH = [\vh_1, \vh_2, \ldots, \vh_T] \in \real^{d\times T}$ by embedding matrices $\mP, \mU$ where 
%     \[
%     \vh_i = \mU^\transpose\ve_{s_i} + \mP^\transpose\ve_{i}, \forall i \in [T].
%     \]
%     For each layer $l \in [L]$, the output of layer $l$, $\mH^{(l)} \in \real^{d\times T}$, is obtained by 
%     \begin{align*}
%         &\mH^{(l)} =  \mH^{(l-1/2)} + \mlp(\mH^{(l-1/2)};\mW_1^{(l)},\mW_2^{(l)}), \\
%         & \mH^{(l-1/2)} = \mH^{(l-1)} + \\ & \quad \Attn(\mH^{(l-1)}; \{(\mQ_m^{(l)},\mK_m^{(l)},\mV_m^{(l)},\mO_m^{(l)}) \}_{m\in[M]}), 
%     \end{align*}
%     where the input $\mH^{(l-1)}$ is the output of the previous layer $l-1$ for $l > 1$ and the input of the first layer $\mH^{(0)} = \mH$. Finally, the output of the transformer is obtained by 
%     \begin{align*}
%         \transformer(s_{1:T}) = \softmax(\Wreadout\vh_T^{(L)})
%     \end{align*}
%     which is a $V$-dimensional vector after softmax representing a distribution over $\vocab$, and $\vh_T^{(L)}$ is the $T$-th column of the output of the last layer, $\mH^{(L)}$.
% \end{definition}



% For each token $v \in \vocab$, there is a corresponding $d_t$-dimensional token embedding vector $\embed(v) \in \mathbb{R}^{d_t}$. Assume the maximum length of the sequence studied in this paper does not exceed $T$. For each position $t \in [T]$, there is a corresponding positional embedding  







\section{Berkeley Case: No Latent Confounding}\label{sec:modeling}
 \section{Additional Figures}

In this section, we present some additional figures that demonstrate the negative effects of random edge-dropping, particularly focusing on providing empirical evidence for scenarios not covered by the theory in \autoref{sec:theory}. Additionally, we provide visual evidence of the negative effects of DropNode, despite the fact that it preserves sensitivity between nodes (in expectation).

\subsection{Symmetrically Normalized Propagation Matrix}
\label{sec:fig-sym-norm}

\begin{figure}
    \centering
    \includegraphics[width=\linewidth]{assets/linear-gcn_symmetric_Proteins.png}
    % \includegraphics[width=0.48\linewidth]{assets/linear-gcn_symmetric_MUTAG.png}
    \caption{Entries of $\ddot{\transition}^6$, averaged after binning node-pairs by their shortest distance.}
    \label{fig:linear-gcn_symmetric}
\end{figure}

The results in \autoref{sec:linear} correspond to the use of $\hat{\adjacency} = \propagation^\asym$ for aggregating messages -- in each message passing step, only the in-degree of node $i$ is used to compute the aggregation weights of the incoming messages. In practice, however, it is more common to use the symmetrically normalized propagation matrix, $\propagation = \propagation^\sym$, which ensures that nodes with high degree do not dominate the information flow in the graph \cite{kipf2017gcn}. As in \autoref{thm:sensitivity-l-layer}, we are looking for 
$$
    \expectation_{\mask\layer{1}, \ldots, \mask\layer{L}} \sb{\prod_{l=1}^L \propagation\layer{\ell}} = \ddot{\transition}^L
$$
where $\ddot{\transition} \coloneqq \expectation_{\mathsf{DE}} [\propagation^{\sym}]$. While $\ddot{\transition}$ is analytically intractable, we can approximate it using Monte-Carlo sampling. Accordingly, we use 20 samples of $\mask$ to compute an approximation of $\ddot{\transition}$, and plot out the entries of $\ddot{\transition}^L$, as we did for $\dot{\transition}^L$ in \autoref{fig:linear-gcn_asymmetric}. The results are presented in \autoref{fig:linear-gcn_symmetric}, which shows that while the sensitivity between nearby nodes is affected to a lesser extent compared to those observed in \autoref{fig:linear-gcn_asymmetric}, that between far-off nodes is significantly reduced, same as earlier.

\subsection{Upper Bound on Expected Sensitivity}

\begin{figure}
    \centering
    \includegraphics[width=\linewidth]{assets/linear-gcn_black-extension_Proteins.png}
    % \includegraphics[width=0.48\linewidth]{assets/linear-gcn_black-extension_MUTAG.png}
    \caption{Entries of $\sum_{l=0}^6 \dot{\transition}^l$, averaged after binning node-pairs by their shortest distance.}
    \label{fig:black_extension}
\end{figure}

\citet{black2023resistance} showed that the sensitivity between any two nodes in a graph can be bounded using the sum of the powers of the propagation matrix. In \autoref{sec:nonlinear}, we extended this bound to random edge-dropping methods with independent edge masks smapled in each layer:

\begin{align*}
    \expectation_{\mask\layer{1}, \ldots, \mask\layer{L}} \sb{\norm{\frac{\partial \representation\layer{L}_i}{\partial \feature_j}}_1}
    =
    \zeta_3^{\rb{L}} \rb{\sum_{\ell=0}^L \expectation \sb{\propagation}^{\ell}}_{ij}
\end{align*}

Although this bound does not have a closed form, we can use real-world graphs to study its entries. We randomly sample 100 molecular graphs from the Proteins dataset \cite{dobson2003proteins} and plot the entries of $\sum_{l=0}^6 \dot{\transition}^\ell$ (corresponding to \inline{DropEdge}) against the shortest distance between node-pairs. The results are presented in \autoref{fig:black_extension}. We observe a polynomial decline in sensitivity as the \inline{DropEdge} probability increases, suggesting that it is unsuitable for capturing LRIs.

\subsection{Test Accuracy versus DropNode Probability}

\begin{figure}[t]
    \centering
    \subcaptionbox{Homophilic datasets}{\includegraphics[width=0.7\linewidth]{assets/DropNode_homophilic.png}} \\
    \vspace{2mm}
    \subcaptionbox{Heterophilic datasets}{\includegraphics[width=\linewidth]{assets/DropNode_heterophilic.png}}
    \caption{Dropping probability versus test accuracy of DropNode-GCN.}
    \label{fig:dropnode}
\end{figure}

In \autoref{eqn:dropnode}, we noted that the expectation of sensitivity remains unchanged when using DropNode. However, these results were only in expectation. In practice, a high DropNode probability will result in poor communication between distant nodes, preventing the model from learning to effectively model LRIs. This is supported by the results in \autoref{tab:correlation}, where we observed a negative correlation between the test accuracy and DropNode probability. Moreover, DropNode was the only algorithm which recorded negative correlations on homophilic datasets. In \autoref{fig:dropnode}, we visualize these relationships, noting the stark contrast with \autoref{fig:acc-trends}, particularly in the trends with homophilic datasets.

Under the semantic framework of SCMs, we first make the same causal modeling assumptions that are commonplace in works that mention the Berkeley admissions case. We compare fairness notions that are tied to these modeling assumptions, with the view that modeling assumptions describe a family of SCMs and fairness notions define a subset of this family. We relate existing general notions of fairness in the literature to this viewpoint. While this is a re-examination of the various existing analyses of the Berkeley admissions case, in the next section, we relax the causal modeling assumptions and consider the more general family of models that allow for confounding between department choice and admissions outcome.

The set of endogenous variables consists of the protected attribute, namely sex of the applicant, $\sex$, the department they applied to, $\dept$, and the decision of the admissions committee, $\outcome$. We assume that $\sex, \outcome$ are binary variables and $\dept$ is a discrete-valued variable taking finite number of values, where $\sex=0,1$ corresponds to male, female applicants, respectively, and $\outcome=0,1$ corresponds to reject and accept, respectively.\footnote{The assumption of binary sex is purely for mathematical simplicity.} Given that, possibly, societal biases nudge applicants to departments at differing rates depending on their sex, we assume that $\sex$ affects $\dept$. Since departments are the primary decision-making units and have different admission rates, we also assume that $\dept$ affects $\outcome$. The question of whether acceptance decisions discriminate against sex centers around the direct causal effect of $\sex$ on $\outcome$, and therefore we allow such an effect in the model. We assume the absence of any confounding between the variables (in addition to the absence of any selection bias). The structural equations are given by
\ifdefined\SINGLE
\begin{align}\label{eq:no-cf-edge}
    \sex &= f_{\sex}(U_{\sex}) \nonumber, \\
    \dept &= f_{\dept}(\sex, U_{\dept}), \\
    \outcome &= f_{\outcome}(\sex,\dept,U_{\outcome}), \nonumber
\end{align}
\else $\sex = f_{\sex}(U_{\sex}), 
    \dept = f_{\dept}(\sex, U_{\dept}),
    \outcome = f_{\outcome}(\sex,\dept,U_{\outcome})$
\fi 
where $U_{\sex},U_{\dept}$ and $U_{\outcome}$ denote independent exogenous random variables. We denote the family of SCMs parameterized by the \ifdefined\SINGLE functions in \eqref{eq:no-cf-edge} \else above functions\fi and the exogenous distribution as $\modelsunconfedge$. 
%Let $\modelsunconfnoedge$ represent the same but where $\sex$ does not affect $\outcome$. 
For $\model \in \modelsunconfedge$, the causal graph $G(\model)$ is a directed acyclic graph (DAG), a subgraph of the one in Figure~\ref{fig:no-cf-edge}. 

\subsection{Fairness Notions}
We define a fairness notion to be a certain condition that is required to be satisfied by a causal model to be deemed fair. These conditions can take the form of observational, interventional, counterfactual or graphical queries on the SCMs in the families of causal models defined by modeling assumptions, in our case, $\modelsunconfedge$. While the criteria for the fairness notions in Section~\ref{sec:modeling} are phrased in terms of queries corresponding to different rungs of the causal ladder, in our case, any condition can only be tested using observational data.

% \ifdefined\SINGLE
% \begin{definition}[Demographic Parity]
% $\model \in \modelsunconfedge$ is fair according to demographic parity if it belongs to the null hypothesis set $\nullobsunconf \triangleq \left \lbrace \model \in \modelsunconfedge : P_{\model}\Paren{\outcome=1 \mid \sex =0} = P_{\model}\Paren{\outcome=1 \mid \sex=1} \right \rbrace $.
% \end{definition}
% \else
% \begin{definition}[Observational Notion of Fairness]
% $\model \in \modelsunconfedge$ is fair according to the observational notion of fairness if it belongs to 
% \begin{align*}
% \nullobsunconf &\triangleq \left \lbrace \model \in \modelsunconfedge : \right.\\
% &\left. P_{\model}\Paren{\outcome=1 \mid \sex =0} = P_{\model}\Paren{\outcome=1 \mid \sex=1} \right \rbrace.
% \end{align*}
% \end{definition}
% \fi

% The observational notion of fairness defined above is termed demographic parity \citep{DworkHPRZ12} in modern-day fairness literature and has implicitly been mentioned in earlier works \citep{HutchinsonMitchell19}. This is the notion of fairness that prompted the investigation of \cite{BickelHO75} into the Berkeley admissions data. We have already mentioned that this fairness notion falls prey to Simpson's paradox, i.e. Simpson's paradox translates into a fairness paradox where aggregated over the departments, the decision-making is unfair whereas for each department separately, it is fair. 

% We now mention another observational notion of fairness that is similar to a conditional version of demographic parity. 

The investigation of Berkeley's admission data was initiated on the observation that the well-known fairness notion of demographic parity $P_{\model}\Paren{\outcome=1 \mid \sex = 0} = P_{\model}\Paren{\outcome=1 \mid \sex = 1}$ did not hold. This fairness notion is based purely on observational data and we have already noted that it falls prey to Simpson's paradox. We now present another observational notion of fairness that can be interpreted as a conditional version of demographic parity.
\ifdefined\SINGLE
\begin{definition}[Observational Notion of Fairness]
$\model \in \modelsunconfedge$ is fair according to the observational notion of fairness if it belongs to the null hypothesis set 
\begin{align*}
\nullobsunconf \triangleq &\left \lbrace \model \in \modelsunconfedge :  \forall \ldept, \lsex, P_{\model}\Paren{\dept = \ldept, \sex = \lsex} >0 \right.\\
&\left. \implies P_{\model}\Paren{\outcome=1 \mid \sex =\lsex, \dept = \ldept} = P_{\model}\Paren{\outcome=1 \mid \dept = \ldept} \right \rbrace.
\end{align*}
% $$\nullobsunconf \triangleq \left \lbrace \model \in \modelsunconfedge : \forall \ldept, \lsex: P_{\model}\Paren{\dept = \ldept, \sex = \lsex} >0,  P_{\model}\Paren{\outcome=1 \mid \sex =\lsex, \dept = \ldept} = P_{\model}\Paren{\outcome=1 \mid \dept = \ldept} \right \rbrace. $$
\end{definition}
\else
\begin{definition}[Observational Notion of Fairness]
$\model \in \modelsunconfedge$ is fair according to the observational notion of fairness if it belongs to 
\begin{align*}
&\nullobsunconf \triangleq \left \lbrace \model \in \modelsunconfedge :  \forall \ldept, \lsex,  P_{\model}\Paren{\dept = \ldept, \sex = \lsex} >0 \right.\\
&\left. \implies P_{\model}\Paren{\outcome=1 \mid \sex =\lsex, \dept = \ldept} \right. \\
&\left.= P_{\model}\Paren{\outcome=1 \mid \dept = \ldept} \right \rbrace.
\end{align*}
\end{definition}
\fi

\citet{BickelHO75} proposed this notion for the Berkeley data. A valid test for this notion is a conditional independence test for $\outcome \indep \sex \mid \dept$. Indeed, the analysis of \citet{BickelHO75} shows that the data contain not enough evidence to reject the null hypothesis that this conditional independence holds, and therefore, concludes fairness. 

% \renewcommand{\thedefinition}{\thesection.\arabic{definition}} 
From the causal graph of $\modelsunconfedge$ in Figure~\ref{fig:no-cf-edge}, a natural subset of fair causal models is those without the edge $\sex \rightarrow \outcome$. 

\begin{definition}[Graphical Notion of Fairness]\label{def:graph_fairness}
     $\model \in \modelsunconfedge$ is fair according to the graphical notion of fairness if it belongs to the null hypothesis set $\nullgraphunconf \triangleq \left \lbrace \model \in \modelsunconfedge : \sex \rightarrow \outcome \notin \cg{\model} \right \rbrace$.
\end{definition}

% $\model \in \nullgraphunconf$ imposes a conditional independence constraint on the observational distribution, namely $\sex \indep \outcome \mid \dept$ by the global Markov property. Therefore, \citet{BickelHO75}'s analysis can be thought of as a valid test for the graphical notion of fairness albeit with an implicit faithfulness assumption.

\citet[Section 4.5.3]{Pearl09} discusses the direct effect in the context of the Berkeley admissions example, where he objects to conditioning for the department and instead proposes intervening on department choice, which corresponds to the controlled direct effect (CDE) \citep{Pearl01} of the `treatment', $\sex$, on the outcome, $\outcome$, for every value of the mediator, i.e., every department choice $\ldept$. 

\ifdefined\SINGLE
\begin{definition}[Interventional Notion of Fairness]
    $\model \in \modelsunconfedge$ is fair according to the interventional notion of fairness if it belongs to 
     \begin{equation*}\label{eq:interfairunconf}
    \nullinterunconf \triangleq \left \lbrace \model \in \modelsunconfedge: \forall \ldept,\lsex, P_{\model}\Paren{\outcome=1\mid\doop{\sex=\lsex},\doop{\dept=\ldept}} = P_{\model}\Paren{\outcome=1\mid\doop{\dept = \ldept}} \right \rbrace.
    \end{equation*}
    \end{definition}
\else
\begin{definition}[Interventional Notion of Fairness]
    $\model \in \modelsunconfedge$ is fair according to the interventional notion of fairness if it belongs to the null hypothesis set 
     \begin{align*}\label{eq:interfairunconf}
    &\nullinterunconf \triangleq \left \lbrace \model \in \modelsunconfedge: \forall \ldept,\lsex \right. \nonumber\\
   & \left. P_{\model}\Paren{\outcome=1\mid\doop{\sex=\lsex},\doop{\dept=\ldept}} \right. \nonumber\\
   &\left. = P_{\model}\Paren{\outcome=1\mid\doop{\dept = \ldept}} \right \rbrace.
    \end{align*}
\end{definition}
\fi
 % Note that, for the Berkeley example, a valid test for the interventional notion of fairness is again the conditional independence test $\outcome \indep \sex \mid \dept$.\footnote{This can be seen by applying the do-calculus and holds under positivity assumptions, which are met by the data at hand.}

Recent analyses of the Berkeley example emphasize counterfactual notions of fairness. In \citet[Section 4.5.4]{Pearl09}, \citet{PearlMackenzie18}, Pearl considers a counterfactual quantity, namely the natural direct effect (NDE) \citep{RobinsG92, Pearl01}  by motivating a hypothetical experiment where ``all female candidates retain their department preferences but change their gender [sex] identification (on the application form) from female to male''. Subsequent causal fairness works \citep{NabiShpitser18, Chiappa19} build on this and propose fairness notions based on known path-specific versions of NDE where the `direct path' from $\sex$ to $\outcome$ is viewed as `unfair' as opposed to the `fair' path $\sex \rightarrow \dept \rightarrow \outcome$. For the Berkeley example, the NDE$(s' \rightarrow s)$ is given by 
\begin{equation*}
P_{\model}\Paren{\outcome^{\doop{\sex = \lsex', \dept = \dept^{\doop{\sex=\lsex}}}}=1} - P_{\model}\Paren{\outcome^{\doop{\sex=s}}=1}
\end{equation*}
for $\lsex \neq \lsex'$. Note that by Pearl's mediation formula \citep{Pearl01}, the above is identified (assuming $\forall \ldept, \lsex, P_{\model}\Paren{\dept = \ldept, \sex = \lsex} > 0$) as 
\ifdefined\SINGLE
\begin{equation*}
    \sum\limits_{\ldept} \left( P_{\model}\Paren{\outcome=1 \mid \dept = \ldept, \sex = \lsex}  - P_{\model}\Paren{\outcome=1 \mid \dept = \ldept, \sex = \lsex'}\right)P_{\model}\Paren{\dept = \ldept \mid \sex= \lsex}.
\end{equation*}
\else
\begin{align*}
    \sum\limits_{\ldept} & \left( P_{\model}\Paren{\outcome=1 \mid \dept = \ldept, \sex = \lsex} \right. \\
    &\left. - P_{\model}\Paren{\outcome=1 \mid \dept = \ldept, \sex = \lsex'}\right)P_{\model}\Paren{\dept = \ldept \mid \sex= \lsex}.
\end{align*}
\fi
This implies that if the observational notion of fairness and positivity hold, the NDE is $0$. However, the converse is not necessarily true. For example, if one department favors male applicants and another favors female applicants, then the NDE could be $0$ while it is not necessary that the observational notion of fairness holds. 

Other counterfactual notions of fairness include those by \citet{KusnerLRS17}. The authors define a counterfactual fairness notion that implies demographic parity (see Section~\ref{app:kusnerctrfdemo} for a proof) for the Berkeley example; we have already seen that this particular fairness notion falls prey to Simpson's paradox. In the appendix, however, they define a path-dependent notion of counterfactual fairness.\footnote{This notion is specifically motivated by the Berkeley example.} In Section~\ref{app:kusnerpathnocf} we show that, in our setting, testing for the path-dependent counterfactual fairness notion is equivalent to testing for the conditional independence $A \indep S \mid D$. We now propose an alternate counterfactual notion of fairness and later compare testing of the same.
% Both the above notions correspond to not having a direct effect of $\sex$ on $\outcome$. Due to space constraints, we show the equivalence in supplementary material.

%that demands the invariance of the distribution of a decision, in a given context, with respect to a forced change in the protected attribute. 

\begin{definition}[Counterfactual Notion of Fairness]\label{def:ctrf-nocf}
$\model \in \modelsunconfedge$ is fair according to the counterfactual notion of fairness if it belongs to the null hypothesis set
\ifdefined\SINGLE
\begin{equation*}\label{eq:nullctrf}
    \nullctrfunconf \triangleq \left \lbrace \model \in \modelsunconfedge: \forall \ldept, \lsex, P_M(A^{\doop{S=s,D=d}} = A^{\doop{D=d}}) = 1 \right \rbrace.
\end{equation*}
\else
\begin{align*}\label{eq:nullctrf}
    \nullctrfunconf &\triangleq \left \lbrace \model \in \modelsunconfedge:\forall \ldept, \lsex,  \right. \nonumber\\
    &\left. P_M(A^{\doop{S=s,D=d}} = A^{\doop{D=d}}) = 1 \right \rbrace
\end{align*}
\fi
\end{definition}
The alternate hypotheses are given by the complement of the null hypotheses w.r.t. $\modelsunconfedge$. Given that the notions are defined on different rungs of the causal hierarchy, it is perhaps not surprising that they are nested accordingly. The assumption of no confounding simplifies the relations as we can prove equivalence of a few notions under positivity. The proof is deferred to Section~\ref{app:nested-nocf}.
\begin{restatable}{lemma}{unconfnested}\label{lem:notion_equiv}
\begin{equation*}\nullgraphunconf = \nullctrfunconf \subset \nullinterunconf \subset \nullobsunconf.\end{equation*}
If for all $s,d$, $P_{\model}(s,d) > 0$, then in addition, we have $\nullinterunconf = \nullobsunconf$.
\end{restatable}

Despite the nested nature of the fairness notions at different rungs of the causal hierarchy, we prove that the sets of observational distributions that these notions induce are identical. The proof is in Section~\ref{app:equiv-nocf}.
\begin{restatable}{theorem}{unconfequiv}\label{thm:unconf_test_equiv}
Let
% Let $P_{\model}\Paren{s,d} > 0$ for all $s,d$, and
\begin{align*} 
\distgraphunconf &\triangleq \left \lbrace P_{\model}\Paren{\dept,\outcome, \sex} : \model \in \nullgraphunconf \right \rbrace, \\
\distctrfunconf &\triangleq \left \lbrace P_{\model}\Paren{\dept,\outcome, \sex} : \model \in \nullctrfunconf \right \rbrace, \\
\distinterunconf &\triangleq \left \lbrace P_{\model}\Paren{\dept,\outcome, \sex} : \model \in \nullinterunconf \right \rbrace, \\
\distobsunconf &\triangleq \left \lbrace P_{\model}\Paren{\dept,\outcome, \sex} : \model \in \nullobsunconf \right \rbrace.
\end{align*}
Then $\distgraphunconf = \distctrfunconf = \distinterunconf = \distobsunconf.$
\end{restatable}
% \ifdefined\SINGLE
% \begin{equation}\label{eq:nullctrf}
%     \nullctrfunconf \triangleq \left \lbrace \model \in \modelsunconfedge: \forall \ldept, \lsex, g_{\outcome}\Paren{\ldept, \lsex, X_{\exrv}} = h_{\outcome}\Paren{\ldept,X_{\exrv}} \hspace{5mm}\text{a.s.} \right \rbrace,
% \end{equation}
% \else
% \begin{align}\label{eq:nullctrf}
%     \nullctrfunconf &\triangleq \left \lbrace \model \in \modelsunconfedge: \right. \nonumber\\
%     &\left. \forall \ldept, \lsex, g_{\outcome}\Paren{\ldept, \lsex, X_{\exrv}} = h_{\outcome}\Paren{\ldept,X_{\exrv}} \hspace{2mm}\text{a.s.} \right \rbrace,
% \end{align}
% \fi
% where $g_{\outcome}, h_{\outcome}$ are solution functions of $\model_{\doop{\dept,\sex}}$ and $\model_{\doop{\dept}}$ w.r.t. $\outcome$. 

% \todo{Given that the other counterfactual notions are formulated using potential outcome notation, we can also do that here for notational consistency.
% }
% In potential-outcome notation, the constraint on $M$ can also be written as:
% $$P_M(A^{\doop{S=s,D=d}} = A^{\doop{D=d}}) = 1$$
% \end{definition}

% Note that $\distobsunconf = \left \lbrace P(\outcome,\dept,\sex): \outcome \indep \sex \mid \dept \right \rbrace$. 
In summary, despite the fact that we analyze the Berkeley admissions case using multiple fairness notions, under the assumption of no confounding, with observational data, they can all be tested using a conditional independence test. 

If the data contains enough evidence to reject conditional independence, then the data generating mechanism is unfair w.r.t.\ the observational notion of fairness. On the other hand, if the data does not contain enough evidence to reject conditional independence, then the data generating mechanism is fair w.r.t.\ the observational notion of fairness. However, this extrapolation of the outcome of the statistical test on the fairness implications does not hold for the interventional, counterfactual and graphical notions. The following example illustrates that for the graphical notion of fairness, an unfaithful causal model, where $\outcome$ is directly affected by $\sex$, could satisfy conditional independence. 
\begin{example}\label{ex:unconfexample}
   Let $\model \in \modelsunconfedge$ be defined as $U_{\sex} \sim \text{Ber}(\frac{1}{2}), U_{\outcome} \sim \text{Ber}(\frac{1}{2}), U_{\dept} \sim \text{Ber}(\varepsilon)$ where $\varepsilon \in \left[0,\frac{1}{2}\right)$ and $\sex = U_{\sex}, \dept=\sex \oplus U_{\dept}, \outcome = \sex \oplus \dept \oplus U_{\outcome}$. Here, $\outcome \indep \sex \mid \dept$ but $\sex$ is a parent of $\outcome$, i.e., $\model \in \nullobsunconf$, but $\model \notin \nullgraphunconf$. 
\end{example}
For the interventional notion of fairness, the following example illustrates that a causal model that violates positivity could satisfy conditional independence but not the interventional notion of fairness.
\begin{example}\label{ex:posexample}
    Let $\model \in \modelsunconfedge$ be defined as $U_{\sex}=0, U_{\dept}=0,U_{\outcome} \sim \text{Ber}\Paren{\varepsilon}$ where $\varepsilon \in [0,\frac{1}{2})$, and $\sex=0,\dept=0,\outcome=\sex \oplus U_A$. Here $\outcome \indep \sex \mid \dept$, but 
 for all $d$, $P_{\model}\Paren{\outcome = 1 \mid \doop{\sex=1}, \doop{D=d}} = 1-\varepsilon \neq P_{\model}\Paren{\outcome = 1 \mid \doop{D=d}} = \varepsilon$. Therefore, $\model \in \nullobsunconf$, but $\model \notin \nullinterunconf$.
\end{example}

So, if the outcome of the test is that conditional independence cannot be rejected ($\model \in \nullobsunconf$), then due to the aforementioned observations, we cannot conclude that the underlying causal model belongs to the causal null hypothesis of the interventional or counterfactual or graphical fairness notions, i.e., our conclusion is that fairness is ``undecidable''. However, if the outcome of the statistical test is that there is enough evidence in the data to reject conditional independence($\model \notin \nullobsunconf$), then we can conclude that the underlying causal model does not belong to the causal null hypothesis of \textit{any} of the fairness notions, i.e., there is unfairness. 

In the next section, we enlarge the class of models to allow for confounding between $\dept$ and $\outcome$ and perform a similar reasoning exercise. 

%by motivating a hypothetical experiment where "all female candidates retain their department preferences but change their gender (sex) identification (on the application form) from female to male. 

%  $\model \in \modelsedge$ is fair according to the interventional notion of fairness if it belongs to    \begin{equation}\label{eq:interfair}
% \nullinter \triangleq \left \lbrace \model \in \modelsedge : \forall \ldept, \Pr\Paren{\outcome=1|\doop{\formsex=1},\doop{\dept=\ldept}} = \Pr\Paren{\outcome=1|\doop{\formsex=0},\doop{\dept = \ldept}} \right \rbrace.
% \end{equation}

%===============================================================
%LATER SECTION - WITH CONFOUNDING

% \paragraph{With Confounding:} Allowing for confounding between the department choice and admissions outcome affects how we test the fairness notions that we define in Section~\ref{sec:notions}. In particular, Kruskal \cite[Pg 128-129]{FairleyMosteller77} demonstrated an example where the existence of a confounder, such as state of residence, can render the result of the analysis of \cite{BickelHO75} incorrect. Other natural confounders include, for example, level of department-specific technical skills (for example, maths skills) that influence both, the department choice of an applicant and the admissions outcome. We combine all such possible confounders into a single exogenous random variable, $U$.

% Inspired by \cite{Pearl09} \todo{I think this trick has been proposed by others before Pearl, e.g. Geneletti, S. and A. P. Dawid (2007). Defining and identifying the effect of
% treatment on the treated, and Robins, J. M., T. J. VanderWeele, and T. S. Richardson (2007). Discussion
% of “Causal effects in the presence of non compliance a latent variable
% interpretation” by Forcina, A. Metron LXIV (3), 288–298. By the way, where does Pearl 2009 do this?}, we introduce $\formsex$, interpreted as the reported sex of the applicant on the form. In the observational data, we assume that $\formsex$ is just a copy of ``birth" sex, $\sex$. However, the difference lies in functional dependencies on $\dept$ and $\outcome$;  birth sex affects department choice whereas the reported sex affects admissions outcome. This ``node-splitting" operation helps us define testable fairness notions as we shall see in subsequent sections. Note that, while \cite{Pearl09} mentions the distinction between reported sex and birth sex, the analysis does not treat them as different.\todo{Couldn't find this... is it in paragraph 4.5?} The structural equations are given by 

% % \todo{Maybe mention that while Pearl mentions this in words, he doesn't mention use this node-splitting operation in the analysis.} 


% \begin{align}\label{eq:SCMfunctions}
%         \sex &= f_{\sex}(U_{\sex}), \nonumber\\
%         \formsex &= f_{\formsex}(\sex) = \sex, \nonumber \\
%         \dept &= f_{\dept}(\sex,U,U_{\dept}), \nonumber \\
%         \outcome &= f_{\outcome}(\formsex,\dept,U,U_{\outcome}),
% \end{align}
% where $U$ denotes the confounder between $\dept$ and $\outcome$. Note that we make no assumptions about the dimension and nature of the confounder. The exogenous distribution is a product distribution over all the exogenous variables, namely $\left \lbrace U_{\sex}, U_{\dept}, U_{\outcome}, U \right \rbrace$. We denote the family of models parameterized by the functional dependencies and the exogenous distributions as described above as $\modelsedge$. We also define $\modelsnoedge$ to be the family of models parameterized by the exogenous distribution and the functional dependencies listed above with the modification in the structural equation of $\outcome$ where $\outcome = f_{\outcome}\Paren{\dept,U,U_{\outcome}}$.

% While there might be other variables in the system that are observed, we assume that the resulting SCM obtained by marginalizing all variables except $\sex, \formsex, \dept, \outcome$ is given as above. Despite the fact that the data obtained from the Berkeley Graduate School admissions is in the form of a finite dataset, we will often assume that we have access to the observational distribution of $\sex, \dept, \outcome$ and denote it by $\obsdis$. Specifically, we assume that we can draw samples from $\obsdis$ which we call observational data. We address statistical tests with finite data in Section~\ref{sec:tests}.

%=============================================================

%PREVIOUS VERSION: CAUSAL MODELING SECTION 

% As a warm-up, we first make the same causal modeling assumptions that are commonplace in works that mention the Berkeley admissions case. We then consider the more general family of models that allow for confounding between the department choice and the admissions outcome. For both cases, we operate under the semantic framework of structural causal models (SCM).
% \paragraph{Without Confounding:} The set of endogenous variables include the protected attribute, namely sex of the applicant, $\sex$, the department they applied to, $\dept$ and the decision of the admissions committee, $\outcome$. We assume that $\sex, \outcome$ are binary variables and $\dept$ is a discrete-valued variable taking finite number of values, where for women applicants, $\sex=1$ and acceptance decisions are denoted by $\outcome=1$.\todo{it is more precise to say $\sex=0,1$ corresponds with male,female, respectively, and $\outcome=0,1$ corresponds with reject/accept, respectively; to remain sufficiently inclusive perhaps it's good to add a comment that for mathematical simplicity we will assume sex is binary} Given the evidence  \cite{BickelHO75} that societal biases nudge applicants to departments at differing rates depending on their sex, we assume a functional dependency of $\sex$ on $\dept$ \todo{I think you mean vice versa: ``$\dept$ on $\sex$''. `Functional dependency' is SCM-specific terminology, you could also say `$S$ affects $D$'}. Since departments are the primary decision-making units and have differing admission rates, we also assume a functional dependency of $\dept$ on $\outcome$ \todo{$\outcome$ on $\dept$. Etcetera in the following}. Since models that don't have a functional dependency of $\sex$ on $\outcome$ are subsumed by models that do, we consider models with the functional dependency. The structural equations are given by
% \begin{align}\label{eq:no-cf-edge}
%     \sex &= f_{\sex}(U_{\sex}), \\
%     \dept &= f_{\dept}(\sex, U_{\dept}), \\
%     \outcome &= f_{\outcome}(\sex,\dept,U_{\outcome}),
% \end{align}
% where $U_{\sex},U_{\dept}$ and $U_{\outcome}$ denote independent exogenous random variables. We denote the family of models parameterized by the functional dependencies given by \eqref{eq:no-cf-edge} and the exogenous distribution as $\modelsunconfedge$. Let $\modelsunconfnoedge$ represent the same but where $\outcome$ does not have a functional dependency on $\sex$. For a SCM $\model \in \modelsunconfedge$, the causal graph $G(\model)$ is a directed acyclic graph (DAG) as shown in Figure~\ref{fig:no-cf-edge}. 

% \paragraph{With Confounding:} Allowing for confounding between the department choice and admissions outcome affects how we test the fairness notions that we define in Section~\ref{sec:notions}. In particular, Kruskal \cite[Pg 128-129]{FairleyMosteller77} demonstrated an example where the existence of a confounder, such as state of residence, can render the result of the analysis of \cite{BickelHO75} incorrect. Other natural confounders include, for example, level of department-specific technical skills (for example, maths skills) that influence both, the department choice of an applicant and the admissions outcome. We combine all such possible confounders into a single exogenous random variable, $U$.

% Inspired by \cite{Pearl09} \todo{I think this trick has been proposed by others before Pearl, e.g. Geneletti, S. and A. P. Dawid (2007). Defining and identifying the effect of
% treatment on the treated, and Robins, J. M., T. J. VanderWeele, and T. S. Richardson (2007). Discussion
% of “Causal effects in the presence of non compliance a latent variable
% interpretation” by Forcina, A. Metron LXIV (3), 288–298.}. By the way, where does Pearl 2009 do this?}, we introduce $\formsex$, interpreted as the reported sex of the applicant on the form. In the observational data, we assume that $\formsex$ is just a copy of ``birth" sex, $\sex$. However, the difference lies in functional dependencies on $\dept$ and $\outcome$;  birth sex affects department choice whereas the reported sex affects admissions outcome. This ``node-splitting" operation helps us define testable fairness notions as we shall see in subsequent sections. Note that, while \cite{Pearl09} mentions the distinction between reported sex and birth sex, the analysis does not treat them as different.\todo{Couldn't find this... is it in paragraph 4.5?} The structural equations are given by 

% % \todo{Maybe mention that while Pearl mentions this in words, he doesn't mention use this node-splitting operation in the analysis.} 


% \begin{align}\label{eq:SCMfunctions}
%         \sex &= f_{\sex}(U_{\sex}), \nonumber\\
%         \formsex &= f_{\formsex}(\sex) = \sex, \nonumber \\
%         \dept &= f_{\dept}(\sex,U,U_{\dept}), \nonumber \\
%         \outcome &= f_{\outcome}(\formsex,\dept,U,U_{\outcome}),
% \end{align}
% where $U$ denotes the confounder between $\dept$ and $\outcome$. Note that we make no assumptions about the dimension and nature of the confounder. The exogenous distribution is a product distribution over all the exogenous variables, namely $\left \lbrace U_{\sex}, U_{\dept}, U_{\outcome}, U \right \rbrace$. We denote the family of models parameterized by the functional dependencies and the exogenous distributions as described above as $\modelsedge$. We also define $\modelsnoedge$ to be the family of models parameterized by the exogenous distribution and the functional dependencies listed above with the modification in the structural equation of $\outcome$ where $\outcome = f_{\outcome}\Paren{\dept,U,U_{\outcome}}$.

% While there might be other variables in the system that are observed, we assume that the resulting SCM obtained by marginalizing all variables except $\sex, \formsex, \dept, \outcome$ is given as above. Despite the fact that the data obtained from the Berkeley Graduate School admissions is in the form of a finite dataset, we will often assume that we have access to the observational distribution of $\sex, \dept, \outcome$ and denote it by $\obsdis$. Specifically, we assume that we can draw samples from $\obsdis$ which we call observational data. We address statistical tests with finite data in Section~\ref{sec:tests}.


%==============================================================
%OLD VERSION:   

% \subsection{Selection Bias}
% The Berkeley dataset released in \cite{??} \todo{Cite the R package UCBAdmissions} contains data about the top $6$ departments. However, in \cite{BickelHO75}, there is data about $101$ departments. This is an example of selection bias and not being cognizant about this in our analysis leads to biased results. \todo{What are other examples of selection bias?} Other forms of selection bias on sex could exist, for example, through existence of specific prior preparatory programs that affect the chances of choosing a department. In our work, we consider selection bias to be an important aspect and evaluate our fairness notions and the associated tests based on whether they are robust to selection bias on the department and sex. 

% We handle robustness to selection bias in the type of observational data that we assume access to where the different types are differentiated by sets of variables that we condition on. For example, a test that only assumes access to the conditional distribution of $\outcome$ given $\dept, \sex$ is robust to selection bias on $\dept$ and $\sex$. We shall see in Section~\ref{sec:bounds} how the fairness notions defined in Section~\ref{sec:notions} change with change in these types of observational data. 





% \section{Fairness Notions}\label{sec:notions}\todo{not mentioned counterfactual notions. Redo this section after tying in the counterfactual fairness and path-dependent counterfactual fairness notions from Kusner et al's paper.}
% \todo{Readability suffers from that you're telling two stories at once, the one with-cf and the one without-cf. Didactically it seems better to first tell the no-cf story (esentially reproducing Bickel et al's story) and then the with-cf story.}
%We define a fairness notion to be a certain condition that is required to be satisfied by a causal model to be deemed fair. These conditions can take the form of observational, interventional, counterfactual or graphical queries on the SCMs in the families of causal models defined by modeling assumptions in Section~\ref{sec:modeling}. 


% \todo{Should we explicitly also discuss ``observational'' fairness notions like demographic parity? Most of the fairness literature looks at those notions... Perhaps all readers will know that the notion of demographic parity may fall prone to Simpson's paradox, but perhaps we need to add that discussion for completeness? Anyhow, this is something that can be added later on, let's focus on the causal fairness notions first.}
% Given the SCM in the previous section, there are multiple fairness notions that we could consider. In this section, we motivate a few fairness notions to consider. \todo{Is this the rationale for us to consider all the notions that we do? - Natural?} We leave the issue of coming up with statistical tests to test each of the notions for Section~\ref{sec:tests}.
\section{Berkeley Case: With Latent Confounding Between Department And Outcome}\label{sec:confounder}
\begin{figure}[!th]
    \centering
    \begin{tikzpicture}
\tikzstyle{vertex}=[circle,fill=none,draw=black,minimum size=17pt,inner sep=0pt]
\node[vertex] (S) at (0,0) {$S$};
\node[vertex] (A) at (2,0) {$A$};
\node[vertex] (D) at (1,1) {$D$};
%\node[vertex] (S') at (1,-0.5) {$S'$};
\path (S) edge[bend left=10] (D);
\path (D) edge[bend left=10] (S);
\path (D) edge[bend left=10] (A);
\path (A) edge[bend left=10] (D);
\path[bidirected] (D) edge[bend left=60] (A);
\path[bidirected] (S) edge[bend left=60] (D);
\path[bidirected] (S) edge[bend right=60] (A);
\path (S) edge[bend left=10] (A);
\path (A) edge[bend left=10] (S);
    \end{tikzpicture}
    \caption{Causal graph of a model without assumptions}
    \label{fig:causal_modeling}
\end{figure}

We now take a more careful causal modeling approach. Instead of starting from variables and reasoning about structural equations that we allow, we start with assuming that all structural equations exist.\footnote{Since this allows for causal cycles, this would require using the framework of simple SCMs \citep{BongersFPM21}.} For the Berkeley example, Figure~\ref{fig:causal_modeling} shows a causal graph of an SCM that we start with. \ifdefined\SINGLE We now provide rationale for ruling out few structural equations. Based on chronology of events, we rule out those where $D$ directly affects $S$, where $A$ directly affects $D$ and where $A$ directly affects $S$. \else Based on chronology of events, we rule out structural equations where $D$ directly affects $S$, where $A$ directly affects $D$ and where $A$ directly affects $S$.\fi We rule out unobserved common causes of $S$ and $D$, and $S$ and $A$ since we model $S$ to be sex at birth. While latent selection bias might introduce bidirected edges \citep{ChenZM24} that are incident on $S$, we assume for now that there is no selection bias in the dataset. The resulting class of SCMs has structural equations of the form
\ifdefined\SINGLE
\begin{align}\label{eq:cf-edge}
    \sex &= f_{\sex}(U_{\sex}) \nonumber, \\
    \dept &= f_{\dept}(\sex,U, U_{\dept}), \\
    \outcome &= f_{\outcome}(\sex,\dept,U,U_{\outcome}), \nonumber
\end{align}
\else 
$\sex = f_{\sex}(U_{\sex}), \dept = f_{\dept}(\sex,U, U_{\dept}), \outcome = f_{\outcome}(\sex,\dept,U,U_{\outcome})$
\fi 
where $U,U_{\sex},U_{\dept}$ and $U_{\outcome}$ denote independent exogenous random variables. 
We define $\modelsedgerelax$ to be the family of models parameterized by the above structural equations and the exogenous distribution. Further, we define $\modelsedge = \left\{ \model \in \modelsedgerelax : \forall s, P_{\model}(S=s) > 0 \right\}$. For $\model \in \modelsedgerelax$ (and $\modelsedge$), the causal graph is a subgraph of the one shown in Figure~\ref{fig:cf-edge}. 

Although we arrived at allowing confounding between department and outcome through a careful causal modeling approach, this is not a novel consideration. In particular, Kruskal \citep[Pg 128-129]{FairleyMosteller77} demonstrated an example where the existence of a latent confounder, such as state of residence, can render \citet{BickelHO75}'s analysis incorrect. Other natural latent confounders include, for example, level of department-specific technical skills that influence both the department choice of an applicant and the admissions outcome. 

%We denote all such latent confounders by a single exogenous random variable, $U$ which is independent of other exogenous random variables. We modify the causal mechanisms of $\dept, \outcome$ in \eqref{eq:no-cf-edge} to include $U$ as input.

%In Section~\ref{app:modeling}, we show how we end up with $\modelsedge$ using a more careful causal modeling approach. 

% From the causal graphs of $\modelsunconfedge$ and $
% \modelsedge$ in Figures~\ref{fig:no-cf-edge} and \ref{fig:cf-edge}, a natural subset of fair causal models are those where the red edge does not exist in the causal graph.\todo{It is important to mention that this coincides with the notion that Pearl proposes in his 2009 book (no direct effect of $\sex$ on $\outcome$ w.r.t.\ $\{\sex,\dept,\outcome\}$).}

% Our first notion of fairness stems from the natural question, "Did the admissions committee consider $\formsex$ as a deciding factor while deciding the outcome?". In the SCM framework, this can be formulated as whether the structural equation for $\outcome$ includes $\formsex$ or equivalently, does the edge $\formsex \rightarrow \outcome$ exist in the causal graph of $\model$, $\cG(\model)$? 

% \noindent\textbf{Graphical Notion}: If $\sex \rightarrow \outcome \notin \cG(\model)$, then $M$ is fair. 

% Denote the class of models that are fair according to the 

% \begin{definition}[Graphical Notion of Fairness]\label{def:graph_fairness}
%      $\model \in \modelsunconfedge$ is fair according to the graphical notion of fairness if it belongs to $\nullgraphunconf \triangleq \left \lbrace \model \in \modelsunconfedge : \sex \rightarrow \outcome \notin \cg{\model} \right \rbrace$. $\model \in \modelsedge$ is fair according to the graphical notion of fairness if it belongs to $\nullgraph \triangleq \left \lbrace \model \in \modelsedge : \formsex \rightarrow \outcome \notin \cg{\model} \right \rbrace$.
% \end{definition}
% \todo{I got a bit lost in redundancies in the notation. $H^0_{no-cf-graph} = \model_{no-cf-no-edge}$ and $H^0_{graph}=\model_{no-edge}$ that were already introduced before.}

Since our modeling assumptions expand the family of SCMs under consideration to $\modelsedge$, the fairness notions that we discussed in the previous section are modified accordingly to obtain null hypothesis sets $ \nullgraph, \nullinter$ and $\nullctrf$. 
\ifdefined\SINGLE
\begin{definition}[Fairness Notions with Confounding]\label{def:notions-cf}
For $\model \in \modelsedge$ the null hypothesis set corresponding to the interventional, counterfactual and graphical notion of fairness are 
\begin{align*}\label{eq:cf-def}
    \nullinter &\triangleq \left \lbrace \model \in \modelsedge: \forall \ldept,\lsex, P_{\model}\Paren{\outcome=1\mid\doop{\sex=\lsex},\doop{\dept=\ldept}} = P_{\model}\Paren{\outcome=1\mid\doop{\dept = \ldept}} \right \rbrace, \\
     \nullctrf &\triangleq \left \lbrace \model \in \modelsedge: \forall \ldept, \lsex, P_M(A^{\doop{S=s,D=d}} = A^{\doop{D=d}}) = 1 \right \rbrace, \\
     \nullgraph&\triangleq \left \lbrace \model \in \modelsedge : \sex \rightarrow \outcome \notin \cg{\model} \right \rbrace.
\end{align*}
\end{definition}
\else
\begin{definition}[Fairness Notions with Confounding]\label{def:notions-cf}
For $\model \in \modelsedge$ the null hypothesis set corresponding to the interventional, counterfactual and graphical notion of fairness are 
\begin{align*}\label{eq:cf-def}
    \nullinter &\triangleq \left \lbrace \model \in \modelsedge: \forall \ldept,\lsex, \right.\\
    &\left. P_{\model}\Paren{\outcome=1\mid\doop{\sex=\lsex},\doop{\dept=\ldept}} \right.\\
    &\left. = P_{\model}\Paren{\outcome=1\mid\doop{\dept = \ldept}} \right \rbrace, \\
     \nullctrf &\triangleq \left \lbrace \model \in \modelsedge: \forall \ldept, \lsex, \right.\\
     &\left. P_M(A^{\doop{S=s,D=d}} = A^{\doop{D=d}}) = 1 \right \rbrace, \\
     \nullgraph &\triangleq \left \lbrace \model \in \modelsedge : \sex \rightarrow \outcome \notin \cg{\model} \right \rbrace.
\end{align*}
\end{definition}
\fi
% \nullobs &\triangleq \left \lbrace \model \in \modelsedge : \forall \ldept, P_{\model}\Paren{\outcome=1 \mid \sex =0, \dept = \ldept} = P_{\model}\Paren{\outcome=1 \mid \sex=1, \dept = \ldept} \right \rbrace, \\
% \nullobs &\triangleq \left \lbrace \model \in \modelsedge : \forall \ldept, P_{\model}\Paren{\outcome=1 \mid \sex =0, \dept = \ldept} \right. \\
%     &\left. = P_{\model}\Paren{\outcome=1 \mid \sex=1,\dept = \ldept} \right \rbrace, \\
While the above notions generalize straightforwardly from the no-confounder setting, this is no longer the case for the observational notion. In addition, while the statistical tests for the no-confounder model are straightforward, this is no longer the case for the aforementioned null hypotheses since $\outcome \not\!\perp\!\!\!\perp \sex \mid \dept$ in general. We first consider the graphical notion of fairness and develop a corresponding statistical test.

\subsection{Graphical Notion and the Instrumental Variable (IV) Inequalities}\label{subsec:graph_iv}

In the presence of latent confounding, graphical queries, such as absence of edges, impose equality or inequality constraints \citep{Evans16, WolfeSF19} in addition to conditional independence constraints which are the only constraints imposed by a DAG. For the Berkeley case with confounding, since the path $\sex \rightarrow \dept \leftrightarrow \outcome$ is open when conditioned on $\dept$, we have $\sex \not\!\perp\!\!\!\perp \outcome \mid \dept$ in general. 
%For example, for $\model \in \nullgraphunconf$, by the global Markov property, the graph implies the conditional independence $\sex \indep \outcome \mid \dept$ for any distribution that is Markov with respect to $\model$ \todo{in particular, for the observational distribution of $\model$ (the other distributions are not relevant to consider here)}.
%However, this no longer holds for the case with confounding since the path $\sex \rightarrow \dept \leftrightarrow \outcome$ is open when conditioned on $\dept$. 
Our test for the graphical notion of fairness for $\modelsedge$ stems from the observation that a model $\model \in \nullgraph$, lies in the instrumental variable (IV) model class $\modeliv$ where $\sex$ is considered the instrument, $\dept$ the treatment, and $\outcome$ the effect. If all modeled endogenous variables are discrete-valued, a necessary condition for the observational distribution\footnote{While we express the IV inequalities as a condition satisfied by the observational distribution, in Section~\ref{app:iv} we reason that they are more appropriately expressed as conditions in terms of $P_{\model}(X,Y \mid \doop{Z})$.} resulting from $\model \in \modeliv$ is to satisfy the IV inequalities \citep{Pearl95}, which in the context of Figure~\ref{fig:iv} are given by  
\begin{equation}\label{eq:iv}
    \max_{x} \sum_{y} \max_{z} P_{\model}\Paren{X=x,Y=y\mid Z=z} \leq 1. 
\end{equation}
% \st{This implies that a violation of the instrumental variable inequalities is evidence of unfairness.} \todo{Not necessarily, violations of the IV model class need not imply that there is an edge from $\sex \rightarrow \outcome$. Can we then not conclude anything from the IV inequality violation? }
Since the IV inequalities are only necessary conditions, an arbitrary distribution on $X,Y,Z$ that satisfies the IV inequality does not necessarily imply that it is an entailed distribution of a model from the IV model class. \citet{Bonet01} showed that for the binary instrument, treatment and effect case, the IV inequalities are also sufficient conditions. In Theorem~\ref{thm:iv_tight}, we show that for the case where the instrument and outcome are binary and the treatment is discrete-valued with finite support, any distribution that satisfies the IV inequality is also entailed by some causal model from the IV model class. To the best of our knowledge, Theorem~\ref{thm:iv_tight} is a novel result. We defer the proof to Section~\ref{app:ivsharp}.
\ifdefined\SINGLE
\begin{restatable}{theorem}{ivtight}\label{thm:iv_tight}
Let $X,Y,Z$ be discrete random variables defined on $\cX,\cY,\cZ$ respectively, with $|\cX| = n\geq 2, |\cY|=2, |\cZ| =2$. Let the set of joint distributions that satisfy the IV inequalities be defined as $\distiv \triangleq \left \lbrace P(X,Y,Z) : P(X,Y \mid Z)\text{ satisfies }\eqref{eq:iv} \text{ and } \forall z, P(Z=z)>0 \right \rbrace$. Define $\distmodeliv \triangleq \left \lbrace P_{\model}(X,Y,Z) : \model \in \modeliv \right \rbrace.$ Then $\distiv = \distmodeliv.$
\end{restatable}
\else
\begin{restatable}{theorem}{ivtight}\label{thm:iv_tight}
Let $X,Y,Z$ be discrete random variables defined on $\cX,\cY,\cZ$ respectively, with $|\cX| = n\geq 2, |\cY|=2, |\cZ| =2$. Define $\distmodeliv \triangleq \left \lbrace P_{\model}(X,Y,Z) : \model \in \modeliv \right \rbrace.$ and the set of joint distributions that satisfy the IV inequalities as \begin{align*}\distiv &\triangleq \left \lbrace P(X,Y,Z) : P(X,Y \mid Z)\text{ satisfies }\eqref{eq:iv} \right.\\
&\left.\text{ and } \forall z, P(Z=z)>0 \right \rbrace.\end{align*}  Then $\distiv = \distmodeliv.$
\end{restatable}
\fi
% We make a few observations that help us understand the implications of Theorem~\ref{thm:iv_tight} for concluding finally concluding about  
For the Berkeley admissions case, assuming for now that the true observational distribution over $\sex,\dept,\outcome$ is known, the observational distribution satisfying the IV inequalities implies that there exists a causal explanation (model) where the directed edge $\sex \rightarrow \outcome$ is absent, i.e., given that $P(\outcome,\dept, \sex) \in \distiv$, there exists $\model \in \modeliv$ such that $P_{\model}(\outcome,\dept,\sex) = P(\outcome,\dept,\sex)$. On the other hand, the observational distribution violating the IV inequalities does not necessarily imply that the edge $\sex \rightarrow \outcome$ is present since the IV model class, $\modeliv$, is only a subset of all the models that do not contain the edge $\sex \rightarrow \outcome$ in the causal graph. For example, the existence of latent confounding between $\sex$ and $\outcome$ in a model $\model$ may result in $\model \notin \modeliv$, even though $\cg{\model}$ does not necessarily contain the directed edge $\sex \rightarrow \outcome$. However, the causal modeling assumption that defined $\modelsedge$ rules out latent confounding between $\sex$ and $\outcome$. Therefore, given our modeling assumptions, $\nullgraph = \modeliv$, and in turn, we conclude that violating the IV inequalities implies that $\model \in \modelsedge \backslash \nullgraph$. As in the previous section, it is possible that causal models that lie outside $\nullgraph$  (``unfair'' models) induce observational distributions that lie in $\distiv$, i.e., satisfy the IV inequalities. Therefore, satisfying the IV inequalities is not conclusive evidence that the data-generating mechanism is fair, i.e., our conclusion should be that fairness is undecidable. In Section~\ref{sec:bayesiantest} we introduce a Bayesian test for the IV inequalities.

\subsection{Bounds on Interventional Notion of Fairness}\label{subsec:bounds}
For $\model \in \modelsedge$, the interventional notion of fairness is the CDE, which is not identifiable in our case.
By a response-function parameterization \citep{Balke95, BalkePearl97} of $\model \in \modelsedge$, we can express the interventional distributions in Definition~\ref{def:notions-cf} as a linear function of response variables. Further, the observational distribution is also expressed as a linear function of the response variables. Using the symbolic linear programming approach of \citet{Balke95}, we obtain upper and lower bounds in terms of the observational distribution, specifically, $P_{\model}\Paren{\outcome,\dept \mid \sex}$. Indeed, \citet{CaiKPT08} express the same bounds which we reproduce below. The CDE given by
\ifdefined\SINGLE
\begin{equation*}P_{\model}\Paren{\outcome=1\mid \doop{\sex=1}, \doop{\dept=\ldept}} - P_{\model}\Paren{\outcome=1\mid \doop{\sex=0}, \doop{\dept=\ldept}},
\end{equation*}
\else
\begin{align*}
&P_{\model}\Paren{\outcome=1\mid \doop{\sex=1}, \doop{\dept=\ldept}}\\
&- P_{\model}\Paren{\outcome=1\mid \doop{\sex=0}, \doop{\dept=\ldept}},
\end{align*}
\fi
lies in the interval
\ifdefined\SINGLE
\begin{align*}
    &\left[ \Pr\left(\outcome=1,\dept=\ldept\mid\sex=1\right) + \Pr\left(\outcome=0,\dept=\ldept\mid \sex=0\right) - 1,\right.\\
    & \left.1 - \Pr\left(\outcome=0,\dept=\ldept\mid\sex=1\right) - \Pr\left(\outcome=1,\dept=\ldept\mid\sex=0\right)\right].
\end{align*}
\else
% \begin{align*}
%     &\left[ \Pr\left(\outcome=1,\dept=\ldept\mid \sex=1\right) \right. \\
%     &\left.+ \Pr\left(\outcome=0,\dept=\ldept\mid\sex=0\right) - 1,\right.\\
%     & \left.1 - \Pr\left(\outcome=0,\dept=\ldept\mid\sex=1\right) \right. \\
%     &\left. -\Pr\left(\outcome=1,\dept=\ldept\mid\sex=0\right)\right].
% \end{align*}
\begin{align*}
    &\left[ \Pr\left(\outcome=1,\ldept\mid \sex=1\right) + \Pr\left(\outcome=0,\ldept\mid\sex=0\right) - 1,\right.\\
    & \left.1 - \Pr\left(\outcome=0,\ldept\mid\sex=1\right) -\Pr\left(\outcome=1,\ldept\mid\sex=0\right)\right].
\end{align*}
\fi 
For the interventional notion of fairness, the CDE must be $0$ for all $\ldept$. By setting the lower bound to be at most $0$ and the upper bound to be at least $0$, we recover the IV inequalities in \eqref{eq:iv}. While \citet{CaiKPT08} do not point out the connection to the IV inequalities, they find it ``remarkable that we [they] get such a simple formula, consisting of only one additive expression in the lower bound and one additive expression in the upper bound". In the next subsection, we show that the connection to the IV inequalities is not a coincidence.   

\subsection{A Family of Equivalent Tests}\label{subsec:equivalence}

% \todo{I'm not sure if this is the best way to present things... you spend 1.5 page on deriving bounds for the interventional H0. But then later on you see that you could have skipped that... Does the reader need to go through the same chronological order of understanding? Or is it better to take a different perspective (that would have saved us work if we would have had it initially)?}

The graphical and interventional fairness notions end up imposing identical constraints on the observational distribution.
%\todo{Is it clear they are actually different? Perhaps people would believe them to be equivalent!}
However, note that $\nullinter \supseteq \nullgraph$. In fact, we prove in Section~\ref{app:nested-cf} that $ \nullgraph = \nullctrf \subset \nullinter$. Models in $\nullinter \backslash \nullgraph$ (Example~\ref{ex:unconfexample}) are such that the edge $\sex \rightarrow \outcome$ exists in the causal graph and yet, the interventional queries in Definition~\ref{def:notions-cf} are equal.
% Therefore, models in $\nullinter \backslash \nullgraph$ violate certain faithfulness assumptions. 
Given that the null hypothesis of the interventional fairness notion is a strict superset of that of the graphical fairness notion, we might expect the same relation to hold in the resulting set of observational distributions for these hypotheses, thus giving us potentially different tests. In contrast, like in Section~\ref{sec:modeling}, we show that the corresponding sets of observational distributions resulting from models in $\nullinter$, $\nullctrf$, $\nullgraph$ are identical. Section~\ref{app:equiv-cf} contains the proof.

\begin{restatable}{theorem}{confequiv}\label{thm:equivalence}
Let 
\begin{align*}
\distgraph &\triangleq \left \lbrace P_{\model}\Paren{\dept,\outcome, \sex} : \model \in \nullgraph \right \rbrace, \\
\distinter &\triangleq \left \lbrace P_{\model}\Paren{\dept,\outcome, \sex} : \model \in \nullinter \right \rbrace, \\
\distctrf &\triangleq \left \lbrace P_{\model}\Paren{\dept,\outcome, \sex} : \model \in \nullctrf \right \rbrace.
\end{align*}
Then $\distinter = \distctrf = \distgraph = \distiv,$ where $\distiv$ is defined in Theorem~\ref{thm:iv_tight}.
\end{restatable}

In summary, testing for the graphical, interventional and counterfactual notions of fairness, with confounding, all boil down to testing the IV inequalities.

\subsection{Comparison With Existing Fairness Notions}

The utility of considering statistical tests is that we can now compare different fairness notions for a particular case with respect to the same causal modeling assumptions. In this section, we consider the three existing counterfactual fairness notions, namely the NDE \citep{NabiShpitser18, Chiappa19}, and the counterfactual and path-dependent counterfactual fairness notions in \citet{KusnerLRS17}. 

For NDE, \citet{KaufmanKMGP05} obtain bounds for the all-binary setting. Using these bounds, we obtain a strictly weaker test than the IV inequalities.\footnote{Intuitively, the reason is the same as in Section 3; the NDE averages over departments, and a positive bias in one department may cancel out against a similarly strong negative bias in another department. Hence, vanishing NDE does not imply that each department takes fair decisions.} We show in Section~\ref{app:ctrfkusner-cf} that the counterfactual notion of fairness of \citet{KusnerLRS17} implies demographic parity even when confounding is allowed. In Section~\ref{app:pathwise-cf} we show that testing the path-dependent counterfactual fairness notion of \citet{KusnerLRS17} is equivalent to testing the IV inequalities.

% and only provide a proof sketch here. We first show that $\nullinter \supseteq \nullctrf \superseteq \nullgraph$ which implies the same relations for $\distinter, \distctrf, \distgraph$. We express the observational distributions in $\distinter$ in terms of the response-function parameterization and show that 

% \begin{equation}\Pr\left(\outcome=1,\dept=\ldept|\sex=1\right) + \Pr\left(\outcome=0,\dept=\ldept|\sex=0\right) - 1 \leq 0,$$

% $$1 - \Pr\left(\outcome=0,\dept=\ldept|\sex=1\right) - \Pr\left(\outcome=1,\dept=\ldept|\sex=0\right) \geq 0.
% $$

% The above conditions, for all $\ldept$, are the same as the IV inequalities in \eqref{eq:iv}. As we shall see in the next subsection, this is not coincidence. 



% \begin{equation}\label{eq:inter_resp}
%         \Pr\Paren{\outcome=\loutcome\mid \doop{\formsex=\lsex'}, \doop{\dept=\ldept}} = \sum\limits_{\Paren{\respfunc_1,\respfunc_2}  \in \cX_{\response}}\bm{1}\Brack{\respfunc_2\Paren{\lsex',\ldept}=\loutcome}\tilde{P}\Paren{\respfunc_1,\respfunc_2}.
%    \Pr\Paren{\outcome=\loutcome\mid \dept=\ldept, \sex = \lsex, \doop{\formsex=\lsex'}} &= \sum\limits_{\Paren{\respfunc_1,\respfunc_2}  \in \cX_{\response}}\bm{1}\Brack{\respfunc_2\Paren{\lsex',\ldept}=\loutcome, \respfunc_1\Paren{\lsex} = \ldept}\tilde{P}\Paren{\respfunc_1,\respfunc_2}.\label{eq:cond_resp}
%\end{equation}

% Likewise, the conditional distribution $P_{\model}\Paren{\outcome,\dept|\sex}$ can also be expressed as a linear function of $\tilde{P}$, 

% \begin{equation}\label{eq:obs_resp}
%     P_{\model}\Paren{\outcome=\loutcome,\dept=\ldept|\sex=\lsex} = \sum\limits_{\Paren{\respfunc_1,\respfunc_2}  \in \cX_{\response}}\bm{1}\Brack{\respfunc_2\Paren{\lsex,\ldept}=\loutcome, \respfunc_1\Paren{\lsex} = \ldept}\tilde{P}\Paren{\respfunc_1,\respfunc_2}.
% \end{equation}

% Similar to \cite{Balke95}, we setup a symbolic linear program that derives upper and lower bounds in terms of the observational distribution $P_{\model}\Paren{\outcome,\dept \mid \sex}$ by maximizing and minimizing \eqref{eq:inter_resp} subject to symbolic constraints obtained from \eqref{eq:obs_resp}. These bounds can be used to bound the difference between the interventional queries in \eqref{eq:interfair}. The code to obtain these bounds is in the supplementary material.

% For the conditional notion of fairness, we first rewrite $\Pr\Paren{\outcome|\dept, \sex, \doop{\formsex}} = \frac{\Pr\Paren{\outcome, \dept| \sex, \doop{\formsex}}}{\Pr\Paren{\dept| \sex, \doop{\formsex}}}$. Note that the denominator, $\Pr\Paren{\dept| \sex, \doop{\formsex}}$ is identifiable and equal to $\Pr\Paren{\dept| \sex}$. The numerator can be expressed in terms of a linear function of $\tilde{P}$, 

% \begin{equation}\label{eq:cond_resp}
%        \Pr\Paren{\outcome=\loutcome, \dept=\ldept \mid \sex = \lsex, \doop{\formsex=\lsex'}} = \sum\limits_{\Paren{\respfunc_1,\respfunc_2}  \in \cX_{\response}}\bm{1}\Brack{\respfunc_2\Paren{\lsex',\ldept}=\loutcome, \respfunc_1\Paren{\lsex} = \ldept}\tilde{P}\Paren{\respfunc_1,\respfunc_2}.
% \end{equation}

% With a similar linear program setup, we obtain bounds on the difference of the interventional queries in \eqref{eq:cndfair}. In the next section we see how the IV inequalities and the bounds obtained in this subsection can be turned into a statistical test.

% For the interventional notion of fairness we express the bounds in closed form below. Due to space constraints, for the conditional notion of fairness the bounds are in the supplementary material. The bounds on $\Pr\Paren{\outcome=\loutcome\mid \doop{\formsex=\lsex'}, \doop{\dept=\ldept}}$ are 






% As we shall see in Section~\ref{sec:tests}, testing the graphical notion of fairness poses several challenges. These stem from the difficulty of obtaining constraints that a graphical query imposes on the set of observational distributions since a particular observational distribution can arise from multiple SCMs where the form of

% Since, it is easier to test fairness notions based on observational, interventional or counterfactual queries, we define fairness notions based on these probabilistic queries below. Another possible advantage of defining fairness notions based on probabilistic queries is the ability to analyze them under selection bias. 

% \begin{definition}[Interventional Notion of Fairness]
%     $\model \in \modelsunconfedge$ is fair according to the interventional notion of fairness if it belongs to 
%      \begin{equation}\label{eq:interfairunconf}
%     \nullinterunconf \triangleq \left \lbrace \model \in \modelsunconfedge: \forall \ldept, \Pr\Paren{\outcome=1|\doop{\sex=1},\doop{\dept=\ldept}} = \Pr\Paren{\outcome=1|\doop{\sex=0},\doop{\dept = \ldept}} \right \rbrace.
%     \end{equation}
%     $\model \in \modelsedge$ is fair according to the interventional notion of fairness if it belongs to
%     \begin{equation}\label{eq:interfair}
%     \nullinter \triangleq \left \lbrace \model \in \modelsedge : \forall \ldept, \Pr\Paren{\outcome=1|\doop{\formsex=1},\doop{\dept=\ldept}} = \Pr\Paren{\outcome=1|\doop{\formsex=0},\doop{\dept = \ldept}} \right \rbrace.
%     \end{equation}
%\end{definition}
% \todo{Notation: Pr() or P()? The latter was introduced earlier on as referring to the observational distribution of the SCM, so better to stick to that. And personally I usually write $P_M$ to emphasize the parameter(SCM)-dependence.}
% \todo{I suggest strengthening these to $P(A|do(S),do(D)) = P(A|do(D))$ (and $S'$ instead of $S$, respectively).}

% The criteria for the interventional notion of fairness for $\modelsunconfedge$ is that the controlled direct effect (CDE) \cite{?} of the ``treatment", $\sex$, on the ``outcome", $\outcome$, is $0$ for every value of the mediator, i.e., every department choice $\ldept$. We compare this notion with the counterfactual notion of natural direct effect (NDE) \cite{Pearl01} in the appendix. Although the NDE is identifiable for models in $\modelsunconfedge$, we show that since the expression of the NDE is a weighted average over the department choice, it is possible for the NDE to be $0$ despite there being discrimination \todo{'discrimination' is yet undefined, replace by specific fairness notion}.  

% If confounding between $\dept$ and $\outcome$ is allowed, the CDE and NDE are both unidentifiable. While bounds on the CDE and NDE are known \cite{KaufmanKMGP05, CaiKPT08}, \todo{sentence stops prematurely?} For $\modelsedge$, we now see the utility of modeling the reported sex on the application form separately. While interventions on $\sex$ are hypothetical and come with issues of interpretation \cite{HuKohlerHausmann20}, interventions on the reported sex, $\formsex$, are practically conceivable. Further, by disentangling the reported sex from the birth sex, we no longer require relying on counterfactual queries such as NDE.   

% In \cite{PearlMackenzie18}, Pearl mentions a disadvantage of CDE, namely intervening on department choice. An applicant's preparation might be targeted to a particular department of their choosing and this would influence the admissions committee's decision largely.\todo{I don't see how this would be problematic?} To circumvent this issue, we define a conditional notion of fairness where we condition on $\sex, \dept$ in addition to intervening on $\formsex$.

% \begin{definition}[Conditional Notion of Fairness]
%     $\model \in \modelsedge$ is fair according to the conditional notion of fairness if it belongs to 
%     \begin{equation}\label{eq:cndfair}
%     \nullcond \triangleq \left \lbrace \model \in \modelsedge : \forall \ldept, \lsex, \Pr\Paren{\outcome=1|\dept = \ldept, \sex = \lsex, \doop{\formsex=1}} = \Pr\Paren{\outcome=1|\dept = \ldept, \sex = \lsex, \doop{\formsex=0}} \right \rbrace.
%     \end{equation}
% \end{definition}

% The conditional notion of fairness is interpreted as mandating that for every sex-department combination, the applicants of a particular sex who apply to a certain department of their choice would have the same probability of admission if their sex on the application form were changed. Notice that despite being an interventional query, $\Pr\Paren{\outcome=1|\dept = \ldept, \sex = \lsex, \doop{\formsex\neq \lsex}}$ has a counterfactual `flavor'.  Further, despite conditioning on $\dept$, we escape the pitfalls that come along with it. We can allow for latent confounding, too, since by also conditioning on $\sex$, there are no longer any open path from $\formsex$ to $\outcome$ with $\dept$ as the collider. Note that this is made possible because of the node-splitting modeling trick.\todo{From my perspective, the conditional version was only relevant when dealing when selection bias, so can be left out of this story. Instead, we should define the counterfactual notion in order to relate to Kushner et al.}



\documentclass{article}
\usepackage{graphicx} % Required for inserting images

\title{XFlow}
\author{tailin Wu}
\date{January 2025}

\begin{document}

\maketitle

\section{Introduction}

\end{document}

\subsubsection{Bayesian Games}
\deni{Makes distributions bold.}
A \mydef{Bayesian game} \cite{harsanyi1968bayesian} $\bgames \doteq (\numplayers, \numactions, \numtypes, \typespace, \typedistrib, \param, \actionspace, \util)$ is a simultaneous-move game which consists of $\numplayers \in \N_+$ players each of whom is characterized by a type space $\typespace[\player] \subset \R^\numtypes$. \amy{characterized by a type in a type space, n'est-ce pas?}\deni{I think these comments are not relevant now that we are reverting to the classical Bayesian game setting no?}\amy{yes, maybe not.}
The players share a \mydef{common prior distribution} \amy{the players don't need a common prior if they see each other's types. they only need this in the independent case, where they see only their own type.}
\amy{the setup does not feel very Bayesian to me. it feels stochastic. like there is a distribution over complete-info games, as determined by the prior and the types, both(?) of which the players observe.} $\typedistrib[\param] \sim \simplex(\typespace)$ over the the \mydef{joint type space} $\typespace \doteq \bigtimes_{\player \in \players} \typespace[\player]$ defined by a \mydef{parameter} $\param \in \params$ coming from a parameter space $\params \subset \R^{\numparams}$.\deni{Make sure ``numparams'' dimensionality does not conflict with other var. def'n}
Each player $\player \in \players$ takes an action $\action[\player] \in \actionspace[\player]$ from its action space $\actionspace[\player] \subset \R^{\numactions}$, and receives a payoff $\util[\player](\action; \type)$ given by the payoff function $\util[\player]: \actionspace \times \typespace \to \R$. We denote the players' \mydef{joint action space} by $\actionspace \doteq \bigtimes_{\player \in \players} \actionspace[\player] \subset \R^{\numplayers \numactions}$ and the vector-valued function of all players' utilities $\util(\action; \type) \doteq \left( \util[\player](\action; \type) \right)_{\player \in \players}$.


% and $\typespace \doteq \bigtimes_{\player \in \players} \typespace[\player] \subset \R^{\numplayers \numtypes}$, and refer to any collection of actions $\action = (\action[1], \hdots, \action[\numplayers]) \in \actionspace$ and types $\type = (\type[1], \hdots, \type[\numplayers]) \in \typespace$ as an \mydef{action profile} and \mydef{type profile} respectively.
% At each time-step $\iter = 0, 1, \hdots$, each player $\player \in \players$ takes an action $\inner[\player][][][\iter] \in \innerset[\player](\context[\iter])$ from an action space $\innerset[\player](\context[\iter]) \subset \actionspace[\player]$\footnote{Going forward, for simplicity, without loss of generality, we drop the dependence of the action space $\innerset[\player]$ on the context $\context$ and take for all player $\player \in \players$, $\innerset[\player] \doteq \actionspace[\player]$.} simultaneously observe a \mydef{context} $\typerv[][\iter] \sim \initcontexts$, i.e., a type profile drawn from a distribution $\initcontexts \in \simplex(\typespace)$ over the set of types. Each player $\player \in \players$ , it takes an action $\inner[\player][][][\iter] \in \innerset[\player](\context[\iter])$ from an action space $\innerset[\player](\context[\iter]) \subset \actionspace[\player]$\footnote{Going forward, for simplicity, without loss of generality, we drop the dependence of the action space $\innerset[\player]$ on the context $\context$ and take for all player $\player \in \players$, $\innerset[\player] \doteq \actionspace[\player]$.}

% Once player $\player \in \players$ has observed the realized contexts $\context[\iter]$, it takes an action $\inner[\player][][][\iter] \in \innerset[\player](\context[\iter])$ from an action space $\innerset[\player](\context[\iter]) \subset \actionspace[\player]$\footnote{Going forward, for simplicity, without loss of generality, we drop the dependence of the action space $\innerset[\player]$ on the context $\context$ and take for all player $\player \in \players$, $\innerset[\player] \doteq \actionspace[\player]$.} and receive reward $\reward[\player](\context, \inner; \type[\player])$ given by a reward function $\reward[\player]: \contexts \times \innerset \times \typespace \to \R$. For a given context, action profile, and type profile tuple, we denote the vector of all  players utilities by $(\context, \action, \type) \mapsto \reward(\context, \action; \type)$  

A Bayesian game is \mydef{continuous} if for all $\type \in \contexts$, $\util(\action; \type)$ is continuous in $\action$ and $\actionspace$ is non-empty and compact.
A Bayesian game is \mydef{concave} if in addition to being continuous, for all types $\type \in \typespace$ and for all players $\player \in \players$, $\util[\player](\action; \type)$ is concave in $\action[\player]$ and $\actionspace[\player]$ is convex.

% A \mydef{joint strategy profile} $\strat: \typespace \to \actionspace$ is a mapping from the joint type space to the joint action space s.t.\ $\strat[\player](\type) \in \actionspace[\player]$ denotes the action played by player $\player$ under type profile $\type \in \typespace$. 
An \mydef{strategy} 
% \deni{Decide if we want discuss the whole decentralized/centralized issue} \deni{I think yes?}
$\strat[\player]: \typespace[\player] \to \actionspace[\player]$ for a player $\player \in \players$, is a mapping from player $\player$'s type to an action s.t. $\strat[\player](\type[\player]) \in \actionspace[\player]$ denotes the action played by player $\player$ when it is of type $\type[\player]$. 
A \mydef{strategy profile} $\strat \doteq \left( \strat[1], \hdots, \strat[\numplayers]\right)$ is a collection of independent strategies, one-per-player, s.t. for any type profile $\type \in \typespace$, $\strat(\type) \in \actionspace$ denotes the action profile played by the players.
% \amy{i think you need a different letter for independent strategy profiles. maybe $\tau$. overloading with $\sigma$ will probably be too confusing (and you can always change the macro later if it isn't).}

An \mydef{ex-ante $\varepsilon$-Bayesian Nash equilibrium} 
% \amy{maybe parameterized instead of Bayesian}\deni{We said no!} 
($\varepsilon$\mydef{-BNE}) is a strategy profile $\strat[][][*] \in \actionspace^\typespace$ s.t.\ for all players $\player \in \players$ and strategy profiles $\strat \in \actionspace^\typespace$, $\Ex_{\typerv \sim \typedistrib[\param]} \left[ \util[\player] (\strat[][][*](\typerv); \typerv)\right] \geq \Ex_{\typerv \sim \typedistrib[\param]} \left[ \util[\player] (\strat[\player][][](\typerv), \strat[-\player][][*] (\typerv); \typerv)\right] - \varepsilon$. 
A $\varepsilon$-ex-post Nash equilibrium ($\varepsilon$-EPNE) is a strategy profile $\strat[][][*] \in \actionspace^\typespace$ s.t. for all players $\player \in \players$, types $\type \in \typespace$, and strategy profiles $\strat \in \actionspace^\typespace$, $\util[\player](\strat[][][*](\type); \type) \geq \util[\player](\strat[\player][][](\type), \strat[-\player][][*](\type); \type) - \varepsilon$. 
A $0$-BNE and $0$-EPNE are simply called BNE and EPNE, respectively.
For concave games, a BNE is guaranteed to exist, while EPNE are not guaranteed exist.%
% \footnote{Traditionally, BNE and EPNE are only defined for independent strategies, in which case EPNE may not exist. Our more general definition applies to strategies that depend on all players' types. \amy{again, it doesn't feel very ex-post.}} 

\deni{We have to define the set of Nash equilibria and EPNE?} \deni{Update: Maybe not}

% \deni{Ignore next para, Old def'n, to remove, keeping it for visual reminder}
% An \mydef{ex-ante $\varepsilon$-Bayesian Nash equilibrium} (joint $\varepsilon$\mydef{-BNE}) is a strategy profile $\strat[][][*] \in \actionspace^\typespace$ s.t. for all players $\player \in \players$ and strategy profiles $\strat \in \actionspace^\typespace$, $\Ex_{\typerv \sim \typedistrib} \left[ \util[\player](\strat[][][*](\typerv); \typerv)\right] \geq \Ex_{\typerv \sim \typedistrib} \left[ \util[\player](\strat[\player][][](\typerv[\player]), \strat[-\player][][*](\typerv[-\player]); \typerv)\right]$. A $\varepsilon$-ex-post Nash equilibrium ($\varepsilon$-EPNE) is a strategy profile $\strat[][][*] \in \actionspace^\typespace$ s.t. for all players $\player \in \players$, types $\type \in \typespace$ and strategy profiles $\strat \in \actionspace^\typespace$, $\util[\player](\strat[][][*](\type); \type) \geq \util[\player](\strat[\player][][](\type[\player]), \strat[-\player][][*](\type[-\player]); \type)$.

% \amy{BNE seems different to me, b/c i don't think $\strat[][][*](\type) = (\strat[\player][][*](\type[\player]), \strat[-\player][][*](\type[-\player]))$. i think it might equal $(\strat[\player][][*](\type), \strat[-\player][][*](\type))$, in which case you have $\util[\player](\strat[][][*](\type); \type) = \util[\player](\strat[\player][][*](\type), \strat[-\player][][*](\type); \type)$ not $\util[\player](\strat[][][*](\type); \type) = \util[\player](\strat[\player][][*](\type[\player]), \strat[-\player][][*](\type[-\player]); \type)$. in the former case, parameters are known to all---it is a complete-info game; in the latter, players know only their own parameters, which is very different from an info-theoretic point of view (so it feels like we are comparing apples to oranges).}

% A \mydef{(Markov or stationary) policy} \cite{maskin2001markov} for player $\player \in \players$, $\policy[\player]: \contexts \to \innerset[\player]$ is a mapping from contexts to actions such that $\policy[\player](\context) \in \innerset[\player]$ denotes the action taken by player $\player$ when it observes context $\context$. We define a \mydef{policy profile} as the collection of policies, i.e., $\policy = (\policy[1], \hdots, \policy[\numplayers]) : \contexts \to \innerset$ such that $\policy(\context) \in \innerset$ denotes the action profile played by the players under context $\context \in \contexts$. The goal of all players $\player \in \players$ is to play a policy $\policy[\player][][*] \in \innerset[\player]^\context$ which maximizes their \mydef{expected cumulative payoff} $\util[\player](\policy[\player], \policy[-\player]; \type[\player]) \doteq \Ex_{\contextrv \sim \initcontexts} \left[ \reward[\player](\contextrv, \policy[\player](\contextrv), \policy[-\player](\contextrv); \type[\player])\right]$. 
% % 


% A \mydef{$\varepsilon$-subgame perfect Nash equilibrium ($\varepsilon$-SPNE)} of a contextual game is a policy profile $\policy[][][*] \in \innerset^\contexts$ such that for all player $\player \in \players$ and for all contexts $\context \in \contexts$,  $\reward[\player](\context, \policy[][][*](\context); \type[\player]) \geq \reward[\player](\policy[\player](\context), \policy[-\player][][*](\context); \type[\player]) - \varepsilon$. A $0$-SPNE is called a \mydef{subgame perfect Nash equilibrium (SPNE)}. A SPNE is guaranteed to exist in concave contextual games. Additionally, note that when the context space is a singleton, then the SPNE of contextual game simply reduce to the NE of the game played at that context. Further, notice that the SPNE of contextual games with different context distributions are all the same.

% \deni{Ex-ante definition is commented out below. Might want to make the introduction of the type inside the paranthesis not as a superscript.}

\deni{Need to make this a function of the parameter distribution and not a function of the parameter, and us the same expectation overload as the Markov game paper.}
Fixing the payoff functions of the players $\util$, for any type $\type \in \typespace$, we define the 
% \mydef{regret} $\regret[][\util]: \actionspace \times \actionspace \to \R^\numplayers$ for playing an action profile $\action$ as compared to another action profile $\otheraction$, as follows: for all followers $\player \in \players$,
% $\regret[\player][](\action, \otheraction; \outer) = \util[\player](\outer, (\otheraction[\player], \action[-\player])) - \util[\player](\outer, \action)$. The 
\mydef{cumulative regret}, $\cumulregret[][]: \actionspace \times \actionspace \times \typespace \to \R$ between two action profiles $\action \in \actionspace$ and $\otheraction \in \actionspace$ across all players in a game as $\cumulregret[][](\action, \otheraction; \type) = \sum_{\player \in \players} \util[\player](\otheraction[\player], \action[-\player]; \type) - \util[\player](\action; \type)$.
Further, the \mydef{exploitability} or (Nikaido-Isoda potential function \cite{nikaido1955note}) of an action profile $\action \in \actionspace$ is defined as 
$\exploit[][](\action; \type) = \max_{\otheraction \in \actionspace} \cumulregret[][](\action, \otheraction; \type)$ \cite{goktas2022exploit}. 
Overloading notation,
for any common prior distribution parameter $\param \in \params$, we define the \mydef{ex-ante cumulative regret} at any given type profile $\type \in \typespace$, $\cumulregret[][\param]: \actionspace^\typespace \times \actionspace^\typespace \to \R$ between two strategy profiles $\strat \in \actionspace^\typespace$ and $\otherstrat \in \actionspace^\typespace$ across all players as $\cumulregret[][\param](\strat, \otherstrat) = \sum_{\player \in \players} \left( \Ex_{\typerv \sim \typedistrib[\param]} \left[ \util[\player](\otherstrat[\player][][](\typerv[\player]), \strat[-\player][][](\typerv[-\player]); \typerv)\right] -\Ex_{\typerv \sim \typedistrib[\param]} \left[ \util[\player](\strat(\typerv); \typerv)\right] \right)$.
Further, the \mydef{ex-ante exploitability} or (Nikaido-Isoda potential function \cite{nikaido1955note}) of a strategy profile $\strat \in \actionspace^\typespace$ is defined as 
$\exploit[][\param](\strat) = \max_{\otherstrat \in \actionspace^\typespace} \cumulregret[][\param](\strat, \otherstrat)$ \cite{goktas2022exploit}. We note that for all $\strat \in \actionspace^\typespace$, $\exploit[][\param](\strat) \geq 0$, and $\strat[][][*]$ is a BNE of $\game$ iff $\exploit[][\param](\strat[][][*]) = 0$. 
% For
% for any type profile $\type \in \typespace$,
% we define the \mydef{ex-post cumulative regret} and the \mydef{ex-post exploitability} respectively as 
% $\cumulregret[][\type](\strat, \otherstrat) \doteq \sum_{\player \in \players} \left[ \util[\player](\otherstrat[\player][][](\type[\player]), \strat[-\player][][](\type[-\player]); \type) - \util[\player](\strat(\type); \type) \right]$
% $\cumulregret[][] (\action, \otheraction; \type) = \sum_{\player \in \players} \util[\player] (\otheraction[\player], \action[-\player]; \type) - \util[\player] (\action; \type)$
% and $\exploit[][](\strat; \type) \doteq \max_{\otherstrat \in \actionspace} \cumulregret[][] (\strat, \otherstrat; \type)$.
\deni{I think these two comments are now answered! Seems to be a typo from before! Alec, I also added two macros into ``auxiliary/filecommands.sty'' for you to add comments make edits if you prefer that!}\amy{not sure we are maxing over the right thing here. still wondering about strategies. actually, maybe we are, but $\otherstrat$ should be $\otheraction$. and i feel like we need an expectation of types $T$.}
\alec{$\rho$ seems undefined here. Also unclear what is the difference between $\mathcal{A}$ and $\mathcal{A}^{\mathcal{T}}$ } \deni{$\actionspace$ is the joint action space, while $\actionspace^\typespace$ is the space of joint strategy profiles, i.e. mappings from typespace to action space. } \deni{Although might just need to define the strategy space because there is a problem with def'n, i.e., the current notation also includes centralized strategies.}

% \subsubsection{Contextual Games}
% A \mydef{(simultaneous-move) contextual game} \cite{sessa2020contextual} $(\numplayers, \numactions, \contexts, \initcontexts, \innerset, \typespace, \type, \reward)$ is a repeated game played over an infinite horizon which comprises of $\numplayers \in \N_+$ players each of whom is characterized by a type space $\typespace[\player] \in \subset \R^\numtypes$. At each time-step $\iter = 0, 1, \hdots$, each player $\player \in \players$ takes an action $\inner[\player][][][\iter] \in \innerset[\player](\context[\iter])$ from an action space $\innerset[\player](\context[\iter]) \subset \actionspace[\player]$\footnote{Going forward, for simplicity, without loss of generality, we drop the dependence of the action space $\innerset[\player]$ on the context $\context$ and take for all player $\player \in \players$, $\innerset[\player] \doteq \actionspace[\player]$.} simultaneously observe a \mydef{context} $\typerv[][\iter] \sim \initcontexts$, i.e., a type profile drawn from a distribution $\initcontexts \in \simplex(\typespace)$ over the set of types. Each player $\player \in \players$ , it takes an action $\inner[\player][][][\iter] \in \innerset[\player](\context[\iter])$ from an action space $\innerset[\player](\context[\iter]) \subset \actionspace[\player]$\footnote{Going forward, for simplicity, without loss of generality, we drop the dependence of the action space $\innerset[\player]$ on the context $\context$ and take for all player $\player \in \players$, $\innerset[\player] \doteq \actionspace[\player]$.}

% % Once player $\player \in \players$ has observed the realized contexts $\context[\iter]$, it takes an action $\inner[\player][][][\iter] \in \innerset[\player](\context[\iter])$ from an action space $\innerset[\player](\context[\iter]) \subset \actionspace[\player]$\footnote{Going forward, for simplicity, without loss of generality, we drop the dependence of the action space $\innerset[\player]$ on the context $\context$ and take for all player $\player \in \players$, $\innerset[\player] \doteq \actionspace[\player]$.} and receive reward $\reward[\player](\context, \inner; \type[\player])$ given by a reward function $\reward[\player]: \contexts \times \innerset \times \typespace \to \R$. For a given context, action profile, and type profile tuple, we denote the vector of all  players utilities by $(\context, \action, \type) \mapsto \reward(\context, \action; \type)$  

% A contextual game is \mydef{continuous} if for all $\context \in \contexts$, $\reward(\context, \inner; \type)$ is continuous in $\inner$ and $\innerset$ is non-empty and compact.
% A contextual game is \mydef{concave} if in addition to being continuous, for all contexts $\context \in \contexts$ and for all players $\player \in \players$, $\reward[\player](\context, \inner)$ is concave in $\inner[\player]$ and $\innerset[\player]$ is convex.

% A \mydef{(Markov or stationary) policy} \cite{maskin2001markov} for player $\player \in \players$, $\policy[\player]: \contexts \to \innerset[\player]$ is a mapping from contexts to actions such that $\policy[\player](\context) \in \innerset[\player]$ denotes the action taken by player $\player$ when it observes context $\context$. We define a \mydef{policy profile} as the collection of policies, i.e., $\policy = (\policy[1], \hdots, \policy[\numplayers]) : \contexts \to \innerset$ such that $\policy(\context) \in \innerset$ denotes the action profile played by the players under context $\context \in \contexts$. The goal of all players $\player \in \players$ is to play a policy $\policy[\player][][*] \in \innerset[\player]^\context$ which maximizes their \mydef{expected cumulative payoff} $\util[\player](\policy[\player], \policy[-\player]; \type[\player]) \doteq \Ex_{\contextrv \sim \initcontexts} \left[ \reward[\player](\contextrv, \policy[\player](\contextrv), \policy[-\player](\contextrv); \type[\player])\right]$. 
% % 


% A \mydef{$\varepsilon$-subgame perfect Nash equilibrium ($\varepsilon$-SPNE)} of a contextual game is a policy profile $\policy[][][*] \in \innerset^\contexts$ such that for all player $\player \in \players$ and for all contexts $\context \in \contexts$,  $\reward[\player](\context, \policy[][][*](\context); \type[\player]) \geq \reward[\player](\policy[\player](\context), \policy[-\player][][*](\context); \type[\player]) - \varepsilon$. A $0$-SPNE is called a \mydef{subgame perfect Nash equilibrium (SPNE)}. A SPNE is guaranteed to exist in concave contextual games. Additionally, note that when the context space is a singleton, then the SPNE of contextual game simply reduce to the NE of the game played at that context. Further, notice that the SPNE of contextual games with different context distributions are all the same.

% \sdeni{}{Overloading notation, we define the \mydef{contextual cumulative regret} at any given type profile $\type \in \typespace$, $\cumulregret[][\type]: \contexts \times \actionspace \times \actionspace \to \R$ between two action profiles $\action \in \actionspace$ and $\otheraction \in \actionspace$ across all players in context $\context$ of a contextual game as $\cumulregret[][\type](\context, \action, \otheraction) = \sum_{\player \in \players} \reward[\player](\context, (\otheraction[\player], \action[-\player]); \type[\player]) - \reward[\player](\context, \action; \type[\player])$.
% Further, the \mydef{contextual exploitability} or (Nikaido-Isoda potential function \cite{nikaido1955note}) of an action profile $\action \in \actionspace$ is defined as 
% $\exploit[][\type](\context, \action) = \max_{\otheraction \in \actionspace} \cumulregret[][\type](\context, \action, \otheraction)$ \cite{goktas2022exploit}. We note that for all $\context \in \contexts$, $\action \in \actionspace$, $\exploit[][\type](\context, \action) \geq 0$, and $\action[][][][*]$ is a SPNE of $(\numplayers, \numactions, \actionspace, \typespace, \type, \util)$ iff $\exploit[][\type](\action[][][][*]) = 0$.} 
% $\util[\player](\policy[][][*]) \geq \util[\player](\policy[\player], \policy[-\player][][*]) - \varepsilon$

% \footnote{Note that although the set of $\varepsilon$-SPNE of any contextual game $(\numplayers, \numactions, \contexts, \initcontexts, \innerset, \reward)$ is also a Nash game $(\numplayers, \numactions, \innerset^\contexts, \util)$, this reduction is mostly vacuous, as Nash's theorem \cite{nash1950existence} or Arrow-Debreu's lemma on abstract economies \cite{arrow-debreu} does not provide existence in this constructed Nash game.}
 % is a policy profile $\policy[][][*] \in \innerset^\contexts$ s.t.\ for all players $\player \in \players$ and actions $\action[\player] \in \actionspace[\player]$, $\util[\player](\action[][][][*]) \geq \util[\player](\action[\player], \action[-\player][][][*]) - \varepsilon$.
% , and for each player $\player \in \players$, an action correspondence $\innerset[\player]: \contexts \rightrightarrows \actionspace$ s.t. for any context $\context \in \contexts$ and any player $\player \in \players$, $\innerset[\player](\context) \subset \actionspace$ denotes the set of actions player $\player$ can choose under context $\context$,  who , encounter a Nash game $(\numplayers, \numactions, \innerset(\contextrv), \util())$ 

\if 0 
\deni{Decide how to incorporate this into the big story.}
\paragraph{Stackelberg-Nash Games}
An $(\numplayers + 1)$-player \mydef{Stackelberg-Nash game} $\stackgame \doteq (\numplayers, \numactions, \outerset, \innerset, \util)$ comprises one player called the \mydef{leader} and $\numplayers \in \N_{++}$ players called \mydef{followers}.
In a Stackelberg-Nash game, the leader first commits to an action $\outer \in \outerset$ from an action space $\outerset \subset \R^{\outerdim}$.
Then, having observed the leader's action, each follower $\player \in \players$, responds with an action $\inner[\player]$ in their \mydef{action space} $\innerset[\player] \subset \R^{\numactions}$.
% determined by the \mydef{feasible action correspondence} $\innerset[\player]: \outerset \rightrightarrows \actionspace[\player]$ which takes as input the leader's action $\outer$ and outputs a subset of \mydef{the action space} $\actionspace[\player] \subset \R^{\numactions}$. 
We define the \mydef{followers' joint action space} $\innerset \doteq \bigtimes_{\player \in \players} \innerset[\player]$. 
% and the \mydef{follower joint feasible action correspondence} by $\innerset(\outer) = \bigtimes_{\player \in \players} \innerset[\player](\outer) \subset \bigtimes_{\player \in \players} \actionspace[] \subset \R^{\numplayers \numactions}$
We refer to a collection of actions $\inner = (\inner[1], \hdots, \inner[\numplayers]) \in \innerset$ as a \mydef{follower action profile}, and to a collection $(\outer, \inner) \in \outerset \times \innerset$ comprising an action for the leader and a follower action profile as simply an \mydef{action profile}. 
% \deni{Maybe also define feasible action profile.} A Stackelberg-Nash game is said to have \mydef{independent action sets} if the feasible action correspondence of each player $\player \in \players$, $\innerset[\player]$ is a constant correspondence, i.e., $\innerset[\player](\outer) = \innerset[\player](\outer[][][\prime]) = \actionspace$ for all leader actions $\outer, \outer[][][\prime] \in \outerset$.

After all players choose an action, the leader receives payoff $\util[0](\outer, \inner) \in \R$, while each follower $\player \in \players$ receives payoff $\util[\player](\outer, \inner) \in \R$. 
Each player $\player \in \allplayers$ aims to maximize her payoff $\util[\player]: \outerset \times \innerset \to \R$. 
For all followers $\player \in \players$, we define the $\delta$-\mydef{best-response correspondence} $\br[\player][\delta] (\outer, \inner[-\player]) \doteq \left\{ \inner[\player] \in \innerset \mid \util[\player](\outer, \inner[][][]) \geq \max_{\inner[\player] \in \innerset[\player]} \util[\player] (\outer, (\inner[\player], \inner[-\player][][])) - \delta \right\}$ and the \mydef{joint $\delta$-best-response correspondence} $\br[][\delta](\outer, \inner) \doteq \bigtimes_{\player \in \players} \br[\player][\delta] (\outer, \inner[-\player])$. 

%\amy{we use the semi-colon notation when talking about exploitability. should we use it here as well? i tend to think yes. e.g., $\br[\player][\delta] (\inner[-\player]; \outer)$}\deni{I feel like we don't have to because this is the follower's best response in the Stackelberg game. While for exploitability that is the exploitability of the lower level game which the leader's action parametrizes so it makes sense.}

Accordingly, we define the \mydef{follower regret} $\regret[][]: \innerset \times \innerset \times \outerset \to \R^\numplayers$ for playing an action profile $\inner$ as compared to another action profile $\otherinner$ when the leader plays $\outer \in \outerset$, as follows: for all followers $\player \in \players$,
$\regret[\player][](\inner, \otherinner; \outer) = \util[\player](\outer, (\otherinner[\player], \inner[-\player])) - \util[\player](\outer, \inner)$. 
The \mydef{follower cumulative regret}, $\cumulregret: \innerset \times \innerset \times \outerset \to \R$ between two action profiles $\inner \in \innerset$ and $\otherinner \in \innerset$ across all players in a game is given by $\cumulregret(\inner, \otherinner; \outer) = \sum_{\player \in \players} \util[\player](\outer, (\otherinner[\player], \inner[-\player])) - \util[\player](\outer, \inner)$.
Further, the \mydef{follower exploitability} or (Nikaido-Isoda potential function \cite{nikaido1955note}) of a follower action profile $\inner \in \innerset$ is defined as 
$\exploit(\inner) = \max_{\otherinner \in \innerset} \cumulregret(\inner, \otherinner)$ \cite{goktas2022exploit}. 

% \if 0
% The canonical solution concept for Stackelberg-Nash games is the $(\varepsilon, \delta)$-\mydef{Stackelberg-Nash equilibrium (SNE)}, an action profile $(\outer[][][*], \inner[][][*]) \in \innerset \times \outerset$ such that:
% \begin{align}
%     \util[0](\outer[][][*], \inner[][][*]) &\geq \max_{\outer \in \outerset: \inner \in \br[][\delta](\outer, \inner)} \util[0](\outer, \inner) - \varepsilon \\
%     \util[\player](\outer[][][*], \inner[][][*]) &\geq \max_{\inner[\player] \in \innerset[\player]} \util[\player](\outer[][][*], (\inner[\player], \inner[-\player][][*])) - \delta&& \forall \player \in \players
% \end{align}
% A $(0,0)$-Stackelberg-Nash equilibrium is simply called a Stackelberg-Nash equilibrium.
% Intuitively, a $(\varepsilon, \delta)$-SNE is an action profile at which the followers play a Nash equilibrium, while the leader $\varepsilon$-approximately maximizes its payoff over its action space, assuming that the followers will play a $\varepsilon$-Nash equilibrium for any of its actions. 
% \fi


% \deni{Can remove the eqm defs, since they're in the intro.}
% \amy{not sure, b/c we don't define $\epsilon-\delta$-SE.}


As the joint best-response correspondence is not necessarily singleton-valued, the leader's objective is likewise a correspondence: i.e., multiple values could be associated with a fixed strategy $\outer \in \outerset$.
As a result, we cannot re-formulate this problem as a single objective optimization without fixing a selection criteria over the followers' joint best-responses.
The results we prove in this paper rely on the strong Stackelberg-Nash Equilibrium as a solution concept:

\begin{definition}[Strong Stackelberg-Nash Equilibrium]
A $(\varepsilon, \delta)$-\mydef{strong Stackelberg-Nash equilibrium (SSNE)} is an action profile $(\outer, \inner) \in \outerset \times \innerset$ s.t.
% 
%\begin{align}
     $\util[0](\outer[][][*], \inner[][][*]) \geq \max_{\outer \in \outerset} \max_{\inner \in \br[][\delta](\outer, \inner)} \util[0](\outer, \inner) - \varepsilon$ and
     $\util[\player](\outer[][][*], \inner[][][*]) \geq \max_{\inner[\player] \in \innerset[\player]} \util[\player](\outer[][][*], (\inner[\player], \inner[-\player][][*])) - \delta$, for all $\player \in \players$.
%\end{align}

% \begin{align}
%     \outer[][][*] &\in \argmax_{\outer \in \outerset} \max_{\inner \in \br(\outer, \inner)} \util[0](\outer, \inner)\\
%     \inner[\player][][*] &\in \argmax_{\inner[\player] \in \innerset[\player]} \util[\player](\outer[][][*], (\inner[\player], \inner[-\player][][*])) && \forall \player \in \players
% \end{align}
\end{definition}
\fi
% \begin{definition}[Weak Stackelberg Equilibrium]
% A $(\varepsilon, \delta)$-\mydef{weak Stackelberg-Nash equilibrium (WSNE)} is an action profile $(\outer, \inner) \in \outerset \times \innerset$ s.t.
% %
% %\begin{align}
%      $\util[0](\outer[][][*], \inner[][][*]) \geq \min_{\inner \in \br[][\delta](\outer, \inner)} \util[0](\outer, \inner) - \varepsilon$, for all $\outer \in \outerset$, and
%      $\util[\player](\outer[][][*], \inner[][][*]) \geq  \max_{\inner[\player] \in \innerset[\player]}\util[\player](\outer[][][*], (\inner[\player], \inner[-\player][][*])) - \delta$, for all $\player \in \players$.
% %\end{align}
% \end{definition}

%%% SPACE
% %
% \if 0
% In these definitions, the leader approximately optimizes its strategy assuming the followers approximately optimize theirs, in which case $\delta \geq 0$. \amy{i think what this means is that $\epsilon$ is a function of $\delta$.} 
% As a result, a $(0,0)$-SSNE/WSNE might not be a $(0, \delta)$-SSNE/WSNE in general.
% \deni{Can we please discuss this, I want to add a footnote but not sure how to phrase it. There is this interesting phenonmenon in the general sum-setting where computing an approximate equilibrium might be hard because the follower's approximate best-response can induce discontinuous change in the *equilibrium* strategy of the leader, and as a result a $(0,0)$-SSNE/WSNE might not be a $(0, \delta)$-SSNE/WSNE. }
% \sdeni{}{Note that an important difference above approximate Stackelberg-Nash definitions are much harder to to compute }
% \amy{i know we discussed yesterday, but need to discuss again.} \amy{i think you might want to move this discussion to right after Obs 1. it might be easier to explain there.}
% \fi

% \deni{The reason why we need the joint convexity assumption is because our goal is to compute a VE, which does not generally exist (and because projection onto non-convex sets is often hard!).}

% We also define local versions of these equilibrium concepts, which are more tractable in general. A pseudo-game is simply called a \mydef{game} if for all players $\player \in \players$, $\actions(\naction[\player])$ is a constant correspondence, i.e., for all players $\player \in \players$, and strategy profiles $\action, \otheraction \in \innerset, \actions(\naction[\player]) = \actions(\otheraction[-\player])$.


%A \mydef{local generalized Nash equilibrium (local GNE)} is an strategy profile $\inner[][][*]$ s.t.\ for all players $\player \in \players$, and $\inner[\player] \in \actions[\player](\naction[\player][][][*]) \cap \ball[\varepsilon][{\inner[\player]}]$, \amy{fix me!} $\util[\player](\inner[][][*]) \geq \util[\player](\inner[\player], \naction[\player][][][*])$.
% A \mydef{local variational equilibrium (local VE) or normalized GNE} is an strategy profile $\inner[][][*]$ s.t.\ for all strategy profiles and $\action \in \innerset \cap \ball[\varepsilon][\action]$ s.t.\ $\actionconstr(\action) \geq \zeros$, $\util[\player](\inner[][][*]) \geq \util[\player](\inner[\player], \naction[\player][][][*])$. 


% $\hypothesis \in \argmin_{\R^{\states \times \actionspace}} \nicefrac{1}{\numsamples} \sum_{\numsample \in [\numsamples]} \left\| \mathrm{Nash}\left(\hypothesis[][*](\state[\numsample]\right) - \inner[][][][\numsample]  \right\|^2$ and for all $\state \in \states$, $\hypothesis[][*](\state[\numsample])$ is a Nash equilibrium of $$.
\section{Discussion of Assumptions}\label{sec:discussion}
In this paper, we have made several assumptions for the sake of clarity and simplicity. In this section, we discuss the rationale behind these assumptions, the extent to which these assumptions hold in practice, and the consequences for our protocol when these assumptions hold.

\subsection{Assumptions on the Demand}

There are two simplifying assumptions we make about the demand. First, we assume the demand at any time is relatively small compared to the channel capacities. Second, we take the demand to be constant over time. We elaborate upon both these points below.

\paragraph{Small demands} The assumption that demands are small relative to channel capacities is made precise in \eqref{eq:large_capacity_assumption}. This assumption simplifies two major aspects of our protocol. First, it largely removes congestion from consideration. In \eqref{eq:primal_problem}, there is no constraint ensuring that total flow in both directions stays below capacity--this is always met. Consequently, there is no Lagrange multiplier for congestion and no congestion pricing; only imbalance penalties apply. In contrast, protocols in \cite{sivaraman2020high, varma2021throughput, wang2024fence} include congestion fees due to explicit congestion constraints. Second, the bound \eqref{eq:large_capacity_assumption} ensures that as long as channels remain balanced, the network can always meet demand, no matter how the demand is routed. Since channels can rebalance when necessary, they never drop transactions. This allows prices and flows to adjust as per the equations in \eqref{eq:algorithm}, which makes it easier to prove the protocol's convergence guarantees. This also preserves the key property that a channel's price remains proportional to net money flow through it.

In practice, payment channel networks are used most often for micro-payments, for which on-chain transactions are prohibitively expensive; large transactions typically take place directly on the blockchain. For example, according to \cite{river2023lightning}, the average channel capacity is roughly $0.1$ BTC ($5,000$ BTC distributed over $50,000$ channels), while the average transaction amount is less than $0.0004$ BTC ($44.7k$ satoshis). Thus, the small demand assumption is not too unrealistic. Additionally, the occasional large transaction can be treated as a sequence of smaller transactions by breaking it into packets and executing each packet serially (as done by \cite{sivaraman2020high}).
Lastly, a good path discovery process that favors large capacity channels over small capacity ones can help ensure that the bound in \eqref{eq:large_capacity_assumption} holds.

\paragraph{Constant demands} 
In this work, we assume that any transacting pair of nodes have a steady transaction demand between them (see Section \ref{sec:transaction_requests}). Making this assumption is necessary to obtain the kind of guarantees that we have presented in this paper. Unless the demand is steady, it is unreasonable to expect that the flows converge to a steady value. Weaker assumptions on the demand lead to weaker guarantees. For example, with the more general setting of stochastic, but i.i.d. demand between any two nodes, \cite{varma2021throughput} shows that the channel queue lengths are bounded in expectation. If the demand can be arbitrary, then it is very hard to get any meaningful performance guarantees; \cite{wang2024fence} shows that even for a single bidirectional channel, the competitive ratio is infinite. Indeed, because a PCN is a decentralized system and decisions must be made based on local information alone, it is difficult for the network to find the optimal detailed balance flow at every time step with a time-varying demand.  With a steady demand, the network can discover the optimal flows in a reasonably short time, as our work shows.

We view the constant demand assumption as an approximation for a more general demand process that could be piece-wise constant, stochastic, or both (see simulations in Figure \ref{fig:five_nodes_variable_demand}).
We believe it should be possible to merge ideas from our work and \cite{varma2021throughput} to provide guarantees in a setting with random demands with arbitrary means. We leave this for future work. In addition, our work suggests that a reasonable method of handling stochastic demands is to queue the transaction requests \textit{at the source node} itself. This queuing action should be viewed in conjunction with flow-control. Indeed, a temporarily high unidirectional demand would raise prices for the sender, incentivizing the sender to stop sending the transactions. If the sender queues the transactions, they can send them later when prices drop. This form of queuing does not require any overhaul of the basic PCN infrastructure and is therefore simpler to implement than per-channel queues as suggested by \cite{sivaraman2020high} and \cite{varma2021throughput}.

\subsection{The Incentive of Channels}
The actions of the channels as prescribed by the DEBT control protocol can be summarized as follows. Channels adjust their prices in proportion to the net flow through them. They rebalance themselves whenever necessary and execute any transaction request that has been made of them. We discuss both these aspects below.

\paragraph{On Prices}
In this work, the exclusive role of channel prices is to ensure that the flows through each channel remains balanced. In practice, it would be important to include other components in a channel's price/fee as well: a congestion price  and an incentive price. The congestion price, as suggested by \cite{varma2021throughput}, would depend on the total flow of transactions through the channel, and would incentivize nodes to balance the load over different paths. The incentive price, which is commonly used in practice \cite{river2023lightning}, is necessary to provide channels with an incentive to serve as an intermediary for different channels. In practice, we expect both these components to be smaller than the imbalance price. Consequently, we expect the behavior of our protocol to be similar to our theoretical results even with these additional prices.

A key aspect of our protocol is that channel fees are allowed to be negative. Although the original Lightning network whitepaper \cite{poon2016bitcoin} suggests that negative channel prices may be a good solution to promote rebalancing, the idea of negative prices in not very popular in the literature. To our knowledge, the only prior work with this feature is \cite{varma2021throughput}. Indeed, in papers such as \cite{van2021merchant} and \cite{wang2024fence}, the price function is explicitly modified such that the channel price is never negative. The results of our paper show the benefits of negative prices. For one, in steady state, equal flows in both directions ensure that a channel doesn't loose any money (the other price components mentioned above ensure that the channel will only gain money). More importantly, negative prices are important to ensure that the protocol selectively stifles acyclic flows while allowing circulations to flow. Indeed, in the example of Section \ref{sec:flow_control_example}, the flows between nodes $A$ and $C$ are left on only because the large positive price over one channel is canceled by the corresponding negative price over the other channel, leading to a net zero price.

Lastly, observe that in the DEBT control protocol, the price charged by a channel does not depend on its capacity. This is a natural consequence of the price being the Lagrange multiplier for the net-zero flow constraint, which also does not depend on the channel capacity. In contrast, in many other works, the imbalance price is normalized by the channel capacity \cite{ren2018optimal, lin2020funds, wang2024fence}; this is shown to work well in practice. The rationale for such a price structure is explained well in \cite{wang2024fence}, where this fee is derived with the aim of always maintaining some balance (liquidity) at each end of every channel. This is a reasonable aim if a channel is to never rebalance itself; the experiments of the aforementioned papers are conducted in such a regime. In this work, however, we allow the channels to rebalance themselves a few times in order to settle on a detailed balance flow. This is because our focus is on the long-term steady state performance of the protocol. This difference in perspective also shows up in how the price depends on the channel imbalance. \cite{lin2020funds} and \cite{wang2024fence} advocate for strictly convex prices whereas this work and \cite{varma2021throughput} propose linear prices.

\paragraph{On Rebalancing} 
Recall that the DEBT control protocol ensures that the flows in the network converge to a detailed balance flow, which can be sustained perpetually without any rebalancing. However, during the transient phase (before convergence), channels may have to perform on-chain rebalancing a few times. Since rebalancing is an expensive operation, it is worthwhile discussing methods by which channels can reduce the extent of rebalancing. One option for the channels to reduce the extent of rebalancing is to increase their capacity; however, this comes at the cost of locking in more capital. Each channel can decide for itself the optimum amount of capital to lock in. Another option, which we discuss in Section \ref{sec:five_node}, is for channels to increase the rate $\gamma$ at which they adjust prices. 

Ultimately, whether or not it is beneficial for a channel to rebalance depends on the time-horizon under consideration. Our protocol is based on the assumption that the demand remains steady for a long period of time. If this is indeed the case, it would be worthwhile for a channel to rebalance itself as it can make up this cost through the incentive fees gained from the flow of transactions through it in steady state. If a channel chooses not to rebalance itself, however, there is a risk of being trapped in a deadlock, which is suboptimal for not only the nodes but also the channel.

\section{Conclusion}
This work presents DEBT control: a protocol for payment channel networks that uses source routing and flow control based on channel prices. The protocol is derived by posing a network utility maximization problem and analyzing its dual minimization. It is shown that under steady demands, the protocol guides the network to an optimal, sustainable point. Simulations show its robustness to demand variations. The work demonstrates that simple protocols with strong theoretical guarantees are possible for PCNs and we hope it inspires further theoretical research in this direction.
\section{Acknowledgements}
This work was supported by Booking.com.

 % \bibliographystyle{alpha}
 \bibliographystyle{plainnat}
  \bibliography{biblio.bib}
  \appendix
\appendix
\section*{Appendix}
\section{Discussion: Scope and Ethics}
\label{appendix:scope}
In this work, we evaluate our method on six core scene-aware tasks: existence, count, position, color, scene, and HOI reasoning. We select these tasks as they represent core aspects of multimodal understanding which are essential for many applications. Meanwhile, we do not extend our evaluation to more complex reasoning tasks, such as numerical calculations or code generation, because SOTA diffusion models like SDXL are not yet capable of handling these tasks effectively. Fine-tuning alone cannot overcome the fundamental limitations of these models in generating images that require symbolic logic or complex reasoning. Additionally, we avoid tasks with ethical concerns, such as generating images of specific individuals (e.g., for celebrity recognition task), to mitigate risks related to privacy and misuse. Our goal was to ensure that our approach focuses on technically feasible and responsible AI applications. Expanding to other tasks will require significant advancements in diffusion model capabilities and careful consideration of ethical implications.

\section{Limitations and Future Work}
While our Multimodal Context Evaluator proves effective in enhancing the fidelity of generated images and maintaining diversity, \method is built using pre-trained diffusion models such as SDXL and MLLMs like LLaVA, it inherently shares the limitations of these foundation models. \method still faces challenges with complex reasoning tasks such as numerical calculations or code generation due to the symbolic logic limitations inherent to SDXL. Additionally, during inference, the MLLM context descriptor occasionally generates incorrect information or ambiguous descriptions initially, which can lead to lower fidelity in the generated images. Figure~\ref{fig:failure} further illustrates these observations.

\method currently focuses on single attributes like count, position, and color as part of the multimodal context. This is because this task alone poses significant challenges to existing methods, which \method effectively addresses. A potential direction for future work is to broaden the applicability of \method to synthesize images with multiple scene attributes in the multimodal context as part of compositional reasoning tasks.


\begin{figure}[!h]
    \centering
    \includegraphics[width=\linewidth]{figures/failures.pdf}
    \caption{Failure cases of \method. (a) Our method fails due to the symbolic logic limitation of existing pre-trained SDXL. (b) Initially incorrect descriptions generated by MLLMs lead to low fidelity of generated images. (c) Context description generated by MLLMs is ambiguous and does not directly relate to the text guidance, the spoon can be both inside or outside the bowl.}
    \label{fig:failure}
\end{figure}

\section{Prompt Templates}
\label{appendix:prompts}
Figure~\ref{fig:prompt_templates}~(a-c) showcases the prompt templates used by \method to fine-tune diffusion models specifically on each task including VQA, HOI Reasoning, and Object-Centric benchmarks. It's worth noting that we designed the prompt such that it provides detailed instruction to MLLMs on which scene attributes to focus. We also evaluate the effectiveness of our designed prompt templates by fine-tuning \method with a generic prompt as illustrated in Figure~\ref{fig:prompt_templates}~(d). Table~\ref{table:prommpt} indicates that without using our designed prompt template, the MLLM is not properly instructed to generate specific context description thus leading to reduced performance after fine-tuning on MME tasks. We believe that when using a generic prompt, MLLM is not able to receive sufficient grounding about the multimodal context leading to information loss on key scene attributes.


\begin{table}[!h]
\centering
\footnotesize
\caption{Effectiveness of the prompt template on fine-tuning \method on MME Perception.}
\resizebox{1\linewidth}{!}{
\begin{tabular}{clcccccccccc}
\toprule
 \textbf{MLLM} & \multirow{2}{*}{\textbf{\method}} & \multicolumn{2}{c}{\textbf{Existence}} & \multicolumn{2}{c}{\textbf{Count}} & \multicolumn{2}{c}{\textbf{Position}} & \multicolumn{2}{c}{\textbf{Color}} & \multicolumn{2}{c}{\textbf{Scene}} \\
 \textbf{Name} & & ACC & ACC+ & ACC & ACC+ & ACC & ACC+ & ACC & ACC+ & ACC & ACC+ \\
 \midrule
 \multirow{3}{*}{\makecell{\textbf{LLaVA }  \\ \textbf{v1.6 7B} \\ \citep{liu2024improved}}}
 &w/ prompt template & \textbf{96.67}  & \textbf{93.33}  & \textbf{83.33}  & \textbf{70.00}  & \textbf{81.67}  & \textbf{66.67} & \textbf{95.00}  & \textbf{93.33}  & \textbf{87.75} & \textbf{74.00} \\
 \cmidrule{2-12}
 & \multirow{2}{*}{w/ generic prompt} & 91.67 & 83.33 & 75.00 & 56.67 & \textbf{81.67} & 63.33 & 91.67 & 83.33 & 87.25 & 73.00 \\
 & & {\scriptsize \color{red}\textbf{$\downarrow$ 5.00}} & {\scriptsize \color{red}\textbf{$\downarrow$ 10.00}} & {\scriptsize \color{red}\textbf{$\downarrow$ 8.33}} & {\scriptsize \color{red}\textbf{$\downarrow$ 13.33}} & - &  {\scriptsize \color{red}\textbf{$\downarrow$ 3.34}} & {\scriptsize \color{red}\textbf{$\downarrow$ 3.33}} & {\scriptsize \color{red}\textbf{$\downarrow$ 10.00}} & {\scriptsize \color{red}\textbf{$\downarrow$ 0.50}} & {\scriptsize \color{red}\textbf{$\downarrow$ 1.00}}\\
 \midrule
 \multirow{3}{*}{\makecell{\textbf{InternVL }  \\ \textbf{2.0 8B}\\ \citep{chen2024internvl}}} 
 &w/ prompt template & \textbf{98.33}  & \textbf{96.67} & \textbf{86.67} & \textbf{73.33}  & \textbf{78.33}  & \textbf{63.33}  & \textbf{98.33}  & \textbf{96.67}  & \textbf{86.25} & \textbf{71.00} \\
 \cmidrule{2-12}
 & \multirow{2}{*}{w/ generic prompt} & 91.67 & 83.33 & 80.00 & 60.00 & 71.67 & 50.00 & 91.67 & 83.33 & 84.50 & 69.00 \\
 & & {\scriptsize \color{red}\textbf{$\downarrow$ 6.66}} &  {\scriptsize \color{red}\textbf{$\downarrow$ 13.34}} & {\scriptsize \color{red}\textbf{$\downarrow$ 6.67}} & {\scriptsize \color{red}\textbf{$\downarrow$ 13.33}} & {\scriptsize \color{red}\textbf{$\downarrow$ 6.66}} & {\scriptsize \color{red}\textbf{$\downarrow$ 13.33}} & {\scriptsize \color{red}\textbf{$\downarrow$ 6.66}} & {\scriptsize \color{red}\textbf{$\downarrow$ 13.34}} & {\scriptsize \color{red}\textbf{$\downarrow$ 1.75}} & {\scriptsize \color{red}\textbf{$\downarrow$ 2.00}}\\
\bottomrule
\end{tabular}
}
\label{table:prommpt}
\end{table}

\begin{figure}[!h]
    \centering
    \includegraphics[width=\linewidth]{figures/prompt_template.pdf}
    \caption{Prompt templates (a-c) used by \method to fine-tune the diffusion model on each task including VQA, HOI Reasoning, and Object Centric benchmarks. The generic prompt (d) is also included to evaluate the effectiveness of prompt template.}
    \label{fig:prompt_templates}
\end{figure}
\section{Inference Pipeline}
\label{appendix:inference}
In the inference pipeline of \method (Figure~\ref{fig:inference}), the text guidance $\mathbf{g}$ includes only the question corresponding to the reference image $\mathbf{x}$. The answer is excluded for fair evaluation. Moreover, we remove Multimodal Context Evaluator, and the generated image $\hat{\mathbf{x}}$ is the final output.
\begin{figure}[!h]
    \centering
    \includegraphics[width=\linewidth]{figures/inference.pdf}
    \caption{Inference pipeline of \method}
    \label{fig:inference}
\end{figure}

\begin{figure}[!h]
    \centering
    \includegraphics[width=\linewidth]{figures/diversity_compact_caption.pdf}
    \vspace{-5mm}
    \caption{Examples of context description from MLLM in the inference pipeline where answers are not included in text guidance.}
    \label{fig:diversity_compact_caption}
\end{figure}



\section{Ablation Study on BLIP-2 QFormer}
Our design choice to leverage BLIP-2 QFormer in \method as the multimodal context evaluator facilitates the formulation of our novel Global Semantic and Fine-grained Consistency Rewards. These rewards enable \method to be effective across all tasks as seen in Table~\ref{table:clip}. On replace with a less powerful multimodal context encoder such as CLIP ViT-G/14, we can only implement the global semantic reward as the cosine similarity between the text features and generated image features. As a result, while the setting can maintain performance on coarse-level tasks such as Scene and Existence, there is a noticeable decline on fine-grained tasks like Count and Position. This demonstrates the effectiveness of our design choices in \method and shows that using less powerful alternatives, without the ability to provide both global and fine-grained alignment, affects the fidelity of generated images.

\begin{figure*}[t]
\centering
\includegraphics[width=15.5cm]{figures/clip_zeroshot.png}\\
\caption{CLIP a) training and b) zero-shot inference framework}
\label{fig:clip} 
\end{figure*}


\section{Additional Evaluation on MME Artwork}

To explore the method's ability to work on tasks involving more nuanced or abstract text guidance beyond factual scene attributes, we evaluate \method on an additional task of MME Artwork. This task focuses on image style attributes that are more nuanced/abstract such as the following question-answer pair -- Question: ``Does this artwork exist in the form of mosaic?'', Answer: ``No''.

Table~\ref{table:artwork_reasoning} summarizes the evaluation. We can observe that \method outperforms all existing methods on both ACC and ACC+, implying its higher effectiveness in generating images with high fidelity (in this case, image style preservation) compared to existing methods. This provides evidence that \method can generalize to tasks involving abstract/nuanced attributes such as image style. Figure~\ref{fig:artwork} further shows qualitative comparison between image generation methods on the MME Artwork task.

\begin{table}[h]
\centering
\caption{Comparison on Artwork benchmark and Visual Reasoning task. \method outperforms SOTA image generation and augmentation techniques.}
\resizebox{\linewidth}{!}{
\begin{tabular}{@{}l@{ }ccccccc@{}}
\toprule
\textbf{Method} & \textbf{Real only} & \textbf{RandAugment} &  \textbf{Image Variation} & \textbf{Image Translation} & \textbf{Textual Inversion} & \textbf{I2T2I SDXL} & \textbf{\method} \\
\midrule
\textbf{Artwork ACC} & 69.50 & 69.25 & 69.00 & 67.00 & 66.75 & 68.00 & \textbf{70.25} \\
\textbf{Artwork ACC+} & 41.00 & 41.00 & 40.00 & 38.00 & 37.50 & 38.00 & \textbf{41.50} \\
\midrule
\textbf{Reasoning ACC} & 69.29 & 67.86 & 69.29 & 69.29 & 67.14 & 72.14 & \textbf{72.86} \\
\textbf{Reasoning ACC+} & 42.86 & 40.00 & 41.40 & 40.00 & 37.14 & 47.14 & \textbf{48.57} \\

\bottomrule
\end{tabular}
}
\label{table:artwork_reasoning}
\end{table}


\begin{figure}[!h]
    \centering
    \includegraphics[width=\linewidth]{figures/artwork.pdf}
    \caption{Qualitative comparison on the Artwork task between image generation method. \method can preserve both diversity and fidelity of the reference image in a more abstract domain.}
    \label{fig:artwork}
\end{figure}


\section{Additional Evaluation on MME Commonsense Reasoning}
We have additionally performed our evaluation to more complex tasks such as Visual Reasoning using the MME Commonsense Reasoning benchmark. Results in Table~\ref{table:artwork_reasoning} highlight \method's ability to generalize effectively across diverse domains and complex reasoning tasks, demonstrating its broader applicability. Figure~\ref{fig:reasoning} further shows qualitative comparison between image generation methods on the MME Commonsense Reasoning task.

\begin{figure}[!h]
    \centering
    \includegraphics[width=\linewidth]{figures/reasoning.pdf}
    \caption{Qualitative comparison on the Commonsense Reasoning task between image generation method. \method can preserve both diversity and fidelity of the reference image in a more abstract domain.}
    \label{fig:reasoning}
\end{figure}
\section{FID Scores}
% \textcolor{blue}{We compute FID scores of traditional augmentation and image generation methods. Table~\ref{table:fid} shows that the data distribution of generated images by RandAugment and Image Translation are closer to the real distribution as these methods only change images minimally. We also want to emphasize that even though the FID metric evaluates the quality of generated images, it can not measure the diversity of generated images. \method with rewards fine-tuning achieves a competitive score. As we showed in the diversity analysis in Table~\ref{table:diversity} in the main paper, \method performs significantly better than these ``minimal change" methods while still achieving a competitive FID score. We believe this is a worldwide trade-off.}

We compute FID scores for \method and the different baselines (traditional augmentation and image generation methods) and tabulate the numbers in Table~\ref{table:fid}. FID is a valuable metric for assessing the quality of generated images and how closely the distribution of generated images matches the real distribution. However, \textit{FID does not account for the diversity among the generated images}, which is a critical aspect of the task our work targets~(i.e., how can we generate high fidelity images, preserving certain scene attributes, while still maintaining high diversity?). We also illustrate the shortcomings of FID for the task in Figure~\ref{fig:fid_diversity} where we compare generated images across methods. We observe that RandAugment and Image Translation achieve lower FID scores than \method~(w/ finetuning) because they compromise on diversity by only minimally changing the input image, allowing their generated image distribution to be much closer to the real distribution. While \method has a higher FID score than RandAugment and Image Translation, Figure~\ref{fig:fid_diversity} shows that it is able to preserve the scene attribute w.r.t.~multimodal context while generating an image that is significantly different from than original input image. Therefore, it accomplishes the targeted task more effectively, with both high fidelity and high diversity.

\begin{table}[h]
\centering
\caption{FID scores of traditional augmentation and image generation methods. Lower is better.}
\resizebox{\linewidth}{!}{
\begin{tabular}{@{}l@{ }ccccccc@{}}
\toprule
\multirow{2}{*}{\textbf{Method}} & \multirow{2}{*}{\textbf{RandAugment}} & \multirow{2}{*}{\textbf{I2T2I SDXL}} & \multirow{2}{*}{\textbf{Image Variation}} & \multirow{2}{*}{\textbf{Image Translation}} & \multirow{2}{*}{\textbf{Textual Inversion}} & \multicolumn{2}{c}{\textbf{\method}} \\
& & & & & & \ding{55} fine-tuning & \ding{51} fine-tuning\\
\midrule
\textbf{FID score $\downarrow$} & \textbf{15.93} & 18.35 & 17.66 & 16.29 & 20.84 & 17.78 & 16.55 \\
\bottomrule
\end{tabular}
}
\label{table:fid}
\end{table}

\begin{figure}[!h]
    \centering
    \includegraphics[width=\linewidth]{figures/fid_diversity.pdf}
    \caption{While RandAugment and Image Translation achieve lower FID scores, \method balances fidelity and diversity effectively.}
    \label{fig:fid_diversity}
\end{figure}

\section{User Study}
% \textcolor{blue}{We created a survey form with 50 questions (10 questions per MME task). In each survey question, users were shown: a reference image, a related question, and two generated images from different methods (I2T2I SDXL vs. \method). Users are asked to select the generated image(s) that preserve the attribute referred to by the question in relation to reference image. We collected form responses from 70 people. Table~\ref{table:user_study} shows that \method significantly outperforms I2T2I SDXL in terms of fidelity across all tasks on MME benchmark. We have some examples of survey questions in Figure~\ref{fig:user_study_examples}.}

We conduct a user study where we create a survey form with 50 questions (10 questions per MME Perception task). In each survey question, we show users a reference image, a related question, and a generated image each from two different methods (baseline I2T2I SDXL vs \method). We ask users to select the generated images(s) (either one or both or neither of them) that preserve the attribute referred to by the question in relation to the reference image. If an image is selected, it denotes high fidelity in generation. We collect form responses from 70 people for this study. We compute the percentage of total generated images for each method that were selected by the users as a measure of fidelity. Table~\ref{table:user_study} summarizes the results and shows that \method significantly outperforms I2T2I SDXL in terms of fidelity across all tasks on the MME Perception benchmark. We have some examples of survey questions in Figure~\ref{fig:user_study_examples}.

\begin{figure}[htp]
  \centering
   \includegraphics[width=\columnwidth]{Assets/userstudy_grid.pdf}
   
   \caption{\textbf{User study results.} Users preference percentage of our method compared to other methods in terms of text alignment, visual quality, and overall preference.
   }
   \label{fig:user_study}
\end{figure}
\begin{figure}[!h]
    \centering
    \includegraphics[width=\linewidth]{figures/user_study_examples.pdf}
    \caption{Some examples of our survey questions to evaluate the fidelity of generated images from I2T2I SDXL and \method.}
    \label{fig:user_study_examples}
\end{figure}
\section{Training Performance on Bongard HOI Dataset}
% \textcolor{blue}{We conducted an additional experiment by training a CNN baseline ResNet50 \citep{he2016deep} model on the Bongard-HOI training set with traditional augmentation and other image generation methods, using the same number of training iterations. As shown in Table~\ref{table:hoi_training}, \method consistently outperforms other methods across all test splits. However, as discussed in Subsection~\ref{sec:benchmark_formulation}, our primary focus on test-time evaluation ensures fair comparisons by avoiding variability in training behavior caused by differences in model architectures, data distributions, and training configurations.}

Following the existing method \citep{shu2022testtime}, we conduct an additional experiment by training a ResNet50 \citep{he2016deep} model on the Bongard-HOI \citep{jiang2022bongard} training set with traditional augmentation and Hummingbird generated images. We compare the performance with other image generation methods, using the same
number of training iterations. As shown in Table~\ref{table:hoi_training}, \method consistently outperforms all the baselines across all test splits. In the paper, as discussed in Section~\ref{sec:benchmark_formulation}, we focus primarily on test-time evaluation because it eliminates the variability introduced by model training due to multiple external variables such as model architecture, data distribution, and training configurations, and allows for a fairer comparison where the evaluation setup remains fixed.

\begin{table}[!h]
\centering
\footnotesize
\caption{Comparison on Human-Object Interaction~(HOI) Reasoning by training a CNN-baseline ResNet50 with image augmentation and generation methods. \method outperforms SOTA methods on all $4$ test splits of Bongard-HOI dataset.}
\resizebox{0.8\linewidth}{!}{
\begin{tabular}{lccccc}
\toprule
\multirow{3}{*}{Method} & \multicolumn{4}{c}{Test Splits} & \multirow{3}{*}{Average} \\
\cmidrule{2-5}
 & seen act., & unseen act., & seen act., & unseen act., &  \\
 & seen obj. & seen obj. & unseen obj. & unseen obj. & \\
  % & seen act., seen obj. & unseen act., seen obj. & seen act., unseen obj. & unseen act., unseen obj. &  \\
 % &  &  &  & & \\
\midrule
CNN-baseline (ResNet50) & 50.03\xspace\xspace\xspace\xspace\xspace\xspace\xspace\xspace\xspace\xspace & 49.89\xspace\xspace\xspace\xspace\xspace\xspace\xspace\xspace\xspace\xspace & 49.77\xspace\xspace\xspace\xspace\xspace\xspace\xspace\xspace\xspace\xspace & 50.01\xspace\xspace\xspace\xspace\xspace\xspace\xspace\xspace\xspace\xspace & 49.92\xspace\xspace\xspace\xspace\xspace\xspace\xspace\xspace\xspace\xspace \\
RandAugment \citep{cubuk2020randaugment} & 51.07 {\scriptsize \color{ForestGreen}$\uparrow$ 1.04} & 51.14 {\scriptsize \color{ForestGreen}$\uparrow$ 1.25} & 51.79 {\scriptsize \color{ForestGreen}$\uparrow$ 2.02} & 51.73 {\scriptsize \color{ForestGreen}$\uparrow$ 1.72} & 51.43 {\scriptsize \color{ForestGreen}$\uparrow$ 1.51} \\
Image Variation \citep{xu2023versatile} & 41.78 {\scriptsize \color{red}$\downarrow$ 8.25} & 41.29 {\scriptsize \color{red}$\downarrow$ 8.60} & 41.15 {\scriptsize \color{red}$\downarrow$ 8.62} & 41.25 {\scriptsize \color{red}$\downarrow$ 8.76} & 41.37 {\scriptsize \color{red}$\downarrow$ 8.55} \\
Image Translation \citep{pan2023boomerang} & 46.60 {\scriptsize \color{red}$\downarrow$ 3.43} & 46.94 {\scriptsize \color{red}$\downarrow$ 2.95} & 46.38 {\scriptsize \color{red}$\downarrow$ 3.39} & 46.50 {\scriptsize \color{red}$\downarrow$ 3.51} & 46.61 {\scriptsize \color{red}$\downarrow$ 3.31} \\
Textual Inversion \citep{gal2022image} & \xspace37.67 {\scriptsize \color{red}$\downarrow$ 12.36} & \xspace37.52 {\scriptsize \color{red}$\downarrow$ 12.37} & \xspace38.12 {\scriptsize \color{red}$\downarrow$ 11.65} & \xspace38.06 {\scriptsize \color{red}$\downarrow$ 11.95} & \xspace37.84 {\scriptsize \color{red}$\downarrow$ 12.08} \\
I2T2I SDXL \citep{podell2023sdxl} & 51.92 {\scriptsize \color{ForestGreen}$\uparrow$ 1.89} & 52.18 {\scriptsize \color{ForestGreen}$\uparrow$ 2.29} & 52.25 {\scriptsize \color{ForestGreen}$\uparrow$ 2.48} & 52.15 {\scriptsize \color{ForestGreen}$\uparrow$ 2.14} & 52.13 {\scriptsize \color{ForestGreen}$\uparrow$ 2.21}\\
\textbf{\method} & \textbf{53.71 {\scriptsize \color{ForestGreen}$\uparrow$ 3.68}} & \textbf{53.55 {\scriptsize \color{ForestGreen}$\uparrow$ 3.66}} & \textbf{53.69 {\scriptsize \color{ForestGreen}$\uparrow$ 3.92}} & \textbf{53.41 {\scriptsize \color{ForestGreen}$\uparrow$ 3.40}} & \textbf{53.59 {\scriptsize \color{ForestGreen}$\uparrow$ 3.67}} \\
\bottomrule
\end{tabular}
}
\label{table:hoi_training}
\end{table}



\section{Random Seeds Selection Analysis}
We conduct an additional experiment, varying the number of random seeds from $10$ to $100$. The results are presented in the boxplot in Figure~\ref{fig:boxplot}, which shows the distribution of the mean L2 distances of generated image features from Hummingbird across different numbers of seeds.


The figure demonstrates that the difference in the distribution of the diversity (L2) scores across the different numbers of random seeds is statistically insignificant. So while it is helpful to increase the number of seeds for improved confidence, we observe that it stabilizes at 20 random seeds. This analysis suggests that using $20$ random seeds also suffices to capture the diversity of generated images without significantly affecting the robustness of the analysis.

% We conduct an additional experiment where we vary the number of seeds from 10 to 100. We present the results as a boxplot in Appendix K, Figure 15 which shows the distribution of the mean L2 distances of generated image features from Hummingbird across different numbers of seeds.

% The figure demonstrates that the difference in the distribution of the diversity (L2) scores across the different numbers of random seeds is statistically insignificant. So while it is helpful to increase the number of seeds for improved confidence, we observe that it stabilizes at 20 random seeds. This analysis suggests that using 20 random seeds also suffices to capture the diversity of generated images without significantly affecting the robustness of the analysis.

\begin{figure}[!h]
    \centering
    \includegraphics[width=0.8\linewidth]{figures/diversity_boxplot_rectangular.pdf}
    \caption{Diversity analysis across varying numbers of random seeds (10 to 100) using mean L2 distances of generated image features from \method. The box plot demonstrates consistent diversity scores as the number of seeds increases, indicating that performance stabilizes around 20 random seeds.}
    \label{fig:boxplot}
\end{figure}

\section{Further Explanation of Multimodal Context Evaluator}
The Global Semantic Reward, \(\mathcal{R}_\textrm{global}\), ensures alignment between the global semantic features of the generated image \(\mathbf{\hat{x}}\) and the textual context description \(\mathcal{C}\). This reward leverages cosine similarity to measure the directional alignment between two feature vectors, which can be interpreted as maximizing the mutual information \(I(\mathbf{\hat{x}}, \mathcal{C})\) between the generated image \(\mathbf{\hat{x}}\) and the context description \(\mathcal{C}\). Mutual information quantifies the dependency between the joint distribution \(p_{\theta}(\mathbf{\hat{x}}, \mathcal{C})\) and the marginal distributions. In conditional diffusion models, the likelihood \(p_{\theta}(\mathbf{\hat{x}} \vert \mathcal{C})\) of generating \(\mathbf{\hat{x}}\) given \(\mathcal{C}\) is proportional to the joint distribution:
\[
p_{\theta}(\mathbf{\hat{x}} \vert \mathcal{C}) = \frac{p_{\theta}(\mathbf{\hat{x}}, \mathcal{C})}{p(\mathcal{C})} \propto p_{\theta}(\mathbf{\hat{x}}, \mathcal{C}),
\]
where \(p(\mathcal{C})\) is the marginal probability of the context description, treated as a constant during optimization. By maximizing \(\mathcal{R}_\textrm{global}\), which aligns global semantic features, the model indirectly maximizes the mutual information \(I(\mathbf{\hat{x}}, \mathcal{C})\), thereby enhancing the likelihood \(p_{\theta}(\mathbf{\hat{x}} \vert \mathcal{C})\) in the conditional diffusion model.


The Fine-Grained Consistency Reward, $\mathcal{R}_{\textrm{fine-grained}}$, captures detailed multimodal alignment between the generated image $\mathbf{\hat{x}}$ and the textual context description $\mathcal{C}$. It operates at a token level, leveraging bidirectional self-attention and cross-attention mechanisms in the BLIP-2 QFormer, followed by the Image-Text Matching (ITM) classifier to maximize the alignment score.

\textbf{Self-Attention on Text Tokens:}
    Text tokens $\mathcal{T}_{\mathrm{tokens}}$ undergo self-attention, allowing interactions among words to capture intra-text dependencies:
    \begin{equation}
        \mathcal{T}_{\mathrm{attn}} = \tt{SelfAttention}(\mathcal{T}_{\mathrm{tokens}})
    \end{equation}

\textbf{Self-Attention on Image Tokens:}
    Image tokens $\mathcal{Z}$ are derived from visual features of the generated image $\mathbf{\hat{x}}$ using a cross-attention mechanism:
    \begin{equation}
        \mathcal{Z} = \tt{CrossAttention}(\mathcal{Q}_{\mathrm{learned}}, \mathcal{I}_{\mathrm{tokens}}(\mathbf{\hat{x}}))
    \end{equation}
    These tokens then pass through self-attention to extract intra-image relationships:
    \begin{equation}
        \mathcal{Z}_{\mathrm{attn}} = \tt{SelfAttention}(\mathcal{Z})
    \end{equation}

\textbf{Cross-Attention between Text and Image Tokens:}
    The text tokens $\mathcal{T}_{\mathrm{attn}}$ and image tokens $\mathcal{Z}_{\mathrm{attn}}$ interact through cross-attention to focus on multimodal correlations:
    \begin{equation}
        \mathcal{F} = \tt{CrossAttention}(\mathcal{T}_{\mathrm{attn}}, \mathcal{Z}_{\mathrm{attn}})
    \end{equation}

\textbf{ITM Classifier for Alignment:}
    The resulting multimodal features $\mathcal{F}$ are fed into the ITM classifier, which outputs two logits: one for positive match ($j=1$) and one for negative match ($j=0$). The positive class ($j=1$) indicates strong alignment between the image-text pair, while the negative class ($j=0$) indicates misalignment:
    \begin{equation}
        \mathcal{R}_{\textrm{fine-grained}} = \tt{ITM\_Classifier}(\mathcal{F})_{j=1}
    \end{equation}

The ITM classifier predicts whether the generated image and the textual context description match. Maximizing the logit for the positive class $j=1$ corresponds to maximizing the log probability $\log p(\mathbf{\hat{x}}, \mathcal{C})$ of the joint distribution of image and text. This process aligns the fine-grained details in $\mathbf{\hat{x}}$ with $\mathcal{C}$, increasing the mutual information between the generated image and the text features.

\textbf{Improving fine-grained relationships of CLIP.} While the CLIP Text Encoder, at times, struggles to accurately capture spatial features when processing longer sentences in the Multimodal Context Description, \method addresses this limitation by distilling the global semantic and fine-grained semantic rewards from BLIP-2 QFormer into a specific set of UNet denoiser layers, as mentioned in the implementation details under Appendix~\ref{appendix:impl}~(i.e., Q, V transformation layers including $\tt{to\_q, to\_v, query, value}$). This strengthens the alignment between the generated image tokens~(Q) and input text tokens from the Multimodal Context Description~(K, V) in the cross-attention mechanism of the UNet denoiser. As a result, we obtain generated images with improved fidelity, particularly w.r.t.~spatial relationships, thereby helping to mitigate the shortcomings of vanilla CLIP Text Encoder in processing the long sentences of the Multimodal Context Description.

To illustrate further, a Context Description like “the dog under the pool” is processed in three steps: (1) self-attention is applied to the text tokens (K, V), enabling spatial terms like “dog,” “under,” and “pool” to interact; (2) self-attention is applied to visual features represented by the generated image tokens (Q) to extract intra-image relationships (3) cross-attention aligns this text features with visual features. The resulting alignment scores are used to compute the mean and select the positive class for the reward. Our approach to distill this reward into the cross-attention layers therefore ensures that spatial relationships and other fine-grained attributes are effectively captured, improving the fidelity of generated images.


\section{The Choice of Text Encoder in SDXL and BLIP-2 QFormer}

The choice of text encoder in our pipeline is to leverage pre-trained models for their respective strengths. SDXL inherently uses the CLIP Text Encoder for its generative pipeline, as it is designed to process text embeddings aligned with the CLIP Image Encoder. In the Multimodal Context Evaluator, we use the BLIP-2 QFormer, which is pre-trained with a BERT-based text encoder.

\section{Textual Inversion for Data Augmentation}
In our experiments, we applied Textual Inversion for data augmentation as follows: given a reference image, Textual Inversion learns a new text embedding that captures the context of the reference image (denoted as $<$context$>$). This embedding is then used to generate multiple augmented images by employing the prompt: ``a photo of $<$context$>$". This approach allows Textual Inversion to create context-relevant augmentations for comparison in our experiments.

\section{Convergence Curve}
To evaluate convergence, we monitor the training process using the Global Semantic Reward and Fine-Grained Consistency Reward as criteria. Specifically, we observe the stabilization of these rewards over training iterations. Figure~\ref{fig:convergence} presents the convergence curves for both rewards, illustrating their gradual increase followed by stabilization around 50k iterations. This steady state indicates that the model has learned to effectively align the generated images with the multimodal context.

\begin{figure}[!h]
    \centering
    \includegraphics[width=\linewidth]{figures/convergence.pdf}
    \caption{Convergence curves of Global Semantic and Fine-Grained Consistency Rewards}
    \label{fig:convergence}
\end{figure}


\section{Fidelity Evaluation using GPT-4o}
In addition to the results above, we compute additional metrics for fidelity, which measure how well the model preserves scene attributes when generating new images from a reference image. For this, we use GPT-4o (model version: 2024-05-13) as the MLLM oracle for a VQA task on the MME Perception benchmark \citep{fu2024mme}. 
% We use a MLLM as an oracle for a visual question-answering (VQA) task on the MME Perception benchmark \citep{fu2024mme}. In this experiment, we use GPT-4o (model version: 2024-05-13) as the oracle. 
We evaluate \method with and without fine-tuning process.

The MME dataset consists of Yes/No questions, with a positive and a negative question for every reference image. To measure fidelity, we measure the rate at which the oracle's answer remains consistent across the reference and the generated image for every image in the dataset. We run the experiment multiple times and report the average numbers in Table~\ref{table:fidelity_comparison}. We see that fine-tuning the base SDXL with our novel rewards results in an average increase of $2.99\%$ in fidelity.

\begin{table}[!h]
\centering
\footnotesize
\caption{Fidelity between reference and generated images from \method with and without fine-tuning.}
\resizebox{0.9\linewidth}{!}{
\begin{tabular}{clccc}
\toprule
 \textbf{MLLM Oracle} & \textbf{\method} & \textbf{Fidelity on ``Yes"} & \textbf{Fidelity on ``No"} & \textbf{Overall Fidelity} \\
 \midrule
 \multirow{2}{*}{\makecell{\textbf{GPT-4o}\\\textbf{Ver: 2024-05-13}}}
 & w/o fine-tuning & 68.33\xspace\xspace\xspace\xspace\xspace\xspace\xspace\xspace\xspace\xspace & 70.55\xspace\xspace\xspace\xspace\xspace\xspace\xspace\xspace\xspace\xspace & 71.18\xspace\xspace\xspace\xspace\xspace\xspace\xspace\xspace\xspace\xspace \\
 \cmidrule{2-5}
 % \cmidrule{2-12}
 & w/ fine-tuning & \textbf{69.72} {\scriptsize \color{ForestGreen}\textbf{$\uparrow$ 1.39}}  & \textbf{73.61} {\scriptsize \color{ForestGreen}\textbf{$\uparrow$ 3.06}}  & \textbf{74.17} {\scriptsize \color{ForestGreen}\textbf{$\uparrow$ 2.99}} \\
\bottomrule
\end{tabular}
}
\label{table:fidelity_comparison}
\end{table}


\section{Implementation Details}
\label{appendix:impl}
We implement \method using PyTorch \citep{paszke2019pytorch} and HuggingFace diffusers \citep{huggingface2023diffusers} libraries. For the generative model, we utilize the SDXL Base $1.0$ which is a standard and commonly used pre-trained diffusion model in natural images domain. In the pipeline, we employ CLIP ViT-G/14 as image encoder and both CLIP-L/14 \& CLIP-G/14 as text encoders \citep{radford2021learning}. We perform LoRA fine-tuning on the following modules of SDXL UNet denoiser including $Q$, $V$ transformation layers, fully-connected layers ($\tt{to\_q, to\_v, query, value, ff.net.0.proj}$) with rank parameter $r = 8$, which results in $11$M trainable parameters $\approx 0.46\%$ of total $2.6$B parameters. The fine-tuning is done on $8$ NVIDIA A100 80GB GPUs using AdamW \citep{loshchilov2017decoupled} optimizer, a learning rate of \texttt{5e-6}, and gradient accumulation steps of $8$.

\section{Additional Qualitative Results}
\label{appendix:visuals}
Figure~\ref{fig:diversity_compact_caption} showcases two examples of context description from MLLM in the inference pipeline where answers are not included in text guidance. Figure~\ref{fig:diversity_full} illustrates additional qualitative results highlighting the diversity and multimodal context fidelity between reference and synthetic images, as well as across images generated by \method with different random seeds. Figure~\ref{fig:qualitative_full} shows additional qualitative comparisons between \method and SOTA image generation methods on VQA and HOI Reasoning tasks.
\begin{figure}[!h]
    \centering
    \includegraphics[width=\linewidth]{figures/diversity_full.pdf}
    \vspace{-5mm}
    \caption{Diversity and multimodal context fidelity between reference and synthetic image and across generated ones from \method with different random seeds.}
    \label{fig:diversity_full}
\end{figure}
\begin{figure}[!h]
    \centering
    \includegraphics[width=\linewidth]{figures/qualitative_full_v1.pdf}
    \vspace{-5mm}
    \caption{Qualitative comparison between \method and other image generation methods on MME Perception and HOI Reasoning benchmarks.}
    \label{fig:qualitative_full}
\end{figure}
% %\section{Proof of Equivalence}

\begin{proof}[Proof of Lemma~\ref{lem:mk_equivalence}]
% From the proof of Theorem~\ref{thm:iv_tight}, $\distiv = \distivpos \hspace{2pt} \dot{\cup} \hspace{2pt} \distivzero$ where $\distivpos$ and $\distivzero$ are as defined in \eqref{eq:distivpos} and \eqref{eq:distivzero}, respectively. Similarly, we can decompose $\distinter, \distctrf, \distgraph$ as $\distnotion = \distnotionpos \hspace{2pt} \dot{\cup} \hspace{2pt} \distnotionzero$ where 
% \begin{align}
% \distnotionpos &\triangleq \left \lbrace P_{\model}(D,A,S) : \model \in H^{0}_{\text{cf-notion}} \text{ and }\forall s, P_{\model}(S=s) > 0 \right \rbrace, \label{eq:distnotionpos} \\
% \distnotionzero &\triangleq  \left \lbrace P_{\model}(D,A,S) : \model \in H^{0}_{\text{cf-notion}} \text{ and } \exists s \text{ s.t. } P_{\model}(S=s) = 0 \right \rbrace,\label{eq:distnotionzero}
% \end{align}
% and we use ``notion'' as a placeholder for ``inter, ctrf, graph'' for clarity. For each of these notions, it is clear that if $P_{\model}(S=s) = 0$, for some $s$, then $\model \in H^{0}_{\text{cf-notion}}$ does not impose additional constraints on $P_{\model}\Paren{A,D \mid S=s'}$ where $s \neq s'$. Therefore, $\distivzero = \distnotionzero$ and it is sufficient to restrict attention to proving the equality of $\distnotioncnd$ and $\distivposcnd$ where the former is defined as 
% $$\distnotioncnd = \left \lbrace P_{\model}(D,A \mid S): \model \in H^{0}_{\text{cf-notion}} \right \rbrace.$$

%Again, like in the proof of Theorem~\ref{thm:iv_tight}, since $P(X,Y,Z) = P(Z) \otimes P(X,Y \mid Z)$ and $P(D,A,S) = P(S) \otimes P(D,A \mid S)$, 

For $\model \in \modelsedgerelax$, the response-function parameterization yields a counterfactually equivalent SCM, $\tilde{\model}$ represented by the tuple $(\enop,\tilde{\exrv},\tilde{\spc},\tilde{f},\tilde{P})$, where $\enop = \left \lbrace \sex, \dept, \outcome \right \rbrace, \tilde{\exrv} = \left \lbrace \response, U_{\sex} \right \rbrace, \tilde{\spc} =\spc_{\enop}\times\spc_{\tilde{\exrv}}, \tilde{f} = \Paren{\tilde{f}_{\sex}, \tilde{f}_{\dept}, \tilde{f}_{\outcome}}$ where we define $\spc_{\response}, \tilde{f},\tilde{P}$ through the function $\Phi: \spc_{\exrv} \mapsto \spc_{\tilde{\exrv}}$ where
\begin{align*}
    \spc_{\response} &\triangleq \spc_{\dept}^{\spc_{\sex}} \times \spc_{\outcome}^{\spc_{\sex}\times \spc_{\dept}},\\
    \forall u_S,u_D,u_A,u, \Phi\Paren{u_S,u_D,u_A,u} &\triangleq \Paren{\Paren{s \mapsto f_D(s,u,u_D),(s,d) \mapsto f_A(s,d,u,u_A)},u_S},\\
    \forall u_{\sex}, \tilde{f}_{\sex}\Paren{u_S}&\triangleq f_{\sex}(u_{\sex}),\\
\forall \lsex, \tilde{f}_{\dept}\Paren{\respfunc,\lsex} &\triangleq \respfunc_1\Paren{\lsex}, \\
\forall \lsex, \ldept, \tilde{f}_{\outcome}\Paren{\respfunc,\lsex,\ldept} &\triangleq \respfunc_2\Paren{\lsex,\ldept},
\end{align*}
where $\respfunc = \Paren{\respfunc_1,\respfunc_2}$ and $\tilde{P}$ is the push-forward distribution $\Phi_{*}(P)$.
Note that $\spc_{\response}$ is a discrete space, $\response$ a discrete random variable, and $\tilde{P}(\response)$  a discrete distribution over $\spc_{\response}$. Under the response-function parameterization, only $\tilde{P}(\response)$ is a parameter and we will abuse notation and denote it as $\tilde{P}$ henceforth. Therefore, we can represent $\nullgraphrelax$ in the parameter space as 

% \begin{equation}\label{eq:respfunc_graph_edge}
%     \nullgraphresp 
%     \triangleq \left \lbrace \tilde{P} \in \triangle\Paren{\cX_{\response}} : \tilde{P}\Paren{\respfunc_1,\respfunc_2} = 0 \text{ where } \respfunc_2\Paren{.,.} \text{ is such that } \exists \ldept 
%     \text{ such that }\respfunc_2(m,\ldept) \neq \respfunc_2\Paren{f,\ldept} \right \rbrace.
% \end{equation}

\begin{equation}\label{eq:respfunc_graph_edge}
    \nullgraphresp 
    \triangleq \left \lbrace \tilde{P} \in \triangle\Paren{\cX_{\response}} : \tilde{P}\Paren{\respfunc_1,\respfunc_2} \neq 0 \text{ implies } \forall \ldept, \respfunc_2(0,\ldept) = \respfunc_2\Paren{1,\ldept} \right \rbrace.
\end{equation}

To express $\nullinterrelax$, we express the interventional Markov kernels $P_{\tilde{\model}}\Paren{\outcome\mid \doop{\sex}, \doop{\dept}}$ in terms of $\tilde{P}$. Since counterfactual equivalence implies interventional equivalence, for all $\lsex, \ldept$, $P_{\model}\Paren{\outcome=1\mid \doop{\sex=\lsex}, \doop{\dept=\ldept}} = P_{\tilde{\model}}\Paren{\outcome=1\mid \doop{\sex=\lsex}, \doop{\dept=\ldept}}$, where 
\begin{align}
    P_{\tilde{\model}}\Paren{\outcome=1\mid \doop{\sex=\lsex}, \doop{\dept=\ldept}} &= \sum\limits_{\Paren{\respfunc_1,\respfunc_2}  \in \cX_{\response}}\bm{1}\Brack{\respfunc_2\Paren{\lsex,\ldept}=1}\tilde{P}\Paren{\respfunc_1,\respfunc_2}, \label{eq:inter_resp}\\
    P_{\tilde{\model}}\Paren{\outcome=1\mid \doop{\dept=\ldept}} &= \sum\limits_{\lsex^*}\sum\limits_{\Paren{\respfunc_1,\respfunc_2}  \in \cX_{\response}}\bm{1}\Brack{\respfunc_2\Paren{\lsex^*,\ldept}=1}\tilde{P}\Paren{\respfunc_1,\respfunc_2}P_{\tilde{\model}}\Paren{\lsex^*} \label{eq:inter_resp_doD},
\end{align}
Subtracting \eqref{eq:inter_resp} from  \eqref{eq:inter_resp_doD} we get 
\begin{align}
    &P_{\tilde{\model}}\Paren{\outcome=1\mid \doop{\sex=\lsex}, \doop{\dept=\ldept}} - P_{\tilde{\model}}\Paren{\outcome=1\mid \doop{\dept=\ldept}} \nonumber \\
    & =\Paren{\sum\limits_{\Paren{\respfunc_1,\respfunc_2}  \in \cX_{\response}} \Paren{\bm{1}\Brack{\respfunc_2\Paren{0,\ldept}=1} - \bm{1}\Brack{\respfunc_2\Paren{1,\ldept}=1}}\tilde{P}\Paren{\respfunc_1,\respfunc_2}}P_{\tilde{\model}}\Paren{s'} = 0 \label{eq:inter_resp_s}
\end{align}
for $\model \in \nullinterrelax$, where $s' \neq s$. Similarly, 
\begin{align}
    &P_{\tilde{\model}}\Paren{\outcome=1\mid \doop{\sex=\lsex'}, \doop{\dept=\ldept}} - P_{\tilde{\model}}\Paren{\outcome=1\mid \doop{\dept=\ldept}} \nonumber\\
    & =\Paren{\sum\limits_{\Paren{\respfunc_1,\respfunc_2}  \in \cX_{\response}} \Paren{\bm{1}\Brack{\respfunc_2\Paren{0,\ldept}=1} - \bm{1}\Brack{\respfunc_2\Paren{1,\ldept}=1}}\tilde{P}\Paren{\respfunc_1,\respfunc_2}}P_{\tilde{\model}}\Paren{s} = 0 \label{eq:inter_resp_sp}.
\end{align}
Since both \eqref{eq:inter_resp_s} and \eqref{eq:inter_resp_sp} hold, 
 the response-function parameterized analogue of $\nullinterrelax$ is 
\begin{equation}\label{eq:respfun_inter_edge}
    \nullinterresp \triangleq \left \lbrace \tilde{P} \in \triangle\Paren{\cX_{\response}} : \forall \ldept, \sum\limits_{\Paren{\respfunc_1,\respfunc_2}  \in \cX_{\response}} \Paren{\bm{1}\Brack{\respfunc_2\Paren{0,\ldept}=1} - \bm{1}\Brack{\respfunc_2\Paren{1,\ldept}=1}}\tilde{P}\Paren{\respfunc_1,\respfunc_2} = 0 \right \rbrace. 
\end{equation}

Note that both $\nullgraphresp$ and $\nullinterresp$ are polyhedra in $\triangle\Paren{\cX_{\response}}$. Further, $\nullgraphresp \subseteq \nullinterresp$. While, $\nullgraphresp, \nullinterresp$ are collections of distributions, we will also refer to them as collection of response-function-parameterized SCMs. 

%So far, we looked at the response-function parameterization for models in $\modelsedge$. However, the instrumental-variable inequalities arise from 
% While we have framed the hypotheses in terms of the exogenous distribution of the response-function parameterization, for a statistical test, we only have access to the observed Markov kernels $\Pr\Paren{\outcome,\dept,\formsex \mid \doop{\sex}}$. Therefore, we now characterize the sets of observed Markov kernels 
% It can be shown that the set of observed Markov kernels that are solutions of SCMs in $\nullgraph$ is the same as $\distiv$ where we define the former as 
From interventional equivalence (which follows as a result of counterfactual equivalence) of the response-function-parameterization, we have 
\begin{align*}
    \mkgraph &= \left \lbrace P_{\tilde{\model}}\Paren{\dept,\outcome\mid \doop{\sex}} : \tilde{\model} \in \nullgraphresp \right \rbrace \\
    \mkinter &= \left \lbrace P_{\tilde{\model}}\Paren{\dept,\outcome\mid \doop{\sex}} : \tilde{\model} \in \nullinterresp \right \rbrace.
\end{align*}

% Therefore, $\distgraph = \distiv$. The set of observed Markov kernels that are solutions of SCMs in $\nullinter$ is given by 
% \begin{equation}\label{eq:distinter}
%     \distinter \triangleq \left \lbrace P_{\model}\Paren{\outcome,\dept,\formsex \mid \doop{\sex}} : \model \in \nullinter \right \rbrace =  
% \end{equation}

We now show that $\mkinter = \mkgraph = \mkiv$. First, notice that $\mkinter \supseteq \mkgraph$ since $\nullinterresp \supseteq \nullgraphresp$. We first show that $\mkinter \subseteq \mkiv$ and then $\mkgraph = \mkiv$ which concludes the argument. 

\bm{$\mkinter \subseteq \mkiv$}: 
The solution function of the response-function parameterized SCM, $g_{A,D}: \cX_{\sex} \times \cX_{\response} \mapsto \cX_{\outcome} \times \cX_{\dept}$ induces a mapping from $\triangle\Paren{\cX_{\response}}$ which can be considered as a subset of $\RR^{\# \cX_{\response}}$ to the set of Markov kernels $P_{\tilde{\model}}\Paren{\dept,\outcome \mid \doop{\sex}}$ which  can be considered to be a subset of $\RR^{\#\Paren{\cX_{\outcome}}\times \#\Paren{\cX_{\dept}}\times \#\Paren{\cX_{\sex}}}$.
% We denote this map by $G: \RR^{\#\Paren{\cX_{\response}}} \mapsto \RR^{\#\Paren{\cX_{\outcome}}\times \#\Paren{\cX_{\dept}}\times \#\Paren{\cX_{\sex}}} $. 
% \begin{align*}
% g_{A,D}(\lsex,\respfunc) &= \Paren{\respfunc_2\Paren{\lsex,\respfunc_1\Paren{\lsex}},\respfunc_1\Paren{\lsex}}
% \end{align*}
% G\Paren{e_{\respfunc}} &= \sum\limits_{\lsex} e_{g_{A,D}\Paren{\lsex,\respfunc}}.
% First, note that for all $\tilde{\model} \in \nullinterresp$, $P_{\tilde{\model}}\Paren{\outcome,\dept,\formsex \mid \doop{\sex}} = P_{\tilde{\model}}\Paren{\outcome,\dept\mid \sex} \times\delta_{\sex}\Paren{\formsex}$. Therefore, we only restrict attention to $P_{\tilde{\model}}\Paren{\outcome,\dept\mid \sex = \formsex}$. 
The condition in \eqref{eq:respfun_inter_edge} implies that for all $\ldept$,
\begin{equation}\label{eq:constraint_outcome_one}
    \sum\limits_{\respfunc: \respfunc_2\Paren{0,\ldept}=1} \tilde{P}(\respfunc) = \sum\limits_{\respfunc: \respfunc_2\Paren{1,\ldept}=1} \tilde{P}(\respfunc). 
\end{equation}
Since, $\sum\limits_{\respfunc} \tilde{P}\Paren{\respfunc} = 1$, 
\begin{equation}\label{eq:constraint_outcome_zero}
    \sum\limits_{\respfunc: \respfunc_2\Paren{0,\ldept}=0} \tilde{P}(\respfunc) = \sum\limits_{\respfunc: \respfunc_2\Paren{1,\ldept}=0} \tilde{P}(\respfunc). 
\end{equation}
Denote $P_{\tilde{\model}}\Paren{\dept = \ldept, \outcome = \loutcome \mid \doop{\sex = \lsex}} $ by $P_{\tilde{\model}}\Paren{d,a || s}$. For  $P_{\tilde{\model}}\Paren{d,a || s} \in \mkinter$,
\begin{equation*}
    P_{\tilde{\model}}\Paren{d,a || s} = \sum\limits_{\respfunc: \respfunc_1\Paren{\lsex}=\ldept, \respfunc_2\Paren{\lsex,\ldept} = \loutcome
    } \tilde{P}(\respfunc). 
\end{equation*}
Therefore, from \eqref{eq:constraint_outcome_one}, 
\begin{align}
    \sum\limits_{\respfunc: \respfunc_2\Paren{0,\ldept}=1} \tilde{P}(\respfunc) &= P_{\tilde{\model}}(1,\ldept || 0) + \sum\limits_{\respfunc: \respfunc_1(0) \neq \ldept, \respfunc_2\Paren{0,\ldept}=1} \tilde{P}(\respfunc) \label{eq:1d0}\\
    &= \sum\limits_{\respfunc: \respfunc_2\Paren{1,\ldept}=1} \tilde{P}(\respfunc) \nonumber \\
    &= P_{\tilde{\model}}(1,\ldept || 1) + \sum\limits_{\respfunc: \respfunc_1(1) \neq \ldept, \respfunc_2\Paren{1,\ldept}=1} \tilde{P}(\respfunc) \label{eq:1d1}.
\end{align}
From \eqref{eq:constraint_outcome_zero}, 
\begin{align}
    \sum\limits_{\respfunc: \respfunc_2\Paren{0,\ldept}=0} \tilde{P}(\respfunc) &= P_{\tilde{\model}}(0,\ldept || 0) + \sum\limits_{\respfunc: \respfunc_1(0) \neq \ldept, \respfunc_2\Paren{0,\ldept}=0} \tilde{P}(\respfunc) \label{eq:0d0} \\
    &= \sum\limits_{\respfunc: \respfunc_2\Paren{1,\ldept}=0} \tilde{P}(\respfunc) \nonumber \\
    &= P_{\tilde{\model}}(0,\ldept || 1) + \sum\limits_{\respfunc: \respfunc_1(1) \neq \ldept, \respfunc_2\Paren{1,\ldept}=0} \tilde{P}(\respfunc) \label{eq:0d1}.
\end{align}
Since from \eqref{eq:constraint_outcome_one},
\begin{equation*}
    \sum\limits_{\respfunc} \tilde{P}\Paren{\respfunc} = \sum\limits_{\respfunc: \respfunc_2\Paren{0,\ldept}=0} \tilde{P}(\respfunc) + \sum\limits_{\respfunc: \respfunc_2\Paren{0,\ldept}=1} \tilde{P}(\respfunc) = \sum\limits_{\respfunc: \respfunc_2\Paren{0,\ldept}=0} \tilde{P}(\respfunc) + \sum\limits_{\respfunc: \respfunc_2\Paren{1,\ldept}=1} \tilde{P}(\respfunc) = 1.
\end{equation*}
Substituting from \eqref{eq:0d0} and \eqref{eq:1d1}, 
\begin{equation*}
    P_{\tilde{\model}}(0,\ldept || 0) + \sum\limits_{\respfunc: \respfunc_1(0) \neq \ldept, \respfunc_2\Paren{0,\ldept}=0} \tilde{P}(\respfunc) + P_{\tilde{\model}}(1,\ldept || 1) + \sum\limits_{\respfunc: \respfunc_1(1) \neq \ldept, \respfunc_2\Paren{1,\ldept}=1} \tilde{P}(\respfunc) =1. 
\end{equation*}
Similarly, substituting from \eqref{eq:0d1} and \eqref{eq:1d0}, 
\begin{equation*}
    P_{\tilde{\model}}(0,\ldept || 1) + \sum\limits_{\respfunc: \respfunc_1(1) \neq \ldept, \respfunc_2\Paren{1,\ldept}=0} \tilde{P}(\respfunc)+ P_{\tilde{\model}}(1,\ldept || 0) + \sum\limits_{\respfunc: \respfunc_1(0) \neq \ldept, \respfunc_2\Paren{0,\ldept}=1} \tilde{P}(\respfunc)=1. 
\end{equation*}
 This implies $P_{\tilde{\model}}(0,\ldept || 0) + P_{\tilde{\model}}(1,\ldept || 1) \leq 1, P_{\tilde{\model}}(0,\ldept || 1) + P_{\tilde{\model}}(1,\ldept || 0) \leq 1$. These are precisely the IV inequalities and they are satisfied. Therefore, $\mkinter \subseteq \mkiv$.

 \bm{$\mkgraph = \mkiv$} follows from Theorem~\ref{thm:iv_tight} since $\model \in \nullgraphrelax$ implies $\model \in \modelivrelax$. By Proposition~\ref{prop:cfnotions}, the lemma follows. 
 %We first show that $\distgraph$ is the same as the set of observed Markov kernels that are solutions of SCMs in $\modelsnoedge$, i.e.,

%  \begin{equation}
%      \distgraphnoedge \triangleq \left \lbrace P_{\model}\Paren{\outcome,\dept \mid \sex} : \model \in \modelsnoedge \right \rbrace = \distgraph.
%  \end{equation}
 
%  We then show that $\distgraphnoedge = \distiv$. 
 
% For $\model \in \modelsnoedge$, the response-function parameterization yields a counterfactually equivalent SCM, $\tilde{\model^*}$ represented by the tuple $(\exip,\enop,\tilde{\exrv}^*,\tilde{\spc}^*,\tilde{f}^*,\tilde{P}^*)$, where $\exip = \left \lbrace \sex \right \rbrace, \enop = \left \lbrace \formsex, \dept, \outcome \right \rbrace, \tilde{\exrv}^* = \left \lbrace \response^*\right \rbrace, \tilde{\spc} = \spc_{\exip}\times\spc_{\enop}\times\spc_{\tilde{\exrv}^*}, \tilde{f}^* = \left \lbrace \tilde{f}_{\formsex}^*, \tilde{f}_{\dept}^*, \tilde{f}_{\outcome}^* \right \rbrace$ where we define $\spc_{\response}^*, \tilde{f}^*,\tilde{P}^*$ as

% \begin{align*}
%     \spc_{\response^*} &\triangleq \spc_{\dept}^{\spc_{\sex}} \times \spc_{\outcome}^{\spc_{\dept}}, \\
%     \tilde{f}^*_{\formsex}(\sex) &\triangleq  f_{\formsex}(\sex) = \sex,\\
%     \tilde{f}^*_{\dept}\Paren{\respfunc^*,\sex} &\triangleq \respfunc_1^*\Paren{\sex} = f_{\dept}\Paren{\sex,U,U_{\dept}} = \respfunc_1(\sex), \\
%     \tilde{f}^*_{\outcome}\Paren{\respfunc^*,\dept} &\triangleq \respfunc_2^*\Paren{\dept} = f_{\outcome}\Paren{\dept,U,U_{\outcome}},
% \end{align*}
% where $\respfunc^* = \Paren{\respfunc_1^*,\respfunc_2^*}$.
% Note that $\spc_{\response^*}$ is a discrete space, $\response^*$ a discrete random variable, and $\tilde{P}^*$  a discrete distribution over $\spc_{\response^*}$. Under the response-function parameterization, only $\tilde{P}$ is a parameter. Therefore, we can represent $\modelsnoedge$ in the parameter space, $\modelsnoedgeresp \in \triangle\Paren{\spc_{\response^*}}$. By observational equivalence of the response-function parameterization,

% \begin{equation}
%     \distgraphnoedge = \left \lbrace P_{\tilde{\model}^*}\Paren{\outcome,\dept,\formsex \mid \doop{\sex}} : \tilde{\model}^* \in \modelsnoedgeresp \right \rbrace.
% \end{equation}

% Consider the set $D = \left \lbrace \respfunc \in \cX_{\response} \text{ such that } \exists \ldept \text{ where } \respfunc_2(0,\ldept) \neq \respfunc_2\Paren{1,\ldept} \right \rbrace.$ For any $\tilde{P} \in \nullgraphresp$, $\tilde{P}\Paren{D}=0$. Therefore, the function $h: \nullgraphresp \mapsto \modelsnoedgeresp$ such that $$h(\tilde{P})(\respfunc_1^*(\lsex), \respfunc_2^*(\ldept)) = \tilde{P}\Paren{\respfunc_1(\lsex),\respfunc_2(0,\ldept)=\respfunc_2(1,\ldept)}$$
% is well-defined and bijective since for any $\tilde{P}^* \in \modelsnoedgeresp$, $$h^{-1}(\tilde{P}^*)\Paren{\respfunc_1(\lsex),\respfunc_2(0,\ldept),\respfunc_2(1,\ldept)} = \bm{1}\left[ \respfunc_2(0,\ldept) = \respfunc_2(1,\ldept)\right] \tilde{P}^*\Paren{\respfunc_1(\lsex),\respfunc_2(0,d)}. $$ 

% The solution function of the response-function parameterized SCM $\tilde{\model^*}$ denoted by $g^*: \cX_{\sex} \times \cX_{\response^*} \mapsto \cX_{\outcome} \times \cX_{\dept} \times \cX_{\formsex}$ induces a mapping from $\triangle\Paren{\cX_{\response^*}}$ which can be considered as a subset of $\RR^{\# \cX_{\response^*}}$ to the set of conditional distributions $\Pr\Paren{\outcome,\dept,\formsex \mid \sex}$ which  can be considered to be a subset of $\RR^{\#\Paren{\cX_{\outcome}}\times \#\Paren{\cX_{\dept}}\times \#\Paren{\cX_{\formsex}}}$. We denote this map by $G^*: \RR^{\#\Paren{\cX_{\response^*}}} \mapsto \RR^{\#\Paren{\cX_{\outcome}}\times \#\Paren{\cX_{\dept}}\times \#\Paren{\cX_{\formsex}}} $ where 

% \begin{align*}
% g^*(\lsex,\respfunc^*) &= \Paren{\respfunc_2^*\Paren{\respfunc_1^*\Paren{\lsex}},\respfunc_1^*\Paren{\lsex},\lsex} =\Paren{\respfunc_2^*\Paren{\respfunc_1\Paren{\lsex}},\respfunc_1\Paren{\lsex},\lsex} , \\
% G^*\Paren{e_{\respfunc^*}} &= \sum\limits_{\lsex} e_{g^*\Paren{\lsex,\respfunc^*}}. 
% \end{align*}
% For $\respfunc \notin D$, note that $g(\lsex,\respfunc) = g^*(\lsex,(\respfunc_1,\respfunc_2'))$ where $\respfunc_2'(\ldept) = \respfunc_2(0,\ldept) = \respfunc_2(1,\ldept)$. Further, for all $\respfunc^* \in \cX_{\response^*}, g^*(\lsex,\respfunc^*) = g(\lsex,\respfunc')$ where $\respfunc' \notin D$ and is defined as $(\respfunc_1' = \respfunc_1^*, \respfunc_2'(0,\ldept) = \respfunc_2'(1,\ldept) =\respfunc_2^*(\ldept))$ Therefore, for any $\tilde{P} \in \nullgraphresp$, since $\tilde{P}(D) =0, G(\tilde{P}) = G^*(h(\tilde{P}))$ thus implying that $\distgraphnoedge = \distgraph$. 

% \begin{lemma}
% \begin{equation}
%     \distgraphnoedge = \distiv. 
% \end{equation}
% \end{lemma}

%\todo{Add proof from lecture notes.}

% We prove a bijection between $\modelsnoedgeresp$ and $\nullgraphresp$. For $\tilde{P} \in \nullgraphresp$, 


%  \begin{equation}\label{eq:respfunc_graph_noedge}
%     \modelsnoedgeresp
%     \triangleq \left \lbrace \tilde{P} \in \triangle\Paren{\cX_{\response}} : \tilde{P}\Paren{\respfunc_1,\respfunc_2} \neq 0 \text{ implies } \forall \ldept, \respfunc_2(0,\ldept) = \respfunc_2\Paren{1,\ldept} \right \rbrace.
% \end{equation}

\end{proof}




% For $\model \in \modelsedge$, the response-function parameterization yields a counterfactually equivalent \todo{Define in Preliminaries?}SCM, $\tilde{\model}$ represented by the tuple $(\exip,\enop,\tilde{\exrv},\tilde{\spc},\tilde{f},\tilde{P})$, where $\exip = \left \lbrace \sex \right \rbrace, \enop = \left \lbrace \formsex, \dept, \outcome \right \rbrace, \tilde{\exrv} = \left \lbrace \response\right \rbrace, \tilde{\spc} = \spc_{\exip}\times\spc_{\enop}\times\spc_{\tilde{\exrv}}, \tilde{f} = \left \lbrace \tilde{f}_{\formsex}, \tilde{f}_{\dept}, \tilde{f}_{\outcome} \right \rbrace$ where we define $\spc_{\response}, \tilde{f},\tilde{P}$ as

% \begin{align*}
%     \spc_{\response} &\triangleq \spc_{\dept}^{\spc_{\sex}} \times \spc_{\outcome}^{\spc_{\formsex}\times \spc_{\dept}}, \\
%     \tilde{f}_{\formsex}(\sex) &\triangleq  f_{\formsex}(\sex) = \sex,\\
%     \tilde{f}_{\dept}\Paren{\respfunc,\sex} &\triangleq \respfunc_1\Paren{\sex} = f_{\dept}\Paren{\sex,U,U_{\dept}}, \\
%     \tilde{f}_{\outcome}\Paren{\respfunc,\formsex,\dept} &\triangleq \respfunc_2\Paren{\formsex,\dept} = f_{\outcome}\Paren{\formsex,\dept,U,U_{\outcome}},
% \end{align*}
% where $\respfunc = \Paren{\respfunc_1,\respfunc_2}$.
% Note that $\spc_{\response}$ is a discrete space, $\response$ a discrete random variable, and $\tilde{P}$  a discrete distribution over $\spc_{\response}$. Under the response-function parameterization, only $\tilde{P}$ is a parameter. Therefore, we can represent $\nullgraph$ in the parameter space, $\nullgraphresp$, defined as \todo{Might need to add definition of parent in a casual graph, in preliminaries?} 

% % \begin{equation}\label{eq:respfunc_graph_edge}
% %     \nullgraphresp 
% %     \triangleq \left \lbrace \tilde{P} \in \triangle\Paren{\cX_{\response}} : \tilde{P}\Paren{\respfunc_1,\respfunc_2} = 0 \text{ where } \respfunc_2\Paren{.,.} \text{ is such that } \exists \ldept 
% %     \text{ such that }\respfunc_2(m,\ldept) \neq \respfunc_2\Paren{f,\ldept} \right \rbrace.
% % \end{equation}

% \begin{equation}\label{eq:respfunc_graph_edge}
%     \nullgraphresp 
%     \triangleq \left \lbrace \tilde{P} \in \triangle\Paren{\cX_{\response}} : \tilde{P}\Paren{\respfunc_1,\respfunc_2} \neq 0 \text{ implies } \forall \ldept, \respfunc_2(0,\ldept) = \respfunc_2\Paren{1,\ldept} \right \rbrace.
% \end{equation}

% To express $\nullinter$, we express the interventional Markov kernels $\Pr\Paren{\outcome\mid \doop{\formsex}, \doop{\dept}}$ in terms of $\tilde{P}$, 
% \begin{equation}\label{eq:inter_resp}
%     \Pr\Paren{\outcome=1\mid \doop{\formsex=\lsex'}, \doop{\dept=\ldept}} = \sum\limits_{\Paren{\respfunc_1,\respfunc_2}  \in \cX_{\response}}\bm{1}\Brack{\respfunc_2\Paren{\lsex',\ldept}=1}\tilde{P}\Paren{\respfunc_1,\respfunc_2}.
% \end{equation}

% Therefore the response-function parameterized analogue of $\nullinter$ is 
% \begin{equation}\label{eq:respfun_inter_edge}
%     \nullinterresp \triangleq \left \lbrace \tilde{P} \in \triangle\Paren{\cX_{\response}} : \forall \ldept, \sum\limits_{\Paren{\respfunc_1,\respfunc_2}  \in \cX_{\response}} \Paren{\bm{1}\Brack{\respfunc_2\Paren{0,\ldept}=1} - \bm{1}\Brack{\respfunc_2\Paren{1,\ldept}=1}}\tilde{P}\Paren{\respfunc_1,\respfunc_2} = 0 \right \rbrace. 
% \end{equation}

% Note that both $\nullgraphresp$ and $\nullinterresp$ are polyhedra in $\triangle\Paren{\cX_{\response}}$. Further, $\nullgraphresp \subseteq \nullinterresp$. While, $\nullgraphresp, \nullinterresp$ are collections of distributions, we will abuse notation and also refer to them as collection of response-function-parameterized SCMs. 

% %So far, we looked at the response-function parameterization for models in $\modelsedge$. However, the instrumental-variable inequalities arise from 

% % While we have framed the hypotheses in terms of the exogenous distribution of the response-function parameterization, for a statistical test, we only have access to the observed Markov kernels $\Pr\Paren{\outcome,\dept,\formsex \mid \doop{\sex}}$. Therefore, we now characterize the sets of observed Markov kernels 


% % It can be shown that the set of observed Markov kernels that are solutions of SCMs in $\nullgraph$ is the same as $\distiv$ where we define the former as 
% From the observational equivalence of the response-function-parameterization, we have \todo{Observational equivalence also in preliminaries.}
% \begin{align*}
%     \distgraph &= \left \lbrace P_{\tilde{\model}}\Paren{\outcome,\dept,\formsex \mid \doop{\sex}} : \tilde{\model} \in \nullgraphresp \right \rbrace \\
%     \distinter &= \left \lbrace P_{\tilde{\model}}\Paren{\outcome,\dept,\formsex \mid \doop{\sex}} : \tilde{\model} \in \nullinterresp \right \rbrace.
% \end{align*}

% % Therefore, $\distgraph = \distiv$. The set of observed Markov kernels that are solutions of SCMs in $\nullinter$ is given by 
% % \begin{equation}\label{eq:distinter}
% %     \distinter \triangleq \left \lbrace P_{\model}\Paren{\outcome,\dept,\formsex \mid \doop{\sex}} : \model \in \nullinter \right \rbrace =  
% % \end{equation}

% We now show that $\distinter = \distgraph = \distiv$. First, notice that $\distinter \supseteq \distgraph$ since $\nullinterresp \supseteq \nullgraphresp$. We first show that $\distinter \subseteq \distiv$ and then $\distgraph = \distiv$ which concludes the argument. 

% \bm{$\distinter \subseteq \distiv$}: 
% The solution function of the response-function parameterized SCM, $g: \cX_{\sex} \times \cX_{\response} \mapsto \cX_{\outcome} \times \cX_{\dept} \times \cX_{\formsex}$ induces a mapping from $\triangle\Paren{\cX_{\response}}$ which can be considered as a subset of $\RR^{\# \cX_{\response}}$ to the set of conditional distributions $\Pr\Paren{\outcome,\dept,\formsex \mid \sex}$ which  can be considered to be a subset of $\RR^{\#\Paren{\cX_{\outcome}}\times \#\Paren{\cX_{\dept}}\times \#\Paren{\cX_{\formsex}}}$. We denote this map by $G: \RR^{\#\Paren{\cX_{\response}}} \mapsto \RR^{\#\Paren{\cX_{\outcome}}\times \#\Paren{\cX_{\dept}}\times \#\Paren{\cX_{\formsex}}} $ where 

% \begin{align*}
% g(\lsex,\respfunc) &= \Paren{\respfunc_2\Paren{\lsex,\respfunc_1\Paren{\lsex}},\respfunc_1\Paren{\lsex},\lsex}, \\
% G\Paren{e_{\respfunc}} &= \sum\limits_{\lsex} e_{g\Paren{\lsex,\respfunc}}. 
% \end{align*}

% First, note that for all $\tilde{\model} \in \nullinterresp$, $P_{\tilde{\model}}\Paren{\outcome,\dept,\formsex \mid \doop{\sex}} = P_{\tilde{\model}}\Paren{\outcome,\dept\mid \sex} \times\delta_{\sex}\Paren{\formsex}$. Therefore, we only restrict attention to $P_{\tilde{\model}}\Paren{\outcome,\dept\mid \sex = \formsex}$. The condition in \eqref{eq:distinter} implies that for all $\ldept$,
% \begin{equation}\label{eq:constraint_outcome_one}
%     \sum\limits_{\respfunc: \respfunc_2\Paren{0,\ldept}=1} \tilde{P}(\respfunc) = \sum\limits_{\respfunc: \respfunc_2\Paren{1,\ldept}=1} \tilde{P}(\respfunc). 
% \end{equation}

% Since, $\sum\limits_{\respfunc} \tilde{P}\Paren{\respfunc} = 1$, 

% \begin{equation}\label{eq:constraint_outcome_zero}
%     \sum\limits_{\respfunc: \respfunc_2\Paren{0,\ldept}=0} \tilde{P}(\respfunc) = \sum\limits_{\respfunc: \respfunc_2\Paren{1,\ldept}=0} \tilde{P}(\respfunc). 
% \end{equation}

% Expressing the marginal over $\outcome, \dept$ 
%  of $P \in \distinter$,
% \begin{equation*}
%     p\Paren{\loutcome,\ldept \mid \lsex} = \sum\limits_{\respfunc: \respfunc_1\Paren{\lsex}=\ldept, \respfunc_2\Paren{\lsex,\ldept} = \loutcome
%     } \tilde{P}(\respfunc). 
% \end{equation*}

% Therefore, from \eqref{eq:constraint_outcome_one} and \eqref{eq:constraint_outcome_zero}, 
% \begin{align*}
%     \sum\limits_{\respfunc: \respfunc_2\Paren{0,\ldept}=1} \tilde{P}(\respfunc) &= p(1,\ldept \mid 0) + \sum\limits_{\respfunc: \respfunc_1(0) \neq \ldept, \respfunc_2\Paren{0,\ldept}=1} \tilde{P}(\respfunc) = \sum\limits_{\respfunc: \respfunc_2\Paren{1,\ldept}=1} \tilde{P}(\respfunc) = p(1,\ldept \mid 1) + \sum\limits_{\respfunc: \respfunc_1(1) \neq \ldept, \respfunc_2\Paren{1,\ldept}=1} \tilde{P}(\respfunc)\\
%     \sum\limits_{\respfunc: \respfunc_2\Paren{0,\ldept}=0} \tilde{P}(\respfunc) &= p(0,\ldept \mid 0) + \sum\limits_{\respfunc: \respfunc_1(0) \neq \ldept, \respfunc_2\Paren{0,\ldept}=0} \tilde{P}(\respfunc) = \sum\limits_{\respfunc: \respfunc_2\Paren{1,\ldept}=0} \tilde{P}(\respfunc) = p(0,\ldept \mid 1) + \sum\limits_{\respfunc: \respfunc_1(1) \neq \ldept, \respfunc_2\Paren{1,\ldept}=0} \tilde{P}(\respfunc)
% \end{align*}

% Since
% \begin{equation*}
%     \sum\limits_{\respfunc} \tilde{P}\Paren{\respfunc} = \sum\limits_{\respfunc: \respfunc_2\Paren{0,\ldept}=0} \tilde{P}(\respfunc) + \sum\limits_{\respfunc: \respfunc_2\Paren{0,\ldept}=1} \tilde{P}(\respfunc) = \sum\limits_{\respfunc: \respfunc_2\Paren{0,\ldept}=0} \tilde{P}(\respfunc) + \sum\limits_{\respfunc: \respfunc_2\Paren{1,\ldept}=1} \tilde{P}(\respfunc) = 1, 
% \end{equation*}
% we have 
% \begin{equation*}
%     p(0,\ldept \mid 0) + \sum\limits_{\respfunc: \respfunc_1(0) \neq \ldept, \respfunc_2\Paren{0,\ldept}=0} \tilde{P}(\respfunc) + p(1,\ldept \mid 1) + \sum\limits_{\respfunc: \respfunc_1(1) \neq \ldept, \respfunc_2\Paren{1,\ldept}=1} \tilde{P}(\respfunc) =1. 
% \end{equation*}

% Similarly, 
% \begin{equation*}
%     p(0,\ldept \mid 1) + \sum\limits_{\respfunc: \respfunc_1(1) \neq \ldept, \respfunc_2\Paren{1,\ldept}=0} \tilde{P}(\respfunc)+ p(1,\ldept \mid 0) + \sum\limits_{\respfunc: \respfunc_1(0) \neq \ldept, \respfunc_2\Paren{0,\ldept}=1} \tilde{P}(\respfunc)=1. 
% \end{equation*}
%  Since the IV inequalities are satisfied, $\distinter \subseteq \distiv$.

%  \bm{$\distgraph = \distiv$}: We first show that $\distgraph$ is the same as the set of observed Markov kernels that are solutions of SCMs in $\modelsnoedge$, i.e.,

%  \begin{equation}
%      \distgraphnoedge \triangleq \left \lbrace P_{\model}\Paren{\outcome,\dept,\formsex \mid \doop{\sex}} : \model \in \modelsnoedge \right \rbrace = \distgraph.
%  \end{equation}
 
%  We then show that $\distgraphnoedge = \distiv$. 
 
% For $\model \in \modelsnoedge$, the response-function parameterization yields a counterfactually equivalent SCM, $\tilde{\model^*}$ represented by the tuple $(\exip,\enop,\tilde{\exrv}^*,\tilde{\spc}^*,\tilde{f}^*,\tilde{P}^*)$, where $\exip = \left \lbrace \sex \right \rbrace, \enop = \left \lbrace \formsex, \dept, \outcome \right \rbrace, \tilde{\exrv}^* = \left \lbrace \response^*\right \rbrace, \tilde{\spc} = \spc_{\exip}\times\spc_{\enop}\times\spc_{\tilde{\exrv}^*}, \tilde{f}^* = \left \lbrace \tilde{f}_{\formsex}^*, \tilde{f}_{\dept}^*, \tilde{f}_{\outcome}^* \right \rbrace$ where we define $\spc_{\response}^*, \tilde{f}^*,\tilde{P}^*$ as

% \begin{align*}
%     \spc_{\response^*} &\triangleq \spc_{\dept}^{\spc_{\sex}} \times \spc_{\outcome}^{\spc_{\dept}}, \\
%     \tilde{f}^*_{\formsex}(\sex) &\triangleq  f_{\formsex}(\sex) = \sex,\\
%     \tilde{f}^*_{\dept}\Paren{\respfunc^*,\sex} &\triangleq \respfunc_1^*\Paren{\sex} = f_{\dept}\Paren{\sex,U,U_{\dept}} = \respfunc_1(\sex), \\
%     \tilde{f}^*_{\outcome}\Paren{\respfunc^*,\dept} &\triangleq \respfunc_2^*\Paren{\dept} = f_{\outcome}\Paren{\dept,U,U_{\outcome}},
% \end{align*}
% where $\respfunc^* = \Paren{\respfunc_1^*,\respfunc_2^*}$.
% Note that $\spc_{\response^*}$ is a discrete space, $\response^*$ a discrete random variable, and $\tilde{P}^*$  a discrete distribution over $\spc_{\response^*}$. Under the response-function parameterization, only $\tilde{P}$ is a parameter. Therefore, we can represent $\modelsnoedge$ in the parameter space, $\modelsnoedgeresp \in \triangle\Paren{\spc_{\response^*}}$. By observational equivalence of the response-function parameterization,

% \begin{equation}
%     \distgraphnoedge = \left \lbrace P_{\tilde{\model}^*}\Paren{\outcome,\dept,\formsex \mid \doop{\sex}} : \tilde{\model}^* \in \modelsnoedgeresp \right \rbrace.
% \end{equation}

% Consider the set $D = \left \lbrace \respfunc \in \cX_{\response} \text{ such that } \exists \ldept \text{ where } \respfunc_2(0,\ldept) \neq \respfunc_2\Paren{1,\ldept} \right \rbrace.$ For any $\tilde{P} \in \nullgraphresp$, $\tilde{P}\Paren{D}=0$. Therefore, the function $h: \nullgraphresp \mapsto \modelsnoedgeresp$ such that $$h(\tilde{P})(\respfunc_1^*(\lsex), \respfunc_2^*(\ldept)) = \tilde{P}\Paren{\respfunc_1(\lsex),\respfunc_2(0,\ldept)=\respfunc_2(1,\ldept)}$$
% is well-defined and bijective since for any $\tilde{P}^* \in \modelsnoedgeresp$, $$h^{-1}(\tilde{P}^*)\Paren{\respfunc_1(\lsex),\respfunc_2(0,\ldept),\respfunc_2(1,\ldept)} = \bm{1}\left[ \respfunc_2(0,\ldept) = \respfunc_2(1,\ldept)\right] \tilde{P}^*\Paren{\respfunc_1(\lsex),\respfunc_2(0,d)}. $$ 

% The solution function of the response-function parameterized SCM $\tilde{\model^*}$ denoted by $g^*: \cX_{\sex} \times \cX_{\response^*} \mapsto \cX_{\outcome} \times \cX_{\dept} \times \cX_{\formsex}$ induces a mapping from $\triangle\Paren{\cX_{\response^*}}$ which can be considered as a subset of $\RR^{\# \cX_{\response^*}}$ to the set of conditional distributions $\Pr\Paren{\outcome,\dept,\formsex \mid \sex}$ which  can be considered to be a subset of $\RR^{\#\Paren{\cX_{\outcome}}\times \#\Paren{\cX_{\dept}}\times \#\Paren{\cX_{\formsex}}}$. We denote this map by $G^*: \RR^{\#\Paren{\cX_{\response^*}}} \mapsto \RR^{\#\Paren{\cX_{\outcome}}\times \#\Paren{\cX_{\dept}}\times \#\Paren{\cX_{\formsex}}} $ where 

% \begin{align*}
% g^*(\lsex,\respfunc^*) &= \Paren{\respfunc_2^*\Paren{\respfunc_1^*\Paren{\lsex}},\respfunc_1^*\Paren{\lsex},\lsex} =\Paren{\respfunc_2^*\Paren{\respfunc_1\Paren{\lsex}},\respfunc_1\Paren{\lsex},\lsex} , \\
% G^*\Paren{e_{\respfunc^*}} &= \sum\limits_{\lsex} e_{g^*\Paren{\lsex,\respfunc^*}}. 
% \end{align*}
% For $\respfunc \notin D$, note that $g(\lsex,\respfunc) = g^*(\lsex,(\respfunc_1,\respfunc_2'))$ where $\respfunc_2'(\ldept) = \respfunc_2(0,\ldept) = \respfunc_2(1,\ldept)$. Further, for all $\respfunc^* \in \cX_{\response^*}, g^*(\lsex,\respfunc^*) = g(\lsex,\respfunc')$ where $\respfunc' \notin D$ and is defined as $(\respfunc_1' = \respfunc_1^*, \respfunc_2'(0,\ldept) = \respfunc_2'(1,\ldept) =\respfunc_2^*(\ldept))$ Therefore, for any $\tilde{P} \in \nullgraphresp$, since $\tilde{P}(D) =0, G(\tilde{P}) = G^*(h(\tilde{P}))$ thus implying that $\distgraphnoedge = \distgraph$. 

% \begin{lemma}
% \begin{equation}
%     \distgraphnoedge = \distiv. 
% \end{equation}
% \end{lemma}

% %\todo{Add proof from lecture notes.}

% % We prove a bijection between $\modelsnoedgeresp$ and $\nullgraphresp$. For $\tilde{P} \in \nullgraphresp$, 


% %  \begin{equation}\label{eq:respfunc_graph_noedge}
% %     \modelsnoedgeresp
% %     \triangleq \left \lbrace \tilde{P} \in \triangle\Paren{\cX_{\response}} : \tilde{P}\Paren{\respfunc_1,\respfunc_2} \neq 0 \text{ implies } \forall \ldept, \respfunc_2(0,\ldept) = \respfunc_2\Paren{1,\ldept} \right \rbrace.
% % \end{equation}

% \end{proof}
% \section{Encompassing Prior Method}

For the sake of completeness, we outline the computation of Bayes Factor using the encompassing prior method \cite{KlugkistKH05}. We also refer the reader to \cite{HeckDavisStrober19} for theoretical justification of computing the Bayes factor using approximations that we do in our procedure. 

Consider two models $\MM_{c}, \MM_{uc}$ where $\MM_{c}$ is a model that is ``encompassed" in $\MM_{uc}$ such that $\MM_{c}$ is constrained by inequality constraints on the unknown parameters. We formally define this as follows. 

\begin{definition}[Inequality-Constrained Encompassing Models]
    A pair of models $\Paren{\MM_{uc},\MM_{c}}$ are inequality-constrained encompassing models if 
\end{definition}

\begin{proposition}\label{prop:BFencompassing}
Let $\Paren{\MM_{uc},\MM_{c}}$ be inequality-constrained encompassing models. Then the Bayes factor is given by 

\begin{equation}
    \bayesfactor\Paren{\MM_{c},\MM_{uc}} = \frac{C_{pri}}{C_{post}},
\end{equation}
where $C_{pri}$ and $C_{post}$ are normalizing constants defined as 
\begin{align}\label{eq:normalizingconstants}
1/C_{pri} &= \int_{\theta \in \MM_{c}} P\Paren{\theta|\MM_{uc}} d\theta, \\
1/C_{post} &= \int_{\theta \in \MM_{c}} P\Paren{\theta|X_1,X_2, \cdots, X_m, \MM_{uc}} d\theta.
\end{align}
\end{proposition}

\begin{proof}
    We know that 
    \begin{equation}\label{eq:bayesfactorratiolikelihood}
        \bayesfactor\Paren{\MM_{c},\MM_{uc}} = \frac{P\Paren{X_1,X_2,\cdots, X_m \mid \MM_{uc}}}{P\Paren{X_1,X_2,\cdots, X_m \mid \MM_c}}.
    \end{equation}
From Bayes Theorem, for a model $\MM$, dataset $\dataset = \Paren{X_1,X_2,\cdots, X_m}$ and parameter $\theta$,

\begin{equation*}
    P\Paren{\theta | \dataset, \MM} = \frac{P\Paren{\dataset|\theta,\MM}P\Paren{\theta|\MM}}{P\Paren{\dataset|\MM}}.
\end{equation*}
    By rearranging, we can express the marginal likelihood of the model, 
\begin{equation*}
    P\Paren{\dataset|\MM} =  \frac{P\Paren{\dataset|\theta,\MM}P\Paren{\theta|\MM}}{P\Paren{\theta | \dataset, \MM}}.
\end{equation*}
Substituting in equation~\ref{eq:bayesfactorratiolikelihood}, we have 
    \begin{equation}\label{eq:bayesfactorsub}
        \bayesfactor\Paren{\MM_{c},\MM_{uc}} = \frac{P\Paren{X_1,X_2,\cdots, X_m \mid \MM_{uc},\theta} P\Paren{\theta|\MM_{uc}}/P\Paren{\theta|X_1,X_2,\cdots, X_m, \MM_{uc}}}{P\Paren{X_1,X_2,\cdots, X_m \mid \MM_{c},\theta} P\Paren{\theta|\MM_{c}}/P\Paren{\theta|X_1,X_2,\cdots, X_m, \MM_{c}}}.
    \end{equation}
Since $\MM_{c}$ is nested in $\MM_{uc}$, for a particular $\theta'$ that is in both the models, we have
\begin{equation*}
    P\Paren{X_1,X_2,\cdots, X_m \mid \MM_{uc},\theta'} = P\Paren{X_1,X_2,\cdots, X_m \mid \MM_{c},\theta'}.
\end{equation*}
 Therefore,
\begin{equation}\label{eq:bayesfactorreduce}
        \bayesfactor\Paren{\MM_{c},\MM_{uc}} = \frac{P\Paren{\theta'|\MM_{uc}}/P\Paren{\theta'|X_1,X_2,\cdots, X_m, \MM_{uc}}}{ P\Paren{\theta'|\MM_{c}}/P\Paren{\theta'|X_1,X_2,\cdots, X_m, \MM_{c}}}.
    \end{equation}
We can express the prior and the posterior for the constrained model as truncated versions of the same for the unconstrained model. 

\begin{align*}
    P\Paren{\theta'|\MM_{c}} &= C_{pri} P\Paren{\theta'|\MM_{uc}}\bm{1}\Brack{\theta' \in \MM_{c}},\\
    P\Paren{\theta'|X_1,X_2,\cdots, X_m, \MM_{c}} &= C_{post}P\Paren{\theta'|X_1,X_2,\cdots, X_m, \MM_{uc}}\bm{1}\Brack{\theta' \in \MM_{c}},
\end{align*}
where $C_{pri}$, $C_{post}$ are normalizing constants defined in equation~\ref{eq:normalizingconstants}. 
Substituting the above in equation~\ref{eq:bayesfactorreduce}, 
\begin{equation}\label{eq:bayesfactorfinal}
        \bayesfactor\Paren{\MM_{c},\MM_{uc}} = \frac{C_{pri}}{C_{post}}.
    \end{equation}
\end{proof}

Proposition~\ref{prop:BFencompassing} lends itself to a simple computational procedure to compute the Bayes factor given in Algorithm~\ref{alg:bayesfactor}.

% \begin{algorithm}[t]\label{alg:bayesfactor}
%     \SetKwInOut{Input}{Input}
%     \SetKwInOut{Output}{Output}
%     \Input{$D = \Paren{X_1,X_2,\ldots, X_m},\alpha=\Paren{1,1,\cdots,1},\smp_{pri},\smp_{post}$}
%     \Output{$\bayesfactor\Paren{\MM_c,\MM_{uc}}$}
%     \BlankLine
%     Set $R_{pri} = 0, R_{post}=0$
    
%     \For{$j \in \left[\smp_{pri}\right]$}
%     {
%     Sample $\hat{p}_j \sim \dir\Paren{\alpha}$ \;
%     \If{$\hat{p}_j$ satisfies inequality constraint}
%     {$R_{pri} = R_{pri}+1$ \;
%     }
%     }
%     $\hat{C}_{pri} = \frac{R_{pri}}{\smp_{pri}}$ \;
%     $\Paren{N_1,N_2,\cdots, N_k} = count(D)$ \;
%     \For{$j \in \left[\smp_{post}\right]$}
%     {
%     Sample $\hat{p}_j \sim \dir\Paren{\alpha+\Paren{N_1,N_2,\cdots, N_k}}$ \;
%     \If{$\hat{p}_j$ satisfies inequality constraint}
%     {$R_{post} = R_{post}+1$ \;
%     }
%     }
%     $\hat{C}_{post} = \frac{R_{post}}{\smp_{post}}$\;
%     \Return{$\bayesfactor\Paren{\MM_c,\MM_{uc}} = \frac{\hat{C}_{post}}{\hat{C}_{pri}}$}
%     \caption{}
% \end{algorithm}
\end{document}
