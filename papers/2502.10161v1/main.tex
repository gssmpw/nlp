\documentclass[11pt]{article}
\def\withcolors{1}
\def\withnotes{1}
\def\withbiblioconversion{1} % long names for the journals
\usepackage[round]{natbib}
\renewcommand{\bibname}{References}
\renewcommand{\bibsection}{\subsubsection*{\bibname}}
\usepackage{amsmath,amsthm,enumitem,nicefrac,bbm,verbatim,amssymb,amsfonts,amscd, graphicx,algpseudocode,hyperref,float,mathtools,bm,xcolor,appendix,subcaption}
\usepackage{tikz}
\usepackage{tikz-3dplot}
\usetikzlibrary{positioning, decorations.pathreplacing, calc, intersections, pgfplots.groupplots}
\usetikzlibrary{positioning, decorations.markings}
\usepackage{pgfplots}
\usepackage{tikz-network}
\usetikzlibrary{shapes,decorations,arrows,calc,arrows.meta,fit,positioning}
\usepackage{thmtools}
\usepackage{thm-restate}
\pgfplotsset{compat=1.5}
\pgfplotsset{scaled x ticks=false}
\tikzset{
	-Latex,auto,node distance =1 cm and 1 cm,semithick,
	state/.style ={ellipse, draw, minimum width = 0.7 cm},
	point/.style = {circle, draw, inner sep=0.04cm,fill,node contents={}},
	bidirected/.style={Latex-Latex},
	el/.style = {inner sep=2pt, align=left, sloped}
}
\usepackage[utf8]{inputenc}
\usepackage[T1]{fontenc}
\usepackage[ruled, lined, linesnumbered, commentsnumbered, longend]{algorithm2e}

\input{glodef}
% Frequently used symbols
\newcommand{\perturb}{\gamma}
\newcommand{\dims}{d}
\newcommand{\zdims}{k}
\newcommand{\nsamps}{m}
\newcommand{\nb}{t}
\newcommand{\nspu}{m}
\newcommand{\nin}{r}
\newcommand{\nms}{{N_S}}
\newcommand{\nbb}{{t'}}
\newcommand{\nbbb}{{t''}}
\newcommand{\gbit}{s}
\newcommand{\unif}{{\mathbf{u}}}
\newcommand{\ngr}{{n_0}}
\newcommand{\DiffS}{{\Delta_S}}
\newcommand{\epsnew}{{\eps_0}}
\newcommand{\alphanew}{{\alpha_0}}
\newcommand{\rank}{{\text{rank}}}

\newcommand{\Proj}{{\Pi}}
\newcommand{\povmset}{{\mathfrak{M}}}
\newcommand{\nqubits}{{N}}
\newcommand{\pauliI}{{\sigma_I}}
\newcommand{\pauliX}{{\sigma_X}}
\newcommand{\pauliY}{{\sigma_Y}}
\newcommand{\pauliZ}{{\sigma_Z}}
\newcommand{\pauliObsSet}{{\mathcal{P}}}
\newcommand{\prPauli}[2]{{\p_{#2}(#1)}}

\newcommand{\opnorm}[1]{{\left\|#1\right\|}_{\text{op}}}
\newcommand{\tracenorm}[1]{{\left\|#1\right\|}_{1}}
\newcommand{\hsnorm}[1]{{\left\|#1\right\|}_{\text{HS}}}
\newcommand{\barDelta}{{\overline{\Delta}}}
\newcommand{\ptb}{{z}}
\newcommand{\ptbDistr}{{\mathcal{D}_{\ell,\cd}}}

\newcommand{\supparen}[1]{^{(#1)}}
\newcommand{\subparen}[1]{_{(#1)}}

% Constants
\newcommand{\cd}{{c}}
\newcommand{\cop}{\kappa}


\newcommand{\isthestate}{\texttt{YES}}
\newcommand{\notthestate}{\texttt{NO}}


\newcommand{\cA}{\mathcal{A}}

% Font
\newcommand{\tst}{t}  % Time: to be used when not for a sum index (e.g., 
%"at time $tst$")
\newcommand{\ts}{r} % Time step: index. To be used  for a sum index (e.g., 
%"$\sum_{\ts=1}^\ts")

\newcommand{\etaa}{\gamma}
%\newcommand{\etab}{\eta}
\newcommand{\signA}{{\bf{}1}^\star}
\newcommand{\signB}{{\bf{}0}^\star}
\newcommand{\setS}{S}
\newcommand{\vecu}{u}
\newcommand{\zero}{\mathbf{0}}

\newcommand{\chd}[1]{\Delta_{#1}^y}
\newcommand{\ratioparam}[1]{\kappa_{\scalebox{0.5}{\ensuremath{#1}}}}%
\newcommand{\indbig}[1]{\one\left\{#1\right\}}
\newcommand{\bfP}{\mathbf{P}}
\newcommand{\bfQ}{\mathbf{Q}}
\newcommand{\hbP}{\hat{\bfP}}
\newcommand{\trans}{\mathcal{T}}
\newcommand{\odiag}{E}
\newcommand{\x}{\mathbf{x}}
\newcommand{\out}{{x}}


\newcommand{\risk}{\mathcal{R}}
\def\multiset#1#2{\ensuremath{\left(\kern-.3em\left(\genfrac{}{}{0pt}{}{#1}{#2}\right)\kern-.3em\right)}}

\newcommand{\hp}{\widehat{\p}}
 
\newcommand{\ham}[2]{\operatorname{d}_{\rm Ham}(#1,#2)}
\newcommand{\variance}[2]{\var_{#1}{\mleft[#2\mright]}}

% quantum
\newcommand{\bm}{{\mathbf{m}}}
\newcommand{\qs}{\rho}
\newcommand{\qmm}{{\rho_{\text{mm}}}}
\newcommand{\qkn}{{\rho_0}}
\newcommand{\blambda}{{\boldsymbol{\lambda}}}
\newcommand{\SW}{\textbf{SW}}
\newcommand{\ngroups}{N}
\newcommand{\pnew}{\mathbf{p}}


\newcommand{\HH}{\mathbb{H}}
\newcommand{\Herm}[1]{{\HH_{#1}}}
\newcommand{\F}{\mathbb{F}}
\newcommand{\Pow}{\mathbb{P}}
\newcommand{\mD}{\mathcal{D}}
\newcommand{\OPT}{\text{OPT}}

\newcommand{\qbit}[1]{|{#1}\rangle}
\newcommand{\qadjoint}[1]{\langle{#1}|}
\newcommand{\qproj}[1]{\qbit{#1}\qadjoint{#1}}
\newcommand{\qoutprod}[2]{\qbit{#1}\qadjoint{#2}}

\newcommand{\qdotprod}[2]{\langle#1|#2\rangle}
\newcommand{\hdotprod}[2]{\left\langle#1,#2\right\rangle}
\newcommand{\matdotprod}[3]{\langle#1|#2|#3\rangle}
\newcommand{\supop}[2]{\mathcal{N}_{{#1}\rightarrow{#2} }}
\newcommand{\bA}{\mathbf{A}}
\newcommand{\bB}{\mathbf{B}}
\newcommand{\bD}{\mathbf{D}}
\newcommand{\bC}{\mathbf{C}}
\newcommand{\bG}{\mathbf{G}}
\newcommand{\bH}{\mathbf{H}}
\newcommand{\bM}{\mathbf{M}}
\newcommand{\bS}{\mathbf{S}}
\newcommand{\bT}{\mathbf{T}}
\newcommand{\bU}{\mathbf{U}}
\newcommand{\bV}{\mathbf{V}}
\newcommand{\bW}{\mathbf{W}}
\newcommand{\bX}{\mathbf{X}}
\newcommand{\bY}{\mathbf{Y}}
\newcommand{\EE}{\mathbb{E}}
\newcommand{\Var}{\text{Var}}
\newcommand{\eye}{\mathbb{I}}
\newcommand{\Real}{\text{Re}}
\newcommand{\Img}{\text{Im}}
\newcommand{\id}{\text{id}}
\newcommand{\img}{\text{i}}

\newcommand{\rk}{{r}}
\newcommand{\VecOp}{\text{vec}}
\newcommand{\vvec}[1]{|#1\rangle\rangle}
\newcommand{\vadj}[1]{\langle\langle#1|}
\newcommand{\vvdotprod}[2]{\left\langle\left\langle#1|#2\right\rangle\right\rangle}

\newcommand{\bv}{\mathbf{v}}
\newcommand{\bx}{\mathbf{x}}
\newcommand{\outset}{{\mathcal{X}}}

\newcommand{\spec}{\text{spec}}
\newcommand{\vsigma}{\vec{\sigma}}
\newcommand{\Luders}{\mathcal{H}}
\newcommand{\avgLuders}{{\overline{\Luders}}}
\newcommand{\Choi}{{\mathcal{C}}}
\newcommand{\avgChoi}{{\overline{\Choi}}}
\newcommand{\hbasis}{{\mathcal{V}}}
\newcommand{\qest}{{\hat{\rho}}}

\DeclareMathOperator{\diam}{diam}
\DeclareMathOperator{\diag}{diag}
\DeclareMathOperator{\iSWAP}{iSWAP}
\DeclareMathOperator{\Span}{span}

% Representation theory
\newcommand{\PiRank}{r}
\newcommand{\Wg}{\text{Wg}}
\newcommand{\U}{\mathbb{U}}
\newcommand{\bi}{\mathbf{i}}
\newcommand{\bj}{\mathbf{j}}
\newcommand{\Sim}{\mathcal{S}}
\newcommand{\Mob}{\text{Mob}}
\newcommand{\Cat}{\text{Cat}}
\newcommand{\Haar}[1]{{\mathcal{U}_{#1}}}
\newcommand{\POVM}{\mathcal{M}}
\newcommand{\permProd}[2]{{\left\langle#1\right\rangle_{#2}}}
\newcommand{\cycle}{\mathcal{C}}
\newcommand{\ObsPOVM}{\mathcal{N}}

\newcommand{\Xset}{\mathcal{I}_X}
\newcommand{\Yset}{\mathcal{I}_Y}
\newcommand{\zest}{\hat{\ptb}}
\usepackage{utopia}
\usepackage{xifthen}
\usepackage{soul}
\usepackage{hyperref}
\usepackage{authblk}
\newcommand*{\SINGLE}{}
\title{Revisiting the Berkeley Admissions data: Statistical Tests for Causal Hypotheses}
% \author{
%  Sourbh Bhadane\\
%  Korteweg-de Vries Institute for Mathematics, University of Amsterdam, The Netherlands\\
%  \tt{s.n.bhadane@uva.nl}\\
%  \and
%  Joris M. Mooij \\
%  Korteweg-de Vries Institute for Mathematics, University of Amsterdam, The Netherlands \\
%  \tt{j.m.mooij@uva.nl}\\
%  \and
%  Onno Zoeter\\
%  Booking.com, The Netherlands \\
%  \tt{onno.zoeter@booking.com}
% }

\author[1]{Sourbh Bhadane} 
\author[1]{Joris M. Mooij}
\author[1]{Philip Boeken}
\author[2]{Onno Zoeter}
\affil[1]{Korteweg-de Vries Institute for Mathematics, University of Amsterdam}
\affil[2]{Booking.com, Amsterdam, The Netherlands}


% \affil{Korteweg-de Vries Institute for Mathematics, University of Amsterdam \and Booking.com, Amsterdam, The Netherlands}
% \address{Korteweg-de Vries Institute for Mathematics, University of Amsterdam \and Booking.com, Amsterdam, The Netherlands}

\begin{document}
\maketitle
\begin{abstract}
Reasoning about fairness through correlation-based notions is rife with pitfalls. The 1973 University of California, Berkeley graduate school admissions case from \citet{BickelHO75} is a classic example of one such pitfall, namely Simpson’s paradox. The discrepancy in admission rates among male and female applicants, in the aggregate data over all departments, vanishes when admission rates per department are examined. We reason about the Berkeley graduate school admissions case through a causal lens. In the process, we introduce a statistical test for causal hypothesis testing based on Pearl's instrumental-variable inequalities \citep{Pearl95}. We compare different causal notions of fairness that are based on graphical, counterfactual and interventional queries on the causal model, and develop statistical tests for these notions that use only observational data. We study the logical relations between notions, and show that while notions may not be equivalent, their corresponding statistical tests coincide for the case at hand. We believe that a thorough case-based causal analysis helps develop a more principled understanding of both causal hypothesis testing and fairness.
\end{abstract}

\section{Introduction}
\label{sec:intro}

\begin{figure*}[tb]
    \centering
    \includegraphics[width=0.848\linewidth]{figs/circuitnn.pdf} 
    \caption{Illustration of differentiable CircuitNN. CircuitNN is designed based on differentiable NAND gates. After DAS is guided by PI and PO pairs of the truth table, CircuitNN can get the precise circuit architecture logic equivalent to the truth table.}
    \label{fig:circuitnn}
\end{figure*}

% 1. Describe the importance of logic synthesis
% 2. Existing Problems
% (a) Neural Architecture Search: Unstable, Predefined Setting, etc.
% (b) Circuit Generation: Probabilistic Model, Logic Equivalence

With the rapid advancement of technology, the scale of integrated circuits (ICs) has expanded exponentially. 
This expansion has introduced significant challenges in chip manufacturing, particularly concerning power and area metrics.
A primary objective in IC design is achieving the same circuit function with fewer transistors, thereby reducing power usage and area occupancy.

Logic synthesis~\cite{hachtel2005logicsynth}, a critical step in electronic design automation (EDA), transforms behavioral-level circuit designs into optimized gate-level circuits, ultimately yielding the final IC layout. 
The primary goal of logic synthesis is to identify the physical implementation with the fewest gates for a given circuit function. 
This task constitutes a challenging NP-hard combinatorial optimization problem. 
Current logic synthesis tools~\cite{brayton2010abc, wolf2013yosys} rely on human-designed heuristics, often leading to sub-optimal outcomes.

Differentiable architecture search (DAS) techniques~\cite{liu2018darts, chu2020darts} offer novel perspectives on addressing challenges in this problem.
Circuit functions can be represented through truth tables, which map binary inputs to their corresponding outputs. 
Truth tables provide a precise representation of input-output relationships, ensuring the design of functionally equivalent circuits.
Inspired by this, researchers~\cite{deepmind2024ai4sys, wang2024tnet} have begun exploring the application of DAS to synthesize circuits directly from truth tables.
Specifically, \citet{deepmind2024ai4sys} proposed CircuitNN, a framework that learns differentiable connection structures with logic gates, enabling the automatic generation of logic circuits from truth tables.
This approach significantly reduces the complexity of traditional circuit generation. 
Building on this, \citet{wang2024tnet} introduced T-Net, a triangle-shaped variant of CircuitNN, incorporating regularization techniques to enhance the efficiency of DAS.

Despite these advancements, several challenges remain. 
The computational complexity of DAS grows quadratically with the number of gates, posing scalability issues.
Although triangle-shaped architecture~\cite{wang2024tnet} partially mitigates this problem, redundancy persists. 
%Additionally, DAS is susceptible to converging to local optima, limiting the ability to search architectures that satisfy the given truth tables~\cite{liu2018darts}. 
%Furthermore, hyperparameters (network depth and layer width) require extensive searches, introducing complexity and prolonging the synthesis process. 
Additionally, DAS is susceptible to converging to local optima~\cite{liu2018darts} and hyperparameters (network depth and layer width) require extensive searches. 
The challenges arise from the vast search space in DAS. 
% Even with predefined settings for CircuitNN, finding a configuration that meets the truth table requires extensive trial and error during the DAS process. 
Intuitively, limiting the search space through predefined parameters (network depth, gates per layer, and connection probabilities) can significantly reduce the complexity.

Recent advances~\cite{openai2023gpt4, abramson2024alphafold3, esser2024sd3, li2024mar} in conditional generative models have demonstrated remarkable performance across language, vision, and graph generation tasks. 
Motivated by these developments, we propose a novel approach to circuit generation that generates preliminary circuit structures to guide DAS in generating refined circuits matching specified truth tables. 
Firstly, we introduce CircuitVQ, a tokenizer with a discrete codebook for circuit tokenization. 
Built upon our Circuit AutoEncoder framework~\cite{hou2022graphmae,li2023maskgae,wu2025mgvga}, CircuitVQ is trained through a circuit reconstruction task. 
Specifically, the CircuitVQ encoder encodes input circuits into discrete tokens using a learnable codebook, while the decoder reconstructs the circuit adjacency matrix based on these tokens.
Subsequently, the CircuitVQ encoder serves as a circuit tokenizer for CircuitAR pretraining, which employs a masked autoregressive modeling paradigm~\cite{chang2022maskgit, li2023mage}. 
In this process, the discrete codes function as supervision signals. 
After training, CircuitAR can generate discrete tokens progressively, which can be decoded into initial circuit structures by the decoder of the CircuitVQ. 
These prior insights can guide DAS in producing refined circuits that match the target truth tables precisely.

Our key contributions can be summarized as follows:
\begin{itemize}
\item We introduce CircuitVQ, a circuit tokenizer that facilitates graph autoregressive modeling for circuit generation, based on our Circuit AutoEncoder framework;
\item Develop CircuitAR, a model trained using masked autoregressive modeling, which generates initial circuit structures conditioned on given truth tables;
\item Propose a refinement framework that integrates differentiable architecture search to produce functionally equivalent circuits guided by target truth tables;
\item Comprehensive experiments demonstrating the scalability and capability emergence of our CircuitAR and the superior performance of the proposed circuit generation approach.
\end{itemize}

% Motivation
% (a) Diffusion (Vision, Graph), Autoregressive (Language, Vision)
% (b) Circuit Generation for Predefined Setting
% (c) Neural Architecture Search for Strict Logic Equivalence

% Contribution
% (a) Circuit Tokenizer (new transformer arch, training strategy)
% (b) CircuitAR (train and gen strategies, post-ar strategy)
% (c) Extensive Evaluation including BitD (Bit Distance) for Scalability

\section{Preliminary}

\paragraph{Notation} Consider a sentence of $T$ tokens $\vx=\{\vx_1,\ldots, \vx_T\}\in\gX$, and let $P$ be the unknown target language distribution, $\tilde P(\vx)$ be the empirical distribution of the training data (which is an approximation of $P$), and $Q$ be the distribution of our model at hand. Since our paper is also closely related to RLHF, we will also use $\pi$ to represent the distributions. In particular, we sometimes write $\pi_\theta$ for a distribution that is parameterized by $\theta$, where $\theta$ is usually the set of trainable parameters of the LLM; we write $\pr$ for a reference distribution that should be clear given the context. The next token prediction loss is minimizing the forward-KL between $P$ and $Q$. 




\section{Berkeley Case: No Latent Confounding}\label{sec:modeling}
 % \begin{figure}[ht]
%      \centering
%      \begin{subfigure}{0.9\linewidth}
%      \centering
%             \begin{tikzpicture}
%             \tikzstyle{vertex}=[circle,fill=none,draw=black,minimum size=17pt,inner sep=0pt]
% \node[vertex] (S) at (0,0) {$S$};
% \node[vertex] (A) at (2,0) {$A$};
% \node[vertex] (D) at (1,1) {$D$};
% \path (S) edge (D);
% \path (D) edge (A);
% \path[red] (S) edge (A);
%             \end{tikzpicture}
%         \caption{Causal graph for $\model \in \modelsunconfedge$ illustrating all possible functional dependencies.}
%         \label{fig:no-cf-edge}
%         \end{subfigure}    \hfill
% %              \begin{subfigure}{0.45\linewidth}
% %              \centering
% %             \begin{tikzpicture}
% %             \tikzstyle{vertex}=[circle,fill=none,draw=black,minimum size=17pt,inner sep=0pt]
% % \node[vertex] (S) at (0,0) {$S$};
% % \node[vertex] (A) at (2,0) {$A$};
% % \node[vertex] (D) at (1,1) {$D$};
% % \path (S) edge (D);
% % \path (D) edge (A);
% % %\path[red] (S) edge (A);
% %             \end{tikzpicture}
% %         \caption{Causal graph for $\model \in \nullgraphunconf$ illustrating all possible functional dependencies.}
% %         \label{fig:no-cf-no-edge}
% %         \end{subfigure}
% \end{figure}

% \begin{figure}[h]
%      \centering
%             \begin{tikzpicture}
%             \tikzstyle{vertex}=[circle,fill=none,draw=black,minimum size=17pt,inner sep=0pt]
% \node[vertex] (S) at (0,0) {$S$};
% \node[vertex] (A) at (2,0) {$A$};
% \node[vertex] (D) at (1,1) {$D$};
% \path (S) edge (D);
% \path (D) edge (A);
% \path[bidirected] (D) edge[bend left=60] (A);
% \path[red] (S) edge (A);
% % \draw[->, line width=0.3mm]  (S)--(D);
% % \draw[->, line width=0.3mm]  (D)--(A);
% % \draw[->, line width=0.3mm]  (S)--(A);
% % \draw[<->, line width=0.3mm]  (D)--(A);
%             \end{tikzpicture}
%         \caption{Causal graph for $\model \in \modelsunconfedge$ illustrating all possible functional dependencies.}
%         \label{fig:cf-no-edge}
% \end{figure}

% \begin{figure}[h]
%      \centering
%             \begin{tikzpicture}
%             \tikzstyle{vertex}=[circle,fill=none,draw=black,minimum size=17pt,inner sep=0pt]
% \node[vertex] (S) at (0,0) {$S$};
% \node[vertex] (A) at (3,-0.5) {$A$};
% \node[vertex] (D) at (1,1) {$D$};
% \node[vertex] (S') at (1,-0.5) {$S'$};
% \path (S) edge (D);
% \path (D) edge (A);
% \path[bidirected] (D) edge[bend left=60] (A);
% \path[red] (S') edge (A);
% %\path (S) edge (S'); 
%  \path (S) edge node[near start, below] {=} (S');
% % \draw[->, line width=0.3mm]  (S)--(D);
% % \draw[->, line width=0.3mm]  (D)--(A);
% % \draw[->, line width=0.3mm]  (S)--(A);
% % \draw[<->, line width=0.3mm]  (D)--(A);
%             \end{tikzpicture}
%         \caption{Causal graph for $\model \in \modelsedge$ illustrating all possible functional dependencies.} 
%         \label{fig:cf-edge}
% \end{figure}


%              \begin{subfigure}{0.45\linewidth}
%              \centering
%             \begin{tikzpicture}
%             \tikzstyle{vertex}=[circle,fill=none,draw=black,minimum size=17pt,inner sep=0pt]
% \node[vertex] (S) at (0,0) {$S$};
% \node[vertex] (A) at (2,0) {$A$};
% \node[vertex] (D) at (1,1) {$D$};
% \path (S) edge (D);
% \path (D) edge (A);
% %\path[red] (S) edge (A);
%             \end{tikzpicture}
%         \caption{Causal graph for $\model \in \nullgraphunconf$ illustrating all possible functional dependencies.}
%         \label{fig:no-cf-no-edge}
%         \end{subfigure}
%\end{figure}

\begin{figure*}[t]
     \centering
     \begin{subfigure}{0.32\linewidth}
     \centering
            \begin{tikzpicture}
            \tikzstyle{vertex}=[circle,fill=none,draw=black,minimum size=17pt,inner sep=0pt]
\node[vertex] (S) at (0,0) {$S$};
\node[vertex] (A) at (2,0) {$A$};
\node[vertex] (D) at (1,1) {$D$};
\path (S) edge (D);
\path (D) edge (A);
\path (S) edge (A);
            \end{tikzpicture}
        \caption{$\model \in \modelsunconfedge$}
        \label{fig:no-cf-edge}
\end{subfigure}
     \begin{subfigure}{0.32\linewidth}
     \centering
            \begin{tikzpicture}
            \tikzstyle{vertex}=[circle,fill=none,draw=black,minimum size=17pt,inner sep=0pt]
\node[vertex] (S) at (0,0) {$S$};
\node[vertex] (A) at (2,0) {$A$};
\node[vertex] (D) at (1,1) {$D$};
%\node[vertex] (S') at (1,-0.5) {$S'$};
\path (S) edge (D);
\path (D) edge (A);
\path[bidirected] (D) edge[bend left=60] (A);
\path (S) edge (A);
%\path (S) edge (S'); 
% \path (S) edge node[near start, below] {=} (S');
% \draw[->, line width=0.3mm]  (S)--(D);
% \draw[->, line width=0.3mm]  (D)--(A);
% \draw[->, line width=0.3mm]  (S)--(A);
% \draw[<->, line width=0.3mm]  (D)--(A);
            \end{tikzpicture}
        \caption{$\model \in \modelsedgerelax$} 
        \label{fig:cf-edge}
        \end{subfigure}
         \begin{subfigure}{0.32\linewidth}
     \centering
            \begin{tikzpicture}
            \tikzstyle{vertex}=[circle,fill=none,draw=black,minimum size=17pt,inner sep=0pt]
\node[vertex] (S) at (0,0) {$S$};
\node[vertex] (A) at (2,0) {$A$};
\node[vertex] (D) at (1,1) {$D$};
%\node[vertex] (S') at (1,-0.5) {$S'$};
\path (S) edge (D);
\path (D) edge (A);
\path[bidirected] (D) edge[bend left=60] (A);
%\path (S) edge (A);
%\path (S) edge (S'); 
% \path (S) edge node[near start, below] {=} (S');
% \draw[->, line width=0.3mm]  (S)--(D);
% \draw[->, line width=0.3mm]  (D)--(A);
% \draw[->, line width=0.3mm]  (S)--(A);
% \draw[<->, line width=0.3mm]  (D)--(A);
            \end{tikzpicture}
        \caption{$\model \in \nullgraph$ and $\model \in \modeliv$} 
        \label{fig:cf-edge-iv}
        \end{subfigure}
        \caption{Causal graphs, $\cg{\model}$, assumed in various model classes.}
\end{figure*}

% \begin{figure}
%      \centering
%             \begin{tikzpicture}
%             \tikzstyle{vertex}=[circle,fill=none,draw=black,minimum size=17pt,inner sep=0pt]
% \node[vertex] (Z) at (0,0) {$Z$};
% \node[vertex] (Y) at (3,0) {$Y$};
% \node[vertex] (X) at (1.5,0) {$X$};
% %\node[vertex] (S') at (1,-0.5) {$S'$};
% \path (Z) edge (X);
% \path (X) edge (Y);
% \path[bidirected] (X) edge[bend left=60] (Y);
% %\path[red] (S') edge (A);
% %\path (S) edge (S'); 
%  %\path (S) edge node[near start, below] {=} (S');
% % \draw[->, line width=0.3mm]  (S)--(D);
% % \draw[->, line width=0.3mm]  (D)--(A);
% % \draw[->, line width=0.3mm]  (S)--(A);
% % \draw[<->, line width=0.3mm]  (D)--(A);
%             \end{tikzpicture}
%         \caption{Causal graph of $M \in \modeliv$} 
%         \label{fig:iv}
%         \end{figure}

Under the semantic framework of SCMs, we first make the same causal modeling assumptions that are commonplace in works that mention the Berkeley admissions case. We compare fairness notions that are tied to these modeling assumptions, with the view that modeling assumptions describe a family of SCMs and fairness notions define a subset of this family. We relate existing general notions of fairness in the literature to this viewpoint. While this is a re-examination of the various existing analyses of the Berkeley admissions case, in the next section, we relax the causal modeling assumptions and consider the more general family of models that allow for confounding between department choice and admissions outcome.

The set of endogenous variables consists of the protected attribute, namely sex of the applicant, $\sex$, the department they applied to, $\dept$, and the decision of the admissions committee, $\outcome$. We assume that $\sex, \outcome$ are binary variables and $\dept$ is a discrete-valued variable taking finite number of values, where $\sex=0,1$ corresponds to male, female applicants, respectively, and $\outcome=0,1$ corresponds to reject and accept, respectively.\footnote{The assumption of binary sex is purely for mathematical simplicity.} Given that, possibly, societal biases nudge applicants to departments at differing rates depending on their sex, we assume that $\sex$ affects $\dept$. Since departments are the primary decision-making units and have different admission rates, we also assume that $\dept$ affects $\outcome$. The question of whether acceptance decisions discriminate against sex centers around the direct causal effect of $\sex$ on $\outcome$, and therefore we allow such an effect in the model. We assume the absence of any confounding between the variables (in addition to the absence of any selection bias). The structural equations are given by
\ifdefined\SINGLE
\begin{align}\label{eq:no-cf-edge}
    \sex &= f_{\sex}(U_{\sex}) \nonumber, \\
    \dept &= f_{\dept}(\sex, U_{\dept}), \\
    \outcome &= f_{\outcome}(\sex,\dept,U_{\outcome}), \nonumber
\end{align}
\else $\sex = f_{\sex}(U_{\sex}), 
    \dept = f_{\dept}(\sex, U_{\dept}),
    \outcome = f_{\outcome}(\sex,\dept,U_{\outcome})$
\fi 
where $U_{\sex},U_{\dept}$ and $U_{\outcome}$ denote independent exogenous random variables. We denote the family of SCMs parameterized by the \ifdefined\SINGLE functions in \eqref{eq:no-cf-edge} \else above functions\fi and the exogenous distribution as $\modelsunconfedge$. 
%Let $\modelsunconfnoedge$ represent the same but where $\sex$ does not affect $\outcome$. 
For $\model \in \modelsunconfedge$, the causal graph $G(\model)$ is a directed acyclic graph (DAG), a subgraph of the one in Figure~\ref{fig:no-cf-edge}. 

\subsection{Fairness Notions}
We define a fairness notion to be a certain condition that is required to be satisfied by a causal model to be deemed fair. These conditions can take the form of observational, interventional, counterfactual or graphical queries on the SCMs in the families of causal models defined by modeling assumptions, in our case, $\modelsunconfedge$. While the criteria for the fairness notions in Section~\ref{sec:modeling} are phrased in terms of queries corresponding to different rungs of the causal ladder, in our case, any condition can only be tested using observational data.

% \ifdefined\SINGLE
% \begin{definition}[Demographic Parity]
% $\model \in \modelsunconfedge$ is fair according to demographic parity if it belongs to the null hypothesis set $\nullobsunconf \triangleq \left \lbrace \model \in \modelsunconfedge : P_{\model}\Paren{\outcome=1 \mid \sex =0} = P_{\model}\Paren{\outcome=1 \mid \sex=1} \right \rbrace $.
% \end{definition}
% \else
% \begin{definition}[Observational Notion of Fairness]
% $\model \in \modelsunconfedge$ is fair according to the observational notion of fairness if it belongs to 
% \begin{align*}
% \nullobsunconf &\triangleq \left \lbrace \model \in \modelsunconfedge : \right.\\
% &\left. P_{\model}\Paren{\outcome=1 \mid \sex =0} = P_{\model}\Paren{\outcome=1 \mid \sex=1} \right \rbrace.
% \end{align*}
% \end{definition}
% \fi

% The observational notion of fairness defined above is termed demographic parity \citep{DworkHPRZ12} in modern-day fairness literature and has implicitly been mentioned in earlier works \citep{HutchinsonMitchell19}. This is the notion of fairness that prompted the investigation of \cite{BickelHO75} into the Berkeley admissions data. We have already mentioned that this fairness notion falls prey to Simpson's paradox, i.e. Simpson's paradox translates into a fairness paradox where aggregated over the departments, the decision-making is unfair whereas for each department separately, it is fair. 

% We now mention another observational notion of fairness that is similar to a conditional version of demographic parity. 

The investigation of Berkeley's admission data was initiated on the observation that the well-known fairness notion of demographic parity $P_{\model}\Paren{\outcome=1 \mid \sex = 0} = P_{\model}\Paren{\outcome=1 \mid \sex = 1}$ did not hold. This fairness notion is based purely on observational data and we have already noted that it falls prey to Simpson's paradox. We now present another observational notion of fairness that can be interpreted as a conditional version of demographic parity.
\ifdefined\SINGLE
\begin{definition}[Observational Notion of Fairness]
$\model \in \modelsunconfedge$ is fair according to the observational notion of fairness if it belongs to the null hypothesis set 
\begin{align*}
\nullobsunconf \triangleq &\left \lbrace \model \in \modelsunconfedge :  \forall \ldept, \lsex, P_{\model}\Paren{\dept = \ldept, \sex = \lsex} >0 \right.\\
&\left. \implies P_{\model}\Paren{\outcome=1 \mid \sex =\lsex, \dept = \ldept} = P_{\model}\Paren{\outcome=1 \mid \dept = \ldept} \right \rbrace.
\end{align*}
% $$\nullobsunconf \triangleq \left \lbrace \model \in \modelsunconfedge : \forall \ldept, \lsex: P_{\model}\Paren{\dept = \ldept, \sex = \lsex} >0,  P_{\model}\Paren{\outcome=1 \mid \sex =\lsex, \dept = \ldept} = P_{\model}\Paren{\outcome=1 \mid \dept = \ldept} \right \rbrace. $$
\end{definition}
\else
\begin{definition}[Observational Notion of Fairness]
$\model \in \modelsunconfedge$ is fair according to the observational notion of fairness if it belongs to 
\begin{align*}
&\nullobsunconf \triangleq \left \lbrace \model \in \modelsunconfedge :  \forall \ldept, \lsex,  P_{\model}\Paren{\dept = \ldept, \sex = \lsex} >0 \right.\\
&\left. \implies P_{\model}\Paren{\outcome=1 \mid \sex =\lsex, \dept = \ldept} \right. \\
&\left.= P_{\model}\Paren{\outcome=1 \mid \dept = \ldept} \right \rbrace.
\end{align*}
\end{definition}
\fi

\citet{BickelHO75} proposed this notion for the Berkeley data. A valid test for this notion is a conditional independence test for $\outcome \indep \sex \mid \dept$. Indeed, the analysis of \citet{BickelHO75} shows that the data contain not enough evidence to reject the null hypothesis that this conditional independence holds, and therefore, concludes fairness. 

% \renewcommand{\thedefinition}{\thesection.\arabic{definition}} 
From the causal graph of $\modelsunconfedge$ in Figure~\ref{fig:no-cf-edge}, a natural subset of fair causal models is those without the edge $\sex \rightarrow \outcome$. 

\begin{definition}[Graphical Notion of Fairness]\label{def:graph_fairness}
     $\model \in \modelsunconfedge$ is fair according to the graphical notion of fairness if it belongs to the null hypothesis set $\nullgraphunconf \triangleq \left \lbrace \model \in \modelsunconfedge : \sex \rightarrow \outcome \notin \cg{\model} \right \rbrace$.
\end{definition}

% $\model \in \nullgraphunconf$ imposes a conditional independence constraint on the observational distribution, namely $\sex \indep \outcome \mid \dept$ by the global Markov property. Therefore, \citet{BickelHO75}'s analysis can be thought of as a valid test for the graphical notion of fairness albeit with an implicit faithfulness assumption.

\citet[Section 4.5.3]{Pearl09} discusses the direct effect in the context of the Berkeley admissions example, where he objects to conditioning for the department and instead proposes intervening on department choice, which corresponds to the controlled direct effect (CDE) \citep{Pearl01} of the `treatment', $\sex$, on the outcome, $\outcome$, for every value of the mediator, i.e., every department choice $\ldept$. 

\ifdefined\SINGLE
\begin{definition}[Interventional Notion of Fairness]
    $\model \in \modelsunconfedge$ is fair according to the interventional notion of fairness if it belongs to 
     \begin{equation*}\label{eq:interfairunconf}
    \nullinterunconf \triangleq \left \lbrace \model \in \modelsunconfedge: \forall \ldept,\lsex, P_{\model}\Paren{\outcome=1\mid\doop{\sex=\lsex},\doop{\dept=\ldept}} = P_{\model}\Paren{\outcome=1\mid\doop{\dept = \ldept}} \right \rbrace.
    \end{equation*}
    \end{definition}
\else
\begin{definition}[Interventional Notion of Fairness]
    $\model \in \modelsunconfedge$ is fair according to the interventional notion of fairness if it belongs to the null hypothesis set 
     \begin{align*}\label{eq:interfairunconf}
    &\nullinterunconf \triangleq \left \lbrace \model \in \modelsunconfedge: \forall \ldept,\lsex \right. \nonumber\\
   & \left. P_{\model}\Paren{\outcome=1\mid\doop{\sex=\lsex},\doop{\dept=\ldept}} \right. \nonumber\\
   &\left. = P_{\model}\Paren{\outcome=1\mid\doop{\dept = \ldept}} \right \rbrace.
    \end{align*}
\end{definition}
\fi
 % Note that, for the Berkeley example, a valid test for the interventional notion of fairness is again the conditional independence test $\outcome \indep \sex \mid \dept$.\footnote{This can be seen by applying the do-calculus and holds under positivity assumptions, which are met by the data at hand.}

Recent analyses of the Berkeley example emphasize counterfactual notions of fairness. In \citet[Section 4.5.4]{Pearl09}, \citet{PearlMackenzie18}, Pearl considers a counterfactual quantity, namely the natural direct effect (NDE) \citep{RobinsG92, Pearl01}  by motivating a hypothetical experiment where ``all female candidates retain their department preferences but change their gender [sex] identification (on the application form) from female to male''. Subsequent causal fairness works \citep{NabiShpitser18, Chiappa19} build on this and propose fairness notions based on known path-specific versions of NDE where the `direct path' from $\sex$ to $\outcome$ is viewed as `unfair' as opposed to the `fair' path $\sex \rightarrow \dept \rightarrow \outcome$. For the Berkeley example, the NDE$(s' \rightarrow s)$ is given by 
\begin{equation*}
P_{\model}\Paren{\outcome^{\doop{\sex = \lsex', \dept = \dept^{\doop{\sex=\lsex}}}}=1} - P_{\model}\Paren{\outcome^{\doop{\sex=s}}=1}
\end{equation*}
for $\lsex \neq \lsex'$. Note that by Pearl's mediation formula \citep{Pearl01}, the above is identified (assuming $\forall \ldept, \lsex, P_{\model}\Paren{\dept = \ldept, \sex = \lsex} > 0$) as 
\ifdefined\SINGLE
\begin{equation*}
    \sum\limits_{\ldept} \left( P_{\model}\Paren{\outcome=1 \mid \dept = \ldept, \sex = \lsex}  - P_{\model}\Paren{\outcome=1 \mid \dept = \ldept, \sex = \lsex'}\right)P_{\model}\Paren{\dept = \ldept \mid \sex= \lsex}.
\end{equation*}
\else
\begin{align*}
    \sum\limits_{\ldept} & \left( P_{\model}\Paren{\outcome=1 \mid \dept = \ldept, \sex = \lsex} \right. \\
    &\left. - P_{\model}\Paren{\outcome=1 \mid \dept = \ldept, \sex = \lsex'}\right)P_{\model}\Paren{\dept = \ldept \mid \sex= \lsex}.
\end{align*}
\fi
This implies that if the observational notion of fairness and positivity hold, the NDE is $0$. However, the converse is not necessarily true. For example, if one department favors male applicants and another favors female applicants, then the NDE could be $0$ while it is not necessary that the observational notion of fairness holds. 

Other counterfactual notions of fairness include those by \citet{KusnerLRS17}. The authors define a counterfactual fairness notion that implies demographic parity (see Section~\ref{app:kusnerctrfdemo} for a proof) for the Berkeley example; we have already seen that this particular fairness notion falls prey to Simpson's paradox. In the appendix, however, they define a path-dependent notion of counterfactual fairness.\footnote{This notion is specifically motivated by the Berkeley example.} In Section~\ref{app:kusnerpathnocf} we show that, in our setting, testing for the path-dependent counterfactual fairness notion is equivalent to testing for the conditional independence $A \indep S \mid D$. We now propose an alternate counterfactual notion of fairness and later compare testing of the same.
% Both the above notions correspond to not having a direct effect of $\sex$ on $\outcome$. Due to space constraints, we show the equivalence in supplementary material.

%that demands the invariance of the distribution of a decision, in a given context, with respect to a forced change in the protected attribute. 

\begin{definition}[Counterfactual Notion of Fairness]\label{def:ctrf-nocf}
$\model \in \modelsunconfedge$ is fair according to the counterfactual notion of fairness if it belongs to the null hypothesis set
\ifdefined\SINGLE
\begin{equation*}\label{eq:nullctrf}
    \nullctrfunconf \triangleq \left \lbrace \model \in \modelsunconfedge: \forall \ldept, \lsex, P_M(A^{\doop{S=s,D=d}} = A^{\doop{D=d}}) = 1 \right \rbrace.
\end{equation*}
\else
\begin{align*}\label{eq:nullctrf}
    \nullctrfunconf &\triangleq \left \lbrace \model \in \modelsunconfedge:\forall \ldept, \lsex,  \right. \nonumber\\
    &\left. P_M(A^{\doop{S=s,D=d}} = A^{\doop{D=d}}) = 1 \right \rbrace
\end{align*}
\fi
\end{definition}
The alternate hypotheses are given by the complement of the null hypotheses w.r.t. $\modelsunconfedge$. Given that the notions are defined on different rungs of the causal hierarchy, it is perhaps not surprising that they are nested accordingly. The assumption of no confounding simplifies the relations as we can prove equivalence of a few notions under positivity. The proof is deferred to Section~\ref{app:nested-nocf}.
\begin{restatable}{lemma}{unconfnested}\label{lem:notion_equiv}
\begin{equation*}\nullgraphunconf = \nullctrfunconf \subset \nullinterunconf \subset \nullobsunconf.\end{equation*}
If for all $s,d$, $P_{\model}(s,d) > 0$, then in addition, we have $\nullinterunconf = \nullobsunconf$.
\end{restatable}

Despite the nested nature of the fairness notions at different rungs of the causal hierarchy, we prove that the sets of observational distributions that these notions induce are identical. The proof is in Section~\ref{app:equiv-nocf}.
\begin{restatable}{theorem}{unconfequiv}\label{thm:unconf_test_equiv}
Let
% Let $P_{\model}\Paren{s,d} > 0$ for all $s,d$, and
\begin{align*} 
\distgraphunconf &\triangleq \left \lbrace P_{\model}\Paren{\dept,\outcome, \sex} : \model \in \nullgraphunconf \right \rbrace, \\
\distctrfunconf &\triangleq \left \lbrace P_{\model}\Paren{\dept,\outcome, \sex} : \model \in \nullctrfunconf \right \rbrace, \\
\distinterunconf &\triangleq \left \lbrace P_{\model}\Paren{\dept,\outcome, \sex} : \model \in \nullinterunconf \right \rbrace, \\
\distobsunconf &\triangleq \left \lbrace P_{\model}\Paren{\dept,\outcome, \sex} : \model \in \nullobsunconf \right \rbrace.
\end{align*}
Then $\distgraphunconf = \distctrfunconf = \distinterunconf = \distobsunconf.$
\end{restatable}
% \ifdefined\SINGLE
% \begin{equation}\label{eq:nullctrf}
%     \nullctrfunconf \triangleq \left \lbrace \model \in \modelsunconfedge: \forall \ldept, \lsex, g_{\outcome}\Paren{\ldept, \lsex, X_{\exrv}} = h_{\outcome}\Paren{\ldept,X_{\exrv}} \hspace{5mm}\text{a.s.} \right \rbrace,
% \end{equation}
% \else
% \begin{align}\label{eq:nullctrf}
%     \nullctrfunconf &\triangleq \left \lbrace \model \in \modelsunconfedge: \right. \nonumber\\
%     &\left. \forall \ldept, \lsex, g_{\outcome}\Paren{\ldept, \lsex, X_{\exrv}} = h_{\outcome}\Paren{\ldept,X_{\exrv}} \hspace{2mm}\text{a.s.} \right \rbrace,
% \end{align}
% \fi
% where $g_{\outcome}, h_{\outcome}$ are solution functions of $\model_{\doop{\dept,\sex}}$ and $\model_{\doop{\dept}}$ w.r.t. $\outcome$. 

% \todo{Given that the other counterfactual notions are formulated using potential outcome notation, we can also do that here for notational consistency.
% }
% In potential-outcome notation, the constraint on $M$ can also be written as:
% $$P_M(A^{\doop{S=s,D=d}} = A^{\doop{D=d}}) = 1$$
% \end{definition}

% Note that $\distobsunconf = \left \lbrace P(\outcome,\dept,\sex): \outcome \indep \sex \mid \dept \right \rbrace$. 
In summary, despite the fact that we analyze the Berkeley admissions case using multiple fairness notions, under the assumption of no confounding, with observational data, they can all be tested using a conditional independence test. 

If the data contains enough evidence to reject conditional independence, then the data generating mechanism is unfair w.r.t.\ the observational notion of fairness. On the other hand, if the data does not contain enough evidence to reject conditional independence, then the data generating mechanism is fair w.r.t.\ the observational notion of fairness. However, this extrapolation of the outcome of the statistical test on the fairness implications does not hold for the interventional, counterfactual and graphical notions. The following example illustrates that for the graphical notion of fairness, an unfaithful causal model, where $\outcome$ is directly affected by $\sex$, could satisfy conditional independence. 
\begin{example}\label{ex:unconfexample}
   Let $\model \in \modelsunconfedge$ be defined as $U_{\sex} \sim \text{Ber}(\frac{1}{2}), U_{\outcome} \sim \text{Ber}(\frac{1}{2}), U_{\dept} \sim \text{Ber}(\varepsilon)$ where $\varepsilon \in \left[0,\frac{1}{2}\right)$ and $\sex = U_{\sex}, \dept=\sex \oplus U_{\dept}, \outcome = \sex \oplus \dept \oplus U_{\outcome}$. Here, $\outcome \indep \sex \mid \dept$ but $\sex$ is a parent of $\outcome$, i.e., $\model \in \nullobsunconf$, but $\model \notin \nullgraphunconf$. 
\end{example}
For the interventional notion of fairness, the following example illustrates that a causal model that violates positivity could satisfy conditional independence but not the interventional notion of fairness.
\begin{example}\label{ex:posexample}
    Let $\model \in \modelsunconfedge$ be defined as $U_{\sex}=0, U_{\dept}=0,U_{\outcome} \sim \text{Ber}\Paren{\varepsilon}$ where $\varepsilon \in [0,\frac{1}{2})$, and $\sex=0,\dept=0,\outcome=\sex \oplus U_A$. Here $\outcome \indep \sex \mid \dept$, but 
 for all $d$, $P_{\model}\Paren{\outcome = 1 \mid \doop{\sex=1}, \doop{D=d}} = 1-\varepsilon \neq P_{\model}\Paren{\outcome = 1 \mid \doop{D=d}} = \varepsilon$. Therefore, $\model \in \nullobsunconf$, but $\model \notin \nullinterunconf$.
\end{example}

So, if the outcome of the test is that conditional independence cannot be rejected ($\model \in \nullobsunconf$), then due to the aforementioned observations, we cannot conclude that the underlying causal model belongs to the causal null hypothesis of the interventional or counterfactual or graphical fairness notions, i.e., our conclusion is that fairness is ``undecidable''. However, if the outcome of the statistical test is that there is enough evidence in the data to reject conditional independence($\model \notin \nullobsunconf$), then we can conclude that the underlying causal model does not belong to the causal null hypothesis of \textit{any} of the fairness notions, i.e., there is unfairness. 

In the next section, we enlarge the class of models to allow for confounding between $\dept$ and $\outcome$ and perform a similar reasoning exercise. 

%by motivating a hypothetical experiment where "all female candidates retain their department preferences but change their gender (sex) identification (on the application form) from female to male. 

%  $\model \in \modelsedge$ is fair according to the interventional notion of fairness if it belongs to    \begin{equation}\label{eq:interfair}
% \nullinter \triangleq \left \lbrace \model \in \modelsedge : \forall \ldept, \Pr\Paren{\outcome=1|\doop{\formsex=1},\doop{\dept=\ldept}} = \Pr\Paren{\outcome=1|\doop{\formsex=0},\doop{\dept = \ldept}} \right \rbrace.
% \end{equation}

%===============================================================
%LATER SECTION - WITH CONFOUNDING

% \paragraph{With Confounding:} Allowing for confounding between the department choice and admissions outcome affects how we test the fairness notions that we define in Section~\ref{sec:notions}. In particular, Kruskal \cite[Pg 128-129]{FairleyMosteller77} demonstrated an example where the existence of a confounder, such as state of residence, can render the result of the analysis of \cite{BickelHO75} incorrect. Other natural confounders include, for example, level of department-specific technical skills (for example, maths skills) that influence both, the department choice of an applicant and the admissions outcome. We combine all such possible confounders into a single exogenous random variable, $U$.

% Inspired by \cite{Pearl09} \todo{I think this trick has been proposed by others before Pearl, e.g. Geneletti, S. and A. P. Dawid (2007). Defining and identifying the effect of
% treatment on the treated, and Robins, J. M., T. J. VanderWeele, and T. S. Richardson (2007). Discussion
% of “Causal effects in the presence of non compliance a latent variable
% interpretation” by Forcina, A. Metron LXIV (3), 288–298. By the way, where does Pearl 2009 do this?}, we introduce $\formsex$, interpreted as the reported sex of the applicant on the form. In the observational data, we assume that $\formsex$ is just a copy of ``birth" sex, $\sex$. However, the difference lies in functional dependencies on $\dept$ and $\outcome$;  birth sex affects department choice whereas the reported sex affects admissions outcome. This ``node-splitting" operation helps us define testable fairness notions as we shall see in subsequent sections. Note that, while \cite{Pearl09} mentions the distinction between reported sex and birth sex, the analysis does not treat them as different.\todo{Couldn't find this... is it in paragraph 4.5?} The structural equations are given by 

% % \todo{Maybe mention that while Pearl mentions this in words, he doesn't mention use this node-splitting operation in the analysis.} 


% \begin{align}\label{eq:SCMfunctions}
%         \sex &= f_{\sex}(U_{\sex}), \nonumber\\
%         \formsex &= f_{\formsex}(\sex) = \sex, \nonumber \\
%         \dept &= f_{\dept}(\sex,U,U_{\dept}), \nonumber \\
%         \outcome &= f_{\outcome}(\formsex,\dept,U,U_{\outcome}),
% \end{align}
% where $U$ denotes the confounder between $\dept$ and $\outcome$. Note that we make no assumptions about the dimension and nature of the confounder. The exogenous distribution is a product distribution over all the exogenous variables, namely $\left \lbrace U_{\sex}, U_{\dept}, U_{\outcome}, U \right \rbrace$. We denote the family of models parameterized by the functional dependencies and the exogenous distributions as described above as $\modelsedge$. We also define $\modelsnoedge$ to be the family of models parameterized by the exogenous distribution and the functional dependencies listed above with the modification in the structural equation of $\outcome$ where $\outcome = f_{\outcome}\Paren{\dept,U,U_{\outcome}}$.

% While there might be other variables in the system that are observed, we assume that the resulting SCM obtained by marginalizing all variables except $\sex, \formsex, \dept, \outcome$ is given as above. Despite the fact that the data obtained from the Berkeley Graduate School admissions is in the form of a finite dataset, we will often assume that we have access to the observational distribution of $\sex, \dept, \outcome$ and denote it by $\obsdis$. Specifically, we assume that we can draw samples from $\obsdis$ which we call observational data. We address statistical tests with finite data in Section~\ref{sec:tests}.

%=============================================================

%PREVIOUS VERSION: CAUSAL MODELING SECTION 

% As a warm-up, we first make the same causal modeling assumptions that are commonplace in works that mention the Berkeley admissions case. We then consider the more general family of models that allow for confounding between the department choice and the admissions outcome. For both cases, we operate under the semantic framework of structural causal models (SCM).
% \paragraph{Without Confounding:} The set of endogenous variables include the protected attribute, namely sex of the applicant, $\sex$, the department they applied to, $\dept$ and the decision of the admissions committee, $\outcome$. We assume that $\sex, \outcome$ are binary variables and $\dept$ is a discrete-valued variable taking finite number of values, where for women applicants, $\sex=1$ and acceptance decisions are denoted by $\outcome=1$.\todo{it is more precise to say $\sex=0,1$ corresponds with male,female, respectively, and $\outcome=0,1$ corresponds with reject/accept, respectively; to remain sufficiently inclusive perhaps it's good to add a comment that for mathematical simplicity we will assume sex is binary} Given the evidence  \cite{BickelHO75} that societal biases nudge applicants to departments at differing rates depending on their sex, we assume a functional dependency of $\sex$ on $\dept$ \todo{I think you mean vice versa: ``$\dept$ on $\sex$''. `Functional dependency' is SCM-specific terminology, you could also say `$S$ affects $D$'}. Since departments are the primary decision-making units and have differing admission rates, we also assume a functional dependency of $\dept$ on $\outcome$ \todo{$\outcome$ on $\dept$. Etcetera in the following}. Since models that don't have a functional dependency of $\sex$ on $\outcome$ are subsumed by models that do, we consider models with the functional dependency. The structural equations are given by
% \begin{align}\label{eq:no-cf-edge}
%     \sex &= f_{\sex}(U_{\sex}), \\
%     \dept &= f_{\dept}(\sex, U_{\dept}), \\
%     \outcome &= f_{\outcome}(\sex,\dept,U_{\outcome}),
% \end{align}
% where $U_{\sex},U_{\dept}$ and $U_{\outcome}$ denote independent exogenous random variables. We denote the family of models parameterized by the functional dependencies given by \eqref{eq:no-cf-edge} and the exogenous distribution as $\modelsunconfedge$. Let $\modelsunconfnoedge$ represent the same but where $\outcome$ does not have a functional dependency on $\sex$. For a SCM $\model \in \modelsunconfedge$, the causal graph $G(\model)$ is a directed acyclic graph (DAG) as shown in Figure~\ref{fig:no-cf-edge}. 

% \paragraph{With Confounding:} Allowing for confounding between the department choice and admissions outcome affects how we test the fairness notions that we define in Section~\ref{sec:notions}. In particular, Kruskal \cite[Pg 128-129]{FairleyMosteller77} demonstrated an example where the existence of a confounder, such as state of residence, can render the result of the analysis of \cite{BickelHO75} incorrect. Other natural confounders include, for example, level of department-specific technical skills (for example, maths skills) that influence both, the department choice of an applicant and the admissions outcome. We combine all such possible confounders into a single exogenous random variable, $U$.

% Inspired by \cite{Pearl09} \todo{I think this trick has been proposed by others before Pearl, e.g. Geneletti, S. and A. P. Dawid (2007). Defining and identifying the effect of
% treatment on the treated, and Robins, J. M., T. J. VanderWeele, and T. S. Richardson (2007). Discussion
% of “Causal effects in the presence of non compliance a latent variable
% interpretation” by Forcina, A. Metron LXIV (3), 288–298.}. By the way, where does Pearl 2009 do this?}, we introduce $\formsex$, interpreted as the reported sex of the applicant on the form. In the observational data, we assume that $\formsex$ is just a copy of ``birth" sex, $\sex$. However, the difference lies in functional dependencies on $\dept$ and $\outcome$;  birth sex affects department choice whereas the reported sex affects admissions outcome. This ``node-splitting" operation helps us define testable fairness notions as we shall see in subsequent sections. Note that, while \cite{Pearl09} mentions the distinction between reported sex and birth sex, the analysis does not treat them as different.\todo{Couldn't find this... is it in paragraph 4.5?} The structural equations are given by 

% % \todo{Maybe mention that while Pearl mentions this in words, he doesn't mention use this node-splitting operation in the analysis.} 


% \begin{align}\label{eq:SCMfunctions}
%         \sex &= f_{\sex}(U_{\sex}), \nonumber\\
%         \formsex &= f_{\formsex}(\sex) = \sex, \nonumber \\
%         \dept &= f_{\dept}(\sex,U,U_{\dept}), \nonumber \\
%         \outcome &= f_{\outcome}(\formsex,\dept,U,U_{\outcome}),
% \end{align}
% where $U$ denotes the confounder between $\dept$ and $\outcome$. Note that we make no assumptions about the dimension and nature of the confounder. The exogenous distribution is a product distribution over all the exogenous variables, namely $\left \lbrace U_{\sex}, U_{\dept}, U_{\outcome}, U \right \rbrace$. We denote the family of models parameterized by the functional dependencies and the exogenous distributions as described above as $\modelsedge$. We also define $\modelsnoedge$ to be the family of models parameterized by the exogenous distribution and the functional dependencies listed above with the modification in the structural equation of $\outcome$ where $\outcome = f_{\outcome}\Paren{\dept,U,U_{\outcome}}$.

% While there might be other variables in the system that are observed, we assume that the resulting SCM obtained by marginalizing all variables except $\sex, \formsex, \dept, \outcome$ is given as above. Despite the fact that the data obtained from the Berkeley Graduate School admissions is in the form of a finite dataset, we will often assume that we have access to the observational distribution of $\sex, \dept, \outcome$ and denote it by $\obsdis$. Specifically, we assume that we can draw samples from $\obsdis$ which we call observational data. We address statistical tests with finite data in Section~\ref{sec:tests}.


%==============================================================
%OLD VERSION:   

% \subsection{Selection Bias}
% The Berkeley dataset released in \cite{??} \todo{Cite the R package UCBAdmissions} contains data about the top $6$ departments. However, in \cite{BickelHO75}, there is data about $101$ departments. This is an example of selection bias and not being cognizant about this in our analysis leads to biased results. \todo{What are other examples of selection bias?} Other forms of selection bias on sex could exist, for example, through existence of specific prior preparatory programs that affect the chances of choosing a department. In our work, we consider selection bias to be an important aspect and evaluate our fairness notions and the associated tests based on whether they are robust to selection bias on the department and sex. 

% We handle robustness to selection bias in the type of observational data that we assume access to where the different types are differentiated by sets of variables that we condition on. For example, a test that only assumes access to the conditional distribution of $\outcome$ given $\dept, \sex$ is robust to selection bias on $\dept$ and $\sex$. We shall see in Section~\ref{sec:bounds} how the fairness notions defined in Section~\ref{sec:notions} change with change in these types of observational data. 





% \section{Fairness Notions}\label{sec:notions}\todo{not mentioned counterfactual notions. Redo this section after tying in the counterfactual fairness and path-dependent counterfactual fairness notions from Kusner et al's paper.}
% \todo{Readability suffers from that you're telling two stories at once, the one with-cf and the one without-cf. Didactically it seems better to first tell the no-cf story (esentially reproducing Bickel et al's story) and then the with-cf story.}
%We define a fairness notion to be a certain condition that is required to be satisfied by a causal model to be deemed fair. These conditions can take the form of observational, interventional, counterfactual or graphical queries on the SCMs in the families of causal models defined by modeling assumptions in Section~\ref{sec:modeling}. 


% \todo{Should we explicitly also discuss ``observational'' fairness notions like demographic parity? Most of the fairness literature looks at those notions... Perhaps all readers will know that the notion of demographic parity may fall prone to Simpson's paradox, but perhaps we need to add that discussion for completeness? Anyhow, this is something that can be added later on, let's focus on the causal fairness notions first.}
% Given the SCM in the previous section, there are multiple fairness notions that we could consider. In this section, we motivate a few fairness notions to consider. \todo{Is this the rationale for us to consider all the notions that we do? - Natural?} We leave the issue of coming up with statistical tests to test each of the notions for Section~\ref{sec:tests}.
\section{Berkeley Case: With Latent Confounding Between Department And Outcome}\label{sec:confounder}
\begin{figure}[!th]
    \centering
    \begin{tikzpicture}
\tikzstyle{vertex}=[circle,fill=none,draw=black,minimum size=17pt,inner sep=0pt]
\node[vertex] (S) at (0,0) {$S$};
\node[vertex] (A) at (2,0) {$A$};
\node[vertex] (D) at (1,1) {$D$};
%\node[vertex] (S') at (1,-0.5) {$S'$};
\path (S) edge[bend left=10] (D);
\path (D) edge[bend left=10] (S);
\path (D) edge[bend left=10] (A);
\path (A) edge[bend left=10] (D);
\path[bidirected] (D) edge[bend left=60] (A);
\path[bidirected] (S) edge[bend left=60] (D);
\path[bidirected] (S) edge[bend right=60] (A);
\path (S) edge[bend left=10] (A);
\path (A) edge[bend left=10] (S);
    \end{tikzpicture}
    \caption{Causal graph of a model without assumptions}
    \label{fig:causal_modeling}
\end{figure}

We now take a more careful causal modeling approach. Instead of starting from variables and reasoning about structural equations that we allow, we start with assuming that all structural equations exist.\footnote{Since this allows for causal cycles, this would require using the framework of simple SCMs \citep{BongersFPM21}.} For the Berkeley example, Figure~\ref{fig:causal_modeling} shows a causal graph of an SCM that we start with. \ifdefined\SINGLE We now provide rationale for ruling out few structural equations. Based on chronology of events, we rule out those where $D$ directly affects $S$, where $A$ directly affects $D$ and where $A$ directly affects $S$. \else Based on chronology of events, we rule out structural equations where $D$ directly affects $S$, where $A$ directly affects $D$ and where $A$ directly affects $S$.\fi We rule out unobserved common causes of $S$ and $D$, and $S$ and $A$ since we model $S$ to be sex at birth. While latent selection bias might introduce bidirected edges \citep{ChenZM24} that are incident on $S$, we assume for now that there is no selection bias in the dataset. The resulting class of SCMs has structural equations of the form
\ifdefined\SINGLE
\begin{align}\label{eq:cf-edge}
    \sex &= f_{\sex}(U_{\sex}) \nonumber, \\
    \dept &= f_{\dept}(\sex,U, U_{\dept}), \\
    \outcome &= f_{\outcome}(\sex,\dept,U,U_{\outcome}), \nonumber
\end{align}
\else 
$\sex = f_{\sex}(U_{\sex}), \dept = f_{\dept}(\sex,U, U_{\dept}), \outcome = f_{\outcome}(\sex,\dept,U,U_{\outcome})$
\fi 
where $U,U_{\sex},U_{\dept}$ and $U_{\outcome}$ denote independent exogenous random variables. 
We define $\modelsedgerelax$ to be the family of models parameterized by the above structural equations and the exogenous distribution. Further, we define $\modelsedge = \left\{ \model \in \modelsedgerelax : \forall s, P_{\model}(S=s) > 0 \right\}$. For $\model \in \modelsedgerelax$ (and $\modelsedge$), the causal graph is a subgraph of the one shown in Figure~\ref{fig:cf-edge}. 

Although we arrived at allowing confounding between department and outcome through a careful causal modeling approach, this is not a novel consideration. In particular, Kruskal \citep[Pg 128-129]{FairleyMosteller77} demonstrated an example where the existence of a latent confounder, such as state of residence, can render \citet{BickelHO75}'s analysis incorrect. Other natural latent confounders include, for example, level of department-specific technical skills that influence both the department choice of an applicant and the admissions outcome. 

%We denote all such latent confounders by a single exogenous random variable, $U$ which is independent of other exogenous random variables. We modify the causal mechanisms of $\dept, \outcome$ in \eqref{eq:no-cf-edge} to include $U$ as input.

%In Section~\ref{app:modeling}, we show how we end up with $\modelsedge$ using a more careful causal modeling approach. 

% From the causal graphs of $\modelsunconfedge$ and $
% \modelsedge$ in Figures~\ref{fig:no-cf-edge} and \ref{fig:cf-edge}, a natural subset of fair causal models are those where the red edge does not exist in the causal graph.\todo{It is important to mention that this coincides with the notion that Pearl proposes in his 2009 book (no direct effect of $\sex$ on $\outcome$ w.r.t.\ $\{\sex,\dept,\outcome\}$).}

% Our first notion of fairness stems from the natural question, "Did the admissions committee consider $\formsex$ as a deciding factor while deciding the outcome?". In the SCM framework, this can be formulated as whether the structural equation for $\outcome$ includes $\formsex$ or equivalently, does the edge $\formsex \rightarrow \outcome$ exist in the causal graph of $\model$, $\cG(\model)$? 

% \noindent\textbf{Graphical Notion}: If $\sex \rightarrow \outcome \notin \cG(\model)$, then $M$ is fair. 

% Denote the class of models that are fair according to the 

% \begin{definition}[Graphical Notion of Fairness]\label{def:graph_fairness}
%      $\model \in \modelsunconfedge$ is fair according to the graphical notion of fairness if it belongs to $\nullgraphunconf \triangleq \left \lbrace \model \in \modelsunconfedge : \sex \rightarrow \outcome \notin \cg{\model} \right \rbrace$. $\model \in \modelsedge$ is fair according to the graphical notion of fairness if it belongs to $\nullgraph \triangleq \left \lbrace \model \in \modelsedge : \formsex \rightarrow \outcome \notin \cg{\model} \right \rbrace$.
% \end{definition}
% \todo{I got a bit lost in redundancies in the notation. $H^0_{no-cf-graph} = \model_{no-cf-no-edge}$ and $H^0_{graph}=\model_{no-edge}$ that were already introduced before.}

Since our modeling assumptions expand the family of SCMs under consideration to $\modelsedge$, the fairness notions that we discussed in the previous section are modified accordingly to obtain null hypothesis sets $ \nullgraph, \nullinter$ and $\nullctrf$. 
\ifdefined\SINGLE
\begin{definition}[Fairness Notions with Confounding]\label{def:notions-cf}
For $\model \in \modelsedge$ the null hypothesis set corresponding to the interventional, counterfactual and graphical notion of fairness are 
\begin{align*}\label{eq:cf-def}
    \nullinter &\triangleq \left \lbrace \model \in \modelsedge: \forall \ldept,\lsex, P_{\model}\Paren{\outcome=1\mid\doop{\sex=\lsex},\doop{\dept=\ldept}} = P_{\model}\Paren{\outcome=1\mid\doop{\dept = \ldept}} \right \rbrace, \\
     \nullctrf &\triangleq \left \lbrace \model \in \modelsedge: \forall \ldept, \lsex, P_M(A^{\doop{S=s,D=d}} = A^{\doop{D=d}}) = 1 \right \rbrace, \\
     \nullgraph&\triangleq \left \lbrace \model \in \modelsedge : \sex \rightarrow \outcome \notin \cg{\model} \right \rbrace.
\end{align*}
\end{definition}
\else
\begin{definition}[Fairness Notions with Confounding]\label{def:notions-cf}
For $\model \in \modelsedge$ the null hypothesis set corresponding to the interventional, counterfactual and graphical notion of fairness are 
\begin{align*}\label{eq:cf-def}
    \nullinter &\triangleq \left \lbrace \model \in \modelsedge: \forall \ldept,\lsex, \right.\\
    &\left. P_{\model}\Paren{\outcome=1\mid\doop{\sex=\lsex},\doop{\dept=\ldept}} \right.\\
    &\left. = P_{\model}\Paren{\outcome=1\mid\doop{\dept = \ldept}} \right \rbrace, \\
     \nullctrf &\triangleq \left \lbrace \model \in \modelsedge: \forall \ldept, \lsex, \right.\\
     &\left. P_M(A^{\doop{S=s,D=d}} = A^{\doop{D=d}}) = 1 \right \rbrace, \\
     \nullgraph &\triangleq \left \lbrace \model \in \modelsedge : \sex \rightarrow \outcome \notin \cg{\model} \right \rbrace.
\end{align*}
\end{definition}
\fi
% \nullobs &\triangleq \left \lbrace \model \in \modelsedge : \forall \ldept, P_{\model}\Paren{\outcome=1 \mid \sex =0, \dept = \ldept} = P_{\model}\Paren{\outcome=1 \mid \sex=1, \dept = \ldept} \right \rbrace, \\
% \nullobs &\triangleq \left \lbrace \model \in \modelsedge : \forall \ldept, P_{\model}\Paren{\outcome=1 \mid \sex =0, \dept = \ldept} \right. \\
%     &\left. = P_{\model}\Paren{\outcome=1 \mid \sex=1,\dept = \ldept} \right \rbrace, \\
While the above notions generalize straightforwardly from the no-confounder setting, this is no longer the case for the observational notion. In addition, while the statistical tests for the no-confounder model are straightforward, this is no longer the case for the aforementioned null hypotheses since $\outcome \not\!\perp\!\!\!\perp \sex \mid \dept$ in general. We first consider the graphical notion of fairness and develop a corresponding statistical test.

\subsection{Graphical Notion and the Instrumental Variable (IV) Inequalities}\label{subsec:graph_iv}

In the presence of latent confounding, graphical queries, such as absence of edges, impose equality or inequality constraints \citep{Evans16, WolfeSF19} in addition to conditional independence constraints which are the only constraints imposed by a DAG. For the Berkeley case with confounding, since the path $\sex \rightarrow \dept \leftrightarrow \outcome$ is open when conditioned on $\dept$, we have $\sex \not\!\perp\!\!\!\perp \outcome \mid \dept$ in general. 
%For example, for $\model \in \nullgraphunconf$, by the global Markov property, the graph implies the conditional independence $\sex \indep \outcome \mid \dept$ for any distribution that is Markov with respect to $\model$ \todo{in particular, for the observational distribution of $\model$ (the other distributions are not relevant to consider here)}.
%However, this no longer holds for the case with confounding since the path $\sex \rightarrow \dept \leftrightarrow \outcome$ is open when conditioned on $\dept$. 
Our test for the graphical notion of fairness for $\modelsedge$ stems from the observation that a model $\model \in \nullgraph$, lies in the instrumental variable (IV) model class $\modeliv$ where $\sex$ is considered the instrument, $\dept$ the treatment, and $\outcome$ the effect. If all modeled endogenous variables are discrete-valued, a necessary condition for the observational distribution\footnote{While we express the IV inequalities as a condition satisfied by the observational distribution, in Section~\ref{app:iv} we reason that they are more appropriately expressed as conditions in terms of $P_{\model}(X,Y \mid \doop{Z})$.} resulting from $\model \in \modeliv$ is to satisfy the IV inequalities \citep{Pearl95}, which in the context of Figure~\ref{fig:iv} are given by  
\begin{equation}\label{eq:iv}
    \max_{x} \sum_{y} \max_{z} P_{\model}\Paren{X=x,Y=y\mid Z=z} \leq 1. 
\end{equation}
% \st{This implies that a violation of the instrumental variable inequalities is evidence of unfairness.} \todo{Not necessarily, violations of the IV model class need not imply that there is an edge from $\sex \rightarrow \outcome$. Can we then not conclude anything from the IV inequality violation? }
Since the IV inequalities are only necessary conditions, an arbitrary distribution on $X,Y,Z$ that satisfies the IV inequality does not necessarily imply that it is an entailed distribution of a model from the IV model class. \citet{Bonet01} showed that for the binary instrument, treatment and effect case, the IV inequalities are also sufficient conditions. In Theorem~\ref{thm:iv_tight}, we show that for the case where the instrument and outcome are binary and the treatment is discrete-valued with finite support, any distribution that satisfies the IV inequality is also entailed by some causal model from the IV model class. To the best of our knowledge, Theorem~\ref{thm:iv_tight} is a novel result. We defer the proof to Section~\ref{app:ivsharp}.
\ifdefined\SINGLE
\begin{restatable}{theorem}{ivtight}\label{thm:iv_tight}
Let $X,Y,Z$ be discrete random variables defined on $\cX,\cY,\cZ$ respectively, with $|\cX| = n\geq 2, |\cY|=2, |\cZ| =2$. Let the set of joint distributions that satisfy the IV inequalities be defined as $\distiv \triangleq \left \lbrace P(X,Y,Z) : P(X,Y \mid Z)\text{ satisfies }\eqref{eq:iv} \text{ and } \forall z, P(Z=z)>0 \right \rbrace$. Define $\distmodeliv \triangleq \left \lbrace P_{\model}(X,Y,Z) : \model \in \modeliv \right \rbrace.$ Then $\distiv = \distmodeliv.$
\end{restatable}
\else
\begin{restatable}{theorem}{ivtight}\label{thm:iv_tight}
Let $X,Y,Z$ be discrete random variables defined on $\cX,\cY,\cZ$ respectively, with $|\cX| = n\geq 2, |\cY|=2, |\cZ| =2$. Define $\distmodeliv \triangleq \left \lbrace P_{\model}(X,Y,Z) : \model \in \modeliv \right \rbrace.$ and the set of joint distributions that satisfy the IV inequalities as \begin{align*}\distiv &\triangleq \left \lbrace P(X,Y,Z) : P(X,Y \mid Z)\text{ satisfies }\eqref{eq:iv} \right.\\
&\left.\text{ and } \forall z, P(Z=z)>0 \right \rbrace.\end{align*}  Then $\distiv = \distmodeliv.$
\end{restatable}
\fi
% We make a few observations that help us understand the implications of Theorem~\ref{thm:iv_tight} for concluding finally concluding about  
For the Berkeley admissions case, assuming for now that the true observational distribution over $\sex,\dept,\outcome$ is known, the observational distribution satisfying the IV inequalities implies that there exists a causal explanation (model) where the directed edge $\sex \rightarrow \outcome$ is absent, i.e., given that $P(\outcome,\dept, \sex) \in \distiv$, there exists $\model \in \modeliv$ such that $P_{\model}(\outcome,\dept,\sex) = P(\outcome,\dept,\sex)$. On the other hand, the observational distribution violating the IV inequalities does not necessarily imply that the edge $\sex \rightarrow \outcome$ is present since the IV model class, $\modeliv$, is only a subset of all the models that do not contain the edge $\sex \rightarrow \outcome$ in the causal graph. For example, the existence of latent confounding between $\sex$ and $\outcome$ in a model $\model$ may result in $\model \notin \modeliv$, even though $\cg{\model}$ does not necessarily contain the directed edge $\sex \rightarrow \outcome$. However, the causal modeling assumption that defined $\modelsedge$ rules out latent confounding between $\sex$ and $\outcome$. Therefore, given our modeling assumptions, $\nullgraph = \modeliv$, and in turn, we conclude that violating the IV inequalities implies that $\model \in \modelsedge \backslash \nullgraph$. As in the previous section, it is possible that causal models that lie outside $\nullgraph$  (``unfair'' models) induce observational distributions that lie in $\distiv$, i.e., satisfy the IV inequalities. Therefore, satisfying the IV inequalities is not conclusive evidence that the data-generating mechanism is fair, i.e., our conclusion should be that fairness is undecidable. In Section~\ref{sec:bayesiantest} we introduce a Bayesian test for the IV inequalities.

\subsection{Bounds on Interventional Notion of Fairness}\label{subsec:bounds}
For $\model \in \modelsedge$, the interventional notion of fairness is the CDE, which is not identifiable in our case.
By a response-function parameterization \citep{Balke95, BalkePearl97} of $\model \in \modelsedge$, we can express the interventional distributions in Definition~\ref{def:notions-cf} as a linear function of response variables. Further, the observational distribution is also expressed as a linear function of the response variables. Using the symbolic linear programming approach of \citet{Balke95}, we obtain upper and lower bounds in terms of the observational distribution, specifically, $P_{\model}\Paren{\outcome,\dept \mid \sex}$. Indeed, \citet{CaiKPT08} express the same bounds which we reproduce below. The CDE given by
\ifdefined\SINGLE
\begin{equation*}P_{\model}\Paren{\outcome=1\mid \doop{\sex=1}, \doop{\dept=\ldept}} - P_{\model}\Paren{\outcome=1\mid \doop{\sex=0}, \doop{\dept=\ldept}},
\end{equation*}
\else
\begin{align*}
&P_{\model}\Paren{\outcome=1\mid \doop{\sex=1}, \doop{\dept=\ldept}}\\
&- P_{\model}\Paren{\outcome=1\mid \doop{\sex=0}, \doop{\dept=\ldept}},
\end{align*}
\fi
lies in the interval
\ifdefined\SINGLE
\begin{align*}
    &\left[ \Pr\left(\outcome=1,\dept=\ldept\mid\sex=1\right) + \Pr\left(\outcome=0,\dept=\ldept\mid \sex=0\right) - 1,\right.\\
    & \left.1 - \Pr\left(\outcome=0,\dept=\ldept\mid\sex=1\right) - \Pr\left(\outcome=1,\dept=\ldept\mid\sex=0\right)\right].
\end{align*}
\else
% \begin{align*}
%     &\left[ \Pr\left(\outcome=1,\dept=\ldept\mid \sex=1\right) \right. \\
%     &\left.+ \Pr\left(\outcome=0,\dept=\ldept\mid\sex=0\right) - 1,\right.\\
%     & \left.1 - \Pr\left(\outcome=0,\dept=\ldept\mid\sex=1\right) \right. \\
%     &\left. -\Pr\left(\outcome=1,\dept=\ldept\mid\sex=0\right)\right].
% \end{align*}
\begin{align*}
    &\left[ \Pr\left(\outcome=1,\ldept\mid \sex=1\right) + \Pr\left(\outcome=0,\ldept\mid\sex=0\right) - 1,\right.\\
    & \left.1 - \Pr\left(\outcome=0,\ldept\mid\sex=1\right) -\Pr\left(\outcome=1,\ldept\mid\sex=0\right)\right].
\end{align*}
\fi 
For the interventional notion of fairness, the CDE must be $0$ for all $\ldept$. By setting the lower bound to be at most $0$ and the upper bound to be at least $0$, we recover the IV inequalities in \eqref{eq:iv}. While \citet{CaiKPT08} do not point out the connection to the IV inequalities, they find it ``remarkable that we [they] get such a simple formula, consisting of only one additive expression in the lower bound and one additive expression in the upper bound". In the next subsection, we show that the connection to the IV inequalities is not a coincidence.   

\subsection{A Family of Equivalent Tests}\label{subsec:equivalence}

% \todo{I'm not sure if this is the best way to present things... you spend 1.5 page on deriving bounds for the interventional H0. But then later on you see that you could have skipped that... Does the reader need to go through the same chronological order of understanding? Or is it better to take a different perspective (that would have saved us work if we would have had it initially)?}

The graphical and interventional fairness notions end up imposing identical constraints on the observational distribution.
%\todo{Is it clear they are actually different? Perhaps people would believe them to be equivalent!}
However, note that $\nullinter \supseteq \nullgraph$. In fact, we prove in Section~\ref{app:nested-cf} that $ \nullgraph = \nullctrf \subset \nullinter$. Models in $\nullinter \backslash \nullgraph$ (Example~\ref{ex:unconfexample}) are such that the edge $\sex \rightarrow \outcome$ exists in the causal graph and yet, the interventional queries in Definition~\ref{def:notions-cf} are equal.
% Therefore, models in $\nullinter \backslash \nullgraph$ violate certain faithfulness assumptions. 
Given that the null hypothesis of the interventional fairness notion is a strict superset of that of the graphical fairness notion, we might expect the same relation to hold in the resulting set of observational distributions for these hypotheses, thus giving us potentially different tests. In contrast, like in Section~\ref{sec:modeling}, we show that the corresponding sets of observational distributions resulting from models in $\nullinter$, $\nullctrf$, $\nullgraph$ are identical. Section~\ref{app:equiv-cf} contains the proof.

\begin{restatable}{theorem}{confequiv}\label{thm:equivalence}
Let 
\begin{align*}
\distgraph &\triangleq \left \lbrace P_{\model}\Paren{\dept,\outcome, \sex} : \model \in \nullgraph \right \rbrace, \\
\distinter &\triangleq \left \lbrace P_{\model}\Paren{\dept,\outcome, \sex} : \model \in \nullinter \right \rbrace, \\
\distctrf &\triangleq \left \lbrace P_{\model}\Paren{\dept,\outcome, \sex} : \model \in \nullctrf \right \rbrace.
\end{align*}
Then $\distinter = \distctrf = \distgraph = \distiv,$ where $\distiv$ is defined in Theorem~\ref{thm:iv_tight}.
\end{restatable}

In summary, testing for the graphical, interventional and counterfactual notions of fairness, with confounding, all boil down to testing the IV inequalities.

\subsection{Comparison With Existing Fairness Notions}

The utility of considering statistical tests is that we can now compare different fairness notions for a particular case with respect to the same causal modeling assumptions. In this section, we consider the three existing counterfactual fairness notions, namely the NDE \citep{NabiShpitser18, Chiappa19}, and the counterfactual and path-dependent counterfactual fairness notions in \citet{KusnerLRS17}. 

For NDE, \citet{KaufmanKMGP05} obtain bounds for the all-binary setting. Using these bounds, we obtain a strictly weaker test than the IV inequalities.\footnote{Intuitively, the reason is the same as in Section 3; the NDE averages over departments, and a positive bias in one department may cancel out against a similarly strong negative bias in another department. Hence, vanishing NDE does not imply that each department takes fair decisions.} We show in Section~\ref{app:ctrfkusner-cf} that the counterfactual notion of fairness of \citet{KusnerLRS17} implies demographic parity even when confounding is allowed. In Section~\ref{app:pathwise-cf} we show that testing the path-dependent counterfactual fairness notion of \citet{KusnerLRS17} is equivalent to testing the IV inequalities.

% and only provide a proof sketch here. We first show that $\nullinter \supseteq \nullctrf \superseteq \nullgraph$ which implies the same relations for $\distinter, \distctrf, \distgraph$. We express the observational distributions in $\distinter$ in terms of the response-function parameterization and show that 

% \begin{equation}\Pr\left(\outcome=1,\dept=\ldept|\sex=1\right) + \Pr\left(\outcome=0,\dept=\ldept|\sex=0\right) - 1 \leq 0,$$

% $$1 - \Pr\left(\outcome=0,\dept=\ldept|\sex=1\right) - \Pr\left(\outcome=1,\dept=\ldept|\sex=0\right) \geq 0.
% $$

% The above conditions, for all $\ldept$, are the same as the IV inequalities in \eqref{eq:iv}. As we shall see in the next subsection, this is not coincidence. 



% \begin{equation}\label{eq:inter_resp}
%         \Pr\Paren{\outcome=\loutcome\mid \doop{\formsex=\lsex'}, \doop{\dept=\ldept}} = \sum\limits_{\Paren{\respfunc_1,\respfunc_2}  \in \cX_{\response}}\bm{1}\Brack{\respfunc_2\Paren{\lsex',\ldept}=\loutcome}\tilde{P}\Paren{\respfunc_1,\respfunc_2}.
%    \Pr\Paren{\outcome=\loutcome\mid \dept=\ldept, \sex = \lsex, \doop{\formsex=\lsex'}} &= \sum\limits_{\Paren{\respfunc_1,\respfunc_2}  \in \cX_{\response}}\bm{1}\Brack{\respfunc_2\Paren{\lsex',\ldept}=\loutcome, \respfunc_1\Paren{\lsex} = \ldept}\tilde{P}\Paren{\respfunc_1,\respfunc_2}.\label{eq:cond_resp}
%\end{equation}

% Likewise, the conditional distribution $P_{\model}\Paren{\outcome,\dept|\sex}$ can also be expressed as a linear function of $\tilde{P}$, 

% \begin{equation}\label{eq:obs_resp}
%     P_{\model}\Paren{\outcome=\loutcome,\dept=\ldept|\sex=\lsex} = \sum\limits_{\Paren{\respfunc_1,\respfunc_2}  \in \cX_{\response}}\bm{1}\Brack{\respfunc_2\Paren{\lsex,\ldept}=\loutcome, \respfunc_1\Paren{\lsex} = \ldept}\tilde{P}\Paren{\respfunc_1,\respfunc_2}.
% \end{equation}

% Similar to \cite{Balke95}, we setup a symbolic linear program that derives upper and lower bounds in terms of the observational distribution $P_{\model}\Paren{\outcome,\dept \mid \sex}$ by maximizing and minimizing \eqref{eq:inter_resp} subject to symbolic constraints obtained from \eqref{eq:obs_resp}. These bounds can be used to bound the difference between the interventional queries in \eqref{eq:interfair}. The code to obtain these bounds is in the supplementary material.

% For the conditional notion of fairness, we first rewrite $\Pr\Paren{\outcome|\dept, \sex, \doop{\formsex}} = \frac{\Pr\Paren{\outcome, \dept| \sex, \doop{\formsex}}}{\Pr\Paren{\dept| \sex, \doop{\formsex}}}$. Note that the denominator, $\Pr\Paren{\dept| \sex, \doop{\formsex}}$ is identifiable and equal to $\Pr\Paren{\dept| \sex}$. The numerator can be expressed in terms of a linear function of $\tilde{P}$, 

% \begin{equation}\label{eq:cond_resp}
%        \Pr\Paren{\outcome=\loutcome, \dept=\ldept \mid \sex = \lsex, \doop{\formsex=\lsex'}} = \sum\limits_{\Paren{\respfunc_1,\respfunc_2}  \in \cX_{\response}}\bm{1}\Brack{\respfunc_2\Paren{\lsex',\ldept}=\loutcome, \respfunc_1\Paren{\lsex} = \ldept}\tilde{P}\Paren{\respfunc_1,\respfunc_2}.
% \end{equation}

% With a similar linear program setup, we obtain bounds on the difference of the interventional queries in \eqref{eq:cndfair}. In the next section we see how the IV inequalities and the bounds obtained in this subsection can be turned into a statistical test.

% For the interventional notion of fairness we express the bounds in closed form below. Due to space constraints, for the conditional notion of fairness the bounds are in the supplementary material. The bounds on $\Pr\Paren{\outcome=\loutcome\mid \doop{\formsex=\lsex'}, \doop{\dept=\ldept}}$ are 






% As we shall see in Section~\ref{sec:tests}, testing the graphical notion of fairness poses several challenges. These stem from the difficulty of obtaining constraints that a graphical query imposes on the set of observational distributions since a particular observational distribution can arise from multiple SCMs where the form of

% Since, it is easier to test fairness notions based on observational, interventional or counterfactual queries, we define fairness notions based on these probabilistic queries below. Another possible advantage of defining fairness notions based on probabilistic queries is the ability to analyze them under selection bias. 

% \begin{definition}[Interventional Notion of Fairness]
%     $\model \in \modelsunconfedge$ is fair according to the interventional notion of fairness if it belongs to 
%      \begin{equation}\label{eq:interfairunconf}
%     \nullinterunconf \triangleq \left \lbrace \model \in \modelsunconfedge: \forall \ldept, \Pr\Paren{\outcome=1|\doop{\sex=1},\doop{\dept=\ldept}} = \Pr\Paren{\outcome=1|\doop{\sex=0},\doop{\dept = \ldept}} \right \rbrace.
%     \end{equation}
%     $\model \in \modelsedge$ is fair according to the interventional notion of fairness if it belongs to
%     \begin{equation}\label{eq:interfair}
%     \nullinter \triangleq \left \lbrace \model \in \modelsedge : \forall \ldept, \Pr\Paren{\outcome=1|\doop{\formsex=1},\doop{\dept=\ldept}} = \Pr\Paren{\outcome=1|\doop{\formsex=0},\doop{\dept = \ldept}} \right \rbrace.
%     \end{equation}
%\end{definition}
% \todo{Notation: Pr() or P()? The latter was introduced earlier on as referring to the observational distribution of the SCM, so better to stick to that. And personally I usually write $P_M$ to emphasize the parameter(SCM)-dependence.}
% \todo{I suggest strengthening these to $P(A|do(S),do(D)) = P(A|do(D))$ (and $S'$ instead of $S$, respectively).}

% The criteria for the interventional notion of fairness for $\modelsunconfedge$ is that the controlled direct effect (CDE) \cite{?} of the ``treatment", $\sex$, on the ``outcome", $\outcome$, is $0$ for every value of the mediator, i.e., every department choice $\ldept$. We compare this notion with the counterfactual notion of natural direct effect (NDE) \cite{Pearl01} in the appendix. Although the NDE is identifiable for models in $\modelsunconfedge$, we show that since the expression of the NDE is a weighted average over the department choice, it is possible for the NDE to be $0$ despite there being discrimination \todo{'discrimination' is yet undefined, replace by specific fairness notion}.  

% If confounding between $\dept$ and $\outcome$ is allowed, the CDE and NDE are both unidentifiable. While bounds on the CDE and NDE are known \cite{KaufmanKMGP05, CaiKPT08}, \todo{sentence stops prematurely?} For $\modelsedge$, we now see the utility of modeling the reported sex on the application form separately. While interventions on $\sex$ are hypothetical and come with issues of interpretation \cite{HuKohlerHausmann20}, interventions on the reported sex, $\formsex$, are practically conceivable. Further, by disentangling the reported sex from the birth sex, we no longer require relying on counterfactual queries such as NDE.   

% In \cite{PearlMackenzie18}, Pearl mentions a disadvantage of CDE, namely intervening on department choice. An applicant's preparation might be targeted to a particular department of their choosing and this would influence the admissions committee's decision largely.\todo{I don't see how this would be problematic?} To circumvent this issue, we define a conditional notion of fairness where we condition on $\sex, \dept$ in addition to intervening on $\formsex$.

% \begin{definition}[Conditional Notion of Fairness]
%     $\model \in \modelsedge$ is fair according to the conditional notion of fairness if it belongs to 
%     \begin{equation}\label{eq:cndfair}
%     \nullcond \triangleq \left \lbrace \model \in \modelsedge : \forall \ldept, \lsex, \Pr\Paren{\outcome=1|\dept = \ldept, \sex = \lsex, \doop{\formsex=1}} = \Pr\Paren{\outcome=1|\dept = \ldept, \sex = \lsex, \doop{\formsex=0}} \right \rbrace.
%     \end{equation}
% \end{definition}

% The conditional notion of fairness is interpreted as mandating that for every sex-department combination, the applicants of a particular sex who apply to a certain department of their choice would have the same probability of admission if their sex on the application form were changed. Notice that despite being an interventional query, $\Pr\Paren{\outcome=1|\dept = \ldept, \sex = \lsex, \doop{\formsex\neq \lsex}}$ has a counterfactual `flavor'.  Further, despite conditioning on $\dept$, we escape the pitfalls that come along with it. We can allow for latent confounding, too, since by also conditioning on $\sex$, there are no longer any open path from $\formsex$ to $\outcome$ with $\dept$ as the collider. Note that this is made possible because of the node-splitting modeling trick.\todo{From my perspective, the conditional version was only relevant when dealing when selection bias, so can be left out of this story. Instead, we should define the counterfactual notion in order to relate to Kushner et al.}



\section{Test Design}

Our proposed evaluation framework consists of two distinct phases designed to validate an AI system's understanding of student misconceptions. In the first phase, we collect unbiased samples of student misconceptions through open-ended responses. The second phase tests whether the AI system can accurately predict how these misconceptions will manifest in a newly generated question.

Let $S$ denote our set of students and $D$ our domain of questions. For any student $s \in S$ and question $q \in D$, we define $A(s,q)$ as the student's response and $C(q)$ as the correct answer. 

In Phase 1, each student $s$ responds to a set of questions $Q_s \subset D$ without multiple choice options. For each incorrect response, we record the tuple $(s,q,A(s,q))$ where $A(s,q) \neq C(q)$. These tuples provide unbiased samples of natural student misconceptions, uninfluenced by the presence of pre-selected answer choices.

For Phase 2, given each Phase 1 tuple $(s,q,A(s,q))$:

1. The AI system implements a function $f_{LLM}: D \times A \to D$ that generates a new question $q'$ based on the original question and incorrect answer.

2. Given $q'$, both our AI system and a human expert independently generate predicted wrong answers. The AI system implements $g_{AI}: D \times D \times A \to A$ that generates prediction $a'$, while the human expert implements $g_H: D \times D \times A \to A$ generating prediction $a''$. A random wrong answer $r$ is also generated to serve as a control.

3. The student $s$ then receives question $q'$ as a multiple choice question with four options presented in random order: the correct answer $C(q')$, the AI-predicted wrong answer $a'$, the expert-predicted wrong answer $a''$, and the random wrong answer $r$. The student's selection is denoted $P_2(s,q,q')$.

\subsubsection{Bayesian Games}
\deni{Makes distributions bold.}
A \mydef{Bayesian game} \cite{harsanyi1968bayesian} $\bgames \doteq (\numplayers, \numactions, \numtypes, \typespace, \typedistrib, \param, \actionspace, \util)$ is a simultaneous-move game which consists of $\numplayers \in \N_+$ players each of whom is characterized by a type space $\typespace[\player] \subset \R^\numtypes$. \amy{characterized by a type in a type space, n'est-ce pas?}\deni{I think these comments are not relevant now that we are reverting to the classical Bayesian game setting no?}\amy{yes, maybe not.}
The players share a \mydef{common prior distribution} \amy{the players don't need a common prior if they see each other's types. they only need this in the independent case, where they see only their own type.}
\amy{the setup does not feel very Bayesian to me. it feels stochastic. like there is a distribution over complete-info games, as determined by the prior and the types, both(?) of which the players observe.} $\typedistrib[\param] \sim \simplex(\typespace)$ over the the \mydef{joint type space} $\typespace \doteq \bigtimes_{\player \in \players} \typespace[\player]$ defined by a \mydef{parameter} $\param \in \params$ coming from a parameter space $\params \subset \R^{\numparams}$.\deni{Make sure ``numparams'' dimensionality does not conflict with other var. def'n}
Each player $\player \in \players$ takes an action $\action[\player] \in \actionspace[\player]$ from its action space $\actionspace[\player] \subset \R^{\numactions}$, and receives a payoff $\util[\player](\action; \type)$ given by the payoff function $\util[\player]: \actionspace \times \typespace \to \R$. We denote the players' \mydef{joint action space} by $\actionspace \doteq \bigtimes_{\player \in \players} \actionspace[\player] \subset \R^{\numplayers \numactions}$ and the vector-valued function of all players' utilities $\util(\action; \type) \doteq \left( \util[\player](\action; \type) \right)_{\player \in \players}$.


% and $\typespace \doteq \bigtimes_{\player \in \players} \typespace[\player] \subset \R^{\numplayers \numtypes}$, and refer to any collection of actions $\action = (\action[1], \hdots, \action[\numplayers]) \in \actionspace$ and types $\type = (\type[1], \hdots, \type[\numplayers]) \in \typespace$ as an \mydef{action profile} and \mydef{type profile} respectively.
% At each time-step $\iter = 0, 1, \hdots$, each player $\player \in \players$ takes an action $\inner[\player][][][\iter] \in \innerset[\player](\context[\iter])$ from an action space $\innerset[\player](\context[\iter]) \subset \actionspace[\player]$\footnote{Going forward, for simplicity, without loss of generality, we drop the dependence of the action space $\innerset[\player]$ on the context $\context$ and take for all player $\player \in \players$, $\innerset[\player] \doteq \actionspace[\player]$.} simultaneously observe a \mydef{context} $\typerv[][\iter] \sim \initcontexts$, i.e., a type profile drawn from a distribution $\initcontexts \in \simplex(\typespace)$ over the set of types. Each player $\player \in \players$ , it takes an action $\inner[\player][][][\iter] \in \innerset[\player](\context[\iter])$ from an action space $\innerset[\player](\context[\iter]) \subset \actionspace[\player]$\footnote{Going forward, for simplicity, without loss of generality, we drop the dependence of the action space $\innerset[\player]$ on the context $\context$ and take for all player $\player \in \players$, $\innerset[\player] \doteq \actionspace[\player]$.}

% Once player $\player \in \players$ has observed the realized contexts $\context[\iter]$, it takes an action $\inner[\player][][][\iter] \in \innerset[\player](\context[\iter])$ from an action space $\innerset[\player](\context[\iter]) \subset \actionspace[\player]$\footnote{Going forward, for simplicity, without loss of generality, we drop the dependence of the action space $\innerset[\player]$ on the context $\context$ and take for all player $\player \in \players$, $\innerset[\player] \doteq \actionspace[\player]$.} and receive reward $\reward[\player](\context, \inner; \type[\player])$ given by a reward function $\reward[\player]: \contexts \times \innerset \times \typespace \to \R$. For a given context, action profile, and type profile tuple, we denote the vector of all  players utilities by $(\context, \action, \type) \mapsto \reward(\context, \action; \type)$  

A Bayesian game is \mydef{continuous} if for all $\type \in \contexts$, $\util(\action; \type)$ is continuous in $\action$ and $\actionspace$ is non-empty and compact.
A Bayesian game is \mydef{concave} if in addition to being continuous, for all types $\type \in \typespace$ and for all players $\player \in \players$, $\util[\player](\action; \type)$ is concave in $\action[\player]$ and $\actionspace[\player]$ is convex.

% A \mydef{joint strategy profile} $\strat: \typespace \to \actionspace$ is a mapping from the joint type space to the joint action space s.t.\ $\strat[\player](\type) \in \actionspace[\player]$ denotes the action played by player $\player$ under type profile $\type \in \typespace$. 
An \mydef{strategy} 
% \deni{Decide if we want discuss the whole decentralized/centralized issue} \deni{I think yes?}
$\strat[\player]: \typespace[\player] \to \actionspace[\player]$ for a player $\player \in \players$, is a mapping from player $\player$'s type to an action s.t. $\strat[\player](\type[\player]) \in \actionspace[\player]$ denotes the action played by player $\player$ when it is of type $\type[\player]$. 
A \mydef{strategy profile} $\strat \doteq \left( \strat[1], \hdots, \strat[\numplayers]\right)$ is a collection of independent strategies, one-per-player, s.t. for any type profile $\type \in \typespace$, $\strat(\type) \in \actionspace$ denotes the action profile played by the players.
% \amy{i think you need a different letter for independent strategy profiles. maybe $\tau$. overloading with $\sigma$ will probably be too confusing (and you can always change the macro later if it isn't).}

An \mydef{ex-ante $\varepsilon$-Bayesian Nash equilibrium} 
% \amy{maybe parameterized instead of Bayesian}\deni{We said no!} 
($\varepsilon$\mydef{-BNE}) is a strategy profile $\strat[][][*] \in \actionspace^\typespace$ s.t.\ for all players $\player \in \players$ and strategy profiles $\strat \in \actionspace^\typespace$, $\Ex_{\typerv \sim \typedistrib[\param]} \left[ \util[\player] (\strat[][][*](\typerv); \typerv)\right] \geq \Ex_{\typerv \sim \typedistrib[\param]} \left[ \util[\player] (\strat[\player][][](\typerv), \strat[-\player][][*] (\typerv); \typerv)\right] - \varepsilon$. 
A $\varepsilon$-ex-post Nash equilibrium ($\varepsilon$-EPNE) is a strategy profile $\strat[][][*] \in \actionspace^\typespace$ s.t. for all players $\player \in \players$, types $\type \in \typespace$, and strategy profiles $\strat \in \actionspace^\typespace$, $\util[\player](\strat[][][*](\type); \type) \geq \util[\player](\strat[\player][][](\type), \strat[-\player][][*](\type); \type) - \varepsilon$. 
A $0$-BNE and $0$-EPNE are simply called BNE and EPNE, respectively.
For concave games, a BNE is guaranteed to exist, while EPNE are not guaranteed exist.%
% \footnote{Traditionally, BNE and EPNE are only defined for independent strategies, in which case EPNE may not exist. Our more general definition applies to strategies that depend on all players' types. \amy{again, it doesn't feel very ex-post.}} 

\deni{We have to define the set of Nash equilibria and EPNE?} \deni{Update: Maybe not}

% \deni{Ignore next para, Old def'n, to remove, keeping it for visual reminder}
% An \mydef{ex-ante $\varepsilon$-Bayesian Nash equilibrium} (joint $\varepsilon$\mydef{-BNE}) is a strategy profile $\strat[][][*] \in \actionspace^\typespace$ s.t. for all players $\player \in \players$ and strategy profiles $\strat \in \actionspace^\typespace$, $\Ex_{\typerv \sim \typedistrib} \left[ \util[\player](\strat[][][*](\typerv); \typerv)\right] \geq \Ex_{\typerv \sim \typedistrib} \left[ \util[\player](\strat[\player][][](\typerv[\player]), \strat[-\player][][*](\typerv[-\player]); \typerv)\right]$. A $\varepsilon$-ex-post Nash equilibrium ($\varepsilon$-EPNE) is a strategy profile $\strat[][][*] \in \actionspace^\typespace$ s.t. for all players $\player \in \players$, types $\type \in \typespace$ and strategy profiles $\strat \in \actionspace^\typespace$, $\util[\player](\strat[][][*](\type); \type) \geq \util[\player](\strat[\player][][](\type[\player]), \strat[-\player][][*](\type[-\player]); \type)$.

% \amy{BNE seems different to me, b/c i don't think $\strat[][][*](\type) = (\strat[\player][][*](\type[\player]), \strat[-\player][][*](\type[-\player]))$. i think it might equal $(\strat[\player][][*](\type), \strat[-\player][][*](\type))$, in which case you have $\util[\player](\strat[][][*](\type); \type) = \util[\player](\strat[\player][][*](\type), \strat[-\player][][*](\type); \type)$ not $\util[\player](\strat[][][*](\type); \type) = \util[\player](\strat[\player][][*](\type[\player]), \strat[-\player][][*](\type[-\player]); \type)$. in the former case, parameters are known to all---it is a complete-info game; in the latter, players know only their own parameters, which is very different from an info-theoretic point of view (so it feels like we are comparing apples to oranges).}

% A \mydef{(Markov or stationary) policy} \cite{maskin2001markov} for player $\player \in \players$, $\policy[\player]: \contexts \to \innerset[\player]$ is a mapping from contexts to actions such that $\policy[\player](\context) \in \innerset[\player]$ denotes the action taken by player $\player$ when it observes context $\context$. We define a \mydef{policy profile} as the collection of policies, i.e., $\policy = (\policy[1], \hdots, \policy[\numplayers]) : \contexts \to \innerset$ such that $\policy(\context) \in \innerset$ denotes the action profile played by the players under context $\context \in \contexts$. The goal of all players $\player \in \players$ is to play a policy $\policy[\player][][*] \in \innerset[\player]^\context$ which maximizes their \mydef{expected cumulative payoff} $\util[\player](\policy[\player], \policy[-\player]; \type[\player]) \doteq \Ex_{\contextrv \sim \initcontexts} \left[ \reward[\player](\contextrv, \policy[\player](\contextrv), \policy[-\player](\contextrv); \type[\player])\right]$. 
% % 


% A \mydef{$\varepsilon$-subgame perfect Nash equilibrium ($\varepsilon$-SPNE)} of a contextual game is a policy profile $\policy[][][*] \in \innerset^\contexts$ such that for all player $\player \in \players$ and for all contexts $\context \in \contexts$,  $\reward[\player](\context, \policy[][][*](\context); \type[\player]) \geq \reward[\player](\policy[\player](\context), \policy[-\player][][*](\context); \type[\player]) - \varepsilon$. A $0$-SPNE is called a \mydef{subgame perfect Nash equilibrium (SPNE)}. A SPNE is guaranteed to exist in concave contextual games. Additionally, note that when the context space is a singleton, then the SPNE of contextual game simply reduce to the NE of the game played at that context. Further, notice that the SPNE of contextual games with different context distributions are all the same.

% \deni{Ex-ante definition is commented out below. Might want to make the introduction of the type inside the paranthesis not as a superscript.}

\deni{Need to make this a function of the parameter distribution and not a function of the parameter, and us the same expectation overload as the Markov game paper.}
Fixing the payoff functions of the players $\util$, for any type $\type \in \typespace$, we define the 
% \mydef{regret} $\regret[][\util]: \actionspace \times \actionspace \to \R^\numplayers$ for playing an action profile $\action$ as compared to another action profile $\otheraction$, as follows: for all followers $\player \in \players$,
% $\regret[\player][](\action, \otheraction; \outer) = \util[\player](\outer, (\otheraction[\player], \action[-\player])) - \util[\player](\outer, \action)$. The 
\mydef{cumulative regret}, $\cumulregret[][]: \actionspace \times \actionspace \times \typespace \to \R$ between two action profiles $\action \in \actionspace$ and $\otheraction \in \actionspace$ across all players in a game as $\cumulregret[][](\action, \otheraction; \type) = \sum_{\player \in \players} \util[\player](\otheraction[\player], \action[-\player]; \type) - \util[\player](\action; \type)$.
Further, the \mydef{exploitability} or (Nikaido-Isoda potential function \cite{nikaido1955note}) of an action profile $\action \in \actionspace$ is defined as 
$\exploit[][](\action; \type) = \max_{\otheraction \in \actionspace} \cumulregret[][](\action, \otheraction; \type)$ \cite{goktas2022exploit}. 
Overloading notation,
for any common prior distribution parameter $\param \in \params$, we define the \mydef{ex-ante cumulative regret} at any given type profile $\type \in \typespace$, $\cumulregret[][\param]: \actionspace^\typespace \times \actionspace^\typespace \to \R$ between two strategy profiles $\strat \in \actionspace^\typespace$ and $\otherstrat \in \actionspace^\typespace$ across all players as $\cumulregret[][\param](\strat, \otherstrat) = \sum_{\player \in \players} \left( \Ex_{\typerv \sim \typedistrib[\param]} \left[ \util[\player](\otherstrat[\player][][](\typerv[\player]), \strat[-\player][][](\typerv[-\player]); \typerv)\right] -\Ex_{\typerv \sim \typedistrib[\param]} \left[ \util[\player](\strat(\typerv); \typerv)\right] \right)$.
Further, the \mydef{ex-ante exploitability} or (Nikaido-Isoda potential function \cite{nikaido1955note}) of a strategy profile $\strat \in \actionspace^\typespace$ is defined as 
$\exploit[][\param](\strat) = \max_{\otherstrat \in \actionspace^\typespace} \cumulregret[][\param](\strat, \otherstrat)$ \cite{goktas2022exploit}. We note that for all $\strat \in \actionspace^\typespace$, $\exploit[][\param](\strat) \geq 0$, and $\strat[][][*]$ is a BNE of $\game$ iff $\exploit[][\param](\strat[][][*]) = 0$. 
% For
% for any type profile $\type \in \typespace$,
% we define the \mydef{ex-post cumulative regret} and the \mydef{ex-post exploitability} respectively as 
% $\cumulregret[][\type](\strat, \otherstrat) \doteq \sum_{\player \in \players} \left[ \util[\player](\otherstrat[\player][][](\type[\player]), \strat[-\player][][](\type[-\player]); \type) - \util[\player](\strat(\type); \type) \right]$
% $\cumulregret[][] (\action, \otheraction; \type) = \sum_{\player \in \players} \util[\player] (\otheraction[\player], \action[-\player]; \type) - \util[\player] (\action; \type)$
% and $\exploit[][](\strat; \type) \doteq \max_{\otherstrat \in \actionspace} \cumulregret[][] (\strat, \otherstrat; \type)$.
\deni{I think these two comments are now answered! Seems to be a typo from before! Alec, I also added two macros into ``auxiliary/filecommands.sty'' for you to add comments make edits if you prefer that!}\amy{not sure we are maxing over the right thing here. still wondering about strategies. actually, maybe we are, but $\otherstrat$ should be $\otheraction$. and i feel like we need an expectation of types $T$.}
\alec{$\rho$ seems undefined here. Also unclear what is the difference between $\mathcal{A}$ and $\mathcal{A}^{\mathcal{T}}$ } \deni{$\actionspace$ is the joint action space, while $\actionspace^\typespace$ is the space of joint strategy profiles, i.e. mappings from typespace to action space. } \deni{Although might just need to define the strategy space because there is a problem with def'n, i.e., the current notation also includes centralized strategies.}

% \subsubsection{Contextual Games}
% A \mydef{(simultaneous-move) contextual game} \cite{sessa2020contextual} $(\numplayers, \numactions, \contexts, \initcontexts, \innerset, \typespace, \type, \reward)$ is a repeated game played over an infinite horizon which comprises of $\numplayers \in \N_+$ players each of whom is characterized by a type space $\typespace[\player] \in \subset \R^\numtypes$. At each time-step $\iter = 0, 1, \hdots$, each player $\player \in \players$ takes an action $\inner[\player][][][\iter] \in \innerset[\player](\context[\iter])$ from an action space $\innerset[\player](\context[\iter]) \subset \actionspace[\player]$\footnote{Going forward, for simplicity, without loss of generality, we drop the dependence of the action space $\innerset[\player]$ on the context $\context$ and take for all player $\player \in \players$, $\innerset[\player] \doteq \actionspace[\player]$.} simultaneously observe a \mydef{context} $\typerv[][\iter] \sim \initcontexts$, i.e., a type profile drawn from a distribution $\initcontexts \in \simplex(\typespace)$ over the set of types. Each player $\player \in \players$ , it takes an action $\inner[\player][][][\iter] \in \innerset[\player](\context[\iter])$ from an action space $\innerset[\player](\context[\iter]) \subset \actionspace[\player]$\footnote{Going forward, for simplicity, without loss of generality, we drop the dependence of the action space $\innerset[\player]$ on the context $\context$ and take for all player $\player \in \players$, $\innerset[\player] \doteq \actionspace[\player]$.}

% % Once player $\player \in \players$ has observed the realized contexts $\context[\iter]$, it takes an action $\inner[\player][][][\iter] \in \innerset[\player](\context[\iter])$ from an action space $\innerset[\player](\context[\iter]) \subset \actionspace[\player]$\footnote{Going forward, for simplicity, without loss of generality, we drop the dependence of the action space $\innerset[\player]$ on the context $\context$ and take for all player $\player \in \players$, $\innerset[\player] \doteq \actionspace[\player]$.} and receive reward $\reward[\player](\context, \inner; \type[\player])$ given by a reward function $\reward[\player]: \contexts \times \innerset \times \typespace \to \R$. For a given context, action profile, and type profile tuple, we denote the vector of all  players utilities by $(\context, \action, \type) \mapsto \reward(\context, \action; \type)$  

% A contextual game is \mydef{continuous} if for all $\context \in \contexts$, $\reward(\context, \inner; \type)$ is continuous in $\inner$ and $\innerset$ is non-empty and compact.
% A contextual game is \mydef{concave} if in addition to being continuous, for all contexts $\context \in \contexts$ and for all players $\player \in \players$, $\reward[\player](\context, \inner)$ is concave in $\inner[\player]$ and $\innerset[\player]$ is convex.

% A \mydef{(Markov or stationary) policy} \cite{maskin2001markov} for player $\player \in \players$, $\policy[\player]: \contexts \to \innerset[\player]$ is a mapping from contexts to actions such that $\policy[\player](\context) \in \innerset[\player]$ denotes the action taken by player $\player$ when it observes context $\context$. We define a \mydef{policy profile} as the collection of policies, i.e., $\policy = (\policy[1], \hdots, \policy[\numplayers]) : \contexts \to \innerset$ such that $\policy(\context) \in \innerset$ denotes the action profile played by the players under context $\context \in \contexts$. The goal of all players $\player \in \players$ is to play a policy $\policy[\player][][*] \in \innerset[\player]^\context$ which maximizes their \mydef{expected cumulative payoff} $\util[\player](\policy[\player], \policy[-\player]; \type[\player]) \doteq \Ex_{\contextrv \sim \initcontexts} \left[ \reward[\player](\contextrv, \policy[\player](\contextrv), \policy[-\player](\contextrv); \type[\player])\right]$. 
% % 


% A \mydef{$\varepsilon$-subgame perfect Nash equilibrium ($\varepsilon$-SPNE)} of a contextual game is a policy profile $\policy[][][*] \in \innerset^\contexts$ such that for all player $\player \in \players$ and for all contexts $\context \in \contexts$,  $\reward[\player](\context, \policy[][][*](\context); \type[\player]) \geq \reward[\player](\policy[\player](\context), \policy[-\player][][*](\context); \type[\player]) - \varepsilon$. A $0$-SPNE is called a \mydef{subgame perfect Nash equilibrium (SPNE)}. A SPNE is guaranteed to exist in concave contextual games. Additionally, note that when the context space is a singleton, then the SPNE of contextual game simply reduce to the NE of the game played at that context. Further, notice that the SPNE of contextual games with different context distributions are all the same.

% \sdeni{}{Overloading notation, we define the \mydef{contextual cumulative regret} at any given type profile $\type \in \typespace$, $\cumulregret[][\type]: \contexts \times \actionspace \times \actionspace \to \R$ between two action profiles $\action \in \actionspace$ and $\otheraction \in \actionspace$ across all players in context $\context$ of a contextual game as $\cumulregret[][\type](\context, \action, \otheraction) = \sum_{\player \in \players} \reward[\player](\context, (\otheraction[\player], \action[-\player]); \type[\player]) - \reward[\player](\context, \action; \type[\player])$.
% Further, the \mydef{contextual exploitability} or (Nikaido-Isoda potential function \cite{nikaido1955note}) of an action profile $\action \in \actionspace$ is defined as 
% $\exploit[][\type](\context, \action) = \max_{\otheraction \in \actionspace} \cumulregret[][\type](\context, \action, \otheraction)$ \cite{goktas2022exploit}. We note that for all $\context \in \contexts$, $\action \in \actionspace$, $\exploit[][\type](\context, \action) \geq 0$, and $\action[][][][*]$ is a SPNE of $(\numplayers, \numactions, \actionspace, \typespace, \type, \util)$ iff $\exploit[][\type](\action[][][][*]) = 0$.} 
% $\util[\player](\policy[][][*]) \geq \util[\player](\policy[\player], \policy[-\player][][*]) - \varepsilon$

% \footnote{Note that although the set of $\varepsilon$-SPNE of any contextual game $(\numplayers, \numactions, \contexts, \initcontexts, \innerset, \reward)$ is also a Nash game $(\numplayers, \numactions, \innerset^\contexts, \util)$, this reduction is mostly vacuous, as Nash's theorem \cite{nash1950existence} or Arrow-Debreu's lemma on abstract economies \cite{arrow-debreu} does not provide existence in this constructed Nash game.}
 % is a policy profile $\policy[][][*] \in \innerset^\contexts$ s.t.\ for all players $\player \in \players$ and actions $\action[\player] \in \actionspace[\player]$, $\util[\player](\action[][][][*]) \geq \util[\player](\action[\player], \action[-\player][][][*]) - \varepsilon$.
% , and for each player $\player \in \players$, an action correspondence $\innerset[\player]: \contexts \rightrightarrows \actionspace$ s.t. for any context $\context \in \contexts$ and any player $\player \in \players$, $\innerset[\player](\context) \subset \actionspace$ denotes the set of actions player $\player$ can choose under context $\context$,  who , encounter a Nash game $(\numplayers, \numactions, \innerset(\contextrv), \util())$ 

\if 0 
\deni{Decide how to incorporate this into the big story.}
\paragraph{Stackelberg-Nash Games}
An $(\numplayers + 1)$-player \mydef{Stackelberg-Nash game} $\stackgame \doteq (\numplayers, \numactions, \outerset, \innerset, \util)$ comprises one player called the \mydef{leader} and $\numplayers \in \N_{++}$ players called \mydef{followers}.
In a Stackelberg-Nash game, the leader first commits to an action $\outer \in \outerset$ from an action space $\outerset \subset \R^{\outerdim}$.
Then, having observed the leader's action, each follower $\player \in \players$, responds with an action $\inner[\player]$ in their \mydef{action space} $\innerset[\player] \subset \R^{\numactions}$.
% determined by the \mydef{feasible action correspondence} $\innerset[\player]: \outerset \rightrightarrows \actionspace[\player]$ which takes as input the leader's action $\outer$ and outputs a subset of \mydef{the action space} $\actionspace[\player] \subset \R^{\numactions}$. 
We define the \mydef{followers' joint action space} $\innerset \doteq \bigtimes_{\player \in \players} \innerset[\player]$. 
% and the \mydef{follower joint feasible action correspondence} by $\innerset(\outer) = \bigtimes_{\player \in \players} \innerset[\player](\outer) \subset \bigtimes_{\player \in \players} \actionspace[] \subset \R^{\numplayers \numactions}$
We refer to a collection of actions $\inner = (\inner[1], \hdots, \inner[\numplayers]) \in \innerset$ as a \mydef{follower action profile}, and to a collection $(\outer, \inner) \in \outerset \times \innerset$ comprising an action for the leader and a follower action profile as simply an \mydef{action profile}. 
% \deni{Maybe also define feasible action profile.} A Stackelberg-Nash game is said to have \mydef{independent action sets} if the feasible action correspondence of each player $\player \in \players$, $\innerset[\player]$ is a constant correspondence, i.e., $\innerset[\player](\outer) = \innerset[\player](\outer[][][\prime]) = \actionspace$ for all leader actions $\outer, \outer[][][\prime] \in \outerset$.

After all players choose an action, the leader receives payoff $\util[0](\outer, \inner) \in \R$, while each follower $\player \in \players$ receives payoff $\util[\player](\outer, \inner) \in \R$. 
Each player $\player \in \allplayers$ aims to maximize her payoff $\util[\player]: \outerset \times \innerset \to \R$. 
For all followers $\player \in \players$, we define the $\delta$-\mydef{best-response correspondence} $\br[\player][\delta] (\outer, \inner[-\player]) \doteq \left\{ \inner[\player] \in \innerset \mid \util[\player](\outer, \inner[][][]) \geq \max_{\inner[\player] \in \innerset[\player]} \util[\player] (\outer, (\inner[\player], \inner[-\player][][])) - \delta \right\}$ and the \mydef{joint $\delta$-best-response correspondence} $\br[][\delta](\outer, \inner) \doteq \bigtimes_{\player \in \players} \br[\player][\delta] (\outer, \inner[-\player])$. 

%\amy{we use the semi-colon notation when talking about exploitability. should we use it here as well? i tend to think yes. e.g., $\br[\player][\delta] (\inner[-\player]; \outer)$}\deni{I feel like we don't have to because this is the follower's best response in the Stackelberg game. While for exploitability that is the exploitability of the lower level game which the leader's action parametrizes so it makes sense.}

Accordingly, we define the \mydef{follower regret} $\regret[][]: \innerset \times \innerset \times \outerset \to \R^\numplayers$ for playing an action profile $\inner$ as compared to another action profile $\otherinner$ when the leader plays $\outer \in \outerset$, as follows: for all followers $\player \in \players$,
$\regret[\player][](\inner, \otherinner; \outer) = \util[\player](\outer, (\otherinner[\player], \inner[-\player])) - \util[\player](\outer, \inner)$. 
The \mydef{follower cumulative regret}, $\cumulregret: \innerset \times \innerset \times \outerset \to \R$ between two action profiles $\inner \in \innerset$ and $\otherinner \in \innerset$ across all players in a game is given by $\cumulregret(\inner, \otherinner; \outer) = \sum_{\player \in \players} \util[\player](\outer, (\otherinner[\player], \inner[-\player])) - \util[\player](\outer, \inner)$.
Further, the \mydef{follower exploitability} or (Nikaido-Isoda potential function \cite{nikaido1955note}) of a follower action profile $\inner \in \innerset$ is defined as 
$\exploit(\inner) = \max_{\otherinner \in \innerset} \cumulregret(\inner, \otherinner)$ \cite{goktas2022exploit}. 

% \if 0
% The canonical solution concept for Stackelberg-Nash games is the $(\varepsilon, \delta)$-\mydef{Stackelberg-Nash equilibrium (SNE)}, an action profile $(\outer[][][*], \inner[][][*]) \in \innerset \times \outerset$ such that:
% \begin{align}
%     \util[0](\outer[][][*], \inner[][][*]) &\geq \max_{\outer \in \outerset: \inner \in \br[][\delta](\outer, \inner)} \util[0](\outer, \inner) - \varepsilon \\
%     \util[\player](\outer[][][*], \inner[][][*]) &\geq \max_{\inner[\player] \in \innerset[\player]} \util[\player](\outer[][][*], (\inner[\player], \inner[-\player][][*])) - \delta&& \forall \player \in \players
% \end{align}
% A $(0,0)$-Stackelberg-Nash equilibrium is simply called a Stackelberg-Nash equilibrium.
% Intuitively, a $(\varepsilon, \delta)$-SNE is an action profile at which the followers play a Nash equilibrium, while the leader $\varepsilon$-approximately maximizes its payoff over its action space, assuming that the followers will play a $\varepsilon$-Nash equilibrium for any of its actions. 
% \fi


% \deni{Can remove the eqm defs, since they're in the intro.}
% \amy{not sure, b/c we don't define $\epsilon-\delta$-SE.}


As the joint best-response correspondence is not necessarily singleton-valued, the leader's objective is likewise a correspondence: i.e., multiple values could be associated with a fixed strategy $\outer \in \outerset$.
As a result, we cannot re-formulate this problem as a single objective optimization without fixing a selection criteria over the followers' joint best-responses.
The results we prove in this paper rely on the strong Stackelberg-Nash Equilibrium as a solution concept:

\begin{definition}[Strong Stackelberg-Nash Equilibrium]
A $(\varepsilon, \delta)$-\mydef{strong Stackelberg-Nash equilibrium (SSNE)} is an action profile $(\outer, \inner) \in \outerset \times \innerset$ s.t.
% 
%\begin{align}
     $\util[0](\outer[][][*], \inner[][][*]) \geq \max_{\outer \in \outerset} \max_{\inner \in \br[][\delta](\outer, \inner)} \util[0](\outer, \inner) - \varepsilon$ and
     $\util[\player](\outer[][][*], \inner[][][*]) \geq \max_{\inner[\player] \in \innerset[\player]} \util[\player](\outer[][][*], (\inner[\player], \inner[-\player][][*])) - \delta$, for all $\player \in \players$.
%\end{align}

% \begin{align}
%     \outer[][][*] &\in \argmax_{\outer \in \outerset} \max_{\inner \in \br(\outer, \inner)} \util[0](\outer, \inner)\\
%     \inner[\player][][*] &\in \argmax_{\inner[\player] \in \innerset[\player]} \util[\player](\outer[][][*], (\inner[\player], \inner[-\player][][*])) && \forall \player \in \players
% \end{align}
\end{definition}
\fi
% \begin{definition}[Weak Stackelberg Equilibrium]
% A $(\varepsilon, \delta)$-\mydef{weak Stackelberg-Nash equilibrium (WSNE)} is an action profile $(\outer, \inner) \in \outerset \times \innerset$ s.t.
% %
% %\begin{align}
%      $\util[0](\outer[][][*], \inner[][][*]) \geq \min_{\inner \in \br[][\delta](\outer, \inner)} \util[0](\outer, \inner) - \varepsilon$, for all $\outer \in \outerset$, and
%      $\util[\player](\outer[][][*], \inner[][][*]) \geq  \max_{\inner[\player] \in \innerset[\player]}\util[\player](\outer[][][*], (\inner[\player], \inner[-\player][][*])) - \delta$, for all $\player \in \players$.
% %\end{align}
% \end{definition}

%%% SPACE
% %
% \if 0
% In these definitions, the leader approximately optimizes its strategy assuming the followers approximately optimize theirs, in which case $\delta \geq 0$. \amy{i think what this means is that $\epsilon$ is a function of $\delta$.} 
% As a result, a $(0,0)$-SSNE/WSNE might not be a $(0, \delta)$-SSNE/WSNE in general.
% \deni{Can we please discuss this, I want to add a footnote but not sure how to phrase it. There is this interesting phenonmenon in the general sum-setting where computing an approximate equilibrium might be hard because the follower's approximate best-response can induce discontinuous change in the *equilibrium* strategy of the leader, and as a result a $(0,0)$-SSNE/WSNE might not be a $(0, \delta)$-SSNE/WSNE. }
% \sdeni{}{Note that an important difference above approximate Stackelberg-Nash definitions are much harder to to compute }
% \amy{i know we discussed yesterday, but need to discuss again.} \amy{i think you might want to move this discussion to right after Obs 1. it might be easier to explain there.}
% \fi

% \deni{The reason why we need the joint convexity assumption is because our goal is to compute a VE, which does not generally exist (and because projection onto non-convex sets is often hard!).}

% We also define local versions of these equilibrium concepts, which are more tractable in general. A pseudo-game is simply called a \mydef{game} if for all players $\player \in \players$, $\actions(\naction[\player])$ is a constant correspondence, i.e., for all players $\player \in \players$, and strategy profiles $\action, \otheraction \in \innerset, \actions(\naction[\player]) = \actions(\otheraction[-\player])$.


%A \mydef{local generalized Nash equilibrium (local GNE)} is an strategy profile $\inner[][][*]$ s.t.\ for all players $\player \in \players$, and $\inner[\player] \in \actions[\player](\naction[\player][][][*]) \cap \ball[\varepsilon][{\inner[\player]}]$, \amy{fix me!} $\util[\player](\inner[][][*]) \geq \util[\player](\inner[\player], \naction[\player][][][*])$.
% A \mydef{local variational equilibrium (local VE) or normalized GNE} is an strategy profile $\inner[][][*]$ s.t.\ for all strategy profiles and $\action \in \innerset \cap \ball[\varepsilon][\action]$ s.t.\ $\actionconstr(\action) \geq \zeros$, $\util[\player](\inner[][][*]) \geq \util[\player](\inner[\player], \naction[\player][][][*])$. 


% $\hypothesis \in \argmin_{\R^{\states \times \actionspace}} \nicefrac{1}{\numsamples} \sum_{\numsample \in [\numsamples]} \left\| \mathrm{Nash}\left(\hypothesis[][*](\state[\numsample]\right) - \inner[][][][\numsample]  \right\|^2$ and for all $\state \in \states$, $\hypothesis[][*](\state[\numsample])$ is a Nash equilibrium of $$.
This work identifies signal collapse as a critical bottleneck in one-shot neural network pruning. Performance loss in pruned networks is due to \textbf{signal collapse} in addition to the removal of critical parameters. We propose \textbf{REFLOW} (\textbf{Re}storing \textbf{F}low of \textbf{Low}-variance signals), a simple yet effective method that mitigates signal collapse without computationally expensive weight updates. By focusing on signal preservation, REFLOW highlights the importance of mitigating signal collapse in sparse networks and enables magnitude pruning to match or surpass state-of-the-art one-shot pruning methods such as CHITA, CBS, and WF.

REFLOW consistently achieves state-of-the-art accuracy across diverse architectures, restoring ResNeXt-101 from under 4.1\% to 78.9\% top-1 accuracy at 80\% sparsity on ImageNet. Its lightweight design makes it a practical solution for both research and deployment, delivering high-quality sparse models without the overhead of traditional approaches. These findings challenge the traditional emphasis on weight selection strategies and underscore the critical role of signal propagation for achieving high-quality sparse networks in the context of one-shot pruning.



\section{Acknowledgements}
This work was supported by Booking.com.

 % \bibliographystyle{alpha}
 \bibliographystyle{plainnat}
  \bibliography{biblio.bib}
  \appendix
% \documentclass[twoside]{article}

% \usepackage{aistats2025}
% If your paper is accepted, change the options for the package
% aistats2025 as follows:
%
%\usepackage[accepted]{aistats2025}
%
% This option will print headings for the title of your paper and
% headings for the authors names, plus a copyright note at the end of
% the first column of the first page.

% If you set papersize explicitly, activate the following three lines:
%\special{papersize = 8.5in, 11in}
%\setlength{\pdfpageheight}{11in}
%\setlength{\pdfpagewidth}{8.5in}

% If you use natbib package, activate the following three lines:
%\usepackage[round]{natbib}
%\renewcommand{\bibname}{References}
%\renewcommand{\bibsection}{\subsubsection*{\bibname}}

% If you use BibTeX in apalike style, activate the following line:
%\bibliographystyle{apalike}

% \begin{document}

% If your paper is accepted and the title of your paper is very long,
% the style will print as headings an error message. Use the following
% command to supply a shorter title of your paper so that it can be
% used as headings.
%
%\runningtitle{I use this title instead because the last one was very long}

% If your paper is accepted and the number of authors is large, the
% style will print as headings an error message. Use the following
% command to supply a shorter version of the authors names so that
% they can be used as headings (for example, use only the surnames)
%
%\runningauthor{Surname 1, Surname 2, Surname 3, ...., Surname n}

% Supplementary material: To improve readability, you must use a single-column format for the supplementary material.
\onecolumn
\appendix
\aistatstitle{From Deep Additive Kernel Learning to Last-Layer \\ Bayesian Neural Networks via Induced Prior Approximation: \\
Supplementary Materials}

\section{SPARSE CHOLESKY DECOMPOSITION}
\label{sec:sparse chol decompose}
In this section, we present the algorithm for constructing the induced grids $\mathbf{U}$ as defined in \cref{eq:GPlayer} by using sorted dyadic points, and obtaining the sparse Choleksy decomposition of the Laplace kernel in one dimension, as proposed in \citep{ding2024sparse}.

A set of one-dimensional level-$L$ dyadic points $\Xv_L$ in increasing order over the interval $[0,1]$ is defined as:
\begin{align}
    \Xv_{L}:= \left\{ \frac{1}{2^{L}}, \frac{2}{2^{L}}, \frac{3}{2^{L}}, \ldots, \frac{2^{L}-1}{2^{L}} \right\}.
\end{align}
However, this increasing order does not yield a sparse representation of the Markov kernel $k(\cdot,\cdot)$ on the points $\Xv_L$, i.e., Cholesky decomposition of the covariance matrix $k(\Xv_L, \Xv_L)$ is not sparse. To achieve a sparse hierarchical expansion, we first sort the dyadic points $\Xv_L$ according to their levels.

\paragraph{Sorted Dyadic Points}
For level-$\ell$ dyadic points $\Xv_{\ell}$ where $ \ell=1,\ldots,L$, we first define the set $\rho(\ell)$ consisting of odd numbers as follows:
\begin{align}
    \rho(\ell) = \left\{ 1,3,5,\ldots,2^{\ell}-1 \right\}.
\end{align}
Next, we define the sorted incremental set $\Dv_{\ell}$ (with $\Xv_{0}:= \varnothing$) as:
\begin{align}
    \Dv_{\ell} = 
    \left\{ \frac{i}{2^{\ell}}: i\in \rho(\ell) \right\} = \Xv_{\ell} - \Xv_{\ell-1}, \quad  \ell=1,\ldots L.
\end{align}
Thus, the level-$L$ dyadic points $\Xv_L$ can be decomposed into disjoint incremental sets $\{ \Dv_{\ell} \}_{\ell=1}^{L}$:
\begin{align}
    \Xv_{L} = \cup_{\ell=1}^{L} \Dv_{\ell}, \quad \Dv_{i} \cap \Dv_{j} = \varnothing \text{ for $i\neq j$}.
\end{align}
Therefore, we can define the sorted level-$L$ dyadic points using these incremental sets as:
\begin{align}\label{eq:sorted dyadic}
    \Xv_{L}^{\text{sort}}:= \left\{ \Dv_1,\Dv_2, \ldots, \Dv_{L} \right\} 
    = \left\{ \frac{i \in \rho(\ell) }{2^{\ell}}, \ell=1,\ldots,L \right\}.
\end{align}
For example, the sorted level-3 dyadic points are given by:
\begin{align}
    \Xv_{3}^{\text{sort}} 
    = \bigg\{ 
    \begingroup
        \color{blue}
        \underbracket{
            \color{black}
            \frac{1}{2^1}
        }_{\color{blue}
            \Dv_1
        }
    \endgroup
    , 
    \begingroup
        \color{blue}
        \underbracket{
            \color{black}
            \frac{1}{2^2}, \frac{3}{2^2}
        }_{\color{blue}
            \Dv_2
        }
    \endgroup
    ,
    \begingroup
        \color{blue}
        \underbracket{
            \color{black}
            \frac{1}{2^3}, \frac{3}{2^3}, \frac{5}{2^3}, \frac{7}{2^3}
        }_{\color{blue}
            \Dv_3
        }
    \endgroup
     \bigg\}.
\end{align}

\paragraph{Algorithm}
We now present the algorithm for computing the inverse of the upper triangular Cholesky factor $[ \Lv_{\Xv_{L}^{\text{sort}}}^{\top} ]^{-1}$ of the covariance matrix $k(\Xv_{L}^{\text{sort}}, \Xv_{L}^{\text{sort}})$ in \Cref{alg:cholesky}, where $\Lv_{\Xv_{L}^{\text{sort}}} \Lv_{\Xv_{L}^{\text{sort}}}^{\top} = k(\Xv_{L}^{\text{sort}}, \Xv_{L}^{\text{sort}})$.. The corresponding proof can be found in \citep{ding2024sparse}. The output of \Cref{alg:cholesky} is a sparse matrix with $\Oc(3 \cdot (2^{L}-1))$ nonzero entries. Since each iteration of the for-loop only requires solving a $3 \times 3$ linear system, which costs $\Oc(3^3)$ time, the total computational complexity of \Cref{alg:cholesky} is $\Oc(2^L-1)$. This implies that the complexity of computing $\left[ \Lv_{\Uv}^{\top} \right]^{-1}$ in \cref{eq:GPlayer} is $\Oc(M)$ when $\Uv$, the induced grid of size $M$, consists of sorted dyadic points as defined in \cref{eq:sorted dyadic}.

\begin{algorithm}[hbt!]
\caption{Computation of the inverse Cholesky factor for the Markov kernel $k(\cdot, \cdot)$ on sorted one-dimensional level-$L$ dyadic points $\Xv_L^{\text{sort}}$.}
\label{alg:cholesky}
\setstretch{0.99} % set the line spacing to 0.99
\begin{algorithmic}[1]
    \STATE {\bfseries Input:} Markov kernel $k(\cdot,\cdot)$, sorted level-$L$ dyadic points $\Xv_{L}^{\text{sort}}$
    \STATE {\bfseries Output:} inverse of the upper triangular Cholesky factor $\Rv:= [ \Lv_{\Xv_{L}^{\text{sort}}}^{\top} ]^{-1}$, s.t. $\Lv_{\Xv_{L}^{\text{sort}}} \Lv_{\Xv_{L}^{\text{sort}}}^{\top} = k(\Xv_{L}^{\text{sort}}, \Xv_{L}^{\text{sort}})$
    \STATE Initialize $\Rv \leftarrow \text{zeros($2^L-1$,$2^L-1$)}$;
    \STATE Define $k(\pm \infty, \cdot) = k(\cdot, \pm \infty) = 0$;
    \FOR{$\ell=1$ {\bfseries to} $L$}
        \FOR{$i \in \rho(\ell)=\{1,3,\ldots,2^{\ell}-1\}$}
            \STATE $x_{\text{mid}} := \frac{i}{2^{\ell}}$;\quad
            $x_{\text{left}}:=\frac{i-1}{2^{\ell}}$ {\bfseries if} $i>1$ {\bfseries else} $-\infty$;\quad
            $x_{\text{right}}:=\frac{i+1}{2^{\ell}}$ {\bfseries if} $i<2^{\ell}-1$ {\bfseries else} $+\infty$;
            \STATE Get $i_{\text{mid}}$, $i_{\text{left}}$, $i_{\text{right}}$, the indices of the points $x_{\text{mid}}$, $x_{\text{left}}$, $x_{\text{right}}$ in the sorted set $\Xv_{L}^{\text{sort}}$ respectively;
            \STATE Get the coefficients $c_1$, $c_2$, $c_3$ by solving the following linear system:
            \begin{align}
                \begin{bmatrix}
                     & k(x_{\text{left}}, x_{\text{left}})
                     & k(x_{\text{left}}, x_{\text{mid}})
                     & k(x_{\text{left}}, x_{\text{right}}) \\
                     & k(x_{\text{mid}}, x_{\text{left}})
                     & k(x_{\text{mid}}, x_{\text{mid}})
                     & k(x_{\text{mid}}, x_{\text{right}}) \\
                     & k(x_{\text{right}}, x_{\text{left}})
                     & k(x_{\text{right}}, x_{\text{mid}})
                    &k(x_{\text{right}}, x_{\text{right}})
                \end{bmatrix}
                \begin{bmatrix}
                    c1\\
                    c2\\
                    c3
                \end{bmatrix}=
                \begin{bmatrix}
                    0\\
                    1\\
                    0
                \end{bmatrix}.
            \end{align}
            \STATE $[c_1,c_2,c_3] := [c_1,c_2,c_3] / \sqrt{c_2}$;
            \STATE {\bfseries if} $x_{\text{left}} \neq - \infty$, 
            {\bfseries then} $\Rv[i_{\text{left}} ,i_{\text{mid}}] = c_1$; \quad
            {\bfseries if} $x_{\text{right}} \neq + \infty$, 
            {\bfseries then} $\Rv[i_{\text{right}} ,i_{\text{mid}}] = c_3$;
            \STATE $\Rv[i_{\text{mid}} ,i_{\text{mid}}] = c_2$;
        \ENDFOR
    \ENDFOR
\end{algorithmic}
\end{algorithm}


\section{REPARAMETERIZATION OF KERNEL LENGTHSCALES}
\label{sec:theo}
Considering the additive Laplace kernel with fixed lengthscale $\tilde{\theta}$ for all base kernels, applying linear projections $\left\{ \wv_{p}^{\top}\xv \right\}_{p=1}^{P}$ on inputs $\xv\in \Rb^D$ will give:
\begin{align}
    &\sum_{p=1}^{P}\sigma^2_p k_p\left( \wv^{\top}_{p}\xv,\wv^{\top}_{p}\xv^{\prime} \right)\nonumber \\
    = & \sum_{p=1}^{P} \sigma^2_p\exp \left( -  \frac{\sum_{d=1}^{D} \left| w_{p,d}\left( x_{d}-x_{d}^{\prime} \right) \right|}{\tilde{\theta}} \right)\nonumber \\
    = & \sum_{p=1}^{P} \prod_{d=1}^{D} \sigma^2_p\exp \left( - \frac{\left| x_{d}-x_{d}^{\prime} \right|}{\tilde{\theta} / \left| w_{p,d}\right| } \right)\nonumber \\
    = & \sum_{p=1}^{P} \prod_{d=1}^{D} \sigma^2_p\exp \left( - \frac{\left| x_{d}-x_{d}^{\prime} \right|}{\theta_{p,d}} \right),
\end{align}
This still leads to an additive Laplace kernel but with adaptive lengthscale $\theta_{p,d}$ for base kernels. The resulting kernel also retains \emph{sparse} Cholesky decomposition by the properties of Markov kernels so that the complexity of inference is $\Oc(M)$.

\section{INFERENCE OF PREDICTIVE DISTRIBUTION}
\label{sec:uq of inference}
Given an input $\xv \in \Rb^D$, the prediction of the DAK model can be written in the following equation according to \cref{eq:DAK prediction}: 
\begin{align}
    \tilde{f}_{\xv}
    &= \sum_{p=1}^{P}
    \sigma_p \Big(
        \phi(h_{\psi}^{[p]}(\xv)) \zv_p
    \Big) + \mu \nonumber\\
    &= \sum_{p=1}^{P}
    \sigma_p \Big(
        \bm{\phi}_{p}^{\top} \zv_p
    \Big) + \mu,
\end{align}
where $\bm{\phi}_{p}^{\top}:=\phi(h_{\psi}^{[p]}(\xv)) \in \Rb^{1 \times M}$
% , $\mu_p:=\mu_p(h_{\psi}^{[p]}(\xv)) \in \Rb$
. We assume the variational distribution over the independent Gaussian weights $\zv_p \sim \Nc(\bm{m}_{\zv_p}, \Sv_{\zv_p})$ and the bias $\mu \sim \Nc(m_{\mu}, \sigma_{\mu}^2)$. Then it's straighforward to deduce that
\begin{align}
    \bm{\phi}_{p}^{\top} \zv_p + \mu 
    &\sim
    \Nc\left(
    \bm{\phi}_{p}^{\top} \bm{m}_{\zv_p} + m_{\mu},\hspace{0.2em}
    \bm{\phi}_{p}^{\top} \Sv_{\zv_p} \bm{\phi}_{p} + \sigma_{\mu}^2
    \right), \\
    \sigma_p \left(
    \bm{\phi}_{p}^{\top} \zv_p 
    \right) + \mu
    & \sim
    \Nc\left(
    \sigma_p ( \bm{\phi}_{p}^{\top} \bm{m}_{\zv_p} )+ m_{\mu} ,\hspace{0.2em}
    \sigma_p^2( \bm{\phi}_{p}^{\top} \Sv_{\zv_p} \bm{\phi}_{p}) + \sigma_{\mu}^2
    \right), \\
    \tilde{f}_{\xv} = 
    \sum_{p=1}^{P}
    \sigma_p \left(
    \bm{\phi}_{p}^{\top} \zv_p
    \right) + \mu
    & \sim
    \Nc\left(
    \sum_{p=1}^{P}
    \sigma_p ( \bm{\phi}_{p}^{\top} \bm{m}_{\zv_p}) + m_{\mu} ,\hspace{0.2em}
    \sum_{p=1}^{P}
    \sigma_p^2( \bm{\phi}_{p}^{\top} \Sv_{\zv_p} \bm{\phi}_{p} ) + \sigma_{\mu}^2
    \right).
\end{align}
Therefore, we obtain the predictive distribution of the $\tilde{f}(\xv)$ at the point $\xv \in \Rb^D$ and its mean and variance are given by:
\begin{subequations}
\label{eq:dak inference closed form}
\begin{align}
    \Eb\left[ \tilde{f}_{\xv} \right]
        = \sum_{p=1}^{P}
        \sigma_p ( \bm{\phi}_{p}^{\top} \bm{m}_{\zv_p}) + m_{\mu},
\end{align}
\begin{align}
    \text{Var}\left[ \tilde{f}_{\xv} \right]
        =\sum_{p=1}^{P}
        \sigma_p^2( \bm{\phi}_{p}^{\top} \Sv_{\zv_p} \bm{\phi}_{p}) + \sigma_{\mu}^2.
\end{align}
\end{subequations}
% \begin{subequations}
% \label{eq:dak inference closed form}
%     \begin{align}
%         \Eb\left[ \tilde{f}(\xv) \right]
%         = \sum_{p=1}^{P}
%         \sigma_p ( \bm{\phi}_{p}^{\top} \bm{m}_{\zv_p} + m_{\mu_p} ),
%     \end{align}
%     \begin{align}
%         \text{Var}\left[ \tilde{f}(\xv) \right]
%         =\sum_{p=1}^{P}
%         \sigma_p^2( \bm{\phi}_{p}^{\top} \Sigma_{\zv_p} \bm{\phi}_{p} + \sigma_{\mu_p}^2).
%     \end{align}
% \end{subequations}


\section{TRAINING OF VARIATIONAL INFERENCE}
\label{sec:training}
Given the dataset $\mathcal{D}=\{ \Xv, \yv \}$ where $\Xv:=\{ \xv_i \}_{i=1}^N$, $\yv=(y_1,\ldots,y_N)^{\top}$, $\xv_i \in \Rb^D$, $y_i\in\Rb$, the prediction $\tilde{f}_{\Xv}\in \Rb^N$ of DAK is given by all the parameters $\bm{\theta}=\left\{ \psi, \bm{\sigma} \right\}$, $\bm{\eta}=\left\{ \{ \mv_{\zv_{p}},\Sv_{\zv_{p}}\}_{p=1}^{P} , \{m_{\mu},\sigma_{\mu} \} \right\}$ according to \cref{eq:DAK prediction}:
\begin{align}
    \tilde{f}_{\Xv}:= \tilde{f}(\Xv; \bm{\theta}, \bm{\eta})
    = \sum_{p=1}^{P}
    \sigma_p \Big(
        \phi(h_{\psi}^{[p]}(\Xv)) \zv_p
    \Big) + \mu,
\end{align}
where $\zv_{p} \sim \mathcal{N} (\bm{m}_{\zv_p} ,\Sv_{\zv_p})$, $p=1,\ldots,P$, and $\mu \sim \mathcal{N} ( m_{\mu},\sigma^2_{\mu} )$ are variational variables $\Theta_{\text{var}}$ parameterized by $\bm{\eta}$. The variational distribution is denoted by $q_{\bm{\eta}}(\Theta_{\text{var}})= q(\mu)\prod_{p=1}^{P} q(\zv_{p}) = \Nc ( m_{\mu} ,\sigma_{\mu}^2 )\prod_{p=1}^{P} 
\Nc ( \bm{m}_{\zv_p} ,\Sv_{\zv_p} )$, and the variational prior is denoted by $p(\Theta_{\text{var}})$.

We consider the KL divergence between $q_{\bm{\eta}}(\Theta_{\text{var}})$ and the true posterior $p(\Theta_{\text{var}}\vert \yv, \Xv, \bm{\theta})$:
\begin{align}
& \qquad \text{KL} \left[ q_{\bm{\eta}}(\Theta_{\text{var}}) \| p(\Theta_{\text{var}} \vert \yv,\Xv, \bm{\theta} ) \right] \nonumber \\
= & \int q_{\bm{\eta}}(\Theta_{\text{var}} )\log \frac{q_{\bm{\eta}}(\Theta_{\text{var}} )}{p(\Theta_{\text{var}} \vert \yv,\Xv,\bm{\theta} )} d\Theta_{\text{var}} \nonumber \\
= & \int q_{\bm{\eta}}(\Theta_{\text{var}} )\log \frac{q_{\bm{\eta}}(\Theta_{\text{var}} )p(\yv \vert \Xv,\bm{\theta})}{p(\yv \vert \Xv,\bm{\theta} ,\Theta_{\text{var}} )p(\Theta_{\text{var}} )} d\Theta_{\text{var}} \nonumber \\
= & \int q_{\bm{\eta}}(\Theta_{\text{var}} )\log \frac{q_{\bm{\eta}}(\Theta_{\text{var}} )}{p(\Theta_{\text{var}} )} d\Theta_{\text{var}} -\int q_{\bm{\eta}}(\Theta_{\text{var}} )\log p(\yv \vert \tilde{f}_{\Xv} )d\Theta_{\text{var}} +\log p(\yv\vert \Xv,\bm{\theta}).
\end{align}
Using the fact that $\text{KL}[\cdot \| \cdot] \geq 0$, we have
\begin{align}
\label{eq:variational lower bound}
    \log p(\yv\vert \Xv,\bm{\theta}) & \geq \int q_{\bm{\eta}}(\Theta_{\text{var}} )\log p(\yv \vert \tilde{f}_{\Xv} )d\Theta_{\text{var}} - \text{KL} \left[ q_{\bm{\eta}}(\Theta_{\text{var}} ) \| p(\Theta_{\text{var}}) \right] \nonumber \\
    & = \Eb_{q_{\bm{\eta}}(\Theta_{\text{var}} )} \left[ \log p(\yv \vert \tilde{f}_{\Xv} ) \right] - \text{KL} \left[ q_{\bm{\eta}}(\Theta_{\text{var}} ) \| p(\Theta_{\text{var}}) \right].
\end{align}

\paragraph{Full-training.}
Firstly, we present the joint training of $\bm{\theta}$ and $\bm{\eta}$. The most common approach optimizes the marginal log-likelihood (the left-hand side of \cref{eq:variational lower bound}):
\begin{align}
    \bm{\theta}^{\ast} &=\argmax_{\bm{\theta}} \log p(\yv\vert \Xv,\bm{\theta} ) \\
    &= \argmax_{\bm{\theta}} \log \int p\left( y\vert X,\bm{\theta},\Theta_{\text{var}} \right) p(\Theta_{\text{var}})d\Theta_{\text{var}},
\end{align}
which involves intractable integral in some tasks such as classification. Instead, we optimize the variational lower bound (the right-hand side of \cref{eq:variational lower bound}):
\begin{align}
    \Theta^{\ast} := \argmax_{\bm{\theta},\bm{\eta}} \mathcal{L}(\bm{\theta},\bm{\eta}) =\argmax_{\bm{\theta},\bm{\eta}}\left\{ E_{q_{\bm{\eta}}(\Theta_{\text{var}} )}\left[ \log p(\yv|\tilde{f}_{\Xv} ) \right] -\text{KL} \left[ q_{\bm{\eta}}(\Theta_{\text{var}} )\| p(\Theta_{\text{var}} ) \right] \right\}.
\end{align}

\paragraph{Fine-tuning.}
An alternative training approach is to firstly pre-train the deterministic parameters of feature extractor by standard neural network training, with mean squared error for regression or cross-entropy for classification as the loss function, and then fine-tune the last layer additive GP with fixed features. The objective function is identical to \cref{eq:elbo}, but $\bm{\theta}$ is learned during the pre-training step and is no longer optimized during fine-tuning.


\section{ELBO}%{DERIVATION OF ELBO}
\label{sec:elbo}
\subsection{Assumptions}
Consider the model $y_i = \tilde{f}(\xv_i) + \epsilon_i$ with the i.i.d. noise $\epsilon_i \overset{\text{i.i.d.}}{\sim} \Nc(0, \sigma_{f}^2)$ and $\tilde{f} : \Rb^D \rightarrow \Rb$ is defined in \cref{eq:DAK prediction}. The training dataset is $\mathcal{D} = \{ \Xv, \yv \}$ where $\Xv:=\{ \xv_i \}_{i=1}^N$, $\yv=(y_1,\ldots,y_N)^{\top}$, $\xv_i \in \Rb^D$, $y_i\in\Rb$. $\Theta_{\text{var}}:= \{ \mu ,\{ \zv_{p}\}_{p=1}^{P} \}$ are the variational random variables consisting of Gaussian weights and bias of $P$ units, $\psi$ are the parameters of the NN, $\bm{\sigma}:=(\sigma_1, \ldots, \sigma_p)^{\top}$ are the scale parameters of base GP layers. The variational distributions are $q(\mu)=\Nc(m_{\mu}, \sigma_{\mu}^2)$, $q(\zv_p)=\Nc(\bm{m}_{\zv_p}, \Sv_{\zv_p})$ and the variational priors are $p(\mu)=\Nc(\check{m}_{\mu} ,\check{\sigma}^2_{\mu})$, $p(\zv_p)=\Nc(\check{\bm{m}}_{\zv_p} ,\check{\Sv}_{\zv_p})$. Note that $\Sv_{\zv_p}\in\Rb^{M \times M}$ is a diagonal covariance matrix due to the independence of $\zv_p$, $M$ is the number of inducing points $\Uv$ defined in \cref{eq:GPlayer}, and $\bm{m}_{\zv_p} \in \Rb^M$, $m_{\mu} \in \Rb$, $\sigma_{\mu}^2 \in \Rb$. We derive the ELBO in VI to learn the preditive posterior over the variational variables $\Theta_{\text{var}}:= \{ \mu ,\{ \zv_{p}\}_{p=1}^{P} \}$ parameterized by $\bm{\eta}:=\left\{ \{ \mv_{\zv_{p}},\Sv_{\zv_{p}}\}_{p=1}^{P} , \{m_{\mu},\sigma_{\mu} \} \right\}$, and optimize the deterministic parameters $\bm{\theta}:=\{\psi, \bm{\sigma}\}$.

\subsection{Expected Log Likelihood}
\paragraph{Closed Form}
The \emph{expected log likelihood}, which is the first term in ELBO defined in \cref{eq:elbo}, is given by 
\begin{align}
    {\Eb}_{q_{\bm{\eta}}(\Theta_{\text{var}})} \left[ \log \text{Pr} (\yv \vert \tilde{f}_{\Xv} ) \right]
    &= {\Eb}_{q_{\bm{\eta}}(\Theta_{\text{var}})} \left[ 
    \log \prod_{i=1}^{N} 
    p (y_i \vert \tilde{f}_{\xv_i} )
    \right] \nonumber\\
    &= \sum_{i=1}^{N} 
    {\Eb}_{q_{\bm{\eta}}(\Theta_{\text{var}})} \left[ 
    \log
    p (y_i \vert \tilde{f}_{\xv_i} )
    \right] \nonumber\\
    &= \sum_{i=1}^{N} 
    {\Eb}_{q_{\bm{\eta}}(\Theta_{\text{var}})} \left[ 
    \log
    \Nc( \tilde{f}_i,\hspace{0.2em} \sigma_{f}^2 )
    \right] \nonumber\\
    &= \sum_{i=1}^{N} 
    {\Eb}_{q_{\bm{\eta}}(\Theta_{\text{var}})} \left[ 
    \log \left(
    (2\pi \sigma_{f}^2)^{-\frac{1}{2}}
    \exp\left\{  
        -\frac{ (y_i - \tilde{f}_i)^2 }{2 \sigma_{f}^2}
    \right\}
    \right)
    \right] \nonumber\\
    &= \sum_{i=1}^{N} 
    {\Eb}_{q_{\bm{\eta}}(\Theta_{\text{var}})} \left[
    -\frac{1}{2} \log(2\pi) 
    - \frac{1}{2}\log(\sigma_{f}^2)
    - \frac{1}{2 \sigma_{f}^2}
    (y_i - \tilde{f}_i)^2
    \right] \nonumber\\
    &= - \frac{N}{2} \log(2\pi)
    - \frac{N}{2} \log(\sigma_{f}^2)
    - \frac{1}{2 \sigma_{f}^2}
    \sum_{i=1}^{N}
    {\Eb}_{q_{\bm{\eta}}(\Theta_{\text{var}})} \left[
    (y_i - \tilde{f}_i)^2
    \right] \nonumber\\
    &= - \frac{N}{2} \log(2\pi)
    - \frac{N}{2} \log(\sigma_{f}^2)
    - \frac{1}{2 \sigma_{f}^2}
    \sum_{i=1}^{N} \left(
    \left({\Eb}_{q(\Theta_{\text{var}})} \left[
    (y_i - \tilde{f}_i)
    \right] \right)^2
    + \text{Var}_{q(\Theta_{\text{var}})} \left[
    (y_i - \tilde{f}_i)
    \right]
    \right) \label{eq:evidence halfway},
\end{align}
where
\begin{align}
    \tilde{f}_i
    % \mu_{\tilde{f}_i} &:= \tilde{f}(\xv_i;\Theta_{\text{var}}, \Theta_{\text{det}} ) \nonumber\\
    &= \sum_{p=1}^{P} \sigma_p \Big(
    \begingroup
        \color{blue}
        \underbracket{
            \color{black}
            \phi(h_{\psi}^{[p]}(\xv_i))
        }_{\color{blue}
            :=\bm{\phi}_{i,p}^{\top} \in \Rb^{1 \times M}
        }
    \endgroup
    \zv_p
    \Big)
    + \mu
    % \begingroup
    %     \color{blue}
    %     \underbracket{
    %         \color{black}
    %         \mu_{p}(h_{\psi}^{[p]}(\xv_i))
    %     }_{\color{blue}
    %         :=\mu_{i,p} \in \Rb
    %     }
    % \endgroup 
    \nonumber\\
    &= \sum_{p=1}^{P} \sigma_p \left(
    \bm{\phi}_{i,p}^{\top} \zv_p 
    \right) + \mu.
\end{align}
Recall that the variational assumptions $q(\zv_p)=\Nc(\bm{m}_{\zv_p}, \Sv_{\zv_p})$ and $q(\mu)=\Nc(m_{\mu}, \sigma_{\mu}^2)$, we can infer that
\begin{align}
    \bm{\phi}_{i,p}^{\top} \zv_p + \mu 
    &\sim
    \Nc\left(
    \bm{\phi}_{i,p}^{\top} \bm{m}_{\zv_p} + m_{\mu},\hspace{0.2em}
    \bm{\phi}_{i,p}^{\top} \Sv_{\zv_p} \bm{\phi}_{i,p} + \sigma_{\mu}^2
    \right), \\
    \sigma_p \left(
    \bm{\phi}_{i,p}^{\top} \zv_p 
    \right) + \mu
    & \sim
    \Nc\left(
    \sigma_p ( \bm{\phi}_{i,p}^{\top} \bm{m}_{\zv_p} ) + m_{\mu},\hspace{0.2em}
    \sigma_p^2( \bm{\phi}_{i,p}^{\top} \Sv_{\zv_p} \bm{\phi}_{i,p} ) + \sigma_{\mu}^2
    \right), \\
    \tilde{f}_i = 
    \sum_{p=1}^{P}
    \sigma_p \left(
    \bm{\phi}_{i,p}^{\top} \zv_p 
    \right)+ \mu
    & \sim
    \Nc\left(
    \sum_{p=1}^{P}
    \sigma_p ( \bm{\phi}_{i,p}^{\top} \bm{m}_{\zv_p} )+ m_{\mu},\hspace{0.2em}
    \sum_{p=1}^{P}
    \sigma_p^2( \bm{\phi}_{i,p}^{\top} \Sv_{\zv_p} \bm{\phi}_{i,p} ) + \sigma_{\mu}^2
    \right), \\
    y_i - \tilde{f}_i
    & \sim 
    \Nc\left(
    y_i - 
    \sum_{p=1}^{P}
    \sigma_p ( \bm{\phi}_{i,p}^{\top} \bm{m}_{\zv_p} ) -m_{\mu},\hspace{0.2em}
    \sum_{p=1}^{P}
    \sigma_p^2( \bm{\phi}_{i,p}^{\top} \Sv_{\zv_p} \bm{\phi}_{i,p} ) + \sigma_{\mu}^2
    \right).
\end{align}
Therefore, 
\begin{subequations}\label{eq:exp and var in evidence}
    \begin{align}
        \left({\Eb}_{q(\Theta_{\text{var}})} \left[
        (y_i - \tilde{f}_i)
        \right] \right)^2
        = \left(
         y_i - 
        \sum_{p=1}^{P}
        \sigma_p ( \bm{\phi}_{i,p}^{\top} \bm{m}_{\zv_p} ) -m_{\mu}
        \right)^2,
    \end{align}
    \begin{align}
        \text{Var}_{q(\Theta_{\text{var}})}
        \left[
        (y_i - \tilde{f}_i)
        \right]
        = \sum_{p=1}^{P}
        \sigma_p^2( \bm{\phi}_{i,p}^{\top} \Sv_{\zv_p} \bm{\phi}_{i,p} ) + \sigma_{\mu}^2.
    \end{align}
\end{subequations}
By applying \cref{eq:exp and var in evidence} to \cref{eq:evidence halfway}, we derive the analytical formula for the expected evidence, expressed as
\begin{align}
    {\Eb}_{q_{\bm{\eta}}(\Theta_{\text{var}})} \left[ \log \text{Pr} (\yv \vert \tilde{f}_{\Xv} ) \right]
    &= - \frac{N}{2} \log(2\pi)
    - \frac{N}{2} \log(\sigma_{f}^2) \nonumber\\
    &- \frac{1}{2 \sigma_{f}^2}
    \sum_{i=1}^{N} \left(
        \Big(
         y_i - 
        \sum_{p=1}^{P}
        \sigma_p ( \bm{\phi}_{i,p}^{\top} \bm{m}_{\zv_p} ) -m_{\mu}
        \Big)^2
        + \sum_{p=1}^{P}
        \sigma_p^2( \bm{\phi}_{i,p}^{\top} \Sv_{\zv_p} \bm{\phi}_{i,p} )+ \sigma_{\mu}^2
    \right). \label{eq:evidence final}
\end{align}

\paragraph{Monte Carlo Approximation}
For comparison, we provide the equation for computing the Monte Carlo estimate of the ELBO in the paragraph that follows.
\begin{align}
    {\Eb}_{q_{\bm{\eta}}(\Theta_{\text{var}})} \left[ \log \text{Pr} (\yv \vert \tilde{f}_{\Xv} ) \right]
    % &= {\Eb}_{q(\Theta)} \left[ 
    % \log \prod_{i=1}^{N} 
    % p (y_i \vert \xv_i,\Theta, \psi, \bm{\sigma})
    % \right] \nonumber\\
    &= \sum_{i=1}^{N} 
    {\Eb}_{q_{\bm{\eta}}(\Theta_{\text{var}} )} \left[ 
    \log
    p (y_i \vert \tilde{f}_{\xv_i} )
    \right] \nonumber\\
    & \approx \sum_{i=1}^{N}
    \frac{1}{S}
     \sum_{s=1}^{S}
    \log
    p (y_i \vert \xv_i,\tilde{\Theta}^{(s)}_{\text{var}}, \bm{\theta} ) \nonumber\\
    &= \frac{1}{S} \sum_{i=1}^{N} 
    \sum_{s=1}^{S} 
    \log
    \Nc(y_i \left\vert\right. \tilde{f}_{i}^{(s)},\hspace{0.2em} \sigma_{f}^2 )
    \nonumber\\
    &= \frac{1}{S} \sum_{i=1}^{N} 
    \sum_{s=1}^{S} 
    \log \left(
    (2\pi \sigma_{f}^2)^{-\frac{1}{2}}
    \exp\left\{  
        -\frac{ (y_i - \tilde{f}_{i}^{(s)})^2 }{2 \sigma_{f}^2}
    \right\}
    \right)
    \nonumber\\
    &= \frac{1}{S} \sum_{i=1}^{N} 
    \sum_{s=1}^{S} \left(
    -\frac{1}{2} \log(2\pi) 
    - \frac{1}{2}\log(\sigma_{f}^2)
    - \frac{1}{2 \sigma_{f}^2}
    (y_i - \tilde{f}_{i}^{(s)})^2
    \right) \nonumber\\
    &= - \frac{N}{2} \log(2\pi)
    - \frac{N}{2} \log(\sigma_{f}^2)
    - \frac{1}{2 \sigma_{f}^2}
    \sum_{i=1}^{N}
    \frac{1}{S} \sum_{s=1}^{S}
    (y_i - \tilde{f}_{i}^{(s)})^2, \label{eq:evidence halfway mc approx}
\end{align}
where $S$ is the number of Monte Carlo samples, $\{  \tilde{\mu}^{(s)} ,\{ \tilde{\zv}_{p}^{(s)} \}_{p=1}^{P} \} := \tilde{\Theta}^{(s)}_{\text{var}}$ are the $s$-th Monte Carlo samplings over the variational parameters $\Theta_{\text{var}}$ and $\tilde{\Theta}^{(s)}_{\text{var}} \sim q_{\bm{\eta}}(\Theta_{\text{var}})$, $\tilde{f}_{i}^{(s)}$ is given as follows:
\begin{align}
    \tilde{f}_{i}^{(s)} &:= \tilde{f}(\xv_i;\tilde{\Theta}^{(s)}_{\text{var}},\bm{\theta} ) \nonumber\\
    &= \sum_{p=1}^{P} \sigma_p \Big(
    \begingroup
        \color{blue}
        \underbracket{
            \color{black}
            \phi(h_{\psi}^{[p]}(\xv_i))
        }_{\color{blue}
            :=\bm{\phi}_{i,p}^{\top} \in \Rb^{1 \times M}
        }
    \endgroup
    \tilde{\zv}_p^{(s)} 
    \Big) + \tilde{\mu}^{(s)} \nonumber\\
    &= \sum_{p=1}^{P} \sigma_p \left(
    \bm{\phi}_{i,p}^{\top} \tilde{\zv}_p^{(s)} 
    \right)+ \tilde{\mu}^{(s)}. \label{eq:mc approx mean}
\end{align}
Therefore, we plug \cref{eq:mc approx mean} into \cref{eq:evidence halfway mc approx} and get the the Monte Carlo estimate of the ELBO written in the following formula:
\begin{align}
    {\Eb}_{q_{\bm{\eta}}(\Theta_{\text{var}})} \left[ \log \text{Pr} (\yv \vert \tilde{f}_{\Xv} ) \right]
    &\approx
    - \frac{N}{2} \log(2\pi)
    - \frac{N}{2} \log(\sigma_{f}^2)
    - \frac{1}{2 \sigma_{f}^2}
    \sum_{i=1}^{N}
    \frac{1}{S} \sum_{s=1}^{S}
    \Big(y_i - 
    \sum_{p=1}^{P} \sigma_p \left(
    \bm{\phi}_{i,p}^{\top} \tilde{\zv}_p^{(s)} 
    \Big)- \tilde{\mu}^{(s)}
    \right)^2, \label{eq:evidence final mc approx} \\
    \tilde{\zv}_p^{(s)} &\sim \Nc(\bm{m}_{\zv_p}, \Sv_{\zv_p}),\qquad
    \tilde{\mu}^{(s)} \sim \Nc(m_{\mu}, \sigma_{\mu}^2).
\end{align}


\subsection{KL Divergence}
Since we place Gaussian assumptions over the variational parameters $\Theta_{\text{var}}$,  the \emph{KL divergence}, which is the second term in ELBO defined in \cref{eq:elbo}, is then given by
\begin{align}
    \text{KL} \left[ q(\Theta_{\text{var}} ) \| p(\Theta_{\text{var}}) \right]
    &= \text{KL} \left[ q( \mu ,\{ \zv_{p}\}_{p=1}^{P} ) \Vert p( \mu ,\{ \zv_{p}\}_{p=1}^{P}) \right] \nonumber\\
    & =  
    \text{KL} \left[ q(\mu) \Vert p(\mu) \right] 
    + \sum_{p=1}^{P} 
    \text{KL} \left[ q(\zv_{p}) \Vert p(\zv_{p}) \right],
\end{align}

\begin{align}
     \text{KL} \left[ q(\mu) \Vert p(\mu) \right]
     = \frac{1}{2} \left(
     \frac{\sigma_{\mu}^2}{\check{\sigma}_{\mu}^2} 
     + \frac{(m_{\mu} - \check{m}_{\mu})^2}{\check{\sigma}_{\mu}^2} 
     -\log\left( \frac{\sigma_{\mu}^2}{\check{\sigma}_{\mu}^2} \right)
     -1
     \right),
\end{align}

\begin{align}
    \text{KL} \left[ q(\zv_{p}) \Vert p(\zv_{p}) \right]
    = \frac{1}{2} \sum_{i=1}^{M} \left(
     \frac{[\Sv_{\zv_p}]_{ii}}{[\check{\Sv}_{\zv_p}]_{ii}} 
     + \frac{([\bm{m}_{\zv_p}]_{i} - [\check{\bm{m}}_{\zv_p}]_i)^2}{[\check{\Sv}_{\zv_p}]_{ii}}
     -\log\left( 
     \frac{[\Sv_{\zv_p}]_{ii}}{[\check{\Sv}_{\zv_p}]_{ii}}  
     \right)
     -1
     \right),
\end{align}
where $[\Sv_{\zv_p}]_{ii}$ is the $(i,i)$-th element of the diagonal covariance matrix $\Sv_{\zv_p} \in \Rb^{M \times M}$, $[\bm{m}_{\zv_p}]_{i}$ is the $i$-th element of the mean vector $\bm{m}_{\zv_p} \in \Rb^M$, the approximated posteriors are $q(\mu)=\Nc(m_{\mu}, \sigma_{\mu}^2)$, $q(\zv_p)=\Nc(\bm{m}_{\zv_p}, \Sv_{\zv_p})$ and the priors are $p(\mu)=\Nc(\check{m}_{\mu} ,\check{\sigma}^2_{\mu})$, $p(\zv_p)=\Nc(\check{\bm{m}}_{\zv_p} ,\check{\Sv}_{\zv_p})$.

% \subsection{Performance Comparison}
% \label{sec:toy exp compare}
% We compare the perforamce of computing the ELBO in \cref{eq:elbo} by using closed form in \cref{eq:evidence final} and using Monte Carlo approximation in \cref{eq:evidence final mc approx} in a toy example.
% \textcolor{red}{Table or Figure to add if time available}


\subsection{Limitations of the Closed-Form ELBO}

The closed-form ELBO is only applicable to regression problems. In classification, applying the softmax function to $\tilde{f}(\xv;\bm{\theta}, \bm{\eta})$ results in a non-analytic predictive distribution, meaning the ELBO must still be computed via Monte Carlo sampling during training. Similarly, the closed-form expressions for the predictive mean and variance, as provided in \cref{eq:dak inference closed form} in \Cref{sec:uq of inference}, are not applicable to classification but only apply to regression problems.


\section{COMPUTATIONAL COMPLEXITY}
\label{sec:complexity}
In this section, we discuss the computational complexity of various DKL models compared to the proposed DAK method, focusing on the GP layer as the most computationally demanding component. \Cref{tab:complexity supp} shows the computational complexity of our model compared to other state-of-the-art GP and DKL methods.

\begin{table}[ht]
    \caption{Computational complexity of the DKL models for $N$ training points. The reported training complexity is for one iteration. $\hat{M}$ is the number of inducing points in SVGP and KISS-GP, while $M$ is the size of induced grids in DAK, $M < \hat{M}$. $S$ is the number of Monte Carlo samples, $B$ is the size of mini-batch, $D_w$ is the dimension of the NN outputs in DKL, $P$ is the dimension of the outputs after applying linear transformations to the NN outputs in the proposed DAK model. DAK-MC refers to the DAK model using Monte Carlo approximation, while DAK-CF refers to the DAK model using closed-form inference and ELBO.}
    \centering
    \begin{tabular}{lcc}
    \toprule[1pt]
                  & \textbf{Inference}       & \textbf{Training} (per iteration) \\
    \midrule[0.5pt]
    NN + SVGP     & $\Oc(\hat{M}^2 N)$    & $\Oc( S D_w MB + \hat{M}^3)$ \\
    NN + KISS-GP  & $\Oc(D_w \hat{M}^{1+\frac{1}{D_w}})$  & $\Oc(S D_w MB + D_w \hat{M}^{\frac{3}{D_w}})$ \\
    DAK-MC (ours) & $\Oc(SM)$       & $\Oc(SPMB + PM)$   \\
    DAK-CF (ours) & $\Oc(M)$        & $\Oc(PMB + PM)$    \\
    \bottomrule[1pt]
    \end{tabular}
    \label{tab:complexity supp}
\end{table}

\paragraph{Inference Complexity.}
In inference based on induced approximation, computing the multiplication of the inverse of the covariance matrix $k(\Uv, \Uv)$ and a vector takes $\Oc(\hat{M}^2N)$ time for $\hat{M}$ inducing points $\Uv$ and $N$ training points when using SVGP. This cost is reduced by KISS-GP to $\Oc(D \hat{M}^{1+\frac{1}{D}})$ by decomposing the covariance matrix into a Kronecker product of $D$ one-dimensional covariance matrices of the inducing points: $k(\Uv, \Uv) = \bigotimes_{d=1}^{D} k(\Uv^{[d]}, \Uv^{[d]})$. Despite the significant reduction on complexity, it requires inducing points $\Uv$ arranged on a Cartesian grid of size $\hat{M} = \prod_{d=1}^{D} \hat{M}_d$, where $\hat{M}_d$ is the number of inducing points in the $d$-th dimension. In high-dimensional spaces, fixing $\hat{M}$ leads to very small $\hat{M}_d$ per dimension, which can degrade model performance. To address this, we propose the DAK model via sparse finite-rank approximation, which employs an additive Laplace kernel for GPs. The inverse Cholesky factor $\Lv_{\Uv}^{\top}$ for one-dimensional induced grids $\Uv$ of size $M$, where $M < \hat{M}$, as defined in \cref{eq:GPlayer}, is sparse and can be computed in $\Oc(M)$ time.

\paragraph{Training Complexity.}
In training, VI requires computing the ELBO as described in \cref{eq:elbo}, which consists of two terms: the \emph{expected log likelihood} and the \emph{KL divergence} between the variational distributions and priors. 

1) The \emph{expected log likelihood} is usually approximated via Monte Carlo sampling at a cost of $\Oc(S N_{\Theta} N)$, where $S$ is the number of Monte Carlo samples, $N_{\Theta}$ is the total number of variational parameters $\Theta_{\text{var}}$, and $N$ is the number of training points. This complexity can be reduced to $\Oc(S N_{\Theta} B)$ by applying stochastic variational inference with a mini-batch of size $B \ll N$. For DKL models using SVGP and KISS-GP, $\Theta_{\text{var}}$ are inducing variables, and the expectation does not have a closed form, requiring Monte Carlo sampling. In contrast, in the proposed DAK model, $\Theta_{\text{var}}= \{ \{ \zv_{p}\}_{p=1}^{P}, \mu \}$ consists of independent Gaussian weights $\zv_p\in \Rb^M$ and bias $\mu$. This allows us to derive an analytical form for this term, as shown in \cref{eq:evidence final} in \Cref{sec:elbo}, reducing the computational cost to $\Oc(N_{\Theta} B) = \Oc(PM B)$ when using a mini-batch of size $B$.

2) The \emph{KL divergence} between two Gaussian distributions can be computed in closed form. This leads to a linear time complexity of $\Oc(N_{\Theta})$ if the parameters $\Theta_{\text{var}}$ are independent, or cubic time $\Oc(N_{\Theta}^3)$ if they are fully correlated. In SVGP and KISS-GP, $\Theta_{\text{var}}$ represents fully correlated Gaussian distributed inducing variables, so computing the KL divergence takes $\Oc(\hat{M}^3)$ for SVGP. In KISS-GP, this can be reduced to $\Oc(D \hat{M}^{\frac{3}{D}})$ using fast eigendecomposition of Kronecker matrices. In the DAK model, the weights $\{\zv_p\}_{p=1}^{P}$ as defined in \cref{eq:GPlayer} are independent Gaussian random variables, allowing the KL divergence to be computed in $\Oc(N_{\Theta}) = \Oc(PM)$ time, where $P$ is the number of base GP layers.


\section{ADDITIONAL DISCUSSIONS}

Although interpretability is one advantage of additive models, the main motivation for replacing a GP layer with an additive GP layer in our work is to handle high-dimensional data. When the input dimension is low, it is reasonable that GPs are superior to additive GPs since the additive kernel is an approximated and restrictive kernel. However, when the input dimension increases, the computational complexity grows considerably even in GPs with sparse approximation. For example, in DKL, the output dimension of NN encoder is usually chosen as small as 2, while in pixel data experiments, DKL cannot handle the computation associated with the dimensionality when the output dimension of ResNet is 512 or more. Although DKL is superior in low-dimensional and simple cases, we view additive structure as a necessary component to achieve scalability and good performance with high-dimensional data.

\subsection{Why choosing the induced grids instead of learning the inducing points?}

From an approximation accuracy point of view, there are two separate strategies to increase the accuracy. The first one is to learn the inducing point locations. The second one, however, is to simply increase the number of inducing points on a pre-specified finer grid. The second method is much easier to implement and has a theoretical guarantee by the GP regression theory: as the inducing points become dense in the input region, the approximation will become exact. In contrast, the first approach does not have such a favorable theoretical guarantee. 

The second approach would become difficult to use for many existing methodologies as in general the computational cost would scale as $\mathcal{O}(M^3)$ with $M$ inducing points, which is particularly problematic in high dimensions. 
% The first approach can be viewed as a compromise in those situations, and that is why many existing methods chose to learn the locations of the inducing points instead.
This difficulty is resolved by additive GPs, since approximating an additive GP boils down to approximating one dimensional GPs, which can be accomplished by using a set of pre-specified inducing points on a fine grid in 1-D. One major benefit of the proposed methodology is that the computation now scales at $\mathcal{O}(M)$, enabled by the Markov kernel and the additive kernel. Therefore, a large number of inducing points can be used in an efficient way. 

The proposed method also has several additional benefits: 1) It can decouple to some extent the neural network component and GP component by avoiding learning the inducing points, which may help reduce overfitting/overconfidence; 2) The equivalence to BNN holds exactly with the fixed inducing points, whereas for learned inducing points, this BNN equivalence breaks down, and the proposed computation/training framework would not be possible to carry through; 3) It can simplify the overall optimization since there is no need to learn the inducing points.

\subsection{Limitations and future directions}

Generally, a finer grid will lead to better approximations, but the number of parameters to be trained will also increase. Therefore, there is a trade-off between the accuracy and the computational cost that we can afford. This current work is using a specific Laplace kernel, which can utilize sparse Cholesky decomposition. More general kernels may result in more computational complexity but better representation power of the model. In addition, the current variational family is restricted under mean-field assumptions. A more general variational family, e.g. full/low-rank covariance, may lead to superior performance in some applications. 


\section{EXPERIMENTAL DETAILS}
\label{sec:expdetail}
In this section, we provide additional details regarding the experiments.

\subsection{Benchmarks for Regression}
\label{subsec:regression supp}
\paragraph{Experiment Setup}
For all models, the NN architecture is a fully connected NN with rectified linear unit (ReLU) activation function \citep{nair2010rectified} and two hidden layers containing 64 and 32 neurons, respectively, structured as $D \rightarrow 64 \rightarrow 32 \rightarrow D_w$, where $D$ is the input feature size (also the size of input $\Xv$) and $D_w$ is the output feature size. The models are evaluated with $D_w=16$, 64, and 256, respectively. The number of Monte Carlo samples is set to 8 during training and 20 during inference.

The NN is a deterministic model, and we use the negative Gaussian log-likelihood as the loss function to quantify the uncertainty of the NN outputs and compute the NLPD.

For NN+SVGP, the inducing points are set to the size of 64 in $D_w$ dimension. We implement the \texttt{ApproximateGP} model in GPyTorch \citep{gardner2018gpytorch}, defining the inducing variables as variational parameters, and use \texttt{VariationalELBO} in GPyTorch to perform variational inference and compute the loss.

SV-DKL is originally designed for classification, so for a fair comparison in regression tasks, we modify it by first applying a linear embedding layer $\Wv: \Rb^{D_w} \rightarrow \Rb^P$ with $P=16$ and normalizing the outputs to the interval $[0,1]$ for each base GP, similar to the DAK model. To adapt the additive GP layer for regression, we remove the softmax function from the model in eq. (1) of \citep{wilson2016stochastic}. Given training data $\{ \xv_i, \yv_i \}_{i=1}^{N}$, the model is modified as follows:
\begin{align}
    p(\yv_i \vert \fv_i, A) = \mathcal{A}(\fv_i)^{\top} \yv_i
\end{align}
where $\fv_i \in \Rb^P$ is a vector of independent GPs followed by a linear mixing layer $\mathcal{A}(\fv_i) = A \fv_i$, with $A \in \Rb^{C \times P}$ as the transformation matrix. Here, $C=1$ for single-task regression. For each $p$-th GP ($1 \leq p \leq P$) in the additive GP layer, the corresponding inducing variables $\uv_p$ are set to the size of 64 and treated as variational parameters for training. We use the \texttt{GridInterpolationVariationalStrategy} model with \texttt{LMCVariationalStrategy} in GPyTorch to perform KISS-GP with variational inducing variables, augmented by a linear mixing layer.

For AV-DKL, the inducing points are set to size of 64 in $D_{w}$ dimension. We implement the AV-DKL model based on the source code~\cite{matias2024amortized}.

Both DAK-MC and DAK-CF use the same additive GP layer size as SV-DKL, with $P=16$, and employ fixed induced grids $\Uv = \{1/8, 2/8, \ldots, 7/8\}$ of size 7 for each base GP, which is much smaller than that of SV-DKL.

\paragraph{Metrics}
Let $\{\xv_t, y_t\}_{t=1}^{T}$ represent a test dataset of size $T$, where $\mu_t$ and $\sigma_t^2$ are the predictive mean and variance. We evaluate model performance using two common metrics: Root Mean Squared Error (RMSE) and Negative Log Predictive Density (NLPD).

RMSE is widely used to assess the accuracy of predictions, measuring how far predictions deviate from the true target values. It is calculated as:
\begin{align}
    \text{RMSE} = \sqrt{ \frac{1}{T} \sum_{t=1}^{T}(y_t - \mu_t)^2 }.
\end{align}

NLPD is a standard probabilistic metric for evaluating the quality of a model's uncertainty quantification. It represents the negative log likelihood of the test data given the predictive distribution. For GPs, NLPD is calculated as:
\begin{align}
    \text{NLPD}
    &= - \sum_{t=1}^{T} \log p(y_t = \mu_t \vert \xv_t) \\
    &= \frac{1}{T}
    \sum_{t=1}^{T} \Big[
    \frac{(y_t - \mu_t)^2}{2\sigma_t^2} + \frac{1}{2} \log(2\pi \sigma_t^2)
    \Big].
\end{align}
Both RMSE and NLPD are widely used in the GP regression literature, where smaller values indicate better model performance.

\paragraph{Computing Infrastructure}
The experiments for regression were run on Macbook Pro M1 with 8 cores and 16GB RAM.

\subsection{Benchmarks for Classification}
\label{subsec:classification supp}
We use PyTorch \citep{paszke2019pytorch} baseline of NN models, GPyTorch \citep{gardner2018gpytorch} baseline of SVGP and SV-DKL models. In classification tasks, we apply a softmax likelihood to normalize the output digits to probability distributions. The NN is a deterministic model trained via negative log-likelihood loss, while DKL and DAK models are trained via ELBO loss. The setting of all training tasks are described in \Cref{tab:model classification} and \Cref{tab:optimizer classification}.

SVGP is originally designed for single-output regression. To make it fit for multi-output classification, we used \texttt{IndependentMultitaskVariationalStrategy} in GPyTorch to implement the multi-task \texttt{ApproximateGP} model, and use \texttt{VariationalELBO} with \texttt{SoftmaxLikelihood} in GPyTorch to perform variational inference and compute the loss. 

For SV-DKL, we employed the same \texttt{VariationalELBO} with \texttt{SoftmaxLikelihood} as the variational loss objective. \texttt{GridInterpolationVariationalStrategy} is applied within \texttt{IndependentMultitaskVariationalStrategy} to perform additive KISS-GP approximation. For each KISS-GP unit, we used $64$ variational inducing points initialized on a grid of size $[-1,1]$. 

For DAK, we implemented DAK-MC using Monte Carlo estimation given the intractable softmax likelihood. We employed fixed induced grids $\Uv=\{ -31/32, -30/32, \ldots, 30/32, 31/32 \}$ of size 63 for each base GP component.

\begin{table}[ht]
\caption{Model architectures for image classification on MNIST, CIFAR-10 and CIFAR-100.}
\centering
\resizebox{0.7\linewidth}{!}{
\begin{tabular}{l|l|ccc}
\toprule[1pt]
Model                   & Hyper-parameter          & MNIST       & CIFAR-10    & CIFAR-100   \\
\midrule[0.5pt]
\multirow{4}{*}{NN+SVGP}   & Feature extractor        & CNN         & ResNet-18   & ResNet-34   \\
                        & NN out features $D_w$         & 128         & 512         & 512         \\
                        & Embedding features $P$               & 16          & 64          & 128         \\
                        & \# inducing points $\hat{M}$      & 512         & 512         & 512         \\
                        & \# epochs       & 20         & 200         & 200         \\
                        & Training strategy      & Full-training         & Full-training         & Fine-tuning         \\
\midrule[0.5pt]
\multirow{5}{*}{SV-DKL} & Feature extractor        & CNN         & ResNet-18   & ResNet-34   \\
                        & NN out features $D_w$         & 128         & 512         & 512         \\
                        & Embedding features $P$               & 16          & 64          & 128         \\
                        & \# inducing points $\hat{M}$      & 64          & 64          & 64          \\
                        & Grid bounds              & {[}-1,1{]} & {[}-1,1{]} & {[}-1,1{]} \\
                        & \# epochs       & 20         & 200         & 200         \\
                        & Training strategy       & Full-training         & Full-training         & Fine-tuning         \\
\midrule[0.5pt]
\multirow{4}{*}{DAK}    & Feature extractor        & CNN         & ResNet-18   & ResNet-34   \\
                        & NN out features $D_w$         & 128         & 512         & 512         \\
                        & Embedding features $P$               & 16          & 64          & 128         \\
                        & \# induced interpolation $M$ & 63          & 63          & 63         \\
                        & \# epochs       & 20         & 200         & 200         \\
                        & Training strategy      & Full-training         & Full-training         & Full-training         \\
\bottomrule[1pt]
\end{tabular}

}
\label{tab:model classification}
\end{table}

\paragraph{MNIST} We used a CNN implemented in PyTorch as the feature extractor: \texttt{Conv2d}(1,32,3) $\rightarrow$ \texttt{Conv2d}(32,64,3) $\rightarrow$ \texttt{MaxPool2d}(2) $\rightarrow$ \texttt{Dropout}(0.25) $\rightarrow$ \texttt{Linear}(9216,128) $\rightarrow$ \texttt{Dropout}(0.5). To make a fair comparison, for both SV-DKL and DAK, we applied an embedding module through a linear layer that transform $128$ output features into $P=16$ base GP channels. 

\paragraph{CIFAR-10} We used a ResNet-18 as the feature extractor followed by a linear embedding layer that compressed the $512$ output features into $P=64$ base GP channels. 

\paragraph{CIFAR-100} We used a pretrained ResNet-34 as the feature extractor for SV-DKL and fine-tuned GP output layers since SV-DKL struggled to fit using full-training. For proposed DAK, we used full-training. The number of base GP channels is selected as $P=128$. 

\begin{table}[ht]
\caption{Details of training optimizer for image classification on MNIST, CIFAR-10 and CIFAR-100.}
\centering
\resizebox{0.7\linewidth}{!}{

\begin{tabular}{l|ccc}
\toprule[1pt]
Optimization      & MNIST                                                             & CIFAR-10                                                                                                  & CIFAR-100                                                                                                 \\
\midrule[0.5pt]
Optimizer         & Adadelta                                                          & SGD                                                                                                       & SGD                                                                                                       \\
Initial lr.       & 1.0                                                               & 0.1                                                                                                       & 0.1                                                                                                       \\
Weight decay      & 0.0001                                                            & 0.0001                                                                                                    & 0.0001                                                                                                    \\
Scheduler         & StepLR                                                            & CosineAnnealingLR                                                                                         & CosineAnnealingLR                                                                                         \\
\midrule[0.5pt]
Data Augmentation & MNIST                                                             & CIFAR-10                                                                                                  & CIFAR-100                                                                                                 \\
\midrule[0.5pt]
RandomCrop        & -                                                                 & size=32, padding=4                                                                                        & size=32, padding=4                                                                                        \\
HorizontalFlip    & -                                                                 & p=0.5                                                                                                     & p=0.5                                                                                                     \\
% Normalization     & \begin{tabular}[c]{@{}l@{}}mean=0.1307,\\ std=0.3081\end{tabular} & \begin{tabular}[c]{@{}l@{}}mean={[}0.4914,0.4822,0.4465{]},\\ std={[}0.2023,0.1994,0.2010{]}\end{tabular} & \begin{tabular}[c]{@{}l@{}}mean={[}0.5071,0.4867,0.4408{]},\\ std={[}0.2675,0.2565,0.2761{]}\end{tabular} \\
\bottomrule[1pt]
\end{tabular}
}
\label{tab:optimizer classification}
\end{table}

\paragraph{Additional Benchmark.}  \citet{matias2024amortized} proposed Amortized Variational DKL (AV-DKL), which is a variant SV-DKL using amortization network to compute the inducing locations and variational parameters, thus attenuating the overcorrelation of NN extracted features. AV-DKL is included as the additional benchmark for classification tasks in \Cref{tab:img avdkl}. The training recipe is the same with SV-DKL. 


\begin{table*}[ht]
\caption{\small{Accuracy, NLL, ECE for AV-DKL, SV-DKL, DAK-MC on CIFAR-10/100 averaged over 3 runs. CIFAR-10 uses ResNet-18 with 64 features extracted; CIFAR-100 uses ResNet-34 with 512 features. The best results are highlighted in \textbf{bold}; the second best results are highlighted by \underline{underline}.}}
\centering
\vspace{-0.1cm}
\resizebox{\linewidth}{!}{%
\begin{tabular}{rccclccc}
\toprule[1pt]
\multicolumn{1}{l}{} & \multicolumn{3}{c}{Batch size: 128}  &  & \multicolumn{3}{c}{Batch size: 1024} \\ \cline{2-4} \cline{6-8} \vspace{-8pt} \\
\multicolumn{1}{l}{} & AV-DKL & SV-DKL & \cellcolor{Gray} DAK-MC &   & AV-DKL  & SV-DKL & \cellcolor{Gray} DAK-MC \\ 
\midrule[1pt]
CIFAR-10 - Acc. (\%) $\uparrow$    & \underline{94.23 $\pm$ 0.65}  & 93.44 $\pm$ 0.28    &  \cellcolor{Gray} \textbf{94.81 $\pm$ 0.13}   &     &  \textbf{93.32} $\pm$ \textbf{0.13}        & 90.22 $\pm$ 1.42       & \cellcolor{Gray} \underline{93.02 $\pm$ 0.18}        \\
NLL $\downarrow$     & 0.352 $\pm$ 0.084    & \underline{0.312 $\pm$ 0.033}       &  \cellcolor{Gray} \textbf{0.256} $\pm$ \textbf{0.014}     &      & \underline{0.439 $\pm$ 0.022}         & 0.485 $\pm$ 0.061       & \cellcolor{Gray} \textbf{0.345 $\pm$ 0.001}    \\
ECE $\downarrow$      & 0.048 $\pm$ 0.006    & \underline{0.046 $\pm$ 0.003}       &  \cellcolor{Gray} \textbf{0.039 $\pm$ 0.002}          &     & \underline{0.054 $\pm$ 0.001}       & 0.060 $\pm$ 0.004       & \cellcolor{Gray} \textbf{0.052 $\pm$ 0.001}           \\
\midrule[1pt]
CIFAR-100 -  Acc. (\%) $\uparrow$    & \textbf{77.47 $\pm$ 0.19}  & 74.52 $\pm$ 0.13       & \cellcolor{Gray}  \underline{76.75 $\pm$ 0.18}     &     &  \textbf{77.07 $\pm$ 0.10}        & 66.54 $\pm$ 0.74       & \cellcolor{Gray} \underline{70.38 $\pm$ 1.25}        \\
NLL $\downarrow$     & 1.787 $\pm$ 0.011    & \underline{1.041 $\pm$ 0.007}       & \cellcolor{Gray}  \textbf{1.001 $\pm$ 0.027}     &      & 2.326 $\pm$ 0.030    & \underline{1.738 $\pm$  0.058}      & \cellcolor{Gray} \textbf{1.203 $\pm$ 0.040}        \\
ECE $\downarrow$      & 0.166 $\pm$ 0.002    & \underline{0.049 $\pm$ 0.002}       & \cellcolor{Gray}  \textbf{0.041 $\pm$ 0.004}        &     & 0.175 $\pm$ 0.001         & \underline{0.148 $\pm$ 0.007}       &\cellcolor{Gray}  \textbf{0.056 $\pm$ 0.006}           \\
\bottomrule[1pt]
\end{tabular}
}
\vspace{-0.2cm}
\label{tab:img avdkl}
\end{table*}

\paragraph{Metrics} 
We evaluate model performance using four common metrics: Top-1 accuracy, ELBO, Negative Log Likelihood (NLL), and Expected Calibration Error (ECE). 

ECE is a metric used to quantify the degree of ``calibration'' of a probabilistic model in UQ, specifically for classification problems. It is defined as the weighted average of the absolute difference between the model's predicted probability (confidence) and the actual outcome (accuracy) over several bins of predicted probability. Mathematically, ECE is given by:
\begin{align}
    \text{ECE} =\sum_{m=1}^{M} \frac{\left| B_{m} \right|}{n} \left| \text{acc} (B_{m})-\text{conf} (B_{m}) \right|,
\end{align}
where $M$ is the number of bins into which the confidence values are partitioned, $B_m$ is the set of indices of samples whose predicted confidence falls into the $m$-th bin, $n$ is the total number of samples.

\paragraph{Computing Infrastructure}
The experiments for classification were run on a Linux machine with NVIDIA RTX4080 GPU, and 32GB of RAM.




\subsection{Additional Tables and Figures}
\label{sec:additional exp results}

\paragraph{Choices of learning rates.}
We evaluate the choices of learning rates on 1D regression examples. DKL requires a separate tuning of the learning rate of the GP covariance parameters, which differs from the learning rate of the NN feature extractor. In \Cref{fig:dkl lr}, we choose the learning rate of the NN feature extractor as $0.01$, while the learning rate of the GP covariance is set to different values. (a)-(c) show that different learning rates of covariance in DKL result in different predictive posterior. In particular, although the training losses for DKL in both (a) and (b) are minimal, the regressions do not fit well. On the other hand, DAK does not need a distinct recipe for tuning GP covariances because of the BNN interpretation. Furthermore, the poor posterior is indicated by the higher training loss, as illustrated in (d)-(f).

\begin{figure}[ht]
\centering
\subfloat[$\begin{gathered}\text{DKL: last-layer lr} =0.01.\\ \text{Training loss:} -0.21.\end{gathered}$]{\includegraphics[width=.3\textwidth]{toy_dkl_lr_01.pdf}}
\subfloat[$\begin{gathered}\text{DKL: last-layer lr} =0.001.\\ \text{Training loss: } -0.07.\end{gathered}$]{\includegraphics[width=.3\textwidth]{toy_dkl_lr_001.pdf}}
\subfloat[$\begin{gathered}\text{DKL: last-layer lr} =0.0001.\\ \text{Training loss: } 0.22.\end{gathered}$]{\includegraphics[width=.3\textwidth]{toy_dkl_lr_0001.pdf}}

\subfloat[$\begin{gathered}\text{DAK: last-layer lr} =0.1.\\ \text{Training loss: } 0.10.\end{gathered}$]{\includegraphics[width=.3\textwidth]{toy_dak_lr_1.pdf}}
\subfloat[$\begin{gathered}\text{DAK: last-layer lr} =0.01.\\ \text{Training loss: } 0.10.\end{gathered}$]{\includegraphics[width=.3\textwidth]{toy_dak_lr_01.pdf}}
\subfloat[$\begin{gathered}\text{DAK: last-layer lr} =0.001.\\ \text{Training loss: } 0.22.\end{gathered}$]{\includegraphics[width=.3\textwidth]{toy_dak_lr_001.pdf}}

\caption{Results on 1D regression with different last-layer learning rates. The learning rate of NN feature extractor is set as $0.01$. (a)--(f) shows the regression fits and corresponding training losses. DAK fits for the same learning rate strategy with NN feature extractor (lr=0.01), while DKL requires a separate tuning for last-layer learning rate of GPs. Additionally, a better training loss does not necessarily prevent overfitting for DKL.}
\label{fig:dkl lr}
\end{figure}


\paragraph{Learning curves.} We plot the learning curves of CIFAR-10/100 in \Cref{fig:cifar10 curves} and \ref{fig:cifar100 curves}. The learning curves of SVDKL in \Cref{fig:cifar10 curves} is more unstable, with many significant spikes, and the convergence is slower than DAK. Futhermore, SVDKL struggles to fit with full-training in CIFAR-100, and a pretrained feature extractor is used in CIFAR-100. Therefore, the learning curves of SVDKL look smoothing, but DAK fits well with full-training in CIFAR-100.


\begin{figure}[ht]
\centering
\subfloat[Test Error (\%).]{\includegraphics[width=.3\textwidth]{CIFAR_10_test_error.pdf}}
\subfloat[Test NLL.]{\includegraphics[width=.3\textwidth]{CIFAR_10_nll.pdf}}
\subfloat[ELBO.]{\includegraphics[width=.3\textwidth]{CIFAR_10_elbo.pdf}}
\caption{Test errors, test NLLs, ELBOs of NN, SVDKL, and DAK curves with batch size of 128/1024 for CIFAR-10 averaged on 3 runs. DAK outperforms SVDKL on both test error and NLL along the training epochs. Additionally, SVDKL degrades more and struggles to fit when the batch size becomes larger.}
\label{fig:cifar10 curves}
\end{figure}

\begin{figure}[ht]
\centering
\subfloat[Test Error (\%).]{\includegraphics[width=.3\textwidth]{CIFAR_100_test_error.pdf}}
\subfloat[Test NLL.]{\includegraphics[width=.3\textwidth]{CIFAR_100_nll.pdf}}
\subfloat[ELBO.]{\includegraphics[width=.3\textwidth]{CIFAR_100_elbo.pdf}}
\caption{Test errors, test NLLs, ELBOs of NN, SVDKL, and DAK curves with batch size of 128/1024 for CIFAR-100 averaged on 3 runs. DAK trained NN and last-layer additive GPs jointly, while SVDKL used the pre-trained NN and fine-tuned the last-layer GP since SVDKL struggles to fit using full-training. DAK outperforms SVDKL on both test error and NLL along the training epochs. SVDKL struggled to fit in high-dimensional multitask cases, indicating the necessity of pre-training in SVDKL. However, DAK fitted well with high dimensionality and large batch sizes.}
\label{fig:cifar100 curves}
\end{figure}







% \end{document}

% %\section{Proof of Equivalence}

\begin{proof}[Proof of Lemma~\ref{lem:mk_equivalence}]
% From the proof of Theorem~\ref{thm:iv_tight}, $\distiv = \distivpos \hspace{2pt} \dot{\cup} \hspace{2pt} \distivzero$ where $\distivpos$ and $\distivzero$ are as defined in \eqref{eq:distivpos} and \eqref{eq:distivzero}, respectively. Similarly, we can decompose $\distinter, \distctrf, \distgraph$ as $\distnotion = \distnotionpos \hspace{2pt} \dot{\cup} \hspace{2pt} \distnotionzero$ where 
% \begin{align}
% \distnotionpos &\triangleq \left \lbrace P_{\model}(D,A,S) : \model \in H^{0}_{\text{cf-notion}} \text{ and }\forall s, P_{\model}(S=s) > 0 \right \rbrace, \label{eq:distnotionpos} \\
% \distnotionzero &\triangleq  \left \lbrace P_{\model}(D,A,S) : \model \in H^{0}_{\text{cf-notion}} \text{ and } \exists s \text{ s.t. } P_{\model}(S=s) = 0 \right \rbrace,\label{eq:distnotionzero}
% \end{align}
% and we use ``notion'' as a placeholder for ``inter, ctrf, graph'' for clarity. For each of these notions, it is clear that if $P_{\model}(S=s) = 0$, for some $s$, then $\model \in H^{0}_{\text{cf-notion}}$ does not impose additional constraints on $P_{\model}\Paren{A,D \mid S=s'}$ where $s \neq s'$. Therefore, $\distivzero = \distnotionzero$ and it is sufficient to restrict attention to proving the equality of $\distnotioncnd$ and $\distivposcnd$ where the former is defined as 
% $$\distnotioncnd = \left \lbrace P_{\model}(D,A \mid S): \model \in H^{0}_{\text{cf-notion}} \right \rbrace.$$

%Again, like in the proof of Theorem~\ref{thm:iv_tight}, since $P(X,Y,Z) = P(Z) \otimes P(X,Y \mid Z)$ and $P(D,A,S) = P(S) \otimes P(D,A \mid S)$, 

For $\model \in \modelsedgerelax$, the response-function parameterization yields a counterfactually equivalent SCM, $\tilde{\model}$ represented by the tuple $(\enop,\tilde{\exrv},\tilde{\spc},\tilde{f},\tilde{P})$, where $\enop = \left \lbrace \sex, \dept, \outcome \right \rbrace, \tilde{\exrv} = \left \lbrace \response, U_{\sex} \right \rbrace, \tilde{\spc} =\spc_{\enop}\times\spc_{\tilde{\exrv}}, \tilde{f} = \Paren{\tilde{f}_{\sex}, \tilde{f}_{\dept}, \tilde{f}_{\outcome}}$ where we define $\spc_{\response}, \tilde{f},\tilde{P}$ through the function $\Phi: \spc_{\exrv} \mapsto \spc_{\tilde{\exrv}}$ where
\begin{align*}
    \spc_{\response} &\triangleq \spc_{\dept}^{\spc_{\sex}} \times \spc_{\outcome}^{\spc_{\sex}\times \spc_{\dept}},\\
    \forall u_S,u_D,u_A,u, \Phi\Paren{u_S,u_D,u_A,u} &\triangleq \Paren{\Paren{s \mapsto f_D(s,u,u_D),(s,d) \mapsto f_A(s,d,u,u_A)},u_S},\\
    \forall u_{\sex}, \tilde{f}_{\sex}\Paren{u_S}&\triangleq f_{\sex}(u_{\sex}),\\
\forall \lsex, \tilde{f}_{\dept}\Paren{\respfunc,\lsex} &\triangleq \respfunc_1\Paren{\lsex}, \\
\forall \lsex, \ldept, \tilde{f}_{\outcome}\Paren{\respfunc,\lsex,\ldept} &\triangleq \respfunc_2\Paren{\lsex,\ldept},
\end{align*}
where $\respfunc = \Paren{\respfunc_1,\respfunc_2}$ and $\tilde{P}$ is the push-forward distribution $\Phi_{*}(P)$.
Note that $\spc_{\response}$ is a discrete space, $\response$ a discrete random variable, and $\tilde{P}(\response)$  a discrete distribution over $\spc_{\response}$. Under the response-function parameterization, only $\tilde{P}(\response)$ is a parameter and we will abuse notation and denote it as $\tilde{P}$ henceforth. Therefore, we can represent $\nullgraphrelax$ in the parameter space as 

% \begin{equation}\label{eq:respfunc_graph_edge}
%     \nullgraphresp 
%     \triangleq \left \lbrace \tilde{P} \in \triangle\Paren{\cX_{\response}} : \tilde{P}\Paren{\respfunc_1,\respfunc_2} = 0 \text{ where } \respfunc_2\Paren{.,.} \text{ is such that } \exists \ldept 
%     \text{ such that }\respfunc_2(m,\ldept) \neq \respfunc_2\Paren{f,\ldept} \right \rbrace.
% \end{equation}

\begin{equation}\label{eq:respfunc_graph_edge}
    \nullgraphresp 
    \triangleq \left \lbrace \tilde{P} \in \triangle\Paren{\cX_{\response}} : \tilde{P}\Paren{\respfunc_1,\respfunc_2} \neq 0 \text{ implies } \forall \ldept, \respfunc_2(0,\ldept) = \respfunc_2\Paren{1,\ldept} \right \rbrace.
\end{equation}

To express $\nullinterrelax$, we express the interventional Markov kernels $P_{\tilde{\model}}\Paren{\outcome\mid \doop{\sex}, \doop{\dept}}$ in terms of $\tilde{P}$. Since counterfactual equivalence implies interventional equivalence, for all $\lsex, \ldept$, $P_{\model}\Paren{\outcome=1\mid \doop{\sex=\lsex}, \doop{\dept=\ldept}} = P_{\tilde{\model}}\Paren{\outcome=1\mid \doop{\sex=\lsex}, \doop{\dept=\ldept}}$, where 
\begin{align}
    P_{\tilde{\model}}\Paren{\outcome=1\mid \doop{\sex=\lsex}, \doop{\dept=\ldept}} &= \sum\limits_{\Paren{\respfunc_1,\respfunc_2}  \in \cX_{\response}}\bm{1}\Brack{\respfunc_2\Paren{\lsex,\ldept}=1}\tilde{P}\Paren{\respfunc_1,\respfunc_2}, \label{eq:inter_resp}\\
    P_{\tilde{\model}}\Paren{\outcome=1\mid \doop{\dept=\ldept}} &= \sum\limits_{\lsex^*}\sum\limits_{\Paren{\respfunc_1,\respfunc_2}  \in \cX_{\response}}\bm{1}\Brack{\respfunc_2\Paren{\lsex^*,\ldept}=1}\tilde{P}\Paren{\respfunc_1,\respfunc_2}P_{\tilde{\model}}\Paren{\lsex^*} \label{eq:inter_resp_doD},
\end{align}
Subtracting \eqref{eq:inter_resp} from  \eqref{eq:inter_resp_doD} we get 
\begin{align}
    &P_{\tilde{\model}}\Paren{\outcome=1\mid \doop{\sex=\lsex}, \doop{\dept=\ldept}} - P_{\tilde{\model}}\Paren{\outcome=1\mid \doop{\dept=\ldept}} \nonumber \\
    & =\Paren{\sum\limits_{\Paren{\respfunc_1,\respfunc_2}  \in \cX_{\response}} \Paren{\bm{1}\Brack{\respfunc_2\Paren{0,\ldept}=1} - \bm{1}\Brack{\respfunc_2\Paren{1,\ldept}=1}}\tilde{P}\Paren{\respfunc_1,\respfunc_2}}P_{\tilde{\model}}\Paren{s'} = 0 \label{eq:inter_resp_s}
\end{align}
for $\model \in \nullinterrelax$, where $s' \neq s$. Similarly, 
\begin{align}
    &P_{\tilde{\model}}\Paren{\outcome=1\mid \doop{\sex=\lsex'}, \doop{\dept=\ldept}} - P_{\tilde{\model}}\Paren{\outcome=1\mid \doop{\dept=\ldept}} \nonumber\\
    & =\Paren{\sum\limits_{\Paren{\respfunc_1,\respfunc_2}  \in \cX_{\response}} \Paren{\bm{1}\Brack{\respfunc_2\Paren{0,\ldept}=1} - \bm{1}\Brack{\respfunc_2\Paren{1,\ldept}=1}}\tilde{P}\Paren{\respfunc_1,\respfunc_2}}P_{\tilde{\model}}\Paren{s} = 0 \label{eq:inter_resp_sp}.
\end{align}
Since both \eqref{eq:inter_resp_s} and \eqref{eq:inter_resp_sp} hold, 
 the response-function parameterized analogue of $\nullinterrelax$ is 
\begin{equation}\label{eq:respfun_inter_edge}
    \nullinterresp \triangleq \left \lbrace \tilde{P} \in \triangle\Paren{\cX_{\response}} : \forall \ldept, \sum\limits_{\Paren{\respfunc_1,\respfunc_2}  \in \cX_{\response}} \Paren{\bm{1}\Brack{\respfunc_2\Paren{0,\ldept}=1} - \bm{1}\Brack{\respfunc_2\Paren{1,\ldept}=1}}\tilde{P}\Paren{\respfunc_1,\respfunc_2} = 0 \right \rbrace. 
\end{equation}

Note that both $\nullgraphresp$ and $\nullinterresp$ are polyhedra in $\triangle\Paren{\cX_{\response}}$. Further, $\nullgraphresp \subseteq \nullinterresp$. While, $\nullgraphresp, \nullinterresp$ are collections of distributions, we will also refer to them as collection of response-function-parameterized SCMs. 

%So far, we looked at the response-function parameterization for models in $\modelsedge$. However, the instrumental-variable inequalities arise from 
% While we have framed the hypotheses in terms of the exogenous distribution of the response-function parameterization, for a statistical test, we only have access to the observed Markov kernels $\Pr\Paren{\outcome,\dept,\formsex \mid \doop{\sex}}$. Therefore, we now characterize the sets of observed Markov kernels 
% It can be shown that the set of observed Markov kernels that are solutions of SCMs in $\nullgraph$ is the same as $\distiv$ where we define the former as 
From interventional equivalence (which follows as a result of counterfactual equivalence) of the response-function-parameterization, we have 
\begin{align*}
    \mkgraph &= \left \lbrace P_{\tilde{\model}}\Paren{\dept,\outcome\mid \doop{\sex}} : \tilde{\model} \in \nullgraphresp \right \rbrace \\
    \mkinter &= \left \lbrace P_{\tilde{\model}}\Paren{\dept,\outcome\mid \doop{\sex}} : \tilde{\model} \in \nullinterresp \right \rbrace.
\end{align*}

% Therefore, $\distgraph = \distiv$. The set of observed Markov kernels that are solutions of SCMs in $\nullinter$ is given by 
% \begin{equation}\label{eq:distinter}
%     \distinter \triangleq \left \lbrace P_{\model}\Paren{\outcome,\dept,\formsex \mid \doop{\sex}} : \model \in \nullinter \right \rbrace =  
% \end{equation}

We now show that $\mkinter = \mkgraph = \mkiv$. First, notice that $\mkinter \supseteq \mkgraph$ since $\nullinterresp \supseteq \nullgraphresp$. We first show that $\mkinter \subseteq \mkiv$ and then $\mkgraph = \mkiv$ which concludes the argument. 

\bm{$\mkinter \subseteq \mkiv$}: 
The solution function of the response-function parameterized SCM, $g_{A,D}: \cX_{\sex} \times \cX_{\response} \mapsto \cX_{\outcome} \times \cX_{\dept}$ induces a mapping from $\triangle\Paren{\cX_{\response}}$ which can be considered as a subset of $\RR^{\# \cX_{\response}}$ to the set of Markov kernels $P_{\tilde{\model}}\Paren{\dept,\outcome \mid \doop{\sex}}$ which  can be considered to be a subset of $\RR^{\#\Paren{\cX_{\outcome}}\times \#\Paren{\cX_{\dept}}\times \#\Paren{\cX_{\sex}}}$.
% We denote this map by $G: \RR^{\#\Paren{\cX_{\response}}} \mapsto \RR^{\#\Paren{\cX_{\outcome}}\times \#\Paren{\cX_{\dept}}\times \#\Paren{\cX_{\sex}}} $. 
% \begin{align*}
% g_{A,D}(\lsex,\respfunc) &= \Paren{\respfunc_2\Paren{\lsex,\respfunc_1\Paren{\lsex}},\respfunc_1\Paren{\lsex}}
% \end{align*}
% G\Paren{e_{\respfunc}} &= \sum\limits_{\lsex} e_{g_{A,D}\Paren{\lsex,\respfunc}}.
% First, note that for all $\tilde{\model} \in \nullinterresp$, $P_{\tilde{\model}}\Paren{\outcome,\dept,\formsex \mid \doop{\sex}} = P_{\tilde{\model}}\Paren{\outcome,\dept\mid \sex} \times\delta_{\sex}\Paren{\formsex}$. Therefore, we only restrict attention to $P_{\tilde{\model}}\Paren{\outcome,\dept\mid \sex = \formsex}$. 
The condition in \eqref{eq:respfun_inter_edge} implies that for all $\ldept$,
\begin{equation}\label{eq:constraint_outcome_one}
    \sum\limits_{\respfunc: \respfunc_2\Paren{0,\ldept}=1} \tilde{P}(\respfunc) = \sum\limits_{\respfunc: \respfunc_2\Paren{1,\ldept}=1} \tilde{P}(\respfunc). 
\end{equation}
Since, $\sum\limits_{\respfunc} \tilde{P}\Paren{\respfunc} = 1$, 
\begin{equation}\label{eq:constraint_outcome_zero}
    \sum\limits_{\respfunc: \respfunc_2\Paren{0,\ldept}=0} \tilde{P}(\respfunc) = \sum\limits_{\respfunc: \respfunc_2\Paren{1,\ldept}=0} \tilde{P}(\respfunc). 
\end{equation}
Denote $P_{\tilde{\model}}\Paren{\dept = \ldept, \outcome = \loutcome \mid \doop{\sex = \lsex}} $ by $P_{\tilde{\model}}\Paren{d,a || s}$. For  $P_{\tilde{\model}}\Paren{d,a || s} \in \mkinter$,
\begin{equation*}
    P_{\tilde{\model}}\Paren{d,a || s} = \sum\limits_{\respfunc: \respfunc_1\Paren{\lsex}=\ldept, \respfunc_2\Paren{\lsex,\ldept} = \loutcome
    } \tilde{P}(\respfunc). 
\end{equation*}
Therefore, from \eqref{eq:constraint_outcome_one}, 
\begin{align}
    \sum\limits_{\respfunc: \respfunc_2\Paren{0,\ldept}=1} \tilde{P}(\respfunc) &= P_{\tilde{\model}}(1,\ldept || 0) + \sum\limits_{\respfunc: \respfunc_1(0) \neq \ldept, \respfunc_2\Paren{0,\ldept}=1} \tilde{P}(\respfunc) \label{eq:1d0}\\
    &= \sum\limits_{\respfunc: \respfunc_2\Paren{1,\ldept}=1} \tilde{P}(\respfunc) \nonumber \\
    &= P_{\tilde{\model}}(1,\ldept || 1) + \sum\limits_{\respfunc: \respfunc_1(1) \neq \ldept, \respfunc_2\Paren{1,\ldept}=1} \tilde{P}(\respfunc) \label{eq:1d1}.
\end{align}
From \eqref{eq:constraint_outcome_zero}, 
\begin{align}
    \sum\limits_{\respfunc: \respfunc_2\Paren{0,\ldept}=0} \tilde{P}(\respfunc) &= P_{\tilde{\model}}(0,\ldept || 0) + \sum\limits_{\respfunc: \respfunc_1(0) \neq \ldept, \respfunc_2\Paren{0,\ldept}=0} \tilde{P}(\respfunc) \label{eq:0d0} \\
    &= \sum\limits_{\respfunc: \respfunc_2\Paren{1,\ldept}=0} \tilde{P}(\respfunc) \nonumber \\
    &= P_{\tilde{\model}}(0,\ldept || 1) + \sum\limits_{\respfunc: \respfunc_1(1) \neq \ldept, \respfunc_2\Paren{1,\ldept}=0} \tilde{P}(\respfunc) \label{eq:0d1}.
\end{align}
Since from \eqref{eq:constraint_outcome_one},
\begin{equation*}
    \sum\limits_{\respfunc} \tilde{P}\Paren{\respfunc} = \sum\limits_{\respfunc: \respfunc_2\Paren{0,\ldept}=0} \tilde{P}(\respfunc) + \sum\limits_{\respfunc: \respfunc_2\Paren{0,\ldept}=1} \tilde{P}(\respfunc) = \sum\limits_{\respfunc: \respfunc_2\Paren{0,\ldept}=0} \tilde{P}(\respfunc) + \sum\limits_{\respfunc: \respfunc_2\Paren{1,\ldept}=1} \tilde{P}(\respfunc) = 1.
\end{equation*}
Substituting from \eqref{eq:0d0} and \eqref{eq:1d1}, 
\begin{equation*}
    P_{\tilde{\model}}(0,\ldept || 0) + \sum\limits_{\respfunc: \respfunc_1(0) \neq \ldept, \respfunc_2\Paren{0,\ldept}=0} \tilde{P}(\respfunc) + P_{\tilde{\model}}(1,\ldept || 1) + \sum\limits_{\respfunc: \respfunc_1(1) \neq \ldept, \respfunc_2\Paren{1,\ldept}=1} \tilde{P}(\respfunc) =1. 
\end{equation*}
Similarly, substituting from \eqref{eq:0d1} and \eqref{eq:1d0}, 
\begin{equation*}
    P_{\tilde{\model}}(0,\ldept || 1) + \sum\limits_{\respfunc: \respfunc_1(1) \neq \ldept, \respfunc_2\Paren{1,\ldept}=0} \tilde{P}(\respfunc)+ P_{\tilde{\model}}(1,\ldept || 0) + \sum\limits_{\respfunc: \respfunc_1(0) \neq \ldept, \respfunc_2\Paren{0,\ldept}=1} \tilde{P}(\respfunc)=1. 
\end{equation*}
 This implies $P_{\tilde{\model}}(0,\ldept || 0) + P_{\tilde{\model}}(1,\ldept || 1) \leq 1, P_{\tilde{\model}}(0,\ldept || 1) + P_{\tilde{\model}}(1,\ldept || 0) \leq 1$. These are precisely the IV inequalities and they are satisfied. Therefore, $\mkinter \subseteq \mkiv$.

 \bm{$\mkgraph = \mkiv$} follows from Theorem~\ref{thm:iv_tight} since $\model \in \nullgraphrelax$ implies $\model \in \modelivrelax$. By Proposition~\ref{prop:cfnotions}, the lemma follows. 
 %We first show that $\distgraph$ is the same as the set of observed Markov kernels that are solutions of SCMs in $\modelsnoedge$, i.e.,

%  \begin{equation}
%      \distgraphnoedge \triangleq \left \lbrace P_{\model}\Paren{\outcome,\dept \mid \sex} : \model \in \modelsnoedge \right \rbrace = \distgraph.
%  \end{equation}
 
%  We then show that $\distgraphnoedge = \distiv$. 
 
% For $\model \in \modelsnoedge$, the response-function parameterization yields a counterfactually equivalent SCM, $\tilde{\model^*}$ represented by the tuple $(\exip,\enop,\tilde{\exrv}^*,\tilde{\spc}^*,\tilde{f}^*,\tilde{P}^*)$, where $\exip = \left \lbrace \sex \right \rbrace, \enop = \left \lbrace \formsex, \dept, \outcome \right \rbrace, \tilde{\exrv}^* = \left \lbrace \response^*\right \rbrace, \tilde{\spc} = \spc_{\exip}\times\spc_{\enop}\times\spc_{\tilde{\exrv}^*}, \tilde{f}^* = \left \lbrace \tilde{f}_{\formsex}^*, \tilde{f}_{\dept}^*, \tilde{f}_{\outcome}^* \right \rbrace$ where we define $\spc_{\response}^*, \tilde{f}^*,\tilde{P}^*$ as

% \begin{align*}
%     \spc_{\response^*} &\triangleq \spc_{\dept}^{\spc_{\sex}} \times \spc_{\outcome}^{\spc_{\dept}}, \\
%     \tilde{f}^*_{\formsex}(\sex) &\triangleq  f_{\formsex}(\sex) = \sex,\\
%     \tilde{f}^*_{\dept}\Paren{\respfunc^*,\sex} &\triangleq \respfunc_1^*\Paren{\sex} = f_{\dept}\Paren{\sex,U,U_{\dept}} = \respfunc_1(\sex), \\
%     \tilde{f}^*_{\outcome}\Paren{\respfunc^*,\dept} &\triangleq \respfunc_2^*\Paren{\dept} = f_{\outcome}\Paren{\dept,U,U_{\outcome}},
% \end{align*}
% where $\respfunc^* = \Paren{\respfunc_1^*,\respfunc_2^*}$.
% Note that $\spc_{\response^*}$ is a discrete space, $\response^*$ a discrete random variable, and $\tilde{P}^*$  a discrete distribution over $\spc_{\response^*}$. Under the response-function parameterization, only $\tilde{P}$ is a parameter. Therefore, we can represent $\modelsnoedge$ in the parameter space, $\modelsnoedgeresp \in \triangle\Paren{\spc_{\response^*}}$. By observational equivalence of the response-function parameterization,

% \begin{equation}
%     \distgraphnoedge = \left \lbrace P_{\tilde{\model}^*}\Paren{\outcome,\dept,\formsex \mid \doop{\sex}} : \tilde{\model}^* \in \modelsnoedgeresp \right \rbrace.
% \end{equation}

% Consider the set $D = \left \lbrace \respfunc \in \cX_{\response} \text{ such that } \exists \ldept \text{ where } \respfunc_2(0,\ldept) \neq \respfunc_2\Paren{1,\ldept} \right \rbrace.$ For any $\tilde{P} \in \nullgraphresp$, $\tilde{P}\Paren{D}=0$. Therefore, the function $h: \nullgraphresp \mapsto \modelsnoedgeresp$ such that $$h(\tilde{P})(\respfunc_1^*(\lsex), \respfunc_2^*(\ldept)) = \tilde{P}\Paren{\respfunc_1(\lsex),\respfunc_2(0,\ldept)=\respfunc_2(1,\ldept)}$$
% is well-defined and bijective since for any $\tilde{P}^* \in \modelsnoedgeresp$, $$h^{-1}(\tilde{P}^*)\Paren{\respfunc_1(\lsex),\respfunc_2(0,\ldept),\respfunc_2(1,\ldept)} = \bm{1}\left[ \respfunc_2(0,\ldept) = \respfunc_2(1,\ldept)\right] \tilde{P}^*\Paren{\respfunc_1(\lsex),\respfunc_2(0,d)}. $$ 

% The solution function of the response-function parameterized SCM $\tilde{\model^*}$ denoted by $g^*: \cX_{\sex} \times \cX_{\response^*} \mapsto \cX_{\outcome} \times \cX_{\dept} \times \cX_{\formsex}$ induces a mapping from $\triangle\Paren{\cX_{\response^*}}$ which can be considered as a subset of $\RR^{\# \cX_{\response^*}}$ to the set of conditional distributions $\Pr\Paren{\outcome,\dept,\formsex \mid \sex}$ which  can be considered to be a subset of $\RR^{\#\Paren{\cX_{\outcome}}\times \#\Paren{\cX_{\dept}}\times \#\Paren{\cX_{\formsex}}}$. We denote this map by $G^*: \RR^{\#\Paren{\cX_{\response^*}}} \mapsto \RR^{\#\Paren{\cX_{\outcome}}\times \#\Paren{\cX_{\dept}}\times \#\Paren{\cX_{\formsex}}} $ where 

% \begin{align*}
% g^*(\lsex,\respfunc^*) &= \Paren{\respfunc_2^*\Paren{\respfunc_1^*\Paren{\lsex}},\respfunc_1^*\Paren{\lsex},\lsex} =\Paren{\respfunc_2^*\Paren{\respfunc_1\Paren{\lsex}},\respfunc_1\Paren{\lsex},\lsex} , \\
% G^*\Paren{e_{\respfunc^*}} &= \sum\limits_{\lsex} e_{g^*\Paren{\lsex,\respfunc^*}}. 
% \end{align*}
% For $\respfunc \notin D$, note that $g(\lsex,\respfunc) = g^*(\lsex,(\respfunc_1,\respfunc_2'))$ where $\respfunc_2'(\ldept) = \respfunc_2(0,\ldept) = \respfunc_2(1,\ldept)$. Further, for all $\respfunc^* \in \cX_{\response^*}, g^*(\lsex,\respfunc^*) = g(\lsex,\respfunc')$ where $\respfunc' \notin D$ and is defined as $(\respfunc_1' = \respfunc_1^*, \respfunc_2'(0,\ldept) = \respfunc_2'(1,\ldept) =\respfunc_2^*(\ldept))$ Therefore, for any $\tilde{P} \in \nullgraphresp$, since $\tilde{P}(D) =0, G(\tilde{P}) = G^*(h(\tilde{P}))$ thus implying that $\distgraphnoedge = \distgraph$. 

% \begin{lemma}
% \begin{equation}
%     \distgraphnoedge = \distiv. 
% \end{equation}
% \end{lemma}

%\todo{Add proof from lecture notes.}

% We prove a bijection between $\modelsnoedgeresp$ and $\nullgraphresp$. For $\tilde{P} \in \nullgraphresp$, 


%  \begin{equation}\label{eq:respfunc_graph_noedge}
%     \modelsnoedgeresp
%     \triangleq \left \lbrace \tilde{P} \in \triangle\Paren{\cX_{\response}} : \tilde{P}\Paren{\respfunc_1,\respfunc_2} \neq 0 \text{ implies } \forall \ldept, \respfunc_2(0,\ldept) = \respfunc_2\Paren{1,\ldept} \right \rbrace.
% \end{equation}

\end{proof}




% For $\model \in \modelsedge$, the response-function parameterization yields a counterfactually equivalent \todo{Define in Preliminaries?}SCM, $\tilde{\model}$ represented by the tuple $(\exip,\enop,\tilde{\exrv},\tilde{\spc},\tilde{f},\tilde{P})$, where $\exip = \left \lbrace \sex \right \rbrace, \enop = \left \lbrace \formsex, \dept, \outcome \right \rbrace, \tilde{\exrv} = \left \lbrace \response\right \rbrace, \tilde{\spc} = \spc_{\exip}\times\spc_{\enop}\times\spc_{\tilde{\exrv}}, \tilde{f} = \left \lbrace \tilde{f}_{\formsex}, \tilde{f}_{\dept}, \tilde{f}_{\outcome} \right \rbrace$ where we define $\spc_{\response}, \tilde{f},\tilde{P}$ as

% \begin{align*}
%     \spc_{\response} &\triangleq \spc_{\dept}^{\spc_{\sex}} \times \spc_{\outcome}^{\spc_{\formsex}\times \spc_{\dept}}, \\
%     \tilde{f}_{\formsex}(\sex) &\triangleq  f_{\formsex}(\sex) = \sex,\\
%     \tilde{f}_{\dept}\Paren{\respfunc,\sex} &\triangleq \respfunc_1\Paren{\sex} = f_{\dept}\Paren{\sex,U,U_{\dept}}, \\
%     \tilde{f}_{\outcome}\Paren{\respfunc,\formsex,\dept} &\triangleq \respfunc_2\Paren{\formsex,\dept} = f_{\outcome}\Paren{\formsex,\dept,U,U_{\outcome}},
% \end{align*}
% where $\respfunc = \Paren{\respfunc_1,\respfunc_2}$.
% Note that $\spc_{\response}$ is a discrete space, $\response$ a discrete random variable, and $\tilde{P}$  a discrete distribution over $\spc_{\response}$. Under the response-function parameterization, only $\tilde{P}$ is a parameter. Therefore, we can represent $\nullgraph$ in the parameter space, $\nullgraphresp$, defined as \todo{Might need to add definition of parent in a casual graph, in preliminaries?} 

% % \begin{equation}\label{eq:respfunc_graph_edge}
% %     \nullgraphresp 
% %     \triangleq \left \lbrace \tilde{P} \in \triangle\Paren{\cX_{\response}} : \tilde{P}\Paren{\respfunc_1,\respfunc_2} = 0 \text{ where } \respfunc_2\Paren{.,.} \text{ is such that } \exists \ldept 
% %     \text{ such that }\respfunc_2(m,\ldept) \neq \respfunc_2\Paren{f,\ldept} \right \rbrace.
% % \end{equation}

% \begin{equation}\label{eq:respfunc_graph_edge}
%     \nullgraphresp 
%     \triangleq \left \lbrace \tilde{P} \in \triangle\Paren{\cX_{\response}} : \tilde{P}\Paren{\respfunc_1,\respfunc_2} \neq 0 \text{ implies } \forall \ldept, \respfunc_2(0,\ldept) = \respfunc_2\Paren{1,\ldept} \right \rbrace.
% \end{equation}

% To express $\nullinter$, we express the interventional Markov kernels $\Pr\Paren{\outcome\mid \doop{\formsex}, \doop{\dept}}$ in terms of $\tilde{P}$, 
% \begin{equation}\label{eq:inter_resp}
%     \Pr\Paren{\outcome=1\mid \doop{\formsex=\lsex'}, \doop{\dept=\ldept}} = \sum\limits_{\Paren{\respfunc_1,\respfunc_2}  \in \cX_{\response}}\bm{1}\Brack{\respfunc_2\Paren{\lsex',\ldept}=1}\tilde{P}\Paren{\respfunc_1,\respfunc_2}.
% \end{equation}

% Therefore the response-function parameterized analogue of $\nullinter$ is 
% \begin{equation}\label{eq:respfun_inter_edge}
%     \nullinterresp \triangleq \left \lbrace \tilde{P} \in \triangle\Paren{\cX_{\response}} : \forall \ldept, \sum\limits_{\Paren{\respfunc_1,\respfunc_2}  \in \cX_{\response}} \Paren{\bm{1}\Brack{\respfunc_2\Paren{0,\ldept}=1} - \bm{1}\Brack{\respfunc_2\Paren{1,\ldept}=1}}\tilde{P}\Paren{\respfunc_1,\respfunc_2} = 0 \right \rbrace. 
% \end{equation}

% Note that both $\nullgraphresp$ and $\nullinterresp$ are polyhedra in $\triangle\Paren{\cX_{\response}}$. Further, $\nullgraphresp \subseteq \nullinterresp$. While, $\nullgraphresp, \nullinterresp$ are collections of distributions, we will abuse notation and also refer to them as collection of response-function-parameterized SCMs. 

% %So far, we looked at the response-function parameterization for models in $\modelsedge$. However, the instrumental-variable inequalities arise from 

% % While we have framed the hypotheses in terms of the exogenous distribution of the response-function parameterization, for a statistical test, we only have access to the observed Markov kernels $\Pr\Paren{\outcome,\dept,\formsex \mid \doop{\sex}}$. Therefore, we now characterize the sets of observed Markov kernels 


% % It can be shown that the set of observed Markov kernels that are solutions of SCMs in $\nullgraph$ is the same as $\distiv$ where we define the former as 
% From the observational equivalence of the response-function-parameterization, we have \todo{Observational equivalence also in preliminaries.}
% \begin{align*}
%     \distgraph &= \left \lbrace P_{\tilde{\model}}\Paren{\outcome,\dept,\formsex \mid \doop{\sex}} : \tilde{\model} \in \nullgraphresp \right \rbrace \\
%     \distinter &= \left \lbrace P_{\tilde{\model}}\Paren{\outcome,\dept,\formsex \mid \doop{\sex}} : \tilde{\model} \in \nullinterresp \right \rbrace.
% \end{align*}

% % Therefore, $\distgraph = \distiv$. The set of observed Markov kernels that are solutions of SCMs in $\nullinter$ is given by 
% % \begin{equation}\label{eq:distinter}
% %     \distinter \triangleq \left \lbrace P_{\model}\Paren{\outcome,\dept,\formsex \mid \doop{\sex}} : \model \in \nullinter \right \rbrace =  
% % \end{equation}

% We now show that $\distinter = \distgraph = \distiv$. First, notice that $\distinter \supseteq \distgraph$ since $\nullinterresp \supseteq \nullgraphresp$. We first show that $\distinter \subseteq \distiv$ and then $\distgraph = \distiv$ which concludes the argument. 

% \bm{$\distinter \subseteq \distiv$}: 
% The solution function of the response-function parameterized SCM, $g: \cX_{\sex} \times \cX_{\response} \mapsto \cX_{\outcome} \times \cX_{\dept} \times \cX_{\formsex}$ induces a mapping from $\triangle\Paren{\cX_{\response}}$ which can be considered as a subset of $\RR^{\# \cX_{\response}}$ to the set of conditional distributions $\Pr\Paren{\outcome,\dept,\formsex \mid \sex}$ which  can be considered to be a subset of $\RR^{\#\Paren{\cX_{\outcome}}\times \#\Paren{\cX_{\dept}}\times \#\Paren{\cX_{\formsex}}}$. We denote this map by $G: \RR^{\#\Paren{\cX_{\response}}} \mapsto \RR^{\#\Paren{\cX_{\outcome}}\times \#\Paren{\cX_{\dept}}\times \#\Paren{\cX_{\formsex}}} $ where 

% \begin{align*}
% g(\lsex,\respfunc) &= \Paren{\respfunc_2\Paren{\lsex,\respfunc_1\Paren{\lsex}},\respfunc_1\Paren{\lsex},\lsex}, \\
% G\Paren{e_{\respfunc}} &= \sum\limits_{\lsex} e_{g\Paren{\lsex,\respfunc}}. 
% \end{align*}

% First, note that for all $\tilde{\model} \in \nullinterresp$, $P_{\tilde{\model}}\Paren{\outcome,\dept,\formsex \mid \doop{\sex}} = P_{\tilde{\model}}\Paren{\outcome,\dept\mid \sex} \times\delta_{\sex}\Paren{\formsex}$. Therefore, we only restrict attention to $P_{\tilde{\model}}\Paren{\outcome,\dept\mid \sex = \formsex}$. The condition in \eqref{eq:distinter} implies that for all $\ldept$,
% \begin{equation}\label{eq:constraint_outcome_one}
%     \sum\limits_{\respfunc: \respfunc_2\Paren{0,\ldept}=1} \tilde{P}(\respfunc) = \sum\limits_{\respfunc: \respfunc_2\Paren{1,\ldept}=1} \tilde{P}(\respfunc). 
% \end{equation}

% Since, $\sum\limits_{\respfunc} \tilde{P}\Paren{\respfunc} = 1$, 

% \begin{equation}\label{eq:constraint_outcome_zero}
%     \sum\limits_{\respfunc: \respfunc_2\Paren{0,\ldept}=0} \tilde{P}(\respfunc) = \sum\limits_{\respfunc: \respfunc_2\Paren{1,\ldept}=0} \tilde{P}(\respfunc). 
% \end{equation}

% Expressing the marginal over $\outcome, \dept$ 
%  of $P \in \distinter$,
% \begin{equation*}
%     p\Paren{\loutcome,\ldept \mid \lsex} = \sum\limits_{\respfunc: \respfunc_1\Paren{\lsex}=\ldept, \respfunc_2\Paren{\lsex,\ldept} = \loutcome
%     } \tilde{P}(\respfunc). 
% \end{equation*}

% Therefore, from \eqref{eq:constraint_outcome_one} and \eqref{eq:constraint_outcome_zero}, 
% \begin{align*}
%     \sum\limits_{\respfunc: \respfunc_2\Paren{0,\ldept}=1} \tilde{P}(\respfunc) &= p(1,\ldept \mid 0) + \sum\limits_{\respfunc: \respfunc_1(0) \neq \ldept, \respfunc_2\Paren{0,\ldept}=1} \tilde{P}(\respfunc) = \sum\limits_{\respfunc: \respfunc_2\Paren{1,\ldept}=1} \tilde{P}(\respfunc) = p(1,\ldept \mid 1) + \sum\limits_{\respfunc: \respfunc_1(1) \neq \ldept, \respfunc_2\Paren{1,\ldept}=1} \tilde{P}(\respfunc)\\
%     \sum\limits_{\respfunc: \respfunc_2\Paren{0,\ldept}=0} \tilde{P}(\respfunc) &= p(0,\ldept \mid 0) + \sum\limits_{\respfunc: \respfunc_1(0) \neq \ldept, \respfunc_2\Paren{0,\ldept}=0} \tilde{P}(\respfunc) = \sum\limits_{\respfunc: \respfunc_2\Paren{1,\ldept}=0} \tilde{P}(\respfunc) = p(0,\ldept \mid 1) + \sum\limits_{\respfunc: \respfunc_1(1) \neq \ldept, \respfunc_2\Paren{1,\ldept}=0} \tilde{P}(\respfunc)
% \end{align*}

% Since
% \begin{equation*}
%     \sum\limits_{\respfunc} \tilde{P}\Paren{\respfunc} = \sum\limits_{\respfunc: \respfunc_2\Paren{0,\ldept}=0} \tilde{P}(\respfunc) + \sum\limits_{\respfunc: \respfunc_2\Paren{0,\ldept}=1} \tilde{P}(\respfunc) = \sum\limits_{\respfunc: \respfunc_2\Paren{0,\ldept}=0} \tilde{P}(\respfunc) + \sum\limits_{\respfunc: \respfunc_2\Paren{1,\ldept}=1} \tilde{P}(\respfunc) = 1, 
% \end{equation*}
% we have 
% \begin{equation*}
%     p(0,\ldept \mid 0) + \sum\limits_{\respfunc: \respfunc_1(0) \neq \ldept, \respfunc_2\Paren{0,\ldept}=0} \tilde{P}(\respfunc) + p(1,\ldept \mid 1) + \sum\limits_{\respfunc: \respfunc_1(1) \neq \ldept, \respfunc_2\Paren{1,\ldept}=1} \tilde{P}(\respfunc) =1. 
% \end{equation*}

% Similarly, 
% \begin{equation*}
%     p(0,\ldept \mid 1) + \sum\limits_{\respfunc: \respfunc_1(1) \neq \ldept, \respfunc_2\Paren{1,\ldept}=0} \tilde{P}(\respfunc)+ p(1,\ldept \mid 0) + \sum\limits_{\respfunc: \respfunc_1(0) \neq \ldept, \respfunc_2\Paren{0,\ldept}=1} \tilde{P}(\respfunc)=1. 
% \end{equation*}
%  Since the IV inequalities are satisfied, $\distinter \subseteq \distiv$.

%  \bm{$\distgraph = \distiv$}: We first show that $\distgraph$ is the same as the set of observed Markov kernels that are solutions of SCMs in $\modelsnoedge$, i.e.,

%  \begin{equation}
%      \distgraphnoedge \triangleq \left \lbrace P_{\model}\Paren{\outcome,\dept,\formsex \mid \doop{\sex}} : \model \in \modelsnoedge \right \rbrace = \distgraph.
%  \end{equation}
 
%  We then show that $\distgraphnoedge = \distiv$. 
 
% For $\model \in \modelsnoedge$, the response-function parameterization yields a counterfactually equivalent SCM, $\tilde{\model^*}$ represented by the tuple $(\exip,\enop,\tilde{\exrv}^*,\tilde{\spc}^*,\tilde{f}^*,\tilde{P}^*)$, where $\exip = \left \lbrace \sex \right \rbrace, \enop = \left \lbrace \formsex, \dept, \outcome \right \rbrace, \tilde{\exrv}^* = \left \lbrace \response^*\right \rbrace, \tilde{\spc} = \spc_{\exip}\times\spc_{\enop}\times\spc_{\tilde{\exrv}^*}, \tilde{f}^* = \left \lbrace \tilde{f}_{\formsex}^*, \tilde{f}_{\dept}^*, \tilde{f}_{\outcome}^* \right \rbrace$ where we define $\spc_{\response}^*, \tilde{f}^*,\tilde{P}^*$ as

% \begin{align*}
%     \spc_{\response^*} &\triangleq \spc_{\dept}^{\spc_{\sex}} \times \spc_{\outcome}^{\spc_{\dept}}, \\
%     \tilde{f}^*_{\formsex}(\sex) &\triangleq  f_{\formsex}(\sex) = \sex,\\
%     \tilde{f}^*_{\dept}\Paren{\respfunc^*,\sex} &\triangleq \respfunc_1^*\Paren{\sex} = f_{\dept}\Paren{\sex,U,U_{\dept}} = \respfunc_1(\sex), \\
%     \tilde{f}^*_{\outcome}\Paren{\respfunc^*,\dept} &\triangleq \respfunc_2^*\Paren{\dept} = f_{\outcome}\Paren{\dept,U,U_{\outcome}},
% \end{align*}
% where $\respfunc^* = \Paren{\respfunc_1^*,\respfunc_2^*}$.
% Note that $\spc_{\response^*}$ is a discrete space, $\response^*$ a discrete random variable, and $\tilde{P}^*$  a discrete distribution over $\spc_{\response^*}$. Under the response-function parameterization, only $\tilde{P}$ is a parameter. Therefore, we can represent $\modelsnoedge$ in the parameter space, $\modelsnoedgeresp \in \triangle\Paren{\spc_{\response^*}}$. By observational equivalence of the response-function parameterization,

% \begin{equation}
%     \distgraphnoedge = \left \lbrace P_{\tilde{\model}^*}\Paren{\outcome,\dept,\formsex \mid \doop{\sex}} : \tilde{\model}^* \in \modelsnoedgeresp \right \rbrace.
% \end{equation}

% Consider the set $D = \left \lbrace \respfunc \in \cX_{\response} \text{ such that } \exists \ldept \text{ where } \respfunc_2(0,\ldept) \neq \respfunc_2\Paren{1,\ldept} \right \rbrace.$ For any $\tilde{P} \in \nullgraphresp$, $\tilde{P}\Paren{D}=0$. Therefore, the function $h: \nullgraphresp \mapsto \modelsnoedgeresp$ such that $$h(\tilde{P})(\respfunc_1^*(\lsex), \respfunc_2^*(\ldept)) = \tilde{P}\Paren{\respfunc_1(\lsex),\respfunc_2(0,\ldept)=\respfunc_2(1,\ldept)}$$
% is well-defined and bijective since for any $\tilde{P}^* \in \modelsnoedgeresp$, $$h^{-1}(\tilde{P}^*)\Paren{\respfunc_1(\lsex),\respfunc_2(0,\ldept),\respfunc_2(1,\ldept)} = \bm{1}\left[ \respfunc_2(0,\ldept) = \respfunc_2(1,\ldept)\right] \tilde{P}^*\Paren{\respfunc_1(\lsex),\respfunc_2(0,d)}. $$ 

% The solution function of the response-function parameterized SCM $\tilde{\model^*}$ denoted by $g^*: \cX_{\sex} \times \cX_{\response^*} \mapsto \cX_{\outcome} \times \cX_{\dept} \times \cX_{\formsex}$ induces a mapping from $\triangle\Paren{\cX_{\response^*}}$ which can be considered as a subset of $\RR^{\# \cX_{\response^*}}$ to the set of conditional distributions $\Pr\Paren{\outcome,\dept,\formsex \mid \sex}$ which  can be considered to be a subset of $\RR^{\#\Paren{\cX_{\outcome}}\times \#\Paren{\cX_{\dept}}\times \#\Paren{\cX_{\formsex}}}$. We denote this map by $G^*: \RR^{\#\Paren{\cX_{\response^*}}} \mapsto \RR^{\#\Paren{\cX_{\outcome}}\times \#\Paren{\cX_{\dept}}\times \#\Paren{\cX_{\formsex}}} $ where 

% \begin{align*}
% g^*(\lsex,\respfunc^*) &= \Paren{\respfunc_2^*\Paren{\respfunc_1^*\Paren{\lsex}},\respfunc_1^*\Paren{\lsex},\lsex} =\Paren{\respfunc_2^*\Paren{\respfunc_1\Paren{\lsex}},\respfunc_1\Paren{\lsex},\lsex} , \\
% G^*\Paren{e_{\respfunc^*}} &= \sum\limits_{\lsex} e_{g^*\Paren{\lsex,\respfunc^*}}. 
% \end{align*}
% For $\respfunc \notin D$, note that $g(\lsex,\respfunc) = g^*(\lsex,(\respfunc_1,\respfunc_2'))$ where $\respfunc_2'(\ldept) = \respfunc_2(0,\ldept) = \respfunc_2(1,\ldept)$. Further, for all $\respfunc^* \in \cX_{\response^*}, g^*(\lsex,\respfunc^*) = g(\lsex,\respfunc')$ where $\respfunc' \notin D$ and is defined as $(\respfunc_1' = \respfunc_1^*, \respfunc_2'(0,\ldept) = \respfunc_2'(1,\ldept) =\respfunc_2^*(\ldept))$ Therefore, for any $\tilde{P} \in \nullgraphresp$, since $\tilde{P}(D) =0, G(\tilde{P}) = G^*(h(\tilde{P}))$ thus implying that $\distgraphnoedge = \distgraph$. 

% \begin{lemma}
% \begin{equation}
%     \distgraphnoedge = \distiv. 
% \end{equation}
% \end{lemma}

% %\todo{Add proof from lecture notes.}

% % We prove a bijection between $\modelsnoedgeresp$ and $\nullgraphresp$. For $\tilde{P} \in \nullgraphresp$, 


% %  \begin{equation}\label{eq:respfunc_graph_noedge}
% %     \modelsnoedgeresp
% %     \triangleq \left \lbrace \tilde{P} \in \triangle\Paren{\cX_{\response}} : \tilde{P}\Paren{\respfunc_1,\respfunc_2} \neq 0 \text{ implies } \forall \ldept, \respfunc_2(0,\ldept) = \respfunc_2\Paren{1,\ldept} \right \rbrace.
% % \end{equation}

% \end{proof}
% \section{Encompassing Prior Method}

For the sake of completeness, we outline the computation of Bayes Factor using the encompassing prior method \cite{KlugkistKH05}. We also refer the reader to \cite{HeckDavisStrober19} for theoretical justification of computing the Bayes factor using approximations that we do in our procedure. 

Consider two models $\MM_{c}, \MM_{uc}$ where $\MM_{c}$ is a model that is ``encompassed" in $\MM_{uc}$ such that $\MM_{c}$ is constrained by inequality constraints on the unknown parameters. We formally define this as follows. 

\begin{definition}[Inequality-Constrained Encompassing Models]
    A pair of models $\Paren{\MM_{uc},\MM_{c}}$ are inequality-constrained encompassing models if 
\end{definition}

\begin{proposition}\label{prop:BFencompassing}
Let $\Paren{\MM_{uc},\MM_{c}}$ be inequality-constrained encompassing models. Then the Bayes factor is given by 

\begin{equation}
    \bayesfactor\Paren{\MM_{c},\MM_{uc}} = \frac{C_{pri}}{C_{post}},
\end{equation}
where $C_{pri}$ and $C_{post}$ are normalizing constants defined as 
\begin{align}\label{eq:normalizingconstants}
1/C_{pri} &= \int_{\theta \in \MM_{c}} P\Paren{\theta|\MM_{uc}} d\theta, \\
1/C_{post} &= \int_{\theta \in \MM_{c}} P\Paren{\theta|X_1,X_2, \cdots, X_m, \MM_{uc}} d\theta.
\end{align}
\end{proposition}

\begin{proof}
    We know that 
    \begin{equation}\label{eq:bayesfactorratiolikelihood}
        \bayesfactor\Paren{\MM_{c},\MM_{uc}} = \frac{P\Paren{X_1,X_2,\cdots, X_m \mid \MM_{uc}}}{P\Paren{X_1,X_2,\cdots, X_m \mid \MM_c}}.
    \end{equation}
From Bayes Theorem, for a model $\MM$, dataset $\dataset = \Paren{X_1,X_2,\cdots, X_m}$ and parameter $\theta$,

\begin{equation*}
    P\Paren{\theta | \dataset, \MM} = \frac{P\Paren{\dataset|\theta,\MM}P\Paren{\theta|\MM}}{P\Paren{\dataset|\MM}}.
\end{equation*}
    By rearranging, we can express the marginal likelihood of the model, 
\begin{equation*}
    P\Paren{\dataset|\MM} =  \frac{P\Paren{\dataset|\theta,\MM}P\Paren{\theta|\MM}}{P\Paren{\theta | \dataset, \MM}}.
\end{equation*}
Substituting in equation~\ref{eq:bayesfactorratiolikelihood}, we have 
    \begin{equation}\label{eq:bayesfactorsub}
        \bayesfactor\Paren{\MM_{c},\MM_{uc}} = \frac{P\Paren{X_1,X_2,\cdots, X_m \mid \MM_{uc},\theta} P\Paren{\theta|\MM_{uc}}/P\Paren{\theta|X_1,X_2,\cdots, X_m, \MM_{uc}}}{P\Paren{X_1,X_2,\cdots, X_m \mid \MM_{c},\theta} P\Paren{\theta|\MM_{c}}/P\Paren{\theta|X_1,X_2,\cdots, X_m, \MM_{c}}}.
    \end{equation}
Since $\MM_{c}$ is nested in $\MM_{uc}$, for a particular $\theta'$ that is in both the models, we have
\begin{equation*}
    P\Paren{X_1,X_2,\cdots, X_m \mid \MM_{uc},\theta'} = P\Paren{X_1,X_2,\cdots, X_m \mid \MM_{c},\theta'}.
\end{equation*}
 Therefore,
\begin{equation}\label{eq:bayesfactorreduce}
        \bayesfactor\Paren{\MM_{c},\MM_{uc}} = \frac{P\Paren{\theta'|\MM_{uc}}/P\Paren{\theta'|X_1,X_2,\cdots, X_m, \MM_{uc}}}{ P\Paren{\theta'|\MM_{c}}/P\Paren{\theta'|X_1,X_2,\cdots, X_m, \MM_{c}}}.
    \end{equation}
We can express the prior and the posterior for the constrained model as truncated versions of the same for the unconstrained model. 

\begin{align*}
    P\Paren{\theta'|\MM_{c}} &= C_{pri} P\Paren{\theta'|\MM_{uc}}\bm{1}\Brack{\theta' \in \MM_{c}},\\
    P\Paren{\theta'|X_1,X_2,\cdots, X_m, \MM_{c}} &= C_{post}P\Paren{\theta'|X_1,X_2,\cdots, X_m, \MM_{uc}}\bm{1}\Brack{\theta' \in \MM_{c}},
\end{align*}
where $C_{pri}$, $C_{post}$ are normalizing constants defined in equation~\ref{eq:normalizingconstants}. 
Substituting the above in equation~\ref{eq:bayesfactorreduce}, 
\begin{equation}\label{eq:bayesfactorfinal}
        \bayesfactor\Paren{\MM_{c},\MM_{uc}} = \frac{C_{pri}}{C_{post}}.
    \end{equation}
\end{proof}

Proposition~\ref{prop:BFencompassing} lends itself to a simple computational procedure to compute the Bayes factor given in Algorithm~\ref{alg:bayesfactor}.

% \begin{algorithm}[t]\label{alg:bayesfactor}
%     \SetKwInOut{Input}{Input}
%     \SetKwInOut{Output}{Output}
%     \Input{$D = \Paren{X_1,X_2,\ldots, X_m},\alpha=\Paren{1,1,\cdots,1},\smp_{pri},\smp_{post}$}
%     \Output{$\bayesfactor\Paren{\MM_c,\MM_{uc}}$}
%     \BlankLine
%     Set $R_{pri} = 0, R_{post}=0$
    
%     \For{$j \in \left[\smp_{pri}\right]$}
%     {
%     Sample $\hat{p}_j \sim \dir\Paren{\alpha}$ \;
%     \If{$\hat{p}_j$ satisfies inequality constraint}
%     {$R_{pri} = R_{pri}+1$ \;
%     }
%     }
%     $\hat{C}_{pri} = \frac{R_{pri}}{\smp_{pri}}$ \;
%     $\Paren{N_1,N_2,\cdots, N_k} = count(D)$ \;
%     \For{$j \in \left[\smp_{post}\right]$}
%     {
%     Sample $\hat{p}_j \sim \dir\Paren{\alpha+\Paren{N_1,N_2,\cdots, N_k}}$ \;
%     \If{$\hat{p}_j$ satisfies inequality constraint}
%     {$R_{post} = R_{post}+1$ \;
%     }
%     }
%     $\hat{C}_{post} = \frac{R_{post}}{\smp_{post}}$\;
%     \Return{$\bayesfactor\Paren{\MM_c,\MM_{uc}} = \frac{\hat{C}_{post}}{\hat{C}_{pri}}$}
%     \caption{}
% \end{algorithm}
\end{document}
