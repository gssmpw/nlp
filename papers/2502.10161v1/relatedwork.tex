\section{Related Work}
The question of fairness in decision-making and predictive systems has received increased attention since the past few decades. See \citet{HutchinsonMitchell19,BarocasHN23} for an excellent historical  and technical overview, respectively. While attempts at formalizing fairness lead to correlation-based notions such as fairness through unawareness \citep{DworkHPRZ12}, demographic parity, equality of odds \citep{HardtPS16} etc., purely observational notions of fairness are at odds with each other \cite{Chouldechova17, KleinbergMR17} and are prone to erroneous conclusions. On the other hand, observational notions of fairness are readily translated to statistical tests.

Causal analysis tools such as counterfactuals and interventions provide a framework suitable for fairness. As a result, multiple general fairness notions based on counterfactuals were proposed.  \citet{KusnerLRS17} defined a counterfactual fairness notion that required invariance of the distribution of the decision in a given context, with respect to hypothetical changes in the protected attribute. \citet{NabiShpitser18} and \citet{ZhangWW17} consider path-specific effects. \citet{Chiappa19} proposes a path-specific counterfactual fairness notion and a related notion appears in the appendix of \citet{KusnerLRS17}. Another separate line of work seeks to explain observed disparity through causal discrimination mechanisms \citep{ZhangBareinboim18, plecko2022causal}.

The Berkeley graduate admissions case makes an appearance in multiple papers to motivate the need for causal fairness notions. \cite{KilbertusRPHJS17, plecko2022causal,KusnerLRS17,Chiappa19,BerkKT23} are a few among many works. In addition, the Berkeley example also serves as a motivation to introduce path-specific notions given the assumption that the direct effect of sex on admissions outcome is the only `unfair' path. Also, see \citet{BarocasHN23} for a critique of this common assumption. \citet{Pearl09} considers the Berkeley example at length and illustrates the objection to controlling for the mediator by positing an observed confounder. 

Despite the fact that most causal fairness works mention the Berkeley example, to the best of our knowledge, no previous work gives a definitive answer to the question of fairness for the Berkeley dataset under unobserved confounding. \citet{KilbertusBKWS20} discusses the impact of unmeasured confounding under restrictive parametric assumptions. \citet{ZhangBareinboim18, plecko2022causal} consider fairness models that allow for specific forms of unobserved confounding. \citet{SchroderFF24} build on this by providing sensitivity analysis on fairness of prediction models. However, the kinds of unobserved confounding that they allow affects the sensitive attribute which is different from the kind we allow for in the Berkeley dataset. 

% Other related areas under fairness that consider unobserved confounding are risk assessments \citep{RambachanCK22} and auditing \citep{ByunSOLW24}. 


% in particular, he frames the question of discrimination in causal terms--Given the presence of an indirect effect of sex on admissions outcome (through the choice of department), is there a direct effect of sex on admissions outcome?