\section{RELATED WORKS}
Gaze is important for social interaction in daily life because people understand others' intentions through eye contact and use it to communicate~\cite{rauthmann2012eyes}.
In social interactions, gaze behavior, such as eye contact, also reflects different personality traits~\cite{libby1973personality,brooks1986effects}.

In the transportation field, gaze behavior has been widely studied in pedestrian crossing scenarios. 
For example, by analyzing pedestrians' gaze behavior to understand their crossing strategies, researchers have explored how pedestrians allocate visual attention during road-crossing tasks and how gaze behavior influences decision-making~\cite{zhao2023pedestrian}.
Furthermore, study~\cite{geruschat2003gaze} had examined how normally sighted pedestrians use gaze to navigate and safely cross various types of intersection.


Moreover, eye contact is one of gaze behavior, which serves as an important communication method between drivers, pedestrians, and cyclists~\cite{rasouli2017agreeing,li2021autonomous}.
Vulnerable road users, such as pedestrians and cyclists, often seek eye contact with the driver when encountering a vehicle~\cite{sahai2022crossing, de2021pedestrians}.
This eye contact from the driver indicates that the driver has recognized the pedestrian and assures the pedestrian's safety.
For example, the study \cite{onkhar2022effect} reported that eye contact between the driver and the pedestrian increased the percentage of participants indicating that it was safe to cross.



In research on human-personal mobility vehicle interactions, studies focusing on gaze analysis are limited.
The study~\cite{maekawa2019analysis} analyzed the gaze behavior of drivers of manually driven personal mobility vehicles. 
The researchers found that skilled drivers tend to focus on multiple potential risks in the driving environment at the same time.
Additionally, study~\cite{liu2022implicit} analyzed the gaze behavior of pedestrians when encountering an APMV. 
They found that when pedestrians do not understand the driving intentions of the APMV or feel threatened, the duration of their gaze toward the APMV increases significantly. 
They also suggested that the gaze behavior of pedestrians toward the APMV represents the pedestrians' attempt to required information about driving intentions from the APMV.

However, to the best of the authors' knowledge, there has been no research analyzing the gaze behavior of APMV passengers toward pedestrians during interaction between APMVs and pedestrians. Moreover, the impact of the APMV's eHMI designs on pedestrian gaze behavior in such interaction has yet to be explored.




% eye tracking 
%Understanding how humans use their eyes to gather information from their surroundings is fundamental to exploring the cognitive and perceptual processes that support ongoing behavior \cite{tatler2019eye}.

%The way people allocate visual attention in dynamic environments plays a critical role in guiding decision making, facilitating interactions, and ensuring safety.

% Wearable eye tracking glasses have emerged as a powerful tool to study these processes, offering insights into where and how people focus their attention in real-world human-vehicle interaction scenarios~\cite{liu2022implicit, geruschat2003gaze, DeyGazePatterns2019, de2021pedestrians}.



% In the field of human-vehicle interaction research, the analysis of gaze behavior has mostly focused on encounter scenarios between pedestrians and vehicles.
% Gaze plays a fundamental role in pedestrian-vehicle communication \cite{geruschat2003gaze, zhao2023pedestrian}. 
% Pedestrian safety and traffic efficiency largely depend on pedestrians' ability to accurately interpret the motion and intent of approaching vehicles.
% Traditionally, pedestrians make crossing decisions based on vehicle kinematics \cite{moore2019case}. 
% Moreover, pedestrians often rely on eye contact with drivers to communicate and express their intention~\cite{rasouli2017agreeing}. 



%gaze bahavior and personality.