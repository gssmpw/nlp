\section{Related Work}
\label{related work}
Addressing Backdoor vulnerabilities remains a significant challenge, as existing defenses often compromise the semantic accuracy of clean inputs. For example, techniques such as neuron pruning and model fine-tuning aim to remove malicious neurons introduced by Backdoor attacks. However, these methods often come at the cost of performance degradation, reducing the accuracy and fidelity of clean sample reconstruction \cite{10622193}. Peng \emph{et at.} introduced an adversarial reinforcement learning-based defense framework for task-oriented multi-user semantic communication systems. Their method employs adversarial training to simulate poisoning attacks and iteratively improve model robustness, enabling the system to effectively detect and counteract poisoned data while maintaining communication efficiency\cite{10643310}. While effective, this methods rely on extensive adversarial training, which significantly increase computational overhead. This approach is therefore not suitable for real-time semantic communication tasks, where low latency and high efficiency are key. More recently, Ishmam  \emph{et al.} introduced ``Semantic Shield", which enforces knowledge alignment between image patches and captions to defend against backdoor attacks, being limited to tasks involving paired image-text data. \cite{10655192}. However, semantic communication systems typically operate on low-dimensional semantic vectors that represent compressed meanings without accompanying textual descriptions. 

Due to the above shortcomings, existing methods are not suitable for semantic communication systems. In this work, we are motivated to design a new defense mechanism that can effectively detect Backdoor attacks without modifying the structure of machine learning model or imposing data format constraints. 

\begin{figure}[htbp]
\centering
\begin{subfigure}[b]{\linewidth}
\centering
\includegraphics[width=1\columnwidth]{semantic_communication_framework.png}
\caption{Flow of communication system}
\end{subfigure}

\begin{subfigure}[b]{\linewidth}
\centering
\includegraphics[width=1\columnwidth]{semantic_communication_framework_2.png}
\caption{Poisoned training phase II}
\end{subfigure}
\caption{Semantic Communication Framework.}
\label{framework 1}

\end{figure}