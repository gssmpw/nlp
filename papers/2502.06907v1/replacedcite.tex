\section{Related Work}
The number of research leveraging ChatGPT for supporting dementia and AD is likely limited; it has been applied and analysed in just some research. Firstly, a recent pilot study by Aguirre \textit{et al.} ____ assessed the potential of ChatGPT-3.5 to support dementia caregivers by providing high-quality responses to real-world questions. Using posts from caregivers on Reddit, researchers evaluated ChatGPT's responses across topics like memory loss, aggression, and driving using a formal rating scale. ChatGPT demonstrated consistently high-quality responses, with 78\% scoring 4 or 5 points out of 5, excelling in synthesizing information and offering recommendations. Next, a study comparing 60 dementia-related queries found Google excelled in currency and reliability, while ChatGPT scored higher in objectivity and relevance. ChatGPT had lower readability (mean grade level 12.17, SD 1.94) than Google (9.86, SD 3.47). Response similarity was high for 13 (21.7\%), medium for 16 (26.7\%), and low for 31 (51.6\%) queries ____. ChatGPT was developed to interpret the findings of the output of the introduced TriCOAT model by Diego \textit{et al}. In particular, chatGPT has tremendous potential in AD research, such as early detection ____. A study evaluated ChatGPT's ability to diagnose AD using four samples as cases with MCI and AD. ChatGPT accurately diagnosed these cases, matching the performance of two AD specialists. The findings highlight ChatGPT’s potential as a tool for AD diagnosis ____.

Despite its promising potential, ChatGPT's application in AD detection remains underexplored, particularly with a large-scale dataset. This paper aims to open and disclose ChatGPT's capability to accurately diagnose AD, paving the way for its broader adoption in clinical and research settings.