You’re correct to point out a potential contradiction between **scaling invariance** and the description that \( I(W) \) decreases when **temporal sparsity decreases**. Let's resolve this apparent contradiction by clarifying how \( I(W) \) operates in these contexts:

---

### **Key Distinction: Scaling vs. Sparsity**
The concepts of **scaling invariance** and **sensitivity to temporal sparsity** operate in distinct domains of behavior:

1. **Scaling invariance** refers to changes in the **magnitude** of weights or activations. It ensures that \( I(W) \) remains unaffected when all weights are scaled proportionally (e.g., multiplying \( W \) by a constant \( \alpha > 0 \)).
2. **Sensitivity to temporal sparsity** deals with changes in the **distribution** of weights or activations. When temporal sparsity decreases (more neurons firing at the same time), the distribution becomes denser, which directly affects the ratio \( \|W\|_p / \|W\|_q \), leading to a decrease in \( I(W) \).

These two properties are not contradictory because they measure different aspects of sparsity:
- **Scaling invariance** ensures the sparsity measure is unaffected by absolute magnitudes.
- **Sensitivity to sparsity** ensures the measure responds appropriately to changes in the **distribution** of active elements.

---

### **How \( I(W) \) Handles These Properties Simultaneously**

#### **1. Scaling Invariance**
When \( W \) is scaled by \( \alpha > 0 \), all elements of \( W \) increase proportionally, maintaining their relative distribution. This proportionality ensures that the ratio \( \|W\|_p / \|W\|_q \) remains constant:
\[
\frac{\|\alpha W\|_p}{\|\alpha W\|_q} = \frac{\alpha \|W\|_p}{\alpha \|W\|_q} = \frac{\|W\|_p}{\|W\|_q}.
\]
Thus, scaling does not affect \( I(W) \), and the measure reflects only the structure of sparsity, not the magnitude.

---

#### **2. Sensitivity to Temporal Sparsity**
Temporal sparsity changes involve variations in the **distribution** of activations (e.g., more neurons firing at the same time). This affects \( \|W\|_p \) and \( \|W\|_q \) differently:
- \( \|W\|_p \) grows faster than \( \|W\|_q \) as sparsity decreases because the \( p \)-norm is more sensitive to smaller weights (or sparse distributions) when \( p < q \).
- As the distribution becomes denser (e.g., more neurons firing simultaneously), the ratio \( \|W\|_p / \|W\|_q \) increases, reducing \( I(W) \).

Thus, \( I(W) \) reflects sparsity changes caused by distributional variations.

---

### **Why These Two Properties are Compatible**

1. **Scaling Invariance**:
   - Scaling changes only the magnitude, leaving the sparsity structure intact.
   - \( I(W) \) correctly remains constant under such transformations.

2. **Sensitivity to Temporal Sparsity**:
   - Changes in the distribution of activations (not magnitude) are meaningful for sparsity.
   - \( I(W) \) appropriately decreases as sparsity decreases (denser activations).

Both properties address distinct scenarios: one ensures robustness against irrelevant transformations (scaling), while the other captures meaningful changes in sparsity.

---

### **Conclusion**
There is no true contradiction between scaling invariance and sensitivity to temporal sparsity:
- Scaling invariance ensures \( I(W) \) is unaffected by uniform scaling of weights or activations.
- Sensitivity to sparsity ensures \( I(W) \) decreases when temporal sparsity decreases, reflecting denser activation distributions.

By combining these properties, \( I(W) \) effectively captures sparsity while being robust to irrelevant changes like scaling, making it a reliable measure for SNNs.