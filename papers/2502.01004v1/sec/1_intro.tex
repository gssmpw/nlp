\section{Introduction}
\label{sec:intro}
%
6D object pose estimation plays a pivotal role for robots to interact with objects in real-world environments robustly~\cite{MVBPICRA, ST6DICRA, st6deccv, robi}. As a crucial yet challenging practice application, bin-picking requires robots to first accurately detect the workpieces and estimate their 6D poses before grasping. Herein the workpieces are typically textureless and stacked in a bin randomly. Numerous learning-based methods~\cite{mpaae, aae, ST6DICRA, st6deccv,miretr, dcnet} have been proposed to \noteb{handle this task} by training an object-specific model \noteb{for each workpiece}. When deployed for a novel object, these methods need days to collect real-world data or generate synthetic data for this object and retrain the model. Such an expensive and time-consuming deployment process limits the practical application of these approaches. 
\note{Besides, these methods may fail when target objects are only available during inference.}

Recently, zero-shot 6D pose estimation~\cite{chen2023zeropose, sam6d, labbe2022megapose} has been widely discussed. It aims to detect unseen objects and estimate their pose given only the CAD models, alleviating the deployment efficiency issue. Nevertheless, existing zero-shot methods are typically designed for daily applications, such as augmented reality. They mainly leverage the rich texture information on daily objects and perform RGB-D feature matching between the detected instance and the CAD model for pose estimation. However, for the manufactured workpieces in bin-picking, which lack sufficient texture and contain ambiguous local regions with similar shapes and appearances, these zero-shot methods driven by local feature matching usually suffer from mismatches, as shown in Fig.~\ref{fig:fig1} (b). 
\noteb{The pose estimation accuracy of these zero-shot methods can be substantially degenerated by the noisy correspondence.} It is necessary to design a zero-shot method specifically for bin-picking.


\begin{figure}[t]
    \centering
    \includegraphics[width=0.49 \textwidth]{figures/fig1v4.jpg}
    \caption{(a) The brief overview of ZeroBP. Given the CAD model of an unseen object, the model can be generalized to the real-world bin-picking scene for 6D pose estimation without requiring retraining. (b) The existing local feature matching for 6D pose estimation. (c) The proposed learning position-aware correspondence for 6D pose estimation. \note{"GPEnc" refers to Globally Positional Encoding, and "PA Cross-Attention" refers to Position-Aware Cross-Attention.}}
    \label{fig:fig1}
    \vspace{-2mm}
\end{figure}

%
In this paper, we propose a zero-shot 6D pose estimation method for bin-picking, named ZeroBP, which learns robust Position-Aware Correspondence (PAC) between the scene instance and CAD model, as shown in Fig.~\ref{fig:fig1} (a) and (c). It leverages both the local features and global positions to distinguish between two points with similar shapes and appearances but located far apart. The key to PAC is encoding the global position of the heterogeneous point clouds, \emph{i.e.}, the points of the scene instance and CAD model, in a comparable manner. That is, we need to project the heterogeneous point clouds into a shared coordinate system to encode the global position, whereas the projection itself requires knowing the pose. This raises an intriguing cyclical dependency issue between pose and global position. To address this issue, we start with an approximate initial pose and alternately refine the pose and global position iteratively. \noteb{To represent the point position, we propose a multiplicative positional encoding, defined as the directional vector from the object centroid to the surface point. To leverage the point position, we design a position-aware cross-attention, which effectively integrates positional encodings with local features for correspondence modeling.} During alternate refinement, the multiplicative positional encoding and position-aware cross-attention allow ZeroBP to gradually establish robust correspondence.

\noteb{Experiments on the real-world dataset ROBI~\cite{robi} demonstrate that ZeroBP
outperforms state-of-the-art zero-shot 6D pose estimation methods, achieving an improvement of 9.1\% in average recall of correct pose.} To summarize, we make the following contributions:
\begin{itemize}
    \item \note{We propose a zero-shot 6D pose estimation method for bin-picking, which learns robust position-aware correspondence to alleviate mismatch in ambiguous regions.}
    \item We design the multiplicative positional encoding and position-aware cross-attention to effectively represent and leverage the global position for pose estimation.
    \item Extensive experiments show that our method substantially improves the pose estimation accuracy and outperforms state-of-the-art zero-shot methods.
\end{itemize}
