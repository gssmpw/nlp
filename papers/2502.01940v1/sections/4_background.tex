\section{Background}
% \vspace{-11px}
\subsection{Radar Detectors}
% \vspace{-11px}
% A typical radar pipeline is composed of three modules: the hardware setup, the signal processor, and the detector (point cloud extractor). Each is explained below:

% \textbf{Hardware Setup: }The radar hardware setup features a 1D or 2D antenna array for transmitting and receiving electromagnetic signals. At its core is a transceiver, which generates high-power RF signals, transmits them, and switches to receive mode to capture and digitize reflected echoes for post-signal processing \cite{richards2010principles}.

% \textbf{Signal Processor: }The radar signal processor extracts key information—range, velocity, and angle—from digitized signals. Range is calculated from echo time delays, velocity from Doppler frequency shifts, and angle through beamforming techniques like Bartlett's method or FFT. Together, these processes convert raw signals into actionable radar data.
Radar detectors process signals to differentiate targets from noise by applying a decision threshold, comparing the signal strength to a predefined value, and outputting results as point clouds. Conventional radar detectors, like CA-CFAR and OS-CFAR, aim to maintain a consistent false alarm rate by dynamically adjusting decision thresholds based on surrounding noise and clutter. CA-CFAR works well in homogeneous environments but struggles with the heterogeneity of vehicular surroundings. OS-CFAR handles heterogeneous environments but requires precise prior knowledge of target numbers, which might be challenging to estimate. However, a recent deep learning-based radar detector by \cite{roldan2024see} trained on lidar point clouds addresses these limitations, but struggles with sparse representations of its surroundings, compared to lidar. In this work, we produce denser depth maps using these sparse representations.

%-----------------------%
\subsection{Bartlett's Algorithm for Spatial Power Spectrum Estimation}
Bartlett's algorithm, also known as periodogram averaging \cite{Bartlett1948}, is used in signal processing and time series analysis to estimate the power spectral density of a random sequence. It divides the sequence into \( M \) overlapping segments, computes their periodograms, and averages them, with the number of segments being proportional to the spectral resolution. For received signals at spatially distant receptors, like signals received at different camera physical pixels or signals received in multi-antenna wireless systems, the $M$ segments correspond to the signals received at the $M$ receptor. The time difference between segments introduces a phase shift proportional to the spatial frequency \(\omega\), referenced against the complex sinusoidal signal at the first receptor\cite{AoA}, described as:

\begin{equation}
x_1(n) = e^{-j\omega n} , n=0,1,...,N
\end{equation}
where $N$ is the number of samples in the signal. Assuming that we have $M$ segments, each segment has a time delay that is translated into a phase shift in $x_1(n)$, expressed as:
\begin{equation}
x_m(n) = x_1(n)e^{-jm\phi} =e^{-j(\omega n + m\phi)} , m=0,1,...,M-1
\end{equation}
Hence, the $M$ segments matrix $\textbf{S}\in \mathbb{C}^{N\times M}$ is found as:
\begin{equation}
    \textbf{S} = \begin{bmatrix} \textbf{x}_{1} & \textbf{x}_{2} & \cdots & \textbf{x}_{M}  \end{bmatrix}
\end{equation}
Since we are computing the spatial power spectrum, our goal is to calculate the spectrum of the signal that is described by relative phases between segments, which can be found in the covariance matrix $\textbf{C}\in \mathbb{C}^{M\times M}$ of $\textbf{S}$ as:
\begin{equation}
    \textbf{C}= \frac{1}{N}\textbf{S}^H\textbf{S}
\end{equation}
where $H$ denotes the Hermitian or complex conjugate. Therefore, the power spectral density at $\phi$, $P(\phi)$, is found as:
\begin{equation}
    P(\phi) = \textbf{a}(\phi)^H\textbf{C}\textbf{a}(\phi)
\end{equation}
where $\textbf{a}(\phi) = \begin{bmatrix} 1 & e^{-j\phi} & e^{-j2\phi} & \cdots & e^{-j(M-1)\phi}  \end{bmatrix}^T$ is our basis vector. Note that the choice of basis vector depends on the application, pattern of interest, and desired resolution \cite{priestley1981spectral}. A different basis vector is used for our proposed approach, defined in the next section.
% Therefore, the Discrete Fourier Transform (DFT) \cite{mitra2006digital} can be used to calculate the spatial power spectrum as:
% \begin{equation}
%     \textbf{P}(\omega)= \left|\sum_{k=0}^{K-1} \textbf{FCF}^H\right|
% \end{equation}
% where $\textbf{F}$ is the DFT matrix and $K$ is its number of spatial frequency bins ranging from 0 to $M$ with a step of $\frac{m}{M}$.


