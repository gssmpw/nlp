\documentclass[conference]{IEEEtran}
\IEEEoverridecommandlockouts
% The preceding line is only needed to identify funding in the first footnote. If that is unneeded, please comment it out.
%Template version as of 6/27/2024

\usepackage{cite}
\usepackage{amsmath,amssymb,amsfonts}
\usepackage{algorithmic}
\usepackage{graphicx}
\usepackage{textcomp}
\usepackage{xcolor}
\usepackage{graphicx}
\usepackage{subcaption}
\usepackage[font=small]{caption}
\usepackage{algorithm}
\usepackage{algorithmic}
\usepackage{subcaption}
\usepackage{amsmath,algorithm}
\usepackage{mathrsfs}
\usepackage{subcaption}  % For subfigures
\usepackage{graphicx}    % For including graphics
\usepackage{xcolor}
\usepackage{booktabs} % For thicker lines
\usepackage{tabu}
\usepackage{tabularray}
% \usepackage{slashbox}






\def\BibTeX{{\rm B\kern-.05em{\sc i\kern-.025em b}\kern-.08em
    T\kern-.1667em\lower.7ex\hbox{E}\kern-.125emX}}
\begin{document}

\title{Toward a Low-Cost Perception System in Autonomous Vehicles: A Spectrum Learning Approach\\
}

\author{
    %Authors
    % All authors must be in the same font size and format.
    Mohammed Alsakabi\textsuperscript{\rm 1},
    Aidan Erickson\textsuperscript{\rm 1},
    John M. Dolan\textsuperscript{\rm 2},
    Ozan K. Tonguz\textsuperscript{\rm 1} \\
    
    \textsuperscript{\rm 1}Department of Electrical and Computer Engineering, College of Engineering \\
    \textsuperscript{\rm 2}The Robotics Institute, School of Computer Science \\
    Carnegie Mellon University, Pittsburgh, PA, United States\\
    \{malsakabi, aerickson\}@cmu.edu , \{jdolan, tonguz\}@andrew.cmu.edu
}

\maketitle
In practice,  physical spatiotemporal forecasting can suffer from data scarcity, because collecting large-scale data is non-trivial, especially for extreme events. 
Hence, we propose \method{}, a novel probabilistic framework to realize iterative self-training with new self-ensemble strategies, 
achieving better physical consistency and generalization on extreme events. 
Following any base forecasting model, 
we can encode its deterministic outputs into a latent space and retrieve multiple codebook entries to generate probabilistic outputs. 
Then \method{} extends the beam search from discrete spaces to the continuous state spaces in this field.
We can further employ domain-specific metrics (e.g., Critical Success Index for extreme events) to filter out the top-k candidates and develop the new self-ensemble strategy by combining the high-quality candidates. 
The self-ensemble can not only improve the inference quality and robustness but also iteratively augment the training datasets during continuous self-training. 
Consequently, \method{} realizes the exploration of rare but critical phenomena beyond the original dataset. 
Comprehensive experiments on different benchmarks and backbones show that \method{} consistently reduces forecasting MSE (up to 39\%), enhancing extreme events detection and proving its effectiveness in handling data scarcity. Our codes are available at~\url{https://github.com/easylearningscores/BeamVQ}.



% 在气象预报、流体模拟以及基于偏微分方程(PDE)的多物理系统模型中,数据稀缺下的时空预测仍然是一个关键挑战。本文提出了\method{},一个统一的框架,旨在同时解决标注数据有限以及在确保物理一致性的前提下捕捉极端事件的难题。首先,我们训练了一个确定性的基础模型,从小规模数据中学习主要动力学。随后,通过Top-K 向量量化变分自编码器(VQ-VAE)对基础模型的输出进行增强,该模块将确定性预测编码到潜在空间,并检索多个码本条目以生成多样化且物理上合理的重构结果。一个新颖的联合优化过程利用领域特定的指标(例如关键成功指数)引导基础模型向更准确且对极端事件敏感的预测方向优化。在推理阶段,我们采用束搜索策略,维持多个候选轨迹并通过指标感知评分进行迭代剪枝,从而在探索罕见但关键现象与利用最可能的系统轨迹之间实现平衡。在多个气象和流体流动基准数据集上的大量实验表明,\method{}显著提升了预测精度,增强了对极端状态的检测能力,并保持了物理合理性,证明了其在数据稀缺场景下进行时空预测的优越性。

%%%%%%%%%%%%%%%%%%%%%%%%%%%%%%%%%%%%%%%%%%%%%%%%%%%%%%%%%%%%%%%%%%%%%%%%
\section{Introduction}
Recommender systems are essential components of contemporary digital landscape, enabling personalized services across a diverse range of fields, including e-commerce, social media, and entertainment \cite{zhang2023robust}. The data in RS generally consist of both structural information (\textit{e.g.}, user-item interactions) and textual information (\textit{e.g.}, user attributes and item descriptions).
% Based on technology and utilization of data, RS can be divided into three categories: content-based RS, collaborative filtering RS and hybrid filtering RS~\cite{adomavicius2005toward}.
\begin{figure}[ht]
    \centering
    \includegraphics[width=\linewidth]{figures/overview-yuanhao.pdf}
    \caption{An overview of GFM-based RS. Compared with GNN-based or LLM-based RS, GFM-based RS are positioned as integrating both approaches to create more comprehensive recommendations.}
    \label{fig:overview}
\end{figure}
With the rapid development of graph learning, GNN-based methods have emerged as an important technology in RS, which can further enhance the collaborative signals of collaborative filtering and extend the signals to higher-order structures and external knowledge~\cite{wu2022graph}. However, due to the inherent structural bias, they struggle to handle textual information.
This is where the powerful capabilities of large language models, which have made significant impacts in the field of natural language processing (NLP) and come into play in the realm of RS~\cite{yang2023palr,zhai2024actions}. Leveraging the advanced text capabilities of LLM, these methods efficiently capture user and item textual information while integrating world knowledge for improved recommendations. However, their reasoning limitations restrict the collaborative signals they can comprehend.
Inspired by the success of LLM in the NLP field, the graph domain has also been undergoing transformation, leading to the emergence of graph foundation models (GFMs)~\cite{liu2023towards}. By integrating GNN and LLM technologies, GFM-based RS can efficiently utilize data to align user preferences and make more precise recommendations with minimized bias, as depicted in Figure~\ref{fig:overview}. By appropriately integrating key information from both graph structures and text, GFM-based RS hold significant potential to emerge as a new paradigm in RS.
% \section{Taxonomy}

% As illustrated by Fig. \ref{}, the typical process of vision models based time series analysis has five components: (1) normalization/scaling; (2) time series to image transformation; (3) image modeling; (4) image to time series recovery; and (5) task processing. In the rest of this paper, we will discuss the typical methods for each of these components. The detailed taxonomy of the methods are summarized in Table \ref{tab.taxonomy}.

%Typical step: normalization/scaling, transformation, vision modeling, task-specific head, inverse transformation (for tasks that output time series, e.g., forecasting, generation, imputation, anomaly detection). Normalization is to fit the arbitrary range of time series values to RGB representation.

\begin{figure*}[!t]
\centering
\includegraphics[width=1.0\textwidth]{fig/fig_3.pdf}
% \vspace{-1em}
\caption{An illustration of different methods for imaging time series with a sample (length=336) from the \textit{Electricity} benchmark dataset \protect\cite{nie2023time}. (a)(c)(d)(e)(f) %are univariate methods.
visualize the same variate. (b) visualizes all 321 variates. Filterbank is omitted due to its %high
similarity to STFT.}\label{fig.tsimage}
\vspace{-0.2cm}
\end{figure*}

\begin{table*}[t]
\centering
\scriptsize
\setlength{\tabcolsep}{2.7pt}{
% \begin{tabular}{llllllllllll}
\begin{tabular}{llcccccccccl}
\toprule[1pt]
\multirow{2}{*}{Method} & \multirow{2}{*}{TS-Type} & \multirow{2}{*}{Imaging} & \multicolumn{5}{c}{Imaged Time Series Modeling} & \multirow{2}{*}{TS-Recover} & \multirow{2}{*}{Task} & \multirow{2}{*}{Domain} & \multirow{2}{*}{Code}\\ \cmidrule{4-8}
 & & & Multi-modal & Model & Pre-trained & Fine-tune & Prompt & & & & \\ \midrule
\cite{silva2013time} & UTS & RP & \xmark & \texttt{K-NN} & \xmark & \xmark & \xmark & \xmark & Classification & General & \xmark\\
\cite{wang2015encoding} & UTS & GAF & \xmark & \texttt{CNN} & \xmark & \cmark$^{\flat}$ & \xmark & \cmark & Classification & General & \xmark\\
\cite{wang2015imaging} & UTS & GAF & \xmark & \texttt{CNN} & \xmark & \cmark$^{\flat}$ & \xmark & \cmark & Multiple & General & \xmark\\
% \multirow{2}{*}{\cite{wang2015imaging}} & \multirow{2}{*}{UTS} & \multirow{2}{*}{GAF} & \multirow{2}{*}{\xmark} & \multirow{2}{*}{\texttt{CNN}} & \multirow{2}{*}{\xmark} & \multirow{2}{*}{\cmark$^{\flat}$} & \multirow{2}{*}{\xmark} & \multirow{2}{*}{\cmark} & Classification & \multirow{2}{*}{General} & \multirow{2}{*}{\xmark}\\
% & & & & & & & & & \& Imputation & & \\
\cite{ma2017learning} & MTS & Heatmap & \xmark & \texttt{CNN} & \xmark & \cmark$^{\flat}$ & \xmark & \cmark & Forecasting & Traffic & \xmark\\
\cite{hatami2018classification} & UTS & RP & \xmark & \texttt{CNN} & \xmark & \cmark$^{\flat}$ & \xmark & \xmark & Classification & General & \xmark\\
\cite{yazdanbakhsh2019multivariate} & MTS & Heatmap & \xmark & \texttt{CNN} & \xmark & \cmark$^{\flat}$ & \xmark & \xmark & Classification & General & \cmark\textsuperscript{\href{https://github.com/SonbolYb/multivariate_timeseries_dilated_conv}{[1]}}\\
MSCRED \cite{zhang2019deep} & MTS & Other ($\S$\ref{sec.othermethod}) & \xmark & \texttt{ConvLSTM} & \xmark & \cmark$^{\flat}$ & \xmark & \xmark & Anomaly & General & \cmark\textsuperscript{\href{https://github.com/7fantasysz/MSCRED}{[2]}}\\
\cite{li2020forecasting} & UTS & RP & \xmark & \texttt{CNN} & \cmark & \cmark & \xmark & \xmark & Forecasting & General & \cmark\textsuperscript{\href{https://github.com/lixixibj/forecasting-with-time-series-imaging}{[3]}}\\
\cite{cohen2020trading} & UTS & LinePlot & \xmark & \texttt{Ensemble} & \xmark & \cmark$^{\flat}$ & \xmark & \xmark & Classification & Finance & \xmark\\
% \cite{du2020image} & UTS & Spectrogram & \xmark & \texttt{CNN} & \xmark & \cmark$^{\flat}$ & \xmark & \xmark & Classification & Finance & \xmark\\
\cite{barra2020deep} & UTS & GAF & \xmark & \texttt{CNN} & \xmark & \cmark$^{\flat}$ & \xmark & \xmark & Classification & Finance & \xmark\\
% \cite{barra2020deep} & UTS & GAF & \xmark & \texttt{VGG-16} & \xmark & \cmark$^{\flat}$ & \xmark & \xmark & Classification & Finance & \xmark\\
% \cite{cao2021image} & UTS & RP & \xmark & \texttt{CNN} & \xmark & \cmark$^{\flat}$ & \xmark & \xmark & Classification & General & \xmark\\
VisualAE \cite{sood2021visual} & UTS & LinePlot & \xmark & \texttt{CNN} & \xmark & \cmark$^{\flat}$ & \xmark & \cmark & Forecasting & Finance & \xmark\\
% VisualAE \cite{sood2021visual} & UTS & LinePlot & \xmark & \texttt{CNN} & \xmark & \cmark$^{\flat}$ & \xmark & \xmark & Img-Generation & Finance & \xmark\\
\cite{zeng2021deep} & MTS & Heatmap & \xmark & \texttt{CNN,LSTM} & \xmark & \cmark$^{\flat}$ & \xmark & \cmark & Forecasting & Finance & \xmark\\
% \cite{zeng2021deep} & MTS & Heatmap & \xmark & \texttt{SRVP} & \xmark & \cmark$^{\flat}$ & \xmark & \cmark & Forecasting & Finance & \xmark\\
AST \cite{gong2021ast} & UTS & Spectrogram & \xmark & \texttt{DeiT} & \cmark & \cmark & \xmark & \xmark & Classification & Audio & \cmark\textsuperscript{\href{https://github.com/YuanGongND/ast}{[4]}}\\
TTS-GAN \cite{li2022tts} & MTS & Heatmap & \xmark & \texttt{ViT} & \xmark & \cmark$^{\flat}$ & \xmark & \cmark & Ts-Generation & Health & \cmark\textsuperscript{\href{https://github.com/imics-lab/tts-gan}{[5]}}\\
SSAST \cite{gong2022ssast} & UTS & Spectrogram & \xmark & \texttt{ViT} & \cmark$^{\natural}$ & \cmark & \xmark & \xmark & Classification & Audio & \cmark\textsuperscript{\href{https://github.com/YuanGongND/ssast}{[6]}}\\
MAE-AST \cite{baade2022mae} & UTS & Spectrogram & \xmark & \texttt{MAE} & \cmark$^{\natural}$ & \cmark & \xmark & \xmark & Classification & Audio & \cmark\textsuperscript{\href{https://github.com/AlanBaade/MAE-AST-Public}{[7]}}\\
AST-SED \cite{li2023ast} & UTS & Spectrogram & \xmark & \texttt{SSAST,GRU} & \cmark & \cmark & \xmark & \xmark & EventDetection & Audio & \xmark\\
\cite{jin2023classification} & UTS & %Multiple
LinePlot & \xmark & \texttt{CNN} & \cmark & \cmark & \xmark & \xmark & Classification & Physics & \xmark\\
ForCNN \cite{semenoglou2023image} & UTS & LinePlot & \xmark & \texttt{CNN} & \xmark & \cmark$^{\flat}$ & \xmark & \xmark & Forecasting & General & \xmark\\
Vit-num-spec \cite{zeng2023pixels} & UTS & Spectrogram & \xmark & \texttt{ViT} & \xmark & \cmark$^{\flat}$ & \xmark & \xmark & Forecasting & Finance & \xmark\\
% \cite{wimmer2023leveraging} & MTS & LinePlot & \xmark & \texttt{CLIP,LSTM} & \cmark & \cmark & \xmark & \xmark & Classification & Finance & \xmark\\
ViTST \cite{li2023time} & MTS & LinePlot & \xmark & \texttt{Swin} & \cmark & \cmark & \xmark & \xmark & Classification & General & \cmark\textsuperscript{\href{https://github.com/Leezekun/ViTST}{[8]}}\\
MV-DTSA \cite{yang2023your} & UTS\textsuperscript{*} & LinePlot & \xmark & \texttt{CNN} & \xmark & \cmark$^{\flat}$ & \xmark & \cmark & Forecasting & General & \cmark\textsuperscript{\href{https://github.com/IkeYang/machine-vision-assisted-deep-time-series-analysis-MV-DTSA-}{[9]}}\\
TimesNet \cite{wu2023timesnet} & MTS & Heatmap & \xmark & \texttt{CNN} & \xmark & \cmark$^{\flat}$ & \xmark & \cmark & Multiple & General & \cmark\textsuperscript{\href{https://github.com/thuml/TimesNet}{[10]}}\\
ITF-TAD \cite{namura2024training} & UTS & Spectrogram & \xmark & \texttt{CNN} & \cmark & \xmark & \xmark & \xmark & Anomaly & General & \xmark\\
\cite{kaewrakmuk2024multi} & UTS & GAF & \xmark & \texttt{CNN} & \cmark & \cmark & \xmark & \xmark & Classification & Sensing & \xmark\\
HCR-AdaAD \cite{lin2024hierarchical} & MTS & RP & \xmark & \texttt{CNN,GNN} & \xmark & \cmark$^{\flat}$ & \xmark & \xmark & Anomaly & General & \xmark\\
FIRTS \cite{costa2024fusion} & UTS & Other ($\S$\ref{sec.othermethod}) & \xmark & \texttt{CNN} & \xmark & \cmark$^{\flat}$ & \xmark & \xmark & Classification & General & \cmark\textsuperscript{\href{https://sites.google.com/view/firts-paper}{[11]}}\\
% \multirow{2}{*}{FIRTS \cite{costa2024fusion}} & \multirow{2}{*}{UTS} & Spectrogram & \multirow{2}{*}{\xmark} & \multirow{2}{*}{\texttt{CNN}} & \multirow{2}{*}{\xmark} & \multirow{2}{*}{\cmark$^{\flat}$} & \multirow{2}{*}{\xmark} & \multirow{2}{*}{\xmark} & \multirow{2}{*}{Classification} & \multirow{2}{*}{General} & \multirow{2}{*}{\cmark\textsuperscript{\href{https://sites.google.com/view/firts-paper}{[2]}}}\\
%  & & \& GAF,RP,MTF & & & & & & & & & \\
% \cite{homenda2024time} & UTS\textsuperscript{*} & Multiple & \xmark & \texttt{CNN} & \xmark & \cmark$^{\flat}$ & \xmark & \xmark & Classification & General & \xmark\\
CAFO \cite{kim2024cafo} & MTS & RP & \xmark & \texttt{CNN,ViT} & \xmark & \cmark$^{\flat}$ & \xmark & \xmark & Explanation & General & \cmark\textsuperscript{\href{https://github.com/eai-lab/CAFO}{[12]}}\\
% \multirow{2}{*}{CAFO \cite{kim2024cafo}} & \multirow{2}{*}{MTS} & \multirow{2}{*}{RP} & \multirow{2}{*}{\xmark} & \texttt{ShuffleNet,ResNet} & \multirow{2}{*}{\cmark} & \multirow{2}{*}{\cmark} & \multirow{2}{*}{\xmark} & \multirow{2}{*}{\xmark} & Classification & \multirow{2}{*}{General} & \multirow{2}{*}{\cmark}\\
%  & & & & \texttt{MLP-Mixer,ViT} & & & & & \& Explanation & & \\
ViTime \cite{yang2024vitime} & UTS\textsuperscript{*} & LinePlot & \xmark & \texttt{ViT} & \cmark$^{\natural}$ & \cmark & \xmark & \cmark & Forecasting & General & \cmark\textsuperscript{\href{https://github.com/IkeYang/ViTime}{[13]}}\\
ImagenTime \cite{naiman2024utilizing} & MTS & Other ($\S$\ref{sec.othermethod}) & \xmark & %\texttt{Diffusion}
\texttt{CNN} & \xmark & \cmark$^{\flat}$ & \xmark & \cmark & Ts-Generation & General & \cmark\textsuperscript{\href{https://github.com/azencot-group/ImagenTime}{[14]}}\\
TimEHR \cite{karami2024timehr} & MTS & Heatmap & \xmark & \texttt{CNN} & \xmark & \cmark$^{\flat}$ & \xmark & \cmark & Ts-Generation & Health & \cmark\textsuperscript{\href{https://github.com/esl-epfl/TimEHR}{[15]}}\\
VisionTS \cite{chen2024visionts} & UTS\textsuperscript{*} & Heatmap & \xmark & \texttt{MAE} & \cmark & \cmark & \xmark & \cmark & Forecasting & General & \cmark\textsuperscript{\href{https://github.com/Keytoyze/VisionTS}{[16]}}\\ \midrule
InsightMiner \cite{zhang2023insight} & UTS & LinePlot & \cmark & \texttt{LLaVA} & \cmark & \cmark & \cmark & \xmark & Txt-Generation & General & \xmark\\
\cite{wimmer2023leveraging} & MTS & LinePlot & \cmark & \texttt{CLIP,LSTM} & \cmark & \cmark & \xmark & \xmark & Classification & Finance & \xmark\\
% \cite{dixit2024vision} & UTS & Spectrogram & \cmark & \texttt{GPT4o,Gemini} & \cmark & \xmark & \cmark & \xmark & Classification & Audio & \xmark\\
\multirow{2}{*}{\cite{dixit2024vision}} & \multirow{2}{*}{UTS} & \multirow{2}{*}{Spectrogram} & \multirow{2}{*}{\cmark} & \texttt{GPT4o,Gemini} & \multirow{2}{*}{\cmark} & \multirow{2}{*}{\xmark} & \multirow{2}{*}{\cmark} & \multirow{2}{*}{\xmark} & \multirow{2}{*}{Classification} & \multirow{2}{*}{Audio} & \multirow{2}{*}{\xmark}\\
 & & & & \& \texttt{Claude3} & & & & & & & \\
\cite{daswani2024plots} & MTS & LinePlot & \cmark & \texttt{GPT4o,Gemini} & \cmark & \xmark & \cmark & \xmark & Multiple & General & \xmark\\
TAMA \cite{zhuang2024see} & UTS & LinePlot & \cmark & \texttt{GPT4o} & \cmark & \xmark & \cmark & \xmark & Anomaly & General & \xmark\\
\cite{prithyani2024feasibility} & MTS & LinePlot & \cmark & \texttt{LLaVA} & \cmark & \cmark & \cmark & \xmark & Classification & General & \cmark\textsuperscript{\href{https://github.com/vinayp17/VLM_TSC}{[17]}}\\
\bottomrule[1pt]
\end{tabular}}
\vspace{-0.25cm}
\caption{Taxonomy of vision models on time series. The top panel includes single-modal models. The bottom panel includes multi-modal models. {\bf TS-Type} denotes type of time series. {\bf TS-Recover} denotes %whether time series recovery ($\S$\ref{sec.processing}) has been performed.
recovering time series from predicted images ($\S$\ref{sec.processing}). \textsuperscript{*}: %the model has been %applied on MTSs by %processing %modeling the individual UTSs of each MTS.
the method has been used to model the individual UTSs of an MTS. $^{\natural}$: a new pre-trained model was proposed in the work. $^{\flat}$: %without using a pre-trained model, fine-tune means training from scratch.
when pre-trained models were unused, ``Fine-tune'' refers to train a task-specific model from scratch. %In the
{\bf Model} column: \texttt{CNN} could be regular CNN, ResNet, VGG-Net, %U-Net,
{\em etc.}}\label{tab.taxonomy}
% The code only include verified official code from the authors.
\vspace{-0.3cm}
\end{table*}

\begin{table*}[t]
\centering
\small
\setlength{\tabcolsep}{2.9pt}{
\begin{tabular}{l|l|l|l}\hline
% \toprule[1pt]
\rowcolor{gray!20}
{\bf Method} & {\bf TS-Type} & {\bf Advantages} & {\bf Limitations}\\ \hline
Line Plot ($\S$\ref{sec.lineplot}) & UTS, MTS & matches human perception of time series & limited to MTSs with a small number of variates\\ \hline
Heatmap ($\S$\ref{sec.heatmap}) & UTS, MTS & straightforward for both UTSs and MTSs & the order of variates may affect their correlation learning\\ \hline
Spectrogram ($\S$\ref{sec.spectrogram}) & UTS & encodes the time-frequency space & limited to UTSs; needs a proper choice of window/wavelet\\ \hline
GAF ($\S$\ref{sec.gaf}) & UTS & encodes the temporal correlations in a UTS & limited to UTSs; $O(T^{2})$ time and space complexity\\ \hline% for long time series\\ \hline
% RP ($\S$\ref{sec.rp}) & UTS & flexibility in image size by tuning $m$ and $\tau$ & limited to UTSs; the pattern has a threshold-dependency\\ \hline
RP ($\S$\ref{sec.rp}) & UTS & flexibility in image size by tuning $m$ and $\tau$ & limited to UTSs; information loss after thresholding\\ \hline
% \bottomrule[1pt]
\end{tabular}}
\vspace{-0.2cm}
\caption{Summary of the five primary methods for transforming time series to images. {\bf TS-Type} denotes type of time series.}\label{tab.tsimage}
\vspace{-0.2cm}
\end{table*}

\section{Time Series To Image Transformation}\label{sec.tsimage}

% This section summarizes 5 major methods for imaging time series ($\S$\ref{sec.lineplot}-$\S$\ref{sec.rp}). We also discuss some other methods ($\S$\ref{sec.othermethod}) and how to model MTS with these methods ($\S$\ref{sec.modelmts}).
This section summarizes the methods for imaging time series ($\S$\ref{sec.lineplot}-$\S$\ref{sec.othermethod}) and their extensions to encode MTSs ($\S$\ref{sec.modelmts}).

% This section summarizes 5 major methods for transforming time series to images, including Line Plot, Heatmap, Spetrogram, GAF and RP, and several minor methods. We discuss their pros and cons and how to deal with MTS.

% This section discusses the advantages and limitations of different methods for time series to image transformation (invertible, efficiency, information preservation, MTS, long-range time series, parametric, etc.).

%\subsection{Methods}

\vspace{-0.08cm}

\subsection{Line Plot}\label{sec.lineplot}

Line Plot is a straightforward way for visualizing UTSs for human analysis ({\em e.g.}, stocks, power consumption, {\em etc.}). As illustrated by Fig. \ref{fig.tsimage}(a), the simplest approach is to draw a 2D image with x-axis representing %the time horizon
time steps and y-axis representing %the magnitude of the normalized time series.
time-wise values, %A line is used to connect all values of the series over time.
with a line connecting all values of the series over time. This image can be %represented by either three-channel pixels or single-channel pixels
either three-channel ({\em i.e.}, RGB) or single-channel as the colors may not %provide additional information
be informative %\cite{cohen2020trading,sood2021visual,jin2023classification,zhang2023insight,zhuang2024see}.
\cite{cohen2020trading,sood2021visual,jin2023classification,zhang2023insight}. ForCNN \cite{semenoglou2023image} even uses a single 8-bit integer to represent each pixel for black-white images. So far, there is no consensus on whether other graphical components, such as legend, grids and tick labels, could provide extra benefits in any task. For example, ViTST \cite{li2023time} finds these components are superfluous in a classification task, while TAMA \cite{zhuang2024see} finds grid-like auxiliary lines help enhance anomaly detection.

In addition to the regular Line Plot, MV-DTSA \cite{yang2023your} and ViTime \cite{yang2024vitime} divide an image into $h\times L$ grids, %where $h$ is the number of rows and $L$ is the number of columns,
and %introduced
define a function to map each time step of a UTS to a grid, producing a grid-like Line Plot. Also, we include methods that use Scatter Plot \cite{daswani2024plots,prithyani2024feasibility} in this category because %the only difference between a Scatter Plot and a Line Plot is whether the time-wise values are connected by lines.
a Scatter Plot resembles a Line Plot but doesn't connect %time-wise values
data points with a line. By comparing them, \cite{prithyani2024feasibility} finds a Line Plot could induce better time series classification.

For MTSs, we defer the discussion on Line Plot to $\S$\ref{sec.modelmts}.

% For MTS, some methods use the channel-independence assumption proposed in \cite{nie2023time} and represent each variate in MTS with an individual Line Plot \cite{yang2023your,yang2024vitime}. ViTST \cite{li2023time} also uses an individual Line Plot per variate, but colors different lines and assembles all plots to form a bigger image. The method in \cite{wimmer2023leveraging} plots %the time series of
% all variates in a single Line Plot and distinguish them by %use different
% types of lines ({\em e.g.}, solid, dashed, dotted, {\em etc.}). %to distinguish them.
% However, these methods only work for a small number of variates. For example, in \cite{wimmer2023leveraging}, there are only 4 variates in its financial MTSs.

%\cite{li2023time} space-costly because of blank pixels. scatter plot.

%Invertible with a numeric prediction head \cite{sood2021visual}. It fits tasks such as forecasting, imputation, etc.

\vspace{-0.08cm}

\subsection{Heatmap}\label{sec.heatmap}

As shown in Fig. \ref{fig.tsimage}(b), Heatmap is a 2D visualization of the magnitude of the values in a matrix using color. %The variation of color represents the intensity of each value. %Therefore,
It has been used to %directly
represent the matrix of an MTS, {\em i.e.}, $\mat{X} \in \mathbb{R}^{d\times T}$, as a one-channel $d\times T$ image \cite{li2022tts,yazdanbakhsh2019multivariate}. Similarly, TimEHR \cite{karami2024timehr} represents an {\em irregular} MTS, where the intervals between time steps are uneven, as a $d\times H$ Heatmap image by grouping the uneven time steps into $H$ even time bins. In \cite{zeng2021deep}, a different method is used for visualizing a 9-variate financial %time series.
MTS. It reshapes the 9 variates at each time step to a $3\times 3$ Heatmap image, and uses the sequence of images to forecast future %image
frames, achieving %time series
%MTS
time series forecasting. In contrast, VisionTS \cite{chen2024visionts} uses Heatmap to visualize UTSs. %instead.
Similar to TimesNet \cite{wu2023timesnet}, it first segments a length-$T$ UTS into $\lfloor T/P\rfloor$ length-$P$ subsequences, where $P$ is a parameter representing a periodicity of the UTS. Then the subsequences are stacked into a $P\times \lfloor T/P\rfloor$ matrix, %and duplicated 3 times to produce a 3-channel
with 3 duplicated channels, to produce a grayscale image %which serves as an
input to %a vision foundation model.
an LVM. To encode MTSs, VisionTS adopts the channel independence assumption \cite{nie2023time} and individually models each variate in an MTS.

\vspace{0.2cm}

\noindent{\bf Remark.} Heatmap can be used to visualize matrices of various forms. It is also used for matrices generated by the subsequent methods ({\em e.g.}, Spectrogram, GAF, RP) in this section. In this paper, the name Heatmap refers specifically to images that use color to visualize the (normalized) values in UTS $\mat{x}$ or MTS $\mat{X}$ without performing other transformations.

%\cite{chen2024visionts,karami2024timehr} bin version of TSH \cite{karami2024timehr}, DE and STFT \cite{naiman2024utilizing} (DE can be used for constructing RP), rearrange variates for video version of TSH \cite{zeng2021deep}.

%\vspace{0.2cm}

\subsection{Spectrogram}\label{sec.spectrogram}

A {\em spectrogram} is a visual representation of the spectrum of frequencies of a signal as it varies with time, which are extensively used for analyzing audio signals \cite{gong2021ast}. Since audio signals are a type of UTS, spectrogram can be considered as a method for imaging a UTS. As shown in Fig. \ref{fig.tsimage}(c), a common format is a 2D heatmap image with x-axis representing time steps and y-axis representing frequency, {\em a.k.a.} a time-frequency space. %The color at each point
Each pixel in the image represents the (logarithmic) amplitude of a specific frequency at a specific time point. Typical methods for %transforming a UTS to
producing a spectrogram include {\bf Short-Time Fourier Transform (STFT)} \cite{griffin1984signal}, {\bf Wavelet Transform} \cite{daubechies1990wavelet}, and {\bf Filterbank} \cite{vetterli1992wavelets}.

\vspace{0.2cm}

\noindent{\bf STFT.} %Discrete Fourier transform (DFT) can be used to represent a UTS signal %$\mat{x}=[x_{1}, ..., x_{T}]$
%$\mat{x}\in\mathbb{R}^{1\times T}$ as a sum of sinusoidal components. The output of the transform is a function of frequency $f(w)$, describing the intensity of each constituent frequency $w$ of the entire UTS. 
Discrete Fourier transform (DFT) can be used to describe the intensity $f(w)$ of each constituent frequency $w$ of a UTS signal $\mat{x}\in\mathbb{R}^{1\times T}$. However, $f(w)$ has no time dependency. It cannot provide dynamic information such as when a specific frequency appear in the UTS. STFT addresses this deficiency by sliding a window function $g(t)$ over the time steps in %the UTS,
$\mat{x}$, and computing the DFT within each window by
\begin{equation}\label{eq.stft}
\small
\begin{aligned}
f(w,\tau) = \sum_{t=1}^{T}x_{t}g(t - \tau)e^{-iwt}
\end{aligned}
\end{equation}
where $w$ is frequency, $\tau$ is the position of the window, $f(w,\tau)$ describes the intensity of frequency $w$ at time step $\tau$.

%With a proper selection of the
By selecting a proper window function $g(\cdot)$ ({\em e.g.}, Gaussian/Hamming/Bartlett window), %({\em e.g.}, Gaussian window, Hamming window, Bartlett window), %{\em etc.}),
a 2D spectrogram ({\em e.g.}, Fig. \ref{fig.tsimage}(c)) can be drawn via a heatmap on the squared values $|f(w,\tau)|^{2}$, with $w$ as the y-axis, and $\tau$ as the x-axis. For example, \cite{dixit2024vision} uses STFT based spectrogram as an input to LMMs %\hh{do you mean LVMs? check}
for time series classification.

%Fourier transform is a powerful data analysis tool that represents any complex signal as a sum of sines and cosines and transforms the signal from the time domain to the frequency domain. However, Fourier transform can only show which frequencies are present in the signal, but not when these frequencies appear. The STFT divides original signal into several parts using a sliding window to fix this problem. STFT involves a sliding window for extracting frequency components within the window.

\vspace{0.2cm}

\noindent{\bf Wavelet Transform.} %Like Fourier transform, %\hh{this paragraph needs a citation}
Continuous Wavelet Transform (CWT) uses the inner product to measure the similarity between a signal function $x(t)$ and an analyzing function. %In STFT (Eq.~\eqref{eq.stft}), the analyzing function is a windowed exponential $g(t - \tau)e^{-iwt}$.
%In CWT,
The analyzing function is a {\em wavelet} $\psi(t)$, where the typical choices include Morse wavelet, Morlet wavelet, %Daubechies wavelet, %Beylkin wavelet, 
{\em etc.} %The
CWT compares $x(t)$ to the shifted and scaled ({\em i.e.}, stretched or shrunk) versions of the wavelet, and output a CWT coefficient by
\begin{equation}\label{eq.cwt}
\small
\begin{aligned}
c(s,\tau) = \int_{-\infty}^{\infty}x(t)\frac{1}{s}\psi^{*}(\frac{t - \tau}{s})dt
\end{aligned}
\end{equation}
where $*$ denotes complex conjugate, $\tau$ is the time step to shift, and $s$ represents the scale. In practice, a discretized version of CWT in Eq.~\eqref{eq.cwt} is implemented for UTS $[x_{1}, ..., x_{T}]$.

It is noteworthy that the scale $s$ controls the frequency encoded in a wavelet -- a larger $s$ leads to a stretched wavelet with a lower frequency, and vice versa. As such, by varying $s$ and $\tau$, a 2D spectrogram ({\em e.g.}, Fig. \ref{fig.tsimage}(d)) can be drawn %, often with a heatmap
on $|c(s,\tau)|$, where $s$ is the y-axis and $\tau$ is the x-axis. Compared to STFT, which uses a fixed window size, Wavelet Transform allows variable wavelet sizes -- a larger size %region
for more precise low frequency information. 
%Usually, $s$ and $\tau$ vary dependently -- a larger $s$ leads to a stretched wavelet that shifts slowly, {\em i.e.}, a smaller $\tau$. This property %of CWT
%yields a spectrogram that balances the resolutions of frequency %$s$
%and time, %$\tau$,
%which is an advantage over the fixed time resolution in STFT.
% Thus, both of the methods in %\cite{du2020image}
% \cite{namura2024training} and \cite{zeng2023pixels} choose CWT (with Morlet wavelet) to generate the spectrogram.
Thus, the methods in \cite{du2020image,namura2024training,zeng2023pixels} choose CWT (with Morlet wavelet) to generate the spectrogram.

%A wavelet is a wave-like oscillation that has zero mean and is localized in both time and frequency space.

\vspace{0.2cm}

\noindent{\bf Filterbank.} This method %is relevant to
resembles STFT and is often used in processing audio signals. Given an audio signal, it firstly goes through a {\em pre-emphasis filter} to boost high frequencies, which helps improve the clarity of the signal. Then, STFT is applied on the signal. %with a sliding window $g(t)$ of size $k$ that shifts in a fixed stride $\tau$. %where the adjacent windows may overlap in $k$ time length.
%Finally, filterbank features are computed by applying multiple ``triangle-shaped'' filters spaced on the Mel-scale to the STFT output $f(w, \tau)$. %where Mel-scale is a method to make the filters more discriminative on lower frequencies, %than higher frequencies,
%imitating the non-linear human ear perception of sound.
Finally, multiple ``triangle-shaped'' filters spaced on a Mel-scale are applied to the STFT power spectrum $|f(w, \tau)|^{2}$ to extract frequency bands. The outcome filterbank features $\hat{f}(w, \tau)$ can be used to yield a spectrogram with $w$ as the y-axis, and $\tau$ as the x-axis.

%Filterbank was introduced in AST \cite{gong2021ast} with %$k$=25ms
Filterbank was adopted in AST \cite{gong2021ast} with 
a 25ms Hamming window that shifts every 10ms for classifying audio signals using Vision Transformer (ViT). It then becomes widely used in the follow-up works such as SSAST \cite{gong2022ssast}, MAE-AST \cite{baade2022mae}, and AST-SED \cite{li2023ast}, as summarized in Table \ref{tab.taxonomy}.



%Use MLP to predict TS directly \cite{zeng2023pixels}.

%\vspace{0.2cm}

% \vspace{0.2cm}

\subsection{Gramian Angular Field (GAF)}\label{sec.gaf}

GAF was introduced for classifying UTSs using CNNs %using %image based CNNs
by \cite{wang2015encoding}. It was then extended %with an extension
to an imputation task in \cite{wang2015imaging}. Similarly, \cite{barra2020deep} applied GAF for financial time series forecasting.

Given a UTS $\mat{x}\in\mathbb{R}^{1\times T}$, %$[x_{1}, ..., x_{T}]$,
the first step %before GAF
is to rescale each $x_{t}$ to a value $\tilde{x}_{t}$ %in the interval of
within $[0, 1]$ (or $[-1, 1]$). %by min-max normalization.
This range enables mapping $\tilde{x}_{t}$ to polar coordinates by $\phi_{t}=\text{arccos}(\tilde{x}_{i})$, with a radius $r=t/N$ encoding the time stamp, where $N$ is a constant factor to regularize the span of the polar coordinates. %system. Then,
Two types of GAF, Gramian Sum Angular Field (GASF) and Gramian Difference Angular Field (GADF) are defined as
\begin{equation}\label{eq.gaf}
\small
\begin{aligned}
&\text{GASF:}~~\text{cos}(\phi_{t} + \phi_{t'})=x_{t}x_{t'} - \sqrt{1 - x_{t}^{2}}\sqrt{1 - x_{t'}^{2}}\\
&\text{GADF:}~~\text{sin}(\phi_{t} - \phi_{t'})=x_{t'}\sqrt{1 - x_{t}^{2}} - x_{t}\sqrt{1 - x_{t'}^{2}}
\end{aligned}
\end{equation}
which exploits the pairwise temporal correlations in the UTS. Thus, the outcome is a $T\times T$ matrix $\mat{G}$ with $\mat{G}_{t,t'}$ specified by either type in Eq.~\eqref{eq.gaf}. A GAF image is a heatmap on $\mat{G}$ with both axes representing time, as illustrated by Fig. \ref{fig.tsimage}(e).

% Invertible.

% \vspace{0.2cm}

\subsection{Recurrence Plot (RP)}\label{sec.rp}

%RP \cite{eckmann1987recurrence} is a method to encode a UTS into an image that aims to capture the periodic patterns in the UTS by using its reconstructed {\em phase space}. The phase space of a UTS $[x_{1}, ..., x_{T}]$ can be reconstructed by {\em time delay embedding}, which is a set of new vectors $\mat{v}_{1}$, ..., $\mat{v}_{l}$ with

RP \cite{eckmann1987recurrence} encodes a UTS into an image that captures its periodic patterns by using its reconstructed {\em phase space}. The phase space of %a UTS %$[x_{1}, ..., x_{T}]$
$\mat{x}\in\mathbb{R}^{1\times T}$ can be reconstructed by {\em time delay embedding} -- a set of new vectors $\mat{v}_{1}$, ..., $\mat{v}_{l}$ with
\begin{equation}\label{eq.de}
\small
\begin{aligned}
\mat{v}_{t}=[x_{t}, x_{t+\tau}, x_{t+2\tau}, ..., x_{t+(m-1)\tau}]\in\mathbb{R}^{m\tau},~~~1\le t \le l
\end{aligned}
\end{equation}
where $\tau$ is the time delay, $m$ is the dimension of the phase space, both %of which
are hyperparameters. Hence, $l=T-(m-1)\tau$. With vectors $\mat{v}_{1}$, ..., $\mat{v}_{l}$, an RP image %is constructed by measuring
measures their pairwise distances, results in an $l\times l$ image whose element
\begin{equation}\label{eq.rp}
\small
\begin{aligned}
\text{RP}_{i,j}=\Theta(\varepsilon - \|\mat{v}_{i} - \mat{v}_{j}\|),~~~1\le i,j\le l
\end{aligned}
\end{equation}
where $\Theta(\cdot)$ is the Heaviside step function, $\varepsilon$ is a threshold, and $\|\cdot\|$ is a norm function such as $\ell_{2}$ norm. Eq.~\eqref{eq.rp} %states RP produces a heatmap image on a binary matrix with $\text{RP}_{i,j}=1$ if $\mat{v}_{i}$ and $\mat{v}_{j}$ are sufficiently similar.
generates a binary matrix with $\text{RP}_{i,j}=1$ if $\mat{v}_{i}$ and $\mat{v}_{j}$ are sufficiently similar, producing a black-white image ({\em e.g.}, Fig. \ref{fig.tsimage}(f)).% ({\em e.g.}, a periodic pattern).

An advantage of RP is its flexibility in image size by tuning $m$ and $\tau$. Thus it has been used for time series classification %\cite{cao2021image},
\cite{silva2013time,hatami2018classification}, forecasting \cite{li2020forecasting}, anomaly detection \cite{lin2024hierarchical} and %feature-wise
explanation \cite{kim2024cafo}. Moreover, the method in \cite{hatami2018classification}, and similarly in HCR-AdaAD \cite{lin2024hierarchical}, omit the thresholding in Eq.~\eqref{eq.rp} and uses $\|\mat{v}_{i} - \mat{v}_{j}\|$ to produce continuously valued images %in a classification task
to avoid information loss.


% \vspace{0.2cm}

\subsection{Other Methods}\label{sec.othermethod}

%There are some less commonly used methods. For example, in
Additionally, %there are some peripheral methods. %In addition to GAF,
\cite{wang2015encoding} introduces Markov Transition Field (MTF) for imaging a UTS. %$\mat{x}\in\mathbb{R}^{1\times T}$. 
%MTF first assigns each $x_{t}$ to one of $Q$ quantile bins, then builds a $Q\times Q$ Markov transition matrix $\mat{M}$ {\em s.t.} $\mat{M}_{i,j}$ represents the frequency %with which
%of the case when a point $x_{t}$ in the $i$-th bin is followed by a point $x_{t'}$ in the $j$-th bin, {\em i.e.}, $t=t'+1$. Matrix $\mat{M}$ serves as the input of a heatmap image.
MTF is a matrix $\mat{M}\in\mathbb{R}^{Q\times Q}$ encoding the transition probabilities over time segments, where $Q$ is the number of segments. %Moreover,
ImagenTime \cite{naiman2024utilizing} stacks the delay embeddings $\mat{v}_{1}$, ..., $\mat{v}_{l}$ in Eq.~\eqref{eq.de} to an $l\times m\tau$ matrix for visualizing UTSs. %It also uses a variant of STFT.
% The method in \cite{homenda2024time} introduces five different 2D images by counting, rearranging, replicating the values in a UTS. 
MSCRED \cite{zhang2019deep} uses heatmaps on the $d\times d$ correlation matrices of MTSs with $d$ variates for anomaly detection. 
Furthermore, some methods use a mixture of imaging methods by stacking different transformations. \cite{wang2015imaging} stacks GASF, GADF, MTF to a 3-channel image. %Similarly,
FIRTS \cite{costa2024fusion} builds a 3-channel image by stacking GASF, MTF and RP. %the GASF, MTF, RP representations of each UTS.
%\cite{jin2023classification} combines Line Plot with Constant-Q Transform (CQT) \cite{brown1991calculation}, a method related to wavelet transform ($\S$\ref{sec.spectrogram}), to generate 2-channel images.
The mixture methods encode a UTS with multiple views and were found more robust than single-view images in these works for %time series
classification tasks.

\subsection{How to Model MTS}\label{sec.modelmts}

In the above methods, Heatmap ($\S$\ref{sec.heatmap}) can be %directly
used to visualize the %2D
variate-time matrices, $\mat{X}$, of MTSs ({\em e.g.}, Fig. \ref{fig.structure}(b)), where correlated variates %are better to
should be spatially close to each other. Line Plot ($\S$\ref{sec.lineplot}) can be used to visualize MTSs by plotting all variates in the same image \cite{wimmer2023leveraging,daswani2024plots} or combining all univariate images to compose a bigger %1-channel
image \cite {li2023time}, but these methods only work for a small number of variates. Spectrogram ($\S$\ref{sec.spectrogram}), GAF ($\S$\ref{sec.gaf}), and RP ($\S$\ref{sec.rp}) were designed specifically for UTSs. For these methods and Line Plot, which are not straightforward %for MTS transformation,
in imaging MTSs, the general approaches %to use them %for MTS
include using channel independence assumption to model each variate individually \cite{nie2023time}, %like VisionTS \cite{chen2024visionts},
or stacking the images of $d$ variates to form a $d$-channel image %as did by
\cite{naiman2024utilizing,kim2024cafo}. %\cite{prithyani2024feasibility,naiman2024utilizing,kim2024cafo}.
However, the latter does not fit some vision models pre-trained on RGB images which requires 3-channel inputs (more discussions are deferred to $\S$\ref{sec.processing}).

\vspace{0.2cm}

\noindent{\bf Remark.} As a summary, Table \ref{tab.tsimage} recaps the salient advantages and limitations of the five primary imaging methods that are introduced in this section.

% \hh{can we have a table (e.g., rows are different imaging methods and columns are a few desirable propoerties) or a short paragraph to discuss/summarize/compare the strenths and weakness of different imaging methods for ts? This might bring some structure/comprehension to this section (as opposed to, e.g., some reviewer might complain that what we do here is a laundry list)}

\section{Imaged Time Series Modeling}\label{sec.model}

With image representations, time series analysis can be readily performed with vision models. This section discusses such solutions from %traditional vision models %($\S$\ref{sec.cnns})
%to the recent large vision models %($\S$\ref{sec.lvms})
%and large multimodal models.% ($\S$\ref{sec.lmms}).
the traditional models to the SOTA models.

\begin{figure*}[!t]
\centering
\includegraphics[width=0.9\textwidth]{fig/fig_2.pdf}
% \vspace{-1em}
\caption{An illustration of different modeling strategies on imaged time series in (a)(b)(c) and task-specific heads in (d).}\label{fig.models}
\vspace{-0.2cm}
\end{figure*}

\subsection{Conventional Vision Models}\label{sec.cnns}

%Similar to
Following traditional %methods on
image classification, \cite{silva2013time} applies a K-NN classifier on the RPs of time series, \cite{cohen2020trading} applies an ensemble of fundamental classifiers such as %linear regression, SVM, Ada Boost, {\em etc.}
SVM and AdaBoost on the Line Plots %images
for time series classification. As an image encoder, %a typical encoder, %of images,
CNNs have been %extensively
widely used for learning image representations. %\cite{he2016deep}.
Different from using 1D CNNs on sequences %UTS or MTS
\cite{bai2018empirical}, %regular
2D or 3D CNNs can be applied on imaged time series as shown in Fig. \ref{fig.models}(a). %to learn time series representations by encoding their image transformations.
For example, %standard
regular CNNs have been used on Spectrograms \cite{du2020image}, tiled CNNs have been used on GAF images \cite{wang2015encoding,wang2015imaging}, dilated CNNs have been used on Heatmap images \cite{yazdanbakhsh2019multivariate}. More frequently, ResNet \cite{he2016deep}, Inception-v1 \cite{szegedy2015going}, and VGG-Net \cite{simonyan2014very} have been used on Line Plots \cite{jin2023classification,semenoglou2023image}, Heatmap images \cite{zeng2021deep}, RP images \cite{li2020forecasting,kim2024cafo}, GAF images \cite{barra2020deep,kaewrakmuk2024multi}, 
% Heatmaps \cite{zeng2021deep}, RPs \cite{li2020forecasting,kim2024cafo}, GAFs \cite{barra2020deep,kaewrakmuk2024multi},
and even a mixture of GAF, MTF and RP images \cite{costa2024fusion}. In particular, for time series generation tasks, %a diffusion model with U-Nets \cite{naiman2024utilizing} and GAN frameworks of CNNs \cite{li2022tts,karami2024timehr} have also been explored.%investigated.
GAN frameworks of CNNs \cite{li2022tts,karami2024timehr} and a diffusion model with U-Nets \cite{naiman2024utilizing} have also been explored.

Due to their small to medium sizes, these models are often trained from scratch using task-specific training data. %per task using the task's training set. %of time series images.
Meanwhile, fine-tuning {\em pre-trained vision models}  %such as those pre-trained on ImageNet, %\cite{deng2009imagenet}, 
have already been found promising in cross-modality knowledge transfer for time series anomaly detection \cite{namura2024training}, forecasting \cite{li2020forecasting} and classification \cite{jin2023classification}.

% \cite{li2020forecasting} uses ImageNet pretrained CNNs.

\subsection{Large Vision Models (LVMs)}\label{sec.lvms}

Vision Transformer (ViT) \cite{dosovitskiy2021image} has %given birth to
inspired the development of %some
modern LVMs %large vision models (LVMs)
such as %DeiT \cite{touvron2021training}, 
Swin \cite{liu2021swin}, BEiT \cite{bao2022beit}, and MAE \cite{he2022masked}. %Given an input image, ViT splits it
As Fig. \ref{fig.models}(b) shows, ViT splits an %input
image into {\em patches} of fixed size, then embeds each patch and augments it with a positional embedding. The %resulting
vectors of patches are processed by a Transformer %encoder
as if they were token embeddings. Compared to CNNs, ViTs are less data-efficient, but have higher capacity. %Consequently,
Thus, %the
{\em pre-trained} ViTs have been explored for modeling %the images of time series.
imaged time series. For example, AST \cite{gong2021ast} fine-tunes DeiT \cite{touvron2021training} on the filterbank spetrogram of audios %signals
for classification tasks and finds %using
ImageNet-pretrained DeiT is remarkably effective in knowledge transfer. The fine-tuning paradigm has also been %similarly
adopted in \cite{zeng2023pixels,li2023time} but with different pre-trained models %initializations
such as Swin by \cite{li2023time}. 
VisionTS \cite{chen2024visionts} %explains
attributes %the superiority of LVMs
LVMs' superiority over LLMs in knowledge transfer %over LLMs %as an outcome of
to the small gap between the pre-trained images and imaged time series. %the patterns learned from the large-scale pre-trained images and the patterns in the images of time series.
It %also
finds that with one-epoch fine-tuning, MAE becomes the SOTA time series forecasters on %many
some benchmark datasets.

Similar to %build
time series foundation models %\cite{das2024decoder,goswami2024moment,ansari2024chronos,shi2024time}, %such as TimesFM \cite{das2024decoder}, MOMENT \cite{goswami2024moment}, Chronos \cite{ansari2024chronos} and Time-MoE \cite{shi2024time},
such as TimesFM \cite{das2024decoder}, %and MOMENT \cite{goswami2024moment}, 
there are some initial efforts in pre-training ViT architectures with imaged time series. Following AST, SSAST \cite{gong2022ssast} introduced a %joint discriminative and generative
%masked spectrogram patch prediction self-supervised learning framework
masked spectrogram patch prediction framework for pre-training ViT on a large dataset -- AudioSet-2M. Then it becomes a backbone of some follow-up works such as AST-SED \cite{li2023ast} for sound event detection. %To be effective for UTSs,
For UTSs, ViTime \cite{yang2024vitime} generates a large set of Line Plots of synthetic UTSs for pre-training ViT, which was found superior over TimesFM in zero-shot forecasting tasks on benchmark datasets.

\subsection{Large Multimodal Models (LMMs)}\label{sec.lmms}

%As Large Multimodal Models (LMMs)
As LMMs %are getting
get growing attentions, some %of the
notable LMMs, such as LLaVA \cite{liu2023visual}, Gemini \cite{team2023gemini}, GPT-4o \cite{achiam2023gpt} and Claude-3 \cite{anthropic2024claude}, have been explored to consolidate the power of LLMs %on time series
and LVMs in time series analysis. 
Since LMMs support multimodal input via prompts, methods in this thread typically prompt LMMs with the textual and imaged representations of time series, %textual representation of time series and their %image transformations, transformed images,
%then instruct LMMs
and instructions on what tasks to perform ({\em e.g.}, Fig. \ref{fig.models}(c)).

InsightMiner \cite{zhang2023insight} is a pioneer work that uses the LLaVA architecture to generate %textual descriptions about
texts describing the trend of each input UTS. It extracts the trend of a UTS by Seasonal-Trend decomposition, encodes the Line Plot of the trend, and concatenates the embedding of the Line Plot with the embeddings of a textual instruction, which includes a sequence of numbers representing the UTS, {\em e.g.}, ``[1.1, 1.7, ..., 0.3]''. The concatenated embeddings are taken by a language model for generating trend descriptions. %It also fine-tunes a few layers with the generated texts to align LLaVA checkpoints with time series domain.
Similarly, \cite{prithyani2024feasibility} adopts the LLaVA architecture, but for MTS classification. An MTS is encoded by %a sequence of
the visual %token
embeddings of the stacked Line Plots of all variates. %meanwhile
%The method also stacks
%The time series of all variate are also stacked in a prompt % of all variates in a prompt
The matrix of the MTS is also verbalized in a prompt 
as the textual modality. %By manipulating token embeddings,
By integrating token embeddings, both %of these %works propose to
methods fine-tune some layers of the LMMs with some synthetic data.

Moreover, zero-shot and in-context learning performance of several commercial LMMs have been evaluated for audio classification \cite{dixit2024vision}, anomaly detection \cite{zhuang2024see}, and some synthetic tasks \cite{daswani2024plots}, where the image %({\em e.g.}, spectrograms, Line Plots)
and textual representations of a query %UTS or MTS
time series are integrated into a prompt. For in-context learning, these methods inject the images of a few example time series and their labels ({\em e.g.}, classes) %({\em e.g.}, classes, normal status)
into an instruction to prompt LMMs for assisting the prediction of the query time series.

\subsection{Task-Specific Heads}\label{sec.task}

%With the image embedding of a time series, the next step is to produce its prediction.
For classification tasks, most of the methods in Table \ref{tab.taxonomy} adopt a fully connected (FC) layer or multilayer perceptron (MLP) to transform an embedding into a probability distribution over all classes. For forecasting tasks, there are two approaches: (1) using a $d_{e}\times W$ MLP/FC layer to directly predict (from the $d_{e}$-dimensional embedding) the time series values in a future time window of size $W$ \cite{li2020forecasting,semenoglou2023image}; (2) predicting the pixel values that represent the future part of the time series and then recovering the time series from the predicted image \cite{yang2023your,chen2024visionts,yang2024vitime} ($\S$\ref{sec.processing} discusses the recovery methods). Imputation and generation tasks resemble forecasting %in the sense of predicting
as they also predict time series values. Thus approach (2) has been used for imputation \cite{wang2015imaging} and generation \cite{naiman2024utilizing,karami2024timehr}. %LMMs have been used for classification, text generation, and anomaly detection. For these tasks,
When using LMMs for classification, text generation, and anomaly detection, most of the methods prompt LMMs to produce the desired outputs in textual answers, circumventing task-specific heads \cite{zhang2023insight,dixit2024vision,zhuang2024see}.

%Forecasting: MLP, FC to predict numerical values using embeddings. Imputation of images (TSH). Classification: MLP, FC using embeddings.

\section{Pre-Processing and Post-Processing}\label{sec.processing}

To be successful in using vision models, some subtle design desiderata %to be considered
include {\bf time series normalization}, {\bf image alignment} and {\bf time series recovery}.

\vspace{0.2cm}

\noindent{\bf Time Series Normalization.} Vision models are usually trained on %images after Gaussian normalization (GN).
standardized images. To be aligned, the images introduced in $\S$\ref{sec.tsimage} should be normalized with a controlled mean and standard deviation, as did by \cite{gong2021ast} on spectrograms. In particular, as Heatmap is built on raw time series values, the commonly used Instance Normalization (IN) \cite{kim2022reversible} can be applied on the time series as suggested by VisionTS \cite{chen2024visionts} since IN share similar merits as Standardization. %although min-max normalization was used by \cite{karami2024timehr,zeng2021deep}.
Using Line Plot requires a proper range of y-axis. In addition to rescaling time series %by min-max or GN
\cite{zhuang2024see}, ViTST \cite{li2023time} introduced several methods to remove extreme values from the plot. GAF requires min-max normalization on its input, as it transforms time series values withtin $[0, 1]$ to polar coordinates ({\em i.e.}, arccos). In contrast, input to RP is usually normalization-free as an $\ell_{2}$ norm is involved in Eq.~\eqref{eq.rp} before thresholding.%for a comparison with a threshold.

\vspace{0.2cm}

\noindent{\bf Image Alignment.} When using pre-trained models, it is imperative to fit the image size to the input requirement of the models. This is especially true for Transformer based models as they use a fixed number of positional embeddings to encode the spacial information of image patches. For 3-channel RGB images such as Line Plot, it is straightforward to meet a pre-defined size by adjusting the resolution when producing the image. For images built upon matrices such as Heatmap, Spectrogram, GAF, RP, the number of channels and matrix size need adjustment. For the channels, one method is to duplicate a matrix to 3 channels \cite{chen2024visionts}, another way is to average the weights of the 3-channel patch embedding layer into a 1-channel layer \cite{gong2021ast}. For the image size, bilinear interpolation is a common method to resize input images \cite{chen2024visionts}. Alternatively, AST \cite{gong2021ast} %use cut and bilinear interpolation on
resizes the positional embeddings instead of the images to fit the model to a desired input size. However, the interpolation in these methods may either alter the time series or the spacial information in positional embeddings.

% single-channel (UTS), RGB channel (UTS), duplicate channels (UTS), multi-channel (MTS).

%Bilinear interpolation.

%Correlated variates are better to be spatially close to each other.

%\subsection{Pre-training}

\vspace{0.2cm}

\noindent{\bf Time Series Recovery.} As stated in $\S$\ref{sec.task}, tasks such as forecasting, imputation and generation requires predicting time series values. For models that predict pixel values of images, post-processing involves recovering time series from the predicted images. Recovery from Line Plots is tricky, it requires locating pixels that %correspond to
represent time series and mapping them back to the original values. This can be done by manipulating a grid-like Line Plot as introduced in \cite{yang2023your,yang2024vitime}, which has a recovery function. In contrast, recovery from Heatmap is straightforward as it directly stores the predicted time series values \cite{zeng2021deep,chen2024visionts}. Spectrogram is underexplored in these tasks and it remains open on how to recover time series from it. The existing work \cite{zeng2023pixels} uses Spectrogram for forecasting only with an MLP head that directly predicts time series. %predicts time series values.
GAF supports accurate recovery by an inverse mapping from polar coordinates to normalized time series \cite{wang2015imaging}. However, RP lost time series information during thresholding (Eq.~\ref{eq.rp}), thus may not fit recovery-demanded tasks without using an {\em ad-hoc} prediction head.


% Line Plot was regarded as matrices with rows and columns for mapping in \cite{sood2021visual}.


%\section{Tasks and Time Series Recovery}

%\subsection{Task-Specific Head}

% \subsection{Time Series Recovery}




The GFM-based RS effectively utilize the technological complementarity of GNN and LLM. GNNs struggle to model textual information, while the reasoning capabilities of LLMs do not support their comprehension of higher-order structural information. These two technologies complement each other's shortcomings in GFM, which emerges as a future opportunity in the field of recommendations. For example, LLMGR~\cite{guo2024integrating} injects the embeddings learned by GNN into the token embedding sequence of LLM, and adapts the GFM to the recommendation task through two-stage fine-tuning. LLMRG~\cite{wang2023enhancing} constructs inference graphs and divergence graphs based on user interaction history using LLM, which are then encoded by GNN for recommendations. DALR~\cite{peng2024denoising} aligns the embeddings encoded by GNN and those encoded by LLM in various ways, using the aligned embeddings for subsequent recommendations.

In this survey, we comprehensively investigate the relevant work of GFM-based RS, and provide a clear taxonomy based on the synergistic relationship between the graph and LLM in GFM: \textbf{Graph-augmented LLM}, \textbf{LLM-augmented graph} and \textbf{graph-LLM harmonization}.
Graph-augmented LLM methods can be viewed as utilizing the structural information of the graph to aid the knowledge obtained from LLM pre-training for recommendations. LLM-augmented graph methods, on the other hand, is led by the structural information of the graph, with the world knowledge of LLM serving as auxiliary information. Graph-LLM harmonization methods involve the equal transformation of these two types of information in the representation space. 

% As an evergreen topic in both academia and industry, numerous surveys have been conducted on recommender systems \cite{gao2023survey,wu2024survey}. The former provides a comprehensive review of graph-based recommender systems, representing traditional methodologies, while the latter offers an overview of LLM based recommender systems, representing a new paradigm. While the previous two surveys offer detailed insights into the respective technologies, they were unaware of the rapid development of GFM in the field of recommendations. Therefore, our survey offers a broader perspective for extensive research related to recommendations.

As an evergreen topic in both academia and industry, RS have been the subject of numerous surveys (e.g., \cite{gao2023survey,wu2024survey,liu2023towards,li2023survey}). \cite{gao2023survey,wu2024survey} focus on specific methodologies, such as GNN-based RS or the more recent LLM-based RS. \cite{li2023survey} concentrates on utilizing LLM to enhance graphs for tackling tasks related to graphs. However, the field is rapidly evolving with GFMs emerging as a crucial technique of the RS research. \cite{liu2023towards} systematically outlines the existing GFMs from the perspectives of pre-training and adaptation, while overlooking the recommendation which is one of the significant downstream tasks for GFM. This survey provides a timely and comprehensive overview that covers the landscape of GFM-based recommender systems.

The contributions of this survey can be summarized in the following aspects:\textbf{1)} \textit{Pioneering overview}: Our survey fills the blank in comprehensive work in the field of GFM-based RS. \textbf{2)} \textit{Clear taxonomy}: The comprehensive survey presents a well-structured taxonomy of GFM-based RS, allowing future work to be easily categorized within the corresponding branches. \textbf{3)} \textit{Promising outlook}: We present the challenges and future research directions in this field, which can serve as a valuable reference for research in this rapidly evolving area.
% This survey provides the first systematic review of graph foundation models for recommendation, offering several key contributions to the field:  

% 1. \textbf{A Novel Classification Framework}: We propose a comprehensive framework to categorize the GFM into three paradigms: Graph-augmented LLMs, LLM-augmented graphs, and LLM-graph harmonization in recommendation. This taxonomy provides a clear roadmap for understanding the field and guiding future research.

% 2. \textbf{Methodological Review}: We conduct an in-depth analysis of methodologies within each paradigm, discussing their theoretical foundations, design strategies, and real-world applications. Representative studies are examined to highlight their contributions to solving key recommendation challenges.

% 3. \textbf{Challenges and Future Directions}: Through meticulous literature synthesis, we unveil major challenges in this field, such as alignment of representations, computational efficiency, scalability, and integration complexity. Simultaneously, we spotlight prospective avenues for future research, including adaptive integration methods, cross-modal fusion, and efficient large-scale deployment strategies. Our analysis and insights aim to both address these current challenges and inspire future innovation, guiding researchers to unlock the full potential of integrating graph and LLM technologies.


% \begin{figure*}[!t]
%   \centering
%   \includegraphics[width=0.9\linewidth]{figures/Radar_Pipeline.pdf}
%   \caption{Radar Pipeline includes three modules: the hardware setup, the signal processor, and the detector.}
%   \label{fig:RSUencountered}
%   \vspace{-11px}
%   \centering
% \end{figure*}
%%%%%%%%%%%%%%%%%%%%%%%%%%%%%%%%%%%%%%%%%%%%%%%%%%%%%%%%%%%%%%%%%%%%%%%%
\section{Related Work}
\label{sec:rw}

Our work lies at the intersection of three lines of inquiry: research on technologies supporting health services (Section \ref{sec:rw:tech-services}), mental health data collection and storage (Section \ref{sec:rw:data}), and value-based mental healthcare (Section \ref{sec:rw:vbc}).

\subsection{Designing Technologies for Health Services}
\label{sec:rw:tech-services}

In this work, we studied technologies that support value-based care and the delivery of \textit{health services}, which encompass the people, organizations, and technology involved in healthcare delivery \cite{issues_working_1994, sanford_schwartz_chapter_2017}.
These people and organizations include \textit{healthcare providers}, the clinicians or hospital systems that provide treatments or preventive care (the ``services''); as well as \textit{healthcare payers}, the government agencies or private health insurance companies that pay for health services.
We review specific technologies supporting mental health services in Section \ref{sec:rw:data}.
To design technologies for health services, we need to confront more than the hardware or software capabilities of a specific technology, or the effectiveness of interventions that use technologies to improve health outcomes.
We also need to confront sociotechnical factors that affect the implementation and effectiveness of these technologies in real-world care. 
Norman and Stappers categorize sociotechnical factors that affect technology implementation as political, economic, cultural, organizational, and structural \cite{norman_designx_2015}.
Blandford states that, for health services specifically, HCI scholars should \textit{``consider stages (of identifying technical possibilities or early adopters and planning for adoption and diffusion) that are rarely discussed in HCI, but that are necessary to deliver real impact from HCI innovations in healthcare''} \cite{blandford_hci_2019}.
Thus, we were motivated to improve the design of technologies supporting health services by understanding factors that affect their implementation and adoption in care.

Recently, HCI scholars have considered adopting ideas from health services research to improve both the design and effectiveness of health technologies.
Scholars have considered how HCI research can integrate aspects of \textit{implementation science} -- the health services field examining the real-world adoption of evidence-based interventions \cite{lyon_bridging_2023}. 
Interviews with HCI and implementation science researchers uncovered that HCI tends to de-prioritize factors that influence long-term adoption of technologies in their initial design, including the financial incentives that affect adoption, and an understanding of how technologies support providers after implementation \cite{dopp_aligning_2020}.
Moreover, HCI scholars have stated that if technologies are to impact real-world care, HCI researchers should focus on how technology is consumed in care, including developing an understanding of the technical and market incentives to use new tools \cite{colusso_translational_2019}.
Inspired by this work, we considered these aspects of adoption in the initial design of technologies that support value-based mental healthcare.
Specifically, we considered how technologies can support healthcare providers -- practicing clinicians -- including how these technologies can be integrated into clinicians' workflows to support care, and the financial incentives that influence HIT adoption as a part of value-based care.

\subsection{Health Information Technologies for Collecting and Storing Mental Health Data}
\label{sec:rw:data}

HCI, health informatics, and mental health researchers have collaborated to build health information technologies (HITs) for collecting and storing mental health data.
In this work, we focus on three categories of mental health data: clinical data, active data, and passive data.
\textit{Clinical data} can be retrieved from \textit{electronic health records} (EHRs), which record information collected during clinical visits including patient demographics, diagnoses, health and family history, treatments provided, and unstructured clinical notes \cite{birkhead_uses_2015}.
\rev{That said, to protect patient privacy, not all mental health data may be contained within the EHR, and exporting EHR data for VBC may require patient consent \cite{shenoy_safeguarding_2017, leventhal_designing_2015}.}
Clinical data can also be retrieved from \textit{administrative claims databases}, which log diagnostic, treatment, and medication information used to bill healthcare payers \cite{karve_prospective_2009, davis_can_2016}.
Clinics or hospitals may also collect measures of patient satisfaction to understand patients' perceptions of their care \cite{carr-hill_measurement_1992}.

\textit{Active data} require active patient or clinician engagement to be collected, and can be collected with technologies that support digital surveys (eg, smartphones, iPads, computers, \rev{patient portals}) and pen-and-paper questionnaires.
This data include validated self-reported \textit{measures of mental health symptoms}, which quantify symptom presence and/or severity for specific mental health disorders, such as the PHQ-9 for major depressive disorder \cite{kroenke_phq-9_2001}, or the GAD-7 for generalized anxiety disorder \cite{spitzer_brief_2006}.
Active data can also include clinician-rated scales, collected during clinical interviews \cite{andersen_brief_1986}.
Outside of symptoms, self-reported and clinician-rated measures can also quantify \textit{functioning}, as mental health symptoms can impair functioning including cognition, mobility, self-care, and sociality \cite{ustun_measuring_2010}. 
Self-reported measures can also quantify how well patients and their mental health clinicians collaborate towards shared goals, complete tasks, and bond, called \textit{working alliance} \cite{hatcher_development_2006}.
The discussed scales typically quantify persistent symptoms or functional impairment.
Researchers have used everyday devices, such as smartphones, to collect more in-the-moment symptoms via questionnaires called ecological momentary assessments (EMAs) \cite{wang_crosscheck_2016, hsieh_using_2008}.
EMAs can also collect \textit{engagement data}, measuring, for example, medication adherence, or participation in behavioral interventions, such as mindfulness exercises \cite{militello_digital_2022, klasnja_how_2011}.
Active data can be stored in clinical records, like an EHR, but significant investments have not been made to build structured EHR fields for storing active data \cite{pincus_quality_2016}.

In addition to active data, sensors embedded in devices (eg, smartphones, wearables) and online platforms have created opportunities to collect \textit{passive data} -- data collected with little-to-no effort -- on behavior and physiology \cite{nghiem_understanding_2023}.
Passive data can be used to estimate signals related to functioning, including social behaviors, mobility, and sleep \cite{mohr_personal_2017, saeb_relationship_2016, saeb_scalable_2017}, and more recently, researchers have investigated if passive data can measure engagement in therapeutic exercises \cite{evans_using_2024}.
Prior work has also studied whether passive data can estimate symptom severity \cite{adler_measuring_2024, das_swain_semantic_2022, meyerhoff_evaluation_2021, currey_digital_2022}.
The use of passive data in treatment is limited: \rev{while passive data can be collected within EHRs \cite{apple_healthcare_2024, metrohealth_track_2024, pennic_novant_2015}, established clinical guidelines for passive data use in care do not exist, and use is often limited to patients who are motivated to share passive data with their healthcare provider \cite{nghiem_understanding_2023}}.

It is challenging to identify what mental health data are most relevant to HITs in certain contexts, given their variety.
Li et al. proposed a 5-stage model to work through these challenges, specifically in the context of \textit{personal informatics systems}, where users collect data for self-reflection and gaining self-knowledge.
These five stages are preparation, collection, integration, reflection, and action \cite{li_stage-based_2010}.
In this work, we study how HITs can support mental health outcomes data as a part of value-based mental healthcare, inspired by three out of these five stages, specifically \textit{preparation}, understanding what data to collect; \textit{collection}, gathering data; and \textit{action}, how data is used.
We focus on these three stages because they capture existing challenges to design HITs that support VBC, which we review in Section \ref{sec:rw:vbc}.

\subsection{Value-based Mental Healthcare}
\label{sec:rw:vbc}
The World Economic Forum defines \textit{value-based care} (VBC) as a \textit{``patient-centric way to design and manage health systems''} and \textit{``align industry stakeholders around the shared objective of improving health outcomes delivered to patients at a given cost''} \cite{world_economic_forum_value_2017}.
VBC intends to change how healthcare is paid for, away from \textit{fee-for-service} payment models -- where payers reimburse providers for the number of services they provide -- towards paying for services if they deliver ``value'' to the healthcare system \cite{brown_key_2017}.
In practice, VBC is implemented by paying providers a set rate for managing patients' health, sharing savings if specific cost or utilization targets are met, and/or by offering financial incentives for payers and providers based upon \textit{quality measures}, which quantify the ``value'' of care \cite{world_economic_forum_moment_2023, health_care_payment_learning__action_network_alternative_2017}.
These changes shift some of the financial risk of healthcare from payers to providers.
In fee-for-service models, providers continue to be paid as they provide more services.
In VBC, providers may lose money if services cost more than set rates, specific cost/utilization targets are not met, or if care quality suffers \cite{novikov_historical_2018, health_care_payment_learning__action_network_alternative_2017}.

Standardized quality measures guide payers and providers to deliver services that improve health outcomes and reduce cost.
% Quality measures can be derived from administrative claims, EHRs, and patient self-report; are validated for their reliability and validity; importance for improving quality; feasibility to collect; and are certified by country-specific organizations like the NCQA in the United States, or the National Institute for Health and Care Excellence (NICE) in the UK \cite{center_for_medicare__medicaid_services_your_2021, national_institute_for_health_and_care_excellence_nice_2019}
The Donabedian model categorizes quality measures into three areas: (1) \textit{structure} -- the material, human, and organizational resources used in care (eg, the ratio of patients to providers); (2) \textit{process} -- the services provided in care (eg, the percentage of patients receiving immunizations); and (3) \textit{outcomes} -- measuring the effectiveness of care (eg, surgical mortality rates) \cite{donabedian_quality_1988, endeshaw_healthcare_2020,agency_for_healthcare_research_and_quality_types_2015}.
% Each category of measures has strengths and weaknesses.
While structure and process measures are more actionable -- hospital systems can hire more staff, or modify care practices -- their relationship to outcomes can be ambiguous \cite{quentin_measuring_2019}. 
In contrast, outcome measures most clearly represent the goals of care, but can be biased by factors outside of providers' direct control, including co-occurring health conditions that complicate treatment success \cite{lilienfeld_why_2013, quentin_measuring_2019}.
To reduce bias, statisticians apply a \textit{risk-adjustment} to outcome measures, using regression to model expected care outcomes observed in real-world data, based upon variables known to moderate treatment effects \cite{lane-fall_outcomes_2013}.
The quality of provided health services for a specific patient can then be determined based upon whether a patient's health outcomes exceed or underperform expectations.

Mental healthcare has faced specific challenges implementing VBC.
Some of these challenges can be attributed to ambiguity on how to design health information technologies (HITs) that store outcomes data tying provided services to value \cite{world_economic_forum_value_2017}.
\textit{Preparation challenges} revolve around identifying standardized outcome metrics to store in HITs.
Current quality monitoring programs incentivize using symptom scales as standardized care outcomes \cite{morden_health_2022}.
Patients often experience a unique constellation of symptoms that cut across multiple disorders (eg, major depressive disorder and generalized anxiety disorder) \cite{boschloo_network_2015, cramer_comorbidity_2010, barkham_routine_2023}, making it difficult to identify a limited set of symptom scales to track outcomes across patients.
Given these challenges, researchers have proposed using other data types as an alternative to symptom scales within VBC \cite{hobbs_knutson_driving_2021, oslin_provider_2019}. 
For example, scholars and healthcare providers have argued that functional and engagement outcomes may be a promising alternative to symptom scales. 
Engagement is the proximal outcome of many mental health treatments, improved functioning is often more important to patients than symptom reduction, and functional outcomes measure treatment progress across patients living with different mental health symptoms or disorders \cite{stewart_can_2017, tauscher_what_2021, pincus_quality_2016}.

In terms of \textit{data collection}, it is estimated that less than 20\% of mental health clinicians practice measurement-based care (MBC) -- the process of collecting, planning, and adjusting treatment based on outcomes data -- specifically symptom scales \cite{zimmerman_why_2008, fortney_tipping_2017}, despite evidence that MBC improves outcomes \cite{barkham_routine_2023}. 
MBC is usually implemented by having patients routinely self-report symptoms during clinical encounters using validated symptom scales, like the PHQ-9 for depression, or the GAD-7 for anxiety \cite{wray_enhancing_2018}.
Mental health clinicians choose to not practice MBC for many reasons. 
Electronic health records (EHRs) often do not have standardized fields to support symptom data collection, clinicians perceive that symptom scale administration disrupts the therapeutic relationship, and clinicians are often not paid to administer symptom scales \cite{lewis_implementing_2019, desimone_impact_2023, oslin_provider_2019}.
These barriers call for work centering mental health providers in designing HITs that effectively engage providers in outcomes data collection.

\textit{Action} challenges stem from both perceptions of how outcomes data could be used in care, and challenges towards attributing accountability for care.
For example, clinicians are often not trained to use outcomes data in care, and worry that they will be held accountable and penalized if outcomes data reveal that their patients are not improving \cite{lewis_implementing_2019, desimone_impact_2023}.
There are also concerns that outcomes data could be gamed: biased reporting that artificially inflates performance metrics \cite{kilbourne_measuring_2018}.
In addition, it is difficult in mental healthcare to attribute accountability to specific actors (eg, specific providers) in care systems.
Mental healthcare is often ``siloed'' from physical healthcare, though both physical and mental health outcomes are strongly intertwined (eg, individuals living with schizophrenia suffer from chronic physical health conditions) \cite{pincus_quality_2016}.
Thus, existing value-based mental healthcare programs may hold both physical and mental health clinicians \textit{jointly accountable} by sharing cost savings across different types of providers \cite{hobbs_knutson_driving_2021}.

Taken together, this prior work demonstrates challenges designing HITs that support value-based mental healthcare.
Integral to the design of these HITs are mental health clinicians, who are asked to participate in outcomes data collection, which clinicians have found challenging, and will be held financially accountable to the outcomes data HITs store.
Given these challenges, this work centers mental health clinicians' perspectives on how to design HITs that support value-based mental healthcare.
By centering clinicians' perspectives, we looked to gain a deeper understanding of their workflows and incentives to adopt HITs, and integrate this knowledge into the design and development of HITs supporting value-based care. 
The following section details the methodology used in this study.

\begin{figure*}[!t]
  \centering
  \begin{minipage}{\linewidth}
    \centering
    \begin{subfigure}[b]{\textwidth}
      \centering
      \includegraphics[width=0.83\linewidth]{figures/our_pipeline.pdf}
      \caption{}
      \label{fig:RSUencountered1}
    \end{subfigure}
    \vspace{1em} % Space between images
    \begin{subfigure}[b]{\textwidth}
      \centering
      \includegraphics[width=0.83\linewidth]{figures/operational_scheme.pdf}
      \caption{}
      \label{fig:RSUencountered2}
      \vspace{-11px}
    \end{subfigure}
  \end{minipage}
  \caption{The pipeline for our proposed method: (a) The offline network training scheme is divided into four modules: the predefined radar signal processing and detection (point-clouds extractor) by \cite{roldan2024see}, the pre-processing of camera and radar data defined under 'Proposed Approach', and the training process which takes the 'Radar Data Pre-processing' output as input and the filtered 'Camera Image Pre-processing' output as the ground truth for training. (b) The proposed operational deployment of the trained network in (a) considering the required lower-level processes, where radar and camera work independently of each other and provide data for any required further process.}
  \label{our_pipeline}
  \vspace{-11px}
\end{figure*}
%%%%%%%%%%%%%%%%%%%%%%%%%%%%%%%%%%%%%%%%%%%%%%%%%%%%%%%%%%%%%%%%%%%%%%%%
\section{Background}
% \vspace{-11px}
\subsection{Radar Detectors}
% \vspace{-11px}
% A typical radar pipeline is composed of three modules: the hardware setup, the signal processor, and the detector (point cloud extractor). Each is explained below:

% \textbf{Hardware Setup: }The radar hardware setup features a 1D or 2D antenna array for transmitting and receiving electromagnetic signals. At its core is a transceiver, which generates high-power RF signals, transmits them, and switches to receive mode to capture and digitize reflected echoes for post-signal processing \cite{richards2010principles}.

% \textbf{Signal Processor: }The radar signal processor extracts key information—range, velocity, and angle—from digitized signals. Range is calculated from echo time delays, velocity from Doppler frequency shifts, and angle through beamforming techniques like Bartlett's method or FFT. Together, these processes convert raw signals into actionable radar data.
Radar detectors process signals to differentiate targets from noise by applying a decision threshold, comparing the signal strength to a predefined value, and outputting results as point clouds. Conventional radar detectors, like CA-CFAR and OS-CFAR, aim to maintain a consistent false alarm rate by dynamically adjusting decision thresholds based on surrounding noise and clutter. CA-CFAR works well in homogeneous environments but struggles with the heterogeneity of vehicular surroundings. OS-CFAR handles heterogeneous environments but requires precise prior knowledge of target numbers, which might be challenging to estimate. However, a recent deep learning-based radar detector by \cite{roldan2024see} trained on lidar point clouds addresses these limitations, but struggles with sparse representations of its surroundings, compared to lidar. In this work, we produce denser depth maps using these sparse representations.

%-----------------------%
\subsection{Bartlett's Algorithm for Spatial Power Spectrum Estimation}
Bartlett's algorithm, also known as periodogram averaging \cite{Bartlett1948}, is used in signal processing and time series analysis to estimate the power spectral density of a random sequence. It divides the sequence into \( M \) overlapping segments, computes their periodograms, and averages them, with the number of segments being proportional to the spectral resolution. For received signals at spatially distant receptors, like signals received at different camera physical pixels or signals received in multi-antenna wireless systems, the $M$ segments correspond to the signals received at the $M$ receptor. The time difference between segments introduces a phase shift proportional to the spatial frequency \(\omega\), referenced against the complex sinusoidal signal at the first receptor\cite{AoA}, described as:

\begin{equation}
x_1(n) = e^{-j\omega n} , n=0,1,...,N
\end{equation}
where $N$ is the number of samples in the signal. Assuming that we have $M$ segments, each segment has a time delay that is translated into a phase shift in $x_1(n)$, expressed as:
\begin{equation}
x_m(n) = x_1(n)e^{-jm\phi} =e^{-j(\omega n + m\phi)} , m=0,1,...,M-1
\end{equation}
Hence, the $M$ segments matrix $\textbf{S}\in \mathbb{C}^{N\times M}$ is found as:
\begin{equation}
    \textbf{S} = \begin{bmatrix} \textbf{x}_{1} & \textbf{x}_{2} & \cdots & \textbf{x}_{M}  \end{bmatrix}
\end{equation}
Since we are computing the spatial power spectrum, our goal is to calculate the spectrum of the signal that is described by relative phases between segments, which can be found in the covariance matrix $\textbf{C}\in \mathbb{C}^{M\times M}$ of $\textbf{S}$ as:
\begin{equation}
    \textbf{C}= \frac{1}{N}\textbf{S}^H\textbf{S}
\end{equation}
where $H$ denotes the Hermitian or complex conjugate. Therefore, the power spectral density at $\phi$, $P(\phi)$, is found as:
\begin{equation}
    P(\phi) = \textbf{a}(\phi)^H\textbf{C}\textbf{a}(\phi)
\end{equation}
where $\textbf{a}(\phi) = \begin{bmatrix} 1 & e^{-j\phi} & e^{-j2\phi} & \cdots & e^{-j(M-1)\phi}  \end{bmatrix}^T$ is our basis vector. Note that the choice of basis vector depends on the application, pattern of interest, and desired resolution \cite{priestley1981spectral}. A different basis vector is used for our proposed approach, defined in the next section.
% Therefore, the Discrete Fourier Transform (DFT) \cite{mitra2006digital} can be used to calculate the spatial power spectrum as:
% \begin{equation}
%     \textbf{P}(\omega)= \left|\sum_{k=0}^{K-1} \textbf{FCF}^H\right|
% \end{equation}
% where $\textbf{F}$ is the DFT matrix and $K$ is its number of spatial frequency bins ranging from 0 to $M$ with a step of $\frac{m}{M}$.




\begin{figure*}[t]
  \centering
  \includegraphics[width=0.92\linewidth]{figures/Data_preprocessing_results.pdf}
  \caption{Results for $M=10, 70$ and $200$, and $\Phi=\Theta=(-70,70)$ against the original RGB scene on the left.}
  \label{fig:algorithm_results}
  \vspace{-11px}
  \centering
\end{figure*}

%%%%%%%%%%%%%%%%%%%%%%%%%%%%%%%%%%%%%%%%%%%%%%%%%%%%%%%%%%%%%%%%%%%%%%%%
\section{Proposed Approach}
Unlike lidars, 4D imaging radars used in AV suffer from sparse scene representations. Our goal in this work is to bypass lidars and produce sharp 4D radar depth maps by passing the radar output to a data-driven depth map generator. Since the camera RGB images have different characteristics from radar depth maps (i.e., they come from different pixel image subspaces), we propose to compute their unified, constitutive basis vectors and transform them into their spatial spectrum representations using Bartlett’s algorithm. This bridges the gap between their original characteristics. With non-linear frequency progression basis vectors, we propose to encode the semantic segmentations of camera images and their corresponding radar depth maps to estimate their spatial spectrum. Interestingly, the spatial spectrum of both images includes frequency components proportional to $M$ caused by spectral leakage. These frequency components complete the depth for sparse point clouds, as well as introduce frequency bias that helps with fitting highly oscillatory data (the sharp camera images), as shown in Figure \ref{fig:algorithm_results} \cite{xu2024overview}. Figure \ref{fig:RSUencountered1} depicts this transformation and network training pipeline, while \ref{fig:RSUencountered2} shows the deployment pipeline for real-time operation, noting that the 4D radar and camera work independently but synchronously.

%-----------------------------------------------------------%

\subsection{Pixel Positional Encoding and Spectrum Estimation}
This encoding method aims to facilitate the transformation into the spatial spectrum of both the radar and camera images. Fast implementation of this encoding process starts with an initialization of a non-linear phase progression, complex sinusoidal basis vectors for $M$ segments for horizontal and vertical axes, $\phi$ and $\theta$. Our basis functions differ from the standard Fourier basis functions, since we require higher resolution for the output \cite{priestley1981spectral}. For that, we change the phase progression across basis functions from standard linear to non-linear, resulting in changing frequency across receptors. This changing frequency leads to higher resolution spectrum \cite{richards2010principles}. Our segments are described as:
\begin{equation}
    x(m,\phi_n)=e^{-j\pi m sin(\phi_n)} , x(m,\theta_k)=e^{-j\pi m sin(\theta_k)}
    \label{eq:horizontal_basis}
\end{equation}
% \begin{equation}
%     x(m,\theta_i)=e^{-j\pi m sin(\theta_i)}
%     \label{eq:vertical_basis}
% \end{equation}
where $m$ is the segment index, noting that $M$ is proportional to spectrum resolution, and $\phi_n$ and $\theta_k \in \Phi$ and $\Theta$ are variation angles from the set $(-90, 90)$ with lengths $N$ and $K$, respectively. The covariance matrices of every $\phi$ and $\theta$ are defined as $\textbf{C}(\Phi)$ and $\textbf{C}(\Theta) \in \mathbb{C}^{N \times N}$ and $\mathbb{C}^{K \times K}$, in which each of their rows represents the periodograms $\textbf{y}(\phi_n)$ and $\textbf{y}(\theta_k)$. The joint 2D periodogram is:
\begin{equation}
    \textbf{Y}(\phi_n, \theta_k) = \textbf{y}(\theta_n)^T\textbf{y}(\phi_k)
\end{equation}
To calculate the final 2D spatial power spectrum $\textbf{P} \in \mathbb{R}^{N \times K}$ for an input image $\textbf{I}$, we iteratively encode all pixels of $\textbf{I}$ and calculate $P(n,k)$ as:
\begin{equation}
    P(n,k)= \sum_{n=0}^{N-1} \sum_{k=0}^{K-1} \left| \textbf{Y}(\phi_n, \theta_k) \circ \textbf{I}\right|
\end{equation}
where $\circ$ denotes an element-wise multiplication. Experimental results are presented in the next section.
%----------------------------

% \begin{algorithm}[tb]
% \caption{Pixel Encoding and Spectrum Estimation}
% \label{alg:algorithm}


% \textbf{Input}: \textbf{I}, $M$, $\boldsymbol{\Phi}$, and $\boldsymbol{\Theta}$\\
% \textbf{Output}: 2D spectrum estimation for basis $x(m,\phi)$ and $x(m,\theta)$, \textbf{P}
% \begin{algorithmic}[1] %[1] enables line numbers
% \STATE Initialize $x(m,\phi)$, $x(m,\theta)$, and \textbf{P}\\
%         \STATE Compute $\textbf{C}(\boldsymbol{\Phi})$ and $\textbf{C}(\boldsymbol{\Theta})$\\
%         \FOR{each row $\textbf{y}(\theta_k)$ in $\textbf{C}(\boldsymbol{\Theta})$}
%             \FOR{each row $\textbf{y}(\phi_n)$ in $\textbf{C}(\boldsymbol{\Phi})$}
%                 \STATE $P(n,k) \leftarrow sum(|(\textbf{y}(\theta_k)^T \textbf{y}(\phi_n)| \circ \textbf{I})$
%             \ENDFOR
%         \ENDFOR

% \STATE \textbf{return} \textbf{P}

% \end{algorithmic}
% \end{algorithm}
%-----------------------------------------------------------%

\subsection{Radar Data Preprocessing}
This module conditions the DNN detector's depth map, $\textbf{I}_{radar}$, in the predecessor radar pipeline. The objective is to transform 2D depth maps into a spatial spectrum representation of its constitutional bases, $\textbf{P}_{radar}$, to satisfy the input characteristics in the following module, the 'Training Process'. This process of spatial spectrum estimation is defined as $\mathscr{F}(\textbf{I}, M)$, noting that $M$ is proportional to the output resolution.
\begin{equation}
    \textbf{P}_{radar}=\mathscr{F}(\textbf{I}_{radar}, M_{radar})
\end{equation}
% Example input and output are found in Figure 4.

%-----------------------------------------------------------%

\subsection{Camera Image Preprocessing}
In order to obtain the spatial spectrum representations for objects of interest, we transform the RGB image, $\textbf{I}_{cam}$, into its semantic segmentations, $\textbf{Seg}$, considering that this is highly dependent on the semantic segmentation accuracy and classes of the model in use. We use Deeplab v3 \cite{yurtkulu2019semantic} with ResNet101 \cite{chen2017rethinking} trained on the PASCAL Visual Object Classes (VOC) 2012 dataset \cite{Everingham10}. We then transform the semantic image into its spatial spectrum representation $\textbf{P}_{cam}$ through $\mathscr{H}(\textbf{I}, M)$, which is the process of spatial spectrum estimation of camera images.
\begin{equation}
    \textbf{P}_{cam}=\mathscr{H}(\textbf{I}_{cam}, M_{cam})=\mathscr{F}(\textbf{Seg}, M_{cam})
\end{equation}
Note that we require $M_{cam} > M_{radar}$ so that the radar spectrum images have a lower resolution, which leaves room for enhancement with deep learning models. 
% Example results are present in Figure \ref{fig:algorithm_results}.

%-----------------------------------------------------------%

\subsection{Network Training Process}
This module focuses on training a generative model that produces a denser and contour-accurate version of $\textbf{P}_{radar}$. As there are some objects captured by the semantic segmentation model that are not detectable by radar, and vice versa, element-wise multiplication produces the mutuality between both spectrum images which, thereafter, is fed into the learning as ground truth for training a ResNet, being optimized to reduce the difference through L2 loss. The process is described as:
\begin{equation}
    \textbf{P}_{radar} \circ \textbf{P}_{cam} = \text{ResNet}(\textbf{P}_{radar})
\end{equation}

\begin{figure*}[!t]
	\centering
	\begin{subfigure}{0.30\linewidth}
		\includegraphics[width=\linewidth]{figures/Correlation.pdf}
		\caption{}
		\label{fig:Correlation}
	\end{subfigure}
        \hspace{0.01\textwidth}
	\begin{subfigure}{0.30\linewidth}
		\includegraphics[width=\linewidth]{figures/Mutual_Information.pdf}
		\caption{}
		\label{fig: Mutual Information}
	\end{subfigure}
        \hspace{0.01\textwidth}
	\begin{subfigure}{0.30\linewidth}
	        \includegraphics[width=\linewidth]{figures/UCD.pdf}
	        \caption{}
	        \label{fig: UCD}
         \end{subfigure}
	\caption{(a) Correlation and (b) mutual information between several depth map pairs. 'SoTA' refers to depth map obtained by \cite{roldan2024see} while 'Encoded' refers to the spectrum of encoded images using the described bases.}
	\label{fig:subfigures}
    \vspace{-14px}
\end{figure*}

%%%%%%%%%%%%%%%%%%%%%%%%%%%%%%%%%%%%%%%%%%%%%%%%%%%%%%%%%%%%%%%%%%%%%%%%
\section{Experimental Results and Analysis}
We apply the proposed approach to the Radelft dataset's RGB images and radar depth maps of Scene 2, which includes 3400 frames. We performed two main experiments: a test of the spatial spectrum generation using the proposed approach and an enhancement of these spatial spectrum images. 
\par We evaluate the performance of the data preprocessing by computing the Pearson correlation and mutual information metrics between the generated spectrums of RGB pixel semantic segmentation and the corresponding radar depth maps for $M_{radar}=50$ and $M_{cam}=200$, noting that a higher correlation value indicates a smaller discrepancy between the images, while mutual information indicates the learning potential of one modality from the other.

The performance of the depth map generation is measured by MAE, REL, RMSE, and UCD against the lidar point clouds and depth maps. MAE evaluates the average magnitude of errors in predictions, providing a straightforward measure of accuracy. REL normalizes the error by comparing it to the mean of the actual values, offering insight into the relative performance of the predictions. RMSE emphasizes larger errors by squaring the differences before averaging, making it sensitive to outliers. UCD measures the geometric similarity between the generated depth maps and the ground truth point clouds by calculating the average distance from each predicted point to its closest corresponding point in the lidar data. As our generation is dependent on semantic segmentations, this approach eliminates the average distance inflation that is caused by detectable object discrepancies between the semantic segmentation model, and radar and lidar point clouds when using Bidirectional Chamfer Distance (BCD).
% \vspace{-3px}

%---------------------------------------------%
\subsection{Data Preprocessing}
Our data preprocessing pipeline includes an input data conditioning sub-module followed by the proposed encoding approach explained in the previous sections. We performed experiments for $M = 10, 50, 70, 200$ and $\Phi=\Theta=(-70,70)$ degrees that truncate significant spectrum leakage at higher angles. The radar input is a simple data structure transformation from projected 3D point clouds coordinates to 2D depth maps with the pixel value being inversely proportional to depth. We test our data preprocessing against Pearson correlation and mutual information. The higher value of Pearson correlation represents a stronger linear relationship between the two variables, while the higher value of mutual information indicates a greater reduction in the entropy when predicting one variable from another.

\par In Figure \ref{fig:algorithm_results}, one can observe that a greater $M$ produces higher-resolution images that preserve the contours of objects. The ripples in both horizontal and vertical axes are due to spectral leakage. Figure \ref{fig:Correlation} plots the Pearson correlation values per frame between the camera and radar modalities for $M_{radar}=50$ and $M_{cam}=200$. It also shows that the correlation is still higher when both radar depth map and camera semantic segmentation are encoded with the proposed approach. It also shows that the correlation is significantly higher for single modality encoding. Figure \ref{fig: Mutual Information} shows that mutual information is significantly improved when we encode both modalities. It also shows that mutual information is still significantly improved when we encode only a single modality.

\par Tables \ref{tab:Correlation} and \ref{tab:Mutual Information} show the averages (mean values) of the plots in Figures \ref{fig:Correlation} and \ref{fig: Mutual Information}. The results show that there are improvements by factors of 3.88 and 76.69 in Pearson correlation and mutual information, respectively.
\vspace{-3px}

% \begin{table}[h]
% \renewcommand{\arraystretch}{1.2} % Adjust the row height (1.5x the default height)
% \parbox{.45\linewidth}{
% \centering
% \begin{tabular}{|c||c c|}
% \hline
%     & \textbf{I}_{cam} & \textbf{P}_{cam} \\ \hline\hline
% \textbf{I}_{radar} & 0.1646 & 0.5503 \\
% \textbf{P}_{radar} & 0.0809 & \textbf{0.6396} \\ \hline
% \end{tabular}
% \caption{Average Pearson Correlation for different pairs.}
% \label{tab:Correlation}
% }
% \hfill
% \parbox{.45\linewidth}{
% \centering
% \begin{tabular}{|c||c c|}
% \hline
%     & \textbf{I}_{cam} & \textbf{P}_{cam} \\ \hline\hline
% \textbf{I}_{radar} & 0.0359 & 0.6117 \\ 
% \textbf{P}_{radar} & 0.4863 & \textbf{2.7533} \\ \hline
% \end{tabular}
% \caption{Average Mutual Information for different pairs.}
% \label{tab:Mutual Information}
% }
% \vspace{-12px}
% \end{table}

% \begin{table}[h]
% \renewcommand{\arraystretch}{1.2} % Adjust the row height
% \parbox{.45\linewidth}{
% \centering
% \begin{tabular}{|c||c c|}
% \hline
%     & \textbf{I}_{cam} & \textbf{P}_{cam} \\ \hline\hline
% \textbf{I}_{radar} & 0.1646 & 0.5503 \\
% \textbf{P}_{radar} & 0.0809 & \textbf{0.6396} \\ \hline
% \end{tabular}
% \caption{Average Pearson Correlation for different pairs.}
% \label{tab:Correlation}
% }
% \hfill
% \parbox{.45\linewidth}{
% \centering
% \begin{tabular}{|c||c c|}
% \hline
%     & \textbf{I}_{cam} & \textbf{P}_{cam} \\ \hline\hline
% \textbf{I}_{radar} & 0.0359 & 0.6117 \\ 
% \textbf{P}_{radar} & 0.4863 & \textbf{2.7533} \\ \hline
% \end{tabular}
% \caption{Average Mutual Information for different pairs.}
% \label{tab:Mutual_Information}
% }
% \vspace{-10pt} % Slightly reduce spacing without excessive compression
% \end{table}

\begin{table}[h]
\renewcommand{\arraystretch}{1.2} % Adjust the row height
\parbox{.45\linewidth}{
\centering
\begin{tabular}{|c||c c|}
\hline
    & $\mathbf{I}_{\text{cam}}$ & $\mathbf{P}_{\text{cam}}$ \\ \hline\hline
$\mathbf{I}_{\text{radar}}$ & 0.1646 & 0.5503 \\
$\mathbf{P}_{\text{radar}}$ & 0.0809 & \textbf{0.6396} \\ \hline
\end{tabular}
\caption{Average Pearson Correlation for different pairs.}
\label{tab:Correlation}
}
\hfill
\parbox{.45\linewidth}{
\centering
\begin{tabular}{|c||c c|}
\hline
    & $\mathbf{I}_{\text{cam}}$ & $\mathbf{P}_{\text{cam}}$ \\ \hline\hline
$\mathbf{I}_{\text{radar}}$ & 0.0359 & 0.6117 \\ 
$\mathbf{P}_{\text{radar}}$ & 0.4863 & \textbf{2.7533} \\ \hline
\end{tabular}
\caption{Average Mutual Information for different pairs.}
\label{tab:Mutual_Information}
}
\vspace{-10pt} % Slightly reduce spacing without excessive compression
\end{table}



%---------------------------------------------%
\begin{figure*}[t!]
  \centering
  \includegraphics[width=1\linewidth]{figures/Results.pdf}
  \caption{Results from the training module for four example frames. From left to right for each example frame: Scene in RGB, Camera spatial spectrum ($\textbf{P}_{cam}$), original radar depth map, output from trained ResNet101. In each of the 4 frames, observe that our approach leads to sharper depth maps.}
  \vspace{-11px}
  \label{fig:training_results}
  \centering
  \vspace{-6px}
\end{figure*}

\subsection{Depth Map Generation}
% \vspace{-5px}
The depth map generation is achieved with the ResNet101 network trained as described in the previous section with $M_{cam}=200$, $M_{radar}=20$ and $\Phi=\Theta=(-70,70)$. The ResNet101 is trained for 10,000 epochs while the input data are compressed with the natural logarithm to make different features comparable in terms of scale. Qualitatively, from the results in Figure \ref{fig:training_results}, one can observe that the intensity locations in the radar depth maps are comparable to locations in camera spectrum images. Also, the output depth maps from the ResNet101 show object contours clearly compared to the original radar depth map. However, the magnitude of the ripples is significantly higher due to the logarithmic feature compression used at the input, which can be rescaled by exponentiation.
\par Quantitatively, we measure the performance of the output with MAE, REL, RMSE, and UCD. Specifically for UCD, we transform the spectrum images into 3D point clouds (in which the depth is inversely proportional to the pixel value) and apply the calculations. Table \ref{tab:UCD} presents the results for different methods including the state-of-the-art (SOTA) DNN detector \cite{roldan2024see} and the OS-CFAR. Figure \ref{fig: UCD} shows the per frame UCD for our method and SOTA. We observe that our approach improves MAE, REL, UCD by 21.13\%, 7.9\% and 27.95\%, respectively. However, the performance degrades by 12\% in terms of RMSE. We believe that this is due to the fact that, compared to other approaches, our generated depth maps are dense, inflating the RMSE result as there are gaps in the corresponding lidar depth maps.

\renewcommand{\arraystretch}{1.4}
\begin{table}[h]
    \centering
    \begin{tabular}{|c|c|c|c|c|}
        \hline
        \textbf{Method} & \textbf{MAE} & \textbf{REL} & \textbf{RMSE} & \textbf{UCD ($m$)}\\
        \hline\hline
        Proposed Approach  & \textbf{0.026} & \textbf{0.025} & 0.111 & \textbf{3.48} \\
        SOTA DNN         & 0.033 & 0.027 & \textbf{0.099} & 4.83 \\
        OS-CFAR          & 0.029 & 0.152 & 0.258 & 18.02 \\
        % Peak Detector   &  -- \\
        \hline\hline
        \textbf{Improvement (\%)} & 21.13 & 7.9 & -12 & 27.95\\
        \hline
    \end{tabular}
    \caption{MAE, REL, RMSE, and UCD for different methods}
    \label{tab:UCD}
    \vspace{-5px}
\end{table}







%%%%%%%%%%%%%%%%%%%%%%%%%%%%%%%%%%%%%%%%%%%%%%%%%%%%%%%%%%%%%%%%%%%%%%%%


\section{Discussion}\label{sec:discussion}



\subsection{From Interactive Prompting to Interactive Multi-modal Prompting}
The rapid advancements of large pre-trained generative models including large language models and text-to-image generation models, have inspired many HCI researchers to develop interactive tools to support users in crafting appropriate prompts.
% Studies on this topic in last two years' HCI conferences are predominantly focused on helping users refine single-modality textual prompts.
Many previous studies are focused on helping users refine single-modality textual prompts.
However, for many real-world applications concerning data beyond text modality, such as multi-modal AI and embodied intelligence, information from other modalities is essential in constructing sophisticated multi-modal prompts that fully convey users' instruction.
This demand inspires some researchers to develop multimodal prompting interactions to facilitate generation tasks ranging from visual modality image generation~\cite{wang2024promptcharm, promptpaint} to textual modality story generation~\cite{chung2022tale}.
% Some previous studies contributed relevant findings on this topic. 
Specifically, for the image generation task, recent studies have contributed some relevant findings on multi-modal prompting.
For example, PromptCharm~\cite{wang2024promptcharm} discovers the importance of multimodal feedback in refining initial text-based prompting in diffusion models.
However, the multi-modal interactions in PromptCharm are mainly focused on the feedback empowered the inpainting function, instead of supporting initial multimodal sketch-prompt control. 

\begin{figure*}[t]
    \centering
    \includegraphics[width=0.9\textwidth]{src/img/novice_expert.pdf}
    \vspace{-2mm}
    \caption{The comparison between novice and expert participants in painting reveals that experts produce more accurate and fine-grained sketches, resulting in closer alignment with reference images in close-ended tasks. Conversely, in open-ended tasks, expert fine-grained strokes fail to generate precise results due to \tool's lack of control at the thin stroke level.}
    \Description{The comparison between novice and expert participants in painting reveals that experts produce more accurate and fine-grained sketches, resulting in closer alignment with reference images in close-ended tasks. Novice users create rougher sketches with less accuracy in shape. Conversely, in open-ended tasks, expert fine-grained strokes fail to generate precise results due to \tool's lack of control at the thin stroke level, while novice users' broader strokes yield results more aligned with their sketches.}
    \label{fig:novice_expert}
    % \vspace{-3mm}
\end{figure*}


% In particular, in the initial control input, users are unable to explicitly specify multi-modal generation intents.
In another example, PromptPaint~\cite{promptpaint} stresses the importance of paint-medium-like interactions and introduces Prompt stencil functions that allow users to perform fine-grained controls with localized image generation. 
However, insufficient spatial control (\eg, PromptPaint only allows for single-object prompt stencil at a time) and unstable models can still leave some users feeling the uncertainty of AI and a varying degree of ownership of the generated artwork~\cite{promptpaint}.
% As a result, the gap between intuitive multi-modal or paint-medium-like control and the current prompting interface still exists, which requires further research on multi-modal prompting interactions.
From this perspective, our work seeks to further enhance multi-object spatial-semantic prompting control by users' natural sketching.
However, there are still some challenges to be resolved, such as consistent multi-object generation in multiple rounds to increase stability and improved understanding of user sketches.   


% \new{
% From this perspective, our work is a step forward in this direction by allowing multi-object spatial-semantic prompting control by users' natural sketching, which considers the interplay between multiple sketch regions.
% % To further advance the multi-modal prompting experience, there are some aspects we identify to be important.
% % One of the important aspects is enhancing the consistency and stability of multiple rounds of generation to reduce the uncertainty and loss of control on users' part.
% % For this purpose, we need to develop techniques to incorporate consistent generation~\cite{tewel2024training} into multi-modal prompting framework.}
% % Another important aspect is improving generative models' understanding of the implicit user intents \new{implied by the paint-medium-like or sketch-based input (\eg, sketch of two people with their hands slightly overlapping indicates holding hand without needing explicit prompt).
% % This can facilitate more natural control and alleviate users' effort in tuning the textual prompt.
% % In addition, it can increase users' sense of ownership as the generated results can be more aligned with their sketching intents.
% }
% For example, when users draw sketches of two people with their hands slightly overlapping, current region-based models cannot automatically infer users' implicit intention that the two people are holding hands.
% Instead, they still require users to explicitly specify in the prompt such relationship.
% \tool addresses this through sketch-aware prompt recommendation to fill in the necessary semantic information, alleviating users' workload.
% However, some users want the generative AI in the future to be able to directly infer this natural implicit intentions from the sketches without additional prompting since prompt recommendation can still be unstable sometimes.


% \new{
% Besides visual generation, 
% }
% For example, one of the important aspect is referring~\cite{he2024multi}, linking specific text semantics with specific spatial object, which is partly what we do in our sketch-aware prompt recommendation.
% Analogously, in natural communication between humans, text or audio alone often cannot suffice in expressing the speakers' intentions, and speakers often need to refer to an existing spatial object or draw out an illustration of her ideas for better explanation.
% Philosophically, we HCI researchers are mostly concerned about the human-end experience in human-AI communications.
% However, studies on prompting is unique in that we should not just care about the human-end interaction, but also make sure that AI can really get what the human means and produce intention-aligned output.
% Such consideration can drastically impact the design of prompting interactions in human-AI collaboration applications.
% On this note, although studies on multi-modal interactions is a well-established topic in HCI community, it remains a challenging problem what kind of multi-modal information is really effective in helping humans convey their ideas to current and next generation large AI models.




\subsection{Novice Performance vs. Expert Performance}\label{sec:nVe}
In this section we discuss the performance difference between novice and expert regarding experience in painting and prompting.
First, regarding painting skills, some participants with experience (4/12) preferred to draw accurate and fine-grained shapes at the beginning. 
All novice users (5/12) draw rough and less accurate shapes, while some participants with basic painting skills (3/12) also favored sketching rough areas of objects, as exemplified in Figure~\ref{fig:novice_expert}.
The experienced participants using fine-grained strokes (4/12, none of whom were experienced in prompting) achieved higher IoU scores (0.557) in the close-ended task (0.535) when using \tool. 
This is because their sketches were closer in shape and location to the reference, making the single object decomposition result more accurate.
Also, experienced participants are better at arranging spatial location and size of objects than novice participants.
However, some experienced participants (3/12) have mentioned that the fine-grained stroke sometimes makes them frustrated.
As P1's comment for his result in open-ended task: "\emph{It seems it cannot understand thin strokes; even if the shape is accurate, it can only generate content roughly around the area, especially when there is overlapping.}" 
This suggests that while \tool\ provides rough control to produce reasonably fine results from less accurate sketches for novice users, it may disappoint experienced users seeking more precise control through finer strokes. 
As shown in the last column in Figure~\ref{fig:novice_expert}, the dragon hovering in the sky was wrongly turned into a standing large dragon by \tool.

Second, regarding prompting skills, 3 out of 12 participants had one or more years of experience in T2I prompting. These participants used more modifiers than others during both T2I and R2I tasks.
Their performance in the T2I (0.335) and R2I (0.469) tasks showed higher scores than the average T2I (0.314) and R2I (0.418), but there was no performance improvement with \tool\ between their results (0.508) and the overall average score (0.528). 
This indicates that \tool\ can assist novice users in prompting, enabling them to produce satisfactory images similar to those created by users with prompting expertise.



\subsection{Applicability of \tool}
The feedback from user study highlighted several potential applications for our system. 
Three participants (P2, P6, P8) mentioned its possible use in commercial advertising design, emphasizing the importance of controllability for such work. 
They noted that the system's flexibility allows designers to quickly experiment with different settings.
Some participants (N = 3) also mentioned its potential for digital asset creation, particularly for game asset design. 
P7, a game mod developer, found the system highly useful for mod development. 
He explained: "\emph{Mods often require a series of images with a consistent theme and specific spatial requirements. 
For example, in a sacrifice scene, how the objects are arranged is closely tied to the mod's background. It would be difficult for a developer without professional skills, but with this system, it is possible to quickly construct such images}."
A few participants expressed similar thoughts regarding its use in scene construction, such as in film production. 
An interesting suggestion came from participant P4, who proposed its application in crime scene description. 
She pointed out that witnesses are often not skilled artists, and typically describe crime scenes verbally while someone else illustrates their account. 
With this system, witnesses could more easily express what they saw themselves, potentially producing depictions closer to the real events. "\emph{Details like object locations and distances from buildings can be easily conveyed using the system}," she added.

% \subsection{Model Understanding of Users' Implicit Intents}
% In region-sketch-based control of generative models, a significant gap between interaction design and actual implementation is the model's failure in understanding users' naturally expressed intentions.
% For example, when users draw sketches of two people with their hands slightly overlapping, current region-based models cannot automatically infer users' implicit intention that the two people are holding hands.
% Instead, they still require users to explicitly specify in the prompt such relationship.
% \tool addresses this through sketch-aware prompt recommendation to fill in the necessary semantic information, alleviating users' workload.
% However, some users want the generative AI in the future to be able to directly infer this natural implicit intentions from the sketches without additional prompting since prompt recommendation can still be unstable sometimes.
% This problem reflects a more general dilemma, which ubiquitously exists in all forms of conditioned control for generative models such as canny or scribble control.
% This is because all the control models are trained on pairs of explicit control signal and target image, which is lacking further interpretation or customization of the user intentions behind the seemingly straightforward input.
% For another example, the generative models cannot understand what abstraction level the user has in mind for her personal scribbles.
% Such problems leave more challenges to be addressed by future human-AI co-creation research.
% One possible direction is fine-tuning the conditioned models on individual user's conditioned control data to provide more customized interpretation. 

% \subsection{Balance between recommendation and autonomy}
% AIGC tools are a typical example of 
\subsection{Progressive Sketching}
Currently \tool is mainly aimed at novice users who are only capable of creating very rough sketches by themselves.
However, more accomplished painters or even professional artists typically have a coarse-to-fine creative process. 
Such a process is most evident in painting styles like traditional oil painting or digital impasto painting, where artists first quickly lay down large color patches to outline the most primitive proportion and structure of visual elements.
After that, the artists will progressively add layers of finer color strokes to the canvas to gradually refine the painting to an exquisite piece of artwork.
One participant in our user study (P1) , as a professional painter, has mentioned a similar point "\emph{
I think it is useful for laying out the big picture, give some inspirations for the initial drawing stage}."
Therefore, rough sketch also plays a part in the professional artists' creation process, yet it is more challenging to integrate AI into this more complex coarse-to-fine procedure.
Particularly, artists would like to preserve some of their finer strokes in later progression, not just the shape of the initial sketch.
In addition, instead of requiring the tool to generate a finished piece of artwork, some artists may prefer a model that can generate another more accurate sketch based on the initial one, and leave the final coloring and refining to the artists themselves.
To accommodate these diverse progressive sketching requirements, a more advanced sketch-based AI-assisted creation tool should be developed that can seamlessly enable artist intervention at any stage of the sketch and maximally preserve their creative intents to the finest level. 

\subsection{Ethical Issues}
Intellectual property and unethical misuse are two potential ethical concerns of AI-assisted creative tools, particularly those targeting novice users.
In terms of intellectual property, \tool hands over to novice users more control, giving them a higher sense of ownership of the creation.
However, the question still remains: how much contribution from the user's part constitutes full authorship of the artwork?
As \tool still relies on backbone generative models which may be trained on uncopyrighted data largely responsible for turning the sketch into finished artwork, we should design some mechanisms to circumvent this risk.
For example, we can allow artists to upload backbone models trained on their own artworks to integrate with our sketch control.
Regarding unethical misuse, \tool makes fine-grained spatial control more accessible to novice users, who may maliciously generate inappropriate content such as more realistic deepfake with specific postures they want or other explicit content.
To address this issue, we plan to incorporate a more sophisticated filtering mechanism that can detect and screen unethical content with more complex spatial-semantic conditions. 
% In the future, we plan to enable artists to upload their own style model

% \subsection{From interactive prompting to interactive spatial prompting}


\subsection{Limitations and Future work}

    \textbf{User Study Design}. Our open-ended task assesses the usability of \tool's system features in general use cases. To further examine aspects such as creativity and controllability across different methods, the open-ended task could be improved by incorporating baselines to provide more insightful comparative analysis. 
    Besides, in close-ended tasks, while the fixing order of tool usage prevents prior knowledge leakage, it might introduce learning effects. In our study, we include practice sessions for the three systems before the formal task to mitigate these effects. In the future, utilizing parallel tests (\textit{e.g.} different content with the same difficulty) or adding a control group could further reduce the learning effects.

    \textbf{Failure Cases}. There are certain failure cases with \tool that can limit its usability. 
    Firstly, when there are three or more objects with similar semantics, objects may still be missing despite prompt recommendations. 
    Secondly, if an object's stroke is thin, \tool may incorrectly interpret it as a full area, as demonstrated in the expert results of the open-ended task in Figure~\ref{fig:novice_expert}. 
    Finally, sometimes inclusion relationships (\textit{e.g.} inside) between objects cannot be generated correctly, partially due to biases in the base model that lack training samples with such relationship. 

    \textbf{More support for single object adjustment}.
    Participants (N=4) suggested that additional control features should be introduced, beyond just adjusting size and location. They noted that when objects overlap, they cannot freely control which object appears on top or which should be covered, and overlapping areas are currently not allowed.
    They proposed adding features such as layer control and depth control within the single-object mask manipulation. Currently, the system assigns layers based on color order, but future versions should allow users to adjust the layer of each object freely, while considering weighted prompts for overlapping areas.

    \textbf{More customized generation ability}.
    Our current system is built around a single model $ColorfulXL-Lightning$, which limits its ability to fully support the diverse creative needs of users. Feedback from participants has indicated a strong desire for more flexibility in style and personalization, such as integrating fine-tuned models that cater to specific artistic styles or individual preferences. 
    This limitation restricts the ability to adapt to varied creative intents across different users and contexts.
    In future iterations, we plan to address this by embedding a model selection feature, allowing users to choose from a variety of pre-trained or custom fine-tuned models that better align with their stylistic preferences. 
    
    \textbf{Integrate other model functions}.
    Our current system is compatible with many existing tools, such as Promptist~\cite{hao2024optimizing} and Magic Prompt, allowing users to iteratively generate prompts for single objects. However, the integration of these functions is somewhat limited in scope, and users may benefit from a broader range of interactive options, especially for more complex generation tasks. Additionally, for multimodal large models, users can currently explore using affordable or open-source models like Qwen2-VL~\cite{qwen} and InternVL2-Llama3~\cite{llama}, which have demonstrated solid inference performance in our tests. While GPT-4o remains a leading choice, alternative models also offer competitive results.
    Moving forward, we aim to integrate more multimodal large models into the system, giving users the flexibility to choose the models that best fit their needs. 
    


\section{Conclusion}\label{sec:conclusion}
In this paper, we present \tool, an interactive system designed to help novice users create high-quality, fine-grained images that align with their intentions based on rough sketches. 
The system first refines the user's initial prompt into a complete and coherent one that matches the rough sketch, ensuring the generated results are both stable, coherent and high quality.
To further support users in achieving fine-grained alignment between the generated image and their creative intent without requiring professional skills, we introduce a decompose-and-recompose strategy. 
This allows users to select desired, refined object shapes for individual decomposed objects and then recombine them, providing flexible mask manipulation for precise spatial control.
The framework operates through a coarse-to-fine process, enabling iterative and fine-grained control that is not possible with traditional end-to-end generation methods. 
Our user study demonstrates that \tool offers novice users enhanced flexibility in control and fine-grained alignment between their intentions and the generated images.


\bibliographystyle{IEEEtran}
\bibliography{bibliography}

\end{document}
