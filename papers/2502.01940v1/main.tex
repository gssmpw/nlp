\documentclass[conference]{IEEEtran}
\IEEEoverridecommandlockouts
% The preceding line is only needed to identify funding in the first footnote. If that is unneeded, please comment it out.
%Template version as of 6/27/2024

\usepackage{cite}
\usepackage{amsmath,amssymb,amsfonts}
\usepackage{algorithmic}
\usepackage{graphicx}
\usepackage{textcomp}
\usepackage{xcolor}
\usepackage{graphicx}
\usepackage{subcaption}
\usepackage[font=small]{caption}
\usepackage{algorithm}
\usepackage{algorithmic}
\usepackage{subcaption}
\usepackage{amsmath,algorithm}
\usepackage{mathrsfs}
\usepackage{subcaption}  % For subfigures
\usepackage{graphicx}    % For including graphics
\usepackage{xcolor}
\usepackage{booktabs} % For thicker lines
\usepackage{tabu}
\usepackage{tabularray}
% \usepackage{slashbox}






\def\BibTeX{{\rm B\kern-.05em{\sc i\kern-.025em b}\kern-.08em
    T\kern-.1667em\lower.7ex\hbox{E}\kern-.125emX}}
\begin{document}

\title{Toward a Low-Cost Perception System in Autonomous Vehicles: A Spectrum Learning Approach\\
}

\author{
    %Authors
    % All authors must be in the same font size and format.
    Mohammed Alsakabi\textsuperscript{\rm 1},
    Aidan Erickson\textsuperscript{\rm 1},
    John M. Dolan\textsuperscript{\rm 2},
    Ozan K. Tonguz\textsuperscript{\rm 1} \\
    
    \textsuperscript{\rm 1}Department of Electrical and Computer Engineering, College of Engineering \\
    \textsuperscript{\rm 2}The Robotics Institute, School of Computer Science \\
    Carnegie Mellon University, Pittsburgh, PA, United States\\
    \{malsakabi, aerickson\}@cmu.edu , \{jdolan, tonguz\}@andrew.cmu.edu
}

\maketitle
\begin{abstract}

We introduce \ours, a novel framework for scene-level appearance transfer from a single style image to a real-world scene represented by multiple views. The method combines explicit semantic correspondences with multi-view consistency to achieve precise and coherent stylization.
Unlike conventional stylization methods that apply a reference style globally, \ours uses open-vocabulary segmentation to establish dense, instance-level correspondences between the style and real-world images. This ensures that each object is stylized with semantically matched textures.
\ours first transfers the style to a single view using a training-free semantic-attention mechanism in a diffusion model.
It then lifts the stylization to additional views via a learned warp-and-refine network guided by monocular depth and pixel-wise correspondences.
Experiments show that \ours consistently outperforms prior methods in structure preservation, perceptual style similarity, and multi-view coherence.
User studies further validate its ability to produce photo-realistic, semantically faithful results.
Our code, pretrained models, and dataset will be publicly released, to support new applications in interior design, virtual staging, and 3D-consistent stylization.

\end{abstract}

%%%%%%%%%%%%%%%%%%%%%%%%%%%%%%%%%%%%%%%%%%%%%%%%%%%%%%%%%%%%%%%%%%%%%%%%
\section{Introduction}

\par Autonomous Vehicles (AV) employ various sensors for comprehensive navigation and environmental perception, each contributing distinct advantages and limitations. RGB cameras are attractive due to their affordability and high accuracy in optimal lighting conditions but they struggle in low-visibility scenarios like darkness, heavy rain, or fog, which impede obstacle detection. Lidars offer precise 3D mapping and depth measurements, yet they consume high power, are susceptible to adverse weather conditions, and sometimes require mechanical parts for rotational scanning, raising the overall manufacturing and operational costs. Radars, however, stand out for their robustness in adverse weather, cost-effectiveness, low power consumption, and can offer a theoretical angular resolution comparable to lidars. 

\begin{figure}[t!]
  \centering
  \includegraphics[width=\linewidth]{figures/pixel_image_space1.pdf}
  \caption{Conceptual figure depicting the proposed spectrum-based transformation. The 'RGB' and 'Depth Maps' subspaces represent natural images captured by conventional cameras and the windshield view depth maps that lidars and radars produce, respectively. The 'Spatial Spectrum' subspace includes the special frequency 2D spectrum of bases that constitute images in the other subspaces.}
  \label{fig: image space}
  \vspace{-22px} % Adjust this value to reduce the space
\end{figure}

\par Automotive radars measure range, azimuth, and velocity, with 4D radars adding elevation to their traditional measurements. Unlike 3D radars, 4D radars are capable of estimating object heights without using speculative models. For such radars, extracting precise angular information involves a two-stage process: spatial spectrum estimations followed by Constant False Alarm Rate (CFAR) detectors like cell-averaging CFAR (CA-CFAR) and order-statistic CFAR (OS-CFAR) \cite{richards2010principles}. However, these conventional methods struggle in complex vehicular settings, producing sparse point clouds that limit accurate environmental representation \cite{khan2022comprehensive}. To address these limitations, data-driven approaches using deep neural networks (DNNs) have been reported \cite{brodeski2019deep, cheng2022novel, roldan2024see}. For instance, \cite{roldan2024see} employs a ResNet18 network \cite{chen2017rethinking} trained on dense lidar point clouds, generating denser radar point clouds that more accurately represent object shapes and sizes.

\par In this paper, we introduce a data-driven approach for generating radar depth maps by integrating radar point clouds with camera images. Leveraging the similar field-of-view (FoV) between radar and camera images, employing a non-linear frequency pixel positional encoding algorithm and Bartlett's spatial spectrum estimation \cite{Bartlett1948} transforms radar depth maps and camera RGB images into a shared spatial spectrum subspace, as shown in Figure \ref{fig: image space}. This transformation can resolve the differences between the 4D radar image and camera images, thus enabling spectrum-based learning. The method enables the use of high-resolution cameras to effectively train radar depth map generators. After this off-line training, the 4D radar model can operate independently of the camera, generating sharper and denser depth maps that are critical for perception, tracking, and rendering in AVs. Our contributions can be summarized as:

\begin{itemize}
\item We propose a pixel positional encoding algorithm that helps resolve the differences between a 4D radar image and RGB camera image, thus enabling spectrum-based learning for 4D radar images.

% \item We introduce a holistic framework for producing depth map generative models that are based on the newly developed DNN detector in \cite{roldan2024see} as a sub-module. Although the DNN detector is trained on lidar data, our framework does not require further training on lidar data and camera images for operation.

\item We present experimental results for high-resolution spectrum estimations and depth map generations. Our results show that our approach is capable of producing qualitatively sharp depth maps and significantly outperforms the state-of-the-art (SOTA), resulting in a reduction of 21.13\%, 7.9\% and 27.95\% in Mean Absolute Error (MAE), Relative Absolute Error (REL), and Unidirectional Chamfer Distance (UCD), respectively, which is quite significant.\cite{zhang2021unsupervised}. We also show that the estimated spectrum of camera and radar images results in an increase of the Pearson correlation and mutual information by a factor of 3.88 and 76.69, respectively.

\end{itemize}

% \begin{figure*}[!t]
%   \centering
%   \includegraphics[width=0.9\linewidth]{figures/Radar_Pipeline.pdf}
%   \caption{Radar Pipeline includes three modules: the hardware setup, the signal processor, and the detector.}
%   \label{fig:RSUencountered}
%   \vspace{-11px}
%   \centering
% \end{figure*}
%%%%%%%%%%%%%%%%%%%%%%%%%%%%%%%%%%%%%%%%%%%%%%%%%%%%%%%%%%%%%%%%%%%%%%%%
\section{Related Work}
\subsection{Deep Learning based Weather Forecasting}
\textbf{Global Weather Forecasting.} Global weather forecasting has seen significant progress with deep learning models. FourCastNet, based on Fourier neural operators, provides global forecasts comparable to traditional numerical methods like IFS, but at much higher speeds~\cite{pathak2022fourcastnet}. Pangu, utilizing the Swin Transformer, exceeds NWP methods, incorporating earth-specific location embeddings for better performance~\cite{bi2023accurate}. The Spherical Fourier Neural Operator (SFNO) extends Fourier methods using spherical harmonics, offering more stable long-term predictions~\cite{bonev2023spherical}. FuXi focuses on long-term forecasting, achieving a 15-day forecasts comparable to ECMWF~\cite{chen2023fuxi}. GraphCast leverages message-passing networks to improve efficiency and forecasting accuracy~\cite{lam2023learning}, and GenCast builds on this to enhance ensemble forecasting~\cite{price2023gencast}. Further, diffusion models like those in~\cite{li2024generative} generate probabilistic ensembles by sampling, while NeuralGCM~\cite{kochkov2024neural} focuses on atmospheric circulation with a dynamic core, offering climate simulation capabilities but at higher training and inference costs. 

\textbf{Regional Weather Forecasting.} The goal of regional weather forecasting is to enhance local prediction accuracy with high-resolution models. CorrDiff~\cite{mardani2023generative} combines U-Net and diffusion models to improve local forecasts. MetaWeather~\cite{kim2024metaweather} adapts global forecasts to regional contexts using meta-learning. GNNs are also widely applied in regional forecasting, with Graphcast~\cite{lam2023learning} enhancing accuracy by modeling complex spatial dependencies. MetNet-3~\cite{espeholt2022deep} offers high-accuracy forecasts for weather variables, such as precipitation, temperature, and wind speed, at 2-minute intervals and 1–4 km resolution, outperforming traditional models like HRRR. NowcastNet~\cite{zhang2023skilful} and DGMR~\cite{ravuri2021skilful} excel in short-term extreme precipitation forecasts using deep generative models and radar data. In spatiotemporal prediction, NMO~\cite{wu2024neural} models the evolution of physical dynamics, providing new insights for local weather forecasting. Similarly, SimVP~\cite{gao2022simvp} and PastNet~\cite{wu2024pastnet} achieve good results in forecasting local precipitation evolution using spatiotemporal convolution methods.
    
% Despite these advances, none of these methods effectively address the challenge of balancing global and regional high-resolution forecasts or capturing the fine-grained, dynamic interactions important for extreme event prediction.
    
\subsection{Numerical analysis methods}
Multigrid methods~\cite{mccormick1987multigrid,wesseling1995introduction,hackbusch2013multi,bramble2019multigrid,hiptmair1998multigrid,brandt1983multigrid,borzi2009multigrid} and nested grid strategies~\cite{miyakoda1977one,zhang2012nested,sullivan1996grid} are widely used to solve PDEs and handle multi-scale problems~\cite{debreu2008two,xue2000advanced}. Multigrid methods use grids of different resolutions to transfer information and accelerate iterations. They efficiently solve large-scale problems and improve computational accuracy. By eliminating low-frequency errors on coarse grids and high-frequency errors on fine grids, multigrid methods effectively handle error convergence at different scales~\cite{he2019mgnet,he2023mgno,shao2022fast}. Nested grid strategies embed higher-resolution fine grids into regions of interest based on a global coarse grid to capture local complex physical phenomena in detail. In weather forecasting, this method provides large-scale background fields on a global scale while refining the grid for target regions to accurately simulate the evolution of local weather systems and the occurrence of extreme events~\cite{bacon2000dynamically}. 

% Our proposed neural nested grid method helps address challenges like boundary information loss in regional forecasting and multi-scale feature capture.

\section{Additional Results}
%
We present more additional results in Figure \ref{fig_0.25-day}, \ref{fig_0.5-day}, \ref{fig_1.0-day} \ref{fig_1.5-day}, \ref{fig_2.0-day}, \ref{fig_2.5-day}, \ref{fig_3.0-day}, \ref{fig_3.5-day}, \ref{fig_4.0-day}, \ref{fig_4.5-day}, \ref{fig_5.0-day}, \ref{fig_5.5-day}, \ref{fig_6.0-day}, \ref{fig_6.5-day}, \ref{fig_7.0-day}, \ref{fig_7.5-day},
\ref{fig_8.0-day}, \ref{fig_8.5-day}, \ref{fig_9.0-day}, \ref{fig_9.5-day},
\ref{fig_10.0-day}, including 18 variables that are importmant to weather forecasting, each with results ranging from 6 hours to 10 days. These additional results further demonstrate the effectiveness of OneForecast. Same as the Figure \ref{fig:visual_results}
, the initial conditions is 00:00 UTC, 1 January 2020.


\begin{figure*}[h]
\centering
\includegraphics[width=1\linewidth]{figures/fig_0.25-day.jpg}
\vspace{-20pt}
\caption{6-hour forecast results of different models.}
\label{fig_0.25-day}
\end{figure*}

\begin{figure*}[h]
\centering
\includegraphics[width=1\linewidth]{figures/fig_0.5-day.jpg}
\vspace{-20pt}
\caption{0.5-day forecast results of different models.}
\label{fig_0.5-day}
\end{figure*}

\begin{figure*}[h]
\centering
\includegraphics[width=1\linewidth]{figures/fig_1.0-day.jpg}
\vspace{-20pt}
\caption{1-day forecast results of different models.}
\label{fig_1.0-day}
\end{figure*}

\begin{figure*}[h]
\centering
\includegraphics[width=1\linewidth]{figures/fig_1.5-day.jpg}
\vspace{-20pt}
\caption{1.5-day forecast results of different models.}
\label{fig_1.5-day}
\end{figure*}

\begin{figure*}[h]
\centering
\includegraphics[width=1\linewidth]{figures/fig_2.0-day.jpg}
\vspace{-20pt}
\caption{2-day forecast results of different models.}
\label{fig_2.0-day}
\end{figure*}


\begin{figure*}[h]
\centering
\includegraphics[width=1\linewidth]{figures/fig_2.5-day.jpg}
\vspace{-20pt}
\caption{2.5-day forecast results of different models.}
\label{fig_2.5-day}
\end{figure*}

\begin{figure*}[h]
\centering
\includegraphics[width=1\linewidth]{figures/fig_3.0-day.jpg}
\vspace{-20pt}
\caption{3-day forecast results of different models.}
\label{fig_3.0-day}
\end{figure*}

\begin{figure*}[h]
\centering
\includegraphics[width=1\linewidth]{figures/fig_3.5-day.jpg}
\vspace{-20pt}
\caption{3.5-day forecast results of different models.}
\label{fig_3.5-day}
\end{figure*}

\begin{figure*}[h]
\centering
\includegraphics[width=1\linewidth]{figures/fig_4.0-day.jpg}
\vspace{-20pt}
\caption{4-day forecast results of different models.}
\label{fig_4.0-day}
\end{figure*}

\begin{figure*}[h]
\centering
\includegraphics[width=1\linewidth]{figures/fig_4.5-day.jpg}
\vspace{-20pt}
\caption{4.5-day forecast results of different models.}
\label{fig_4.5-day}
\end{figure*}


\begin{figure*}[h]
\centering
\includegraphics[width=1\linewidth]{figures/fig_5.0-day.jpg}
\vspace{-20pt}
\caption{5.0-day forecast results of different models.}
\label{fig_5.0-day}
\end{figure*}

\begin{figure*}[h]
\centering
\includegraphics[width=1\linewidth]{figures/fig_5.5-day.jpg}
\vspace{-20pt}
\caption{5.5-day forecast results of different models.}
\label{fig_5.5-day}
\end{figure*}

\begin{figure*}[h]
\centering
\includegraphics[width=1\linewidth]{figures/fig_6.0-day.jpg}
\vspace{-20pt}
\caption{6.0-day forecast results of different models.}
\label{fig_6.0-day}
\end{figure*}

\begin{figure*}[h]
\centering
\includegraphics[width=1\linewidth]{figures/fig_6.5-day.jpg}
\vspace{-20pt}
\caption{6.5-day forecast results of different models.}
\label{fig_6.5-day}
\end{figure*}

\begin{figure*}[h]
\centering
\includegraphics[width=1\linewidth]{figures/fig_7.0-day.jpg}
\vspace{-20pt}
\caption{7.0-day forecast results of different models.}
\label{fig_7.0-day}
\end{figure*}

\begin{figure*}[h]
\centering
\includegraphics[width=1\linewidth]{figures/fig_7.5-day.jpg}
\vspace{-20pt}
\caption{7.5-day forecast results of different models.}
\label{fig_7.5-day}
\end{figure*}

\begin{figure*}[h]
\centering
\includegraphics[width=1\linewidth]{figures/fig_8.0-day.jpg}
\vspace{-20pt}
\caption{8.0-day forecast results of different models.}
\label{fig_8.0-day}
\end{figure*}

\begin{figure*}[h]
\centering
\includegraphics[width=1\linewidth]{figures/fig_8.5-day.jpg}
\vspace{-20pt}
\caption{8.5-day forecast results of different models.}
\label{fig_8.5-day}
\end{figure*}

\begin{figure*}[h]
\centering
\includegraphics[width=1\linewidth]{figures/fig_9.0-day.jpg}
\vspace{-20pt}
\caption{9.0-day forecast results of different models.}
\label{fig_9.0-day}
\end{figure*}

\begin{figure*}[h]
\centering
\includegraphics[width=1\linewidth]{figures/fig_9.5-day.jpg}
\vspace{-20pt}
\caption{9.5-day forecast results of different models.}
\label{fig_9.5-day}
\end{figure*}

\begin{figure*}[h]
\centering
\includegraphics[width=1\linewidth]{figures/fig_10.0-day.jpg}
\vspace{-20pt}
\caption{10.0-day forecast results of different models.}
\label{fig_10.0-day}
\end{figure*}

\begin{figure*}[!t]
  \centering
  \begin{minipage}{\linewidth}
    \centering
    \begin{subfigure}[b]{\textwidth}
      \centering
      \includegraphics[width=0.83\linewidth]{figures/our_pipeline.pdf}
      \caption{}
      \label{fig:RSUencountered1}
    \end{subfigure}
    \vspace{1em} % Space between images
    \begin{subfigure}[b]{\textwidth}
      \centering
      \includegraphics[width=0.83\linewidth]{figures/operational_scheme.pdf}
      \caption{}
      \label{fig:RSUencountered2}
      \vspace{-11px}
    \end{subfigure}
  \end{minipage}
  \caption{The pipeline for our proposed method: (a) The offline network training scheme is divided into four modules: the predefined radar signal processing and detection (point-clouds extractor) by \cite{roldan2024see}, the pre-processing of camera and radar data defined under 'Proposed Approach', and the training process which takes the 'Radar Data Pre-processing' output as input and the filtered 'Camera Image Pre-processing' output as the ground truth for training. (b) The proposed operational deployment of the trained network in (a) considering the required lower-level processes, where radar and camera work independently of each other and provide data for any required further process.}
  \label{our_pipeline}
  \vspace{-11px}
\end{figure*}
%%%%%%%%%%%%%%%%%%%%%%%%%%%%%%%%%%%%%%%%%%%%%%%%%%%%%%%%%%%%%%%%%%%%%%%%
\section{Background}\label{sec_background}
\subsection{Comparative Analysis of Sequential Data Models: RNNs, Transformers, and Mamba}
Recurrent Neural Networks (RNNs) have been widely used for sequential data processing due to their ability to retain internal memory. At each time step, an RNN processes a vector alongside the hidden state of the previous time step, enabling it to capture temporal dependencies. However, RNNs struggle with complex dependencies and suffer from slow training times, limiting their scalability and efficiency~\cite{rnn_limitations}. Transformers, in contrast, use self-attention mechanisms to capture global dependencies across input sequences. While transformers excel at modeling long-range dependencies and are highly parallelizable, they suffer from quadratic scaling, making them inefficient for long sequences~\cite{transformer_limitations}. 

State-Space Models (SSMs)~\cite{mamba2} offer an alternative approach, characterized by their linear structure and associative properties. Unlike RNNs, which are nonlinear, SSMs efficiently model sequential data using linear state equations. They capture temporal dependencies using a discretized model, typically employing a Zero-Order Hold (ZOH) approach to convert continuous parameters into discrete intervals. This linearity allows SSMs to perform inference more efficiently than transformers and RNNs, especially for long sequences. Mamba~\cite{mamba}, a leading SSM-based model, provides linear-time efficiency, adaptive memory compression, and significantly faster inference—up to 5$\times$ faster than transformers. This makes Mamba particularly suitable for modern AI workloads, addressing the limitations of traditional models.

\subsection{Performance and Efficiency of Mamba Compared to Transformers}
Mamba offers significant advantages over transformer-based models in computational efficiency and task performance. It achieves higher token throughput across all batch sizes. For example, with a batch size of 16, Mamba 1.4B processes 1089 tokens/second, outperforming transformer models of similar sizes, such as Transformer 1.3B (323 tokens/second) and Transformer 6.9B (120 tokens/second)~\cite{mamba_performance}. Additionally, Mamba scales efficiently to larger batch sizes, while transformers often encounter out-of-memory (OOM) issues. 

In terms of task performance, Mamba maintains high accuracy across benchmarks. The Mamba-2.8B model achieves an average accuracy of 63.3\%, with key benchmarks including LAMBADA (69.2\%), HellaSwag (66.1\%), and PIQA (75.2\%)~\cite{mamba_benchmarks}. Mamba's efficiency stems from its avoidance of the quadratic complexity of attention mechanisms, enabling it to handle long-range dependencies with lower resource requirements. This makes Mamba ideal for low-power, latency-critical environments where energy efficiency and high throughput are essential.

\subsection{Detailed Architecture of the Mamba Block}
The Mamba block is a fundamental component of the Mamba model. It begins by processing the input sequence, which is normalized using RMS normalization to stabilize training. A skip connection bypasses the block, enabling residual learning. The normalized input undergoes transformations through projection layers and a convolutional layer, followed by the SiLU activation function to introduce non-linearity. 

At the core of the block is the selective state-space model (SSM), which operates on sequences using state equations. The state equation $h_k = A h_{k-1} + B x_k$ updates the hidden state by modeling temporal dependencies through the matrix $A$ and incorporating new input $x_k$ via matrix $B$. The output equation $y_k = C h_k$ maps the hidden state to the output using the matrix $C$, with optional augmentation by another learnable matrix $D$. The output from the selective SSM block is combined with the original input using element-wise operations, followed by another RMS normalization. Finally, the output is passed through a task-specific linear layer and a softmax function to generate predictions. This block is repeated $n$ times in the overall Mamba model, forming a modular and scalable architecture for efficient sequence modeling~\cite{mamba_architecture}.

\begin{figure*}[t]
  \centering
  \includegraphics[width=0.92\linewidth]{figures/Data_preprocessing_results.pdf}
  \caption{Results for $M=10, 70$ and $200$, and $\Phi=\Theta=(-70,70)$ against the original RGB scene on the left.}
  \label{fig:algorithm_results}
  \vspace{-11px}
  \centering
\end{figure*}

%%%%%%%%%%%%%%%%%%%%%%%%%%%%%%%%%%%%%%%%%%%%%%%%%%%%%%%%%%%%%%%%%%%%%%%%
\section{Proposed Approach}
Unlike lidars, 4D imaging radars used in AV suffer from sparse scene representations. Our goal in this work is to bypass lidars and produce sharp 4D radar depth maps by passing the radar output to a data-driven depth map generator. Since the camera RGB images have different characteristics from radar depth maps (i.e., they come from different pixel image subspaces), we propose to compute their unified, constitutive basis vectors and transform them into their spatial spectrum representations using Bartlett’s algorithm. This bridges the gap between their original characteristics. With non-linear frequency progression basis vectors, we propose to encode the semantic segmentations of camera images and their corresponding radar depth maps to estimate their spatial spectrum. Interestingly, the spatial spectrum of both images includes frequency components proportional to $M$ caused by spectral leakage. These frequency components complete the depth for sparse point clouds, as well as introduce frequency bias that helps with fitting highly oscillatory data (the sharp camera images), as shown in Figure \ref{fig:algorithm_results} \cite{xu2024overview}. Figure \ref{fig:RSUencountered1} depicts this transformation and network training pipeline, while \ref{fig:RSUencountered2} shows the deployment pipeline for real-time operation, noting that the 4D radar and camera work independently but synchronously.

%-----------------------------------------------------------%

\subsection{Pixel Positional Encoding and Spectrum Estimation}
This encoding method aims to facilitate the transformation into the spatial spectrum of both the radar and camera images. Fast implementation of this encoding process starts with an initialization of a non-linear phase progression, complex sinusoidal basis vectors for $M$ segments for horizontal and vertical axes, $\phi$ and $\theta$. Our basis functions differ from the standard Fourier basis functions, since we require higher resolution for the output \cite{priestley1981spectral}. For that, we change the phase progression across basis functions from standard linear to non-linear, resulting in changing frequency across receptors. This changing frequency leads to higher resolution spectrum \cite{richards2010principles}. Our segments are described as:
\begin{equation}
    x(m,\phi_n)=e^{-j\pi m sin(\phi_n)} , x(m,\theta_k)=e^{-j\pi m sin(\theta_k)}
    \label{eq:horizontal_basis}
\end{equation}
% \begin{equation}
%     x(m,\theta_i)=e^{-j\pi m sin(\theta_i)}
%     \label{eq:vertical_basis}
% \end{equation}
where $m$ is the segment index, noting that $M$ is proportional to spectrum resolution, and $\phi_n$ and $\theta_k \in \Phi$ and $\Theta$ are variation angles from the set $(-90, 90)$ with lengths $N$ and $K$, respectively. The covariance matrices of every $\phi$ and $\theta$ are defined as $\textbf{C}(\Phi)$ and $\textbf{C}(\Theta) \in \mathbb{C}^{N \times N}$ and $\mathbb{C}^{K \times K}$, in which each of their rows represents the periodograms $\textbf{y}(\phi_n)$ and $\textbf{y}(\theta_k)$. The joint 2D periodogram is:
\begin{equation}
    \textbf{Y}(\phi_n, \theta_k) = \textbf{y}(\theta_n)^T\textbf{y}(\phi_k)
\end{equation}
To calculate the final 2D spatial power spectrum $\textbf{P} \in \mathbb{R}^{N \times K}$ for an input image $\textbf{I}$, we iteratively encode all pixels of $\textbf{I}$ and calculate $P(n,k)$ as:
\begin{equation}
    P(n,k)= \sum_{n=0}^{N-1} \sum_{k=0}^{K-1} \left| \textbf{Y}(\phi_n, \theta_k) \circ \textbf{I}\right|
\end{equation}
where $\circ$ denotes an element-wise multiplication. Experimental results are presented in the next section.
%----------------------------

% \begin{algorithm}[tb]
% \caption{Pixel Encoding and Spectrum Estimation}
% \label{alg:algorithm}


% \textbf{Input}: \textbf{I}, $M$, $\boldsymbol{\Phi}$, and $\boldsymbol{\Theta}$\\
% \textbf{Output}: 2D spectrum estimation for basis $x(m,\phi)$ and $x(m,\theta)$, \textbf{P}
% \begin{algorithmic}[1] %[1] enables line numbers
% \STATE Initialize $x(m,\phi)$, $x(m,\theta)$, and \textbf{P}\\
%         \STATE Compute $\textbf{C}(\boldsymbol{\Phi})$ and $\textbf{C}(\boldsymbol{\Theta})$\\
%         \FOR{each row $\textbf{y}(\theta_k)$ in $\textbf{C}(\boldsymbol{\Theta})$}
%             \FOR{each row $\textbf{y}(\phi_n)$ in $\textbf{C}(\boldsymbol{\Phi})$}
%                 \STATE $P(n,k) \leftarrow sum(|(\textbf{y}(\theta_k)^T \textbf{y}(\phi_n)| \circ \textbf{I})$
%             \ENDFOR
%         \ENDFOR

% \STATE \textbf{return} \textbf{P}

% \end{algorithmic}
% \end{algorithm}
%-----------------------------------------------------------%

\subsection{Radar Data Preprocessing}
This module conditions the DNN detector's depth map, $\textbf{I}_{radar}$, in the predecessor radar pipeline. The objective is to transform 2D depth maps into a spatial spectrum representation of its constitutional bases, $\textbf{P}_{radar}$, to satisfy the input characteristics in the following module, the 'Training Process'. This process of spatial spectrum estimation is defined as $\mathscr{F}(\textbf{I}, M)$, noting that $M$ is proportional to the output resolution.
\begin{equation}
    \textbf{P}_{radar}=\mathscr{F}(\textbf{I}_{radar}, M_{radar})
\end{equation}
% Example input and output are found in Figure 4.

%-----------------------------------------------------------%

\subsection{Camera Image Preprocessing}
In order to obtain the spatial spectrum representations for objects of interest, we transform the RGB image, $\textbf{I}_{cam}$, into its semantic segmentations, $\textbf{Seg}$, considering that this is highly dependent on the semantic segmentation accuracy and classes of the model in use. We use Deeplab v3 \cite{yurtkulu2019semantic} with ResNet101 \cite{chen2017rethinking} trained on the PASCAL Visual Object Classes (VOC) 2012 dataset \cite{Everingham10}. We then transform the semantic image into its spatial spectrum representation $\textbf{P}_{cam}$ through $\mathscr{H}(\textbf{I}, M)$, which is the process of spatial spectrum estimation of camera images.
\begin{equation}
    \textbf{P}_{cam}=\mathscr{H}(\textbf{I}_{cam}, M_{cam})=\mathscr{F}(\textbf{Seg}, M_{cam})
\end{equation}
Note that we require $M_{cam} > M_{radar}$ so that the radar spectrum images have a lower resolution, which leaves room for enhancement with deep learning models. 
% Example results are present in Figure \ref{fig:algorithm_results}.

%-----------------------------------------------------------%

\subsection{Network Training Process}
This module focuses on training a generative model that produces a denser and contour-accurate version of $\textbf{P}_{radar}$. As there are some objects captured by the semantic segmentation model that are not detectable by radar, and vice versa, element-wise multiplication produces the mutuality between both spectrum images which, thereafter, is fed into the learning as ground truth for training a ResNet, being optimized to reduce the difference through L2 loss. The process is described as:
\begin{equation}
    \textbf{P}_{radar} \circ \textbf{P}_{cam} = \text{ResNet}(\textbf{P}_{radar})
\end{equation}

\begin{figure*}[!t]
	\centering
	\begin{subfigure}{0.30\linewidth}
		\includegraphics[width=\linewidth]{figures/Correlation.pdf}
		\caption{}
		\label{fig:Correlation}
	\end{subfigure}
        \hspace{0.01\textwidth}
	\begin{subfigure}{0.30\linewidth}
		\includegraphics[width=\linewidth]{figures/Mutual_Information.pdf}
		\caption{}
		\label{fig: Mutual Information}
	\end{subfigure}
        \hspace{0.01\textwidth}
	\begin{subfigure}{0.30\linewidth}
	        \includegraphics[width=\linewidth]{figures/UCD.pdf}
	        \caption{}
	        \label{fig: UCD}
         \end{subfigure}
	\caption{(a) Correlation and (b) mutual information between several depth map pairs. 'SoTA' refers to depth map obtained by \cite{roldan2024see} while 'Encoded' refers to the spectrum of encoded images using the described bases.}
	\label{fig:subfigures}
    \vspace{-14px}
\end{figure*}

%%%%%%%%%%%%%%%%%%%%%%%%%%%%%%%%%%%%%%%%%%%%%%%%%%%%%%%%%%%%%%%%%%%%%%%%
\section{Experimental Results and Analysis}
We apply the proposed approach to the Radelft dataset's RGB images and radar depth maps of Scene 2, which includes 3400 frames. We performed two main experiments: a test of the spatial spectrum generation using the proposed approach and an enhancement of these spatial spectrum images. 
\par We evaluate the performance of the data preprocessing by computing the Pearson correlation and mutual information metrics between the generated spectrums of RGB pixel semantic segmentation and the corresponding radar depth maps for $M_{radar}=50$ and $M_{cam}=200$, noting that a higher correlation value indicates a smaller discrepancy between the images, while mutual information indicates the learning potential of one modality from the other.

The performance of the depth map generation is measured by MAE, REL, RMSE, and UCD against the lidar point clouds and depth maps. MAE evaluates the average magnitude of errors in predictions, providing a straightforward measure of accuracy. REL normalizes the error by comparing it to the mean of the actual values, offering insight into the relative performance of the predictions. RMSE emphasizes larger errors by squaring the differences before averaging, making it sensitive to outliers. UCD measures the geometric similarity between the generated depth maps and the ground truth point clouds by calculating the average distance from each predicted point to its closest corresponding point in the lidar data. As our generation is dependent on semantic segmentations, this approach eliminates the average distance inflation that is caused by detectable object discrepancies between the semantic segmentation model, and radar and lidar point clouds when using Bidirectional Chamfer Distance (BCD).
% \vspace{-3px}

%---------------------------------------------%
\subsection{Data Preprocessing}
Our data preprocessing pipeline includes an input data conditioning sub-module followed by the proposed encoding approach explained in the previous sections. We performed experiments for $M = 10, 50, 70, 200$ and $\Phi=\Theta=(-70,70)$ degrees that truncate significant spectrum leakage at higher angles. The radar input is a simple data structure transformation from projected 3D point clouds coordinates to 2D depth maps with the pixel value being inversely proportional to depth. We test our data preprocessing against Pearson correlation and mutual information. The higher value of Pearson correlation represents a stronger linear relationship between the two variables, while the higher value of mutual information indicates a greater reduction in the entropy when predicting one variable from another.

\par In Figure \ref{fig:algorithm_results}, one can observe that a greater $M$ produces higher-resolution images that preserve the contours of objects. The ripples in both horizontal and vertical axes are due to spectral leakage. Figure \ref{fig:Correlation} plots the Pearson correlation values per frame between the camera and radar modalities for $M_{radar}=50$ and $M_{cam}=200$. It also shows that the correlation is still higher when both radar depth map and camera semantic segmentation are encoded with the proposed approach. It also shows that the correlation is significantly higher for single modality encoding. Figure \ref{fig: Mutual Information} shows that mutual information is significantly improved when we encode both modalities. It also shows that mutual information is still significantly improved when we encode only a single modality.

\par Tables \ref{tab:Correlation} and \ref{tab:Mutual Information} show the averages (mean values) of the plots in Figures \ref{fig:Correlation} and \ref{fig: Mutual Information}. The results show that there are improvements by factors of 3.88 and 76.69 in Pearson correlation and mutual information, respectively.
\vspace{-3px}

% \begin{table}[h]
% \renewcommand{\arraystretch}{1.2} % Adjust the row height (1.5x the default height)
% \parbox{.45\linewidth}{
% \centering
% \begin{tabular}{|c||c c|}
% \hline
%     & \textbf{I}_{cam} & \textbf{P}_{cam} \\ \hline\hline
% \textbf{I}_{radar} & 0.1646 & 0.5503 \\
% \textbf{P}_{radar} & 0.0809 & \textbf{0.6396} \\ \hline
% \end{tabular}
% \caption{Average Pearson Correlation for different pairs.}
% \label{tab:Correlation}
% }
% \hfill
% \parbox{.45\linewidth}{
% \centering
% \begin{tabular}{|c||c c|}
% \hline
%     & \textbf{I}_{cam} & \textbf{P}_{cam} \\ \hline\hline
% \textbf{I}_{radar} & 0.0359 & 0.6117 \\ 
% \textbf{P}_{radar} & 0.4863 & \textbf{2.7533} \\ \hline
% \end{tabular}
% \caption{Average Mutual Information for different pairs.}
% \label{tab:Mutual Information}
% }
% \vspace{-12px}
% \end{table}

% \begin{table}[h]
% \renewcommand{\arraystretch}{1.2} % Adjust the row height
% \parbox{.45\linewidth}{
% \centering
% \begin{tabular}{|c||c c|}
% \hline
%     & \textbf{I}_{cam} & \textbf{P}_{cam} \\ \hline\hline
% \textbf{I}_{radar} & 0.1646 & 0.5503 \\
% \textbf{P}_{radar} & 0.0809 & \textbf{0.6396} \\ \hline
% \end{tabular}
% \caption{Average Pearson Correlation for different pairs.}
% \label{tab:Correlation}
% }
% \hfill
% \parbox{.45\linewidth}{
% \centering
% \begin{tabular}{|c||c c|}
% \hline
%     & \textbf{I}_{cam} & \textbf{P}_{cam} \\ \hline\hline
% \textbf{I}_{radar} & 0.0359 & 0.6117 \\ 
% \textbf{P}_{radar} & 0.4863 & \textbf{2.7533} \\ \hline
% \end{tabular}
% \caption{Average Mutual Information for different pairs.}
% \label{tab:Mutual_Information}
% }
% \vspace{-10pt} % Slightly reduce spacing without excessive compression
% \end{table}

\begin{table}[h]
\renewcommand{\arraystretch}{1.2} % Adjust the row height
\parbox{.45\linewidth}{
\centering
\begin{tabular}{|c||c c|}
\hline
    & $\mathbf{I}_{\text{cam}}$ & $\mathbf{P}_{\text{cam}}$ \\ \hline\hline
$\mathbf{I}_{\text{radar}}$ & 0.1646 & 0.5503 \\
$\mathbf{P}_{\text{radar}}$ & 0.0809 & \textbf{0.6396} \\ \hline
\end{tabular}
\caption{Average Pearson Correlation for different pairs.}
\label{tab:Correlation}
}
\hfill
\parbox{.45\linewidth}{
\centering
\begin{tabular}{|c||c c|}
\hline
    & $\mathbf{I}_{\text{cam}}$ & $\mathbf{P}_{\text{cam}}$ \\ \hline\hline
$\mathbf{I}_{\text{radar}}$ & 0.0359 & 0.6117 \\ 
$\mathbf{P}_{\text{radar}}$ & 0.4863 & \textbf{2.7533} \\ \hline
\end{tabular}
\caption{Average Mutual Information for different pairs.}
\label{tab:Mutual_Information}
}
\vspace{-10pt} % Slightly reduce spacing without excessive compression
\end{table}



%---------------------------------------------%
\begin{figure*}[t!]
  \centering
  \includegraphics[width=1\linewidth]{figures/Results.pdf}
  \caption{Results from the training module for four example frames. From left to right for each example frame: Scene in RGB, Camera spatial spectrum ($\textbf{P}_{cam}$), original radar depth map, output from trained ResNet101. In each of the 4 frames, observe that our approach leads to sharper depth maps.}
  \vspace{-11px}
  \label{fig:training_results}
  \centering
  \vspace{-6px}
\end{figure*}

\subsection{Depth Map Generation}
% \vspace{-5px}
The depth map generation is achieved with the ResNet101 network trained as described in the previous section with $M_{cam}=200$, $M_{radar}=20$ and $\Phi=\Theta=(-70,70)$. The ResNet101 is trained for 10,000 epochs while the input data are compressed with the natural logarithm to make different features comparable in terms of scale. Qualitatively, from the results in Figure \ref{fig:training_results}, one can observe that the intensity locations in the radar depth maps are comparable to locations in camera spectrum images. Also, the output depth maps from the ResNet101 show object contours clearly compared to the original radar depth map. However, the magnitude of the ripples is significantly higher due to the logarithmic feature compression used at the input, which can be rescaled by exponentiation.
\par Quantitatively, we measure the performance of the output with MAE, REL, RMSE, and UCD. Specifically for UCD, we transform the spectrum images into 3D point clouds (in which the depth is inversely proportional to the pixel value) and apply the calculations. Table \ref{tab:UCD} presents the results for different methods including the state-of-the-art (SOTA) DNN detector \cite{roldan2024see} and the OS-CFAR. Figure \ref{fig: UCD} shows the per frame UCD for our method and SOTA. We observe that our approach improves MAE, REL, UCD by 21.13\%, 7.9\% and 27.95\%, respectively. However, the performance degrades by 12\% in terms of RMSE. We believe that this is due to the fact that, compared to other approaches, our generated depth maps are dense, inflating the RMSE result as there are gaps in the corresponding lidar depth maps.

\renewcommand{\arraystretch}{1.4}
\begin{table}[h]
    \centering
    \begin{tabular}{|c|c|c|c|c|}
        \hline
        \textbf{Method} & \textbf{MAE} & \textbf{REL} & \textbf{RMSE} & \textbf{UCD ($m$)}\\
        \hline\hline
        Proposed Approach  & \textbf{0.026} & \textbf{0.025} & 0.111 & \textbf{3.48} \\
        SOTA DNN         & 0.033 & 0.027 & \textbf{0.099} & 4.83 \\
        OS-CFAR          & 0.029 & 0.152 & 0.258 & 18.02 \\
        % Peak Detector   &  -- \\
        \hline\hline
        \textbf{Improvement (\%)} & 21.13 & 7.9 & -12 & 27.95\\
        \hline
    \end{tabular}
    \caption{MAE, REL, RMSE, and UCD for different methods}
    \label{tab:UCD}
    \vspace{-5px}
\end{table}







%%%%%%%%%%%%%%%%%%%%%%%%%%%%%%%%%%%%%%%%%%%%%%%%%%%%%%%%%%%%%%%%%%%%%%%%


\section{Conclusion and future directions} \label{sec:conclusion}

In this paper we proposed a nested MLMC framework that offers important computational savings by performing most calculations in low precision and exploiting approximate random normal variables for the low precision path calculations. The low precision calculations could be performed in fixed precision on an FPGA for greater efficiency, and we suggested a procedure to optimise the bit-widths of every variable at each Monte Carlo level. This is an important improvement over previous mixed precision MLMC frameworks which held the lower precision fixed \cite{Rounding_error_oliver} or defined uniform bit-width at every level heuristically \cite{brugger2014mixed}. Our numerical results suggest that for the first levels our procedure reduces the cost at these levels by a factor 5 or 7. Hence the overall savings are significant since most paths are calculated on the first levels. Our approach would be even more efficient for the Milstein scheme because its higher order strong convergence leads to a greater proportion of the computational costs being on the coarsest levels.

The next stage of the research project will be to implement the RNG methods and the nested framework on FPGAs to determine the hardware requirements and confirm the extent of the computational savings. It would also be good to compare the performance benefits to using half-precision floating point arithmetic on GPUs or CPUs for the low-accuracy computations.




\bibliographystyle{IEEEtran}
\bibliography{bibliography}

\end{document}
