\section{Related Work}
\subsection{KANs}
After the recent introduction of \acp{kan} by Liu et al. \cite{liu2024kan,liu2024kan20} there has been a great variety of attempts to apply the ideas of \acp{kan} to different fields like image processing \cite{bodner2024convolutional}, satellite image segmentation \cite{cambrin2024kan}, graph neural networks \cite{kiamari2024gkan,bresson2024kagnns} and even transformers \cite{yang2024kolmogorov}. Hou et al. \cite{hou2024comprehensive} give a great overview of the different applications and extensions of \acp{kan}. Their explainability can be of great value in fields where machine learning approaches are strongly regulated like survival analysis in medicine or engineering \cite{knottenbelt2024coxkan}. Multiple works have already benchmarked the performance of \acp{kan} against \acp{mlp} \cite{yu2024kan, poeta2024benchmarking} and found \acp{kan} to be a more suitable alternative in some fields. Alter et al. \cite{alter2024robustness} found that large-scale \acp{kan} are more robust against adversarial attacks as \acp{mlp} and thus form an interesting direction for further research in multiple fields.


In \cite{wang2025kolmogorov} the authors explore different partial differential equation forms based on \ac{kan} instead of \ac{mlp} for solving forward and inverse problems in computational physics. A systematical comparison demonstrates that the \ac{kan} approach significantly outperforms \ac{mlp} regarding accuracy and convergence 
speed. Further successful applications of \ac{kan} can be found for operator learning in computational mechanics \cite{Abueidda2025} and image classification \cite{Jamali2024}.


\subsection{Complex-Valued Neural Networks}
After early introduction of \acp{cvnn} \cite{hirose2003complex} they have lately risen in popularity since the introduction of building blocks for deep learning architectures \cite{trabelsi2018deep}. Since then a lot of theoretical contributions have been made \cite{tan2022real,eilers2023building,zhang2021optical} to enable a multitude of applications \cite{zhong2023real,chen2023spectral,liu2023pixelwise,xing2023phase,yakupouglu2024comparison}.

The most closely related prior work in the complex domain are deep \acp{crbfn} \cite{chen2008fully, soares2024deep}. However, in \acp{crbfn}, the \acp{rbf} are applied to all inputs of a neuron (e.g. vertex of the computational graph) simultaneously, while in \acp{ckan} the \acp{rbf} are applied on the edges of the computational graph to each value individually. Thus \acp{crbfn} are architecturally more similar to classical \acp{mlp} with \acp{rbf} as activations functions, where we aim to adopt the \ac{kan} framework to the complex domain.