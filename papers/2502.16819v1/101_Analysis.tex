\section{Main Claim and Proof Roadmap}
\begin{theorem} 
Let $\mc M \subset \mathbb{R}^D$ be a complete and connected $d$-dimensional manifold whose extrinsic geodesic curvature is bounded by $\kappa$. 
\begin{itemize}
    \item {\bf Assumptions on $\mc M$}
    Assume $\kappa \diam(\mc M) \ge 1$ and $\sigma \sqrt{D} \leq \frac{1}{640} \tau_{\mc M}$.
    \item {\bf Assumptions on $Q$} Suppose that the landmarks $Q=\left\{\mb q_1, \dots, \mb q_{ M}\right\} \subset \mc M$ are $\delta$-separated, and form a $\delta$-net for $\mc M$. Assume $\delta \le \diam(\mc M)$.
    \item {\bf Assumptions on $E^1$} Further suppose that the first-order graph $E^1$ satisfies that $u \overset{1}{\rightarrow} v \in E^1$ when $\left\|\mb q_u - \mb q_v\right\|_2 \leq R_{\text{nbrs}}$, and ${a\delta} = R_{\text{nbrs}} \leq \frac{\sqrt{2}}{64}\sigma\sqrt{d}$ for some $a\ge40,$ and $\tau_{\mc M}$ is the reach of the manifold $\mc M$.  
    \item {\bf Assumptions on $E^0$}  Suppose for every $\mb q \in Q$ and every $\mb q' \in \mb q + N^{\eta}_{\mb q}\mc M$ with 
    \begin{equation}  \eta \geq  \epsilon_1 + \delta + 14\sigma\sqrt{d}\sqrt{\kappa\diam(\mc M)+\log(\diam(\mc M))-\log(\delta)+\log(7)}
    \end{equation} there exists a ZOE $\mb q \overset{0}{\rightarrow} \mb q''$ with $\mb q'' \in B_{\mc M}( \mb q', \frac{1}{80}\tau_{\mc M} )$
\end{itemize} 
With probability at least $ 1-4e^{-\frac{d}{16}} - \left(\frac{e}{2}\right)^{-\frac{D}{2}}$ in the noise $\mb z$, the Algorithm \ref{algo:1-0-1} with parameters 
\begin{align} 
R_a &= R_{\mr{nbrs}} - \delta\\
\eps_1 &>  \frac{2R_{\text{nbrs}}}{0.55}\left(\frac{2}{3} + \frac{8}{3}\kappa \diam(\mc M) + \right.\\
&\qquad \qquad \left.+16\kappa\sigma\sqrt{d}\sqrt{\kappa \diam(\sM) + \log(a) - \log(\delta) + \log(\diam(\sM)) + \log(100)}\right) \\ 
\epsilon_2 &> C\max\{\kappa,1\} \sigma \sqrt{d} 
\end{align} 
produces an output $\mb q_\star$ satisfying \begin{equation}
    d_{\mc M}(\mb q_\star, \mb x_\natural) \leq 2\epsilon_2
\end{equation}

with an overall number of arithmetic operations bounded by 
    \begin{equation}
       \left(Dd + d \paren{1 + 4\sqrt{2} a e }^d\right)\left( \frac{10 \mr{diam}^2(\mc M) + 20 \sigma^2 D}{ \epsilon_1 \delta } + \frac{8}{\kappa \delta}\right) + D \times \$^{\eta}_{\frac{1}{80} \tau_{\mc M}} (\mc M)
    \end{equation}
\end{theorem} 

\begin{proof}
With the above assumptions and the combination of \cref{prop:phase I}, \cref{prop:phase II prop}, and \cref{prop:phase III}, it's easy to show that with probability at least $ 1-4e^{-\frac{d}{16}} - \left(\frac{e}{2}\right)^{-\frac{D}{2}}$ in the noise $\mb z$, the output of the algorithm $\mb q_\star$ satisfies
\begin{equation}
    d_{\mc M}(\mb q_\star, \mb x_\natural) \leq 2\epsilon_2,
\end{equation}

and phase $\mr{I}$ and phase $\mr{III}$ use at most $ \frac{\diam^2(\mc M) + 2 \sigma^2 D} {0.1375 \epsilon_1 \delta} + \frac{8}{\kappa\delta}$ steps, and phase $\mr{II}$ takes at most $ D \times \$^{\eta}_{\frac{1}{80} \tau_{\mc M}} (\mc M)$ operations.

And we note that projection of the gradient onto the tangent space takes the cost of $D*d$ number of operations, and choosing the first-order neighbor takes the cost of $d * \max_{\vq_u \in \sQ}\abs{E_{u}^1}$. The number of arithmetic operations  of zero-order step is $D*\max_{\mb q\in Q} \deg^0(\mb q)$, where $D$ comes from the Euclidean distance calculation, and $\max_{\mb q\in Q} \deg^0(\mb q)$ is bounded by the defined geometric quantity $\$^{\eta}_{\frac{1}{80} \tau_{\mc M}} (\mc M)$. Combined all of terms above, we end up with the bound on the number of operations performed by our algorithm.
\end{proof}



\begin{fact} 
    For a manifold $\mc M$ with reach $\tau_{\mc M}$ and extrinsic curvature bounded by $\kappa$, we have
    \begin{equation}
        \tau_{\mc M} \leq 1/\kappa.
    \end{equation}
See Proposition 2.3 in \cite{aamari2019estimating}. We used this fact directly in the following proofs.
\end{fact}




\begin{proposition}\label{prop:phase I}
 Let $\mc M \subset \mathbb{R}^D$ be a complete and connected $d$-dimensional manifold whose extrinsic geodesic curvature is bounded by $\kappa$. Suppose the landmarks $\{ \mb q_u \}$ are $\delta$-separated, and form a $\delta$-net for $\mc M$, and that the first order graph $E^1$ satisfies $u \overset{1}{\rightarrow} v \in E^1$ when $\| \mb q_u - \mb q_v \|_2 \le R_{\mr{nbrs}}$, and that $40 \delta \le R_{\text{nbrs}} \le \tau_{\mc M}$. Assume $\delta \leq \diam(\mc M)$ and $\kappa \diam(\mc M) \geq 1$. Then with probability at least 
  \begin{equation}
      1-e^{-\frac{9d}{2}} - \left(\frac{e}{2}\right)^{-\frac{D}{2}}
  \end{equation}
  in the noise $\mb z$, the first phase of first-order optimization in Algorithm \ref{algo:1-0-1} with parameters  
  \begin{align}
        R_a &= R_{\text{nbrs}}-\delta,  \\
      \epsilon_1 &>  \frac{2R_{\text{nbrs}}}{0.55}\left(\frac{2}{3} + \frac{8}{3}\kappa \diam(\mc M) + \right.\\
      &\qquad \qquad \left. + 16\kappa\sigma\sqrt{d}\sqrt{\kappa \diam(\sM) + \log(a) - \log(\delta) + \log(\diam(\sM)) + \log(100)}\right), \\ 
  \end{align}
  produces $\mb q_{i_{\mr{I}}}$ satisfying 
  \begin{equation}
        \grad[\varphi_{\mb x}](\mb q_{i_{\mr{I}}})
        =\| \mc P_{T_{i_{\mr{I}}}}(\mb x - \mb q_{i_{\mr{I}}}) \|_2 
        \leq \epsilon_1
    \end{equation}
    using at most 
    \begin{equation}
        \frac{\diam^2(\mc M) + 2 \sigma^2 D} {0.1375 \epsilon_1 \delta}
    \end{equation}
    steps. 
\end{proposition}


\begin{proof}
    
    We can bound the total number of steps by dividing maximum initial distance by the minimal distance decrease over each first-order step.

    We note that \begin{equation}
    \begin{aligned}
        \varphi_{\mb x}(\mb q_{i^0})
        &= \frac{1}{2}\left\| \mb q_{i^0} - \mb x \right\|_2^2\\
        &\leq \left\| \mb q_{i^0} - \mb x_\natural \right\|_2^2 + \left\| \mb z \right\|_2^2\\
        &\leq \text{diam}^2(\mc M) + \left\| \mb z \right\|_2^2
    \end{aligned}
\end{equation} 

     We let $\mb{\bar z}$ be the standard unit variance Gaussian variable, for any $0<t<1/2$, we have
\begin{equation}
\begin{aligned}
    \mathbb P[\left\| \mb z \right\|_2^2 \ge 2\sigma^2D] &= \mathbb P[\left\| \mb{\bar z} \right\|_2^2 \ge 2D]\\ &= \mathbb P[e^{t \left\| \mb{\bar z} \right\|_2^2} \ge e^{2tD}] \\ &\le \frac{\mathbb E[e^{t \left\| \mb{\bar z} \right\|_2^2}]}{e^{2tD}} \\ 
    &= \frac{e^{-2tD}}{(1-2t)^{\frac{D}{2}}},
\end{aligned}
\end{equation}
where we used Markov's inequality from the second line to the third line. In particular, we can pick $t = \frac{1}{4}$, so with probability at least $1 - (\frac{e}{2})^{-\frac{D}{2}}$ in the noise $\mb z$, we have
\begin{equation}\label{eq:bound on initial distance}
        \varphi_{\mb x}(\mb q_{i^0}) \leq \text{diam}^2(\mc M) + 2\sigma^2D
\end{equation}

Now we will analyze the decrease over each first-order step from $u$ to $u^+$. We have
\begin{equation}
    \begin{aligned}
        \varphi_{\mb x}(\mb q_{u^+}) - \varphi_{\mb x}(\mb q_u) 
    & = \frac{1}{2}(\|\mb x - \mb q_{u^+}\|_2^2 - \|\mb x - \mb q_u \|_2^2) \\
    & = \frac{1}{2} \| \mb q_{u^+} - \mb q_u \|_2^2
        - \innerprod{\mc P_{T_u}(\mb x - \mb q_u)}{\mc P_{T_u}(\mb q_{u^+} - \mb q_u)}
        - \innerprod{\mc P_{N_u}(\mb x - \mb q_u)}{\mc P_{N_u}(\mb q_{u^+} - \mb q_u)}
    \end{aligned}
\end{equation}

We apply Lemma \ref{lem:edge bounded by tangent term} to bound the first term and Lemma \ref{lem:bound dot product of gradient and edge in the normal direction} to get a high probability bound on the last term, setting $t = 3\sigma\sqrt{d}$. 



Next, we apply \cref{lem:tangent space covering}and \cref{lem:bound dot product of gradient and edge in tangent direction} to bound the second term, which dominates the decrease in the values of the objective function after taking a first-order step. As a reminder, the step rule in \cref{algo:1-0-1} is 
\begin{equation}
    u^+ = \arg\min_{u \overset{1}{\rightarrow} v}
     \left\| \mc P_{B(0,R_a)} \mc P_{T_u} (\mb x -\mb q_u) - \mc P_{T_u} (\mb q_v - \mb q_u) \right\|_2.
\end{equation}
We observe that before phase I in \cref{algo:1-0-1} terminates, $\left\|\mc P_{T_u} (\mb x -\mb q_u)\right\| \ge \epsilon_1 > R_a$ holds. Then the step rule is equivalent to 
\begin{equation}
    u^+ = \arg\min_{u \overset{1}{\rightarrow} v}
     \left\| \frac{\mc P_{T_u} \paren{\mb x - \mb q_u}}{\| \mc P_{T_u} \paren{\mb x - \mb q_u}\|_2} R_a - \mc P_{T_u} (\mb q_v - \mb q_u) \right\|_2,
\end{equation}
in which we have $\left\|\frac{\mc P_{T_u} \paren{\mb x - \mb q_u}}{\| \mc P_{T_u} \paren{\mb x - \mb q_u}\|_2} R_a\right\|_2 = R_a$. Applying \cref{lem:tangent space covering}, there exists $u \overset{1}{\rightarrow} v \in E^1$, such that $\left\| \frac{\mc P_{T_u} \paren{\mb x - \mb q_u}}{\| \mc P_{T_u} \paren{\mb x - \mb q_u}\|_2} R_a - \mc P_{T_u}(\mb q_v - \mb q_u)\right\|_2 \leq \delta + \frac{1}{2}\kappa R_a^2$. From the construction of this step rule, we have  $\left\| \frac{\mc P_{T_u} \paren{\mb x - \mb q_u}}{\| \mc P_{T_u} \paren{\mb x - \mb q_u}\|_2} R_a - \mc P_{T_u}(\mb q_{u^+} - \mb q_u)\right\|_2 \leq \left\| \frac{\mc P_{T_u} \paren{\mb x - \mb q_u}}{\| \mc P_{T_u} \paren{\mb x - \mb q_u}\|_2} R_a - \mc P_{T_u}(\mb q_v - \mb q_u)\right\|_2 \leq \delta + \frac{1}{2}\kappa R_a^2$. Thus when $\mb x \neq \mb q_u$, applying \cref{lem:bound dot product of gradient and edge in tangent direction}, we have $\innerprod{\mc P_{T_u} \paren{\mb x - \mb q_u}}{\mc P_{T_u}\paren{\mb q_v - \mb q_u}} \ge 0.55 \| \mc P_{T_u} \paren{\mb x - \mb q_u} \|_2   \| \mc P_{T_u}\paren{\mb q_v - \mb q_u} \|_2$.





Combining all the above results, we conclude that with probability at least $1-e^{-\frac{9d}{2}}$ in the noise $\vz$, we have 
\begin{equation}
    \begin{aligned}
        \varphi_{\mb x}(\mb q_{u^+}) - \varphi_{\mb x}(\mb q_u)
        &\leq \frac{2}{3} \| \mc P_{T_u}(\mb q_{u^+} - \mb q_u) \|_2^2 - 0.55 \| \mc P_{T_u}(\mb x - \mb q_u) \|_2 \| \mc P_{T_u}(\mb q_{u^+} - \mb q_u) \|_2\\ 
        &+ \frac{8}{3} \kappa (\diam(\mc M) + \sqrt{2} \sigma \sqrt{\log |E^1|} + 3\sigma\sqrt{d}) \| \mc P_{T_u}(\mb q_{u^+} - \mb q_u) \|_2^2.
    \end{aligned}
\end{equation}

Given our assumption on $\epsilon_1$ and applying the bound of $\abs{E^1}$ in \cref{lem:app:foe_bound}, together with the assumption that $\delta \leq \diam(\mc M)$ and $\kappa \diam(\mc M)>1$, it follows that before Phase I ends, we have 
\begin{equation}
\begin{aligned}
0.55 \| \mc P_{T_u}(\mb x - \mb q_u) \|_2
&> 0.55 \epsilon_1\\
&> 2R_{\mr{nbrs}}\left(\frac{2}{3} + \frac{8}{3}\kappa \diam(\mc M) + \right.\\
      &\qquad \qquad \left. + 16\kappa\sigma\sqrt{d}\sqrt{\kappa \diam(\sM) + \log(a) - \log(\delta) + \log(\diam(\sM)) + \log(100)}\right)\\
&> 2R_{\mr{nbrs}} \left( \frac{2}{3} + \frac{8}{3}\kappa \diam({\mc M}) + 8\kappa \sigma \sqrt{d} + 8\kappa\sigma\sqrt{d}\sqrt{\kappa \diam(\sM) + \log(a) - \log(\delta) + \log(\diam(\sM)) + \log(100)}\right)\\
&> 2R_{\mr{nbrs}} \left( \frac{2}{3} + \frac{8}{3}\kappa \diam({\mc M}) + 8\kappa \sigma \sqrt{d} + 8\kappa\sigma \sqrt{\log|E^1|}\right)\\
&>2R_{\text{nbrs}}\paren{\frac{2}{3} + \frac{8}{3}\kappa \left(\mathrm{diam}(\mc M) + \sigma\sqrt{2\log\left|E^1\right|} + 3\sigma\sqrt{d}\right)},
\end{aligned}
\end{equation}
together with $\| \mc P_{T_u}(\mb q_{u^+} - \mb q_u) \|_2 \leq \|\mb q_{u^+}-\mb q_u\|_2 \le R_{\text{nbrs}}$, then we have 
\begin{equation}\label{eq: bound of sufficient decrease in each step}
    \varphi_{\mb x}(\mb q_{u^+}) - \varphi_{\mb x}(\mb q_u) 
    \le -\frac{0.55}{2} \| \mc P_{T_u}(\mb q_{u^+} - \mb q_u) \|_2 \| \mc P_{T_u}(\mb x - \mb q_u) \|_2.
\end{equation}
In the following, we develop a lower bound for $\| \mc P_{T_u}(\mb q_{u^+} - \mb q_u) \|_2$.
Let $\mb\gamma:[0,1] \rightarrow \mc M$ be a minimum length geodesic joining $\mb q_u$ and $\mb q_{u^+}$ with constant speed $d_{\mc M}(\mb q_u, \mb q_{u^+})$, where $\mb\gamma(0)=\mb q_u, \mb\gamma(1)=\mb q_{u^+}$. Then we have

\begin{equation}\label{eq: bound norm of edge embedding}
    \begin{aligned}
        \left\|P_{T_u}\left(\mb q_{u^+}-\mb q_u\right)\right\|_2
        &= \left\|
        P_{T_u}\left(\mb\gamma(1)-\mb\gamma(0)
        \right)\right\|_2\\
        &= \left\|
        P_{T_u}\left(\int_{a=0}^1 \Dot{\mb\gamma}(a)da
        \right)\right\|_2\\
        &= \left\|\Dot{\mb\gamma}(0) + \int_{a=0}^1 \int_{b=0}^a P_{T_u}\Ddot{\mb\gamma}(b) db da\right\|_2\\
        &\geq \left\|\Dot{\mb\gamma}(0)\right\|_2 - \left\|\int_{a=0}^1 \int_{b=0}^a P_{T_u}\Ddot{\mb\gamma}(b) db da\right\|_2\\
        &\geq \left\|\Dot{\mb\gamma}(0)\right\|_2 - \int_{a=0}^1 \int_{b=0}^a \left\|P_{T_u}\Ddot{\mb\gamma}(b)\right\|_2 db da\\
        &\geq \left\|\Dot{\mb\gamma}(0)\right\|_2 - \int_{a=0}^1 \int_{b=0}^a \kappa \left\|\Dot{\mb\gamma}(0)\right\|_2^2 db da\\
        &= d_{\mc M}(\mb q_{u^+}, \mb q_u) - \frac{1}{2}\kappa d_{\mc M}^2(\mb q_{u^+}, \mb q_u)\\
        &\geq \frac{1}{2} d_{\mc M}(\mb q_{u^+}, \mb q_u)\\
        &\geq \frac{1}{2} \delta,
    \end{aligned}
\end{equation}
where in the third line we've used the fact that $\Dot{\mb\gamma}(0)$ lies in the tangent space of $q_u$, in the forth and fifth lines we applied triangle inequality, and in the final lines we've used our assumption that $d_{\mc M}(\mb q_{u^+}, \mb q_u) \leq \tau_{\mc M} \le \frac{1}{\kappa}$ and the landmarks are $\delta$-separated.\\
\vspace{0.2in}
Plugging this back into Equation \eqref{eq: bound of sufficient decrease in each step}, we have 
\begin{equation}
        \varphi_{\mb x}(\mb q_{u^+}) - \varphi_{\mb x}(\mb q_u) \le \frac{\delta}{2} \left(-\frac{0.55}{2}\| \mc P_{T_u}(\mb x - \mb q_u) \|_2 \right) \le \frac{-0.55}{4}\delta\epsilon_1
\end{equation}
Lastly, combining this result with Equation \eqref{eq:bound on initial distance} yields the desired upper bound on the number of iterations.


\end{proof}

\noindent In the following proposition, we let 
\begin{equation}
    N^\eta_{\mb q} \mc M 
\end{equation}
denote the $\eta$-dilated normal space: 
\begin{equation}
    N^{\eta}_{\mb q} \mc M = \set{ \mb v \in \bb R^D \mid \| \mc P_{T_{\mb q}\mc M} \mb v \|_2 \le \eta }
\end{equation}


\begin{proposition} \label{prop:phase II prop}
Assume $\delta \leq \diam(\mc M)$, $\kappa \diam(\mc M)\geq 1$ and $\sigma\sqrt{D} \leq \frac{1}{640} \tau_{\mc M}$. Suppose for every $\mb q \in Q$ and every $\mb q' \in \mb q + N^{\eta}_{\mb q}\mc M$ with 
\begin{equation}  
\eta \geq  \epsilon_1 + \delta + 14\sigma\sqrt{d}\sqrt{\kappa\diam(\mc M)+\log(\diam(\mc M))-\log(\delta)+\log(7)},
\end{equation} 
there exists a ZOE $\mb q \overset{0}{\rightarrow} \mb q''$ with $\mb q'' \in  B_{\mc M}( \mb q', \frac{1}{80}\tau_{\mc M} )$ Then, with probability at least {$1- 2e^{-\frac{9d}{2}}$} in the noise $\mb z$, whenever $\mb q_{i_{\mr{I}}}$ satisfies 
    \begin{equation}
        \| \mc P_{T_{\mb q_{i_{\mr{I}}}}\mc M} (\mb x - \mb q_{i_{\mr{I}}} ) \| \le \epsilon_1, 
    \end{equation}
    the second phase (zero-order step) produces $\mb q_{i_{\mr{II}}}$ satisfying
    \begin{equation}
        d_{\mc M}(\mb q_{i_{\mr{II}}}, \mb x_\natural) \leq 0.1/\kappa.
    \end{equation}
\end{proposition}
\begin{proof}
From \cref{lemma:bound of number of landmarks within R ball}, we have
\begin{equation}
    \abs{\sQ} \le \paren{1 + 4\sqrt{2}\delta^{-1} \diam(\sM) e^{\kappa \diam(\sM)} }^d,
\end{equation}
which gives us that
\begin{equation}
    \begin{aligned}
    \log \abs{\sQ} 
    &\leq d \log \paren{1 + 4\sqrt{2}\delta^{-1} \diam(\sM) e^{\kappa \diam(\sM)}}\\
    &\leq d \log \paren{\paren{\frac{1}{e}+4 \sqrt{2}}\delta^{-1}\diam(\mc M)e^{\kappa \diam(\mc M)}}\\
    &\leq d \paren{\log \paren{ 7 \delta^{-1} \diam(\mc M)} + \kappa \diam(\mc M)},
    \end{aligned}
\end{equation}
where we used the assumption that $\delta \leq \diam(\mc M)$ and $\kappa \diam(\mc M) \geq 1$ from the first line to the second line. Together with the assumption on $\eta$, we have
\begin{equation}
    \eta \geq \epsilon_1 + \delta + \sigma \paren{4 \sqrt{\log \abs{\sQ}} + 10\sqrt{d}}.
\end{equation}


Under these conditions and the assumption that {$\sigma\sqrt{D} \leq \frac{1}{640} \tau_{\mc M} $}, applying \cref{Phase II ZOE gives an acceptable landmark}, we conclude that with probability at least {$1 - 2e^{\frac{-9d}{2}}$} in the noise,  $\mb q_{i_{\mr{II}}} = \arg\min_{\mb q': \mb q_{i_{\mr{I}}} \overset{0}{\rightarrow} \mb q'}\left\|\mb q' -\mb x\right\|_2$ satisfies 
\begin{equation}
    d_{\mc M}(\mb x_\natural, \mb q_{i_{\mr{II}}}) \le \frac{1}{20}\tau_{\mc M}+2\delta \leq \frac{1}{20}\tau_{\mc M}+\frac{1}{20}R_{\text{nbrs}} \le \frac{1}{20}\tau_{\mc M}+\frac{1}{20}\tau_{\mc M} = \frac{1}{10}\tau_{\mc M}.
\end{equation} 
\end{proof}

\begin{proposition}\label{prop:phase III}
      Assume $\sigma\sqrt{D} \leq \frac{1}{640} \tau_{\mc M}$. Suppose the landmarks $\{ \mb q_u \}$ are $\delta$-separated, and form a $\delta$-net for $\mc M$, and that the first order graph $E^1$ satisfies $u \overset{1}{\rightarrow} v \in E^1$ when $\| \mb q_u - \mb q_v \|_2 \le R_{\mr{nbrs}}$, and that 
      \begin{equation}\label{eqn:PIII_RNBR_COND}
          40 \delta \le R_{nbrs} \le \frac{\sqrt{2}}{64}\sigma \sqrt{d}.
      \end{equation}
       
      Initializing at $\mb q_{i_{\mr{II}}} \in B_{\mc M}(\mb x_\natural, 0.1/\kappa)$,
       with probability at least 
      \begin{equation}
          1 - e^{-\tfrac{9}{2} d} - e^{-\frac{1}{16}d}, 
      \end{equation} 
      the third phase of first-order optimization in Algorithm \ref{algo:1-0-1} with parameters
      \begin{equation}
          R_a = R_{\mr{nbrs}} - \delta,
      \end{equation}
      \begin{equation}
          \epsilon_2 > { C\max\{\kappa,1\} } \sigma \sqrt{d},
      \end{equation}      
      for some $C > 0$, produces $\mb q_{i_{\mr{III}}}$ which satisfies
    \begin{equation}
        d_{\mc M}(\mb q_{i_{\mr{III}}}, \mb x_\natural) \leq {2\epsilon_2},
    \end{equation}
using at most
\begin{equation}
    \frac{8}{\kappa\delta}
\end{equation}
steps.
\end{proposition}




\begin{proof}
Let 
\begin{equation}
    \mb q_{i_{\mr{II}}} = \mb q^0, \mb q^1, \mb q^2, \dots 
\end{equation}
denote the sequence of landmarks $\mb q^j$ produced by first order optimization in Phase III. Let $j_\star \in \bb Z \cup \{ \infty \}$ denote the number of iterations taken by this phase of optimization. If the algorithm terminates at some finite $j_\star$, we have 
\begin{equation}
    \| P_{T_{\mb q^{j_\star}}} (\mb x - \mb q^{j_\star} ) \| \le \epsilon_2. 
\end{equation}
Below, we will prove that indeed the algorithm terminates, and bound the number of steps $j_\star$ required. To this end, we will prove that on an event of probability at least 
\begin{equation}
    1 - e^{-\tfrac{9}{2} d} - e^{-\tfrac{1}{16} d},
\end{equation}
the iterates $\mb q^j$ ($j = 0,1,\dots, j_\star$) satisfy the following property:
\begin{equation} \label{eqn:decrement-induction}
    d_{\mc M}(\mb q^j, \mb x_\natural) \le \max\left\{ \frac{1}{10\kappa} -  c_1 j \delta, C_1 \max\{ \kappa, 1 \} \sigma \sqrt{d} \right\}.  
\end{equation}
This immediately implies that there exists $C_2$ such that for all 
\begin{equation}
j \ge  \frac{C_2}{\kappa \delta},
\end{equation} 
the iterates  $\mb q^j$ satisfies 
\begin{equation} 
d_{\mc M}( \mb q^j, \mb x_\natural ) \le C_1 \max\{ \kappa, 1 \} \sigma \sqrt{d}. 
\end{equation}

\paragraph{Step Sizes.} Our next task is to verify \eqref{eqn:decrement-induction}. We begin by noting some bounds on the step size $d_{\mc M}(\mb q^{j},\mb q^{j-1})$. By construction, for each $j$, 
\begin{equation}
    \|\mb q^{j+1} - \mb q^j \|_2 
    \leq R_{\mr{nbrs}} \le \frac{\sqrt{2}}{64}\sigma \sqrt{d}
\end{equation}
Since we know $\frac{1}{\kappa} \ge \tau_{\mc M}$, $D \ge d$, and we've assumed $
\tau_{\mc M}\ge 640 \sigma\sqrt{D}$, we have 
\begin{equation}
R_{\mr{nbrs}} \le \min\left\{\frac{\sqrt{2}}{64}\sigma \sqrt{d}, \: \tau_{\mc M}, \: \frac{1}{200\kappa}\right\}.
\end{equation}

Since $\mb q^j, \mb q^{j+1} \in \mc M$ and $\|\mb q^{j+1}-\mb q^j \|_2 \leq \tau_{\mc M}$, applying Lemma \ref{lem:intrinsic dist bounded by extrinsic dist}, we have
\begin{equation}
     d_{\mc M}(\mb q^{j+1}, \mb q^j ) 
     \leq 2 \|\mb q^{j+1} - \mb q^j \|_2 
     \leq \min \left\{ \frac{\sqrt{2}}{32}\sigma \sqrt{d}, \frac{1}{100\kappa} \right\}
\end{equation}
From Lemma \ref{lem:lower bound for Tmax}, take $t=\frac{\sqrt{2}}{4} \sigma \sqrt{d}$, with probability at least $1-e^{-\frac{d}{16}}$, the random variable
\begin{equation}
    T_{\max} = \sup_{\mb y \in B_{\mc M}(\mb x_\natural, 1/\kappa), \mb v \in T_{\mb y } \mc M, \| \mb v \|_2 =1} \innerprod{\mb v}{\mb z},
\end{equation}
satisfies 
\begin{equation}
    T_{\max} \geq \frac{\sqrt{2}}{4} \sigma \sqrt{d},
\end{equation}
which gives us $\frac{\sqrt{2}}{32}\sigma \sqrt{d} \leq \frac{1}{8}T_{\max}$. Hence on an event of probability at least $1-e^{-\frac{d}{16}}$, for every $j$, 
\begin{equation}
    d_{\mc M}(\mb q^{j+1}, \mb q^j ) 
     \leq \min \left\{ \frac{\sqrt{2}}{32}\sigma \sqrt{d}, \frac{1}{100\kappa} \right\}
     \leq \min \left\{\frac{T_{\max}}{8}, \frac{1}{100\kappa}\right\}.
\end{equation}
In particular, this implies that the condition on the step size in Lemma \ref{prop:one step in phase III} is satisfied. 

\paragraph{Radius of Decrease.} From Lemma \ref{lem:upper bound for Tmax}, with probability at least $1 - e^{-\tfrac{9}{2} d}$, $T_{\max}$ satisfies 
\begin{equation}
T_{\max} \le C_4 \max\{ \kappa, 1 \} \sigma \sqrt{d}, 
\end{equation}
and hence the distance condition in Lemma \ref{prop:one step in phase III} is satisfied whenever 
\begin{equation}
    C_5 \max\{ \kappa, 1 \} \sigma \sqrt{d} \le d_{\mc M}(\mb q^j, \mb x_\natural ) \le \frac{1}{10\kappa}. 
\end{equation}

\paragraph{Proof of Equation \eqref{eqn:decrement-induction}.} We proceed by induction on $j$. For $j = 0$, $\mb q^0 = \mb q_{i_{\mr{II}}} \in B_{\mc M}(\mb x_\natural, \tfrac{1}{10\kappa} )$, and so \eqref{eqn:decrement-induction} holds. Now, suppose that \eqref{eqn:decrement-induction} condition holds for iterates $0, 1, \dots, j-1$. By Lemma \ref{prop:one step in phase III}, {\em if} 
\begin{equation}
    C_5 \max\{ \kappa, 1 \} \sigma \sqrt{d} \le d_{\mc M}(\mb q^j, \mb x_\natural ) \le \frac{1}{10\kappa}. 
\end{equation}
then 
\begin{eqnarray}
    d_{\mc M}(\mb q^{j}, \mb x_\natural) &\le& d_{\mc M}(\mb q^{j-1},\mb x_\natural) - c_2 d_{\mc M}(\mb q^j,\mb q^{j-1}) \\ 
    &\le& d_{\mc M}(\mb q^{j-1},\mb x_\natural) - c_2 \delta \\
    &\le& \frac{1}{10\kappa} - c_2 (j-1) \delta - c_2 j \delta \\
    &\le& \frac{1}{10\kappa} - c_2 j \delta
\end{eqnarray}
On the other hand, if $d_{\mc M}(\mb q^{j-1}, \mb x_\natural) < C_5 \max\{ \kappa, 1 \} \sigma \sqrt{d}$, we have 
\begin{eqnarray}
    d_{\mc M}( \mb q^{j}, \mb x_\natural ) &\le&     d_{\mc M}( \mb q^{j-1}, \mb x_\natural ) + d_{\mc M}( \mb q^{j}, \mb q^{j-1} ) \\
    &\le& C_5 \max\{ \kappa, 1 \} \sigma \sqrt{d} + \frac{\sqrt{2}}{32} \sigma \sqrt{d} \\
    &\le& C_6 \max\{ \kappa, 1 \} \sigma \sqrt{d}. 
\end{eqnarray}
Combining, we have 
\begin{equation}
    d_{\mc M}(\mb q^{j}, \mb x_\natural ) \le \max \left\{ \frac{1}{10\kappa} - c_2 j \delta, C_6 \max\{ \kappa, 1 \} \sigma \sqrt{d} \right\},  
\end{equation}
and \eqref{eqn:decrement-induction} is verified. 

\paragraph{Proof of Termination.} We next verify that under our assumptions on $\epsilon_2$, the algorithm terminates. For an arbitrary point $\mb q \in B_{\mc M}( \mb x_\natural, 1/\kappa)$, let $\phi(t)$ denote a geodesic with $\phi(0) = \mb x_\natural$, $\phi(1) =\mb q$ and constant speed $d_{\mc M}(\mb x_\natural, \mb q)$. Then 
\begin{equation}\label{eqn:tangent-upper} 
    \begin{aligned}
        \left\| P_{T_{\mb q} \mc M}(\mb x_\natural - \mb q) \right\|_2
        &= \left\| P_{T_{\mb q}\mc M} \Dot{\mb \phi}(0) + \int_{a=0}^1 \int_{b=0}^a P_{T_{\mb q }\mc M} \Ddot{\phi}(b) db da\right\|_2\\
        &= \left\| \Dot{\mb \phi}(0) + \int_{a=0}^1 \int_{b=0}^a P_{T_{\mb q}\mc M} \Ddot{\phi}(b) db da\right\|_2\\
        &\le \left\| \Dot{\mb \phi}(0) \right\|_2 + \left\| \int_{a=0}^1 \int_{b=0}^a P_{T_{\mb q}\mc M} \Ddot{\phi}(b) db da \right\|_2\\
        &\le d_{\mc M}(\mb x_\natural, \mb q ) +  \int_{a=0}^1 \int_{b=0}^a \left\|P_{T_{\mb q}\mc M} \Ddot{\phi}(b)\right\|_2 db da\\
        &\le d_{\mc M}(\mb x_\natural, \mb q) + \int_{a=0}^1 \int_{b=0}^a \kappa d_{\mc M}^2(\mb x_\natural, \mb q ) db da\\
        &\le d_{\mc M}(\mb x_\natural, \mb q ) + \frac{1}{2} \kappa d_{\mc M}^2(\mb x_\natural, \mb q )\\
        &\le \frac{3}{2} d_{\mc M}(\mb x_\natural, \mb q )
    \end{aligned}
\end{equation}
Comparing this lower bound to \eqref{eqn:decrement-induction}, we obtain that for any $\epsilon_2 > \tfrac{3}{2} C_1 \max\{ \kappa, 1 \} \sigma \sqrt{d}$, after at most 
\begin{equation}
\wh{j} > \frac{1}{10 c_2 \kappa \delta} 
\end{equation}
steps, the algorithm produces a point $\mb q^{j_\star}$ satisfying 
\begin{equation}
    \| P_{T_{\mb q^{j_\star}} \mc M} (\mb x - \mb q^{j_\star} ) \| \le \epsilon_2,
\end{equation}
and hence the algorithm terminates after at most $\frac{1}{10c_2\kappa \delta}$ steps. To be more specific, 
Lemma \ref{prop:one step in phase III} gives $c_2 = \frac{1}{80}$, so the algorithm stops after at most  $\frac{8}{\kappa \delta}$ steps.
\paragraph{Quality of Terminal Point.} Finally, we bound the distance of $\mb q_{i_{\mr{III}}} = \mb q^{j_\star}$ to $\mb x_\natural$. By reasoning analogous to \eqref{eqn:tangent-upper}, we obtain that for $\mb q \in B_{\mc M}(\mb x_\natural, 1/\kappa)$
\begin{equation}
        \left\| P_{T_{\mb q} \mc M}(\mb x_\natural - \mb q) \right\|_2 \ge \frac{1}{2} d_{\mc M}(\mb x_\natural, \mb q ). 
\end{equation}
This immediately implies that 
\begin{equation}
    d_{\mc M}(\mb q^{j_\star}, \mb x_\natural ) \le 2 \eps_2, 
\end{equation}
as claimed. 
\end{proof}



\section{Phase I Analysis}

\begin{lemma}\label{lem:edge bounded by tangent term}
With the assumption $R_{\text{nbrs}} \leq \tau_{\mc M}$ , for any $\mb q_u, \mb q_v \in \mc M$ and $u\overset{1}{\rightarrow} v \in E^1$, we have
    \begin{equation}
        \| \mb q_u - \mb q_v \|_2^2
        \leq \frac{4}{3} \| \mc P_{T_u} \paren{\mb q_u - \mb q_v} \|_2^2
    \end{equation}
where $T_u$ is the tangent space at $\mb q_u \in \mc M$, and $\mc P_{T_u}$ is the orthogonal projection onto  $T_u$.
\end{lemma}
\begin{proof}
    For the self edge $u\overset{1}{\rightarrow} u \in E^1$, it's easy to show that $ \| \mb q_u - \mb q_u \|_2^2 = \frac{4}{3} \| \mc P_{T_u} \paren{\mb q_u - \mb q_u} \|_2^2 =0$.\\
    Now we consider the situation when $u\overset{1}{\rightarrow} v \in E^1 (u \neq v)$.  By the theorem 2.2 in \cite{aamari2019estimating} and theorem 4.18 in \cite{federer1959curvature}, we know the reach $\tau_{\mc M}$ of the manifold $\mc M$ satisfies
    \begin{equation}
    \tau_{\mc M} = \inf_{\mb q_u \neq \mb q_v \in \mc M}
    \frac{\| \mb q_u - \mb q_v \|_2^2}{2 \| \mc P_{N_u } \paren{\mb q_u - \mb q_v}\|_2}
\end{equation}
where $N_u$ is the normal space at $\mb q_u \in \mc M$, and $\mc P_{N_u}$ is the orthogonal projection onto the $N_u$.\\
Thus, for any $u\overset{1}{\rightarrow} v \in E^1 (u \neq v)$, we have
\begin{equation}
    \tau_{\mc M}
    \leq \frac{\| \mb q_u - \mb q_v \|_2^2}{2 \mc \| \mc P_{N_u} \paren{\mb q_u - \mb q_v} \|_2}
\end{equation}
By the rule of connecting first-order edges, for any edge $u\overset{1}{\rightarrow} v \in E^1$, we have $\| \mb q_u - \mb q_v \|_2 \leq R_{\text{nbrs}}$.\\
By assumption, $R_{\text{nbrs}} \leq \tau_{\mc M}$, thus for first-order edges $u\overset{1}{\rightarrow} v (u \neq v)$, we have:
\begin{equation}
    \begin{aligned}
    \| \mb q_u - \mb q_v \|
    &\leq R_{\text{nbrs}}\\
    &\leq \tau_{\mc M}\\
    &\leq \frac{\| \mb q_u - \mb q_v \|_2^2}{2 \mc \| \mc P_{N_u} \paren{\mb q_u - \mb q_v} \|_2}
    \end{aligned}
\end{equation}
Square both sides, we obtain:
\begin{equation}
    \begin{aligned}
        \| \mb q_u - \mb q_v \|_2^2
        \geq 4 \| \mc P_{N_u} \paren{\mb q_u - \mb q_v} \|_2^2
    \end{aligned}
\end{equation}
Since the tangent and normal components are orthogonal, we have $ \| \mb q_u - \mb q_v \|_2^2 = \| \mc P_{T_u} \paren{\mb q_u - \mb q_v} \|_2^2 + \| \mc P_{N_u} \paren{\mb q_u - \mb q_v} \|_2^2$. Therefore, we have:
\begin{equation}
     \| P_{T_u} \paren{\mb q_u - \mb q_v} \|_2^2
     \geq 3  \| P_{N_u} \paren{\mb q_u - \mb q_v} \|_2^2
\end{equation}
Consequently, we have
\begin{equation}
    \begin{aligned}
    \| \mb q_u - \mb q_v \|_2^2
    &= \| \mc P_{T_u} \paren{\mb q_u - \mb q_v} \|_2^2 + \| \mc P_{N_u} \paren{\mb q_u - \mb q_v} \|_2^2\\
    &\leq \| \mc P_{T_u} \paren{\mb q_u - \mb q_v} \|_2^2 + \frac{1}{3}\| \mc P_{T_u} \paren{\mb q_u - \mb q_v} \|_2^2\\
    &= \frac{4}{3} \| \mc P_{T_u} \paren{\mb q_u - \mb q_v} \|_2^2
    \end{aligned}
\end{equation}
This completes the proof.
\end{proof}


\begin{lemma}\label{lem:small Euclidean ball covering} For any $\mb w_\natural \in B(\mb q_u, R_a) \cap \mc M$, there exists $v: u \overset{1}{\rightarrow} v \in E^1$, such that 
    \begin{equation}
        \| \mb w_\natural - \mb q_v \|_2 \leq \delta
    \end{equation}
\end{lemma}
\begin{proof}
    Since the landmarks form a $\delta$- cover of $\mc M$, and $R_{\text{nbrs}} = a \cdot \delta (a \geq 40)$, there exists $v \in V$ such that $\| \mb q_v - \mb w_\natural \|_2 \leq \delta = \frac{1}{a} R_{\text{nbrs}}$.  \\
    From the triangle inequality, we have
    \begin{equation}
        \begin{aligned}
            \| \mb q_v - \mb q_u \|_2
            &\leq \| \mb q_v - \mb w_\natural \|_2 + \| \mb w_\natural - \mb q_u \|_2\\
            &\leq \frac{1}{a} R_{\text{nbrs}} + R_a\\
            &= \frac{1}{a} R_{\text{nbrs}} + \frac{a-1}{a} R_{\text{nbrs}}\\
            &= R_{\text{nbrs}}
        \end{aligned}
    \end{equation}
From the rule of connecting first-order edges, we know $u \overset{1}{\rightarrow} v \in E^1$. This completes the proof.
\end{proof}

\begin{lemma}\label{lem:tangent space covering}
 For any $\mb w \in T_u$, with $\| \mb w \|_2 \leq R_a$, there exists $v: u \overset{1}{\rightarrow} v \in E^1$, such that 
 \begin{equation}
     \| \mb w - \mc P_{T_u} \paren{\mb q_v - \mb q_u}\|_2
     \leq \delta + \frac{1}{2} \kappa R_a^2
 \end{equation}
\end{lemma}

\begin{proof}
    Consider a constant-speed geodesic $\mb \gamma: [0,1] \rightarrow \mc M$, $\mb \gamma(0)=\mb q_u, \dot{\mb \gamma}(0) =\mb w$ with $\| \mb w \|_2 \leq R_a$. Let $\mb \gamma(1)= \mb w_\natural$.
From the fundemental theorem of calculus, we have
\begin{equation}
    \begin{aligned}
        \mb \gamma(1) = \mb \gamma(0) + \int_{t=0}^1 \dot{\mb \gamma}(t) dt
    \end{aligned}
\end{equation}
Therefore,
\begin{equation}
    \begin{aligned}
       \left\| \mb w_\natural - \mb q_u \right\|_2 
        &= \left\| \int_{t=0}^1 \dot{\mb \gamma}(t) dt \right\|_2\\
        &\leq \int_{t=0}^1 \left\| \dot{\mb \gamma}(t)\right\| dt\\
        &= \| \mb w \|_2\\
        &\leq R_a
    \end{aligned}
\end{equation}
This means $\mb w_\natural \in B(\mb q_u, R_a)$. By lemma \ref{lem:small Euclidean ball covering}, there exists $v: u \overset{1}{\rightarrow} v \in E^1$, such that $\| \mb w_\natural - \mb q_v \|_2 \leq \delta$.\\
By fundamental theorem of calculus, we also know
\begin{equation}\label{eq:geodesic taylor expansion}
    \begin{aligned}
    \mb \gamma(1) 
    &= \mb \gamma(0) + \int_{t=0}^1 \dot{\mb \gamma}(t) dt\\
    &= \mb \gamma(0) + \int_{t=0}^1 \paren{ \dot{\mb \gamma}(0) + \int_{s=0}^t \ddot{\mb \gamma}(s) ds } dt\\
    &= \mb \gamma(0) + \dot{\mb \gamma}(0) + \int_{t=0}^1 \int_{s=0}^t \ddot{\mb\gamma}(s) ds dt
    \end{aligned}
\end{equation}
Then we have
\begin{equation}
    \begin{aligned}
        \mb w = \mb w_\natural - \mb q_u - \int_{t=0}^1 \int_{s=0}^t \ddot{\mb\gamma}(s) ds dt
    \end{aligned}
\end{equation}
Since $\mb w \in T_u$, we have
\begin{equation}
    \begin{aligned}
        \mb w = \mc P_{T_u} (\mb w)
        = \mc P_{T_u} \paren{\mb w_\natural - \mb q_u} - \mc P_{T_u} \paren{\int_{t=0}^1 \int_{s=0}^t \ddot{\mb\gamma}(s) ds dt}
    \end{aligned}
\end{equation}
Hence we know
\begin{equation}
    \begin{aligned}
    \left\| \mb w - \mc P_{T_u} \paren{\mb q_v - \mb q_u}\right\|_2
        &= \left\| \mc P_{T_u}\paren{\mb w_\natural - \mb q_u} - \mc P_{T_u} \paren{\int_{t=0}^1 \int_{s=0}^t \ddot{\mb\gamma}(s) ds dt} - \mc P_{T_u} \paren{\mb q_v - \mb q_u} \right\|_2\\
        &= \left\| \mc P_{T_u} \paren{\mb w_\natural - \mb q_v} - \mc P_{T_u} \paren{\int_{t=0}^1 \int_{s=0}^t \ddot{\mb\gamma}(s) ds dt}\right\|_2\\
        &\leq \left\| \mb w_\natural - \mb q_v \right\|_2 + \left\| \int_{t=0}^1 \int_{s=0}^t \ddot{\mb\gamma}(s) ds dt\right\|_2\\
        &\leq \delta + \int_{t=0}^1 \int_{s=0}^t \| \ddot{\mb\gamma}(s)\|_2 ds dt\\
        &\leq \delta + \frac{1}{2} \kappa \|\mb w\|_2^2\\
        &\leq \delta + \frac{1}{2} \kappa R_a^2
    \end{aligned}
\end{equation}
\end{proof}

\begin{lemma}\label{lem:bound dot product of gradient and edge in tangent direction}
    For any $\mb q_u\in \mc M$ and $\vx \ne \vq_u$, let $v: u\overset{1}{\rightarrow} v \in E^1$ satisfies:
    \begin{align}\label{eq:app:bound_dot_prod_grad_edge_tangent_dir_condition}
        \norm*{ \frac{\mc P_{T_u} \paren{\mb x - \mb q_u}}{\| \mc P_{T_u} \paren{\mb x - \mb q_u}\|_2} R_a - \mc P_{T_u} \paren{\mb q_v - \mb q_u}}_2 \le \delta + \frac{1}{2}\kappa R_a^2.
    \end{align}
    Then we have
    \begin{equation}
         \innerprod{\mc P_{T_u} \paren{\mb x - \mb q_u}}{\mc P_{T_u}\paren{\mb q_v - \mb q_u}}
        \ge 0.55
        \| \mc P_{T_u} \paren{\mb x - \mb q_u} \|_2   \| \mc P_{T_u}\paren{\mb q_v - \mb q_u} \|_2. 
    \end{equation}
\end{lemma}


\begin{proof}
Let $ \delta_T = \delta + \frac{1}{2} \kappa R_a^2$. Given our global constraints that $\delta = \frac{1}{a} R_{\text{nbrs}}$, $a\ge 40$ and $R_{\mr{nbrs}} \le \tau_{\mc M} \le 1/\kappa$,
\begin{equation}\label{eq:bound on delta_T over R_a}
    \begin{aligned}
        \frac{\delta_T}{R_a} 
        &= \frac{\frac{1}{a}R_{\text{nbrs}} + \frac{1}{2}\kappa (1-\frac{1}{a})^2 R_{\text{nbrs}}^2}{(1-\frac{1}{a})R_{\text{nbrs}}}\\
        &=\frac{1}{a(1-\frac{1}{a})} + \frac{1}{2} \kappa (1-\frac{1}{a}) R_{\text{nbrs}}\\
        &\leq \frac{1}{a-1} + \frac{1}{2} (1-\frac{1}{a})\\
        & \le \frac{5}{6} \\ 
    \end{aligned}
\end{equation}

From the assumption we have
 \begin{equation}\label{eq:tangent space covering assumption in tangent term}
     \norm*{ \frac{\mc P_{T_u} \paren{\mb x - \mb q_u}}{\| \mc P_{T_u} \paren{\mb x - \mb q_u}\|_2} R_a - \mc P_{T_u} \paren{\mb q_v - \mb q_u}}_2
     \leq \delta_T
 \end{equation}

Square both sides of the equation \eqref{eq:tangent space covering assumption in tangent term}, we have
\begin{equation}
    R_a^2 + \| \mc P_{T_u} \paren{\mb q_v - \mb q_u }\|_2^2 - 2 \innerprod{\frac{\mc P_{T_u} \paren{\mb x - \mb q_u}}{\| \mc P_{T_u} \paren{\mb x - \mb q_u}\|_2} R_a}{\mc P_{T_u} \paren{\mb q_v - \mb q_u}} \leq \delta_T^2
\end{equation}
Then we have
\begin{equation}
    \begin{aligned}
        \innerprod{\frac{\mc P_{T_u} \paren{\mb x - \mb q_u}}{\| \mc P_{T_u} \paren{\mb x - \mb q_u}\|_2} R_a}{\mc P_{T_u} \paren{\mb q_v - \mb q_u}}
        &\geq \frac{R_a^2 - \delta_T^2 + \| \mc P_{T_u} \paren{\mb q_v - \mb q_u }\|_2^2}{2}\\
        &\geq \sqrt{R_a^2 - \delta_T^2} \| \mc P_{T_u} \paren{\mb q_v - \mb q_u }\|_2
    \end{aligned}
\end{equation}
Therefore, we have
\begin{equation}
\label{eq:dot product on the tangent space large compared to the norm}
    \begin{aligned}
         \innerprod{\mc P_{T_u} \paren{\mb x - \mb q_u}}{\mc P_{T_u} \paren{\mb q_v - \mb q_u}} 
        \geq \sqrt{1 - \frac{\delta_T^2}{R_a^2}}
        \| \mc P_{T_u} \paren{\mb x - \mb q_u }\|_2
        \| \mc P_{T_u} \paren{\mb q_v - \mb q_u }\|_2
    \end{aligned}
\end{equation}
Substituting our result from eq(\ref{eq:bound on delta_T over R_a} to) eq \eqref{eq:dot product on the tangent space large compared to the norm}, we have the following inequality:
\begin{equation}
     \innerprod{\mc P_{T_u} \paren{\mb x - \mb q_u}}{\mc P_{T_u} \paren{\mb q_v - \mb q_u}} 
    \geq 0.55
    \| \mc P_{T_u} \paren{\mb x - \mb q_u }\|_2
    \| \mc P_{T_u} \paren{\mb q_v - \mb q_u }\|_2
\end{equation}
\end{proof}

\begin{lemma}\label{lem:intrinsic dist bounded by extrinsic dist}
     If $\mb q_u, \mb q_v \in \mc M$ and $\| \mb q_u - \mb q_v \|_2 \leq \tau_{\mc M}$, we have
    \begin{equation}
        d_{\mc M}(\mb q_u, \mb q_v) \leq 2 \| \mb q_u - \mb q_v \|_2
    \end{equation}
where $d_{\mc M}(\cdot,\cdot)$ is the intrinsic distance along the manifold.
\end{lemma}
\begin{proof}
    Let $\mb q_t = \mb q_u + t (\mb q_v - \mb q_u), t\in[0,1]$. From Theorem C in \cite{leobacher2021existence}, we know 
    \begin{equation}
        \frac{d}{dt} \mc P_{\mc M} [\mb q_t]=
        \paren{\mb I - \sff^*[\mb q_t - \mc P_{\mc M}[\mb q_t]]}^{-1}  \mc P_{T_{\mc P_{\mc M} [\mb q_t]} \mc M} (\mb q_v - \mb q_u) 
    \end{equation}
Therefore, we have
\begin{equation}
    \begin{aligned}
        \| \frac{d}{dt} \mc P_{\mc M} [\mb q_t] \|_2
        &\leq \| \paren{\mb I - \sff^*[\mb q_t - \mc P_{\mc M}[\mb q_t]]}^{-1} \|_{\text{op}} \| \mc P_{T_{\mb q_t} \mc M} (\mb q_v - \mb q_u)  \|_2\\
        &\leq \frac{\| \mb q_v - \mb q_u \|_2}{1-\kappa \|\mb q_t - \mc P_{\mc M}[\mb q_t] \|_2}\\
        &= \frac{\| \mb q_v - \mb q_u \|_2}{1-\kappa d(\mb q_t, \mc M)}\\
        &\leq \frac{\| \mb q_v - \mb q_u \|_2}{1-\kappa \min\{\| \mb q_t - \mb q_u\|_2, \| \mb q_t - \mb q_v \|_2\}}\\
        &\leq \frac{\| \mb q_v - \mb q_u \|_2}{1-\frac{1}{2} \kappa \| \mb q_v - \mb q_u \|_2}\\
        & \leq \frac{\| \mb q_v - \mb q_u \|_2}{1-\frac{1}{2} \kappa \tau_{\mc M}}\\
        &\leq 2 \| \mb q_v - \mb q_u\|_2
    \end{aligned}
\end{equation}
where in the second line we've used the property that $\forall \mb \eta$, $\| \sff^*[\mb \eta] \| \le \kappa \| \mb \eta \|_2$, and in the last line we've applied the fact that $\tau_{\mc M} \le 1/ \kappa$. 
Then the intrinsic distance along the manifold is 
\begin{equation}
    \begin{aligned}
        d_{\mc M}(\mb q_u, \mb q_v) 
        &\leq \int_{t=0}^1 \| \frac{d}{dt} \mc P_{\mc M} [\mb q_t] \|_2 dt\\
        &\leq \int_{t=0}^1 2\| \mb q_v - \mb q_u \|_2 dt\\
        &= 2\| \mb q_v - \mb q_u \|_2
    \end{aligned}
\end{equation}
This completes the proof.
\end{proof}






\begin{lemma}\label{lem:bound norm edge term}
    For any $\mb q_u, \mb q_v \in \mc M$, we have
    \begin{equation}
        \| \mc P_{N_u} \paren{\mb q_v - \mb q_u} \|_2
        \leq \frac{1}{2} \kappa d_{\mc M}^2(\mb q_u, \mb q_v)
    \end{equation}
\end{lemma}
\begin{proof}
    Consider the geodesic $\mb \gamma:[0,1] \rightarrow M \subset \mathbb{R}^D$ with constant speed $\| \mb v\|_2$, where $\mb \gamma \paren{0} = \mb q_u, \mb \gamma \paren{1} = \mb q_v$.
    Then the intrinsic distance between the $\mb q_u$ and $\mb q_v$ is:
    \begin{equation}
            d_{\mc M}(\mb q_u, \mb q_v)
            = \int_{t=0}^1 \| \mb v\|_2 dt = \| \mb v\|_2
    \end{equation}
From equation \eqref{eq:geodesic taylor expansion} and the fact $\dot{\mb\gamma}(0) \in T_{u}$, we have
\begin{equation}
    \| \mc P_{N_u }(\mb\gamma(1) - \mb\gamma(0))\|_2
    = \norm*{ \int_{t=0}^1 \int_{s=0}^t \ddot{\mb\gamma}(s) ds dt }_2
\end{equation}


\begin{equation}
    \begin{aligned}
        \| \mc P_{N_u} \paren{\mb q_v - \mb q_u} \|_2
        &= \norm*{ \mc P_{N_u} \paren{\mb \gamma \paren{1} - \mb \gamma \paren{0}} }_2\\
        &= \norm*{  \mc P_{N_u} \int_{t=0}^1 \int_{s=0}^t \Ddot{\mb \gamma} \paren{s}  ds dt }_2\\
        & \leq \norm*{ \int_{t=0}^1 \int_{s=0}^t \Ddot{\mb \gamma} \paren{s}  ds dt }_2\\
        & \leq \int_{t=0}^1 \int_{s=0}^t \| \Ddot{\mb \gamma} \paren{s} \|_2  ds dt\\
        & \leq \int_{t=0}^1 \int_{s=0}^t \kappa \| \vv\|_2 ds dt\\
        &= \frac{1}{2} \kappa \|\mb v\|_2^2\\
        &= \frac{1}{2} \kappa d^2_{\mc M}(\mb q_u, \mb q_v)
    \end{aligned}
\end{equation}
This completes the proof.
\end{proof}

\begin{lemma} \label{lem:bound normal edge by tangent edge square}
    If $\mb q_u, \mb q_v \in \mc M$ and $\| \mb q_u - \mb q_v \|_2 \leq \tau_{\mc M}$, we have
    \begin{equation}
        \| \mc P_{N_u} \paren{\mb q_v - \mb q_u} \|_2 \leq \frac{8}{3} \kappa \| \mc P_{T_u} \paren{\mb q_v - \mb q_u} \|_2^2
    \end{equation}
\end{lemma}
\begin{proof}
    Given $\mb q_u, \mb q_v \in \mc M$ and $\| \mb q_u - \mb q_v \|_2 \leq \tau_{\mc M}$, from lemma \ref{lem:intrinsic dist bounded by extrinsic dist}, we know $d_{\mc M}(\mb q_u, \mb q_v) \leq 2 \| \mb q_u - \mb q_v \|_2$.
    From lemma \ref{lem:bound norm edge term}, we have
    $\| \mc P_{N_u} \paren{\mb q_v - \mb q_u} \|_2 \leq \frac{1}{2} \kappa d_{\mc M}^2(\mb q_u, \mb q_v)$. By lemma \ref{lem:edge bounded by tangent term}, we have  $ \| \mb q_u - \mb q_v \|_2^2 \leq \frac{4}{3} \| \mc P_{T_u} \paren{\mb q_u - \mb q_v} \|_2^2$. Then we have 

    \begin{equation}
        \begin{aligned}
        \| \mc P_{N_u} \paren{\mb q_v - \mb q_u} \|_2
        &\leq \frac{1}{2} \kappa d_{\mc M}^2(\mb q_u, \mb q_v)\\
        &\leq 2 \kappa \| \mb q_u - \mb q_v \|_2^2\\
        &= \frac{8}{3} \kappa \| \mc P_{T_u} \paren{\mb q_u - \mb q_v} \|_2^2
        \end{aligned}
    \end{equation}
This completes the proof.
\end{proof}

\begin{lemma}\label{lem:bound dot product of gradient and edge in the normal direction}
    For any $t > 0$, with probability at least $1-e^{-\frac{t^2}{2\sigma^2}}$ in the noise, for any $\mb q_u, \mb q_v \in \mc M$ and $u\overset{1}{\rightarrow} v \in E^1$,
    \begin{equation}\label{eq:high prob bound for normal term}
         - \innerprod{\mc P_{N_u} \paren{\mb x - \mb q_u}}{\mc P_{N_u} \paren{\mb q_v - \mb q_u}} 
         \leq \frac{8}{3} \kappa (\diam(\mc M) + \sqrt{2} \sigma \sqrt{\log |E^1|} + t) \| \mc P_{T_u}(\mb q_v - \mb q_u) \|_2^2
    \end{equation}
\end{lemma}
\begin{proof}
As $u \firstE v \in E^1$, we have $\norm{\vq_u - \vq_v} \le \firstER \le \tau_{\sM}$ from the construction of $E^1$. We decompose the left hand side in equation \eqref{eq:high prob bound for normal term} as follows: 

\begin{align}
- \innerprod{\mc P_{N_u} \paren{\mb x - \mb q_u}}{\mc P_{N_u} \paren{\mb q_v - \mb q_u}} 
= - \innerprod{\mc P_{N_u} \paren{\mb x_\natural - \mb q_u}}{\mc P_{N_u} \paren{\mb q_v - \mb q_u}} - \innerprod{\mb z}{\mc P_{N_u} \paren{\mb q_v - \mb q_u}}. \label{eq:dot_prod_of_gd_and_edge_normal_dir_decompose}
\end{align}

From lemma \ref{lem:bound normal edge by tangent edge square}, we have $\| \mc P_{N_u} \paren{\mb q_v - \mb q_u} \|_2 \leq \frac{8}{3} \kappa \| \mc P_{T_u} \paren{\mb q_v - \mb q_u} \|_2^2$.
Then the first component on the right hand side can be rewritten as:
\begin{equation}\label{eq:bound signal component in normal term}
    \begin{aligned}
         - \innerprod{\mc P_{N_u} \paren{\mb x_\natural - \mb q_u}}{\mc P_{N_u} \paren{\mb q_v - \mb q_u}}
         &\leq \| \mb x_\natural - \mb q_u \|_2 \| \mc P_{N_u} \paren{\mb q_v - \mb q_u} \|_2\\
         &\leq \frac{8}{3} \kappa \diam \mc M \| \mc P_{T_u} \paren{\mb q_v - \mb q_u} \|_2^2
    \end{aligned}
\end{equation}
Now consider a set of random variables 
\begin{equation}
    X_{\mb y} = - \innerprod{\mb z}{\mb y} \quad \mb y \in \mc Y
\end{equation}
with \[\mc Y = \Biggl\{ \frac{\mc P_{N_{u}} (\mb q_v - \mb q_u)}{\| \mc P_{N_{u}} (\mb q_v - \mb q_u) \|_2} \Big\vert \ \forall u \overset{1}{\rightarrow} v \in E^1, u \ne v,  \forall u, v \in \mc V \Biggl\}.\] 
Because $\mb z \sim \mathcal{N}(\mb 0, \sigma^2 \mb I)$, $X_{\mb y}$ follows the distribution $X_{\mb y} \sim \mathcal{N}(0, \sigma^2)$, and $(X_{\mb y})_{\mb y \in \mc Y}$ is a Gaussian process. We first bound the expectation of the supremum of this Gaussian process $ \mathbb E [\sup_{\mb y \in \mc Y} X_{\mb y}]$ by starting from
\begin{equation}
    \begin{aligned}
        e^{t' \mathbb E[\sup_{\mb y \in \mc Y} X_{\mb y}]}
        &= e^{\mathbb E[t'\sup_{\mb y \in \mc Y} X_{\mb y}]}\\
        &\leq \mathbb E[e^{t' \sup_{\mb y \in \mc Y} X_{\mb y}}] \\
        &\leq \sum_{i=1}^{|\mc Y|} \mathbb E[e^{t' X_{\mb y}}] \\
        &= |Y| e^{\frac{1}{2}\sigma^2 t'^2}, \quad \forall t' > 0. 
    \end{aligned}
\end{equation}
Take logarithm of both sides, we have
\begin{equation}\label{eq:take log}
    \E{\sup_{\mb y \in \mc Y} X_{\mb y}}
    \leq \frac{\log|\mc Y|}{t} + \frac{ \sigma^2 t'}{2}
\end{equation}
When $t' = \frac{\sqrt{2 \log |Y|}}{\sigma}$, the right hand side of equation \eqref{eq:take log} is minimized, and we obtain
\begin{equation}
     \mathbb E[\sup_{\mb y \in \mc Y} X_{\mb y}]
     \leq \sqrt{2} \sigma \sqrt{\log |Y|}
\end{equation}
By using Borell–TIS inequality \cite{adler2007gaussian}, for any $t > 0$ we have
\begin{equation}
    \mathbb P(\sup_{\mb y \in \mc Y} X_{\mb y} - \mathbb E[\sup_{\mb y \in \mc Y} X_{\mb y}] > t) 
    \leq e^{\frac{-t^2}{2\sigma^2}}
\end{equation}
This implies 
\begin{equation}
    \mathbb P \left( -\innerprod{\mb z}{\frac{\mc P_{N_u}(\mb q_v - \mb q_u)}{\| \mc P_{N_u}(\mb q_v - \mb q_u) \|_2}} -  \mathbb E[\sup_{\mb y \in \mc Y} X_{\mb y}] \leq t \right) 
    > 1-  e^{\frac{-t^2}{2\sigma^2}}
\end{equation}
Therefore, with probability at least $1-e^{-\frac{t^2}{2\sigma^2}}$, we have
\begin{equation}\label{eq:bound noise component in normal term}
    \begin{aligned}
     -\innerprod{\mb z}{\mc P_{N_u}(\mb q_v - \mb q_u)}
    &\leq \| \mc P_{N_u}(\mb q_v - \mb q_u) \|_2 (\mathbb E[\sup_{\mb y \in \mc Y} X_{\mb y}] +  t) \\
    &\leq  \| \mc P_{N_u}(\mb q_v - \mb q_u) \|_2 (\sqrt{2} \sigma \sqrt{\log |Y|} + t)\\
    &\leq \frac{8}{3} \kappa \| \mc P_{T_u}(\mb q_v - \mb q_u) \|_2^2 (\sqrt{2} \sigma \sqrt{\log |Y|} + t) \quad \text{by lemma \ref{lem:bound normal edge by tangent edge square}}
    \end{aligned}
\end{equation}
Plug equation \eqref{eq:bound signal component in normal term} and equation \eqref{eq:bound noise component in normal term}, with probability at least $1-e^{-\frac{t^2}{2\sigma^2}}$, we have
\begin{equation}
    \begin{aligned}
        &- \innerprod{\mc P_{N_u} \paren{\mb x_\natural - \mb q_u}}{\mc P_{N_u} \paren{\mb q_v - \mb q_u}} - \innerprod{\mb z}{\mc P_{N_u} \paren{\mb q_v - \mb q_u}}
        \\
        &\qquad \qquad \leq - \innerprod{\mc P_{N_u} \paren{\mb x_\natural - \mb q_u}}{\mc P_{N_u} \paren{\mb q_v - \mb q_u}} + \sup_{u\overset{1}{\rightarrow}v} -\innerprod{\mb z}{\mc P_{N_u}(\mb q_v - \mb q_u)}\\
        &\qquad \qquad \leq \frac{8}{3} \kappa (\text{diam}(\mc M) + \sqrt{2} \sigma \sqrt{\log |Y|} + t) \| \mc P_{T_u}(\mb q_v - \mb q_u) \|_2^2
    \end{aligned}
\end{equation}
We conclude the proof by noting that $\abs{Y} < \abs{E^1}$. 
\end{proof}










\section{Phase II Analysis} 

{
\begin{lemma}\label{Phase II Probability bound on H}
Given noise vector $\mb z \simiid \mc N(0,\sigma^2)$, let $H = \max_{\mb q \in Q}\left\|P_{T_{\mb q}\mc M} \mb z \right\|$. 
Then, with probability at least $1 - e^{-\frac{9d}{2}}$, we have
\begin{equation}
 H \le \sigma\left(4\sqrt{\log|Q|}+10\sqrt{d}\right).
\end{equation}
\end{lemma}
\begin{proof}
For any landmark $\mb q$, we define set $S_{\mb q} = \left\{\mb v \in T_qM: \|\mb v\| = 1\right\}$. From Corrollary 4.2.13 in \cite{vershynin2018high}, we know that for any $\delta$, $S_q$ can be $\delta$-covered by at most $\left(\frac{1+\delta/2}{\delta/2}\right)^d = \left(1+\frac{2}{\delta}\right)^d$ points. In particular, we can choose $\delta =\frac{1}{2}$ and let $C_{\mb q}$ to be a 1/2 cover of $S_{\mb q}$ such that $|C_q| \le 5^d$. We let $S_Q = \cup_{\mb q\in Q}S_q$ and $C_Q = \cup_{\mb q\in Q}C_q$. Then, we have
\begin{align}
H &= \max_{\mb u \in S_Q} \innerprod{\mb u}{\mb z} \\
  &\le \max_{\mb v \in C_Q} \innerprod{\mb v}{\mb z} + \frac{1}{2}H, \\
\end{align}
this implies 
\begin{equation}\label{upper bound H by covering}
H \le 2 \max_{\mb v \in C_Q} \innerprod{\mb v}{\mb z}.
\end{equation}
Let $h(\mb z) = \max_{\mb v \in C_Q} \innerprod{\mb v}{\mb z}$, and since it's a 1-lipshitz function in $\mb z$, by applying gaussian concentration inequality, we have 
\begin{equation}\label{eq:Gaussian Concentration Bounds for Liptshiz function}
    \mathbb P\left(h(\mb z) - \mathbb E[h(\mb z)] \geq s \right) \leq e^{-\frac{s^2}{2\sigma^2}}.
\end{equation}
We  then bound its expectation. For any $t>0$,
\begin{equation}
    \begin{aligned}
        e^{t \mathbb E[h(\mb z)]}
        &\le \mathbb E[e^{t [h(\mb z)]}]\\
        &= \mathbb E[e^{t\max_{\mb v \in C_Q} \innerprod{\mb v}{\mb z}}] \\
        &\leq \sum_{\mb v \in C_Q} \mathbb E[e^{t\innerprod{\mb v}{\mb z}}] \\
        &= |C_Q| e^{\frac{1}{2}\sigma^2 t^2}.
    \end{aligned}
\end{equation}
Take logarithm of both sides, we have
\begin{equation}\label{eq:take log again}
    \E{h(\mb z)}
    \leq \frac{\log|C_Q|}{t} + \frac{ \sigma^2 t}{2}.
\end{equation}
When $t = \frac{\sqrt{2 \log |C_Q|}}{\sigma}$, the right hand side of equation \eqref{eq:take log again} is minimized, and we obtain
\begin{equation}
    \E{h(\mb z)}
     \leq \sqrt{2} \sigma \sqrt{\log |C_Q|} 
     = \sqrt{2} \sigma \sqrt{d\log5 +\log |Q|}.
\end{equation}
Plugging this result back to \ref{eq:Gaussian Concentration Bounds for Liptshiz function}, we have 
\begin{equation}
    \mathbb P\left(h(\mb z) - \sqrt{2} \sigma \sqrt{d\log5 +\log |Q|}  \geq s \right) \leq e^{-\frac{s^2}{2\sigma^2}}.
\end{equation}
Picking $s= 3\sigma\sqrt{d}$, this implies with probability at least $1 - e^{-\frac{9d}{2}}$, we have
\begin{equation}
    \max_{\mb v \in C_Q} \innerprod{\mb v}{\mb z} = h(\mb z) \le \sigma\left(\sqrt{(2\log5)d +2\log |Q|}+3\sqrt{d}\right) \le \sigma\left(\sqrt{2\log|Q|}+5\sqrt{d}\right).
\end{equation}
Plugging this result back to \ref{upper bound H by covering}, we have with probability $\ge 1 - e^{-\frac{9d}{2}}$,
\begin{equation}
H \le  \sigma\left(2\sqrt{2\log|Q|}+10\sqrt{d}\right).
\end{equation}
\end{proof}
}

{

\begin{lemma}\label{lem:existence of good ZOE}
    Suppose for any landmark $\mb q'' \in \mb q + N_{\mb q}^{\eta}$ \footnote{$N^{\eta}_{\mb q} \mc M = \set{ \mb v \in \bb R^D \mid \| \mc P_{T_{\mb q}\mc M} \mb v \|_2 \le \eta}$},  with 
    \begin{equation}
    \eta \geq  \epsilon + \delta + \sigma\left(4\sqrt{\log|Q|}+10\sqrt{d}\right),     
    \end{equation}
    there exists a zero-order edge $\mb q \overset{0}{\rightarrow} \mb q'$, such that
    \begin{equation}
        \mb q' \in B_{\mc M}(\mb q'', c_4 \tau_{\mc M}),
    \end{equation}
    where $\delta$ is the covering radius for $Q$, $\epsilon> 0$, 
    and $c_4 \le \tfrac{1}{40}$. \\
    Then, for $\mb x = \mb x_\natural + \mb z$, with probability at least $1 - e^{-\frac{9d}{2}}$ in the noise $\mb z$, the following property obtains: for every point $\mb q$ such that $\mb x \in \mb q + N^{\epsilon}_{\mb q}$, there exists a zero-order edge $\mb q \overset{0}{\rightarrow} \mb q'''$ for some $\mb q''' \in B_{\mc M}(\mb x_\natural, c_4 \tau_{\mc M} + \delta)$.
\end{lemma}




\begin{proof}
    For $\mb x = \mb x_\natural + \mb z \in \mb q+N_{\mb q}^{\epsilon} \mc M$, we have
    \begin{equation}
        \norm*{P_{T_{\mb q}}(\mb x - \mb q)}_2 \leq \epsilon.
    \end{equation}
By the covering property, there exists $\mb q'' \in Q$, such that
\begin{equation}
    d_{\mc M}(\mb x_\natural, \mb q'') \leq \delta. 
\end{equation}
Since
    \begin{equation}
        \norm*{P_{T_{\mb q}}(\mb x - \mb q)}_2 
        \geq \norm*{P_{T_{\mb q}}(\mb q'' - \mb q)}_2 - \norm*{P_{T_{\mb q}}(\mb x_\natural - \mb q'')}_2 - \norm*{P_{T_{\mb q}}\mb z}_2,
    \end{equation}
 we have
\begin{equation}
    \norm*{P_{T_{\mb q}}(\mb q'' - \mb q)}_2 - \norm*{P_{T_{\mb q}}(\mb x_\natural - \mb q'')}_2 - \norm*{P_{T_{\mb q}}\mb z}_2
    \leq \epsilon.
\end{equation}
Together with $\norm*{P_{T_{\mb q}}(\mb x_\natural - \mb q'')}_2 \leq \norm*{\mb x_\natural - \mb q''}_2 \leq \delta$, and \Cref{Phase II Probability bound on H}, then with probability at least $1-e^{\frac{-9d}{2}}$, we have
\begin{equation}
    \begin{aligned}
    \norm*{P_{T_{\mb q}}(\mb q'' - \mb q)}_2
    &\leq \epsilon + \delta + H\\
    &\leq \eta.
    \end{aligned}
\end{equation}
By assumption, there exists a zero-order edge $\mb q \overset{0}{\rightarrow} \mb q'$, such that $\mb q' \in B_{\mc M}(\mb q'', c_4\tau_{\mc M})$. Then
\begin{equation}
    \begin{aligned}
        d_{\mc M}(\mb q', \mb x_\natural)
        &\leq d_{\mc M}(\mb q', \mb q'') + d_{\mc M}(\mb q'', \mb x_\natural)\\
        &\leq c_4 \tau_{\mc M} + \delta.
    \end{aligned}
\end{equation}
\end{proof}
}


\begin{lemma}\label{Phase II ZOE gives an acceptable landmark}
{Assume $\sigma \sqrt{D} \leq c_1 \tau_{\mc M}$ for some $c_1 \leq \frac{1}{640}$} and for any landmark $\mb q'' \in \mb q + N_{\mb q}^{\eta}$, with {$ \eta \geq  \epsilon + \delta + \sigma\left(4\sqrt{\log|Q|}+10\sqrt{d}\right)$}, where $\delta$ is the covering radius of Q and $\epsilon$ > 0, there exists a zero-order edge $\mb q \overset{0}{\rightarrow} \mb q'$, such that $\mb q' \in B_{\mc M}(\mb q'', c_4\tau_{\mc M})$, where $c_4 \le \frac{1}{40}$. Then, for any noisy $\mb x = \mb x_\natural + \mb z$ where $\mb x \in \mb q+N_{\mb q}^{\epsilon}$, with probability at least {$ 1 - 2e^{\frac{-9d}{2}}$} in the noise,  $\mb q^* = \arg\min_{\mb q': \mb q \overset{0}{\rightarrow} \mb q'}\left\|\mb q' -\mb x\right\|_2$ satisfies landmark  $\mb q^* \in B_{\mc M}(\mb x_\natural,{(2c_4 + 16 c_1)\tau_{\mc M}+2\delta})$
\end{lemma}

\begin{proof}
     Given the assumption,  lemma \ref{lem:existence of good ZOE} guarantees that with probability at least $1 - e^{\frac{-9d}{2}}$, there exists a landmark $\mb q'''$ and a zero-order edge $\mb q \overset{0}{\rightarrow} \mb q'''$, such that $\mb q''' \in B_{\mc M}(\mb x_\natural, c_4\tau_{\mc M}+\delta)$.
    Since $\mb q^* = \arg\min_{\mb q': \mb q \overset{0}{\rightarrow} \mb q'}\left\|\mb q' -\mb x\right\|_2$, we know that
    \begin{equation}\label{eq:ZOS output is closer than good edge}
        \norm{\mb q^* - \mb x}_2 \leq \norm{\mb q''' - \mb x}_2.
    \end{equation} 
    which is the same as 
    \begin{equation}
        \norm{\mb q^* - \mb x_\natural - \mb z}_2 \leq \norm{\mb q''' - \mb x_\natural - \mb z}_2.
    \end{equation} 
From triangular inequality, we have 
\begin{equation}
 \norm{\mb q^* - \mb x_\natural}_2 \le \norm{\mb q''' - \mb x_\natural}_2 + 2\norm{\mb z}_2.  
\end{equation}
Since $g(\mb z)=\norm{\mb z}_2$ is 1-lipschitz function in $\mb z$ and $\mathbb E[\norm{\mb z}_2] \leq \sigma \sqrt{D}$, then with probability at least $1-e^{-\frac{t^2}{2\sigma^2}}$, we have
\begin{equation}
    \norm{\mb z}_2 
    \leq \sigma \sqrt{D} + t.
\end{equation}
Take $t=3\sigma \sqrt{D}$, then with probability at least $1-e^{-\frac{9D}{2}}$, we have
\begin{equation}\label{eq:high prob bound for norm of z}
     \norm{\mb z}_2 
    \leq 4\sigma\sqrt{D} \le 4c_1 \tau_{\mc M},
\end{equation}
which implies with probability at least $1-e^{-\frac{9D}{2}}$, we have
\begin{equation}
 \norm{\mb q^* - \mb x_\natural}_2 \le \norm{\mb q''' - \mb x_\natural}_2 + 8c_1 \tau_{\mc M}  \le (c_4 + 8 c_1) \tau_{\mc M}+\delta  \le \tau_{\mc M}.
\end{equation}
The last inequality follows from the conditions $c_4 \leq 1/40, c_1 \leq 1/640, \delta = R_{\mr{nbrs}}/a\leq \tau_{\mc M}/40$. Together with the fact that on this scale $d_{\mc M}(\mb x_\natural, \mb q^*) \le 2 \norm{\mb q^* - \mb x_\natural}_2$, we have 
\begin{equation}
d_{\mc M}(\mb x_\natural, \mb q^*) \le (2c_4 + 16 c_1)\tau_{\mc M}+2\delta.
\end{equation}
\end{proof}






    



\section{Phase III Analysis} 
{
\begin{lemma}\label{lemma:velocity-edgeembedding-comparison}
    Consider the first-order edge $u \overset{1}{\rightarrow} v$. Let $\mb\xi_{uv} = P_{T_u} (\mb q_v - \mb q_u)$ denote the edge embedding, where $P_{T_u}$ is the projection operator onto the tangent space $T_u$ of the manifold $\mc M$ at landmark $\mb q_u$. Let $\mb\gamma:[0,1] \rightarrow \mc M$ be a geodesic joining $\mb q_u$ and $\mb q_v$ with constant speed $\|\mb v\|_2$, where $\mb\gamma(0)=\mb q_u, \mb\gamma(1)=\mb q_v$. Then, for all $t\in[0,1]$, we have
    \begin{equation}
        \|\dot{\mb\gamma}(t)-\mb\xi_{uv}\|_2 \leq \frac{3}{2} \kappa d_{\mc M}^2(\mb q_u, \mb q_v).
    \end{equation}
\end{lemma}
\begin{proof}
    From Taylor expansion, we have
    \begin{equation}
        \mb q_v = \mb q_u + \int_{a=0}^1 \dot{\mb\gamma}(a) da.
    \end{equation}
Therefore $\mb \xi_{uv} = P_{T_u} \int_{a=0}^1 \dot{\mb\gamma}(a) da$. Then


    \begin{equation}
        \begin{aligned}
            \|\dot{\mb\gamma}(t)-\mb\xi_{uv}\|_2
            &= \left\| \dot{\mb\gamma}(t) -  P_{T_u} \int_{a=0}^1 \dot{\mb\gamma}(a) da \right\|_2\\
            &=\left\| \dot{\mb\gamma}(0) + \int_{b=0}^t \Ddot{\mb\gamma}(b)db -  P_{T_u} \int_{a=0}^1 \left(\dot{\mb\gamma}(0) + \int_{b=0}^a \Ddot{\mb\gamma}(b)db \right)da  \right\|_2\\
            &=\left\| \int_{b=0}^t \Ddot{\mb\gamma}(b)db -  \int_{a=0}^1 \int_{b=0}^a P_{T_u} \Ddot{\mb\gamma}(b)dbda\right\|_2\\
            &= \left\| \int_{a=0}^1 \int_{b=0}^t \Ddot{\mb\gamma}(b)db da - \int_{a=0}^1 \left(\int_{b=0}^t P_{T_u} \Ddot{\mb\gamma}(b)db\right) + \left( \int_{b=t}^a P_{T_u} \Ddot{\mb\gamma}(b)db\right)da\right\|_2\\
            &=\left\|\int_{a=0}^1 \left( \int_{b=0}^t P_{N_u}\Ddot{\mb\gamma}(b)db - \int_{b=t}^a P_{T_u} \Ddot{\mb\gamma}(b)db\right)da\right\|_2\\
            &\leq \int_{a=0}^1 \left\| \int_{b=0}^t P_{N_u}\Ddot{\mb\gamma}(b)db \right\|_2 + \left\|\int_{b=t}^a P_{T_u} \Ddot{\mb\gamma}(b)db\right\|_2 da \\
            &\leq \int_{a=0}^1 (t+|a - t|)\kappa \|\mb v\|_2^2 \, da\\
            &= \left(  \int_{a = 0}^t [ 2t - a ] \, da + \int_{a = t}^1 a \, da \right) \kappa \|\mb v\|_2^2 \\
            &= \left( 2 t^2 - \tfrac{1}{2} t^2 + \tfrac{1}{2} - \tfrac{1}{2} t^2 \right) \kappa \|\mb v\|_2^2  \\
            &= \left(\frac{1}{2} + t^2 \right) \kappa \|\mb v\|_2^2 \\ 
            &\le \frac{3}{2} \kappa \|\mb v\|_2^2\\
            &=\frac{3}{2} \kappa d_{\mc M}^2(\mb q_u, \mb q_v)
        \end{aligned}
    \end{equation}
\end{proof}
}




\begin{lemma}\label{lemma:log-grad-comparison}
    For any $\mb z$ and $\mb x_\natural, \mb y \in \mc M$, let
\begin{equation}
    T_{\max} = \sup_{\mb y \in B_{\mc M}(\mb x_\natural, 1/\kappa), \mb v \in T_{\mb y } \mc M, \| \mb v \|_2 =1} \innerprod{\mb v}{\mb z},
\end{equation}
Then if $d_{\mc M}(\mb y, \mb x_\natural ) \le 1/\kappa$, we have 
    \begin{equation}
        \left\|-\log_{\mb y}\mb x_\natural - \grad[\varphi_{\mb x}](\mb y)\right\|_2 \leq 
        \frac{1}{2} \kappa d_{\mc M}^2 (\mb y, \mb x_\natural) + T_{\max}
    \end{equation}
\end{lemma}

\begin{proof}
We first decompose the left hand side into
\begin{equation}
    \begin{aligned}
        \left\|-\log_{\mb y}\mb x_\natural - \grad[\varphi_{\mb x}](\mb y)\right\|_2
        &= \left\| -\log_{\mb y}\mb x_\natural + P_{T_{\mb y}\mc M} (\mb x-\mb y)\right\|_2\\
        &\leq \left\| -\log_{\mb y}\mb x_\natural + P_{T_{\mb y}\mc M} (\mb x_\natural -\mb y) \right\|_2 + \left\| P_{T_{\mb y}\mc M} \mb z \right\|_2.
    \end{aligned}
\end{equation}
 Consider a geodesic $\mb\eta:[0,1] \rightarrow \mc M$ with constant speed $\|\mb v_{\mb\eta}\|_2$, where $\mb\eta(0)=\mb y, \mb\eta(1)=\mb x_\natural$,
then we bound the signal term (first term) of the above bound:
\begin{equation}
    \begin{aligned}
        \left\| -\log_{\mb y}\mb x_\natural + P_{T_{\mb y}\mc M} (\mb x_\natural -\mb y) \right\|_2
        &= \left\| -\Dot{\mb\eta}(0) + P_{T_{\mb y}\mc M} \int_{a=0}^1 \Dot{\mb\eta}(a) da\right\|_2\\
        &=\left\| -\Dot{\mb\eta}(0) + P_{T_{\mb\eta(0)}\mc M} \int_{a=0}^1 \left(\Dot{\mb\eta}(0) + \int_{b=0}^a \Ddot{\mb\eta}(b) db\right) da \right\|_2\\
        &=\left\| P_{T_{\mb\eta(0)}\mc M} \int_{a=0}^1 \int_{b=0}^a \Ddot{\mb\eta}(b) db da\right\|_2\\
        &\leq \int_{a=0}^1 \int_{b=0}^a \left\|\Ddot{\mb\eta}(b) \right\|_2 db da\\
        &\leq \int_{a=0}^1 \int_{b=0}^a \kappa \|\mb v_{\mb \eta}\|_2^2  db da\\
        &=\frac{1}{2} \kappa d_{\mc M}^2 (\mb y, \mb x_\natural).
    \end{aligned}
\end{equation}
Since $\mb y \in B_{\mc M}(\mb x_\natural, 1/\kappa)$, we have 
\begin{equation}
    \begin{aligned}
        \left\| P_{T_{\mb y}\mc M} \mb z\right\|_2
        \leq T_{\max}
    \end{aligned}
\end{equation}
Combining the above, we end up with the result.
\end{proof}


\begin{lemma}\label{lemma:grad-bound} 
For any point $\mb z$ and $\mb x_\natural, \mb y \in \mc M$, let
\begin{equation}
    T_{\max} = \sup_{\mb y \in B_{\mc M}(\mb x_\natural, 1/\kappa), \mb v \in T_{\mb y } \mc M, \| \mb v \|_2 =1} \innerprod{\mb v}{\mb z},
\end{equation}
if $d_{\mc M}(\mb y, \mb x_\natural ) \le 1/\kappa$,  then we have
    \begin{equation}
        \left\|\grad[\varphi_{\mb x}](\mb y) \right\|_2\leq d_{\mc M}(\mb y,\mb x_\natural) + T_{\max}.
    \end{equation}
\end{lemma}

\begin{proof}   
    \begin{equation}
        \begin{aligned}
             \left\|\grad[\varphi_{\mb x}](\mb y) \right\|_2
            &=  \left\| P_{T_{\mb y}\mc M} (\mb y - \mb x)\right\|_2\\
            &\le \left\| P_{T_{\mb y}\mc M} (\mb y - \mb x_\natural)\right\|_2 
            + \left\| P_{T_{\mb y}\mc M} \mb z\right\|_2
        \end{aligned}
    \end{equation}
It's easy to show that $\left\| P_{T_{\mb y}\mc M} (\mb y - \mb x_\natural)\right\|_2 \leq d_{\mc M}(\mb y,\mb x_\natural)$. Let 
\begin{equation}
    T_{\max} = \sup_{\mb y \in B_{\mc M}(\mb x_\natural, 1/\kappa), \mb v \in T_{\mb y } \mc M, \| \mb v \|_2 =1} \innerprod{\mb v}{\mb z},
\end{equation}
then we have
\begin{equation}
    \left\| P_{T_{\mb y}\mc M} \mb z\right\|_2
    \leq T_{\max}
\end{equation}
Combining the above two bounds, we end up with the desired result.
\end{proof}




\begin{lemma}\label{lemma:vecotr-parallel transported vector-comparison}
    Let $\zeta:[0,1] \rightarrow \mc M$ be a geodesic in a Riemannian manifold $\mc M$. Take any initial vector $\mb v_0 \in T_{\mb\zeta(0)}\mc M$, and let $\mb v_t$ be its parallel transport along $\zeta$ up to time $t$. Then 
    \begin{equation}
        \left\| \mb v_t - \mb v_0 \right\|_2
        \leq 3\kappa t \|\mb v_0\|_2 \|\Dot{\mb\zeta}\|_2 
    \end{equation}
\end{lemma}

\begin{proof}
When paralleling transport a $\mb v_0 \in T_{\mb\zeta(0)}\mc M$ along the geodesic $\mb\zeta$, we have
\begin{equation}
    \frac{d}{dt} \mb v_t = \Pi(\mb v_t, \Dot{\mb\zeta}(t)).
\end{equation}
From fundamental theorem of calculus, we have
\begin{equation}
    \mb v_t = \mb v_0 + \int_{a=0}^t \Pi(\mb v_a, \Dot{\mb\zeta}(a)) da.
\end{equation}
From above and applying Lemma 8 in \cite{yan2023tpopt}, we have 
\begin{equation}
    \begin{aligned}
        \left\| \mb v_t - \mb v_0 \right\|_2
        &= \left\| \int_{a=0}^t \Pi(\mb v_a, \Dot{\mb\zeta}(a)) da \right\|_2\\
        &\leq \int_{a=0}^t \left\|\Pi(\mb v_a, \Dot{\mb\zeta}(a))\right\|_2 da\\
        &\leq { \int_{a=0}^t 3\kappa \|\mb v_0\|_2 \|\Dot{\mb\zeta}\|_2da}\\
        &= 3\kappa t \|\mb v_0\|_2 \|\Dot{\mb\zeta}\|_2 
    \end{aligned}
\end{equation}
\end{proof}

\begin{lemma}\label{lemma:tangnet space diff bound}
    Let $\mb\gamma:[0,1] \rightarrow \mc M$ be a geodesic joining $\mb q_u$ and $\mb q_v$ with constant speed, where $\mb\gamma(0)=\mb q_u, \mb\gamma(1)=\mb q_v$. Then, for all $t\in[0,1]$, we have
    \begin{equation}
    \left\| P_{T_{\mb\gamma(t)} \mc M} - P_{T_{\mb\gamma(0)} \mc M}\right\|_{op}
    \leq 3\sqrt{2} \kappa t \|\Dot{\mb\gamma}\|_2  .
    \end{equation}
\end{lemma}


\begin{proof}
\begin{equation}
    \begin{aligned}
        \left\| P_{T_{\mb\gamma(t)} \mc M} - P_{T_{\mb\gamma(0)} \mc M}\right\|_{op}
        &= \sup_{\|\mb w\|_2=1} \left\| \left( P_{T_{\mb\gamma(t)} \mc M} - P_{T_{\mb\gamma(0)} \mc M}\right) \mb w\right\|_2\\
        &= \sup_{\|\mb w\|_2=1} \left\| \left( P_{T_{\mb\gamma(t)} \mc M} - P_{T_{\mb\gamma(0)} \mc M}\right) (\mb w_{||} + \mb w_{\perp}) \right\|_2\\
        &\leq \sup_{\|\mb w\|_2=1} \left\| \left( P_{T_{\mb\gamma(t)} \mc M} - P_{T_{\mb\gamma(0)} \mc M}\right) \mb w_{||} \right\|_2 + \left\| \left( P_{T_{\mb\gamma(t)} \mc M} - P_{T_{\mb\gamma(0)} \mc M}\right) \mb w_{\perp} \right\|_2\\
        &= \sup_{\|\mb w\|_2=1} \left\| P_{T_{\mb\gamma(t)} \mc M} \mb w_{||} -  \mb w_{||} \right\|_2 
        + \left\| P_{T_{\mb\gamma(t)} \mc M} \mb w_{\perp}  \right\|_2
    \end{aligned}
\end{equation}
where $\mb w_{||} = P_{T_{\mb\gamma (0)}\mc M} \mb w, \mb w_{\perp} = P_{T_{\mb\gamma (0)}^{\perp} \mc M} \mb w$.

Together with Lemma \ref{lemma:vecotr-parallel transported vector-comparison}, we bound the first term $\left\| \left( P_{T_{\mb\gamma(t)} \mc M} - P_{T_{\mb\gamma(0)} \mc M}\right) \mb w_{||} \right\|_2$.
\begin{equation}
    \begin{aligned}
        \left\| P_{T_{\mb\gamma(t)} \mc M} \mb w_{||} -  \mb w_{||} \right\|_2
        &= \min_{\mb h \in T_{\mb\gamma(t)}\mc M} \left\| \mb h - \mb w_{||}\right\|_2\\
        &\leq \left\|  \Pi_{0\rightarrow t} \mb w_{||} - \mb w_{||}\right\|_2\\
        &\leq 3 \kappa t\|\mb w_{||}\|_2 \|\Dot{\mb\gamma}\|_2 
    \end{aligned}
\end{equation}
Here $\Pi_{0\rightarrow t}$ is a transport operator which transports a vector in $T_{\mb\gamma(0)} \mc M$ to the $T_{\mb\gamma(t)} \mc M$ along the geodesic $\mb\gamma$.

We next bound the second term $\left\| P_{T_{\mb\gamma(t)} \mc M} \mb w_{\perp}  \right\|_2$.
\begin{equation}
    \begin{aligned}
        \left\| P_{T_{\mb\gamma(t)} \mc M} \mb w_{\perp}  \right\|_2
        &= \left\| P_{T_{\mb\gamma(t)} \mc M} P_{T_{\mb\gamma(0)}^{\perp}\mc M}\mb w_{\perp}  \right\|_2\\
        &\leq \|\mb w_{\perp}\|_2 \left\| P_{T_{\mb\gamma(t)} \mc M} P_{T_{\mb\gamma(0)}^{\perp}\mc M}\right\|_{op}\\
        &= \|\mb w_{\perp}\|_2 \left\| P_{T_{\mb\gamma(0)}^{\perp} \mc M} P_{T_{\mb\gamma(t)}\mc M}\right\|_{op} \\
        &= \|\mb w_{\perp}\|_2 \sup_{\|\mb u\|_2=1} \left\| P_{T_{\mb\gamma(0)}^{\perp} \mc M} P_{T_{\mb\gamma(t)}\mc M} \mb u\right\|_2\\
        &= \|\mb w_{\perp}\|_2 \sup_{\|\mb u\|_2=1, \mb u \in T_{\mb\gamma(t)}\mc M} \left\| P_{T_{\mb\gamma(0)}^{\perp} \mc M} \mb u\right\|_2\\
        &= \|\mb w_{\perp}\|_2 \sup_{\|\mb u\|_2=1, \mb u \in T_{\mb\gamma(t)}\mc M} \left\| \left(\mb I - P_{T_{\mb\gamma(0)} \mc M} \right) \mb u\right\|_2\\
        &= \|\mb w_{\perp}\|_2 \sup_{\|\mb u\|_2=1, \mb u \in T_{\mb\gamma(t)}\mc M} \min_{\mb h \in T_{\mb\gamma(0)}\mc M}\left\| \mb h - \mb u\right\|_2\\
        &\leq \|\mb w_{\perp}\|_2 \sup_{\|\mb u\|_2=1, \mb u \in T_{\mb\gamma(t)}\mc M} \left\|
        \Pi_{t\rightarrow 0} \mb u - \mb u\right\|_2\\
        &\leq 3 \kappa t \| \|\mb w_{\perp}\|_2 \|\Dot{\mb\gamma}\|_2 
    \end{aligned}
\end{equation}

Therefore,
\begin{equation}
    \begin{aligned}
        \sup_{\|\mb w\|_2=1} \left\| P_{T_{\mb\gamma(t)} \mc M} \mb w_{||} -  \mb w_{||} \right\|_2 
        + \left\| P_{T_{\mb\gamma(t)} \mc M} \mb w_{\perp}  \right\|_2
        &\leq \sup_{\|\mb w\|_2=1} 3 \kappa t\|\Dot{\mb\gamma}\|_2  (\|\mb w_{||}\|_2 + \|\mb w_{\perp}\|_2)\\
        &\leq \sup_{\|\mb w\|_2=1} 3 \sqrt{2} \kappa t \|\Dot{\mb\gamma}\|_2 \|\mb w\|_2\\
        &= 3\sqrt{2} \kappa t \|\Dot{\mb\gamma}\|_2 
    \end{aligned}
\end{equation}
Hence
\begin{equation}
    \left\| P_{T_{\mb\gamma(t)} \mc M} - P_{T_{\mb\gamma(0)} \mc M}\right\|_{op}
    \leq 3\sqrt{2} \kappa t \|\Dot{\mb\gamma}\|_2  
\end{equation}

\end{proof}


\begin{lemma}\label{lemma:main term bound}
Let $\mb\gamma:[0,1] \rightarrow \mc M$ be a minimum length geodesic joining $\mb q_u$ and $\mb q_{u^+}$ with constant speed $d_{\mc M}(\mb q_u, \mb q_{u^+})$, where $\mb\gamma(0)=\mb q_u, \mb\gamma(1)=\mb q_{u^+}$. Suppose for any $t \in [0,1], \: d_{\mc M}( \mb \gamma(t), \mb x_\natural ) \le 1 /\kappa$, then, we have
\begin{equation}
    \begin{aligned}\innerprod{\grad[\varphi_{\mb x}](\mb\gamma(t))}{\mb \xi_{uu^+}}
    &\leq -\frac{0.55}{2} \left( \frac{1}{2}d_{\mc M}(\mb x_\natural, \mb\gamma(0))  - T_{\max}\right) 
        d_{\mc M}(\mb q_u, \mb q_{u^+})\\
        &+ \left( 3\sqrt{2} \kappa t d_{\mc M}(\mb q_u, \mb q_{u^+}) d_{\mc M}(\mb\gamma(t),\mb x_\natural)+2T_{\max}+ t d_{\mc M}(\mb q_u, \mb q_{u^+}) + \frac{1}{2} \kappa t^2 d_{\mc M}^2 (\mb q_u, \mb q_{u^+})\right) d_{\mc M}(\mb q_u, \mb q_{u^+})
        \end{aligned}
\end{equation}
\end{lemma}



\begin{proof}
    We note that
    \begin{equation}
    \begin{aligned}
        \innerprod{\grad[\varphi_{\mb x}](\mb\gamma(t))}{\mb \xi_{uu^+}}
        &= \innerprod{\grad[\varphi_{\mb x}](\mb\gamma(0))}{\mb \xi_{uu^+}} 
        + \innerprod{\grad[\varphi_{\mb x}](\mb\gamma(t)) - \grad[\varphi_{\mb x}](\mb\gamma(0))}{\mb \xi_{uu^+}}\\
        &\leq \innerprod{\grad[\varphi_{\mb x}](\mb\gamma(0))}{\mb \xi_{uu^+}} 
        + \left\| \grad[\varphi_{\mb x}](\mb\gamma(t)) - \grad[\varphi_{\mb x}](\mb\gamma(0)) \right\|_2 d_{\mc M}(\mb q_u, \mb q_{u^+}).
    \end{aligned}
\end{equation}


For any $\mb x \ne \mb q_u$, our first-order step rule is equivalent of choosing \[u^+ = \arg \min_{v : u \overset{1}{\rightarrow} v \in E^1} \left\| R_a\frac{\mc P_{T_u} (\mb x -\mb q_u)}{\left\|\mc P_{T_u} (\mb x -\mb q_u) \right\|_2} - \mc P_{T_u} (\mb q_v - \mb q_u) \right\|_2\] 
Then, lemma \ref{lem:tangent space covering} guarantees that there exists a $v$ satisfies \Cref{eq:app:bound_dot_prod_grad_edge_tangent_dir_condition}. From our step rule, we know that $q_{u^+}$ also satisfies \Cref{eq:app:bound_dot_prod_grad_edge_tangent_dir_condition}. Together with Lemma \ref{lem:bound dot product of gradient and edge in tangent direction}  and $d_{\mc M}(\mb\gamma(0),\mb x_\natural) \leq 1/\kappa$, we have
\begin{equation}
    \begin{aligned}
        \innerprod{\grad[\varphi_{\mb x}](\mb\gamma(0))}{\mb \xi_{uu^+}} 
        &\leq -0.55 \left\|\grad[\varphi_{\mb x}](\mb\gamma(0)) \right\|_2 \left\|\mb \xi_{uu^+} \right\|_2\\
        &\leq -0.55 \left( \left\| P_{T_{\mb\gamma(0)}\mc M}  \left(\mb x_\natural - \mb\gamma(0)\right)\right\|_2 - \left\| P_{T_{\mb\gamma(0)}\mc M} \mb z\right\|_2\right) \left\|\mb \xi_{uu^+} \right\|_2\\
        &\leq -0.55\left( d_{\mc M}(\mb\gamma(0),\mb x_\natural)\left( 1-\frac{1}{2}\kappa d_{\mc M}(\mb\gamma(0),\mb x_\natural)\right) - T_{\max}\right) \frac{1}{2}d_{\mc M}(\mb q_u, \mb q_{u^+})\\
        &\leq \frac{-0.55}{2}\left(\frac{1}{2}d_{\mc M}(\mb\gamma(0),\mb x_\natural) -T_{\max} \right) d_{\mc M}(\mb q_u, \mb q_{u^+}).
    \end{aligned}
\end{equation}
where from the second to the third line we've used the same argument from equation \eqref{eq: bound norm of edge embedding} to lower bound $\left\| P_{T_{\mb\gamma(0)}\mc M}  \left(\mb x_\natural - \mb\gamma(0)\right)\right\|_2$ by $d_{\mc M}(\mb\gamma(0),\mb x_\natural)\left( 1-\frac{1}{2}\kappa d_{\mc M}(\mb\gamma(0),\mb x_\natural)\right)$ and $\left\|\mb \xi_{uu^+} \right\|_2$ by $\frac{1}{2}d_{\mc M}(\mb q_u, \mb q_{u^+})$
And we can bound $\left\| \grad[\varphi_{\mb x}](\mb\gamma(t)) - \grad[\varphi_{\mb x}](\mb\gamma(0)) \right\|_2$:

\begin{equation}
    \begin{aligned}
        \left\| \grad[\varphi_{\mb x}](\mb\gamma(t)) - \grad[\varphi_{\mb x}](\mb\gamma(0)) \right\|_2 
        &\leq \left\| \left(P_{T_{\mb\gamma(t)} \mc M} - P_{T_{\mb\gamma(0)} \mc M}\right)(\mb\gamma(t)-\mb x_\natural)\right\|_2\\
        &+  \left\| \left(P_{T_{\mb\gamma(t)} \mc M} - 
        P_{T_{\mb\gamma(0)} \mc M}\right) \mb z\right\|_2\\
        &+\left\|P_{T_{\mb\gamma(0)} \mc M}(\mb\gamma(t)-\mb\gamma(0)) \right\|_2\\
        &\leq \left\| P_{T_{\mb\gamma(t)} \mc M} - P_{T_{\mb\gamma(0)} \mc M}\right\|_{op} d_{\mc M}(\mb\gamma(t),\mb x_\natural)\\
        &+ \left\| P_{T_{\mb\gamma(t)} \mc M}  \mb z\right\|_2 + \left\| P_{T_{\mb\gamma(0)} \mc M}  \mb z\right\|_2\\
        &+ \left\|P_{T_{\mb\gamma(0)} \mc M}(\mb\gamma(t)-\mb\gamma(0)) \right\|_2
    \end{aligned}
\end{equation}

Together with Lemma \ref{lemma:tangnet space diff bound}, we have
\begin{equation}
    \left\| P_{T_{\mb\gamma(t)} \mc M} - P_{T_{\mb\gamma(0)} \mc M}\right\|_{op} d_{\mc M}(\mb\gamma(t),\mb x_\natural)
    \leq 3\sqrt{2} \kappa t d_{\mc M}(\mb q_u, \mb q_{u^+}) d_{\mc M}(\mb\gamma(t),\mb x_\natural)
\end{equation}

Since $\mb\gamma(t)\in B_{\mc M}(\mb x_\natural, 1/\kappa) \quad \forall t\in[0,1]$, we have
\begin{equation}
    \left\| P_{T_{\mb\gamma(t)} \mc M}  \mb z\right\|_2 \leq T_{\max},
\end{equation}
hence
\begin{equation}
    \left\| P_{T_{\mb\gamma(t)} \mc M}  \mb z\right\|_2+
    \left\| P_{T_{\mb\gamma(0)} \mc M}  \mb z\right\|_2 
    \leq 2T_{\max}.
\end{equation}

From Taylor expansion on the geodesic $\mb\gamma$, we have
\begin{equation}
    \begin{aligned}
        \left\|P_{T_{\mb\gamma(0)} \mc M}(\mb\gamma(t)-\mb\gamma(0)) \right\|_2
        &= \left\|P_{T_{\mb\gamma(0)} \mc M} \int_{a=0}^t \Dot{\mb\gamma}(a) da \right\|_2\\
        &= \left\|P_{T_{\mb\gamma(0)} \mc M} \int_{a=0}^t \left(\Dot{\mb\gamma}(0) + \int_{b=0}^a \Ddot{\mb\gamma}(b) db\right) da \right\|_2\\
        &= \left\| \Dot{\mb\gamma}(0) t + \int_{a=0}^t \int_{b=0}^a 
        P_{T_{\mb\gamma(0)} \mc M} \Ddot{\mb\gamma}(b) db  da \right\|_2\\
        &\leq t d_{\mc M}(\mb q_u, \mb q_{u^+}) + \frac{1}{2} \kappa t^2 d_{\mc M}^2 (\mb q_u, \mb q_{u^+})
    \end{aligned}
\end{equation}
Therefore, 
\begin{equation}
    \begin{aligned}
     \left\| \grad[\varphi_{\mb x}](\mb\gamma(t)) - \grad[\varphi_{\mb x}](\mb\gamma(0)) \right\|_2 
      \leq 3\sqrt{2} \kappa t d_{\mc M}(\mb q_u, \mb q_{u^+}) d_{\mc M}(\mb\gamma(t),\mb x_\natural)
      + 2T_{\max}
      + t d_{\mc M}(\mb q_u, \mb q_{u^+}) + \frac{1}{2} \kappa t^2 d_{\mc M}^2 (\mb q_u, \mb q_{u^+})
    \end{aligned}
\end{equation}

Combining all of things above, we have
\begin{equation}
    \begin{aligned}
        \innerprod{\grad[\varphi_{\mb x}](\mb\gamma(t))}{\mb \xi_{uu^+}}
        &\leq \frac{-0.55}{2} \left( \frac{1}{2}d_{\mc M}(\mb x_\natural, \mb\gamma(0))  - T_{\max}\right) 
        d_{\mc M}(\mb q_u, \mb q_{u^+})\\
        &+ \left( 3\sqrt{2} \kappa t d_{\mc M}(\mb q_u, \mb q_{u^+}) d_{\mc M}(\mb\gamma(t),\mb x_\natural)+2T_{\max}+ t d_{\mc M}(\mb q_u, \mb q_{u^+}) + \frac{1}{2} \kappa t^2 d_{\mc M}^2 (\mb q_u, \mb q_{u^+})\right) d_{\mc M}(\mb q_u, \mb q_{u^+})
    \end{aligned}
\end{equation}
\end{proof}






\begin{lemma}\label{prop:one step in phase III}
Suppose
\begin{equation}
    \mb q_u \in B_{\mc M}(\mb x_\natural, c/\kappa) \setminus B_{\mc M}(\mb x_\natural, C T_{\max} ),
\end{equation}
for some constant $c \leq 0.1, \: C \ge 321$, with 
\begin{equation}
    T_{\max} = \sup_{\mb y \in B_{\mc M}(\mb x_\natural, 1/\kappa), \mb v \in T_{\mb y} \mc M, \| \mb v \|_2 =1} \innerprod{\mb v}{\mb z},
\end{equation}
and \begin{equation}
    d_{\mc M}(\mb q_u, \mb q_{u^+}) \leq \min \left\{ \frac{1}{100\kappa}, \frac{1}{8} T_{\max}\ \right\}, 
    \end{equation}
    then 
    \begin{equation}
        d_{\mc M}(\mb q_{u^+},\mb x_\natural) \leq  d_{\mc M}(\mb q_{u},\mb x_\natural) - \frac{1}{80} d_{\mc M}(\mb q_u, \mb q_{u^+}).    
    \end{equation}
\end{lemma}

\begin{proof} Let $\mb\gamma:[0,1] \rightarrow \mc M$ be a minimum length geodesic joining $\mb q_u$ and $\mb q_{u^+}$ with constant speed $d_{\mc M}(\mb q_u, \mb q_{u^+})$, where $\mb\gamma(0)=\mb q_u, \mb\gamma(1)=\mb q_{u^+}$. Then we have
\begin{equation}
    \begin{aligned}
        d_{\mc M}(\mb x_\natural, \mb q_{u^+}) - d_{\mc M}(\mb x_\natural, \mb q_u)
        &= \int_{t=0}^1 \frac{d}{dt}d_{\mc M}(\mb x_\natural, \mb\gamma(t)) dt\\
        &=\int_{t=0}^1 \innerprod{\frac{d}{d \mb y}d_{\mc M}(\mb x_\natural,\mb y)\Big|_{\mb y=\mb\gamma(t)}}{\dot{\mb\gamma}(t)} dt\\
        &= \int_{t=0}^1 \innerprod{\frac{-\log_{\mb \gamma(t)}\mb x_\natural}{\| \log_{\mb \gamma(t)}\mb x_\natural\|_2}}{\dot{\mb\gamma}(t)} dt.
    \end{aligned}
\end{equation}
We can further decompose the integrand as follows:
\begin{equation}
    \begin{aligned}
        &\innerprod{\frac{-\log_{\mb \gamma(t)}\mb x_\natural}{\| \log_{\mb \gamma(t)}\mb x_\natural\|_2}}{\dot{\mb\gamma}(t)}\\
        &= \frac{1}{\| \log_{\mb \gamma(t)}\mb x_\natural\|_2} \innerprod{-\log_{\mb \gamma(t)}\mb x_\natural + \grad[\varphi_{\mb x}](\mb\gamma(t)) - \grad[\varphi_{\mb x}](\mb\gamma(t))}{\dot{\mb\gamma}(t) + \mb \xi_{uu^+} - \mb \xi_{uu^+}}\\
        &= \frac{1}{d_{\mc M}(\mb\gamma(t), \mb x_\natural)}  
        \left(\innerprod{\grad[\varphi_{\mb x}](\mb\gamma(t))}{\mb \xi_{uu^+}}
        + \innerprod{-\log_{\mb \gamma(t)}\mb x_\natural - \grad[\varphi_{\mb x}](\mb\gamma(t))}{\dot{\mb\gamma}(t)}
        + \innerprod{\grad[\varphi_{\mb x}](\mb\gamma(t))}{\dot{\mb\gamma}(t)-\mb\xi_{uu^+}} \right) \label{eqn:dist-deriv-split1}.
    \end{aligned}
\end{equation}
We will proceed to use Lemma \ref{lemma:main term bound} to bound the first term, Lemma \ref{lemma:log-grad-comparison} to bound the second term, and Lemmas \ref{lemma:velocity-edgeembedding-comparison} and \ref{lemma:grad-bound} to bound the last term. In order to apply these lemmas, we observe that $\forall t\in[0,1], d_{\mc M}(\mb\gamma(t),\mb x_\natural) \leq d_{\mc M}(\mb\gamma(0),\mb x_\natural) + d_{\mc M}(\mb\gamma(0),\mb\gamma(t)) \leq d_{\mc M}(\mb\gamma(0),\mb x_\natural) + d_{\mc M}(\mb\gamma(0),\mb\gamma(1)) \leq \frac{c}{\kappa} + \frac{1}{100\kappa}\leq \frac{1}{8\kappa}.$ 



Thus from Lemma \ref{lemma:main term bound} we have
 \begin{equation}
    \begin{aligned}
        \innerprod{\grad[\varphi_{\mb x}](\mb\gamma(t))}{\mb \xi_{uu^+}}
        &\leq {\frac{-0.55}{2}} \left( \frac{1}{2}d_{\mc M}(\mb x_\natural, \mb\gamma(0))  - T_{\max}\right) 
        d_{\mc M}(\mb q_u, \mb q_{u^+})\\
        &+ \left( 3\sqrt{2} \kappa t d_{\mc M}(\mb q_u, \mb q_{u^+}) d_{\mc M}(\mb\gamma(t),\mb x_\natural)+2T_{\max}+ t d_{\mc M}(\mb q_u, \mb q_{u^+}) + \frac{1}{2} \kappa t^2 d_{\mc M}^2 (\mb q_u, \mb q_{u^+})\right) d_{\mc M}(\mb q_u, \mb q_{u^+}) \\
        &\leq \left(-\frac{1}{8}d_{\mc M}(\mb x_\natural, \mb\gamma(t))+\frac{1}{8}d_{\mc M}(\mb q_u, \mb q_{u^+}) + \frac{1}{2} T_{\max}\right)d_{\mc M}(\mb q_u, \mb q_{u^+})\\
        &+ \left( \frac{3\sqrt{2}}{8}d_{\mc M}(\mb q_u, \mb q_{u^+})+2T_{\max}+ d_{\mc M}(\mb q_u, \mb q_{u^+}) + \frac{1}{200} d_{\mc M} (\mb q_u, \mb q_{u^+})\right) d_{\mc M}(\mb q_u, \mb q_{u^+}) \\
        &\leq \left(-\frac{1}{8}d_{\mc M}(\mb x_\natural, \mb\gamma(t))+ \frac{5}{2} T_{\max} + 2d_{\mc M}(\mb q_u, \mb q_{u^+}) \right)d_{\mc M}(\mb q_u, \mb q_{u^+})\\
    \end{aligned}
\end{equation}
Similarly, from Lemma \ref{lemma:log-grad-comparison} we have 
\begin{equation}
    \begin{aligned}
    \innerprod{-\log_{\mb \gamma(t)}\mb x_\natural - \grad[\varphi_{\mb x}](\mb\gamma(t))}{\dot{\mb\gamma}(t)}
    &\leq \left\|-\log_{\mb \gamma(t)}\mb x_\natural - \grad[\varphi_{\mb x}](\mb\gamma(t))\right\| \cdot \left\|\dot{\mb\gamma}(t)\right\| \\ 
    &\leq \left(\frac{1}{2} \kappa d_{\mc M}^2 (\mb\gamma(t), \mb x_\natural) + T_{\max}\right)\cdot d_{\mc M}(\mb q_u, \mb q_{u^+})\\
    &\leq \left(\frac{1}{16}d_{\mc M} (\mb\gamma(t), \mb x_\natural) + T_{\max}\right)\cdot d_{\mc M}(\mb q_u, \mb q_{u^+})
    \end{aligned}
\end{equation}
and Lemmas \ref{lemma:velocity-edgeembedding-comparison} and \ref{lemma:grad-bound} also give
\begin{equation}
    \begin{aligned}
    \innerprod{\grad[\varphi_{\mb x}](\mb\gamma(t))}{\dot{\mb\gamma}(t)-\mb\xi_{uu^+}}
    &\leq \left\|\grad[\varphi_{\mb x}](\mb\gamma(t))\right\| \cdot \left\|\dot{\mb\gamma}(t)-\mb\xi_{uu^+}\right\| \\ 
    &\leq \left(d_{\mc M} (\mb\gamma(t), \mb x_\natural) + T_{\max} \right)\cdot\left(\frac{3\kappa}{2}d_{\mc M}{^2}(\mb q_u, \mb q_{u^+})\right)\\
    &\leq \left(\frac{3}{200}d_{\mc M} (\mb\gamma(t), \mb x_\natural) +\frac{3}{200} T_{\max} \right)d_{\mc M}(\mb q_u, \mb q_{u^+})
    \end{aligned}
\end{equation}
Lastly, we combine the terms to get
\begin{equation}
    \begin{aligned}
        & \innerprod{\grad[\varphi_{\mb x}](\mb\gamma(t))}{\mb \xi_{uu^+}}
        + \innerprod{-\log_{\mb \gamma(t)}\mb x_\natural - \grad[\varphi_{\mb x}](\mb\gamma(t))}{\dot{\mb\gamma}(t)}
        + \innerprod{\grad[\varphi_{\mb x}](\mb\gamma(t))}{\dot{\mb\gamma}(t)-\mb\xi_{uu^+}} \\
        \leq& \left(-\frac{1}{40}d_{\mc M} (\mb\gamma(t), \mb x_\natural) +\frac{15}{4}T_{\max}+ 2d_{\mc M}(\mb q_u, \mb q_{u^+}) \right)d_{\mc M}(\mb q_u, \mb q_{u^+})\\
        \leq& \left(-\frac{1}{40}d_{\mc M} (\mb\gamma(t), \mb x_\natural) +4T_{\max}\right)d_{\mc M}(\mb q_u, \mb q_{u^+})\\
        \leq& \left(-\frac{1}{80}d_{\mc M} (\mb\gamma(t), \mb x_\natural)\right)d_{\mc M}(\mb q_u, \mb q_{u^+})
    \end{aligned}
\end{equation}
where we've used our assumption that $d_{\mc M}(\mb q_u, \mb q_{u^+})\leq \frac{1}{8}T_{\max}$ and $d_{\mc M} (\mb\gamma(t), \mb x_\natural) \ge d_{\mc M} (\mb\gamma(0), \mb x_\natural) - d_{\mc M}(\mb q_u, \mb q_{u^+}) \ge C T_{\max} - \frac{1}{8} T_{\max} \ge 320 T_{\max}$. Finanly, plugging this result back to Equation \ref{eqn:dist-deriv-split1}, we observe 
\begin{equation}
    \begin{aligned}
        d_{\mc M}(\mb x_\natural, \mb q_{u^+}) - d_{\mc M}(\mb x_\natural, \mb q_u)
        &= \int_{t=0}^1 \innerprod{\frac{-\log_{\mb \gamma(t)}\mb x_\natural}{\| \log_{\mb \gamma(t)}\mb x_\natural\|_2}}{\dot{\mb\gamma}(t)} dt\\
        & \leq \int_{t=0}^1\frac{1}{d_{\mc M}(\mb\gamma(t), \mb x_\natural)}  \left(-\frac{1}{80}d_{\mc M} (\mb\gamma(t), \mb x_\natural)\right)d_{\mc M}(\mb q_u, \mb q_{u^+}) dt\\
        & \leq -\frac{1}{80}d_{\mc M}(\mb q_u, \mb q_{u^+})
    \end{aligned}
\end{equation}
This completes the proof
\end{proof}




\appendix 


\section{Supporting Results} 


\paragraph{Preliminiaries on the logarithmic map.}

{
The following sequence of lemmas provides an upper bound on the number of landmarks $|Q|$, under the assumption that the landmarks are $\delta$-separated. Our argument will assume that the manifold $\mc M$ is {\em connected} and {\em geodesically complete}. Under these assumptions, the exponential map 
\begin{equation}
    \exp_{\mb x_\natural}(\cdot) : T_{\mb x_\natural} \mc M \to \mc M
\end{equation}
is surjective, i.e., for every $\mb q \in \mc M$, there exists $\mb v \in T_{\mb x_\natural}\mc M$ such that 
\begin{equation} \label{eqn:v-q}
    \exp_{\mb x_\natural}(\mb v) = \mb q.
\end{equation} 
Moreover, by the Hopf-Rinow theorem, there exists a length-minimizing geodesic joining $\mb x_\natural$ and $\mb q$, and hence there exists $\mb v \in T_{\mb x_\natural} \mc M$ of norm $\| \mb v \| = d_{\mc M}(\mb x_\natural, \mb q )$ satisfying \eqref{eqn:v-q}. In particular, for every $\mb q \in \mc M$, there exists $\mb v \in T_{\mb x_\natural} \mc M$ of norm at most $\| \mb v \| \le \mr{diam}(\mc M)$ satisfying \eqref{eqn:v-q}. 

The {\em logarithmic map}
\begin{equation}
    \wt{\log}_{\mb x_\natural}  : \mc M \to T_{\mb x_\natural} \mc M
\end{equation}
is defined, in the broadest generality, as the inverse of the exponential map. This mapping can be multi-valued, since for a given $\mb q$ there may be multiple tangent vectors $\mb v$ satisfying \eqref{eqn:v-q}. Notice that because $\exp$ is surjective, its inverse, $\wt{\log}$ is well defined for all $\mb q \in \mc M$. When $d_{\mc M}(\mb x_\natural,\mb q) \le r_{\mr{inj}}$ is smaller than the injectivity radius of the exponential map at $\mb x_\natural$ \footnote{$\text{inj}(\mb x_\natural) = \sup \{r >0 :\exp_{\mb x_\natural} \text{is a diffeomorphism on} B(0,r) \subset T_{\mb x_\natural}\mc M\}$}, there is a unique minimum norm element $\mb v_\star$ of the set $\wt{\log}_{\mb x_\natural}(\mb q)$. This is typically denoted 
\begin{equation}
    \log_{\mb x_\natural}(\mb q)
\end{equation}
and satisfies $\| \mb v_\star \| = d_{\mc M}(\mb x_\natural, \mb q )$.\footnote{This is often taken as the {\em definition} of the logarithmic map.} We can extend this notation from $\mb q \in B_{\mc M}(\mb x_\natural, r_{\mr{inj}})$ to all of $\mc M$, by letting 
\begin{equation}
    \log_{\mb x_\natural}(\mb q)
\end{equation}
denote a minimum norm element of the set $\wt{\log}_{\mb x_\natural}(\mb q)$, chosen arbitrarily in the case that there are multiple minimizers.\footnote{This selection is possible thanks to the axiom of choice.} With this choice, $\log_{\mb x_\natural}(\mb q)$ is well-defined, single-valued over all of $\mc M$, and defines a mapping 
\begin{equation}
    \log_{\mb x_\natural} : \mc M \to B_{T_{\mb x_\natural} \mc M} \Bigl( \mb 0, \mr{diam}(\mc M) \Bigr ) 
\end{equation}
Our analysis will assume that the landmarks $Q$ are $\delta$-separated on $\mc M$, i.e., $d_{\mc M}(\mb q_i, \mb q_j) \ge \delta$ for all $i \ne j$. We will show under this assumption that $\log_{\mb x_\natural}(\mb q_i)$ and $\log_{\mb x_\natural}(\mb q_j)$ are  $\delta'$-separated, albeit with a radius of separation $\delta'$ which could be significantly smaller than $\delta$. 

This argument makes heavy use of properties of geodesic triangles -- in particular, Toponogov's theorem, which compares side lengths of geodesic triangles in spaces of {\em bounded} sectional curvature to side lengths of triangles in spaces of {\em constant} sectional curvature. Our argument uses the following properties of the mapping $\log$ defined above: 
\begin{itemize}
    \item {\em Inverse Property}: for $\mb v = \log_{\mb x_\natural}(\mb q)$, $\exp_{\mb x_\natural}(\mb v) = \mb q$
    \item {\em Minimum Norm Property}: $\| \log_{\mb x_\natural}(\mb q) \| = d_{\mc M}(\mb x_\natural, \mb q) \le \mr{diam}(\mc M)$.
\end{itemize}
Our analysis {\em does not} require analytical properties of the logarithmic map, such as continuity, which do not obtain beyond the injectivity radius of the exponential map. 


}



\begin{lemma}\label{lemma:tangent space seperation}
    For any $R > 0$, and any $\delta$-separated pair of points $\mb q, \mb q' \in \mb B_{\mc M}(\mb x_\natural, R)$ (i.e., pair of points satisfying $d_{\mc M}(\mb q,\mb q') \geq \delta$), we have 
    \begin{align}
        \norm{\log_{\target}\vq - \log_{\target}\vq'} \ge \frac{\sqrt{2}}{4}\exp(-\kappa R)\delta. 
    \end{align}
\end{lemma}


\begin{proof}
    We will prove this claim by applying Toponogov's theorem, a fundamental result in Riemannian geometry. Toponogov's theorem is a comparison theorem for triangles, which allows us to compare side lengths of geodesic triangles in an arbitrary manifold of bounded sectional curvature to the side lengths of geodesic triangles in a model space of {\em constant} sectional curvature. 
    From Lemma 10 in \cite{yan2023tpopt}, the sectional curvatures of $\mc M$ are bounded from below by the extrinsic curvature $\kappa$, i.e.,
    \begin{equation}
        \kappa_s \geq -\kappa^2.
    \end{equation}
    Our plan is as follows: form a geodesic triangle $\triangle(\mb x, \mb q_{-\kappa^2}, \mb q'_{-\kappa^2} )$ in the model $M_{-\kappa^2}$ with constant section curvature $-\kappa^2$, whose side lengths satisfy
    \begin{equation}
        d_{\mc M_{-\kappa^2}}(\mb x, \mb q_{-\kappa^2}) = d_{\mc M}(\mb x_\natural, \mb q), \qquad d_{\mc M_{-\kappa^2}}(\mb x, \mb q'_{-\kappa^2}) = d_{\mc M}(\mb x_\natural, \mb q'),
    \end{equation}
    and whose angle satisfies 
    \begin{equation}
    \angle(\mb q_{-\kappa^2},\mb x, \mb q'_{-\kappa^2}) = \angle(\mb q,\mb x_\natural, \mb q')
    \end{equation} 
    Then by Toponogov's theorem, the third sides of these pair of triangles satisfy the inequality 
    \begin{equation}
    d_{\mc M}(\mb q,\mb q') \le d_{\mc M_{-\kappa^2}}( \mb q_{-\kappa^2}, \mb q'_{-\kappa^2} ), \label{eqn:topo}
    \end{equation} 
    i.e., the third side in the constant curvature model space is larger than that in $\mc M$. 

    We construct the triangle $\triangle( \mb x, \mb q_{-\kappa^2}, \mb q'_{-\kappa^2} )$ more explicitly as follows: fix a arbitrary base point $\mb x \in  M_{-\kappa^2}$. Let $\mb v, \mb v'$ be two distinct tangent vectors in the tangent space $T_{\mb x}\mc M_{-\kappa^2}$ satisfying 
\begin{equation}
    \|\mb v\|_2 = \|\log_{\mb x_\natural} \mb q\|_2, \;\;
    \|\mb v'\|_2 = \|\log_{\mb x_\natural} \mb q'\|_2,
\end{equation}
and $\theta = \angle(\log_{\mb x_\natural} \mb q, \log_{\mb x_\natural} \mb q')=\angle(\mb v, \mb v')$. Set $\mb q_{-\kappa^2} = \exp_{\mb x}(\mb v) \in \mc M_{-\kappa^2}, \mb q_{-\kappa^2}' = \exp_{\mb x}(\mb v') \in \mc M_{-\kappa^2}$. 

Notice that $\| \mb v - \mb v' \| = \| \log_{\mb x_\natural} \mb q - \log_{\mb x_\natural} \mb q' \|$. We would like to {\em lower bound} this quantity. From \eqref{eqn:topo} and the fact that $d_{\mc M}(\mb q,\mb q') \ge \delta$, we have $d_{\mc M_{-\kappa^2}}( \mb q_{-\kappa^2}, \mb q'_{-\kappa^2} ) \ge \delta$, and the task becomes one of lower bounding $\| \mb v - \mb v' \|$ in terms of this quantity. To facilitate this bound, we move $\mc M_{-\kappa^2}$ to hyperbolic 
space $\mc M_{-1}$, where we can apply standard results from hyperbolic trigonometry, by scaling all side lengths by $\kappa$. Namely, form a third geodesic triangle in $\mc M_{-1}$, by taking an arbitrary $\mb x_{-1} \in \mc M_{-1}$, choosing $\mb v_{-1}, \mb v'_{-1} \in T_{\mb x_{-1}} \mc M_{-1} $ with $\angle( \mb v_{-1}, \mb v'_{-1} ) = \theta$ and $\| \mb v_{-1} \|_2 = \kappa \| \mb v \|$, $\| \mb v'_{1} \|_2 = \kappa \| \mb v' \|$. As above, set $\mb q_{-1} = \exp_{\mb x_{-1}}(\mb v_{-1})$, and $\mb q'_{-1} = \exp_{\mb x_{-1}}(\mb v'_{-1})$. Then 
\begin{equation}
    d_{\mc M_{-1}}(\mb q_{-1},\mb q'_{-1}) = \kappa d_{\mc M_{-\kappa^2}}( \mb q_{-\kappa^2}, \mb q'_{-\kappa^2} ). 
\end{equation}
Moreover, 
\begin{equation}
    \| \mb v_{-1} - \mb v'_{-1} \| = \kappa \| \mb v - \mb v' \|. 
\end{equation}


\noindent For compactness of notation, let $L'$ denote the third sidelength of $\triangle_{-1}$, 
\begin{equation}
    L' =  d_{\mc M_{-1}}(\mb q_{-1},\mb q'_{-1})  = \kappa *  d_{\mc M_{-\kappa^2}}(\mb q_{-\kappa^2}, \mb q_{-\kappa^2}').
\end{equation}
The lengths of the other two side are $a =\kappa* \|\mb v\|_2 \leq \kappa R, b=\kappa* \|\mb v'\|_2 \leq \kappa R$, and angle between these two sides is equal to $\theta$. In the corresponding Euclidean triangle on the tangent space, we also have the two sides are of length $a$ and $b$, and the third side has length
\begin{equation}
   L = \kappa * \|\mb v -\mb v'\|_2. 
\end{equation}


As $\sM_{-1}$ is hyperbolic, from hyperbolic law of cosines, we have
\begin{equation}
    \cosh L' = \cosh a \cosh b - \sinh a \sinh b \cos \theta.
\end{equation}
From the fact that $\cosh(a - b) = \cosh a \cosh b - \sinh a \sinh b$, we could further get
\begin{equation}
    \begin{aligned}
        \cosh L' = \cosh(a-b) + \sinh a \sinh b (1 - \cos \theta).
    \end{aligned}
\end{equation}
Since $\sinh t $ is convex over $t \in [0,\infty)$, we have $\sinh t \leq \frac{t}{\kappa R} \sinh (\kappa R)$ for $t \le \kappa R$, hence $\sinh a \sinh b \leq \frac{ab}{(\kappa R)^2} \sinh^2(\kappa R)$. And $\cosh (a-b) = \cosh|a-b| \leq \cosh L$. 
Then we have 
\begin{equation}
    \cosh L' \leq \cosh L + \frac{ab}{(\kappa R)^2} \sinh^2(\kappa R)(1 - \cos \theta)
\end{equation}

From the law of cosines applying on Euclidean triangle with length of two sides $a,b$ and the length of the third side $L$, we know
\begin{equation}\label{eq:relationship between L and L'}
    \begin{aligned}
        L^2 
        &= a^2 + b^2 - 2ab \cos \theta\\
        &\geq 2ab(1-\cos\theta)\\
        &\geq 2ab \frac{(\cosh L' - \cosh L)(\kappa R)^2}{ab\sinh^2(\kappa R)}\\
        &= 2(\kappa R)^2 \frac{(\cosh L' - \cosh L)}{\sinh^2(\kappa R)}
    \end{aligned}
\end{equation}
Since $d_{\mc M}(\mb q,\mb q')\geq \delta$, then $L' = \kappa *  d_{\mc M_{-\kappa^2}}(\mb q_{-\kappa^2}, \mb q_{-\kappa^2}') \geq \kappa * d_{\mc M}(\mb q,\mb q')\geq \kappa\delta$. Then equation \eqref{eq:relationship between L and L'} implies 
\begin{align}\label{eq:app:tangent_space_sep_L_bound_1}
    L^2 + \frac{2(\kappa R)^2}{\sinh^2(\kappa R)}\cosh L \geq \frac{2(\kappa R)^2}{\sinh^2(\kappa R)} \cosh \kappa\delta.
\end{align}
\vspace{.1in}\\
By triangle inequality, we know $L \leq a + b \leq 2\kappa R$. From the mean value form of the Taylor series, we have $\sinh(\kappa R) = \kappa R \cosh(r_1)$ for some $0 \le r_1 \le \kappa R$ and 
\begin{align}
    \cosh L &\le 1 + \frac{L^2}{2} \cosh(r_2)
\end{align}
for some $0 \le r_2 \le L$. Multiply equation \eqref{eq:app:tangent_space_sep_L_bound_1} both side by $\frac{\sinh^2(\kappa R)}{2(\kappa R)^2}$ and plug in the value above, we get
\begin{align}
\cosh \kappa\delta &\le \frac{\sinh^2(\kappa R)}{2(\kappa R)^2}L^2 + \cosh L \\
&\le \frac{\cosh^2(r_1)}{2}L^2 + \paren{1 + \frac{L^2}{2}\cosh(r_2)}.
\end{align}
Rearrange the terms, we get
\begin{align}
    L^2 &\ge 2\paren{\cosh^2(r_1) + \cosh(r_2)}^{-1} \paren{\cosh(\kappa \delta) - 1} \\
    &\ge 2\paren{\cosh^2(\kappa R) + \cosh(2\kappa R)}^{-1} \frac{\kappa^2 \delta^2}{2} \\
    &\ge \frac{1}{2} \exp(-2\kappa R) \kappa^2 \delta^2.\\
\end{align}
where from the first to second line we used that $\cosh(t) \ge 1 + \frac{t^2}{2}$. 
As a result, we have $\norm{\log_{\target}\vq - \log_{\target}\vq'} = \norm{\vv - \vv'} = L/\kappa \ge \frac{\sqrt{2}}{2}\exp(-\kappa R)\delta \ge \frac{\sqrt{2}}{4}\exp(-\kappa R)\delta$.
\end{proof}

\begin{lemma}\label{lemma:bound of number of landmarks within R ball}
    Consider $\mb x_\natural \in \mc M$ and let $\mc Q_{R} = \{\mb q: \mb q\in  B_{\mc M}(\mb x_\natural, R) \cap Q\}$. Then the number of landmarks within $R$ ball centering at $\mb x_\natural$, i.e. $|\mc Q_{R}|$, satisfies 
    \begin{equation}
    |\mc Q_{R}| 
    \leq \left( 1 + 4\sqrt{2} R \e^{\kappa R}/ \delta\right)^d, \forall R > 0.
    \end{equation}
    In particular, we have
    \begin{align}\label{eq:app:Q_size_bound}
        \abs{\sQ} \le \paren{1 + 4\sqrt{2}\delta^{-1} \diam(\sM) e^{\kappa \diam(\sM)} }^d 
    \end{align}
\end{lemma}
\begin{proof}
    For every $\mb q \in \mc Q_{R}$, there is a {\em unique} tangent vector $\log_{\mb x_\natural} \mb q \in T_{\mb x_\natural} \mc M$. Now we define the set $\mc V_{R} = \{\log_{\mb x_\natural} \mb q \in T_{\mb x_\natural} \mc M \quad \forall \mb q \in \mc Q_{R}\}$. Then the number of landmarks within the intrinsic ball $\mb B_{\mc M}(\mb x_\natural, R)$ is $|\mc Q_{R}| = |\mc V_{R}|$.
    \vspace{.1in}\\
    Let $\delta_R = \frac{\sqrt{2}}{4}\exp(-\kappa R)\delta$. From Lemma \ref{lemma:tangent space seperation}, we know that $\mc V_{R}$ forms a $\delta_R$-separated subset of $B(0,R)$ on $T_{\mb x_\natural} \mc M$. And for any $\log_{\mb x_\natural}\mb q\in \mc V_{R}$, we have $\left\| \log_{\mb x_\natural}\mb q\right\|_2 = d_{\mc M}(\mb x_\natural, \mb q) \leq R$. Then it's natural to notice that $|\mc V_{R}| \leq P(B(0,R), \delta_R)$, where $P(B(0,R), \delta_R)$ is the largest cardinality of a $\delta_R$-separated subset of $B(0,R) \in T_{\mb x_\natural} \mc M$.
    \vspace{.1in}\\
    Since $P(B(0,R),\delta_R)$ is the largest number of closed disjoint balls with centers in $B(0,R)$ and radii $\delta_R$/2, then by volume comparison, we have
    \begin{equation}
         P(B(0,R),\delta_R) * \text{vol}(B_{\delta_R/2}) \leq \text{vol}(B_{\delta_R/2+R})
    \end{equation}
Then we will have $|\mc V_{R}| \leq P(B(0,R), \delta_R) \leq \left( 1+\frac{2R}{\delta_R}\right)^d$ which gives the bound we need. 

To bound $\sQ$, we can simply take $R = \diam(\sM)$ and notice $\abs{\sQ} =  \abs{\sQ_{\diam(\sM)}}$. 
\end{proof}



\begin{lemma}\label{lem:app:foe_bound}

    Let $\mc M \in \mathbb{R}^D$ be a complete $d$-dimensional manifold. Suppose the set of landmarks $Q = \{\mb q\} \subset \mc M$ forms a $\delta$-net for $\mc M$, and first -order edges $E^1$ satisfies that $u \overset{1}{\rightarrow} v \in E^1$ if $\|\mb q_u - \mb q_v\|_2 \leq R_{\text{nbrs}}$, where $R_{\text{nbrs}} \leq \tau_{\mc M}$, and $\tau_{\mc M}$ is the reach of the manifold. Assume $\delta \leq \diam{\mc M}$, and  $\kappa \diam(\sM) \ge 1$. Then the number of first-order edges $\abs{E^1}$ satisfies
    \begin{align}
        \log \abs{E^1} &\le \paren{\kappa \diam(\sM) + \log(a) - \log(\delta) + \log(\diam(\sM)) + \log(100)}d. 
    \end{align}
\end{lemma}
\begin{proof}
From construction, we have 
\begin{align}
    \abs{E^1} \le \abs{Q} \max_{\vq_u \in \sQ}\abs{E_{u}^1},
\end{align} 
where $E^1_u$ denotes the first-order edges at landmark $\mb q_u$.
As we have $\firstER \le \reach \le 1/\kappa$, following \cref{lemma:bound of number of landmarks within R ball} and \cref{eq:app:Q_size_bound}, we get
\begin{align}
    \abs{E^1} 
    &\le \paren{1 + 4\sqrt{2}\delta^{-1} \firstER e^{\kappa \firstER} }^d \paren{1 + 4\sqrt{2}\delta^{-1} \diam(\sM) e^{\kappa \diam(\sM)} }^d \\
    &= \paren{1 + 4\sqrt{2}a e^{\kappa \firstER} }^d \paren{1 + 4\sqrt{2}\delta^{-1} \diam(\sM) e^{\kappa \diam(\sM)} }^d 
\end{align}



We recall that $a = \firstER / \delta \ge 40$. Since $e^{\kappa \firstER} \ge 1$, we have $\frac{1}{40} a e^{\kappa \firstER} \ge 1$, which means  $1 + 4\sqrt{2}a e^{\kappa \firstER} \le (\frac{1}{40} + 4\sqrt{2})a e^{\kappa \firstER} \le (\frac{1}{40} + 4\sqrt{2})a\cdot e$, since $\kappa \firstER \le 1$. Similarly, since $\delta \leq \diam{\mc M}$, and $\kappa \diam(\sM) \ge 1$, $1 + 4\sqrt{2}\delta^{-1} \diam(\sM) e^{\kappa \diam(\sM)} \le (\frac{1}{e} + 4\sqrt{2})\delta^{-1} \diam(\sM) e^{\kappa \diam(\sM)}$, which gives 
\begin{align}
    \abs{E^1} 
    &\le\paren{(\frac{e}{40} + 4\sqrt{2}e)a}^d \paren{(\frac{1}{e} + 4\sqrt{2})\delta^{-1} \diam(\sM) e^{\kappa \diam(\sM)}}^d \\ 
    &\le\paren{(\frac{e}{40} + 4\sqrt{2}e)(\frac{1}{e} + 4\sqrt{2})}^d \paren{a\delta^{-1} \diam(\sM) e^{\kappa \diam(\sM)}}^d \\
    &\le\paren{100}^d \paren{a\delta^{-1} \diam(\sM) e^{\kappa \diam(\sM)}}^d
\end{align}

Taking the log,  we get
\begin{align}
    \log \abs{E^1} &  \le \paren{\kappa \diam(\sM) + \log(a) - \log(\delta) + \log(\diam(\sM)) + \log(100)}d.    
\end{align}  


\end{proof}





{
\begin{lemma}\label{lem:upper bound for Tmax} There exists a numerical constant $C$ such that with probabilty at least $1 - e^{-9d/2}$, 
    \begin{equation}
    T_{\max} = \sup \, \Bigl\{ \, \innerprod{ \mb v }{\mb z } \, \mid  \begin{array}{l} d_{\mc M}(\mb y, \mb x_\natural) \le 1/\kappa, \\ \mb v \in T_{\mb y} \mc M, \, \| \mb v \|_2 = 1 \end{array} \Bigr\}. 
\end{equation}
satisfies 
\begin{equation}
    T_{\max} 
    \leq C \max\{ \kappa, 1 \} \times \sigma \sqrt{d}.
\end{equation}
\end{lemma}

\begin{proof}
Let $\bar{\kappa} = \max \{ \kappa, 1 \}$. From Theorem 10 of \cite{yan2023tpopt}, we have that  with probability at least $1-e^{-\frac{x^{2}}{2\sigma^{2}}}$,  
\begin{equation}
    T_{\max} \le 12\sigma\left(\bar{\kappa} \sqrt{2\pi(d+1)}+\sqrt{\log12\bar{\kappa}}\right)+x. 
\end{equation}
We note that there exists numerical constants $C_1, C_2$, such that $\sqrt{\log 12\bar{\kappa}} \leq C_1 \bar{\kappa}$, and $\sqrt{d+1}\leq C_2 \sqrt{d}$. Setting $x = 3\sigma\sqrt{d}$, then we have
\begin{equation}
    \begin{aligned}                         
       12\sigma(\bar{\kappa}\sqrt{2\pi(d+1)}+\sqrt{\log12\bar{\kappa}})+x
        &\leq 12\sigma\bar{\kappa}\sqrt{2\pi} C_2 \sqrt{d} + \sigma C_1 \bar{\kappa} + 3\sigma\sqrt{d}\\
        &=\left( 12\bar{\kappa}\sqrt{2\pi} C_2 + C_1 \bar{\kappa} +3\right) \sigma \sqrt{d}
    \end{aligned},
\end{equation}
yielding the claimed bound. 

\end{proof}
}




{
\begin{lemma}\label{lem:lower bound for Tmax}
    Let
    \begin{equation}
    T_{\max} = \sup \, \Bigl\{ \, \innerprod{ \mb v }{\mb z } \, \mid  \begin{array}{l} d_{\mc M}(\mb y, \mb x_\natural) \le 1/\kappa, \\ \mb v \in T_{\mb y} \mc M, \, \| \mb v \|_2 = 1 \end{array} \Bigr\}. 
\end{equation}
Then with probability at least $1 - e^{-\frac{t^2}{2\sigma^2}}$, we have 
\begin{align}
    T_{\max} \ge \sigma \sqrt{d/2} - t. 
\end{align}
\end{lemma}
}

\begin{proof}
Since $T_{\max}$ is the supremum of a $1$-Lipschitz function and is therefore $1$-Lipschitz in $\mb{z}$, it follows that
\begin{equation}\label{eq:tmax_lb_lip}
    \prob{T_{\max} \leq \bb E[T_{\max}] - t} \le e^{-t^2/ 2 \sigma^2}.
\end{equation}
By setting $\vy = \target$ and $\vv = \sP_{\tangent{\target}{\sM}}\vz$ we obtain \(
    T_{\max} \geq \left\| \sP_{T_{\mb x_\natural} \mc M} \mb z\right\|_2 \) and thus \(\bb E[T_{\max}] \geq \bb E\left[\left\| \sP_{T_{\mb x_\natural} \mc M} \mb z\right\|_2\right]\). 

Since $\vz$ is $\iid$ Gaussian with variance $\sigma^2$, the rotational invariance of Gaussian distributions implies that $\E{\norm{\sP_{T_{\mb x_\natural} \mc M} \mb z}} = \E{\norm{\sigma \vg_d}}$, where $\vg_d \sim \sN(0, \vI_d)$ is a $d$ dimensional standard Gaussian vector. 
As $\norm{\vg_d}$ is $1$-Lipschitz, from \cref{lem:app:lipschitz_bounded_variance} we have
\begin{align}
    1 \ge \Var{\norm{\vg_d}} = \E{\norm{\vg_d}^2} - \E{\norm{\vg_d}}^2 = d - \E{\norm{\vg_d}}^2
\end{align}
and thus $\E{\norm{\vg_d}} \ge \sqrt{d - 1}$. 
Therefore, 
\begin{align}
    \E{\norm{\sP_{T_{\mb x_\natural} \mc M} \mb z}} = \E{\sigma \norm{\vg_d}} \ge \sigma \sqrt{d - 1} \ge \sigma \sqrt{d/2} , \quad \text{for } d \geq 2. 
\end{align}

For the case where $d = 1$, we compute directly:
\begin{align}
    \E{\norm{\sP_{T_{\target} \sM}}} &= \sqrt{2} \sigma \frac{\Gamma((d+1)/2)}{\Gamma(d/2)} = \sqrt{2/\pi} \sigma \ge \sqrt{d/2} \sigma. 
\end{align}
Combining the cases for $d \geq 2$ and $d = 1$, and substituting into \cref{eq:tmax_lb_lip}, we conclude that
\begin{equation}
        T_{\max} \geq \sigma \sqrt{d / 2} - t \quad \text{with probability at least } 1 - e^{-t^2 / 2\sigma^2}.
    \end{equation}
\end{proof}

\begin{lemma}[Bounded Variance of $1$-Lipschitz Function]\label{lem:app:lipschitz_bounded_variance}
    Let $\vg \sim \sN(\mb 0, \vI)$ be a standard Gaussian, $f$ is a $1$-Lipschitz function, then we have
    \begin{align}
        \Var{f(\vg)} \le 1.  
    \end{align}
\end{lemma}
\begin{proof}
    In the prove, we utilize the Gaussian Poincar\'e inequality \cite{boucheron2003concentration}[Theorem 3.20] which says that
\begin{align}
    \Var{h(\vg)} \le \E{\norm{\nabla h(\vg)}^2}
\end{align}
for any $C^1$ function $h$. Let $\rho_{\eps}$ be the standard Gaussian mollifier \(\rho_{\eps}(\vz) = \frac{1}{(2\pi\eps)^d}e^{-\frac{\norm{\vz}^2}{2\eps}}\)
and let \[f_{\eps} = f \ast \rho_{\eps} = \int_{\vz} f(\vx - \vx)\rho_{\eps}(\vz) d\vz,\]then $f_{\eps}$ is smooth. As
\begin{align}
    \abs{f_{\eps}(\vx) - f_{\eps}(\vy)} &= \abs{\int_{\vz} \paren{f_{\eps}(\vx - \vz) - f_{\eps}(\vy - \vz)}\rho_{\eps}(\vz)d\vz} \\
    &\le \int_{\vz} \abs{\paren{f_{\eps}(\vx - \vz) - f_{\eps}(\vy - \vz)}}\rho_{\eps}(\vz)d\vz\\
    &\le \norm{\vx - \vy} \int_{\vz}\rho_{\eps}(\vz)d\vz = \norm{\vx - \vy},
\end{align}
$f_{\eps}$ is also $1$-Lipschitz. 
Following the  Gaussian Poincar\'e inequality we have
\begin{align}
    \Var{f_{\eps}(\vg)} &\le \E{\norm{\nabla f_{\eps}(\vg)}^2} \le 1.
\end{align}
To conclude the result, we need to show the interchangeability of the interation and the limit. As $f_{\eps}$ is $1$-Lipschitz, we have 
\begin{align}
    \abs{f_{\eps}(\vx)}^2 &\le \norm{f_{\eps}(0)} + \norm{\vx} \\
    &\le \abs{\int_{\vz} f(-\vz) \rho_{\eps}(\vz)d\vz} + \norm{\vx}\\
    &\le \int_{\vz} \norm{\vz}\rho_{\eps}(\vz)d\vz + \norm{\vx} \\
    &= \sqrt{\eps} \E{\norm{\vg}} + \norm{\vx}
\end{align}
As the moments of a standard Gaussian are upper bounded, $\abs{f_{\eps}(\vg) - \E{f_{\eps}(\vg)}}^2$ can be uniformly upper bounded by some integrable function for all $\eps \le 1$. And thus
\begin{align}
    \Var{f[\vg]} &= \int_{\eps \to 0} \Var{f_{\eps}(\vg)} \le 1. 
\end{align}
\end{proof}

\newpage 












\section{Experimental Details}

\subsection{Gravitational Waves Data Generation} \label{sec:data_generation}

We generate synthetic gravitational waveforms with the PyCBC package \cite{nitz2023gwastro} with masses drawn from a Gaussian distribution with mean 35 and variance 15. We use rejection sampling to limit masses to the range [20, 50]. Each waveform is sampled at 2048Hz, padded or truncated to 1 second, and normalized to have unit $\ell^2$ norm. We simulate noise as i.i.d.\ Gaussian with standard deviation $\sigma = 0.01$ \cite{yan2023tpopt}. The training set consists of 100,000 noisy waveforms, the test set contains 5,000 noisy waveforms.





\section{Traversal Networks on Synthetic Manifolds} \label{sec:manifold_vis}

In this section, we present traversal networks created by \cref{alg:mtn-growth} on the following synthetic manifolds: sphere (Figure \ref{fig:sphere_visual}), torus (Figure \ref{fig:torus_visual}).

\begin{figure}[htbp]
    \centering
    \scalebox{0.8}{  % Scale to 80% of original size
    % First row
    \begin{minipage}[t]{0.2\textwidth}
        \centering
        \includegraphics[width=\textwidth]{figs/synthetic_mfs/torus/torus_clean.png}
    \end{minipage}%
    \hfill%
    \begin{minipage}[t]{0.2\textwidth}
        \centering
        \includegraphics[width=\textwidth]{figs/synthetic_mfs/torus/torus_LM.png}
    \end{minipage}%
    \hfill%
    \begin{minipage}[t]{0.2\textwidth}
        \centering
        \includegraphics[width=\textwidth]{figs/synthetic_mfs/torus/torus_FOE.png}
    \end{minipage}
    
    \vspace{1em}
    % Second row
    \begin{minipage}[t]{0.2\textwidth}
        \centering
        \includegraphics[width=\textwidth]{figs/synthetic_mfs/torus/torus_ZOE.png}
    \end{minipage}%
    \hfill%
    \begin{minipage}[t]{0.2\textwidth}
        \centering
        \includegraphics[width=\textwidth]{figs/synthetic_mfs/torus/torus_network.png}
    \end{minipage}%
    \hfill%
    \begin{minipage}[t]{0.2\textwidth}
        \centering
        \includegraphics[width=\textwidth]{figs/synthetic_mfs/torus/torus_all.png}
    \end{minipage}
    }
    
    \caption{Traversal network on the torus with noise level $\sigma=0.01$. Left-to-Right, Top-to-Bottom: Clean manifold, landmarks (blue dots), first-order edges (blue lines), zero-order edges (green lines), final traversal network, and final traversal network overlayed with clean manifold.}
    \label{fig:torus_visual}
\end{figure}



\begin{figure}[htbp]
    \centering
    \scalebox{0.8}{
    % First row
    \begin{minipage}[t]{0.2\textwidth}
        \centering
        \includegraphics[width=\textwidth]{figs/synthetic_mfs/sphere/sphere_clean.png}
    \end{minipage}%
    \hfill%
    \begin{minipage}[t]{0.2\textwidth}
        \centering
        \includegraphics[width=\textwidth]{figs/synthetic_mfs/sphere/sphere_LM.png}
    \end{minipage}%
    \hfill%
    \begin{minipage}[t]{0.2\textwidth}
        \centering
        \includegraphics[width=\textwidth]{figs/synthetic_mfs/sphere/sphere_FOE.png}
    \end{minipage}
    \vspace{1em}
    % Second row
    \begin{minipage}[t]{0.2\textwidth}
        \centering
        \includegraphics[width=\textwidth]{figs/synthetic_mfs/sphere/sphere_ZOE.png}
    \end{minipage}%
    \hfill%
    \begin{minipage}[t]{0.2\textwidth}
        \centering
        \includegraphics[width=\textwidth]{figs/synthetic_mfs/sphere/sphere_network.png}
    \end{minipage}%
    \hfill%
    \begin{minipage}[t]{0.2\textwidth}
        \centering
        \includegraphics[width=\textwidth]{figs/synthetic_mfs/sphere/sphere_all.png}
    \end{minipage}
    }
    
    \caption{Traversal network on the sphere created based on 100,000 noisy points with noise level $\sigma=0.01$. Left-to-Right, Top-to-Bottom: Clean manifold, landmarks (blue dots), first-order edges (blue lines), zero-order edges (green lines), final traversal network, and final traversal network overlayed with clean manifold.}
    \label{fig:sphere_visual}
\end{figure}















\section{Incremental PCA for Efficient Tangent Space Approximation} \label{sec:ipca_description}

\begin{algorithm}[tb]
      \caption{$\mathtt{IncrPCAonMatrix}(\mb X, d)$} \label{alg:TISVD_X}
      \begin{algorithmic}[1]
        \STATE \textbf{Input:} $\mb X = [\mb x_1, \dots, \mb x_n] \in \R^{D \times n}$ as a collection of points, $d$ as intrinsic dimension.
        \STATE $\mb U_1 \leftarrow [\mb x_1 / \| \mb x_1 \|]$
        \STATE $\mb S_1 \leftarrow [\|\mb x_1\|^2_2]$
        \FOR{i = 1, \dots, n-1}
        \STATE $\mb U_{i+1}, \mb S_{i+1} \leftarrow \mathtt{IncrPCA}(\mb x_{i+1}, \mb U_{i}, \mb S_{i}, i+1, d)$
    \ENDFOR
    \STATE \textbf{Output:} $\mb U_n, \mb S_n$
    \end{algorithmic}
\end{algorithm}

In this section, we detail the tangent space approximation implementation mentioned in \cref{alg:mtn-growth} and detailed in \cref{alg:TISVD_X} and \cref{alg:TISVD_one_point)}. We use incremental Principal Component Analysis (PCA) to efficiently process streaming high-dimensional data. Below we present the mathematics and algorithmic details of our implementation.

\paragraph{Initializing Local Model Parameters at New Landmarks:} If a newly created landmark $\mb q_M$ in \cref{alg:mtn-growth} has no other landmarks within $R_\mr{nbrs}$ distance, then its tangent space $T_{\mb q_M}$ is initialized randomly. Otherwise, we establish first-order connections to all existing landmarks within radius $R_\mr{nbrs}$, and the local parameters $T_{\mb q_M}$ and $\Xi_{\mb q_M}$ are then initialized in the following ways.



Let $ \left \{\mb q_i \right \}_{i=1}^k$ denote the set of first-order neighbors of landmark $\mb q_M$. We compute the normalized difference vectors:
\begin{align}
    \mb h_i = \frac{\mb q_i - \mb q_M}{\|\mb q_i - \mb q_M\|}
\end{align}
\noindent and assemble them into a matrix $\mb H = \left[ \begin{array}{cccc} \mb h_1 & \mb h_2 & \dots & \mb h_k \end{array} \right]$. The tangent space $T_{\mb q_M}$ is spanned by the orthonormal matrix $\mb U_{\mb q_M}$ obtained through truncated singular value decomposition of $\mb H$, which ensures $\mb U_{\mb q_M} \in \R^{D \times d}$, where $D$ and $d$ represent ambient and intrinsic dimensions, respectively. Edge embeddings $\Xi_{\mb q_M}$ are then created via projecting difference vectors $\mb q_i - \mb q_M$ onto $T_{\mb q_M}$.

\paragraph{Updating Tangent Space Approximations Efficiently:} Now that a new landmark $\mb q_M$ has been created along with $T_{\mb q_M}$ and $\Xi_{\mb q_M}$, we must update all three of them as more points arrive within radius $R(i)$ of $\mb q_M$. As \cref{alg:mtn-growth} proceeds, each new point $\mb x_{n+1}$ that appears within radius $R(i)$ of $\mb q_M$ is used to update the local parameters at vertex $M$. To approximate the local parameters at $\mb q_M$, we could consider the $n$ noisy points $\{\mb x_i\}_{i=1}^n$ which already lie within radius $R(i)$ of landmark $\mb q_M$, and local parameters $T_{\mb q_M}$ and $\Xi_{\mb q_M}$ can be established using these points, with $T_{\mb q_M} = \mr{span}(\mb U_n)$ where $\mb U_n \in \R^{D \times d}$. Landmark $\mb q_M$ is updated to be the average of all $n+1$ points. A straightforward way to approximate $T_{\mb q_M}$ would be to form $\mb X_{n+1} = \left[ \begin{array}{cccc} \mb x_1 & \mb x_2 & \dots & \mb x_{n+1} \end{array} \right]$ and simply let $\mb U_{n+1}, \mb S_{n+1}, \mb V_{n+1}^\ast \leftarrow \texttt{svd} \left(\mb X_{n+1} \right)$, and $ T_{\mb q_M} = \mr{span}(\mb U_{n+1})$. However, this presents a computational challenge, given dimensions of matrix $\mb X_{n+1}$ and computational complexity of SVD. Moreover, performing SVD on the entire set of points within $R(i)$ of $\mb q_M$ every time a new point is seen would be computationally redundant. This is why we implement tangent space estimation updates using the incremental PCA \cite{brand2006fast, arora2012stochastic}, detailed below.

Let vertex $M$ have local parameters $\mb q_M, T_{\mb q_M}, \Xi_{\mb q_M}$ with $T_{\mb q_M} = \mr{span}(\mb U_n)$. Let $\mb X_n = \left[ \begin{array}{cccc} \mb x_1 & \mb x_2 & \dots & \mb x_n \end{array} \right] \in \R^{D \times n}$, which is expanded by a new sample $\mb x_{n+1}$ to form the matrix $\mb X_{n+1} = \left[ \begin{array}{cc} \mb X_n & \mb x_{n+1}
\end{array} \right]$. Assume that we have the truncated singular value decomposition $\mb X_n \approx \mb U_n \mb S_n \mb V_n^T$, with orthonormal $\mb U_n \in \R^{D \times d}, \mb V_n \in \R^{n \times d}$, and diagonal $\mb S_n \in \R^{d \times d}$, with $\mb U_n$ spanning the tangent space at $\mb q_M$ prior the arrival of $\mb x_{n+1}$. Our goal is to compute the truncated SVD of $\mb X_{n+1} \approx \mb U_{n+1} \mb S_{n+1} \mb V_{n+1}^T$, and to do so \emph{efficiently}. We represent matrices $\mb U_{n+1}, \mb S_{n+1}, \mb V_{n+1}$ in terms of $\mb U_n, \mb S_n, \mb V_n$. 
\begin{eqnarray}
    \mb X_{n+1} &=& 
    \left[ \begin{array}{cc} \mb X_n & \mb x_{n+1} \end{array} \right] \\
    &=& {\left[ \begin{array}{cc} \mb X_n & \mb 0 \end{array} \right]} + {\mb x_{n+1}} \left[ \begin{array}{cc} \mb 0^T & 1 \end{array} \right] \\
    &=& {\left[ \begin{array}{cc} \mb U_n \mb S_n \mb V^T_n & \mb 0 \end{array} \right]} + {\mb x_{n+1}} {\left[ \begin{array}{cc} \mb 0^T & 1 \end{array} \right]} \label{eqn:sum1} \\
    &=& \mb U_{n+1} \mb S_{n+1} \mb V^T_{n+1}
\end{eqnarray}

for matrices $\mb U_{n+1} \in \R^{D \times d}, ~~ \mb S_{n+1} \in \R^{d \times d}$ and $ \mb V_{n+1} \in \R^{d \times (n+1)}$. Thus, finding the SVD of $\mb X_{n+1}$ is equivalent to finding the SVD of the sum in \eqref{eqn:sum1}.




\begin{algorithm}[tb]
      \caption{$\mathtt{IncrPCA}(\mb x_{i+1}, \mb U_{i}, \mb S_{i}, i+1, d)$}\label{alg:TISVD_one_point)}
      \begin{algorithmic}[1]
      \STATE \textbf{Inputs:} $\mb x_{i+1}$, $\mb U_{i}, \mb S_{i}$, $i+1$, $d$
      \STATE $\mb S_{\text{exp}} \leftarrow \left[  \begin{array}{cc}
       \mb S_i & \mb 0  \\
        \mb 0^T   &  0
      \end{array}\right]$
      \STATE $\mb x_{i+1}^\perp = \mb x_{i+1} - \mb U_i \mb U_i^T \mb x_{i+1}$
      \STATE $\mb K_{i+1} = \mb S_{exp} + \left[ \begin{array}{c} \mb U^T_i \mb x_{i+1} \\ \left\| \mb x_{i+1}^\perp \right\| \end{array} \right] \left[ \begin{array}{c} \mb U^T_i \mb x_{i+1} \\ \left\| \mb x_{i+1}^\perp \right\| \end{array} \right] ^T$
    \STATE $\mb U_{K_{i+1}}, \mb S_{K_{i+1}} \leftarrow \mathrm{svd}(\mb K_{i+1})$
    \STATE $\mb U_{i+1} \leftarrow \left[\begin{array}{cc} \mb U_i  & \frac{\mb x_{i+1}^\perp}{\left \| \mb x_{i+1}^\perp \right\|} \end{array} \right] \mb U_{\mb K_{i+1}}$
    \STATE $\mb S_{i+1} \leftarrow \mb S_{\mb K_{i+1}}$
    \IF{$i \geq d$}
        \STATE $\mb U_{i+1} \leftarrow  \mb U_{i+1}[:, :d]$
        \STATE $\mb S_{i+1} \leftarrow  \mb S_{i+1}[:d, :d]$
    \ENDIF

    \STATE \textbf{Output:} $\mb U_{i+1}, \mb S_{i+1}$
    \end{algorithmic}
\end{algorithm}


We then define the vector  $\mb b \in \R^{(n+1) \times 1}$ and the expand the matrix $\mb V_n$ to be

\begin{eqnarray}
    \mb b = \left[ \begin{array}{c} \mb 0 \\ 1 \end{array} \right] \in \R^{(n+1) \times 1}, \quad \mb V_{\text{exp}} = \left[ \begin{array}{c} \mb V_n \\ \mb 0^T \end{array}
\right] \in \R^{(n + 1) \times d}
\end{eqnarray}

and rewrite $\eqref{eqn:sum1}$ to be


\begin{eqnarray}
    \mb X_{n+1} &=& \mb U_n {\mb S_n} {\mb V_{\text{exp}}^T} + {\mb x_{n+1}} {\mb b^T} \\
    &=& \left[ \begin{array}{cc} {\mb U_n} & {\mb x_{n+1}} \end{array} \right] {\left[ \begin{array}{cc}
    \mb S_n & \mb 0  \\
    \mb 0^T     &  1
    \end{array} \right]} {\left[ \begin{array}{cc} \mb V_{\text{exp}} & \mb b \end{array} \right]^T}. \label{eqn:matrix_mult}
\end{eqnarray}

We now consider the first and the last matrices in the product above. Note that for a given point $\mb x_n$, we have  $\mb x_{n+1}^\perp = (\mb I - \mb U_n \mb U_n^T) \mb x_{n+1}$.

\begin{eqnarray}
    \left[ \begin{array}{cc} \mb U_n & \mb x_{n+1} \end{array} \right] &=& \left[ \begin{array}{cc} \mb U_n & \mb U_n \mb U^T_n \mb x_{n+1} + \underbrace{\mb x_{n+1} - \mb U_n \mb U^T_n \mb x_{n+1}}_\text{$\mb x^\perp_{n+1}$} \end{array} \right] \\
    &=& \left[ \begin{array}{cc} \mb U_n & \mb U_n \mb U^T_n \mb x_{n+1} + \mb x^\perp_{n+1} \end{array} \right] \\
    &=& \left[ \begin{array}{cc} \mb U_n & \frac{\mb x_{n+1}^\perp}{\left \|\mb x_{n+1}^\perp \right \|} \end{array} \right] \left[ \begin{array}{cc} \mb I & \mb U_n^T \mb x_{n+1} \\
    \mb 0^T & \left\| \mb x^\perp_{n+1} \right|\|  \end{array} \right]
\end{eqnarray}

Similarly,

\begin{eqnarray}
    \left[ \begin{array}{cc} \mb V_{\text{exp}} & \mb b \end{array} \right] &=& \left[ \begin{array}{cc} \mb V_{\text{exp}} & \frac{\mb b^\perp}{\left \|\mb b^\perp \right \|} \end{array} \right] \left[ \begin{array}{cc} \mb I & \mb V_{\text{exp}}^T \mb b \\
    \mb 0^T & \left\| \mb b^\perp \right \|  \end{array} \right]
\end{eqnarray}
Putting these together, \eqref{eqn:matrix_mult} becomes
\begin{eqnarray}
    \mb X_{n+1} &=& \left[ \begin{array}{cc} \mb U_n & \frac{\mb x_{n+1}^\perp}{\left \|\mb x_{n+1}^\perp \right \|} \end{array} \right] \left[ \begin{array}{cc} \mb I & \mb U_n^T \mb x_{n+1} \\
    \mb 0^T & \left\| \mb x^\perp_{n+1} \right|\|  \end{array} \right] \left[ \begin{array}{cc}
    \mb S_n & \mb 0  \\
    \mb 0^T     &  1
    \end{array} \right] \Bigg( \left[ \begin{array}{cc} \mb V_{\text{exp}} & \frac{\mb b^\perp}{\left \|\mb b^\perp \right \|} \end{array} \right] \left[ \begin{array}{cc} \mb I & \mb V_{\text{exp}}^T \mb b \\
    \mb 0^T & \left\| \mb b^\perp \right \|  \end{array} \right] \Bigg)^T \\
    &=& \left[ \begin{array}{cc} \mb U_n & \frac{\mb x_{n+1}^\perp}{\left \|\mb x_{n+1}^\perp \right \|} \end{array} \right] \underbrace{ \left[ \begin{array}{cc} \mb I & \mb U_n^T \mb x_{n+1} \\
    \mb 0^T & \left\| \mb x^\perp_{n+1} \right|\|  \end{array} \right] \left[ \begin{array}{cc}
    \mb S_n & \mb 0  \\
    \mb 0^T     &  1
    \end{array} \right]  \left[ \begin{array}{cc} \mb I & \mb 0  \\
    \mb b^T \mb V_{\text{exp}} & \left\| \mb b^\perp \right \|  \end{array} \right] }_\text{$\mb K$} \left[ \begin{array}{cc} \mb V_{\text{exp}} & \frac{\mb b^\perp}{\left \|\mb b^\perp \right \|} \end{array} \right]^T \\
    &=& \left[ \begin{array}{cc} \mb U_n & \frac{\mb x_{n+1}^\perp}{\left \|\mb x_{n+1}^\perp \right \|} \end{array} \right] \mb K \left[ \begin{array}{cc} \mb V_{\text{exp}} & \frac{\mb b^\perp}{\left \|\mb b^\perp \right \|} \end{array} \right]^T \label{eqn:X_with_K}
\end{eqnarray}
where 
\begin{eqnarray}
    \mb K &=& \left[ \begin{array}{cc} \mb I & {\mb U_n^T \mb x_{n+1}} \\
    \mb 0^T & \left\| \mb x^\perp_{n+1} \right|\|  \end{array} \right] {\left[ \begin{array}{cc}
    {\mb S_n} & {\mb 0}  \\
    {\mb 0^T} &  {1}
    \end{array} \right]} \left[ \begin{array}{cc} \mb I & \mb 0  \\
    {\mb b^T \mb V_{\text{exp}}} & \left\| \mb b^\perp \right \|  \end{array} \right] \label{eqn:genera_K}
\end{eqnarray}
This is a general form for matrix $
\mb K$. We can further simplify it via
\begin{eqnarray}
    \mb K &=& \left[ \begin{array}{cc} \mb S_n & \mb U_n^T \mb x_{n+1} \\
    \mb 0^T & \left\| \mb x^\perp_{n+1} \right|\|  \end{array} \right] \left[ \begin{array}{cc} \mb I & \mb 0  \\
    \mb b^T \mb V_{\text{exp}} & \left\| \mb b^\perp \right \|  \end{array} \right] \\
    &=& \left[ \begin{array}{cc} \mb S_n + \mb U_n^T \mb x_{n+1} \mb b^T \mb V_{\text{exp}} & \mb U_n^T \mb x_{n+1} \left\| \mb b^\perp \right \| \\
    \left\| \mb x^\perp_{n+1} \right\| \mb b^T \mb V_{\text{exp}} & \left\| \mb x^\perp_{n+1} \right\| \left\| \mb b^\perp \right \|  \end{array} \right] \\
    &=& \left[ \begin{array}{cc} \mb S_n & \mb 0 \\
    \mb 0^T & 0  \end{array} \right] + \left[ \begin{array}{cc} \mb U_n^T \mb x_{n+1} \mb b^T \mb V_{\text{exp}} & \mb U_n^T \mb x_{n+1} \left\| \mb b^\perp \right \| \\
    \left\| \mb x^\perp_{n+1} \right\| \mb b^T \mb V_{\text{exp}} & \left\| \mb x^\perp_{n+1} \right\| \left\| \mb b^\perp \right \|  \end{array} \right] \\
    &=& \left[ \begin{array}{cc} \mb S_n & \mb 0 \\
    \mb 0^T & 0  \end{array} \right] + \left[ \begin{array}{c} \mb U_n^T \mb x_{n+1} \\ \left\| \mb x^\perp_{n+1} \right\|  \end{array} \right] \left[ \begin{array}{cc} \mb b^T  \mb V_{\text{exp}} & \left\| \mb b^\perp \right\|  \end{array} \right] \\
    &=& \left[ \begin{array}{cc} \mb S_n & \mb 0 \\
    \mb 0^T & 0  \end{array} \right] + \left[ \begin{array}{c} \mb U_n^T \mb x_{n+1} \\ \left\| \mb x^\perp_{n+1} \right\|  \end{array} \right] \left[ \begin{array}{c}  \mb V_{\text{exp}}^T \mb b  \\ \left\| \mb b^\perp \right\|  \end{array} \right]^T \label{eqn:sparse_K}
\end{eqnarray}
Equation \eqref{eqn:sparse_K} is another general form for the matrix $\mb K$. Note that $\mb K$ is highly structured and sparse\cite{brand2006fast}. Since it is of size $(d +1) \times (d + 1)$, the $\mb U_K, \mb S_K, \mb V_K \leftarrow \texttt{svd}(\mb K)$ will merely cost $\mathcal{O} (d^3)$.
Finally, we rewrite \eqref{eqn:X_with_K} as
\begin{eqnarray}
    \mb X_{n+1} &=& \left[ \begin{array}{cc} \mb U_n & \frac{\mb x_{n+1}^\perp}{\left \|\mb x_{n+1}^\perp \right \|} \end{array} \right] \mb U_K \mb S_K \mb V_K^T \left[ \begin{array}{cc} \mb V_{\text{exp}} & \frac{\mb b^\perp}{\left \|\mb b^\perp \right \|} \end{array} \right]^T \\
    &=& \mb U_{n+1} \mb S_{n+1} \mb V_{n+1}^T
\end{eqnarray}



\begin{table}[t]
\caption{These are the simulation results which yield the performance/complexity tradeoff curve in Figure \ref{fig:complexity_accuracy}. The experiment setup is described in that section of the paper. We describe the hyperparameters defining each denoiser in the following table.}
\label{complexity_accuracy_table}
\vskip 0.15in
\begin{center}
\begin{small}
\begin{sc}
\begin{tabular}{rccccc}
\toprule
Denoiser \# & \# Landmarks & MT Complexity & NN Complexity & MT Performance & NN Performance \\
\midrule
1 & 1221 & 7.99E04 & 2.50E06 & 1.13E-02 & 1.02E-02 \\
2 & 212 & 6.40E04 & 4.34E05 & 2.22E-02 & 2.06E-02 \\
3 & 1356 & 7.22E04 & 2.78E06 & 1.23E-02 & 9.67E-03 \\
4 & 54 & 5.43E04 & 1.11E05 & 4.4E-02 & 3.97E-02 \\
5 & 392 & 6.84E04 & 8.03E05 & 1.49E-02 & 1.30E-02 \\
6 & 3844 & 7.03E04 & 7.87E06 & 1.45E-02 & 1.04E-02 \\
7 & 84 & 5.00E04 & 1.72E05 & 2.91E-02 & 2.70E-02 \\
8 & 237 & 6.39E04 & 4.85E05 & 1.69E-02 & 1.28E-02 \\
9 & 358 & 6.44E04 & 7.33E05 & 2.35E-02 & 2.11E-02 \\
10 & 125 & 6.25E04 & 2.56E05 & 2.58E-02 & 2.22E-02 \\
11 & 44 & 5.01E04 & 9.01E04 & 5.32E-02 & 5.11E-02 \\
12 & 634 & 6.19E04 & 1.30E06 & 2.41E-02 & 2.19E-02 \\
\bottomrule
\end{tabular}
\end{sc}
\end{small}
\end{center}
\vskip -0.1in
\end{table}


Thus, final update equations implemented in \cref{alg:TISVD_one_point)} are
\begin{eqnarray}
    \mb U_{n+1} &=& \left[ \begin{array}{cc} \mb U_n & \frac{\mb x_{n+1}^\perp}{\left \|\mb x_{n+1}^\perp \right \|} \end{array} \right] \mb U_K \label{eqn:update_U} \\
    \mb S_{n+1} &=& \mb S_K \label{eqn:update_S} \\
    \mb V_{n+1} &=&  \left[ \begin{array}{cc} \mb V_{\text{exp}} & \frac{\mb b^\perp}{\left \|\mb b^\perp \right \|} \end{array} \right] \mb V_K \label{eqn:update_V}
\end{eqnarray}

Equations \eqref{eqn:update_U} and  \eqref{eqn:update_S} define the final update equations. Note that in this specific case, $\mb b^\perp = \pmb b$, and $\left\| \mb b^\perp \right\| = 1 $. Here, the computational cost is $\mc{O}(nDd^2)$ with storage of $\mc{O}(Dd)$. 








\section{Choosing Denoising Radius} \label{sec:R_i_description}

The parameter called denoising radius $R(i)$ in \cref{alg:mtn-growth} controls complexity by determining the number of landmarks created. Conceptually, as the online algorithm learns, the error in landmarks decreases, which means that $R(i)$ needs to be decreased as the landmark gets learned. This is why we define a general formula for $R(i)$ as follows:

\begin{equation}
    R(i)^2 = c_1 \left(\sigma^2 D + \frac{\sigma^2 D}{{N_i}^k} + c_2\sigma^2 d \right)
\end{equation}

\noindent where $N_i$ denotes the number of points assigned to landmark $\mb q_i$. The power parameter $k$ helps us control the speed of decay of $R(i)$, making it adaptable to different datasets. Table \ref{complexity_accuracy_table} and \ref{parameter_choices_table} show the specific constants used to create the $R(i)$ parameter to produce Figure \ref{fig:complexity_accuracy}.


\begin{table}[t]
\caption{The choice of hyperparameters yielding each denoiser. $N_{i}$ corresponds to the number of points assigned to a landmark $q_i$. For all experiments, $\sigma=0.01$, $d=2$,and $D=2048$.}
\label{parameter_choices_table}
\vskip 0.15in
\begin{center}
\begin{small}
\begin{sc}
\begin{tabular}{rccccc}
\toprule
Denoiser \# & $R_{\text{denoising}}^2$ & $R_{\text{nbrs}}$ \\
\midrule
1 & $1.2(\sigma^2D + \frac{\sigma^2 D}{N_i^{1/2}}+20\sigma^2d)$ & $2.39\sigma^2D$ \\
2 & $2.06\sigma^2D$ & $2.39\sigma^2D$ \\
3 & $1.2(\sigma^2D + \frac{\sigma^2 D}{N_i^{1/2}}+8\sigma^2d)$ & $2.39\sigma^2D$ \\
4 & $2.75 \sigma^2 D$ & $3.13\sigma^2D$ \\
5 & $1.3(\sigma^2D + \frac{\sigma^2 D}{N_i^{1/3}}+20\sigma^2d)$& $2.39\sigma^2D$ \\
6 & $1.15(\sigma^2D + \frac{\sigma^2 D}{N_i^{1/2}}+4\sigma^2d)$  & $2.39\sigma^2D$ \\
7 & $2.39\sigma^2D$ & $2.75\sigma^2D$ \\
8 &  $1.5 ( \sigma^2 D + \frac{\sigma^2 D}{Ni^1/2} +30 \sigma^2 d)$ & $2.39\sigma^2D$ \\
9 & $2\sigma^2D$ & $2.39\sigma^2D$ \\
10 & $2.19\sigma^2D$ & $2.39\sigma^2D$ \\
11 & $3.13\sigma^2D$ & $3.53\sigma^2D$ \\
12 & $1.94\sigma^2D$ & $2.39\sigma^2D$ \\
\bottomrule
\end{tabular}
\end{sc}
\end{small}
\end{center}
\vskip -0.1in
\end{table}














