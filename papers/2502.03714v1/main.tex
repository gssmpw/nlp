\documentclass{article}

\usepackage{microtype}
\usepackage{graphicx}
\usepackage{subfigure}
\usepackage{comment}
\usepackage{booktabs} %

\usepackage{hyperref}
\usepackage{float}
\usepackage{caption}

\newcommand{\theHalgorithm}{\arabic{algorithm}}


\usepackage[accepted]{icml2025}
\newcommand{\CG}{\mathcal{G}\xspace}
\newcommand{\CV}{\mathcal{V}\xspace}
\newcommand{\CE}{\mathcal{E}\xspace}
\newcommand{\CA}{\mathcal{A}\xspace}
\newcommand{\CF}{\mathcal{F}\xspace}
\newcommand{\CR}{\mathcal{R}\xspace}
\newcommand{\CB}{\mathcal{B}\xspace}
\newcommand{\CX}{\mathcal{X}\xspace}
\newcommand{\CK}{\mathcal{K}\xspace}
\newcommand{\CM}{\mathcal{M}\xspace}
\newcommand{\CC}{\mathcal{C}\xspace}
\newcommand{\CL}{\mathcal{L}\xspace}
\newcommand{\CI}{\mathcal{I}\xspace}
\newcommand{\CQ}{\mathcal{Q}\xspace}
\newcommand{\CO}{\mathcal{O}\xspace}
\newcommand{\CP}{\mathcal{P}\xspace}
\newcommand{\CS}{\mathcal{S}\xspace}
\newcommand{\CT}{\mathcal{T}\xspace}
\newcommand{\CJ}{\mathcal{J}\xspace}
\usepackage[para]{footmisc}
\usepackage{subfig}
% \usepackage{subcaption}
% \usepackage{array}
% \usepackage{colortbl}


\usepackage{code}
\usepackage[capitalize,noabbrev]{cleveref}


\theoremstyle{plain}
\newtheorem{theorem}{Theorem}[section]
\newtheorem{proposition}[theorem]{Proposition}
\newtheorem{lemma}[theorem]{Lemma}
\newtheorem{corollary}[theorem]{Corollary}
\theoremstyle{definition}
\newtheorem{definition}[theorem]{Definition}
\newtheorem{assumption}[theorem]{Assumption}
\theoremstyle{remark}
\newtheorem{remark}[theorem]{Remark}




\icmltitlerunning{Universal Sparse Autoencoders: Interpretable Cross-Model Concept Alignment}

\begin{document}

\twocolumn[{
\icmltitle{Universal Sparse Autoencoders: Interpretable Cross-Model Concept Alignment}

\begin{icmlauthorlist}
\icmlauthor{Harrish Thasarathan}{york,vector}
\icmlauthor{Julian Forsyth}{york}
\icmlauthor{Thomas Fel}{harvard}
\icmlauthor{Matthew Kowal}{york,vector,far}
\icmlauthor{Konstantinos Derpanis}{york,vector}
\end{icmlauthorlist}

\icmlaffiliation{york}{EECS York University, Toronto, Canada}
\icmlaffiliation{harvard}{Kempner Institute, Harvard University, Boston, USA}
\icmlaffiliation{vector}{Vector Institute, Toronto, Canada}
\icmlaffiliation{far}{FAR AI}
\icmlcorrespondingauthor{Harrish Thasarathan}{harryt@yorku.ca}

\vspace{5mm} 

\begin{center}
    \includegraphics[width=0.98\textwidth]{images/big_picture_v6.jpg}
    \vspace{-0.4cm}
    \captionof{figure}{\textbf{Overview of Universal Sparse Autoencoders.} (A) We introduce \textit{Universal Sparse Autoencoders} (USAEs), a method for discovering common concepts across multiple different deep neural networks. USAEs are simultaneously trained on the activations of multiple models and are constrained to share an aligned and interpretable dictionary of discovered concepts. (B) We also demonstrate one immediate application of USAEs, \textit{Coordinated Activation Maximization}, where optimizing the inputs of multiple models to activate the same concepts reveals how different models encode the same concept. Visualization reveals interesting concepts at various levels of abstraction, such as `curves' (top), `animal haunch' (middle) and `the faces of crowds' (bottom). Better viewed with zoom.}
    \label{figure:method_overview}
    \vspace{-5mm}
\end{center}

\icmlkeywords{Machine Learning, ICML}
\vskip 0.3in
}]

\printAffiliationsAndNotice{} %





\begin{abstract}

% Recent works to jointly reconstruct 3D human and object from a single RGB image, are mostly model-based, that fail to capture the fine details of the clothed human body and object surface. In this paper, we introduce ReCHOR, a novel, model-free, first-method to produce realistic clothed human-object reconstructions from a monocular view. This is extremely challenging due to human-object occlusions, diverse interactions and depth ambiguity, as it needs to infer both 3D spatial awareness and high resolution details. Our core idea is based on estimating neural implicit representations for human and object respectively by an attention-based neural implicit model that attends to pixel-aligned features from both the global human-object image for spatial awareness and  the local separate view of human and object images for high quality details. Additionally, the network is conditioned on semantic features from an initial estimated human-object pose prior and a generative diffusion model that inpaints occluded regions, thus enabling the retrieval of details from them.
% We also propose a synthetic dataset with rendered scenes of diverse, inter-occluded 3D human and object scans, to train our network. We evaluate our method on the synthetic and real world BEHAVE dataset. Our experiments show that our method outperforms the SOTA in achieving realistic clothed human-object reconstructions.
Recent approaches to jointly reconstruct 3D humans and objects from a single RGB image represent 3D shapes with template-based or coarse models, which fail to capture details of loose clothing on human bodies. In this paper, we introduce a novel implicit approach for jointly reconstructing realistic 3D clothed humans and objects from a monocular view. For the first time, we model both the human and the object with an implicit representation, allowing to capture more realistic details such as clothing. This task is extremely challenging due to human-object occlusions and the lack of 3D information in 2D images, often leading to poor detail reconstruction and depth ambiguity. To address these problems, we propose a novel attention-based neural implicit model that leverages image pixel alignment from both the input human-object image for a global understanding of the human-object scene and from local separate views of the human and object images to improve realism with, for example, clothing details. Additionally, the network is conditioned on semantic features derived from an estimated human-object pose prior, which provides 3D spatial information about the shared space of humans and objects. To handle human occlusion caused by objects, we use a generative diffusion model that inpaints the occluded regions, recovering otherwise lost details. For training and evaluation, we introduce a synthetic dataset featuring rendered scenes of inter-occluded 3D human scans and diverse objects. Extensive evaluation on both synthetic and real-world datasets demonstrates the superior quality of the proposed human-object reconstructions over competitive methods.
\end{abstract}   
\section{Introduction}
\label{sec:intro}
% Image editing methods in diffusion models depend on user-defined control directions - users can unlock their creativity using these methods by specifying the desired manipulation through prompts~\cite{gandikota2023concept}, reference images~\cite{ruiz2022dreambooth, kumari2022customdiffusion, gal2022image, chen2024trainingfreeregionalpromptingdiffusion}, or attribute vectors~\cite{parmar2023zero,hertz2022prompt}. In this work, we ask a fundamentally different question: \emph{Can we automatically discover the underlying visual structure of a concept within diffusion model's knowledge?} %Rather than requiring user-specified controls, we aim to decompose the model's internal knowledge into meaningful directions.

% This question touches on a fundamental limitation in how we interact with diffusion models. Current control methods ~\cite{zhang2023addingconditionalcontroltexttoimage, gandikota2023concept, ye2023ipadaptertextcompatibleimage,ye2023ipadaptertextcompatibleimage, hertz2024stylealignedimagegeneration, li2023photomaker, shi2024instantbooth, chen2024trainingfreeregionalpromptingdiffusion} require users to specify their desired manipulations in advance, limiting interactive creativity. This contrasts with natural human artistic workflows, where creators dynamically explore creative ideas while jointly refining them toward meaningful artistic outcomes~\cite{hoffmann2016modeling}. This synergy between specification and exploration is not new to generative models. Early GAN architectures naturally developed disentangled latent spaces that enabled continuous\cite{harkonen2020ganspace,radford2015unsupervised, wu2021stylespace, shen2020interfacegan}, compositional control over generated images. Users could explore these spaces to discover interesting variations that would be difficult to describe in words~\cite{wu2021stylespace}, then combine them to achieve their creative goals~\cite{grabe2022towards}. 


% While diffusion models have largely superseded GANs in conditional image synthesis~\cite{dhariwal2021diffusion},  their underlying structure remains less understood. Diffusion models achieve remarkable diversity through high-dimensional latents, unlike GANs' compact latent spaces.  With a single prompt, diffusion models can generate radically different variations through different random initializations of input noise. We ask - Is it possible to discover interpretable structure within this vast space of variations?

Text-to-image diffusion models are capable of generating remarkable visual variations from a single prompt through different random initializations. However, this vast creative potential remains largely opaque to users---while we can generate diverse images, we lack understanding of the underlying structure of these variations. This presents a fundamental challenge: how can we discover and expose the latent visual capabilities encoded within these models?

\let\thefootnote\relax \footnote{$^{*}$Correspondence to \texttt{gandikota.ro@northeastern.edu}}

The challenge touches on a key limitation in how we interact with diffusion models today. Current control methods require users to explicitly specify their desired edits in advance through prompts~\cite{gandikota2023concept}, reference images~\cite{zhang2023addingconditionalcontroltexttoimage, chen2024trainingfreeregionalpromptingdiffusion, ruiz2022dreambooth,kumari2022customdiffusion, Ryu_lora, hu2021lora}, or attribute vectors~\cite{ye2023ipadaptertextcompatibleimage, hertz2024stylealignedimagegeneration, li2023photomaker, shi2024instantbooth,parmar2023zero,hertz2022prompt}. That contrasts sharply with natural human creative workflows, where artists dynamically explore creative ideas and jointly refine them toward meaningful artistic outcomes~\cite{hoffmann2016modeling}. The need for pre-specified controls creates a barrier between users and the full creative potential of these models.

Interestingly, earlier generative models like GANs~\cite{gans,karras2019style,brock2018large} naturally developed more interpretable internal structures. Their compact latent spaces often exhibited emergent disentanglement~\cite{harkonen2020ganspace,radford2015unsupervised, wu2021stylespace, shen2020interfacegan}, enabling continuous and compositional control over generated images. Users could explore these spaces to discover interesting variations that would be difficult to describe in words~\cite{wu2021stylespace}, then combine them to achieve their creative goals~\cite{grabe2022towards}.

Diffusion models have largely superseded GANs in conditional image synthesis~\cite{dhariwal2021diffusion}, achieving greater diversity through much higher-dimensional latents. And yet an understanding of the underlying structure of these larger latent spaces has remained elusive. In this work, we ask a fundamental question: \emph{Can we automatically discover the visual structure within a diffusion model's knowledge of a concept?} Rather than requiring user-specified controls, we aim to decompose the model's internal representations into expressive directions that users can explore and combine.

To address these needs, we present \textbf{SliderSpace}, a framework that brings systematic explorability to diffusion models. Given just a text prompt, SliderSpace discovers a canonical set of meaningful, diverse, and controllable directions within the model's knowledge of that concept. Each direction is implemented as a low-rank adapter~\cite{hu2021lora} that can be scaled and composed with others, allowing users to explore and smoothly combine different aspects of variation, as shown in Figure~\ref{fig:intro}.

We ground SliderSpace discovery in three key requirements for meaningful decomposition of a diffusion model's visual manifold: 
\begin{enumerate}
    \item \textbf{Unsupervised Discovery:} The decomposition process should emerge from the intrinsic structure of the model's learned representation, rather than being guided by predefined attributes. This ensures we capture the true topology of the model's knowledge space rather than projecting our assumptions onto it.
    
    \item \textbf{Semantic Orthogonality:} Each discovered control must represent a distinct semantic direction. This is enforced in a semantic feature space, like CLIP, where every slider has an orthogonal effect in embeddings. This prevents discovering multiple controls that create similar semantic effects, making the system more efficient and easier.
    
    \item \textbf{Distribution Consistency:} Directions must induce consistent transformations across both random seeds and prompt variations. 
\end{enumerate}

These requirements naturally lead to our proposed framework, which we formalize in Section~\ref{sec:method}. As we show in our experiments, SliderSpace is architecture-agnostic, working with both conventional U-Net based models like Stable Diffusion~\cite{rombach2022high, rombach2022sd20, podell2023sdxl, turbo, dmd} and recent transformer-based architectures like Flux~\cite{flux}.

We demonstrate the expressiveness of SliderSpace through three applications: First, we show how SliderSpace can decompose high-level concepts into diverse and expressive components, revealing the natural axes of variation in the model's understanding. Second, we explore artistic style variation, where SliderSpace discovers directions that match or exceed the diversity of manually curated artist lists while being judged more useful by human evaluators. Finally, we show how SliderSpace can help reverse the mode collapse commonly observed in distilled diffusion models, restoring diversity while maintaining generation speed.

Beyond providing practical creative control, SliderSpace opens new avenues for understanding and utilizing the latent capabilities of diffusion models. By mapping these models' visual potential into intuitive, composable directions, we take a step toward making their creative possibilities more accessible and interpretable to users.

% Image editing methods in diffusion models unlock the creativity of users. In this work we ask an alternate question: \emph{Can we organize and expose what of the diffusion model is already capable of?}.
% Existing methods for controlling image generation typically require users to manually specify edit directions for desired changes. This process is time-consuming, requires technical expertise, and limits the spontaneity of the creative process. For instance, if a user wants to adjust the smile of a generated person, they must explicitly request this edit, often through imprecise prompt engineering or model fine-tuning. This approach of predefined controls or manual specifications restricts users from fully exploring the latent capabilities of the model. There may be interesting stylistic variations or attributes that the model can generate, but users have no easy way to discover or utilize these.

% Natural visual disentanglement was an emergent property in the latent space of Generative Adversarial Models (GANs) \cite{harkonen2020ganspace,radford2015unsupervised, wu2021stylespace, shen2020interfacegan}. In particular, it has been observed that StyleGAN~\cite{karras2019style} stylespace neurons offer detailed control over many meaningful aspects of images that would be difficult to describe in words~\cite{wu2021stylespace}. However, diffusion models do not share such a compact latent space~\cite{park2023unsupervised}; and efforts to uncover such a space in the semantic embeddings of the text conditioning have met with limited success \nik{Nick - is there a specific citation you were thinking about?}.

% In this work we introduce \textbf{SliderSpace}, which takes a step towards uncovering an analogous low dimensional representation of diffusion models' visual breadth; in essence treating the diffusion model as many generators sharing parameters, where a particular generator is defined by a specific prompt. For a given prompt we sample many random seeds (and optionally prompt expansions using an LLM), generate the corresponding images, and apply an off the shelf feature extractor (in this work CLIP, but our method can be applied to any differentiable feature extractor). We use PCA to analyze these features, and for each of the leading $k$ principal components we train a LoRA \cite{} which causes the diffusion model to produces images which increase the feature magnitude along that component when passed back through the same feature extractor. This leads to a 'Slider' for each principal component, because each LoRA can be scaled and applied to the original diffusion model, continuously varying those visual features in the generated results (as measured, in our case, by CLIP).

% There are many other works that enhance the controllability of diffusion models. One common approach is enabling users to add spatial constraints to a generation either manually, or via a reference image \cite{zhang2023addingconditionalcontroltexttoimage, chen2024trainingfreeregionalpromptingdiffusion}, a second is leveraging more abstract embeddings (e.g. identity, style) extracted from a reference image \cite{ye2023ipadaptertextcompatibleimage, hertz2024stylealignedimagegeneration, li2023photomaker, shi2024instantbooth}, a third is finetuning a foundation model to better generate a concept important to the user \cite{ruiz2022dreambooth, kumari2022customdiffusion, Ryu_lora, hu2021lora}, and a fourth (most relevant to this work) is finding low-rank adaptors of the model based on a prompt or small training set which can be scaled to provide continous control over one aspect of generated image (e.g. night vs day, basic vs luxury, etc.) \cite{gandikota2023concept}. SliderSpace is complementary to all of these methods and offers something distinct. All of the other methods we are aware require the user (and / or model designer) to know in advance what type of control they want. In contrast SliderSpace assists users in discovering and controlling hidden capabilities present in the diffusion model's distribution of possible generations.

%We propose that truly intuitive creative control in a text-to-image model should meet three key criteria: \emph{discoverability}, \emph{intuitiveness}, and \emph{specificity}. The model should reveal controllable attributes that may not be immediately obvious, offer controls that are easy to understand and manipulate, and ensure each control affects a distinct attribute of the generated image.

% We demonstrate the utility and power of SliderSpace using three applications built on top of SDXL-DMD \cite{dmd}, because its fast generation speed lends itself well to the continuous control offered by SliderSpace.

% First, we study concept decomposition (Section \ref{sec:concept_exp}), where we learn sliders for a specific concept (e.g. 'monster', 'waterfall', 'car'). Through quantitative metrics of diversity and text alignment we demonstrate that the learned sliders dramatically boost the diversity of generations when randomly applied without harming text alignment; we also ask humans to qualitatively judge these results in a user study where they find the SliderSpace results to be more 'Diverse', 'Useful', and 'Creative' than our baselines.

% Second, we attempt to compare the automatic discoveries of SliderSpace to a large scale manual study of artistic styles (Section \ref{sec:art_exp}), open-sourced by ParrotZone \cite{parrotzone}. In this study SDXL was prompted with over 4300 artist names,  and based on visual inspection the cases of successful stylistic mimicry recorded. Quantitatively SliderSpace more closely matches the distribution of artistic variation discovered by ParrotZone than other baselines, and in our user studies was judged to be significantly more 'Diverse' and 'Useful' than the baselines. To our surprise humans even judged SliderSpace results to be slightly more 'Diverse' than the results generated by the manually discovered artist names of \cite{parrotzone}.

% Third, we attempt to use SliderSpace to reverse the mode collapse commonly observed in distilled few-step diffusion models relative to the original teacher model (Section \ref{sec:diverse_exp}). We quantitatively demonstrate that applying SliderSpace to SDXL-DMD leads to more closely matching the distribution of images by the original teacher, SDXL.

%Through extensive experiments on various state-of-the-art text-to-image models, we demonstrate that SliderSpace significantly enhances user control and creative expression in AI-assisted image generation tasks. Our method enables a range of applications, including concept decomposition and control, diversity improvement in generated images, customization dissection and edits, and the exploration of artistic styles inherent in the model.

% SliderSpace goes beyond providing a practical tool for enhanced creative control. By mapping the visual potential of diffusion models it can open new avenues for generative creativity and deepens our understanding of each model's hidden potential.
\section{Related Work}
\label{sec:related_work}

The original investigation \cite{gibson1979ecological} on the relationship between visual perception and human action defines \emph{affordance} as the opportunities for interaction with the surrounding environment. Behavioral studies on regular and cognitively impaired persons have shown evidence that perception results in both visual and motor signals in the human brain. An extended study \cite{anderson2002attentional} shows that visual attention to the spatial characteristics of the perceived objects initiates automatic motor signals for different actions. In computer vision, human affordance learning involves novel pose prediction such that the estimated pose represents a valid human action within the scene context. The task is fundamental to many problems requiring robust semantic reasoning about the environment, such as human motion synthesis \cite{wang2021scene} and scene-aware human pose generation \cite{wang2017binge, roy2016multi, zhang2022inpaint, yao2023scene}.

Earlier methods of affordance learning have explored knowledge mining \cite{zhu2014reasoning} and multimodal feature cues \cite{roy2016multi} to address the problem. In \cite{zhu2014reasoning}, the authors use a Markov Logic Network for constructing a knowledge base by extracting several object attributes from different image and metadata sources, which can perform various downstream visual inference tasks without any additional classifier, including zero-shot affordance prediction. In \cite{roy2016multi}, the authors use depth map, surface normals, and segmentation map as multimodal cues to train a multi-scale convolutional neural network (CNN) for scene-level semantic label assignment associated with specific human actions. In \cite{do2018affordancenet}, the authors design a multi-branch end-to-end CNN with two separate pathways for object detection and affordance label assignment to achieve high real-time inference throughput. Researchers \cite{chuang2018learning} have also explored socially imposed constraints for affordance learning. In \cite{chuang2018learning}, the authors propose a graph neural network (GNN) to propagate contextual scene information from egocentric views for action-object affordance reasoning.

Probabilistic modeling of scene-aware human motion generation also involves semantic reasoning of human interaction with the environment. Initial works on human motion synthesis have taken different architectural approaches, such as sequence-to-sequence models \cite{barsoum2018hp}, generative adversarial networks (GAN) \cite{barsoum2018hp, cai2018deep, yang2018pose}, graph convolutional networks (GCN) \cite{yan2019convolutional}, and variational autoencoders (VAE) \cite{guo2020action2motion}. However, these methods have mostly ignored the role of environmental semantics. Due to potential uncertainty in human motion, in a recent approach \cite{wang2021scene}, the authors address such motion synthesis with a GAN conditioned on scene attributes and motion trajectory to predict probable body pose dynamics.

One key challenge of human affordance generation in 2D scenes is the lack of large-scale datasets with rich pose annotations. In \cite{wang2017binge}, the authors compile the only public dataset of annotated human body poses in complex 2D indoor scenes by extracting frames from sitcom videos. Aiming to generate a contextually valid human affordance at a user-defined location, the authors propose sampling the scale and deformation parameters for an existing human pose template using a VAE conditioned on the localized image patches as scene context. In \cite{zhang2022inpaint}, the authors introduce a two-stage GAN architecture for achieving a similar goal by estimating the affine bounding box parameters to localize a probable human in the scene and then generating a potential body pose at that location. The method uses the input scene, corresponding depth, and segmentation maps as semantic guidance. In \cite{yao2023scene}, the authors propose a transformer-based approach with knowledge distillation for generating human affordances in 2D indoor scenes.




\section{Methodology}
\paragraph{Preliminaries.}
We primarily focus on the homologous model merging, in which $\boldsymbol{\theta}_i$ all come from the same base model $\boldsymbol{\theta}_{\rm{base}}$. Given $K$ tasks $\{T_1,T_2,\cdots,T_K\}$ and $K$ corresponding fine-tuned models with parameters $\{\boldsymbol{\theta}_1,\boldsymbol{\theta}_2,\cdots,\boldsymbol{\theta}_K\}$, model merging aims to combine $K$ fine-tuned models into one single model simultaneously performing on $\{T_1,T_2,\cdots,T_K\}$ without post-training~\cite{method_p1_1,method_p1_2}.
Task vector~\cite{ilharco2023editing,yang2024adamerging} is a key element in merging method which could enhances the base model‘s ability or enable the model to handle other tasks. Specifically, for task $T_i$, the task vector $\boldsymbol\tau_i\in \mathbb{R}^D$ is defined as the vector obtained by subtracting the SFT weights $\boldsymbol{\theta}_i$ from the base model weight
$\boldsymbol{\theta}_{\rm{base}}$, \emph{i.e.}, $\boldsymbol\tau_i=\boldsymbol{\theta}_i-\boldsymbol{\theta}_{\rm{base}}$. The merged model could be denoted as $\boldsymbol{\theta}_m=\boldsymbol{\theta}_{\rm{base}}+\sum_i \lambda_i\boldsymbol{\tau}_i$, which $\lambda_i$ is the scaling factor measuring the importance of task vector. For clarification, we also denote the neuron set in $\boldsymbol{\theta}_i$ as $\mathcal{N}_i$, the neuron set in $\boldsymbol{\tau}_i$ as $\mathcal{T}_i$.



\begin{algorithm}[!ht]
    \caption{LED-Merging}
    \label{alg1}
    \begin{algorithmic}[1]
        \REQUIRE  base model $\boldsymbol{\theta}_{\rm{base}}$, SFT models $\{\boldsymbol{\theta}_{i}\mid i\in [K]\}$, mask ratios \{$r_{i} \mid i\in [K]\}$, scaling factors $\{\lambda_i\mid i\in[K]\}$, location datasets $\{\mathcal{X}_{i}\mid i\in[K]\}$
        \ENSURE merged parameter $\boldsymbol{\theta}_{m}$
        \STATE $\mathcal{M}\leftarrow\phi$
        \STATE $\boldsymbol{\theta}_{m}\leftarrow \boldsymbol{\theta}_{\rm{base}}$
        \FOR{$i\in [K]$}
        \STATE $I(\boldsymbol{\theta}_i)=\mathbb{E}_{x\sim \mathcal{X}_i}|\boldsymbol{\theta}_{i}\odot \nabla_{\boldsymbol{\theta}_i}\mathcal{L}(x)|$
        \STATE $I(\boldsymbol{\theta}_{\rm{base}})=\mathbb{E}_{x\sim \mathcal{X}_i}|\boldsymbol{\theta}_{\rm{base}}\odot \nabla_{\boldsymbol{\theta}_{\rm{base}}}\mathcal{L}(x)|$
        
        \STATE calculate $\mathcal{T}^{r_i}_{i}$ following Equation \ref{vote}
        \STATE  $\mathcal{M}\leftarrow \mathcal{M}\cup\{\mathcal{T}^{r_i}_i\}$
       
        
   
        
        
        \ENDFOR  
        \FOR{$i\in [K]$}
        
        \STATE calculate $\text{Disjoint}(\mathcal{T}_i^{r_i})$ use Equation~\ref{disjoint_safety}
        \STATE $\boldsymbol{m}_i \leftarrow \boldsymbol{0}$
        \FOR{$d\in \mathcal{T}_i^{r_i}$}
        \STATE $\boldsymbol{m}_{i,d}=1$
        \ENDFOR
        \STATE $\boldsymbol{\theta}_{m}\leftarrow \boldsymbol{\theta}_{m}+\lambda_i \boldsymbol{\tau}_i\odot \boldsymbol{m}_{i}$
        \ENDFOR
    \end{algorithmic}
\end{algorithm}
    %\vspace{-5pt}
\begin{figure*}[h!]
    \centering
    \includegraphics[width=\linewidth]{figs/pipeline_v2.pdf}
    \vspace{-40mm}
    \caption{Overview of our two-stage training pipeline {\ours}.}
    \label{fig:pipeline}
\end{figure*}


\paragraph{LED-Merging: Location, Election, and Disjoint Merging}
To address the neuron misidentification and interference issues in existing model merging methods, we propose LED-Merging (Location, Election, and Disjoint Merging). Specifically, previous studies \cite{modelstock, ilharco2023editing, tiesmerging} fail to accurately identify safety-related neurons in task vectors with a single magnitude score, namely \textit{neuron misidentification}. Meanwhile, there exists an interference between safety-related and utility-related task vector neurons during the merging process, namely \textit{neuron interference}. To address neuron misidentification, we first locate important neurons both in the base and fine-tuned models and then elect neurons from the task vector considering these two scores together. Subsequently, to mitigate the interference, we introduce a disjoint step, isolating these important neurons so that they influence different base neurons. The whole process is illustrated in Figure~\ref{fig:method}. 




In the location and election step, we consider the importance score from base and fine-tuned models simultaneously to locate task-specific neurons. In this way, it is more accurate than relying on the magnitude score alone because task-specific neurons with high importance score in the fine-tuned model may not necessarily score high in the base model, and vice versa.

{\textbf{Location}}.  We first calculate importance scores for each neuron in a base/fine-tuned model. Given a location dataset $\mathcal{X}_i=\{(x,y)_k\}$, where $x$ is the question and $y$ is the answer, we calculate the importance scores for the weight $\boldsymbol{\theta}_i\in\mathbb{R}^D$ in any  layer as follows~\cite{snip,spareseGPT,sun2024a}:
\begin{equation}
    I(\boldsymbol{\theta}_i)=\mathbb{E}_{x\sim \mathcal{X}_i}[\boldsymbol{\theta}_i\odot \nabla _{\boldsymbol{\theta}_i}\mathcal{L}(x)],
    \label{location}
\end{equation}
which $\mathcal{L}(x)=-\log p(y\mid x)$ is the conditional negative log-likelihood loss. We choose the SNIP score~\cite{snip} because it balances computational efficiency and performance~\cite{cq}. Please refer to Sec.~\ref{sec:ablation} for the comparison between different location methods. After computing importance scores, we choose top-$r_i$ neurons as the important neuron subset $\mathcal{N}_{i}^{r_i}$ from $I(\boldsymbol{\theta}_i)$.
 
 % After computing locating scores, we select the neurons scoring both high in base and fine-tuned models as important neurons in task vectors. Then in the disjoint step,  with preventing  polysemantic neurons  from receiving gradient updates towards different directions,
 % we use set difference to isolate the safety   and utility-related neurons  and construct corresponding masks for merging process,

{\textbf{Election}}. A natural question is how to select important neurons in the task vector $\boldsymbol{\tau}_i$ based on $I(\boldsymbol{\theta}_{\rm{base}})$ and $I(\boldsymbol{\theta}_{i})$. The important neurons in the base model may be different from neurons in the fine-tuned model. Therefore, we introduce the following election strategy to select neurons with high scores in both base and fine-tuned models:
\begin{equation}
    \mathcal{T}_i^{r_i}=\mathcal{N}_i^{r_i}\cap \mathcal{N}_{\rm{base}}^{r_i}.
    \label{vote}
\end{equation}
\emph{Remark}. We compare different choosing methods, including scoring low or high in base or fine-tuned model in Section~\ref{sec:ablation} and find that Equation \ref{vote} achieves the best performance.





{\textbf{Disjoint}}. As important neurons from different task vectors may conflict with each other at the same position, we use the set difference to disjoint the neurons from others to prevent interference:
\begin{equation}
    \text{Disjoint}(\mathcal{T}^{r_i}_{i})=\mathcal{T}^{r_i}_{i}-\mathop{\cup}\limits_{{J}\subsetneqq [K],|J|\geq 2}\mathop{\cap}\limits_{j\in {J}}\mathcal{T}^{r_j}_{j}.
    \label{disjoint_safety}
\end{equation}

Next, we construct a mask $\boldsymbol{m}_i\in\mathbb{R}^D$ to implement disjoint in the merging process. Specifically, this mask $\boldsymbol{m}_i$ is used to select neurons from $\mathcal{T}_i$. The mask ratio is $r_i$, where $r\in(0,1]$. The mask $\boldsymbol{m}_i$ can be derived from:
\begin{equation}
    \boldsymbol{m}_{i,d}=\begin{aligned} &\left\{ \begin{array}{ll} 1, & \text{if } d\in \text{Disjoint}(\mathcal{T}_{i}^{r_i}), \\ 0, & \text{otherwise}. \end{array} \right. \end{aligned}
    \label{mask_safety}
\end{equation}


% \subsection{Merging Models with Masks}
{\textbf{Merging}}. The final
merged task vector $\boldsymbol{\tau}_m$ is as follows:
\begin{equation}
    \boldsymbol{\tau}_m= \sum_i \lambda_i\boldsymbol{\tau}_{i}\odot\boldsymbol{m}_i.
    \label{merged_task_vector}
\end{equation}
We summarize the workflow in Algorithm \ref{alg1}.



\section{Experiments}
\label{sec:experiments}

\begin{figure*}[t]
\vspace{-6mm}
    \centering
    \includegraphics[width=0.8\linewidth]{figs/compare.pdf}
    \vspace{-4mm}
    \caption{\textbf{Qualitative comparison} with the baseline for generating a sequence of novel view images.  
    The results demonstrate that our method synthesizes more consistent multi-view images compared to our baseline model (Zero123). In addition, compared to SyncDreamer, our method visually maintains better similarity to the conditioned image and appears more natural.}
    \label{fig:sota_compare}
\vspace{-5mm}
\end{figure*}

\subsection{Experimental Setups}
\textbf{Dataset.}
Following previous work~\cite{zero123, SyncDreamer}, we evaluate our work on the Google Scanned Object (GSO)~\cite{GSO} dataset to verify the zero-shot novel view image synthesis capability. 
We also provide results for additional datasets in the Supplementary Material.
Specifically, we randomly select 30 objects from the GSO dataset with various object categories. 
Unlike recent approaches~\cite{mvdream, SyncDreamer} that aim to enhance the consistency of novel view synthesis models by generating multiple fixed-view images, our method can generate images from any camera pose and any number of views. Therefore, we conduct experiments under different camera pose settings to validate our approach:
specifically, 
1) \textit{16-views with free camera pose}: for each object, we circularly render 16 views with the elevation angles ranging in $[-10\degree, 40\degree]$ and the azimuth angles are evenly distributed in $[0\degree, 360\degree]$. 
2) \textit{16-views with fixed camera pose}: We maintain a constant elevation angle of $30\degree$ and uniformly sample azimuth angles (same as SyncDreamer~\cite{SyncDreamer}).
3) \textit{32-views with free camera pose}: Similar to the first setting, but we sample 32 views.
It's important to note that our method does not require additional training or fine-tuning on any datasets.

\noindent\textbf{Metrics.}
To validate the effectiveness of our method, we mainly evaluate it based on three criteria:
1) \textit{Quality Score}. We evaluate the image quality of synthesized multi-view images by measuring their similarity with ground truth images. Following prior research~\cite{zero123, sparsefusion}, we report the similarity between the synthesized images and the ground truth images with standard metrics: PSNR, SSIM~\cite{ssim}, and LPIPS~\cite{lpips}.
2) \textit{Multi-view Consistency Score}. As the primary goal of our work is to improve the consistency of generated images, we also employ the 3D consistency score~\cite{3dim} to verify the consistency among the synthesized images. Specifically, we train an Instant-NGP~\cite{instant_ngp} with the input image and part of the synthesized novel view images of our model and evaluate the similarity between the remaining synthesized images and the rendered images of Instant-NGP. For the synthesized multi-view images of each object, we allocate $3/4$ for training and reserve the remaining $1/4$ for validation.
Intuitively, if the consistency of synthesized images is improved, the NeRF-like model will train a better object representation, and the re-rendered images will agree more with the validation images.
3) \textit{Input Consistency Score}. To assess the faithfulness of synthesized images in preserving the identity of the input condition image, we introduce the input consistency score. This score calculates the similarity of each synthesized image with the input condition image, utilizing the LPIPS metric.

In addition, we use synthesized multi-view images to train a neural 3D reconstruction model (NeuS~\cite{neus}) and report commonly used Chamfer Distances (CD) and Volume IoUs between the trained 3D model and the ground truth.

\noindent\textbf{Baselines.}
Given that our main goal is to improve the consistency of the trained baseline model without further fine-tuning, we mainly compare our approach with the used baseline model Zero123~\cite{zero123}. Additionally, we compare our method to the SOTA approaches such as PGD~\cite{tseng2023consistent} and SyncDreamer~\cite{SyncDreamer} using the same Zero123 base model.

\noindent\textbf{Implementation Details.}
We use the official checkpoint provided by Zero123~\cite{zero123}, which is trained on objaverse~\cite{objaverse} for 165,000 steps. We inject our epipolar attention layer after step $T=4$ and layer $L=10$ by default. We find that feature fusion weight $\alpha=0.5$, and the number of context views $M=2$ work better.

\begin{table}[t]
\centering
\caption{Comparison of multi-view consistency, image quality, and input consistency of synthesized multi-view images at the 16-view setting with free camera pose.}
\label{tab:view16_free_compare}
\vspace{-2mm}
\scalebox{0.6}{
\begin{tabular}{c ccc ccc c}
\toprule
              & \multicolumn{3}{c}{Multi-view Consistency} & \multicolumn{3}{c}{Quality Score} & \multicolumn{1}{c}{Input Consis.} \\
              \cmidrule(lr){2-4} \cmidrule(lr){5-7} \cmidrule(lr){8-8}
              & PSNR$\uparrow$  & SSIM$\uparrow$ & LPIPS$\downarrow$ 
              & PSNR$\uparrow$  & SSIM$\uparrow$ & LPIPS$\downarrow$ 
              & LPIPS$\downarrow$ 
              \\ \midrule

Zero123
& 15.225        & 0.645       & 0.408
& 14.255        & 0.747       &	0.208
& 0.303         
\\
SyncDreamer
& 14.830        & 0.626       & 0.434
& 12.650        & 0.713       &	0.254
& 0.317         
\\
Ours 
& \best{18.300}	& \best{0.734}	& \best{0.355}
& \best{14.947}	& \best{0.763}	& \best{0.191}
& \best{0.282}
\\

\bottomrule
\end{tabular}
}
\end{table}

\begin{table}[t]
\vspace{-1mm}
\centering
\caption{Comparison of multi-view consistency, image quality, and input consistency at the 16-view setting with fixed camera pose as SyncDreamer~\cite{SyncDreamer}.}
\label{tab:view16_fxied_compare}
\vspace{-3mm}
\scalebox{0.6}{
\begin{tabular}{c ccc ccc c}
\toprule
              & \multicolumn{3}{c}{Multi-view Consistency} & \multicolumn{3}{c}{Quality Score} & \multicolumn{1}{c}{Input Consis.} \\
              \cmidrule(lr){2-4} \cmidrule(lr){5-7} \cmidrule(lr){8-8}
              & PSNR$\uparrow$  & SSIM$\uparrow$ & LPIPS$\downarrow$ 
              & PSNR$\uparrow$  & SSIM$\uparrow$ & LPIPS$\downarrow$ 
              & LPIPS$\downarrow$ 
              \\ \midrule

Zero123
& 16.556        & 0.682       & 0.378
& 14.592        & 0.750       &	0.207
& 0.305         
\\
SyncDreamer
& \best{22.424}        & \best{0.812}       & \best{0.268}
& 15.269        & 0.749       &	0.196
& 0.300         
\\
Ours 
& 21.151	& 0.780	& 0.302
& \best{15.293}	& \best{0.764}	& \best{0.184}
& \best{0.287}
\\

\bottomrule
\end{tabular}
}
\vspace{-4mm}
\end{table}


\subsection{Comparison With Baseline Models}
The quantitative comparison on three settings are shown in Tab.~\ref{tab:view16_free_compare}, Tab.~\ref{tab:view16_fxied_compare}, and Tab.~\ref{tab:view32_free_compare}. The qualitative comparison is shown in Fig.~\ref{fig:sota_compare}.

\begin{table}[t]
\centering
\caption{Comparison of multi-view consistency and image quality scores of synthesized multi-view images at the 32-view setting with free camera pose.}
\vspace{-3mm}
\label{tab:view32_free_compare}
\scalebox{0.7}{
\begin{tabular}{c ccc ccc}
\toprule
              & \multicolumn{3}{c}{Multi-view Consistency} & \multicolumn{3}{c}{Quality Score} \\
              \cmidrule(lr){2-4} \cmidrule(lr){5-7}
              & PSNR$\uparrow$  & SSIM$\uparrow$ & LPIPS$\downarrow$ 
              & PSNR$\uparrow$  & SSIM$\uparrow$ & LPIPS$\downarrow$ 
              \\ \midrule

Zero123
& 16.515        & 0.694       & 0.378
& 15.142        & 0.733       &	0.211
\\
PGD~\cite{tseng2023consistent}
& 18.481        & 0.720       & 0.343
& 15.281        & 0.739       &	0.205
\\
Ours 
& \best{20.655}	& \best{0.792}	& \best{0.305}
& \best{15.268}	& \best{0.742}	& \best{0.203}
\\

\bottomrule
\end{tabular}
}
\vspace{-3mm}
\end{table}

\begin{table*}
  [t]
  \centering
  \resizebox{\textwidth}{!}{%
  \begin{tabular}{cccccccccccc}
    \toprule \multicolumn{2}{c}{Components}                                                             & \multicolumn{5}{c}{Re-executability Rate (\%)} & \multicolumn{5}{c}{Readability (\#)} \\
    \cmidrule(lr){1-2} \cmidrule(lr){3-7} \cmidrule(lr){8-12}        \hspace{8pt}\labelemoji\hspace{8pt}                                                                & \hspace{8pt}\toolemoji\hspace{8pt}                                      & O0                                 & O1             & O2             & O3             & AVG            & O0             & O1             & O2             & O3             & AVG            \\
    \hline
    \rowcolor[rgb]{0.93,0.93,0.93}\multicolumn{12}{c}{\textbf{Initialize with LLM4Decompile-End-6.7B~\citep{llm4decompile}}}   \\
    \xmark                                                                                              & \xmark                                    & 69.51                              & 46.95          & 50.61          & 46.34          & 53.35          & 3.98 & 3.41 & 3.44 & 3.38 & 3.55 \\
    \cmark                                                                                              & \xmark                                    & 75.61                              & 50.61          & 50.00          & 50.00          & 56.55          & 4.01 & 3.44 & 3.39 & \textbf{3.49} & 3.58 \\
    \xmark                                                                                              & \cmark                                    & 83.54                     & \textbf{56.10}          & 51.22          & 50.61 & 60.37 & 4.05 & 3.51 & 3.51 & 3.42 & 3.62 \\
    \cmark                                                                                              & \cmark                                    & \textbf{85.37}                            & \textbf{56.10}                     & \textbf{51.83} & \textbf{52.43}          & \textbf{61.43} & \textbf{4.13} & \textbf{3.60} & \textbf{3.54} & \textbf{3.49} & \textbf{3.69} \\

    \rowcolor[rgb]{0.93,0.93,0.93}\multicolumn{12}{c}{\textbf{Initialize with Deepseek-Coder-6.7B-base~\citep{deepseekcoder}}} \\
    \xmark                                                                                              & \xmark                                    & 59.15                              & 35.98          & 39.02          & 37.80          & 42.99          & 3.71 & 3.05 & 3.16 & 3.05 & 3.24 \\
    \cmark                                                                                              & \xmark                                    & 66.46                              & 41.46          & 38.41          & 36.59          & 45.73          & 3.76 & 3.17 & \textbf{3.21} & 3.08 & 3.31 \\
    \xmark                                                                                              & \cmark                                    & 70.73                              & 39.63          & 39.02          & 40.24          & 47.41          & 3.90 & 3.17 & 3.08 & 3.11 & 3.31 \\
    \cmark                                                                                              & \cmark                                    & \textbf{79.88}                     & \textbf{45.73} & \textbf{43.90} & \textbf{42.68} & \textbf{53.05} & \textbf{3.96} & \textbf{3.21} & 3.18 & \textbf{3.19} & \textbf{3.38} \\
    \bottomrule
  \end{tabular}%
  }
  \caption{The ablation study of different methods across four optimization levels
  (O0, O1, O2, O3), as well as their average scores (AVG). The results in bold represent the optimal performance. The ~\labelemoji~ and ~\toolemoji~ means Relabedling and Function Call. \textbf{Bold} denotes the best performance.}
  \label{tab:ablation}
\end{table*}



\begin{figure*}[ht]
    \centering
    \begin{minipage}{0.65\textwidth}
        \centering
        \includegraphics[width=0.95\linewidth]{figs/ablation.pdf}
        \vspace{-2mm}
        \captionof{figure}{Qualitative Comparison for different design choices. Our method, employing multi-view epipolar attention, demonstrates the best consistency.}
        \label{fig:ablation}
    \end{minipage}\hfill
    \begin{minipage}{0.33\textwidth}
        \centering
        \includegraphics[width=0.8\linewidth]{figs/neus_ver.pdf}
        \vspace{-3mm}
        \caption{Our method shows better direct 3D reconstruction~\cite{neus}.}
        \label{fig:neus}
    \end{minipage}
    \vspace{-5mm}
\end{figure*}

\noindent\textbf{Multi-view Consistency.}
Tab.~\ref{tab:view16_fxied_compare} presents the 3D consistency scores compared to our baseline model (Zero123) and SyncDreamer. The results indicate a significant improvement across all three metrics achieved by our method when compared with Zero123.
While our method exhibits a marginally lower numerical consistency score compared to SyncDreamer, it enables the synthesis of images with arbitrary camera poses.	
This capability is illustrated in Tab.~\ref{tab:view16_free_compare}, where our method consistently enhances consistency with changes in camera pose settings, whereas SyncDreamer fails to do so and exhibits inferior results compared to Zero123.
Furthermore, our method facilitates the synthesis of multi-view images with any number of camera views. This versatility is demonstrated in Tab.~\ref{tab:view32_free_compare}, where our method continues to achieve significant improvements in consistency scores, while SyncDreamer is unable to operate under such conditions.	

Meanwhile, Fig.~\ref{fig:sota_compare} provides a qualitative comparison with the baseline. While both our method and SyncDreamer enhance consistency, our method visually preserves better similarity to the input image, including color and texture details. The input consistency score further corroborates this.

\noindent\textbf{Image Quality.}
While our primary goal centers around enhancing the consistency of synthesized multi-view images, we also evaluate the image quality by comparing the similarity with the ground truth images. The results shown in Tab.~\ref{tab:view16_free_compare}, Tab.~\ref{tab:view16_fxied_compare}, and Tab.~\ref{tab:view32_free_compare} indicate that our method also enhances the image quality under different settings besides improving the consistency.
Moreover, our method shows better image quality compared with SyncDreamer even in the 16-view setting with fixed camera pose.

\noindent\textbf{Input Consistency.}
Input consistency terms whether the results align with the input image.
Fig.~\ref{fig:sota_compare} illustrates that both our method and SyncDreamer enhance multi-view consistency. However, the color and texture details of SyncDreamer's results diverge from the input image and appear visually unnatural.
This discrepancy is evident in the input consistency score presented in Tab.~\ref{tab:view16_fxied_compare}, indicating lower similarity with the condition image in the SyncDreamer results.	

\subsection{Ablation Study}
The overall quantitative results are shown in Tab.~\ref{tab:ablation}, and the qualitative comparisons are shown in Fig.~\ref{fig:ablation}.

\noindent \textbf{Full Attention \vs Epipolar Attention.}
The results presented in Tab.\ref{tab:ablation} and Fig.\ref{fig:ablation} demonstrate that our epipolar attention mechanism can synthesize more consistent multi-view images compared with full attention. Furthermore, our epipolar attention achieves a greater performance improvement compared to full attention when using multiple reference images. This could be attributed to the fact that our epipolar attention more effectively localizes target information, as depicted in Fig.~\ref{fig:full_attn_compare}, thereby reducing noise from the reference images. In the multi-view setting, where multiple reference images are utilized, this noise reduction becomes particularly crucial.
Moreover, it is noteworthy that the epipolar attention mechanism consumes less GPU memory compared to our baseline, as discussed in Sec.~\ref{sec:attn_analysis}.

\noindent \textbf{Attending Single-View \vs Multi-View.}
Applying the epipolar attention significantly improves the consistency between the input and target views. However, the consistency between different views in the unobserved regions of the input view is not well preserved.
After implementing our epipolar attention in the multi-view setting, the consistency across the generated multi-view images is further improved. The last row in Tab.~\ref{tab:ablation} shows that after applying our multi-view epipolar attention, the consistency score is further improved compared with the single-view setting. Besides, the qualitative result in Fig.~\ref{fig:ablation} also shows better consistency among different target views.



\begin{table}[t]
\centering
\vspace{-1mm}
\caption{Comparison of 3D reconstruction results. Our method significantly improves the reconstruction quality.}
\vspace{-3mm}
\label{tab:neus}
\scalebox{0.7}{
\begin{tabular}{c cc}
\toprule
              &  Chamfer Dist.$\downarrow$  & Volume IoU$\uparrow$
\\ \midrule

            Zero123         & 0.017         & 0.819    \\
            SyncDreamer     & \best{0.013}         & \best{0.847}    \\
            Ours            & 0.014	& 0.842 \\

\bottomrule
\end{tabular}
}
\vspace{-5mm}
\end{table}


\vspace{-2mm}
\subsection{Downstream Application}
\vspace{-2mm}
To demonstrate the effectiveness of our method, we also applied it to the downstream 3D reconstruction task. Specifically, we trained the NeuS model~\cite{neus} directly using images synthesized by our method, Zero123, and SyncDreamer, respectively.
The quantitative results in Tab.~\ref{tab:neus} show that the consistent multi-view images synthesized by our method can significantly improve the 3D reconstruction quality.
Additionally, our method exhibits similar performance to SyncDreamer which requires time-consuming re-training.
The qualitative results in Fig.~\ref{fig:neus} show that it is challenging to train the NeuS model directly due to the lack of consistency in the images generated by Zero123. In contrast, our method generates more consistent multi-view images and, therefore, better reconstructs the geometry and texture details.
We show improvements on other downstream applications such as image-to-3D in the Supplementary Material.


\section{Concluding Remarks}
In this paper, we proposed a novel approach utilizing multimodal LLMs to generate gesture-aware speech recognition transcripts for patients with language disorders. Our framework integrates verbal speech and iconic gestures, enabling the generation of enriched transcripts that capture the latent meaning conveyed through both modalities. Through extensive experimentation, we demonstrated that the proposed method effectively contextualizes incomplete or disfluent speech by incorporating gesture information, leading to more accurate and meaningful representations of the speaker's intent. These findings highlight the potential of our approach to significantly contribute to the field of speech and language therapy, offering innovative tools that can enhance the quality of life for individuals with language disorders by facilitating better communication and assessment methods.

\subsection{Ethical Statement} 
Our dataset was obtained from AphasiaBank with the approval of the Institutional Review Board (IRB) and adheres to the data sharing guidelines set by TalkBank\footnote{https://talkbank.org/share/ethics.html}. This includes complying with the Ground Rules for all TalkBank databases, which are based on the American Psychological Association Code of Ethics~\cite{american2002ethical}.

\subsection{Limitation \& Future Work} 
%This study represents a preliminary investigation into using multimodal LLMs to generate gesture-aware speech recognition transcripts. 
While the results are promising, we recognize several limitations and outline our plans to extend this work further.

One primary limitation is the absence of a definitive ground truth for quantitative evaluation. Since our model generates transcripts by synthesizing speech and gesture data from scratch, traditional benchmarks, such as comparisons with standard speech recognition outputs, are insufficient. Moreover, existing original transcripts lack gesture annotations, making direct comparisons challenging. In future work, we aim to address this gap by collaborating with certified pathologists to conduct qualitative assessments, such as A-B preference tests, to evaluate the effectiveness of gesture-enriched transcripts in accurately conveying the speaker's intentions.

To support quantitative evaluations, we plan to develop novel metrics that assess transcript quality, including grammar accuracy, semantic consistency, and the integration of multimodal information. Such metrics will provide a more objective basis for assessing our model's performance and facilitate comparisons with other multimodal and unimodal approaches.

Another limitation of this study is its focus on structured gestures from a specific task, the Peanut Butter Sandwich Task. While this task offers a controlled context for testing our approach, it does not encompass the diversity of gestures and communication patterns seen in everyday scenarios. As part of our future work, we plan to expand the scope of our model to include tasks such as the Cinderella Story Recall Task~\cite{bird1996cinderella}, which involves unstructured and complex narrative gestures. This expansion will allow us to evaluate the adaptability and robustness of our model in handling varied linguistic and gestural contexts.

In summary, while this study establishes a strong foundation for gesture-aware speech recognition, we aim to refine and extend our methods through collaborative qualitative evaluations, the development of robust quantitative metrics, and broader task applications. These efforts will ensure that our approach continues to evolve, ultimately contributing to more effective communication tools and interventions for individuals with language disorders.








\bibliography{main,dictionarylearning,refs_shorthand}
\bibliographystyle{icml2025}



% \newpage
\clearpage
% Appendix
\appendix
\label{sec:appendix}

\subsection{Real World Setup}

\noindent\textbf{Deployment hardware}
% DOF of the robot, workspace, etc.
The humanoid robot on which we deploy \our is Unitree H1~\cite{H1-page}.
Following \citet{cheng2024tv}, we assembled two DYNAMIXEL XL330-M288-T motors~\citep{dynamixel-page} with 3D printed gimble parts and a ZED Mini stereo camera ~\cite{zed-page} for two-DoF (yaw and pitch) active sensing.
Each arm of H1 has 5 DoFs and a 6-DoF end-effector from~\cite{dexterous-page}, and other DoFs on the robot are not used.


\noindent\textbf{Motion capture system}
\label{app:mocap}
We use ArUco markers and two cameras to build a simple motion capture system.
We put four ArUco markers on the four corners of the workspace table to locate the cameras.
Each human in the workspace wears two 3D-printed wristbands with four ArUco markers on each of them, illustrated in \Cref{fig:mocap_sys}. 
The cameras are calibrated and located using the OpenCV library and capture the human wristband's position in real time.
Each wristband has an additional IMU sensor to capture the orientation of the human's wrist.
To reduce the noise in the collected data, we use a Kalman filter to smooth the data.

\begin{figure}[b]
    \centering
    % \includegraphics[width=0.8\linewidth]{example-image}
    \includegraphics[width=\linewidth]{figs/src/mocap.pdf}
    \caption{\small \textbf{Setup of motion capture system.} a) The two Motion Capture Cameras are used to detect the ArUco markers. The video recorded by the RGB-D Camera is used to process human motion and human hand details of the Leader. b) Follower and Leader both wear wristbands with ArUco markers for hand position detection. c) We 3D-print our wristbands with 4 ArUco markers on 4 surfaces and embed a IMU beneath the upper surface.}
    % \label{fig:intent_corner}
    \label{fig:mocap_sys}
\end{figure}

\begin{figure}
    \centering
    \includegraphics[width=0.85\linewidth]{figs/src/body_hand_obj.pdf}
    \caption{\small \textbf{Visualization results of human body motion detection, object detection, and human hand detection.} The upper image demonstrates the human motion detection (only upper body is used) and object detection (in Dining scenario). The lower image demonstrate the human hand detection.}
    \label{fig:zed_detect}
\end{figure}

\noindent\textbf{Motion Detection and Object Detection}
As is illustrated in \Cref{method:real-robot}, we use the Body Tracking feature in ZED API to detect the body motion of the human and a fine-tuned YoloV11~\cite{khanam2024yolov11} model to detect objects on the table. For hand detection, we use HaMeR\cite{pavlakos2024reconstructing} to obtain the human hand motion and then retarget\cite{qin2024anyteleopgeneralvisionbaseddexterous} it to the 6-DoF robot hand. The visualization results are shown in \Cref{fig:zed_detect}.


\subsection{Skill Descriptions}
\label{app:skills}
%  task description, succ. condition, reverse task(if exist), and instruction
In this section, we describe the skills that we deploy on the humanoid robot. The details include the description, success condition, reverse skill (if exists), and the human intention related to the skill.
Note that the intention is inferred mainly from human behavior, hand positions, and the relative location of objects.
The latter two can be concluded trivially to the start condition and end transition, and are shown by skill in \Cref{tab:skills}.
We concern the intention mainly on the human body motion in the narration as follows:


\noindent\textbf{1) Scenario 1: Humanoid as a Dining Waiter}

% Description of this scenario.
In this scenario, the leader human and the robot sit face-to-face at the side of a dining table. 
There are plates with food, a Coke can, a tissue box, and a sponge on the table.
In the following skill descriptions, the humanoid robot takes the role of a helpful waiter and serves the leader human a meal with these objects.
$10$ skills related to $4$ objects are listed below:

\begin{itemize}[leftmargin=*]
    \item \textit{Pick Can}: The robot picks up a can with its right arm from the table. 
    The success condition is that the can is lifted off the table.
    The interruption data includes the human taking the can away or the human putting his hand on the can. 
    \textit{Place Can} is the corresponding reverse skill.
    The leader human shows the intention by pointing to the can when the right robot arm is empty.
    
    \item \textit{Place Can}: The robot places the can back on the table with its right arm. 
    The success condition is that the can is placed on the table and the robot's hand is lifted off the can.
    The intention of this skill is shown by the leader human pointing to the place on the table where the can was placed before.
    
    \item \textit{Get Plate from Human}: The robot fetches a plate from the hand of the human with its left arm.
    The success condition is the plate in the dexterous hand when the human loosens the grip of the plate.
    The leader human shows the intention by handing a plate forward.
    
    \item \textit{Place Plate to Stack}: The robot stacks the plate in its right hand onto a pile of plates on the table.
    The success condition is the plate settled on the top of the plate pile without slipping.
    The intention is given by the leader human pointing to the stack.
    Interruption data, in which the human touches the plate to stick the motion, is added to the collected dataset.
    
    \item \textit{Pick Place from Table}: The robot lifts a plate on the table by both arms and holds it in the left hand.
    Application of this skill succeeds if there is no slippage in the motion until the plate is held.
    The leader human points to the plate to show the intention.
    
    \item \textit{Handover Plate}: The robot protracts its left arm to give the leader human the plate on it.
    The showcase of this skill ends when the plate is put into the human hand.
    The leader human simply stretches the right hand out to show the intention.
    It is the reverse skill of \textit{Get Plate from Human}.
    
    \item \textit{Pick Sponge}: The robot picks up the sponge with its right arm. 
    The sponge is placed beside the can.
    When it is lifted off the table the skill showcase succeeds.
    % \textit{Place Sponge} is the reverse skill.
    The leader human shows intention by mimicking washing.
    Data where the human snatches the sponge before the robot reaches it adds to the dataset. 
    In case this happens during deployment, the robot withdraws its hand to the idle state.

    \item \textit{Brush with Sponge}: It is a complex skill using both arms.
    The start condition is a plate in the left hand and a sponge in the right one when the leader human makes the washing gesture (same as that in \textit{Pick Sponge}) again.
    To apply this skill, the robot moves the sponge close to the plate and rubs the sponge on the plate to brush it.
    The success condition is the robot keeps the periodic brushing motion for over $10$ seconds.

    \item \textit{Place Sponge}: The reverse skill of \textit{Pick Sponge}.
    The robot puts the sponge in the right hand back onto the table to complete the skill demonstration.
    The intention is shown by the leader human pointing to the place on the table where the sponge was placed before (similar to skill \textit{Place Can}).

    \item \textit{Pick a Piece of Tissue}: The leader human points to the tissue box to express the intention. 
    Then the robot uses its left hand to pull a tissue from the tissue box placed on the table corner and gives it to the leader human.
    The skill showcase succeeds when the leader human receives the tissue.
\end{itemize}

\begin{table*}
    \begin{center}
        \caption{Description of the skills. Notes: The \textbf{Start Condition} or \textbf{End Transition} \texttt{[A, B]} means that object \texttt{A} is in the left hand of the robot and \texttt{B} is in the right hand. \texttt{empty} for this hand must be empty, \texttt{any} for this hand could hold any object or be empty, and \texttt{-} for this object remains unchanged after the skill is completed.}
        \label{tab:skills}
        % \begin{tabular}{c|c|c|c|c|c|c}
        \begin{tabular}{ccccccc}
        \toprule
        \textbf{Scenarios} & \textbf{Object} & \textbf{Skill Name} & \textbf{Start Condition} & \textbf{End Transition} & \textbf{Num. of Data} & \textbf{Arm}\\
            \midrule
            % can
            \multirow{11}{*}{\begin{tabular}{c}
                 Scenario 1 \\
                 Dining Waiter
            \end{tabular}}  & \multirow{2}{*}{can} & Pick Can & \texttt{[any, empty]} & \texttt{[-, can]} & $107$ & Right\\
            & & Place Can & \texttt{[any, can]} & \texttt{[-, empty]} & $100$ & Right\\
            % plate
            \cmidrule{2-7}
            & \multirow{4}{*}{plate} & Get Plate from Human  & \texttt{[empty, any]} & \texttt{[plate, -]} & $100$ & Left\\
            & & Place Plate to Stack & \texttt{[plate, any]} & \texttt{[empty, -]} &  $98$ & Left\\
            & & Pick Plate from Table & \texttt{[empty, empty]} & \texttt{[empty, plate]} & $115$ & Dual-Arm\\
            & & Handover Plate & \texttt{[plate, any]} & \texttt{[empty, -]} & $115$ & Left\\
            % sponge
            \cmidrule{2-7}
            & \multirow{3}{*}{sponge} & Pick Sponge  & \texttt{[any, empty]} & \texttt{[-, sponge]} & $89$ & Right\\
            & & Brush with Sponge & \texttt{[plate, sponge]} & \texttt{[-, -]} & $81$ & Dual-Arm\\
            & & Place Sponge & \texttt{[any, sponge]} & \texttt{[-, empty]} & $82$ & Right\\
            % tissue
            \cmidrule{2-7}
            & tissue & Pick a Piece of Tissue  & \texttt{[empty, any]} & \texttt{[-, -]} & $105$ & Left\\
            \midrule
            % cap
        \multirow{8}{*}{\begin{tabular}{c}
             Scenario 2  \\
             Office Assistant
        \end{tabular}} & \multirow{2}{*}{cap} & Settle Cap & \texttt{[any, empty]} & \texttt{[-, -]} & $111$ & Right\\
            & & Handover Cap & \texttt{[any, empty]} & \texttt{[-, -]} & $110$ & Right\\
            % book
            \cmidrule{2-7}
            & book & Pick Book & \texttt{[empty, any]} & \texttt{[-, -]} & $115$ & Left\\
            % stamp
            \cmidrule{2-7}
            & \multirow{3}{*}{stamp} & Pick Stamp & \texttt{[any, empty]} & \texttt{[-, stamp]} & $92$ & Right\\
            & & Stamp the Paper & \texttt{[any, stamp]} & \texttt{[-, -]} & $87$ & Right\\
            & & Place Stamp & \texttt{[any, stamp]} & \texttt{[-, empty]} & $89$ & Right\\
            % lamp
            \cmidrule{2-7}
            & lamp & Turn off/on the Lamp & \texttt{[empty, any]} & \texttt{[-, -]} & $85$ & Left\\
            \midrule 
            \multirow{6}{*}{\begin{tabular}{c}
                 Expressive Motions
            \end{tabular}
            } & \multirow{6}{*}{None} & Cheers & \texttt{[any, can]} & \texttt{[-, -]} & $66$ & Dual-Arm\\
            & & Wave & \texttt{[any, empty]} & \texttt{[-, -]} & $39$ & Dual-Arm\\
            & & Shake Hands & \texttt{[any, empty]} & \texttt{[-, -]} & $51$ & Dual-Arm\\
            & & Take Photo & \texttt{[any, empty]} & \texttt{[-, -]}  & $31$ & Dual-Arm\\
            & & Thumb Up & \texttt{[empty, empty]} & \texttt{[-, -]}  & $22$ & Dual-Arm\\
            & & Spread out Hands & \texttt{[empty, empty]} & \texttt{[-, -]} & $26$ & Dual-Arm\\
            \bottomrule
        \end{tabular}
    \end{center}
\end{table*}

\noindent\textbf{2) Scenario 2: Humanoid as an Office Assistant}

In this scenario, the leader human sits across the humanoid robot at an office table.
This time the robot transforms into an office assistant and deals with complicated cases such as stamping paper for approval, settling a baseball cap on the rack, picking and handing a book over, and reacting properly if the human takes a snap in working.
There are $7$ skills related to $4$ objects in this scenario.

\begin{itemize}[leftmargin=*]

\item \textit{Settle Cap}: The robot gets a cap from the leader human's hand and settles it on a hat rack with its right arm.
The skill showcase begins with the human holding the cap with both hands and ends with the robot pulling its hand back from the hat rack.

\item \textit{Handover Cap}: The robot takes the cap off the hat rack and sends it to the leader human.
It is the reverse skill of \textit{Settle Cap}.
The related intention is inferred when seeing the human pointing to the rack.
The success condition is that the human has received the cap.

\item \textit{Pick Book}: The robot picks a book from the shelf and hands it over.
The skill begins with the human gesturing toward the book.
When the human takes the book, this skill is completed successfully.

\item \textit{Pick Stamp}: The robot picks up the stamp on the table with its right hand. 
The skill succeeds when the stamp is lifted near the hand in an idle posture.
The leader human instructs the execution of this skill by passing along the paper.

\item \textit{Stamp the Paper}: It is a delicate operation to make an issue for approval.
The robot presses the stamp down onto the paper to mark a sign.
This skill is considered successful only if one mark is imprinted.
It is noteworthy that printing more than one mark in a single execution means that the model fails to predict the ending, thus being treated as a failure case.
The sign of the related intention is the leader human pointing at the paper.
To make an in-skill interruption, the human covers the paper with a hand to make the robot withdraw its hand if the pressing is not done.

\item \textit{Place Stamp}: The robot places the stamp back with its right hand.
It is the reverse skill of \textit{Pick Stamp} and is triggered by withdrawing the paper.

\item \textit{Turn off/on the Lamp}: Turning on and off the lamp share the same motion, and thus are trained as one skill.
When the human slumps over the office desk to take a nap, the robot taps the switch of the lamp to turn off it.
And when the human wakes up and lifts the head, the robot operates the same motion to turn on the lamp.
    
\end{itemize}

\noindent\textbf{3) Interactive Motion Skills}

Some skills are not involved with object operation and, thus, are not trained as manipulation skills. 
They are noted as motion skills.
These skills are considered successful when the robot performs the motion properly as the human shows the intention and recovers to the idle posture when the intention no longer sustains.

\begin{itemize}[leftmargin=*]
\item \textit{Cheers}: The robot reaches out the right hand to touch the bottle held by the right hand of the human.
Though holding a Coke can in the right hand during deployment, the robot does not manipulate the object.
For this reason, this skill is not trained in a manipulation demonstration model.

\item \textit{Wave}: The robot lifts up the right hand and waves the right hand when the leader human is waving also.

\item \textit{Shake Hands}: The robot stretches its right hand out to touch the hand of the leader human with a handshaking posture.

\item \textit{Take Photo}: The robot lifts up the right hand and makes a V-sign when the human raises the phone to take a photo, and puts the hand done as the human puts away the phone.

\item \textit{Thumb Up}: The robot reaches both hands out with the thumbs up as the human gives it a thumb-up.
Human intention with the left hand, right hand, or both is approved.

\item \textit{Spread out Hands}: The robot stretches its arms out to the sides with palms up when the leader human spreads its hands out.
    
\end{itemize}

\subsection{Prompt for VLMs}
Here are the prompts we give to Qwen and GPT-4o-mini in evaluation of the intention prediction module.

% For the \textbf{Dining} scenario, our prompt is:
\noindent\begin{tcolorbox}[
    colframe=darkgray, % Dark grey frame color
    boxrule=0.5pt, % Frame thickness
    colback=lightgray!20, %
    arc=3pt, % Rounded corners
    fontupper=\small,
    % width=.475\textwidth,
    breakable, title={Prompt for the \textbf{Dining} scenario},
    % height=9cm,
    ]
    
\textit{You are a humanoid robot sitting in front of a human and equipped with a camera slightly tilted downward on your head, providing a first-person perspective. I am assigning you a new task to recognize to human gestures in front of you. Remember, the person is sitting facing you, so be mindful of their gestures. If the person is holding a cup to you and trying to cheer with you, answer `Cheers'. If the person is giving you a thumbs-up, answer `Thumbup'. If the person extends their right hand to shake hands with you, answer `ShakingHand'.If the person is waving to you with the right hand, answer `Waving`. If the person is taking a photo of you with a cellphone, answer `Taking Photo`. If the person is spreading out both hands in a gesture of resignation, answer `Spreading Hands`. If the person is pointing to a Coke can in the middle of the table (on your right side), answer `Pointing Can'. If the person is pointing to an empty spot on the table with no objects (on your right side), answer `Pointing Table`. If the person is pointing to a tissue box at the far left of the table, answer `Pointing TissueBox'. If the person is pointing to a plate in the middle of the table (just in front of you), answer `Pointing Plate'. If the person is holding out the right hand with the palm open toward you, answer `Palmup'. If the person is handing you a plate, answer `Handing Plate'. If the person is clenching their right fist, holding their left hand open and upward, and placing their right hand above the left as if pretending to wash a plate, answer `Washing'. If the person is pointing at a stack of plates on the left side of the table, answer `Pointing Plates'. If the person is pointing at a sponge on the right side of the table, answer `Pointing Sponge'. If the person is crossing his arms to form an X shape, answer `Cancel'. If no significant gestures are made, answer `Idle'. 
Respond directly with the corresponding options [Cheers, Thumbup, ShakingHand, Pointing Can, Pointing TissueBox, Pointing Plate, Palmup, Handing Plate, Washing, Pointing Plates, Pointing Sponge, Cancel, Idle] based on the current image and observed gestures. Directly reply with the chosen answer only, without any additional characters.}
\end{tcolorbox}

% For the \textbf{Office} scenario, our prompt is:
\noindent\begin{tcolorbox}[
    colframe=darkgray, % Dark grey frame color
    boxrule=0.5pt, % Frame thickness
    colback=lightgray!20, %
    arc=3pt, % Rounded corners
    fontupper=\small,
    % width=.475\textwidth,
    breakable, title={Prompt for the \textbf{Office} scenario},
    % height=9cm,
    ]
\textit{You are a humanoid robot sitting in front of a human and equipped with a camera slightly tilted downward on your head, providing a first-person perspective. I am assigning you a new task to recognize human gestures in front of you. Remember, the person is sitting facing you, so be mindful of their gestures. If the person is giving you a thumbs-up, answer `Thumbup`. If the person extends their right hand to shake hands with you, answer `ShakingHand`. If the person is waving to you with the right hand, answer `Waving`. If the person is taking a photo of you with a cellphone, answer `Taking Photo`. If the person is spreading out both hands in a gesture of resignation, answer `Spreading Hands`. If the person is handing you a cap, answer `Handing Cap`. If the person is pointing at a cap place on the right of the table, answer `Pointing Cap`. If the person is handing a document to you with both hands and you are NOT holding a stamp, answer `Handing File`. If a document is placed in the center of the table in front of you, and the person is pointing to it with the right hand, answer `Pointing File`. If the person retrieves the document from your side of the table to the other side, directly across from you, and you are still holding the stamp, answer `Retrieve File`. If the person is lying down on the table and the lamp is ON, answer `Lie Down`. If the person is sitting up from the table and the lamp is OFF, answer `Sit up`. If the person is pointing at the books standing in the top left corner of the table, answer `Pointing Book`. If the person is crossing the arms to form an X shape, answer `Cancel`. If no significant gestures are made, answer `Idle`. You are NOT holding a stamp right now and the lamp is now ON, observe the image and gestures carefully. Respond directly with the corresponding options [Pointing Book, Handing Cap, Pointing Cap, Handing File, Pointing File, Retrieve File, Lie Down, Sit up, ShakingHand, Thumbup, Cancel, Idle]. Directly reply with the chosen answer ONLY, without any additional characters.}
\end{tcolorbox}

The sentence `\textit{You are NOT holding a stamp right
now and the lamp is now ON}' is modified at each query according to the current situation (whether the robot is holding a stamp and whether the lamp is on).

\subsection{Implementation of RHINO Modules}

\label{app:implementation}

\noindent\textbf{1) Reactive Planner}
% Initialize skillState ← IDLE  
% Initialize occupancyState ← EMPTY  
% Initialize lastIntention ← NONE  

% function reactivePlanner(currentInput):  
%     predictedIntent ← intentionModel.predict(currentInput)  
%     if predictedIntent is stable for K steps and predictedIntent ≠ lastIntention:  
%         if skillState is a manipulation skill and interruptionAllowed:  
%             skillState ← reverseOf(skillState)  
%             occupancyState ← updateOccupancyState()  

%         if startConditionFor(predictedIntent) is satisfied by occupancyState:  
%             skillState ← correspondingSkill(predictedIntent)  
%         else:  
%             path ← findPath(occupancyState, startConditionFor(predictedIntent))  
%             skillState ← executePath(path)  

%         lastIntention ← predictedIntent  

%     if currentSkillSucceeded(skillState) or currentSkillFailed(skillState):  
%         skillState ← IDLE  

%     return skillState  

% \begin{algorithm}[t]
%     \caption{Reactive Planner Pseudo-code}
%     \label{alg:reactive_planner}
%     \SetAlgoLined
%     \KwIn{
%       \\
%       \quad $M_t, H_t$: human body and hand posture at time $t$\\
%       \quad $B^i_t$: 3D bounding boxes of objects/hands at time $t$\\
%       \quad $P_t$: human head pose at time $t$\\
%       \quad $\text{IntentionPredictor}(\cdot)$: Transformer-based model for human intention\\
%       \quad $k$: number of consecutive frames required to confirm an intention\\
%       \quad $\mathcal{S}$: set of all possible skills\\
%       \quad $\text{startCondition}(s)$: the hand-occupancy requirement for skill $s$\\
%       \quad $\text{endTransition}(s)$: the change of hand occupancy after $s$ completes\\
%       \quad $\mathcal{G}$: directed graph of hand-occupancy transitions
%     }
%     \KwOut{Next robot action (skill execution)}
    
%     \textbf{Initialize:}\\
%     \quad $\text{currentSkill} \leftarrow \text{idle}$ \\
%     \quad $\text{currentOccupancy} \leftarrow [~]$ \ \tcp{Empty if no object in either hand}
%     \quad $\text{consecutiveCount}[i] \leftarrow 0, \forall i \in \text{AllIntentions}$ \\
%     \quad $\text{pendingSkill} \leftarrow \text{None}$ \\
    
%     \While{robot is running}{
%         \textbf{Step 1: Capture human behavior.}\\
%         \quad Retrieve sensor data $(M_t, H_t, B^i_t, P_t)$ from the RGB-D camera and detection models.\\
        
%         \textbf{Step 2: Predict human intention.}\\
%         \quad $I_t \leftarrow \text{IntentionPredictor}(M_t, H_t, B^i_t, P_t)$\\
%         \quad \ForEach{possible intention $i$}{
%             \uIf{$i = I_t$}{
%                 $\text{consecutiveCount}[i] \leftarrow \text{consecutiveCount}[i] + 1$\\
%             }
%             \Else{
%                 $\text{consecutiveCount}[i] \leftarrow 0$\\
%             }
%         }
        
%         \textbf{Step 3: Confirm intention for skill switching.}\\
%         \quad $I^* \leftarrow \arg\max_{i \in \text{AllIntentions}} \text{consecutiveCount}[i]$\\
%         \quad \If{$\text{consecutiveCount}[I^*] \geq k$}{
%             \textcolor{blue}{\tcp{We have a stable intention $I^*$}}
%             \quad $\text{pendingSkill} \leftarrow \text{mapIntentionToSkill}(I^*)$\;
%         }
        
%         \textbf{Step 4: Check if a skill switch is needed.}\\
%         \quad \uIf{$\text{pendingSkill} \neq \text{None}$}{
%             \quad \textcolor{blue}{\tcp{Interrupt current skill if needed}}
%             \quad \If{$\text{currentSkill} \neq \text{idle}$ \textbf{and} $\text{pendingSkill} \neq \text{currentSkill}$}{
%                 \quad \textcolor{blue}{\tcp{Handle interruption of manipulation skill if active}}
%                 \quad \text{reverseSkill}(\text{currentSkill})\;
%             }
%             \quad \textcolor{blue}{\tcp{Attempt to start the pending skill}}
%             \quad $s \leftarrow \text{pendingSkill}$\;
%             \quad \If(\tcp*[f]{Check start condition}){$\text{startCondition}(s)$ \textbf{is not satisfied by} $\text{currentOccupancy}$}{
%                 \quad \textcolor{blue}{\tcp{Find shortest path in occupancy graph $\mathcal{G}$ to fulfill start condition}}
%                 \quad $path \leftarrow \text{findOccupancyPath}(\mathcal{G}, \text{currentOccupancy}, \text{startCondition}(s))$\;
%                 \quad \ForEach{skill $p$ \textbf{in} $path$}{
%                     \quad Execute $p$ to modify $\text{currentOccupancy}$\;
%                 }
%             }
%             \quad \textcolor{blue}{\tcp{Now that occupancy requirements are met, start skill $s$}}
%             \quad $\text{currentSkill} \leftarrow s$\;
%             \quad $\text{pendingSkill} \leftarrow \text{None}$\;
%         }
        
%         \textbf{Step 5: Execute the current skill.}\\
%         \quad \uIf(\tcp*[f]{For manipulation skill}){$\text{currentSkill}$ \text{is manipulation}}{
%             \quad \If{\text{skill succeeds}}{
%                 \quad $\text{endTransition}(\text{currentSkill}) \rightarrow \text{currentOccupancy}$\;
%                 \quad $\text{currentSkill} \leftarrow \text{idle}$\;
%             }
%             \quad \ElseIf{\text{skill fails or times out}}{
%                 \quad \text{handleFailure}(\text{currentSkill})\;
%                 \quad $\text{currentSkill} \leftarrow \text{idle}$\;
%             }
%         }
%         \quad \ElseIf(\tcp*[f]{For motion skill}){$\text{currentSkill}$ \text{is motion}}{
%             \quad \If(\tcp*[f]{Stop if new intention or canceled}){a new stable intention emerges}{
%                 \quad $\text{currentSkill} \leftarrow \text{idle}$\;
%             }
%             \quad \Else{
%                 \quad Continue motion skill execution\;
%             }
%         }
%         \quad \Else{
%             \quad \textcolor{blue}{\tcp{Idle state: do nothing unless new skill arrives}}
%             \quad \text{currentSkill} \leftarrow \text{idle}\;
%         }
        
%         \textbf{Step 6: Repeat at 30Hz.}\\
%         \quad \textcolor{blue}{\tcp{Proceed to the next time step}}
%     }
%     \end{algorithm}
    
    % \begin{algorithm}[t]
    %     \caption{Reactive Planner Simplified Pseudo-code}
    %     \label{alg:reactive_planner_simplified}
    %     \SetAlgoLined
    %     \KwIn{
    %       \quad $M_t, H_t$: human body and hand posture at time $t$ \\
    %       \quad $B^i_t$: 3D bounding boxes of objects/hands at time $t$ \\
    %       \quad $P_t$: human head pose at time $t$ \\
    %       \quad $\mathcal{S}$: set of skills \\
    %       \quad $k$: number of frames for stable intention
    %     }
    %     \KwOut{Next robot action (skill execution)}
        
    %     \textbf{Initialize:} \\
    %     \quad $\text{currentSkill} \leftarrow \text{idle}$, $\text{currentOccupancy} \leftarrow [~]$ \\
    %     \quad $\text{pendingSkill} \leftarrow \text{None}$, $\text{consecutiveCount}[i] \leftarrow 0, \forall i$ \\
    %     \While{robot is running}{
    %         \textbf{Capture data:} \\
    %         \quad Retrieve $(M_t, H_t, B^i_t, P_t)$ from sensors.\\
            
    %         \textbf{Predict intention:} \\
    %         \quad $I_t \leftarrow \text{IntentionPredictor}(M_t, H_t, B^i_t, P_t)$. \\
    %         \quad \ForEach{intention $i$}{
    %             \quad $\text{consecutiveCount}[i] \leftarrow \text{consecutiveCount}[i] + 1$ if $i = I_t$ else 0.\\
    %         }
            
    %         \textbf{Confirm intention:} \\
    %         \quad $I^* \leftarrow \arg\max \text{consecutiveCount}[i]$ \\
    %         \quad \If{$\text{consecutiveCount}[I^*] \geq k$}{
    %             \quad $\text{pendingSkill} \leftarrow \text{mapIntentionToSkill}(I^*)$. \\
    %         }
            
    %         \textbf{Skill switch:} \\
    %         \quad \If{$\text{pendingSkill} \neq \text{None}$}{
    %             \quad \If{$\text{currentSkill} \neq \text{idle}$ \textbf{and} $\text{pendingSkill} \neq \text{currentSkill}$}{
    %                 \quad $\text{reverseSkill}(\text{currentSkill})$. \\
    %             }
    %             \quad \If{$\text{startCondition}(\text{pendingSkill})$ is met by $\text{currentOccupancy}$}{
    %                 \quad $\text{currentSkill} \leftarrow \text{pendingSkill}$. \\
    %                 \quad $\text{pendingSkill} \leftarrow \text{None}$. \\
    %             }
    %             \quad \Else{
    %                 \quad \text{Find and execute preparatory skills using the occupancy transition graph}. \\
    %             }
    %         }
            
    %         \textbf{Execute skill:} \\
    %         \quad \If{$\text{currentSkill}$ is manipulation}{
    %             \quad \If{skill succeeds}{ 
    %                 \quad $\text{endTransition}(\text{currentSkill})$. \\
    %                 \quad $\text{currentSkill} \leftarrow \text{idle}$. \\
    %             }
    %             \quad \ElseIf{skill fails or times out}{
    %                 \quad \text{handle failure}. \\
    %                 \quad $\text{currentSkill} \leftarrow \text{idle}$. \\
    %             }
    %         }
    %         \quad \ElseIf{$\text{currentSkill}$ is motion}{
    %             \quad \If{new intention emerges}{
    %                 \quad $\text{currentSkill} \leftarrow \text{idle}$. \\
    %             }
    %         }
    %         \quad \Else{
    %             \quad \text{Stay idle until next skill request}. \\
    %         }
    %     }
    %     \end{algorithm}

\begin{algorithm}
\caption{Pseudo-code for Skill Transitions of Reactive Planner.}
\begin{algorithmic}[1]
\label{alg:planner}
\STATE $Skill \gets \text{Idle}$
\WHILE{$true$}
    \STATE $human\_intention \gets \text{Recognize\_Human\_Intention()}$
    \IF{human intention is stable for $k$ frames and human intention != current intention}
        \IF{human intention $=$ Idle and Skill $=$ Manipulation}
            % continue the loop
            \STATE Continue
        \ENDIF
        \IF{Skill $=$ Manipulation and interruptionAllowed}
            \STATE $Skill \gets \text{Reverse\_Skill}(Skill)$
        \ENDIF
        \IF{$\text{Start\_Condition}(human\_intention)$ is not satisfied by hand occupancy}
             \STATE $path \gets$  FindPath(occupancy, StartCondition $(human\_intention))$
            % \STATE $\begin{aligned}
            %     path \gets & \text{FindPath}(\text{occupancy}, \\
            %     & \text{StartCondition}(human\_intention))
            % \end{aligned}$
            \STATE $Skill \gets \text{Execute\_Path}(path)$
        \ELSE
            \STATE $Skill \gets \text{Corresponding Skill} (human\_intention)$
        \ENDIF
    \ELSIF{SkillSucceeded(Skill) or SkillTimeout(Skill)}
        \STATE $Skill \gets \text{Idle}$
        \IF{SkillSucceeded(Skill)}
            \STATE $\text{Hand Occupancy} \gets \text{End\_Transition}(Skill)$
        \ENDIF
    \ENDIF
\ENDWHILE
\end{algorithmic}
\end{algorithm}


% \begin{figure}[b]
%     \centering
%     % \includegraphics[width=0.8\linewidth]{example-image}
%     \includegraphics[width=\linewidth]{figs/src/grp1.pdf}
%     \caption{\small \textbf{Setup of motion capture system.} a) The two Motion Capture Cameras are used to detect the ArUco markers. The video recorded by the RGB-D Camera is used to process human motion and human hand details of the Leader. b) Follower and Leader both wear wristbands with ArUco markers for hand position detection. c) We 3D-print our wristbands with 4 ArUco markers on 4 surfaces and embed a IMU beneath the upper surface.}
    
%     % \label{fig:intent_corner}
%     \label{fig:mocap_sys}
% \end{figure}
\begin{figure}[t!]
    \centering
    \begin{subfigure}[t]{\linewidth}
        \centering
        \includegraphics[width=\linewidth]{figs/src/grp1.pdf}
        \caption{Dining scenario.}
    \end{subfigure}%
    \\
    \begin{subfigure}[t]{\linewidth}
        \centering
        \includegraphics[width=0.8\linewidth]{figs/src/grp2.pdf}
        \caption{Office scenario.}
    \end{subfigure}
    \caption{Occupancy graph of two different scenarios.}\label{fig:occupancy}
\end{figure}

% \begin{table}
    \begin{center}
     \caption{Hyper-parameters of the Transformer backbone for Reactive Planner.}
        \begin{tabular}{cc}
            \toprule
            hyper-parameter & value \\
            \midrule
            latent dimension  & $128$ \\
            num head &$8$ \\
            num layers & $3$ \\
            batch size & $256$ \\
            feed-forward dimension & $128$ \\
            maximum epoch & $300$ \\
            learning rate & $0.0001$ \\
            \bottomrule
        \end{tabular}
    \end{center}
    \label{tab:planner-hyper}
\end{table}
The switching logic of the skill planner is listed in ~\Cref{alg:planner}.
The directed graphs of occupancy are shown in~\Cref{fig:occupancy}.
Here we further explain the \texttt{Recognize\_Human\_Intention()} function in detail, which is implemented as a transformer-based classifier. The model input includes:
\begin{itemize}[leftmargin=*]
    \item \textbf{Upper Body Human Posture}: a 36-dim human upper body skeleton, namely the 6D rotation of the wrist, elbow and shoulder joints for each arm.
    \item \textbf{Human Hand Pose}: a 12-dim human hand pose vector. For each hand, we retarget the detected human hand pose to our robot hand with IK, and take the 6 joint pos as human hand pose vector.
    \item \textbf{Robot Hand Occupancy}: a 10-dim robot hand occupancy label. Since we have at most 5 objects in total (Can, Cup, Plate, Sponge, Tissue), we use a 5-dim one-hot label for each hand to represent the object held in the robot's hand. If the robot is not holding anything, the label will be all-zeros.
    \item \textbf{Human Details}: a 19-dim vector, including the x and y-axis of each human hand position, the z-axis (height) of the human head position, and a 7-dim label for the nearest object to each hand. The nearest object label is concatenated by a 5-dim one-hot label of the object type, the distance from the object to the human hand, and the average of IOU and IOFs of the object bounding box and the human hand bounding box.
\end{itemize}
We use an MLP encoder to encode the concatenated vector of  \textbf{Upper Body Human Posture}, \textbf{Human Hand Pose} and \textbf{Robot Hand Occupancy}, and another MLP to encode \textbf{Human Details} to latent dimension. The concatenated latent vector is processed by a Transformer backbone, followed by a final MLP layer to predict the human intention class. The 
hyper-parameters of the Transformer backbone are listed in ~\Cref{tab:planner-hyper}.

% \begin{table}[t]
%     \begin{center}
%      \caption{Hyper-parameters of the ACT model for manipulation skills.}
%      \label{tab:act-hyper}
%         \begin{tabular}{cc}
%             \toprule
%             hyper-parameter & value \\
%             \midrule
%             KL weight  & 10 \\
%             Cross-entropy weight &1 \\
%             chunk size & 30 \\
%             hidden dimension & 512 \\
%             batch size & 45 \\
%             feed-forward dimension & 3200 \\
%             maximum epoch & 50000 \\
%             learning rate & 0.00005 \\
%             \bottomrule
%         \end{tabular}
%     \end{center}
% \end{table}

\begin{table*}[t]
    \begin{minipage}[t]{0.3\linewidth}
        \begin{center}
            \caption{Hyper-parameters of the Reactive Planner.}
            \label{tab:planner-hyper}
            \resizebox{0.9\linewidth}{!}{
            \begin{tabular}{cc}
                \toprule
                hyper-parameter&value\\
                \midrule
                latent dimension&128\\
                num head&8\\
                num layers&3\\
                batch size&256\\
                feed-forward dimension&128\\
                maximum epoch&300\\
                learning rate&0.0001\\
                \bottomrule
            \end{tabular}
            }
        \end{center}
    \end{minipage}
    \hfill
    \begin{minipage}[t]{0.3\linewidth}
        \begin{center}
            \caption{Hyper-parameters of the Motion Generation Model.}
            \label{tab:motion-hyper}
            \resizebox{0.9\linewidth}{!}{
            \begin{tabular}{cc}
                \toprule
                hyper-parameter&value\\
                \midrule
                latent dimension&256\\
                num head&8\\
                num layers&4\\
                feed-forward dimension&256\\
                diffusion steps&300\\
                sampling steps&30\\
                batch size&512\\
                maximum epoch&4000\\
                learning rate&0.0001\\
                \bottomrule
            \end{tabular}
            }
        \end{center}
    \end{minipage}
    \hfill
    \begin{minipage}[t]{0.3\linewidth}
        \begin{center}
            \caption{Hyper-parameters of the ACT model for manipulation skills.}
            \label{tab:act-hyper}
            \resizebox{0.9\linewidth}{!}{
            \begin{tabular}{cc}
                \toprule
                hyper-parameter&value\\
                \midrule
                KL weight&10\\
                Cross-entropy weight&1\\
                chunk size&30\\
                hidden dimension&512\\
                batch size&45\\
                feed-forward dimension&3200\\
                maximum epoch&50000\\
                learning rate&0.00005\\
                \bottomrule
            \end{tabular}
            }
        \end{center}
    \end{minipage}
\end{table*}



\noindent\textbf{2) Interactive Motion Generation}

% \begin{table}
    \begin{center}
     \caption{Hyper-parameters of the Interactive Motion Generation Model.}
        \begin{tabular}{cc}
            \toprule
            hyper-parameter & value \\
            \midrule
            latent dimension  & $256$ \\
            num head &$8$ \\
            num layers & $4$ \\
            feed-forward dimension & $256$ \\
            diffusion steps & $300$ \\
            sampling steps & $30$ \\
            batch size & $512$ \\
            maximum epoch & $4000$ \\
            learning rate & $0.0001$ \\
            \bottomrule
        \end{tabular}
    \end{center}
    \label{tab:motion-hyper}
\end{table}
For interactive motion generation, we use a transformer-based diffusion model, which denoises the past 30 frames of human and robot motions and future 5 frames of robot motions. Both human motion and robot motion consist of upper-body motion (36-dim for humans and 10-dim for humanoid), hand motion (6-dim for each hand,) and hand occupancy label (5-dim one-hot label for each hand). Besides, the predicted human intention label is also conditioned during the diffusion process. The hyper-parameters of our model are listed in \Cref{tab:motion-hyper}.


\noindent\textbf{3) Manipulation Skills}

Thanks to the stability of model training, most of the hyper-parameters are basically consistent across all skills.
The volume of data for training each skill is shown as a column in \Cref{tab:skills}.
The hyper-parameters in training ACT models~\cite{zhao2023learning} are shown as \Cref{tab:act-hyper}.
% We mainly adopted those from ~\cite{cheng2024tv} and adjusted some important hyper-parameters.

\begin{table*}[htb]
    \begin{center}
        \caption{Performance of manipulation module across manipulation skills. }
        \label{tab:manipulation_detailed}
        % \begin{tabular}{c|c|c|c|c|c|c}
        \begin{tabular}{ccccccc}
        \toprule
        \textbf{Scenarios} & \textbf{Object} & \textbf{Skill Name} & \textbf{Success Rate} & \textbf{Average Time} & \begin{tabular}{c}
             \textbf{Success Rate}  \\
             \textbf{(Human)}
        \end{tabular} & \begin{tabular}{c}
             \textbf{Average Time}  \\
             \textbf{(Human)}
        \end{tabular} \\
            \midrule
            % can
            \multirow{11}{*}{\begin{tabular}{c}
                 Scenario 1 \\
                 Dining Waiter
            \end{tabular}}  & \multirow{2}{*}{can} & Pick Can & 1.00 & 5.31 & 1.00 & 5.77\\
            & & Place Can & 1.00 & 4.10 & 0.93 & 4.65\\
            % plate
            \cmidrule{2-7}
            & \multirow{4}{*}{plate} & Get Plate from Human  & 1.00 & 4.86 & 0.98 & 5.12 \\
            & & Place Plate to Stack & 0.95 & 8.19 & 0.97 & 6.91 \\
            & & Pick Plate from Table & 0.90 & 10.75 & 0.96 & 8.60 \\
            & & Handover Plate & 1.00 & 5.79 & 1.00 & 5.14 \\
            % sponge
            \cmidrule{2-7}
            & \multirow{3}{*}{sponge} & Pick Sponge  & 0.95 & 8.19 & 1.00 & 7.45 \\
            & & Brush with Sponge & 0.90 & 10.02 & 1.00 & 4.18 \\
            & & Place Sponge & 0.85 & 5.57 & 0.98 & 5.41 \\
            % tissue
            \cmidrule{2-7}
            & tissue & Pick a Piece of Tissue  & 0.95 & 9.43 & 0.91 & 9.54\\
            \midrule
            % cap
            \multirow{8}{*}{\begin{tabular}{c}
             Scenario 2  \\
             Office Assistant
            \end{tabular}} & \multirow{2}{*}{cap} & Settle Cap & 1.00 & 7.50 & 0.91 & 8.50\\
            & & Handover Cap & 0.85 & 8.64 & 0.90 & 10.48 \\
            % book
            \cmidrule{2-7}
            & book & Pick Book & 0.95 & 10.81 & 0.93 & 10.21 \\
            % stamp
            \cmidrule{2-7}
            & \multirow{3}{*}{stamp} & Pick Stamp & 1.00 & 4.80 & 0.92 & 3.91 \\
            & & Stamp the Paper & 0.80 & 5.64 & 0.92 & 3.11 \\
            & & Place Stamp & 1.00 & 4.74 & 0.93 & 4.83 \\
            % lamp
            \cmidrule{2-7}
            & lamp & Turn off/on the Lamp & 1.00 & 5.06 & 0.96 & 3.53 \\
            % \midrule 
            % \multirow{6}{*}{\begin{tabular}{c}
            %      Expressive Motions
            % \end{tabular}
            % } & \multirow{6}{*}{None} & Cheers & & & $0$ & Dual-Arm\\
            % & & Wave & & & $0$ & Dual-Arm\\
            % & & Shake Hands & & & $0$ & Dual-Arm\\
            % & & Take Photo & & & $0$ & Dual-Arm\\
            % & & Thumb Up & & & $0$ & Dual-Arm\\
            % & & Spread out Hands & & & $0$ & Dual-Arm\\
            \bottomrule
        \end{tabular}
    \end{center}
\end{table*}

For the prediction of the success signal, we marked the last $n_{s}$ frames of the recorded data as $1$ (completed) and other frames as $0$ to generate a 0/1 label.
$n_{s}$ is set to $25$ in most of the skills and shifted to $10$ in three of them of which the ending frames changed sharply in motion.
The special skills are \textit{Pick Stamp}, \textit{Stamp the Paper}, and \textit{Place Stamp}.

% Jingxiao: Ignore some details
% \begin{table*}
    \begin{center}
        \caption{Hyper-parameters in real-world deployment across the manipulation tasks. Notes: \textbf{threshold} indicated the \textbf{progress threshold}, \textbf{initial} for \textbf{initial joint position}, \textbf{warm-up} and \textbf{time-out} means the two timings in termination prediction respectively.}
        \label{tab:skill_hyper}
        % \begin{tabular}{c|c|c|c|c|c|c}
        \begin{tabular}{ccccccc}
        \toprule
        \textbf{Scenarios} & \textbf{Object} & \textbf{Task Name} & \begin{tabular}{c}
             \textbf{threshold}  % \\
             % \textbf{threshold}
        \end{tabular} & \begin{tabular}{c}
             \textbf{initial}  % \\
             % \textbf{(Human)}
        \end{tabular} & \begin{tabular}{c}
             \textbf{warm-up}  % \\
             % \textbf{(Human)}
        \end{tabular} & \begin{tabular}{c}
             \textbf{time-out} % \\
             % \textbf{(Human)}
        \end{tabular} \\
            \midrule
            % can
            \multirow{11}{*}{\begin{tabular}{c}
                 Scenario 1 \\
                 Dining Waiter
            \end{tabular}}  & \multirow{2}{*}{can} & Pick Can & $0.85$ & \texttt{predicted} & $2$ & $15$\\
            & & Place Can & $0.85$ & \texttt{fixed} & $2$ & $15$\\
            % plate
            \cmidrule{2-7}
            & \multirow{4}{*}{plate} & Get Plate from Human  & $0.90$ & \texttt{fixed} & $2$ & $15$ \\
            & & Place Plate to Stack & $0.85$ & \texttt{fixed} & $2$ & $15$ \\
            & & Pick Plate from Table & $0.85$ & \texttt{fixed} & $2$ & $15$ \\
            & & Handover Plate & $0.85$ & \texttt{fixed} & $2$ & $15$ \\
            % sponge
            \cmidrule{2-7}
            & \multirow{3}{*}{sponge} & Pick Sponge  & $0.85$ & \texttt{fixed} & $2$ & $15$ \\
            & & Brush with Sponge & \texttt{unused} & \texttt{predicted} & $2$ & $10$ \\
            & & Place Sponge & $0.85$ & \texttt{fixed} & $2$ & $15$ \\
            % tissue
            \cmidrule{2-7}
            & tissue & Pick a Piece of Tissue  & $0.85$ & \texttt{fixed} & $2$ & $15$\\
            \midrule
            % cap
        \multirow{8}{*}{\begin{tabular}{c}
             Scenario 2  \\
             Office Assistant
        \end{tabular}} & \multirow{2}{*}{cap} & Settle Cap & $0.85$ & \texttt{fixed} & $2$ & $15$\\
            & & Handover Cap & $0.85$ & \texttt{predicted} & $2$ & $15$ \\
            % book
            \cmidrule{2-7}
            & book & Pick Book & $0.85$ & \texttt{predicted} & $2$ & $15$ \\
            % stamp
            \cmidrule{2-7}
            & \multirow{3}{*}{stamp} & Pick Stamp & $0.95$ & \texttt{fixed} & $2$ & $15$ \\
            & & Stamp the Paper & $0.85$ & \texttt{predicted} & $2$ & $10$ \\
            & & Place Stamp & $0.85$ & \texttt{predicted} & $2$ & $15$ \\
            % lamp
            \cmidrule{2-7}
            & lamp & Turn off/on the Lamp & $0.85$ & \texttt{fixed} & $2$ & $15$ \\
            % \midrule 
            % \multirow{6}{*}{\begin{tabular}{c}
            %      Expressive Motions
            % \end{tabular}
            % } & \multirow{6}{*}{None} & Cheers & & & $0$ & Dual-Arm\\
            % & & Wave & & & $0$ & Dual-Arm\\
            % & & Shake Hands & & & $0$ & Dual-Arm\\
            % & & Take Photo & & & $0$ & Dual-Arm\\
            % & & Thumb Up & & & $0$ & Dual-Arm\\
            % & & Spread out Hands & & & $0$ & Dual-Arm\\
            \bottomrule
        \end{tabular}
    \end{center}
\end{table*}

% Some simple but effective tricks are utilized to make skills perform better when deployed on the robot. 
% Firstly, we set a \textbf{progress threshold} to filter the noise of terminal condition prediction. 
% During deployment, the skill execution is terminated when the model predicts a success signal larger than the threshold.
% The threshold is typically $0.85$ and varies in certain skills.
% Moreover, to avoid verbose discussion that the robot should start from any state to finish the skill, we set the control target to an \textbf{initial joint position} before performing a skill.
% The initial position is typically selected as the first action of one slice in the collected dataset (noted as \texttt{fixed}) and is replaced by the model prediction of the first state (noted as \texttt{predicted}) in certain skills in consideration of robustness.
% However, there are some cases where the start and end state of the same skill are similar, leading to termination soon or dragging on in skill execution.
% We assign a pair of \textbf{warm-up time} and \textbf{time-out period} to avoid this phenomenon.
% A skill can not be predicted to be completed before the former and is regarded as terminated after the latter.
% They are typically $2$ and $15$ seconds respectively, and adjusted in only a few skills.
% In some skills with periodic features in the motion, the \textbf{time-out period} takes the place of \textbf{progress threshold}.
% In those skills, the robot executes a series of periodic actions until time out.

% The detailed selection of hyper-parameters across the manipulation skills is shown as \Cref{tab:skill_hyper}. 
% It shows that the manipulation module needs only a few simple adjustments to work on various skills.




\noindent\textbf{4) Safety Supervisor}


\begin{figure}[t]
    \centering
    \includegraphics[width=0.9\linewidth]{figs/src/safe.pdf}
    \caption{The interface displays the work of the safety supervisor, with sphere markers representing the collision boxes of the human hands and robot arms. These markers move in sync with the interaction. When an unsafe collision is detected, the human hand markers change color from green to red.}
    \label{fig:safe}
\end{figure}

The collision box is calculated using $14$ key points across each arm. The key points at specific joints and their midpoints are identified as follows:

\begin{itemize}[leftmargin=*]
    \item The origins of the shoulder pitch, shoulder yaw, elbow, and wrist joints are defined as key points.
    \item Additional key points include the midpoints between the shoulder yaw and elbow joints, and between the elbow and wrist joints.
    \item A further key point is defined at one-third the distance beyond the elbow towards the wrist, extending from the segment between these two joints.
\end{itemize}
This structured delineation allows for precise calculations pertinent to robotic arm movements within a predefined spatial configuration.

The human hands are shaped by the detected key points from body detection model of ZED API, from which each hand is reconstructed as $5$ points.
Once one of the points is close to any robot key point in $0.1$ meters, an unsafe signal is broadcast to pause the robot control.

We also provide the visualization of the safety supervisor, of which the interface shown in \Cref{fig:safe}. When the human hand key-points collide with any collision box, the supervisor will send an unsafe signal to halt the robot.
Our safety supervisor runs at $30$Hz.


\subsection{Detailed Experiment Results}

\label{app:result}

\noindent\textbf{1) Planner}
\begin{figure*}[t]
    \vspace{-3.0em}

    \centering
    \begin{minipage}{0.48\textwidth}
        \centering
        \includegraphics[height=8.5cm, keepaspectratio]{figs/src/confusion_matrix_ours_dining.pdf}
        \label{fig:image1}
    \end{minipage}
    \begin{minipage}{0.48\linewidth}
        \centering
        \includegraphics[height=8.5cm, keepaspectratio]{figs/src/confusion_matrix_ours_office.pdf}
        \label{fig:image2}
    \end{minipage}

    \vspace{-1.5em}

    \begin{minipage}{0.48\textwidth}
        \centering
        \includegraphics[height=8.5cm, keepaspectratio]{figs/src/confusion_matrix_wopos_dining.pdf}
        \label{fig:image3}
    \end{minipage}
    \begin{minipage}{0.48\linewidth}
        \centering
        \includegraphics[height=8.5cm, keepaspectratio]{figs/src/confusion_matrix_wopos_office.pdf}
        \label{fig:image4}
    \end{minipage}

    \vspace{-1.5em}

    \begin{minipage}{0.48\textwidth}
        \centering
        \includegraphics[height=8.5cm, keepaspectratio]{figs/src/confusion_matrix_dining_IDFalse_finetuneFalse_GPTTrue.pdf}
        \label{fig:image5}
    \end{minipage}
    \begin{minipage}{0.48\linewidth}
        \centering
        \includegraphics[height=8.5cm, keepaspectratio]{figs/src/confusion_matrix_office_IDFalse_finetuneFalse_GPTTrue.pdf}
        \label{fig:image6}
    \end{minipage}
    
    \caption{\textbf{Confusion Matrices of Our Model (with and without Human Details) and GPT-4o-mini.} To show the results more clearly, we did not color the cell in the top left corner since ”idle” accounts for a significant proportion in the data.}
    \label{fig:Confusion Matrix}
\end{figure*}


We use confusion matrices to show the classification performance of our planner on the test dataset. The confusion matrices for our model, our model without human details and GPT-4o-mini on the test datasets of the dining and office scenarios are shown in \Cref{fig:Confusion Matrix}. 
% To show the results more clearly, We did not color the cell in the top left corner (the cell enclosed by a dashed line) since "idle" accounts for a significant proportion in the data. 

As is shown in the confusion matrices, although the model mainly relies on human body motion and human hand motion input for classification, human details can help the model better deal with certain situations, such as avoiding mis-classification into Idle.

\noindent\textbf{2) Objects Manipulation}

The detailed success rates and average execution times across skills are presented in \Cref{tab:manipulation_detailed}, from which the statistics in \Cref{tab:manipulation} are derived.

In most skills, the manipulation module of \our autonomously executes motions following the patterns of teleoperation data within a comparable time frame.
Trained exclusively on successful human teleoperation cases, the module demonstrates both effectiveness and robustness to slight scene variations during deployment. 
As a result, it achieves higher success rates in skills involving simple motions with abundant training data, such as \textit{Pick Can}, \textit{Handover Plate}, and \textit{Place Stamp}.

However, certain skills pose challenges for the manipulation module. In \textit{Place Plate to Stack} and \textit{Stamp the Paper}, the robot hesitates to drop the plate or press the stamp due to prediction noise. 
In \textit{Pick Plate from Table}, it must overcome increased friction against the table when joint positions deviate from those in the collected data.
Another challenge arises in \textit{Brush with Sponge}, where the success signal predictor struggles to assess the progress of the periodic motion accurately. 
As a result, termination is constrained by a $10$-second timeout. 
These various factors contribute to a longer average execution time for these four skills compared to human performance.

Referring to the experiment on in-skill interruption data presented in \Cref{tab:safety_manipulation}, we select \textit{Pick Can}, \textit{Stamp the Paper}, and \textit{Place Plate to Stack} as representative skills for interruptions occurring during the stages of fetching an object, operating with the object, and returning the object, respectively.

To ensure a controlled data volume across different interruption ratios, we assign a fixed data amount of $M$ to each skill. 
In a full data collection for any given skill, the total data amount is $N$, with $N_{in}$ representing the portion containing in-skill interruptions. 
The ratio of interrupted data in a selected subset is denoted as $\alpha$, meaning that $\lceil\alpha M\rceil$ slices contain interruptions. 
To maintain this proportion, we set $M=N-N_{in}+1$. 
Specifically, $M$ is set to $69$, $66$, and $76$ for the three skills, respectively.

It is worth noting that the assigned data amount is smaller than that used in full data collection models (see \Cref{tab:skills}), which results in a decrease in skill success rate and interrupt success rate compared to the final model.

Each ACT model in this experiment uses the same hyper-parameters as those employed for the corresponding skill in both training and deployment with the full data collection.


\noindent\textbf{3) End2End Policy}


\begin{table}[htb]
    \begin{center}
        \caption{Detailed success rate of \our framework and end-to-end model on different skills.}
        \label{tab:framework_detailed}
        \begin{tabular}{cc|ccccc}
            \toprule
            \multicolumn{2}{c|}{\textbf{Method}} & \textbf{cheers} & \textbf{pick} & \textbf{place} & \textbf{handshake} & \textbf{wave} \\
            \midrule
\multicolumn{2}{c|}{Ours} & 1.00 & 1.00 & 0.85 & 1.00 & 1.00 \\
            \midrule
\multirow{2}{*}{E2E/1} & I.D. & 1.00 & -  & -  & -  & -  \\ & O.O.D.  & 0.95 & -  & -  & -  & -  \\
            \midrule
\multirow{2}{*}{E2E/3} & I.D. & 1.00 & 0.75 & 0.35 & -  & -  \\ & O.O.D.  & 0.95 & 0.30 & 0.45 & -  & -  \\
            \midrule
\multirow{2}{*}{E2E/5} & I.D. & 0.95 & 0.70 & 0.80 & 0.85 & 0.90 \\ & O.O.D.  & 1.00 & 0.55 & 0.35 & 0.60 & 0.85 \\
            \bottomrule
        \end{tabular}
    \end{center}
\end{table}

The detailed result to derive \Cref{tab:framework} is shown in \Cref{tab:framework_detailed}.
For each single skill, we collected $100$ slices of motion in the dataset, which is close to the average volume of all the manipulation skills.
As the small model is not capable of the skills which are totally unseen, we only examine each E2E model with the success rate of skills in the training data.





\end{document}
