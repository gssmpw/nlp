


\section{Conclusion and Future Work}

The work represents a pilot study of integrating retrieval-augmented generation (RAG) into the private training of generative models. We present \system, a novel approach for training differentially private (DP) diffusion models. Through extensive evaluation using benchmark datasets and models, we demonstrate that \system largely outperforms state-of-the-art methods in terms of generative quality, memory footprint, and inference efficiency. The findings suggest that integrating RAG with DP training represents a promising direction for designing privacy-preserving generative models.

This work also opens up several avenues for future research. \mct{i}
Like other approaches for training DP diffusion models and the broader pre-training/fine-tuning paradigm, \system relies on access to a diverse public dataset that captures a range of patterns and shares similar high-level layouts with the private data. It is worth exploring scenarios with highly dissimilar public/private data~\citep{liu2021leveraging,liu2021iterative,fuentes2024joint}.
\mct{ii} In its current implementation, \system retrieves only the top-1 nearest trajectory in RAG. Exploring ways to effectively aggregate multiple neighboring trajectories could improve generative quality and diversity. \mct{iii} While \system's privacy accounting focuses on privatizing the fine-tuning stage, it is worth accounting for random noise introduced by the diffusion process to further improve its privacy guarantee~\citep{wangdp}. \mct{iv} Although this work primarily focuses on image synthesis tasks, given the increasingly widespread use of diffusion models, extending \system to other tasks (e.g., text-to-video generation) presents an intriguing opportunity. 

% ii) With the increasing use of diffusion models (e.g., text-to-video generation~\citep{bar2024lumiere} and protein structure prediction~\citep{abramson2024accurate}), it becomes critical to ensure the generated samples adhere to certain objectives (e.g., physical rules). We consider integrating such compliance into the generative process of diffusion models.


% In this paper, we introduce \system, a \ul{R}etrieval-\ul{A}ugmented \ul{P}r\ul{I}vate \ul{D}iffusion model which integrates retrieval augmentation with Latent Diffusion models with privacy guarantee provided by DP-SGD. We first designed a feasible solution to the bottlenecks to the retrieval of diffusion trajectories: \newline1, We no longer need to generate a brand new trajectory knowledge for each unique initialization of the Sampling process. \newline 2, By replacing the trajectories generated by a pre-trained model with the trajectories of forward generation processes, we significantly reduce the time and resource requirement for each trajectory knowledge base generation. We then integrate our retrieval augmentation with the Differentially Private Latent Diffusion Model \dpldm. Through extensive experiments we have demonstrated that our method improves the performances of Differentially Private Diffusion models while reducing the requirements of excessively large batch sizes. 
