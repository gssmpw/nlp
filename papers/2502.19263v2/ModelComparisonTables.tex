\newcommand{\firsttable}{
\begin{table*}[h]
  \centering
  \renewcommand{\arraystretch}{1.2}
  \resizebox{\textwidth}{!}
  {
    \begin{tabular}{p{0.07\linewidth} p{0.31\linewidth} p{0.31\linewidth} p{0.31\linewidth}} % Adjust column widths as needed
      \textbf{}
       & \begin{minipage}{.25\textwidth}
          \includegraphics[width=\linewidth]{raw_art/rainbow.png}
          \textbf{\textit{Rainbow}}
          \Description{A child's artwork of a rainbow with seven colors, a pot of gold on the bottom right of the rainbow, a green grassy ground, a blue sky, and a yellow sun at the top right of the artwork.}
        \end{minipage} 
        & \begin{minipage}{.25\textwidth}
          \includegraphics[width=\linewidth]{raw_art/abstract_blue_white_red.png}
          \textbf{\textit{AbstractBlueWhiteRed}}
          \Description{An abstract artwork with a dark blue background, some white paintbrush strokes in a circular shape, and a red ghost-like figure in the center-right that has been done on a cut-out piece of paper on top of the blue background paper.}
          \end{minipage} 
        & \begin{minipage}{.2\textwidth}
          \includegraphics[width=\linewidth]{raw_art/peeps.png}
          \textbf{\textit{Peeps}}
          \Description{A coloring book page with several "Peeps" outlined in black, and a few peeps colored in throughout the page in various shades of red, green, yellow, blue, and purple. The bottom right of the page has some words that read: "Express your Peepsonality".}
          \end{minipage} 
        \\
      \hline
        \textbf{Claude 3.5 Sonnet}
        &
        \begin{tabular}{| p{0.17\linewidth} | p{0.16\linewidth} | p{0.17\linewidth} | p{0.17\linewidth} |}
          \textbf{R.A} & \textbf{R.B} & \textbf{R.C} & \textbf{R.D} \\
          \hline
          3 & 4 & 4 & 4 \\
          \hline
        \end{tabular}
        % \vspace{0.2em}
        \newline
        Makes assumptions, such as the black object being a "pot or cauldron" and the yellow line representing "gold or something valuable inside."
        & 
        \begin{tabular}{| p{0.17\linewidth} | p{0.16\linewidth} | p{0.17\linewidth} | p{0.17\linewidth} |}
          \textbf{R.A} & \textbf{R.B} & \textbf{R.C} & \textbf{R.D} \\
          \hline
          2 & 4 & 4 & 4 \\
          \hline
        \end{tabular}
        % \vspace{0.2em}
        \newline
        Assumptions include the red figure being a "person or creature" and suggesting themes like "isolation, adventure, or standing out in a chaotic world." 
        &
        \begin{tabular}{| p{0.17\linewidth} | p{0.16\linewidth} | p{0.17\linewidth} | p{0.17\linewidth} |}
          \textbf{R.A} & \textbf{R.B} & \textbf{R.C} & \textbf{R.D} \\
          \hline
          2 & 4 & 4 & 3 \\
          \hline
        \end{tabular}
        %\vspace{0.2em}
        \newline
        Assumes blue shape may be a "gingerbread man or a little person." Assumes purpose of the text. Should specify the placement of colors. \hfill

        \textit{-1 Misc}: Misidentifies the blue shape.
        \\
        \hline
        \textbf{Score}
        &
        \textbf{15/16}
        & 
        \textbf{14/16}
        &
        \textbf{12/16}
        \\
        \hline
        \textbf{GPT-4 Turbo}
        &
        \begin{tabular}{| p{0.17\linewidth} | p{0.16\linewidth} | p{0.17\linewidth} | p{0.17\linewidth} |}
          \textbf{R.A} & \textbf{R.B} & \textbf{R.C} & \textbf{R.D} \\
          \hline
          3 & 4 & 4 & 4 \\
          \hline
        \end{tabular}
        %\vspace{0.2em}
        \newline
        Makes an assumption about child intent: the black pot with a yellow band is "the classic tale of a pot of gold at the end of the rainbow."
        & 
        \begin{tabular}{| p{0.17\linewidth} | p{0.16\linewidth} | p{0.17\linewidth} | p{0.17\linewidth} |}
          \textbf{R.A} & \textbf{R.B} & \textbf{R.C} & \textbf{R.D} \\
          \hline
          2 & 4 & 4 & 4 \\
          \hline
        \end{tabular}
        %\vspace{0.2em}
        \newline
        Assumptions about the intent and themes: "a moment of calm... within a stormy or chaotic environment," and "themes such as peace amid chaos..."
        &
        \begin{tabular}{| p{0.17\linewidth} | p{0.16\linewidth} | p{0.17\linewidth} | p{0.17\linewidth} |}
          \textbf{R.A} & \textbf{R.B} & \textbf{R.C} & \textbf{R.D} \\
          \hline
          3 & 4 & 4 & 4 \\
          \hline
        \end{tabular}
        %\vspace{0.2em}
        \newline
        Assumes intent, such as: the choice to color one Peep differently is due to "a sense of standing out" and "mirroring a personal expression."
        \\
        \hline
        \textbf{Score}
        &
        \textbf{15/16}
        & 
        \textbf{14/16}
        &
        \textbf{15/16}
        \\
        \hline
        \textbf{GPT-4o}
        &
        \begin{tabular}{| p{0.17\linewidth} | p{0.16\linewidth} | p{0.17\linewidth} | p{0.17\linewidth} |}
          \textbf{R.A} & \textbf{R.B} & \textbf{R.C} & \textbf{R.D} \\
          \hline
          4 & 4 & 4 & 4 \\
          \hline
        \end{tabular}
        %\vspace{0.2em}
        \newline
        Respectful and celebrates the child's creativity with positive language like "joyful" and "vibrant", is detailed and covers colors and brushstrokes, and captures all major elements.
        & 
       \begin{tabular}{| p{0.17\linewidth} | p{0.16\linewidth} | p{0.17\linewidth} | p{0.17\linewidth} |}
          \textbf{R.A} & \textbf{R.B} & \textbf{R.C} & \textbf{R.D} \\
          \hline
          2 & 4 & 4 & 4 \\
          \hline
        \end{tabular}
        %\vspace{0.2em}
        \newline
        Suggesting the background is a "night sky or deep space" and that the red figure "might be a character wearing a hood or a cloak." \hfill

        \textit{-1 Misc}: Asking "What does your child say about their creation?"
        &
        \begin{tabular}{| p{0.17\linewidth} | p{0.16\linewidth} | p{0.17\linewidth} | p{0.17\linewidth} |}
          \textbf{R.A} & \textbf{R.B} & \textbf{R.C} & \textbf{R.D} \\
          \hline
          4 & 4 & 4 & 4 \\
          \hline
        \end{tabular}
        %\vspace{0.2em}
        \newline
        Avoids making assumptions, is respectful and uses positive language like "fun and eye-catching pattern", is detailed and covers aspects like colors and placement, and captures all major elements.
        \\
        \hline
        \textbf{Score}
        &
        \textbf{16/16}
        & 
        \textbf{13/16}
        &
        \textbf{16/16}
        \\
        \hline
        \textbf{Gemini 1.5 Flash}
        &
        \begin{tabular}{| p{0.17\linewidth} | p{0.16\linewidth} | p{0.17\linewidth} | p{0.17\linewidth} |}
          \textbf{R.A} & \textbf{R.B} & \textbf{R.C} & \textbf{R.D} \\
          \hline
          2 & 4 & 3 & 4 \\
          \hline
        \end{tabular}
        %\vspace{0.2em}
        \newline
        Too simple, makes assumptions about the intent behind the artwork ("fun day"), asks speculative questions. \hfill

        \textit{-1 Misc}: Asking "What do you think it represents to your child?".
        & 
        \begin{tabular}{| p{0.17\linewidth} | p{0.16\linewidth} | p{0.17\linewidth} | p{0.17\linewidth} |}
          \textbf{R.A} & \textbf{R.B} & \textbf{R.C} & \textbf{R.D} \\
          \hline
          1 & 3 & 2 & 3 \\
          \hline
        \end{tabular}
        %\vspace{0.2em}
        \newline
        Assumptions about the white areas "look like clouds", the scene is "in the sky". Saying "maybe this is a scene in the sky?" could be minimizing to the child's effort. Also too simple and not all major elements captured. \hfill

        \textit{-1 Misc}: Asks "What do you think?"
        &
        \begin{tabular}{| p{0.17\linewidth} | p{0.16\linewidth} | p{0.17\linewidth} | p{0.17\linewidth} |}
          \textbf{R.A} & \textbf{R.B} & \textbf{R.C} & \textbf{R.D} \\
          \hline
          4 & 4 & 3 & 3 \\
          \hline
        \end{tabular}
        %\vspace{0.2em}
        \newline
        Could provide more detail about the specific arrangement and the overall pattern of the figures. Also misses identifying the text, "Express your Peepsonality".
        \\
        \hline
        \textbf{Score}
        &
        \textbf{12/16}
        & 
        \textbf{8/16}
        &
        \textbf{14/16}
        \\
        \hline
    \end{tabular}
  }
  %\vspace{0.4em}
  \caption{Three example images from our dataset with the LLM Scorer scores for each Rubric guideline (R.A---presumptive, R.B---reductive, R.C---too simple, R.D---all elements captured), the reasoning for points lost, and the total scores.}
  \label{MasterTable}
\end{table*}
}
