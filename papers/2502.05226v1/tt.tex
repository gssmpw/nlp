\subsection{Findings and Implications}

\subsubsection{Experiment 1: n200\_T10\_k3\_B60\_C10}
This experiment presents a comparative analysis of computational performance between digital computing -- Binary Quadratic Programming (BQP) and Quadratic Unconstrained Binary Optimization (QUBO) and quantum computing -- quantum annealing in Dwave? IBM qiskit? InfinityQ? -- in optimizing a medium-sized portfolio allocation among 200 assets and cash over 10 trading days. BQP problems were solved using Gurobi v11.0 with a 10,000-second time limit, while QUBO problems were addressed using qoqo ABS2 within a 3,600-second limit.

\begin{table}[ht]
\centering
\renewcommand{\arraystretch}{1.1}
\resizebox{\textwidth}{!}{
\begin{tabular}{l|r|r|r|c|r|r|c}
\hline\hline
$q$ & \textbf{LB} & \textbf{UB} & \textbf{Gap} & \textbf{Gurobi TTS [s]} & \textbf{QUBO Obj.} & \textbf{Gap*} & \textbf{ABS2 TTS [s]} \\ \hline
0 & -1,855,197 & -1,855,197 & 0.000\% & 0.22 & -1,855,197 & 0.000\% & 102 \\ \hline
0.000001 & -1,839,242 & -1,839,242 & 0.000\% & 34 & -1,839,012 & 0.010\% & 210 \\ \hline
0.00001 & -1,706,529 & -1,697,030 & 0.560\% & 125 & -1,701,223 & 0.312\% & 105 \\ \hline
0.00005 & -1,289,791 & -1,265,722 & 1.891\% & 3,212 & -1,269,382 & 1.608\% & 1,191 \\ \hline
0.0001 & -988,003 & -950,586 & 3.936\% & 9,591 & -953,432 & 3.626\% & 1,778 \\ \hline
0.0005 & -356,010 & -296,540 & 20.055\% & 10,000 & -272,033 & 30.870\% & 3,562 \\ \hline
0.001 & -231,724 & -168,674 & 37.380\% & 10,000 & -104,856 & 120.992\% & 3,552 \\ \hline
0.01 & -108,821 & -1,000 & 10,782\% & 33 & -1,000 & 10,782\% & 53 \\ \hline\hline
\end{tabular}
}
\caption{BQP vs QUBO in portfolio optimization of n200\_T10\_k3\_B60\_C10}
\label{tab:a200_t10}
\end{table}
As shown in Table~\ref{tab:a200_t10}, the computational performance varies with the risk aversion parameter \(q\). Notably, as \(q\) approaches extremes (either focusing solely on profit or risk), the problem becomes significantly easier to solve, with solvers quickly finding solutions. Particularly, for extreme risk aversion (\(q=0.01\)), both solvers converge to a solution in under a minute, albeit with a considerable gap indicative of the distance from the theoretical optimum under relaxed constraints. Figure~\ref{fig:profit_var_200} shows the Pareto optimal frontier, aligning with the computed results and showing similar profit-risk dependence against various values of \(q\).
\begin{figure}[ht]
    \centering
    \includegraphics[width=0.8\textwidth]{image_bench/profit_var_200.png}
    \caption{Pareto optimal frontier for n200\_T10\_k3\_B60\_C10}
    \label{fig:profit_var_200}
\end{figure}
When \(q\) is large enough, e.g., at \(q=0.01\), it effectively minimizes risk to nearly zero. Both methods can solve the problem within 1 minute, but exhibit an extremely large gap of 10,782\%. However, the gap is merely an indicator reflecting the distance to the theoretical optimum under relaxed constraints and does not necessarily imply that the solutions are suboptimal. In this extreme risk-averse scenario, the best strategy should be holding cash over time. Both the BQP and QUBO strategies consistently recommend holding cash up to the maximum of \(C=10\) blocks throughout the period, as illustrated in Figure~\ref{fig:n200_T10}. This consistency validates that the solutions are globally optimal.
\begin{figure}[h!]
    \centering
    \begin{subfigure}[b]{0.47\textwidth}
        \includegraphics[width=\textwidth]{image_bench/pf10_bqp_a200_t10_u3_q0.01_s60_c10.png}
        \caption{BQP model strategy}
    \end{subfigure}
    \hfill
    \begin{subfigure}[b]{0.47\textwidth}
        \includegraphics[width=\textwidth]{image_bench/pf10_qoqo_a200_t10_u3_q0.01_s60_c10.png}
        \caption{QUBO model strategy}
    \end{subfigure}
    \caption{Stock positions for n200\_T10\_k3\_B60\_C10 with $q = 0.01$}
    \label{fig:n200_T10}
\end{figure}
For \(q = 0\) up to \(q = 0.00001\), risk considerations become irrelevant, and the focus shifts entirely towards maximizing profit. Both Gurobi and ABS2 efficiently find solutions within 2 minutes, closely matching or achieving the theoretical optimal benchmark under relaxed constraints (gap close to or equal to 0\%). Up to \(q = 0.0001\), the problem remains solvable within 2.5 hours with a reasonable gap of less than 5\%. Figure~\ref{fig:position_200_q0} illustrates the stock holdings and cash positions of the two models. Due to space constraints, the figures only display the stocks selected in the first time period (\(t=1\)) and their corresponding positions over time up to \(T\). For both models, the dynamic positions are identical.
\begin{figure}[h!]
    \centering
    \begin{subfigure}[b]{0.47\textwidth}
        \includegraphics[width=\textwidth]{image_bench/pf10_bqp_a200_t10_u3_q0_s60_c10.png}
        \caption{BQP model strategy}
    \end{subfigure}
    \hfill
    \begin{subfigure}[b]{0.47\textwidth}
        \includegraphics[width=\textwidth]{image_bench/pf10_qoqo_a200_t10_u3_q0_s60_c10.png}
        \caption{QUBO model strategy}
    \end{subfigure}
    \caption{Stock positions for n200\_T10\_k3\_B60\_C10 with \(q = 0\)}
    \label{fig:position_200_q0}
\end{figure}

While Gurobi generally finds solutions faster than qoqo ABS2 for simpler problems, in scenarios where \(q\) is set to manage both profit and risk, the optimization problem becomes challenging, demanding more computational resources. For \(q\) values such as 0.0005 and 0.001, both solvers nearly exhaust the computational limits set. Even though Gurobi generally provides a slightly better solution given longer computational time limit, it requires substantially more time to reach to a good solution with optimal \(\pm\epsilon\) compared to qoqo ABS2, as detailed in the convergence plots, see Figures~\ref{fig:a200_t10_q0.0005} and~\ref{fig:a200_t10_q0.001}. Specifically, ABS2 converges towards a good solution significantly faster than Gurobi, e.g., within 71 seconds for \( q = 0.0005 \) and within 35 seconds for \( q = 0.001 \). Both solvers are unable to significantly improve upon them further until the computational time limit is reached.
\begin{figure}[ht]
    \centering
    \begin{subfigure}[b]{0.47\textwidth}
        \centering
        \includegraphics[width=\textwidth]{image_bench/Objective_Value_for_a200_t10_q0.0005.png}
        \caption{Objective Change for \( q = 0.0005 \) in a200.}
        \label{fig:a200_t10_q0.0005}
    \end{subfigure}
    \hfill
    \begin{subfigure}[b]{0.47\textwidth}
        \centering
        \includegraphics[width=\textwidth]{image_bench/Objective_Value_for_a200_t10_q0.001.png}
        \caption{Objective Change for \( q = 0.001 \) in a200.}
        \label{fig:a200_t10_q0.001}
    \end{subfigure}
    \caption{Convergence of objective values for different risk aversion parameters.}
\end{figure}
When we further vary the computational time limits to 120, 600, 1800, and 3600 seconds, we observe that at a very short time limit of 120 seconds, the objective values of QUBO and BQP are very close for different \( q \) values. At 600, 1800, and 3600 seconds, except for the most challenging problem at \( q = 0.001 \), where the objective value of QUBO optimized by ABS2 is noticeably higher than that of BQP optimized by Gurobi, the differences between the objective values of QUBO and BQP for \( q = 0.001 \) decrease as the time limit increases.

Figures \ref{fig:bqp_200_q0005} and \ref{fig:qubo_200_q0005} display the trading trajectories at \( q = 0.0005 \), where both risk and profit are considered. The two models provide different strategies. On the first day, the BQP model selected 44 stocks, while the QUBO model selected 50 stocks. Nevertheless, their positions on the same stocks over the period were quite similar. Specifically, both models held long positions in APD from day 1 to day 7, long positions in CME from day 1 to day 8, short positions in LMT on all days except the fifth, and short positions in PEP for the first three days.
\begin{figure}[h!]
    \centering
    \begin{subfigure}[b]{0.47\textwidth}
        \includegraphics[width=\textwidth]{image_bench/pf10_bqp_a200_t10_u3_q0.0005_s60_c10.png}
        \caption{BQP model strategy}
        \label{fig:bqp_200_q0005}
    \end{subfigure}
    \hfill
    \begin{subfigure}[b]{0.47\textwidth}
        \includegraphics[width=\textwidth]{image_bench/pf10_qoqo_a200_t10_u3_q0.0005_s60_c10.png}
        \caption{QUBO model strategy}
        \label{fig:qubo_200_q0005}
    \end{subfigure}
    \caption{Stock positions for n200\_T10\_k3\_B60\_C10 with $q = 0.0005$}
\end{figure}

\subsubsection{Experiment 2: n499\_T15\_k3\_B60\_C10}
In this larger-scale experiment, we address the portfolio optimization problem involving the allocation of capital among 499 stocks and cash over a period of three weeks, while keeping other parameters same. Given the current limitations of quantum computing hardware, implementing quantum solutions for such extensive and complex scenarios remains out of reach. However, we establish a robust benchmark using digital computing as a reference point. This benchmark is designed with the anticipation of rapid advancements in quantum computing technologies. Our goal is to provide a transparent and fair platform to evaluate genuine breakthroughs in quantum computing as they emerge, helping to determine when quantum solutions can realistically surpass the capabilities of traditional digital algorithms in handling large-scale financial optimization tasks.

For this experiment, only digital computing solvers were utilized. BQP problems were addressed using Gurobi v11.0, with a time limit of 10,000 seconds, while QUBO problems were solved using the qoqo ABS2 solver, with a time limit of 3,600 seconds. The performance comparison between these digital computing approaches in portfolio optimization is presented in Table~\ref{tab:a499_t15}. A Pareto optimal frontier, displayed in Figure~\ref{fig:profit_var_499}, illustrates that the solvers provide comparable results under various scenarios.
\begin{table}[ht]
\centering
\renewcommand{\arraystretch}{1.1}
\resizebox{\textwidth}{!}{
\begin{tabular}{r|r|r|r|c|r|r|c}
\hline\hline
$q$     & \textbf{LB}      & \textbf{UB}      & \textbf{Gap}  & \textbf{Gurobi TTS [s]} & \textbf{QUBO Obj.} & \textbf{Gap*} & \textbf{ABS2 TTS [s]} \\ \hline
0        & -4,841,118 & -4,841,118 & 0.000\%  & 0.93           & -4,831,302  & 0.203\%   & 3,615        \\ \hline
0.000001 & -4,795,977 & -4,785,071 & 0.228\%  & 588             & -4,787,080  & 0.186\%   & 3,568        \\ \hline
0.00001  & -4,343,457 & -4,324,080 & 0.448\%  & 834            & -4,322,507  & 0.484\%   & 3,621        \\ \hline
0.00005  & -2,927,666 & -2,867,493 & 2.099\%  & 3,060          & -2,871,591  & 1.953\%   & 3,606      \\ \hline
0.0001   & -1,996,723   & -1,903,969   & 4.872\%  & 2,663          & -1,887,560    & 5.783\%   & 3,545      \\ \hline
0.0005   & -708,516   & -563,441   & 25.748\% & 4,843         & -451,312    & 56.990\%  & 3,579      \\ \hline
0.001    & -507,219   & -332,172   & 52.698\% & 5,243         & -132,613    & 282.481\% & 3,595      \\ \hline
0.01     & -316,320   & -1,500     & 20,988\% & 337             & -1,500      & 20,988\%  & 1,026         \\ \hline\hline
\end{tabular}
}
\caption{BQP vs QUBO in portfolio optimization of n499\_T15\_k3\_B60\_C10}
\label{tab:a499_t15}
\end{table}
\begin{figure}[ht]
    \centering
    \begin{subfigure}[b]{0.8\textwidth}
        \centering
        \includegraphics[width=\textwidth]{image_bench/profit_var_499.png}
        \caption{n499\_T15\_k3\_B60\_C10}
        \label{fig:profit_var_499}
    \end{subfigure}
    \caption{Profits and Variances at different \( q \) values.}
\end{figure}
As the risk aversion parameter \(q\) approaches extremities -- focusing solely on profit or risk, the problem is easier to solve. When \(q \rightarrow 0\), the small optimality gap verifies the effectiveness of the solutions; whereas at a high \(q\), e.g., \(q=0.01\), both solvers suggest holding cash, effectively minimizing risk to zero and thereby justifying the optimality of the solutions despite a large gap. Generally, Gurobi offers better efficiency with shorter TTS for simpler problems, while QUBO resolves more challenging scenarios at \(q=0.0005\) and \(q=0.001\) more effectively.

An in-depth analysis of the objective value changes over time for \(q = 0.0005\) and \(q = 0.001\) is depicted in Figures~\ref{fig:a499_t15_q0.0005} and \ref{fig:a499_t15_q0.001}. Although BQP achieves a better final objective value with longer computational time, QUBO reaches good solutions in substantially less time. After 200 seconds for both \(q = 0.0005\) and \(q = 0.001\), there is only marginal improvement, whereas Gurobi requires more than 1400 seconds and 500 seconds, respectively, to stabilize.
\begin{figure}[ht]
    \centering
    \begin{subfigure}[b]{0.47\textwidth}
        \centering
        \includegraphics[width=\textwidth]{image_bench/Objective_Value_for_a499_t15_q0.0005.png}
        \caption{Objective Change for \( q = 0.0005 \) in a499.}
        \label{fig:a499_t15_q0.0005}
    \end{subfigure}
    \hfill
    \begin{subfigure}[b]{0.47\textwidth}
        \centering
        \includegraphics[width=\textwidth]{image_bench/Objective_Value_for_a499_t15_q0.001.png}
        \caption{Objective Change for \( q = 0.001 \) in a499.}
        \label{fig:a499_t15_q0.001}
    \end{subfigure}
    \caption{}
\end{figure}
We further explored the objective values of QUBO and BQP across different \(q\) values and various computational time limits of 120, 600, 1800, and 3600 seconds, as shown in Figures~\ref{fig:comparison_a499_t15_120s}-\ref{fig:comparison_a499_t15_3600s}. At 120 seconds, except for \(q = 0\), QUBO typically achieves better objective values than BQP, supporting observations that QUBO problems tend to reach better (lower) objective values in the early stages of optimization compared to BQP. At 600 seconds, except for \(q = 0.01\) where the QUBO objective value is noticeably higher, the results for QUBO and BQP are quite similar. By 1800 seconds, the optimization results for QUBO and BQP converge across all \(q\) values.
\begin{figure}[ht]
    \centering
    \begin{subfigure}[b]{0.47\textwidth}
        \centering
        \includegraphics[width=\textwidth]{image_bench/Comparison_120s_n499_T15_k3_B60_C10.png}
        \caption{}
        \label{fig:comparison_a499_t15_120s}
    \end{subfigure}
    \hfill
    \begin{subfigure}[b]{0.47\textwidth}
        \centering
        \includegraphics[width=\textwidth]{image_bench/Comparison_600s_n499_T15_k3_B60_C10.png}
        \caption{}
        \label{fig:comparison_a499_t15_600s}
    \end{subfigure}
    \vfill
    \begin{subfigure}[b]{0.47\textwidth}
        \centering
        \includegraphics[width=\textwidth]{image_bench/Comparison_1800s_n499_T15_k3_B60_C10.png}
        \caption{}
        \label{fig:comparison_a499_t15_1800s}
    \end{subfigure}
    \hfill
    \begin{subfigure}[b]{0.47\textwidth}
        \centering
        \includegraphics[width=\textwidth]{image_bench/Comparison_3600s_n499_T15_k3_B60_C10.png}
        \caption{}
        \label{fig:comparison_a499_t15_3600s}
    \end{subfigure}
    \caption{Objective Values for n499\_T15\_k3\_B60\_C10 at different times.}
\end{figure}
Figure~\ref{fig:bqp_499_q0005} and Figure~\ref{fig:qubo_499_q0005} illustrate the trading strategy trajectories for the two models. Due to space limitations, we only display the stocks selected on the first day as well as cash positions. The two models suggest differing strategies; the BQP model chose 45 stocks on the first day, while the QUBO model opted for 53 stocks. Nevertheless, their positions on the same stocks throughout the period showed notable similarities. Specifically, both models maintained long positions in DAL from days 1 to 10 and on day 12, and held short positions in BA over the same days, with no positions in BA from days 13 to 15.
\begin{figure}[h!]
    \centering
    \begin{subfigure}[b]{0.47\textwidth}
        \includegraphics[width=\textwidth]{image_bench/pf10_bqp_a200_t10_u3_q0.0005_s60_c10.png}
        \caption{BQP model strategy}
        \label{fig:bqp_200_q0005}
    \end{subfigure}
    \hfill
    \begin{subfigure}[b]{0.47\textwidth}
        \includegraphics[width=\textwidth]{image_bench/pf10_qoqo_a200_t10_u3_q0.0005_s60_c10.png}
        \caption{QUBO model strategy}
        \label{fig:qubo_200_q0005}
    \end{subfigure}
    \caption{Stock positions for n200\_T10\_k3\_B60\_C10 with $q = 0.0005$}
\end{figure}

\begin{figure}[h!]
    \centering
    \begin{subfigure}[b]{0.47\textwidth}
        \includegraphics[width=\textwidth]{image_bench/pf10_bqp_a499_t15_u3_q0.0005_s60_c10.png}
        \caption{BQP model strategy}
        \label{fig:bqp_499_q0005}
    \end{subfigure}
    \hfill
    \begin{subfigure}[b]{0.47\textwidth}
        \includegraphics[width=\textwidth]{image_bench/pf10_qoqo_a499_t15_u3_q0.0005_s60_c10.png}
        \caption{QUBO model strategy}
        \label{fig:qubo_499_q0005}
    \end{subfigure}
    \caption{Stock positions for n499\_T15\_k3\_B60\_C10 with $q = 0.0005$}
\end{figure}



In conclusion, while both quantum and digital solvers are capable of effectively handling the portfolio optimization problem, their performance significantly depends on the complexity induced by the risk aversion parameter \(q\). This experiment provides a clear insight into the current computational capabilities and limitations, guiding future developments in optimization algorithms.
