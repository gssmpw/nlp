We propose a novel framework that exploits quantum computing to navigate the complexities of large-scale dynamic portfolio optimization, incorporating various market frictions that pose significant challenges to traditional computational methods. Our investigation reveals not only the theoretical potential but also the practical applicability of quantum algorithms in solving financial optimization processes, potentially setting a benchmark for future research in the confluence of finance, quantum computing, and optimization theory.

Our model's capability to employ up to 5000 qubits for solving portfolio optimization problems can be considered as a milestone in the field, illustrating the scalability and computational power of quantum technologies. This advance opens avenues for addressing the multi-faceted challenges of financial markets, incorporating market frictions such as transaction costs, integer share constraints, and path-dependent capital constraints, which have long stymied conventional optimization methods.

Through experiments encompassing a portfolio of 50 stocks, a time frame of 1 and 2 weeks, and an upper (lower) bound of 16 (-15) shares per stock, we demonstrate that our model, powered by a hybrid quantum computing approach, consistently achieves good optimization outcomes, specifically in minimizing volatility as the objective function. It secures the highest Sharpe ratio compared to traditional investment strategies, indicating its efficacy in generating optimized, risk-adjusted returns. This underscores the potential of quantum computing in enhancing financial decision-making and portfolio management.

Moreover, our findings show the accelerated convergence capabilities of the quantum-based model (60s) relative to classical optimization solvers such as Gurobi and CPLEX (more than 600s). This efficiency gain highlights the quantum algorithm's ability to navigate complex optimization landscapes in computational finance by reducing solution times and enhancing performance.

Our research sheds light on the application of adiabatic quantum computing, particularly hybrid quantum annealing, in solving QUBO problems. Despite the current limitations of quantum hardware, especially in the gate model's capacity to address large-scale challenges, quantum annealing emerges as a powerful tool for real-world financial applications, offering a new paradigm for portfolio optimization. As quantum technology continues to evolve, we anticipate its increasingly prominent role in finance, driving innovation and offering solutions to complex problems. 