
\documentclass{article} % For LaTeX2e
\usepackage{iclr2025_conference,times}

% Optional math commands from https://github.com/goodfeli/dlbook_notation.
% %%%%% NEW MATH DEFINITIONS %%%%%

\usepackage{amsmath,amsfonts,bm}
\usepackage{derivative}
% Mark sections of captions for referring to divisions of figures
\newcommand{\figleft}{{\em (Left)}}
\newcommand{\figcenter}{{\em (Center)}}
\newcommand{\figright}{{\em (Right)}}
\newcommand{\figtop}{{\em (Top)}}
\newcommand{\figbottom}{{\em (Bottom)}}
\newcommand{\captiona}{{\em (a)}}
\newcommand{\captionb}{{\em (b)}}
\newcommand{\captionc}{{\em (c)}}
\newcommand{\captiond}{{\em (d)}}

% Highlight a newly defined term
\newcommand{\newterm}[1]{{\bf #1}}

% Derivative d 
\newcommand{\deriv}{{\mathrm{d}}}

% Figure reference, lower-case.
\def\figref#1{figure~\ref{#1}}
% Figure reference, capital. For start of sentence
\def\Figref#1{Figure~\ref{#1}}
\def\twofigref#1#2{figures \ref{#1} and \ref{#2}}
\def\quadfigref#1#2#3#4{figures \ref{#1}, \ref{#2}, \ref{#3} and \ref{#4}}
% Section reference, lower-case.
\def\secref#1{section~\ref{#1}}
% Section reference, capital.
\def\Secref#1{Section~\ref{#1}}
% Reference to two sections.
\def\twosecrefs#1#2{sections \ref{#1} and \ref{#2}}
% Reference to three sections.
\def\secrefs#1#2#3{sections \ref{#1}, \ref{#2} and \ref{#3}}
% Reference to an equation, lower-case.
\def\eqref#1{equation~\ref{#1}}
% Reference to an equation, upper case
\def\Eqref#1{Equation~\ref{#1}}
% A raw reference to an equation---avoid using if possible
\def\plaineqref#1{\ref{#1}}
% Reference to a chapter, lower-case.
\def\chapref#1{chapter~\ref{#1}}
% Reference to an equation, upper case.
\def\Chapref#1{Chapter~\ref{#1}}
% Reference to a range of chapters
\def\rangechapref#1#2{chapters\ref{#1}--\ref{#2}}
% Reference to an algorithm, lower-case.
\def\algref#1{algorithm~\ref{#1}}
% Reference to an algorithm, upper case.
\def\Algref#1{Algorithm~\ref{#1}}
\def\twoalgref#1#2{algorithms \ref{#1} and \ref{#2}}
\def\Twoalgref#1#2{Algorithms \ref{#1} and \ref{#2}}
% Reference to a part, lower case
\def\partref#1{part~\ref{#1}}
% Reference to a part, upper case
\def\Partref#1{Part~\ref{#1}}
\def\twopartref#1#2{parts \ref{#1} and \ref{#2}}

\def\ceil#1{\lceil #1 \rceil}
\def\floor#1{\lfloor #1 \rfloor}
\def\1{\bm{1}}
\newcommand{\train}{\mathcal{D}}
\newcommand{\valid}{\mathcal{D_{\mathrm{valid}}}}
\newcommand{\test}{\mathcal{D_{\mathrm{test}}}}

\def\eps{{\epsilon}}


% Random variables
\def\reta{{\textnormal{$\eta$}}}
\def\ra{{\textnormal{a}}}
\def\rb{{\textnormal{b}}}
\def\rc{{\textnormal{c}}}
\def\rd{{\textnormal{d}}}
\def\re{{\textnormal{e}}}
\def\rf{{\textnormal{f}}}
\def\rg{{\textnormal{g}}}
\def\rh{{\textnormal{h}}}
\def\ri{{\textnormal{i}}}
\def\rj{{\textnormal{j}}}
\def\rk{{\textnormal{k}}}
\def\rl{{\textnormal{l}}}
% rm is already a command, just don't name any random variables m
\def\rn{{\textnormal{n}}}
\def\ro{{\textnormal{o}}}
\def\rp{{\textnormal{p}}}
\def\rq{{\textnormal{q}}}
\def\rr{{\textnormal{r}}}
\def\rs{{\textnormal{s}}}
\def\rt{{\textnormal{t}}}
\def\ru{{\textnormal{u}}}
\def\rv{{\textnormal{v}}}
\def\rw{{\textnormal{w}}}
\def\rx{{\textnormal{x}}}
\def\ry{{\textnormal{y}}}
\def\rz{{\textnormal{z}}}

% Random vectors
\def\rvepsilon{{\mathbf{\epsilon}}}
\def\rvphi{{\mathbf{\phi}}}
\def\rvtheta{{\mathbf{\theta}}}
\def\rva{{\mathbf{a}}}
\def\rvb{{\mathbf{b}}}
\def\rvc{{\mathbf{c}}}
\def\rvd{{\mathbf{d}}}
\def\rve{{\mathbf{e}}}
\def\rvf{{\mathbf{f}}}
\def\rvg{{\mathbf{g}}}
\def\rvh{{\mathbf{h}}}
\def\rvu{{\mathbf{i}}}
\def\rvj{{\mathbf{j}}}
\def\rvk{{\mathbf{k}}}
\def\rvl{{\mathbf{l}}}
\def\rvm{{\mathbf{m}}}
\def\rvn{{\mathbf{n}}}
\def\rvo{{\mathbf{o}}}
\def\rvp{{\mathbf{p}}}
\def\rvq{{\mathbf{q}}}
\def\rvr{{\mathbf{r}}}
\def\rvs{{\mathbf{s}}}
\def\rvt{{\mathbf{t}}}
\def\rvu{{\mathbf{u}}}
\def\rvv{{\mathbf{v}}}
\def\rvw{{\mathbf{w}}}
\def\rvx{{\mathbf{x}}}
\def\rvy{{\mathbf{y}}}
\def\rvz{{\mathbf{z}}}

% Elements of random vectors
\def\erva{{\textnormal{a}}}
\def\ervb{{\textnormal{b}}}
\def\ervc{{\textnormal{c}}}
\def\ervd{{\textnormal{d}}}
\def\erve{{\textnormal{e}}}
\def\ervf{{\textnormal{f}}}
\def\ervg{{\textnormal{g}}}
\def\ervh{{\textnormal{h}}}
\def\ervi{{\textnormal{i}}}
\def\ervj{{\textnormal{j}}}
\def\ervk{{\textnormal{k}}}
\def\ervl{{\textnormal{l}}}
\def\ervm{{\textnormal{m}}}
\def\ervn{{\textnormal{n}}}
\def\ervo{{\textnormal{o}}}
\def\ervp{{\textnormal{p}}}
\def\ervq{{\textnormal{q}}}
\def\ervr{{\textnormal{r}}}
\def\ervs{{\textnormal{s}}}
\def\ervt{{\textnormal{t}}}
\def\ervu{{\textnormal{u}}}
\def\ervv{{\textnormal{v}}}
\def\ervw{{\textnormal{w}}}
\def\ervx{{\textnormal{x}}}
\def\ervy{{\textnormal{y}}}
\def\ervz{{\textnormal{z}}}

% Random matrices
\def\rmA{{\mathbf{A}}}
\def\rmB{{\mathbf{B}}}
\def\rmC{{\mathbf{C}}}
\def\rmD{{\mathbf{D}}}
\def\rmE{{\mathbf{E}}}
\def\rmF{{\mathbf{F}}}
\def\rmG{{\mathbf{G}}}
\def\rmH{{\mathbf{H}}}
\def\rmI{{\mathbf{I}}}
\def\rmJ{{\mathbf{J}}}
\def\rmK{{\mathbf{K}}}
\def\rmL{{\mathbf{L}}}
\def\rmM{{\mathbf{M}}}
\def\rmN{{\mathbf{N}}}
\def\rmO{{\mathbf{O}}}
\def\rmP{{\mathbf{P}}}
\def\rmQ{{\mathbf{Q}}}
\def\rmR{{\mathbf{R}}}
\def\rmS{{\mathbf{S}}}
\def\rmT{{\mathbf{T}}}
\def\rmU{{\mathbf{U}}}
\def\rmV{{\mathbf{V}}}
\def\rmW{{\mathbf{W}}}
\def\rmX{{\mathbf{X}}}
\def\rmY{{\mathbf{Y}}}
\def\rmZ{{\mathbf{Z}}}

% Elements of random matrices
\def\ermA{{\textnormal{A}}}
\def\ermB{{\textnormal{B}}}
\def\ermC{{\textnormal{C}}}
\def\ermD{{\textnormal{D}}}
\def\ermE{{\textnormal{E}}}
\def\ermF{{\textnormal{F}}}
\def\ermG{{\textnormal{G}}}
\def\ermH{{\textnormal{H}}}
\def\ermI{{\textnormal{I}}}
\def\ermJ{{\textnormal{J}}}
\def\ermK{{\textnormal{K}}}
\def\ermL{{\textnormal{L}}}
\def\ermM{{\textnormal{M}}}
\def\ermN{{\textnormal{N}}}
\def\ermO{{\textnormal{O}}}
\def\ermP{{\textnormal{P}}}
\def\ermQ{{\textnormal{Q}}}
\def\ermR{{\textnormal{R}}}
\def\ermS{{\textnormal{S}}}
\def\ermT{{\textnormal{T}}}
\def\ermU{{\textnormal{U}}}
\def\ermV{{\textnormal{V}}}
\def\ermW{{\textnormal{W}}}
\def\ermX{{\textnormal{X}}}
\def\ermY{{\textnormal{Y}}}
\def\ermZ{{\textnormal{Z}}}

% Vectors
\def\vzero{{\bm{0}}}
\def\vone{{\bm{1}}}
\def\vmu{{\bm{\mu}}}
\def\vtheta{{\bm{\theta}}}
\def\vphi{{\bm{\phi}}}
\def\va{{\bm{a}}}
\def\vb{{\bm{b}}}
\def\vc{{\bm{c}}}
\def\vd{{\bm{d}}}
\def\ve{{\bm{e}}}
\def\vf{{\bm{f}}}
\def\vg{{\bm{g}}}
\def\vh{{\bm{h}}}
\def\vi{{\bm{i}}}
\def\vj{{\bm{j}}}
\def\vk{{\bm{k}}}
\def\vl{{\bm{l}}}
\def\vm{{\bm{m}}}
\def\vn{{\bm{n}}}
\def\vo{{\bm{o}}}
\def\vp{{\bm{p}}}
\def\vq{{\bm{q}}}
\def\vr{{\bm{r}}}
\def\vs{{\bm{s}}}
\def\vt{{\bm{t}}}
\def\vu{{\bm{u}}}
\def\vv{{\bm{v}}}
\def\vw{{\bm{w}}}
\def\vx{{\bm{x}}}
\def\vy{{\bm{y}}}
\def\vz{{\bm{z}}}

% Elements of vectors
\def\evalpha{{\alpha}}
\def\evbeta{{\beta}}
\def\evepsilon{{\epsilon}}
\def\evlambda{{\lambda}}
\def\evomega{{\omega}}
\def\evmu{{\mu}}
\def\evpsi{{\psi}}
\def\evsigma{{\sigma}}
\def\evtheta{{\theta}}
\def\eva{{a}}
\def\evb{{b}}
\def\evc{{c}}
\def\evd{{d}}
\def\eve{{e}}
\def\evf{{f}}
\def\evg{{g}}
\def\evh{{h}}
\def\evi{{i}}
\def\evj{{j}}
\def\evk{{k}}
\def\evl{{l}}
\def\evm{{m}}
\def\evn{{n}}
\def\evo{{o}}
\def\evp{{p}}
\def\evq{{q}}
\def\evr{{r}}
\def\evs{{s}}
\def\evt{{t}}
\def\evu{{u}}
\def\evv{{v}}
\def\evw{{w}}
\def\evx{{x}}
\def\evy{{y}}
\def\evz{{z}}

% Matrix
\def\mA{{\bm{A}}}
\def\mB{{\bm{B}}}
\def\mC{{\bm{C}}}
\def\mD{{\bm{D}}}
\def\mE{{\bm{E}}}
\def\mF{{\bm{F}}}
\def\mG{{\bm{G}}}
\def\mH{{\bm{H}}}
\def\mI{{\bm{I}}}
\def\mJ{{\bm{J}}}
\def\mK{{\bm{K}}}
\def\mL{{\bm{L}}}
\def\mM{{\bm{M}}}
\def\mN{{\bm{N}}}
\def\mO{{\bm{O}}}
\def\mP{{\bm{P}}}
\def\mQ{{\bm{Q}}}
\def\mR{{\bm{R}}}
\def\mS{{\bm{S}}}
\def\mT{{\bm{T}}}
\def\mU{{\bm{U}}}
\def\mV{{\bm{V}}}
\def\mW{{\bm{W}}}
\def\mX{{\bm{X}}}
\def\mY{{\bm{Y}}}
\def\mZ{{\bm{Z}}}
\def\mBeta{{\bm{\beta}}}
\def\mPhi{{\bm{\Phi}}}
\def\mLambda{{\bm{\Lambda}}}
\def\mSigma{{\bm{\Sigma}}}

% Tensor
\DeclareMathAlphabet{\mathsfit}{\encodingdefault}{\sfdefault}{m}{sl}
\SetMathAlphabet{\mathsfit}{bold}{\encodingdefault}{\sfdefault}{bx}{n}
\newcommand{\tens}[1]{\bm{\mathsfit{#1}}}
\def\tA{{\tens{A}}}
\def\tB{{\tens{B}}}
\def\tC{{\tens{C}}}
\def\tD{{\tens{D}}}
\def\tE{{\tens{E}}}
\def\tF{{\tens{F}}}
\def\tG{{\tens{G}}}
\def\tH{{\tens{H}}}
\def\tI{{\tens{I}}}
\def\tJ{{\tens{J}}}
\def\tK{{\tens{K}}}
\def\tL{{\tens{L}}}
\def\tM{{\tens{M}}}
\def\tN{{\tens{N}}}
\def\tO{{\tens{O}}}
\def\tP{{\tens{P}}}
\def\tQ{{\tens{Q}}}
\def\tR{{\tens{R}}}
\def\tS{{\tens{S}}}
\def\tT{{\tens{T}}}
\def\tU{{\tens{U}}}
\def\tV{{\tens{V}}}
\def\tW{{\tens{W}}}
\def\tX{{\tens{X}}}
\def\tY{{\tens{Y}}}
\def\tZ{{\tens{Z}}}


% Graph
\def\gA{{\mathcal{A}}}
\def\gB{{\mathcal{B}}}
\def\gC{{\mathcal{C}}}
\def\gD{{\mathcal{D}}}
\def\gE{{\mathcal{E}}}
\def\gF{{\mathcal{F}}}
\def\gG{{\mathcal{G}}}
\def\gH{{\mathcal{H}}}
\def\gI{{\mathcal{I}}}
\def\gJ{{\mathcal{J}}}
\def\gK{{\mathcal{K}}}
\def\gL{{\mathcal{L}}}
\def\gM{{\mathcal{M}}}
\def\gN{{\mathcal{N}}}
\def\gO{{\mathcal{O}}}
\def\gP{{\mathcal{P}}}
\def\gQ{{\mathcal{Q}}}
\def\gR{{\mathcal{R}}}
\def\gS{{\mathcal{S}}}
\def\gT{{\mathcal{T}}}
\def\gU{{\mathcal{U}}}
\def\gV{{\mathcal{V}}}
\def\gW{{\mathcal{W}}}
\def\gX{{\mathcal{X}}}
\def\gY{{\mathcal{Y}}}
\def\gZ{{\mathcal{Z}}}

% Sets
\def\sA{{\mathbb{A}}}
\def\sB{{\mathbb{B}}}
\def\sC{{\mathbb{C}}}
\def\sD{{\mathbb{D}}}
% Don't use a set called E, because this would be the same as our symbol
% for expectation.
\def\sF{{\mathbb{F}}}
\def\sG{{\mathbb{G}}}
\def\sH{{\mathbb{H}}}
\def\sI{{\mathbb{I}}}
\def\sJ{{\mathbb{J}}}
\def\sK{{\mathbb{K}}}
\def\sL{{\mathbb{L}}}
\def\sM{{\mathbb{M}}}
\def\sN{{\mathbb{N}}}
\def\sO{{\mathbb{O}}}
\def\sP{{\mathbb{P}}}
\def\sQ{{\mathbb{Q}}}
\def\sR{{\mathbb{R}}}
\def\sS{{\mathbb{S}}}
\def\sT{{\mathbb{T}}}
\def\sU{{\mathbb{U}}}
\def\sV{{\mathbb{V}}}
\def\sW{{\mathbb{W}}}
\def\sX{{\mathbb{X}}}
\def\sY{{\mathbb{Y}}}
\def\sZ{{\mathbb{Z}}}

% Entries of a matrix
\def\emLambda{{\Lambda}}
\def\emA{{A}}
\def\emB{{B}}
\def\emC{{C}}
\def\emD{{D}}
\def\emE{{E}}
\def\emF{{F}}
\def\emG{{G}}
\def\emH{{H}}
\def\emI{{I}}
\def\emJ{{J}}
\def\emK{{K}}
\def\emL{{L}}
\def\emM{{M}}
\def\emN{{N}}
\def\emO{{O}}
\def\emP{{P}}
\def\emQ{{Q}}
\def\emR{{R}}
\def\emS{{S}}
\def\emT{{T}}
\def\emU{{U}}
\def\emV{{V}}
\def\emW{{W}}
\def\emX{{X}}
\def\emY{{Y}}
\def\emZ{{Z}}
\def\emSigma{{\Sigma}}

% entries of a tensor
% Same font as tensor, without \bm wrapper
\newcommand{\etens}[1]{\mathsfit{#1}}
\def\etLambda{{\etens{\Lambda}}}
\def\etA{{\etens{A}}}
\def\etB{{\etens{B}}}
\def\etC{{\etens{C}}}
\def\etD{{\etens{D}}}
\def\etE{{\etens{E}}}
\def\etF{{\etens{F}}}
\def\etG{{\etens{G}}}
\def\etH{{\etens{H}}}
\def\etI{{\etens{I}}}
\def\etJ{{\etens{J}}}
\def\etK{{\etens{K}}}
\def\etL{{\etens{L}}}
\def\etM{{\etens{M}}}
\def\etN{{\etens{N}}}
\def\etO{{\etens{O}}}
\def\etP{{\etens{P}}}
\def\etQ{{\etens{Q}}}
\def\etR{{\etens{R}}}
\def\etS{{\etens{S}}}
\def\etT{{\etens{T}}}
\def\etU{{\etens{U}}}
\def\etV{{\etens{V}}}
\def\etW{{\etens{W}}}
\def\etX{{\etens{X}}}
\def\etY{{\etens{Y}}}
\def\etZ{{\etens{Z}}}

% The true underlying data generating distribution
\newcommand{\pdata}{p_{\rm{data}}}
\newcommand{\ptarget}{p_{\rm{target}}}
\newcommand{\pprior}{p_{\rm{prior}}}
\newcommand{\pbase}{p_{\rm{base}}}
\newcommand{\pref}{p_{\rm{ref}}}

% The empirical distribution defined by the training set
\newcommand{\ptrain}{\hat{p}_{\rm{data}}}
\newcommand{\Ptrain}{\hat{P}_{\rm{data}}}
% The model distribution
\newcommand{\pmodel}{p_{\rm{model}}}
\newcommand{\Pmodel}{P_{\rm{model}}}
\newcommand{\ptildemodel}{\tilde{p}_{\rm{model}}}
% Stochastic autoencoder distributions
\newcommand{\pencode}{p_{\rm{encoder}}}
\newcommand{\pdecode}{p_{\rm{decoder}}}
\newcommand{\precons}{p_{\rm{reconstruct}}}

\newcommand{\laplace}{\mathrm{Laplace}} % Laplace distribution

\newcommand{\E}{\mathbb{E}}
\newcommand{\Ls}{\mathcal{L}}
\newcommand{\R}{\mathbb{R}}
\newcommand{\emp}{\tilde{p}}
\newcommand{\lr}{\alpha}
\newcommand{\reg}{\lambda}
\newcommand{\rect}{\mathrm{rectifier}}
\newcommand{\softmax}{\mathrm{softmax}}
\newcommand{\sigmoid}{\sigma}
\newcommand{\softplus}{\zeta}
\newcommand{\KL}{D_{\mathrm{KL}}}
\newcommand{\Var}{\mathrm{Var}}
\newcommand{\standarderror}{\mathrm{SE}}
\newcommand{\Cov}{\mathrm{Cov}}
% Wolfram Mathworld says $L^2$ is for function spaces and $\ell^2$ is for vectors
% But then they seem to use $L^2$ for vectors throughout the site, and so does
% wikipedia.
\newcommand{\normlzero}{L^0}
\newcommand{\normlone}{L^1}
\newcommand{\normltwo}{L^2}
\newcommand{\normlp}{L^p}
\newcommand{\normmax}{L^\infty}

\newcommand{\parents}{Pa} % See usage in notation.tex. Chosen to match Daphne's book.

\DeclareMathOperator*{\argmax}{arg\,max}
\DeclareMathOperator*{\argmin}{arg\,min}

\DeclareMathOperator{\sign}{sign}
\DeclareMathOperator{\Tr}{Tr}
\let\ab\allowbreak


\usepackage{hyperref}       % hyperlinks
\usepackage{url}            % simple URL typesetting
\usepackage{booktabs}       % professional-quality tables
\usepackage{amsfonts}       % blackboard math symbols
\usepackage{nicefrac}       % compact symbols for 1/2, etc.
\usepackage{microtype}      % microtypography
\usepackage{xcolor}         % colors

\usepackage{booktabs}
\usepackage{multirow}
\usepackage{graphicx}
\usepackage{caption}
\usepackage{subcaption}
\usepackage{amsfonts}
\usepackage{algorithm}
% \usepackage{algorithmic}
\usepackage{algpseudocode}
\usepackage{amssymb}
\usepackage{pifont}
\usepackage{xcolor}
\usepackage{makecell}
\usepackage{graphicx}
\usepackage{subcaption}
\usepackage{booktabs} % for professional tables
\usepackage{amsmath}
\usepackage{amsthm}
\usepackage{amssymb}
\usepackage{mathtools}
% \usepackage{natbib}
\usepackage{thmtools}
\usepackage{thm-restate}
% \usepackage{pdfsync}
\usepackage{hyperref}
\usepackage{xcolor}
\usepackage{color}
\usepackage{amsmath}
\usepackage{amsthm}
\usepackage{thm-restate}
\usepackage{multicol}
\usepackage{wrapfig}
% \definecolor{rebuttal}{HTML}{0000FF}


\newcommand{\bX}{\text{\boldmath{$X$}}}
\newcommand{\bw}{\text{\boldmath{$w$}}}
\newcommand{\bW}{\text{\boldmath{$W$}}}
\newcommand{\bH}{\text{\boldmath{$H$}}}
\newcommand{\bHtil}{\tilde{\text{\boldmath{$H$}}}}
\newcommand{\bytil}{\tilde{\text{\boldmath{$y$}}}}
\newcommand{\bxtil}{\tilde{\text{\boldmath{$x$}}}}
\newcommand{\bwtil}{\tilde{\text{\boldmath{$w$}}}}
\newcommand{\bB}{\text{\boldmath{$B$}}}
\newcommand{\bD}{\text{\boldmath{$D$}}}
\newcommand{\bR}{\text{\boldmath{$R$}}}
\newcommand{\br}{\text{\boldmath{$r$}}}
\newcommand{\bQ}{\text{\boldmath{$Q$}}}
\newcommand{\bJ}{\text{\boldmath{$J$}}}
\newcommand{\bZ}{\text{\boldmath{$Z$}}}
\newcommand{\bmu}{\text{\boldmath{$\mu$}}}


\newcommand{\bz}{\boldsymbol{z}}
\newcommand{\bx}{\boldsymbol{x}}
\newcommand{\bk}{\boldsymbol{k}}
\newcommand{\bbf}{\boldsymbol{f}}
\newcommand{\be}{\boldsymbol{e}}
\newcommand{\bdelta}{\boldsymbol{\delta}}
\newcommand{\bepsilon}{\boldsymbol{\epsilon}}
\newcommand{\bxi}{\boldsymbol{\xi}}
\newcommand{\bXi}{\boldsymbol{\Xi}}
\newcommand{\bzeta}{\boldsymbol{\zeta}}
\newcommand{\wpre}{\blodsymbol{w}_{\text{pre}}}
\newcommand{\ba}{\boldsymbol{a}}
\newcommand{\bs}{\boldsymbol{s}}
\newcommand{\bu}{\boldsymbol{u}}
\newcommand{\bb}{\boldsymbol{b}}
\newcommand{\bg}{\boldsymbol{g}}
\newcommand{\by}{\boldsymbol{y}}
\newcommand{\bY}{\boldsymbol{Y}}
\newcommand{\bS}{\boldsymbol{S}}
\newcommand{\bU}{\boldsymbol{U}}
\newcommand{\bzero}{\boldsymbol{0}}
\newcommand{\bone}{\boldsymbol{1}}
\newcommand{\beps}{\boldsymbol{\epsilon}}
\newcommand{\btheta}{\boldsymbol{\theta}}
\newcommand{\bphi}{\boldsymbol{\phi}}

\newcommand{\bl}{\boldsymbol{l}}
\newcommand{\bI}{\boldsymbol{I}}
\newcommand{\bh}{\boldsymbol{h}}
\newcommand{\bq}{\boldsymbol{q}}
\newcommand{\bA}{\boldsymbol{A}}
\newcommand{\bM}{\boldsymbol{M}}
\newcommand{\bV}{\boldsymbol{V}}
\newcommand{\bbP}{\mathbb{P}}
\newcommand{\bbI}{\mathbb{I}}
\newcommand{\bbR}{\mathbb{R}}
\newcommand{\bbC}{\mathbb{C}}
\newcommand{\grad}{\mathrm{grad}}
\newcommand{\bt}{\boldsymbol{t}}
\newcommand{\bv}{\boldsymbol{v}}
\newcommand{\col}{\text{col}}
\newcommand{\imcol}{\text{im2col}}
\newcommand{\row}{\text{row}}
\newcommand{\median}{\mathsf{med}}
\newcommand{\dist}{\text{dist}}
\newcommand{\erf}{\text{erf}}
\newcommand{\erfc}{\text{erfc}}
\newcommand{\vect}{\mathsf{vec}}
\newcommand{\diag}{\mathsf{diag}}
\newcommand{\back}{\mathsf{Back}}
\newcommand{\sW}{\mathsf{W}}
\newcommand{\dirac}{\delta_{\text{Dirac}}}

\DeclareMathOperator*{\argmin}{arg\,min}

\newtheorem{result}{\indent \em Result}
\newtheorem{conj}{\textbf{Conjecture}}
\newtheorem{assumption}{\textbf{Assumption}}
\newtheorem{definition}{\textbf{Definition}}
\newtheorem{corollary}{\textbf{Corollary}}
\newtheorem{lemma}{\textbf{Lemma}}
\newtheorem{theorem}{\textbf{Theorem}}
\newtheorem{proposition}{\textbf{Proposition}}
\newtheorem{remark}{\textbf{Remark}}
\newtheorem{apprx}{\textbf{Approximation}}
\newtheorem{example}{\textbf{Example}}
\newtheorem{claim}{\textbf{Claim}}
\newtheorem{fact}{\indent \em Fact}
\newtheorem{step}{\em Step}


\newcommand{\nn}{\nonumber}
\newcommand{\scr}{\scriptstyle}
\newcommand{\disp}{\displaystyle}
\newcommand{\mE}{\mathbb{E}}
\newcommand{\rank}{\mathrm{rank}}
\newcommand{\Var}{\mathsf{Var}}
%\newcommand{\Var}{\textrm{Var}}
\newcommand{\Cov}{\textrm{Cov}}
\newcommand{\TV}{\mathsf{TV}}
\newcommand{\cE}{\mathcal{E}}
\newcommand{\cD}{\mathcal{D}}
\newcommand{\cM}{\mathcal{M}}
\newcommand{\cV}{\mathcal{V}}
\newcommand{\cU}{\mathcal{U}}
\newcommand{\cR}{\mathcal{R}}
\newcommand{\cX}{\mathcal{X}}
\newcommand{\cY}{\mathcal{Y}}
\newcommand{\cZ}{\mathcal{Z}}
\newcommand{\cL}{\mathcal{L}}
\newcommand{\cW}{\mathcal{W}}
\newcommand{\cS}{\mathcal{S}}
\newcommand{\cC}{\mathcal{C}}
\newcommand{\cN}{\mathcal{N}}
\newcommand{\cB}{\mathcal{B}}
\newcommand{\cP}{\mathcal{P}}
\newcommand{\cI}{\mathcal{I}}
\newcommand{\cF}{\mathcal{F}}
\newcommand{\cA}{\mathcal{A}}
\newcommand{\cT}{\mathcal{T}}
\newcommand{\cJ}{\mathcal{J}}
\newcommand{\cK}{\mathcal{K}}
\newcommand{\cQ}{\mathcal{Q}}
\newcommand{\cH}{\mathcal{H}}
\newcommand{\cG}{\mathcal{G}}
\newcommand{\cO}{\mathcal{O}}
\newcommand{\dalpha}{\dot{\alpha}}
\newcommand{\dbeta}{\dot{\beta}}

\newcommand{\tx}{\tilde{\bx}}
\newcommand{\tg}{\tilde{g}}
\newcommand{\ty}{\tilde{y}}
\newcommand{\tz}{\tilde{z}}
\newcommand{\ts}{\tilde{s}}
\newcommand{\tw}{\tilde{w}}
\newcommand{\tu}{\tilde{u}}
\newcommand{\tp}{\tilde{p}}
\newcommand{\tq}{\tilde{q}}
\newcommand{\tR}{\Tilde{R}}
\newcommand{\tX}{\Tilde{X}}
\newcommand{\tY}{\Tilde{Y}}
\newcommand{\tU}{\Tilde{U}}
\newcommand{\tP}{\Tilde{P}}
\newcommand{\tZ}{\Tilde{Z}}
\newcommand{\tN}{\Tilde{N}}
\newcommand{\tf}{\tilde{f}}
\newcommand{\talpha}{\tilde{\alpha}}
\newcommand{\tbeta}{\tilde{\beta}}
\newcommand{\tlambda}{\tilde{\lambda}}
\newcommand{\tepsilon}{\tilde{\epsilon}}
\newcommand{\ttheta}{\Tilde{\boldsymbol{\theta}}}
\newcommand{\tr}{\mathrm{tr}}



\newcommand{\chZ}{\check{Z}}
\newcommand{\chN}{\check{N}}
\newcommand{\chR}{\check{R}}

\newcommand{\ut}{\underline{t}}
\newcommand{\ux}{\underline{x}}
\newcommand{\uy}{\underline{y}}
\newcommand{\uu}{\underline{u}}
\newcommand{\uh}{\underline{h}}
\newcommand{\uhy}{\underline{\hat{y}}}
\newcommand{\up}{\underline{p}}

\newcommand{\brw}{\breve{w}}
\newcommand{\brR}{\breve{R}}

\newcommand{\hW}{\hat{W}}
\newcommand{\hR}{\hat{R}}
\newcommand{\hs}{\hat{s}}
\newcommand{\hw}{\hat{w}}
\newcommand{\hY}{\hat{Y}}
\newcommand{\hy}{\hat{y}}
\newcommand{\hz}{\hat{z}}
\newcommand{\hZ}{\hat{Z}}
\newcommand{\hv}{\hat{v}}
\newcommand{\hN}{\hat{N}}
\newcommand{\hX}{\hat{X}}
\newcommand{\hhz}{\hat{\hat{z}}}
\newcommand{\hhs}{\hat{\hat{s}}}
\newcommand{\hhw}{\hat{\hat{w}}}
\newcommand{\hhv}{\hat{\hat{v}}}

\newcommand{\baralpha}{\bar{\alpha}}
\newcommand{\barbeta}{\bar{\beta}}
\newcommand{\bargamma}{\bar{\gamma}}
\newcommand{\bartheta}{\bar{\theta}}

\newenvironment{oneshot}[1]{\@begintheorem{#1}{\unskip}}{\@endtheorem}

\providecommand{\lemmaname}{Lemma}
\providecommand{\theoremname}{Theorem}





\providecommand{\lemmaname}{Lemma}
\providecommand{\theoremname}{Theorem}

\makeatother

\providecommand{\lemmaname}{Lemma}
\providecommand{\remarkname}{Remark}
\providecommand{\theoremname}{Theorem}

% \newcommand{\yi}[1]{\textcolor{red}{#1}}
% \newcommand{\revise}[1]{\textcolor{blue}{#1}}
% \newcommand{\fix}{\marginpar{FIX}}
% \newcommand{\new}{\marginpar{NEW}}

\title{Improved Diffusion-based Generative Model with Better Adversarial Robustness}


% \title{Formatting Instructions for ICLR 2025 \\ Conference Submissions}

% Authors must not appear in the submitted version. They should be hidden
% as long as the \iclrfinalcopy macro remains commented out below.
% Non-anonymous submissions will be rejected without review.
\renewcommand\footnotemark{}
\author{
    \!\!Zekun Wang$^{*1}$, Mingyang Yi$^{*\dagger2}$, Shuchen Xue$^{3,4}$, Zhenguo Li$^{5}$, Ming Liu$^{\dagger1}$, Bing Qin$^{1}$, \\
    \textbf{Zhi-Ming Ma$^{3,4}$}%
    \thanks{$^*$Equal contribution}%
    \thanks{$^{\dagger}$Corresponding author}\\
    $^{1}$Harbin Institute of Technology ~~~~ $^{2}$Renmin University of China \\
    $^{3}$Academy of Mathematics and Systems Science, Chinese Academy of Sciences \\
    $^{4}$University of Chinese Academy of Sciences ~~~~$^{5}$Huawei Noah’s Ark Lab \\
    \footnotesize{\texttt{zkwang@ir.hit.edu.cn}} ~~~\footnotesize{\texttt{yimingyang@ruc.edu.cn}}
    \\\footnotesize{\texttt{xueshuchen17@mails.ucas.ac.cn}}
}

% The \author macro works with any number of authors. There are two commands
% used to separate the names and addresses of multiple authors: \And and \AND.
%
% Using \And between authors leaves it to \LaTeX{} to determine where to break
% the lines. Using \AND forces a linebreak at that point. So, if \LaTeX{}
% puts 3 of 4 authors names on the first line, and the last on the second
% line, try using \AND instead of \And before the third author name.

\newcommand{\fix}{\marginpar{FIX}}
\newcommand{\new}{\marginpar{NEW}}

\iclrfinalcopy % Uncomment for camera-ready version, but NOT for submission.
\begin{document}


\maketitle

\maketitle
\begin{abstract}
  Diffusion Probabilistic Models (DPMs) have achieved significant success in generative tasks. However, their training and sampling processes suffer from the issue of distribution mismatch. During the denoising process, the input data distributions differ between the training and inference stages, potentially leading to inaccurate data generation. To obviate this, we analyze the training objective of DPMs and theoretically demonstrate that this mismatch can be alleviated through Distributionally Robust Optimization (DRO), which is equivalent to performing robustness-driven Adversarial Training (AT) on DPMs. Furthermore, for the recently proposed Consistency Model (CM), which distills the inference process of the DPM, we prove that its training objective also encounters the mismatch issue. Fortunately, this issue can be mitigated by AT as well. Based on these insights, we propose to conduct efficient AT on both DPM and CM. Finally, extensive empirical studies validate the effectiveness of AT in diffusion-based models. The code is available at \url{https://github.com/kugwzk/AT_Diff}.
\end{abstract}


\section{Introduction}\label{sec:intro}
Diffusion Probabilistic Models (DPMs)~\citep{ ho2020denoising, song2020score,yi2024towards} have achieved remarkable success across a wide range of generative tasks such as image synthesis~\citep{dhariwal2021diffusion, Rombach_2022_CVPR, ho2022cascaded}, video generation~\citep{ho2022vdm,blattmann2023videoldm}, text-to-image generation~\citep{2021glide,dalle2,imagen}, \emph{etc}. The core mechanism of DPMs involves a forward diffusion process that progressively injects noise into the data, followed by a reverse process that learns to generate data by denoising the noise.
Unlike traditional generative models such as GANs\citep{goodfellow2014generative} or VAEs \citep{kingma2013auto}, which directly map an easily sampled latent variable (e.g., Gaussian noise) to the target data through a single network function evaluation (NFE), DPMs adopt a gradual denoising approach that requires multiple NFEs~\citep{song2020denoising, salimans2022progressive, lu2022dpmsa, ma2024surprising}. However, this noising-then-denoising process introduces a distribution mismatch between the training and sampling stages, potentially leading to inaccuracies in the generated outputs.
\par
Concretely, during the training stage, the model is learned to predict the noise in ground-truth noisy data derived from the training set. In contrast, during the inference stage, the input distribution is obtained from the output generated by the DPM in the previous step, which differs from the training phase, caused by the inaccurate estimation of the score function due to training \citep{song2021maximum,yi2023generalization} and the discretization error \citep{chen2022sampling,li2023towards,xue2024sa,xue2024accelerating} brought by sampling. 
Such distribution mismatches are referred to as \textit{Exposure Bias}, which has been discussed in auto-regressive language models \citep{bengio2015scheduled, ranzato2016sequence}. 
\par
Recently, the aforementioned distribution mismatch problem in diffusion has been also recognized by \citep{diffusion-ip,li2024on,ren2024multistep,es,li2024alleviating,Reflected_diffusion_models}. However, these studies are either rely on strong mismatch distributional assumptions (e.g., Gaussian) \citep{diffusion-ip,es,ren2024multistep} or incur significant additional computational costs \citep{li2024on}. 
This indicates that a more practical solution to this problem has been overlooked until now. To bridge this gap, we begin with the discrete DPM introduced in \citep{ho2020denoising}. Intuitively, although there is a mismatch between training and inference, the distributions of intermediate noise generated during the inference stage are close to the ground-truth distributions observed during training. Therefore, improving the distributional robustness \citep{yi2021improved,namkoong2019reliable,shapiro2017distributionally} (which measures the robustness of the model to distributional perturbations in training data) of DPM mitigates the distribution mismatch problem. To achieve this, we refer to Distribution Robust Optimization (DRO) \citep{shapiro2017distributionally,namkoong2019reliable}, which aims to improve the distributional robustness of models. We then prove that applying DRO to DPM is mathematically equivalent to implementing \emph{robustness-driven} Adversarial Training (AT) \citep{madry2018towards,freeat,yi2021improved} on DPM. \footnote{Note that the ``adversarial'' here refers to perturbation to input training data, instead of the adversarial of generator-discriminator in GAN \citep{goodfellow2014generative}.} Following the DRO framework, we also analyze the recently proposed diffusion-based Consistency Model (CM)~\citep{song2023consistency,luo2023latent} which distills the trajectory of DPM into a model with one NFE generation. We first prove that the training objective of CM similarly suffers from the mismatch issue as in multi-step DPM. Moreover, the issue can also be mitigated by implementing AT. Therefore, for both DPM and CM, we propose to apply efficient AT (e.g., ``Free-AT'' \citep{freeat}) during their training stages to mitigate the distribution mismatch problem.\footnote{Notably, the standard AT \citep{madry2018towards} solves a minimax problem that slows the training process. The efficient AT has no extra computational cost compared to the standard training ones \citep{freeat}.} Finally, we summarize our contributions as follows.
\begin{itemize}
\item We conduct an in-depth analysis of the diffusion-based models (DPM and CM) from a theoretical perspective and systematically characterize its distribution mismatch problem. 
% We find that minimizing the evidence lower bound of the negative log likelihood of training data implicitly optimizes the likelihood of all latent variables.
\item For both DPM and CM, we theoretically show that their mismatch problem is mitigated by DRO, which is equivalent to implementing AT with proved error bounds during training. 
\item We propose to conduct efficient AT on both DPM and CM in various tasks, including image generation on \texttt{CIFAR10} 32$\times$32\citep{krizhevsky2009learning} and \texttt{ImageNet} 64$\times$64 \citep{deng2009imagenet}, and zero-shot Text-to-Image (T2I) generation on MS-COCO 512$\times$512~\citep{mscoco}. Extensive experimental results illustrate the effectiveness of the proposed AT training method in alleviating the distribution mismatch of DPM and CM. 
\end{itemize}
\section{RELATED WORK}
\label{sec:relatedwork}
In this section, we describe the previous works related to our proposal, which are divided into two parts. In Section~\ref{sec:relatedwork_exoplanet}, we present a review of approaches based on machine learning techniques for the detection of planetary transit signals. Section~\ref{sec:relatedwork_attention} provides an account of the approaches based on attention mechanisms applied in Astronomy.\par

\subsection{Exoplanet detection}
\label{sec:relatedwork_exoplanet}
Machine learning methods have achieved great performance for the automatic selection of exoplanet transit signals. One of the earliest applications of machine learning is a model named Autovetter \citep{MCcauliff}, which is a random forest (RF) model based on characteristics derived from Kepler pipeline statistics to classify exoplanet and false positive signals. Then, other studies emerged that also used supervised learning. \cite{mislis2016sidra} also used a RF, but unlike the work by \citet{MCcauliff}, they used simulated light curves and a box least square \citep[BLS;][]{kovacs2002box}-based periodogram to search for transiting exoplanets. \citet{thompson2015machine} proposed a k-nearest neighbors model for Kepler data to determine if a given signal has similarity to known transits. Unsupervised learning techniques were also applied, such as self-organizing maps (SOM), proposed \citet{armstrong2016transit}; which implements an architecture to segment similar light curves. In the same way, \citet{armstrong2018automatic} developed a combination of supervised and unsupervised learning, including RF and SOM models. In general, these approaches require a previous phase of feature engineering for each light curve. \par

%DL is a modern data-driven technology that automatically extracts characteristics, and that has been successful in classification problems from a variety of application domains. The architecture relies on several layers of NNs of simple interconnected units and uses layers to build increasingly complex and useful features by means of linear and non-linear transformation. This family of models is capable of generating increasingly high-level representations \citep{lecun2015deep}.

The application of DL for exoplanetary signal detection has evolved rapidly in recent years and has become very popular in planetary science.  \citet{pearson2018} and \citet{zucker2018shallow} developed CNN-based algorithms that learn from synthetic data to search for exoplanets. Perhaps one of the most successful applications of the DL models in transit detection was that of \citet{Shallue_2018}; who, in collaboration with Google, proposed a CNN named AstroNet that recognizes exoplanet signals in real data from Kepler. AstroNet uses the training set of labelled TCEs from the Autovetter planet candidate catalog of Q1–Q17 data release 24 (DR24) of the Kepler mission \citep{catanzarite2015autovetter}. AstroNet analyses the data in two views: a ``global view'', and ``local view'' \citep{Shallue_2018}. \par


% The global view shows the characteristics of the light curve over an orbital period, and a local view shows the moment at occurring the transit in detail

%different = space-based

Based on AstroNet, researchers have modified the original AstroNet model to rank candidates from different surveys, specifically for Kepler and TESS missions. \citet{ansdell2018scientific} developed a CNN trained on Kepler data, and included for the first time the information on the centroids, showing that the model improves performance considerably. Then, \citet{osborn2020rapid} and \citet{yu2019identifying} also included the centroids information, but in addition, \citet{osborn2020rapid} included information of the stellar and transit parameters. Finally, \citet{rao2021nigraha} proposed a pipeline that includes a new ``half-phase'' view of the transit signal. This half-phase view represents a transit view with a different time and phase. The purpose of this view is to recover any possible secondary eclipse (the object hiding behind the disk of the primary star).


%last pipeline applies a procedure after the prediction of the model to obtain new candidates, this process is carried out through a series of steps that include the evaluation with Discovery and Validation of Exoplanets (DAVE) \citet{kostov2019discovery} that was adapted for the TESS telescope.\par
%



\subsection{Attention mechanisms in astronomy}
\label{sec:relatedwork_attention}
Despite the remarkable success of attention mechanisms in sequential data, few papers have exploited their advantages in astronomy. In particular, there are no models based on attention mechanisms for detecting planets. Below we present a summary of the main applications of this modeling approach to astronomy, based on two points of view; performance and interpretability of the model.\par
%Attention mechanisms have not yet been explored in all sub-areas of astronomy. However, recent works show a successful application of the mechanism.
%performance

The application of attention mechanisms has shown improvements in the performance of some regression and classification tasks compared to previous approaches. One of the first implementations of the attention mechanism was to find gravitational lenses proposed by \citet{thuruthipilly2021finding}. They designed 21 self-attention-based encoder models, where each model was trained separately with 18,000 simulated images, demonstrating that the model based on the Transformer has a better performance and uses fewer trainable parameters compared to CNN. A novel application was proposed by \citet{lin2021galaxy} for the morphological classification of galaxies, who used an architecture derived from the Transformer, named Vision Transformer (VIT) \citep{dosovitskiy2020image}. \citet{lin2021galaxy} demonstrated competitive results compared to CNNs. Another application with successful results was proposed by \citet{zerveas2021transformer}; which first proposed a transformer-based framework for learning unsupervised representations of multivariate time series. Their methodology takes advantage of unlabeled data to train an encoder and extract dense vector representations of time series. Subsequently, they evaluate the model for regression and classification tasks, demonstrating better performance than other state-of-the-art supervised methods, even with data sets with limited samples.

%interpretation
Regarding the interpretability of the model, a recent contribution that analyses the attention maps was presented by \citet{bowles20212}, which explored the use of group-equivariant self-attention for radio astronomy classification. Compared to other approaches, this model analysed the attention maps of the predictions and showed that the mechanism extracts the brightest spots and jets of the radio source more clearly. This indicates that attention maps for prediction interpretation could help experts see patterns that the human eye often misses. \par

In the field of variable stars, \citet{allam2021paying} employed the mechanism for classifying multivariate time series in variable stars. And additionally, \citet{allam2021paying} showed that the activation weights are accommodated according to the variation in brightness of the star, achieving a more interpretable model. And finally, related to the TESS telescope, \citet{morvan2022don} proposed a model that removes the noise from the light curves through the distribution of attention weights. \citet{morvan2022don} showed that the use of the attention mechanism is excellent for removing noise and outliers in time series datasets compared with other approaches. In addition, the use of attention maps allowed them to show the representations learned from the model. \par

Recent attention mechanism approaches in astronomy demonstrate comparable results with earlier approaches, such as CNNs. At the same time, they offer interpretability of their results, which allows a post-prediction analysis. \par



% Despite their superior performance in generation tasks, one of the drawbacks of DPM is they require a considerable number of network function evaluations (NFEs) when compared to alternative methodologies like Generative Adversarial Networks (GANs) \citep{goodfellow2014generative} or Variational Autoencoders (VAEs) \citep{kingma2013auto}, which is a notable computational limitation for their broader application. A series of training-based and sampling-based methods \citep{song2020denoising,salimans2022progressive,lu2022dpmsa,xue2024sa,luo2024diff,xue2024accelerating,ma2024surprising} have been proposed to tackle the issues. Another drawback is the distribution mismatch phenomenon during the training and sampling stages. The input distribution of the training stage is the true noisy data distribution derived from the training set. In contrast, during the inference stage, the input distribution is derived from the output generated by the Diffusion model in the previous step, which differs from the training phase due to the inaccurate score function estimation during training and discretization error during sampling. Moreover, the discrepancy error accumulates across the whole generation process. Such distribution mismatches are commonly referred to as Exposure Bias or Distribution Shift. Some studies \citep{diffusion-ip,li2024on,ren2024multistep,es,li2024alleviating} attempt to alleviate the exposure bias in DPM. 
% \par
% In this work, we focus on modeling and mitigating the distribution mismatch in Diffusion Models. Our main contributions are summarized as follows:


\section{Preliminary}

\paragraph{Diffusion Probabilistic Models.} DPM~\citep{sohl2015deep, ho2020denoising} constructs the Markov chain $\bx_{t}$ by transition kernel $q(\bx_{t+1}\mid\bx_{t}) = \cN(\sqrt{\alpha_{t+1}}  \bx_{t}, (1-\alpha_{t+1})\bI)$, where $\alpha_1, \cdots, \alpha_T$ are in $[0, 1]$. Let $\baralpha_t := \Pi_{s=1}^t \alpha_s$, and $\bx_{0}\sim q$ be ground-truth data. Then, for $\bx_{t}$, it holds   
\begin{equation}\label{eq:xt}
    \small
        \bx_{t} = \sqrt{\baralpha_{t}}\bx_{0} + \sqrt{1 - \baralpha_{t}}\beps_{t} \qquad t=1, \cdots, T,
\end{equation}
with $\beps_{t}\sim \cN(0, \bI)$. The reverse process $p_{\btheta}(\bx_{t} \mid \bx_{t + 1})$ is parameterized as
\begin{equation}
    p_{\btheta}(\bx_{t} \mid \bx_{t + 1}) = \cN(\mu_{\btheta}(\bx_{t + 1}, t+1), \sigma_{t+1}^2 \bI),
\end{equation}
where $\sigma_{t+1}^2 = 1 - \alpha_{t+1}$.  %$\frac{1-\baralpha_t}{1-\baralpha_{t+1}}(1 - \alpha_{t+1})$. 
To learn $p_{\btheta}(\bx_{t} \mid \bx_{t + 1})$, a standard method is to minimize the following evidence lower bound of negative log-likelihood (NLL) \citep{ho2020denoising}, 
\begin{equation}\label{eq:nll loss}
    \small
    \begin{aligned}
        -\mE_{q}\left[\log{p}_{\btheta}(\bx_{0})\right] \leq \mE_{q}\left[-\log{\frac{p_{\btheta}(\bx_{0:T})}{q(\bx_{1:T}\mid \bx_{0})}}\right].
    \end{aligned}
\end{equation}
Here, minimizing the ELBO in the r.h.s. of above inequality links to $p_{\btheta}(\bx_{t} \mid \bx_{t+1})$ since it is equivalent to minimizing the following rewritten objective  
\begin{equation}\label{eq:rewrite nll upper bound}
    \small
    \min_{\btheta} \left\{D_{KL}(q(\bx_{T})\parallel p_{\btheta}(\bx_{T})) + \sum_{t = 0}^{T - 1}\underbrace{D_{KL}(q(\bx_{t}\mid \bx_{t + 1}) \parallel p_{\btheta}(\bx_{t}\mid \bx_{t + 1}))}_{L_{t}}\right\}, % + H(\bx_0),
\end{equation}
as in \citep{ho2020denoising,bao2022analytic,yi2023generalization}. Here, the conditional Kullback–Leibler (KL) divergence $D_{KL}(q(\bx_{t}\mid \bx_{t + 1})\parallel p(\bx_{t}\mid \bx_{t + 1})) = \int q(\bx_{t}\mid \bx_{t + 1})\log{\frac{q(\bx_{t}\mid \bx_{t + 1})}{p(\bx_{t}\mid \bx_{t + 1})}}d\bx_{t} d\bx_{t + 1}$ \citep{duchi2016lecture}, and minimizing $L_{t}$ is equivalent to solve the following noise prediction problem
\begin{equation}\label{eq:noise prediction}
    \small
    \min_{\btheta}\mE\left[\left\|\beps_{\btheta}(\sqrt{\baralpha_{t}}\bx_{0} + \sqrt{1 - \baralpha_{t}}\beps_{t}, t) - \beps_{t}\right\|^{2}\right]. 
\end{equation}
We use $\|\cdot\|_{p}$ to denote $\ell_{p}$-norm. Unless specified, the norm $\|\cdot\|$ refers to the $\ell_2$-norm $\|\cdot\|_2$. Since $\baralpha_{t}\rightarrow 0$ for $t\to T$, $\bx_{0}$ is obtained by conducting the reverse diffusion process $p_{\btheta}(\bx_{t}\mid \bx_{t + 1})$ starting from $\bx_{T}\sim\cN(0, \bI)$ and $\beps\sim\cN(0, \bI)$, under the learned model $\beps_{\btheta}$ with 
\begin{equation}\label{eq:transition}
    \small
       \bx_{t} = \frac{1}{\sqrt{\alpha_{t + 1}}}\left(\bx_{t + 1} - \frac{1 - \alpha_{t + 1}}{\sqrt{1 - \baralpha_{t + 1}}} \beps_{\btheta}(\bx_{t + 1}, t+1)\right) + \sqrt{1 - \alpha_{t + 1}}\beps.
\end{equation} 

\paragraph{Wasserstein Distance.} For integer $p>0$, $\Gamma(\mu, \nu)$ as the set of union distributions with marginal $\mu$ and $\nu$, the Wasserstein $p$-distance \citep{villani2009optimal} between distributions $\mu$ and $\nu$ with finite $p$-moments is
\begin{equation}
    \sW_{p}^{p}(\mu, \nu) = \inf_{\gamma \in \Gamma(\mu, \nu)} \mE_{(\bx,\by) \sim \gamma} \|\bx - \by\|_{p}^p.
\end{equation}
% where $\Gamma(\mu, \nu)$ is the set of all joint distributions whose marginal distributions correspond to $\mu$ and $\nu$.
% Furthermore, there exists a deterministic (implicit) reverse process pointed out by \citep{song2020denoising}
% \begin{equation}
%     \small
%         \bx_{t} = \sqrt{\baralpha_{t}}\left( \frac{\bx_{t+1} - \sqrt{1-\baralpha_{t+1}} \beps_{\btheta}(\bx_{t + 1}, t+1) }{\sqrt{\baralpha_{t+1}}} \right) + \sqrt{1 - \baralpha_{t}}\beps_{\btheta}(\bx_{t + 1}, t+1).
% \end{equation}

\section{Robustness-driven Adversarial Training of Diffusion Models}\label{sec:diffusion model as multi-step}
In this section, we formally show that the success of DPM relies on specific conditions, i.e., $\bx_{t}$ is close to $\bx_{t+1}$. Next, to mitigate the drawbacks brought by the restriction, we propose to consider the distribution mismatch problem as discussed in Section \ref{sec:intro}, and connect the problem to a rewritten ELBO. Finally, we apply DRO for this ELBO to mitigate the distribution mismatch problem and finally link it to AT to be implemented in practice. 

\subsection{How Does DPM Works in Practice?}\label{sec:How Does DPM Works in Practice}
Notably, minimizing \eqref{eq:rewrite nll upper bound} potentially obtains a sharp NLL under target distribution $q(\bx_{0})$. However, in the following proposition, we show that \eqref{eq:rewrite nll upper bound} also implicitly minimizes the NLL of each $\bx_{t}$.
\begin{restatable}{proposition}{elboupperbound}\label{pro:elbo upper bound}
    The minimization problem \eqref{eq:rewrite nll upper bound} is equivalent to minimizing an upper bound of $\mE_{q}[-\log{p_{\btheta}}(\bx_{t})]$ for any $0\leq t \leq T$.
\end{restatable}
The proof is provided in Appendix \ref{app:proofs in sec:diffusion model as multi-step}. It shows that though \eqref{eq:rewrite nll upper bound} is proposed to generate $\bx_{0}\sim q(\bx_{0})$, it also guides the model to generate $\bx_{t}$ such that $p_{\btheta}(\bx_{t})$ approximates the ground-truth distribution $q(\bx_{t})$. The conclusion is nontrivial as minimizing the ELBO of NLL $\mE_{q}\left[-\log{p}_{\btheta}(\bx_{0})\right]$ does not necessarily impose any restrictions on $\bx_{t}$ for $t \geq 1$. % Moreover, it does not necessarily make $\bx_{t}$ follow $q(\bx_{t})$ during the inference stage, to generate $\bx_{0}\sim q(\bx_{0})$.   
\par
Next, we will further explain why \eqref{eq:rewrite nll upper bound} leads to a small NLL of $\bx_{t}$. In $L_{t}$ of \eqref{eq:rewrite nll upper bound}, $p_{\btheta}(\bx_{t}\mid \bx_{t + 1})$ approximates $q(\bx_{t}\mid \bx_{t + 1})$ with $\bx_{t + 1}\sim q(\bx_{t + 1})$ representing ground-truth data. Consequently, $p_{\btheta}(\bx_{t})$ approximates $q(\bx_{t})$ by recursively applying such a relationship as in the following proposition. 
% \begin{remark}
% If the NLL $-\mE_{q}\left[\log{p}_{\btheta}(\bx_{0})\right]$ is the optimization target, it is not necessary for $p_{\btheta}(\bx_{t})$ to approximate $q(\bx_{t})$ in order to generate the desired $\bx_{0}$ because NLL  does not impose any requirements on $\bx_{t}$ ($t \geq 1$).
% \end{remark}
\par
\begin{restatable}{proposition}{elboupperboundxt}\label{pro:elbo upper bound xt}
    Suppose $p_{\btheta}(\bx_{t}\mid \bx_{t + 1})$ matches $q(\bx_{t}\mid \bx_{t + 1})$ well such that
    \begin{equation}
        \small
        L_{t} = D_{KL}(q(\bx_{t}\mid \bx_{t + 1}) \parallel p_{\btheta}(\bx_{t}\mid \bx_{t + 1}))\le \frac{\gamma}{T},
    \end{equation}
    and the discrepancy satisfies $D_{KL}(q(\bx_{T})\parallel p_{\btheta}(\bx_{T})) \le \gamma_0$, then for any $0\leq t \leq T$, we have  
    \begin{equation}\label{eq:accumulated error}
        \small
        D_{KL}(q(\bx_{t})\parallel p_{\btheta}(\bx_{t})) \leq D_{KL}(q(\bx_{T})\parallel p_{\btheta}(\bx_{T})) + L_{t} \le \gamma_0 + \frac{(T-t)\gamma}{T}.
    \end{equation}
\end{restatable}
The results is similarly obtained in \citep{chen2023improved}, while their result is applied for $D_{KL}(q(\bx_{0})\parallel p_{\btheta_{0}})$, which is narrowed compared with Proposition \ref{pro:elbo upper bound xt}. The proof is provided in Appendix \ref{app:proofs in sec:diffusion model as multi-step}, which formally explains why \eqref{eq:rewrite nll upper bound} results in $p_{\btheta}(\bx_{t})$ approximating $q(\bx_{t})$. However, this proposition is built upon small $L_{t}$, and notably, the error introduced by $L_{t}$ will be accumulated on the r.h.s. of \eqref{eq:accumulated error}, as it increases w.r.t. $t$. This phenomenon is caused by the \emph{distribution mismatch problem} discussed in Section \ref{sec:intro}. Concretely, in \eqref{eq:rewrite nll upper bound}, minimizing $L_{t}$ learns the transition probability $p_{\btheta}(\bx_{t}\mid \bx_{t + 1})$ based on $\bx_{t + 1}\sim q(\bx_{t + 1})$, while in practice, $\bx_{t}$ in \eqref{eq:transition} is generated from $\bx_{t + 1}\sim p_{\btheta}(\bx_{t + 1})$. The error between $p_{\btheta}(\bx_{t + 1})$ and $q(\bx_{t + 1})$ will propagates into the error between $p_{\btheta}(\bx_{t})$ and $q(\bx_{t})$ as in \eqref{eq:accumulated error}.  
\par
Therefore, owing to the existence of distribution mismatch, only if $L_{t}$ is minimized, the gap between $p_{\btheta}(\bx_{t})$ and $q(\bx_{t})$ can be guaranteed. However, the following proposition proved in Appendix \ref{app:proofs in sec:diffusion model as multi-step} indicates that $L_{t}$ is theoretically minimized with restrictions. 
\begin{restatable}{proposition}{gaussianinverse}
    $L_{t}$ in \eqref{eq:rewrite nll upper bound} is well minimized, only if $q(\bx_{t + 1})$ is Gaussian or $\|\bx_{t + 1} - \bx_{t}\|\to 0$.
\end{restatable}
\par
In practice, the $q(\bx_{t + 1})$ is usually non-Gaussian. Besides, the gap $\|\bx_{t + 1} - \bx_{t}\|$ is not necessarily small, especially for samplers with few sampling steps, e.g., DDIM \citep{song2020denoising}, DPM-Solver \citep{lu2022dpm}. Therefore, in practice, the accumulated error in \eqref{eq:accumulated error} caused by the distribution mismatch problem may become large, and degenerate the quality of $\bx_{0}$. 


\begin{figure}[t]
		\centering
        \vspace{-0.2in}
    	\includegraphics[scale=0.5]{./pic/adversarial_training.pdf}
		\caption{A comparison between standard training and the proposed distributional robust optimization in \eqref{eq:dro objective}. When minimizing $D_{KL}(\tq_{t}(\bx_{t}\mid \bx_{t + 1})\parallel p_{\btheta}(\bx_{t}\mid \bx_{t + 1}))$, the $\bx_{t + 1}$ is sampled from $\tq_{t}(\bx_{t + 1})$, such that both $\tq_{t}(\bx_{t + 1})$ in training stage and $p_{\btheta}(\bx_{t + 1})$ in inference stage are in $B_{D_{KL}}(q(x_{t + 1}), \eta_{0})$, so that $p_{\btheta}(\bx_{t})$ tends to locates in $B_{D_{KL}}(q(x_{t}), \eta_{0})$ as well as $\tq_{t}(\bx_{t})$. Then, the distributional robustness captured by \eqref{eq:dro objective} guarantees the generated $p_{\btheta}(\bx_{t})$ always locates around $q(\bx_{t})$ for all $t$.}
		\vspace{-0.2in}
		\label{fig:adversarial training}
	\end{figure}

\subsection{Distributional Robustness in DPM}\label{sec:Distributional Robustness in DPM}
% Then the condition $D_{KL}(q(\bx_{t - 1}\mid \bx_{t}) \parallel p_{\btheta}(\bx_{t - 1}\mid \bx_{t})) \approx 0$ that enables the stable generating process is broken. Thus learning the transition probability $p_{\btheta}(\bx_{t - 1}\mid \bx_{t})$ by matching $q(\bx_{t - 1}\mid \bx_{t})$ is not a wise choice. Next, we consider reformulate the formulation of $p_{\btheta}(\bx_{t - 1}\mid \bx_{t})$. 
% \par
Inspired by the discussion above, we propose a new training objective as the sum of NLLs under $\bx_{t}$,  
\begin{equation}\label{eq:new objective}
    \small
    \min_{\btheta}\cL(\btheta) = \sum_{t = 0}^{T}\mE_{q}\left[-\log{p_{\btheta}}(\bx_{t})\right].
\end{equation} 
Then the following proposition constructs ELBOs for each of $\mE_{q}[-\log{p_{\btheta}}(\bx_{t})]$. 
\begin{restatable}{proposition}{advelboupperbound}\label{pro:adv elbo upper bound}
    For any distribution $\tq$ satisfies $\tq(\bx_{t}) = q(\bx_{t})$ for specific $t$, we have  
    \begin{equation}\label{eq:new elbo}
        \small
        \mE_{q}\left[-\log{p_{\btheta}}(\bx_{t})\right] \le \underbrace{D_{KL}(\tq(\bx_{t}\mid \bx_{t + 1}) \parallel p_{\btheta}(\bx_{t}\mid \bx_{t + 1}))}_{L^{\tq}_{t}} + C, 
    \end{equation}
    for a constant $C$ independent of $\btheta$. 
\end{restatable}
The proof is in Appendix \ref{app:proofs in sec:Distributional Robustness in DPM}. This proposition generalizes the results in Proposition \ref{pro:elbo upper bound} since $\tq$ can be taken as $q$ in Proposition \ref{pro:elbo upper bound}. During minimizing $L^{\tq}_{t}$, the transition probability $p_{\btheta}(\bx_{t}\mid \bx_{t + 1})$ matches $\tq(\bx_{t}\mid \bx_{t + 1})$, while $\bx_{t + 1}\sim \tq(\bx_{t + 1})$ in the training stage has no restriction. Thus, one may take $\tq(\bx_{t + 1}) \approx p_{\btheta}(\bx_{t + 1})$, then in $L_{t}^{\tq}$, $p_{\btheta}(\bx_{t}\mid \bx_{t + 1})$ matches $\tq(\bx_{t}\mid \bx_{t + 1})$ leads $p_{\btheta}(\bx_{t})\approx \tq(\bx_{t}) = q(\bx_{t})$, which mitigates the distribution mismatch problem, when minimizing such $L_{t}^{\tq}$. 
\par
Unfortunately, for each $t$, obtaining such specific $\tq_{t}(\bx_{t + 1}) =  p_{\btheta}(\bx_{t + 1})$ is computationally expensive \citep{li2024on}, which prevents us using desired $\tq_{t}(\bx_{t + 1})$. However, we know $p_{\btheta}(\bx_{t + 1})$ is around $q(\bx_{t + 1})$. Therefore, by borrowing the idea from DRO \citep{shapiro2017distributionally}, for each $t$, we propose to minimize the maximal value of $L_{t}^{\tq_{t}}$ over all possible $\tq_{t}(\bx_{t + 1})$ around $q(\bx_{t + 1})$. This leads to a small $L_{t}^{p_{\btheta}}$, as $p_{\btheta}(\bx_{t + 1})$ locates around $q(\bx_{t + 1})$, so that is included in the ``maximal range''. Technically, the DRO-based EBLO of \eqref{eq:new elbo} is formulated as follows. Here $p_{\btheta}(x_{t + 1})$ is supposed in $B_{D_{KL}}(q(\bx_{t + 1}), \eta_{0})$, and it capatures the distributional robustness of $p_{\btheta}(\bx_{t}\mid \bx_{t + 1})$ w.r.t. input $\bx_{t + 1}$. 
\begin{equation}\label{eq:dro objective}
    \small
    \begin{aligned}
         & \min_{\btheta} \sum_{t = 0}^{T - 1} L_{t}^{\mathrm{DRO}}(\btheta) = \min_{\btheta} \sum_{t = 0}^{T - 1} \sup_{\tq_{t}(\bx_{t + 1})\in B_{D_{KL}}(q(\bx_{t + 1}), \eta_{0})}D_{KL}(\tq_{t}(\bx_{t}\mid \bx_{t + 1})\parallel p_{\btheta}(\bx_{t}\mid \bx_{t + 1})); \\
         & s.t. \qquad \tq_{t}(\bx_{t}) = q(\bx_{t}).
    \end{aligned}
\end{equation}
Here $\tq_{t}(\bx_{t + 1})\in B_{D_{KL}}(q(\bx_{t + 1}), \eta_{0})$ means $D_{KL}(q(\bx_{t + 1})\parallel \tq_{t}(\bx_{t + 1})) \leq \eta_{0}$. By solving problem \eqref{eq:dro objective}, if the desired $\tq_{t}(\bx_{t + 1}) = p_{\btheta}(\bx_{t + 1})$ is in $B_{D_{KL}}(q(\bx_{t + 1}), \eta_{0})$, then the conditional probability in \eqref{eq:dro objective} transfers $\bx_{t + 1}\sim p_{\btheta}(\bx_{t + 1})$ to target $\bx_{t}\sim q(\bx_{t})$ is learned, which mitigates the distribution mismatch problem. The theoretical clarification is in the following Proposition proved in Appendix \ref{app:proofs in sec:Distributional Robustness in DPM}, which indicates that small DRO loss \eqref{eq:dro objective} guarantees the quality of generated $\bx_{0}$. 
\begin{restatable}{proposition}{effectivenessofdro}\label{pro:effectiveness}
    If $L_{t}^{\mathrm{DRO}}(\btheta) \leq \eta_{0}$ in \eqref{eq:dro objective} for all $t$, and $D_{KL}(q(\bx_{T})\parallel p_{\btheta}(\bx_{T})) \leq \eta_{0}$, then $D_{KL}(q(x_{0})\parallel p_{\btheta}(\bx_{0})) \leq \eta_{0}$.
\end{restatable}
\par
Up to now, we do not know how to compute the DRO-based training objective \eqref{eq:dro objective} we derived. Fortunately, the following theorem corresponds \eqref{eq:dro objective} to a ``perturbed'' noise prediction problem similar to \eqref{eq:noise prediction}. The theorem is proved in Appendix \ref{app:proofs in sec:Distributional Robustness in DPM}. 
\begin{restatable}{theorem}{equivalencedro}\label{thm:equivalence}
    There exists $\bdelta_{t}$ depends on $\bx_{0}$ and $\beps_{t}$ makes \eqref{eq:eps dro} equivalent to problem \eqref{eq:dro objective}. 
    \begin{equation}\label{eq:eps dro}
    \small
        \min_{\btheta}\sum_{t=0}^{T - 1}\mE_{q(\bx_{0}),\beps_{t}}\left[\left\|\beps_{\btheta}(\sqrt{\baralpha_{t}}\bx_{0} + \sqrt{1 - \baralpha_{t}}\beps_{t} + \bdelta_{t}, t) - \beps_{t} - \frac{\bdelta_{t}}{\sqrt{1 - \baralpha_{t}}}\right\|^{2}\right].
\end{equation}
\end{restatable}
This theorem connects the proposed DRO problem \eqref{eq:dro objective} with noise prediction problem \eqref{eq:eps dro}. Naturally, we can solve \eqref{eq:eps dro}, if we know the exact $\bdelta_{t}$. Fortunately, we have the following proposition to characterize the range of $\bdelta_{t}$, and it is proved in Appendix \ref{app:proofs in sec:Distributional Robustness in DPM}. 
% In fact, the proposed objective \eqref{eq:eps dro} originate$s from the distributional robust optimization (DRO) \citep{yi2021improved}. However, minimizing \eqref{eq:eps dro} requires solving the inner maximization problem over a distributions set, which can be hard to be implemented in practice \citep{levy2020large}. To tackle this, we consider borrowing the training objective from adversarial training \citep{madry2018towards}, which is shown to be an approximation to DRO \citep{yi2021improved}. Before presenting the adversarial training objective, we need the following proposition. 
\begin{restatable}{proposition}{wtokl}\label{pro:ddpm adv}
    For $\eta > 0$ and $\bdelta_{t}$ in \eqref{eq:eps dro}, $\|\bdelta_{t}\|_{1} \leq \eta$ holds with probability at least $1 - \sqrt{2(1 - \baralpha_{t}) / \eta}$. 
\end{restatable}
The proposition indicates that for any $\bdelta_{t}$ depends on $\bx_{0}, \beps_{t}$ in \eqref{eq:eps dro}, it is likely in a small range (measured under any $\ell_{p}$-norm, since they can bound each other in Euclidean space). Thus, to resolve \eqref{eq:eps dro} (so that \eqref{eq:dro objective}), we propose to directly consider the following adversarial training \citep{madry2018towards} objective with the perturbation $\bdelta$ is taken over its possible range as proved in Proposition \ref{pro:ddpm adv}, which captures the input (instead of distribution) robustness of model $\beps_{\btheta}$. 
\begin{equation}\label{eq:dpm at}
    \small
    \min_{\btheta}\sum_{t=0}^{T - 1}\mE_{q(\bx_{0})}\left[\mE_{q(\bx_{t}\mid \bx_{0})}\left[\sup_{\bdelta: \|\bdelta\| \leq \eta}\left\|\beps_{\btheta}(\sqrt{\baralpha_{t}}\bx_{0} + \sqrt{1 - \baralpha_{t}}\beps_{t} + \bdelta) - \beps_{t} - \frac{\bdelta}{\sqrt{1 - \baralpha_{t}}}\right\|^{2}\right]\right].
\end{equation}
We present a fine-grained connection between \eqref{eq:dpm at} and classical AT in Appendix \ref{app:connection to AT}. Notably, our objective \eqref{eq:dpm at} is different from the ones in \citep{diffusion-ip}, whereas $\bdelta$ in it is a Gaussian, and $\beps_{\btheta}$ predicts $\beps_{t}$ instead of $\beps_{t} + \bdelta / \sqrt{1 - \baralpha_{t}}$ as ours. 
\par
To make it clear, we summarize the rationale from DRO objective \eqref{eq:dro objective} to AT our objective \eqref{eq:dpm at}. Since Theorem \ref{thm:equivalence} shows solving \eqref{eq:dro objective} is equivalent to \eqref{eq:eps dro}, which conducts noise prediction \eqref{eq:noise prediction} with a perturbation $\bdelta_{t}$ in a small range added (Proposition \ref{pro:ddpm adv}). Thus, we propose to minimize the maximal loss over the possible $\bdelta_{t}$, which is indeed our AT objective \eqref{eq:dpm at}.   

\section{Adversarial Training under Consistency Model}\label{sec:adversarial under consistency model}
Although the DPM generates high-quality target data $\bx_{0}$, the multi-step denoising process \eqref{eq:transition} requires numerous model evaluations, which can be computationally expensive. To resolve this, the diffusion-based consistency model (CM) is proposed in \citep{song2023consistency}. Consistency model $f_{\btheta}(\bx_{t}, t)$ transfers $\bx_{t}\sim q(\bx_{t})$ into a distribution that approximates the target $q(\bx_{0})$. $f_{\btheta}$ is optimized by the following consistency distillation (CD) loss \footnote{In practice,  \eqref{eq:cm objective} is updated under target model $f_{\btheta^{-}}(\Phi_{t}(\bx_{t + 1}), t)$ with exponential moving average (EMA) $\btheta^-$ under a stop gradient operation. \citep{song2023consistency} find that it greatly stabilizes the training process. In this section, we focus on the theory of consistency model and still use $\btheta$ in formulas.}
\begin{equation}\label{eq:cm objective}
    \small
    \min_{\btheta}\cL_{CD}(\btheta) = \sum_{t = 0}^{T - 1}\mE_{\bx_{t + 1}\sim q(\bx_{t + 1})}\left[d\left(f_{\btheta}(\Phi_{t}(\bx_{t + 1}), t), f_{\btheta}(\bx_{t + 1}, t + 1)\right)\right],
\end{equation}
where $\Phi_{t}(\bx_{t + 1})$ is a solution of a specific ordinary differential equation (ODE) (\eqref{eq:sde} in Appendix \ref{app:proofs of consistency model}) which is a deterministic function transfers $\bx_{t + 1}$ to $\bx_{t}$, i.e., $\Phi_{t}(\bx_{t + 1})\sim q(\bx_{t})$, and $d(\bx, \by)$ is a distance between $\bx$ and $\by$ e.g., $\ell_{1}, \ell_{2}$ distance. 
\begin{remark}
    In \citep{song2023consistency,luo2023latent}, the noisy data $\bx_{t}$ in \eqref{eq:cm objective} is described by an ODE  \eqref{eq:sde} in Appendix \ref{app:proofs of consistency model}. However, we use the discrete $\bx_{t}$ \eqref{eq:xt} here to unify the notations with Section \ref{sec:diffusion model as multi-step}. The two frameworks are mathematically equivalent as all $\bx_{t}$ in \eqref{eq:xt} located in the trajectory of ODE in \citep{song2023consistency}. More details of this claim refer to Appendix \ref{app:proofs of consistency model}.   
\end{remark}

Next, we use the following theorem to illustrate that solving problem \eqref{eq:cm objective} indeed creates $f_{\btheta}(\bx_{t}, t)$ with distribution close target $q(\bx_{0})$. The theorem is proved in Appendix \ref{app:proofs of consistency model}. 

\begin{restatable}{theorem}{expectedcdgap}\label{thm:expected cd gap}
    For $\cL_{CD}(\btheta)$ in \eqref{eq:cm objective} with $d(\cdot, \cdot)$ is $\ell_{2}$ distance, then $\sW_{1}(f_{\btheta}(\bx_{t}, t), \bx_{0}) \leq \sqrt{t\cL_{CD}(\btheta)}$ \footnote{Here $\sW_{1}(f_{\btheta}(\bx_{t}, t), \bx_{0})$ is the Wasserstein 1-distance between distributions of $f_{\btheta}(\bx_{t}, t)$ and $\bx_{0}$.}. 
\end{restatable}

Though solving problem \eqref{eq:cm objective} creates the desired CM $f_{\btheta}$, computing the exact $\Phi_{t}(\bx_{t + 1})$ involves solving an ODE as pointed out in Appendix \ref{app:proofs of consistency model}. Thus, in practice \citep{song2023consistency,luo2023latent}, the $\Phi_{t}(\bx_{t + 1})$ is approximated by a computable numerical estimation $\hat{\Phi}_{t}(\bx_{t + 1}, \beps_{\bphi})$ of it, e.g., Euler (\eqref{eq:estimated phi} in Appendix \ref{app:proof of cd upper bound}) or DDIM \citep{song2023consistency}, where $\beps_{\bphi}$ is a pretrained noise prediction model as in \eqref{eq:noise prediction}. Therefore, the practical training objective of \eqref{eq:cm objective} becomes
\begin{equation}\label{eq:cd objective}
    \small
        \min_{\btheta}\sum_{t = 0}^{T - 1}\hat{\cL}_{CD}(\btheta) = \mE_{\bx_{t + 1}\sim q(\bz_{t})}\left[d\left(f_{\btheta}(\hat{\Phi}_{t}(\bx_{t + 1}, \beps_{\bphi}), t), f_{\btheta}(\bx_{t + 1}, t + 1)\right)\right].
\end{equation}

In \eqref{eq:cd objective}, $\hat{\Phi}_{t}(\bx_{t + 1}, \beps_{\bphi})$ is an estimation to $\Phi_{t}(\bx_{t + 1})$, which causes an inaccurate training objective $\hat{\cL}_{CD}$ in \eqref{eq:cd objective}, compared with target $\cL_{CD}$ \eqref{eq:cm objective}. Thus, this results in the distribution mismatch problem in CM, as in DPM of Section \ref{sec:diffusion model as multi-step}. However, similar to Section \ref{sec:Distributional Robustness in DPM}, if we train $f_{\btheta}$ with robustness to the gap between $\hat{\Phi}_{t}(\bx_{t + 1}, \beps_{\bphi})$ and $\Phi_{t}(\bx_{t + 1})$, the distribution mismatch problem in CM is mitigated. 
\par
Technically, suppose $\Phi_{t}(\bx_{t + 1}) = \hat{\Phi}_{t}(\bx_{t + 1}, \beps_{\bphi}) +  \bdelta_{t}(\bx_{t + 1})$, we can consider minimizing the following adversarial training objective of CM, if $\|\bdelta_{t}(\bx_{t + 1})\| \leq \eta$ uniformly over $t$, for some constant $\eta$, so that the target $\Phi_{t}(\bx_{t + 1})$ is included in the maximal range as well.    
\begin{equation}\label{eq:cd at}
        \small
            \hat{\cL}_{CD}^{Adv}(\btheta) = \sum_{t = 0}^{T - 1}\mE_{\bx_{t + 1}}\left[\sup_{\|\bdelta\| \leq \eta}d\left(f_{\btheta}(\hat{\Phi}_{t}(\bx_{t + 1}, \beps_{\bphi}) + \bdelta, t), f_{\btheta}(\bx_{t + 1}, t + 1)\right)\right].
\end{equation}

By doing so, the learned model $f_{\btheta}$ can be robust to the perturbation brought by $\bdelta_{t}(\bx_{t + 1})$, so that results in a small $\cL_{CD}(\btheta)$, as well as the small $\sW_{1}(f_{\btheta}(\bx_{T}, T), \bx_{0})$ as proved in Theorem \ref{thm:expected cd gap}. Next, we use the following theorem to show that $\|\bdelta_{t}(\bx_{t + 1})\|$ is indeed small, and minimizing $\hat{\cL}_{CD}^{Adv}(\btheta)$ results in $f_{\btheta}(\bx_{T}, T)$ with distribution approximates $\bx_{0}$.  

 \begin{restatable}{theorem}{adversarialcd}\label{thm:cd upper bound}
    Under proper regularity conditions, for $0\leq t< T$, we have $\mE_{\bx_{t+1}}[\|\bdelta_{t}(\bx_{t + 1})\|] \leq o(1)$. On the other hand, it holds 
    \begin{equation}
        \small
        \sW_{1}(f_{\btheta}(\bx_{T}, T), \bx_{0}) \leq \sqrt{T\hat{\cL}_{CD}^{Adv}(\btheta) + o(1)}.
    \end{equation}
 \end{restatable}
The theorem is proved in Appendix \ref{app:proof of cd upper bound}, and it indicates that using the proposed adversarial training objective \eqref{eq:cd at} of CM indeed guarantees the learned CM transfers $\bx_{T}$ into data from $q(\bx_{0})$.  

\begin{algorithm}[ht!]
\caption{\textit{NovelSelect}}
\label{alg:novelselect}
\begin{algorithmic}[1]
\State \textbf{Input:} Data pool $\mathcal{X}^{all}$, data budget $n$
\State Initialize an empty dataset, $\mathcal{X} \gets \emptyset$
\While{$|\mathcal{X}| < n$}
    \State $x^{new} \gets \arg\max_{x \in \mathcal{X}^{all}} v(x)$
    \State $\mathcal{X} \gets \mathcal{X} \cup \{x^{new}\}$
    \State $\mathcal{X}^{all} \gets \mathcal{X}^{all} \setminus \{x^{new}\}$
\EndWhile
\State \textbf{return} $\mathcal{X}$
\end{algorithmic}
\end{algorithm}


\section{Experiments}
\subsection{Algorithms} 
In the standard adversarial training method like Projected Gradient Descent (PGD) \citep{madry2018towards}, the perturbation $\bdelta$ is constructed by implementing numbers (3-8) of gradient ascents to $\bdelta$ before updating the model, which slows down the training process. To resolve this, we adopt an efficient implementation \citep{freeat} in Algorithms \ref{alg:adv dpm}, \ref{alg:adv cm} to solve AT \eqref{eq:dpm at} and \eqref{eq:cd at} of DPM and CM, \emph{which has similar computational cost compared to standard training}, and significantly accelerate standard AT. Notably, unlike PGD, in Algorithms \ref{alg:adv dpm} and \ref{alg:adv cm}, every maximization step of perturbation $\bdelta$ follows an update step of the model $\btheta$. 
% Therefore, the AT used in our work has faster convergence rates compared to \citep{madry2018towards}. 
Thus, the efficient AT do not require further back propagations to construct adversarial samples as in PGD.   
We provide a comparison between our efficient AT and standard AT (PGD) with the same update iterations of model $\btheta$ in Appendix~\ref{app:at_ablation}. Moreover, we observe that efficient AT can yield comparable and even better performance than PGD while accelerating the training (2.6$\times$ speed-up), further verifying the benefits of our efficient AT. \footnote{For the experts in AT, they would recognize that the AT in Algorithms \ref{alg:adv dpm}, \ref{alg:adv cm} actually constructs the adversarial augmented data to improve the performance of the model \citep{freelb,jiang2020smart,yi2021improved}.} 
% Thus, we adopt the efficient adversarial training method to train models, ensuring that the updated iterations of models are synchronized with the actual training iterations. 
  
\subsection{Performance on DPM}
\label{sec:dpm_exp}
\paragraph{Settings.} The experiments are conducted on the unconditional generation on \texttt{CIFAR-10} 32$\times$32 \citep{krizhevsky2009learning} and the class-conditional generation on \texttt{ImageNet} $64\times64$ \citep{deng2009imagenet}. Our model and training pipelines in adopted from ADM \citep{dhariwal2021diffusion} paper, where ADM is a UNet-type network \citep{ronneberger2015u}, with strong performance in image generation under diffusion model.

To save training costs, our methods and baselines are fine-tuned from pretrained models, rather than training from scratch.
By doing so, we can efficiently assess the performance of methods, which is more practical for general scenarios.
% For \texttt{CIFAR-10}, the pretrained ADM is trained using a batch size of 128 for 250K iterations with a learning rate set to 1e-4.
% For \texttt{ImageNet}, the pretrained model is trained with a batch size of 1024 for 400K iterations, employing a learning rate of 3e-4.
We also explore training from scratch in Appendix \ref{app:convergence}, which also verifies the effectiveness of our method in this regime. 
During training, we fine-tune the pretrained models (details are in Appendix~\ref{app:hyper_dm}) with batch size 128 for 150K iterations under learning rate 1e-4 on \texttt{CIFAR-10}, and batch size 1024 for 50K iterations under learning rate of 3e-4 on \texttt{ImageNet}.
For the hyperparameters of AT, we select the adversarial learning rate $\alpha$ from $\left\{0.05, 0.1, 0.5\right\}$ and the adversarial step $K$ from $\left\{3, 5\right\}$. 
% We report the best results among the checkpoints for all methods.
More details are in Appendix~\ref{app:hyper_dm}.

We use the Frechet Inception Distance (FID)~\citep{heusel2017gans} to evaluate image quality. Unless otherwise specified, 50K images are sampled for evaluation.
Other results of metric Classification Accuracy Score (CAS)~\citep{ravuri2019cas}, sFID, Inception Score, Precision, and Recall are in Appendix~\ref{app:cas} and~\ref{app:more_metrics} for comprehensive evaluation.

\paragraph{Baselines.}
For experiments on diffusion models, we consider the following baselines.
1): the original pretrained model.
Compared with it, we verify whether the models are overfitting during fine-tuning.
2): continue fine-tuning the pretrained model, which is fine-tuned with the standard diffusion objective \eqref{eq:noise prediction}. 
Compared to it, we validate whether performance improvements come only from more training costs.
We also compare with the existing typical method to alleviate the DPM distribution mismatch, 3): ADM-IP~\citep{diffusion-ip}, which adds a Gaussian perturbation to the input data to simulate mismatch errors during the training process.
The last two fine-tuning baselines are based on \textbf{the same} pretrained model and hyperparameters as in the original literature. 

\begin{table}[!t]
    \caption{Sample quality measured by FID $\downarrow$ of different sampling methods of DPM under different NFEs on \texttt{CIFAR10} 32x32. All models are trained with same  iterations (computational costs).}
    \begin{subtable}[h]{0.48\textwidth}
        \centering
        \caption{IDDPM}
        \scalebox{0.85}{
        \small
        \begin{tabular}{l l l l l l}
        \toprule
         Methods $\backslash$ NFEs & 5 & 8 & 10 & 20 & 50 \\
        \midrule
         ADM (original) & 37.99 & 26.75 & 22.62 & 10.52 & 4.55\\
         \midrule
         ADM (finetune) & \bf36.91 & 26.06 & 21.94 & 10.58 & 4.34 \\
         ADM-IP & 47.57 & 26.91 & 20.09 & 7.81 & 3.42\\
         ADM-AT (Ours) & 37.15 & \bf23.59 & \bf15.88 & \bf6.60 & \bf3.34\\
        \bottomrule
       \end{tabular}
       }
    \end{subtable}
    \hfill
    \begin{subtable}[h]{0.48\textwidth}
        \centering
        \caption{DDIM}
        \scalebox{0.85}{
        \small
        \begin{tabular}{l l l l l l}
        \toprule
         Methods $\backslash$ NFEs & 5 & 8 & 10 & 20 & 50 \\
        \midrule
         ADM (original) & 34.28 & 14.34 & 11.66 & 7.00 & 4.68\\
         \midrule
         ADM (finetune) & 29.30 & 15.08 & 12.06 & 6.80 & 4.15 \\
         ADM-IP & 43.15 & 15.72 & 10.47 & 4.58 & 4.89\\
         ADM-AT (Ours) & \bf26.38 & \bf12.98 & \bf9.30 & \bf4.40 & \bf3.07\\
        \bottomrule
        \end{tabular}
        }
     \end{subtable}
    \begin{subtable}[h]{0.48\textwidth}
        \centering
        \caption{ES}
        \scalebox{0.85}{
        \small
        \begin{tabular}{l l l l l l}
        \toprule
         Methods $\backslash$ NFEs & 5 & 8 & 10 & 20 & 50 \\
        \midrule
         ADM (original) & 82.18 & 29.28 & 17.73 & 5.11 & 2.70\\
         \midrule
         ADM (finetune) & 63.46 & 24.80 & 17.03 & 5.19 & 2.52 \\
         ADM-IP & 91.10 & 31.44 & 18.72 & 5.19 & 2.89\\
         ADM-AT (Ours) & \bf41.07 & \bf21.62 & \bf14.68 & \bf4.36 & \bf2.48\\
        \bottomrule
        \end{tabular}
        }
     \end{subtable}
    \hfill
    \begin{subtable}[h]{0.48\textwidth}
        \centering
        \caption{DPM-Solver}
        \scalebox{0.85}{
        \small
        \begin{tabular}{l l l l l l}
        \toprule
         Methods $\backslash$ NFEs & 5 & 8 & 10 & 20 & 50 \\
        \midrule
         ADM (original) & 23.95 & 8.00 & 5.46 & 3.46 & 3.14\\
         \midrule
         ADM (finetune) & 22.98 & 7.61 & 5.29 & 3.41 & 3.12 \\
         ADM-IP & 43.83 & 6.70 & 6.80 & 9.78 & 10.91\\
         ADM-AT (Ours) & \bf18.40 & \bf5.84 & \bf4.81 & \bf3.28 & \bf3.01\\
        \bottomrule
        \end{tabular}
        }
     \end{subtable}
    \label{tab:C10}
\end{table}
\begin{table}[!t]
    % \label{tab:I64}
    \caption{Sample quality measured by FID $\downarrow$ of different sampling methods of DPM under different NFEs on \texttt{ImageNet} 64x64. All models are trained with the same iterations (computational costs).}
    \begin{subtable}[h]{0.48\textwidth}
        \centering
        \caption{IDDPM}
        \scalebox{0.85}{
        \small
        \begin{tabular}{l l l l l l}
        \toprule
         Methods $\backslash$ NFEs & 5 & 8 & 10 & 20 & 50 \\
        \midrule
         ADM (original) & 76.92 & 33.74 & 27.63 & 12.85 & 5.30\\
         \midrule
         ADM (finetune) & 78.87 & 33.99 & 27.82 & 12.80 & 5.26 \\
         ADM-IP & 67.12 & 29.96 & 22.60 & 8.66 & \bf3.83\\
         ADM-AT (Ours) & \bf45.65 & \bf23.79 & \bf19.18 & \bf8.28 & 4.01\\
        \bottomrule
       \end{tabular}
       }
    \end{subtable}
    \hfill
    \begin{subtable}[h]{0.48\textwidth}
        \centering
        \caption{DDIM}
        \scalebox{0.85}{
        \small
        \begin{tabular}{l l l l l l}
        \toprule
         Methods $\backslash$ NFEs & 5 & 8 & 10 & 20 & 50 \\
        \midrule
         ADM (original) & 60.07 & 20.10 & 14.97 & 8.41 & 5.65\\
         \midrule
         ADM (finetune) & 60.32 & 20.26 & 15.04 & 8.32 & 5.48 \\
         ADM-IP & 76.51 & 26.25 & 18.05 & 8.40 & 6.94\\
         ADM-AT (Ours) & \bf43.04 & \bf16.08 & \bf12.15 & \bf6.20 & \bf4.67\\
        \bottomrule
        \end{tabular}
        }
     \end{subtable}
    \begin{subtable}[h]{0.48\textwidth}
        \centering
        \caption{ES}
        \scalebox{0.85}{
        \small
        \begin{tabular}{l l l l l l}
        \toprule
         Methods $\backslash$ NFEs & 5 & 8 & 10 & 20 & 50 \\
        \midrule
         ADM (original) & 71.31 & 28.97 & 21.10 & 8.23 & 3.76\\
         \midrule
         ADM (finetune) & 72.30 & 29.24 & 21.58 & 8.25 & 3.64 \\
         ADM-IP & 88.37 & 33.91 & 23.32 & 7.80 & 3.54\\
         ADM-AT (Ours) & \bf43.95 & \bf19.57 & \bf14.12 & \bf6.16 & \bf3.45\\
        \bottomrule
        \end{tabular}
        }
     \end{subtable}
    \hfill
    \begin{subtable}[h]{0.48\textwidth}
        \centering
        \caption{DPM-Solver}
        \scalebox{0.85}{
        \small
        \begin{tabular}{l l l l l l}
        \toprule
         Methods $\backslash$ NFEs & 5 & 8 & 10 & 20 & 50 \\
        \midrule
         ADM (original) & 27.72 & 10.06 & 7.21 & 4.69 & 4.24\\
         \midrule
         ADM (finetune) & 27.82 & 9.97 & 7.22 & 4.64 & \bf4.15 \\
         ADM-IP & 32.43 & 9.94 & 8.87 & 9.16 & 9.68\\
         ADM-AT (Ours) & \bf17.36 & \bf6.55 & \bf5.78 & \bf4.56 & 4.34\\
        \bottomrule
        \end{tabular}
        }
     \end{subtable}
    \label{tab:I64}
\end{table}

\paragraph{Results.} To verify the effectiveness of our AT method, we conduct experiments with four diffusion samplers: IDDPM \citep{dhariwal2021diffusion}, DDIM \citep{song2020denoising}, DPM-Solver \citep{lu2022dpmsa}, and ES \citep{es} under various NFEs. The sampler choices contain the three most popular samplers: IDDPM, DDIM, DPM-Solver, and ES, a sampler that scales down the norm of predicted noise to mitigate the distribution mismatch from the perspective of sampling. The experimental results of \texttt{CIFAR-10} and \texttt{ImageNet} are shown in Table~\ref{tab:C10} and Table~\ref{tab:I64}, respectively. Results of more than hundreds of NFEs are shown in Appendix~\ref{app:more_nfes}
\par
%具体数字
As can be seen, the proposed AT for DPM significantly improves the performance of the original pretrained model and outperforms the other baselines (continue fine-tuning and ADM-IP) overall for all diffusion samplers and NFEs we take. Moreover, we have the following observarions. 
% For example, we improve FID from $22.98$ to $18.40$ with DPM-Solver at only 5 NFEs and improve FID from $4.15$ to $3.07$ with DDIM at 50 NFEs on \texttt{CIFAR-10}. 
% On \texttt{ImageNet}, we improve the FID from $27.72$ to $17.36$ with DPM-Solver for only 5 NFEs and improve the FID from $5.48$ to $4.67$ with DDIM for 50 NFEs. 
%少步数好
\par
1): Fewer (practically used) sampling steps (5,10) will result in larger mismatching errors, while our AT method demonstrates significant improvements in this regime across various samplers, e.g., AT improves FID 27.72 to 17.36 under 5 NFEs DPM-Solver on \texttt{ImageNet}. This suggests that our method is indeed effective in alleviating the distribution mismatch of DPM.
% 在各个sampler上都好
The results also indicate that our method consistently beats the baseline methods, regardless of stochastic (IDDPM) or deterministic samplers (DDIM, DPM-Solver). 
2): The ES sampler results show that our AT is orthogonal to the sampling-based method to mitigate the distribution mismatch problem and can be combined to further alleviate the issue. Notably, we further verify in Appendix \ref{app:convergence} that our methods will not slow the convergence unlike AT in classification \citep{madry2018towards}. 
%收敛性
% In addition, we report the convergence of our methods in Appendix~\ref{app:convergence}.
% Under the continual fine-tuning setting, we observe that our efficient AT method achieves better performance with the same training iterations.
We also perform ablation analysis of hyperparameters in our AT framework in Appendix~\ref{app:ablation_hyperparams}.

\subsection{Performance on Latent Consistency Models}

\paragraph{Settings.} 
We further evaluate the proposed AT for consistency models on text-to-image generation tasks with Latent Consistency Models \citep{luo2023latent} Stable Diffusion (SD) v1.5~\citep{Rombach_2022_CVPR} backbone, which generates 512$\times$512 images.
Both our AT and the original LCM training (baseline) are trained from scratch with the same hyperparameters (the IP method \citep{diffusion-ip} is not applied straightforwardly).
The training set is LAION-Aesthetics-6.5+~\citep{laion} with hyperparameters following \citet{song2023consistency,luo2023latent}. 
%The learning rate is set at 8e-6 and the EMA rate of the target model at 0.95, following \citet{song2023consistency} and \citet{luo2023latent}. 
%We set the range of the guidance scale $[w_{min}, w_{max}] = [3,5]$ during training and use $w = 4$ in sampling because it performs better in our preliminary experiments, which is similar to DMD \citep{yin2024onestep}.
We select the adversarial learning rate $\alpha$ from $\left\{0.02, 0.05\right\}$ and adversarial step $K$ from $\left\{2, 3\right\}$.
The models are trained with a batch size of 64 for 100K iterations. More details are shown in Appendix~\ref{app:hyper_lcm}.

Following~\citet{luo2023latent} and \citet{ chen2024pixart}, we evaluate models on MS-COCO 2014~\citep{lin2014microsoft} at a resolution of 512$\times$512 by randomly drawing 30K prompts from its validation set. Then, we report the FID between the generated samples under these prompts and the reference samples from the full validation set following~\citet{imagen}. 
% \revise{a little redundant?}
We also report CLIP scores~\citep{clipscore} to evaluate the text-image alignment by CLIP-ViT-B/16.

\paragraph{Results.}
\begin{table*}[t!]
\centering
\caption{Results of LCM on MS-COCO 2014 validation set at 512$\times$512 resolution in terms of FID $\downarrow$ and CLIP score $\uparrow$. All models are trained with the same setting (computational costs).}
\scalebox{0.8}{
\begin{tabular}{lcccccccc}
\toprule
    \multirow{2}{*}{Methods} & \multicolumn{4}{c}{FID $\downarrow$} & \multicolumn{4}{c}{CLIP Score $\uparrow$}\\ 
    & 1 step & 2 step & 4 step & 8 step & 1 step & 2 step & 4 step & 8 step\\
    \midrule
    LCM & 25.43 & 12.61 & 11.61 & 12.62 & 29.25 & 30.24 & 30.40 & 30.47 \\
    LCM-AT (Ours) & \bf{23.34} & \bf{11.28} & \bf{10.31} & \bf{10.68} & \bf{29.63} & \bf{30.43} & \bf{30.49} & \bf{30.53} \\
    \bottomrule
    \end{tabular}
}
\label{tab:lcm_res}
\end{table*}
The methods are evaluated under various sampling steps in Table~\ref{tab:lcm_res}, which shows that the LCM with AT consistently improves FID under various sampling steps. Besides, though the AT is not specified to improve text-image alignment, we observe that it has comparable or even better CLIP scores across various sampling steps, which shows that AT will not degenerate text-image alignment. 


\section{Conclusion}
In this paper, we novelly introduce efficient Adversarial Training (AT) in the training of DPM and CM to mitigate the issue of distribution mismatch between training and sampling. We conduct an in-depth analysis of the DPM training objective and systematically characterize the distribution mismatch problem. Furthermore, we prove that the training objective of CM similarly faces the distribution mismatch issue. We theoretically prove that DRO can mitigate the mismatch for both DPM and CM, which is equivalent to conducting AT. Experiments on image generation and text-to-image generation benchmarks verify the effectiveness of the proposed AT method in alleviating the distribution mismatch of DPM and CM.

\subsubsection*{Acknowledgments}
We thank anonymous reviewers for insightful feedback that helped improve the paper.
% Ming Liu and Mingyang Yi are the corresponding authors.
Zekun Wang, Ming Liu, Bing Qin are supported by the National Science Foundation of China (U22B2059, 62276083), the Human-Machine Integrated Consultation System for Cardiovascular Diseases (2023A003). 
They also appreciate the support from China Mobile Group Heilongjiang Co., Ltd.


\bibliography{reference}
\bibliographystyle{iclr2025_conference}

\newpage
\appendix
\subsection{Lloyd-Max Algorithm}
\label{subsec:Lloyd-Max}
For a given quantization bitwidth $B$ and an operand $\bm{X}$, the Lloyd-Max algorithm finds $2^B$ quantization levels $\{\hat{x}_i\}_{i=1}^{2^B}$ such that quantizing $\bm{X}$ by rounding each scalar in $\bm{X}$ to the nearest quantization level minimizes the quantization MSE. 

The algorithm starts with an initial guess of quantization levels and then iteratively computes quantization thresholds $\{\tau_i\}_{i=1}^{2^B-1}$ and updates quantization levels $\{\hat{x}_i\}_{i=1}^{2^B}$. Specifically, at iteration $n$, thresholds are set to the midpoints of the previous iteration's levels:
\begin{align*}
    \tau_i^{(n)}=\frac{\hat{x}_i^{(n-1)}+\hat{x}_{i+1}^{(n-1)}}2 \text{ for } i=1\ldots 2^B-1
\end{align*}
Subsequently, the quantization levels are re-computed as conditional means of the data regions defined by the new thresholds:
\begin{align*}
    \hat{x}_i^{(n)}=\mathbb{E}\left[ \bm{X} \big| \bm{X}\in [\tau_{i-1}^{(n)},\tau_i^{(n)}] \right] \text{ for } i=1\ldots 2^B
\end{align*}
where to satisfy boundary conditions we have $\tau_0=-\infty$ and $\tau_{2^B}=\infty$. The algorithm iterates the above steps until convergence.

Figure \ref{fig:lm_quant} compares the quantization levels of a $7$-bit floating point (E3M3) quantizer (left) to a $7$-bit Lloyd-Max quantizer (right) when quantizing a layer of weights from the GPT3-126M model at a per-tensor granularity. As shown, the Lloyd-Max quantizer achieves substantially lower quantization MSE. Further, Table \ref{tab:FP7_vs_LM7} shows the superior perplexity achieved by Lloyd-Max quantizers for bitwidths of $7$, $6$ and $5$. The difference between the quantizers is clear at 5 bits, where per-tensor FP quantization incurs a drastic and unacceptable increase in perplexity, while Lloyd-Max quantization incurs a much smaller increase. Nevertheless, we note that even the optimal Lloyd-Max quantizer incurs a notable ($\sim 1.5$) increase in perplexity due to the coarse granularity of quantization. 

\begin{figure}[h]
  \centering
  \includegraphics[width=0.7\linewidth]{sections/figures/LM7_FP7.pdf}
  \caption{\small Quantization levels and the corresponding quantization MSE of Floating Point (left) vs Lloyd-Max (right) Quantizers for a layer of weights in the GPT3-126M model.}
  \label{fig:lm_quant}
\end{figure}

\begin{table}[h]\scriptsize
\begin{center}
\caption{\label{tab:FP7_vs_LM7} \small Comparing perplexity (lower is better) achieved by floating point quantizers and Lloyd-Max quantizers on a GPT3-126M model for the Wikitext-103 dataset.}
\begin{tabular}{c|cc|c}
\hline
 \multirow{2}{*}{\textbf{Bitwidth}} & \multicolumn{2}{|c|}{\textbf{Floating-Point Quantizer}} & \textbf{Lloyd-Max Quantizer} \\
 & Best Format & Wikitext-103 Perplexity & Wikitext-103 Perplexity \\
\hline
7 & E3M3 & 18.32 & 18.27 \\
6 & E3M2 & 19.07 & 18.51 \\
5 & E4M0 & 43.89 & 19.71 \\
\hline
\end{tabular}
\end{center}
\end{table}

\subsection{Proof of Local Optimality of LO-BCQ}
\label{subsec:lobcq_opt_proof}
For a given block $\bm{b}_j$, the quantization MSE during LO-BCQ can be empirically evaluated as $\frac{1}{L_b}\lVert \bm{b}_j- \bm{\hat{b}}_j\rVert^2_2$ where $\bm{\hat{b}}_j$ is computed from equation (\ref{eq:clustered_quantization_definition}) as $C_{f(\bm{b}_j)}(\bm{b}_j)$. Further, for a given block cluster $\mathcal{B}_i$, we compute the quantization MSE as $\frac{1}{|\mathcal{B}_{i}|}\sum_{\bm{b} \in \mathcal{B}_{i}} \frac{1}{L_b}\lVert \bm{b}- C_i^{(n)}(\bm{b})\rVert^2_2$. Therefore, at the end of iteration $n$, we evaluate the overall quantization MSE $J^{(n)}$ for a given operand $\bm{X}$ composed of $N_c$ block clusters as:
\begin{align*}
    \label{eq:mse_iter_n}
    J^{(n)} = \frac{1}{N_c} \sum_{i=1}^{N_c} \frac{1}{|\mathcal{B}_{i}^{(n)}|}\sum_{\bm{v} \in \mathcal{B}_{i}^{(n)}} \frac{1}{L_b}\lVert \bm{b}- B_i^{(n)}(\bm{b})\rVert^2_2
\end{align*}

At the end of iteration $n$, the codebooks are updated from $\mathcal{C}^{(n-1)}$ to $\mathcal{C}^{(n)}$. However, the mapping of a given vector $\bm{b}_j$ to quantizers $\mathcal{C}^{(n)}$ remains as  $f^{(n)}(\bm{b}_j)$. At the next iteration, during the vector clustering step, $f^{(n+1)}(\bm{b}_j)$ finds new mapping of $\bm{b}_j$ to updated codebooks $\mathcal{C}^{(n)}$ such that the quantization MSE over the candidate codebooks is minimized. Therefore, we obtain the following result for $\bm{b}_j$:
\begin{align*}
\frac{1}{L_b}\lVert \bm{b}_j - C_{f^{(n+1)}(\bm{b}_j)}^{(n)}(\bm{b}_j)\rVert^2_2 \le \frac{1}{L_b}\lVert \bm{b}_j - C_{f^{(n)}(\bm{b}_j)}^{(n)}(\bm{b}_j)\rVert^2_2
\end{align*}

That is, quantizing $\bm{b}_j$ at the end of the block clustering step of iteration $n+1$ results in lower quantization MSE compared to quantizing at the end of iteration $n$. Since this is true for all $\bm{b} \in \bm{X}$, we assert the following:
\begin{equation}
\begin{split}
\label{eq:mse_ineq_1}
    \tilde{J}^{(n+1)} &= \frac{1}{N_c} \sum_{i=1}^{N_c} \frac{1}{|\mathcal{B}_{i}^{(n+1)}|}\sum_{\bm{b} \in \mathcal{B}_{i}^{(n+1)}} \frac{1}{L_b}\lVert \bm{b} - C_i^{(n)}(b)\rVert^2_2 \le J^{(n)}
\end{split}
\end{equation}
where $\tilde{J}^{(n+1)}$ is the the quantization MSE after the vector clustering step at iteration $n+1$.

Next, during the codebook update step (\ref{eq:quantizers_update}) at iteration $n+1$, the per-cluster codebooks $\mathcal{C}^{(n)}$ are updated to $\mathcal{C}^{(n+1)}$ by invoking the Lloyd-Max algorithm \citep{Lloyd}. We know that for any given value distribution, the Lloyd-Max algorithm minimizes the quantization MSE. Therefore, for a given vector cluster $\mathcal{B}_i$ we obtain the following result:

\begin{equation}
    \frac{1}{|\mathcal{B}_{i}^{(n+1)}|}\sum_{\bm{b} \in \mathcal{B}_{i}^{(n+1)}} \frac{1}{L_b}\lVert \bm{b}- C_i^{(n+1)}(\bm{b})\rVert^2_2 \le \frac{1}{|\mathcal{B}_{i}^{(n+1)}|}\sum_{\bm{b} \in \mathcal{B}_{i}^{(n+1)}} \frac{1}{L_b}\lVert \bm{b}- C_i^{(n)}(\bm{b})\rVert^2_2
\end{equation}

The above equation states that quantizing the given block cluster $\mathcal{B}_i$ after updating the associated codebook from $C_i^{(n)}$ to $C_i^{(n+1)}$ results in lower quantization MSE. Since this is true for all the block clusters, we derive the following result: 
\begin{equation}
\begin{split}
\label{eq:mse_ineq_2}
     J^{(n+1)} &= \frac{1}{N_c} \sum_{i=1}^{N_c} \frac{1}{|\mathcal{B}_{i}^{(n+1)}|}\sum_{\bm{b} \in \mathcal{B}_{i}^{(n+1)}} \frac{1}{L_b}\lVert \bm{b}- C_i^{(n+1)}(\bm{b})\rVert^2_2  \le \tilde{J}^{(n+1)}   
\end{split}
\end{equation}

Following (\ref{eq:mse_ineq_1}) and (\ref{eq:mse_ineq_2}), we find that the quantization MSE is non-increasing for each iteration, that is, $J^{(1)} \ge J^{(2)} \ge J^{(3)} \ge \ldots \ge J^{(M)}$ where $M$ is the maximum number of iterations. 
%Therefore, we can say that if the algorithm converges, then it must be that it has converged to a local minimum. 
\hfill $\blacksquare$


\begin{figure}
    \begin{center}
    \includegraphics[width=0.5\textwidth]{sections//figures/mse_vs_iter.pdf}
    \end{center}
    \caption{\small NMSE vs iterations during LO-BCQ compared to other block quantization proposals}
    \label{fig:nmse_vs_iter}
\end{figure}

Figure \ref{fig:nmse_vs_iter} shows the empirical convergence of LO-BCQ across several block lengths and number of codebooks. Also, the MSE achieved by LO-BCQ is compared to baselines such as MXFP and VSQ. As shown, LO-BCQ converges to a lower MSE than the baselines. Further, we achieve better convergence for larger number of codebooks ($N_c$) and for a smaller block length ($L_b$), both of which increase the bitwidth of BCQ (see Eq \ref{eq:bitwidth_bcq}).


\subsection{Additional Accuracy Results}
%Table \ref{tab:lobcq_config} lists the various LOBCQ configurations and their corresponding bitwidths.
\begin{table}
\setlength{\tabcolsep}{4.75pt}
\begin{center}
\caption{\label{tab:lobcq_config} Various LO-BCQ configurations and their bitwidths.}
\begin{tabular}{|c||c|c|c|c||c|c||c|} 
\hline
 & \multicolumn{4}{|c||}{$L_b=8$} & \multicolumn{2}{|c||}{$L_b=4$} & $L_b=2$ \\
 \hline
 \backslashbox{$L_A$\kern-1em}{\kern-1em$N_c$} & 2 & 4 & 8 & 16 & 2 & 4 & 2 \\
 \hline
 64 & 4.25 & 4.375 & 4.5 & 4.625 & 4.375 & 4.625 & 4.625\\
 \hline
 32 & 4.375 & 4.5 & 4.625& 4.75 & 4.5 & 4.75 & 4.75 \\
 \hline
 16 & 4.625 & 4.75& 4.875 & 5 & 4.75 & 5 & 5 \\
 \hline
\end{tabular}
\end{center}
\end{table}

%\subsection{Perplexity achieved by various LO-BCQ configurations on Wikitext-103 dataset}

\begin{table} \centering
\begin{tabular}{|c||c|c|c|c||c|c||c|} 
\hline
 $L_b \rightarrow$& \multicolumn{4}{c||}{8} & \multicolumn{2}{c||}{4} & 2\\
 \hline
 \backslashbox{$L_A$\kern-1em}{\kern-1em$N_c$} & 2 & 4 & 8 & 16 & 2 & 4 & 2  \\
 %$N_c \rightarrow$ & 2 & 4 & 8 & 16 & 2 & 4 & 2 \\
 \hline
 \hline
 \multicolumn{8}{c}{GPT3-1.3B (FP32 PPL = 9.98)} \\ 
 \hline
 \hline
 64 & 10.40 & 10.23 & 10.17 & 10.15 &  10.28 & 10.18 & 10.19 \\
 \hline
 32 & 10.25 & 10.20 & 10.15 & 10.12 &  10.23 & 10.17 & 10.17 \\
 \hline
 16 & 10.22 & 10.16 & 10.10 & 10.09 &  10.21 & 10.14 & 10.16 \\
 \hline
  \hline
 \multicolumn{8}{c}{GPT3-8B (FP32 PPL = 7.38)} \\ 
 \hline
 \hline
 64 & 7.61 & 7.52 & 7.48 &  7.47 &  7.55 &  7.49 & 7.50 \\
 \hline
 32 & 7.52 & 7.50 & 7.46 &  7.45 &  7.52 &  7.48 & 7.48  \\
 \hline
 16 & 7.51 & 7.48 & 7.44 &  7.44 &  7.51 &  7.49 & 7.47  \\
 \hline
\end{tabular}
\caption{\label{tab:ppl_gpt3_abalation} Wikitext-103 perplexity across GPT3-1.3B and 8B models.}
\end{table}

\begin{table} \centering
\begin{tabular}{|c||c|c|c|c||} 
\hline
 $L_b \rightarrow$& \multicolumn{4}{c||}{8}\\
 \hline
 \backslashbox{$L_A$\kern-1em}{\kern-1em$N_c$} & 2 & 4 & 8 & 16 \\
 %$N_c \rightarrow$ & 2 & 4 & 8 & 16 & 2 & 4 & 2 \\
 \hline
 \hline
 \multicolumn{5}{|c|}{Llama2-7B (FP32 PPL = 5.06)} \\ 
 \hline
 \hline
 64 & 5.31 & 5.26 & 5.19 & 5.18  \\
 \hline
 32 & 5.23 & 5.25 & 5.18 & 5.15  \\
 \hline
 16 & 5.23 & 5.19 & 5.16 & 5.14  \\
 \hline
 \multicolumn{5}{|c|}{Nemotron4-15B (FP32 PPL = 5.87)} \\ 
 \hline
 \hline
 64  & 6.3 & 6.20 & 6.13 & 6.08  \\
 \hline
 32  & 6.24 & 6.12 & 6.07 & 6.03  \\
 \hline
 16  & 6.12 & 6.14 & 6.04 & 6.02  \\
 \hline
 \multicolumn{5}{|c|}{Nemotron4-340B (FP32 PPL = 3.48)} \\ 
 \hline
 \hline
 64 & 3.67 & 3.62 & 3.60 & 3.59 \\
 \hline
 32 & 3.63 & 3.61 & 3.59 & 3.56 \\
 \hline
 16 & 3.61 & 3.58 & 3.57 & 3.55 \\
 \hline
\end{tabular}
\caption{\label{tab:ppl_llama7B_nemo15B} Wikitext-103 perplexity compared to FP32 baseline in Llama2-7B and Nemotron4-15B, 340B models}
\end{table}

%\subsection{Perplexity achieved by various LO-BCQ configurations on MMLU dataset}


\begin{table} \centering
\begin{tabular}{|c||c|c|c|c||c|c|c|c|} 
\hline
 $L_b \rightarrow$& \multicolumn{4}{c||}{8} & \multicolumn{4}{c||}{8}\\
 \hline
 \backslashbox{$L_A$\kern-1em}{\kern-1em$N_c$} & 2 & 4 & 8 & 16 & 2 & 4 & 8 & 16  \\
 %$N_c \rightarrow$ & 2 & 4 & 8 & 16 & 2 & 4 & 2 \\
 \hline
 \hline
 \multicolumn{5}{|c|}{Llama2-7B (FP32 Accuracy = 45.8\%)} & \multicolumn{4}{|c|}{Llama2-70B (FP32 Accuracy = 69.12\%)} \\ 
 \hline
 \hline
 64 & 43.9 & 43.4 & 43.9 & 44.9 & 68.07 & 68.27 & 68.17 & 68.75 \\
 \hline
 32 & 44.5 & 43.8 & 44.9 & 44.5 & 68.37 & 68.51 & 68.35 & 68.27  \\
 \hline
 16 & 43.9 & 42.7 & 44.9 & 45 & 68.12 & 68.77 & 68.31 & 68.59  \\
 \hline
 \hline
 \multicolumn{5}{|c|}{GPT3-22B (FP32 Accuracy = 38.75\%)} & \multicolumn{4}{|c|}{Nemotron4-15B (FP32 Accuracy = 64.3\%)} \\ 
 \hline
 \hline
 64 & 36.71 & 38.85 & 38.13 & 38.92 & 63.17 & 62.36 & 63.72 & 64.09 \\
 \hline
 32 & 37.95 & 38.69 & 39.45 & 38.34 & 64.05 & 62.30 & 63.8 & 64.33  \\
 \hline
 16 & 38.88 & 38.80 & 38.31 & 38.92 & 63.22 & 63.51 & 63.93 & 64.43  \\
 \hline
\end{tabular}
\caption{\label{tab:mmlu_abalation} Accuracy on MMLU dataset across GPT3-22B, Llama2-7B, 70B and Nemotron4-15B models.}
\end{table}


%\subsection{Perplexity achieved by various LO-BCQ configurations on LM evaluation harness}

\begin{table} \centering
\begin{tabular}{|c||c|c|c|c||c|c|c|c|} 
\hline
 $L_b \rightarrow$& \multicolumn{4}{c||}{8} & \multicolumn{4}{c||}{8}\\
 \hline
 \backslashbox{$L_A$\kern-1em}{\kern-1em$N_c$} & 2 & 4 & 8 & 16 & 2 & 4 & 8 & 16  \\
 %$N_c \rightarrow$ & 2 & 4 & 8 & 16 & 2 & 4 & 2 \\
 \hline
 \hline
 \multicolumn{5}{|c|}{Race (FP32 Accuracy = 37.51\%)} & \multicolumn{4}{|c|}{Boolq (FP32 Accuracy = 64.62\%)} \\ 
 \hline
 \hline
 64 & 36.94 & 37.13 & 36.27 & 37.13 & 63.73 & 62.26 & 63.49 & 63.36 \\
 \hline
 32 & 37.03 & 36.36 & 36.08 & 37.03 & 62.54 & 63.51 & 63.49 & 63.55  \\
 \hline
 16 & 37.03 & 37.03 & 36.46 & 37.03 & 61.1 & 63.79 & 63.58 & 63.33  \\
 \hline
 \hline
 \multicolumn{5}{|c|}{Winogrande (FP32 Accuracy = 58.01\%)} & \multicolumn{4}{|c|}{Piqa (FP32 Accuracy = 74.21\%)} \\ 
 \hline
 \hline
 64 & 58.17 & 57.22 & 57.85 & 58.33 & 73.01 & 73.07 & 73.07 & 72.80 \\
 \hline
 32 & 59.12 & 58.09 & 57.85 & 58.41 & 73.01 & 73.94 & 72.74 & 73.18  \\
 \hline
 16 & 57.93 & 58.88 & 57.93 & 58.56 & 73.94 & 72.80 & 73.01 & 73.94  \\
 \hline
\end{tabular}
\caption{\label{tab:mmlu_abalation} Accuracy on LM evaluation harness tasks on GPT3-1.3B model.}
\end{table}

\begin{table} \centering
\begin{tabular}{|c||c|c|c|c||c|c|c|c|} 
\hline
 $L_b \rightarrow$& \multicolumn{4}{c||}{8} & \multicolumn{4}{c||}{8}\\
 \hline
 \backslashbox{$L_A$\kern-1em}{\kern-1em$N_c$} & 2 & 4 & 8 & 16 & 2 & 4 & 8 & 16  \\
 %$N_c \rightarrow$ & 2 & 4 & 8 & 16 & 2 & 4 & 2 \\
 \hline
 \hline
 \multicolumn{5}{|c|}{Race (FP32 Accuracy = 41.34\%)} & \multicolumn{4}{|c|}{Boolq (FP32 Accuracy = 68.32\%)} \\ 
 \hline
 \hline
 64 & 40.48 & 40.10 & 39.43 & 39.90 & 69.20 & 68.41 & 69.45 & 68.56 \\
 \hline
 32 & 39.52 & 39.52 & 40.77 & 39.62 & 68.32 & 67.43 & 68.17 & 69.30  \\
 \hline
 16 & 39.81 & 39.71 & 39.90 & 40.38 & 68.10 & 66.33 & 69.51 & 69.42  \\
 \hline
 \hline
 \multicolumn{5}{|c|}{Winogrande (FP32 Accuracy = 67.88\%)} & \multicolumn{4}{|c|}{Piqa (FP32 Accuracy = 78.78\%)} \\ 
 \hline
 \hline
 64 & 66.85 & 66.61 & 67.72 & 67.88 & 77.31 & 77.42 & 77.75 & 77.64 \\
 \hline
 32 & 67.25 & 67.72 & 67.72 & 67.00 & 77.31 & 77.04 & 77.80 & 77.37  \\
 \hline
 16 & 68.11 & 68.90 & 67.88 & 67.48 & 77.37 & 78.13 & 78.13 & 77.69  \\
 \hline
\end{tabular}
\caption{\label{tab:mmlu_abalation} Accuracy on LM evaluation harness tasks on GPT3-8B model.}
\end{table}

\begin{table} \centering
\begin{tabular}{|c||c|c|c|c||c|c|c|c|} 
\hline
 $L_b \rightarrow$& \multicolumn{4}{c||}{8} & \multicolumn{4}{c||}{8}\\
 \hline
 \backslashbox{$L_A$\kern-1em}{\kern-1em$N_c$} & 2 & 4 & 8 & 16 & 2 & 4 & 8 & 16  \\
 %$N_c \rightarrow$ & 2 & 4 & 8 & 16 & 2 & 4 & 2 \\
 \hline
 \hline
 \multicolumn{5}{|c|}{Race (FP32 Accuracy = 40.67\%)} & \multicolumn{4}{|c|}{Boolq (FP32 Accuracy = 76.54\%)} \\ 
 \hline
 \hline
 64 & 40.48 & 40.10 & 39.43 & 39.90 & 75.41 & 75.11 & 77.09 & 75.66 \\
 \hline
 32 & 39.52 & 39.52 & 40.77 & 39.62 & 76.02 & 76.02 & 75.96 & 75.35  \\
 \hline
 16 & 39.81 & 39.71 & 39.90 & 40.38 & 75.05 & 73.82 & 75.72 & 76.09  \\
 \hline
 \hline
 \multicolumn{5}{|c|}{Winogrande (FP32 Accuracy = 70.64\%)} & \multicolumn{4}{|c|}{Piqa (FP32 Accuracy = 79.16\%)} \\ 
 \hline
 \hline
 64 & 69.14 & 70.17 & 70.17 & 70.56 & 78.24 & 79.00 & 78.62 & 78.73 \\
 \hline
 32 & 70.96 & 69.69 & 71.27 & 69.30 & 78.56 & 79.49 & 79.16 & 78.89  \\
 \hline
 16 & 71.03 & 69.53 & 69.69 & 70.40 & 78.13 & 79.16 & 79.00 & 79.00  \\
 \hline
\end{tabular}
\caption{\label{tab:mmlu_abalation} Accuracy on LM evaluation harness tasks on GPT3-22B model.}
\end{table}

\begin{table} \centering
\begin{tabular}{|c||c|c|c|c||c|c|c|c|} 
\hline
 $L_b \rightarrow$& \multicolumn{4}{c||}{8} & \multicolumn{4}{c||}{8}\\
 \hline
 \backslashbox{$L_A$\kern-1em}{\kern-1em$N_c$} & 2 & 4 & 8 & 16 & 2 & 4 & 8 & 16  \\
 %$N_c \rightarrow$ & 2 & 4 & 8 & 16 & 2 & 4 & 2 \\
 \hline
 \hline
 \multicolumn{5}{|c|}{Race (FP32 Accuracy = 44.4\%)} & \multicolumn{4}{|c|}{Boolq (FP32 Accuracy = 79.29\%)} \\ 
 \hline
 \hline
 64 & 42.49 & 42.51 & 42.58 & 43.45 & 77.58 & 77.37 & 77.43 & 78.1 \\
 \hline
 32 & 43.35 & 42.49 & 43.64 & 43.73 & 77.86 & 75.32 & 77.28 & 77.86  \\
 \hline
 16 & 44.21 & 44.21 & 43.64 & 42.97 & 78.65 & 77 & 76.94 & 77.98  \\
 \hline
 \hline
 \multicolumn{5}{|c|}{Winogrande (FP32 Accuracy = 69.38\%)} & \multicolumn{4}{|c|}{Piqa (FP32 Accuracy = 78.07\%)} \\ 
 \hline
 \hline
 64 & 68.9 & 68.43 & 69.77 & 68.19 & 77.09 & 76.82 & 77.09 & 77.86 \\
 \hline
 32 & 69.38 & 68.51 & 68.82 & 68.90 & 78.07 & 76.71 & 78.07 & 77.86  \\
 \hline
 16 & 69.53 & 67.09 & 69.38 & 68.90 & 77.37 & 77.8 & 77.91 & 77.69  \\
 \hline
\end{tabular}
\caption{\label{tab:mmlu_abalation} Accuracy on LM evaluation harness tasks on Llama2-7B model.}
\end{table}

\begin{table} \centering
\begin{tabular}{|c||c|c|c|c||c|c|c|c|} 
\hline
 $L_b \rightarrow$& \multicolumn{4}{c||}{8} & \multicolumn{4}{c||}{8}\\
 \hline
 \backslashbox{$L_A$\kern-1em}{\kern-1em$N_c$} & 2 & 4 & 8 & 16 & 2 & 4 & 8 & 16  \\
 %$N_c \rightarrow$ & 2 & 4 & 8 & 16 & 2 & 4 & 2 \\
 \hline
 \hline
 \multicolumn{5}{|c|}{Race (FP32 Accuracy = 48.8\%)} & \multicolumn{4}{|c|}{Boolq (FP32 Accuracy = 85.23\%)} \\ 
 \hline
 \hline
 64 & 49.00 & 49.00 & 49.28 & 48.71 & 82.82 & 84.28 & 84.03 & 84.25 \\
 \hline
 32 & 49.57 & 48.52 & 48.33 & 49.28 & 83.85 & 84.46 & 84.31 & 84.93  \\
 \hline
 16 & 49.85 & 49.09 & 49.28 & 48.99 & 85.11 & 84.46 & 84.61 & 83.94  \\
 \hline
 \hline
 \multicolumn{5}{|c|}{Winogrande (FP32 Accuracy = 79.95\%)} & \multicolumn{4}{|c|}{Piqa (FP32 Accuracy = 81.56\%)} \\ 
 \hline
 \hline
 64 & 78.77 & 78.45 & 78.37 & 79.16 & 81.45 & 80.69 & 81.45 & 81.5 \\
 \hline
 32 & 78.45 & 79.01 & 78.69 & 80.66 & 81.56 & 80.58 & 81.18 & 81.34  \\
 \hline
 16 & 79.95 & 79.56 & 79.79 & 79.72 & 81.28 & 81.66 & 81.28 & 80.96  \\
 \hline
\end{tabular}
\caption{\label{tab:mmlu_abalation} Accuracy on LM evaluation harness tasks on Llama2-70B model.}
\end{table}

%\section{MSE Studies}
%\textcolor{red}{TODO}


\subsection{Number Formats and Quantization Method}
\label{subsec:numFormats_quantMethod}
\subsubsection{Integer Format}
An $n$-bit signed integer (INT) is typically represented with a 2s-complement format \citep{yao2022zeroquant,xiao2023smoothquant,dai2021vsq}, where the most significant bit denotes the sign.

\subsubsection{Floating Point Format}
An $n$-bit signed floating point (FP) number $x$ comprises of a 1-bit sign ($x_{\mathrm{sign}}$), $B_m$-bit mantissa ($x_{\mathrm{mant}}$) and $B_e$-bit exponent ($x_{\mathrm{exp}}$) such that $B_m+B_e=n-1$. The associated constant exponent bias ($E_{\mathrm{bias}}$) is computed as $(2^{{B_e}-1}-1)$. We denote this format as $E_{B_e}M_{B_m}$.  

\subsubsection{Quantization Scheme}
\label{subsec:quant_method}
A quantization scheme dictates how a given unquantized tensor is converted to its quantized representation. We consider FP formats for the purpose of illustration. Given an unquantized tensor $\bm{X}$ and an FP format $E_{B_e}M_{B_m}$, we first, we compute the quantization scale factor $s_X$ that maps the maximum absolute value of $\bm{X}$ to the maximum quantization level of the $E_{B_e}M_{B_m}$ format as follows:
\begin{align}
\label{eq:sf}
    s_X = \frac{\mathrm{max}(|\bm{X}|)}{\mathrm{max}(E_{B_e}M_{B_m})}
\end{align}
In the above equation, $|\cdot|$ denotes the absolute value function.

Next, we scale $\bm{X}$ by $s_X$ and quantize it to $\hat{\bm{X}}$ by rounding it to the nearest quantization level of $E_{B_e}M_{B_m}$ as:

\begin{align}
\label{eq:tensor_quant}
    \hat{\bm{X}} = \text{round-to-nearest}\left(\frac{\bm{X}}{s_X}, E_{B_e}M_{B_m}\right)
\end{align}

We perform dynamic max-scaled quantization \citep{wu2020integer}, where the scale factor $s$ for activations is dynamically computed during runtime.

\subsection{Vector Scaled Quantization}
\begin{wrapfigure}{r}{0.35\linewidth}
  \centering
  \includegraphics[width=\linewidth]{sections/figures/vsquant.jpg}
  \caption{\small Vectorwise decomposition for per-vector scaled quantization (VSQ \citep{dai2021vsq}).}
  \label{fig:vsquant}
\end{wrapfigure}
During VSQ \citep{dai2021vsq}, the operand tensors are decomposed into 1D vectors in a hardware friendly manner as shown in Figure \ref{fig:vsquant}. Since the decomposed tensors are used as operands in matrix multiplications during inference, it is beneficial to perform this decomposition along the reduction dimension of the multiplication. The vectorwise quantization is performed similar to tensorwise quantization described in Equations \ref{eq:sf} and \ref{eq:tensor_quant}, where a scale factor $s_v$ is required for each vector $\bm{v}$ that maps the maximum absolute value of that vector to the maximum quantization level. While smaller vector lengths can lead to larger accuracy gains, the associated memory and computational overheads due to the per-vector scale factors increases. To alleviate these overheads, VSQ \citep{dai2021vsq} proposed a second level quantization of the per-vector scale factors to unsigned integers, while MX \citep{rouhani2023shared} quantizes them to integer powers of 2 (denoted as $2^{INT}$).

\subsubsection{MX Format}
The MX format proposed in \citep{rouhani2023microscaling} introduces the concept of sub-block shifting. For every two scalar elements of $b$-bits each, there is a shared exponent bit. The value of this exponent bit is determined through an empirical analysis that targets minimizing quantization MSE. We note that the FP format $E_{1}M_{b}$ is strictly better than MX from an accuracy perspective since it allocates a dedicated exponent bit to each scalar as opposed to sharing it across two scalars. Therefore, we conservatively bound the accuracy of a $b+2$-bit signed MX format with that of a $E_{1}M_{b}$ format in our comparisons. For instance, we use E1M2 format as a proxy for MX4.

\begin{figure}
    \centering
    \includegraphics[width=1\linewidth]{sections//figures/BlockFormats.pdf}
    \caption{\small Comparing LO-BCQ to MX format.}
    \label{fig:block_formats}
\end{figure}

Figure \ref{fig:block_formats} compares our $4$-bit LO-BCQ block format to MX \citep{rouhani2023microscaling}. As shown, both LO-BCQ and MX decompose a given operand tensor into block arrays and each block array into blocks. Similar to MX, we find that per-block quantization ($L_b < L_A$) leads to better accuracy due to increased flexibility. While MX achieves this through per-block $1$-bit micro-scales, we associate a dedicated codebook to each block through a per-block codebook selector. Further, MX quantizes the per-block array scale-factor to E8M0 format without per-tensor scaling. In contrast during LO-BCQ, we find that per-tensor scaling combined with quantization of per-block array scale-factor to E4M3 format results in superior inference accuracy across models. 

% \section{Appendix}
% You may include other additional sections here.


\end{document}
