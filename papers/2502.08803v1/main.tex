\documentclass{article}
\usepackage{spconf,amsmath,graphicx}

\def\x{{\mathbf x}}
\def\L{{\cal L}}

%
\makeatletter
\let\NAT@parse\undefined
\makeatother

\usepackage{graphics,mathptmx,times,amssymb}
\usepackage{amsmath,epsfig,xcolor,graphicx,url,xspace,booktabs}
\usepackage{url,multirow,bm,bbm,breqn,rotating,color,colortbl,subcaption,pifont,colortbl}

\let\labelindent\relax
\usepackage{enumitem}

\definecolor{Gray}{gray}{0.86}
\newcommand{\cmark}{\ding{51}}
\newcommand{\xmark}{\ding{55}}

\usepackage{hyperref}
\definecolor{citecolor}{HTML}{0071BC}
\definecolor{linkcolor}{HTML}{ED1C24}
\definecolor{Gray}{gray}{0.86}


\hypersetup{colorlinks=True, citecolor=citecolor, urlcolor=magenta}

\title{Deep EEG Super-resolution: Upsampling EEG Spatial Resolution with Generative Adversarial Networks}

\name{Isaac Corley\sthanks{Corresponding author \tt\footnotesize 
\href{mailto:isaac.corley@utsa.edu}{isaac.corley@utsa.edu}}, Yufei Huang}
\address{University of Texas at San Antonio}


\begin{document}

\maketitle
\thispagestyle{empty}
\pagestyle{empty}


\begin{abstract}
\textbf{Electroencephalography (EEG) activity contains a wealth of information about what is happening within the human brain. Recording more of this data has the potential to unlock endless future applications. However, the cost of EEG hardware is increasingly expensive based upon the number of EEG channels being recorded simultaneously. We combat this problem in this paper by proposing a novel deep EEG super-resolution (SR) approach based on Generative Adversarial Networks (GANs). This approach can produce high spatial resolution EEG data from low resolution samples, by generating channel-wise upsampled data to effectively interpolate numerous missing channels, thus reducing the need for expensive EEG equipment. We tested the performance using an EEG dataset from a mental imagery task. Our proposed GAN model provided 10^4 fold and 10^2 fold reduction in mean-squared error (MSE) and mean-absolute error (MAE), respectively, over the baseline bicubic interpolation method. We further validate our method by training a classifier on the original classification task, which displayed minimal loss in accuracy while using the super-resolved data. The proposed SR EEG by GAN is a promising approach to improve the spatial resolution of low density EEG headsets.}
\end{abstract}


\section{Introduction}
Electroencephalography (EEG) is a noninvasive neuroimaging modality widely used for clinical diagnosis of seizures and cognitive neuroscience. It has gained increasing popularity in recent years as a neurofeedback device in   brain-computer interface (BCI) systems with applications including typing interface for locked-in patients, neurorehabilitation~\cite{wolpaw1994multichannel}, brain-controlled drone~\cite{merino2017asynchronous}, and detection of driver fatigue~\cite{hajinoroozi2016eeg}. However, a primary bottleneck to EEG-based BCI research is the cost of hardware. Ideally, EEG devices with high-density channels are preferred in order to obtain recordings of brain activities with high spatial resolution underlying different cognitive events. However, the cost of EEG hardware increases exponentially with channels, with a majority of commercial EEG devices with 32 channels costing more than \$20k. For academia and industry, this can greatly hinder the quality of products and research being performed. This also results in poor generality for EEG-based algorithms because prediction algorithms developed using one headset cannot be used for a headset of different channels even if both are used to measure the same cognitive events. EEG channel interpolation has been proposed in many research efforts~\cite{petrichella2016channel, courellis2016eeg} to recreate missing or defective sensor channels. Although they showed favorable improvement for single selective channel interpolation, research for interpolating many channels at a global scale is scarce.  

Deep learning and its applications have recently become highly popular and rightly so, due to their superior ability to learn representations of complex data~\cite{lecun2015deep}. One of its applications is image super-resolution (SR), where deep learning-based pixel interpolation was developed~\cite{dong2015image} to generate high-resolution (HR) copies from low-resolution (LR) images. The state-of-the-art SR performance is obtained by the new game theoretic deep generative model of Generative Adversarial Networks (GANs)~\cite{goodfellow2014generative}, which established the first framework to achieve photo-realistic natural images for a 4x upscaling factor~\cite{ledig2017photo}. 

Inspired by the similarity between global EEG channel interpolation and image SR, along with the superb image SR performance achieved by GANs, we propose deep EEG super-resolution, a novel framework for generating high spatial resolution EEG data from low resolution recordings using GANs. We compare our work to a baseline of bicubic interpolation and then additionally verify performance by training a classifier using the SR EEG data for the classification purpose of the original EEG dataset. 

\section{Data}
\subsection{Berlin BCI Competition III, Dataset V}
Dataset V of the Berlin Brain Computer Interface Competition III provided by the IDIAP Research Institute~\cite{millan2004need} was used. The dataset consists of 32 EEG channels recorded at 512 Hz for 3 individual subjects, located at the standard positions of the International 10-20 system \textit{(Fp1, AF3, F7, F3, FC1, FC5, T7, C3, CP1, CP5, P7, P3, Pz, PO3, O1, Oz, O2, PO4, P4, P8, CP6, CP2, C4, T8, FC6, FC2, F4, F8, AF4, Fp2, Fz, Cz)}. The dataset was initially used for a mental imagery multiclass classification competition of three labeled tasks. The data is provided with train and test sets for each subject; however, only the train set was used due to the test set not being provided with labels. The train set contained a total of 1.1M samples. 

\subsection{Preprocessing}
Following the epoch extraction procedure in~\cite{millan2004need}, the raw data was separated into epochs of length 512 samples using a moving window with a stride of 32 samples. This resulted in epochs of size (32 channels by 512 samples). The epoched data was then split into train, validation, and test sets for holdout validation using a 75/20/5 percentage split criterion.  

The initial Dataset V classifiers used precomputed features, which consisted of the estimated power spectral density (PSD) of each epoch in the band 8-30 Hz with a frequency resolution of 2 Hz for the 8 centro-parietal channels \textit{(C3, Cz, C4, CP1, CP2, P3, Pz, P4)}, which resulted in a 96-dimensional vector (8 channels, 12 frequency components). 

For use with super-resolution models, all datasets were reshaped to epochs of size (32 channels by 64 samples). To produce the low-resolution (LR) data, the epochs were downsampled by channel based upon the scale factor used, e.g., downsampling 32 channels by a scale factor of 2 would remove every other channel, leaving 16 channels of LR data. The removed channels are then used as the HR data. The input data and its corresponding ground truth were then standard-normalized to a mean $\mu=0$ and standard deviation $\sigma=1$ using the mean and standard deviation of the input channel training set. This was repeated using the same statistics for normalizing the validation and test data. 

\section{Methods}
\section{Model}
\label{sec:model}
Let $[N] = \{1, 2, \dots, N \}$ be a set of $N$ agents.
We examine an environment in which a system interacts with the agents over $T$ rounds.
Every round $t\leq T$ comprises $N$ \emph{sessions}, each session represents an encounter of the system with exactly one agent, and each agent interacts exactly once with the system every round.
I.e., in each round $t$ the agents arrive sequentially. 


\paragraph{Arrival order} The \emph{arrival order} of round $t$, denoted as $\ordv_t=(\ord_t(1),\dots, \ord_t(N))$, is an element from set of all permutations of $[N]$. Each entry $q$ in $\ordv_t$ is the index of the agent that arrives in the $q^{\text{th}}$ session of round $t$.
For example, if $\ord_t(1) = 2$, then agent $2$ arrives in the first session of round $t$.
Correspondingly, $\ord_t^{-1}(i)=q$ implies that agent $i$ arrives in the $q^{\text{th}}$ session of round $t$. 

As we demonstrate later, the arrival order has an immediate impact on agent rewards. We call the mechanism by which the arrival order is set \emph{arrival function} and denote it by $\ordname$. Throughout the paper, we consider several arrival functions such as the \emph{uniform arrival} function, denoted by $\uniord$, and the \emph{nudged arrival} $\sugord$; we introduce those formally in Sections~\ref{sec:uniform} and~\ref{sec:nudge}, respectively.

%We elaborate more on this concept in Section~\ref{sec: arrival}.


\paragraph{Arms} We consider a set of $K \geq 2$ arms, $A = \{a_1, \ldots, a_K\}$. The reward of arm $a_i$ in round $t$ is a random variable $X_i^t \sim \mathcal{D}^t_i$, where the rewards $(X_i^t)_{i,t}$ are mutually independent and bounded within the interval $[0,1]$. The reward distribution $\mathcal{D}^t_i$ of arm $a_i$, $i\in [K]$ at round $t\in T$ is assumed to be non-stationary but independent across arms and rounds. We denote the realized reward of arm $a_i$ in round $t$ by $x_i^t$. We assume \emph{reward consistency}, meaning that rewards may vary between rounds but remain constant within the sessions of a single round. Specifically, if an arm $a_i$ is selected multiple times during round~$t$, each selection yields the same reward $x_i^t$, where the superscript $t$ indicates its dependence on the round rather than the session. This consistency enables the system to leverage information obtained from earlier sessions to make more informed decisions in later sessions within the same round. We provide further details on this principle in Subsection~\ref{subsec:information}.


\paragraph{Algorithms} An algorithm is a mapping from histories to actions. We typically expect algorithms to maximize some aggregated agent metric like social welfare. Let $\mathcal H^{t,q}$ denote the information observed during all sessions of rounds $1$ to $t-1$ and sessions $1$ to $q-1$ in round $t$.  The history $\mathcal H^{t,q}$ is an element from $(A \times [0,1])^{(t-1) \cdot N +q-1}$, consisting of pairs of the form (pulled arm, realized reward). Notice that we restrict our attention to \emph{anonymous} algorithms, i.e., algorithms that do not distinguish between agents based on their identities. Instead, they only respond to the history of arms pulled and rewards observed, without conditioning on which specific agent performed each action.
%In the most general case, algorithms make decisions at session $q$ of round $t$  based on the entire history $\mathcal H^{t,q}$ and the index of the arriving agent $\ord_t(q)$. %Furthermore, we sometimes assume that algorithms have Bayesian information, i.e., algorithms are aware of the distributions $(\mathcal D_i)^K_{i=1}$. 
Furthermore, we sometimes assume that algorithms have Bayesian information, meaning they are aware of the reward distributions $(\mathcal{D}^t_i)_{i,t}$. If such an assumption is required to derive a result, we make it explicit. %Otherwise, we do not assume any additional knowledge about the algorithm’s information. %This distinction allows us to analyze both general algorithms without prior distributional knowledge and specialized algorithms that leverage Bayesian information.


\paragraph{Rewards} Let $\rt{i}$ denote the reward received by agent $i \in [N]$ at round $t$, and let $\Rt{i}$ denote her cumulative reward at the end of round $t$, i.e., $\Rt{i} = \sum_{\tau=1}^{t}{r^{\tau}_{i}}$. We further denote the \emph{social welfare} as the sum of the rewards all agents receive after $T$ rounds. Formally, $\sw=\sum^{N}_{i=1}{R^T_i}$. We emphasize that social welfare is independent of the arrival order. 


\paragraph{Envy}
We denote by $\adift{i}{j}$ the reward discrepancy of agents $i$ and $j$ in round $t$; namely, $\adift{i}{j}= \rt{i} - \rt{j}$. %We call this term \omer{name??} reward discrepancy in round $t$. 
The (cumulative) \emph{envy} between two agents at round $t$ is the difference in their cumulative rewards. Formally, $\env_{i,j}^t= \Rt{i} - \Rt{j}$ is the envy after $t$ rounds between agent $i$ and $j$. We can also formulate envy as the sum of reward discrepancies, $\env_{i,j}^t= \sum^{t}_{\tau=1}{\adif{i}{j}^\tau}$. Notice that envy is a signed quantity and can be either positive or negative. Specifically, if $\env_{i,j}^t < 0$, we say that agent $i$ envies agent $j$, and if $\env_{i,j}^t > 0$, agent $j$ envies agent $i$. The main goal of this paper is to investigate the behavior of the \emph{maximal envy}, defined as
\[
\env^t = \max_{i,j \in [N]} \env^t_{i,j}.
\]
For clarity, the term \emph{envy} will refer to the maximal envy.\footnote{ We address alternative definitions of envy in Section~\ref{sec:discussion}.} % Envy can also be defined in alternative ways, such as by averaging pairwise envy across all agents. We address average envy in Section~\ref{sec:avg_envy}.}
Note that $\env_{i,j}^t$ are random variables that depend on the decision-making algorithm, realized rewards, and the arrival order, and therefore, so is $\env^t$. If a result we obtain regarding envy depends on the arrival order $\ordname$, we write $\env^t(\ordname)$. Similarly, to ease notation, if $\ordname$ can be understood from the context, it is omitted.



\paragraph{Further Notation} We use the subscript $(q)$ to address elements of the $q^{\text{th}}$ session, for $q\in [N]$.
That is, we use the notation $\rt{(q)}$ to denote the reward granted to the agent that arrives in the $q^{\text{th}}$ session of round $t$ and $\Rt{(q)}$ to denote her cumulative reward. %Additionally, we introduce the notation $\at{(q)}$ to denote the arm pulled in that session.
Correspondingly, $\sdift{q}{w} = \rt{(q)} - \rt{(w)}$ is the reward discrepancy of the agents arriving in the $q^{\text{th}}$ and $w^{\text{th}}$ sessions of round $t$, respectively. 
To distinguish agents, arms, sessions and rounds, we use the letters $i,j$ to mark agents and arms, $q,w$ for sessions, and $t,\tau$ for rounds.


\subsection{Example}
\label{sec: example}
To illustrate the proposed setting and notation, we present the following example, which serves as a running example throughout the paper.

\begin{table}[t]
\centering
\begin{tabular}{|c|c|c|c|}
\hline
$t$ (round) & $\ordv_t$ (arrival order) & $x_1^t$ & $x_2^t$ \\ \hline
1           & 2, 1                     & 0.6     & 0.92    \\ \hline
2           & 1, 2                     & 0.48    & 0.1     \\ \hline
3           & 2, 1                     & 0.15    & 0.8     \\ \hline
\end{tabular}
\caption{
    Data for Example~\ref{example 1}.
}
\label{tbl: example}
\end{table}

\begin{algorithm}[t]
\caption{Algorithm for Example~\ref{example 1}}
\label{alguni}
\DontPrintSemicolon 
\For{round $t = 1$ to $T$}{
    pull $a_{1}$ in the first session\label{alguniexample: first}\\
    \lIf{$x^t_1 \geq \frac{1}{2}$}{pull $a_{1}$ again in second session \label{alguniexample: pulling a again}}
    \lElse{pull $a_{2}$ in second session \label{alguniexample: sopt else}}
}
\end{algorithm}


\begin{example}\label{example 1}
We consider $K=2$ uniform arms, $X_1,X_2 \sim \uni{0,1}$, and $N=2$ for some $T\geq 3$. We shall assume arm decision are made by Algorithm~\ref{alguni}: In the first session, the algorithm pulls $a_1$; if it yields a reward greater than $\nicefrac{1}{2}$, the algorithm pulls it again in the second session (the ``if'' clause). Otherwise, it pulls $a_2$.



We further assume that the arrival orders and rewards are as specified in Table~\ref{tbl: example}. Specifically, agent 2 arrives in the first session of round $t=1$, and pulling arm $a_2$ in this round would yield a reward of $x^1_2 = 0.92$. Importantly, \emph{this information is not available to the decision-making algorithm in advance} and is only revealed when or if the corresponding arms are pulled.




In the first round, $\boldsymbol{\eta}^1 = \left(2,1\right)$; thus, agent 2 arrives in the first session.
The algorithm pulls arm $a_1$, which means, $a^1_{(1)} = a_1$, and the agent receives $r_{2}^1=r_{(1)}^1=x_1^1=0.6$.
Later that round, in the second session, agent 1 arrives, and the algorithm pulls the same arm again since $x^1_1 = 0.6 \geq \nicefrac{1}{2}$ due to the ``if'' clause.
I.e., $a^1_{(2)} = a_1$ and $r_{1}^1 = r_{(2)}^1 = x_1^1 = 0.6$.
Even though the realized reward of arm $a_2$ in that round is higher ($0.92$), the algorithm is not aware of that value.
At the end of the first round, $R^1_1 = R^1_{(2)} = R^1_2 = R^1_{(1)} = 0.6$. The reward discrepancy is thus $\adif{1}{2}^1 = \adif{2}{1}^1= \sdif{2}{1}^1 = 0.6 - 0.6 =0$. 



In the second round, agent 1 arrives first, followed by agent 2.
Firstly, the algorithm pulls arm $a_1$ and agent 1 receives a reward of $r_{1}^2 = r_{(1)}^2 = x_1^2 = 0.48$.
Because the reward is lower than $\nicefrac{1}{2}$, in the second session the algorithm pulls the other arm ($a^2_{(2)} = a_2$), granting agent 2 a reward of $r_{2}^2 = r_{(2)}^2 = x_2^2 = 0.1$.
At the end of the second round, $R^2_1 = R^2_{(1)} = 0.6 + 0.48 = 1.08$ and $R^2_2 = R^2_{(2)} = 0.6 + 0.1 = 0.7$. Furthermore, $\sdif{2}{1}^2 = \adif{2}{1}^2 = r^2_{2} - r^2_{1} = 0.1 - 0.48 = -0.38$.

In the third and final round, agent 2 arrives first again, and receives a reward  of $0.15$ from $a_1$. When agent 1 arrives in the second session, the algorithm pulls arm $a_2$, and she receives a reward of $0.8$. As for the reward discrepancy, $\sdif{2}{1}^3 = \adif{2}{1}^3 = r^3_{2} - r^3_{1} = 0.15 - 0.8 = -0.75$. 

Finally, agent 1 has a cumulative reward of $R^3_1 = R^3_{(2)} = 0.6 + 0.48 + 0.8 = 1.88$, whereas agent~2 has a cumulative reward of $R^3_2 = R^3_{(1)} = 0.6 + 0.1 + 0.15 = 0.85$. In terms of envy, $\env^1_{1,2}= \adif{1}{2}^1 =0$, $\env^2_{1,2}=\adif{1}{2}^1+\adif{1}{2}^2= 0.38$, and $\env^3_{1,2} = -\env^3_{2,1} = R^3_1-R^3_2 = 1.88-0.85 = 1.03$, and consequently the envy in round 3 is $\env^3 = 1.03$.
\end{example}


\subsection{Information Exploitation}
\label{subsec:information}

In this subsection, we explain how algorithms can exploit intra-round information.
Since rewards are consistent in the sessions of each round, acquiring information in each session can be used to increase the reward of the following sessions.
In other words, the earlier sessions can be used for exploration, and we generally expect agents arriving in later sessions to receive higher rewards.
Taken to the extreme, an agent that arrives after all arms have been pulled could potentially obtain the highest reward of that round, depending on how the algorithm operates.

To further demonstrate the advantage of late arrival, we reconsider Example~\ref{example 1} and Algorithm~\ref{alguni}. 
The expected reward for the agent in the first session of round $t$ is $\E{\rt{(1)}}=\mu_1=\frac{1}{2}$, yet the expected reward of the agent in the second session is
\begin{align*}
\E{\rt{(2)}}=\E{\rt{(2)}\mid X^t_1 \geq \frac{1}{2} }\prb{X^t_1 \geq \frac{1}{2}} + \E{\rt{(2)}\mid X^t_1 < \frac{1}{2} }\prb{X^t_1 < \frac{1}{2}};
\end{align*}
thus, $\E{\rt{(2)}} =\E{X^t_1\mid X^t_1 \geq \frac{1}{2} }\cdot \frac{1}{2} + \mu_2\cdot\frac{1}{2} = \frac{5}{8}$.
Consequently, the expected welfare per round is $\E{\rt{(1)}+\rt{(2)}}=1+\frac{1}{8}$, and the benefit of arriving in the second session of any round $t$ is $\E{\rt{(2)} - \rt{(1)}} = \frac{1}{8}$. This gap creates envy over time, which we aim to measure and understand.
%This discrepancy generates envy over time, and our paper aims to better understand it.
\subsection{Socially Optimal Algorithms}
\label{sec: sw}
Since our model is novel, particularly in its focus on the reward consistency element, studying social welfare maximizing algorithms represents an important extension of our work. While the primary focus of this paper is to analyze envy under minimal assumptions about algorithmic operations, we also make progress in the direction of social welfare optimization. See more details in Section~\ref{sec:discussion}.%Due to space limitations, we defer the discussion on socially optimal algorithms to  \ifnum\Includeappendix=0{the appendix}\else{Section~\ref{appendix:sociallyopt}}\fi.




% Since our model is novel and specifically the reward consistency element, it might be interesting to study social welfare optimization. While the main focus of our paper is to study envy under minimal assumptions on how the algorithm operates, we take steps toward this direction as well. Due to space limitations, we defer the discussion on socially optimal algorithms to  \ifnum\Includeappendix=0{the appendix}\else{Section~\ref{appendix:sociallyopt}}\fi.  We devise a socially optimal algorithm for the two-agent case, offer efficient and optimal algorithms for special cases of $N>2$ agents, and an inefficient and approximately optimal algorithm for any instance with $N>2$. Moreover, we address the welfare-envy tradeoff in Section~\ref{sec:extensions}.


% Social welfare, unlike envy, is entirely independent of the arrival order. While the main focus of our paper is to study envy under minimal assumptions on how the algorithm operates, socially optimal algorithms might also be of interest. Due to space limitations, we defer the discussion on socially optimal algorithms to  \ifnum\Includeappendix=0{the appendix}\else{Section~\ref{appendix:sociallyopt}}\fi. We devise a socially optimal algorithm for the two-agent case, offer efficient and optimal algorithms for special cases of $N>2$ agents, and an inefficient and approximately optimal algorithm for any instance with $N>2$. %Furthermore, we treat the welfare-envy tradeoff of the special case of Example~\ref{example 1}.




\subsection{Generative Adversarial Networks}
Generative Adversarial Networks (GANs) are an unsupervised deep learning framework recently proposed by Goodfellow et al.~\cite{goodfellow2014generative}. The framework is composed of two networks, a generator G and a discriminator D, optimized to minimize a two-player minimax game, where the generator learns to fool the discriminator and the discriminator learns to prevent itself from being fooled. As Goodfellow et al.~\cite{goodfellow2014generative} describes, \textit{"The generative model can be thought of as analogous to a team of counterfeiters, trying to produce fake currency and use it without detection, while the discriminative model is analogous to the police, trying to detect the counterfeit currency."} During training of GANs, the generator is fed an input noise vector and produces an output distribution $P_G$. The discriminator is then trained to learn to discriminate between $P_G$ and the true data distribution, $P_{Data}$. Additionally, the generator is trained to learn how to further fool the discriminator. Theoretically, $P_G$ will converge to $P_{Data}$ with the discriminator being unable to differentiate between generated and true samples, resulting in an ideal generative model which can produce data following the true data distribution. 

While GANs are a powerful framework, they possess stability issues which cause the adversarial networks to rarely reach convergence. Variations of the framework, namely Wasserstein GANs (WGANs)~\cite{arjovsky2017wasserstein, gulrajani2017improved}, have been developed which use different loss functions with properties that improve training stability. In contrast to the original GAN framework, WGANs minimize the Earth Mover’s Distance (Wasserstein-1 Distance) and attempt to constrain the gradient norm of the discriminator’s output with respect to its input using a gradient penalty in the loss function. We adopt the WGAN framework for training throughout our research as we experienced improved stability over the original GAN framework. 

\subsection{Proposed Wasserstein Generative Adversarial Networks for EEG Super Resolution}

\begin{table}[ht!]
\centering
\caption{\textbf{Generator Model Architecture}}
\label{tab:gen}
\resizebox{0.95\linewidth}{!}{%
\begin{tabular}{@{}cccc@{}}
\toprule
\textbf{Layer Type} & \textbf{Kernels} & \textbf{Dimensions}            & \textbf{Activation} \\
\toprule
\rowcolor{Gray}
Convolution         & 128              & (\# input channels +  1, 1)    & Linear              \\

Convolution         & 128              & (\# input channels / 2 + 1, 1) & ELU                 \\

\rowcolor{Gray}
Upsampling          & -                & (scale factor – 1, 1)          & -                   \\

Convolution         & 128              & (\# input channels + 1, 1)     & ELU                 \\

\rowcolor{Gray}
Convolution         & 128              & (\# input channels / 2 + 1, 1) & ELU                 \\

Concatenate         & -                & -                              & -                   \\

\rowcolor{Gray}
Convolution         & 256              & (\# input channels / 2 + 1, 1) & ELU                 \\

Concatenate         & -                & -                              & -                   \\

\rowcolor{Gray}
Convolution         & 512              & (\# input channels / 2 + 1, 3) & ELU                 \\

Concatenate         & -                & -                              & -                   \\

\rowcolor{Gray}
Convolution         & 1                & (\# input channels + 1, 1)     & None                \\
\bottomrule

\end{tabular}
}
\end{table}

Our proposed WGAN model for EEG SR also consists of a generator and a discriminator. The generator architecture consists of the sequence of layers detailed in Fig. 1 with the parameters detailed in Table~\ref{tab:gen}. Similarly to~\cite{arjovsky2017wasserstein} we adopt a modified sequence of convolutional layers, which allows EEG data to be processed by Convolutional Neural Networks (CNNs) due to correlations across channels. This sequence is composed of convolutional layers with kernel dimensions that find the relationships between channels, (n, 1), where n = (\# input channels + 1). All convolutional layers enforce the same zero-padding to keep the same dimensions throughout. This sequence is then fed into one dense block~\cite{huang2017densely} composed of 3 densely connected convolutional sequences, followed by a convolutional layer whose outputs are the super-resolved channels. Note that the upsampling layer and the first subsequent convolutional layer are unique to SR models for a scale factor of 4 as the input channels need to be upsampled by 3 channel-wise.



The discriminator follows a similar scheme to the generator architecture aside from a few key differences detailed in Table~\ref{tab:disc}. The 4th convolutional layer has a stride of (4, 4) and is fed into a fully-connected layer. The final output activation of the model is linear to comply with the WGAN framework.  



As in~\cite{ledig2017photo}, the generator’s parameters are first initialized through training the network in a supervised manner to map the downsampled LR data to the HR counterparts using a mean-squared error (MSE) loss function. This was found to prevent converging to local minima. The generator is then inserted into the GAN training process to fine-tune the model and find more optimal parameters in comparison to only using a distance metric as a loss function. 


\section{Results}
\subsection{Training \& Hyperparameters}

\begin{table}[ht!]
\centering
\caption{\textbf{Discriminator Model Architecture}}
\label{tab:disc}
\resizebox{0.95\linewidth}{!}{%
\begin{tabular}{@{}cccc@{}}
\toprule
\textbf{Layer Type} & \textbf{Kernels} & \textbf{Dimensions}            & \textbf{Activation} \\
\toprule

\rowcolor{Gray}
Convolution         & 64               & (\# input channels + 1, 1)     & Linear              \\

Convolution         & 64               & (1, 3)                         & ELU                 \\

\rowcolor{Gray}
Concatenate         & -                & -                              & -                   \\

Convolution         & 128              & (\# input channels / 2 + 1, 1) & ELU                 \\

\rowcolor{Gray}
Concatenate         & -                & -                              & -                   \\

Convolution         & 256              & (\# input channels / 4 + 1, 3) & ELU                 \\

\rowcolor{Gray}
Fully-Connected     & 128              & -                              & ELU                 \\

Fully-Connected     & 1                & -                              & Linear              \\

\bottomrule
\end{tabular}%
}
\end{table}

Dropout regularization~\cite{srivastava2014dropout} was applied at the output of every activation within both the generator and discriminator, excluding the output layers. The generator and discriminator had dropout rates of 0.1 and 0.25, respectively. The $\alpha$ parameter for all ELU~\cite{clevert2015fast} activations was set to 1. The Adam optimizer~\cite{kingma2014adam} was used throughout with learning rate $\alpha=10^{-4}$, $\beta_1=0.5$, and $\beta_2=0.9$. The generator network was first pre-trained using a MSE (L2) loss for 50 epochs with a mini-batch size of 64. All hyperparameters were tuned to optimize performance on the validation set.  

The pre-trained generator was then fine-tuned using the WGAN framework losses with a gradient penalty weight of 10. The GAN training ratio was set to 3, which updates the discriminator once for every 3 generator updates. In addition to the WGAN loss function, a modification was made to multiply the WGAN loss by a factor of $10^{-2}$ and add a MSE loss on the generator output. This was inspired by the feature matching procedure from~\cite{salimans2016improved}, which specifies additional objectives for the generator to prevent overtraining on the discriminator. Also from~\cite{salimans2016improved}, the label smoothing technique was incorporated to assist in avoiding convergence to local minima. 

An evaluation of the model outputs of the validation and test datasets is displayed in Table~\ref{tab:results}. The quantitative results are compared to a baseline of bicubic interpolated channel data. Both MSE and mean absolute error (MAE) between upsampled and true EEG signals were calculated.       


\subsection{Dataset V Classification Super Resolution Performance}
To further evaluate the validity of the SR data, we investigated the performance of classifying the mental imagery classes using the SR data. Deep neural network (DNN) classifiers were trained for both the precomputed features of the HR and SR data using the precomputed feature class labels. The DNN classifier consisted of 5 dense layers with 512, 256, 128, 64, and 3 neurons per layer, respectively. All layers contained ReLU~\cite{nair2010rectified} activations excluding the output layer, which consisted of a softmax activation. The classifiers were trained using a categorical cross-entropy loss optimized by the ADAM optimizer with learning rate $\alpha=10^{-3}$, $\beta_1=0.9$, and $\beta_2=0.99$. The class predictions with multiple metrics for the DNNs trained using the ground truth HR data and SR data by WGAN are recorded in Table~\ref{tab:cls}. 


\begin{table*}[t]
\centering
\fontsize{11pt}{11pt}\selectfont
\begin{tabular}{lllllllllllll}
\toprule
\multicolumn{1}{c}{\textbf{task}} & \multicolumn{2}{c}{\textbf{Mir}} & \multicolumn{2}{c}{\textbf{Lai}} & \multicolumn{2}{c}{\textbf{Ziegen.}} & \multicolumn{2}{c}{\textbf{Cao}} & \multicolumn{2}{c}{\textbf{Alva-Man.}} & \multicolumn{1}{c}{\textbf{avg.}} & \textbf{\begin{tabular}[c]{@{}l@{}}avg.\\ rank\end{tabular}} \\
\multicolumn{1}{c}{\textbf{metrics}} & \multicolumn{1}{c}{\textbf{cor.}} & \multicolumn{1}{c}{\textbf{p-v.}} & \multicolumn{1}{c}{\textbf{cor.}} & \multicolumn{1}{c}{\textbf{p-v.}} & \multicolumn{1}{c}{\textbf{cor.}} & \multicolumn{1}{c}{\textbf{p-v.}} & \multicolumn{1}{c}{\textbf{cor.}} & \multicolumn{1}{c}{\textbf{p-v.}} & \multicolumn{1}{c}{\textbf{cor.}} & \multicolumn{1}{c}{\textbf{p-v.}} &  &  \\ \midrule
\textbf{S-Bleu} & 0.50 & 0.0 & 0.47 & 0.0 & 0.59 & 0.0 & 0.58 & 0.0 & 0.68 & 0.0 & 0.57 & 5.8 \\
\textbf{R-Bleu} & -- & -- & 0.27 & 0.0 & 0.30 & 0.0 & -- & -- & -- & -- & - &  \\
\textbf{S-Meteor} & 0.49 & 0.0 & 0.48 & 0.0 & 0.61 & 0.0 & 0.57 & 0.0 & 0.64 & 0.0 & 0.56 & 6.1 \\
\textbf{R-Meteor} & -- & -- & 0.34 & 0.0 & 0.26 & 0.0 & -- & -- & -- & -- & - &  \\
\textbf{S-Bertscore} & \textbf{0.53} & 0.0 & {\ul 0.80} & 0.0 & \textbf{0.70} & 0.0 & {\ul 0.66} & 0.0 & {\ul0.78} & 0.0 & \textbf{0.69} & \textbf{1.7} \\
\textbf{R-Bertscore} & -- & -- & 0.51 & 0.0 & 0.38 & 0.0 & -- & -- & -- & -- & - &  \\
\textbf{S-Bleurt} & {\ul 0.52} & 0.0 & {\ul 0.80} & 0.0 & 0.60 & 0.0 & \textbf{0.70} & 0.0 & \textbf{0.80} & 0.0 & {\ul 0.68} & {\ul 2.3} \\
\textbf{R-Bleurt} & -- & -- & 0.59 & 0.0 & -0.05 & 0.13 & -- & -- & -- & -- & - &  \\
\textbf{S-Cosine} & 0.51 & 0.0 & 0.69 & 0.0 & {\ul 0.62} & 0.0 & 0.61 & 0.0 & 0.65 & 0.0 & 0.62 & 4.4 \\
\textbf{R-Cosine} & -- & -- & 0.40 & 0.0 & 0.29 & 0.0 & -- & -- & -- & -- & - & \\ \midrule
\textbf{QuestEval} & 0.23 & 0.0 & 0.25 & 0.0 & 0.49 & 0.0 & 0.47 & 0.0 & 0.62 & 0.0 & 0.41 & 9.0 \\
\textbf{LLaMa3} & 0.36 & 0.0 & \textbf{0.84} & 0.0 & {\ul{0.62}} & 0.0 & 0.61 & 0.0 &  0.76 & 0.0 & 0.64 & 3.6 \\
\textbf{our (3b)} & 0.49 & 0.0 & 0.73 & 0.0 & 0.54 & 0.0 & 0.53 & 0.0 & 0.7 & 0.0 & 0.60 & 5.8 \\
\textbf{our (8b)} & 0.48 & 0.0 & 0.73 & 0.0 & 0.52 & 0.0 & 0.53 & 0.0 & 0.7 & 0.0 & 0.59 & 6.3 \\  \bottomrule
\end{tabular}
\caption{Pearson correlation on human evaluation on system output. `R-': reference-based. `S-': source-based.}
\label{tab:sys}
\end{table*}



\begin{table}%[]
\centering
\fontsize{11pt}{11pt}\selectfont
\begin{tabular}{llllll}
\toprule
\multicolumn{1}{c}{\textbf{task}} & \multicolumn{1}{c}{\textbf{Lai}} & \multicolumn{1}{c}{\textbf{Zei.}} & \multicolumn{1}{c}{\textbf{Scia.}} & \textbf{} & \textbf{} \\ 
\multicolumn{1}{c}{\textbf{metrics}} & \multicolumn{1}{c}{\textbf{cor.}} & \multicolumn{1}{c}{\textbf{cor.}} & \multicolumn{1}{c}{\textbf{cor.}} & \textbf{avg.} & \textbf{\begin{tabular}[c]{@{}l@{}}avg.\\ rank\end{tabular}} \\ \midrule
\textbf{S-Bleu} & 0.40 & 0.40 & 0.19* & 0.33 & 7.67 \\
\textbf{S-Meteor} & 0.41 & 0.42 & 0.16* & 0.33 & 7.33 \\
\textbf{S-BertS.} & {\ul0.58} & 0.47 & 0.31 & 0.45 & 3.67 \\
\textbf{S-Bleurt} & 0.45 & {\ul 0.54} & {\ul 0.37} & 0.45 & {\ul 3.33} \\
\textbf{S-Cosine} & 0.56 & 0.52 & 0.3 & {\ul 0.46} & {\ul 3.33} \\ \midrule
\textbf{QuestE.} & 0.27 & 0.35 & 0.06* & 0.23 & 9.00 \\
\textbf{LlaMA3} & \textbf{0.6} & \textbf{0.67} & \textbf{0.51} & \textbf{0.59} & \textbf{1.0} \\
\textbf{Our (3b)} & 0.51 & 0.49 & 0.23* & 0.39 & 4.83 \\
\textbf{Our (8b)} & 0.52 & 0.49 & 0.22* & 0.43 & 4.83 \\ \bottomrule
\end{tabular}
\caption{Pearson correlation on human ratings on reference output. *not significant; we cannot reject the null hypothesis of zero correlation}
\label{tab:ref}
\end{table}


\begin{table*}%[]
\centering
\fontsize{11pt}{11pt}\selectfont
\begin{tabular}{lllllllll}
\toprule
\textbf{task} & \multicolumn{1}{c}{\textbf{ALL}} & \multicolumn{1}{c}{\textbf{sentiment}} & \multicolumn{1}{c}{\textbf{detoxify}} & \multicolumn{1}{c}{\textbf{catchy}} & \multicolumn{1}{c}{\textbf{polite}} & \multicolumn{1}{c}{\textbf{persuasive}} & \multicolumn{1}{c}{\textbf{formal}} & \textbf{\begin{tabular}[c]{@{}l@{}}avg. \\ rank\end{tabular}} \\
\textbf{metrics} & \multicolumn{1}{c}{\textbf{cor.}} & \multicolumn{1}{c}{\textbf{cor.}} & \multicolumn{1}{c}{\textbf{cor.}} & \multicolumn{1}{c}{\textbf{cor.}} & \multicolumn{1}{c}{\textbf{cor.}} & \multicolumn{1}{c}{\textbf{cor.}} & \multicolumn{1}{c}{\textbf{cor.}} &  \\ \midrule
\textbf{S-Bleu} & -0.17 & -0.82 & -0.45 & -0.12* & -0.1* & -0.05 & -0.21 & 8.42 \\
\textbf{R-Bleu} & - & -0.5 & -0.45 &  &  &  &  &  \\
\textbf{S-Meteor} & -0.07* & -0.55 & -0.4 & -0.01* & 0.1* & -0.16 & -0.04* & 7.67 \\
\textbf{R-Meteor} & - & -0.17* & -0.39 & - & - & - & - & - \\
\textbf{S-BertScore} & 0.11 & -0.38 & -0.07* & -0.17* & 0.28 & 0.12 & 0.25 & 6.0 \\
\textbf{R-BertScore} & - & -0.02* & -0.21* & - & - & - & - & - \\
\textbf{S-Bleurt} & 0.29 & 0.05* & 0.45 & 0.06* & 0.29 & 0.23 & 0.46 & 4.2 \\
\textbf{R-Bleurt} & - &  0.21 & 0.38 & - & - & - & - & - \\
\textbf{S-Cosine} & 0.01* & -0.5 & -0.13* & -0.19* & 0.05* & -0.05* & 0.15* & 7.42 \\
\textbf{R-Cosine} & - & -0.11* & -0.16* & - & - & - & - & - \\ \midrule
\textbf{QuestEval} & 0.21 & {\ul{0.29}} & 0.23 & 0.37 & 0.19* & 0.35 & 0.14* & 4.67 \\
\textbf{LlaMA3} & \textbf{0.82} & \textbf{0.80} & \textbf{0.72} & \textbf{0.84} & \textbf{0.84} & \textbf{0.90} & \textbf{0.88} & \textbf{1.00} \\
\textbf{Our (3b)} & 0.47 & -0.11* & 0.37 & 0.61 & 0.53 & 0.54 & 0.66 & 3.5 \\
\textbf{Our (8b)} & {\ul{0.57}} & 0.09* & {\ul 0.49} & {\ul 0.72} & {\ul 0.64} & {\ul 0.62} & {\ul 0.67} & {\ul 2.17} \\ \bottomrule
\end{tabular}
\caption{Pearson correlation on human ratings on our constructed test set. 'R-': reference-based. 'S-': source-based. *not significant; we cannot reject the null hypothesis of zero correlation}
\label{tab:con}
\end{table*}

\section{Results}
We benchmark the different metrics on the different datasets using correlation to human judgement. For content preservation, we show results split on data with system output, reference output and our constructed test set: we show that the data source for evaluation leads to different conclusions on the metrics. In addition, we examine whether the metrics can rank style transfer systems similar to humans. On style strength, we likewise show correlations between human judgment and zero-shot evaluation approaches. When applicable, we summarize results by reporting the average correlation. And the average ranking of the metric per dataset (by ranking which metric obtains the highest correlation to human judgement per dataset). 

\subsection{Content preservation}
\paragraph{How do data sources affect the conclusion on best metric?}
The conclusions about the metrics' performance change radically depending on whether we use system output data, reference output, or our constructed test set. Ideally, a good metric correlates highly with humans on any data source. Ideally, for meta-evaluation, a metric should correlate consistently across all data sources, but the following shows that the correlations indicate different things, and the conclusion on the best metric should be drawn carefully.

Looking at the metrics correlations with humans on the data source with system output (Table~\ref{tab:sys}), we see a relatively high correlation for many of the metrics on many tasks. The overall best metrics are S-BertScore and S-BLEURT (avg+avg rank). We see no notable difference in our method of using the 3B or 8B model as the backbone.

Examining the average correlations based on data with reference output (Table~\ref{tab:ref}), now the zero-shoot prompting with LlaMA3 70B is the best-performing approach ($0.59$ avg). Tied for second place are source-based cosine embedding ($0.46$ avg), BLEURT ($0.45$ avg) and BertScore ($0.45$ avg). Our method follows on a 5. place: here, the 8b version (($0.43$ avg)) shows a bit stronger results than 3b ($0.39$ avg). The fact that the conclusions change, whether looking at reference or system output, confirms the observations made by \citet{scialom-etal-2021-questeval} on simplicity transfer.   

Now consider the results on our test set (Table~\ref{tab:con}): Several metrics show low or no correlation; we even see a significantly negative correlation for some metrics on ALL (BLEU) and for specific subparts of our test set for BLEU, Meteor, BertScore, Cosine. On the other end, LlaMA3 70B is again performing best, showing strong results ($0.82$ in ALL). The runner-up is now our 8B method, with a gap to the 3B version ($0.57$ vs $0.47$ in ALL). Note our method still shows zero correlation for the sentiment task. After, ranks BLEURT ($0.29$), QuestEval ($0.21$), BertScore ($0.11$), Cosine ($0.01$).  

On our test set, we find that some metrics that correlate relatively well on the other datasets, now exhibit low correlation. Hence, with our test set, we can now support the logical reasoning with data evidence: Evaluation of content preservation for style transfer needs to take the style shift into account. This conclusion could not be drawn using the existing data sources: We hypothesise that for the data with system-based output, successful output happens to be very similar to the source sentence and vice versa, and reference-based output might not contain server mistakes as they are gold references. Thus, none of the existing data sources tests the limits of the metrics.  


\paragraph{How do reference-based metrics compare to source-based ones?} Reference-based metrics show a lower correlation than the source-based counterpart for all metrics on both datasets with ratings on references (Table~\ref{tab:sys}). As discussed previously, reference-based metrics for style transfer have the drawback that many different good solutions on a rewrite might exist and not only one similar to a reference.


\paragraph{How well can the metrics rank the performance of style transfer methods?}
We compare the metrics' ability to judge the best style transfer methods w.r.t. the human annotations: Several of the data sources contain samples from different style transfer systems. In order to use metrics to assess the quality of the style transfer system, metrics should correctly find the best-performing system. Hence, we evaluate whether the metrics for content preservation provide the same system ranking as human evaluators. We take the mean of the score for every output on each system and the mean of the human annotations; we compare the systems using the Kendall's Tau correlation. 

We find only the evaluation using the dataset Mir, Lai, and Ziegen to result in significant correlations, probably because of sparsity in a number of system tests (App.~\ref{app:dataset}). Our method (8b) is the only metric providing a perfect ranking of the style transfer system on the Lai data, and Llama3 70B the only one on the Ziegen data. Results in App.~\ref{app:results}. 


\subsection{Style strength results}
%Evaluating style strengths is a challenging task. 
Llama3 70B shows better overall results than our method. However, our method scores higher than Llama3 70B on 2 out of 6 datasets, but it also exhibits zero correlation on one task (Table~\ref{tab:styleresults}).%More work i s needed on evaluating style strengths. 
 
\begin{table}%[]
\fontsize{11pt}{11pt}\selectfont
\begin{tabular}{lccc}
\toprule
\multicolumn{1}{c}{\textbf{}} & \textbf{LlaMA3} & \textbf{Our (3b)} & \textbf{Our (8b)} \\ \midrule
\textbf{Mir} & 0.46 & 0.54 & \textbf{0.57} \\
\textbf{Lai} & \textbf{0.57} & 0.18 & 0.19 \\
\textbf{Ziegen.} & 0.25 & 0.27 & \textbf{0.32} \\
\textbf{Alva-M.} & \textbf{0.59} & 0.03* & 0.02* \\
\textbf{Scialom} & \textbf{0.62} & 0.45 & 0.44 \\
\textbf{\begin{tabular}[c]{@{}l@{}}Our Test\end{tabular}} & \textbf{0.63} & 0.46 & 0.48 \\ \bottomrule
\end{tabular}
\caption{Style strength: Pearson correlation to human ratings. *not significant; we cannot reject the null hypothesis of zero corelation}
\label{tab:styleresults}
\end{table}

\subsection{Ablation}
We conduct several runs of the methods using LLMs with variations in instructions/prompts (App.~\ref{app:method}). We observe that the lower the correlation on a task, the higher the variation between the different runs. For our method, we only observe low variance between the runs.
None of the variations leads to different conclusions of the meta-evaluation. Results in App.~\ref{app:results}.

\section{Discussion}
On the topic of CNNs for EEG time-series data, we highlight below some of the important findings throughout our research. Feature scaling techniques besides standard normalization decreased model performance. With regards to convolutional layers, using kernel dimensions that contained weights for each channel in the input and output layers improved performance significantly over the standard kernel dimensions used for images, e.g., $3\times3$, $9\times9$. Implementing concatenation connections instead of residual connections, popular in ResNet~\cite{he2016deep} architectures used in many Super Resolution papers, offered improved performance using a lesser amount of layers. A Linear activation on the input layer followed by ELU activations on the subsequent layers outperformed other popular neural network activation function combinations. 

It was notably difficult and time-consuming to train GANs for EEG data. We observed after testing different variants of GAN that WGAN appeared to be more stable during training. Replacing MSE with MAE in all loss functions produced SR EEG signals which were smoothed and did not contain similar frequency domain statistics as the HR data. It can be concluded that the task of EEG SR is highly sensitive to the loss function components used during training. 

Observing the results in Table~\ref{tab:results}, compared to bicubic interpolation, WGAN achieved $10^4$ fold and $10^2$ fold reduction in MSE and MAE, respectively, demonstrating the remarkable improvement of our proposed WGAN method in simultaneously reconstructing numerous missing EEG signals at a high resolution. Judging from the results in Table~\ref{tab:cls} it can be observed that classification of SR data produces minimal loss of accuracy when compared to ground truth signals, less than 4\% and 9\% for scale factors of 2 and 4, respectively. 

\section{Conclusion}
Our results conclude that our WGAN methods significantly improved over bicubic interpolation for the Dataset V EEG signals. We conclude that SR EEG by GAN is a promising approach to improve the spatial resolution of low-density EEG headsets. However, we intend to expand our work to perform well across multiple datasets for different classification tasks. Considerations for further work also include using different distance metrics than MSE for assessing signal similarity, as well as using other recent variations of the GAN framework to compare results.  

\begin{table}[ht!]
\centering
\caption{\textbf{Classification Performance Results on the Berlin BCI Competition III, Dataset V test set.} We remove 16 (\textit{scale=2)} and 24 (\textit{scale=4)} of the dataset's 32 EEG channels and use our proposed WGAN spatial upsampling method to recreate the missing channels using the remaining channels. We then train and evaluate a classifier on the upsampled channels and show minimal loss in performance compared to using the original dataset channels.}
\label{tab:cls}
\resizebox{0.9\linewidth}{!}{%
\begin{tabular}{@{}ccccc@{}}
\toprule
\textbf{Scale} &
\textbf{Metric} &
\textbf{Class} &
\textbf{HR} &
\textbf{WGAN (Ours)} \\
\toprule
\multirow{7}{*}{2} & Accuracy                   & - & 87.75 & 83.88 \\ \cmidrule(l){2-5} 
                   & \multirow{3}{*}{Precision} & 2 & 86.65 & 82.24 \\
                   &                            & 3 & 87.54 & 82.84 \\
                   &                            & 7 & 88.77 & 86.12 \\ \cmidrule(l){2-5} 
                   & \multirow{3}{*}{Recall}    & 2 & 84.33 & 80.21 \\
                   &                            & 3 & 88.57 & 84.37 \\
                   &                            & 7 & 89.62 & 86.26 \\ \midrule
\multirow{7}{*}{4} & Accuracy                   & - & 87.75 & 82.00 \\ \cmidrule(l){2-5} 
                   & \multirow{3}{*}{Precision} & 2 & 86.65 & 80.21 \\
                   &                            & 3 & 87.54 & 82.03 \\
                   &                            & 7 & 88.77 & 83.28 \\ \cmidrule(l){2-5} 
                   & \multirow{3}{*}{Recall}    & 2 & 84.33 & 77.73 \\
                   &                            & 3 & 88.57 & 81.34 \\
                   &                            & 7 & 89.62 & 85.94 \\ \bottomrule
\end{tabular}%
}
\end{table}

\section*{\centering \large Acknowledgements}
This work was supported by the Army Research Laboratory Cognition and Neuroergonomics Collaborative Technology Alliance (CANCTA) under Cooperative Agreement Number W911NF-10-2-0022.  


\bibliographystyle{IEEEbib}
\bibliography{refs}

\end{document}