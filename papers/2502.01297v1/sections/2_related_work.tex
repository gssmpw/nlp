\section{Related Work}

Currently, there are numerous remarkable works on VIO in both academia and industry, which fall into two main categories: filtering-based and optimization-based approaches. Additionally, there are several typical implementations in the community for the two key modules of initialization and feature matching.

%% copy from RD-VIO as baseline %%
 \textbf{Filtering-based VIO} Early VIO systems, such as MSCKF \cite{mourikis-icra-2007} and ROVIO \cite{bloesch-ijrr-2017}, were built on Kalman filtering principles. These systems, based on sliding window approaches, incorporate frame poses at different time steps into their state vector.  During the update phase, visual constraints are utilized to update the state vector, while landmarks are marginalized. This approach limited the computational complexity required for these systems. ROVIO, in particular, integrated data association with the filter estimation process by utilizing photometric errors for visual observations. 
 
R-VIO \cite{huai2022robocentric} introduced a novel robocentric visual-inertial odometry approach. The key idea is to redesign the VINS framework based on a moving local coordinate system, rather than the fixed global reference frame typically used in conventional world-centric VINS. This redesign aims to achieve higher precision relative motion estimation for updating the global pose. 

OpenVINS \cite{geneva2020openvins} emerged as a recently developed open-source platform that employed the MSCKF filter. Its modular design allowed for flexibility of use and easy expansion. Evaluations on open-source datasets demonstrated its superior precision and robustness.

To boost VIO performance, recent systems, as discussed in \cite{bao2022robust}, have leveraged pre-existing high-precision maps, resulting in significant enhancements in accuracy. Others, such as RNIN-VIO \cite{chen2021rnin}, have capitalized on neural networks in IMU navigation to enhance robustness. These advancements showcase the ongoing efforts in pushing the boundaries of VIO.

\textbf{Optimization-based VIO }OKVIS \cite{leutenegger-ijrr-2015-OKVIS} is a system that operates using a sliding window methodology. During the optimization process, it integrates new keyframes and marginalizes old keyframes. By linearizing old observations as priors and incorporating them into the optimization, the system achieves improved performance. VINS-Mono \cite{qin-tro-2018_VINS-Mono} and VINS-Fusion \cite{qin2019a_VINS_Fusion_Local,qin2019b_VINS_Fusion_Global} are recent influential VI-SLAM systems that also use keyframe-based sliding window methods in their frontends. The backend loop-closure feature helps mitigate accumulated errors, resulting in higher precision. In contrast, VI-ORB-SLAM \cite{murartal-ral-2017-VI-ORB} is a loosely-coupled VI-SLAM system that initially processes visual observations using traditional VSLAM techniques before aligning them with inertial measurements to obtain metric results. ORB-SLAM3 \cite{campos2021orb-slam3} is a tightly-coupled system known for its remarkable accuracy. However, it solves the full SLAM problem, necessitating optimization of early poses based on later observations, making it unsuitable for real-time applications. VI-DSO \cite{von2018direct-VI-DSO} and DM-VIO \cite{stumberg22DM-VIO} are VIO extensions of the original DSO \cite{engel2018direct_DSO}, designed to enhance robustness and provide accurate scaling capabilities.

\textbf{Visual Inertial Initialization} The accurate and robust initialization of VIO is crucial for its normal operation. This process involves using both visual and inertial measurements to calculate the initial states, including scale, speed, gravity direction, etc. In the work of \cite{Dong-Si_initialization}, a tightly coupled closed-form algorithm was proposed, which uses visual and IMU measurements to simultaneously estimate initial states and the depth of feature points. OpenVINS \cite{geneva2020openvins}, a recently released open-source VIO platform, adopts this initialization process and adds Visual Inertial Bundle Adjustment \cite{triggs1999bundle} (VI-BA) after state initialization, as discussed in \cite{genevaopenvins}. On the other hand, VINS-Mono\cite{qin-tro-2018_VINS-Mono} is a typical loosely coupled method, which firstly utilizes visual SfM to solve camera poses using only visual measurements and then aligns them with IMU pre-intergration \cite{forster2017manifold-preintergration} to estimate the initial state. 
The latest approach \cite{Rotation-Translation-Decoupled} employs a decoupled formulation for rotation and translation, along with a pose-only solver, to estimate the initial state. Experiments in \cite{Rotation-Translation-Decoupled} have shown that it is optimal to firstly tightly couple the gyroscope in visual-inertial optimization, and secondly loosely couple the accelerometer in linear VI-alignment (namely DRT-l). Additionally, capitalizing on advancements in monocular depth estimation methods \cite{ranftl2020towards}, \cite{Mono-Depth-VI-Init-2022,merrill2023fast} employ the learned monocular depth as input for multi-view consistent visual-inertial initialization. We provide a detailed comparison of initialization algorithms in \cref{tab:algorithms}. 

\textbf{Feature Matching}
    GFTT\cite{gftt1994shi} and KLT\cite{Lucas_Kanade_1981_KLT} are popular optical flow-based feature matching methods for VIO, both of which are implemented in OpenCV \cite{opencv_library}. VINS-Mono\cite{qin-tro-2018_VINS-Mono}, OpenVINS\cite{geneva2020openvins} and many other VIO systems achieve high accuracy using these implementations.  Optical flow is suitable for consecutive feature matching and often results in long track length. However, it may suffer from frame-by-frame drift, posing a core bottleneck for VIO accuracy.  On the other hand, descriptor-based matching, as utilized by systems like ORB-SLAM\cite{leutenegger-ijrr-2015-OKVIS,campos2021orb-slam3,murartal-ral-2017-VI-ORB}, offers better accuracy and lower drift. BRISK\cite{BRISK} and ORB\cite{rublee2011orb} are two widely used methods for feature extraction and description, both known for their good repeatability and suitability for feature matching. Despite these advantages, descriptor matching tends to have a shorter track length due to changes in image perspective or lighting, which limits the accuracy of VIO tracking.
    A new trend in feature matching now is how to combine optical flow and descriptor matching. Some existing works\cite{bang2017camera,zong2017improved,zhong2023improved} have made preliminary attempts, but they just simply combine the two algorithms together without tight fusion, which limit the efficiency and accuracy of feature matching.