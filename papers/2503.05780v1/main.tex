\documentclass{article}
\usepackage{graphicx} % Required for inserting images

% \documentclass[a4paper,12pt]{article}  % Standard article class with A4 paper size

% Essential packages
\usepackage[a4paper, margin=1in]{geometry}  % Set margins
\usepackage{graphicx}  % For including images
\usepackage{titlesec}  % For section formatting
\usepackage{fancyhdr}  % Custom headers and footers
\usepackage{hyperref}  % Clickable links
\usepackage{amsmath, amssymb}  % Math symbols if needed
\usepackage{todonotes}  % For comments (optional)

\usepackage[edges]{forest}
\usepackage{xcolor}
\usepackage[skins]{tcolorbox}%
\usepackage{tcolorbox}
\usepackage{hyperref}
\newtcolorbox[auto counter, number within=section]{definitionbox}[2][]{%
colframe=black!50,
colback=black!5,
coltitle=white,
fonttitle=\bfseries,
title={#2},
sharp corners=south,
enhanced,
before upper={\noindent},
}

\newcommand{\ignore}[1]{}

% Title and author
%\title{\textbf{AI Risk Atlas (Nexus? open to either) - Taxonomy and Tooling for Navigating AI Risks and Resources}}
\title{\textbf{AI Risk Atlas:\\
Taxonomy and Tooling for\\Navigating AI Risks and Resources}}

\author{Frank Bagehorn, Kristina Brimijoin, Elizabeth M. Daly, Jessica He,\\Michael Hind, Luis Garcés-Erice, Christopher Giblin, Ioana Giurgiu,\\Jacquelyn Martino, Rahul Nair, David Piorkowski, Ambrish Rawat, John Richards,\\Sean Rooney, Dhaval Salwala, Seshu Tirupathi, Peter Urbanetz,\\Kush R. Varshney, Inge Vejsbjerg, Mira L. Wolf-Bauwens}
%\date{\today}

\begin{document}

\maketitle  % Generate title

\thanks{Authors are listed in alphabetical order by last name.}




% \begin{document}

% \maketitle

\begin{abstract}
The rapid evolution of generative AI has expanded the breadth of risks associated with AI systems. While various taxonomies and frameworks exist to classify these risks, the lack of interoperability between them creates challenges for researchers, practitioners, and policymakers seeking to operationalise AI governance. To address this gap, we introduce the \textbf{AI Risk Atlas}, a structured taxonomy that consolidates AI risks from diverse sources and aligns them with governance frameworks. Additionally, we present the \textbf{Risk Atlas Nexus}, a collection of open-source tools designed to bridge the divide between risk definitions, benchmarks, datasets, and mitigation strategies. This knowledge-driven approach leverages ontologies and knowledge graphs to facilitate risk identification, prioritization, and mitigation. By integrating AI-assisted compliance workflows and automation strategies, our framework lowers the barrier to responsible AI adoption. We invite the broader research and open-source community to contribute to this evolving initiative, fostering cross-domain collaboration and ensuring AI governance keeps pace with technological advancements.
%\todo{MH: I decided to write a short abstract to capture what I think the flow of the intro is. Feel free to improve this draft. I also massaged the title to match.}
\end{abstract}

\section{Introduction}
% NET: AI risks are important, but need help to stay current and to bring the risk identification and mitigation communities together.
%
Identifying the risks of AI systems has attracted interest from research~\cite{weizenbaum1976computer, russell2016artificial}, industry~\cite{shahriari2017ieee}, and policy makers~\cite{carricco2018eu}.
%The risks associated with AI systems is something research \cite{weizenbaum1976computer, russell2016artificial}, industry \cite{shahriari2017ieee} and policy makers \cite{carricco2018eu} have understood for some time.
%\todo{MH: I think the previous seems to imply we know everything about the risks of AI systems, so I massage this.}
These perspectives spawned many innovations to aid in the creation and operationalising of responsible AI system design \cite{aif360, arya2021ai, wexler2019if, bird2020fairlearn}. % Despite these efforts, their adoption has remained ad hoc and largely focused on a small subset of AI risks.
%\todo{MH: I'm not sure we need the last 2 sentences. The first sentence is only pointing to toolkits, but there is a lot more in this general space. The 2nd sentence takes a position I'm not sure we need to take. Elizabeth: I removed the 2nd sentence if you feel it is problematic. I think highlighting that attention on risks helps the community focus on buidling algorithmic solutions isn't a bad message...}
With the advent of generative AI and its rapidly evolving capabilities~\cite{bommasani2021opportunities}, the spectrum of risks, and the urgency to mitigate them, has increased. As a result, there is a need to help the community identify and address these risks in parallel, while also creating a mechanism for collaboration.

%NET: Other efforts to identify risks don't solve all the problems
There have been several efforts to catalogue risks associated with AI systems \cite{nist,owasp,airiskrepo,mit-risk-repository,zeng2024ai}. However, connections and relationships to existing risk classification frameworks are missing. This lack of connectivity can present a challenge for practitioners who may want to adopt new risk taxonomies, but have already categorised their assets using existing definitions. 

% NET: there are flourishing communities for sharing datasets and benchmarks, but these haven't been focused on risks, so there is an opportunity to expand the impact of these communitities.
%
Meanwhile, there are currently flourishing communities for sharing datasets~\cite{lhoest2021datasets} and benchmarks~\cite{eval-harness}. Although the focus of datasets effort is to evaluate the correctness of an AI model, typically partitioned by AI task or skill, many of these datasets can also be leveraged to assess various AI risks. Similarly, mapping the results of benchmarks to risk concerns is currently not a part of most benchmarks, but could be accomplished with better collaboration between the benchmark and risk communities.

%%HuggingFace provides thousands of valuable datasets contributed via the AI research community \cite{lhoest2021datasets}, however, while the focus has been on specific tasks or skills a large number may also aid in assessing risks. Similarly, lm-eval-harness contains many benchmarks, however, the interpretation of the outcome of these benchmarks and how they might reflect risk concerns is left to the data scientist to determine by studying related resources and documentation \cite{eval-harness}.

%NET: AI gov in practice is involved. Our tooling info can provide a foundation for future innovations to help automate some of the process
%
%Automation can play a vital role in easing the barrier for entry for developers and practitioners to make responsible AI an integral part of their process, however to support operationalisation, some structure must be put in place.
To help manage risks, the process of putting a new AI system into production often includes multiple stakeholders such as business owners, risk and compliance officers, and ethics officers approving the AI system for a specific usage. Governance frameworks to manage this process typically include multiple manual steps, including curating information needed to assess risks (where will the system be used? who is the target user?) and reviewing outcomes to identify appropriate actions and governance strategies.
Automation can play a vital role in easing the barrier for entry for developers and practitioners to make responsible AI an integral part of their process, however, this automation requires some structure of the underlying information.
For example, AI capabilities can help to create better semi-structured governance information, identify and prioritize risks according to the intended use case, recommend appropriate benchmarks and risk assessments and most importantly recommend mitigation strategies and recommended actions.  Our aim is to develop an AI Systems risk ontology that links
risk entities described using multiple different taxonomies with AI models,
evaluations, mitigations and other important entities. This ontology is manifested in a knowledge graph that associated tooling uses to integrate distinct risk frameworks, thereby providing the community with a way to align their assets with both new and existing risk definitions. 

This paper is organized as follows. Section~\ref{sec-risk-atlas} describes the AI Risk Atlas, which provides a taxonomy of AI Risks. Section~\ref{sec-tools} presents \textbf{Risk Atlas Nexus}, an open source tooling effort we have developed to enable inter- and cross-community collaboration. 
Section~\ref{sec-potential} discusses some of the future directions that can be facilitated by these tools.
Section~\ref{sec-conc} invites the broader community to expand on this initial seeding of tools by bringing different perspectives with different needs and ultimately lowering the barrier to AI governance.


%\textit{Original Outline
%\begin{itemize}
 %   \item Challenges companies, researchers and governments face in the fast changing AI landscape 
  %  \item The purpose and aim of the Risk Atlas and why it is so important
 %   \item Why is a taxonomy useful
  %  \item What can it be used for and what challenges does it address
  %  \item how can tooling help address additional challenges
%\end{itemize}
%}  
    

\section{The AI Risk Landscape} \label{sec-risk-atlas}
This section describes the \textit{AI Risk Atlas} which aims to provide clarity for practitioners of the risks associated with Generative AI systems. 

% High level description of the risk atlas
The AI Risk Atlas is a taxonomy of AI risks collected from prior research, real-world examples, and from experts in the field. It defines risks posed by AI systems and explain potential consequences of those risks. Each risk is grouped into one of four categories based on where the risk originates. The categories are input risks, inference risks, output risks and non-technical risks. Within each category, risks are further grouped into risk dimensions such as accuracy, fairness, or explainability. These dimensions classify the individual risks into groups, and enable a user of the Atlas to focus on the dimensions relevant to them. 


\begin{figure*}[!ht]
    % \section{Taxonomy}

% As illustrated by Fig. \ref{}, the typical process of vision models based time series analysis has five components: (1) normalization/scaling; (2) time series to image transformation; (3) image modeling; (4) image to time series recovery; and (5) task processing. In the rest of this paper, we will discuss the typical methods for each of these components. The detailed taxonomy of the methods are summarized in Table \ref{tab.taxonomy}.

%Typical step: normalization/scaling, transformation, vision modeling, task-specific head, inverse transformation (for tasks that output time series, e.g., forecasting, generation, imputation, anomaly detection). Normalization is to fit the arbitrary range of time series values to RGB representation.

\begin{figure*}[!t]
\centering
\includegraphics[width=1.0\textwidth]{fig/fig_3.pdf}
% \vspace{-1em}
\caption{An illustration of different methods for imaging time series with a sample (length=336) from the \textit{Electricity} benchmark dataset \protect\cite{nie2023time}. (a)(c)(d)(e)(f) %are univariate methods.
visualize the same variate. (b) visualizes all 321 variates. Filterbank is omitted due to its %high
similarity to STFT.}\label{fig.tsimage}
\vspace{-0.2cm}
\end{figure*}

\begin{table*}[t]
\centering
\scriptsize
\setlength{\tabcolsep}{2.7pt}{
% \begin{tabular}{llllllllllll}
\begin{tabular}{llcccccccccl}
\toprule[1pt]
\multirow{2}{*}{Method} & \multirow{2}{*}{TS-Type} & \multirow{2}{*}{Imaging} & \multicolumn{5}{c}{Imaged Time Series Modeling} & \multirow{2}{*}{TS-Recover} & \multirow{2}{*}{Task} & \multirow{2}{*}{Domain} & \multirow{2}{*}{Code}\\ \cmidrule{4-8}
 & & & Multi-modal & Model & Pre-trained & Fine-tune & Prompt & & & & \\ \midrule
\cite{silva2013time} & UTS & RP & \xmark & \texttt{K-NN} & \xmark & \xmark & \xmark & \xmark & Classification & General & \xmark\\
\cite{wang2015encoding} & UTS & GAF & \xmark & \texttt{CNN} & \xmark & \cmark$^{\flat}$ & \xmark & \cmark & Classification & General & \xmark\\
\cite{wang2015imaging} & UTS & GAF & \xmark & \texttt{CNN} & \xmark & \cmark$^{\flat}$ & \xmark & \cmark & Multiple & General & \xmark\\
% \multirow{2}{*}{\cite{wang2015imaging}} & \multirow{2}{*}{UTS} & \multirow{2}{*}{GAF} & \multirow{2}{*}{\xmark} & \multirow{2}{*}{\texttt{CNN}} & \multirow{2}{*}{\xmark} & \multirow{2}{*}{\cmark$^{\flat}$} & \multirow{2}{*}{\xmark} & \multirow{2}{*}{\cmark} & Classification & \multirow{2}{*}{General} & \multirow{2}{*}{\xmark}\\
% & & & & & & & & & \& Imputation & & \\
\cite{ma2017learning} & MTS & Heatmap & \xmark & \texttt{CNN} & \xmark & \cmark$^{\flat}$ & \xmark & \cmark & Forecasting & Traffic & \xmark\\
\cite{hatami2018classification} & UTS & RP & \xmark & \texttt{CNN} & \xmark & \cmark$^{\flat}$ & \xmark & \xmark & Classification & General & \xmark\\
\cite{yazdanbakhsh2019multivariate} & MTS & Heatmap & \xmark & \texttt{CNN} & \xmark & \cmark$^{\flat}$ & \xmark & \xmark & Classification & General & \cmark\textsuperscript{\href{https://github.com/SonbolYb/multivariate_timeseries_dilated_conv}{[1]}}\\
MSCRED \cite{zhang2019deep} & MTS & Other ($\S$\ref{sec.othermethod}) & \xmark & \texttt{ConvLSTM} & \xmark & \cmark$^{\flat}$ & \xmark & \xmark & Anomaly & General & \cmark\textsuperscript{\href{https://github.com/7fantasysz/MSCRED}{[2]}}\\
\cite{li2020forecasting} & UTS & RP & \xmark & \texttt{CNN} & \cmark & \cmark & \xmark & \xmark & Forecasting & General & \cmark\textsuperscript{\href{https://github.com/lixixibj/forecasting-with-time-series-imaging}{[3]}}\\
\cite{cohen2020trading} & UTS & LinePlot & \xmark & \texttt{Ensemble} & \xmark & \cmark$^{\flat}$ & \xmark & \xmark & Classification & Finance & \xmark\\
% \cite{du2020image} & UTS & Spectrogram & \xmark & \texttt{CNN} & \xmark & \cmark$^{\flat}$ & \xmark & \xmark & Classification & Finance & \xmark\\
\cite{barra2020deep} & UTS & GAF & \xmark & \texttt{CNN} & \xmark & \cmark$^{\flat}$ & \xmark & \xmark & Classification & Finance & \xmark\\
% \cite{barra2020deep} & UTS & GAF & \xmark & \texttt{VGG-16} & \xmark & \cmark$^{\flat}$ & \xmark & \xmark & Classification & Finance & \xmark\\
% \cite{cao2021image} & UTS & RP & \xmark & \texttt{CNN} & \xmark & \cmark$^{\flat}$ & \xmark & \xmark & Classification & General & \xmark\\
VisualAE \cite{sood2021visual} & UTS & LinePlot & \xmark & \texttt{CNN} & \xmark & \cmark$^{\flat}$ & \xmark & \cmark & Forecasting & Finance & \xmark\\
% VisualAE \cite{sood2021visual} & UTS & LinePlot & \xmark & \texttt{CNN} & \xmark & \cmark$^{\flat}$ & \xmark & \xmark & Img-Generation & Finance & \xmark\\
\cite{zeng2021deep} & MTS & Heatmap & \xmark & \texttt{CNN,LSTM} & \xmark & \cmark$^{\flat}$ & \xmark & \cmark & Forecasting & Finance & \xmark\\
% \cite{zeng2021deep} & MTS & Heatmap & \xmark & \texttt{SRVP} & \xmark & \cmark$^{\flat}$ & \xmark & \cmark & Forecasting & Finance & \xmark\\
AST \cite{gong2021ast} & UTS & Spectrogram & \xmark & \texttt{DeiT} & \cmark & \cmark & \xmark & \xmark & Classification & Audio & \cmark\textsuperscript{\href{https://github.com/YuanGongND/ast}{[4]}}\\
TTS-GAN \cite{li2022tts} & MTS & Heatmap & \xmark & \texttt{ViT} & \xmark & \cmark$^{\flat}$ & \xmark & \cmark & Ts-Generation & Health & \cmark\textsuperscript{\href{https://github.com/imics-lab/tts-gan}{[5]}}\\
SSAST \cite{gong2022ssast} & UTS & Spectrogram & \xmark & \texttt{ViT} & \cmark$^{\natural}$ & \cmark & \xmark & \xmark & Classification & Audio & \cmark\textsuperscript{\href{https://github.com/YuanGongND/ssast}{[6]}}\\
MAE-AST \cite{baade2022mae} & UTS & Spectrogram & \xmark & \texttt{MAE} & \cmark$^{\natural}$ & \cmark & \xmark & \xmark & Classification & Audio & \cmark\textsuperscript{\href{https://github.com/AlanBaade/MAE-AST-Public}{[7]}}\\
AST-SED \cite{li2023ast} & UTS & Spectrogram & \xmark & \texttt{SSAST,GRU} & \cmark & \cmark & \xmark & \xmark & EventDetection & Audio & \xmark\\
\cite{jin2023classification} & UTS & %Multiple
LinePlot & \xmark & \texttt{CNN} & \cmark & \cmark & \xmark & \xmark & Classification & Physics & \xmark\\
ForCNN \cite{semenoglou2023image} & UTS & LinePlot & \xmark & \texttt{CNN} & \xmark & \cmark$^{\flat}$ & \xmark & \xmark & Forecasting & General & \xmark\\
Vit-num-spec \cite{zeng2023pixels} & UTS & Spectrogram & \xmark & \texttt{ViT} & \xmark & \cmark$^{\flat}$ & \xmark & \xmark & Forecasting & Finance & \xmark\\
% \cite{wimmer2023leveraging} & MTS & LinePlot & \xmark & \texttt{CLIP,LSTM} & \cmark & \cmark & \xmark & \xmark & Classification & Finance & \xmark\\
ViTST \cite{li2023time} & MTS & LinePlot & \xmark & \texttt{Swin} & \cmark & \cmark & \xmark & \xmark & Classification & General & \cmark\textsuperscript{\href{https://github.com/Leezekun/ViTST}{[8]}}\\
MV-DTSA \cite{yang2023your} & UTS\textsuperscript{*} & LinePlot & \xmark & \texttt{CNN} & \xmark & \cmark$^{\flat}$ & \xmark & \cmark & Forecasting & General & \cmark\textsuperscript{\href{https://github.com/IkeYang/machine-vision-assisted-deep-time-series-analysis-MV-DTSA-}{[9]}}\\
TimesNet \cite{wu2023timesnet} & MTS & Heatmap & \xmark & \texttt{CNN} & \xmark & \cmark$^{\flat}$ & \xmark & \cmark & Multiple & General & \cmark\textsuperscript{\href{https://github.com/thuml/TimesNet}{[10]}}\\
ITF-TAD \cite{namura2024training} & UTS & Spectrogram & \xmark & \texttt{CNN} & \cmark & \xmark & \xmark & \xmark & Anomaly & General & \xmark\\
\cite{kaewrakmuk2024multi} & UTS & GAF & \xmark & \texttt{CNN} & \cmark & \cmark & \xmark & \xmark & Classification & Sensing & \xmark\\
HCR-AdaAD \cite{lin2024hierarchical} & MTS & RP & \xmark & \texttt{CNN,GNN} & \xmark & \cmark$^{\flat}$ & \xmark & \xmark & Anomaly & General & \xmark\\
FIRTS \cite{costa2024fusion} & UTS & Other ($\S$\ref{sec.othermethod}) & \xmark & \texttt{CNN} & \xmark & \cmark$^{\flat}$ & \xmark & \xmark & Classification & General & \cmark\textsuperscript{\href{https://sites.google.com/view/firts-paper}{[11]}}\\
% \multirow{2}{*}{FIRTS \cite{costa2024fusion}} & \multirow{2}{*}{UTS} & Spectrogram & \multirow{2}{*}{\xmark} & \multirow{2}{*}{\texttt{CNN}} & \multirow{2}{*}{\xmark} & \multirow{2}{*}{\cmark$^{\flat}$} & \multirow{2}{*}{\xmark} & \multirow{2}{*}{\xmark} & \multirow{2}{*}{Classification} & \multirow{2}{*}{General} & \multirow{2}{*}{\cmark\textsuperscript{\href{https://sites.google.com/view/firts-paper}{[2]}}}\\
%  & & \& GAF,RP,MTF & & & & & & & & & \\
% \cite{homenda2024time} & UTS\textsuperscript{*} & Multiple & \xmark & \texttt{CNN} & \xmark & \cmark$^{\flat}$ & \xmark & \xmark & Classification & General & \xmark\\
CAFO \cite{kim2024cafo} & MTS & RP & \xmark & \texttt{CNN,ViT} & \xmark & \cmark$^{\flat}$ & \xmark & \xmark & Explanation & General & \cmark\textsuperscript{\href{https://github.com/eai-lab/CAFO}{[12]}}\\
% \multirow{2}{*}{CAFO \cite{kim2024cafo}} & \multirow{2}{*}{MTS} & \multirow{2}{*}{RP} & \multirow{2}{*}{\xmark} & \texttt{ShuffleNet,ResNet} & \multirow{2}{*}{\cmark} & \multirow{2}{*}{\cmark} & \multirow{2}{*}{\xmark} & \multirow{2}{*}{\xmark} & Classification & \multirow{2}{*}{General} & \multirow{2}{*}{\cmark}\\
%  & & & & \texttt{MLP-Mixer,ViT} & & & & & \& Explanation & & \\
ViTime \cite{yang2024vitime} & UTS\textsuperscript{*} & LinePlot & \xmark & \texttt{ViT} & \cmark$^{\natural}$ & \cmark & \xmark & \cmark & Forecasting & General & \cmark\textsuperscript{\href{https://github.com/IkeYang/ViTime}{[13]}}\\
ImagenTime \cite{naiman2024utilizing} & MTS & Other ($\S$\ref{sec.othermethod}) & \xmark & %\texttt{Diffusion}
\texttt{CNN} & \xmark & \cmark$^{\flat}$ & \xmark & \cmark & Ts-Generation & General & \cmark\textsuperscript{\href{https://github.com/azencot-group/ImagenTime}{[14]}}\\
TimEHR \cite{karami2024timehr} & MTS & Heatmap & \xmark & \texttt{CNN} & \xmark & \cmark$^{\flat}$ & \xmark & \cmark & Ts-Generation & Health & \cmark\textsuperscript{\href{https://github.com/esl-epfl/TimEHR}{[15]}}\\
VisionTS \cite{chen2024visionts} & UTS\textsuperscript{*} & Heatmap & \xmark & \texttt{MAE} & \cmark & \cmark & \xmark & \cmark & Forecasting & General & \cmark\textsuperscript{\href{https://github.com/Keytoyze/VisionTS}{[16]}}\\ \midrule
InsightMiner \cite{zhang2023insight} & UTS & LinePlot & \cmark & \texttt{LLaVA} & \cmark & \cmark & \cmark & \xmark & Txt-Generation & General & \xmark\\
\cite{wimmer2023leveraging} & MTS & LinePlot & \cmark & \texttt{CLIP,LSTM} & \cmark & \cmark & \xmark & \xmark & Classification & Finance & \xmark\\
% \cite{dixit2024vision} & UTS & Spectrogram & \cmark & \texttt{GPT4o,Gemini} & \cmark & \xmark & \cmark & \xmark & Classification & Audio & \xmark\\
\multirow{2}{*}{\cite{dixit2024vision}} & \multirow{2}{*}{UTS} & \multirow{2}{*}{Spectrogram} & \multirow{2}{*}{\cmark} & \texttt{GPT4o,Gemini} & \multirow{2}{*}{\cmark} & \multirow{2}{*}{\xmark} & \multirow{2}{*}{\cmark} & \multirow{2}{*}{\xmark} & \multirow{2}{*}{Classification} & \multirow{2}{*}{Audio} & \multirow{2}{*}{\xmark}\\
 & & & & \& \texttt{Claude3} & & & & & & & \\
\cite{daswani2024plots} & MTS & LinePlot & \cmark & \texttt{GPT4o,Gemini} & \cmark & \xmark & \cmark & \xmark & Multiple & General & \xmark\\
TAMA \cite{zhuang2024see} & UTS & LinePlot & \cmark & \texttt{GPT4o} & \cmark & \xmark & \cmark & \xmark & Anomaly & General & \xmark\\
\cite{prithyani2024feasibility} & MTS & LinePlot & \cmark & \texttt{LLaVA} & \cmark & \cmark & \cmark & \xmark & Classification & General & \cmark\textsuperscript{\href{https://github.com/vinayp17/VLM_TSC}{[17]}}\\
\bottomrule[1pt]
\end{tabular}}
\vspace{-0.25cm}
\caption{Taxonomy of vision models on time series. The top panel includes single-modal models. The bottom panel includes multi-modal models. {\bf TS-Type} denotes type of time series. {\bf TS-Recover} denotes %whether time series recovery ($\S$\ref{sec.processing}) has been performed.
recovering time series from predicted images ($\S$\ref{sec.processing}). \textsuperscript{*}: %the model has been %applied on MTSs by %processing %modeling the individual UTSs of each MTS.
the method has been used to model the individual UTSs of an MTS. $^{\natural}$: a new pre-trained model was proposed in the work. $^{\flat}$: %without using a pre-trained model, fine-tune means training from scratch.
when pre-trained models were unused, ``Fine-tune'' refers to train a task-specific model from scratch. %In the
{\bf Model} column: \texttt{CNN} could be regular CNN, ResNet, VGG-Net, %U-Net,
{\em etc.}}\label{tab.taxonomy}
% The code only include verified official code from the authors.
\vspace{-0.3cm}
\end{table*}

\begin{table*}[t]
\centering
\small
\setlength{\tabcolsep}{2.9pt}{
\begin{tabular}{l|l|l|l}\hline
% \toprule[1pt]
\rowcolor{gray!20}
{\bf Method} & {\bf TS-Type} & {\bf Advantages} & {\bf Limitations}\\ \hline
Line Plot ($\S$\ref{sec.lineplot}) & UTS, MTS & matches human perception of time series & limited to MTSs with a small number of variates\\ \hline
Heatmap ($\S$\ref{sec.heatmap}) & UTS, MTS & straightforward for both UTSs and MTSs & the order of variates may affect their correlation learning\\ \hline
Spectrogram ($\S$\ref{sec.spectrogram}) & UTS & encodes the time-frequency space & limited to UTSs; needs a proper choice of window/wavelet\\ \hline
GAF ($\S$\ref{sec.gaf}) & UTS & encodes the temporal correlations in a UTS & limited to UTSs; $O(T^{2})$ time and space complexity\\ \hline% for long time series\\ \hline
% RP ($\S$\ref{sec.rp}) & UTS & flexibility in image size by tuning $m$ and $\tau$ & limited to UTSs; the pattern has a threshold-dependency\\ \hline
RP ($\S$\ref{sec.rp}) & UTS & flexibility in image size by tuning $m$ and $\tau$ & limited to UTSs; information loss after thresholding\\ \hline
% \bottomrule[1pt]
\end{tabular}}
\vspace{-0.2cm}
\caption{Summary of the five primary methods for transforming time series to images. {\bf TS-Type} denotes type of time series.}\label{tab.tsimage}
\vspace{-0.2cm}
\end{table*}

\section{Time Series To Image Transformation}\label{sec.tsimage}

% This section summarizes 5 major methods for imaging time series ($\S$\ref{sec.lineplot}-$\S$\ref{sec.rp}). We also discuss some other methods ($\S$\ref{sec.othermethod}) and how to model MTS with these methods ($\S$\ref{sec.modelmts}).
This section summarizes the methods for imaging time series ($\S$\ref{sec.lineplot}-$\S$\ref{sec.othermethod}) and their extensions to encode MTSs ($\S$\ref{sec.modelmts}).

% This section summarizes 5 major methods for transforming time series to images, including Line Plot, Heatmap, Spetrogram, GAF and RP, and several minor methods. We discuss their pros and cons and how to deal with MTS.

% This section discusses the advantages and limitations of different methods for time series to image transformation (invertible, efficiency, information preservation, MTS, long-range time series, parametric, etc.).

%\subsection{Methods}

\vspace{-0.08cm}

\subsection{Line Plot}\label{sec.lineplot}

Line Plot is a straightforward way for visualizing UTSs for human analysis ({\em e.g.}, stocks, power consumption, {\em etc.}). As illustrated by Fig. \ref{fig.tsimage}(a), the simplest approach is to draw a 2D image with x-axis representing %the time horizon
time steps and y-axis representing %the magnitude of the normalized time series.
time-wise values, %A line is used to connect all values of the series over time.
with a line connecting all values of the series over time. This image can be %represented by either three-channel pixels or single-channel pixels
either three-channel ({\em i.e.}, RGB) or single-channel as the colors may not %provide additional information
be informative %\cite{cohen2020trading,sood2021visual,jin2023classification,zhang2023insight,zhuang2024see}.
\cite{cohen2020trading,sood2021visual,jin2023classification,zhang2023insight}. ForCNN \cite{semenoglou2023image} even uses a single 8-bit integer to represent each pixel for black-white images. So far, there is no consensus on whether other graphical components, such as legend, grids and tick labels, could provide extra benefits in any task. For example, ViTST \cite{li2023time} finds these components are superfluous in a classification task, while TAMA \cite{zhuang2024see} finds grid-like auxiliary lines help enhance anomaly detection.

In addition to the regular Line Plot, MV-DTSA \cite{yang2023your} and ViTime \cite{yang2024vitime} divide an image into $h\times L$ grids, %where $h$ is the number of rows and $L$ is the number of columns,
and %introduced
define a function to map each time step of a UTS to a grid, producing a grid-like Line Plot. Also, we include methods that use Scatter Plot \cite{daswani2024plots,prithyani2024feasibility} in this category because %the only difference between a Scatter Plot and a Line Plot is whether the time-wise values are connected by lines.
a Scatter Plot resembles a Line Plot but doesn't connect %time-wise values
data points with a line. By comparing them, \cite{prithyani2024feasibility} finds a Line Plot could induce better time series classification.

For MTSs, we defer the discussion on Line Plot to $\S$\ref{sec.modelmts}.

% For MTS, some methods use the channel-independence assumption proposed in \cite{nie2023time} and represent each variate in MTS with an individual Line Plot \cite{yang2023your,yang2024vitime}. ViTST \cite{li2023time} also uses an individual Line Plot per variate, but colors different lines and assembles all plots to form a bigger image. The method in \cite{wimmer2023leveraging} plots %the time series of
% all variates in a single Line Plot and distinguish them by %use different
% types of lines ({\em e.g.}, solid, dashed, dotted, {\em etc.}). %to distinguish them.
% However, these methods only work for a small number of variates. For example, in \cite{wimmer2023leveraging}, there are only 4 variates in its financial MTSs.

%\cite{li2023time} space-costly because of blank pixels. scatter plot.

%Invertible with a numeric prediction head \cite{sood2021visual}. It fits tasks such as forecasting, imputation, etc.

\vspace{-0.08cm}

\subsection{Heatmap}\label{sec.heatmap}

As shown in Fig. \ref{fig.tsimage}(b), Heatmap is a 2D visualization of the magnitude of the values in a matrix using color. %The variation of color represents the intensity of each value. %Therefore,
It has been used to %directly
represent the matrix of an MTS, {\em i.e.}, $\mat{X} \in \mathbb{R}^{d\times T}$, as a one-channel $d\times T$ image \cite{li2022tts,yazdanbakhsh2019multivariate}. Similarly, TimEHR \cite{karami2024timehr} represents an {\em irregular} MTS, where the intervals between time steps are uneven, as a $d\times H$ Heatmap image by grouping the uneven time steps into $H$ even time bins. In \cite{zeng2021deep}, a different method is used for visualizing a 9-variate financial %time series.
MTS. It reshapes the 9 variates at each time step to a $3\times 3$ Heatmap image, and uses the sequence of images to forecast future %image
frames, achieving %time series
%MTS
time series forecasting. In contrast, VisionTS \cite{chen2024visionts} uses Heatmap to visualize UTSs. %instead.
Similar to TimesNet \cite{wu2023timesnet}, it first segments a length-$T$ UTS into $\lfloor T/P\rfloor$ length-$P$ subsequences, where $P$ is a parameter representing a periodicity of the UTS. Then the subsequences are stacked into a $P\times \lfloor T/P\rfloor$ matrix, %and duplicated 3 times to produce a 3-channel
with 3 duplicated channels, to produce a grayscale image %which serves as an
input to %a vision foundation model.
an LVM. To encode MTSs, VisionTS adopts the channel independence assumption \cite{nie2023time} and individually models each variate in an MTS.

\vspace{0.2cm}

\noindent{\bf Remark.} Heatmap can be used to visualize matrices of various forms. It is also used for matrices generated by the subsequent methods ({\em e.g.}, Spectrogram, GAF, RP) in this section. In this paper, the name Heatmap refers specifically to images that use color to visualize the (normalized) values in UTS $\mat{x}$ or MTS $\mat{X}$ without performing other transformations.

%\cite{chen2024visionts,karami2024timehr} bin version of TSH \cite{karami2024timehr}, DE and STFT \cite{naiman2024utilizing} (DE can be used for constructing RP), rearrange variates for video version of TSH \cite{zeng2021deep}.

%\vspace{0.2cm}

\subsection{Spectrogram}\label{sec.spectrogram}

A {\em spectrogram} is a visual representation of the spectrum of frequencies of a signal as it varies with time, which are extensively used for analyzing audio signals \cite{gong2021ast}. Since audio signals are a type of UTS, spectrogram can be considered as a method for imaging a UTS. As shown in Fig. \ref{fig.tsimage}(c), a common format is a 2D heatmap image with x-axis representing time steps and y-axis representing frequency, {\em a.k.a.} a time-frequency space. %The color at each point
Each pixel in the image represents the (logarithmic) amplitude of a specific frequency at a specific time point. Typical methods for %transforming a UTS to
producing a spectrogram include {\bf Short-Time Fourier Transform (STFT)} \cite{griffin1984signal}, {\bf Wavelet Transform} \cite{daubechies1990wavelet}, and {\bf Filterbank} \cite{vetterli1992wavelets}.

\vspace{0.2cm}

\noindent{\bf STFT.} %Discrete Fourier transform (DFT) can be used to represent a UTS signal %$\mat{x}=[x_{1}, ..., x_{T}]$
%$\mat{x}\in\mathbb{R}^{1\times T}$ as a sum of sinusoidal components. The output of the transform is a function of frequency $f(w)$, describing the intensity of each constituent frequency $w$ of the entire UTS. 
Discrete Fourier transform (DFT) can be used to describe the intensity $f(w)$ of each constituent frequency $w$ of a UTS signal $\mat{x}\in\mathbb{R}^{1\times T}$. However, $f(w)$ has no time dependency. It cannot provide dynamic information such as when a specific frequency appear in the UTS. STFT addresses this deficiency by sliding a window function $g(t)$ over the time steps in %the UTS,
$\mat{x}$, and computing the DFT within each window by
\begin{equation}\label{eq.stft}
\small
\begin{aligned}
f(w,\tau) = \sum_{t=1}^{T}x_{t}g(t - \tau)e^{-iwt}
\end{aligned}
\end{equation}
where $w$ is frequency, $\tau$ is the position of the window, $f(w,\tau)$ describes the intensity of frequency $w$ at time step $\tau$.

%With a proper selection of the
By selecting a proper window function $g(\cdot)$ ({\em e.g.}, Gaussian/Hamming/Bartlett window), %({\em e.g.}, Gaussian window, Hamming window, Bartlett window), %{\em etc.}),
a 2D spectrogram ({\em e.g.}, Fig. \ref{fig.tsimage}(c)) can be drawn via a heatmap on the squared values $|f(w,\tau)|^{2}$, with $w$ as the y-axis, and $\tau$ as the x-axis. For example, \cite{dixit2024vision} uses STFT based spectrogram as an input to LMMs %\hh{do you mean LVMs? check}
for time series classification.

%Fourier transform is a powerful data analysis tool that represents any complex signal as a sum of sines and cosines and transforms the signal from the time domain to the frequency domain. However, Fourier transform can only show which frequencies are present in the signal, but not when these frequencies appear. The STFT divides original signal into several parts using a sliding window to fix this problem. STFT involves a sliding window for extracting frequency components within the window.

\vspace{0.2cm}

\noindent{\bf Wavelet Transform.} %Like Fourier transform, %\hh{this paragraph needs a citation}
Continuous Wavelet Transform (CWT) uses the inner product to measure the similarity between a signal function $x(t)$ and an analyzing function. %In STFT (Eq.~\eqref{eq.stft}), the analyzing function is a windowed exponential $g(t - \tau)e^{-iwt}$.
%In CWT,
The analyzing function is a {\em wavelet} $\psi(t)$, where the typical choices include Morse wavelet, Morlet wavelet, %Daubechies wavelet, %Beylkin wavelet, 
{\em etc.} %The
CWT compares $x(t)$ to the shifted and scaled ({\em i.e.}, stretched or shrunk) versions of the wavelet, and output a CWT coefficient by
\begin{equation}\label{eq.cwt}
\small
\begin{aligned}
c(s,\tau) = \int_{-\infty}^{\infty}x(t)\frac{1}{s}\psi^{*}(\frac{t - \tau}{s})dt
\end{aligned}
\end{equation}
where $*$ denotes complex conjugate, $\tau$ is the time step to shift, and $s$ represents the scale. In practice, a discretized version of CWT in Eq.~\eqref{eq.cwt} is implemented for UTS $[x_{1}, ..., x_{T}]$.

It is noteworthy that the scale $s$ controls the frequency encoded in a wavelet -- a larger $s$ leads to a stretched wavelet with a lower frequency, and vice versa. As such, by varying $s$ and $\tau$, a 2D spectrogram ({\em e.g.}, Fig. \ref{fig.tsimage}(d)) can be drawn %, often with a heatmap
on $|c(s,\tau)|$, where $s$ is the y-axis and $\tau$ is the x-axis. Compared to STFT, which uses a fixed window size, Wavelet Transform allows variable wavelet sizes -- a larger size %region
for more precise low frequency information. 
%Usually, $s$ and $\tau$ vary dependently -- a larger $s$ leads to a stretched wavelet that shifts slowly, {\em i.e.}, a smaller $\tau$. This property %of CWT
%yields a spectrogram that balances the resolutions of frequency %$s$
%and time, %$\tau$,
%which is an advantage over the fixed time resolution in STFT.
% Thus, both of the methods in %\cite{du2020image}
% \cite{namura2024training} and \cite{zeng2023pixels} choose CWT (with Morlet wavelet) to generate the spectrogram.
Thus, the methods in \cite{du2020image,namura2024training,zeng2023pixels} choose CWT (with Morlet wavelet) to generate the spectrogram.

%A wavelet is a wave-like oscillation that has zero mean and is localized in both time and frequency space.

\vspace{0.2cm}

\noindent{\bf Filterbank.} This method %is relevant to
resembles STFT and is often used in processing audio signals. Given an audio signal, it firstly goes through a {\em pre-emphasis filter} to boost high frequencies, which helps improve the clarity of the signal. Then, STFT is applied on the signal. %with a sliding window $g(t)$ of size $k$ that shifts in a fixed stride $\tau$. %where the adjacent windows may overlap in $k$ time length.
%Finally, filterbank features are computed by applying multiple ``triangle-shaped'' filters spaced on the Mel-scale to the STFT output $f(w, \tau)$. %where Mel-scale is a method to make the filters more discriminative on lower frequencies, %than higher frequencies,
%imitating the non-linear human ear perception of sound.
Finally, multiple ``triangle-shaped'' filters spaced on a Mel-scale are applied to the STFT power spectrum $|f(w, \tau)|^{2}$ to extract frequency bands. The outcome filterbank features $\hat{f}(w, \tau)$ can be used to yield a spectrogram with $w$ as the y-axis, and $\tau$ as the x-axis.

%Filterbank was introduced in AST \cite{gong2021ast} with %$k$=25ms
Filterbank was adopted in AST \cite{gong2021ast} with 
a 25ms Hamming window that shifts every 10ms for classifying audio signals using Vision Transformer (ViT). It then becomes widely used in the follow-up works such as SSAST \cite{gong2022ssast}, MAE-AST \cite{baade2022mae}, and AST-SED \cite{li2023ast}, as summarized in Table \ref{tab.taxonomy}.



%Use MLP to predict TS directly \cite{zeng2023pixels}.

%\vspace{0.2cm}

% \vspace{0.2cm}

\subsection{Gramian Angular Field (GAF)}\label{sec.gaf}

GAF was introduced for classifying UTSs using CNNs %using %image based CNNs
by \cite{wang2015encoding}. It was then extended %with an extension
to an imputation task in \cite{wang2015imaging}. Similarly, \cite{barra2020deep} applied GAF for financial time series forecasting.

Given a UTS $\mat{x}\in\mathbb{R}^{1\times T}$, %$[x_{1}, ..., x_{T}]$,
the first step %before GAF
is to rescale each $x_{t}$ to a value $\tilde{x}_{t}$ %in the interval of
within $[0, 1]$ (or $[-1, 1]$). %by min-max normalization.
This range enables mapping $\tilde{x}_{t}$ to polar coordinates by $\phi_{t}=\text{arccos}(\tilde{x}_{i})$, with a radius $r=t/N$ encoding the time stamp, where $N$ is a constant factor to regularize the span of the polar coordinates. %system. Then,
Two types of GAF, Gramian Sum Angular Field (GASF) and Gramian Difference Angular Field (GADF) are defined as
\begin{equation}\label{eq.gaf}
\small
\begin{aligned}
&\text{GASF:}~~\text{cos}(\phi_{t} + \phi_{t'})=x_{t}x_{t'} - \sqrt{1 - x_{t}^{2}}\sqrt{1 - x_{t'}^{2}}\\
&\text{GADF:}~~\text{sin}(\phi_{t} - \phi_{t'})=x_{t'}\sqrt{1 - x_{t}^{2}} - x_{t}\sqrt{1 - x_{t'}^{2}}
\end{aligned}
\end{equation}
which exploits the pairwise temporal correlations in the UTS. Thus, the outcome is a $T\times T$ matrix $\mat{G}$ with $\mat{G}_{t,t'}$ specified by either type in Eq.~\eqref{eq.gaf}. A GAF image is a heatmap on $\mat{G}$ with both axes representing time, as illustrated by Fig. \ref{fig.tsimage}(e).

% Invertible.

% \vspace{0.2cm}

\subsection{Recurrence Plot (RP)}\label{sec.rp}

%RP \cite{eckmann1987recurrence} is a method to encode a UTS into an image that aims to capture the periodic patterns in the UTS by using its reconstructed {\em phase space}. The phase space of a UTS $[x_{1}, ..., x_{T}]$ can be reconstructed by {\em time delay embedding}, which is a set of new vectors $\mat{v}_{1}$, ..., $\mat{v}_{l}$ with

RP \cite{eckmann1987recurrence} encodes a UTS into an image that captures its periodic patterns by using its reconstructed {\em phase space}. The phase space of %a UTS %$[x_{1}, ..., x_{T}]$
$\mat{x}\in\mathbb{R}^{1\times T}$ can be reconstructed by {\em time delay embedding} -- a set of new vectors $\mat{v}_{1}$, ..., $\mat{v}_{l}$ with
\begin{equation}\label{eq.de}
\small
\begin{aligned}
\mat{v}_{t}=[x_{t}, x_{t+\tau}, x_{t+2\tau}, ..., x_{t+(m-1)\tau}]\in\mathbb{R}^{m\tau},~~~1\le t \le l
\end{aligned}
\end{equation}
where $\tau$ is the time delay, $m$ is the dimension of the phase space, both %of which
are hyperparameters. Hence, $l=T-(m-1)\tau$. With vectors $\mat{v}_{1}$, ..., $\mat{v}_{l}$, an RP image %is constructed by measuring
measures their pairwise distances, results in an $l\times l$ image whose element
\begin{equation}\label{eq.rp}
\small
\begin{aligned}
\text{RP}_{i,j}=\Theta(\varepsilon - \|\mat{v}_{i} - \mat{v}_{j}\|),~~~1\le i,j\le l
\end{aligned}
\end{equation}
where $\Theta(\cdot)$ is the Heaviside step function, $\varepsilon$ is a threshold, and $\|\cdot\|$ is a norm function such as $\ell_{2}$ norm. Eq.~\eqref{eq.rp} %states RP produces a heatmap image on a binary matrix with $\text{RP}_{i,j}=1$ if $\mat{v}_{i}$ and $\mat{v}_{j}$ are sufficiently similar.
generates a binary matrix with $\text{RP}_{i,j}=1$ if $\mat{v}_{i}$ and $\mat{v}_{j}$ are sufficiently similar, producing a black-white image ({\em e.g.}, Fig. \ref{fig.tsimage}(f)).% ({\em e.g.}, a periodic pattern).

An advantage of RP is its flexibility in image size by tuning $m$ and $\tau$. Thus it has been used for time series classification %\cite{cao2021image},
\cite{silva2013time,hatami2018classification}, forecasting \cite{li2020forecasting}, anomaly detection \cite{lin2024hierarchical} and %feature-wise
explanation \cite{kim2024cafo}. Moreover, the method in \cite{hatami2018classification}, and similarly in HCR-AdaAD \cite{lin2024hierarchical}, omit the thresholding in Eq.~\eqref{eq.rp} and uses $\|\mat{v}_{i} - \mat{v}_{j}\|$ to produce continuously valued images %in a classification task
to avoid information loss.


% \vspace{0.2cm}

\subsection{Other Methods}\label{sec.othermethod}

%There are some less commonly used methods. For example, in
Additionally, %there are some peripheral methods. %In addition to GAF,
\cite{wang2015encoding} introduces Markov Transition Field (MTF) for imaging a UTS. %$\mat{x}\in\mathbb{R}^{1\times T}$. 
%MTF first assigns each $x_{t}$ to one of $Q$ quantile bins, then builds a $Q\times Q$ Markov transition matrix $\mat{M}$ {\em s.t.} $\mat{M}_{i,j}$ represents the frequency %with which
%of the case when a point $x_{t}$ in the $i$-th bin is followed by a point $x_{t'}$ in the $j$-th bin, {\em i.e.}, $t=t'+1$. Matrix $\mat{M}$ serves as the input of a heatmap image.
MTF is a matrix $\mat{M}\in\mathbb{R}^{Q\times Q}$ encoding the transition probabilities over time segments, where $Q$ is the number of segments. %Moreover,
ImagenTime \cite{naiman2024utilizing} stacks the delay embeddings $\mat{v}_{1}$, ..., $\mat{v}_{l}$ in Eq.~\eqref{eq.de} to an $l\times m\tau$ matrix for visualizing UTSs. %It also uses a variant of STFT.
% The method in \cite{homenda2024time} introduces five different 2D images by counting, rearranging, replicating the values in a UTS. 
MSCRED \cite{zhang2019deep} uses heatmaps on the $d\times d$ correlation matrices of MTSs with $d$ variates for anomaly detection. 
Furthermore, some methods use a mixture of imaging methods by stacking different transformations. \cite{wang2015imaging} stacks GASF, GADF, MTF to a 3-channel image. %Similarly,
FIRTS \cite{costa2024fusion} builds a 3-channel image by stacking GASF, MTF and RP. %the GASF, MTF, RP representations of each UTS.
%\cite{jin2023classification} combines Line Plot with Constant-Q Transform (CQT) \cite{brown1991calculation}, a method related to wavelet transform ($\S$\ref{sec.spectrogram}), to generate 2-channel images.
The mixture methods encode a UTS with multiple views and were found more robust than single-view images in these works for %time series
classification tasks.

\subsection{How to Model MTS}\label{sec.modelmts}

In the above methods, Heatmap ($\S$\ref{sec.heatmap}) can be %directly
used to visualize the %2D
variate-time matrices, $\mat{X}$, of MTSs ({\em e.g.}, Fig. \ref{fig.structure}(b)), where correlated variates %are better to
should be spatially close to each other. Line Plot ($\S$\ref{sec.lineplot}) can be used to visualize MTSs by plotting all variates in the same image \cite{wimmer2023leveraging,daswani2024plots} or combining all univariate images to compose a bigger %1-channel
image \cite {li2023time}, but these methods only work for a small number of variates. Spectrogram ($\S$\ref{sec.spectrogram}), GAF ($\S$\ref{sec.gaf}), and RP ($\S$\ref{sec.rp}) were designed specifically for UTSs. For these methods and Line Plot, which are not straightforward %for MTS transformation,
in imaging MTSs, the general approaches %to use them %for MTS
include using channel independence assumption to model each variate individually \cite{nie2023time}, %like VisionTS \cite{chen2024visionts},
or stacking the images of $d$ variates to form a $d$-channel image %as did by
\cite{naiman2024utilizing,kim2024cafo}. %\cite{prithyani2024feasibility,naiman2024utilizing,kim2024cafo}.
However, the latter does not fit some vision models pre-trained on RGB images which requires 3-channel inputs (more discussions are deferred to $\S$\ref{sec.processing}).

\vspace{0.2cm}

\noindent{\bf Remark.} As a summary, Table \ref{tab.tsimage} recaps the salient advantages and limitations of the five primary imaging methods that are introduced in this section.

% \hh{can we have a table (e.g., rows are different imaging methods and columns are a few desirable propoerties) or a short paragraph to discuss/summarize/compare the strenths and weakness of different imaging methods for ts? This might bring some structure/comprehension to this section (as opposed to, e.g., some reviewer might complain that what we do here is a laundry list)}

\section{Imaged Time Series Modeling}\label{sec.model}

With image representations, time series analysis can be readily performed with vision models. This section discusses such solutions from %traditional vision models %($\S$\ref{sec.cnns})
%to the recent large vision models %($\S$\ref{sec.lvms})
%and large multimodal models.% ($\S$\ref{sec.lmms}).
the traditional models to the SOTA models.

\begin{figure*}[!t]
\centering
\includegraphics[width=0.9\textwidth]{fig/fig_2.pdf}
% \vspace{-1em}
\caption{An illustration of different modeling strategies on imaged time series in (a)(b)(c) and task-specific heads in (d).}\label{fig.models}
\vspace{-0.2cm}
\end{figure*}

\subsection{Conventional Vision Models}\label{sec.cnns}

%Similar to
Following traditional %methods on
image classification, \cite{silva2013time} applies a K-NN classifier on the RPs of time series, \cite{cohen2020trading} applies an ensemble of fundamental classifiers such as %linear regression, SVM, Ada Boost, {\em etc.}
SVM and AdaBoost on the Line Plots %images
for time series classification. As an image encoder, %a typical encoder, %of images,
CNNs have been %extensively
widely used for learning image representations. %\cite{he2016deep}.
Different from using 1D CNNs on sequences %UTS or MTS
\cite{bai2018empirical}, %regular
2D or 3D CNNs can be applied on imaged time series as shown in Fig. \ref{fig.models}(a). %to learn time series representations by encoding their image transformations.
For example, %standard
regular CNNs have been used on Spectrograms \cite{du2020image}, tiled CNNs have been used on GAF images \cite{wang2015encoding,wang2015imaging}, dilated CNNs have been used on Heatmap images \cite{yazdanbakhsh2019multivariate}. More frequently, ResNet \cite{he2016deep}, Inception-v1 \cite{szegedy2015going}, and VGG-Net \cite{simonyan2014very} have been used on Line Plots \cite{jin2023classification,semenoglou2023image}, Heatmap images \cite{zeng2021deep}, RP images \cite{li2020forecasting,kim2024cafo}, GAF images \cite{barra2020deep,kaewrakmuk2024multi}, 
% Heatmaps \cite{zeng2021deep}, RPs \cite{li2020forecasting,kim2024cafo}, GAFs \cite{barra2020deep,kaewrakmuk2024multi},
and even a mixture of GAF, MTF and RP images \cite{costa2024fusion}. In particular, for time series generation tasks, %a diffusion model with U-Nets \cite{naiman2024utilizing} and GAN frameworks of CNNs \cite{li2022tts,karami2024timehr} have also been explored.%investigated.
GAN frameworks of CNNs \cite{li2022tts,karami2024timehr} and a diffusion model with U-Nets \cite{naiman2024utilizing} have also been explored.

Due to their small to medium sizes, these models are often trained from scratch using task-specific training data. %per task using the task's training set. %of time series images.
Meanwhile, fine-tuning {\em pre-trained vision models}  %such as those pre-trained on ImageNet, %\cite{deng2009imagenet}, 
have already been found promising in cross-modality knowledge transfer for time series anomaly detection \cite{namura2024training}, forecasting \cite{li2020forecasting} and classification \cite{jin2023classification}.

% \cite{li2020forecasting} uses ImageNet pretrained CNNs.

\subsection{Large Vision Models (LVMs)}\label{sec.lvms}

Vision Transformer (ViT) \cite{dosovitskiy2021image} has %given birth to
inspired the development of %some
modern LVMs %large vision models (LVMs)
such as %DeiT \cite{touvron2021training}, 
Swin \cite{liu2021swin}, BEiT \cite{bao2022beit}, and MAE \cite{he2022masked}. %Given an input image, ViT splits it
As Fig. \ref{fig.models}(b) shows, ViT splits an %input
image into {\em patches} of fixed size, then embeds each patch and augments it with a positional embedding. The %resulting
vectors of patches are processed by a Transformer %encoder
as if they were token embeddings. Compared to CNNs, ViTs are less data-efficient, but have higher capacity. %Consequently,
Thus, %the
{\em pre-trained} ViTs have been explored for modeling %the images of time series.
imaged time series. For example, AST \cite{gong2021ast} fine-tunes DeiT \cite{touvron2021training} on the filterbank spetrogram of audios %signals
for classification tasks and finds %using
ImageNet-pretrained DeiT is remarkably effective in knowledge transfer. The fine-tuning paradigm has also been %similarly
adopted in \cite{zeng2023pixels,li2023time} but with different pre-trained models %initializations
such as Swin by \cite{li2023time}. 
VisionTS \cite{chen2024visionts} %explains
attributes %the superiority of LVMs
LVMs' superiority over LLMs in knowledge transfer %over LLMs %as an outcome of
to the small gap between the pre-trained images and imaged time series. %the patterns learned from the large-scale pre-trained images and the patterns in the images of time series.
It %also
finds that with one-epoch fine-tuning, MAE becomes the SOTA time series forecasters on %many
some benchmark datasets.

Similar to %build
time series foundation models %\cite{das2024decoder,goswami2024moment,ansari2024chronos,shi2024time}, %such as TimesFM \cite{das2024decoder}, MOMENT \cite{goswami2024moment}, Chronos \cite{ansari2024chronos} and Time-MoE \cite{shi2024time},
such as TimesFM \cite{das2024decoder}, %and MOMENT \cite{goswami2024moment}, 
there are some initial efforts in pre-training ViT architectures with imaged time series. Following AST, SSAST \cite{gong2022ssast} introduced a %joint discriminative and generative
%masked spectrogram patch prediction self-supervised learning framework
masked spectrogram patch prediction framework for pre-training ViT on a large dataset -- AudioSet-2M. Then it becomes a backbone of some follow-up works such as AST-SED \cite{li2023ast} for sound event detection. %To be effective for UTSs,
For UTSs, ViTime \cite{yang2024vitime} generates a large set of Line Plots of synthetic UTSs for pre-training ViT, which was found superior over TimesFM in zero-shot forecasting tasks on benchmark datasets.

\subsection{Large Multimodal Models (LMMs)}\label{sec.lmms}

%As Large Multimodal Models (LMMs)
As LMMs %are getting
get growing attentions, some %of the
notable LMMs, such as LLaVA \cite{liu2023visual}, Gemini \cite{team2023gemini}, GPT-4o \cite{achiam2023gpt} and Claude-3 \cite{anthropic2024claude}, have been explored to consolidate the power of LLMs %on time series
and LVMs in time series analysis. 
Since LMMs support multimodal input via prompts, methods in this thread typically prompt LMMs with the textual and imaged representations of time series, %textual representation of time series and their %image transformations, transformed images,
%then instruct LMMs
and instructions on what tasks to perform ({\em e.g.}, Fig. \ref{fig.models}(c)).

InsightMiner \cite{zhang2023insight} is a pioneer work that uses the LLaVA architecture to generate %textual descriptions about
texts describing the trend of each input UTS. It extracts the trend of a UTS by Seasonal-Trend decomposition, encodes the Line Plot of the trend, and concatenates the embedding of the Line Plot with the embeddings of a textual instruction, which includes a sequence of numbers representing the UTS, {\em e.g.}, ``[1.1, 1.7, ..., 0.3]''. The concatenated embeddings are taken by a language model for generating trend descriptions. %It also fine-tunes a few layers with the generated texts to align LLaVA checkpoints with time series domain.
Similarly, \cite{prithyani2024feasibility} adopts the LLaVA architecture, but for MTS classification. An MTS is encoded by %a sequence of
the visual %token
embeddings of the stacked Line Plots of all variates. %meanwhile
%The method also stacks
%The time series of all variate are also stacked in a prompt % of all variates in a prompt
The matrix of the MTS is also verbalized in a prompt 
as the textual modality. %By manipulating token embeddings,
By integrating token embeddings, both %of these %works propose to
methods fine-tune some layers of the LMMs with some synthetic data.

Moreover, zero-shot and in-context learning performance of several commercial LMMs have been evaluated for audio classification \cite{dixit2024vision}, anomaly detection \cite{zhuang2024see}, and some synthetic tasks \cite{daswani2024plots}, where the image %({\em e.g.}, spectrograms, Line Plots)
and textual representations of a query %UTS or MTS
time series are integrated into a prompt. For in-context learning, these methods inject the images of a few example time series and their labels ({\em e.g.}, classes) %({\em e.g.}, classes, normal status)
into an instruction to prompt LMMs for assisting the prediction of the query time series.

\subsection{Task-Specific Heads}\label{sec.task}

%With the image embedding of a time series, the next step is to produce its prediction.
For classification tasks, most of the methods in Table \ref{tab.taxonomy} adopt a fully connected (FC) layer or multilayer perceptron (MLP) to transform an embedding into a probability distribution over all classes. For forecasting tasks, there are two approaches: (1) using a $d_{e}\times W$ MLP/FC layer to directly predict (from the $d_{e}$-dimensional embedding) the time series values in a future time window of size $W$ \cite{li2020forecasting,semenoglou2023image}; (2) predicting the pixel values that represent the future part of the time series and then recovering the time series from the predicted image \cite{yang2023your,chen2024visionts,yang2024vitime} ($\S$\ref{sec.processing} discusses the recovery methods). Imputation and generation tasks resemble forecasting %in the sense of predicting
as they also predict time series values. Thus approach (2) has been used for imputation \cite{wang2015imaging} and generation \cite{naiman2024utilizing,karami2024timehr}. %LMMs have been used for classification, text generation, and anomaly detection. For these tasks,
When using LMMs for classification, text generation, and anomaly detection, most of the methods prompt LMMs to produce the desired outputs in textual answers, circumventing task-specific heads \cite{zhang2023insight,dixit2024vision,zhuang2024see}.

%Forecasting: MLP, FC to predict numerical values using embeddings. Imputation of images (TSH). Classification: MLP, FC using embeddings.

\section{Pre-Processing and Post-Processing}\label{sec.processing}

To be successful in using vision models, some subtle design desiderata %to be considered
include {\bf time series normalization}, {\bf image alignment} and {\bf time series recovery}.

\vspace{0.2cm}

\noindent{\bf Time Series Normalization.} Vision models are usually trained on %images after Gaussian normalization (GN).
standardized images. To be aligned, the images introduced in $\S$\ref{sec.tsimage} should be normalized with a controlled mean and standard deviation, as did by \cite{gong2021ast} on spectrograms. In particular, as Heatmap is built on raw time series values, the commonly used Instance Normalization (IN) \cite{kim2022reversible} can be applied on the time series as suggested by VisionTS \cite{chen2024visionts} since IN share similar merits as Standardization. %although min-max normalization was used by \cite{karami2024timehr,zeng2021deep}.
Using Line Plot requires a proper range of y-axis. In addition to rescaling time series %by min-max or GN
\cite{zhuang2024see}, ViTST \cite{li2023time} introduced several methods to remove extreme values from the plot. GAF requires min-max normalization on its input, as it transforms time series values withtin $[0, 1]$ to polar coordinates ({\em i.e.}, arccos). In contrast, input to RP is usually normalization-free as an $\ell_{2}$ norm is involved in Eq.~\eqref{eq.rp} before thresholding.%for a comparison with a threshold.

\vspace{0.2cm}

\noindent{\bf Image Alignment.} When using pre-trained models, it is imperative to fit the image size to the input requirement of the models. This is especially true for Transformer based models as they use a fixed number of positional embeddings to encode the spacial information of image patches. For 3-channel RGB images such as Line Plot, it is straightforward to meet a pre-defined size by adjusting the resolution when producing the image. For images built upon matrices such as Heatmap, Spectrogram, GAF, RP, the number of channels and matrix size need adjustment. For the channels, one method is to duplicate a matrix to 3 channels \cite{chen2024visionts}, another way is to average the weights of the 3-channel patch embedding layer into a 1-channel layer \cite{gong2021ast}. For the image size, bilinear interpolation is a common method to resize input images \cite{chen2024visionts}. Alternatively, AST \cite{gong2021ast} %use cut and bilinear interpolation on
resizes the positional embeddings instead of the images to fit the model to a desired input size. However, the interpolation in these methods may either alter the time series or the spacial information in positional embeddings.

% single-channel (UTS), RGB channel (UTS), duplicate channels (UTS), multi-channel (MTS).

%Bilinear interpolation.

%Correlated variates are better to be spatially close to each other.

%\subsection{Pre-training}

\vspace{0.2cm}

\noindent{\bf Time Series Recovery.} As stated in $\S$\ref{sec.task}, tasks such as forecasting, imputation and generation requires predicting time series values. For models that predict pixel values of images, post-processing involves recovering time series from the predicted images. Recovery from Line Plots is tricky, it requires locating pixels that %correspond to
represent time series and mapping them back to the original values. This can be done by manipulating a grid-like Line Plot as introduced in \cite{yang2023your,yang2024vitime}, which has a recovery function. In contrast, recovery from Heatmap is straightforward as it directly stores the predicted time series values \cite{zeng2021deep,chen2024visionts}. Spectrogram is underexplored in these tasks and it remains open on how to recover time series from it. The existing work \cite{zeng2023pixels} uses Spectrogram for forecasting only with an MLP head that directly predicts time series. %predicts time series values.
GAF supports accurate recovery by an inverse mapping from polar coordinates to normalized time series \cite{wang2015imaging}. However, RP lost time series information during thresholding (Eq.~\ref{eq.rp}), thus may not fit recovery-demanded tasks without using an {\em ad-hoc} prediction head.


% Line Plot was regarded as matrices with rows and columns for mapping in \cite{sood2021visual}.


%\section{Tasks and Time Series Recovery}

%\subsection{Task-Specific Head}

% \subsection{Time Series Recovery}



    \caption{AI Risk Atlas Taxonomy. The taxonomy is divided into four categories, which are then further divided into dimensions. Next to each dimension in the parentheses is the number of risks identified for that dimension.}
    \label{fig:atlas-main}
\end{figure*}

\begin{figure*}
    \centering
    \includegraphics[width=\linewidth]{images/Hallucination.jpg}
    \caption{Screenshot of AI Risk Atlas Detail Page for Hallucination}
    \label{fig:atlas-detail-screenshot}
\end{figure*}

\ignore{
\begin{figure*}
\begin{definitionbox}{Hallucination}
Hallucinations generate factually inaccurate or untruthful content with respect to the model's training data or input. This is also sometimes referred to lack of faithfulness or lack of groundedness.\newline\newline
\textbf{Concern: }Hallucinations can be misleading. These false outputs can mislead users and be incorporated into downstream artifacts, further spreading misinformation. False output can harm both owners and users of the AI models. In some uses, hallucinations can be particularly consequential.\newline\newline
\textbf{Type: }output\newline
\textbf{Descriptor: }specific \newline\newline
\textbf{Implementation details: } \newline
ID: atlas-hallucination \newline
Tag: hallucination \newline
URI:  \href{https://www.ibm.com/docs/en/watsonx/saas?topic=SSYOK8/wsj/ai-risk-atlas/hallucination.html}{IBM AI Risk Atlas - Hallucination}\newline
\end{definitionbox}
\caption{AI Risk Atlas Entry Example }
\label{fig:atlas-detail}
\end{figure*}
}

%\begin{figure*}
%    \centering
%    \includegraphics[width=0.99\linewidth]{images/RiskAtlasMain-med.jpg}
%    \caption{AI Risk Atlas Main Page}
%    \label{fig:atlas-main}
%\end{figure*}



Figure~\ref{fig:atlas-main} shows the overall AI Risk Atlas taxonomy where each category may include a subset of risk definitions. 
For example, Figure~\ref{fig:atlas-detail-screenshot} shows a screenshot of the Hallucination risk from the AI Risk Atlas, which is in the Robustness dimension in the Output category in Figure~\ref{fig:atlas-main}. The details contain a description of the risk along with a description of why the risk is a concern, a public example, when available, of the risk being manifested, and any related risks in other popular taxonomies.

% Motivation and a bit of history
The creation of the AI Risk Atlas was motivated by the changing risk landscape due to the emergence and rapid success of generative AI. Before generative AI became ubiquitous, IBM Research had developed the techniques to evaluate and mitigate some risks such as fairness, explainability, adversarial robustness, privacy, and uncertainty for traditional (non-generative) models. This work evolved into open-source toolkits designed to measure, and in some cases, mitigate those risks~\cite{aif360,aix360,art360,aip360,uq360}. However, these toolkits did not initially account for the risks specific to generative models. The large amount of data used to train generative AI, the increasing complexity of the models, and the non-determinism prevalent were some of the factors that led to the identification of new risks for generative models. Prompt-based attacks, poorly curated training data, and hallucinations emerged as early risks identified by researchers and concerned citizens alike. The identification of risks had begun to outpace mechanisms for measuring and understanding the risks. 

%\todo{Should we also include the ethics board white paper that came out here? (The first version of the risk atlas was an evolution from the content there). Elizabeth: Yes I think that is a good idea.}

To address this gap and to provide a foundation for understanding the risks of both traditional and generative AI models, 
IBM's AI Ethics Board created a white paper~\cite{ethics-board-pov} that provided the foundation for the AI Risk Atlas.
The goal for the Atlas was multi-faceted. First, we wanted to have a single source of information for currently known AI risks. This would enable a shared vocabulary when discussing AI risks. Second, we wanted to identify opportunities for measuring and mitigating the newly identified risks. This would help identify research opportunities for underrepresented risks. Third, we wanted a resource to develop usage-based governance. Specifically, we wanted to address the question of what risks are relevant to particular use cases.


The AI Risk Atlas has been used as a conversation starting point with enterprises who are considering deploying AI. It helps these organizations to be aware of the possible risks they need to govern. It provides a palette or vocabulary of risks that enterprises can consider: they can decide the risk is relevant to their use case and develop a plan to mitigate the risk either with tooling, human oversight, or both.
For those risks that aren't applicable for a use case, the organization can document this risk to help demonstrate their risk governance framework. However, the AI Risk Atlas can be used further than just a conversation starter. It can provide the underlying vocabulary for the complete management of the risk from development to deployment to monitoring \cite{watsonx-gov-dec-2024,mra-vision, daly2024usage}.
%IBM has recently released functionality to use the atlas for automated risk identification~\cite{watsonx-gov-dec-2024} and can also be used as the foundation for risk evaluation~\cite{mra-vision}.

% Focus on motivation, categorization, and amplified risks.

\section{Tools for Practitioners} \label{sec-tools}
The IBM AI Risk Atlas has been used many enterprise customers to help them reason about the risks in their AI systems. In order to enable efforts to leverage these risks to  operationalise governance and risk mitigation frameworks we created \textit{Risk Atlas Nexus.}

The \textit{Risk Atlas Nexus} is a collection of tooling to help bring together disparate resources related to governance of foundation models. We aim to support a community-driven approach to curating and cataloguing resources such as datasets, benchmarks and mitigations. Our goal is to turn abstract risk definitions into actionable workflows that streamline AI governance processes. By connecting fragmented resources, Risk Atlas Nexus seeks to fill a critical gap in AI governance, enabling stakeholders to build more robust, transparent, and accountable systems. The Risk Atlas Nexus is a step towards enabling the following.

\textbf{Navigating disparate risk taxonomies:} IBM AI Risk Atlas is one amongst a number of existing risk taxonomies, for example; the OWASP Top 10 for LLMs and Generative AI Apps \cite{owasp}, the NIST AI Risk Management Framework \cite{nist}, the MIT AI Risk Repository \cite{airiskrepo}, the AIR taxonomy 2024 \cite{zeng2024ai}.

To provide a way through this labyrinth of taxonomies, we have constructed an AI risk ontology that allows both the creation of a knowledge graph containing those different taxonomies and the ability to map between them. The ontology has been modeled using LinkML \cite{moxon2021linked}, which allows the generation of different data representations (e.g. RDF, OWL) in a simple way. The risk taxonomies have been stored as LinkML data instance YAML files. To express some semantically meaningful mapping between risks from different taxonomies, we have used the Simple Standard for Sharing Ontological Mappings (SSSOM) \cite{sssom}. Therefore those mappings are maintained in SSOM TSV files and are converted to LinkML data YAML using Python helper scripts.

Sample notebooks demonstrate how to load the LinkML data and user data and how to get details about specific risks and their relations to risks in other taxonomies.

\textbf{Question Answering:} Compliance questionnaires are usually required prior to deploying an AI model into production. These enable a thorough understanding of the specific use case and associated risk exposures~\cite{watsonx-gov-dec-2024,lee2023qb4airaquestionbankai}. The Risk Atlas Nexus supports the development and curation of questionnaires to a desired taxonomy. Additionally, the content can support Large Language Models (LLMs) to assist users in responding to time-consuming compliance questionnaires, thereby reducing manual effort and minimizing errors~\cite{daly2024usage}. Similarly, other aspects like risk identification, guardrail implementation, and identifying security vulnerabilities for specific use cases can be largely automated with human feedback and sign-off provided only when necessary.

\textbf{Use Case to Risk Prioritisation:} To help prioritise which of the many risks are most related to their use case we leverage LLM-as-a-judge capabilities to identify which risks to consider. This information can be used to look for appropriate research papers, benchmarks and metrics. In a similar manner Risk Atlas Nexus can be used to tag disparate resources by passing in text such as a paper abstract or a dataset description as the risks for the basis for an LLM-as-a-judge definition \cite{ashktorab2024aligning,desmond2025evalassist}. 


\textbf{From Risks to Mitigating Actions:} The knowledge graph supports mapping between risks to two types of mitigation strategies: detectors and recommended actions. Detectors such as Granite Guardian~\cite{padhi2024graniteguardian} dimensions could be run in tandem with an AI system to better protect against certain risks such as social bias and prompt injection attacks. We have also mined recommended actions as part of the NIST AI Risk Management  Framework~\cite{nist} to be able to recommend more process driven mitigation strategies. 

\textbf{Bring Your Own Risks, Relationships and Questionnaires:} Risk Atlas Nexus tooling supports several well known risk taxonomy frameworks, however some organisations may wish to define their own custom concerns and definitions.  Risk Atlas Nexus allows users to define custom questionnaire templates as well as taxonomies, risks, mappings, and mitigation actions which should conform to the  \href{ https://github.com/IBM/risk-atlas-nexus/blob/42a42aebf87cdf18232105ab57ffef69331e322d/docs/ontology/index.md }{ontology schema}. We encourage users to contribute their taxonomy definitions and mappings back to the project for others to use through the open-source project.  

\section{Potential for the future} \label{sec-potential}
There is immense potential of automation of various aspects of compliance and risk management processes. While human oversight and manual verification are still essential requirements for compliance, auditing and regulatory purposes, automation can help to bring an efficient execution of intermediate stages in the compliance workflow. AutoML strategies have been employed to automate model training pipelines excelling at tasks such as feature selection,  hyper parameter optimization, model generation and evaluation \cite{he2021automl,wang2020autoai}. In a similar manner AI governance pipelines could be employed to detect and mitigate risks. By starting with identifying the most relevant risks, running the most relevant benchmarks and then assessing the impact of employing real-time mitigation strategies, concrete recommendations can be made to improve safety of AI solutions.

In the context of complex systems LLMs are increasingly being used as part of the validation process, from software testing and assisting in tasks such as test case preparation \cite{wang2024software} to assessing the output of an LLM \cite{van2024field,shankar2024validates, ashktorab2024aligning}. With the increasing capabilities of LLMs their applications have gone beyond single function tasks to being used to address complex problems acting as autonomous-agents \cite{wang2024survey}. ToolLLaMA learns how to call appropriate API based tools \cite{schick2023toolformer} meaning LLMs can orchestrate tasks that leverage existing functionality embedded in other tooling. The research community has begun to use agent flows to design, plan and execute scientific experiments  \cite{boiko2023emergent} and even write papers \cite{lu2024ai}.

%\begin{itemize}
%    \item AutoML opportunities for CI/CD pipelines
%    \item Agentic solutions for lifecycle governance
%    \item Potential links to policy and regulations
%    \item Open questions and challenges. How can practisioners use the information in the risk atlas to communicate to stake holders
%\end{itemize}

Given these evolving capabilities agentic frameworks\cite{lang-graph,crewai,autogen,bee-ai-framework} have the potential to create real-time governance pipelines, from identifying relevant risks and benchmarks to identifying mitigating actions to online monitoring capabilities. To reliably build such pipelines a method to organise and connect these functional components together must be created to curate appropriate meta-data. 
%\todo{Elizabeth: We really need some related work here on LLM Agentic frameworks. Mike: I added 4 citations to frameworks. I'm not sure if you wanted more than that. Elizabeth: Thank you!}



%Further, to ensure adaptive realtime governance of LLMs, an agentic framework can be established. This can enable monitoring of LLM models in production for drift, enhanced risk exposure or security attacks. The agent systems should inherently possess the capacity for self-improvement through mechanisms such as continuous learning from experience, data, and feedback, ensuring refinement of AI governance performance and adaptability. This is essential for maintaining reliability, and overall effectiveness of AI governance systems over time.

% \begin{itemize}
%     \item Potential for Automation and Scaling 
%     \item Intent to Governance for AI Lifecycle pipelines [Seshu]
%         \todo{MH: Here are 2 citations\cite{watsonx-gov-dec-2024,lee2023qb4airaquestionbankai} that can be used as examples for risk identification questionnaires. The first is the best we have for what was shipped in x.gov (Marc will create a blog post in the near future) and the 2nd is an external group.}
%     \item Agentic workflows for real-time dynamic governance [Seshu]
% \end{itemize}

\section{Conclusion and call to action} \label{sec-conc}

Curating the AI Risk Atlas is just the first step in providing a reference framework for researchers and practitioners navigating the rapidly evolving AI landscape. By positioning our risk taxonomy in relation to existing definitions and taxonomies, we aim to encourage the community to map new risk definitions, datasets, benchmarks, research papers, and crucially mitigation and detection strategies into a structured framework. This approach will enhance accessibility and facilitate the operationalisation of AI governance processes.

The initial tools released as part of the Risk Atlas Nexus toolkit represent only the beginning of what is possible. We are committed to ongoing development, enabling the developer community to contribute and expand this initiative. By fostering a community-driven approach, we can lower the barrier to entry for all. We invite the open-source community to enrich the knowledge graph by linking benchmarks, datasets, and research papers to identified risks. Additionally, contributors can request new functionality through GitHub enhancement requests or develop and integrate their own algorithms.





\bibliographystyle{plain}
\bibliography{refs}

\appendix
%\documentclass[a4paper,12pt]{article}

%\begin{document}
\section*{AI Risk Atlas Definitions}

The below is a catalog of potential risks when working with generative AI, foundation models, and machine learning models.


\begin{definitionbox}{Non-disclosure}
Content might not be clearly disclosed as AI generated.\newline\newline
\textbf{Concern: }Users must be notified when they are interacting with an AI system. Not disclosing the AI-authored content can result in a lack of transparency.\newline\newline
\textbf{Type: }output\newline
\textbf{Descriptor: }specific \newline\newline
\textbf{Implementation details: } \newline
ID: atlas-non-disclosure \newline
Tag: non-disclosure \newline
URI:  \href{https://www.ibm.com/docs/en/watsonx/saas?topic=SSYOK8/wsj/ai-risk-atlas/non-disclosure.html}{IBM AI Risk Atlas - Non-disclosure}\newline
\end{definitionbox}
\begin{definitionbox}{Lack of training data transparency}
Without accurate documentation on how a model's data was collected, curated, and used to train a model, it might be harder to satisfactorily explain the behavior of the model with respect to the data.\newline\newline
\textbf{Concern: }A lack of data documentation limits the ability to evaluate risks associated with the data. Having access to the training data is not enough. Without recording how the data was cleaned, modified, or generated, the model behavior is more difficult to understand and to fix. Lack of data transparency also impacts model reuse as it is difficult to determine data representativeness for the new use without such documentation.\newline\newline
\textbf{Type: }training-data\newline
\textbf{Descriptor: }amplified \newline\newline
\textbf{Implementation details: } \newline
ID: atlas-data-transparency \newline
Tag: data-transparency \newline
URI:  \href{https://www.ibm.com/docs/en/watsonx/saas?topic=SSYOK8/wsj/ai-risk-atlas/data-transparency.html}{IBM AI Risk Atlas - Lack of training data transparency}\newline
\end{definitionbox}
\begin{definitionbox}{Model usage rights restrictions}
Terms of service, licenses, or other rules restrict the use of certain models.\newline\newline
\textbf{Concern: }Laws and regulations that concern the use of AI are in place and vary from country to country. Additionally, the usage of models might be dictated by licensing terms or agreements.\newline\newline
\textbf{Type: }non-technical\newline
\textbf{Descriptor: }traditional \newline\newline
\textbf{Implementation details: } \newline
ID: atlas-model-usage-rights \newline
Tag: model-usage-rights \newline
URI:  \href{https://www.ibm.com/docs/en/watsonx/saas?topic=SSYOK8/wsj/ai-risk-atlas/model-usage-rights.html}{IBM AI Risk Atlas - Model usage rights restrictions}\newline
\end{definitionbox}
\begin{definitionbox}{Prompt injection attack}
A prompt injection attack forces a generative model that takes a prompt as input to produce unexpected output by manipulating the structure, instructions, or information contained in its prompt.\newline\newline
\textbf{Concern: }Injection attacks can be used to alter model behavior and benefit the attacker.\newline\newline
\textbf{Type: }inference\newline
\textbf{Descriptor: }specific \newline\newline
\textbf{Implementation details: } \newline
ID: atlas-prompt-injection \newline
Tag: prompt-injection \newline
URI:  \href{https://www.ibm.com/docs/en/watsonx/saas?topic=SSYOK8/wsj/ai-risk-atlas/prompt-injection.html}{IBM AI Risk Atlas - Prompt injection attack}\newline
\end{definitionbox}
\begin{definitionbox}{Incomplete advice}
When a model provides advice without having enough information, resulting in possible harm if the advice is followed.\newline\newline
\textbf{Concern: }A person might act on incomplete advice or worry about a situation that is not applicable to them due to the overgeneralized nature of the content generated. For example, a model might provide incorrect medical, financial, and legal advice or recommendations that the end user might act on, resulting in harmful actions.\newline\newline
\textbf{Type: }output\newline
\textbf{Descriptor: }specific \newline\newline
\textbf{Implementation details: } \newline
ID: atlas-incomplete-advice \newline
Tag: incomplete-advice \newline
URI:  \href{https://www.ibm.com/docs/en/watsonx/saas?topic=SSYOK8/wsj/ai-risk-atlas/incomplete-advice.html}{IBM AI Risk Atlas - Incomplete advice}\newline
\end{definitionbox}
\begin{definitionbox}{Lack of system transparency}
Insufficient documentation of the system that uses the model and the model's purpose within the system in which it is used.\newline\newline
\textbf{Concern: }A lack of documentation makes it difficult to understand how the model's outcomes contribute to the system's or application's functionality.\newline\newline
\textbf{Type: }non-technical\newline
\textbf{Descriptor: }traditional \newline\newline
\textbf{Implementation details: } \newline
ID: atlas-lack-of-system-transparency \newline
Tag: lack-of-system-transparency \newline
URI:  \href{https://www.ibm.com/docs/en/watsonx/saas?topic=SSYOK8/wsj/ai-risk-atlas/lack-of-system-transparency.html}{IBM AI Risk Atlas - Lack of system transparency}\newline
\end{definitionbox}
\begin{definitionbox}{Data usage restrictions}
Laws and other restrictions can limit or prohibit the use of some data for specific AI use cases.\newline\newline
\textbf{Concern: }Data usage restrictions can impact the availability of the data required for training an AI model and can lead to poorly represented data.\newline\newline
\textbf{Type: }training-data\newline
\textbf{Descriptor: }traditional \newline\newline
\textbf{Implementation details: } \newline
ID: atlas-data-usage \newline
Tag: data-usage \newline
URI:  \href{https://www.ibm.com/docs/en/watsonx/saas?topic=SSYOK8/wsj/ai-risk-atlas/data-usage.html}{IBM AI Risk Atlas - Data usage restrictions}\newline
\end{definitionbox}
\begin{definitionbox}{Impact on cultural diversity}
AI systems might overly represent certain cultures that result in a homogenization of culture and thoughts.\newline\newline
\textbf{Concern: }Underrepresented groups' languages, viewpoints, and institutions might be suppressed by that means reducing diversity of thought and culture.\newline\newline
\textbf{Type: }non-technical\newline
\textbf{Descriptor: }specific \newline\newline
\textbf{Implementation details: } \newline
ID: atlas-impact-on-cultural-diversity \newline
Tag: impact-on-cultural-diversity \newline
URI:  \href{https://www.ibm.com/docs/en/watsonx/saas?topic=SSYOK8/wsj/ai-risk-atlas/impact-on-cultural-diversity.html}{IBM AI Risk Atlas - Impact on cultural diversity}\newline
\end{definitionbox}
\begin{definitionbox}{Impact on education: plagiarism}
Easy access to high-quality generative models might result in students that use AI models to plagiarize existing work intentionally or unintentionally.\newline\newline
\textbf{Concern: }AI models can be used to claim the authorship or originality of works that were created by other people in doing so by engaging in plagiarism. Claiming others' work as your own is both unethical and often illegal.\newline\newline
\textbf{Type: }non-technical\newline
\textbf{Descriptor: }specific \newline\newline
\textbf{Implementation details: } \newline
ID: atlas-plagiarism \newline
Tag: plagiarism \newline
URI:  \href{https://www.ibm.com/docs/en/watsonx/saas?topic=SSYOK8/wsj/ai-risk-atlas/plagiarism.html}{IBM AI Risk Atlas - Impact on education: plagiarism}\newline
\end{definitionbox}
\begin{definitionbox}{Personal information in data}
Inclusion or presence of personal identifiable information (PII) and sensitive personal information (SPI) in the data used for training or fine tuning the model might result in unwanted disclosure of that information.\newline\newline
\textbf{Concern: }If not properly developed to protect sensitive data, the model might expose personal information in the generated output.  Additionally, personal, or sensitive data must be reviewed and handled in accordance with privacy laws and regulations.\newline\newline
\textbf{Type: }training-data\newline
\textbf{Descriptor: }traditional \newline\newline
\textbf{Implementation details: } \newline
ID: atlas-personal-information-in-data \newline
Tag: personal-information-in-data \newline
URI:  \href{https://www.ibm.com/docs/en/watsonx/saas?topic=SSYOK8/wsj/ai-risk-atlas/personal-information-in-data.html}{IBM AI Risk Atlas - Personal information in data}\newline
\end{definitionbox}
\begin{definitionbox}{Improper usage}
Improper usage occurs when a model is used for a purpose that it was not originally designed for.\newline\newline
\textbf{Concern: }Reusing a model without understanding its original data, design intent, and goals might result in unexpected and unwanted model behaviors.\newline\newline
\textbf{Type: }output\newline
\textbf{Descriptor: }amplified \newline\newline
\textbf{Implementation details: } \newline
ID: atlas-improper-usage \newline
Tag: improper-usage \newline
URI:  \href{https://www.ibm.com/docs/en/watsonx/saas?topic=SSYOK8/wsj/ai-risk-atlas/improper-usage.html}{IBM AI Risk Atlas - Improper usage}\newline
\end{definitionbox}
\begin{definitionbox}{Extraction attack}
An attribute inference attack is used to detect whether certain sensitive features can be inferred about individuals who participated in training a model. These attacks occur when an adversary has some prior knowledge about the training data and uses that knowledge to infer the sensitive data.\newline\newline
\textbf{Concern: }With a successful extraction attack, the attacker can perform further adversarial attacks to gain valuable information such as sensitive personal information or intellectual property.\newline\newline
\textbf{Type: }inference\newline
\textbf{Descriptor: }amplified \newline\newline
\textbf{Implementation details: } \newline
ID: atlas-extraction-attack \newline
Tag: extraction-attack \newline
URI:  \href{https://www.ibm.com/docs/en/watsonx/saas?topic=SSYOK8/wsj/ai-risk-atlas/extraction-attack.html}{IBM AI Risk Atlas - Extraction attack}\newline
\end{definitionbox}
\begin{definitionbox}{Impact on Jobs}
Widespread adoption of foundation model-based AI systems might lead to people's job loss as their work is automated if they are not reskilled.\newline\newline
\textbf{Concern: }Job loss might lead to a loss of income and thus might negatively impact the society and human welfare. Reskilling might be challenging given the pace of the technology evolution.\newline\newline
\textbf{Type: }non-technical\newline
\textbf{Descriptor: }amplified \newline\newline
\textbf{Implementation details: } \newline
ID: atlas-job-loss \newline
Tag: job-loss \newline
URI:  \href{https://www.ibm.com/docs/en/watsonx/saas?topic=SSYOK8/wsj/ai-risk-atlas/job-loss.html}{IBM AI Risk Atlas - Impact on Jobs}\newline
\end{definitionbox}
\begin{definitionbox}{Jailbreaking}
A jailbreaking attack attempts to break through the guardrails that are established in the model to perform restricted actions.\newline\newline
\textbf{Concern: }Jailbreaking attacks can be used to alter model behavior and benefit the attacker. If not properly controlled, business entities can face fines, reputational harm, and other legal consequences.\newline\newline
\textbf{Type: }inference\newline
\textbf{Descriptor: }specific \newline\newline
\textbf{Implementation details: } \newline
ID: atlas-jailbreaking \newline
Tag: jailbreaking \newline
URI:  \href{https://www.ibm.com/docs/en/watsonx/saas?topic=SSYOK8/wsj/ai-risk-atlas/jailbreaking.html}{IBM AI Risk Atlas - Jailbreaking}\newline
\end{definitionbox}
\begin{definitionbox}{Data acquisition restrictions}
Laws and other regulations might limit the collection of certain types of data for specific AI use cases.\newline\newline
\textbf{Concern: }"There are several ways of collecting data for building a foundation models: web scraping, web crawling, crowdsourcing, and curating public datasets. Data acquisition restrictions can also impact the availability of the data that is required for training an AI model and can lead to poorly represented data."\newline\newline
\textbf{Type: }training-data\newline
\textbf{Descriptor: }amplified \newline\newline
\textbf{Implementation details: } \newline
ID: atlas-data-acquisition \newline
Tag: data-acquisition \newline
URI:  \href{https://www.ibm.com/docs/en/watsonx/saas?topic=SSYOK8/wsj/ai-risk-atlas/data-acquisition.html}{IBM AI Risk Atlas - Data acquisition restrictions}\newline
\end{definitionbox}
\begin{definitionbox}{Data bias}
Historical and societal biases that are present in the data are used to train and fine-tune the model.
\newline\newline
\textbf{Concern: }Training an AI system on data with bias, such as historical or societal bias, can lead to biased or skewed outputs that can unfairly represent or otherwise discriminate against certain groups or individuals.\newline\newline
\textbf{Type: }training-data\newline
\textbf{Descriptor: }amplified \newline\newline
\textbf{Implementation details: } \newline
ID: atlas-data-bias \newline
Tag: data-bias \newline
URI:  \href{https://www.ibm.com/docs/en/watsonx/saas?topic=SSYOK8/wsj/ai-risk-atlas/data-bias.html}{IBM AI Risk Atlas - Data bias}\newline
\end{definitionbox}
\begin{definitionbox}{Uncertain data provenance}
Data provenance refers to tracing history of data, which includes its ownership, origin, and transformations. Without standardized and established methods for verifying where the data came from, there are no guarantees that the data is the same as the original source and has the correct usage terms.\newline\newline
\textbf{Concern: }Not all data sources are trustworthy. Data might be unethically collected, manipulated, or falsified. Verifying that data provenance is challenging due to factors such as data volume, data complexity, data source varieties, and poor data management. Using such data can result in undesirable behaviors in the model.\newline\newline
\textbf{Type: }training-data\newline
\textbf{Descriptor: }amplified \newline\newline
\textbf{Implementation details: } \newline
ID: atlas-data-provenance \newline
Tag: data-provenance \newline
URI:  \href{https://www.ibm.com/docs/en/watsonx/saas?topic=SSYOK8/wsj/ai-risk-atlas/data-provenance.html}{IBM AI Risk Atlas - Uncertain data provenance}\newline
\end{definitionbox}
\begin{definitionbox}{Unrepresentative risk testing}
Testing is unrepresentative when the test inputs are mismatched with the inputs that are expected during deployment.\newline\newline
\textbf{Concern: }If the model is evaluated in a use, context, or setting that is not the same as the one expected for deployment, the evaluations might not accurately reflect the risks of the model.\newline\newline
\textbf{Type: }non-technical\newline
\textbf{Descriptor: }amplified \newline\newline
\textbf{Implementation details: } \newline
ID: atlas-unrepresentative-risk-testing \newline
Tag: unrepresentative-risk-testing \newline
URI:  \href{https://www.ibm.com/docs/en/watsonx/saas?topic=SSYOK8/wsj/ai-risk-atlas/unrepresentative-risk-testing.html}{IBM AI Risk Atlas - Unrepresentative risk testing}\newline
\end{definitionbox}
\begin{definitionbox}{Data usage rights restrictions}
Terms of service, license compliance, or other IP issues may restrict the ability to use certain data for building models.\newline\newline
\textbf{Concern: }Laws and regulations concerning the use of data to train AI are unsettled and can vary from country to country, which creates challenges in the development of models.\newline\newline
\textbf{Type: }training-data\newline
\textbf{Descriptor: }amplified \newline\newline
\textbf{Implementation details: } \newline
ID: atlas-data-usage-rights \newline
Tag: data-usage-rights \newline
URI:  \href{https://www.ibm.com/docs/en/watsonx/saas?topic=SSYOK8/wsj/ai-risk-atlas/data-usage-rights.html}{IBM AI Risk Atlas - Data usage rights restrictions}\newline
\end{definitionbox}
\begin{definitionbox}{Harmful code generation}
Models might generate code that causes harm or unintentionally affects other systems.\newline\newline
\textbf{Concern: }The execution of harmful code might open vulnerabilities in IT systems.\newline\newline
\textbf{Type: }output\newline
\textbf{Descriptor: }specific \newline\newline
\textbf{Implementation details: } \newline
ID: atlas-harmful-code-generation \newline
Tag: harmful-code-generation \newline
URI:  \href{https://www.ibm.com/docs/en/watsonx/saas?topic=SSYOK8/wsj/ai-risk-atlas/harmful-code-generation.html}{IBM AI Risk Atlas - Harmful code generation}\newline
\end{definitionbox}
\begin{definitionbox}{Data contamination}
Data contamination occurs when incorrect data is used for training. For example, data that is not aligned with model's purpose or data that is already set aside for other development tasks such as testing and evaluation.\newline\newline
\textbf{Concern: }Data that differs from the intended training data might skew model accuracy and affect model outcomes.\newline\newline
\textbf{Type: }training-data\newline
\textbf{Descriptor: }amplified \newline\newline
\textbf{Implementation details: } \newline
ID: atlas-data-contamination \newline
Tag: data-contamination \newline
URI:  \href{https://www.ibm.com/docs/en/watsonx/saas?topic=SSYOK8/wsj/ai-risk-atlas/data-contamination.html}{IBM AI Risk Atlas - Data contamination}\newline
\end{definitionbox}
\begin{definitionbox}{Incomplete usage definition}
Since foundation models can be used for many purposes, a model's intended use is important for defining the relevant risks of that model. As the use changes, the relevant risks might correspondingly change.\newline\newline
\textbf{Concern: }It might be difficult to accurately determine and mitigate the relevant risks for a model when its intended use is insufficiently specified. Such as how a model is going to be used, where it is going to be used and what it is going to be used for.\newline\newline
\textbf{Type: }non-technical\newline
\textbf{Descriptor: }specific \newline\newline
\textbf{Implementation details: } \newline
ID: atlas-incomplete-usage-definition \newline
Tag: incomplete-usage-definition \newline
URI:  \href{https://www.ibm.com/docs/en/watsonx/saas?topic=SSYOK8/wsj/ai-risk-atlas/incomplete-usage-definition.html}{IBM AI Risk Atlas - Incomplete usage definition}\newline
\end{definitionbox}
\begin{definitionbox}{Copyright infringement}
A model might generate content that is similar or identical to existing work protected by copyright or covered by open-source license agreement.\newline\newline
\textbf{Concern: }Laws and regulations concerning the use of content that looks the same or closely similar to other copyrighted data are largely unsettled and can vary from country to country, providing challenges in determining and implementing compliance.\newline\newline
\textbf{Type: }output\newline
\textbf{Descriptor: }specific \newline\newline
\textbf{Implementation details: } \newline
ID: atlas-copyright-infringement \newline
Tag: copyright-infringement \newline
URI:  \href{https://www.ibm.com/docs/en/watsonx/saas?topic=SSYOK8/wsj/ai-risk-atlas/copyright-infringement.html}{IBM AI Risk Atlas - Copyright infringement}\newline
\end{definitionbox}
\begin{definitionbox}{Lack of data transparency}
Lack of data transparency is due to insufficient documentation of training or tuning dataset details. \newline\newline
\textbf{Concern: }Transparency is important for legal compliance and AI ethics. Information on the collection and preparation of training data, including how it was labeled and by who are necessary to understand model behavior and suitability. Details about how the data risks were determined, measured, and mitigated are important for evaluating both data and model trustworthiness. Missing details about the data might make it more difficult to evaluate representational harms, data ownership, provenance, and other data-oriented risks. The lack of standardized requirements might limit disclosure as organizations protect trade secrets and try to limit others from copying their models.\newline\newline
\textbf{Type: }non-technical\newline
\textbf{Descriptor: }amplified \newline\newline
\textbf{Implementation details: } \newline
ID: atlas-lack-of-data-transparency \newline
Tag: lack-of-data-transparency \newline
URI:  \href{https://www.ibm.com/docs/en/watsonx/saas?topic=SSYOK8/wsj/ai-risk-atlas/lack-of-data-transparency.html}{IBM AI Risk Atlas - Lack of data transparency}\newline
\end{definitionbox}
\begin{definitionbox}{Impact on affected communities}
It is important to include the perspectives or concerns of communities that are affected by model outcomes when designing and building models. Failing to include these perspectives makes it difficult to understand the relevant context for the model and to engender trust within these communities.\newline\newline
\textbf{Concern: }Failing to engage with communities that are affected by a model's outcomes might result in harms to those communities and societal backlash.\newline\newline
\textbf{Type: }non-technical\newline
\textbf{Descriptor: }traditional \newline\newline
\textbf{Implementation details: } \newline
ID: atlas-impact-on-affected-communities \newline
Tag: impact-on-affected-communities \newline
URI:  \href{https://www.ibm.com/docs/en/watsonx/saas?topic=SSYOK8/wsj/ai-risk-atlas/impact-on-affected-communities.html}{IBM AI Risk Atlas - Impact on affected communities}\newline
\end{definitionbox}
\begin{definitionbox}{Improper retraining}
Using undesirable output (for example, inaccurate, inappropriate, and user content) for retraining purposes can result in unexpected model behavior.\newline\newline
\textbf{Concern: }Repurposing generated output for retraining a model without implementing proper human vetting increases the chances of undesirable outputs to be incorporated into the training or tuning data of the model. In turn, this model can generate even more undesirable output.\newline\newline
\textbf{Type: }training-data\newline
\textbf{Descriptor: }amplified \newline\newline
\textbf{Implementation details: } \newline
ID: atlas-improper-retraining \newline
Tag: improper-retraining \newline
URI:  \href{https://www.ibm.com/docs/en/watsonx/saas?topic=SSYOK8/wsj/ai-risk-atlas/improper-retraining.html}{IBM AI Risk Atlas - Improper retraining}\newline
\end{definitionbox}
\begin{definitionbox}{Spreading toxicity}
Generative AI models might be used intentionally to generate hateful, abusive, and profane (HAP) or obscene content.\newline\newline
\textbf{Concern: }Toxic content might negatively affect the well-being of its recipients. A model that has this potential must be properly governed.\newline\newline
\textbf{Type: }output\newline
\textbf{Descriptor: }specific \newline\newline
\textbf{Implementation details: } \newline
ID: atlas-spreading-toxicity \newline
Tag: spreading-toxicity \newline
URI:  \href{https://www.ibm.com/docs/en/watsonx/saas?topic=SSYOK8/wsj/ai-risk-atlas/spreading-toxicity.html}{IBM AI Risk Atlas - Spreading toxicity}\newline
\end{definitionbox}
\begin{definitionbox}{Inaccessible training data}
Without access to the training data, the types of explanations a model can provide are limited and more likely to be incorrect.\newline\newline
\textbf{Concern: }Low quality explanations without source data make it difficult for users, model validators, and auditors to understand and trust the model.\newline\newline
\textbf{Type: }output\newline
\textbf{Descriptor: }amplified \newline\newline
\textbf{Implementation details: } \newline
ID: atlas-inaccessible-training-data \newline
Tag: inaccessible-training-data \newline
URI:  \href{https://www.ibm.com/docs/en/watsonx/saas?topic=SSYOK8/wsj/ai-risk-atlas/inaccessible-training-data.html}{IBM AI Risk Atlas - Inaccessible training data}\newline
\end{definitionbox}
\begin{definitionbox}{Impact on education: bypassing learning}
Easy access to high-quality generative models might result in students that use AI models to bypass the learning process.\newline\newline
\textbf{Concern: }AI models are quick to find solutions or solve complex problems. These systems can be misused by students to bypass the learning process. The ease of access to these models results in students having a superficial understanding of concepts and hampers further education that might rely on understanding those concepts.\newline\newline
\textbf{Type: }non-technical\newline
\textbf{Descriptor: }specific \newline\newline
\textbf{Implementation details: } \newline
ID: atlas-bypassing-learning \newline
Tag: bypassing-learning \newline
URI:  \href{https://www.ibm.com/docs/en/watsonx/saas?topic=SSYOK8/wsj/ai-risk-atlas/bypassing-learning.html}{IBM AI Risk Atlas - Impact on education: bypassing learning}\newline
\end{definitionbox}
\begin{definitionbox}{Untraceable attribution}
The content of the training data used for generating the model's output is not accessible.\newline\newline
\textbf{Concern: }Without the ability to access training data content, the possibility of using source attribution techniques can be severely limited or impossible. This makes it difficult for users, model validators, and auditors to understand and trust the model.\newline\newline
\textbf{Type: }output\newline
\textbf{Descriptor: }amplified \newline\newline
\textbf{Implementation details: } \newline
ID: atlas-untraceable-attribution \newline
Tag: untraceable-attribution \newline
URI:  \href{https://www.ibm.com/docs/en/watsonx/saas?topic=SSYOK8/wsj/ai-risk-atlas/untraceable-attribution.html}{IBM AI Risk Atlas - Untraceable attribution}\newline
\end{definitionbox}
\begin{definitionbox}{Evasion attack}
Evasion attacks attempt to make a model output incorrect results by slightly perturbing the input data that is sent to the trained model.\newline\newline
\textbf{Concern: }Evasion attacks alter model behavior, usually to benefit the attacker.\newline\newline
\textbf{Type: }inference\newline
\textbf{Descriptor: }amplified \newline\newline
\textbf{Implementation details: } \newline
ID: atlas-evasion-attack \newline
Tag: evasion-attack \newline
URI:  \href{https://www.ibm.com/docs/en/watsonx/saas?topic=SSYOK8/wsj/ai-risk-atlas/evasion-attack.html}{IBM AI Risk Atlas - Evasion attack}\newline
\end{definitionbox}
\begin{definitionbox}{Impact on the environment}
AI, and large generative models in particular, might produce increased carbon emissions and increase water usage for their training and operation.\newline\newline
\textbf{Concern: }Training and operating large AI models, building data centers, and manufacturing specialized hardware for AI can consume large amounts of water and energy, which contributes to carbon emissions. Additionally, water resources that are used for cooling AI data center servers can no longer be allocated for other necessary uses. If not managed, these could exacerbate climate change. \newline\newline
\textbf{Type: }non-technical\newline
\textbf{Descriptor: }amplified \newline\newline
\textbf{Implementation details: } \newline
ID: atlas-impact-on-the-environment \newline
Tag: impact-on-the-environment \newline
URI:  \href{https://www.ibm.com/docs/en/watsonx/saas?topic=SSYOK8/wsj/ai-risk-atlas/impact-on-the-environment.html}{IBM AI Risk Atlas - Impact on the environment}\newline
\end{definitionbox}
\begin{definitionbox}{Over- or under-reliance}
In AI-assisted decision-making tasks, reliance measures how much a person trusts (and potentially acts on) a model's output. Over-reliance occurs when a person puts too much trust in a model, accepting a model's output when the model's output is likely incorrect. Under-reliance is the opposite, where the person doesn't trust the model but should.\newline\newline
\textbf{Concern: }In tasks where humans make choices based on AI-based suggestions, over/under reliance can lead to poor decision making because of the misplaced trust in the AI system, with negative consequences that increase with the importance of the decision.\newline\newline
\textbf{Type: }output\newline
\textbf{Descriptor: }amplified \newline\newline
\textbf{Implementation details: } \newline
ID: atlas-over-or-under-reliance \newline
Tag: over-or-under-reliance \newline
URI:  \href{https://www.ibm.com/docs/en/watsonx/saas?topic=SSYOK8/wsj/ai-risk-atlas/over-or-under-reliance.html}{IBM AI Risk Atlas - Over- or under-reliance}\newline
\end{definitionbox}
\begin{definitionbox}{Incorrect risk testing}
A metric selected to measure or track a risk is incorrectly selected, incompletely measuring the risk, or measuring the wrong risk for the given context.\newline\newline
\textbf{Concern: }If the metrics do not measure the risk as intended, then the understanding of that risk will be incorrect and mitigations might not be applied. If the model's output is consequential, this might result in societal, reputational, or financial harm.\newline\newline
\textbf{Type: }non-technical\newline
\textbf{Descriptor: }amplified \newline\newline
\textbf{Implementation details: } \newline
ID: atlas-incorrect-risk-testing \newline
Tag: incorrect-risk-testing \newline
URI:  \href{https://www.ibm.com/docs/en/watsonx/saas?topic=SSYOK8/wsj/ai-risk-atlas/incorrect-risk-testing.html}{IBM AI Risk Atlas - Incorrect risk testing}\newline
\end{definitionbox}
\begin{definitionbox}{Membership inference attack}
A membership inference attack repeatedly queries a model to determine whether a given input was part of the model's training. More specifically, given a trained model and a data sample, an attacker samples the input space, observing outputs to deduce whether that sample was part of the model's training. \newline\newline
\textbf{Concern: }Identifying whether a data sample was used for training data can reveal what data was used to train a model. Possibly giving competitors insight into how a model was trained and the opportunity to replicate the model or tamper with it. Models that include publicly-available data are at higher risk of such attacks.\newline\newline
\textbf{Type: }inference\newline
\textbf{Descriptor: }amplified \newline\newline
\textbf{Implementation details: } \newline
ID: atlas-membership-inference-attack \newline
Tag: membership-inference-attack \newline
URI:  \href{https://www.ibm.com/docs/en/watsonx/saas?topic=SSYOK8/wsj/ai-risk-atlas/membership-inference-attack.html}{IBM AI Risk Atlas - Membership inference attack}\newline
\end{definitionbox}
\begin{definitionbox}{Confidential data in prompt}
Confidential information might be included as a part of the prompt that is sent to the model.\newline\newline
\textbf{Concern: }If not properly developed to secure confidential data, the model might reveal confidential information or IP in the generated output. Additionally, end users' confidential information might be unintentionally collected and stored.\newline\newline
\textbf{Type: }inference\newline
\textbf{Descriptor: }specific \newline\newline
\textbf{Implementation details: } \newline
ID: atlas-confidential-data-in-prompt \newline
Tag: confidential-data-in-prompt \newline
URI:  \href{https://www.ibm.com/docs/en/watsonx/saas?topic=SSYOK8/wsj/ai-risk-atlas/confidential-data-in-prompt.html}{IBM AI Risk Atlas - Confidential data in prompt}\newline
\end{definitionbox}
\begin{definitionbox}{Data privacy rights alignment}
Existing laws could include providing data subject rights such as opt-out, right to access, and right to be forgotten.\newline\newline
\textbf{Concern: }Improper usage or a request for data removal could force organizations to retrain the model, which is expensive.\newline\newline
\textbf{Type: }training-data\newline
\textbf{Descriptor: }amplified \newline\newline
\textbf{Implementation details: } \newline
ID: atlas-data-privacy-rights \newline
Tag: data-privacy-rights \newline
URI:  \href{https://www.ibm.com/docs/en/watsonx/saas?topic=SSYOK8/wsj/ai-risk-atlas/data-privacy-rights.html}{IBM AI Risk Atlas - Data privacy rights alignment}\newline
\end{definitionbox}
\begin{definitionbox}{IP information in prompt}
Copyrighted information or other intellectual property might be included as a part of the prompt that is sent to the model.\newline\newline
\textbf{Concern: }Inclusion of such data might result in it being disclosed in the model output. In addition to accidental disclosure, prompt data might be used for other purposes like model evaluation and retraining, and might appear in their output if not properly removed.\newline\newline
\textbf{Type: }inference\newline
\textbf{Descriptor: }specific \newline\newline
\textbf{Implementation details: } \newline
ID: atlas-ip-information-in-prompt \newline
Tag: ip-information-in-prompt \newline
URI:  \href{https://www.ibm.com/docs/en/watsonx/saas?topic=SSYOK8/wsj/ai-risk-atlas/ip-information-in-prompt.html}{IBM AI Risk Atlas - IP information in prompt}\newline
\end{definitionbox}
\begin{definitionbox}{Prompt leaking}
A prompt leak attack attempts to extract a model's system prompt (also known as the system message).\newline\newline
\textbf{Concern: }A successful attack copies the system prompt used in the model. Depending on the content of that prompt, the attacker might gain access to valuable information, such as sensitive personal information or intellectual property, and might be able to replicate some of the functionality of the model.\newline\newline
\textbf{Type: }inference\newline
\textbf{Descriptor: }specific \newline\newline
\textbf{Implementation details: } \newline
ID: atlas-prompt-leaking \newline
Tag: prompt-leaking \newline
URI:  \href{https://www.ibm.com/docs/en/watsonx/saas?topic=SSYOK8/wsj/ai-risk-atlas/prompt-leaking.html}{IBM AI Risk Atlas - Prompt leaking}\newline
\end{definitionbox}
\begin{definitionbox}{Hallucination}
Hallucinations generate factually inaccurate or untruthful content with respect to the model's training data or input. This is also sometimes referred to lack of faithfulness or lack of groundedness.\newline\newline
\textbf{Concern: }Hallucinations can be misleading. These false outputs can mislead users and be incorporated into downstream artifacts, further spreading misinformation. False output can harm both owners and users of the AI models. In some uses, hallucinations can be particularly consequential.\newline\newline
\textbf{Type: }output\newline
\textbf{Descriptor: }specific \newline\newline
\textbf{Implementation details: } \newline
ID: atlas-hallucination \newline
Tag: hallucination \newline
URI:  \href{https://www.ibm.com/docs/en/watsonx/saas?topic=SSYOK8/wsj/ai-risk-atlas/hallucination.html}{IBM AI Risk Atlas - Hallucination}\newline
\end{definitionbox}
\begin{definitionbox}{Legal accountability}
Determining who is responsible for an AI model is challenging without good documentation and governance processes.\newline\newline
\textbf{Concern: }If ownership for development of the model is uncertain, regulators and others might have concerns about the model. It would not be clear who would be liable and responsible for the problems with it or can answer questions about it. Users of models without clear ownership might find challenges with compliance with future AI regulation.\newline\newline
\textbf{Type: }non-technical\newline
\textbf{Descriptor: }amplified \newline\newline
\textbf{Implementation details: } \newline
ID: atlas-legal-accountability \newline
Tag: legal-accountability \newline
URI:  \href{https://www.ibm.com/docs/en/watsonx/saas?topic=SSYOK8/wsj/ai-risk-atlas/legal-accountability.html}{IBM AI Risk Atlas - Legal accountability}\newline
\end{definitionbox}
\begin{definitionbox}{Prompt priming}
Because generative models tend to produce output like the input provided, the model can be prompted to reveal specific kinds of information. For example, adding personal information in the prompt increases its likelihood of generating similar kinds of personal information in its output. If personal data was included as part of the model's training, there is a possibility it could be revealed.\newline\newline
\textbf{Concern: }Jailbreaking attacks can be used to alter model behavior and benefit the attacker. \newline\newline
\textbf{Type: }inference\newline
\textbf{Descriptor: }specific \newline\newline
\textbf{Implementation details: } \newline
ID: atlas-prompt-priming \newline
Tag: prompt-priming \newline
URI:  \href{https://www.ibm.com/docs/en/watsonx/saas?topic=SSYOK8/wsj/ai-risk-atlas/prompt-priming.html}{IBM AI Risk Atlas - Prompt priming}\newline
\end{definitionbox}
\begin{definitionbox}{Reidentification}
Even with the removal or personal identifiable information (PII) and sensitive personal information (SPI) from data, it might be possible to identify persons due to correlations to other features available in the data.\newline\newline
\textbf{Concern: }Including irrelevant but highly correlated features to personal information for model training can increase the risk of reidentification.\newline\newline
\textbf{Type: }training-data\newline
\textbf{Descriptor: }traditional \newline\newline
\textbf{Implementation details: } \newline
ID: atlas-reidentification \newline
Tag: reidentification \newline
URI:  \href{https://www.ibm.com/docs/en/watsonx/saas?topic=SSYOK8/wsj/ai-risk-atlas/reidentification.html}{IBM AI Risk Atlas - Reidentification}\newline
\end{definitionbox}
\begin{definitionbox}{Attribute inference attack}
An attribute inference attack repeatedly queries a model to detect whether certain sensitive features can be inferred about individuals who participated in training a model. These attacks occur when an adversary has some prior knowledge about the training data and uses that knowledge to infer the sensitive data.\newline\newline
\textbf{Concern: }With a successful attack, the attacker can gain valuable information such as sensitive personal information or intellectual property.\newline\newline
\textbf{Type: }inference\newline
\textbf{Descriptor: }amplified \newline\newline
\textbf{Implementation details: } \newline
ID: atlas-attribute-inference-attack \newline
Tag: attribute-inference-attack \newline
URI:  \href{https://www.ibm.com/docs/en/watsonx/saas?topic=SSYOK8/wsj/ai-risk-atlas/attribute-inference-attack.html}{IBM AI Risk Atlas - Attribute inference attack}\newline
\end{definitionbox}
\begin{definitionbox}{Poor model accuracy}
Poor model accuracy occurs when a model's performance is insufficient to the task it was designed for. Low accuracy might occur if the model is not correctly engineered, or there are changes to the model's expected inputs.\newline\newline
\textbf{Concern: }Inadequate model performance can adversely affect end users and downstream systems that are relying on correct output. In cases where model output is consequential, this might result in societal, reputational, or financial harm.\newline\newline
\textbf{Type: }inference\newline
\textbf{Descriptor: }amplified \newline\newline
\textbf{Implementation details: } \newline
ID: atlas-poor-model-accuracy \newline
Tag: poor-model-accuracy \newline
URI:  \href{https://www.ibm.com/docs/en/watsonx/saas?topic=SSYOK8/wsj/ai-risk-atlas/poor-model-accuracy.html}{IBM AI Risk Atlas - Poor model accuracy}\newline
\end{definitionbox}
\begin{definitionbox}{Data transfer restrictions}
Laws and other restrictions can limit or prohibit transferring data.\newline\newline
\textbf{Concern: }Data transfer restrictions can also impact the availability of the data that is required for training an AI model and can lead to poorly represented data.\newline\newline
\textbf{Type: }training-data\newline
\textbf{Descriptor: }traditional \newline\newline
\textbf{Implementation details: } \newline
ID: atlas-data-transfer \newline
Tag: data-transfer \newline
URI:  \href{https://www.ibm.com/docs/en/watsonx/saas?topic=SSYOK8/wsj/ai-risk-atlas/data-transfer.html}{IBM AI Risk Atlas - Data transfer restrictions}\newline
\end{definitionbox}
\begin{definitionbox}{Generated content ownership and IP}
Legal uncertainty about the ownership and intellectual property rights of AI-generated content.\newline\newline
\textbf{Concern: }Laws and regulations that relate to the ownership of AI-generated content are largely unsettled and can vary from country to country. Not being able to identify the owner of an AI-generated content might negatively impact AI-supported creative tasks.\newline\newline
\textbf{Type: }non-technical\newline
\textbf{Descriptor: }specific \newline\newline
\textbf{Implementation details: } \newline
ID: atlas-generated-content-ownership \newline
Tag: generated-content-ownership \newline
URI:  \href{https://www.ibm.com/docs/en/watsonx/saas?topic=SSYOK8/wsj/ai-risk-atlas/generated-content-ownership.html}{IBM AI Risk Atlas - Generated content ownership and IP}\newline
\end{definitionbox}
\begin{definitionbox}{Output bias}
Generated content might unfairly represent certain groups or individuals.\newline\newline
\textbf{Concern: }Bias can harm users of the AI models and magnify existing discriminatory behaviors.\newline\newline
\textbf{Type: }output\newline
\textbf{Descriptor: }specific \newline\newline
\textbf{Implementation details: } \newline
ID: atlas-output-bias \newline
Tag: output-bias \newline
URI:  \href{https://www.ibm.com/docs/en/watsonx/saas?topic=SSYOK8/wsj/ai-risk-atlas/output-bias.html}{IBM AI Risk Atlas - Output bias}\newline
\end{definitionbox}
\begin{definitionbox}{Dangerous use}
Generative AI models might be used with the sole intention of harming people.\newline\newline
\textbf{Concern: }Large language models are often trained on vast amounts of publicly-available information that may include information on harming others. A model that has this potential must be carefully evaluated for such content and properly governed.\newline\newline
\textbf{Type: }output\newline
\textbf{Descriptor: }specific \newline\newline
\textbf{Implementation details: } \newline
ID: atlas-dangerous-use \newline
Tag: dangerous-use \newline
URI:  \href{https://www.ibm.com/docs/en/watsonx/saas?topic=SSYOK8/wsj/ai-risk-atlas/dangerous-use.html}{IBM AI Risk Atlas - Dangerous use}\newline
\end{definitionbox}
\begin{definitionbox}{Unexplainable output}
Explanations for model output decisions might be difficult, imprecise, or not possible to obtain.\newline\newline
\textbf{Concern: }Foundation models are based on complex deep learning architectures, making explanations for their outputs difficult. Inaccessible training data could limit the types of explanations a model can provide. Without clear explanations for model output, it is difficult for users, model validators, and auditors to understand and trust the model. Wrong explanations might lead to over-trust.\newline\newline
\textbf{Type: }output\newline
\textbf{Descriptor: }amplified \newline\newline
\textbf{Implementation details: } \newline
ID: atlas-unexplainable-output \newline
Tag: unexplainable-output \newline
URI:  \href{https://www.ibm.com/docs/en/watsonx/saas?topic=SSYOK8/wsj/ai-risk-atlas/unexplainable-output.html}{IBM AI Risk Atlas - Unexplainable output}\newline
\end{definitionbox}
\begin{definitionbox}{Human exploitation}
When workers who train AI models such as ghost workers are not provided with adequate working conditions, fair compensation, and good health care benefits that also include mental health.\newline\newline
\textbf{Concern: }Foundation models still depend on human labor to source, manage, and program the data that is used to train the model. Human exploitation for these activities might negatively impact the society and human welfare. \newline\newline
\textbf{Type: }non-technical\newline
\textbf{Descriptor: }amplified \newline\newline
\textbf{Implementation details: } \newline
ID: atlas-human-exploitation \newline
Tag: human-exploitation \newline
URI:  \href{https://www.ibm.com/docs/en/watsonx/saas?topic=SSYOK8/wsj/ai-risk-atlas/human-exploitation.html}{IBM AI Risk Atlas - Human exploitation}\newline
\end{definitionbox}
\begin{definitionbox}{Toxic output}
Toxic output occurs when the model produces hateful, abusive, and profane (HAP) or obscene content. This also includes behaviors like bullying.\newline\newline
\textbf{Concern: }Hateful, abusive, and profane (HAP) or obscene content can adversely impact and harm people interacting with the model.\newline\newline
\textbf{Type: }output\newline
\textbf{Descriptor: }specific \newline\newline
\textbf{Implementation details: } \newline
ID: atlas-toxic-output \newline
Tag: toxic-output \newline
URI:  \href{https://www.ibm.com/docs/en/watsonx/saas?topic=SSYOK8/wsj/ai-risk-atlas/toxic-output.html}{IBM AI Risk Atlas - Toxic output}\newline
\end{definitionbox}
\begin{definitionbox}{Data poisoning}
A type of adversarial attack where an adversary or malicious insider injects intentionally corrupted, false, misleading, or incorrect samples into the training or fine-tuning datasets.\newline\newline
\textbf{Concern: }Poisoning data can make the model sensitive to a malicious data pattern and produce the adversary's desired output. It can create a security risk where adversaries can force model behavior for their own benefit.\newline\newline
\textbf{Type: }training-data\newline
\textbf{Descriptor: }traditional \newline\newline
\textbf{Implementation details: } \newline
ID: atlas-data-poisoning \newline
Tag: data-poisoning \newline
URI:  \href{https://www.ibm.com/docs/en/watsonx/saas?topic=SSYOK8/wsj/ai-risk-atlas/data-poisoning.html}{IBM AI Risk Atlas - Data poisoning}\newline
\end{definitionbox}
\begin{definitionbox}{Unreliable source attribution}
Source attribution is the AI system's ability to describe from what training data it generated a portion or all its output. Since current techniques are based on approximations, these attributions might be incorrect.\newline\newline
\textbf{Concern: }Low-quality attributions make it difficult for users, model validators, and auditors to understand and trust the model.\newline\newline
\textbf{Type: }output\newline
\textbf{Descriptor: }specific \newline\newline
\textbf{Implementation details: } \newline
ID: atlas-unreliable-source-attribution \newline
Tag: unreliable-source-attribution \newline
URI:  \href{https://www.ibm.com/docs/en/watsonx/saas?topic=SSYOK8/wsj/ai-risk-atlas/unreliable-source-attribution.html}{IBM AI Risk Atlas - Unreliable source attribution}\newline
\end{definitionbox}
\begin{definitionbox}{Harmful output}
A model might generate language that leads to physical harm The language might include overtly violent, covertly dangerous, or otherwise indirectly unsafe statements.\newline\newline
\textbf{Concern: }A model generating harmful output can cause immediate physical harm or create prejudices that might lead to future harm.\newline\newline
\textbf{Type: }output\newline
\textbf{Descriptor: }specific \newline\newline
\textbf{Implementation details: } \newline
ID: atlas-harmful-output \newline
Tag: harmful-output \newline
URI:  \href{https://www.ibm.com/docs/en/watsonx/saas?topic=SSYOK8/wsj/ai-risk-atlas/harmful-output.html}{IBM AI Risk Atlas - Harmful output}\newline
\end{definitionbox}
\begin{definitionbox}{Confidential information in data}
Confidential information might be included as part of the data that is used to train or tune the model.\newline\newline
\textbf{Concern: }If confidential data is not properly protected, there could be an unwanted disclosure of confidential information. The model might expose confidential information in the generated output or to unauthorized users.\newline\newline
\textbf{Type: }training-data\newline
\textbf{Descriptor: }amplified \newline\newline
\textbf{Implementation details: } \newline
ID: atlas-confidential-information-in-data \newline
Tag: confidential-information-in-data \newline
URI:  \href{https://www.ibm.com/docs/en/watsonx/saas?topic=SSYOK8/wsj/ai-risk-atlas/confidential-information-in-data.html}{IBM AI Risk Atlas - Confidential information in data}\newline
\end{definitionbox}
\begin{definitionbox}{Lack of model transparency}
Lack of model transparency is due to insufficient documentation of the model design, development, and evaluation process and the absence of insights into the inner workings of the model.\newline\newline
\textbf{Concern: }Transparency is important for legal compliance, AI ethics, and guiding appropriate use of models. Missing information might make it more difficult to evaluate risks,  change the model, or reuse it.  Knowledge about who built a model can also be an important factor in deciding whether to trust it. Additionally, transparency regarding how the model's risks were determined, evaluated, and mitigated also play a role in determining model risks, identifying model suitability, and governing model usage.\newline\newline
\textbf{Type: }non-technical\newline
\textbf{Descriptor: }traditional \newline\newline
\textbf{Implementation details: } \newline
ID: atlas-lack-of-model-transparency \newline
Tag: lack-of-model-transparency \newline
URI:  \href{https://www.ibm.com/docs/en/watsonx/saas?topic=SSYOK8/wsj/ai-risk-atlas/lack-of-model-transparency.html}{IBM AI Risk Atlas - Lack of model transparency}\newline
\end{definitionbox}
\begin{definitionbox}{Unrepresentative data}
Unrepresentative data occurs when the training or fine-tuning data is not sufficiently representative of the underlying population or does not measure the phenomenon of interest.\newline\newline
\textbf{Concern: }If the data is not representative, then the model will not work as intended.\newline\newline
\textbf{Type: }training-data\newline
\textbf{Descriptor: }traditional \newline\newline
\textbf{Implementation details: } \newline
ID: atlas-unrepresentative-data \newline
Tag: unrepresentative-data \newline
URI:  \href{https://www.ibm.com/docs/en/watsonx/saas?topic=SSYOK8/wsj/ai-risk-atlas/unrepresentative-data.html}{IBM AI Risk Atlas - Unrepresentative data}\newline
\end{definitionbox}
\begin{definitionbox}{Impact on human agency}
AI might affect the individuals' ability to make choices and act independently in their best interests.\newline\newline
\textbf{Concern: }AI can generate false or misleading information that looks real.  It may simplify the ability of nefarious actors to generate realistically looking false or misleading content with intention to manipulate human thoughts and behavior. When false or misleading content that is generated by AI is spread, people might not recognize it as false information leading to a distorted understanding of the truth. People might experience reduced agency when exposed to false or misleading information since they may use false assumptions in their decision process.\newline\newline
\textbf{Type: }non-technical\newline
\textbf{Descriptor: }amplified \newline\newline
\textbf{Implementation details: } \newline
ID: atlas-impact-on-human-agency \newline
Tag: impact-on-human-agency \newline
URI:  \href{https://www.ibm.com/docs/en/watsonx/saas?topic=SSYOK8/wsj/ai-risk-atlas/impact-on-human-agency.html}{IBM AI Risk Atlas - Impact on human agency}\newline
\end{definitionbox}
\begin{definitionbox}{Personal information in prompt}
Personal information or sensitive personal information that is included as a part of a prompt that is sent to the model.\newline\newline
\textbf{Concern: }If personal information or sensitive personal information is included in the prompt, it might be unintentionally disclosed in the models' output. In addition to accidental disclosure, prompt data might be stored or later used for other purposes like model evaluation and retraining, and might appear in their output if not properly removed. \newline\newline
\textbf{Type: }inference\newline
\textbf{Descriptor: }specific \newline\newline
\textbf{Implementation details: } \newline
ID: atlas-personal-information-in-prompt \newline
Tag: personal-information-in-prompt \newline
URI:  \href{https://www.ibm.com/docs/en/watsonx/saas?topic=SSYOK8/wsj/ai-risk-atlas/personal-information-in-prompt.html}{IBM AI Risk Atlas - Personal information in prompt}\newline
\end{definitionbox}
\begin{definitionbox}{Nonconsensual use}
Generative AI models might be intentionally used to imitate people through deepfakes by using video, images, audio, or other modalities without their consent.\newline\newline
\textbf{Concern: }Deepfakes can spread disinformation about a person, possibly resulting in a negative impact on the person's reputation. A model that has this potential must be properly governed.\newline\newline
\textbf{Type: }output\newline
\textbf{Descriptor: }amplified \newline\newline
\textbf{Implementation details: } \newline
ID: atlas-nonconsensual-use \newline
Tag: nonconsensual-use \newline
URI:  \href{https://www.ibm.com/docs/en/watsonx/saas?topic=SSYOK8/wsj/ai-risk-atlas/nonconsensual-use.html}{IBM AI Risk Atlas - Nonconsensual use}\newline
\end{definitionbox}
\begin{definitionbox}{Decision bias}
Decision bias occurs when one group is unfairly advantaged over another due to decisions of the model. This might be caused by biases in the data and also amplified as a result of the model's training.\newline\newline
\textbf{Concern: }Bias can harm persons affected by the decisions of the model.\newline\newline
\textbf{Type: }output\newline
\textbf{Descriptor: }traditional \newline\newline
\textbf{Implementation details: } \newline
ID: atlas-decision-bias \newline
Tag: decision-bias \newline
URI:  \href{https://www.ibm.com/docs/en/watsonx/saas?topic=SSYOK8/wsj/ai-risk-atlas/decision-bias.html}{IBM AI Risk Atlas - Decision bias}\newline
\end{definitionbox}
\begin{definitionbox}{Lack of testing diversity}
AI model risks are socio-technical, so their testing needs input from a broad set of disciplines and diverse testing practices.\newline\newline
\textbf{Concern: }Without diversity and the relevant experience, an organization might not correctly or completely identify and test for AI risks.\newline\newline
\textbf{Type: }non-technical\newline
\textbf{Descriptor: }amplified \newline\newline
\textbf{Implementation details: } \newline
ID: atlas-lack-of-testing-diversity \newline
Tag: lack-of-testing-diversity \newline
URI:  \href{https://www.ibm.com/docs/en/watsonx/saas?topic=SSYOK8/wsj/ai-risk-atlas/lack-of-testing-diversity.html}{IBM AI Risk Atlas - Lack of testing diversity}\newline
\end{definitionbox}
\begin{definitionbox}{Exposing personal information}
When personal identifiable information (PII) or sensitive personal information (SPI) are used in training data, fine-tuning data, or as part of the prompt, models might reveal that data in the generated output. Revealing personal information is a type of data leakage.\newline\newline
\textbf{Concern: }Sharing people's PI impacts their rights and make them more vulnerable.\newline\newline
\textbf{Type: }output\newline
\textbf{Descriptor: }amplified \newline\newline
\textbf{Implementation details: } \newline
ID: atlas-exposing-personal-information \newline
Tag: exposing-personal-information \newline
URI:  \href{https://www.ibm.com/docs/en/watsonx/saas?topic=SSYOK8/wsj/ai-risk-atlas/exposing-personal-information.html}{IBM AI Risk Atlas - Exposing personal information}\newline
\end{definitionbox}
\begin{definitionbox}{Improper data curation}
Improper collection and preparation of training or tuning data includes data label errors and by using data with conflicting information or misinformation.\newline\newline
\textbf{Concern: }Improper data curation can adversely affect how a model is trained, resulting in a model that does not behave in accordance with the intended values. Correcting problems after the model is trained and deployed might be insufficient for guaranteeing proper behavior. \newline\newline
\textbf{Type: }training-data\newline
\textbf{Descriptor: }amplified \newline\newline
\textbf{Implementation details: } \newline
ID: atlas-data-curation \newline
Tag: data-curation \newline
URI:  \href{https://www.ibm.com/docs/en/watsonx/saas?topic=SSYOK8/wsj/ai-risk-atlas/data-curation.html}{IBM AI Risk Atlas - Improper data curation}\newline
\end{definitionbox}
\begin{definitionbox}{Revealing confidential information}
When confidential information is used in training data, fine-tuning data, or as part of the prompt, models might reveal that data in the generated output. Revealing confidential information is a type of data leakage.\newline\newline
\textbf{Concern: }If not properly developed to secure confidential data, the model might reveal confidential information or IP in the generated output and reveal information that was meant to be secret.\newline\newline
\textbf{Type: }output\newline
\textbf{Descriptor: }amplified \newline\newline
\textbf{Implementation details: } \newline
ID: atlas-revealing-confidential-information \newline
Tag: revealing-confidential-information \newline
URI:  \href{https://www.ibm.com/docs/en/watsonx/saas?topic=SSYOK8/wsj/ai-risk-atlas/revealing-confidential-information.html}{IBM AI Risk Atlas - Revealing confidential information}\newline
\end{definitionbox}
\begin{definitionbox}{Spreading disinformation}
Generative AI models might be used to intentionally create misleading or false information to deceive or influence a targeted audience.\newline\newline
\textbf{Concern: }Spreading disinformation might affect human's ability to make informed decisions. A model that has this potential must be properly governed.\newline\newline
\textbf{Type: }output\newline
\textbf{Descriptor: }specific \newline\newline
\textbf{Implementation details: } \newline
ID: atlas-spreading-disinformation \newline
Tag: spreading-disinformation \newline
URI:  \href{https://www.ibm.com/docs/en/watsonx/saas?topic=SSYOK8/wsj/ai-risk-atlas/spreading-disinformation.html}{IBM AI Risk Atlas - Spreading disinformation}\newline
\end{definitionbox}
%\end{document}



\end{document}
