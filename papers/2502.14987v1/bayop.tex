\section{Proof-of-Concept Controller}
\label{sec:cachetrace}
Motivated by our prior findings and modeling work, we present an example controller to help validate our conjecture of a black-box search strategy. This controller uses an established machine learning technique, Bayesian optimization~\cite{frazier, garnett_bayesoptbook_2022}, to find energy-efficient interrupt coalescing and CPU frequency settings that can adapt to the specific application, OS, and hardware while exploiting the stability in offered loads. 

In 1) \cref{sec:cachetrace2}, we illustrate its applicability in optimizing the energy efficiency of a server that supports a realistic datacenter workload trace~\cite{cacheWorkload-OSDI20} over 24 hours by periodically adjusting ITR-delay, and DVFS settings as offered load changes, and in 2) \cref{sec:tailbench} demonstrates the generality of the controller as we apply it across different types of NICs and CPUs (\cref{table:hwsw}) when run on three applications from Tailbench~\cite{tailbench}.

As a proof-of-concept controller, we made simplfying assumptions in its design and leave addressing real deployment scenarios for future work. Our assumptions include ad-hoc thresholds for when to trigger Bayesian optimization and the number of subsequent trials to run. However, our results show that even using straightforward assumptions can yield significant advantages, leaving ample room for improvement.

The architecture of our controller also facilitates the integration of more advanced policies for initiating the Bayesian process. We envision the deployment of this technique in data centers through collaboration with load-balancers that make use of historical usage data. This collaboration would help distribute incoming loads to energy-optimized servers, which have been pre-configured with specific settings, while still meeting SLA objectives. In addition, the load balancers can help mitigate the potential gaming of the learning agent's behavior in response to changing request rates.

\subsection{Design}
Fig.~\ref{fig:bayop} illustrates the design of our  controller: \circled{1} A live system running memcached services requests arriving at varying QPSes from an external source. \circled{2} It then triggers a set of performance and energy measurements of the live system to be shared with an external Ax~\cite{ax, Bakshy2018AEAD} Bayesian optimization platform. \circled{3} This process then computes a penalty $Rp$ of the current (ITR-delay, DVFS) setting and \circled{4} then recommends an update to a new (ITR-delay, DVFS) configuration on the live system that minimizes $Rp$. Once this process completes, the live system is set with a fixed (ITR-delay, DVFS) configuration until the next set of measurements is triggered.

\subsubsection{Penalty Function}
We use a simple function that penalizes the optimization process by the amount of measured energy and magnifies that penalty when measured latency violates the SLA objective:

\begin{equation} \label{eq:bayesopt_reward}
    Rp = m\_energy * max((m\_latency - SLA + 1), 1)
\end{equation}

For example, with an SLA objective of 99\% tail latency < 500 $\mu$s, where measured latency ($m\_latency$) is 600 $\mu$s, the reward $Rp$ will be scaled up by a factor of 100, such that $Rp = m\_energy * (600 - 500)$. If $m\_latency$ is less than $SLA$, then $Rp$ will evaluate to $m\_energy$. Minimizing $Rp$ is indicative that Bayesian optimization is selecting (ITR-delay, DVFS) pairs to meet performance/energy objectives. This reward function enables an operator to express their preference to optimizing for different combinations of energy and performance objectives.

The possibilities of customizing this function further are also ripe for exploration: such as using new combinations of performance/energy or known metrics such as energy-delay-product~\cite{573184,10.1109/40.888701}. One can also imagine developing a rich set of reward functions that capture preferences a service operator might have. In this way, the controller can be reconfigured as priorities change by selecting and tuning from the set of reward functions.
\subsection{Applicability to \textit{cache-trace}}
\label{sec:cachetrace2}
This section presents the results of running our controller against a publicly available KV store workload trace (\textit{cache-trace}~\cite{cacheWorkload-OSDI20}) which exhibits the stable demand curve behavior for our controller. 

\subsubsection{Experimental Setup}
We used the same infrastructure of our study (\cref{sec:study}) but modified the \textit{mutilate} workload generator to generate QPSes following from \textit{cache-trace} instead: first, we extracted a 24-hour sequence of QPS rates from a single trace and binned the data into hourly divisions to capture the mean QPS rate at an hourly basis. As cache-trace QPS rates were often in the tens of thousands of QPS as it was running on limited vCPUs, we scaled up the rates to match our hardware capability. However, \ref{sec:open2} shows that even at low QPS rates where DVFS is fixed at the lowest CPU frequency, ITR-delay can still be used to further reduce energy use. Therefore, we then generate these scaled-up mean QPSes to our live memcached server for which we capture energy-per-second measurements over the entire 24-hour period. 

The controller is configured to trigger its periodic measurements at an hourly rate and run Bayesian optimization for a default of 30 trials - this is due to overheads in our single-thread Python package; which takes around 5 minutes to run. In contrast to our initial energy study (\cref{sec:study}), which was limited to only using up to 340 (ITR-delay,  DVFS) pairs due to experimental scope, our controller allows Bayesian optimization to choose from all available ITR-delay, and DVFS values (a total of \textit{2 million} possible combinations). 

%\paragraph*{}
We evaluate our controller by comparing the energy and performance behavior of five different system configurations:
\begin{itemize}
    \item \textbf{Linux}: Operating in its default state, where the dynamic ITR-delay and DVFS  algorithms are enabled.
    \item \textbf{Linux-BayOp and EbbRT-BayOp}: Operating with Bayesian Optimization to tune both ITR-delay and DVFS, with a target of minimizing overall energy use while maintaining SLA objectives.
    \item \textbf{Linux-DVFS-BayOp and Linux-ITR-BayOp}: Operating with Bayesian Optimization to tune only one of the two settings. We were motivated to explore these configurations to better understand the limitations of the two hardware mechanisms individually. \textit{Linux-DVFS-BayOp} tunes DVFS while enabling the dynamic ITR-delay algorithm. \textit{Linux-ITR-BayOp} tunes ITR-delay while enabling the dynamic DVFS algorithm.
\end{itemize}


\subsubsection{Evaluation}
We evaluate our controller's energy impact across two
applications, namely memcached and silo, in both Linux and EbbRT\footnote{The controller's penalty can also be modified to minimize latency, details can be found in \cref{sec:appendix}}. Silo~\cite{mcdsilo, zygos} is a compute and memory-intensive application that is extended with a web front-end such that every request triggers a corresponding set of TPC-C transactions on an in-memory database~\cite{silo}. We ported Silo to EbbRT and the workload mix and SLA constraints of Silo follow from those used in memcached.


\paragraph{Memcached Results}
Fig.~\ref{fig:mcd_bayop} illustrates our controllers evaluation against three different SLA objectives: 99\% latency < 500 $\mu$s, 90\% latency < 500 $\mu$s, and a even more stringent 99\% latency < 200 $\mu$s. The QPS values, shown on the right, change on an hourly basis, as shown by black line segments. At the beginning of each hourly QPS change, we see spikes in energy usage of \textit{*-BayOp} systems which results from the Bayesian Optimization process searching through (ITR-delay,  DVFS) settings on the memcached server to meet its optimization objective. After this initial energy spike, the system settles into a steady energy consumption state until the next hourly trigger. A key result of this application is the importance of using both ITR-delay and DVFS to meet SLA objectives for optimizing energy efficiency rather than individually.

\begin{figure}[!htb]
\centering
\begin{subfigure}{.45\textwidth}
  \centering
  \includegraphics[width=\linewidth]{bayesopt_mcd_lat500_per99_energy.png}
\end{subfigure}
\begin{subfigure}{.45\textwidth}
  \centering
  \includegraphics[width=\linewidth]{bayesopt_mcd_lat500_per90_energy.png}
\end{subfigure}
\begin{subfigure}{.45\textwidth}
  \centering
  \includegraphics[width=\linewidth]{bayesopt_mcd_lat200_per99_energy.png}
\end{subfigure}%
\caption{\small \texttt{BayOp} applied to memcached; the \textbf{QPS} label is shown on the right side of the graph and the QPS lines show the different offered loads on a per-hour basis. We present results from different SLA objectives studied and illustrate the measured power (energy/second) on the Y-axis as QPS changes across the five system configurations studied over 24 hours (X-axis).}
\label{fig:mcd_bayop}
\vspace{-0.1in}
\end{figure}

\begin{figure*}[ht!]
\centering
\begin{subfigure}{.49\textwidth}
  \includegraphics[width=\linewidth]{bayesopt_mcdsilo_lat500_per99_energy.png}
\end{subfigure}
\begin{subfigure}{.49\textwidth}
  \includegraphics[width=\linewidth]{bayesopt_mcdsilo_lat500_per90_energy.png}
\end{subfigure}
\caption{\small Controller applied to \textit{cache-trace} for Silo. We show two different SLA objectives. The \textbf{QPS} line shows the change in QPS offered load on a per-hour basis. The consumed power (energy/second) of each system configuration on the Y-axis is shown over 24 hours on the X-axis.}
\label{fig:mcdsilo_bayop}
\vspace{-0.15in}
\end{figure*}

\begin{figure}[h!]
\centering
    \includegraphics[width=0.45\textwidth]{bayesopt_mcd_lat200_per99_lat.png}
    \caption{\small Measured 99\% latency across Linux for an SLA of 200 $\mu$s. The latency is shown on a per-hour basis due to how \textit{mutilate} reports its resultant latency measurements. We find that \textit{Linux-DVFS-BayOp} often violates the SLA which suggests only tuning DVFS is not enough to achieve stable system behavior.}
    \label{fig:bayesopt_mcd_lat200_per99_lat}
    \vspace{-0.25in}
\end{figure}

We find that, for an SLA objective of 99\% latency < 500 $\mu$s, \textit{Linux-BayOp} can result in energy savings of up to 50\% over \textit{Linux}. Relaxing the SLA objective to 90\% < 500 $\mu$s enables our controller to find (ITR-delay,  DVFS) configurations that yield even more energy savings of over 60\%. At the most stringent SLA of 99\% latency < 200 $\mu$s, our controller can still adapt while enabling energy savings of up to 30\%. 
The energy savings of \textit{EbbRT-BayOp} are similar to those found in our energy study of memcached (Fig.~\ref{fig:mcd_overview}). Our controller is robust enough to adapt to the software stack of EbbRT and find energy-efficient configurations that consistently result in the lowest energy use (over 2X lower than \textit{Linux}). The measured energy-per-second variability of EbbRT is often lower in contrast to that of Linux (indicated by the thinner red plot in Fig.~\ref{fig:mcd_bayop}), a byproduct of EbbRT's simplified and more optimized network paths.

For \textit{Linux-ITR-BayOp}, allowing our controller to tune only ITR-delay still generally improved energy savings over \textit{Linux}. However, for a stringent SLA of 99\% latency < 200 $\mu$s, the reduced SLA headroom prevents the controller from trading off latency for energy as effectively as it can when tuning alongside DVFS. At the lower QPSes, \textit{Linux-ITR-BayOp} performed worse than \textit{Linux}.

Allowing our controller to tune only DVFS (\textit{Linux-DVFS-BayOp}) results in energy savings comparable with \textit{Linux-BayOp} across SLA objectives. This is further supported by \ref{sec:open2} which illustrates the significant influence of DVFS on overall energy consumption. However, though it may seem that under a more stringent SLA of 99\% latency < 200 $\mu$s, \textit{Linux-DVFS-BayOp} results in the highest energy savings, we found instances where the measured 99\% latency violated the SLA of 200 $\mu$s, as shown in Fig.~\ref{fig:bayesopt_mcd_lat200_per99_lat}; revealing the weakness of relying on DVFS only.

\begin{figure*}[!htb]
\begin{subfigure}{\textwidth}
  \includegraphics[width=\linewidth]{tail1.pdf}
\end{subfigure}
\begin{subfigure}{\textwidth}
  \includegraphics[width=\linewidth]{tail2.pdf}
\end{subfigure}
\caption{\small This figure illustrates the energy use of each application (\textbf{APP}) for each of the hardware platforms (\textbf{NODE: N0 to N3}). The energy is normalized (Y-axis) against Linux default, where lower is better. For each \textbf{APP}, we use two representative offered loads which are 40\% and 80\% of the measured \textbf{Peak Load} of Linux default. Within each representative offered load, we also selected two \textbf{SLAs} (as indicated by \textbf{/////} and \textbf{.....}) for the application to meet while our controller is optimizing its energy efficiency.}
\label{fig:xl170}
\vspace{-0.1in}
\end{figure*}
\paragraph{Silo Results}
We selected another trace from \textit{cache-trace} that was akin to a more computationally intense server. Fig.~\ref{fig:mcdsilo_bayop} shows that the trace peak QPS rates are often lower than those of Fig.~\ref{fig:mcd_bayop} (peak 250K QPS versus 750K QPS). Fig.~\ref{fig:mcdsilo_bayop} does not show results for SLA of 99\% latency < 200 $\mu$s, as the inherent computational cost of Silo's TPC-C transactions resulted in a lower bound of measured latency values that were consistently greater than the SLA objective of 200 $\mu$s. A key result of this application is that it helps expose in computationally intensive cases the limitation of ITR-delay to affect energy savings.

Fig.~\ref{fig:mcdsilo_bayop} illustrates that even for a computationally intensive application with different SLA objectives, \textit{Linux-BayOp} was able to find (ITR-delay,  DVFS) settings that enable 30\% energy savings in Linux for various QPS rates and higher winnings when the SLA is relaxed to 90\% latency < 500 $\mu$s.

The controller was able to adapt to a different OS and application stack and found configurations of \textit{EbbRT-BayOp} that consistently had the lowest energy use over Linux. In contrast to Fig.~\ref{fig:mcd_bayop}, one can see larger variations in energy saved from one QPS to the next (more hilly behavior). This can be partly attributed to the complicated database work that must now be done per request.

In contrast to memcached, we find that tuning ITR-delay alone (\textit{Linux-ITR-BayOp}) while enabling Linux's default DVFS mechanism is largely ineffective at reducing energy. This is likely due to the increased computational cost for each request which limits the potential energy savings gained from interrupt coalescing and prolonged sleep states that are induced by the ITR-delay mechanism.

We find that tuning DVFS alone (\textit{Linux-DVFS-BayOp}) while enabling Linux's default ITR-delay mechanism works surprisingly well for Silo and, in most cases, achieves a slight energy saving over \textit{Linux-BayOp}. This result suggests an interesting compromise between enabling a degree of energy savings that controlling ITR-delay provides to a computationally-driven network application versus abandoning ITR-delay control so that Bayesian optimization can focus on tuning DVFS to maximize energy savings.


\subsection{Black-Box Generality: Diverse Apps and Hardware}
\label{sec:tailbench}
\begin{table}[t]
\small
\begin{tabular}{|c|c|c|c|c|}
    \hline
  Name & CPU & Cores & NIC & RAM\\ \hline
     N0 & Intel E5-2640 & 8 & Mellanox 25GbE & 62GB\\ \hline
     N1 & Intel E5-2660 & 20 & Solarflare 10GbE & 128GB\\ \hline
     N2 & AMD EPYC 7452 & 32 & Mellanox 40GbE & 128GB\\ \hline
     N3 & Ampere ARMv8 & 80 & Mellanox 25GbE & 124 GB \\ \hline
\end{tabular}
\caption{\small Different hardware explored to run Tailbench.}
\label{table:hwsw}
\vspace{-0.2in}
\end{table}

In this section, we further demonstrate the versatility of the controller by applying it to optimize energy efficiency for three applications from Tailbench~\cite{tailbench}. Our motivation was to reveal how externally controlling interrupt coalescing and CPU frequency can be applied agnostically on hardware even across multi-generational divides\footnote{Scripts to reproduce results at https://anonymous.4open.science/r/bayop-188B}.


\subsubsection{Experimental Setup}
For these experiments, we selected four hardware platforms as shown in \cref{table:hwsw}. Nodes N0, N1, and N2 are provided by CloudLab~\cite{cloudlab} and we disable hyperthreads and TurboBoost on all processors to minimize system noise. For each node type, we create a cluster consisting of a single server node, three client nodes that generate traffic to the server node, and an external bootstrap node that launches experiments and runs the \texttt{BayOp} controller to tune interrupt coalescing (ITR-delay) and CPU frequency (DVFS) on the server. All of the nodes were running Linux 5.15 kernel; we only examined Linux as EbbRT does not have the necessary device driver support for Solarflare and Mellanox NICs. Notably, while we used \texttt{ethtool} to set static interrupt rates across all three NICs in this paper, the fundamental implementation may be different depending on the hardware's capability. On the Intel processors, we use the RAPL hardware registers~\cite{rapl} to report its dynamic energy use while for AMD, we use \texttt{amd\_energy} hardware monitor driver~\cite{amdenergy}.  

Node N3 is another experimental node that runs Linux 6.4.13 but we could only get a single client node to generate traffic\footnote{Due to the computation-heavy nature of Tailbench applications, we found this was still able to saturate the single server}. The ARMv8 server provided \texttt{xgene-hwmon}~\cite{armxgene} tool that enabled us to report its power readings. 

For each hardware category, we selected applications from Tailbench~\cite{tailbench}, each designed to fulfill distinct SLA objectives. These applications encompassed \textbf{img-dnn}, a handwriting recognition program built on OpenCV; \textbf{sphinx}, an open-source search engine; and \textbf{xapian}, a speech recognition system. These applications both represent a diverse suite of benchmarks in contrast to the previous examples from our study as well as providing new SLA objectives in the order of milliseconds to seconds. Overall, these selections allowed us to assess the impact of different SLAs and hardware platforms.


\subsubsection{Experimental Results}
In our experiments, we observed that the controller consistently achieves energy savings ranging from 5\% to 36\%, depending on the specific combination of software and hardware. Importantly, our findings underscore the fundamental nature of these two mechanisms, which can be effectively applied across a variety of hardware platforms in different SLA-driven application domains. Further, we found that the generic architecture of our controller meant that it was straightforward to simply deploy this technique in new hardware environments as long as it provided support for energy readings and exposed control of interrupt coalescing and CPU frequency. 

Fig.~\ref{fig:xl170} depicts the resulting energy consumption for Tailbench; we normalize the energy usage relative to the default Linux configurations under different scenarios:
\begin{enumerate}
    \item We selected representative offered loads of 40\% and 80\% of each hardware platform's peak QPS capacity for running the respective applications.
    \item For each of these offered loads, we applied two distinct SLA objectives tailored to each application, as indicated by the labels in each figure. These SLA objectives were derived from default values provided by the authors of Tailbench~\cite{tailbench}.
\end{enumerate}

However, it is worth pointing out that the controller's ability to adapt to applications and offered loads is heavily influenced by the hardware's ability to offer a range of configurations for exploration within this space. One can see an example of this for \textbf{APP: img-dnn} in \textbf{Node: N2} where it did not manage to find an ITR-delay, DVFS pair that managed to further reduce energy consumption. We hypothesize this stems from a combination of the application type as well as the DVFS settings provided by the AMD EPYC 7452 processor. The processor uses AMD's Collaborative Processor Performance Control (CPPC) interface~\cite{amdpstate}, which is an abstracted performance value that isn't tied to specific a CPU frequency; further, we were limited to only three settings in contrast to the hundreds and thousands available on the other processors. However, this limitation can also be mitigated by newer processors that support the AMD P-state EPP~\cite{amdepp} driver, providing finer-grained CPU frequency settings.

\begin{figure}[!htb]
\centering
    \includegraphics[width=0.45\textwidth]{cdf1.png}
    \caption{\small CDF of per request latency between Linux and Linux-BayOp from a single Tailbench application.}
    \label{fig:cdf1}
    \vspace{-0.2in}
\end{figure}

To delve into the energy gains 
we detail the CDF of an example Tailbench application in \cref{fig:cdf1}. In this figure, we illustrate the per-request latency as provided by Tailbench when running the img-dnn application on our ARMv8 server (note this is at a particular peak load and SLA). As the figure shows, the overall request latency of \textit{Linux-BayOP} is about 2X worse than Linux as the controller chose energy-efficient ITR-delay and DVFS settings. While we found Linux was able to support this workload with a 99\% latency of 2.8ms, \textit{Linux-BayOp} was still able to meet the SLA at 99\% latency of 4.8ms while saving 31\% energy.

