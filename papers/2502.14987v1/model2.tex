\section{Modeling ITR-delay and DVFS Effects}
\label{sec:model}
A key takeaway from \cref{sec:study} is that \cref{fig:closed_overview} and \cref{fig:mcd_overview} reveal common shapes ("V" and "L") that are OS-agnostic and share a stable structure in response to changes to ITR-delay and DVFS. This suggests one can develop a formal model that captures OS-agnostic performance and energy profiles and that generic external control mechanisms can then be made feasible for both OSes.

\subsection{Memcached Model Fitting}
Motivated by the implications of these OS shapes, we formulated a mathematical model to explore fitting our experimental data with a set of free parameters and ITR-delay, DVFS settings. We assume a simple model where the offered load is light enough that requests don't batch up in the receive queue and can be treated independently. We model the performance as 99\% tail latency as well as energy consumed \footnote{We have also collected other tail latency values and found that our model can accurately fit them as well.}. 

\subsubsection{Performance:} We define $\triangle t$ as the time it takes to handle a single request:
$$ \triangle t = t_{work} + t_{interrupt} $$

We parameterize $t_{work}$ as a function of DVFS values:

\begin{equation} \label{eq:t_work}
t_{\text{work}} = \frac{Z}{DVFS^{1+\alpha}}
\end{equation}

$Z$ and $\alpha$ are free parameters that change for both the OS and application load. In this model, $Z$ acts as a maximum time limit that each request can take (i.e. SLA objective). $\alpha$ represents a system's dependence on DVFS to trade off performance for energy. For example, if $\alpha = -1$, then that particular system has no dependence on DVFS and can largely use DVFS to lower energy use without sacrificing performance - this is inspired by the study results in \cref{sec:open1} that illustrate how DVFS affects Linux and EbbRT differently in trading off energy for latency.


We parameterize $t_{interrupt}$ as a function of ITR-delay values:
\begin{equation} \label{eq:t_interrupt}
    t_{\text{interrupt}} = \phi * ITRdelay
\end{equation}

As ITR-delay greatly affects the measured tail latency, $\phi$ represents the location in the receive queue where a packet is placed before being processed. For example, if an unlucky packet arrives just as the NIC's $ITR$ value starts counting down, then it will have to artificially wait for a full $ITR$ before being processed, thereby delaying overall request processing time. 

\subsubsection{Energy:} We define $\triangle J$ as the amount of energy it takes to process a single request. This is affected by the voltage and frequency states of DVFS and how ITR-delay can induce prolonged idle periods: 

\begin{equation} \label{eq:open_energy}
\triangle J = \gamma*(\phi*ITR)*DVFS^\beta 
\end{equation}

Note that $\phi$ used here is the same variable from \cref{eq:t_interrupt}. $\gamma$ (units of watts) acts to convert the interactions of ITR-delay and DVFS into energy used. The variable $\beta$ acts as a dependence factor on DVFS in a similar way to $\alpha$ in \cref{eq:t_work}.

Fig.~\ref{fig:mcd_model_600K} illustrates one example result of the model fitting against memcached data for a QPS of 600K. The x-axis shows the set of energy and performance predictions and the y-axis shows their measured values. The diagonal lines show where ideal points would lie if our model's calculations were exact. We use the Adam optimizer from PyTorch~\cite{pytorchadam} in this process and run each fit several times to check the stability of the inferred parameters and avoid getting stuck in local minima. Overall, we find that despite complicated interactions between the hardware and software, our models are expressive enough to exploit the common OS response behaviors to accurately fit both performance and energy data. 

\begin{figure}
\centering
    \includegraphics[width=0.48\textwidth]{mcd_model_600K.png}
    \caption{\small Prediction of energy and performance across both OSes using our model for Memcached @ 600K QPS. The diagonal line indicates a perfect model fit. The negative energy values are due to \texttt{log()} transformations during modeling for regression analysis.}
    \label{fig:mcd_model_600K}
    \vspace{-0.2in}
\end{figure}

\begin{figure*}[htb!]
\centering
\includegraphics[width=0.85\textwidth]{bayop.png}
\caption[]
{\small Our controller designed to optimize energy efficiency for a memcached server.}
\label{fig:bayop}
\vspace{-0.15in}
\end{figure*}

However, the limitation of our model is that it is only practical in highly constrained settings. To replicate this approach for new software and hardware, one would need to exhaustively re-gather data while tuning ITR-delay and DVFS. Nevertheless, the accuracy of our model suggests the viability of using similar approaches to search this space.
