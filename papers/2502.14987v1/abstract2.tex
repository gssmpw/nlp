
In this paper, we demonstrate that a server running a single latency-sensitive application can be treated as a black box to reduce energy consumption while meeting an SLA target. We find that when the mean offered load is stable, one can find the ``sweet spot'' settings in packet batching (via interrupt coalescing) and controlling the processing rate (DVFS) that represents optimal trade-offs in the interactions of the software stack and hardware with the arrival rate and composition of requests currently being served. Trying a few combinations of settings on the live system, an example Bayesian optimizer can find settings that reduce the energy consumption to meet a desired tail latency for the current load.

This research demonstrates that: 1) without software changes, dramatic energy savings (up to 60\%) can be achieved across diverse hardware systems if one controls batching and processing rate, 2) specialized research OSes that have been developed for performance can achieve more than 2x better energy efficiency than general-purpose OSes, and 3) a controller, agnostic to the application and system, can easily find energy-efficient settings for the offered load that meets SLA objectives.