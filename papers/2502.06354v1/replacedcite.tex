\section{Related Work}
\label{sec:relatedwork}
\subsection{Supervised Denoising Methods}
Numerous methods for image noise reduction have been developed to date. Research by Zhang, Zamir, and others has proposed methods using CNNs and Vision Transformers. Furthermore, Luo and colleagues have achieved noise removal using diffusion models. All these methods focus on general images containing synthetic Gaussian noise, and to our knowledge, no studies have specifically targeted the quality improvement of photoacoustic images. Therefore, our approach, which effectively utilizes imaging condition information, is considered superior in improving the quality of photoacoustic images.

\subsection{Guidance Methods}

As a method for conditional image generation based on specific classes using diffusion models, Classifier Guidance____ can be mentioned. This technique mixes the classifier's gradients with the estimates of the diffusion model, enabling more stable image generation that reflects class information. In Classifier-free Guidance____, instead of using classifier gradients, the estimates of the conditional and unconditional diffusion models are mixed based on the following equation:
\begin{equation}
    \tilde{\bm{\epsilon}}_\theta(\bm{x}_t,t,c) = (1+w)\bm{\epsilon}_\theta(\bm{x}_t,t,c) - w\bm{\epsilon}_\theta(\bm{x}_t,t,\phi),
    \label{eq:cfg}
\end{equation}
where $w$ represents the strength of guidance, and $\phi$ signifies the unconditional token used for unconditional generation. Adjusting $w$ achieves a trade-off between diversity in image generation and fidelity to class information. This method is not limited to class-conditional generation and has also been used in Text to Image tasks, based on textual information ____. However, there are few studies applying such guidance in Image to Image tasks.

\begin{figure*}[t]
    \centering
    \includegraphics[scale=0.2, width=\linewidth]{image/proposed4.pdf}
    % \vspace{1mm}
    \caption{}
    % \vspace{1mm}
    \label{fig:proposed}
\end{figure*}