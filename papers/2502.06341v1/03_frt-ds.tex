% This is samplepaper.tex, a sample chapter demonstrating the
% LLNCS macro package for Springer Computer Science proceedings;
% Version 2.20 of 2017/10/04
%
\documentclass[runningheads]{llncs}
%
\usepackage[
    left = `,% 
    right = ',% 
    leftsub = ``,% 
    rightsub = '' %
]{dirtytalk}
\usepackage{longtable}
\usepackage{multirow}
\usepackage{subcaption}
\usepackage{tabularx}   
\usepackage{url}  
\usepackage{graphicx}
% Used for displaying a sample figure. If possible, figure files should
% be included in EPS format.
\usepackage{hyperref}
\hypersetup{
    colorlinks=true,
    linkcolor=blue,
    filecolor=magenta,      
    urlcolor=cyan,
    pdftitle={Overleaf Example},
    pdfpagemode=FullScreen,
    }
\urlstyle{same}
% If you use the hyperref package, please uncomment the following line
% to display URLs in blue roman font according to Springer's eBook style:
\renewcommand\UrlFont{\color{blue}\rmfamily}
%for tables
\usepackage{colortbl}
\usepackage{fancybox}
\usepackage{xcolor}
% \usepackage{fancyhdr}
% \fancyhf{}
% \renewcommand{\headrulewidth}{0pt}
% \fancyfoot[c]{}
% \fancypagestyle{FirstPage}{
% \lfoot{This preprint has not undergone peer review (when applicable) or any post-submission improvements or corrections. The Version of Record of this contribution is published in Machine Learning and Principles and Practice of Knowledge Discovery in Databases. ECML PKDD 2023. Communications in Computer and Information Science, vol 2133. Springer, Cham, and is available online at \href{https://doi.org/10.1007/978-3-031-74630-7_10}{https://doi.org/10.1007/978-3-031-74630-7\_10}} 
% }
\usepackage{tikz}
\newcommand\copyrightnotice[1]{
    \begin{tikzpicture}[remember picture,overlay]
    \node[anchor=south,yshift=10pt] at (current page.south) {\fbox{\parbox{\dimexpr\textwidth-\fboxsep-\fboxrule\relax}{#1}}};
    \end{tikzpicture}
}
\begin{document}

%
\title{Facial Analysis Systems and Down Syndrome}
% \title{Facial Analysis Systems and Down Syndrome\thanks{This study was carried out within the FAIR - Future Artificial Intelligence Research and received funding from the European Union Next-GenerationEU (PIANO NAZIONALE DI RIPRESA E RESILIENZA (PNRR) – MISSIONE 4 COMPONENTE 2, INVESTIMENTO 1.3 – D.D. 1555 11/10/2022, PE00000013). This manuscript reflects only the authors’ views and opinions, neither the European Union nor the European Commission can be considered responsible for them.}}
%
%\titlerunning{Abbreviated paper title}
% If the paper title is too long for the running head, you can set
% an abbreviated paper title here
%
%\author{Anonymous Author(s)}
%\authorrunning{Anon.}
 \author{Marco Rondina\inst{1}\orcidID{0000-1111-2222-3333} \and
 Fabiana Vinci \and
 Antonio Vetrò\inst{1}\orcidID{1111-2222-3333-4444} \and
 Juan Carlos De Martin\inst{1}\orcidID{2222-3333-4444-5555}}
 \institute{Politecnico di Torino, Corso Duca degli Abruzzi 24, 10129 Torino, Italy
 \email{\{marco.rondina,antonio.vetro,demartin\}@polito.it}}

 \authorrunning{M. Rondina, F. Vinci et al.}
%
\maketitle              % typeset the header of the contribution
%
\begin{abstract}
The ethical, social and legal issues surrounding facial analysis technologies have been widely debated in recent years.
Key critics have argued that these technologies can perpetuate bias and discrimination, particularly against marginalized groups.
We contribute to this field of research by reporting on the limitations of facial analysis systems with the faces of people with Down syndrome: this particularly vulnerable group has received very little attention in the literature so far.

This study involved the creation of a specific dataset of face images.
An experimental group with faces of people with Down syndrome, and a control group with faces of people who are not affected by the syndrome.
Two commercial tools were tested on the dataset, along three tasks: gender recognition, age prediction and face labelling.

The results show an overall lower accuracy of prediction in the experimental group, and other specific patterns of performance differences: i) high error rates in gender recognition in the category of males with Down syndrome; ii) adults with Down syndrome were more often incorrectly labelled as children; iii) social stereotypes are propagated in both the control and experimental groups, with labels related to aesthetics more often associated with women, and labels related to education level and skills more often associated with men. 

These results, although limited in scope, shed new light on the biases that alter face classification when applied to faces of people with Down syndrome. 
They confirm the structural limitation of the technology, which is inherently dependent on the datasets used to train the models.
\keywords{Datasets \and Face recognition \and Face attribute estimation \and Gender recognition \and Age estimation \and Image labelling \and AI and disability \and Down syndrome \and AI bias.}
\end{abstract}
%
%
%
\section{Introduction and motivation}\label{sec:introduction}
% \thispagestyle{FirstPage}
\copyrightnotice{This preprint has not undergone peer review (when applicable) or any post-submission improvements or corrections. The Version of Record of this contribution is published in ``Machine Learning and Principles and Practice of Knowledge Discovery in Databases. ECML PKDD 2023. Communications in Computer and Information Science, vol 2133. Springer, Cham.'', and is available online at \href{https://doi.org/10.1007/978-3-031-74630-7_10}{https://doi.org/10.1007/978-3-031-74630-7\_10}}
% Face recognition technologies (FRTs) are technologies focused on detection, verification, and categorization or classification of individuals \cite{europeanparliamentRegulatingFacialRecognition2021}. 
% They belong to the field of biometric technologies, applications that make use of biometric information to perform designed tasks 
% \footnote{According to the features predicted by these technologies, different terminologies are used to identify the specific application. In particular, face attribute classification, is defined as the field of FRTs with the aim of classify people according to the gender, race or ethnicity; face attribute estimation is used when attributes are numbers, like age; face attribute detection are focused on recognize the presence of accessories (scarf, glasses etc.) or face attribute (beard, moustaches). This study investigated some specific fields of FRTs: face attribute classification, detection and estimation and for the sake of simplicity the acronym FRTs is the one used in this paper to refer to all the applications mentioned above.}.  

In recent years, the ethical, social and legal implications of Facial Analysis Systems (FASs) arose in several parts of the world.
% qui manca qualcosa
% Politicians have also taken action.
Several municipalities and governments banned the use of facial recognition technologies in public spaces, such as the city of San Francisco \cite{sanfranciscoboardofsupervisorsOrdinanceNo107192019}. 
In Italy, a moratorium \cite{parlamentoitalianoTestoCoordinatoDecretolegge2021} suspended the use of these systems in public spaces and by private agencies, except for activities related to criminal justice.
The European Commission proposed a restriction to 'real-time' remote biometric identification systems \cite{europeancommissionProposalRegulationEuropean2021} while the European Parliament Research Services published an analysis on the regulation of facial recognition in the EU \cite{europeanparliamentRegulatingFacialRecognition2021}.

One of the main criticisms of FASs is their potential to perpetuate biases and discrimination. 
This is particularly the case for marginalized groups such as people of colour \cite{westDiscriminatingSystemsGender2019,buolamwiniGenderShadesIntersectional2018}, or individuals with non-binary gender and transgender identities \cite{melendezUberDriverTroubles2018,vincentTransgenderYouTubersHad2017}. 
% This is similar to other machine learning and artificial intelligence technologies.
The amount of evidence and the implications of these issues are so significant that major companies have slowed development. 
Some decided to remove gender prediction from their models \cite{hillMicrosoftPlansEliminate2022} or to oppose the use of facial recognition technology for certain purposes \cite{krishnaIBMCEOLetter2019}.

This paper is part of the strand of research into the bias of FASs. 
It focuses on the limitations of FASs in relation to people with Down syndrome, an overlooked vulnerable group in this field.
We built a dataset of images of people with and without Down syndrome (200 in each group, equally divided by binary gender) and used it to test and compare the classification of two commercial tools. 
% The results show that overall performances of predictions for people with Down syndrome are lower than the ones of people without the syndrome, probably due to the low representativeness of this specific category of vulnerable subjects in the training data. A typical answer to the problem would be to include more people with Down syndrome in the training data, or to develop ad-hoc models for them: we are sceptical of this kind of technical fix.
% The motivations of the research was not to merely to test the robustness of technology towards the typical face features of the Down-syndrome and then advise on how to improve classification.
Our motivation was to provide new evidence on the structural limitations of FASs and their high dependability from training data. 
We did this from the perspective of a vulnerable social group that is often excluded from the design process and from the most popular discourses on AI bias and discrimination. 

This paper is organized as follows: in Section \ref{sec:rel_work} we position our work in relation to previous studies related to the topic of the research.
In Section \ref{sec:design}, we describe the design of our study, highlighting the research questions, the methodology used to create the test dataset and the details of the tested models.
In Section \ref{sec:results} we present the results of our experiment for the three different tasks analysed, and the related discussion.
Section \ref{sec:threats-validity-limitation} outlines the threats to validity and ethical concerns associated with the current study. 
Section \ref{sec:conclusions} summarizes the reflections on the whole experiment.
Finally, section \ref{sec:future_work} explores possible future implementations.
%the algorithms should be constructed and fed with data that represent all the differences of the faces in order to avoid discriminations and to have a wider perspective of the diversity of faces of human being.
% The main characteristics of the Down syndrome in terms of facial features are flattened faces, almond-shapes eyes, and many others.\mrnote{fonte?}
% The facial features of the EG should be learned by models in order to have a wider perspective of the diversity of faces of human being.
%It is not possible to know, a priori, whether the EG is included in the training set, but it is interesting to examine the results of the models predicting gender, age and generic labels on this specific dataset.

%The problem of the social discrimination of people with disabilities is common and has been exacerbated by digital and technological discrimination. Some examples are the limited accessibility to media content and web pages \cite{alonso-virgosWebPageDesign2018} and the discriminations about job recruitment area \cite{bakalisRecruitmentAIHas2020}.
%Despite this, typical debates about AI discrimination and problems focus on gender and ethnicity but, rarely, on disability.


\section{Related Works}\label{sec:rel_work}
Several scholars highlighted the ethical issues of FASs.
Crawford in \textit{Atlas of AI} \cite{crawfordAtlasAI2021} reconstructed the history and development of AI in relationships with a variety of impacts (e.g., on the environment, work, health) and highlighted the epistemic issues of FASs and their controversial historical origins.

Other studies focused on the failures of FASs and the associated negative impacts on society. 
In terms of gender and ethnicity, Buolamwini and Gebru \cite{buolamwiniGenderShadesIntersectional2018} evaluated different commercial gender classification systems, and found that darker-skinned women were the most misclassified group.
Similar findings are reported by Klare
et al. \cite{klareFaceRecognitionPerformance2012}, who found that commercial and non-trainable algorithms performed worse for women, blacks and young people.
The issue of diversity and inclusion in FASs can arise from the lack of examples of subpopulations in the train dataset, but also from the definition of classes that incorporate specific values and beliefs, as in a binary formulation of gender \cite{scheuermanHowComputersSee2019}.
The consequences of a non-inclusive operationalization of FASs can be seen in the case of transgender representation \cite{keyesMisgenderingMachinesTrans2018}.

Recent work on the unknown behaviour of neural networks, has shown which are the key features used by commercial face classification services in order to classify gender. 
In fact, lip, eye, cheek structure and make-up are more discriminative than skin and hair length \cite{muthukumarUnderstandingUnequalGender2018}.
As the authors discuss, the fact that the make-up is so important in predicting female gender is a troubling stereotype.

% **********
%Regarding the disability issue.. \avcomment{come sopra, purtroppo non si può fare un elenco senza spiegare perché gli articoli vengono citati, equivale a un farsi dare un reject solo per questo. Non per forza bisogna tenerli tutti, se il legame con questo studio è debole o non chiaro} the following studies by \citeauthor{asimSyndromeInsightDisease2015} \cite{asimSyndromeInsightDisease2015} and \citeauthor{kazemiSyndromeCurrentStatus2016} \cite{kazemiSyndromeCurrentStatus2016}, which give the scientific background of the syndrome. Following these initial studies, an interesting work by \citeauthor{damascenoFacialAnalysisSyndrome2014} \cite{damascenoFacialAnalysisSyndrome2014} in the orthodontic field, analysing the facial features of people with Down syndrome, has been of great help in understanding the physiology of this condition.
% **********

Taking the next step and linking FASs with Down syndrome, several papers are based on recognizing the disability in children in their first years of life.
Agbolade et al. \cite{agboladeSyndromeFaceRecognition2020} presented a performance comparison of different machine learning methods on the task of Down Syndrome detection.
Paredes et al. \cite{paredesEmotionRecognitionSyndrome2022} compared different machine learning and deep learning techniques to perform the emotion detection task on people with Down syndrome.
Finally, Qin et al. \cite{qinAutomaticIdentificationSyndrome2020} presented an identification method based on deep convolutional neural networks. 
The studies mentioned above were aimed at identifying the syndrome through the face or better understanding emotions through technology. 
What is not covered in previous works is whether people with Down syndrome are discriminated against, in terms of lower performance of the FASs, compared to non-affected people.
This is the observable gap that we address in this paper.
% They did not investigate whether discriminations of FASs towards people with Down syndrome occur in terms of lower performance, compared to not affected people.

%At the same time, several articles and websites, show the discriminations and perplexities of different AI technologies with respect to people with disabilities.
%Some examples of discriminations are regarding the job recruitment area, by \citeauthor{bakalisRecruitmentAIHas2020} \cite{bakalisRecruitmentAIHas2020} but also the general accessibility to the internet for people with disabilities by \citeauthor{alonso-virgosWebPageDesign2018} \cite{alonso-virgosWebPageDesign2018}.
%With respect to the experimental group of the current analysis, the problems and consequences are unknown and unexplored, leading to wrong implementations of the algorithms.


\section{Study Design}\label{sec:design}
We describe the research questions that drove the entire analysis (Section \ref{sec:rq}), how the test set was constructed (Section \ref{sec:Dataset}) and how the FASs were selected (Section \ref{sec:models}).

% \footnote{For the creation of the dataset, the guideline provided by \citeauthor{gebruDatasheetsDatasets2021} \cite{gebruDatasheetsDatasets2021} was followed and the answers to the questions, provided by the authors, are better explained. \avnote{Perché è importante citarlo qui? Quali relazioni con questo studio?}}

\subsection{Research Questions}\label{sec:rq}
% The research questions related to the current study are as follows:

\textbf{RQ1: How does the FASs work, with images of Down-syndrome people, regarding the predictions of a) gender and b) age?}

%In order to answer this question, the models of gender recognition and age prediction are applied to the specific dataset. 
%The results of both models are shown and analysed in the sections \ref{sec:gender_recognition} and \ref{sec:age_prediction}.
Previous work has shown how FASs can fail to predict age and gender for vulnerable people (Section \ref{sec:rel_work}), but never for people with disabilities.
In this work, we are interested in understanding whether predicting gender and age for people with Down syndrome work in the same way as for people without the syndrome.
The consequences of incorrect gender and age predictions are diverse and depend on the decisions that are made according to these predictions (e.g., hiring, wrong associations for personalized contents, etc.).
\newline

\textbf{RQ2: Are the labels assigned differently to people with or without Down syndrome by image recognition models?}

The labels generated by the models and associated with an input image, can be used by companies for many purposes.
The labels are generally correlated with gender, objects in the image and emotions of the people.
The aim of this part of the study is to analyse the labels of the images and to evaluate any possible differences between people with and without Down syndrome.
These differences could lead to downstream discrimination effects, depending on the application context in which the FASs are used.

\subsection{Dataset}\label{sec:Dataset}
 In the past, some researchers have used datasets of faces of people with Down syndrome.
 They focused their experiments on classifying people with or without the syndrome \cite{agboladeSyndromeFaceRecognition2020,qinAutomaticIdentificationSyndrome2020}. 
 However, they focused exclusively on children, who are not representative of the population as a whole.

It was therefore necessary to build a set of facial images of people with Down syndrome from scratch: this set is the experimental group (EG). 
Due to a lack of resources, it wasn't possible to build a sample by taking photos directly or by contacting people and asking them to send their photos (in both cases with their explicit consent).
It was therefore necessary to use images already available on the web. 
The images come from Google searches and from websites that offer free stock images, such as iStock and Pexels. 
In this way, it is impossible to know whether the individuals have given their explicit consent: for this reason, we err on the side of caution and do not redistribute them (see the discussion in section \ref{sec:threats-validity-limitation}).
Not every EG image has a referenced age: the age is known for 66 EG males and 64 EG females. 
It is also important to note that the life expectancy of people with Down syndrome is currently around 65 years \cite{kazemiSyndromeCurrentStatus2016}.
The resulting EG set consisted of 200 images.
% BOH: Perché range 0-55 con aspettativa di vita 0-65?

The control group (CG, i.e. images of people not affected by Down syndrome) also consisted of 200 images: these were collected by selecting some of the best quality images from the UTKFace dataset\footnote{\url{https://susanqq.github.io/UTKFace/}} \cite{zhangAgeProgressionRegression2017}.  
This dataset is a well-known large-scale face dataset with a long age range (from 0 to 116 years), constructed from images of famous people. 
% BOH: come sono state selezionate?

A total of 400 images make up the dataset, 100 for each of the following categories: EG male, EG female, CG male and CG female. 
% A.2: \ref{subsec:rules-clarifai}}
% 1: \ref{sec:cropped_images}
Each image in the dataset was stored in two different ways, according to the reference rule of one of the tools used (see section \ref{sec:models} that require cropped  - representing only the part of the images with the face - for gender and age detection \footnote{this is the ClarifAI model, rule number 1, described in Appendix A.2 of the Supplementary Material}).
Thus, on the one hand, the \textit{cropped} version of the images was used for gender and age recognition. 
On the other hand, the \textit{not cropped} version of the images, was used for label detection.

All images in the dataset were paired with gender and age to test the predictions of the models.
In line with the categories used by most FASs, gender is considered to be binary: we are aware of the limitations of such a representation.
Age is the corresponding age of the person at the time the photo was taken.
The information on \textit{age} was the most difficult to find. 
For the EG, we knew the age for 66 and 64 pictures of the female and male, respectively.
Instead, the CG was created using a pre-existing dataset containing at least one image for each age from 3 to 85, so that all images had the age information.
The images without age information were not used to predict age, but were used to predict gender and labels.

One of the most important aspects for good model performance is the quality of the images.
For this reason, all images were manually selected, and as a result, the majority of samples in the dataset meet the Pose, Illumination and Expression (PIE) rules \cite{phillipsIntroductionGoodBad2011}.

\subsection{Models}\label{sec:models}
The selection of services offering facial analysis, was based on a study of the most widely used and well-known commercial services, that meet some initial constraints:
\begin{itemize}
    \item the images should not be retained for use for other purposes, or at a minimum be deleted with account suspension;
    \item it should be possible to predict gender and age;
    \item there should be a free amount of test operations.
\end{itemize}
The two services that met the previous conditions are: ClarifAI and AWS Rekognition \footnote{regarding the first requirement, please refer to the policies on Appendix A in the Supplementary Material, in particular AWSR rule number 4, ClarifAI rule number 2}.
% A: \ref{sec:appendix-rules-reference}
% 4: \ref{sec:aws_data_ret}
% 2: \ref{sec:clarifai_data_ret}
Some other services were considered, but not selected because they did not meet the above conditions.
In particular: the services retain images indefinitely, as in the case of Face++ and Mega Matcher; they require a payment for the operations, Face++ and Cognitec's Face VACS; they do not provide gender recognition, Microsoft Face API, or they are deprecated, IBM Watson Visual Recognition. 

Both ClarifAI and AWS Rekognition provide different models that are used during the analysis. 
The models and their details are shown and summarized in the Supplementary Material, Table 1, Appendix B. % The models and their details are shown and summarized in the Table \ref{tab:names-models}.  
The selected services also provide some suggestions, called \textit{Rules of reference}, for the correct use of models: they are summarized and reported in the Supplementary Material, Appendix A. %\ref{sec:appendix-rules-reference}
The tests were operationalized using the SDKs for Python provided by AWSR and ClarifAI\footnote{\url{https://docs.aws.amazon.com/rekognition/latest/dg/labels-detect-labels-image.html}} \footnote{\url{https://web.archive.org/web/20211130220210/https://docs.clarifai.com/api-guide/predict/images}}. 
% The scripts are available on the websites of both services \footnote{\url{https://docs.aws.amazon.com/rekognition/latest/dg/labels-detect-labels-image.html}}\footnote{\url{https://web.archive.org/web/20211130220210/https://docs.clarifai.com/api-guide/predict/images}}.
The output of the models is a JSON file containing different types of information.

According to the research questions codified in Section \ref{sec:rq}, this paper analyses three different tasks: gender recognition, age prediction and image recognition.
% According to the choice of the services to consider the gender as binary, the two genders evaluated are female and male.
As previously mentioned, gender is categorized in a binary format: male and female.
% \textbf{Gender recognition} is a task of facial analysis, with the aim of predicting the gender of a person from images or videos.
% The gender recognition for Down-syndrome people is analysed in the current analysis and, according to the choice of the services to consider the gender as binary, the two evaluated genders are female and male.
% The models assigned different labels to the two genders, as shown in Table \ref{tab:labels-gender-age}.
Age prediction is performed differently by the two models.
%\ref{tab:labels-gender-age}
ClarifAI, assigns a probability to each of the possible age intervals for each image\footnote{Table 2 of the Appendix B in the Supplementary Material.}: the predicted age interval is the one with the highest probability.
Instead, AWSR predicts the age by assigning to each image a specific range from a \textit{'Low'} value to a \textit{'High'} value.
The consequence of such an output format is that it is not possible to obtain the same specific ranges for both models.
According to the rule of reference number 1 of the AWSR\footnote{Appendix A.1 in the Supplementary Material}, the mathematical mean of the predicted range is taken as the final output value of age.
% \ref{sec:aws_age} 
Image recognition models predict concepts, labels, themes and image properties. 
Analysis of this task provides insight into model training, in particular how the images in the training set were labelled. 
ClarifAI provides a model (\textit{general-image-recognition}), that outputs 20 different \textit{concepts} for each image.
Each \textit{concept} has a corresponding probability value. 
The AWSR model (\textit{detect labels}) assigns different labels to each image.
We will refer to the \textit{concepts} of ClarifAI as "labels" for the results of both models, ClarifAI and AWSR.
A maximum of 20 labels are predicted for each image.
For ClarifAI, the labels were grouped into ad-hoc categories. 
Instead, the AWSR model does the grouping itself by specifying pre-defined categories (as mentioned in its documentation).

All ClarifAI output labels were reviewed, analysed and then grouped according to their meaning and main theme.
In the end, the following categories were created \textit{aesthetic}, \textit{education}, \textit{person description}.
The labels of the first two categories are adjectives related to the people depicted in the photos.
\textit{Person Description} contains the same labels as the homonymous AWSR category.
The similar categories of the AWSR model (\textit{Clothing and Accessories}, \textit{Beauty and Personal Care} and \textit{Education}) contain mainly names of objects and descriptors of the images, which are not considered in the current analysis.

The peculiarity of the AWSR output labels is that they are mostly descriptive.
The list of labels is similar to a list of objects that the algorithm recognizes in the image. 
For example, considering one of its categories, \textit{Apparel and Accessories}, some labels are: Jeans, T-shirt, Hat, Shoes etc. 
Instead, looking at the predicted labels from the ClarifAI model, the labels are mainly adjectives.
Adjectives can be much more ethically dangerous than nouns. 
Therefore, we focused our analysis on the \textit{Aesthetics}, \textit{Education}, \textit{Person descriptors} categories of ClarifAI and on the category \textit{Person descriptors} of the AWSR model.

\section{Results and Discussion}\label{sec:results}

\subsection{RQ1.a - Gender Recognition}\label{sec:gender_recognition}

Table \ref{tab:gender-results} presents the values of \textit{Accuracy}, \textit{Recall}, \textit{Precision} and \textit{F1-score}, for both gender recognition models and groups.
The accuracy scores of the EG were lower than those of the CG for both models: the discrepancy between these scores was about 7\% and 4\% respectively.
In general, the results of the EG were lower than those of the CG for both models. 
The \textit{F1-scores} of the EG were lower than those of the CG.
Female \textit{Precision} and male \textit{Recall} showed lower values between EG and CG.
% In particular, the results of the male EG classes were lower than the others.
% The female class of the EG was not so different from the female class of the CG, it performed well for the AWSR model, and for its values of \textit{Recall}.

% Due to the high difference of gender predictions, a deeper inspection of the misclassified images is done.
A closer examination of the misclassified images was carried out.
On the one hand, all misclassified images of the EG male group represent children and adolescents. 
On the other hand, the misclassified images of EG females by the ClarifAI model represent old people.
% \ref{sec:aws_confidence}
The rule of reference number 2 (Appendix A.1 in the Supplementary Material) regarding the AWSR model suggests that the confidence value assigned to each prediction of gender should be checked and taken into account.
The threshold considered safe for sensible subjects is set at 99.00\% by the rules of the model.
Following the previous recommendation, a detailed examination of the confidence values is carried out on each group of the dataset, as shown in Figure \ref{fig:confidence_values}.

\begin{table}[t]
\centering
\caption{Gender recognition results.}
\label{tab:gender-results}
\begin{tabular}{ll|cccc|cccc|}
\cline{3-10}
                                                     &                 & \multicolumn{4}{c|}{\textbf{Experimental group}}                                                                                                               & \multicolumn{4}{c|}{\textbf{Control group}}                                                                                                                   \\ \hline
\multicolumn{1}{|l|}{\textbf{Model}}                 & \textbf{Gender} & \multicolumn{1}{l|}{\textbf{Acc.}}         & \multicolumn{1}{l|}{\textbf{Prec.}} & \multicolumn{1}{l|}{\textbf{Recall}} & \multicolumn{1}{l|}{\textbf{F1-score}} & \multicolumn{1}{l|}{\textbf{Acc.}}         & \multicolumn{1}{l|}{\textbf{Prec.}} & \multicolumn{1}{l|}{\textbf{Recall}} & \multicolumn{1}{l|}{\textbf{F1-score}} \\ \hline
\multicolumn{1}{|l|}{\multirow{2}{*}{\textbf{AWSR}}} & \textit{Female} & \multicolumn{1}{c|}{\multirow{2}{*}{93\%}} & \multicolumn{1}{c|}{87\%}           & \multicolumn{1}{c|}{100\%}           & 93\%                                 & \multicolumn{1}{c|}{\multirow{2}{*}{97\%}} & \multicolumn{1}{c|}{94\%}          & \multicolumn{1}{c|}{99\%}            & 96\%                                 \\ \cline{2-2} \cline{4-6} \cline{8-10} 
\multicolumn{1}{|l|}{}                               & \textit{Male}   & \multicolumn{1}{c|}{}                      & \multicolumn{1}{c|}{100\%}          & \multicolumn{1}{c|}{85\%}            & 92\%                                 & \multicolumn{1}{c|}{}                      & \multicolumn{1}{c|}{99\%}          & \multicolumn{1}{c|}{94\%}            & 96\%                                 \\ \hline
\multicolumn{1}{|l|}{\multirow{2}{*}{\textbf{CLAI}}} & \textit{Female} & \multicolumn{1}{c|}{\multirow{2}{*}{91\%}} & \multicolumn{1}{c|}{87\%}           & \multicolumn{1}{c|}{97\%}            & 92\%                                 & \multicolumn{1}{c|}{\multirow{2}{*}{98\%}} & \multicolumn{1}{c|}{97\%}          & \multicolumn{1}{c|}{98\%}            & 98\%                                 \\ \cline{2-2} \cline{4-6} \cline{8-10} 
\multicolumn{1}{|l|}{}                               & \textit{Male}   & \multicolumn{1}{c|}{}                      & \multicolumn{1}{c|}{97\%}           & \multicolumn{1}{c|}{85\%}            & 90\%                                 & \multicolumn{1}{c|}{}                      & \multicolumn{1}{c|}{98\%}          & \multicolumn{1}{c|}{97\%}            & 97\%                                 \\ \hline
\end{tabular}
\end{table}


\begin{figure}[t]
    \centering
    \begin{subfigure}[b]{\linewidth}
        \includegraphics[width=\linewidth]{images/graphs/confvalues/ConfValues_AWS.pdf}
        \caption{
            Confidence values computed by the AWSR model.
        }
        \label{fig:AWS_conf}       
    \end{subfigure}
    \\
    \begin{subfigure}[b]{\linewidth}
        \includegraphics[width=\linewidth]{images/graphs/confvalues/ConfValues_ClarifAI.pdf}
        \caption{
            Confidence values computed by the ClarifAI model.
        }
        \label{fig:ClarifAI_conf}
    \end{subfigure}
    \caption{Confidence values for each category, computed by the AWSR and ClarifAI models, regarding the gender prediction task. Green bars refer to correct prediction, purple bars refer to incorrect predictions.}
    \label{fig:confidence_values}
\end{figure}
Figures \ref{fig:AWS_conf} and \ref{fig:ClarifAI_conf} illustrate the distribution of confidence values for each class of the dataset, including the values of correct and incorrect predictions: the colours green and purple represent the correctness and incorrectness of the model prediction in the study, respectively.
The different shades of these colours represent the confidence levels and their own level of error.
% The disparity between the different colours represents a wrongly performed model. 
% The more constant the colour in a bar is, the better the model performs.
Both models performed poorly for the EG male category.
The AWSR model correctly classified 66\% of the images with a confidence level greater than or equal to 99.00\%. 
The ClarifAI model correctly classified 60\% of the images with a confidence level greater than or equal to 99.00\%. 
% The remaining part of the correct predictions gradually decreases to confidence values lower than 50.
We observe that the categories with the best performance were the female ones.
In particular the category EG female, for the AWSR model, does not contain misclassification of gender. 
Looking at the incorrect predictions, we found that  about a 5\% of the predictions of the EG male classes had a confidence value greater than or equal to 99.00\%. 
This means that the model misclassified images for which it was very confident about the prediction.
% The other bar that has more shades of colours is the one related to the category CG male, which is slightly worst compared to the one of control group female.

Finally, Table \ref{tab:cf-results} shows the accuracy values considering only the prediction with a confidence value greater or equal to 99.00\%.
Within females, the discrepancy between EG and CG was reduced.
In fact, in the case of ClarifAI, the difference between EG and CG is about 6\%, while in the case of AWSR, EG performed better than CG (2\%).
The males highlight large performance discrepancies between EG and CG. 
In the AWSR case, the difference in accuracy was 24\%, while in the case of ClarifAI the difference in accuracy was equal to 15\%.
\vspace{0.5cm}

\framebox{
    \begin{minipage}[h]{0.90\linewidth}
        We observed that the prediction of gender performed differently between EG and CG.
        The results showed a high error rate towards the EG male category.
        In particular, both models correctly predicted 85\% of the images in the EG male class, in contrast to 97\% and 94\% of the CG male class for ClarifAI and AWSR respectively.
    \end{minipage}
}

\begin{table}[t]
    \centering
    \caption{
        Accuracy values for the gender prediction task, considering only the prediction with a confidence value greater than or equal to 99.00\%. 
    }
    \label{tab:cf-results}
    \resizebox{\columnwidth}{!}                   & 93\%                                     & \multicolumn{1}{c|}{83\%}                   & 89\%                                     \\ \hline
\multicolumn{1}{|l|}{\textbf{Male}}   & \multicolumn{1}{c|}{66\%}                   & 90\%                                     & \multicolumn{1}{c|}{60\%}                   & 75\%                                     \\ \hline
\end{tabular}%
}
\end{table}


\subsection{RQ1.b - Age Prediction}\label{sec:age_prediction}
In the models analysed, age prediction is a classification problem and the result of the prediction consists of a range of ages rather than a precise value, as described in section \ref{sec:models}. 
% Regarding ClarifAI, the model provides specific ranges and for all of them a probability is calculated. Instead, on the AWSR model, ranges are specific for each image, as described above. 
% \ref{sec:aws_age}
% The guidelines of the model number 1 (Appendix A.1 in the Supplementary Material) suggests considering, as output age, the mean of the values of the predicted range.
\begin{table}[t]
\footnotesize
    \centering
    \caption{
        Accuracy values for the age prediction task. 
    }
    \label{tab:age_total_results}
    \resizebox{\columnwidth}{!}                & 45\%                   & \multicolumn{1}{c|}{48\%}                & 49\%                   \\ \hline
\multicolumn{1}{|l|}{\textbf{Incorrect Age}} & \multicolumn{1}{c|}{48\%}                & 55\%                   & \multicolumn{1}{c|}{52\%}                & 51\%                   \\ \hline
\end{tabular}%
}
\end{table}
The accuracy values are presented in Table \ref{tab:age_total_results}.
The results show that the performances of both models are low: only half of the samples are predicted correctly. 
% both models correctly predicted only half of the samples of the total dataset. 
% The overall conclusion is that the two models did not work well for both EG and CG.

The truth tables were constructed using the ClarifAI ranges for both the true range and the predicted range.
% The true age is one of the information of the dataset (when available, for EG), while the predicted age, in the case of AWSR, is the mathematical mean of the range.
Each value of the truth tables \ref{tab:truth_table_eg_clarifai}, \ref{tab:truth_table_cg_clarifai}, \ref{tab:truth_table_eg_aws}, \ref{tab:truth_table_cg_aws} represent the number of images predicted in the corresponding range of ages.
Looking at the ClarifAI model (Tables \ref{tab:truth_table_cg_clarifai} and \ref{tab:truth_table_eg_clarifai}) we can observe some differences between EG and CG. 
The EG performed worst in the ranges: 20-29, 30-39, 40-49, while the CG performed worst in the ranges between: 40-49, 50-59, 60-69, $\geq70$.
The AWSR predictions (Tables \ref{tab:truth_table_cg_aws} and \ref{tab:truth_table_eg_aws}) were slightly more accurate.

%TRUTH TABLES
%TRUTH TABLE EG ClarifAI
\begin{table}[t]
    \caption{Truth tables regarding age prediction.}
    \label{tab:truth_table_clarifai}
    \begin{subtable}[ht]{0.49\linewidth}
        \caption{ClarifAI Experimental group.}
        \label{tab:truth_table_eg_clarifai}
        \resizebox{\columnwidth}{!}{%
            
            \begin{tabular}{|lccccccccc|}
            \hline
            \multicolumn{10}{|l|}{\textit{\textbf{Experimental group}}} \\ \hline
            \multicolumn{1}{|c|}{\textbf{}} &
              \multicolumn{9}{c|}{\textbf{PREDICTED RANGE}} \\ \hline
            \multicolumn{1}{|l|}{\textbf{\begin{tabular}[c]{@{}l@{}}TRUE\\ RANGE\end{tabular}}} &
              \multicolumn{1}{l|}{\textbf{0-2}} &
              \multicolumn{1}{l|}{\textbf{3-9}} &
              \multicolumn{1}{l|}{\textbf{10-19}} &
              \multicolumn{1}{l|}{\textbf{20-29}} &
              \multicolumn{1}{l|}{\textbf{30-39}} &
              \multicolumn{1}{l|}{\textbf{40-49}} &
              \multicolumn{1}{l|}{\textbf{50-59}} &
              \multicolumn{1}{l|}{\textbf{60-69}} &
              \multicolumn{1}{l|}{\textbf{\textgreater{}70}} \\ \hline
            \multicolumn{1}{|l|}{\textbf{0 - 2}} &
              \multicolumn{1}{c|}{\cellcolor[HTML]{D9D7D7}2} &
              \multicolumn{1}{c|}{} &
              \multicolumn{1}{c|}{} &
              \multicolumn{1}{c|}{} &
              \multicolumn{1}{c|}{} &
              \multicolumn{1}{c|}{} &
              \multicolumn{1}{c|}{} &
              \multicolumn{1}{c|}{} &
               \\ \hline
            \multicolumn{1}{|l|}{\textbf{3 - 9}} &
              \multicolumn{1}{c|}{3} &
              \multicolumn{1}{c|}{\cellcolor[HTML]{D9D7D7}9} &
              \multicolumn{1}{c|}{} &
              \multicolumn{1}{c|}{} &
              \multicolumn{1}{c|}{} &
              \multicolumn{1}{c|}{} &
              \multicolumn{1}{c|}{} &
              \multicolumn{1}{c|}{} &
               \\ \hline
            \multicolumn{1}{|l|}{\textbf{10 - 19}} &
              \multicolumn{1}{c|}{} &
              \multicolumn{1}{c|}{4} &
              \multicolumn{1}{c|}{\cellcolor[HTML]{D9D7D7}15} &
              \multicolumn{1}{c|}{9} &
              \multicolumn{1}{c|}{} &
              \multicolumn{1}{c|}{} &
              \multicolumn{1}{c|}{} &
              \multicolumn{1}{c|}{} &
               \\ \hline
            \multicolumn{1}{|l|}{\textbf{20 - 29}} &
              \multicolumn{1}{c|}{} &
              \multicolumn{1}{c|}{3} &
              \multicolumn{1}{c|}{14} &
              \multicolumn{1}{c|}{\cellcolor[HTML]{D9D7D7}24} &
              \multicolumn{1}{c|}{10} &
              \multicolumn{1}{c|}{} &
              \multicolumn{1}{c|}{} &
              \multicolumn{1}{c|}{} &
               \\ \hline
            \multicolumn{1}{|l|}{\textbf{30 - 39}} &
              \multicolumn{1}{c|}{} &
              \multicolumn{1}{c|}{2} &
              \multicolumn{1}{c|}{2} &
              \multicolumn{1}{c|}{9} &
              \multicolumn{1}{c|}{\cellcolor[HTML]{D9D7D7}6} &
              \multicolumn{1}{c|}{3} &
              \multicolumn{1}{c|}{} &
              \multicolumn{1}{c|}{} &
               \\ \hline
            \multicolumn{1}{|l|}{\textbf{40 - 49}} &
              \multicolumn{1}{c|}{} &
              \multicolumn{1}{c|}{} &
              \multicolumn{1}{c|}{1} &
              \multicolumn{1}{c|}{2} &
              \multicolumn{1}{c|}{2} &
              \multicolumn{1}{c|}{\cellcolor[HTML]{D9D7D7}6} &
              \multicolumn{1}{c|}{2} &
              \multicolumn{1}{c|}{} &
               \\ \hline
            \multicolumn{1}{|l|}{\textbf{50 - 59}} &
              \multicolumn{1}{c|}{} &
              \multicolumn{1}{c|}{} &
              \multicolumn{1}{c|}{} &
              \multicolumn{1}{c|}{} &
              \multicolumn{1}{c|}{} &
              \multicolumn{1}{c|}{1} &
              \multicolumn{1}{c|}{\cellcolor[HTML]{D9D7D7}0} &
              \multicolumn{1}{c|}{1} &
               \\ \hline
            \multicolumn{1}{|l|}{\textbf{60 - 69}} &
              \multicolumn{1}{l|}{} &
              \multicolumn{1}{l|}{} &
              \multicolumn{1}{l|}{} &
              \multicolumn{1}{l|}{} &
              \multicolumn{1}{l|}{} &
              \multicolumn{1}{l|}{} &
              \multicolumn{1}{l|}{} &
              \multicolumn{1}{l|}{} &
              \multicolumn{1}{l|}{} \\ \hline
            \multicolumn{1}{|l|}{\textbf{\textgreater 70}} &
              \multicolumn{1}{l|}{} &
              \multicolumn{1}{l|}{} &
              \multicolumn{1}{l|}{} &
              \multicolumn{1}{l|}{} &
              \multicolumn{1}{l|}{} &
              \multicolumn{1}{l|}{} &
              \multicolumn{1}{l|}{} &
              \multicolumn{1}{l|}{} &
              \multicolumn{1}{l|}{} \\ \hline
            \end{tabular}%  
        }
    \end{subtable}
    \hfill
    \begin{subtable}[ht]{0.49\linewidth}
    \caption{ClarifAI Control group.}
    \label{tab:truth_table_cg_clarifai}
        \resizebox{\columnwidth}{!}{%
            \begin{tabular}{|lccccccccl|}
            \hline
            \multicolumn{10}{|l|}{\textit{\textbf{Control group}}} \\ \hline
            \multicolumn{1}{|l|}{} &
              \multicolumn{9}{c|}{\textbf{PREDICTED RANGE}} \\ \hline
            \multicolumn{1}{|l|}{\textbf{\begin{tabular}[c]{@{}l@{}}TRUE \\ RANGE\end{tabular}}} &
              \multicolumn{1}{l|}{\textbf{0-2}} &
              \multicolumn{1}{l|}{\textbf{3-9}} &
              \multicolumn{1}{l|}{\textbf{10-19}} &
              \multicolumn{1}{l|}{\textbf{20-29}} &
              \multicolumn{1}{l|}{\textbf{30-39}} &
              \multicolumn{1}{l|}{\textbf{40-49}} &
              \multicolumn{1}{l|}{\textbf{50-59}} &
              \multicolumn{1}{l|}{\textbf{60-69}} &
              \textbf{\textgreater{}70} \\ \hline
            \multicolumn{1}{|l|}{\textbf{0 - 2}} &
              \multicolumn{1}{c|}{\cellcolor[HTML]{D9D7D7}0} &
              \multicolumn{1}{l|}{} &
              \multicolumn{1}{l|}{} &
              \multicolumn{1}{l|}{} &
              \multicolumn{1}{l|}{} &
              \multicolumn{1}{l|}{} &
              \multicolumn{1}{l|}{} &
              \multicolumn{1}{l|}{} &
               \\ \hline
            \multicolumn{1}{|l|}{\textbf{3 - 9}} &
              \multicolumn{1}{l|}{} &
              \multicolumn{1}{c|}{\cellcolor[HTML]{D9D7D7}11} &
              \multicolumn{1}{c|}{4} &
              \multicolumn{1}{l|}{} &
              \multicolumn{1}{l|}{} &
              \multicolumn{1}{l|}{} &
              \multicolumn{1}{l|}{} &
              \multicolumn{1}{l|}{} &
               \\ \hline
            \multicolumn{1}{|l|}{\textbf{10 - 19}} &
              \multicolumn{1}{l|}{} &
              \multicolumn{1}{c|}{5} &
              \multicolumn{1}{c|}{\cellcolor[HTML]{D9D7D7}13} &
              \multicolumn{1}{c|}{7} &
              \multicolumn{1}{l|}{} &
              \multicolumn{1}{l|}{} &
              \multicolumn{1}{l|}{} &
              \multicolumn{1}{l|}{} &
               \\ \hline
            \multicolumn{1}{|l|}{\textbf{20 - 29}} &
              \multicolumn{1}{l|}{} &
              \multicolumn{1}{l|}{} &
              \multicolumn{1}{c|}{2} &
              \multicolumn{1}{c|}{\cellcolor[HTML]{D9D7D7}20} &
              \multicolumn{1}{c|}{3} &
              \multicolumn{1}{l|}{} &
              \multicolumn{1}{l|}{} &
              \multicolumn{1}{l|}{} &
               \\ \hline
            \multicolumn{1}{|l|}{\textbf{30 - 39}} &
              \multicolumn{1}{l|}{} &
              \multicolumn{1}{l|}{} &
              \multicolumn{1}{l|}{} &
              \multicolumn{1}{c|}{9} &
              \multicolumn{1}{c|}{\cellcolor[HTML]{D9D7D7}15} &
              \multicolumn{1}{l|}{} &
              \multicolumn{1}{l|}{} &
              \multicolumn{1}{l|}{} &
               \\ \hline
            \multicolumn{1}{|l|}{\textbf{40 - 49}} &
              \multicolumn{1}{l|}{} &
              \multicolumn{1}{l|}{} &
              \multicolumn{1}{l|}{} &
              \multicolumn{1}{c|}{4} &
              \multicolumn{1}{c|}{9} &
              \multicolumn{1}{c|}{\cellcolor[HTML]{D9D7D7}11} &
              \multicolumn{1}{c|}{3} &
              \multicolumn{1}{l|}{} &
               \\ \hline
            \multicolumn{1}{|l|}{\textbf{50 - 59}} &
              \multicolumn{1}{l|}{} &
              \multicolumn{1}{l|}{} &
              \multicolumn{1}{l|}{} &
              \multicolumn{1}{c|}{3} &
              \multicolumn{1}{c|}{6} &
              \multicolumn{1}{c|}{9} &
              \multicolumn{1}{c|}{\cellcolor[HTML]{D9D7D7}10} &
              \multicolumn{1}{c|}{3} &
               \\ \hline
            \multicolumn{1}{|l|}{\textbf{60 - 69}} &
              \multicolumn{1}{l|}{} &
              \multicolumn{1}{l|}{} &
              \multicolumn{1}{l|}{} &
              \multicolumn{1}{l|}{} &
              \multicolumn{1}{c|}{3} &
              \multicolumn{1}{c|}{4} &
              \multicolumn{1}{c|}{12} &
              \multicolumn{1}{c|}{\cellcolor[HTML]{D9D7D7}9} &
               \\ \hline
            \multicolumn{1}{|l|}{\textbf{\textgreater 70}} &
              \multicolumn{1}{l|}{} &
              \multicolumn{1}{l|}{} &
              \multicolumn{1}{l|}{} &
              \multicolumn{1}{c|}{1} &
              \multicolumn{1}{l|}{} &
              \multicolumn{1}{l|}{} &
              \multicolumn{1}{c|}{7} &
              \multicolumn{1}{c|}{11} &
              \cellcolor[HTML]{D9D7D7}6 \\ \hline
            \end{tabular}%
        }
    \end{subtable}
    \\
    \begin{subtable}[ht]{0.49\linewidth}
        \caption{AWSR experimental group.}
        \label{tab:truth_table_eg_aws}
        \resizebox{\columnwidth}{!}{%
            \begin{tabular}{|lccccccccc|}
            \hline
            \multicolumn{10}{|l|}{\textit{\textbf{Experimental group}}} \\ \hline
            \multicolumn{1}{|l|}{} &
              \multicolumn{9}{c|}{\textbf{PREDICTED RANGE}} \\ \hline
            \multicolumn{1}{|l|}{\textbf{\begin{tabular}[c]{@{}l@{}}TRUE \\ RANGE\end{tabular}}} &
              \multicolumn{1}{l|}{\textbf{0-2}} &
              \multicolumn{1}{l|}{\textbf{3-9}} &
              \multicolumn{1}{l|}{\textbf{10-19}} &
              \multicolumn{1}{l|}{\textbf{20-29}} &
              \multicolumn{1}{l|}{\textbf{30-39}} &
              \multicolumn{1}{l|}{\textbf{40-49}} &
              \multicolumn{1}{l|}{\textbf{50-59}} &
              \multicolumn{1}{l|}{\textbf{60-69}} &
              \textbf{\textgreater{}70} \\ \hline
            \multicolumn{1}{|l|}{\textbf{0 - 2}} &
              \multicolumn{1}{c|}{\cellcolor[HTML]{D9D7D7}1} &
              \multicolumn{1}{l|}{1} &
              \multicolumn{1}{l|}{} &
              \multicolumn{1}{l|}{} &
              \multicolumn{1}{l|}{} &
              \multicolumn{1}{l|}{} &
              \multicolumn{1}{l|}{} &
              \multicolumn{1}{l|}{} &
               \\ \hline
            \multicolumn{1}{|l|}{\textbf{3 - 9}} &
              \multicolumn{1}{c|}{1} &
              \multicolumn{1}{c|}{\cellcolor[HTML]{D9D7D7}10} &
              \multicolumn{1}{c|}{1} &
              \multicolumn{1}{l|}{} &
              \multicolumn{1}{l|}{} &
              \multicolumn{1}{l|}{} &
              \multicolumn{1}{l|}{} &
              \multicolumn{1}{l|}{} &
               \\ \hline
            \multicolumn{1}{|l|}{\textbf{10 - 19}} &
              \multicolumn{1}{c|}{} &
              \multicolumn{1}{c|}{3} &
              \multicolumn{1}{c|}{\cellcolor[HTML]{D9D7D7}14} &
              \multicolumn{1}{c|}{9} &
              \multicolumn{1}{l|}{} &
              \multicolumn{1}{l|}{} &
              \multicolumn{1}{l|}{} &
              \multicolumn{1}{l|}{} &
               \\ \hline
            \multicolumn{1}{|l|}{\textbf{20 - 29}} &
              \multicolumn{1}{l|}{} &
              \multicolumn{1}{l|}{} &
              \multicolumn{1}{c|}{15} &
              \multicolumn{1}{c|}{\cellcolor[HTML]{D9D7D7}22} &
              \multicolumn{1}{c|}{12} &
              \multicolumn{1}{c|}{2} &
              \multicolumn{1}{l|}{} &
              \multicolumn{1}{l|}{} &
               \\ \hline
            \multicolumn{1}{|l|}{\textbf{30 - 39}} &
              \multicolumn{1}{l|}{} &
              \multicolumn{1}{c|}{1} &
              \multicolumn{1}{c|}{2} &
              \multicolumn{1}{c|}{10} &
              \multicolumn{1}{c|}{\cellcolor[HTML]{D9D7D7}9} &
              \multicolumn{1}{l|}{} &
              \multicolumn{1}{l|}{} &
              \multicolumn{1}{l|}{} &
               \\ \hline
            \multicolumn{1}{|l|}{\textbf{40 - 49}} &
              \multicolumn{1}{l|}{} &
              \multicolumn{1}{l|}{} &
              \multicolumn{1}{l|}{} &
              \multicolumn{1}{c|}{2} &
              \multicolumn{1}{c|}{5} &
              \multicolumn{1}{c|}{\cellcolor[HTML]{D9D7D7}4} &
              \multicolumn{1}{c|}{2} &
              \multicolumn{1}{l|}{} &
               \\ \hline
            \multicolumn{1}{|l|}{\textbf{50 - 59}} &
              \multicolumn{1}{l|}{} &
              \multicolumn{1}{l|}{} &
              \multicolumn{1}{l|}{} &
              \multicolumn{1}{l|}{} &
              \multicolumn{1}{l|}{} &
              \multicolumn{1}{c|}{1} &
              \multicolumn{1}{c|}{\cellcolor[HTML]{D9D7D7}1} &
              \multicolumn{1}{l|}{} &
               \\ \hline
            \multicolumn{1}{|l|}{\textbf{60 - 69}} &
              \multicolumn{1}{l|}{} &
              \multicolumn{1}{l|}{} &
              \multicolumn{1}{l|}{} &
              \multicolumn{1}{l|}{} &
              \multicolumn{1}{l|}{} &
              \multicolumn{1}{l|}{} &
              \multicolumn{1}{l|}{} &
              \multicolumn{1}{l|}{} &
               \\ \hline
            \multicolumn{1}{|l|}{\textbf{\textgreater 70}} &
              \multicolumn{1}{l|}{} &
              \multicolumn{1}{l|}{} &
              \multicolumn{1}{l|}{} &
              \multicolumn{1}{l|}{} &
              \multicolumn{1}{l|}{} &
              \multicolumn{1}{l|}{} &
              \multicolumn{1}{l|}{} &
              \multicolumn{1}{l|}{} &
               \\ \hline
            \end{tabular}%
        }
    \end{subtable}
    \hfil
    \begin{subtable}[ht]{0.49\linewidth}
        \caption{AWSR control group.}
        \label{tab:truth_table_cg_aws}
        \resizebox{\columnwidth}{!}{%
            \begin{tabular}{|lccccccccc|}
            \hline
            \multicolumn{10}{|l|}{\textit{\textbf{Control group}}} \\ \hline
            \multicolumn{1}{|l|}{} &
              \multicolumn{9}{c|}{\textbf{PREDICTED RANGE}} \\ \hline
            \multicolumn{1}{|l|}{\textbf{\begin{tabular}[c]{@{}l@{}}TRUE \\ RANGE\end{tabular}}} &
              \multicolumn{1}{l|}{\textbf{0-2}} &
              \multicolumn{1}{l|}{\textbf{3-9}} &
              \multicolumn{1}{l|}{\textbf{10-19}} &
              \multicolumn{1}{l|}{\textbf{20-29}} &
              \multicolumn{1}{l|}{\textbf{30-39}} &
              \multicolumn{1}{l|}{\textbf{40-49}} &
              \multicolumn{1}{l|}{\textbf{50-59}} &
              \multicolumn{1}{l|}{\textbf{60-69}} &
              \multicolumn{1}{l|}{\textbf{\textgreater{}70}} \\ \hline
            \multicolumn{1}{|l|}{\textbf{0 - 2}} &
              \multicolumn{1}{c|}{\cellcolor[HTML]{D9D7D7}0} &
              \multicolumn{1}{c|}{} &
              \multicolumn{1}{c|}{} &
              \multicolumn{1}{c|}{} &
              \multicolumn{1}{c|}{} &
              \multicolumn{1}{c|}{} &
              \multicolumn{1}{c|}{} &
              \multicolumn{1}{c|}{} &
               \\ \hline
            \multicolumn{1}{|l|}{\textbf{3 - 9}} &
              \multicolumn{1}{c|}{1} &
              \multicolumn{1}{c|}{\cellcolor[HTML]{D9D7D7}9} &
              \multicolumn{1}{c|}{5} &
              \multicolumn{1}{c|}{} &
              \multicolumn{1}{c|}{} &
              \multicolumn{1}{c|}{} &
              \multicolumn{1}{c|}{} &
              \multicolumn{1}{c|}{} &
               \\ \hline
            \multicolumn{1}{|l|}{\textbf{10 - 19}} &
              \multicolumn{1}{c|}{} &
              \multicolumn{1}{c|}{3} &
              \multicolumn{1}{c|}{\cellcolor[HTML]{D9D7D7}12} &
              \multicolumn{1}{c|}{10} &
              \multicolumn{1}{c|}{} &
              \multicolumn{1}{c|}{} &
              \multicolumn{1}{c|}{} &
              \multicolumn{1}{c|}{} &
               \\ \hline
            \multicolumn{1}{|l|}{\textbf{20 - 29}} &
              \multicolumn{1}{c|}{} &
              \multicolumn{1}{c|}{} &
              \multicolumn{1}{c|}{3} &
              \multicolumn{1}{c|}{\cellcolor[HTML]{D9D7D7}21} &
              \multicolumn{1}{c|}{1} &
              \multicolumn{1}{c|}{} &
              \multicolumn{1}{c|}{} &
              \multicolumn{1}{c|}{} &
               \\ \hline
            \multicolumn{1}{|l|}{\textbf{30 - 39}} &
              \multicolumn{1}{c|}{} &
              \multicolumn{1}{c|}{} &
              \multicolumn{1}{c|}{1} &
              \multicolumn{1}{c|}{10} &
              \multicolumn{1}{c|}{\cellcolor[HTML]{D9D7D7}7} &
              \multicolumn{1}{c|}{3} &
              \multicolumn{1}{c|}{} &
              \multicolumn{1}{c|}{} &
               \\ \hline
            \multicolumn{1}{|l|}{\textbf{40 - 49}} &
              \multicolumn{1}{c|}{} &
              \multicolumn{1}{c|}{} &
              \multicolumn{1}{c|}{} &
              \multicolumn{1}{c|}{2} &
              \multicolumn{1}{c|}{6} &
              \multicolumn{1}{c|}{\cellcolor[HTML]{D9D7D7}17} &
              \multicolumn{1}{c|}{2} &
              \multicolumn{1}{c|}{} &
               \\ \hline
            \multicolumn{1}{|l|}{\textbf{50 - 59}} &
              \multicolumn{1}{c|}{} &
              \multicolumn{1}{c|}{} &
              \multicolumn{1}{c|}{} &
              \multicolumn{1}{c|}{} &
              \multicolumn{1}{c|}{2} &
              \multicolumn{1}{c|}{12} &
              \multicolumn{1}{c|}{\cellcolor[HTML]{D9D7D7}17} &
              \multicolumn{1}{c|}{} &
               \\ \hline
            \multicolumn{1}{|l|}{\textbf{60 - 69}} &
              \multicolumn{1}{c|}{} &
              \multicolumn{1}{c|}{} &
              \multicolumn{1}{c|}{} &
              \multicolumn{1}{c|}{} &
              \multicolumn{1}{c|}{} &
              \multicolumn{1}{c|}{4} &
              \multicolumn{1}{c|}{15} &
              \multicolumn{1}{c|}{\cellcolor[HTML]{D9D7D7}6} &
              1 \\ \hline
            \multicolumn{1}{|l|}{\textbf{\textgreater 70}} &
              \multicolumn{1}{c|}{} &
              \multicolumn{1}{c|}{} &
              \multicolumn{1}{c|}{} &
              \multicolumn{1}{c|}{} &
              \multicolumn{1}{c|}{} &
              \multicolumn{1}{c|}{} &
              \multicolumn{1}{c|}{6} &
              \multicolumn{1}{c|}{15} &
              \cellcolor[HTML]{D9D7D7}3 \\ \hline
            \end{tabular}%
        }
    \end{subtable}
\end{table}

Most of the errors for the EG were in the ranges: 30-39, 40-49 whereas for the CG they were in the ranges: 30-39, 60-69, $\geq70$.
The comparison should take into account that the average life expectancy of people with Down syndrome is around 65 years, and the EG dataset contains images of people with a maximum age in the range 50-59.

Focusing on the predictions of ClarifAI on the EG, Table \ref{tab:truth_table_eg_clarifai} shows that the predicted ranges 3-9 and 10-19 are the two intervals with higher variance in the dataset.
This means that some images with a true age range of 20-29 and 30-39 are labelled with the age range 3-9 or 10-19.
The same can be verified for the age range 40-49 with predictions of 10-19.
Differently, for the CG, Table \ref{tab:truth_table_cg_clarifai}, the age ranges with higher variance are 20-29 and 30-39.
Some images with a true age range of 50-59 or 60-69 are labelled with age ranges of 20-29 and 30-39.
% These are the critical errors that the ClarifAI model makes, but these model decisions can be contextualized.
For the CG, the predicted age ranges are lower than the real ones, but they never coincide with the age ranges of children.
The predictions corresponding to a lower age could be due to the fact that the dataset is made up of images of famous people, i.e. people with facial surgery and make-up who make themselves look younger.
Instead, the images representing adult people with Down syndrome are classified with children in age ranges such as 3-9 and 10-19.

The AWSR model is more stable in the ranges of its predictions.
% The Tables \ref{tab:truth_table_cg_aws} and \ref{tab:truth_table_eg_aws} do not show the same amount of steps from the diagonal. 
The ranges with higher variance for the EG are 3-9 and 10-19, whereas for the CG almost all ranges have the same variance. However, some errors in the AWSR model are quite similar to those in the ClarifAI model for the experimental group.
Some images with a true age range of 30-39 are classified as having an age range of 3-9 or 10-19.
\vspace{0.5cm}

\framebox{
    \begin{minipage}[h]{0.90\linewidth}
        We observed that there are some differences in age estimation between EG and CG. The results lead to the conclusion that both models assign the age range of children to adults belonging to the EG.
    \end{minipage}
}

\subsection{RQ2. Image labelling}\label{sec:general_image_recognition}

\subsubsection{Aesthetics and Education}\label{sec:general_aesthetics}
\begin{figure}[ht]
    \centering
    \begin{subfigure}[b]{0.49\linewidth}
        \includegraphics[width=\linewidth]{images/graphs/concepts/Aesthetic_ClarifAI_total.pdf}
        \caption{
            Comparison between experimental group and control group.
        }
        \label{fig:aesthetic_Clarifai}
    \end{subfigure}
    \hfill
    \begin{subfigure}[b]{0.49\linewidth}
        \centering
        \includegraphics[width=\linewidth]{images/graphs/concepts/Aesthetic_ClarifAI_gendered.pdf}
        \caption{
            Gendered comparison between experimental group and control group.
        }
        \label{fig:aesthetic_Gendered_Clarifai}
    \end{subfigure}
    \caption{Comparison regarding \textit{Aesthetic} labels assigned by the ClarifAI model}
\end{figure}

\begin{figure}[ht]
    \begin{subfigure}[b]{0.48\linewidth}
        \centering
        \includegraphics[width=\linewidth]{images/graphs/concepts/Education_ClarifAI_total.pdf}
        \caption{
            Comparison between experimental group and control group regarding.
        }
        \label{fig:education_Clarifai}
    \end{subfigure}
    \hfill
    \begin{subfigure}[b]{0.49\linewidth}
        \centering
        \includegraphics[width=\linewidth]{images/graphs/concepts/Education_ClarifAI_gendered.pdf}
        \caption{
            Gendered comparison between experimental group and control group.
        }
        \label{fig:education_Gendered_Clarifai}
    \end{subfigure}
    \caption{Comparison regarding \textit{Education} labels assigned by the ClarifAI model.}
\end{figure}

Figure \ref{fig:aesthetic_Clarifai} shows that the EG had a higher number of occurrences than the CG for every concept except the label \textit{Sexy}, although the difference is very small. 
Looking at the gender distinction, Figure \ref{fig:aesthetic_Gendered_Clarifai}, it is noticeable that for both the EG and the CG, women were more likely to be associated with the aesthetic labels than men.
In terms of numbers, 316 vs. 135 labels were assigned to females and males, respectively.
Furthermore, the label \textit{Sexy} is only assigned to those images that were classified as female in the gender recognition task.
The description of this label is linked to the ability to arouse sexual desire or interest, which is only associated with the female gender.
Regarding the labels of the category \textit{Education}, Figure \ref{fig:education_Gendered_Clarifai}, most of the labels were associated with images representing males rather than females.
The overall situation reflects the typical stereotypes associated with gender, with aesthetic labels associated with females and educational labels associated with males.
 
\subsubsection{Person Descriptors}

\begin{figure*}[t]
    \centering
    \begin{subfigure}[b]{0.49\linewidth}
        \includegraphics[width=\linewidth]{images/graphs/concepts/Person_Description_ClarifAI_total.pdf}
        \caption{
            Comparison between experimental group and control group regarding \textit{Person Description} labels assigned by the ClarifAI model.
        }
        \label{fig:PersonalDescription_Clarifai}
    \end{subfigure}
    \hfill
    \begin{subfigure}[b]{0.49\linewidth}
        \includegraphics[width=\linewidth]{images/graphs/concepts/Person_Description_AWS_total.pdf}
        \caption{
            Comparison between experimental group and control group regarding \textit{Person Description} labels assigned by the AWSR model.
        }
    \label{fig:PersonDescription_AWS}
    \end{subfigure}
    \\
    \begin{subfigure}[b]{0.48\linewidth}
        \includegraphics[width=\linewidth]{images/graphs/concepts/Person_Description_ClarifAI_gendered.pdf}
        \caption{
            Gendered comparison between EG and CG regarding \textit{Person Description} labels assigned by the ClarifAI model.
        }
        \label{fig:PersonDescription_Clarifai_gendered}
    \end{subfigure}
    \hfill
    \begin{subfigure}[b]{0.48\linewidth}
        \includegraphics[width=\linewidth]{images/graphs/concepts/Person_Description_AWS_gendered.pdf}
        \caption{
            Gendered comparison between EG and CG regarding \textit{Person Description} labels assigned by the AWSR model.
        }
        \label{fig:PersonDescription_AWS_gendered}
        \end{subfigure}
        \caption{Comparison regarding \textit{Person Description} labels assigned by the ClarifAI and AWSR models.}
\end{figure*}

\vspace{-0.3cm}
The name \textit{Person description} is taken from one of the predefined categories of the AWSR model.
All the labels are common to both models, so a comparison can be made as shown in Figure \ref{fig:PersonalDescription_Clarifai} and Figure \ref{fig:PersonDescription_AWS}.

% The first two labels to analyse are \textit{Adult} and \textit{Child}.
Both models assigned the label \textit{Child} more often to the EG than to the CG, although the number of images representing children is quite balanced between the two groups.
The same consideration can be made for the label \textit{Adult}, where the situation is reversed: this result is consistent with the observations from the age prediction task, i.e. the models were more likely to consider a person in the EG as a child rather than an adult.

% The labels \textit{Person} and \textit{People} that should be assigned if there is the presence of one person or more people, respectively, are assigned in the opposite way, with respect to their meaning, by the ClarifAI model. 
Each image represents only one person, and according to the definition given by the ClarifAI model for the label \textit{Person} - one human being - and the label \textit{People} - (plural) any group of people (men or women or children) together - the label
the label \textit{Person} is the correct one for each image in the dataset. 
ClarifAI used the correct label for only 56 images out of a total of 400 images, while AWSR used the correct label for 399 images out of a total of 400 images. 

The other labels are gendered in the sense that they refer strictly to one gender rather than the other.
For this reason, it may be useful to look at the Figure \ref{fig:PersonDescription_Clarifai_gendered} and \ref{fig:PersonDescription_AWS_gendered} where a comparison is made of the labels assigned between all four different classes of the dataset: EG male, EG female, CG male, CG female.
All of these gendered labels occurred for both genders, meaning that some images representing females were labelled as \textit{Man} and vice versa.
Some images had both labels \textit{Man} and \textit{Woman} or \textit{Boy} and \textit{Girl}.
Both of these considerations reflect a general confusion in the assignment of these types of labels.

% In addition, the gender considered as binary is a choice made during the creation of algorithms, and reflects specific values and beliefs. 
In addition, even if the task of the model is different from gender recognition, gender is always seen as binary, reflecting specific values and beliefs.
Some labels in the \textit{Person Description} category, such as \textit{Male}, \textit{Female}, \textit{Man}, \textit{Woman}, etc., have the power to classify people on the basis of their appearance.
Incorrect gender labelling can have a number of negative consequences for people, especially for groups that are systematically discriminated against. 
It is therefore questionable whether the use of gender labels is appropriate.
\vspace{0.5cm}

\framebox{
    \begin{minipage}[h]{0.95\linewidth}
        The results obtained by the image labelling models do not show significant differences between the EG and the CG, but they show very important differences between the genders. In particular, labels of the \textit{Aesthetics} category are more likely to be associated with female classes than with male classes, and labels of the \textit{Education} category are more likely to be associated with male classes than with female classes.
    \end{minipage}
}

% For example, a socially discriminate group of people as the one of the LGBTQ+ community, would be ethically and profoundly attacked by those estimations.
% The current usage of gender binary diversification is no more applicable in any environment because of its problematic effects.\mrnote{forse riformulerei}.

\section{Threats to Validity and Ethical Concerns}\label{sec:threats-validity-limitation}
Most of the limitations and ethical concerns of the study relate to the construction of the dataset and the choice of models.

Starting from the limitations of the dataset, some images are of poor quality.
% One of the suggestions made in the ClarifAI model guidelines - rule of reference number 1 (Appendix A.2 in the Supplementary Material) - states that a "cropped" version of the images, representing only the face, is mandatory for a well-performing model. The Service's suggestion was followed and, as a result, a cropped version of all images was produced. 
The resizing of the images to obtain the cropped versions inevitably lowered the quality of about 15\% of the images belonging to the CG. 
The impact of the lower quality of these images on the classification results cannot be excluded, but this problem affected a limited number of images.

As described in Section \ref{sec:Dataset}, some images of the EG did not include information about the year in which they were taken or the age of the person depicted. 
The resulting limitation is an unbalanced comparison between the two groups, EG and CG, in terms of the different number of samples for each age range. 

% An additional consideration concerns the CG and EG datasets. 
Since there is no information and no details on which datasets were used for the learning phase of the models, we cannot exclude the possibility that the same images, or some of them, were used in the training process. 
Building a test set with homemade images requires relevant economic and organizational resources.

% A general limitation of the datasets relates to their sources. 
As the dataset has been created using resources available online, we do not have the explicit consent of the people depicted in the images. 
For this reason, the dataset created is not available to the public and it will remain private.

Another important limitation relates to the \textit{ethnicity} of the people represented.
We carefully constructed the dataset to include people from different backgrounds, so that a variety of ethnicities are included in the dataset.
However, given the difficulties in finding images for the EG, we didn't aim at an equal distribution of different ethnicities, leaving this aspect to future work.
% the final paper does not contain an equal distribution of different ethnicities.

\section{Conclusions}\label{sec:conclusions}

% Until now, little attention has been given to the facial recognition and classification issues faced by this group, and our research aims to shed light on these important issues.
The goal of this study was to understand whether people with Down syndrome may experience problems when their facial image is automatically classified. 
By focusing on this specific group of vulnerable people, we identified a gap in the literature on bias in facial analysis systems and further contributed to the investigation of the inherent limitations of this technology. 

To achieve our goal, we created a test set by collecting facial images already available on the web.
We collected 400 images, 200  faces of people with Down syndrome (experimental group, EG) and 200 faces of people without the syndrome (control group, CG).
We then compared the performance of two commercial face recognition tools.
% The tools were run on the whole datasets and information on identified gender, age and other labels -- related to aesthetics, education level and person physical characteristics were collected.  
The results showed that overall the tools performed less well with the EG. 
We also found that: i) the gender prediction showed a higher error rate towards the Down male sample, with an accuracy value of 85\% for both tools; ii) people with Down syndrome were assigned a younger age in relation to their real age; iii) the labels assigned to the experimental group reflected the same gender stereotypes observed in the labels of the control group, in certain cases with a higher frequency.

% Given these results, from the purely technical point of view, an ideal scenario in which this type of technology could perform better is the one in which a dataset includes all "the face of every person in the world" \cite{merlerDiversityFaces2019}.
% : IBM Researchers, during their work \textit{Diversity in Faces} \cite{merlerDiversityFaces2019} stated that \textit{“the challenge of diversity could be solved by building a data set comprised from the face of every person in the world”}. 
% Obviously, this would be not only unfeasible, but also a very debatable from the ethical, legal and social point of view. 
Involving the most vulnerable populations in the design of facial analysis systems can reduce their operational bias.
Improving the transparency of documentation could also allow for better external scrutiny.
However, we should question the technology itself and its implementation. 
Predicting sensitive characteristics such as gender and age, as well as tagging a person's face with pre-defined labels, could have significant consequences for the lives of those people.
Regardless of the level of accuracy achievable, this technological development may not be socially acceptable. 
This work contributes to this debate by shedding further light on the structural limitations of facial analysis systems.

\section{Future Work}\label{sec:future_work}

The way in which the dataset was constructed was, for the time being, the most feasible way of investigating the research questions.
One of the first improvements in the construction of the dataset is to involve people from the selected communities and ask them to take photographs of themselves, thus obtaining their consent and some valuable information. 
This could be a valid solution to some limitations mentioned above, such as knowing the exact age of each person and getting their explicit consent to be part of the research. 
Furthermore, some problems encountered during the process, such as pose, lighting and quality, can be solved by using appropriate cameras and rules for taking photographs.

Another important improvement concerns ethnicity. 
An idea to construct an equally balanced dataset can be inspired by the \textit{Pilot Parliaments Benchmark} dataset \cite{buolamwiniGenderShadesIntersectional2018}.
% , which is based on the dermatologist-approved \textit{Fitzpatrick Skin Type classification system} \cite{fitzpatrickValidityPracticalitySunReactive1988}.

% the ones used in the current analysis are two commercial models that can be used freely and that still offer the gender prediction.
In terms of models, it would be interesting to increase the number of models studied. 
This will make it possible to get a broader picture of how FASs are performing in relation to groups that are under-represented.
% in order to understand the differences between them and to visualize the results.

% In addition to possible solutions to some of the limitations of this research outlined above, 
Further progress could be made on active measures to reduce discrimination against under-represented groups (such as people with Down syndrome).
Since the subgroup performance issues that lead to dangerous discrimination stem most probably from under-representation and unbalanced data, it would be interesting to explore a way to measure data characteristics (such as balance) and provide ad hoc designed labels that provide ethically relevant information. 
Such a tool could be integrated into the AI pipeline to allow developers to be aware of the data issues and take into consideration meaningful countermeasures.
This could be also useful if it is used to publish and disseminate relevant information related to public datasets that are widely used in the AI community, such as those used or mentioned in this article.
% , to put in practice before the model publication (or the publication of the service based on it).This should always be done bearing in mind the severe limitations of this family of technologies and the risk of individual harm. However, ethical discussions should always be thorough in order to determine whether it is right to publish a particular service or not.

\clearpage

% ---- Bibliography ----
%
% BibTeX users should specify bibliography style 'splncs04'.
% References will then be sorted and formatted in the correct style.
%
\bibliographystyle{splncs04}
\bibliography{03_frt-ds}
\end{document}
