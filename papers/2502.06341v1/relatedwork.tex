\section{Related Works}
\label{sec:rel_work}
Several scholars highlighted the ethical issues of FASs.
Crawford in \textit{Atlas of AI} \cite{crawfordAtlasAI2021} reconstructed the history and development of AI in relationships with a variety of impacts (e.g., on the environment, work, health) and highlighted the epistemic issues of FASs and their controversial historical origins.

Other studies focused on the failures of FASs and the associated negative impacts on society. 
In terms of gender and ethnicity, Buolamwini and Gebru \cite{buolamwiniGenderShadesIntersectional2018} evaluated different commercial gender classification systems, and found that darker-skinned women were the most misclassified group.
Similar findings are reported by Klare
et al. \cite{klareFaceRecognitionPerformance2012}, who found that commercial and non-trainable algorithms performed worse for women, blacks and young people.
The issue of diversity and inclusion in FASs can arise from the lack of examples of subpopulations in the train dataset, but also from the definition of classes that incorporate specific values and beliefs, as in a binary formulation of gender \cite{scheuermanHowComputersSee2019}.
The consequences of a non-inclusive operationalization of FASs can be seen in the case of transgender representation \cite{keyesMisgenderingMachinesTrans2018}.

Recent work on the unknown behaviour of neural networks, has shown which are the key features used by commercial face classification services in order to classify gender. 
In fact, lip, eye, cheek structure and make-up are more discriminative than skin and hair length \cite{muthukumarUnderstandingUnequalGender2018}.
As the authors discuss, the fact that the make-up is so important in predicting female gender is a troubling stereotype.

% **********
%Regarding the disability issue.. \avcomment{come sopra, purtroppo non si può fare un elenco senza spiegare perché gli articoli vengono citati, equivale a un farsi dare un reject solo per questo. Non per forza bisogna tenerli tutti, se il legame con questo studio è debole o non chiaro} the following studies by \citeauthor{asimSyndromeInsightDisease2015} \cite{asimSyndromeInsightDisease2015} and \citeauthor{kazemiSyndromeCurrentStatus2016} \cite{kazemiSyndromeCurrentStatus2016}, which give the scientific background of the syndrome. Following these initial studies, an interesting work by \citeauthor{damascenoFacialAnalysisSyndrome2014} \cite{damascenoFacialAnalysisSyndrome2014} in the orthodontic field, analysing the facial features of people with Down syndrome, has been of great help in understanding the physiology of this condition.
% **********

Taking the next step and linking FASs with Down syndrome, several papers are based on recognizing the disability in children in their first years of life.
Agbolade et al. \cite{agboladeSyndromeFaceRecognition2020} presented a performance comparison of different machine learning methods on the task of Down Syndrome detection.
Paredes et al. \cite{paredesEmotionRecognitionSyndrome2022} compared different machine learning and deep learning techniques to perform the emotion detection task on people with Down syndrome.
Finally, Qin et al. \cite{qinAutomaticIdentificationSyndrome2020} presented an identification method based on deep convolutional neural networks. 
The studies mentioned above were aimed at identifying the syndrome through the face or better understanding emotions through technology. 
What is not covered in previous works is whether people with Down syndrome are discriminated against, in terms of lower performance of the FASs, compared to non-affected people.
This is the observable gap that we address in this paper.
% They did not investigate whether discriminations of FASs towards people with Down syndrome occur in terms of lower performance, compared to not affected people.

%At the same time, several articles and websites, show the discriminations and perplexities of different AI technologies with respect to people with disabilities.
%Some examples of discriminations are regarding the job recruitment area, by \citeauthor{bakalisRecruitmentAIHas2020} \cite{bakalisRecruitmentAIHas2020} but also the general accessibility to the internet for people with disabilities by \citeauthor{alonso-virgosWebPageDesign2018} \cite{alonso-virgosWebPageDesign2018}.
%With respect to the experimental group of the current analysis, the problems and consequences are unknown and unexplored, leading to wrong implementations of the algorithms.