\section{Literature Review}
\label{s:lit-rev}
In this section, we review some of the relevant works \textcolor{black}{that are} similar to our work. A new course was designed and shown in the work \cite{19} at the University of Calgary, using RPi to make the learning mechanism of assembly language enjoyable and interactive for the students. In the RPi part of the course, students learn to create an interactive video game using ARM assembly language. The author showed that the distribution of grades in the exams improved significantly after introducing RPi to the curriculum. A voluntary, anonymous survey collected feedback from 198 students in two consecutive semesters, indicating an overall response rate of 58\%, evaluating aspects like enjoyment, learning experience, motivation, and independent learning in the new course format. Quantile-Quantile normality test was performed before and after introducing RPi which showed that the mean grades after introducing RPi were not normally distributed compared to the previous curriculum. Consequently, the authors performed a t-test which revealed that the integration of RPi was significant. Two surveys were also conducted concerning RPi and future studies. In both of them, the students expressed that RPi enhanced their learning curve and they would like to explore more about low-level and game programming in the future. In \cite{20}, the authors designed a MATLAB course while integrating RPi to enhance the learning capability of the BME students while handling real-life data. They proposed six experiments and performed both numerical course evaluation and survey on the students before and after incorporating RPi. After including RPi, the evaluation score increased. The student enjoyed programming more after introducing such hardware experiments and the authors indicated that as the BME students in the senior year have to do more complex real-life analysis, this newly proposed course would help them to understand how real-life projects work.

\begin{table*}[!htb]
\centering
    \caption{Categorization of the present literature review}
    \label{tab:literature}
    \begin{tabular}{|p{0.1\textwidth}|p{0.02\textwidth}|p{0.3\textwidth}|p{0.2\textwidth}|p{0.12\textwidth}|p{0.12\textwidth}|}
    \hline \textbf{Discussed Area/Study Type} & \textbf{Ref.} & \textbf{Learning Emphasis} & \textbf{Assessment Methods} & \textbf{Main Focus/Outputs} & \textbf{Published In}\\
    \hline %\multirow{5}{*}
    {CS/Engineering Courses with hardware} & \cite{19}& Development of Capabilities &Lab Evaluations, Assessments,Exams&Multidisciplinary & SIGCSE - ACM Digital Library\\ \cline{2-6}
    
    & \cite{20}& Real-life Learning Capability Enhancement for BME Students, data processing &Course Evaluation and Exams &Multidisciplinary & 2017 ASEE Annual Conference \& Exposition\\ \cline{2-6}

    
    & \cite{23}& Python-based skill development using RPi for enhance understandings &Lab evaluations and self-developed
    evaluation sheet  &Multidisciplinary & Heliyon\\ \cline{2-6}
    
    & \cite{24}& Project-based Computing Learning capabilities with integration of RPi &Project Evaluation, Student Evaluation
    & Multidisciplinary and Project-based & SIGCSE - ACM Digital Library\\ \cline{2-6}
    
    & \cite{ShekharUnpack} & Determining the Motivating factors of the individuals interested in Engineering and CS Undergraduate Levels &Project-based Learning and Teamwork & Multidisciplinary and Project-based & IEEE Transactions on Education\\ \cline{2-6}

    \hline %\multirow{3{*}
    {Course Curriculum Design} & \cite{21}& Practical Skills Enhancement in I\&C System with Real-life problem-solving capabilities according to the stakeholders &Specific Indicators based Evaluation such as Written Exam, Viva-Voce, Practical Examination, Project report, Survey & Multidisciplinary and Project-based & IEEE Transactions on Education\\ \cline{2-6}
    
    & \cite{22} & Designing real-life industrial Robotic Prototypes &Written Exams, Lab practicals, Students Participations,  and Lab reports &Multidisciplinary& IEEE Transactions on Education\\ \cline{2-6}

    & \cite{chiu2021creation} & AI Curriculum for the school students & Survey and Questionnaires & Multidisciplinary & IEEE Transactions on Education\\

    \hline %\multirow{2}{*}
    {Student Feedback Analysis on Existing Course} & \cite{ZhaoGame} & Game-based learning approach to teach programming in undergraduate levels &Demographic Questionnaires
    &Interactive with feedback mechanism& IEEE Transactions on Education\\ \cline{2-6}
    
    &  \cite{riese2023engineering} & Students Experience Analysis in CS Courses &Surveys considering distinct elements of engineering course &Multidisciplinary and Project-based & IEEE Transactions on Education \\

    \hline
    \end{tabular}
    
\end{table*}


In \cite{21}, the authors proposed a new curriculum for teaching the Instrumentation and Control Systems (I\&C System) for engineering students with four intended learning outcomes. During prototype planning, they contacted I\&C systems industry experts and asked questions relevant to the learning outcomes of their course. According to the experts' survey, the prototype was designed for the students. The design of such a prototype has a multidisciplinary approach keeping the feedback from the industrial organizations' survey. As per the authors’ knowledge, and from studies, such prototypes do not exist in other Indian institutes. Both course structure and criteria were proposed with respect to the ABET criteria and survey results of the industrial leaders. The students were divided into groups to learn and share the lab experiments in a collaborative environment. Two types of evaluations were performed after deploying the course: student survey and evaluation based on specific indicators. Students were asked 10 questions each having ratings from 1 to 5. The authors explained the poor rating in some of the questionnaires. The course has five specific indicators (SI) on which the evaluation was performed. Overall evaluation of this course was based on the Written Exam, Viva-Voce, Practical Examination, and Project report where these segments were mapped to different SIs. A detailed semester-wise comparison was performed based on these SI’s before and after introducing this prototype. The survey questionnaire for the students and their answers were categorized into different Specific Indicators. This was done for the students attending the newly designed course and the previous three years for the old courses. The average of the specific indicators increased in the semester of introducing this new prototype. Also, Cronbach’s $alpha$ test and \textit{t-test} were performed on these specific indicators which shows the reliability and effectiveness of this prototype respectively.  Another work \cite{22} designed a mechatronics course while maintaining a multidisciplinary approach as well as the ABET criterion for an undergraduate program at the Automation Department at Universidad Autónoma de Querétaro, Querétaro, Mexico. The authors focused on teaching and building a robot prototype (a robot arm) for the students. They have prepared six practical sessionals for teaching such industrial robot prototypes which included basic levels of familiarization of robot prototypes to advanced levels of controlling the robotic arm in wireless mode. Two types of evaluations were proposed after teaching the practical sessions which are Student Surveys and Evaluation Based on Specific Indicators (SIs). Individual and team wise (5 students per group) survey were performed for 10 questions each having 5 points (1 means strongly disagree, 5 means strongly agree). The survey showed on average 4 to 5 points while having low standard deviation for most of the questions. On the other hand, the course was designed keeping in mind 5 SIs, and the evaluation for the grading was divided into four parts: Written Exam (20\% marks), Practicals (50\% marks), Lab Report (20\% marks), Student’s participation and their ability to function in a multi-disciplinary team (qualitative-10\% marks) based on these SIs. This prototype was introduced in 2016 and the average number (percentage) in each specific indicator was calculated for three consecutive years: 2014, 2015, and 2016. It was found that the average percentage number seems to improve in the year 2016 compared to the previous prototype which shows that the introduced robot prototype is more effective for the students. However, the authors mentioned that taking surveys from the stakeholders matters too in such analysis which they did not include in the performance analysis of the new prototype. 


Kawash et al.~\cite{23} proposed a Python-based lab course that has been designed to perform image and video processing. The course included seven experiments and used RPi 3 model B for a basic understanding of image and multidimensional data like interpolation, feature detection, segmentation, and transformation. The lab course has been evaluated by the participating students in the evaluation system of the university and a self-developed evaluation sheet. Overall, in both evaluation systems, the lab course has been rated between excellent and very good. Another work \cite{24} proposed a project-based lab course for first-year undergraduate students to learn computing using RPi. The RPi allowed students to perform projects that require a physical computing aspect by using the GPIO headers on the Pi. The proposed lab task included implementing algorithms, mapping and graphing the weather, Twitter posting, and scraping. Assignments in this lab were mostly study-based. They were given topics or videos to study and discuss during discussions. The project marking criteria included application-based project ideas related to the area of computing, relative complexity, and effort, project completion status according to the original proposal, presentation of Project as well as the individual reflection. Both project evaluation and student evaluation were performed which showed that the student found the technology tools utilized were appropriate to the course as well as received positive feedback. 

Other than course design, \textcolor{black}{another} study \cite{riese2023engineering}  focused on how engineering students (non-CSE majors) experience and perceive different aspects of the introductory CS courses. The authors surveyed the students taking six different CS courses where the survey questionnaires focused on 5 different attributes of the course: Lab assignments, Exams, Individual Projects, Course Coordinators, and Teaching Assistants. These attributes influence the learning curve and experience of the students. From these surveys, one of the vital findings was that the students perceived lab assignments as a learning activity rather than “lab assignment as a necessary evil” \cite{inproceedingsRiese}. On the other hand, the project in these courses was experienced as the most enjoyable and evocative one as they solved real-life problems in the projects \cite{riese2023engineering}. However, they also found a gap between the projects and lab experiments. That’s why collaborative teamwork seemed a viable option to them while working on such projects \cite{riese2023engineering}. Another study \cite{ShekharUnpack} was conducted in this paper to find the factors that can motivate high school students in engineering and computer science subjects leveraging project-based learning methods. Motivation in engineering education brings better outcomes from engineering courses. \textcolor{black}{Conversely}, several studies \cite{Millsarticle}, \cite{Chenarticle} suggest that project-based learning is more effective in engaging students to participate in teamwork and motivates them to bring out their best. The authors formulated six diversified focus groups from four different schools in different US Schools. First, the authors \cite{ShekharUnpack} proposed some technical classes on robotics, embedded systems, and IoT-based systems using Arduino, and RPi with Python programming for the focus groups. Students attended these classes and learned basic hardware-software interfacing which enables them to apply basic programming logic to hardware while conducting some real-life projects. Then a survey was conducted which suggested that, while complex projects challenged the students to think deeper and boosted their confidence, problem definitions in the project should be made clear for them. Multidisciplinary approaches such as cross-combining mathematics, programming, and other subjects were found effective in motivating the students. However, project-based learning became one of the key motivating factors according to the survey where the students worked in a team to solve real-life problems which were relevant contextually. Zhao et al. \cite{ZhaoGame} proposed a game-based approach to create an interactive environment with the students to teach programming knowledge to the students. Over 100 students from three institutions studying at first and second-year undergraduate levels participated in the pilot project of the proposed approach where the evaluation was quantified using demographic questionnaires. The findings indicated that the game-based approach helped students to learn the concepts in an interactive environment while captivating their concentration.  However, step-by-step instructions are needed in such a game-based approach, otherwise the user experience can be affected severely. Also, learning capabilities vary from person to person where personalization is needed in the proposed games. These types of game-based learning lack personalization whereas project-based learning motivates personalized learning capabilities. Chiu et al.~\cite{chiu2021creation} discussed creating a pre-tertiary AI curriculum for students in Hong Kong schools. The objective of creating such an AI curriculum for the students was to motivate them to pursue this multidisciplinary technology in the future. The curriculum framework consists of beginner levels to advanced levels while incorporating knowledge of AI, application-based tasks as well as the ethical side of AI, overall delivering the future impact of AI towards these young minds. The authors developed a Robot car, "CUHKiCar" which allows the students to perform AI experiments. Also, reprogramming options allow the students to learn different functionalities of the AI which enables interactive learning. The proposed curriculum in this work is flexible for the teachers and students offering versatile learning approaches. The curriculum was also evaluated after surveys which showed that students became motivated to learn AI, and solve complex projects after completing this curriculum.    
Other relevant works \cite{25, 26, 27, 28} proposed different engineering lab courses while maintaining a similar strategy as discussed in previous literature above.

The majority of previous methods have taken multidisciplinary approaches while designing the courses. However, they have usually taken close-ended approaches which indicate teaching different experiments and evaluating the students based on these experiments. On the other hand, some studies proposed open-ended (project-based) approaches. The overall scenario of our literature review is divided into several segments shown in Table~\ref{tab:literature} where we can see the main focus or outputs of the different studies are multidisciplinary and project-based approaches. However, the assessments in the course design also included close-ended evaluations.


In our work, we have taken a mixture of \textit{close-ended} and \textit{open-ended} approaches for more effective learning. In this approach, we designed the experiments while keeping the multidisciplinary perspective as well as the feedback from the industry and academic experts.