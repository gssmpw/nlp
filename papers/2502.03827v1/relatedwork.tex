\section{Related Work}
% Several prior surveys on Arabic sentiment analysis exist \cite{Almurqren2024ArabicTS}, \cite{ALAYYOUB2019320}, \cite{arabic_survey_subjectivity}, \cite{arabic_survey_sc}, \cite{arabic_survey_asa_social_media}, \cite{arabic_survey_asa}, \cite{arabic_survey_sarcasm_detection}, \cite{asa_survey}. \cite{arabic_survey_subjectivity} provides trends and challenges for subjectivity and sentiment analysis of Arabic, which covers methods before 2014 and therefore does not contain any deep learning based methods. \cite{arabic_survey_sc} organises Arabic sentiment classification methods which are published before 2016, they category the existing methods into lexicon based, machine learning based and hybrid approaches. Similarly, deep learning based methods are not covered. \cite{arabic_survey_asa_social_media} surveys ASA methods that are published before 2017 with a focus on social media. \cite{arabic_survey_asa} provides a survey for ASA which includes the works before 2019. Although deep learning has already been widely adopted in general SA, their survey still focuses on works utilising lexicon based methods and traditional machine learning methods. \cite{arabic_survey_sarcasm_detection} provides a systematic survey for Arabic sarcasm detection, while our work focuses on the broader field of ASA which includes Arabic sarcasm detection. The most related work is the survey by \cite{asa_survey} which provides systematic survey for ASA utilising different methods that includes lexicon based methods, machine learning based methods and deep learning based methods. However, our work distinguishes itself through the following contributions:




%The contributions of this paper are as follows: (1). We introduce a taxonomy for sentiment analysis which can cover most of the sentiment analysis tasks. (2). We provide a systematic and up-to-date survey for Arabic sentiment analysis, especially the works that use deep learning methods. (3). We put Arabic sentiment analysis in a wider context of general sentiment analysis and point out the gap between them. (4). We outline the challenges and future directions for research. 

%We present a detailed taxonomy for sentiment analysis, and categorize existing sentiment analysis approaches into specific, well-defined categories. We then present a comprehensive analysis of sentiment analysis methods proposed for each category according to this taxonomy.  


% \textcolor{red}{Is this the beginning of a new section?}
\begin{comment}
\begin{table*} 
  \centering
  % \captionsetup{justification=centering}
  % \caption{\label{table: datasets} Datasets for Arabic Sentiment Analysis, organised according to our proposed taxonomy in Section \ref{section: taxonomy}.\\ \textit{SC: Sentiment Classification, MAST: Multifacted Analysis of Subjective Text, ABSA: Aspect based Sentiment Analysis.}}
  \begin{tabular}{m{5cm} m{1.2cm} m{2cm} m{2.7cm} m{3.2cm}}
    \hline
    \textbf{Dataset} & \textbf{Modality} & \textbf{Granularity} & \textbf{Context} & \textbf{Dialect} \\
    \hline
    LABR \cite{aly-atiya-2013-labr} & text & SC & document-level & MSA  and various other dialects \\
    ASTD \cite{nabil-etal-2015-astd} & text & SC & document-level & MSA and Egyptian Arabic \\
    ArSentD-LEV \cite{DBLP:journals/corr/abs-1906-01830} & text & SC, ABSA, MAST & document-level & Levantine dialect \\
    ArSarcasm \cite{abu-farha-magdy-2020-arabic} & text & SC, MAST & document-level & various dialects\\
    ArSarcasm-v2 \cite{abufarha-etal-2021-arsarcasm-v2} & text & SC, MAST & document-level & various dialects \\
    Arsen-20 \cite{fang2024arsen} & text & SC & document-level & various dialects \\ 
    Arabic multimodal dataset \cite{Haouhat2023TowardsAM} & text, audio, video & SC & document-level (video segments) & various dialects \\ 
    \hline
  \end{tabular}
  \captionsetup{justification=centering}
  \caption{\label{table: datasets} Datasets for Arabic Sentiment Analysis, organised according to modality, granularity and context.\\ \textit{SC: Sentiment Classification, MAST: Multifacted Analysis of Subjective Text, ABSA: Aspect based Sentiment Analysis.}}
\end{table*}
\end{comment} 

\begin{comment}