%%%%%%%% ICML 2025 EXAMPLE LATEX SUBMISSION FILE %%%%%%%%%%%%%%%%%

\documentclass{article}

% Recommended, but optional, packages for figures and better typesetting:
\usepackage{microtype}
\usepackage{graphicx}
\usepackage{subfigure}
\usepackage{booktabs} % for professional tables
\usepackage{xfrac}
\usepackage{enumitem}
\setlist{itemsep=1pt, topsep=1pt} % or \setlist{noitemsep} to leave space around whole list
% hyperref makes hyperlinks in the resulting PDF.
% If your build breaks (sometimes temporarily if a hyperlink spans a page)
% please comment out the following usepackage line and replace
% \usepackage{icml2025} with \usepackage[nohyperref]{icml2025} above.
\usepackage{hyperref}
\usepackage{sidecap}
%\renewcommand\sidecaptionsep{4mm}

% Attempt to make hyperref and algorithmic work together better:
\newcommand{\theHalgorithm}{\arabic{algorithm}}

% Use the following line for the initial blind version submitted for review:
% \usepackage{icml2025}

% If accepted, instead use the following line for the camera-ready submission:
\usepackage[accepted]{icml2025}
%\usepackage{icml2025}

% For theorems and such
\usepackage{amsmath}
\usepackage{amssymb}
\usepackage{mathtools}
\usepackage{amsthm}
\usepackage{lipsum}
% if you use cleveref..
\usepackage[capitalize,noabbrev]{cleveref}

%%%%%%%%%%%%%%%%%%%%%%%%%%%%%%%%
% THEOREMS
%%%%%%%%%%%%%%%%%%%%%%%%%%%%%%%%
\theoremstyle{plain}
\newtheorem{theorem}{Theorem}[section]
\newtheorem{proposition}[theorem]{Proposition}
\newtheorem{lemma}[theorem]{Lemma}
\newtheorem{corollary}[theorem]{Corollary}
\theoremstyle{definition}
\newtheorem{definition}[theorem]{Definition}
\newtheorem{assumption}[theorem]{Assumption}
\theoremstyle{remark}
\newtheorem{remark}[theorem]{Remark}
% In your preamble, include the following:
\usepackage{amsthm} % For theorem-like environments

% Define the Problem Statement environment
\newtheoremstyle{problemstyle} % Name of the style
  {3pt} % Space above
  {3pt} % Space below
  {\normalfont} % Body font
  {} % Indent amount
  {\bfseries} % Theorem head font
  {.} % Punctuation after theorem head
  {0.5em} % Space after theorem head
  {} % Theorem head spec

\theoremstyle{problemstyle}
\newtheorem{problem}{Problem} % Define the "problem" environment

\newtheorem*{proofsketch}{Proof sketch}
% Todonotes is useful during development; simply uncomment the next line
%    and comment out the line below the next line to turn off comments
%\usepackage[disable,textsize=tiny]{todonotes}
\usepackage[textsize=tiny]{todonotes}

\newcommand{\stas}[1]{\textcolor{teal}{ {\small \ (stas)} #1}}
\newcommand{\point}[1]{\textcolor{red}{ {\small \ (point)} #1}}
\usepackage{subcaption}
\usepackage{tikz,pgfplots}
\newcommand{\fix}{\marginpar{FIX}}
\newcommand{\new}{\marginpar{NEW}}
\newcommand\myeq{\mathrel{\overset{\makebox[0pt]{\mbox{\normalfont\tiny\sffamily d}}}{=}}}
\newcommand\eqdistrib{\mathrel{\overset{\makebox[0pt]{\mbox{\normalfont\tiny\sffamily d}}}{=}}}
% The \icmltitle you define below is probably too long as a header.
% Therefore, a short form for the running title is supplied here:
\icmltitlerunning{Identifying metric structures of deep latent variable models}

\begin{document}

\twocolumn[
\icmltitle{Identifying metric structures of deep latent variable models}

% It is OKAY to include author information, even for blind
% submissions: the style file will automatically remove it for you
% unless you've provided the [accepted] option to the icml2025
% package.

% List of affiliations: The first argument should be a (short)
% identifier you will use later to specify author affiliations
% Academic affiliations should list Department, University, City, Region, Country
% Industry affiliations should list Company, City, Region, Country

% You can specify symbols, otherwise they are numbered in order.
% Ideally, you should not use this facility. Affiliations will be numbered
% in order of appearance and this is the preferred way.
\icmlsetsymbol{equal}{*}

\begin{icmlauthorlist}
\icmlauthor{Stas Syrota}{dtu}
\icmlauthor{Yevgen Zainchkovskyy}{dtu}
\icmlauthor{Johnny Xi}{ubc}
\icmlauthor{Benjamin Bloem-Reddy}{ubc}
\icmlauthor{Søren Hauberg}{dtu}
\end{icmlauthorlist}

\icmlaffiliation{dtu}{Department of Applied Mathematics and Computer Science,  Technical University of Denmark, Lyngby, Denmark}
\icmlaffiliation{ubc}{Department of Statistics, University of British Columbia}

\icmlcorrespondingauthor{Stas Syrota}{stasy@dtu.dk}
% You may provide any keywords that you
% find helpful for describing your paper; these are used to populate
% the "keywords" metadata in the PDF but will not be shown in the document
\icmlkeywords{Machine Learning, ICML}

\vskip 0.3in
]

% this must go after the closing bracket ] following \twocolumn[ ...

% This command actually creates the footnote in the first column
% listing the affiliations and the copyright notice.
% The command takes one argument, which is text to display at the start of the footnote.
% The \icmlEqualContribution command is standard text for equal contribution.
% Remove it (just {}) if you do not need this facility.

\printAffiliationsAndNotice{}  % leave blank if no need to mention equal contribution
%\printAffiliationsAndNotice{\icmlEqualContribution} % otherwise use the standard text.

\begin{abstract}
Deep latent variable models learn condensed representations of data that, hopefully, reflect the inner workings of the studied phenomena. Unfortunately, these latent representations are not statistically identifiable, meaning they cannot be uniquely determined. Domain experts, therefore, need to tread carefully when interpreting these. Current solutions limit the lack of identifiability through additional constraints on the latent variable model, e.g.\@ by requiring labeled training data, or by restricting the expressivity of the model. We change the goal: instead of identifying the latent variables, we identify \emph{relationships} between them such as meaningful distances, angles, and volumes. We prove this is feasible under very mild model conditions and without additional labeled data. We empirically demonstrate that our theory results in more reliable latent distances, offering a principled path forward in extracting trustworthy conclusions from deep latent variable models.

%which gives practitioners a way forward.
\end{abstract}


\section{Introduction}
Latent variable models express the density of observational data through a set of latent, i.e.\@ unobserved, variables that ideally capture the driving mechanisms of the data-generating phenomena. For example, the latent variables of a variational autoencoder \citep{VAE, rezende2014stochastic} trained on protein data, can reveal the underlying protein evolution which can help domain experts understand a problem of study \citep{riesselman2018deep, ding2019deciphering, Detlefsen_2022}. % Similarly, when aiming to explain how a neural network functions, we want to reason about its (latent) weights.

Unfortunately, latent variables are rarely identifiable, i.e.\@ they cannot be uniquely estimated from data. This lack of uniqueness prevents reliable analysis of the learned latent variables as the analysis becomes subject to the arbitrariness of model training. Fig.~\ref{fig:transcriptomic} exemplifies the issue, where two independently trained latent representations are estimated from transcriptomic data \citep{tasic2018shared}. While clusters are similarly estimated across models, the \emph{relationships} between clusters vary significantly, preventing us from extracting intra-cluster information from the plots. This fundamental issue has sparked the development of training heuristics to limit the issue \citep{thearttsne}, and the practice of analyzing latent variables has been both disputed \citep{chari2023specious} and defended \citep{lause2024art}. This entire discussion could have been avoided if only distances between latent variables had been identifiable.\looseness=-1

\begin{figure}[t]
    \centering
    \rotatebox{90}{\footnotesize{~~~~~Latent representation}}
    \includegraphics[width=0.4\linewidth]{pics/teaser/scatter_1.png}
    \includegraphics[width=0.4\linewidth]{pics/teaser/scatter_3.png}\\
    \rotatebox{90}{\footnotesize{Latent Euclidean distances}}
    \includegraphics[width=0.4\linewidth]{pics/teaser/distances_1.png}
    \includegraphics[width=0.4\linewidth]{pics/teaser/distances_3.png}
    \vspace{-2mm}
    \caption{Latent representations of transcriptomic data (top row) changes with model retraining. Each column corresponds to a model trained from scratch. The latent variables are not identifiable and change between training runs. Pairwise Euclidean distances (bottom row, averaged across cell types) also change significantly between runs. This lack of identifiability prevents us from reliably using the latent representations to understand the underlying biology.}
    \label{fig:transcriptomic}
\end{figure}


Providing identifiability guarantees has been heavily investigated. We survey key results in Secs.~\ref{sec:background} and \ref{sec:related_work}, but the executive summary is that current approaches require significant model restrictions (e.g.\@ linearity assumptions), labeled data, or a combination of both. These solutions are underwhelming to practitioners who often lack a labeling mechanism and are keen to leverage contemporary generative models.


\textbf{In this paper}, we change the identifiability question to arrive at working tools that do not require data labels and only impose minimal restrictions on the used models. Instead of identifying the latent variables, we identify the \emph{relationship between latent variables}. For example, instead of identifying the coordinates of the latent variables we identify pairwise distances. In our experience, this matches the needs of domain experts who rarely assign meaning to the coordinates of latent variables. Using differential geometry, we prove strong identifiability guarantees on pairwise distances, angles, and more. We empirically validate our theory on two variational autoencoders. % and release the code.



%\point{Within the general class of deep latent variable models, can we achieve strongly identifiable latent metric spaces in the latent space of the model without compromising its flexibility?}

%\begin{itemize}
%    \item Introduce the reader to what identifiability means in this context and why it is important. 
%    \item Briefly explain the differences between strong and weak identifiability
%    \item briefly mention work so far and why it is, honestly, a disappointment
%    \item Lead the reader from the hopeless "nothing can ever be identifiable, we are doomed" to the pragmatic "we can still do a lot if we could establish identifiability of the structure"
%    \item state point 1 from the above as the main question. 
%    \item When introducing the approach, highlight that we are joining two lines of work (bloem-reddy and diffgeom). We are framing strong identifiability of distances as reparametrization invariance in differential geometry. Mention that this approach is scalable, feasible and has various degrees of "purity" but evidence for its efficacy in any case. 
%    \item End with a teaser like "Is diffgeom the answer to identifiability concerns? We will make such an argument"
%\end{itemize}
%While our main result concerns establishing strongly identifiable distances in the latent space, along the way towards the goal, we establish the following facts:
%\begin{itemize}
%    \item each decoder in $f\in \mathcal{F}$ gives rise to an embedded submanifold of the observation space $\mathcal{D}$. A fact that is frequently alluded to in the literature, whereas we provide the first (to our knowledge) proof of the fact
%    \item the indeterminacy transformations $\mathbf{A}(M)$ for the model defined in Eq.~\ref{eq:StatModLvm} are reparametrizations of the same manifold equivalent to the concept of charts that is well-studied in differential geometry


\section{Background}\label{sec:backgrnd}

\subsection{Cold Start Latency and Mitigation Techniques}

Traditional FaaS platforms mitigate cold starts through snapshotting, lightweight virtualization, and warm-state management. Snapshot-based methods like \textbf{REAP} and \textbf{Catalyzer} reduce initialization time by preloading or restoring container states but require significant memory and I/O resources, limiting scalability~\cite{dong_catalyzer_2020, ustiugov_benchmarking_2021}. Lightweight virtualization solutions, such as \textbf{Firecracker} microVMs, achieve fast startup times with strong isolation but depend on robust infrastructure, making them less adaptable to fluctuating workloads~\cite{agache_firecracker_2020}. Warm-state management techniques like \textbf{Faa\$T}~\cite{romero_faa_2021} and \textbf{Kraken}~\cite{vivek_kraken_2021} keep frequently invoked containers ready, balancing readiness and cost efficiency under predictable workloads but incurring overhead when demand is erratic~\cite{romero_faa_2021, vivek_kraken_2021}. While these methods perform well in resource-rich cloud environments, their resource intensity challenges applicability in edge settings.

\subsubsection{Edge FaaS Perspective}

In edge environments, cold start mitigation emphasizes lightweight designs, resource sharing, and hybrid task distribution. Lightweight execution environments like unikernels~\cite{edward_sock_2018} and \textbf{Firecracker}~\cite{agache_firecracker_2020}, as used by \textbf{TinyFaaS}~\cite{pfandzelter_tinyfaas_2020}, minimize resource usage and initialization delays but require careful orchestration to avoid resource contention. Function co-location, demonstrated by \textbf{Photons}~\cite{v_dukic_photons_2020}, reduces redundant initializations by sharing runtime resources among related functions, though this complicates isolation in multi-tenant setups~\cite{v_dukic_photons_2020}. Hybrid offloading frameworks like \textbf{GeoFaaS}~\cite{malekabbasi_geofaas_2024} balance edge-cloud workloads by offloading latency-tolerant tasks to the cloud and reserving edge resources for real-time operations, requiring reliable connectivity and efficient task management. These edge-specific strategies address cold starts effectively but introduce challenges in scalability and orchestration.

\subsection{Predictive Scaling and Caching Techniques}

Efficient resource allocation is vital for maintaining low latency and high availability in serverless platforms. Predictive scaling and caching techniques dynamically provision resources and reduce cold start latency by leveraging workload prediction and state retention.
Traditional FaaS platforms use predictive scaling and caching to optimize resources, employing techniques (OFC, FaasCache) to reduce cold starts. However, these methods rely on centralized orchestration and workload predictability, limiting their effectiveness in dynamic, resource-constrained edge environments.



\subsubsection{Edge FaaS Perspective}

Edge FaaS platforms adapt predictive scaling and caching techniques to constrain resources and heterogeneous environments. \textbf{EDGE-Cache}~\cite{kim_delay-aware_2022} uses traffic profiling to selectively retain high-priority functions, reducing memory overhead while maintaining readiness for frequent requests. Hybrid frameworks like \textbf{GeoFaaS}~\cite{malekabbasi_geofaas_2024} implement distributed caching to balance resources between edge and cloud nodes, enabling low-latency processing for critical tasks while offloading less critical workloads. Machine learning methods, such as clustering-based workload predictors~\cite{gao_machine_2020} and GRU-based models~\cite{guo_applying_2018}, enhance resource provisioning in edge systems by efficiently forecasting workload spikes. These innovations effectively address cold start challenges in edge environments, though their dependency on accurate predictions and robust orchestration poses scalability challenges.

\subsection{Decentralized Orchestration, Function Placement, and Scheduling}

Efficient orchestration in serverless platforms involves workload distribution, resource optimization, and performance assurance. While traditional FaaS platforms rely on centralized control, edge environments require decentralized and adaptive strategies to address unique challenges such as resource constraints and heterogeneous hardware.



\subsubsection{Edge FaaS Perspective}

Edge FaaS platforms adopt decentralized and adaptive orchestration frameworks to meet the demands of resource-constrained environments. Systems like \textbf{Wukong} distribute scheduling across edge nodes, enhancing data locality and scalability while reducing network latency. Lightweight frameworks such as \textbf{OpenWhisk Lite}~\cite{kravchenko_kpavelopenwhisk-light_2024} optimize resource allocation by decentralizing scheduling policies, minimizing cold starts and latency in edge setups~\cite{benjamin_wukong_2020}. Hybrid solutions like \textbf{OpenFaaS}~\cite{noauthor_openfaasfaas_2024} and \textbf{EdgeMatrix}~\cite{shen_edgematrix_2023} combine edge-cloud orchestration to balance resource utilization, retaining latency-sensitive functions at the edge while offloading non-critical workloads to the cloud. While these approaches improve flexibility, they face challenges in maintaining coordination and ensuring consistent performance across distributed nodes.


\section{Problem Statement}
\label{sec:problem}

This work studies two popular models of opinion exchange on networks. The overarching goal is for a network of truthful, rational agents to learn a binary piece of information, which we call the state of the world or \emph{ground truth}. We can also think of this state as an optimal binary action (buying or selling a stock, voting for a political party's candidate, etc.). The agents are arranged on a directed graph $G=(V,E)$ and broadcast which of the two states they believe is more probable to their neighbors. 

More explicitly, we encode the ground truth in $ \theta \in \{0,1\} $, distributed according to $ \berd{q} $, Bernoulli distribution with probability of $q$ of taking $1$ and $1-q$ of taking $0$. Every agent $v \in V$ initially receives an independent \emph{private signal} $ s_v \in \{0,1\}$ correlated with the ground truth. They then announce a prediction $a_v \in \{ 0,1 \} $ of the ground truth along outgoing edges, so out-neighbors of $v$ may use $a_v$ to improve their own predictions. Importantly, the probabilities $p$ and $q$, as well as the graph $G$ are all common knowledge. 
This is captured in the following formal definition of a network.

\begin{definition}[Social Network]
    A \emph{social network} is $ \network \deq ( G,q,p ) $,
    where \begin{enumerate}
        \item $ G = ( V,E ) $ is a directed graph with agents as vertices,
        \item $ q \in ( 0,1 ) $ is the prior probability of $\theta = 1$,
        \item $ p \in ( \frac 12, 1 ) $ is the accuracy of agents' private signals $s_v \in \{0,1\}$, such that \[\pr{s_v = 1 \suchthat \theta = 1} = \pr{s_v = 0 \suchthat \theta = 0} = p, \quad \forall v \in V.
            \]
    \end{enumerate}
    We further denote $ n \deq \absolute{V} $.
\end{definition}

We consider a classic asynchronous \emph{sequential} model ~\cite{Golub2017-qo}, in which agents announce their predictions in a \emph{decision ordering}, given by a one-to-one mapping $\sigma: V \to [n]$.
We denote the set of all possible orderings by $\Sigma_n$.
At every time step $i$, agent $v = \sigma^{-1}(i)$ makes an announcement $a_v \in \{0,1\}$.
The announcement depends on the agent's private measurement $ s_v $, along with the \emph{previous announcements of in-neighbors}, which we denote as a tuple $ N_v $, defined as \[
    N_v \deq ( a_u \suchthat u \in V \land uv \in E \land \sigma(u)<\sigma(v) ).
\] 
We call the tuple $X_v = (s_v) \cup N_v$ the \emph{inputs} of node $v$. 
This setup of limiting visibility to an agent's neighborhood has been studied in a number of recent papers~\cite{Bahar2020-am,arieli2020social,lu24enabling}.

When making announcements, agents follow an \emph{aggregation rule}, which is a function $ \mu:  ( X_v, G, \sigma ) \mapsto a_v $.
Broadly speaking, aggregation rules can either be Bayesian or non-Bayesian. 
In the \emph{Bayesian} model,  agents are fully rational
and make predictions according to their posterior probability for $\theta$, given their inputs and knowledge of the network topology $G$. In particular, agents take into account the correlation between their inputs resulting from the network topology and from the current ordering $ \sigma $. \[
     \mu^B(X_v, G, \sigma) \deq \begin{cases}
         1 & \text{if $\Pr_{G,\sigma}[\theta = 1 \suchthat X_v] > \frac 12$,} \\
         0 & \text{if $\Pr_{G,\sigma}[\theta = 0 \suchthat X_v] > \frac 12$,} \\
         \berd{\frac 12} & \text{otherwise.}
     \end{cases}
 \]

We also consider a non-Bayesian model, in which agents have bounded rationality and instead use simpler heuristic rules.
This is perhaps a more practical model, as computing posterior probabilities in arbitrary networks can become computationally expensive. 
In particular, we examine the \emph{majority dynamics} model, in which agents simply follow the majority among their inputs~\cite{Bahar2020-am,Shoham1992-ir,Laland2004-ej}. Since this model does not require agents to take into account correlations between their inputs derived from the network topology or the ordering, we omit $G$ and $ \sigma $ as inputs to $\mu^M$: \[
    \mu^M(X_v) \deq \begin{cases}
        1 & \text{if $  \frac 1{\absolute{X_v}}\sum_{x \in X_v} x > \frac 12 $,} \\
        0 & \text{if $  \frac 1{\absolute{X_v}}\sum_{x \in X_v} x < \frac 12 $,} \\
        s_v & \text{otherwise.}
    \end{cases}
\]

Finally, we quantify how successful the network is in predicting the ground truth by defining the following notion of a learning rate.
\begin{definition}
    The \emph{cumulative learning rate} (CLR) of a network $ \network $ under the ordering $ \sigma $ and an aggregation rule $ \mu $ is \[
		\clr (\network, \sigma, \mu) \deq \E_{\theta, s} \left[ \sum_{v \in V}^{} \mathds{1}_{\{a_v = \theta\}} \right] = \sum_{v \in V} \Pr_{\theta, s}\left[a_v = \theta\right],
	\]
    where the equality follows from linearity of expectation.
    Further, the \emph{learning rate} (LR) of a network $ \network $ under the ordering $ \sigma $ is simply \[
		\lr (\network, \sigma, \mu) \deq \tfrac 1 n \clr (\network, \sigma, \mu).
	\]
\end{definition}

We are mainly interested in the \emph{optimal} learning rate of a network, defined as follows.

\begin{definition}[Optimal LRs]
    The \emph{optimal cumulative learning rate} of a network $ \network $ is \[
        \oclr (\network, \mu) \deq \max_{\sigma \in \Sigma_n} \clr (\network, \sigma, \mu),
    \]
    and the \emph{optimal learning rate} of a network $ \network $ is \[
        \olr (\network, \mu) \deq \max_{\sigma \in \Sigma_n} \lr (\network, \sigma, \mu).
    \]
\end{definition}

Note that when $ p $ and $ q $ are clear from the context, we use the learning rate notation with only the graph, for example $ \olr (G, \mu) = \olr(( G,p,q ), \mu) $.
We can now present a formal definition of our main focus, the \netlearnopt{} optimization problem, and \netlearn{}, its decision version.

\begin{definition}[\netlearnopt{}]
    Suppose $\mu$ is a fixed aggregation rule.
    Given a network $ \network $, the \netlearnopt{} problem is to maximize  $ \lr (\network, \sigma, \mu) $, over $ \sigma \in \Sigma_n $.
\end{definition}


\begin{definition}[\netlearn{}]\label{def:decnetlearn}
    Suppose $\mu$ is a fixed aggregation rule.
    Given a network $ \network $ and a constant threshold $ \varepsilon \in (0,1)$,
    the \netlearn{} decision problem asks whether \[
            ( \exists \sigma \in \Sigma_n )\quad \lr (\network, \sigma, \mu) \geq 1-\varepsilon.
        \]
\end{definition}

Note that \Cref{def:decnetlearn} can be formulated equivalently by asking whether an optimal ordering $\sigma^*$ which maximizes the network learning rate achieves LR at least $1-\varepsilon$.

In \Cref{sec:bayes,sec:maj}, we focus on the decision problem, offering a proof that it is \np-hard for $ \mu = \mu^B $ and $ \mu = \mu^M $.
Finally, in \Cref{sec:approx}, we use insights from the \np-hardness proofs to show \netlearnopt{} is hard to even approximate.
Surprisingly, this gives us a stronger \np-hardness statement, showing that \netlearn{} is \np-hard even if we arbitrarily fix the agents' accuracy $ p \in ( \frac 12, 1 ) $.
 
\begin{table*}[t]
\resizebox{\textwidth}{!}{
\begin{tabular}{cccccccccccc}
\hline
\multirow{2}{*}{$\alpha$} & \multirow{2}{*}{Method} & \multicolumn{5}{c}{$\beta = 0.2$}                                                                                                         & \multicolumn{5}{c}{$\beta = 0.4$}                                                                                     \\ \cline{3-12} 
                          &                         & Precision                 & Recall                    & F1                        & Accuracy                  & AUC                       & Precision             & Recall                & F1                    & Accuracy              & AUC                   \\ \hline
\multirow{7}{*}{0.2}      & FedAvg                  & 53.94 $\pm$ 3.56          & 54.05 $\pm$ 3.53          & 53.21 $\pm$ 3.90          & 53.41 $\pm$ 3.71          & 69.39 $\pm$ 2.05          & 53.58 $\pm$ 2.34      & 53.80 $\pm$ 2.77      & 52.30 $\pm$ 2.32      & 52.64 $\pm$ 2.14      & 68.20 $\pm$ 2.72      \\
                          & FedProx                 & 55.81 $\pm$ 1.58          & 54.37 $\pm$ 5.19          & 54.32 $\pm$ 2.21          & 54.62 $\pm$ 2.18          & 68.29 $\pm$ 2.93          & 54.39 $\pm$ 2.69      & 54.36 $\pm$ 1.82      & 53.27 $\pm$ 2.99      & 53.52 $\pm$ 2.97      & 68.11 $\pm$ 2.74      \\
                          & FedMed-GAN              & 54.62 $\pm$ 1.02          & 55.18 $\pm$ 2.73          & 53.46 $\pm$ 0.97          & 53.52 $\pm$ 1.07          & 68.80 $\pm$ 2.16          & 53.87 $\pm$ 3.66      & 54.66 $\pm$ 3.57      & 53.35 $\pm$ 3.61      & 53.41 $\pm$ 3.63      & 68.09 $\pm$ 2.53      \\
                          & FedMI                   & 54.77 $\pm$ 2.94          & 55.22 $\pm$ 4.36          & 54.26 $\pm$ 2.82          & 54.40 $\pm$ 2.83          & 69.15 $\pm$ 2.73          & 54.03 $\pm$ 3.88      & 54.46 $\pm$ 3.56      & 53.54 $\pm$ 4.22      & 53.63 $\pm$ 4.14      & 67.61 $\pm$ 3.93      \\
                          & MFCPL                   & 54.54 $\pm$ 1.88          & \textbf{56.06 $\pm$ 4.14} & 53.88 $\pm$ 1.64          & 54.07 $\pm$ 1.93          & 69.39 $\pm$ 2.71          & 53.86 $\pm$ 5.42      & 54.01 $\pm$ 5.44      & 53.22 $\pm$ 5.16      & 53.30 $\pm$ 5.34      & 67.58 $\pm$ 3.67      \\
                          & PmcmFL                  & 53.17 $\pm$ 1.60          & 52.73 $\pm$ 3.50          & 52.58 $\pm$ 1.81          & 52.64 $\pm$ 2.03          & 67.51 $\pm$ 4.07          & 49.03 $\pm$ 1.55      & 48.01 $\pm$ 3.03      & 48.38 $\pm$ 2.00      & 48.46 $\pm$ 2.31      & 63.71 $\pm$ 2.50      \\
                          & ClusMFL                    & \textbf{57.16 $\pm$ 2.32}     & 54.73 $\pm$ 3.93              & \textbf{56.56 $\pm$ 2.36}     & \textbf{56.92 $\pm$ 2.41}     & \textbf{72.81 $\pm$ 3.64}     & \textbf{56.06 $\pm$ 1.31} & \textbf{55.44 $\pm$ 2.19} & \textbf{55.38 $\pm$ 1.07} & \textbf{55.49 $\pm$ 1.10} & \textbf{72.50 $\pm$ 2.02} \\ \hline
\multirow{7}{*}{0.4}      & FedAvg                  & 53.75 $\pm$ 3.86          & 54.26 $\pm$ 3.76          & 52.76 $\pm$ 4.64          & 53.08 $\pm$ 4.34          & 67.91 $\pm$ 3.48          & 52.60 $\pm$ 3.00      & 53.95 $\pm$ 3.66      & 51.84 $\pm$ 3.06      & 51.98 $\pm$ 3.07      & 67.03 $\pm$ 3.22      \\
                          & FedProx                 & 54.36 $\pm$ 3.50          & 54.28 $\pm$ 4.17          & 53.77 $\pm$ 3.48          & 53.96 $\pm$ 3.51          & 68.26 $\pm$ 3.53          & 52.16 $\pm$ 2.20      & 52.93 $\pm$ 3.76      & 51.35 $\pm$ 2.13      & 51.54 $\pm$ 2.07      & 67.43 $\pm$ 3.64      \\
                          & FedMed-GAN              & 52.97 $\pm$ 1.63          & 52.89 $\pm$ 2.41          & 52.43 $\pm$ 1.74          & 52.53 $\pm$ 1.63          & 67.25 $\pm$ 3.45          & 53.56 $\pm$ 3.57      & 54.02 $\pm$ 3.95      & 52.54 $\pm$ 3.97      & 52.86 $\pm$ 3.63      & 67.91 $\pm$ 2.93      \\
                          & FedMI                   & 55.40 $\pm$ 2.85          & 53.82 $\pm$ 2.45          & 54.84 $\pm$ 2.75          & 54.95 $\pm$ 2.69          & 68.59 $\pm$ 2.48          & 54.01 $\pm$ 2.07      & 52.36 $\pm$ 1.98      & 53.62 $\pm$ 2.14      & 53.85 $\pm$ 2.06      & 66.70 $\pm$ 2.63      \\
                          & MFCPL                   & 55.07 $\pm$ 4.34          & 54.66 $\pm$ 3.93          & 54.34 $\pm$ 4.37          & 54.51 $\pm$ 4.57          & 66.60 $\pm$ 2.58          & 52.84 $\pm$ 4.14      & 54.69 $\pm$ 3.92      & 51.70 $\pm$ 3.89      & 51.87 $\pm$ 3.74      & 67.81 $\pm$ 3.48      \\
                          & PmcmFL                  & 51.71 $\pm$ 5.54          & 49.80 $\pm$ 6.21          & 50.77 $\pm$ 5.21          & 50.88 $\pm$ 5.42          & 65.54 $\pm$ 6.22          & 52.45 $\pm$ 3.96      & 50.03 $\pm$ 5.74      & 50.40 $\pm$ 4.72      & 50.77 $\pm$ 4.66      & 63.31 $\pm$ 5.17      \\
                          & ClusMFL                    & \textbf{56.59 $\pm$ 4.53} & \textbf{54.68 $\pm$ 4.49} & \textbf{56.17 $\pm$ 4.08} & \textbf{56.37 $\pm$ 4.10} & \textbf{73.25 $\pm$ 4.11} & \textbf{54.83 $\pm$ 6.13} & \textbf{54.88 $\pm$ 6.09} & \textbf{54.22 $\pm$ 5.89} & \textbf{54.40 $\pm$ 6.03} & \textbf{72.22 $\pm$ 4.47} \\ \hline
\end{tabular}
}
\caption{Performance Comparison Across Different Federated Learning Methods (Mean $\pm$ Standard Deviation \%) under Different Settings.}
\label{main result}
\end{table*}

\section{Experiments}
\label{sec:experiments}

Theorem~\ref{thrm:identifiable_dist_func} states that geodesic distances are identifiable while Euclidean ones are not. This suggests that geodesic distances should be more stable under model retraining than the Euclidean counterpart. To test this hypothesis, we train 30 models with different initial seeds and compute both Euclidean and geodesic distances between 100 randomly chosen point pairs from the test set. We emphasize that the pairs are the same across all models, which allows us to measure the variances of the distances across models.

\begin{SCfigure}[100][b]
    \includegraphics[width=0.55\linewidth]{pics/8decoders_pair1_v3.pdf}
    \caption{An example latent geodesic from a $\mathcal{M}$-flow model trained on three classes from \textsc{mnist}. The background color indicates model uncertainty.\looseness=-1}
    \label{fig:godesic}
\end{SCfigure}

%To demonstrate the practical benefit of geodesic distances in the latent space, we compare them to the Euclidean. For each of the experiments, we are training 30 models with different initial seeds and compute Euclidean and geodesic distances between 100 randomly chosen and fixed point pairs from the test set. Thus, for all 30 models, the pairs are the same and the experiment produces 30 distance measurements for each point pair. 

To assess the stability of the distance measures, we compute the coefficient of variation for each pair, which is evaluated as $\text{CV}(m_{pair}) = \sfrac{\sigma(m_{pair})}{\mu(m_{pair})}$ with $\sigma$ and $\mu$ denoting standard deviation and the mean, respectively. The coefficient of variation is a unitless measure of variability, where low values indicate less variability. We use this to compare the variability of Euclidean and geodesic distances.


We consider two models and datasets. First, a smaller model that satisfies all our assumptions, and second a larger model where we disregard the injectivity assumption. %Since computing geodesic distances is more involved than computing the Euclidean norm, the primary goal of our experiments is to demonstrate that our methodology is effective and feasible for modern deep latent variable models. The secondary goal is to show the efficacy of our method while relaxing assumption~\ref{ass:2} by choosing a modern and flexible architecture that is not guaranteed to be injective.
To compute geodesics we parametrize them by a spline connecting two points in the latent space and minimize its energy. The discussion around Definition~\ref{def:geodesics} in Appendix~\ref{appendix:diffgeom} covers how this leads to a geodesic. %To compute the energy, we discretize the curve, decode the line segments and compute the sum. 
It has been noted that taking decoder uncertainty into account is key to good performance \citep{arvanitidis2021latentspaceodditycurvature, hauberg:only:2018} and we follow the ensemble-based approach from \citet{pmlr-v251-syrota24a}, implying that we train an ensemble of 8 decoders. Details are in Appendix~\ref{appendix:geodesics} and the code to reproduce our results is available at the \href{https://github.com/mustass/identifiable-latent-metric-space/}{project GitHub repository}. \looseness=-1 

\textbf{\textsc{mnist} with \ref{ass:1}-\ref{ass:4} satisfied.~~}
We use $\mathcal{M}$-flows \citep{brehmer2020flows} to construct a VAE with an injective decoder. %maps lower dimensional representation to the observational space by a decoder that consists of a padding operation followed by a bijection.
%
We train this model on a 3-class subset of \textsc{mnist} with a 2D latent space for visualization purposes. An example of a geodesic curve from digit 7 to digit 0 from the test set is visualized in Fig.~\ref{fig:godesic}. The geodesic crosses class boundaries where they are well-explored by the model and offer little uncertainty.

The left panel of Fig.~\ref{fig:histogram-mnist-celeba} shows a histogram of the coefficient of variation for the 100 point-pairs,  where we see a narrower distribution with both a lower mean and spread for geodesic distances. We perform a one-sided Student's $t$-test for the null hypothesis that geodesic distances vary less than the Euclidean (Table~\ref{tab:t_tests}) and find strong evidence for the hypothesis. This demonstrates that identifiability improves reliability.


\begin{SCfigure*}[1][t]
  \includegraphics[width=0.8\textwidth]{pics/celeba_geodesic.pdf} 
  \caption{Geodesic (top) and Euclidean (bottom) interpolations, are highly similar, but distances still differ significantly (Fig.~\ref{fig:histogram-mnist-celeba}). \looseness=-1}
  \label{fig:celeba_interp}
\end{SCfigure*}


\textbf{\textsc{celeba} with \ref{ass:2} relaxed and \ref{ass:3} verified.~~}
The general VAE model is known to be effective due to its flexible decoder parametrized by a neural network with arbitrary architecture that is not guaranteed to be globally injective (\ref{ass:2}) nor to have full rank Jacobian (\ref{ass:3}). We train this model on the \textsc{celeba} dataset \cite{liu2015faceattributes} where the architecture is composed of convolutional and dense layers with ELU activation functions. The latent space dimension is 64 and we further employ Resize-Conv layers \cite{odena2016deconvolution} to improve image quality.
We follow the approach from \citet{8575533} to validate that the decoder Jacobian is, indeed, always full rank. An example geodesic is shown in Fig.~\ref{fig:celeba_interp} alongside a Euclidean counterpart, where we do not observe a significant difference between generated images.

%Since calculating the geodesics post-training requires \ref{ass:3} to be satisfied, we can verify it using approaches outlined in \cite{8575533}. A necessary prerequisite for the post-training test to be valid is that the activation functions used are monotonically increasing, which excludes e.g. ReLU and GELU functions.

Fig.~\ref{fig:histogram-mnist-celeba}(right) shows that the coefficient of variation for geodesic distances has both lower mean and standard deviation than Euclidean distances. The one-sided Student's $t$-test again validates this observation (Table~\ref{tab:t_tests}). This demonstrates that geodesic distances remain more reliable than Euclidean ones even when the injectivity assumption may be violated.


% \subsection{Implementation and experimental setup}

% To compute geodesics we parametrize them by a spline connecting two points in the latent space and minimize its energy. The discussion around Definition~\ref{def:geodesics} in Appendix~\ref{appendix:diffgeom} covers how this leads to a geodesic. %To compute the energy, we discretize the curve, decode the line segments and compute the sum. 
% It has been noted that taking decoder uncertainty into account is key to good performance \citep{arvanitidis2021latentspaceodditycurvature, hauberg:only:2018} and we follow the ensemble-based approach from \citet{pmlr-v251-syrota24a}, implying that we train an ensemble of 8 decoders. Details are in Appendix~\ref{appendix:geodesics} and the code to reproduce our results is in the supplements. \looseness=-1 %presented in this section is available at [\texttt{link redacted}]
% %Our implementation details are in Appendix~\ref{appendix:models} and code to reproduce the results presented in this section is available at [\texttt{link redacted}].
 
%\subsection{Results}

% The results from both experiments demonstrate that the geodesic distance is more robust than the Euclidean distance in the latent space of a deep latent variable model. Figure~\ref{fig:histogram-mnist-celeba} shows the histograms of coefficient of variation across 100 point pairs for both experiments where we see that on \textsc{mnist} we get a narrower distribution with both a lower mean and spread for geodesic distances.  Geodesic distances on \textsc{celeba} similarly show a distribution with a lower mean, whereas the spread that looks comparable. Another view of the results is offered in Table~\ref{tab:t_tests} that shows a One-sided Student's $t$-test for the Null hypothesis that geodesic distances vary less than the Euclidean and finds strong evidence for the hypothesis. 

\begin{SCtable}[0.7][t]
    \centering
    \begin{tabular}{lccc}
    \toprule
    & \textsc{mnist} & \textsc{celeba} \\ 
    \midrule
    $t$-statistic    & -8.64 & -22.33 \\ 
    $p$-value        & 1.00  & 1.00  \\ 
    \bottomrule
    \end{tabular}
    \caption{One-sided Student's $t$-test for the variability of geodesic versus Euclidean distances} % Geodesic distances vary significantly less.}
    \label{tab:t_tests}
\end{SCtable}


\begin{SCfigure*}[5][t]
    %\centering
        \includegraphics[width=0.8\columnwidth]{pics/mnist_prob.pdf}
        \includegraphics[width=0.8\columnwidth]{pics/celeba_prob.pdf}
    \caption{Histograms of coefficients of variation for Euclidean and geodesic distances on \textsc{mnist} (left) and \textsc{celeba} (right). Geodesic distances vary significantly less, which is quantified in Table~\ref{tab:t_tests}.}
    \label{fig:histogram-mnist-celeba}
\end{SCfigure*}
\putsec{related}{Related Work}

\noindent \textbf{Efficient Radiance Field Rendering.}
%
The introduction of Neural Radiance Fields (NeRF)~\cite{mil:sri20} has
generated significant interest in efficient 3D scene representation and
rendering for radiance fields.
%
Over the past years, there has been a large amount of research aimed at
accelerating NeRFs through algorithmic or software
optimizations~\cite{mul:eva22,fri:yu22,che:fun23,sun:sun22}, and the
development of hardware
accelerators~\cite{lee:cho23,li:li23,son:wen23,mub:kan23,fen:liu24}.
%
The state-of-the-art method, 3D Gaussian splatting~\cite{ker:kop23}, has
further fueled interest in accelerating radiance field
rendering~\cite{rad:ste24,lee:lee24,nie:stu24,lee:rho24,ham:mel24} as it
employs rasterization primitives that can be rendered much faster than NeRFs.
%
However, previous research focused on software graphics rendering on
programmable cores or building dedicated hardware accelerators. In contrast,
\name{} investigates the potential of efficient radiance field rendering while
utilizing fixed-function units in graphics hardware.
%
To our knowledge, this is the first work that assesses the performance
implications of rendering Gaussian-based radiance fields on the hardware
graphics pipeline with software and hardware optimizations.

%%%%%%%%%%%%%%%%%%%%%%%%%%%%%%%%%%%%%%%%%%%%%%%%%%%%%%%%%%%%%%%%%%%%%%%%%%
\myparagraph{Enhancing Graphics Rendering Hardware.}
%
The performance advantage of executing graphics rendering on either
programmable shader cores or fixed-function units varies depending on the
rendering methods and hardware designs.
%
Previous studies have explored the performance implication of graphics hardware
design by developing simulation infrastructures for graphics
workloads~\cite{bar:gon06,gub:aam19,tin:sax23,arn:par13}.
%
Additionally, several studies have aimed to improve the performance of
special-purpose hardware such as ray tracing units in graphics
hardware~\cite{cho:now23,liu:cha21} and proposed hardware accelerators for
graphics applications~\cite{lu:hua17,ram:gri09}.
%
In contrast to these works, which primarily evaluate traditional graphics
workloads, our work focuses on improving the performance of volume rendering
workloads, such as Gaussian splatting, which require blending a huge number of
fragments per pixel.

%%%%%%%%%%%%%%%%%%%%%%%%%%%%%%%%%%%%%%%%%%%%%%%%%%%%%%%%%%%%%%%%%%%%%%%%%%
%
In the context of multi-sample anti-aliasing, prior work proposed reducing the
amount of redundant shading by merging fragments from adjacent triangles in a
mesh at the quad granularity~\cite{fat:bou10}.
%
While both our work and quad-fragment merging (QFM)~\cite{fat:bou10} aim to
reduce operations by merging quads, our proposed technique differs from QFM in
many aspects.
%
Our method aims to blend \emph{overlapping primitives} along the depth
direction and applies to quads from any primitive. In contrast, QFM merges quad
fragments from small (e.g., pixel-sized) triangles that \emph{share} an edge
(i.e., \emph{connected}, \emph{non-overlapping} triangles).
%
As such, QFM is not applicable to the scenes consisting of a number of
unconnected transparent triangles, such as those in 3D Gaussian splatting.
%
In addition, our method computes the \emph{exact} color for each pixel by
offloading blending operations from ROPs to shader units, whereas QFM
\emph{approximates} pixel colors by using the color from one triangle when
multiple triangles are merged into a single quad.


\section{Weaknesses and open questions}
Our work provides strong identifiability guarantees for essential quantities such as pairwise distances in contemporary generative models. However, our approach is not problem-free and we highlight some pitfalls to be aware of.

\textbf{Observation metrics matter.~~}
Our identifiable geometric structure relies on the idea of locally bringing the observation space metric into the latent space. This has many benefits but also raises the question of choosing the observation space metric. This choice will directly impact the final latent distances. Did we then replace one difficult problem (identifiability) with another (choosing observation metric)? We argue that most data is equipped with units of measurement, which greatly simplifies the task of picking a suitable metric in the observation space. Furthermore, being explicit about how data is compared improves the transparency of the conducted data analysis. Finally, we emphasize that we are \emph{not} proposing to bring the Euclidean distance from observation space into the latent space, but only to do so \emph{infinitesimally}, i.e.\@ measuring \emph{along} the manifold.

\textbf{Identifiability comes at a (computational) cost.~~}
Euclidean distances are cheap to compute, unlike the geodesic distances we consider. For any geodesic distance, we must solve an iterative optimization problem. Fortunately, this is a locally convex problem, that only requires estimating a limited number of parameters, so the computation is feasible. Yet it remains significantly more costly than computing a Euclidean distance. We argue that when identifiability is important, e.g.\@ in scientific knowledge discovery and hypothesis generation, the additional computational resources are well-spent. Proposition~\ref{prop:flat_models} effectively tells us that we must choose between (cheap) flat decoders or (expensive) curved ones: we cannot get the best of both worlds.

\textbf{Compression matters.~~}
Our current work rests on the assumption that the latent space has a dimension that is less than or equal to the data space dimension, i.e.\@ $\text{dim}(\mathcal{Z}) \leq \text{dim}(\mathcal{D})$. This remains the standard setting for representation learning and generative models. However, if we, for the sake of argument, wanted to identify distances between the weights of overparametrized neural networks, then our strategies would not directly apply. Early work has begun to appear on understanding the geometric structure of overparametrized models \citep{roy:reparam:2024}, which suggests that perhaps our approach can be adapted.

\textbf{Injectivity remains an issue.~~}
Most of our assumptions are purely technical and easily satisfied in practice. The key exception is Assumption~\ref{ass:2} stating that the decoder $f$ must be \emph{injective}. The decoders of contemporary models such as diffusion models and (continuous) normalizing flows (including those trained with flow matching) are injective and the assumption is satisfied. However, general neural networks cannot be expected to be injective, such that variational autoencoders and similar models are not identifiable `out of the box'. We have demonstrated that (injective) $\mathcal{M}$-flow architectures can be used for such models, and empirically we observe that the geodesic distance increases robustness in non-injective models. This gives hope that theoretical statements can be made without the injectivity assumption. One such path forward may be to consider notions of \emph{weak injectivity} \citep{kivva2022identifiabilitydeepgenerativemodels}, which is less restrictive and more easily satisfied in practice.



\section{Conclusion}
In this paper, we show that latent distances, and similar quantities, can be statistically identified in a large class of generative models that includes contemporary models without imposing unrealistic assumptions. This is a significant improvement over existing work that tends to impose additional restrictions on either model or training data. Our results are significant when seeking to understand the mechanisms that drive the true data-generating process, e.g.\@ in scientific discovery, where reliability is essential.

Practically, it is important to note that our strategy requires no changes to how models are trained. Our constructions are entirely \emph{post hoc}, making them broadly applicable.

Our proof strategy relies on linking identifiability with Riemannian geometry; a link that does not appear to have formally been made elsewhere. This link paves a way forward as many tools readily exist for statistical computations on manifolds. For example, Riemannian counterparts to \emph{averages} \citep{karcher1977riemannian}, \emph{covariances} \citep{pennec2006intrinsic}, \emph{principal components} \citep{fletcher2004principal}, \emph{Kalman filters} \citep{hauberg:jmiv:2013}, and much more readily exist. In principle, it is also possible to devise computational tests to determine if these statistics are identifiable for a given model and dataset. This is, however, future work.




\section*{Impact statement}
%Authors are \textbf{required} to include a statement of the potential broader impact of their work, including its ethical aspects and future societal consequences. This statement should be in an unnumbered section at the end of the paper (co-located with Acknowledgements -- the two may appear in either order, but both must be before References), and does not count toward the paper page limit. In many cases, where the ethical impacts and expected societal implications are those that are well established when advancing the field of Machine Learning, substantial discussion is not required, and a simple statement such as the following will suffice:

This paper improves our collective understanding of which aspects of a statistical model can be identified. As the theoretical understanding translates directly into algorithmic tools, our work has an impact potential beyond the theoretical questions. Being able to identify relationships between latent representations of data can aid in the process of scientific discovery as we increase the reliability of the data analysis. Our work can also help provide robustness to interpretations of neural networks and other statistical models, which may help in explainability efforts.



%``This paper presents work whose goal is to advance the field of Machine Learning. There are many potential societal consequences of our work, none which we feel must be specifically highlighted here.''


\section*{Acknowledgments}

This work was supported by a research grant (42062) from VILLUM FONDEN. This project received funding from the European Research Council (ERC) under the European Union’s Horizon research and innovation programme (grant agreement 101125993). The work was partly funded by the Novo Nordisk Foundation through the Center for Basic Machine Learning Research in Life Science (NNF20OC0062606).
JX is supported by an Natural Sciences and Engineering Research Council of Canada
(NSERC) Canada Graduate Scholarship.
BBR acknowledges the support of NSERC: RGPIN2020-04995, RGPAS-2020-00095. 

\bibliography{citations}
\bibliographystyle{icml2025}
\subsection{Lloyd-Max Algorithm}
\label{subsec:Lloyd-Max}
For a given quantization bitwidth $B$ and an operand $\bm{X}$, the Lloyd-Max algorithm finds $2^B$ quantization levels $\{\hat{x}_i\}_{i=1}^{2^B}$ such that quantizing $\bm{X}$ by rounding each scalar in $\bm{X}$ to the nearest quantization level minimizes the quantization MSE. 

The algorithm starts with an initial guess of quantization levels and then iteratively computes quantization thresholds $\{\tau_i\}_{i=1}^{2^B-1}$ and updates quantization levels $\{\hat{x}_i\}_{i=1}^{2^B}$. Specifically, at iteration $n$, thresholds are set to the midpoints of the previous iteration's levels:
\begin{align*}
    \tau_i^{(n)}=\frac{\hat{x}_i^{(n-1)}+\hat{x}_{i+1}^{(n-1)}}2 \text{ for } i=1\ldots 2^B-1
\end{align*}
Subsequently, the quantization levels are re-computed as conditional means of the data regions defined by the new thresholds:
\begin{align*}
    \hat{x}_i^{(n)}=\mathbb{E}\left[ \bm{X} \big| \bm{X}\in [\tau_{i-1}^{(n)},\tau_i^{(n)}] \right] \text{ for } i=1\ldots 2^B
\end{align*}
where to satisfy boundary conditions we have $\tau_0=-\infty$ and $\tau_{2^B}=\infty$. The algorithm iterates the above steps until convergence.

Figure \ref{fig:lm_quant} compares the quantization levels of a $7$-bit floating point (E3M3) quantizer (left) to a $7$-bit Lloyd-Max quantizer (right) when quantizing a layer of weights from the GPT3-126M model at a per-tensor granularity. As shown, the Lloyd-Max quantizer achieves substantially lower quantization MSE. Further, Table \ref{tab:FP7_vs_LM7} shows the superior perplexity achieved by Lloyd-Max quantizers for bitwidths of $7$, $6$ and $5$. The difference between the quantizers is clear at 5 bits, where per-tensor FP quantization incurs a drastic and unacceptable increase in perplexity, while Lloyd-Max quantization incurs a much smaller increase. Nevertheless, we note that even the optimal Lloyd-Max quantizer incurs a notable ($\sim 1.5$) increase in perplexity due to the coarse granularity of quantization. 

\begin{figure}[h]
  \centering
  \includegraphics[width=0.7\linewidth]{sections/figures/LM7_FP7.pdf}
  \caption{\small Quantization levels and the corresponding quantization MSE of Floating Point (left) vs Lloyd-Max (right) Quantizers for a layer of weights in the GPT3-126M model.}
  \label{fig:lm_quant}
\end{figure}

\begin{table}[h]\scriptsize
\begin{center}
\caption{\label{tab:FP7_vs_LM7} \small Comparing perplexity (lower is better) achieved by floating point quantizers and Lloyd-Max quantizers on a GPT3-126M model for the Wikitext-103 dataset.}
\begin{tabular}{c|cc|c}
\hline
 \multirow{2}{*}{\textbf{Bitwidth}} & \multicolumn{2}{|c|}{\textbf{Floating-Point Quantizer}} & \textbf{Lloyd-Max Quantizer} \\
 & Best Format & Wikitext-103 Perplexity & Wikitext-103 Perplexity \\
\hline
7 & E3M3 & 18.32 & 18.27 \\
6 & E3M2 & 19.07 & 18.51 \\
5 & E4M0 & 43.89 & 19.71 \\
\hline
\end{tabular}
\end{center}
\end{table}

\subsection{Proof of Local Optimality of LO-BCQ}
\label{subsec:lobcq_opt_proof}
For a given block $\bm{b}_j$, the quantization MSE during LO-BCQ can be empirically evaluated as $\frac{1}{L_b}\lVert \bm{b}_j- \bm{\hat{b}}_j\rVert^2_2$ where $\bm{\hat{b}}_j$ is computed from equation (\ref{eq:clustered_quantization_definition}) as $C_{f(\bm{b}_j)}(\bm{b}_j)$. Further, for a given block cluster $\mathcal{B}_i$, we compute the quantization MSE as $\frac{1}{|\mathcal{B}_{i}|}\sum_{\bm{b} \in \mathcal{B}_{i}} \frac{1}{L_b}\lVert \bm{b}- C_i^{(n)}(\bm{b})\rVert^2_2$. Therefore, at the end of iteration $n$, we evaluate the overall quantization MSE $J^{(n)}$ for a given operand $\bm{X}$ composed of $N_c$ block clusters as:
\begin{align*}
    \label{eq:mse_iter_n}
    J^{(n)} = \frac{1}{N_c} \sum_{i=1}^{N_c} \frac{1}{|\mathcal{B}_{i}^{(n)}|}\sum_{\bm{v} \in \mathcal{B}_{i}^{(n)}} \frac{1}{L_b}\lVert \bm{b}- B_i^{(n)}(\bm{b})\rVert^2_2
\end{align*}

At the end of iteration $n$, the codebooks are updated from $\mathcal{C}^{(n-1)}$ to $\mathcal{C}^{(n)}$. However, the mapping of a given vector $\bm{b}_j$ to quantizers $\mathcal{C}^{(n)}$ remains as  $f^{(n)}(\bm{b}_j)$. At the next iteration, during the vector clustering step, $f^{(n+1)}(\bm{b}_j)$ finds new mapping of $\bm{b}_j$ to updated codebooks $\mathcal{C}^{(n)}$ such that the quantization MSE over the candidate codebooks is minimized. Therefore, we obtain the following result for $\bm{b}_j$:
\begin{align*}
\frac{1}{L_b}\lVert \bm{b}_j - C_{f^{(n+1)}(\bm{b}_j)}^{(n)}(\bm{b}_j)\rVert^2_2 \le \frac{1}{L_b}\lVert \bm{b}_j - C_{f^{(n)}(\bm{b}_j)}^{(n)}(\bm{b}_j)\rVert^2_2
\end{align*}

That is, quantizing $\bm{b}_j$ at the end of the block clustering step of iteration $n+1$ results in lower quantization MSE compared to quantizing at the end of iteration $n$. Since this is true for all $\bm{b} \in \bm{X}$, we assert the following:
\begin{equation}
\begin{split}
\label{eq:mse_ineq_1}
    \tilde{J}^{(n+1)} &= \frac{1}{N_c} \sum_{i=1}^{N_c} \frac{1}{|\mathcal{B}_{i}^{(n+1)}|}\sum_{\bm{b} \in \mathcal{B}_{i}^{(n+1)}} \frac{1}{L_b}\lVert \bm{b} - C_i^{(n)}(b)\rVert^2_2 \le J^{(n)}
\end{split}
\end{equation}
where $\tilde{J}^{(n+1)}$ is the the quantization MSE after the vector clustering step at iteration $n+1$.

Next, during the codebook update step (\ref{eq:quantizers_update}) at iteration $n+1$, the per-cluster codebooks $\mathcal{C}^{(n)}$ are updated to $\mathcal{C}^{(n+1)}$ by invoking the Lloyd-Max algorithm \citep{Lloyd}. We know that for any given value distribution, the Lloyd-Max algorithm minimizes the quantization MSE. Therefore, for a given vector cluster $\mathcal{B}_i$ we obtain the following result:

\begin{equation}
    \frac{1}{|\mathcal{B}_{i}^{(n+1)}|}\sum_{\bm{b} \in \mathcal{B}_{i}^{(n+1)}} \frac{1}{L_b}\lVert \bm{b}- C_i^{(n+1)}(\bm{b})\rVert^2_2 \le \frac{1}{|\mathcal{B}_{i}^{(n+1)}|}\sum_{\bm{b} \in \mathcal{B}_{i}^{(n+1)}} \frac{1}{L_b}\lVert \bm{b}- C_i^{(n)}(\bm{b})\rVert^2_2
\end{equation}

The above equation states that quantizing the given block cluster $\mathcal{B}_i$ after updating the associated codebook from $C_i^{(n)}$ to $C_i^{(n+1)}$ results in lower quantization MSE. Since this is true for all the block clusters, we derive the following result: 
\begin{equation}
\begin{split}
\label{eq:mse_ineq_2}
     J^{(n+1)} &= \frac{1}{N_c} \sum_{i=1}^{N_c} \frac{1}{|\mathcal{B}_{i}^{(n+1)}|}\sum_{\bm{b} \in \mathcal{B}_{i}^{(n+1)}} \frac{1}{L_b}\lVert \bm{b}- C_i^{(n+1)}(\bm{b})\rVert^2_2  \le \tilde{J}^{(n+1)}   
\end{split}
\end{equation}

Following (\ref{eq:mse_ineq_1}) and (\ref{eq:mse_ineq_2}), we find that the quantization MSE is non-increasing for each iteration, that is, $J^{(1)} \ge J^{(2)} \ge J^{(3)} \ge \ldots \ge J^{(M)}$ where $M$ is the maximum number of iterations. 
%Therefore, we can say that if the algorithm converges, then it must be that it has converged to a local minimum. 
\hfill $\blacksquare$


\begin{figure}
    \begin{center}
    \includegraphics[width=0.5\textwidth]{sections//figures/mse_vs_iter.pdf}
    \end{center}
    \caption{\small NMSE vs iterations during LO-BCQ compared to other block quantization proposals}
    \label{fig:nmse_vs_iter}
\end{figure}

Figure \ref{fig:nmse_vs_iter} shows the empirical convergence of LO-BCQ across several block lengths and number of codebooks. Also, the MSE achieved by LO-BCQ is compared to baselines such as MXFP and VSQ. As shown, LO-BCQ converges to a lower MSE than the baselines. Further, we achieve better convergence for larger number of codebooks ($N_c$) and for a smaller block length ($L_b$), both of which increase the bitwidth of BCQ (see Eq \ref{eq:bitwidth_bcq}).


\subsection{Additional Accuracy Results}
%Table \ref{tab:lobcq_config} lists the various LOBCQ configurations and their corresponding bitwidths.
\begin{table}
\setlength{\tabcolsep}{4.75pt}
\begin{center}
\caption{\label{tab:lobcq_config} Various LO-BCQ configurations and their bitwidths.}
\begin{tabular}{|c||c|c|c|c||c|c||c|} 
\hline
 & \multicolumn{4}{|c||}{$L_b=8$} & \multicolumn{2}{|c||}{$L_b=4$} & $L_b=2$ \\
 \hline
 \backslashbox{$L_A$\kern-1em}{\kern-1em$N_c$} & 2 & 4 & 8 & 16 & 2 & 4 & 2 \\
 \hline
 64 & 4.25 & 4.375 & 4.5 & 4.625 & 4.375 & 4.625 & 4.625\\
 \hline
 32 & 4.375 & 4.5 & 4.625& 4.75 & 4.5 & 4.75 & 4.75 \\
 \hline
 16 & 4.625 & 4.75& 4.875 & 5 & 4.75 & 5 & 5 \\
 \hline
\end{tabular}
\end{center}
\end{table}

%\subsection{Perplexity achieved by various LO-BCQ configurations on Wikitext-103 dataset}

\begin{table} \centering
\begin{tabular}{|c||c|c|c|c||c|c||c|} 
\hline
 $L_b \rightarrow$& \multicolumn{4}{c||}{8} & \multicolumn{2}{c||}{4} & 2\\
 \hline
 \backslashbox{$L_A$\kern-1em}{\kern-1em$N_c$} & 2 & 4 & 8 & 16 & 2 & 4 & 2  \\
 %$N_c \rightarrow$ & 2 & 4 & 8 & 16 & 2 & 4 & 2 \\
 \hline
 \hline
 \multicolumn{8}{c}{GPT3-1.3B (FP32 PPL = 9.98)} \\ 
 \hline
 \hline
 64 & 10.40 & 10.23 & 10.17 & 10.15 &  10.28 & 10.18 & 10.19 \\
 \hline
 32 & 10.25 & 10.20 & 10.15 & 10.12 &  10.23 & 10.17 & 10.17 \\
 \hline
 16 & 10.22 & 10.16 & 10.10 & 10.09 &  10.21 & 10.14 & 10.16 \\
 \hline
  \hline
 \multicolumn{8}{c}{GPT3-8B (FP32 PPL = 7.38)} \\ 
 \hline
 \hline
 64 & 7.61 & 7.52 & 7.48 &  7.47 &  7.55 &  7.49 & 7.50 \\
 \hline
 32 & 7.52 & 7.50 & 7.46 &  7.45 &  7.52 &  7.48 & 7.48  \\
 \hline
 16 & 7.51 & 7.48 & 7.44 &  7.44 &  7.51 &  7.49 & 7.47  \\
 \hline
\end{tabular}
\caption{\label{tab:ppl_gpt3_abalation} Wikitext-103 perplexity across GPT3-1.3B and 8B models.}
\end{table}

\begin{table} \centering
\begin{tabular}{|c||c|c|c|c||} 
\hline
 $L_b \rightarrow$& \multicolumn{4}{c||}{8}\\
 \hline
 \backslashbox{$L_A$\kern-1em}{\kern-1em$N_c$} & 2 & 4 & 8 & 16 \\
 %$N_c \rightarrow$ & 2 & 4 & 8 & 16 & 2 & 4 & 2 \\
 \hline
 \hline
 \multicolumn{5}{|c|}{Llama2-7B (FP32 PPL = 5.06)} \\ 
 \hline
 \hline
 64 & 5.31 & 5.26 & 5.19 & 5.18  \\
 \hline
 32 & 5.23 & 5.25 & 5.18 & 5.15  \\
 \hline
 16 & 5.23 & 5.19 & 5.16 & 5.14  \\
 \hline
 \multicolumn{5}{|c|}{Nemotron4-15B (FP32 PPL = 5.87)} \\ 
 \hline
 \hline
 64  & 6.3 & 6.20 & 6.13 & 6.08  \\
 \hline
 32  & 6.24 & 6.12 & 6.07 & 6.03  \\
 \hline
 16  & 6.12 & 6.14 & 6.04 & 6.02  \\
 \hline
 \multicolumn{5}{|c|}{Nemotron4-340B (FP32 PPL = 3.48)} \\ 
 \hline
 \hline
 64 & 3.67 & 3.62 & 3.60 & 3.59 \\
 \hline
 32 & 3.63 & 3.61 & 3.59 & 3.56 \\
 \hline
 16 & 3.61 & 3.58 & 3.57 & 3.55 \\
 \hline
\end{tabular}
\caption{\label{tab:ppl_llama7B_nemo15B} Wikitext-103 perplexity compared to FP32 baseline in Llama2-7B and Nemotron4-15B, 340B models}
\end{table}

%\subsection{Perplexity achieved by various LO-BCQ configurations on MMLU dataset}


\begin{table} \centering
\begin{tabular}{|c||c|c|c|c||c|c|c|c|} 
\hline
 $L_b \rightarrow$& \multicolumn{4}{c||}{8} & \multicolumn{4}{c||}{8}\\
 \hline
 \backslashbox{$L_A$\kern-1em}{\kern-1em$N_c$} & 2 & 4 & 8 & 16 & 2 & 4 & 8 & 16  \\
 %$N_c \rightarrow$ & 2 & 4 & 8 & 16 & 2 & 4 & 2 \\
 \hline
 \hline
 \multicolumn{5}{|c|}{Llama2-7B (FP32 Accuracy = 45.8\%)} & \multicolumn{4}{|c|}{Llama2-70B (FP32 Accuracy = 69.12\%)} \\ 
 \hline
 \hline
 64 & 43.9 & 43.4 & 43.9 & 44.9 & 68.07 & 68.27 & 68.17 & 68.75 \\
 \hline
 32 & 44.5 & 43.8 & 44.9 & 44.5 & 68.37 & 68.51 & 68.35 & 68.27  \\
 \hline
 16 & 43.9 & 42.7 & 44.9 & 45 & 68.12 & 68.77 & 68.31 & 68.59  \\
 \hline
 \hline
 \multicolumn{5}{|c|}{GPT3-22B (FP32 Accuracy = 38.75\%)} & \multicolumn{4}{|c|}{Nemotron4-15B (FP32 Accuracy = 64.3\%)} \\ 
 \hline
 \hline
 64 & 36.71 & 38.85 & 38.13 & 38.92 & 63.17 & 62.36 & 63.72 & 64.09 \\
 \hline
 32 & 37.95 & 38.69 & 39.45 & 38.34 & 64.05 & 62.30 & 63.8 & 64.33  \\
 \hline
 16 & 38.88 & 38.80 & 38.31 & 38.92 & 63.22 & 63.51 & 63.93 & 64.43  \\
 \hline
\end{tabular}
\caption{\label{tab:mmlu_abalation} Accuracy on MMLU dataset across GPT3-22B, Llama2-7B, 70B and Nemotron4-15B models.}
\end{table}


%\subsection{Perplexity achieved by various LO-BCQ configurations on LM evaluation harness}

\begin{table} \centering
\begin{tabular}{|c||c|c|c|c||c|c|c|c|} 
\hline
 $L_b \rightarrow$& \multicolumn{4}{c||}{8} & \multicolumn{4}{c||}{8}\\
 \hline
 \backslashbox{$L_A$\kern-1em}{\kern-1em$N_c$} & 2 & 4 & 8 & 16 & 2 & 4 & 8 & 16  \\
 %$N_c \rightarrow$ & 2 & 4 & 8 & 16 & 2 & 4 & 2 \\
 \hline
 \hline
 \multicolumn{5}{|c|}{Race (FP32 Accuracy = 37.51\%)} & \multicolumn{4}{|c|}{Boolq (FP32 Accuracy = 64.62\%)} \\ 
 \hline
 \hline
 64 & 36.94 & 37.13 & 36.27 & 37.13 & 63.73 & 62.26 & 63.49 & 63.36 \\
 \hline
 32 & 37.03 & 36.36 & 36.08 & 37.03 & 62.54 & 63.51 & 63.49 & 63.55  \\
 \hline
 16 & 37.03 & 37.03 & 36.46 & 37.03 & 61.1 & 63.79 & 63.58 & 63.33  \\
 \hline
 \hline
 \multicolumn{5}{|c|}{Winogrande (FP32 Accuracy = 58.01\%)} & \multicolumn{4}{|c|}{Piqa (FP32 Accuracy = 74.21\%)} \\ 
 \hline
 \hline
 64 & 58.17 & 57.22 & 57.85 & 58.33 & 73.01 & 73.07 & 73.07 & 72.80 \\
 \hline
 32 & 59.12 & 58.09 & 57.85 & 58.41 & 73.01 & 73.94 & 72.74 & 73.18  \\
 \hline
 16 & 57.93 & 58.88 & 57.93 & 58.56 & 73.94 & 72.80 & 73.01 & 73.94  \\
 \hline
\end{tabular}
\caption{\label{tab:mmlu_abalation} Accuracy on LM evaluation harness tasks on GPT3-1.3B model.}
\end{table}

\begin{table} \centering
\begin{tabular}{|c||c|c|c|c||c|c|c|c|} 
\hline
 $L_b \rightarrow$& \multicolumn{4}{c||}{8} & \multicolumn{4}{c||}{8}\\
 \hline
 \backslashbox{$L_A$\kern-1em}{\kern-1em$N_c$} & 2 & 4 & 8 & 16 & 2 & 4 & 8 & 16  \\
 %$N_c \rightarrow$ & 2 & 4 & 8 & 16 & 2 & 4 & 2 \\
 \hline
 \hline
 \multicolumn{5}{|c|}{Race (FP32 Accuracy = 41.34\%)} & \multicolumn{4}{|c|}{Boolq (FP32 Accuracy = 68.32\%)} \\ 
 \hline
 \hline
 64 & 40.48 & 40.10 & 39.43 & 39.90 & 69.20 & 68.41 & 69.45 & 68.56 \\
 \hline
 32 & 39.52 & 39.52 & 40.77 & 39.62 & 68.32 & 67.43 & 68.17 & 69.30  \\
 \hline
 16 & 39.81 & 39.71 & 39.90 & 40.38 & 68.10 & 66.33 & 69.51 & 69.42  \\
 \hline
 \hline
 \multicolumn{5}{|c|}{Winogrande (FP32 Accuracy = 67.88\%)} & \multicolumn{4}{|c|}{Piqa (FP32 Accuracy = 78.78\%)} \\ 
 \hline
 \hline
 64 & 66.85 & 66.61 & 67.72 & 67.88 & 77.31 & 77.42 & 77.75 & 77.64 \\
 \hline
 32 & 67.25 & 67.72 & 67.72 & 67.00 & 77.31 & 77.04 & 77.80 & 77.37  \\
 \hline
 16 & 68.11 & 68.90 & 67.88 & 67.48 & 77.37 & 78.13 & 78.13 & 77.69  \\
 \hline
\end{tabular}
\caption{\label{tab:mmlu_abalation} Accuracy on LM evaluation harness tasks on GPT3-8B model.}
\end{table}

\begin{table} \centering
\begin{tabular}{|c||c|c|c|c||c|c|c|c|} 
\hline
 $L_b \rightarrow$& \multicolumn{4}{c||}{8} & \multicolumn{4}{c||}{8}\\
 \hline
 \backslashbox{$L_A$\kern-1em}{\kern-1em$N_c$} & 2 & 4 & 8 & 16 & 2 & 4 & 8 & 16  \\
 %$N_c \rightarrow$ & 2 & 4 & 8 & 16 & 2 & 4 & 2 \\
 \hline
 \hline
 \multicolumn{5}{|c|}{Race (FP32 Accuracy = 40.67\%)} & \multicolumn{4}{|c|}{Boolq (FP32 Accuracy = 76.54\%)} \\ 
 \hline
 \hline
 64 & 40.48 & 40.10 & 39.43 & 39.90 & 75.41 & 75.11 & 77.09 & 75.66 \\
 \hline
 32 & 39.52 & 39.52 & 40.77 & 39.62 & 76.02 & 76.02 & 75.96 & 75.35  \\
 \hline
 16 & 39.81 & 39.71 & 39.90 & 40.38 & 75.05 & 73.82 & 75.72 & 76.09  \\
 \hline
 \hline
 \multicolumn{5}{|c|}{Winogrande (FP32 Accuracy = 70.64\%)} & \multicolumn{4}{|c|}{Piqa (FP32 Accuracy = 79.16\%)} \\ 
 \hline
 \hline
 64 & 69.14 & 70.17 & 70.17 & 70.56 & 78.24 & 79.00 & 78.62 & 78.73 \\
 \hline
 32 & 70.96 & 69.69 & 71.27 & 69.30 & 78.56 & 79.49 & 79.16 & 78.89  \\
 \hline
 16 & 71.03 & 69.53 & 69.69 & 70.40 & 78.13 & 79.16 & 79.00 & 79.00  \\
 \hline
\end{tabular}
\caption{\label{tab:mmlu_abalation} Accuracy on LM evaluation harness tasks on GPT3-22B model.}
\end{table}

\begin{table} \centering
\begin{tabular}{|c||c|c|c|c||c|c|c|c|} 
\hline
 $L_b \rightarrow$& \multicolumn{4}{c||}{8} & \multicolumn{4}{c||}{8}\\
 \hline
 \backslashbox{$L_A$\kern-1em}{\kern-1em$N_c$} & 2 & 4 & 8 & 16 & 2 & 4 & 8 & 16  \\
 %$N_c \rightarrow$ & 2 & 4 & 8 & 16 & 2 & 4 & 2 \\
 \hline
 \hline
 \multicolumn{5}{|c|}{Race (FP32 Accuracy = 44.4\%)} & \multicolumn{4}{|c|}{Boolq (FP32 Accuracy = 79.29\%)} \\ 
 \hline
 \hline
 64 & 42.49 & 42.51 & 42.58 & 43.45 & 77.58 & 77.37 & 77.43 & 78.1 \\
 \hline
 32 & 43.35 & 42.49 & 43.64 & 43.73 & 77.86 & 75.32 & 77.28 & 77.86  \\
 \hline
 16 & 44.21 & 44.21 & 43.64 & 42.97 & 78.65 & 77 & 76.94 & 77.98  \\
 \hline
 \hline
 \multicolumn{5}{|c|}{Winogrande (FP32 Accuracy = 69.38\%)} & \multicolumn{4}{|c|}{Piqa (FP32 Accuracy = 78.07\%)} \\ 
 \hline
 \hline
 64 & 68.9 & 68.43 & 69.77 & 68.19 & 77.09 & 76.82 & 77.09 & 77.86 \\
 \hline
 32 & 69.38 & 68.51 & 68.82 & 68.90 & 78.07 & 76.71 & 78.07 & 77.86  \\
 \hline
 16 & 69.53 & 67.09 & 69.38 & 68.90 & 77.37 & 77.8 & 77.91 & 77.69  \\
 \hline
\end{tabular}
\caption{\label{tab:mmlu_abalation} Accuracy on LM evaluation harness tasks on Llama2-7B model.}
\end{table}

\begin{table} \centering
\begin{tabular}{|c||c|c|c|c||c|c|c|c|} 
\hline
 $L_b \rightarrow$& \multicolumn{4}{c||}{8} & \multicolumn{4}{c||}{8}\\
 \hline
 \backslashbox{$L_A$\kern-1em}{\kern-1em$N_c$} & 2 & 4 & 8 & 16 & 2 & 4 & 8 & 16  \\
 %$N_c \rightarrow$ & 2 & 4 & 8 & 16 & 2 & 4 & 2 \\
 \hline
 \hline
 \multicolumn{5}{|c|}{Race (FP32 Accuracy = 48.8\%)} & \multicolumn{4}{|c|}{Boolq (FP32 Accuracy = 85.23\%)} \\ 
 \hline
 \hline
 64 & 49.00 & 49.00 & 49.28 & 48.71 & 82.82 & 84.28 & 84.03 & 84.25 \\
 \hline
 32 & 49.57 & 48.52 & 48.33 & 49.28 & 83.85 & 84.46 & 84.31 & 84.93  \\
 \hline
 16 & 49.85 & 49.09 & 49.28 & 48.99 & 85.11 & 84.46 & 84.61 & 83.94  \\
 \hline
 \hline
 \multicolumn{5}{|c|}{Winogrande (FP32 Accuracy = 79.95\%)} & \multicolumn{4}{|c|}{Piqa (FP32 Accuracy = 81.56\%)} \\ 
 \hline
 \hline
 64 & 78.77 & 78.45 & 78.37 & 79.16 & 81.45 & 80.69 & 81.45 & 81.5 \\
 \hline
 32 & 78.45 & 79.01 & 78.69 & 80.66 & 81.56 & 80.58 & 81.18 & 81.34  \\
 \hline
 16 & 79.95 & 79.56 & 79.79 & 79.72 & 81.28 & 81.66 & 81.28 & 80.96  \\
 \hline
\end{tabular}
\caption{\label{tab:mmlu_abalation} Accuracy on LM evaluation harness tasks on Llama2-70B model.}
\end{table}

%\section{MSE Studies}
%\textcolor{red}{TODO}


\subsection{Number Formats and Quantization Method}
\label{subsec:numFormats_quantMethod}
\subsubsection{Integer Format}
An $n$-bit signed integer (INT) is typically represented with a 2s-complement format \citep{yao2022zeroquant,xiao2023smoothquant,dai2021vsq}, where the most significant bit denotes the sign.

\subsubsection{Floating Point Format}
An $n$-bit signed floating point (FP) number $x$ comprises of a 1-bit sign ($x_{\mathrm{sign}}$), $B_m$-bit mantissa ($x_{\mathrm{mant}}$) and $B_e$-bit exponent ($x_{\mathrm{exp}}$) such that $B_m+B_e=n-1$. The associated constant exponent bias ($E_{\mathrm{bias}}$) is computed as $(2^{{B_e}-1}-1)$. We denote this format as $E_{B_e}M_{B_m}$.  

\subsubsection{Quantization Scheme}
\label{subsec:quant_method}
A quantization scheme dictates how a given unquantized tensor is converted to its quantized representation. We consider FP formats for the purpose of illustration. Given an unquantized tensor $\bm{X}$ and an FP format $E_{B_e}M_{B_m}$, we first, we compute the quantization scale factor $s_X$ that maps the maximum absolute value of $\bm{X}$ to the maximum quantization level of the $E_{B_e}M_{B_m}$ format as follows:
\begin{align}
\label{eq:sf}
    s_X = \frac{\mathrm{max}(|\bm{X}|)}{\mathrm{max}(E_{B_e}M_{B_m})}
\end{align}
In the above equation, $|\cdot|$ denotes the absolute value function.

Next, we scale $\bm{X}$ by $s_X$ and quantize it to $\hat{\bm{X}}$ by rounding it to the nearest quantization level of $E_{B_e}M_{B_m}$ as:

\begin{align}
\label{eq:tensor_quant}
    \hat{\bm{X}} = \text{round-to-nearest}\left(\frac{\bm{X}}{s_X}, E_{B_e}M_{B_m}\right)
\end{align}

We perform dynamic max-scaled quantization \citep{wu2020integer}, where the scale factor $s$ for activations is dynamically computed during runtime.

\subsection{Vector Scaled Quantization}
\begin{wrapfigure}{r}{0.35\linewidth}
  \centering
  \includegraphics[width=\linewidth]{sections/figures/vsquant.jpg}
  \caption{\small Vectorwise decomposition for per-vector scaled quantization (VSQ \citep{dai2021vsq}).}
  \label{fig:vsquant}
\end{wrapfigure}
During VSQ \citep{dai2021vsq}, the operand tensors are decomposed into 1D vectors in a hardware friendly manner as shown in Figure \ref{fig:vsquant}. Since the decomposed tensors are used as operands in matrix multiplications during inference, it is beneficial to perform this decomposition along the reduction dimension of the multiplication. The vectorwise quantization is performed similar to tensorwise quantization described in Equations \ref{eq:sf} and \ref{eq:tensor_quant}, where a scale factor $s_v$ is required for each vector $\bm{v}$ that maps the maximum absolute value of that vector to the maximum quantization level. While smaller vector lengths can lead to larger accuracy gains, the associated memory and computational overheads due to the per-vector scale factors increases. To alleviate these overheads, VSQ \citep{dai2021vsq} proposed a second level quantization of the per-vector scale factors to unsigned integers, while MX \citep{rouhani2023shared} quantizes them to integer powers of 2 (denoted as $2^{INT}$).

\subsubsection{MX Format}
The MX format proposed in \citep{rouhani2023microscaling} introduces the concept of sub-block shifting. For every two scalar elements of $b$-bits each, there is a shared exponent bit. The value of this exponent bit is determined through an empirical analysis that targets minimizing quantization MSE. We note that the FP format $E_{1}M_{b}$ is strictly better than MX from an accuracy perspective since it allocates a dedicated exponent bit to each scalar as opposed to sharing it across two scalars. Therefore, we conservatively bound the accuracy of a $b+2$-bit signed MX format with that of a $E_{1}M_{b}$ format in our comparisons. For instance, we use E1M2 format as a proxy for MX4.

\begin{figure}
    \centering
    \includegraphics[width=1\linewidth]{sections//figures/BlockFormats.pdf}
    \caption{\small Comparing LO-BCQ to MX format.}
    \label{fig:block_formats}
\end{figure}

Figure \ref{fig:block_formats} compares our $4$-bit LO-BCQ block format to MX \citep{rouhani2023microscaling}. As shown, both LO-BCQ and MX decompose a given operand tensor into block arrays and each block array into blocks. Similar to MX, we find that per-block quantization ($L_b < L_A$) leads to better accuracy due to increased flexibility. While MX achieves this through per-block $1$-bit micro-scales, we associate a dedicated codebook to each block through a per-block codebook selector. Further, MX quantizes the per-block array scale-factor to E8M0 format without per-tensor scaling. In contrast during LO-BCQ, we find that per-tensor scaling combined with quantization of per-block array scale-factor to E4M3 format results in superior inference accuracy across models. 









\end{document}


% This document was modified from the file originally made available by
% Pat Langley and Andrea Danyluk for ICML-2K. This version was created
% by Iain Murray in 2018, and modified by Alexandre Bouchard in
% 2019 and 2021 and by Csaba Szepesvari, Gang Niu and Sivan Sabato in 2022.
% Modified again in 2023 and 2024 by Sivan Sabato and Jonathan Scarlett.
% Previous contributors include Dan Roy, Lise Getoor and Tobias
% Scheffer, which was slightly modified from the 2010 version by
% Thorsten Joachims & Johannes Fuernkranz, slightly modified from the
% 2009 version by Kiri Wagstaff and Sam Roweis's 2008 version, which is
% slightly modified from Prasad Tadepalli's 2007 version which is a
% lightly changed version of the previous year's version by Andrew
% Moore, which was in turn edited from those of Kristian Kersting and
% Codrina Lauth. Alex Smola contributed to the algorithmic style files.
