\newpage
\appendix
\onecolumn


\section{Riemannian Geometry}
\label{appendix:diffgeom}

This appendix covers the necessary concepts and results from differential geometry. For more full treatment, the reader is referred to excellent sources such as the book by \citet{lee2003introduction}.

The notation in this appendix is self-contained and separated from the main sections of the paper. Thus, the symbols that were used in the main body can be reused in the following but denote a different concept. The notation is introduced as we go along and should not lead to confusion. 

To gradually build up the necessary constructions bottom-up, we start from the concept of a topology that we will morph into a Riemannian manifold by progressively adding structure.

\subsection{Topological concepts}

A topology $\mathcal{T}$ on a set $X$ is a collection of subsets of $X$, that defines which sets are open in $X$. This gives rise to the notion of a neighborhood of a point $p\in X$ for an abstract set $X$. 

\begin{definition}\label{def:topology}
    
A topology $\mathcal{T}$ on $X$ is a collection of subsets of $\mathcal{P}(X)$ satisfying the following axioms:

\begin{enumerate}
    \item $X$ and $\varnothing$ are in $\mathcal{T}$.
    \item The union of any family of subsets in $\mathcal{T}$ are in $\mathcal{T}$.
    \item The intersection of any finite family of subsets in $\mathcal{T}$ are in $\mathcal{T}$.
\end{enumerate}
The tuple $(X, \mathcal{T})$ is called a topological space, and the elements of $\mathcal{T}$ are called open sets.
\end{definition}

To construct a topology, we need a basis.

\begin{definition}\label{def:basis}
    A basis for a topology $\mathcal{T}$ on a set $X$ is a collection $\mathcal{B}$ of subsets of $X$ such that:
    \begin{enumerate}
        \item For each $x\in X$, there is at least one $B\in \mathcal{B}$ such that $x\in B$.
        \item If $x\in B_1\cap B_2$ for $B_1, B_2 \in \mathcal{B}$, then there is a $B_3\in \mathcal{B}$ such that $x\in B_3\subseteq B_1\cap B_2$.
    \end{enumerate}
\end{definition}



In cases where we work with subsets of $X$, we can use the topology of $X$ to define a topology on the subsets.

\begin{definition}[Subspace Topology]\label{def:subspace_topology}
    Let $X$ be a topological space with topology $\mathcal{T}$ and $Y\subseteq X$. The subspace topology on $Y$ is defined as $\mathcal{T}_Y=\{Y\cap U|U\in \mathcal{T}\}$.
\end{definition}

The notion of a topological leads to the construction of topological manifold that is the prerequisite to a Riemannian manifold. 


\begin{definition} \label{def:top_manifold}
    Suppose $M$ is a topological space. We say that $M$ is a topological manifold of dimension $n$ or a topological $n$-manifold if it has the following properties:

\begin{enumerate}
    \item $M$ is a Hausdorff space: for every pair of distinct points $p, q \in M$, there are disjoint open subsets $U, V \subseteq M$ such that $p \in U$ and $q \in V$.
    \item  $M$ is second-countable: there exists a countable basis for the topology of $M$.
    \item  $M$ is locally Euclidean of dimension $n$: each point of $M$ has a neighborhood that is homeomorphic to an open subset of $\mathbb{R}^n$.
\end{enumerate}
\end{definition}

Definition~\ref{def:top_manifold} specifies what we mean by a homeomorphism.

\begin{definition}\label{def:homeomorphism}
    Let $X$ and $Y$ be topological spaces, a map $F:X\rightarrow Y$ is 
    \begin{itemize}
        \item continuous if the preimage of every open set in $Y$ is open in $X$. $$\forall V\subseteq Y \text{ open } \Rightarrow F^{-1}(V)\subseteq X \text{ open}$$
        \item injective if $F(x)=F(y)$ implies $x=y$.
        \item surjective if for every $y\in Y$ there is an $x\in X$ such that $F(x)=y$.
        \item bijective if it is both injective and surjective.
        \item homeomorphism if it is bijective and both $F$ and $F^{-1}$ are continuous.
    \end{itemize}
    If $F$ is a homeomorphism, then $X$ and $Y$ are called homeomorphic. If $F$, on the other hand, is not bijective but only injective, we call $F$ a topological embedding. 
\end{definition}



\begin{definition}\label{def:top_embedding}
    Let $X$ and $Y$ be topological spaces; a continuous injective map $F:X\rightarrow Y$ is called a topological embedding if it is a homeomorphism onto its image $F(X) \subseteq Y$ in the subspace topology.
\end{definition}

The notion of local homeomorphism defined in the following is fundamental to the construction of a Riemannian manifold.

\begin{definition}\label{def:loc_homeomorphism}
   For two topological spaces $X$ and $Y$, a continuous map $F:X\rightarrow Y$ is called a local homeomorphism if every point $p\in X$ has a neighborhood $U\subseteq X$ such that $F(U)$ is open in $Y$ and $F$ restricts to a homeomorphism from $U$ to $F(U)$.
\end{definition}

Definition~\ref{def:top_manifold} requires the existence of a local homeomorphism from $M$ to $\mathbb{R}^n$ for every point $p \in M$ defined on a neighborhood $U$ of $p$. Each such local homeomorphism with the corresponding restriction $U \in M$ is called a coordinate chart. 

\begin{definition} \label{def:coordinate_chart}
    A coordinate chart on $M$ is a pair $(U,\phi)$ where $U$ is an open subset of M and $\phi: U \rightarrow \hat{U}$ is a homeomorphism from $U$ to an open subset $\hat{U}=\phi(U)\subseteq \mathbb{R}^n$.
\end{definition}



It follows from the definition of a topological manifold $M$ that every point $p \in M$ lies in the domain of some chart. $U$ is called coordinate domain and $\phi$ local coordinate map. The foundation of a smooth manifold is a maximal smooth atlas containing coordinate charts that are smoothly compatible. 

\begin{definition}\label{def:smooth_charts}
Let $M$ be a topological $n$-manifold. Two charts $(U,\phi)$ and $(V,\psi)$ are called smoothly compatible if $U\cap V = \varnothing$ or their transition map $$\psi \circ \phi^{-1}:\phi(U\cap V)\rightarrow \psi (U\cap V)$$
is a diffeomorphism in case $U\cap V \neq \varnothing$.
\end{definition}

In Definition~\ref{def:smooth_charts}, note how the smoothness of the transition map can be analyzed in terms of the smoothness of maps between open subsets of $\mathbb{R}^n$, namely the domains of the associated coordinate charts. 

\begin{definition}
    A diffeomorphism between two open subsets $U$ and $V$ of $\mathbb{R}^n$ is a bijective map $F:U\rightarrow V$ such that both $F$ and $F^{-1}$ are continuous and differentiable.
\end{definition}

\subsection{Smooth manifolds}

We are now in a position to define a smooth manifold.

\begin{definition}\label{def:smooth_atlas}
Let $M$ be a topological manifold, 
\begin{itemize}
    \item an atlas $\mathcal{A}$ is a collection of charts whose domains cover $M$
    \item a smooth atlas is an atlas $\mathcal{A}$ such that any two charts in $\mathcal{A}$ are smoothly compatible
    \item a maximal smooth atlas is a smooth atlas that is not properly contained in any larger smooth atlas
    \item $M$ together with a maximal smooth atlas $\mathcal{A}$ is called a smooth manifold denoted by $(M,\mathcal{A})$
\end{itemize}
\end{definition}
 In the following, we define smoothness for a map between two smooth manifolds.

\begin{definition} \label{def:smooth_map}
    Let $M, N$ be smooth manifolds, and let $F: M \rightarrow N$ be any map. We say that $F$ is a smooth map if for every $p \in M$, there exist smooth charts $(U, \varphi)$ containing $p$ and $(V, \psi)$ containing $F(p)$ such that $F(U) \subseteq V$ and the composite map $\psi \circ F \circ$ $\varphi^{-1}$ is smooth from $\varphi(U)$ to $\psi(V)$. This means that $\psi \circ F \circ$ $\varphi^{-1}$ is a map between subsets of $\mathbb{R}^n$ and $\mathbb{R}^n$ and we can apply the usual real calculus.
\end{definition}

Since a manifold does not have the usual operations of the Euclidean vector space, one way to construct a tangent vector space to a manifold $M$ at a point $p \in M$ is to define it in terms of a tangent vector to some curve $\gamma: I\rightarrow M$.

\begin{definition}\label{def:tangent_vector}
    Let $M$ be a smooth manifold, and let $\gamma_1, \gamma_2:(-\epsilon,\epsilon)\rightarrow M$ be  smooth curves in $M$. 
    Suppose that $\gamma_1(0)=\gamma_2(0)=p\in M$, then $\gamma_1$ and $\gamma_2$ are said to be equivalent if the following holds:
    $$\left.\frac{d(\varphi \circ \gamma_1)}{d t}\right|_{t=0}=\left.\frac{d(\varphi \circ \gamma_2)}{d t}\right|_{t=0}$$
    This defines an equivalence relation on the set of all smooth curves through $p$, and the equivalence classes are called tangent vectors of $M$ at $p$. The tangent space $T_pM$ to $M$ at $p$ is then defined as the set of all tangent vectors at $p$ and does not depend on the choice of the coordinate chart $\varphi$.
\end{definition}

A vector space structure on the tangent space $T_pM$ is defined by using the coordinate charts that map between subsets of $\mathbb{R}^n$ and allow us to do vector addition and scalar multiplication. \citet{lee2003introduction} shows that the resulting construction is independent of the choice of the charts and that $T_pM$ is an $n$-dimensional real vector space. 

Considering maps between manifolds, we want to link the tangent space of one with the tangent space of the other. This is done by defining the differential $\mathrm{d}F$ of $F$ at a point $p$, which is a linear mapping from one manifold's tangent space to another's. 

\begin{definition}\label{def:diff_of_F}
    Let $M$ and $N$ be smooth manifolds and $F: M \rightarrow N$ be a smooth map. For each $p \in M$ we define a map:
    $$\mathrm{d}F_p:T_pM \rightarrow T_{F(p)}N$$
    called the differential of $F$ at $p$, which is a linear map between the tangent spaces. The following property defines the differential:
    $$\mathrm{d}F_p(v)=\left.\frac{d(\varphi \circ \gamma)}{d t}\right|_{t=0}$$
    where $\gamma$ is smooth a curve in $M$ through $p$ with $\gamma^{\prime}(0)=v$ and $\varphi$ is a coordinate chart around $p$. This construction is independent of the choice of the chart $\varphi$ as shown in \cite{lee2003introduction}.
\end{definition}

The differential allows us to assess the rank of a map between two manifolds.

\begin{definition}\label{def:constant_rank_map}
    Given two smooth manifolds $M$ and $N$ a map $F:M\rightarrow N$ has constant rank $r$ at $p \in M$ if the linear map $dF_p:T_pM \rightarrow T_{F(p)}N$ has rank $r$. $F$ is called a smooth submersion if its differential is surjective at each point (rank $F=\operatorname{dim} N$). It is called a smooth immersion if its differential is injective at each point (rank $F=\operatorname{dim} M$)
\end{definition}

To define new submanifolds as images of maps, we need the concept of a smooth embedding.

\begin{definition}\label{def:smooth_embedding}
    Let $M$ and $N$ be smooth
manifolds, a smooth embedding of $M$ into $N$ is a smooth
immersion $F:M\rightarrow N$ that is also a topological embedding, i.e., a homeomorphism
onto its image $F(M)\subseteq N$ in the subspace topology.
\end{definition}

 Theorem~\ref{thrm:immersion_is_embedding} tells us when an injective smooth immersion is also a smooth embedding. 

\begin{theorem} \label{thrm:immersion_is_embedding}
    (Proposition 4.22 in \cite{lee2003introduction}) Let $M$ and $N$ be smooth manifolds, and $F:M\rightarrow N$ is an injective smooth immersion. If any of the following
holds, then $F$ is a smooth embedding.
\begin{itemize}
    \item $F$ is an open or closed map.
    \item $F$ is a proper map.
    \item $M$ is compact.
    \item $M$ has empty boundary and $\operatorname{dim}M=\operatorname{dim}N$
\end{itemize}
\end{theorem}

Theorem~\ref{thrm:embedding_to_mnfld} tells us that images of smooth embeddings are submanifolds with smooth properties. 

\begin{theorem}\label{thrm:embedding_to_mnfld}
    (Proposition 5.2 in \cite{lee2003introduction}) Suppose $M$ and $N$ are smooth manifolds and $F:M\rightarrow N$ is a smooth embedding. Let $S = F(M)$. With the subspace topology,
$S$ is a topological manifold, and it has a unique smooth structure, making it into an
embedded submanifold of $N$ with the property that $F$ is a diffeomorphism onto its
image.
\end{theorem}


\subsection{Riemannian manifolds}
Proposition 13.3 of \cite{lee2003introduction} proves the existence of a Riemannian metric $g$ in any smooth manifold $N$, where $g$ is a smooth, symmetric covariant 2-tensor field on $M$ that is positive definite at each point $p \in N$ and defines an inner product in a tangent space $T_pN$. The tuple $(N,g)$ is called a Riemannian manifold.

Often, in modeling situations, we think of Theorem~\ref{thrm:embedding_to_mnfld}, that is, we imagine an embedded submanifold $S\subseteq N$ in some ambient space $N$. In this case, we say that $F:M\rightarrow N$ is a parametrization of $S$. If there is a Riemannian metric $g^{N}$ on $N$, there is a way to measure lengths of vectors in a tangent space to a point on $S$ using $g^{N}$ as it is embedded in this larger vector space that has a metric. We can use this to construct a metric on $S$ by pulling  $g^{N}$ by $F$. 


    \begin{definition}
    \label{def:appendix:pullback_metric}
    For a map $F: M \rightarrow F(M)=S\subset N$ between manifolds $M$ and $S$, and a metric $g^{N}$ on $N$, the pullback metric $f^*g^{N}$ on $M$ is defined as:
\begin{equation}
\label{eq:pullback:appendix}
    (F^*g^{N})_p(u, v) = g^{N}_{F(p)}(\mathrm{d}F_p(u), \mathrm{d}F_p(v))
\end{equation}
for any tangent vectors $u, v \in T_pM$.
\end{definition}

Given two Riemannian manifolds, we can check if they are isometric by using the pullback metric.

\begin{definition}
   Given Riemannian manifolds $(M,g^M)$ and $(N,g^N)$, a smooth map $F:M\rightarrow N$ is called an isometry if $F$ is a diffeomorphism such that:
   \begin{equation}
       \label{eq:isometry_def}
       F^*g^N=g^M 
   \end{equation}
   In which case we say $(M,g^M)$ and $(N,g^N)$ are isometric
\end{definition}

A series of results in \cite{lee2003introduction} Chapters 13,15,16 show that if two manifolds are isometric through a diffeomorphism $F$, then $F$ preserves lengths of curves, distances, angles, volumes, and other geometric properties between manifolds. 

 \begin{definition}\label{def:metric_enables}
 Given a Riemannian manifold $(M,g)$ we can define the following:
     \begin{itemize}
    \item length or norm of a tangent vector $v \in T_p M$ is defined to be
$$
|v|_g=\langle v, v\rangle_g^{1 / 2}=g_p(v, v)^{1 / 2} .
$$
\item angle between two nonzero tangent vectors $v, w \in T_p M$ is the unique $\theta \in$ $[0, \pi]$ satisfying
$$
\cos \theta=\frac{\langle v, w\rangle_g}{|v|_g|w|_g} .
$$
\item tangent vectors $v, w \in T_p M$ are said to be orthogonal if $\langle v, w\rangle_g=0$. This means either one or both vectors are zero, or the angle between them is $\pi / 2$.
\item given a smooth curve $\gamma:[a,b]\rightarrow M$ we can define the length of $\gamma$ to be:
$$\mathrm{L}_g(\gamma)=\int_a^b\left|\gamma^{\prime}(t)\right|_g d t$$
and  the energy of $\gamma$ to be:
$$\mathrm{E}_g(\gamma)=\frac12\int_a^b\left|\gamma^{\prime}(t)\right|_g^2 d t$$
\end{itemize}
 \end{definition}

Proposition 13.25 of \citet{lee2003introduction} shows that given a curve $\gamma:[a,b]\rightarrow M$ and a reparametrization $u:[c,d]\rightarrow [a,b]$ that is a diffeomorphism, the length of the curve $\tilde{\gamma} = \gamma \circ u$, $\mathrm{L}_g(\tilde{\gamma})$ is equal to $\mathrm{L}_g(\gamma)$.

The notion of the length of a curve given in Definition~\ref{def:metric_enables} allows us to consider the Riemannian distance from $p$ to $q$ ($p,q \in M$) denoted by $d_g(p,q)$ and defined to be the infimum of $\mathrm{L}_g$ over all piecewise smooth curve segments from $p$ to $q$. A shortest curve is not unique since $\mathrm{L}_g(\gamma)$ is reparametrization invariant, and a set of curves is locally minimizing  $\mathrm{L}_g(\gamma)$. One useful parametrization is by arc-length $s$, which ensures that we move along the curve at a constant speed.

\begin{definition}[Arc-length parametrization]
    A curve $\gamma:[a,b]\rightarrow M$ is said to be parametrized by arc-length if the length of the curve between any two points $t_1$ and $t_2$ is equal to the difference in the parameter values $t_2-t_1$. Formally, $\gamma$ is parametrized by arc-length if:
    $$\left|\gamma^{\prime}(t)\right|_g=1$$
    for all $t\in [a,b]$.
\end{definition}

Curves $\gamma$ that are locally minimizing $\mathrm{L}_g(\gamma)$ and are parametrized by arc-length are called geodesics defined in Definition~\ref{def:geodesics}.

\begin{definition}\label{def:geodesics}
Given a Riemannian manifold $(M,g)$ and $x,y \in M$ with $x\neq y$, a geodesic curve on between $x$ and $y$ on $M$ is formally defined as a curve 
$\gamma:I\rightarrow M$ that locally minimizes the energy functional 
$$\mathrm{E}_g(\gamma)=\frac{1}{2}\int_a^b\left|\gamma^{\prime}(t)\right|_g^2 d t$$
over all smooth curves 
$\gamma:[a,b]\rightarrow M$ connecting two given points 

$\gamma(a)=x$ and $\gamma(b)=y$, where 
$g$ is the Riemannian metric tensor on $M$.
\end{definition}

As discussed by, e.g., \citet{DG_SH} it is a standard result that a minimizer of the energy functional will necessarily be arc-length parametrized and minimize the length functional.

%The notion of the Riemannian distance function allows us to treat $(M,g)$ a metric space defined in Definition~\ref{def:metric_space}. 

%\begin{definition}\label{def:metric_space}
%    A metric space is a set $M$ together with a distance function $d:M\times M\rightarrow \mathbb{R}$ that satisfies the following properties:
%    \begin{itemize}
%        \item $d(p,q)\geq 0$ for all $p,q\in M$ and $d(p,q)=0$ if and only if $p=q$
%        \item $d(p,q)=d(q,p)$ for all $p,q\in M$
%        \item $d(p,q)\leq d(p,r)+d(r,q)$ for all $p,q,r\in M$
%    \end{itemize}
%\end{definition}


%This result is the foundation of our result in the next chapter, and we make it explicit in Theorem~\ref{thrm:metric_space}.

%\begin{theorem}\label{thrm:metric_space}
%(Theorem 13.29 in \cite{lee2003introduction})
%    Let $(M,g)$ be a connected Riemannian manifold. Let the Riemannian distance function be defined as:
%    $$d_g(p,q)=\mathop{\text{inf}}_{\gamma}\mathrm{L}_g(\gamma)=\mathop{\text{inf}}_{\gamma}\int_a^b\left|\gamma^{\prime}(t)\right|_g d t$$
  %  where $\gamma:[a,b]\rightarrow M$ is a smooth curve satisfying $\gamma(a)=p$ and $\gamma(b)=q$. Then, $M$ is a metric space whose metric topology is the same as the original manifold topology.
%\end{theorem}


We can consider the Fr{\'e}chet mean and variance of a set of points on a manifold. 


\begin{definition}\label{def:frechet_variance}
    Let $(M,d_g)$ be a Riemannian metric space and let $\left\{ x_1 \dots x_N \right\} \in M$ be points on the manifold. For any point $p \in M$, Fr{\'e}chet variance is defined to be:
    $$ \Psi (p)=\sum _{i=1}^{N}d_g^{2}\left(p,x_{i}\right)$$
    Karcher means are the points $m\in M$ that locally minimize $\Psi$:
     $$m=\mathop {\text{arg min}} _{p\in M}\sum _{i=1}^{N}d_g^{2}\left(p,x_{i}\right)$$
    If there exists a unique $m \in M$ that globally minimizes $\Psi$, then it is a Fr{\'e}chet mean.
\end{definition} 

\newpage


\section{Proofs}
\label{appendix:proofs}


\begin{theorem}(Lemma~\ref{thrm:f_is_embedding_main} in the main text)\label{thrm:f_is_embedding}

       Let $\mathcal{Z}$ and $\mathcal{D}$ be two smooth manifolds and $f \in \mathcal{F}$, then $f$ is a smooth embedding and $f(\mathcal{Z})\subset \mathcal{D}$ is a submanifold in $\mathcal{D}$. In particular, $f:\mathcal{Z}\rightarrow f(\mathcal{Z})$ is a diffeomorphism.
\end{theorem}

\begin{proof}
    By Definition~\ref{def:constant_rank_map} $f$ is a smooth immersion as it is a smooth map of constant rank with injective differential (assumptions~\ref{ass:2}-\ref{ass:3}). Using Theorem~\ref{thrm:immersion_is_embedding} with the fact that $\mathcal{Z}$ is a compact set (assumption~\ref{ass:1}) gives us that $f$ is a smooth embedding. Finally, Theorem~\ref{thrm:embedding_to_mnfld} gives us that $f(\mathcal{Z})$ is a submanifold of $\mathcal{D}$ and $f$ is a diffeomorphism on its image. 
\end{proof} 


\begin{theorem}(Lemma~\ref{thrm:gen_transf_is_diffeo_main} in the main text) 
\label{thrm:gen_transf_is_diffeo}
    Let $f_a,f_b \in \mathcal{F}$ and consider the generator transform $A_{a, b}:\mathcal{Z}_a\rightarrow \mathcal{Z}_b$ defined by $$A_{a, b}(z) = f_b^{-1}\circ f_a(z)$$ Then $A_{a, b}(z)$ and $A^{-1}_{a, b}(z)=A_{b, a}(z)=f_a^{-1}\circ f_b(z)$ are diffeomorphisms.
\end{theorem}

\begin{proof}
    The result follows from Theorem~\ref{thrm:f_is_embedding} with the fact that $f_a,f_b \in \mathcal{F}$ have the same image due to assumption~\ref{ass:4}.
\end{proof}


\begin{theorem}(Theorem~\ref{thrm:gentransforms_isometries} in the main text)

   Let $\theta_a=\left(f_a, P_{Z_a}\right)$ and $\theta_b=\left(f_b, P_{Z_b}\right)$ with $P_{\theta_a}=P_{\theta_b}$and let $(\mathcal{Z}_a, g^{f_a})$ and $(\mathcal{Z}_b,g^{f_b})$ be the associated Riemannian manifolds, then the generator transform is an isometry and it holds that:
    \begin{equation}
    \label{eq:isometry:appendix}
        \left(A_{a, b}\right)^* g^{f_b}=g^{f_a}
    \end{equation}
    Thus, making $(\mathcal{Z}_a,g^{f_a})$ and $(\mathcal{Z}_b,g^{f_b})$ isometric. This makes Riemannian geometric properties such as lengths of curves, angles, volumes, areas, Ricci curvature tensor, geodesics, parallel transport, and the exponential map identifiable. 
\end{theorem}

\begin{proof}
    We first show that the generator transform is an isometry. Let us recall that by definition of the pullback we have:
    $$g^{f_b}_p(u, v) = g^{\mathcal{D}}_{f_b(p)}(\mathrm{d}f_{b,p}(u), \mathrm{d}f_{b,p}(v))$$
    for $u,v \in T_p\mathcal{Z}_b$ and where $\mathrm{d}f_{b,p}(v)$ denotes the differential of the map $f_b$ at a point $p \in \mathcal{Z}_b$ evaluated on the vector $v \in T_p\mathcal{Z}_b$.
    Using this, we will check Eq.~\ref{eq:isometry:appendix} directly:
    \begin{equation}
        \label{eq:pullback_of_pullback}
        \begin{aligned}
            \left((A_{a,b})^* g^{f_b}\right)_p(y,w)&=g^{\mathcal{D}}_{f_b\circ f_b^{-1}\circ f_a(p)} \left(  \mathrm{d} f_{b,f_b^{-1}\circ f_a(p)}(y),\mathrm{d}f_{b,f_b^{-1}\circ f_a(p)}(w)\right)\\
            &= g^{\mathcal{D}}_{ f_a(p)}\left( 
            \mathrm{d}(f_b\circ f_b^{-1}\circ f_a)_p(y),
            \mathrm{d}(f_b\circ f_b^{-1}\circ f_a)_p(w)\right)\\
            &= g^{\mathcal{D}}_{ f_a(p)}\left( 
            \mathrm{d}f_{a,p}(y),
           \mathrm{d}f_{a,p}(w)\right)\\
           &= g^{f_a}
        \end{aligned}
    \end{equation}
    for $y,w \in T_p\mathcal{Z}_a$.

    Since we have shown that the generator transform is an isometry, we can conclude that $(\mathcal{Z}_a,g^{f_a})$ and $(\mathcal{Z}_b,g^{f_b})$ are isometric and thus their Riemannian metric properties are identical \cite{oneilgeom}[Chapters 6 and 7]. To connect to identifibility, we express any of the properties as a task of the form described in Section~\ref{sec:background} and use the isometry result above to conclude that the output will be same. We will show an example of this and prove the claim for the goedesic distance function (Theorem~\ref{thrm:identifiable_dist_func} in the main text) below.
\end{proof}


\begin{theorem}
(Theorem~\ref{thrm:identifiable_dist_func} in the main text)

    Let $\theta_a=\left(f_a, P_{Z_a}\right)$ and $\theta_b=\left(f_b, P_{Z_b}\right)$ with $P_{\theta_a}=P_{\theta_b}$and let $A_{a, b}$ be the generator transform between the parameters. Furthermore, let $(\mathcal{Z}_a, g^{f_a})$ and $(\mathcal{Z}_b,g^{f_b})$ be the associated connected Riemannian manifolds. Then, the geodesic distance between $\mathbf{z}_1$ and $\mathbf{z}_2$ is identifiable and it holds that: 
    \begin{equation}
        \label{eq:appendix:geodesic_dist}
        d_{g^{f_a}}(\mathbf{z}_1,\mathbf{z}_2)= d_{g^{f_b}}(A_{a, b}(\mathbf{z}_1),A_{a, b}(\mathbf{z}_2))
    \end{equation}
for some $\mathbf{z}_1,\mathbf{z}_2 \in \mathcal{Z}_a$ be two points in the latent space that correspond to some $\mathbf{x}_1,\mathbf{x}_2 \in \mathcal{M}$ on the manifold.
\end{theorem}

\begin{proof}We need to check Definition~\ref{def:task_identifiability} to show that the task of measuring distances in the latent space is identifiable using the geodesic distance function.

Let $\mathbf{x}_1,\mathbf{x}_2\in \mathcal{D}$ be the observed data points and consider the inverse of the decoders as our selection function given $\mathbf{x}_1$:
\begin{equation}
    \mathbf{z}_1^a=s(\theta_a, \mathbf{x}_1)=f_a^{-1}(\mathbf{x}_1) \text{ and } s(A\theta, \mathbf{x}_1)=s(\theta_b, \mathbf{x}_1)=f_b^{-1}(\mathbf{x}_1)=\mathbf{z}_1^b
\end{equation}
in a similar way we obtain $\mathbf{z}_2^a$ and $\mathbf{z}_2^b$. Define the task of measuring distances on the manifold as: 
\begin{equation}
\begin{aligned}
    t(\theta_a, \left\{ \mathbf{x}_1,\mathbf{x}_2 \right\} ,\left\{ \mathbf{z}_1^a,\mathbf{z}_2^a \right\}) &=d_{g^{f_a}}(\mathbf{z}_1^a,\mathbf{z}_2^a)\\
    t(\theta_b, \left\{ \mathbf{x}_1,\mathbf{x}_2 \right\} ,\left\{ \mathbf{z}_1^b,\mathbf{z}_2^b \right\}) &=d_{g^{f_b}}(\mathbf{z}_1^b,\mathbf{z}_2^b)
\end{aligned}
\end{equation}

To check the selection function, take  with $A\in \mathbf{A}(M)$ and some $\mathbf{x}_1\in \mathcal{D}$. Then,
\begin{equation}
    \begin{aligned}
        \mathbf{z}_1^b=s(\theta_b, \mathbf{x}_1) &= f_b^{-1}(\mathbf{x}_1)\\ &= f_b^{-1}\circ f_a \circ f_a^{-1}(\mathbf{x}_1)\\ &= A_{a,b}(s(\theta_a, \mathbf{x}_1))\\
        &= A(s(\theta_a, \mathbf{x}_1))\\
        &=A_{a,b}(f_a^{-1}(\mathbf{x}_1))\\&=A_{a,b}(\mathbf{z}_1^a)
    \end{aligned}
\end{equation}
where we have used that $A$ is almost everywhere equal to the generator transform $A_{a,b}$ due to \citeauthor{xi2023indeterminacy}.


To check the task function, let us recall that:
\begin{equation}
\label{eq:dist:ap}
d_{g^{f_a}}(\mathbf{z}_1^a,\mathbf{z}_2^a)= \inf_{\gamma} \int_a^b | \gamma^{\prime}(t) |_{g^{f_a}} dt    
\end{equation}
where $\gamma:[c,d]\rightarrow\mathcal{Z}_a$ is a curve in $\mathcal{Z}_a$ connecting $\mathbf{z}_1^a$ and $\mathbf{z}_2^a$ such that $\gamma(c)=\mathbf{z}_1^a$ and $\gamma(d)=\mathbf{z}_2^a$.



As a geodesic is not unique, multiple curves result in the infimum in Eq.~\ref{eq:dist:ap}. Let $\gamma_a$ be one solution, then it is by Definition~\ref{def:geodesics} a geodesic curve. By Theorem~\ref{thrm:gentransforms_isometries} we have that $(\mathcal{Z}_a, g^{f_a})$ and $(\mathcal{Z}_b,g^{f_b})$ are isometric, which means that $A_{a, b}$ maps geodesics to geodesics \cite{oneilgeom}, then $\gamma_b:[\tilde{c},\tilde{d}]\rightarrow \mathcal{Z}_b$ constructed from $\gamma_a$ by $A_{a, b}(\gamma_a)$ is a geodesic in $\mathcal{Z}_b$ and thus a solution for $d_{g^{f_b}}(A_{a, b}(\mathbf{z}_1^a),A_{a, b}(\mathbf{z}_2^a))=d_{g^{f_b}}(\mathbf{z}_1^b,\mathbf{z}_2^b)$.
\end{proof}



\section{Computing geodesics}
\label{appendix:geodesics}

A geodesic between $a$ and $b$ is defined to be a curve $\gamma(t)$ defined on some interval (usually $[0,1]$) such that $\gamma(0) = a$ and $\gamma(1) = b$ minimizing the length functional defined in Definition~\ref{def:metric_enables}. In our work, we choose to parametrize a geodesic by a cubic spline \cite{Schoenberg1946ContributionsTT} and optimize the energy functional defined in Definition~\ref{def:metric_enables} with respect to the free parameters using gradient methods. 

\subsection{Gedoesic parametrized by a cubic spline}

Having settled on a cubic spline as a parametrization of a geodesic, we will now describe the construction of the spline and use it to derive the free parameters of the resulting curve that we can use when minimizing the energy of that curve. 

A cubic spline is a piecewise function with pieces that are cubic polynomials. The points where the pieces meet are called the knots, $h$, and we want to construct a continuous spline with continuous first and second derivatives. Individual components are polynomials, so we only need to constrain their behavior at the knots to satisfy the requirements. Suppose the knots are known, and we are using the splines to interpolate a set of points. In that case, these constraints and boundary constraints will usually give a system of linear equations that can be solved to find the coefficients of the polynomials. In our setting, however, we are using splines to define a path between two points, and the knots are the unknown parameters of the problem, as well as the coefficients of the polynomials. Furthermore, given that $a,b \in \mathbb{R}^n$, we will look to parametrize a geodesic curve $\gamma(t) = (\gamma_1(t), \dots, \gamma_n(t)) \in \mathbb{R}^n$ and thus we will have $n$ splines, one for each dimension. In the following, we will describe how such a construction works for one dimension and invite the reader to conceptually repeat this for $n$ dimensions.

Following the idea of \cite{DG_SH}, we will start by connecting two points in the latent space ($a, b \in \mathbb{R}$) by a straight line $l: [0,1] \rightarrow \mathbb{R}$ defined as $l(t) = a + t(b-a)$ and then find a cubic spline that will start and end in 0 to parametrize a deviation from the line. The result will be a curve $\gamma(t) = l(t) + S(t)$ that will connect the two points on the manifold. 

The spline $S(t)$ is defined as a piecewise function with $n$ cubic polynomials with coefficients $a_i, b_i, c_i, d_i \in \mathbb{R}$, each defined on an interval $[h_i, h_{i+1}]$ where $h_i$ are the knots with $h_0 = 0$ and $h_n=1$ set to be the endpoints. 

\begin{equation}
    S(t) = \begin{cases}
        S_1(t) & \text{if } t \in [h_0, h_1] \\
        S_2(t) & \text{if } t \in [h_1, h_2] \\
        \vdots & \vdots \\
        S_n(t) & \text{if } t \in [h_{n-1}, h_n ]
    \end{cases}
\end{equation}

where each $S_i(t)$ is a cubic polynomial:
\begin{equation}
    %S_i(t) = a_i + b_i(t-h_{i-1}) + c_i(t-h_{i-1})^2 + d_i(t-h_{i-1})^3
    S_i(t) = a_i + b_i(t) + c_i(t)^2 + d_i(t)^3 \quad \text{for } t \in [h_{i-1}, h_i]
\end{equation}


In the following, let $\xi = (a_1,b_1,c_1,d_1, \dots, a_n, b_n, c_n, d_n)$ be a vector of all coefficients of the polynomials in our spline and $\xi[i,j]$ be a subvector of $\xi$ containing the coefficients of the $i$-th and $j$-th polynomial.

Boundary conditions mean that we need our first polynomial to start in $(0,0)$ and the last polynomial to end in $(1,0)$. This gives us two equations:
\begin{equation}
    S_1(0) = a_1 = 0 \quad \text{and} \quad S_n(1)= a_n + b_n + d_n + c_n  = 0
\end{equation}

which we translate into the following matrix equation of the coefficients $\xi$ and a $2 \times 4n$ matrix $B$:

\begin{equation}
    B \xi^T =
    \begin{bmatrix}
        0 \\ 0
    \end{bmatrix}
\end{equation}
where 
\begin{equation}
    B = \begin{bmatrix}
        1 & 0 & 0 & 0 & \dots & 0 & 0 & 0 & 0 & 0 \\
        0 & 0 & 0 & 0 & \dots &0 & 1 & 1 & 1 & 1
    \end{bmatrix}
\end{equation}

The continuity conditions are met when the values at the knots are the same for the two meeting polynomials. This can be expressed as: 
\begin{equation}
    S_i(h_i) = S_{i+1}(h_i) \Leftrightarrow S_i(h_i)-S_{i+1}(h_i)=0 \quad \text{for } i = 1, \dots, n-1 
\end{equation}
and for each knot we can write this as a dot product of the coefficients $\xi[i,i+1]$ and a vector $c_i^0$:
\begin{equation}
    \label{eq:continuity_constraint}
    c_i^0 =\begin{bmatrix}
        1 & h_i & h_i^2 & h_i^3 & -1 & -h_i & -h_i^2 & -h_i^3
    \end{bmatrix}
\end{equation}
such that the condition at a knot $i$ becomes: 
\begin{equation}
    c_i^0 \xi[i,i+1]^T= 0
\end{equation}

The conditions of first and second derivatives being continuous can be expressed in a similar way. 
\begin{equation}
    \begin{aligned}
        S_i'(h_i) = S_{i+1}'(h_i) \Leftrightarrow S_i'(h_i)-S_{i+1}'(h_i)=0 \quad \text{for } i = 1, \dots, n-1 \\
        S_i''(h_i) = S_{i+1}''(h_i) \Leftrightarrow S_i''(h_i)-S_{i+1}''(h_i)=0 \quad \text{for } i = 1, \dots, n-1
    \end{aligned}
\end{equation}
and we can write these conditions as dot products of the coefficients $\xi[i,i+1]$ and vectors $c_i^1$ and $c_i^2$:
\begin{equation}
    \label{eq:c1_constraint}
    c_i^1 =\begin{bmatrix}
        0 & 1 & 2h_i & 3h_i^2 & 0 & -1 & -2h_i & -3h_i^2
    \end{bmatrix}
\end{equation}
\begin{equation}
    \label{eq:c2_constraint}
    c_i^2 =\begin{bmatrix}
        0 & 0 & 2 & 6h_i & 0 & 0 & -2 & -6h_i
    \end{bmatrix}
\end{equation}
such that the conditions at a knot $i$ become:
\begin{equation}
    \begin{aligned}
        c_i^1 \xi[i,i+1]^T= 0 \\
        c_i^2 \xi[i,i+1]^T= 0
    \end{aligned}
\end{equation}
Having defined the smoothness constraints for a given knot $i$ we can construct  matrices $C^0, C^1, C^2$ each with dimensions $(n-1) \times 4n$ where each row $i$ corresponds to the respective constraint at the knot $i$ with $4\cdot (i-1)$ zeros before the constraint and $4\cdot (n-1-i)$ zeros after the constraint.
    E.g. for $n=4$ the $C^0$ matrix would look as follows:
    \begin{equation*}
        C^0 = \begin{bmatrix}
            c_1^0 & 0 & 0 & 0 & 0 & 0 & 0 & 0 & 0 \\
                0 & 0 & 0 & 0 & c_2^0 & 0 & 0 & 0 & 0 \\
                0 & 0 & 0 & 0 & 0 & 0 & 0 & 0 & c_3^0
        \end{bmatrix}
    \end{equation*}
where each $c_i^0$ is a row vector as defined in Eq.~\ref{eq:continuity_constraint}. 

Now, the final system of equations can be written as:
\begin{equation}
    \label{eq:spline_system}
    \underbrace{\begin{bmatrix}
         B \\
         C^0 \\
         C^1 \\
         C^2
   \end{bmatrix}}_{:=A}
    \xi^T = \boldsymbol{0}
\end{equation}
resulting in $4n-2$ equations for $4n$ unknowns. To solve this system of equations in an interpolation setting, we usually impose two additional constraints to get a square system of equations. These constraints can be that the second derivative is zero at the endpoints or that the second derivatives at the first and last knots are equal. The former is known in the literature as a natural spline and the latter as not-a-knot spline \cite{kress2012numerical}.

Getting back to our original task of finding free parameters of the curve that we can optimize its energy with respect to, we note that given that we have $4n$ coefficients, the actual number of free parameters is considerably smaller due to the constraints. The problem in Eq.~\ref{eq:spline_system} is known as the problem of finding the Null Space of the row space of matrix $A$. A basis for such null space, denoted by $\mathcal{N}(A)$, can be found by computing the Singular Value Decomposition (SVD) \cite{SVD} of $A$. If $A$ is of rank $r$, then SVD of A is given by $A = U \Sigma V^T$ where $U$ and $V$ are orthogonal matrices with dimensions $((4n-2) \times 4n)$ each and $\Sigma$ is a $(4n \times 4n)$ diagonal matrix with $r$ nonzero singular values in the diagonal. The null space of $A$ is then given by the columns of $V^T$ corresponding to the zero singular values. Treating $\mathcal{N}(A)=:N$ as a $(4n \times (n-r))$ matrix, we have arrived at a set of $n-r$ free parameters $\omega$ that we can optimize with respect to. To recover the full set of coefficients $\xi$, we can use the following equation:
\begin{equation}
    \label{eq:full_coefficients}
    \xi = N\omega
\end{equation} 
and evaluate the spline at the desired points to get the curve $\gamma(t)$.

\subsection{Optimizing the spline to find a geodesic}
\label{sec:problemz_to_solve}
In the previous subsection, we have reduced the infinite set of functions in which we are looking for a geodesic to another but considerably smaller, infinite set of splines. The next step is to use optimization to find the spline that minimizes the energy defined in Definition~\ref{def:metric_enables}. Calculating the energy requires computing an integral, which is, in practice, approximated by a sum over a discretized interval.

In the following treatment we assume that $\mathcal{D}=\mathbb{R}^n$ and let $f_{\theta}:\mathcal{Z}\rightarrow \mathbb{R}^n$ be a decoder parametrized by $\theta$ and $\gamma: [0,1] \rightarrow \mathcal{Z}$ be a spline in the latent space, then the approximation of the energy of $\gamma$ is given by:

\begin{equation}
\label{eq:energy_approx}
    \begin{aligned}
        E(\gamma) &=  \frac{1}{2}\int_0^1 | \gamma^{\prime}(t) |_g^2 dt \\
         &=  \frac{1}{2}\int_0^1 |\frac{\partial}{\partial t} f_{\theta}(\gamma(t))|_{E}^2 dt \\
        &\approx\frac{1}{2 \Delta t}\sum_{i=2}^{n_t} \left\| f_{\theta}(\gamma(\bar{t}_i)) - f_{\theta}(\gamma(\bar{t}_{i-1})) \right\|^2 =: \bar{E}(\gamma)
    \end{aligned}
\end{equation}

where $\left\{\bar{t}_i\right\}_{i=0}^{n_t}$ is a sequence of $n_t$ points in the interval $[0,1]$. Combining this with the discussion in the previous section, we can now define the optimization problem as simply:
\begin{equation}
    \label{eq:optimization_problem}
    \begin{aligned}
        \min_{\boldsymbol{\omega}} \quad & \bar{E}(\gamma_{\boldsymbol{\omega}}) 
    \end{aligned}
\end{equation}

where we use $\boldsymbol{\omega} = \left\{\omega_j\right\}_{j=1}^{k}$ to de note the parameters of the $k$ different splines given the dimensionality of the latent space $\mathcal{Z}\in \mathbb{R}^k$ and remind the reader that $\gamma_{\boldsymbol{\omega}}(t) = (\gamma_{\omega_1}^1(t), \dots, \gamma_{\omega_k}^k(t)) \in \mathcal{Z}$. 

Using optimization to learn the manifold will result in different approximations depending on the initialization of the parameters and the optimization algorithm used. Considering the problem in light of the first line of Eq.~\ref{eq:energy_approx}, we can see that the Riemannian metric becomes the stochastic term. In this sense, the manifold is stochastic, and the resulting distances between points will be affected by this stochasticity.

Following \citep{pmlr-v251-syrota24a}, having access to an ensemble of decoders allows us, in principle, to make the optimization problem in Eq.~\ref{eq:optimization_problem} aware of the uncertainty involved. The methodology effectively uses Monte Carlo methods to compute the energy with respect to the uncorrelated posterior over parameters. This posterior approximated by an ensemble. The following equation is the optimization problem we solve and makes the idea explicit:

\begin{equation}
    \label{eq:ensemble_optimization_problem}
    \begin{aligned}
         \min_{\boldsymbol{\omega}} \quad &\frac{1}{2 \Delta t}\sum_{i=2}^{n_t} \left\| f_{\hat{\theta}_j}(\gamma_{\boldsymbol{\omega}}(\bar{t}_i)) - f_{\hat{\theta}_{k}}(\gamma_{\boldsymbol{\omega}}(\bar{t}_{i-1})) \right\|^2\\
    \end{aligned}
\end{equation}


where $\Delta_t=\bar{t}_i -\bar{t}_{i-1} $ is the step size in the discretization of the interval $[0,1]$ and is assumed to be constant. The decoders $f_{\hat{\theta}_{k}}$ and $f_{\hat{\theta}_{kj}}$ are sampled uniformely and independently from the ensemble.

%\newpage
%\section{Experimental setup}
