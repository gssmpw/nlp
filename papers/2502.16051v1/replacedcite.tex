\section{Related Work}
\label{sec:2_related_work}

Numerous benchmarks and datasets have been introduced to train or evaluate AI systems for medical applications
ranging from genetics, radiology, cardiology, and EMR applications ____ to medical exam-like content such as MedQA ____, MMMU ____, NEJM Image Challenges ____, and Path-VQA ____, alongside exam-based tasks like MedMCQA ____ and MMLU ____. 
Broader efforts include MedS-bench ____ for LMs in clinical contexts and Google’s Gemini initiative ____ or state-of-the-art graduate-level and human expert benchmarks ____. 

In mental health, researchers have compiled datasets of counseling sessions ____, explored AI-driven diagnostic reasoning ____, and automated clinical documentation ____. 
They have also investigated therapy referrals ____, peer support ____, patient attitudes ____, and augmented care via automated psychotherapy, diagnosis, and biometric stress analysis ____, with broader safety considerations ____, concrete safety concerns in mental health emergencies ____, and demographic biases ____ remaining active concerns.

Unlike the existing exam-style benchmarks and multi-specialty medical datasets, our work focuses specifically on capturing the everyday ambiguities of mental healthcare tasks that often lack a single “correct” answer supported by extensive human expert input without intentionally contaminating the data with LM assistance. 
Thus, our work complements large datasets ____ that focus on scale.
While prior efforts have explored broader medical applications or aggregated data from exams, clinical notes, and research publications, our evaluation-first approach emphasizes diverse expert annotations, real-life psychiatric decision-making, and open-source availability, specifically within mental health. 
Finally, we evaluate the impact of demographic diversity on a wide variety of tasks such as triage and documentation—an analysis often overlooked by more extensive, general-purpose medical benchmarks.

% % Copied from TAIMH and Declan's Text


% There is a significant body of prior work regarding AI benchmarks and data sets for medical applications.
% For example, in genetics, radiology, cardiology, and electronic medical record data processing____, 



% In the domain of mental health datasets and summarization, Adhikary et al. introduced a dataset comprising 191 counseling sessions with associated summaries ____




% Google published *Capabilities of Gemini Models in Medicine*, incorporating several prominent medical benchmarks ____. The authors highlight the novelty of their work as “the most comprehensive benchmarking of multimodal medical models to date” based on their use of 14 different medical benchmarks ____. 

% These benchmarks include ECG-QA ____, MedQA ____, GeneTuring ____, MMMU (health medicine) ____, NEJM Image Challenges ____, Path-VQA ____, and others. Additionally, an effort was made to enhance the clinical relevance of these findings by evaluating the models’ ability to summarize complex medical information and generate referral letters for specialists ____. 

% the most popular medical benchmarks, such as medical question answering (MedQA), medical multiple-choice question answering (MedMCQA ____), and massive multitask language understanding (MMLU ____), are derivatives of medical exams such as USMLE or the MCAT

% A related study, published in 2025, identified 11 high-level clinical tasks and created a large benchmark (MedS-bench ____) which aggregates 58 open-source biomedical natural language processing datasets from five text sources, including exams, clinical texts, academic papers, medical knowledge bases, and daily conversations to create a instruction tuning dataset


% Google recently published its Artificial Medical Intelligence Explorer tool, which it describes as an “AI system based on a LLM and optimized for diagnostic reasoning and conversations” in primary care settings ____. 
% Companies utilize AI to automate administrative tasks in healthcare and assist providers in completing necessary documentation ____.

% Researchers are exploring AI's potential impact on various aspects of mental healthcare delivery, including its ability to automate therapy referrals ____, empower peer support specialists ____, and how the technology is viewed by patients ____. 
% They have also discussed its ability to augment clinical decision-making ____, deliver treatment in the form of increasingly automated psychotherapy or conversational agent ____, aid in psychiatric diagnosis ____, and investigated the use of AI to analyze biometric data in the setting of stress ____. 

% Other work creates data sets to evaluate physical/medical-related safety concerns ____, but do not study mental health-related issues or only partially touch on the topic of mental health treatment.



% ____ uses human expert evaluation with trained clinicians to show that LLMs can use implicit and explicit cues to infer patient demographics like race. 


% SOTA benchmarks on graduate level tasks and problem-solving created by human experts
% ____