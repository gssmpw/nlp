%%%%%%%% ICML 2025 EXAMPLE LATEX SUBMISSION FILE %%%%%%%%%%%%%%%%%

\documentclass{article}

% Recommended, but optional, packages for figures and better typesetting:
\usepackage{microtype}
\usepackage{graphicx}
\usepackage{subfigure}
\usepackage{booktabs} % for professional tables
\usepackage{makecell}
\usepackage{framed}
\usepackage{multirow}


% hyperref makes hyperlinks in the resulting PDF.
% If your build breaks (sometimes temporarily if a hyperlink spans a page)
% please comment out the following usepackage line and replace
% \usepackage{icml2025} with \usepackage[nohyperref]{icml2025} above.
\usepackage{hyperref}


% Attempt to make hyperref and algorithmic work together better:
\newcommand{\theHalgorithm}{\arabic{algorithm}}

% Use the following line for the initial blind version submitted for review:
% \usepackage{icml2025}

% If accepted, instead use the following line for the camera-ready submission:
\usepackage[accepted]{icml2025}

% For theorems and such
\usepackage{amsmath}
\usepackage{amssymb}
\usepackage{mathtools}
\usepackage{amsthm}

% if you use cleveref..
\usepackage[capitalize,noabbrev]{cleveref}

%%%%%%%%%%%%%%%%%%%%%%%%%%%%%%%%
% THEOREMS
%%%%%%%%%%%%%%%%%%%%%%%%%%%%%%%%
\theoremstyle{plain}
\newtheorem{theorem}{Theorem}[section]
\newtheorem{proposition}[theorem]{Proposition}
\newtheorem{lemma}[theorem]{Lemma}
\newtheorem{corollary}[theorem]{Corollary}
\theoremstyle{definition}
\newtheorem{definition}[theorem]{Definition}
\newtheorem{assumption}[theorem]{Assumption}
\theoremstyle{remark}
\newtheorem{remark}[theorem]{Remark}

% Todonotes is useful during development; simply uncomment the next line
%    and comment out the line below the next line to turn off comments
\usepackage[disable,textsize=tiny]{todonotes}
% \usepackage[textsize=tiny]{todonotes}


% The \icmltitle you define below is probably too long as a header.
% Therefore, a short form for the running title is supplied here:
\icmltitlerunning{Clinician-Annotated Dataset of Real-World Tasks and Ambiguity in Mental Healthcare}

\begin{document}

\twocolumn[
\icmltitle{Moving Beyond Medical Exam Questions: A Clinician-Annotated Dataset of Real-World Tasks and Ambiguity in Mental Healthcare}

% It is OKAY to include author information, even for blind
% submissions: the style file will automatically remove it for you
% unless you've provided the [accepted] option to the icml2025
% package.

% List of affiliations: The first argument should be a (short)
% identifier you will use later to specify author affiliations
% Academic affiliations should list Department, University, City, Region, Country
% Industry affiliations should list Company, City, Region, Country

% You can specify symbols, otherwise they are numbered in order.
% Ideally, you should not use this facility. Affiliations will be numbered
% in order of appearance and this is the preferred way.
\icmlsetsymbol{equal}{*}

\begin{icmlauthorlist}
\icmlauthor{Max Lamparth}{equal,stanford}
\icmlauthor{Declan Grabb}{equal,stanford}
\icmlauthor{Amy Franks}{colorado}
\icmlauthor{Scott Gershan}{northwestern}
\icmlauthor{Kaitlyn N Kunstman}{northwestern}
\icmlauthor{Aaron Lulla}{stanford}
\icmlauthor{Monika Drummond Roots}{wisconsin}
\icmlauthor{Manu Sharma}{yale}
\icmlauthor{Aryan Shrivastava}{chicago}
\icmlauthor{Nina Vasan}{stanford}
\icmlauthor{Colleen Waickman}{osu}
\end{icmlauthorlist}

\icmlaffiliation{stanford}{Stanford University}
\icmlaffiliation{colorado}{University of Colorado}
\icmlaffiliation{northwestern}{Northwestern University}
\icmlaffiliation{wisconsin}{University of Wisconsin}
\icmlaffiliation{yale}{Yale University}
\icmlaffiliation{chicago}{University of Chicago}
\icmlaffiliation{osu}{Ohio State University}

\icmlcorrespondingauthor{Max Lamparth}{lamparth@stanford.edu}
\icmlcorrespondingauthor{Declan Grabb}{declang@stanford.edu}


% You may provide any keywords that you
% find helpful for describing your paper; these are used to populate
% the "keywords" metadata in the PDF but will not be shown in the document
% \icmlkeywords{Machine Learning, ICML}

\vskip 0.3in
]

% this must go after the closing bracket ] following \twocolumn[ ...

% This command actually creates the footnote in the first column
% listing the affiliations and the copyright notice.
% The command takes one argument, which is text to display at the start of the footnote.
% The \icmlEqualContribution command is standard text for equal contribution.
% Remove it (just {}) if you do not need this facility.

%\printAffiliationsAndNotice{}  % leave blank if no need to mention equal contribution
\printAffiliationsAndNotice{\icmlEqualContribution} % otherwise use the standard text.

\begin{abstract}
Current medical language model (LM) benchmarks often over-simplify the complexities of day-to-day clinical practice tasks and instead rely on evaluating LMs on multiple-choice board exam questions. 
Thus, we present an expert-created and annotated dataset 
spanning five critical domains of decision-making in mental healthcare: treatment, diagnosis, documentation, monitoring, and triage. 
This dataset — created without any LM assistance — is designed to capture the nuanced clinical reasoning and daily ambiguities mental health practitioners encounter, reflecting the inherent complexities of care delivery that are missing from existing datasets. 
Almost all 203 base questions with five answer options each have had the decision-irrelevant demographic patient information removed and replaced with variables (e.g., AGE), and are available for male, female, or non-binary-coded patients. 
For question categories dealing with ambiguity and multiple valid answer options, we create a preference dataset with uncertainties from the expert annotations. 
We outline a series of intended use cases and demonstrate the usability of our dataset by evaluating eleven off-the-shelf and four mental health fine-tuned LMs on category-specific task accuracy, on the impact of patient demographic information on decision-making, and how consistently free-form responses deviate from human annotated samples.
\end{abstract}


\section{Introduction}\label{sec:intro}

In computational finance, Monte Carlo simulations are used extensively to estimate the expected value of financial payoffs based on the solution of stochastic differential equations (SDEs) which model the evolution of stock prices, interest rates, exchange rates and other quantities \cite{glasserman04}.  Monte Carlo methods are very general and flexible, but for high accuracy it requires generating a large number of costly SDE path approximations, which has motivated research into a number of variance reduction or, equivalently, cost reduction techniques. One such method is
Multilevel Monte Carlo (MLMC), which was proposed in \cite{GILES2008} and was adapted for various applications that are summarised in \cite{Giles_overview17} and successfully combined with other methods such as quasi-Monte Carlo methods. The main idea of MLMC is to approximate the payoff using different time stepping resolutions when numerically solving the underlying SDE and to generate an optimal number of samples on each level, such that the overall computational cost is minimised subject to the desired bound on the variance. %, such that the total computational cost is minimised. 
The computational savings come from the fact that most samples are computed on the coarser levels and hence are less expensive while only a few samples from the finest levels are required \cite{GILES2008}.


Among the directions in which the computational cost 
of MLMC methods could further be reduced, an important avenue is the use of lower precision calculations, especially for the first Monte Carlo levels where the targeted accuracy is relatively low. 
 An overview of the research on mixed precision for the standard Monte Carlo (MC) framework is provided in \cite{ChowMixedPrecisionStandardMC} but only a few references study the potential of low precision computation in the MLMC framework \cite{Rounding_error_oliver}. To the best of our knowledge, the only MLMC framework with customised precision in the literature is \cite{brugger2014mixed}, but they use a uniform precision for all operations on each Monte Carlo level instead of optimising 
 the precision of each intermediary variable to reduce as much as possible the cost of path generation.
 
An important motivation for an MLMC framework with variable precision would be performing the low precision computations on reconfigurable hardware devices such as Field Programmable Gate Arrays (FPGAs). FPGAs contain customizable logic blocks and connectors that make it easy to adapt the digital circuit architecture for a specific application, leading to a highly parallel and optimised implementation. Therefore they are successfully exploited in applications that require high speed and have high computational workload, such as signal processing \cite{woods2008fpga}, and real time applications like high frequency trading \cite{HFT1,HFT2}. That is why a number of previous works in hardware architecture design implemented the MLMC algorithm to price financial options using FPGAs as accelerators, which resulted in improved speed and power efficiency compared to full CPU architectures \cite{Schryver2013AMM}. The paper \cite{lindsey2016domain} also proposed 
a Domain Specific Language to automate the configuration of FPGAs for this specific application. However, only \cite{brugger2014mixed} proposed a heuristic to reduce the precision in calculations.

In addition, all aforementioned works considered that the random number generation (RNG) is performed in single or double precision. Yet in most cases an important portion of the workload in the overall MLMC simulation comes from the RNG and in \cite{brugger2014mixed} this limited the total computational savings.
To reduce the cost of MLMC simulations in particular those based on the Geometric Brownian Motion (GBM), \cite{approximateICDF_Oliver, NestedOliver} have proposed to use approximate random numbers that are generated by applying an approximation of the inverse CDF to uniform random numbers. In \cite{NestedOliver}, the authors proposed a way to integrate these lower precision random variables into a \textit{nested} MLMC framework and completed a numerical analysis to bound the resulting error at each MC level by a product of the time step and the error in the random number approximation. The same authors show in \cite{approximateICDF_Oliver} that using approximate random variables reduces the cost of path generation by a factor 7.


In this paper we propose a nested MLMC framework that combines the use of approximate random normal variables and lower precision calculations to reduce the computational cost of MLMC even further than \cite{brugger2014mixed,NestedOliver}. We illustrate the efficiency of our framework in Matlab, after making several assumptions on the cost of operations and size of the errors that we carefully justify. We focus on the case of GBM and use the approximate RNG methods presented in \cite{approximateICDF_Oliver} as well as a new slightly modified method that combines CDF inversion and the central limit theorem. To choose the precision of the variables in the low precision path generation, we introduce a novel method to optimise the bit-widths. This optimisation is performed before the main path generation loop is executed and is based on a linear model of the payoff error  
due to rounding when computing in low precision. The error model relies on algorithmic differentiation in a similar manner to \cite{unifying-bwoptim,bitwidth-AD,ADAPT}. The bit-width optimisation procedure can be performed off-line, so this stage can be excluded from the on-line time complexity of our framework. The user specified desired accuracy is then enforced by calculating on-line the number of samples that need to be generated.

In terms of hardware design, we suggest implementing the low precision path generation on FPGAs and the full-precision ones on a CPU or GPU. 
The FPGA offers enough flexibility to define a separate bit-width for every variable in the low precision path generation, and can be reconfigured periodically to update the bit-widths when the market parameters have changed considerably. 


The paper is organized as follows : \Cref{sec:MLMC} introduces MLMC and nested MLMC to make clear the estimator that is implemented in our framework. Then in \Cref{sec:RNG} we detail the methods that could be used to obtain approximate random normally distributed numbers very cheaply for the low precision path generation. In \Cref{sec:error_model} and \Cref{sec:costModel} we propose an error model and a cost model (resp.) that we then use to formulate the optimisation problem that is solved to obtain the optimal bit-widths of fixed point variables in \Cref{sec:optimisation}. Finally we summarise our results and future directions in \Cref{sec:conclusion}.




\section{Related Work}
\label{sec:related_work}

The original investigation \cite{gibson1979ecological} on the relationship between visual perception and human action defines \emph{affordance} as the opportunities for interaction with the surrounding environment. Behavioral studies on regular and cognitively impaired persons have shown evidence that perception results in both visual and motor signals in the human brain. An extended study \cite{anderson2002attentional} shows that visual attention to the spatial characteristics of the perceived objects initiates automatic motor signals for different actions. In computer vision, human affordance learning involves novel pose prediction such that the estimated pose represents a valid human action within the scene context. The task is fundamental to many problems requiring robust semantic reasoning about the environment, such as human motion synthesis \cite{wang2021scene} and scene-aware human pose generation \cite{wang2017binge, roy2016multi, zhang2022inpaint, yao2023scene}.

Earlier methods of affordance learning have explored knowledge mining \cite{zhu2014reasoning} and multimodal feature cues \cite{roy2016multi} to address the problem. In \cite{zhu2014reasoning}, the authors use a Markov Logic Network for constructing a knowledge base by extracting several object attributes from different image and metadata sources, which can perform various downstream visual inference tasks without any additional classifier, including zero-shot affordance prediction. In \cite{roy2016multi}, the authors use depth map, surface normals, and segmentation map as multimodal cues to train a multi-scale convolutional neural network (CNN) for scene-level semantic label assignment associated with specific human actions. In \cite{do2018affordancenet}, the authors design a multi-branch end-to-end CNN with two separate pathways for object detection and affordance label assignment to achieve high real-time inference throughput. Researchers \cite{chuang2018learning} have also explored socially imposed constraints for affordance learning. In \cite{chuang2018learning}, the authors propose a graph neural network (GNN) to propagate contextual scene information from egocentric views for action-object affordance reasoning.

Probabilistic modeling of scene-aware human motion generation also involves semantic reasoning of human interaction with the environment. Initial works on human motion synthesis have taken different architectural approaches, such as sequence-to-sequence models \cite{barsoum2018hp}, generative adversarial networks (GAN) \cite{barsoum2018hp, cai2018deep, yang2018pose}, graph convolutional networks (GCN) \cite{yan2019convolutional}, and variational autoencoders (VAE) \cite{guo2020action2motion}. However, these methods have mostly ignored the role of environmental semantics. Due to potential uncertainty in human motion, in a recent approach \cite{wang2021scene}, the authors address such motion synthesis with a GAN conditioned on scene attributes and motion trajectory to predict probable body pose dynamics.

One key challenge of human affordance generation in 2D scenes is the lack of large-scale datasets with rich pose annotations. In \cite{wang2017binge}, the authors compile the only public dataset of annotated human body poses in complex 2D indoor scenes by extracting frames from sitcom videos. Aiming to generate a contextually valid human affordance at a user-defined location, the authors propose sampling the scale and deformation parameters for an existing human pose template using a VAE conditioned on the localized image patches as scene context. In \cite{zhang2022inpaint}, the authors introduce a two-stage GAN architecture for achieving a similar goal by estimating the affine bounding box parameters to localize a probable human in the scene and then generating a potential body pose at that location. The method uses the input scene, corresponding depth, and segmentation maps as semantic guidance. In \cite{yao2023scene}, the authors propose a transformer-based approach with knowledge distillation for generating human affordances in 2D indoor scenes.



\begin{figure*}[t]
  \centering
  \includegraphics[width=\textwidth, height=10cm]{images/mindmap2.pdf} 
  \caption{Mindmap showing Data Collection and Rewrite Desiderata}
  \label{fig:mindmap}
\end{figure*}
% \begin{figure*}[t]
%   \centering
%   \includegraphics[width=\textwidth]{images/process.pdf} 
%   \caption{Dataset Creation Pipeline}
%   \label{fig:process}
% \end{figure*}
\section{Constructing a Dataset for Visual Instruction Rewriting}
\label{sec:datasets}

Task-oriented conversational AI systems rely on a semantic parser to interpret user intent and extract structured arguments \cite{louvan2020recent,aghajanyan2020conversational}. For example, when a user says,\textit{ "Add the team meeting to my calendar for Friday at 3 PM"}, the system must parse the intent (\textit{CreateCalendarEvent}) and extract arguments such as the \textit{EventTitle} (``team meeting''), \textit{EventDate} (``Friday''), and \textit{EventTime} (``3 PM'') to schedule the event correctly. Unlike purely text-based interactions, multimodal instructions, particularly those directed at conversational AI assistants on AR/VR devices (\textit{e.g.,} Apple's Siri for Apple Vision Pro), introduce additional challenges such as ellipsis and coreference resolution. For instance, a user may look at a book cover and ask, \textit{“Who wrote this?”} or point at a product in an AR interface and say, \textit{“How much does this cost?”} Traditional text-based semantic parsers struggle with such instructions since critical visual context is missing. Thus, to bridge the gap between multimodal input and existing conversational AI stacks, we introduce a dataset specifically designed for \textit{rewriting multimodal instructions} into structured text that can be processed by standard text-based semantic parsers. Figure \ref{fig:mindmap} illustrates a representation of the dataset collection requirement, highlighting the transformation of multimodal inputs into text-based rewrites.

To construct our dataset, we first define an ontology of intents and arguments, as existing ontologies in conversational AI and semantic parsing are often proprietary and unavailable for research use. We take inspiration from \newcite{goel2023presto} for ontology and extend it to accommodate multimodal task-oriented interactions. Figure \ref{fig:intent_argument_box} (ref. Appendix) presents an overview of the intents and arguments in our ontology. Next, we curate a diverse set of images covering various real-world multimodal interaction scenarios, including book covers, product packaging, paintings, mobile screenshots, flyers, signboards, and landmarks. These images are sourced from publicly available academic datasets, such as OCR-VQA\footnote{\url{https://ocr-vqa.github.io/}}, CD and book cover datasets, Stanford mobile image datasets\footnote{\url{http://web.cs.wpi.edu/~claypool/mmsys-dataset/2011/stanford/}}, flyer OCR datasets\footnote{\url{https://github.com/Skeletonboi/ocr-nlp-flyer.git}}, signboard classification datasets\footnote{\url{https://github.com/madrugado/signboard-classification-dataset}}, Google Landmarks\footnote{\url{https://github.com/cvdfoundation/google-landmark}}, and Products-10K\footnote{\url{https://products-10k.github.io/}}.

\begin{table}[t]
    \centering
    \scriptsize
    \label{tab:dataset_statistics}
    \begin{tabular}{llccc}
        \toprule
        \textbf{Category} & \textbf{Total} & \textbf{Train} & \textbf{Test} \\
        \midrule
        Book              & 485 / 500                               & 386 / 399                               & 101 / 101                               \\
        Business Card     & 26 / 960                                & 26 / 772                                & 26 / 188                                \\
        CD               & 27 / 1,020                              & 27 / 835                                & 27 / 185                                \\
        Flyer & 159 / 5,940                             & 159 / 4,742                             & 159 / 1,198                             \\
        Landmark         & 511 / 19,274                            & 511 / 15,420                            & 511 / 3,854                             \\
        Painting & 27 / 980                                & 27 / 774                                & 27 / 206                                \\
        Product          & 499 / 10,349                            & 499 / 8,276                             & 492 / 2,073                             \\
        \midrule
        \textbf{Total}   & \textbf{1,734 / 39,023}                 & \textbf{1,635 / 31,218}                 & \textbf{1,343 / 7,805}                  \\
        \bottomrule
    \end{tabular}
    \caption{Number of Images/Instructions per Category}
    \label{tab:sources}
\end{table}
\begin{table}[t]
    \centering
    \footnotesize
    \begin{tabular}{l  c}
        \toprule
         \textbf{Annotator}& \textbf{Percentage of Correct Captions}\\ 
         \midrule
         Annotator 1	& 90.62\%\\ 
         Annotator 2	& 87.23\%\\
         Annotator 3	& 86.35\%\\
         \midrule
         \textbf{At least two }& \textbf{92.18}\%\\
         \midrule
         \textit{All three }& \textit{74.63}\% \\
         \bottomrule
    \end{tabular}
    \caption{GPT-4 Instruction Rewriting Validation Results from Amazon Mechanical Turk }
    \label{tab:annotator_data}
\end{table}
\begin{figure}[t]
\includegraphics[width=\columnwidth]{images/intent.png}
  \caption{Dataset Distributions By Intent}
  \label{fig:intent}
\end{figure}
Upon identifying and verifying the images, we employ the GPT-4 model from OpenAI \cite{achiam2023gpt} to systematically generate and refine multimodal instructions into rewritten text-based instructions. The process begins with a bootstrap phase, where GPT-4 is prompted to generate 20 direct questions per image by explicitly referencing visible objects or textual elements while adhering to the intent list defined in Figure \ref{fig:intent_argument_box}. A second prompting phase then validates the generated questions against the corresponding image, filtering out ambiguous or irrelevant instructions to ensure alignment with the visual context. 

In the rewriting phase, GPT-4 is tasked with paraphrasing the validated instructions, ensuring that the transformed questions are fully self-contained and interpretable without requiring the image. This transformation is crucial for enabling multimodal conversational AI systems to process instructions using purely text-based stacks. Finally, a verification phase prompts the model to assess the rewritten questions in relation to both the original instruction and the image, ensuring semantic fidelity and eliminating inconsistencies. This multi-stage prompting strategy resulted in a dataset of 39,023 original-rewritten instruction pairs, derived from 1,734 images, with an 80\%-20\% train-test split. Table \ref{tab:sources} provides a breakdown of image sources.

While automated validation ensures consistency across different stages, human evaluation remains critical for verifying the dataset’s reliability. To this end, we conducted an annotation task via Amazon Mechanical Turk (AMT) to validate rewritten instructions within the test set for indirect image-based instructions. Each annotation task followed a structured validation guideline, where annotators reviewed an image, its original multimodal instruction, and the rewritten text-only instruction, determining whether the reformulation preserved the intent and meaning of the original instruction. Annotators were instructed to select "Accept" if the rewritten instruction was correct or "Reject" if it failed to capture the original meaning. Annotators are incentivized appropriately for this binary grading task. Agreement analysis, as shown in Table \ref{tab:annotator_data}, indicates that in 92.2\% of cases, at least two annotators agreed on "Accept," while 74.6\% of instructions achieved full consensus across all three annotators. Despite a Fleiss' Kappa score of 0.278—suggesting fair inter-annotator agreement—the high rate of majority consensus supports the dataset’s reliability for real-world use. Given these results, we publicly release the full dataset along with raw AMT responses, enabling further analysis, filtering, and refinements by the research community.

Figure \ref{fig:intent} presents the distribution of intents in our dataset, categorized into training and test splits. The distribution reflects practical usage patterns in real-world multimodal conversational AI systems, with a higher occurrence of general QA and web search, alongside diverse task-oriented intents such as reminders, messaging, and navigation, ensuring coverage of frequent user interactions.



% In this study, we utilize a comprehensive multimodal dataset curated from various sources to facilitate research in multimodal instruction rewriting using compact models. Table~\ref{tab:dataset_statistics} provides an overview of the dataset's composition, detailing the number of images and corresponding instructions sourced from different domains. This diverse dataset is designed to challenge models in interpreting and rewriting instructions based on both visual and textual information embedded within images.

% The dataset is organized into a single TSV file, \texttt{all\_data.tsv}, which consolidates all the data for streamlined processing and analysis.

% The dataset is publicly accessible and can be downloaded from our Hugging Face repository:
% \url{https://huggingface.co/datasets/utischoolnlp/multimodal_instruction_rewrites}.

% \begin{table}[h]
%     \centering
%     \caption{Dataset Statistics}
%     \label{tab:dataset_statistics}
%     \resizebox{0.5\textwidth}{!}{%
%         \begin{tabular}{|l|l|c|c|}
%             \hline
%             \textbf{Data Source} & \textbf{Type} & \textbf{Number of Images} & \textbf{Number of instructions} \\ \hline
%             \href{https://github.com/gulvarol/grocerydataset}{Grocery Store Dataset} & Grocery Dataset & 287 & 5,945 \\ \hline
%             \href{https://amazon-berkeley-objects.s3.amazonaws.com/index.html}{Amazon Berkeley Objects} & Amazon Dataset & 187 & 3,890 \\ \hline
%             \href{https://products-10k.github.io/}{Products-10K} & E-commerce Dataset & 23 & 472 \\ \hline
%             \href{https://www.kaggle.com/datasets/vikashrajluhaniwal/fashion-images}{Fashion Images} & Fashion Clothing Dataset & 2 & 42 \\ \hline
%             \textbf{Total} & & \textbf{499} & \textbf{10,349} \\ \hline
%         \end{tabular}
%     }
% \end{table}


% \subsection*{Additional Dataset Statistics}

% To provide a deeper understanding of the dataset's characteristics, we present the following statistics derived from \texttt{all\_data.tsv}:

% \begin{itemize}
%     \item \textbf{Prompt Length}:
%     \begin{itemize}
%         \item \textbf{Average Prompt Length}: 80.99 tokens
%         \item \textbf{Maximum Prompt Length}: 160 tokens
%         \item \textbf{Minimum Prompt Length}: 28 tokens
%     \end{itemize}
    
%     \item \textbf{Rewritten Question Length}:
%     \begin{itemize}
%         \item \textbf{Average Rewritten Question Length}: 56.94 tokens
%         \item \textbf{Maximum Rewritten Question Length}: 160 tokens
%         \item \textbf{Minimum Rewritten Question Length}: 28 tokens
%     \end{itemize}
% \end{itemize}

% These statistics highlight the complexity and variability of the prompts and their corresponding rewritten questions, providing a robust foundation for training and evaluating multimodal instruction rewriting models.

% \subsection*{Dataset Composition}

% The dataset is consolidated into a single TSV file, \texttt{all\_data.tsv}, which includes all image-instruction pairs. This unified format simplifies data handling and ensures consistency across training and evaluation phases. The structure of \texttt{all\_data.tsv} is as follows:


% \begin{itemize}
%     \item \textbf{Columns}:
%     \begin{itemize}
%         \item \texttt{Image\_ID}: Unique identifier for each image.
%         \item \texttt{Image\_URL}: Direct link to the image file.
%         \item \texttt{Prompt}: Original instruction associated with the image.
%         \item \texttt{Rewritten\_Question}: Reformulated version of the original instruction.
%     \end{itemize}
% \end{itemize}

% \subsection*{Dataset Accessibility}

% Researchers and practitioners can access the dataset and its associated resources through our Hugging Face repository:
% \url{https://huggingface.co/datasets/utischoolnlp/multimodal_instruction_rewrites}.

% The dataset is organized in a structured format, including:
% \begin{itemize}
%     \item \texttt{all\_data.tsv}: Consolidated dataset containing all image-instruction pairs.
%     \item \texttt{images.zip}: Compressed archive of all dataset images.
%     \item \texttt{README.md}: Detailed instructions and metadata descriptions for dataset usage.
% \end{itemize}

% \subsection*{Discussion}

% The diversity of data sources, ranging from grocery items to fashion clothing, ensures that the dataset covers a wide array of visual and textual contexts. This variety is crucial for training models that are robust and generalizable across different domains. The substantial number of instructions relative to images indicates that each image is associated with multiple instructions, providing ample data for effective model training and evaluation.

% By consolidating all data into a single TSV file, we streamline the data processing pipeline, facilitating easier integration with various modeling frameworks and tools. The comprehensive statistics on prompt and rewritten question lengths further underscore the dataset's complexity, challenging models to handle a wide range of instruction formulations.

% \section*{Conclusion}

% Our multimodal instruction rewriting dataset offers a comprehensive resource for researchers aiming to develop and evaluate models in this domain. By providing a diverse and sizeable dataset, we aim to facilitate advancements in multimodal understanding and contribute to the broader field of artificial intelligence.

% \section*{References}

% \begin{itemize}
%     \item \href{https://github.com/gulvarol/grocerydataset}{Grocery Store Dataset}
%     \item \href{https://amazon-berkeley-objects.s3.amazonaws.com/index.html}{Amazon Berkeley Objects}
%     \item \href{https://products-10k.github.io/}{Products-10K}
%     \item \href{https://www.kaggle.com/datasets/vikashrajluhaniwal/fashion-images}{Fashion Images Dataset}
% \end{itemize}

% \label{sec:dataset}

\section{Experiments}
\label{sec:experiments}

\begin{figure*}[t]
\vspace{-6mm}
    \centering
    \includegraphics[width=0.8\linewidth]{figs/compare.pdf}
    \vspace{-4mm}
    \caption{\textbf{Qualitative comparison} with the baseline for generating a sequence of novel view images.  
    The results demonstrate that our method synthesizes more consistent multi-view images compared to our baseline model (Zero123). In addition, compared to SyncDreamer, our method visually maintains better similarity to the conditioned image and appears more natural.}
    \label{fig:sota_compare}
\vspace{-5mm}
\end{figure*}

\subsection{Experimental Setups}
\textbf{Dataset.}
Following previous work~\cite{zero123, SyncDreamer}, we evaluate our work on the Google Scanned Object (GSO)~\cite{GSO} dataset to verify the zero-shot novel view image synthesis capability. 
We also provide results for additional datasets in the Supplementary Material.
Specifically, we randomly select 30 objects from the GSO dataset with various object categories. 
Unlike recent approaches~\cite{mvdream, SyncDreamer} that aim to enhance the consistency of novel view synthesis models by generating multiple fixed-view images, our method can generate images from any camera pose and any number of views. Therefore, we conduct experiments under different camera pose settings to validate our approach:
specifically, 
1) \textit{16-views with free camera pose}: for each object, we circularly render 16 views with the elevation angles ranging in $[-10\degree, 40\degree]$ and the azimuth angles are evenly distributed in $[0\degree, 360\degree]$. 
2) \textit{16-views with fixed camera pose}: We maintain a constant elevation angle of $30\degree$ and uniformly sample azimuth angles (same as SyncDreamer~\cite{SyncDreamer}).
3) \textit{32-views with free camera pose}: Similar to the first setting, but we sample 32 views.
It's important to note that our method does not require additional training or fine-tuning on any datasets.

\noindent\textbf{Metrics.}
To validate the effectiveness of our method, we mainly evaluate it based on three criteria:
1) \textit{Quality Score}. We evaluate the image quality of synthesized multi-view images by measuring their similarity with ground truth images. Following prior research~\cite{zero123, sparsefusion}, we report the similarity between the synthesized images and the ground truth images with standard metrics: PSNR, SSIM~\cite{ssim}, and LPIPS~\cite{lpips}.
2) \textit{Multi-view Consistency Score}. As the primary goal of our work is to improve the consistency of generated images, we also employ the 3D consistency score~\cite{3dim} to verify the consistency among the synthesized images. Specifically, we train an Instant-NGP~\cite{instant_ngp} with the input image and part of the synthesized novel view images of our model and evaluate the similarity between the remaining synthesized images and the rendered images of Instant-NGP. For the synthesized multi-view images of each object, we allocate $3/4$ for training and reserve the remaining $1/4$ for validation.
Intuitively, if the consistency of synthesized images is improved, the NeRF-like model will train a better object representation, and the re-rendered images will agree more with the validation images.
3) \textit{Input Consistency Score}. To assess the faithfulness of synthesized images in preserving the identity of the input condition image, we introduce the input consistency score. This score calculates the similarity of each synthesized image with the input condition image, utilizing the LPIPS metric.

In addition, we use synthesized multi-view images to train a neural 3D reconstruction model (NeuS~\cite{neus}) and report commonly used Chamfer Distances (CD) and Volume IoUs between the trained 3D model and the ground truth.

\noindent\textbf{Baselines.}
Given that our main goal is to improve the consistency of the trained baseline model without further fine-tuning, we mainly compare our approach with the used baseline model Zero123~\cite{zero123}. Additionally, we compare our method to the SOTA approaches such as PGD~\cite{tseng2023consistent} and SyncDreamer~\cite{SyncDreamer} using the same Zero123 base model.

\noindent\textbf{Implementation Details.}
We use the official checkpoint provided by Zero123~\cite{zero123}, which is trained on objaverse~\cite{objaverse} for 165,000 steps. We inject our epipolar attention layer after step $T=4$ and layer $L=10$ by default. We find that feature fusion weight $\alpha=0.5$, and the number of context views $M=2$ work better.

\begin{table}[t]
\centering
\caption{Comparison of multi-view consistency, image quality, and input consistency of synthesized multi-view images at the 16-view setting with free camera pose.}
\label{tab:view16_free_compare}
\vspace{-2mm}
\scalebox{0.6}{
\begin{tabular}{c ccc ccc c}
\toprule
              & \multicolumn{3}{c}{Multi-view Consistency} & \multicolumn{3}{c}{Quality Score} & \multicolumn{1}{c}{Input Consis.} \\
              \cmidrule(lr){2-4} \cmidrule(lr){5-7} \cmidrule(lr){8-8}
              & PSNR$\uparrow$  & SSIM$\uparrow$ & LPIPS$\downarrow$ 
              & PSNR$\uparrow$  & SSIM$\uparrow$ & LPIPS$\downarrow$ 
              & LPIPS$\downarrow$ 
              \\ \midrule

Zero123
& 15.225        & 0.645       & 0.408
& 14.255        & 0.747       &	0.208
& 0.303         
\\
SyncDreamer
& 14.830        & 0.626       & 0.434
& 12.650        & 0.713       &	0.254
& 0.317         
\\
Ours 
& \best{18.300}	& \best{0.734}	& \best{0.355}
& \best{14.947}	& \best{0.763}	& \best{0.191}
& \best{0.282}
\\

\bottomrule
\end{tabular}
}
\end{table}

\begin{table}[t]
\vspace{-1mm}
\centering
\caption{Comparison of multi-view consistency, image quality, and input consistency at the 16-view setting with fixed camera pose as SyncDreamer~\cite{SyncDreamer}.}
\label{tab:view16_fxied_compare}
\vspace{-3mm}
\scalebox{0.6}{
\begin{tabular}{c ccc ccc c}
\toprule
              & \multicolumn{3}{c}{Multi-view Consistency} & \multicolumn{3}{c}{Quality Score} & \multicolumn{1}{c}{Input Consis.} \\
              \cmidrule(lr){2-4} \cmidrule(lr){5-7} \cmidrule(lr){8-8}
              & PSNR$\uparrow$  & SSIM$\uparrow$ & LPIPS$\downarrow$ 
              & PSNR$\uparrow$  & SSIM$\uparrow$ & LPIPS$\downarrow$ 
              & LPIPS$\downarrow$ 
              \\ \midrule

Zero123
& 16.556        & 0.682       & 0.378
& 14.592        & 0.750       &	0.207
& 0.305         
\\
SyncDreamer
& \best{22.424}        & \best{0.812}       & \best{0.268}
& 15.269        & 0.749       &	0.196
& 0.300         
\\
Ours 
& 21.151	& 0.780	& 0.302
& \best{15.293}	& \best{0.764}	& \best{0.184}
& \best{0.287}
\\

\bottomrule
\end{tabular}
}
\vspace{-4mm}
\end{table}


\subsection{Comparison With Baseline Models}
The quantitative comparison on three settings are shown in Tab.~\ref{tab:view16_free_compare}, Tab.~\ref{tab:view16_fxied_compare}, and Tab.~\ref{tab:view32_free_compare}. The qualitative comparison is shown in Fig.~\ref{fig:sota_compare}.

\begin{table}[t]
\centering
\caption{Comparison of multi-view consistency and image quality scores of synthesized multi-view images at the 32-view setting with free camera pose.}
\vspace{-3mm}
\label{tab:view32_free_compare}
\scalebox{0.7}{
\begin{tabular}{c ccc ccc}
\toprule
              & \multicolumn{3}{c}{Multi-view Consistency} & \multicolumn{3}{c}{Quality Score} \\
              \cmidrule(lr){2-4} \cmidrule(lr){5-7}
              & PSNR$\uparrow$  & SSIM$\uparrow$ & LPIPS$\downarrow$ 
              & PSNR$\uparrow$  & SSIM$\uparrow$ & LPIPS$\downarrow$ 
              \\ \midrule

Zero123
& 16.515        & 0.694       & 0.378
& 15.142        & 0.733       &	0.211
\\
PGD~\cite{tseng2023consistent}
& 18.481        & 0.720       & 0.343
& 15.281        & 0.739       &	0.205
\\
Ours 
& \best{20.655}	& \best{0.792}	& \best{0.305}
& \best{15.268}	& \best{0.742}	& \best{0.203}
\\

\bottomrule
\end{tabular}
}
\vspace{-3mm}
\end{table}

\begin{table*}
  [t]
  \centering
  \resizebox{\textwidth}{!}{%
  \begin{tabular}{cccccccccccc}
    \toprule \multicolumn{2}{c}{Components}                                                             & \multicolumn{5}{c}{Re-executability Rate (\%)} & \multicolumn{5}{c}{Readability (\#)} \\
    \cmidrule(lr){1-2} \cmidrule(lr){3-7} \cmidrule(lr){8-12}        \hspace{8pt}\labelemoji\hspace{8pt}                                                                & \hspace{8pt}\toolemoji\hspace{8pt}                                      & O0                                 & O1             & O2             & O3             & AVG            & O0             & O1             & O2             & O3             & AVG            \\
    \hline
    \rowcolor[rgb]{0.93,0.93,0.93}\multicolumn{12}{c}{\textbf{Initialize with LLM4Decompile-End-6.7B~\citep{llm4decompile}}}   \\
    \xmark                                                                                              & \xmark                                    & 69.51                              & 46.95          & 50.61          & 46.34          & 53.35          & 3.98 & 3.41 & 3.44 & 3.38 & 3.55 \\
    \cmark                                                                                              & \xmark                                    & 75.61                              & 50.61          & 50.00          & 50.00          & 56.55          & 4.01 & 3.44 & 3.39 & \textbf{3.49} & 3.58 \\
    \xmark                                                                                              & \cmark                                    & 83.54                     & \textbf{56.10}          & 51.22          & 50.61 & 60.37 & 4.05 & 3.51 & 3.51 & 3.42 & 3.62 \\
    \cmark                                                                                              & \cmark                                    & \textbf{85.37}                            & \textbf{56.10}                     & \textbf{51.83} & \textbf{52.43}          & \textbf{61.43} & \textbf{4.13} & \textbf{3.60} & \textbf{3.54} & \textbf{3.49} & \textbf{3.69} \\

    \rowcolor[rgb]{0.93,0.93,0.93}\multicolumn{12}{c}{\textbf{Initialize with Deepseek-Coder-6.7B-base~\citep{deepseekcoder}}} \\
    \xmark                                                                                              & \xmark                                    & 59.15                              & 35.98          & 39.02          & 37.80          & 42.99          & 3.71 & 3.05 & 3.16 & 3.05 & 3.24 \\
    \cmark                                                                                              & \xmark                                    & 66.46                              & 41.46          & 38.41          & 36.59          & 45.73          & 3.76 & 3.17 & \textbf{3.21} & 3.08 & 3.31 \\
    \xmark                                                                                              & \cmark                                    & 70.73                              & 39.63          & 39.02          & 40.24          & 47.41          & 3.90 & 3.17 & 3.08 & 3.11 & 3.31 \\
    \cmark                                                                                              & \cmark                                    & \textbf{79.88}                     & \textbf{45.73} & \textbf{43.90} & \textbf{42.68} & \textbf{53.05} & \textbf{3.96} & \textbf{3.21} & 3.18 & \textbf{3.19} & \textbf{3.38} \\
    \bottomrule
  \end{tabular}%
  }
  \caption{The ablation study of different methods across four optimization levels
  (O0, O1, O2, O3), as well as their average scores (AVG). The results in bold represent the optimal performance. The ~\labelemoji~ and ~\toolemoji~ means Relabedling and Function Call. \textbf{Bold} denotes the best performance.}
  \label{tab:ablation}
\end{table*}



\begin{figure*}[ht]
    \centering
    \begin{minipage}{0.65\textwidth}
        \centering
        \includegraphics[width=0.95\linewidth]{figs/ablation.pdf}
        \vspace{-2mm}
        \captionof{figure}{Qualitative Comparison for different design choices. Our method, employing multi-view epipolar attention, demonstrates the best consistency.}
        \label{fig:ablation}
    \end{minipage}\hfill
    \begin{minipage}{0.33\textwidth}
        \centering
        \includegraphics[width=0.8\linewidth]{figs/neus_ver.pdf}
        \vspace{-3mm}
        \caption{Our method shows better direct 3D reconstruction~\cite{neus}.}
        \label{fig:neus}
    \end{minipage}
    \vspace{-5mm}
\end{figure*}

\noindent\textbf{Multi-view Consistency.}
Tab.~\ref{tab:view16_fxied_compare} presents the 3D consistency scores compared to our baseline model (Zero123) and SyncDreamer. The results indicate a significant improvement across all three metrics achieved by our method when compared with Zero123.
While our method exhibits a marginally lower numerical consistency score compared to SyncDreamer, it enables the synthesis of images with arbitrary camera poses.	
This capability is illustrated in Tab.~\ref{tab:view16_free_compare}, where our method consistently enhances consistency with changes in camera pose settings, whereas SyncDreamer fails to do so and exhibits inferior results compared to Zero123.
Furthermore, our method facilitates the synthesis of multi-view images with any number of camera views. This versatility is demonstrated in Tab.~\ref{tab:view32_free_compare}, where our method continues to achieve significant improvements in consistency scores, while SyncDreamer is unable to operate under such conditions.	

Meanwhile, Fig.~\ref{fig:sota_compare} provides a qualitative comparison with the baseline. While both our method and SyncDreamer enhance consistency, our method visually preserves better similarity to the input image, including color and texture details. The input consistency score further corroborates this.

\noindent\textbf{Image Quality.}
While our primary goal centers around enhancing the consistency of synthesized multi-view images, we also evaluate the image quality by comparing the similarity with the ground truth images. The results shown in Tab.~\ref{tab:view16_free_compare}, Tab.~\ref{tab:view16_fxied_compare}, and Tab.~\ref{tab:view32_free_compare} indicate that our method also enhances the image quality under different settings besides improving the consistency.
Moreover, our method shows better image quality compared with SyncDreamer even in the 16-view setting with fixed camera pose.

\noindent\textbf{Input Consistency.}
Input consistency terms whether the results align with the input image.
Fig.~\ref{fig:sota_compare} illustrates that both our method and SyncDreamer enhance multi-view consistency. However, the color and texture details of SyncDreamer's results diverge from the input image and appear visually unnatural.
This discrepancy is evident in the input consistency score presented in Tab.~\ref{tab:view16_fxied_compare}, indicating lower similarity with the condition image in the SyncDreamer results.	

\subsection{Ablation Study}
The overall quantitative results are shown in Tab.~\ref{tab:ablation}, and the qualitative comparisons are shown in Fig.~\ref{fig:ablation}.

\noindent \textbf{Full Attention \vs Epipolar Attention.}
The results presented in Tab.\ref{tab:ablation} and Fig.\ref{fig:ablation} demonstrate that our epipolar attention mechanism can synthesize more consistent multi-view images compared with full attention. Furthermore, our epipolar attention achieves a greater performance improvement compared to full attention when using multiple reference images. This could be attributed to the fact that our epipolar attention more effectively localizes target information, as depicted in Fig.~\ref{fig:full_attn_compare}, thereby reducing noise from the reference images. In the multi-view setting, where multiple reference images are utilized, this noise reduction becomes particularly crucial.
Moreover, it is noteworthy that the epipolar attention mechanism consumes less GPU memory compared to our baseline, as discussed in Sec.~\ref{sec:attn_analysis}.

\noindent \textbf{Attending Single-View \vs Multi-View.}
Applying the epipolar attention significantly improves the consistency between the input and target views. However, the consistency between different views in the unobserved regions of the input view is not well preserved.
After implementing our epipolar attention in the multi-view setting, the consistency across the generated multi-view images is further improved. The last row in Tab.~\ref{tab:ablation} shows that after applying our multi-view epipolar attention, the consistency score is further improved compared with the single-view setting. Besides, the qualitative result in Fig.~\ref{fig:ablation} also shows better consistency among different target views.



\begin{table}[t]
\centering
\vspace{-1mm}
\caption{Comparison of 3D reconstruction results. Our method significantly improves the reconstruction quality.}
\vspace{-3mm}
\label{tab:neus}
\scalebox{0.7}{
\begin{tabular}{c cc}
\toprule
              &  Chamfer Dist.$\downarrow$  & Volume IoU$\uparrow$
\\ \midrule

            Zero123         & 0.017         & 0.819    \\
            SyncDreamer     & \best{0.013}         & \best{0.847}    \\
            Ours            & 0.014	& 0.842 \\

\bottomrule
\end{tabular}
}
\vspace{-5mm}
\end{table}


\vspace{-2mm}
\subsection{Downstream Application}
\vspace{-2mm}
To demonstrate the effectiveness of our method, we also applied it to the downstream 3D reconstruction task. Specifically, we trained the NeuS model~\cite{neus} directly using images synthesized by our method, Zero123, and SyncDreamer, respectively.
The quantitative results in Tab.~\ref{tab:neus} show that the consistent multi-view images synthesized by our method can significantly improve the 3D reconstruction quality.
Additionally, our method exhibits similar performance to SyncDreamer which requires time-consuming re-training.
The qualitative results in Fig.~\ref{fig:neus} show that it is challenging to train the NeuS model directly due to the lack of consistency in the images generated by Zero123. In contrast, our method generates more consistent multi-view images and, therefore, better reconstructs the geometry and texture details.
We show improvements on other downstream applications such as image-to-3D in the Supplementary Material.



\section{Discussion}
\label{section:discussion}


\subsection{Practical Implications for Feedforward Prompting}

Of course, prompting an LLM continuously before the user submits their prompt is significantly most costly over submitting the prompt just once, once the user is ready.

% But user might not be ready, and the cognitive costs is pretty heavy.


\subsection{}


% Does this work well with Chain of Thought actually?
% Maybe this approach will actually incentivize self-prompt-chaining???
% What are the implications of this?


% A benefit of this is certainly more transparency in the LLM
% LLM is so flexible that adding this kind of structure is still okay for the LLM



% What's more costly, entering a prompt, then responding and saying, no i want this, or typing a prompt, and tuning the prompt/expected output to reduce message exchanges?

% Learning to become a better prompter. One is by trial and error experience. Perhaps another is through this feedforward that tells you what you might be able to anticipate.

% Example figure and table to copy to adhere to style guide:

% % Example figure
% \begin{figure}[ht]
%     \vskip 0.2in
%     \begin{center}
%     \centerline{\includegraphics[width=\columnwidth]{icml_numpapers}}
%     \caption{Historical locations and number of accepted papers for International
%     Machine Learning Conferences (ICML 1993 -- ICML 2008) and International
%     Workshops on Machine Learning (ML 1988 -- ML 1992). At the time this figure was
%     produced, the number of accepted papers for ICML 2008 was unknown and instead
%     estimated.}
%     \label{icml-historical}
%     \end{center}
%     \vskip -0.2in
% \end{figure}


% \begin{table}[t]
% \caption{Classification accuracies for naive Bayes and flexible
% Bayes on various data sets.}
% \label{sample-table}
% \vskip 0.15in
% \begin{center}
% \begin{small}
% \begin{sc}
% \begin{tabular}{lcccr}
% \toprule
% Data set & Naive & Flexible & Better? \\
% \midrule
% Breast    & 95.9$\pm$ 0.2& 96.7$\pm$ 0.2& $\surd$ \\
% Cleveland & 83.3$\pm$ 0.6& 80.0$\pm$ 0.6& $\times$\\
% Glass2    & 61.9$\pm$ 1.4& 83.8$\pm$ 0.7& $\surd$ \\
% Credit    & 74.8$\pm$ 0.5& 78.3$\pm$ 0.6&         \\
% Horse     & 73.3$\pm$ 0.9& 69.7$\pm$ 1.0& $\times$\\
% Meta      & 67.1$\pm$ 0.6& 76.5$\pm$ 0.5& $\surd$ \\
% Pima      & 75.1$\pm$ 0.6& 73.9$\pm$ 0.5&         \\
% Vehicle   & 44.9$\pm$ 0.6& 61.5$\pm$ 0.4& $\surd$ \\
% \bottomrule
% \end{tabular}
% \end{sc}
% \end{small}
% \end{center}
% \vskip -0.1in
% \end{table}

%%%%%%%%%%%%%%%%%%%%%%%%%%%%%%%%%%%%%%%%%%%%%%%%%%%%%%%%%%%%%%%%%%%%%%%%%%%%%%%
%%%%%%%%%%%%%%%%%%%%%%%%%%%%%%%%%%%%%%%%%%%%%%%%%%%%%%%%%%%%%%%%%%%%%%%%%%%%%%%
% Does not count to 8 page limit:
%%%%%%%%%%%%%%%%%%%%%%%%%%%%%%%%%%%%%%%%%%%%%%%%%%%%%%%%%%%%%%%%%%%%%%%%%%%%%%%
%%%%%%%%%%%%%%%%%%%%%%%%%%%%%%%%%%%%%%%%%%%%%%%%%%%%%%%%%%%%%%%%%%%%%%%%%%%%%%%

% Acknowledgements should only appear in the accepted version.
\section*{Acknowledgements}

Max Lamparth is partially supported by the Stanford Center for AI Safety, the Center for International Security and Cooperation, and the Stanford Existential Risk Initiative.


\section*{Impact Statement}

The MENTAT dataset represents a significant step forward in AI evaluation for psychiatry, providing a clinician-annotated, real-world benchmark that moves beyond traditional exam-style questions. By making the raw dataset (fully anonymized), processing code, evaluation framework, and final evaluation sets publicly available, we enable researchers to rigorously test models while allowing for easy modifications and extensions to fit various psychiatric AI applications. This ensures that MENTAT remains a flexible, transparent, and adaptable tool for AI alignment, fairness, and interpretability research.

A major ethical consideration in dataset creation is what to include and exclude—decisions that inevitably shape AI model development. We deliberately did not use LM-generated content, ensuring that all data comes from human clinical expertise rather than AI-reinforced biases. While this approach enhances credibility, bias risks remain—particularly in expert judgments and demographic representation. Although we sought diverse annotators, biases inherent to psychiatric practice or subtle algorithmic tendencies may still persist. By systematically varying demographic attributes, we provide a lens to study how AI models respond to different patient profiles, reinforcing the need for bias mitigation before deployment.

A critical risk is that a good model performance on MENTAT could inadvertently encourage premature AI deployment in psychiatric care. As AI models improve, there may be economic pressures to automate diagnosis, triage, and billing, potentially leading to job displacement and diminished human oversight. Without rigorous safety measures, AI-driven psychiatric tools could reinforce systemic biases, misdiagnose patients, or fail to recognize mental health emergencies. Ethical AI in psychiatry must prioritize human-in-the-loop validation, regulatory oversight, and transparent reporting of model limitations.

By establishing a higher standard for AI evaluation in psychiatry, we hope to guide responsible AI development while preventing premature deployment that could compromise patient care. MENTAT is a foundation for safer, fairer, and clinically meaningful AI—one that must augment, not replace, human expertise in mental healthcare.


\bibliography{icml_references}
\bibliographystyle{icml2025}


%%%%%%%%%%%%%%%%%%%%%%%%%%%%%%%%%%%%%%%%%%%%%%%%%%%%%%%%%%%%%%%%%%%%%%%%%%%%%%%
%%%%%%%%%%%%%%%%%%%%%%%%%%%%%%%%%%%%%%%%%%%%%%%%%%%%%%%%%%%%%%%%%%%%%%%%%%%%%%%
% APPENDIX
%%%%%%%%%%%%%%%%%%%%%%%%%%%%%%%%%%%%%%%%%%%%%%%%%%%%%%%%%%%%%%%%%%%%%%%%%%%%%%%
%%%%%%%%%%%%%%%%%%%%%%%%%%%%%%%%%%%%%%%%%%%%%%%%%%%%%%%%%%%%%%%%%%%%%%%%%%%%%%%
\newpage
\appendix
\onecolumn

\section{How is MENTAT Different from Medical Exam Questions?}
\label{app:medqa_to_mentat}

For years, medical AI benchmarks have focused on fact-based assessments. Most medical evaluations for LMs rely on board exams and medical student tests, primarily measuring knowledge recall rather than real-world clinical decision-making. These exams have little correlation with actual clinical practice, as passing them does not equate to the ability to manage patients effectively even in humans \cite{Saguil2015}.

\begin{figure}[ht]
    \begin{framed}
    A 32-year-old woman with type 1 diabetes mellitus has had progressive renal failure during the past 2 years. 
    She has not yet started dialysis. Examination shows no abnormalities. Her hemoglobin concentration is 9 g/dL, 
    hematocrit is 28\%, and mean corpuscular volume is 94 $\mu$m\textsuperscript{3}. 
    A blood smear shows normochromic, normocytic cells. 
    Which of the following is the most likely cause?
    
    (A) Acute blood loss \\
    (B) Chronic lymphocytic leukemia\\
    (C) Erythrocyte enzyme deficiency\\
    (D) Erythropoietin deficiency\\
    (E) Immunohemolysis\\
    (F) Microangiopathic hemolysis\\
    (G) Polycythemia vera \\
    (H) Sickle cell disease \\
    (I) Sideroblastic anemia \\
    (J) $\beta$-Thalassemia trait\\
    \textbf{(Answer: D)}
    \end{framed}
    \caption{USMLE board exam question example }
    \label{fig:usmle_example_q}
\end{figure}




For example, \Cref{fig:usmle_example_q} presents a classic USMLE board exam question \cite{USMLE2021}, which tests an AI model’s ability to recall factual knowledge rather than apply practical decision-making skills. The question may assess the recognition of a laboratory abnormality in diabetes, but it does not evaluate whether the model can adjust insulin regimens, recognize psychosocial factors, or determine hospitalization needs—key components of real-world patient care. As highlighted in previous research, medical licensing exams do not strongly correlate with clinical competency, reinforcing the need for benchmarks that evaluate accurate decision-making skills rather than memorization.

\begin{table}[h]
    \centering
    \begin{tabular}{llp{10cm}}
        \toprule
        \textbf{Question type} & \textbf{Attribute type} & \textbf{Example template question} \\
        \midrule
        \multirow{6}{*}{Single-Verify} 
        & SCP Code & Does this ECG show symptoms of \textbf{non-specific ST changes}? \\
        & Noise & Does this ECG show \textbf{baseline drift in lead I}? \\
        & Stage of infarction & Does this ECG show \textbf{early stage of myocardial infarction}? \\
        & Extra systole & Does this ECG show \textbf{ventricular extrasystoles}? \\
        & Heart axis & Does this ECG show \textbf{left axis deviation}? \\
        & Numeric feature & Does the \textbf{RR interval} of this ECG fall \textbf{within the normal range}? \\
        \bottomrule
    \end{tabular}
    \caption{Example template questions for different ECG attributes.}
    \label{tab:ecg_questions}
\end{table}
\begin{table}[h]
    \centering
    \begin{tabular}{lp{3.cm}p{3.5cm}p{1.5cm}p{3.5cm}}
        \toprule
        \textbf{Category} & \textbf{Task} & \textbf{Prompt} & \textbf{Result} & \textbf{AI Response} \\
        \midrule
        \multirow{2}{*}{Sequence alignment} 
        & DNA sequence alignment to human genome 
        & Align the DNA sequence to the human genome: \texttt{TGGGCTCA AGTGATCATA……} 
        & chr7 
        & As a language model AI, I do not have the capability to align a DNA sequence to the human genome…… 
        \\
        \midrule
        & DNA sequence alignment to multiple species 
        & Which organism does the DNA sequence come from: \texttt{CGTACACC ATTGGTGC……} 
        & yeast 
        & The organism from which the DNA sequence comes cannot be determined based solely on the DNA sequence…… 
         \\
        \bottomrule
    \end{tabular}
    \caption{DNA Sequence Alignment Tasks and AI Responses}
    \label{tab:sequence_alignment}
\end{table}


\Cref{tab:ecg_questions} and \Cref{tab:sequence_alignment} illustrate additional examples of widely used AI benchmarks, such as ECG-QA \cite{Oh2024} and GeneTuring \cite{Hou2023}, which focus on highly structured, fact-based medical knowledge. These datasets and others like MedQA \cite{Jin2021} have been leveraged by major AI companies, including Google’s Gemini initiative \cite{Saab2024}, to highlight model performance. While these benchmarks evaluate text-based and multimodal AI capabilities, they focus heavily on fact memorization rather than applied clinical reasoning.

Unlike traditional medical AI benchmarks, MENTAT is designed by practicing psychiatrists to reflect real-world clinical scenarios. The dataset also includes ambiguous, multi-choice decision-making tasks rather than a single correct answer, simulating the complex nature of psychiatric practice. Furthermore, MENTAT aims to reduce bias by empowering a diverse group of clinicians in its development from the start, making it less likely to reinforce harmful racial, gender, or sexuality-based biases in mental healthcare. In summary, MENTAT differs from medical exam questions by moving beyond fact recall to assess practical clinical decision-making in mental healthcare. While traditional benchmarks test AI models on medical knowledge, MENTAT evaluates whether AI can handle real-world psychiatric tasks, manage patient uncertainty, and make informed decisions in complex clinical environments.




% For years, benchmark evaluations have been utilized in medical AI to track the progress of new and updated models. However, they have largely focused on genetics, radiology, cardiology, and electronic medical record data processing\cite{Hou2023, Zambrano2023, Oh2024}. Little work has thus far been invested in the creation of benchmark evaluations and datasets for mental healthcare. Most medical evaluations of language and multi-modal models have also only focused on specialty board exams and exams intended for medical students. Both categories of exams assess knowledge but have been noted to have relatively little correlation to the real-world practice of medicine\cite{Saguil2015}. Every medical specialty would benefit from a creation of a dataset of question-answer pairs tailored specifically to clinical practice as opposed to the fact-based assessment that is common in licensing and medical student exams. As an example, Figure 1 demonstrates a classic board exam question directly from the USMLE website \cite{USMLE2021}, a medical licensing exam all medical students must take. It does not assess knowledge about pragmatic clinical management of diabetes; rather, it focuses on fact-based knowledge. Just as Saguil et al highlighted a lack of correlation between medical licensing exams and clinical skills, a model’s ability to answer Figure 1 correctly does not correlate to its ability to care for an individual with diabetes. Construction of datasets that test practical medical knowledge are necessary to robustly evaluate language models’ appropriateness in clinical settings. Furthermore, clinicians have recently called for moving beyond a medical exam benchmark, stating, "it is essential to move beyond medical exams and adopt more grounded, task-specific approaches for evaluation" \cite{Raji2025}. A related study, published in 2025, identified 11 high-level clinical tasks and created a benchmark (MedS-bench) meant to "address this gap" as current benchmarks and evaluation datasets "fail to adequately reflect the practical utility of LLMs in real-world clinical scenarios". Our work is different, but complementary, as our dataset applies this concept (testing for skills required to practice as a clinician as opposed to esoteric medical facts) to mental healthcare. In the realm of mental healthcare, researchers have expanded evaluations of LLM's into the realm of psychotherapy, which is complementary to our approach of evaluating LLM performance in the related field of psychiatry.


% As previously mentioned, most investigations into the evaluation of AI models in the healthcare setting have focused on pre-existing fact-based datasets such as the USMLE exams and specialty-specific board exams. In practice, these knowledge-based tests (e.g., USMLE Step 1) are designed to assess whether human trainees have acquired sufficient baseline knowledge to enter post-graduate training in a selected medical specialty. However, passing these tests alone is not considered sufficient for practicing as a physician in the United States. Residency training is required, during which trainees apply fundamental medical knowledge to real-world clinical cases \cite{Mowery2015}. 
% %
% Similarly, AI models must progress beyond simple recall of medical facts. They should be trained and evaluated on real-world clinical tasks that require the application of baseline medical knowledge. To maximize external validity, datasets should be created and vetted by actively practicing clinicians. Some of the most widely used benchmarks in medical AI are frequently leveraged by major companies to showcase the clinical capabilities of their fine-tuned models. For instance, in 2024, Google published *Capabilities of Gemini Models in Medicine*, incorporating several prominent medical benchmarks \cite{Saab2024}. The authors highlight the novelty of their work as “the most comprehensive benchmarking of multimodal medical models to date” based on their use of 14 different medical benchmarks \cite{Saab2024}. These benchmarks include ECG-QA \cite{Oh2024}, MedQA \cite{Jin2021}, GeneTuring \cite{Hou2023}, MMMU (health medicine) \cite{Yue2023}, NEJM Image Challenges \cite{NEJM2024}, Path-VQA \cite{He2020}, and others. Additionally, an effort was made to enhance the clinical relevance of these findings by evaluating the models’ ability to summarize complex medical information and generate referral letters for specialists \cite{Saab2024}. 
% %
% Examples of questions from these datasets are provided in Figures 3 and 4, which illustrate excerpts from ECG-QA and GeneTuring, respectively. MedQA, on the other hand, is best represented by the example in Figure 1. In the domain of mental health datasets and summarization, Adhikary et al. introduced a dataset comprising 191 counseling sessions with associated summaries \cite{Adhikary2024}. 
% %
% Despite the breadth of existing benchmarks, there is still no robust, clinician-led, and clinician-vetted mental healthcare benchmark for AI models. Current medical benchmarks remain overly narrow and fact-based, limiting their external validity and clinical relevance. As noted earlier, a high score on the MedQA benchmark does not equate to excellence in clinical care. Our dataset shifts the paradigm by moving beyond fact-based assessments (e.g., USMLE exams) and introducing a comprehensive evaluation of clinician-level decision-making in mental healthcare. 
% %
% Our benchmark uniquely assesses an AI model’s ability to **triage, diagnose, treat, and monitor mental health conditions**, establishing a new category of medical benchmarks that we hope other specialties will adopt. Furthermore, because our dataset is developed and overseen by a diverse group of practicing clinicians, it is significantly less likely to perpetuate harmful racial, gender, or sexuality-based biases in mental healthcare. It has been validated by [insert number] practicing psychiatrists.

\newpage

\section{Further Annotation Processing Results}
\label{app:annotation_details}

% \begin{figure}[ht]
%     \vskip 0.2in
%     \begin{center}
%     \centerline{\includegraphics[width=0.5\columnwidth]{figures/raw_annotation_krippendorf.pdf}}
%     \caption{Test.}
%     \label{fig:raw_annotation_krippendorf}
%     \end{center}
%     \vskip -0.2in
% \end{figure}

\begin{figure}[ht!]
    \centering
    \begin{minipage}[b]{0.49\textwidth}
        \centering
        \includegraphics[width=\linewidth]{figures/annotator_scores_hbt_pars.pdf}
        \caption{(Top) We show the average raw annotation score with with bootstrapped (95\% CL) uncertainties for each annotator. All of them deviate from 50 with statistical significance (the random baseline). 
        (Bottom) Fitted individual annotator parameters from the hierarchical Bradley-Terry model.
        Besides regularization in the log-likelihood objective, we bound the individual annotator parameters ($\gamma_a \in [-3.0, 3.0]$, $\alpha_a \in [0.5, 2.0]$) during the optimization to balance the goal of slightly de-noising the resulting preference dataset while keeping the majority of differences between individual annotator preferences.
        These bounds prevent the model from fixing contradictory data by pushing a parameter to an extreme.
        The fact that all annotators have a positive offset $\gamma_a$ indicates that they all tend to choose one answer option to prefer over all others in a single annotation of one question.}
        \label{fig:annotator_scores_hbt_pars}
    \end{minipage}%
    \hfill
    \begin{minipage}[b]{0.49\textwidth}
        \centering
        \includegraphics[width=\linewidth]{figures/raw_annotation_krippendorf.pdf}
        \caption{
        We show the distribution of  Krippendorff's $\alpha$ for raw triage and documentation question annotations.
        We verify that the expert annotators do not converge on one answer option and that there is sufficient inter-annotator disagreement.
        Given our design choices, we expect $\alpha$ to be naturally low as our goal is not to measure the presence of a single ground truth and low $\alpha$ values ($\alpha \leq 0.5$) will not tell us how useful a set of annotations is—only that experts statistically disagree. 
        }
        \label{fig:raw_annotation_krippendorf}
    \end{minipage}
\end{figure}
% \begin{figure}[ht]
%     \vskip 0.2in
%     \begin{center}
%     \centerline{\includegraphics[width=0.5\columnwidth]{figures/annotator_scores_hbt_pars.pdf}}
%     \caption{Test.}
%     \label{fig:annotator_scores_hbt_pars}
%     \end{center}
%     \vskip -0.2in
% \end{figure}
% \begin{figure}[ht]
%     \vskip 0.2in
%     \begin{center}
%     \centerline{\includegraphics[width=0.5\columnwidth]{figures/frac_ct in topk_bt_vs_hbt.pdf}}
%     \caption{Test.}
%     \label{fig:frac_ct in topk_bt_vs_hbt}
%     \end{center}
%     \vskip -0.2in
% \end{figure}

\newpage

\section{Language Model Prompts}
\label{app:prompting}

\begin{figure}[ht]
    \centering
    \begin{minipage}[b]{0.38\textwidth}
        \begin{framed}
        \texttt{
        f"Question: \{q\}\textbackslash n\textbackslash n"\\
        f"A: \{answer\_list[0]\}\textbackslash n"\\
        f"B: \{answer\_list[1]\}\textbackslash n"\\
        f"C: \{answer\_list[2]\}\textbackslash n"\\
        f"D: \{answer\_list[3]\}\textbackslash n"\\
        f"E: \{answer\_list[4]\}\textbackslash n\textbackslash n"\\
        "Answer (single letter): "
        }
        \end{framed}
    \end{minipage}%
    \hfill
    \begin{minipage}[b]{0.62\textwidth}
        % \centering
        \begin{framed}
        \texttt{
        f"Question: \{q\}\textbackslash n\textbackslash n"\\
        f"A: \{answer\_list[0]\}\textbackslash n"\\
        f"B: \{answer\_list[1]\}\textbackslash n"\\
        f"C: \{answer\_list[2]\}\textbackslash n"\\
        f"D: \{answer\_list[3]\}\textbackslash n"\\
        f"E: \{answer\_list[4]\}\textbackslash n\textbackslash n"\\
        "Answer (only reply with a single letter!): "
        }
        \end{framed}
    \end{minipage}
    \caption{(Left) Prompt text MCQA variation A (as used for \textit{gpt-4o-mini-2024-07-18}, \textit{gpt-4o-2024-08-06}, \textit{o1-2024-12-17}, and \textit{o1-mini-2024-09-12}).
    (Right) Prompt text MCQA variation B (all other models).
By looking at the responses from models evaluated with variation A, we verified that the recorded accuracy difference caused by using different promtps was $\leq 1$\%.
The only exception was \textit{o1-mini-2024-09-12}, for which we corrected the evaluation.}
    \label{fig:eval_prompts_mcqa}
\end{figure}

\begin{figure}[ht]
    \vskip 0.2in
    \begin{framed}
        \texttt{
        f"Question: \{q\}\textbackslash n\textbackslash n"\\
        "Answer (write your reply in only one short sentence!): "
        }
        \end{framed}
        \caption{Prompt text free-form (as used for the models evaluated in \Cref{sec:4_4_consistency}).}
    \vskip -0.2in
\end{figure}

\newpage

\section{Annotator Interface}
\label{app:annotator_interface}

\begin{figure}[ht]
    \centering
    \begin{minipage}[b]{0.49\textwidth}
        \centering
        \includegraphics[width=\linewidth]{figures/mentat_q36_question.png}
    \end{minipage}%
    \hfill
    \begin{minipage}[b]{0.49\textwidth}
        \centering
        \includegraphics[width=\linewidth]{figures/mentat_q36_answers.png}
    \end{minipage}
    \caption{Example of the online annotation interface using the \textit{jsPsych} library \citep{de_Leeuw2023} (MIT license). There is also a comment box below the sliders for feedback/comments, that is not shown.}
    \label{fig:mentat_q36_combined}
\end{figure}

\newpage

\section{Further Evaluation Results}
\label{app:more_experiment_results}

\begin{figure}[ht]
    \centering
    \begin{minipage}[b]{0.49\textwidth}
        \centering
        \includegraphics[width=\linewidth]{figures/final_eval_results_by_gender.pdf}
        \caption{Using the $\mathcal{D}_\text{G}$ dataset, we evaluate eleven off-the-shelf instruction-tuned and three (mental) healthcare fine-tuned models for overall accuracy and how it is impacted by different patient genders.}
        \label{fig:final_eval_results_by_gender}
    \end{minipage}%
    \hfill
    \begin{minipage}[b]{0.49\textwidth}
        \centering
        \includegraphics[width=\linewidth]{figures/final_eval_results_by_age.pdf}
        \caption{Using the $\mathcal{D}_\text{A}$ dataset, we evaluate eleven off-the-shelf instruction-tuned and three (mental) healthcare fine-tuned models for overall accuracy and how it is impacted by different patient ages.}
        \label{fig:final_eval_results_by_age}
    \end{minipage}
\end{figure}

% \begin{figure}[ht]
%     \vskip 0.2in
%     \begin{center}
%     \centerline{\includegraphics[width=0.5\columnwidth]{figures/final_eval_results_by_gender.pdf}}
%     \caption{Test.}
%     \label{fig:final_eval_results_by_gender}
%     \end{center}
%     \vskip -0.2in
% \end{figure}
% \begin{figure}[ht]
%     \vskip 0.2in
%     \begin{center}
%     \centerline{\includegraphics[width=0.5\columnwidth]{figures/final_eval_results_by_age.pdf}}
%     \caption{Test.}
%     \label{fig:final_eval_results_by_age}
%     \end{center}
%     \vskip -0.2in
% \end{figure}

\begin{figure}[ht!]
    \vskip 0.2in
    \begin{center}
    \centerline{\includegraphics[width=0.47\columnwidth]{figures/final_eval_results_by_nat.pdf}}
    \caption{Using the $\mathcal{D}_\text{N}$ dataset, we evaluate eleven off-the-shelf instruction-tuned and three (mental) healthcare fine-tuned models for overall accuracy and how it is impacted by different patient ethnicities.}
    \label{fig:final_eval_results_by_nat}
    \end{center}
    \vskip -0.2in
\end{figure}



\end{document}

