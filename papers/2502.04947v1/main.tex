%got all
% revue possible https://www.sciencedirect.com/journal/computer-methods-in-applied-mechanics-and-engineering
% git add --all
%git commit -m "modif"
%git push

%Les commandes pour git... Avant de commit/push :
%- Mettre de côté tes modifications locales
%git stash --include-untracked
%- Récupérer les modifications distantes :
%git pull
%- Réappliquer tes modifications
%git stash pop
%- Ajouter tous les fichiers à commiter
%git add .
%- Commiter les modifications
%git commit -m "Message"
%- Pousser les modifications
%git push

\documentclass{article}

% problème accent commentaires en français
\usepackage[utf8]{inputenc}

% Language setting
% Replace `english' with e.g. `spanish' to change the document language
\usepackage[english]{babel}

% Set page size and margins
% Replace `letterpaper' with`a4paper' for UK/EU standard size
\usepackage[scale=0.8]{geometry}

% Useful packages
\usepackage{mathtools}
\usepackage{amsthm}
\usepackage{thmtools}
\usepackage{amssymb}
\usepackage{graphicx}
\usepackage{multirow}
\usepackage{stmaryrd}
\usepackage{proof}
\usepackage{soul} % add by Frédérique (can be temporary)
\usepackage{subcaption} %add by Frédérique
\usepackage{float} %add by Frédérique (figure[H])
\usepackage{bm}
\usepackage[colorlinks=true, allcolors=blue]{hyperref}
\usepackage[capitalise,nameinlink,noabbrev]{cleveref}

\newcommand{\nico}[1]{\textcolor{magenta}{\textbf{Nicolas :} {#1}}}
\newcommand{\manu}[1]{\textcolor{purple}{\textbf{Emmanuel :} {#1}}}
\newcommand{\Ln}[1]{\textcolor{blue}{\textbf{LN :} {#1}}}
\newcommand{\fred}[1]{\textcolor{teal}{\textbf{Frédérique :} {#1}}}
\newcommand{\upfig}[2]{\textcolor{violet}{\textbf{Update #1  :} \\ {#2}}}
\newcommand{\flo}[1]{\textcolor{orange!95!black}{\textbf{Flo: } {#1}}}
\newcommand{\victor}[1]{\textcolor{green!50!black}{\textbf{Victor :} {#1}}}
\newcommand{\michel}[1]{\textcolor{red}{\textbf{Michel :} {#1}}}
\newcommand{\vanessa}[1]{\textcolor{cyan}{\textbf{Vanessa :} {#1}}}

% \newtheorem{theorem}{Theorem}[section]
\newtheorem{theorem}{Theorem}
\newtheorem{corollary}[theorem]{Corollary}
\newtheorem{lemma}[theorem]{Lemma}
\newtheorem{proposition}[theorem]{Proposition}
\newtheorem{definition}[theorem]{Definition}
\newtheorem{remark}[theorem]{Remark}

\DeclareMathOperator*{\argmin}{argmin}

\newcommand{\R}{\mathbb{R}}
\newcommand{\peclet}{\text{Pe}}

\usepackage{authblk}
\newcommand{\address}{\affil}

\title{Enriching continuous Lagrange finite element approximation spaces using neural networks}

%---------------------- Authors---------------------


\author[2]{Hélène Barucq}
\author[1]{Michel Duprez}
\author[2]{Florian Faucher}
\author[3]{Emmanuel Franck}
\author[1]{Frédérique Lecourtier\footnote{Corresponding author}}
\author[1,4]{Vanessa Lleras}
\author[3]{Victor Michel-Dansac}
\author[2]{Nicolas Victorion}

\address[1]{Project-Team MIMESIS, Inria, University of Strasbourg, ICube, CNRS UMR 7357, Strasbourg, France}
\address[2]{Project-Team Makutu, Inria, University of Pau and Pays de l'Adour, TotalEnergies, CNRS UMR 5142, Pau, France}
\address[3]{Project-Team MACARON, Inria, University of Strasbourg, IRMA, CNRS UMR 7501, Strasbourg, France}
\address[4]{IMAG, University of Montpellier, CNRS UMR 5149, Montpellier, France}

\numberwithin{equation}{section}

\usepackage{amssymb}
\usepackage{mathtools}
%\usepackage[scale=0.8]{geometry}
\usepackage{pgfplots}
\usepackage{pgfplotstable}
% \usepackage{filecontents}
\usepackage{datatool}
\usepackage{fp}

\pgfplotsset{
    compat=newest,
}
\pgfplotsset{
    smaller labels/.style={
        label style={font=\footnotesize},
        tick label style={font=\footnotesize}
    }
}
\tikzset{font=\small}
\usetikzlibrary{
    fpu,
    fixedpointarithmetic,
    babel,
    external,
    arrows.meta,
    plotmarks,
    positioning,
    angles,
    quotes,
    intersections,
    calc,
    spy,
    decorations.pathreplacing,
    matrix,
    fit,
}
\usepgfplotslibrary{fillbetween}

% Define colors
\definecolor{femcolor}{RGB}{51, 138, 55} %Green (27,158,119)
\definecolor{addcolor}{RGB}{217,95,2} %Orange
\definecolor{addsobcolor}{RGB}{199,39,34} %Red (sob or other)
\definecolor{multcolor3}{RGB}{117,112,179} %Purple 
%{RGB}{231,41,138} %Pink
\definecolor{multcolor100}{RGB}{0,0,0} %Black (+ empty marker)
\definecolor{multcolor0weak}{RGB}{49, 73, 181} %Blue
\definecolor{multcolor0strong}{RGB}{49, 181, 161} %Cyan

% Define line styles according to the method 
% FEM : solid
% Add : dashed
% Mult : dotted

% Define marker styles according to the degree
% P1 : square
% P2 : circle
% P3 : triangle

%________________ error lines (by Ricardo Costa) ________________

% argument 1: slopes (e.g. {4,6})
% argument 2: x position of the bottom left corner
% argument 3: y position of the bottom left corner
% argument 4: x length

\makeatletter

\newcommand{\printslopeinv}[4]{
    \tikzset{fixed point arithmetic}
    % get arguments
    \def\nero@printslope@orderlist{#1}
    \edef\nero@printslope@xpos{#2}
    \edef\nero@printslope@ypos{#3}
    \edef\nero@printslope@width{#4}
    % get points position
    \pgfmathparse{\nero@printslope@xpos+\nero@printslope@width}
    \edef\nero@printslope@px{\pgfmathresult}
    \edef\nero@printslope@py{\nero@printslope@ypos}
    \edef\nero@printslope@qx{\pgfmathresult}
    \edef\nero@printslope@ry{\nero@printslope@ypos}
    \foreach \nero@printslope@order in {#1}{
        \pgfmathparse{
        ((\nero@printslope@px/\nero@printslope@xpos)^(\nero@printslope@order))*\nero@printslope@ypos}
        \edef\nero@printslope@qy{\pgfmathresult}
            \edef\nero@aux1{\noexpand\draw[line width=0.6pt]
            (axis cs:\nero@printslope@xpos,\nero@printslope@ypos)
            -- (axis cs:\nero@printslope@qx,\nero@printslope@qy)
            -- (axis cs:\nero@printslope@px,\nero@printslope@py);}
        \nero@aux1
        % slope label
        \pgfmathparse{10^((ln(\nero@printslope@ry)+ln(\nero@printslope@qy))/(ln(10)*2))}
        \edef\nero@printslope@labelpos{\pgfmathresult}
        \edef\nero@aux2{\noexpand\node[anchor=west] at
            (axis cs:\nero@printslope@qx,\nero@printslope@labelpos)
            {\noexpand\tiny \nero@printslope@order};}
        \nero@aux2
        \global\edef\nero@printslope@ry{\nero@printslope@qy}
    }
    % base line
    \draw[line width=0.6pt] (axis cs:\nero@printslope@xpos,\nero@printslope@ypos)
        |- (axis cs:\nero@printslope@px,\nero@printslope@py);
    % label of base line
    \pgfmathparse{10^((ln(\nero@printslope@px)+ln(\nero@printslope@xpos))/(ln(10)*2))}
    \edef\nero@printslope@labelpos{\pgfmathresult}
    \node[anchor=north] at (axis cs:\nero@printslope@labelpos,\nero@printslope@ypos) {\tiny 1};
}

\makeatother

\newlength{\plotwidth}
\setlength{\plotwidth}{0.45\textwidth}
\newlength{\plotheight}
\setlength{\plotheight}{0.3\textwidth}

% \newcommand{\generatexticklabels}[2]{%
%     \pgfplotstableset{create on use/xlabels/.style={
%         create col/expr={##1}
%     }}%
%     \pgfplotstabletypeset[
%         col sep=comma,
%         string type,
%         columns={#2},
%         columns/xlabels/.style={column name={}, string type},
%         every head row/.style={output empty row}
%     ]{#1}%
%     \edef\xlabels{\pgfplotsretval}%
% }

% \newcommand{\generatexticklabels}[2]{%
%     % Number of rows
%     \pgfplotstablegetrowsof{#1}
%     \pgfmathtruncatemacro{\lastrow}{\pgfplotsretval-1}%

%     \edef\xlabels{}
%     \foreach \i in {0,...,\lastrow} {%
%         \pgfplotstablegetelem{\i}{#2}\of{#1}%
%         \edef\temp{\pgfplotsretval}%
%         \ifnum\i=0
%             \xappto\xlabels{\temp}%
%         \else
%             \xappto\xlabels{,\temp}%
%         \fi
%     }
%     \typeout{Xlabels: \xlabels}%
% }

\gdef\iterator{0}

\newenvironment{cvgh}[4]{
    \begin{tikzpicture}
        \edef\filename{#1}
        \edef\legendcolumns{#2}
        \edef\slopes{#3}
        \edef\ypos{#4}

        % Read the CSV file into a table
        \pgfplotstableread[col sep=comma]{\filename}\datatable

        % Add a new column with rounded values of "h"
        % \pgfplotstablecreatecol[
        %     create col/expr={
        %         round(\thisrow{h}*1000)/1000
        %     }
        % ]{h_rounded}{\datatable}

        % \generatexticklabels{\datatable}{h}
        
        % Obtenir le nombre de lignes
        % \pgfplotstablegetrowsof{\datatable}
        % \edef\Nrows{\pgfplotsretval}

        % Obtenir le second élément
        \pgfmathtruncatemacro{\secondrow}{1} % Index de la dernière ligne
        \pgfplotstablegetelem{\secondrow}{h}\of\datatable
        \pgfmathsetmacro{\second}{\pgfplotsretval} % Dernière valeur de h_rounded

        % Obtenir le premier élément
        \pgfmathtruncatemacro{\firstrow}{0} % Index de l'avant-dernière ligne
        \pgfplotstablegetelem{\firstrow}{h}\of\datatable
        \pgfmathsetmacro{\first}{\pgfplotsretval} % Avant-dernière valeur de h_rounded

        % Calculer la différence entre les deux
        \pgfmathsetmacro{\diff}{\first - \second}

        %update iterator
        \pgfmathtruncatemacro{\iterator}{\iterator+1}

        % \pgfkeys{/pgf/number format/.cd,sci,sci e,precision=2}

        \begin{loglogaxis}[
            smaller labels,
            name = left_plot,
            % axis lines
            axis lines = left,
            enlarge x limits={abs=10pt},
            enlarge y limits={abs=10pt},
            axis x line shift = -5pt,
            axis y line shift = -5pt,
            % labels
			xmode=log,
            xlabel = {$h$},
            ylabel = {\rotatebox{270}{$L^2$}},
            xlabel style={at={(ticklabel* cs:1.01)},anchor=west},
            ylabel style={at={(ticklabel* cs:1.01)},anchor=west},
            % ticks and labels
            xtick=data,
            xticklabels from table={\datatable}{h},
            % xticklabel=\pgfmathprintnumber{\tick}, % Conversion en notation scientifique
            % xticklabel style={/pgf/number format/fixed}, % Format fixe pour éviter l'approximation
			% xticklabels={%
            %     \pgfmathsetmacro\temp{\thisrow{h}}%
            %     \pgfmathprintnumber{\temp}%
            % }
            % xticklabels={%
            %     \pgfmathprintnumber[scientific, precision=2]{\thisrow{h_rounded}} 
            % },
            % xticklabel style={rotate=45, anchor=east},
            % size
            width=\plotwidth, height=\plotheight,
            % marks
            mark options={solid, scale=1},
            % grid
            grid = major,
            % legend
            % legend entries={\legendentries},
            legend columns=\legendcolumns,
            legend to name=leg:legendFEMCORR_\iterator,
            legend image post style={mark options={solid, scale=1}},
        ]
        \expandafter\printslopeinv\expandafter{\slopes}{\second}{\ypos}{\diff}
    }
    {
        \end{loglogaxis}
        \node[yshift=-20pt] at (left_plot.outer south) {\pgfplotslegendfromname{leg:legendFEMCORR_\iterator}};

    \end{tikzpicture}
}

\newcommand{\cvgFEMCorrAlldeg}[3]{
    \edef\fem{#1}
    \edef\add{#2}

    \begin{cvgh}{\fem}{3}{2,3,4}{#3}
        % Complete the legend
        \addlegendentry{\,FEM $\mathbb{P}_1$\;}
        \addlegendentry{\,FEM $\mathbb{P}_2$\;}
        \addlegendentry{\,FEM $\mathbb{P}_3$\;}
        \addlegendentry{\,Add $\mathbb{P}_1$\;}
        \addlegendentry{\,Add $\mathbb{P}_2$\;}
        \addlegendentry{\,Add $\mathbb{P}_3$\;}

        % Plot FEM
        \addplot [style={solid}, mark=square*, mark size=2, color=femcolor, line width=0.8pt ]
        table [x=h, y=P1, col sep=comma]
            {\fem};
        
        \addplot [style={solid}, mark=*, mark size=2, color=femcolor, line width=0.8pt ]
        table [x=h, y=P2, col sep=comma]
            {\fem};
        
        \addplot [style={solid}, mark=triangle*, mark size=2, color=femcolor, line width=0.8pt ]
        table [x=h, y=P3, col sep=comma]
            {\fem};

        % Plot Add
        \addplot [style={dashed}, mark=square*, mark size=2, color=addcolor, line width=0.8pt ]
        table [x=h, y=P1, col sep=comma]
            {\add};

        \addplot [style={dashed}, mark=*, mark size=2, color=addcolor, line width=0.8pt ]
        table [x=h, y=P2, col sep=comma]
            {\add};

        \addplot [style={dashed}, mark=triangle*, mark size=2, color=addcolor, line width=0.8pt ]
        table [x=h, y=P3, col sep=comma]
            {\add};

    \end{cvgh}
}


\newcommand{\cvgFEMCorrMultOnedeg}[6]{
    \edef\fem{#1}
    \edef\femsec{#2}
    \edef\add{#3}
    \edef\mult{#4}
    \edef\multHundred{#5}

    \begin{cvgh}{\fem}{3}{2,3}{#6}
        % Complete the legend
        \addlegendentry{\,FEM $\mathbb{P}_1$\;}
        \addlegendentry{\,Mult $\mathbb{P}_1$ (M=3)\;}
        \addlegendentry{\,Add $\mathbb{P}_1$\;}
        \addlegendentry{\,FEM $\mathbb{P}_2$\;}
        \addlegendentry{\,Mult $\mathbb{P}_1$ (M=100)\;}

        % Plot the data
        \addplot [style={solid}, mark=square*, mark size=2, color=femcolor, line width=0.8pt ]
        table [x=h, y=err, col sep=comma]
            {\fem};

        \addplot [style={dotted}, mark=square*, mark size=2, color=multcolor3, line width=1.0pt ]
        table [x=h, y=err, col sep=comma]
            {\mult};

        \addplot [style={dashed}, mark=square*, mark size=2, color=addcolor, line width=0.8pt ]
        table [x=h, y=err, col sep=comma]
            {\add};

        \addplot [style={solid}, mark=*, mark size=2, color=femcolor, line width=0.8pt ]
        table [x=h, y=err, col sep=comma]
            {\femsec};

        \addplot [style={dotted}, mark=square, mark size=2, color=multcolor100, line width=1.0pt ]
        table [x=h, y=err, col sep=comma]
            {\multHundred};
    \end{cvgh}
}


%convergence Ell1D on square (with mul strong and mul weak)
\newcommand{\cvgFEMCorrMultSWOnedeg}[6]{
    \edef\fem{#1}
    \edef\femsec{#2}
    \edef\add{#3}
    \edef\mults{#4}
    \edef\multw{#5}

    \begin{cvgh}{\fem}{3}{2,3}{#6}
        % Complete the legend
        \addlegendentry{\,FEM $\mathbb{P}_1$\;}
        \addlegendentry{\,Mult strong $\mathbb{P}_1$\;}
        \addlegendentry{\,Add $\mathbb{P}_1$\;}
        \addlegendentry{\,FEM $\mathbb{P}_2$\;}
        \addlegendentry{\,Mult weak $\mathbb{P}_1$\;}

        \addplot [style={solid}, mark=square*, mark size=2, color=femcolor, line width=0.8pt ]
        table [x=h, y=err, col sep=comma]
            {\fem};

        \addplot [style={dotted}, mark=square*, mark size=2, color=multcolor0strong, line width=0.8pt ]
        table [x=h, y=err, col sep=comma]
            {\mults};

        \addplot [style={dashed}, mark=square*, mark size=2, color=addcolor, line width=0.8pt ]
        table [x=h, y=err, col sep=comma]
            {\add};

        \addplot [style={solid}, mark=*, mark size=2, color=femcolor, line width=0.8pt ]
        table [x=h, y=err, col sep=comma]
            {\femsec};

        \addplot [style={dotted}, mark=square*, mark size=2, color=multcolor0weak, line width=0.8pt ]
        table [x=h, y=err, col sep=comma]
            {\multw};
    \end{cvgh}
}



% convergence Lap2D on square (Sobolev loss)
\newcommand{\cvgFEMCorrAugFirst}[5]{
    \edef\fem{#1}
    \edef\add{#2}
    \edef\addsob{#3}
    \begin{cvgh}{\fem}{3}{2}{#4}
        % Complete the legend
        \addlegendentry{\,FEM $\mathbb{P}_1$\;}
        \addlegendentry{\,Add $\mathbb{P}_1$ ($u_\theta$)\;}
        \addlegendentry{\,Add $\mathbb{P}_1$ (#5)\;}

        \addplot [style={solid}, mark=square*, mark size=2, color=femcolor, line width=0.8pt ]
            table [x=h, y=P1, col sep=comma]
                {\fem};

        \addplot [style={dashed}, mark=square*, mark size=2, color=addcolor, line width=0.8pt ]
            table [x=h, y=P1, col sep=comma]
                {\add};

        \addplot [style={dashed}, mark=square*, mark size=2, color=addsobcolor, line width=0.8pt ]
            table [x=h, y=P1, col sep=comma]
                {\addsob};
    \end{cvgh}
}

\newcommand{\cvgFEMCorrAugSecond}[5]{
    \edef\fem{#1}
    \edef\add{#2}
    \edef\addsob{#3}

    \begin{cvgh}{\fem}{3}{3}{#4}
        % Complete the legend
        \addlegendentry{\,FEM $\mathbb{P}_2$\;}
        \addlegendentry{\,Add $\mathbb{P}_2$ ($u_\theta$)\;}
        \addlegendentry{\,Add $\mathbb{P}_2$ (#5)\;}

        \addplot [style={solid}, mark=*, mark size=2, color=femcolor, line width=0.8pt ]
            table [x=h, y=P2, col sep=comma]
                {\fem};

        \addplot [style={dashed}, mark=*, mark size=2, color=addcolor, line width=0.8pt ]
            table [x=h, y=P2, col sep=comma]
                {\add};

        \addplot [style={dashed}, mark=*, mark size=2, color=addsobcolor, line width=0.8pt ]
            table [x=h, y=P2, col sep=comma]
                {\addsob};
    \end{cvgh}
}

\newcommand{\cvgFEMCorrAugThird}[5]{
    \edef\fem{#1}
    \edef\add{#2}
    \edef\addsob{#3}

    \begin{cvgh}{\fem}{3}{4}{#4}
        % Complete the legend
        \addlegendentry{\,FEM $\mathbb{P}_3$\;}
        \addlegendentry{\,Add $\mathbb{P}_3$ ($u_\theta$)\;}
        \addlegendentry{\,Add $\mathbb{P}_3$ (#5)\;}

        \addplot [style={solid}, mark=triangle*, mark size=2, color=femcolor, line width=0.8pt ]
            table [x=h, y=P3, col sep=comma]
                {\fem};

        \addplot [style={dashed}, mark=triangle*, mark size=2, color=addcolor, line width=0.8pt ]
            table [x=h, y=P3, col sep=comma]
                {\add};

        \addplot [style={dashed}, mark=triangle*, mark size=2, color=addsobcolor, line width=0.8pt ]
            table [x=h, y=P3, col sep=comma]
                {\addsob};
    \end{cvgh}
}
% \usepackage{amssymb}
% \usepackage{mathtools}
% %\usepackage[scale=0.8]{geometry}
% \usepackage{pgfplots}
% \usepackage{pgfplotstable}
% \usepackage{filecontents}
% \usepackage{datatool}
% \usepackage{fp}
% \pgfplotsset{
%     compat=newest,
% }
% \pgfplotsset{
%     smaller labels/.style={
%         label style={font=\footnotesize},
%         tick label style={font=\footnotesize}
%     }
% }
% \tikzset{font=\small}
% \usetikzlibrary{
%     fpu,
%     fixedpointarithmetic,
%     babel,
%     external,
%     arrows.meta,
%     plotmarks,
%     positioning,
%     angles,
%     quotes,
%     intersections,
%     calc,
%     spy,
%     decorations.pathreplacing,
%     matrix,
%     fit,
% }
% \usepgfplotslibrary{fillbetween}

\usepackage{booktabs}

% gains pour un q
\newcommand{\gainstable}[1]{
	\pgfplotstabletypeset[
		col sep=comma,
		every head row/.style={
		before row={\toprule[1.pt]
		& \multicolumn{4}{c}{\textbf{Gains in $L^2$ rel error}} &
		\multicolumn{4}{c}{\textbf{Gains in $L^2$ rel error}} \\
		& \multicolumn{4}{c}{\textbf{of our method w.r.t. PINN}} &
		\multicolumn{4}{c}{\textbf{of our method w.r.t. FEM}} \\
		\cmidrule(lr){2-5} \cmidrule(lr){6-9}
		},
		after row=\cmidrule(lr){1-1} \cmidrule(lr){2-5} \cmidrule(lr){6-9}},
		every last row/.style={after row=\bottomrule[1.pt]},
		columns/N/.style={
		column name=\textbf{N}%,
		%			postproc cell content/.append style={
			%				/pgfplots/table/@cell content/.add={$\fontfamily{pag}\selectfont}{$}
			%			}
			},
		columns/min_PINNs/.style={column name=\textbf{min},fixed},
		columns/max_PINNs/.style={column name=\textbf{max},fixed},
		columns/mean_PINNs/.style={column name=\textbf{mean},fixed},
		columns/std_PINNs/.style={column name=\textbf{std},fixed},
		columns/min_FEM/.style={column name=\textbf{min},fixed},
		columns/max_FEM/.style={column name=\textbf{max},fixed},
		columns/mean_FEM/.style={column name=\textbf{mean},fixed},
		columns/std_FEM/.style={column name=\textbf{std},fixed},
		columns={N,min_PINNs,max_PINNs,mean_PINNs,std_PINNs,min_FEM,max_FEM,mean_FEM,std_FEM},
		precision=2
	]{#1}
}

% gains pour tous les q
\newcommand{\gainstableallq}[1]{
    \pgfplotstabletypeset[
        col sep=comma,
        every head row/.style={
        before row={\toprule[1.pt]
        & & \multicolumn{4}{c}{\textbf{Gains in $L^2$ rel error}} &
		\multicolumn{4}{c}{\textbf{Gains in $L^2$ rel error}} \\
		& & \multicolumn{4}{c}{\textbf{of our method w.r.t. PINN}} &
		\multicolumn{4}{c}{\textbf{of our method w.r.t. FEM}} \\
		\cmidrule(lr){3-6} \cmidrule(lr){7-10}
        },
        after row=\cmidrule(lr){1-1} \cmidrule(lr){2-2} \cmidrule(lr){3-6} \cmidrule(lr){7-10}},
        every last row/.style={after row=\bottomrule[1.pt]},
        every nth row={2}{before row=\cmidrule(lr){1-1} \cmidrule(lr){2-2} \cmidrule(lr){3-6} \cmidrule(lr){7-10}},
		columns/q/.style={column name=\textbf{k}},
        columns/N/.style={column name=\textbf{N}},
		columns/min_PINNs/.style={column name=\textbf{min},fixed},
        columns/max_PINNs/.style={column name=\textbf{max},fixed},
        columns/mean_PINNs/.style={column name=\textbf{mean},fixed},
		columns/std_PINNs/.style={column name=\textbf{std},fixed},
        columns/min_FEM/.style={column name=\textbf{min},fixed},
        columns/max_FEM/.style={column name=\textbf{max},fixed},
        columns/mean_FEM/.style={column name=\textbf{mean},fixed},
		columns/std_FEM/.style={column name=\textbf{std},fixed},
        columns={q,N,min_PINNs,max_PINNs,mean_PINNs,std_PINNs,min_FEM,max_FEM,mean_FEM,std_FEM},
        precision=2
    ]{#1}
}

% gains pour tous les q avec h (pas N)
\newcommand{\gainstableallqh}[1]{
    \pgfplotstabletypeset[
        col sep=comma,
        every head row/.style={
        before row={\toprule[1.pt]
        & & \multicolumn{4}{c}{\textbf{Gains in $L^2$ rel error}} &
		\multicolumn{4}{c}{\textbf{Gains in $L^2$ rel error}} \\
		& & \multicolumn{4}{c}{\textbf{of our method w.r.t. PINN}} &
		\multicolumn{4}{c}{\textbf{of our method w.r.t. FEM}} \\
		\cmidrule(lr){3-6} \cmidrule(lr){7-10}
        },
        after row=\cmidrule(lr){1-1} \cmidrule(lr){2-2} \cmidrule(lr){3-6} \cmidrule(lr){7-10}},
        every last row/.style={after row=\bottomrule[1.pt]},
        every nth row={2}{before row=\cmidrule(lr){1-1} \cmidrule(lr){2-2} \cmidrule(lr){3-6} \cmidrule(lr){7-10}},
		columns/q/.style={column name=\textbf{k}},
        columns/h/.style={column name=\textbf{h},sci},
		columns/min_PINNs/.style={column name=\textbf{min},fixed},
        columns/max_PINNs/.style={column name=\textbf{max},fixed},
        columns/mean_PINNs/.style={column name=\textbf{mean},fixed},
		columns/std_PINNs/.style={column name=\textbf{std},fixed},
        columns/min_FEM/.style={column name=\textbf{min},fixed},
        columns/max_FEM/.style={column name=\textbf{max},fixed},
        columns/mean_FEM/.style={column name=\textbf{mean},fixed},
		columns/std_FEM/.style={column name=\textbf{std},fixed},
        columns={q,h,min_PINNs,max_PINNs,mean_PINNs,std_PINNs,min_FEM,max_FEM,mean_FEM,std_FEM},
        precision=2
    ]{#1}
}



% gains pour tous les q
\newcommand{\gainstableMult}[1]{
    \pgfplotstabletypeset[
        col sep=comma,
        every head row/.style={
        before row={\toprule[1.pt]
        & & \multicolumn{4}{c}{\textbf{Gains in $L^2$ rel error}} &
		\multicolumn{4}{c}{\textbf{Gains in $L^2$ rel error}} \\
		& & \multicolumn{4}{c}{\textbf{of our method w.r.t. PINN}} &
		\multicolumn{4}{c}{\textbf{of our method w.r.t. FEM}} \\
		\cmidrule(lr){3-6} \cmidrule(lr){7-10}
        },
        after row=\cmidrule(lr){1-1} \cmidrule(lr){2-2} \cmidrule(lr){3-6} \cmidrule(lr){7-10}},
        every last row/.style={after row=\bottomrule[1.pt]},
        every nth row={2}{before row=\cmidrule(lr){1-1} \cmidrule(lr){2-2} \cmidrule(lr){3-6} \cmidrule(lr){7-10}},
		columns/method/.style={column name=\textbf{method},string type},
        columns/N/.style={column name=\textbf{N}},
		columns/min_PINNs/.style={column name=\textbf{min},fixed},
        columns/max_PINNs/.style={column name=\textbf{max},fixed},
        columns/mean_PINNs/.style={column name=\textbf{mean},fixed},
		columns/std_PINNs/.style={column name=\textbf{std},fixed},
        columns/min_FEM/.style={column name=\textbf{min},fixed},
        columns/max_FEM/.style={column name=\textbf{max},fixed},
        columns/mean_FEM/.style={column name=\textbf{mean},fixed},
		columns/std_FEM/.style={column name=\textbf{std},fixed},
        columns={method,N,min_PINNs,max_PINNs,mean_PINNs,std_PINNs,min_FEM,max_FEM,mean_FEM,std_FEM},
        precision=2
    ]{#1}
}

\newcommand{\gainstableMultData}[1]{
    \pgfplotstabletypeset[
        col sep=comma,
        every head row/.style={
        before row={\toprule[1.pt]
        & & \multicolumn{4}{c}{\textbf{Gains in $L^2$ rel error of our}} &
		\multicolumn{4}{c}{\textbf{Gains in $L^2$ rel error}} \\
		& & \multicolumn{4}{c}{\textbf{method w.r.t. Data Network}} &
		\multicolumn{4}{c}{\textbf{of our method w.r.t. FEM}} \\
		\cmidrule(lr){3-6} \cmidrule(lr){7-10}
        },
        after row=\cmidrule(lr){1-1} \cmidrule(lr){2-2} \cmidrule(lr){3-6} \cmidrule(lr){7-10}},
        every last row/.style={after row=\bottomrule[1.pt]},
        every nth row={2}{before row=\cmidrule(lr){1-1} \cmidrule(lr){2-2} \cmidrule(lr){3-6} \cmidrule(lr){7-10}},
		columns/method/.style={column name=\textbf{method},string type},
        columns/N/.style={column name=\textbf{N}},
		columns/min_PINNs/.style={column name=\textbf{min},fixed},
        columns/max_PINNs/.style={column name=\textbf{max},fixed},
        columns/mean_PINNs/.style={column name=\textbf{mean},fixed},
		columns/std_PINNs/.style={column name=\textbf{std},fixed},
        columns/min_FEM/.style={column name=\textbf{min},fixed},
        columns/max_FEM/.style={column name=\textbf{max},fixed},
        columns/mean_FEM/.style={column name=\textbf{mean},fixed},
		columns/std_FEM/.style={column name=\textbf{std},fixed},
        columns={method,N,min_PINNs,max_PINNs,mean_PINNs,std_PINNs,min_FEM,max_FEM,mean_FEM,std_FEM},
        precision=2
    ]{#1}
}

% Lap1D : gains pour comparaison PINNs/Data
\newcommand{\gainsbothNN}[2]{
	\pgfplotstabletypeset[
        col sep=comma,
        every head row/.style={
        before row={\toprule[1.pt]
        & \textbf{FEM} \\
		\cmidrule(lr){2-2}
        },
        after row=\cmidrule(lr){1-1} \cmidrule(lr){2-2}},
        every last row/.style={after row=\bottomrule[1.pt]},
        every nth row={2}{before row=\cmidrule(lr){1-1} \cmidrule(lr){2-2}},
        columns/N/.style={column name=\textbf{N}},
		columns/err/.style={column name=\textbf{error},sci},
        columns={N,err},
        precision=2
    ]{#1} \hspace{20pt}
	\pgfplotstabletypeset[
        col sep=comma,
        every head row/.style={
        before row={\toprule[1.pt]
        & & \multicolumn{2}{c}{\textbf{PINN prior $u_{\theta}$}} &
		\multicolumn{2}{c}{\textbf{Data prior $u_{\theta}^\text{data}$}} \\
		\cmidrule(lr){3-4} \cmidrule(lr){5-6}
        },
        after row=\cmidrule(lr){1-1} \cmidrule(lr){2-2} \cmidrule(lr){3-4} \cmidrule(lr){5-6}},
        every last row/.style={after row=\bottomrule[1.pt]},
        every nth row={2}{before row=\cmidrule(lr){1-1} \cmidrule(lr){2-2} \cmidrule(lr){3-4} \cmidrule(lr){5-6}},
		columns/method/.style={column name=\textbf{method},string type},
        columns/N/.style={column name=\textbf{N}},
		columns/PINNs_err/.style={column name=\textbf{error},sci},
        columns/PINNs_gains/.style={column name=\textbf{gain},fixed},
        columns/NN_err/.style={column name=\textbf{error},sci},
		columns/NN_gains/.style={column name=\textbf{gain},fixed},
        columns={method,N,PINNs_err,PINNs_gains,NN_err,NN_gains},
        precision=2
    ]{#2}
}


% Ell1D : gains pour mu2
\newcommand{\gainsPINNsecond}[2]{
	\pgfplotstabletypeset[
        col sep=comma,
        every head row/.style={
        before row={\toprule[1.pt]
        & \textbf{FEM} \\
		\cmidrule(lr){2-2}
        },
        after row=\cmidrule(lr){1-1} \cmidrule(lr){2-2}},
        every last row/.style={after row=\bottomrule[1.pt]},
        every nth row={2}{before row=\cmidrule(lr){1-1} \cmidrule(lr){2-2}},
        columns/N/.style={column name=\textbf{N}},
		columns/err2/.style={column name=\textbf{error},sci},
        columns={N,err2},
        precision=2
    ]{#1} \hspace{20pt}
	\pgfplotstabletypeset[
        col sep=comma,
        every head row/.style={
        before row={\toprule[1.pt]
        & & \multicolumn{2}{c}{\textbf{PINN prior $u_{\theta}$}} \\
		\cmidrule(lr){3-4}
        },
        after row=\cmidrule(lr){1-1} \cmidrule(lr){2-2} \cmidrule(lr){3-4}},
        every last row/.style={after row=\bottomrule[1.pt]},
        every nth row={2}{before row=\cmidrule(lr){1-1} \cmidrule(lr){2-2} \cmidrule(lr){3-4}},
		columns/method/.style={column name=\textbf{method},string type},
        columns/N/.style={column name=\textbf{N}},
		columns/err2/.style={column name=\textbf{error},sci},
        columns/gains2/.style={column name=\textbf{gain},fixed},
        columns={method,N,err2,gains2},
        precision=2
    ]{#2}
}

% Lap2D : gains pour mu1
\newcommand{\gainsPINNfirst}[2]{
	\pgfplotstabletypeset[
        col sep=comma,
        every head row/.style={
        before row={\toprule[1.pt]
        & \textbf{FEM} \\
		\cmidrule(lr){2-2}
        },
        after row=\cmidrule(lr){1-1} \cmidrule(lr){2-2}},
        every last row/.style={after row=\bottomrule[1.pt]},
        every nth row={2}{before row=\cmidrule(lr){1-1} \cmidrule(lr){2-2}},
        columns/N/.style={column name=\textbf{N}},
		columns/err1/.style={column name=\textbf{error},sci},
        columns={N,err1},
        precision=2
    ]{#1} \hspace{20pt}
	\pgfplotstabletypeset[
        col sep=comma,
        every head row/.style={
        before row={\toprule[1.pt]
        & & \multicolumn{2}{c}{\textbf{PINN prior $u_{\theta}$}} \\
		\cmidrule(lr){3-4}
        },
        after row=\cmidrule(lr){1-1} \cmidrule(lr){2-2} \cmidrule(lr){3-4}},
        every last row/.style={after row=\bottomrule[1.pt]},
        every nth row={2}{before row=\cmidrule(lr){1-1} \cmidrule(lr){2-2} \cmidrule(lr){3-4}},
		columns/method/.style={column name=\textbf{method},string type},
        columns/N/.style={column name=\textbf{N}},
		columns/err1/.style={column name=\textbf{error},sci},
        columns/gains1/.style={column name=\textbf{gain},fixed},
        columns={method,N,err1,gains1},
        precision=2
    ]{#2}
}

% \newcommand{\gainsbothmu}[2]{
% 	\pgfplotstabletypeset[
%         col sep=comma,
%         every head row/.style={
%         before row={\toprule[1.pt]
%         & \multicolumn{2}{c}{\textbf{FEM errors}} \\
% 		\cmidrule(lr){2-3}
%         },
%         after row=\cmidrule(lr){1-1} \cmidrule(lr){2-2} \cmidrule(lr){3-3}},
%         every last row/.style={after row=\bottomrule[1.pt]},
%         every nth row={2}{before row=\cmidrule(lr){1-1} \cmidrule(lr){2-2}},
%         columns/N/.style={column name=\textbf{N}},
% 		columns/err1/.style={column name=$\bm{\mu}^{(1)}$,sci},
%         columns/err2/.style={column name=$\bm{\mu}^{(2)}$,sci},
%         columns={N,err1,err2},
%         precision=2
%     ]{#1} \hspace{20pt}
% 	\pgfplotstabletypeset[
%         col sep=comma,
%         every head row/.style={
%         before row={\toprule[1.pt]
%         & & \multicolumn{2}{c}{$\bm{\mu}^{(1)}$} &
% 		\multicolumn{2}{c}{$\bm{\mu}^{(2)}$} \\
% 		\cmidrule(lr){3-4} \cmidrule(lr){5-6}
%         },
%         after row=\cmidrule(lr){1-1} \cmidrule(lr){2-2} \cmidrule(lr){3-4} \cmidrule(lr){5-6}},
%         every last row/.style={after row=\bottomrule[1.pt]},
%         every nth row={2}{before row=\cmidrule(lr){1-1} \cmidrule(lr){2-2} \cmidrule(lr){3-4} \cmidrule(lr){5-6}},
% 		columns/method/.style={column name=\textbf{method},string type},
%         columns/N/.style={column name=\textbf{N}},
% 		columns/err1/.style={column name=\textbf{error}},
%         columns/gains1/.style={column name=\textbf{gain},fixed},
%         columns/err2/.style={column name=\textbf{error}},
% 		columns/gains2/.style={column name=\textbf{gain},fixed},
%         columns={method,N,err1,gains1,err2,gains2},
%         precision=2
%     ]{#2}
% }
\usepackage{booktabs}

% costs pour tous les q : N et DoFs
\newcommand{\coststableallq}[1]{
    \pgfplotstabletypeset[
        col sep=comma,
        every head row/.style={
        before row={\toprule[1.pt]
        & & \multicolumn{2}{c}{\textbf{$N$}} &
		\multicolumn{2}{c}{\textbf{$N_\text{dofs}$}} \\
		\cmidrule(lr){3-4} \cmidrule(lr){5-6}
        },
        after row=\cmidrule(lr){1-1} \cmidrule(lr){2-2} \cmidrule(lr){3-4} \cmidrule(lr){5-6}},
        every last row/.style={after row=\bottomrule[1.pt]},
        every nth row={2}{before row=\cmidrule(lr){1-1} \cmidrule(lr){2-2} \cmidrule(lr){3-4} \cmidrule(lr){5-6}},
		columns/q/.style={column name=\textbf{q}},
        columns/e/.style={column name=\textbf{e},sci},
		columns/FEM_N/.style={column name=\textbf{FEM},fixed},
        columns/Add_N/.style={column name=\textbf{Add},fixed},
        columns/FEM_dofs/.style={column name=\textbf{FEM},fixed},
		columns/Add_dofs/.style={column name=\textbf{Add},fixed},
        columns={q,e,FEM_N,Add_N,FEM_dofs,Add_dofs},
        precision=2
    ]{#1}
}

% costs pour tous les q : Weights + Dofs (1 and 100 parameters)
\newcommand{\coststableallqhundred}[1]{
    \pgfplotstabletypeset[
        col sep=comma,
        every head row/.style={
        before row={\toprule[1.pt]
        & & \multicolumn{2}{c}{\textbf{$n_p=1$}} &
		\multicolumn{2}{c}{\textbf{$n_p=100$}} \\
		\cmidrule(lr){3-4} \cmidrule(lr){5-6}
        },
        after row=\cmidrule(lr){1-1} \cmidrule(lr){2-2} \cmidrule(lr){3-4} \cmidrule(lr){5-6}},
        every last row/.style={after row=\bottomrule[1.pt]},
        every nth row={2}{before row=\cmidrule(lr){1-1} \cmidrule(lr){2-2} \cmidrule(lr){3-4} \cmidrule(lr){5-6}},
		columns/q/.style={column name=\textbf{q}},
        columns/e/.style={column name=\textbf{e},sci},
		columns/FEM1/.style={column name=\textbf{FEM},fixed},
        columns/Corr1/.style={column name=\textbf{Add},fixed},
        columns/FEM100/.style={column name=\textbf{FEM},fixed},
		columns/Corr100/.style={column name=\textbf{Add},fixed},
        columns={q,e,FEM1,Corr1,FEM100,Corr100},
        precision=2
    ]{#1}
}

% \setlength\parindent{0pt} % OK ? bof (<- Victor)
\usepackage{dashrule}

% ajout des paragraphes dans la table des matières
\setcounter{secnumdepth}{4}
\setcounter{tocdepth}{4}

\crefname{paragraph}{Section}{Sections}
% \Crefname{paragraph}{Paragraph}{Paragraphs}

\crefname{remark}{Remark}{Remarks}
\Crefname{remark}{Remark}{Remarks}
\crefname{lemma}{Lemma}{Lemmas}
\Crefname{lemma}{Lemma}{Lemmas}

\begin{document}

\maketitle

\begin{abstract}
    In this work, we present a preliminary study combining two approaches in the context of solving PDEs: the classical finite element method (FEM) and more recent techniques based on neural networks. Indeed, in recent years, physics-informed neural networks (PINNs) have become particularly interesting for rapidly solving such problems, especially in high dimensions. However, their lack of accuracy is a significant drawback in this context, hence the interest in combining them with FEM, for which error estimators are already known. The complete pipeline proposed here, therefore, consists of modifying classical FEM approximation spaces by taking information from a prior, chosen here as the prediction of a neural network. On the one hand, this combination improves and certifies the prediction of neural networks to obtain a fast and accurate solution.
    On the other hand, error estimates are proven, showing that such strategies outperform classical ones by a factor that depends only on the quality of the prior. We validate our approach with numerical results obtained for this preliminary work on parametric problems with one- and two-dimensional geometries. They demonstrate that to achieve a fixed error target, a coarser mesh can be used with our enhanced FEM compared to the standard one, leading to reduced computation time, particularly for parametric problems.
\end{abstract}

%table of content
\tableofcontents

\section{Introduction%: \flo{new version in progress, reorganizing paragraphs}
}\label{sec:introduction}



The finite element method (FEM, e.g., \cite{ciarlet2002finite,Ern2004TheoryAP,brenner2008mathematical})
is a widely used method for the numerical solution of PDEs.
It requires the construction of the mesh that discretizes the domain of interest into elements, typically simplexes or quadrilaterals/hexahedrals.
Basis functions are associated to each element to define the approximation space in which the numerical solution is sought.
This solution is defined from degrees of freedom, which depend on the elements and basis functions, and by the order of approximation
of the method.
In particular, the method converges according to the characteristic mesh size and, depending on the mesh size, it requires the inversion of a potentially large matrix. Hence, solving the problem thousands of times  (as in optimal control or uncertainty propagation) becomes expensive.
This has motivated intensive research on new finite element methods to combine accuracy and matrices of reduced orders.
We can, for instance, refer to the Trefftz method (e.g., \cite{hiptmair2013error,moiola2018space,ImbMoiSto2022}),
or the hybridizable discontinuous Galerkin method (e.g., \cite{Cockburn2008,hungria2017hdg,Pham2024stabilization})
where the resulting matrix uses degrees of freedom only on the skeleton of the mesh.
In addition to the computational cost, it is important to note that one needs a mesh of the discretized geometry to perform the computation.
This step is not always straightforward (e.g., in geosciences where Earth layers are not clearly localized), and it can be time-consuming.
That is why mesh-less strategies have also been investigated in the last decades, and isogeometric analysis has been employed, e.g.
 \cite{hughes2005isogeometric,Frambati2022practical}.


In recent years, learning-based alternatives have emerged, such as Physics-Informed Neural Networks (PINNs, \cite{RAISSI2019686})
or the Deep Ritz method~\cite{e2017deepritzmethoddeep}.
The idea is to approximate the solution of the PDE under consideration using a neural network trained by minimizing a loss function,
taking the underlying physics into account.
Unlike neural networks trained with more conventional data-driven loss functions, these methods share similarities with traditional
solvers: They require the same inputs, namely the PDE, physical parameters, boundary, and initial conditions.
In addition, the training phase requires approximating the PDE solution in a discrete set of points of the space domain.
Thus, these approaches have some advantages, notably the absence of meshing and their relative dimension-insensitivity.
Since they do not require data (reference solutions), they are particularly well suited to high-dimensional problems on complex domains.
However, at present, these learning-based techniques are not competitive with classical finite element methods (see~\cite{grossmann2023can}), mainly because network-based methods lack precision and convergence guarantees, see \cite{sikora2024comparison} for a comparison between PINNs and FEM.
While FEM has a better error/computation time ratio for a single resolution, PINNs are more advantageous for parametric systems where a multitude of resolutions is needed. We further refer to \cite{cuomo2022scientific} for some analysis on PINNs.


This paper aims to propose a new method that combines learning-based and finite element methods.
More precisely, the main idea is to use a parametric PINN to compute a large family of offline solutions.
This is followed by calculating an online solution for a single parameter using coarse finite elements, with the PINN solution used as a prior information.
The result is a method capable of rapidly predicting a PDE solution while guaranteeing convergence properties, thanks to the FEM framework.
Finite element resolution improves the prediction while remaining cheap as it is performed on a coarse mesh, benefiting from the network prediction.
This paper proposes two ways to enrich the FEM.
In both cases, the finite element error will be exhibited as a function of the network error with respect to the true solution.
These corrections will be called ``additive" and ``multiplicative" depending on how the prediction is incorporated in the FEM spaces.



% Other works use an a priori for FEM resolution. For example, in the $\varphi$-FEM method developed in \cite{duprez2020phi} (see also \cite{cotin2023phi,duprez2023new,DupLleLozVui2023,duprez2023phi} for different contexts), the a prior of the FEM is a level set function used to localized the boundary of the domain. In \cite{brunet2019physics}, the authors initialize Newton's algorithm when solving a hyperelastic equation with a prior derived from the prediction of a neural network. Such a prediction can also be used as a prior for discontinuous Galerkin methods (see \cite{FraMicNav2024}).

Some previous works combine FEM and the use of a prior as neural network prediction.
In \cite{brunet2019physics}, the authors initialize Newton's algorithm when solving a hyperelastic equation with a prior derived from the prediction of a neural network. Such a prediction can also be used as a prior for discontinuous Galerkin methods (see \cite{FraMicNav2024}).
In \cite{feng_hybrid_2024}, the authors solve PDE by using a neural network for the spatial resolution and a FEM scheme for the temporal one. It is also possible to include the shape function of the FEM in a PINN approach as in \cite{skardova_finite_2024}.
In the $\varphi$-FEM method developed in \cite{duprez2020phi} (see also \cite{cotin2023phi,duprez2023new,DupLleLozVui2023,duprez2023phi} for different contexts), the a prior of the FEM is a level set function used to localized the boundary of the domain.
In the FEM, the space of approximation can also be enrich to ensure stability as for instance the introduction of bubble function in mixed problems (see e.g. \cite{Ern2004TheoryAP}).

In this paper, we consider general parametric linear elliptic differential equations
defined on a smooth domain $\Omega \subset \R^d$
with $d$ space dimensions and $\partial\Omega$ the boundary of $\Omega$.
Let a parameter space $\mathcal{M}=\{\bm{\mu}=(\mu_1,\ldots,\mu_p)\in \mathbb{R}^p\}$.
The typical problem of interest is given by:
for one or several  $\bm{\mu}\in \mathcal{M}$, find $u: \Omega\to \R$ such that
\begin{equation}
    \label{eq:ob_pde}
		\mathcal{L}\big(u;\bm{x},\bm{\mu}\big) = f(\bm{x},\bm{\mu}),
\end{equation}
with $\bm{x}=(x_1,\dots,x_d)\in\Omega$ the space variable and where $\mathcal{L}$ is the parametric differential operator defined  by
\begin{equation*}
    \mathcal{L}(\cdot;\bm{x},\bm{\mu}) : u \mapsto R(\bm{x},\bm{\mu}) u + C(\bm{\mu}) \cdot \nabla u - \frac{1}{\peclet} \nabla \cdot (D(\bm{x},\bm{\mu}) \nabla u),
\end{equation*}
with $f(\bm{x},\bm{\mu})\in L^2(\Omega)$ the source term,
$R(\bm{x},\bm{\mu})\in L^{\infty}(\Omega)
$ the reaction coefficient,
$C(\bm{\mu}) \in \R^d%(L^{\infty}(\Omega))^d
$ the convection coefficient,
$D(\bm{x},\bm{\mu}) \in {(W^{1,\infty}(\Omega))}^{d\times d}
$ the diffusion matrix (symmetric positive definite)
and $\peclet \in \R_+^*$ the Péclet number represents the ratio between convection and diffusion.
The differential operator is considered with Dirichlet, Neumann or Robin boundary conditions, which can also depend on $\bm{\mu}$.

The pipeline associated with our approaches is presented in \cref{fig:pipeline}.

\begin{figure}[!ht]
    \centering
%    \begin{minipage}{0.48\textwidth}
%        \centering
%        \includegraphics[width=\linewidth]{fig/pipeline/offline.pdf}
%    \end{minipage}
%    \begin{minipage}{0.48\textwidth}
%        \centering
%        \includegraphics[width=\linewidth]{fig/pipeline/online.pdf}
%    \end{minipage}
    \begin{minipage}{0.48\textwidth}
        \centering
        \includegraphics[width=\linewidth]{fig_pipeline_offline_v2.pdf}
    \end{minipage}
    \begin{minipage}{0.48\textwidth}
        \centering
        \includegraphics[width=\linewidth]{fig_pipeline_online_v2.pdf}
    \end{minipage}
    \caption{Pipeline of the enriched method considered. Left: offline phase (PINN training). Right: online phase (Correction of the PINN prediction).}
    \label{fig:pipeline}
\end{figure}




The manuscript is organized as follows:
In \cref{sec:FEM}, we recall the used finite element method, allowing us to introduce the notations needed in the next sections.
In \cref{sec:additive_prior,sec:multiplicative_prior}, we present the two proposed approaches and prove error estimates.
They both rely on modifying the functions of the FEM approximation space, using information from prior knowledge of the solution.
This prior is introduced first in an additive way, then in a multiplicative way.
Both approaches are compared in \cref{sec:comparison_add_mul}.
\Cref{sec:prior_construction} is devoted to the construction of the prior, justifying the use of PINNs and recalling methods for improving their efficiency.
In \cref{sec:implementation_details}, we give details on the implementation.
Numerical simulations conclude this manuscript in \cref{sec:numerical_results} and show that the proposed methods can significantly reduce the computational cost of solving parametric problems.

In \cref{app:notations}, we introduce the main notations used throughout this manuscript.

% \fred{Références à ajouter (proposées par Emmanuel) : \cite{feng_hybrid_2024} \cite{skardova_finite_2024}}



\section{Continuous finite element method}\label{sec:FEM}

The goal of this section is to recall the classical FEM,
and to introduce the notation that will be used throughout the paper.
To solve the problem~\eqref{eq:ob_pde} under consideration for a fixed parameter $\bm{\mu}$
(which will be omitted for clarity)
with homogeneous Dirichlet boundary conditions using
the continuous FEM, we rewrite it as the following variational problem:
\begin{equation}
	\label{eq:weakform}
	\text{Find } u \in V^0 \text{ such that, } \; \forall v\in V^0, \; a(u,v)=l(v),
\end{equation}
where $V^0=H^1_0(\Omega)$,
and where the bilinear form $a$ is given by
\[
	a(u,v)=
	\frac{1}{\peclet} \int_{\Omega}D \nabla u \cdot  \nabla v+
	\int_{\Omega} R \, u \, v  +
	\int_{\Omega} v \, C \cdot \nabla u 	,
\]
while the linear form $l$ reads
\[
	l(v)=\int_{\Omega} f \, v .
\]

\begin{remark}
	Note that since $a$ is continuous on $V^0\times V^0$ and coercive and $l$ is continuous on $V^0$,
    the existence and uniqueness of the solution $u$ are ensured by the Lax-Milgram theorem.
\end{remark}


Let $\mathcal{T}_h$ be a mesh of the domain $\Omega$ composed of simplexes,
where $h$ denotes the characteristic size of the mesh, i.e.\ the biggest diameter of the simplexes. We suppose that $\mathcal{T}_h$ satisfies the Ciarlet condition (see e.g.\ \cite{Ern2004TheoryAP}) and that its boundary is exactly $\partial\Omega$.
Consider $V_h^0\subset V_h\subset V=H^1(\Omega)$  the two continuous Lagrange finite elements spaces of degree $k\geq 1$ defined by
\begin{equation}
	V_h = \left\{v_h\in C^0(\Omega),\; \forall K\in \mathcal{T}_h,\; v_h\vert_{K}\in\mathbb{P}_k\right\},
	\label{eq:Vh}
\end{equation}
and
\begin{equation*}
	V_h^0 = \left\{v_h\in C^0(\Omega),\; \forall K\in \mathcal{T}_h,\; v_h\vert_{K}\in\mathbb{P}_k,v_h\vert_{\partial\Omega}=0\right\},
	\label{eq:Vh0}
\end{equation*}
with $\mathbb{P}_k$ the space of polynomials with real coefficients of degree at most $k$.
The solution to \eqref{eq:weakform} will be approximated by the solution $u_h$ to
\begin{equation}
	\label{eq:approachform}
	\text{Find } u_h \in V_h^0 \text{ such that, } \; \forall v_h\in V_h^0, \; a(u_h,v_h)=l(v_h).
\end{equation}

Let us now some results used in the next sections.
We first introduce the Lagrange interpolation operator defined by
\begin{equation}
	\mathcal{I}_h  : C^0(\Omega) \ni v \mapsto \sum_{i=1}^{N_{\text{\rm dofs}}} v\big(\bm{x}^{(i)}\big) \psi_i\in V_h,
	\label{eq:Ih}
\end{equation}
with \smash{${(\bm{x}^{(i)})}_{i \in \{1,\ldots,N_{\text{\rm dofs}}\}}$}
the $N_{\text{\rm dofs}}$ degrees of freedom (dofs) associated to the mesh,
and \smash{${(\psi_i)}_{i \in \{1,\ldots,N_{\text{\rm dofs}}\}}$}
the associated Lagrange shape functions of degree $k$.

\begin{remark}
	In the whole manuscript, for a Sobolev space $H$, the notation $|\cdot|_{H}$ and $\|\cdot\|_{H}$ will represent respectively the semi-norm and the norm in $H$.
\end{remark}

The following result gives a bound of the interpolation error:
\begin{theorem}[see e.g.\ \cite{Ern2004TheoryAP}]\label{th:interpol}
There exists $C_q>0$ such that
for all $v\in H^{q+1}(\Omega)$ and $1\leqslant q\leqslant k$,
\begin{equation*}
    \|v-\mathcal{I}_h v\|_{H^1}\leqslant C_q h^q |v|_{H^{q+1}}.
\end{equation*}
\end{theorem}

The next estimate is associated to the elliptic regularity:

\begin{theorem}[see e.g. {\cite[Theorem 4, p. 317]{evans2022partial}}]\label{th:ellip}
There exists $C_e>0$, such that for all $f\in L^2(\Omega)$, the unique solution $w\in H^2(\Omega)$ to
$$\mathcal{L}^* w=\xi$$
with homogeneous Dirichlet boundary condition satisfies
\begin{equation*}
    \|w\|_{H^2}\leqslant C_{e} \|\xi\|_{L^2}.
\end{equation*}
Here $\mathcal{L}^*$ represents the adjoint of the operator $\mathcal{L}$.
\end{theorem}


\noindent These estimates, combined with Céa's Lemma, which uses the continuity and coercivity of $a$, give the following error estimate:
\begin{theorem}[see e.g.\ \cite{Ern2004TheoryAP}]\label{thm:classical_error_estimate}
	Let $u\in H^{q+1}(\Omega)$ and $u_h\in V_h^0$ the solutions to~\eqref{eq:weakform} and~\eqref{eq:approachform}.
	For all $1\leqslant q\leqslant k$, one has %$0\leqslant m\leq q$, one has
	\begin{equation*}
		\label{eq:classical_error_estimate}
		|u-u_h|_{H^1}\leqslant C_q\dfrac{\gamma}{\alpha}h^{q} |u|_{H^{q+1}}
	\end{equation*}
	and
	\begin{equation*}
		\|u-u_h\|_{L^2}\leqslant C_eC_1C_q\dfrac{\gamma^2}{\alpha}h^{q+1} |u|_{H^{q+1}},
	\end{equation*}
	where $\gamma$ and $\alpha$ are respectively
	the constants of continuity and coercivity of $a$.
\end{theorem}



For the sake of simplicity, we consider an elliptic boundary
value problem with homogeneous Dirichlet conditions.
Obviously, we can use more general boundary conditions as Robin-like
conditions depending on the parameters. This will be investigated
in the numerical experiments to show that the proposed methodology
applies to a larger class of boundary value problems, see \cref{sec:Lap2DMixRing}.


% !TeX root = ../main.tex

\section{Enriching the finite element method with additive priors}
\label{sec:additive_prior}

In this section, we assume that a prior knowledge of the solution to \eqref{eq:ob_pde} is available. In what follows, we call this information a ``prior''.
This prior is denoted by $\bm{x} \mapsto u_{\theta}(\bm{x})$ with parameters $\theta$, and we assume that it can be constructed with the desired regularity $u_{\theta} \in H^{q+1}(\Omega)\cap H_0^1(\Omega)$ for $1\leqslant q\leqslant k$, where $k$ is the polynomial degree of the enriched FEM.
In this section and the next two, the prior will be general, but up the \cref{sec:prior_construction}, it will be the prediction of a parametric PINN.
In \cref{sec:modified_problem_add}, we first show how to use this prior to enriching classical finite element spaces.
Then, in \cref{sec:error_estimates_add}, we prove a convergence estimate for the resulting method.

\subsection{Construction of the modified problem}
\label{sec:modified_problem_add}

In the general setting of FEM, we follow the Bobunov--Galerkin method \cite{Ern2004TheoryAP}, where the basis functions and the numerical solutions are in the same space (see \eqref{eq:approachform}, where both $u_h$ and $v_h$ are in $V_h^0$).
As we intend to enrich the classical approximation space, we exploit the idea formalized as Petrov--Galerkin method (e.g., \cite{j2005introduction,brenner2008mathematical,demkowicz2023mathematical}), where the test and trial functions belong to different spaces.
This approach is often used for convection-dominated problems, \cite{ALMEIDA1997291}.
We propose to enrich the trial space using the prior $u_\theta$ such that,
\begin{equation}
    \label{eq:Vh_add}
    V_h^+ = \left\{
    u_h^+= u_{\theta} + p_h^+, \quad p_h^+ \in V_h^0
    \right\},
\end{equation}
and we use the space $V_h^0$ for the test functions.
Since we have assumed that $u_{\theta} \in H^{q+1}(\Omega)\cap H^1_0(\Omega)$,~$V_h^+$ is also a subset of $V^0$, like $V_h^0$.
Plugging this new trial space into the approximate problem \eqref{eq:approachform}, we obtain the formulation
\begin{equation}\label{eq:add1}
    \text{Find } u_h^+ \in V_h^+ \text{ such that, } \; \forall v_h\in V_h^0, \; a(u_h,v_h)=l(v_h),
\end{equation}
which leads to the following approximation problem:
\begin{equation}\label{eq:approachform_add}
    \text{Find } p_h^+ \in V_h^0 \text{ such that, } \;
    \forall v_h \in V_h^0, \; a(p_h^+,v_h) = l(v_h) - a(u_{\theta},v_h).
\end{equation}
Therefore, we obtain a classical Galerkin approximation with a modified source term.
%Indeed, as shown in \eqref{eq:approachform_add}, using the space $V_h^+$ for the solution amounts to writing a classical finite element method approximating the residual between the solution and the prior.
%In practice, we solve the modified problem~\eqref{eq:approachform_add} with modified boundary conditions, as detailed in \cref{sec:boundary_conditions}.\textcolor{red}{clarifier cette phrase?}

\subsection{Convergence analysis}
\label{sec:error_estimates_add}

The objective is to prove that the FEM solution to problem \eqref{eq:approachform_add} converges, with an error depending on the quality of the prior.


\begin{theorem}\label{lem:error_estimation_add}
    Let $u\in H^{q+1}(\Omega)$ be the solution to problem \eqref{eq:weakform} and $u_{\theta}\in H^{q+1}(\Omega)\cap H_0^1(\Omega)$ be a prior on $u$.
    We consider $u_h^+\in V_h^+$ as the solution to the discrete problem \eqref{eq:approachform_add} with $V_h^+$ the modified trial space defined in \eqref{eq:Vh_add}.
    The following estimates hold.
    For all $1\leqslant q\leqslant k$,
    \begin{equation}
        \label{eq:error_add}
        | u-u_h^+|_{H^1} \leqslant C_q\dfrac{\gamma}{\alpha} C_\text{\rm gain}^+ \, h^{q} |u|_{H^{q+1}}
    \end{equation}
    and
    \begin{equation*}
        \label{eq:error_addL2}
        \| u-u_h^+\|_{L^2} \leqslant C_e C_1 C_q\dfrac{\gamma^2}{\alpha} C_\text{\rm gain}^+ \, h^{q+1} |u|_{H^{q+1}},
    \end{equation*}
    with $C_e$, $C_1$, $C_q$, $\gamma$, $\alpha$ defined in \cref{sec:FEM} and
    \begin{equation}
        \label{eq:gain_add}
        C_\text{\rm gain}^+= \frac{| u-u_{\theta} |_{H^{q+1}}}{| u |_{H^{q+1}}}.
    \end{equation}
\end{theorem}
\begin{remark}\label{rmk:gain_add}
    The constant \smash{$C_\text{\rm gain}^+$}
    represents the potential gain compared to the error of the classical FEM presented in \cref{thm:classical_error_estimate}. Note that this constant is the same in $L^2$ norm and $H^1$ semi-norm.
\end{remark}



\begin{proof}[Proof of \cref{lem:error_estimation_add}]
    \textbf{$H^1$-error:}   To prove \eqref{eq:error_add}, we adapt the proof of Céa's lemma to the additive prior case.
    Considering the trial space defined in \eqref{eq:Vh_add}, the numerical solution $u_h^+$ is given by
    \[
        u_h^+=u_{\theta}+p_h^+,
    \]
    with $p_h^+ \in V_h^0 \subset V$ solution to \eqref{eq:approachform_add}.
    We have
    \begin{alignat}{3}
        \notag
        a(u-u_h^+, u-u_h^+)
        % & =
        %a\big(u-u_h^+,(u-u_{\theta})-p_h^+\big) \\
        %\notag
         & =
        a\big(u-u_h^+,(u-u_{\theta})-p_h^+ - v_h+ v_h\big),
         &   & \quad \forall v_h \in V_h^0  \\
        \label{eq:proof_additive_1}
         & =
        a\big(u-u_h^+,(u-u_{\theta})-v_h\big)  +
        a\big(u-u_h^+, v_h-p_h^+\big),
         &   & \quad \forall v_h \in V_h^0.
    \end{alignat}

    Let us first estimate the second term on the right-hand side of \eqref{eq:proof_additive_1}.
    Using the fact that $V_h^0\subset V^0$, we have, by Galerkin orthogonality (difference of the continuous problem \eqref{eq:weakform} and discrete problem \eqref{eq:add1}),
    \begin{equation}\label{eq:orth-gal+}
        a(u-u_h^+, z_h)=0, \quad \forall z_h \in V_h^0.
    \end{equation}
    %   The above equality is valid for all $v_h \in V_h$, and $v_h-p_h^+ \in V_h$.
    Therefore, for  $z_h=v_h-p_h^+\in V_h^0$, we obtain
    \[
        a\big(u-u_h^+, v_h-p_h^+\big)=0, \quad \forall v_h \in V_h^0.
    \]
    Plugging this equality into \eqref{eq:proof_additive_1} yields
    \[
        a(u-u_h^+, u-u_h^+)=a\big(u-u_h^+,(u-u_{\theta})-v_h\big), \quad \forall v_h \in V_h^0.
    \]

    Denoting by $\alpha$ and $\gamma$ the
    coercivity and continuity constants of the bilinear form $a$, we have
    \begin{alignat*}{3}
        \alpha \big| u-u_h^+\big|_{H^1}^2 & \leq
        a\big(u-u_h^+,u-u_h^+\big) =
        a\big(u-u_h^+,(u-u_{\theta})-v_h\big),
                                          &      & \quad \forall v_h \in V_h^0, \\
                                          & \leq
        \gamma \big| u-u_h^+\big|_{H^1} \big| (u-u_{\theta})-v_h \big|_{H^1},
                                          &      & \quad \forall v_h \in V_h^0,
    \end{alignat*}
    which immediately leads to
    \[
        | u-u_h^+|_{H^1} \leq \frac{\gamma}{\alpha} \big| (u-u_{\theta})-v_h \big|_{H^1}, \quad \forall v_h \in V_h^0.
    \]
    Since the above relation is valid for all $v_h \in V_h^0$,
    we apply it to $v_h=\mathcal{I}_h(u-u_\theta) \in V_h^0$
    with $\mathcal{I}_h$ the Lagrange interpolation operator \eqref{eq:Ih} in $V_h$, it holds
    using interpolation estimate given in \cref{th:interpol},
    %classical interpolation results yields, see, e.g., \cite{Ern2004TheoryAP},
    \[
        |u-u_h^+|_{H^1} \leq C_q\frac{\gamma}{\alpha} h^{q} | u-u_{\theta} |_{H^{q+1}},
    \]
    with $C_q$ defined in \cref{sec:FEM}.

    %Rewriting the above expression and introducing the error associated with the classical FEM given in \cref{thm:classical_error_estimate}, we obtain
    The above expression can be rewritten as
    \begin{equation}\label{eq:trucbidule}
        | u-u_h^+|_{H^1} \leq C_q\frac{\gamma}{\alpha}  C_\text{\rm gain}^+ \, h^{q}|u|_{H^{q+1}} \,,
    \end{equation}
    with
    \[
        C_\text{\rm gain}^+ = \frac{| u-u_{\theta} |_{H^{q+1}}}{| u |_{H^{q+1}}},
    \]
    which completes the first part of the proof.

    \textbf{$L^2$-error:}
    We will follow the Aubin-Nitsche technique.
    Consider $w\in H^2(\Omega)$ the solution to
    $$\mathcal{L}^* w=u-u_h^+,$$
    with homogeneous Dirichlet boundary condition.
    Thanks to \cref{th:ellip}, one has
    \begin{equation}\label{eq:wH2}
        \|w\|_{H^2}\leqslant C_e\|u-u_h^+\|_{L^2}.
    \end{equation}
    Using the Galerkin orthogonality \eqref{eq:orth-gal+} and the continuity of the bilinear form $a$,
    $$\|u-u_h^+\|_{L^2}^2
        = a(u-u_h^+,w-I_hw)
        \leq \gamma |u-u_h^+|_{H_1}|w-I_hw|_{H_1}.$$
    Thanks to \cref{th:interpol}
    and \eqref{eq:wH2},
    $$|w-I_hw|_{H_1}\leq C_eC_1\|u-u_h^+\|_{L^2},$$
    which leads to the conclusion by using \eqref{eq:trucbidule}.
\end{proof}

\begin{remark}
    \label{rmk:C_gain_additif}
    The gain constant $C_\text{\rm gain}^+$ defined in \eqref{eq:gain_add} shows that the closer the prior is to the solution,
    the smaller is the error constant associated with the FEM while keeping the same order of accuracy.
    Therefore, as soon as \smash{$C_\text{\rm gain}^+ < 1$}, the FEM with additive prior will be more accurate than the classical one.
    While this gives us a particularly flexible constraint, our objective is to balance this gain by relaxing the contribution $h^q$, using a coarser grid and low-order polynomial, to reduce the computational cost of the FEM while maintaining accuracy.
    Nonetheless, the gain is related to the~$L^2$ error associated with the derivatives of $(q+1)$\textsuperscript{th} order (with $1\leqslant q\leqslant k$) of the prior.
    This shows that the prior must accurately approximate the derivatives of the solution in addition to the solution itself.
    This highlights that we need to build our prior by ensuring a good approximation of the derivatives of the solution.
    It also shows that the higher the order of the finite elements $k$ is, the better our prior should approximate the higher-order derivatives.
    Therefore, it is more appropriate to use only low-order FEM so that $k$ remains small.
\end{remark}


\section{Enriching the finite element method with multiplicative priors}
\label{sec:multiplicative_prior}

This section employs the same assumptions as in \cref{sec:additive_prior}, namely that we have a sufficiently smooth prior $u_\theta$ on the solution~$u$ of the PDE \eqref{eq:ob_pde}.
However, this prior will now be multiplied to elements of $V_h$ rather than added to them.
We construct the underlying modified problem in \cref{sec:modified_problem_mul}.
Then, similarly to the additive approach of \cref{sec:additive_prior}, error estimates are obtained in \cref{sec:error_estimates_mul}.
\subsection{Construction of the modified problem}
\label{sec:modified_problem_mul}

To construct the modified problem in this case, we must ensure that the prior $u_\theta$ never vanishes.
Therefore, we propose to modify the initial problem~\eqref{eq:ob_pde} and consider in this section the problem defined by
\begin{equation}
	\label{eq:ob_pde_M}
	\begin{dcases}
		\mathcal{L}(u_M)=f, &
		\text{\quad in } \Omega,          \\
		u_M = %g +
		M,                  &
		\text{\quad on } \partial \Omega. \\
	\end{dcases}
\end{equation}
Note that \eqref{eq:ob_pde_M} is nothing but the initial problem~\eqref{eq:ob_pde} lifted by a constant $M \in \R_+$, chosen large enough to ensure that $u_M=u+M>0$.
We then introduce the associated variational problem, defined by
\begin{equation}
	\label{eq:weakform2}
	\text{Find } u_M = u + M,~\text{with }u \in V^0 \text{ such that, }
	\forall v\in V^0,~ a(u_M,v)=l(v).
\end{equation}
Therefore, solving~\eqref{eq:weakform2}, we recover the solution $u$ of the initial problem \eqref{eq:ob_pde} by setting \[u = u_M - M.\]
The prior \[u_{\theta,M}=u_\theta+M>0\] is associated with problem \eqref{eq:weakform2}.

%We are now ready to introduce a new finite element space $V_h^\times$ associated to $\mathbb{P}_k$ polynomials.
%This multiplicatively
Let us introduce the following modified finite element space defined by
\begin{equation}
	\label{eq:Vh_mul}
	V_h^\times = \left\{
	u_{h,M}^\times = u_{\theta,M} \; p_h^\times,
	\quad p_h^\times \in 1+V_h^0
	\right\},
\end{equation}
with, for all $\bm{x}\in \Omega$, $u_{\theta,M}(\bm{x})\neq 0$.
From \eqref{eq:weakform2}, this leads to the following approximate formulation:
\begin{equation}\label{eq:approachform_mul}
	\text{Find } p_h^\times \in 1+ V_h^0 \text{ such that, }
	\quad
	\forall v_h \in V_h^0,
	\quad
	\; a \big(u_{\theta,M} \; p_h^\times,u_{\theta,M}  v_h \big) = l(u_{\theta,M} v_h).
\end{equation}
Therefore, solving \eqref{eq:approachform_mul}, we recover the solution $u_h^\times\in V_h^\times-M$ of the original problem \eqref{eq:ob_pde} by setting $u_h^\times = u_{h,M}^\times - M$.

Based on the $N_{\text{dofs}}$ dofs $\big(\bm{x}^{(i)}\big)_{i \in \{1,\ldots,N_{\text{\rm dofs}}\}}$ of the mesh, we consider the interpolation operator on $V_h^\times$ given by
\begin{equation*}
	\label{eq:Ih_tilde}
	\tilde{\mathcal{I}}_h:
	C^0(\Omega) \ni v \mapsto
	\sum_{i=1}^{N_\text{dofs}}\frac{v\big(\bm{x}^{(i)}\big)}{u_{\theta,M}\big(\bm{x}^{(i)}\big)} \tilde{\psi_i} \in V_h^\times,
\end{equation*}
where the shape functions $\tilde{\psi_i}$ associated to $V_h^\times$ are defined by
\[
	\tilde{\psi_i}={u_{\theta,M}} \; {\psi}_i,
\]
with ${\psi}_i$ the classical shape functions presented in \cref{sec:FEM}.
Note that the new interpolation operator $\tilde{\mathcal{I}}_h$ is related to the classical Lagrange interpolation operator defined in \eqref{eq:Ih} as follows
\begin{equation}
	\label{eq:relation_Ih_Ih_tilde}
	\forall v \in C^0(\Omega), \qquad
	\tilde{\mathcal{I}}_h(v)
	=
	u_{\theta,M} \;
	\mathcal{I}_h \left( \frac v {u_{\theta,M}} \right).
\end{equation}



% \subsection{Tools for the convergence analysis}
% \label{sec:tools_for_error_estimates_mul}

% To make the convergence analysis work,
% we need to prove that $V_h^\times$ is a subspace of $M+V^0$,
% and that, for all \smash{$u_{h,M}^\times \in V_h^\times$}, \smash{$\tilde{\mathcal{I}}_h(u_{h,M}^\times) = u_{h,M}^\times$}.
% Indeed, these two properties are necessary to apply Céa's lemma,
% which is essential to prove convergence.

% \begin{lemma}
% 	Suppose that $u_{\theta,M}\in M+W^{1,\infty}(\Omega)\cap H^1_0(\Omega)$. The space $V_h^\times$ is a vector subspace of $M+V^0=M+H_0^1(\Omega)$.
% \end{lemma}
% \begin{proof}
% 	Since $V_h$ is a vector space, it is clear that $V_h^\times$ is also a vector space.
% 	Therefore, we need to show that $V_h^\times \subset M+V^0$, which is itself a vector space.
% 	To that end, take $u_{h,M}^\times \in V_h^\times$;
% 	then, there exists $p_h^\times \in 1+V_h^0$ such that \smash{$u_{h,M}^\times = u_{\theta,M} p_h^\times$},
% 	and we get
% 	\begin{equation*}
% 		\| u_{h,M}^\times \|_{H_0^1}^2
% 		=
% 		\| p_h^\times \ u_{\theta,M}\|_{H_0^1}^2
% 		\leqslant
% 		\| p_h^\times \ u_{\theta,M}\|_{L^2}^2 +
% 		\| u_{\theta,M}\nabla p_h^\times\|_{L^2}^2 +
% 		\| p_h^\times\nabla u_{\theta,M}\|_{L^2}^2.
% 	\end{equation*}
% 	Since $u_{\theta,M}\in M+W^1_{\infty}(\Omega)$,
% 	we know that there exist $C_1, C_2 > 0$ such that
% 	$\| u_{\theta,M}\|_{\infty} \le C_1$
% 	and  $ \| \nabla u_{\theta,M}\|_{\infty} \le C_2$.
% 	Therefore,
% 	\begin{align*}
% 		\| p_h^\times u_{\theta,M}\|_{H_0^1}^2
% 		 & \le C_1\| p_h^\times \|_{L^2}^2+ C_1\|\nabla p_h^\times\|_{L^2}^2  +C_2\| p_h^\times\|_{L^2}^2 \\
% 		 & \le (C_1+C_2)(\| p_h^\times\|_{L^2}^2 +\|\nabla p_h^\times\|_{L^2}^2)                          \\
% 		 & \lesssim \| p_h^\times\|_{H_0^1}^2.
% 	\end{align*}
% 	Since $p_h^\times \in 1+ V_h^0 \subset 1+ V^0 = 1+ H_0^1(\Omega)$,
% 	its $H^1$ norm is finite,
% 	and hence so is the $H^1$ norm of $u_{h,M}^\times$.
% 	We deduce that $u_{h,M}^\times \in M +V^0$, which leads to the conclusion.
% \end{proof}

% \begin{lemma}
% 	For all $u_{h,M}^\times \in V_h^\times$, $\tilde{\mathcal{I}}_h(u_{h,M}^\times) = u_{h,M}^\times$.
% \end{lemma}


% \begin{proof}
% 	Let us consider $u_{h,M}^\times \in V_h^\times$.
% 	By definition
% 	\eqref{eq:Ih_tilde} of the global interpolant,
% 	we know that
% 	\begin{equation*}
% 		\tilde{\mathcal{I}}_h(u_{h,M}^\times)
% 		=
% 		\sum_{i=1}^{N_\text{dofs}} \frac{u_{h,M}^\times\big(\bm{x}^{(i)}\big)}{u_{\theta,M}\big(\bm{x}^{(i)}\big)} \psi_i u_{\theta,M}.
% 	\end{equation*}
% 	Now, using \eqref{eq:Vh_mul}, there exists $p_h^\times \in 1+ V_h^0$ such that $u_{h,M}^\times = u_{\theta,M} p_h^\times$.
% 	Since $p_h^\times \in 1 + V_h^0$, it satisfies
% 	\begin{equation*}
% 		p_h^\times = \sum_{i=1}^{N_\text{dofs}} p_h^\times\big(\bm{x}^{(i)}\big) \psi_i,
% 	\end{equation*}
% 	and therefore
% 	\begin{equation*}
% 		\tilde{\mathcal{I}}_h(u_{h,M}^\times)
% 		=
% 		u_{\theta,M} \sum_{i=1}^{N_\text{dofs}} p_h^\times\big(\bm{x}^{(i)}\big) \psi_i
% 		=
% 		u_{\theta,M} p_h^\times
% 		=
% 		u_{h,M}^\times,
% 	\end{equation*}
% 	which concludes the proof.
% \end{proof}

\subsection{Convergence analysis}
\label{sec:error_estimates_mul}

In this section, we finally prove that the modified FEM \eqref{eq:approachform_mul} converges to the solution to \eqref{eq:ob_pde_M},
and that it satisfies the same type of estimate as the classical one.
% We focus on the homogeneous boundary Dirichlet problem for the sake of simplicity.
Equipped with the lifting trick in \cref{sec:modified_problem_mul},
we can state the following convergence theorem.

\begin{theorem}
	\label{lem:error_estimate_multiplicative}
	Let $u_M\in H^{q+1}(\Omega)$ be the solution of the enhanced problem \eqref{eq:weakform2}
    and $u_{\theta,M}\in M+H^{q+1}(\Omega)\cap H^1_0(\Omega)$ be a prior on $u_M$.
    We consider $u_{h,M}^\times\in V_h^\times$ the solution to the finite element problem \eqref{eq:approachform_mul} with
    \smash{$V_h^\times$} the modified trial space defined in \eqref{eq:Vh_mul},
    considering $\mathbb{P}_k$ polynomials.
    We define $u=u_M-M$ and $u_h^\times=u_{h,M}^\times-M$.
    Then, for all $1\leqslant q\leqslant k$
	\begin{equation*}
		\label{eq:error_mul}
		| u-u_h^\times|_{H^1} \leq C_q\dfrac{\gamma}{\alpha} C_{\text{\rm gain},H^1}^{\times,M} h^{q}| u |_{H^{q+1}}
	\end{equation*}
	and
	\begin{equation*}
        \label{eq:error_mulL2}
        \| u-u_h^\times\|_{L^2} \leqslant C_e C_1 C_q\dfrac{\gamma^2}{\alpha} C_{\text{\rm gain},L^2}^{\times,M} \, h^{q+1} |u|_{H^{q+1}},
    \end{equation*}
    with $C_e$, $C_1$, $C_q$, $\gamma$, $\alpha$ defined in \cref{sec:FEM}, and where
    \begin{equation}
		\label{eq:gain_mul}
		C_{\text{\rm gain},H^1}^{\times,M} = \left| \frac{u_M}{u_{\theta,M}} \right|_{H^{q+1}} \frac{\| u_{\theta,M}\|_{W^{1,\infty}}}{| u |_{H^{q+1}}},
	\end{equation}
	and
    \begin{equation}
		\label{eq:gain_mulL2}
		C_{\text{\rm gain},L^2}^{\times,M} =
		C_{\theta,M}\left| \frac{u_M}{u_{\theta,M}} \right|_{H^{q+1}} \frac{\| u_{\theta,M}\|_{W^{1,\infty}}^2}{| u |_{H^{q+1}}},
	\end{equation}
	with
    \begin{equation}
		\label{eq:CthetaM}C_{\theta,M}=\|u_{\theta,M}^{-1}\|_{L^{\infty}}
		+2|u_{\theta,M}^{-1}|_{W^{1,\infty}}
		+|u_{\theta,M}^{-1}|_{W^{2,\infty}}.
	\end{equation}
\end{theorem}

\begin{remark}\label{rmk:gain_mul}
The constants $C_{\text{\rm gain},H^1}^{\times,M}$ and $C_{\text{\rm gain},L^2}^{\times,M}$
represents the potential gains in both $H^1$ semi-norm and $L^2$ norm
when using the multiplicative approach, compared
to the error of the classical FEM presented in \cref{thm:classical_error_estimate} with $\mathbb{P}_k$ polynomials.
\end{remark}

\begin{proof}[Proof of \cref{lem:error_estimate_multiplicative}]
	\textbf{$H^1$-error:}
	Considering the trial space defined in \eqref{eq:Vh_mul}, the numerical solution $u_{h,M}^\times$ is given by
	\[
		u_{h,M}^\times=u_{\theta,M} \; p_h^\times,
	\]
	with $p_h^\times\in 1+ V_h^0\subset 1 + V^0$  solution to \eqref{eq:approachform_mul}.
By coercivity of $a$,
		\[
		\alpha| u_M-u_{h,M}^\times|_{H^1}^2
		\leq a(u_M-u_{h,M}^\times,u_M-u_{h,M}^\times) .
	\]
	Thanks to \eqref{eq:weakform} and \eqref{eq:approachform_mul}, we have the following Galerkin orthogonality: for all $v_h\in V_h^0$,
\begin{equation}\label{eq:orthGalmult}
a(	u_M-u_{h,M}^\times, u_{\theta,M}v_h)=0.
\end{equation}
	For $v_h=p_h^{\times} -\mathcal{I}_h\left(\frac{u_M}{u_{\theta,M}}\right)$, we deduce by definition \eqref{eq:relation_Ih_Ih_tilde} of $\tilde{\mathcal{I}}_h$ that
			$$%\begin{multline*}
	a(u_M-u_{h,M}^\times,u_M-u_{h,M}^\times)
		=a\left(u_M-u_{\theta,M}\mathcal{I}_h\left(\frac{u_M}{u_{\theta,M}}\right),u_M-u_{h,M}^\times\right)
				=a(u_M-\tilde{\mathcal{I}}_h(u_M),u_M-u_{h,M}^\times) .
$$
By continuity of $a$,
\begin{equation}\label{eq:tructruc}
		| u_M-u_{h,M}^\times|_{H^1} \leq \frac{\gamma}{\alpha}| u_M-\tilde{\mathcal{I}}_h(u_M)|_{H^1} .
	\end{equation}
	Again, using the definition \eqref{eq:relation_Ih_Ih_tilde} of $\tilde{\mathcal{I}}_h$,
	\[
		|u_M-\tilde{\mathcal{I}}_h(u_M)|_{H^{1}} \leq
		%\frac{m!}{\lfloor \frac{m}{2} \rfloor!^2}
		\|u_{\theta,M}\|_{W^{1,\infty}} \left\|\frac{u_M}{u_{\theta,M}}-\mathcal{I}_h\left(\frac{u_M}{u_{\theta,M}}\right)\right\|_{H^{1}}.
	\]


	Finally, applying interpolation estimate given in \cref{th:interpol}, it holds
	\begin{equation}\label{eq:interpol tilde Ih}
		|u_M-\tilde{\mathcal{I}}_h(u_M)|_{H^1} \leq C_q
		\|u_{\theta,M}\|_{W^{1,\infty}} h^q \left|\frac{u_M}{u_{\theta,M}}\right|_{H^{q+1}} \,,
	\end{equation}
	with $C_q$ defined in \cref{sec:FEM}.
	%Introducing the error \eqref{eq:classical_error_estimate} associated with the classical finite element without prior, we obtain,
	Combining the last inequality with \eqref{eq:tructruc}, we obtain
	\begin{equation}\label{eq:H1 mult}
		| u-u_h^\times|_{H^1} = | u_M-u_{h,M}^\times|_{H^1} \leq C_q \dfrac{\gamma}{\alpha} C_{\text{\rm gain},H^1}^{\times,M} h^{q}| u |_{H^{q+1}},
	\end{equation}
	with $C_{\text{\rm gain},H^1}^{\times,M}$ given in \eqref{eq:gain_mul}, which conclude the first part of the proof.

	\textbf{$L^2$-error:}
	Again, we follow the Aubin-Nitsche strategy here. Consider the problem
	$$\mathcal{L}^*w=u-u_h^{\times}=u_M-p_h^{\times}u_{\theta,M},$$
	with $w=M$ on $\partial\Omega$. Then,  using the Galerkin orthogonality \eqref{eq:orthGalmult} for $v_h=\mathcal{I}_h\left(\frac{u_M}{u_{\theta,M}}\right)$,
	$$\|u-u_h^{\times}\|_{L^2}^2
		=\|u_M-p_h^{\times}u_{\theta,M}\|_{L^2}^2
		=a(u_M-p_h^{\times}u_{\theta,M},w)
		=a(u_M-p_h^{\times}u_{\theta,M},w-\tilde I_h(w)).$$
    Hence, by continuity of $a$,
	$$\|u-u_h^{\times}\|_{L^2}^2 \leqslant \gamma |u_M-p_h^{\times}u_{\theta,M}|_{H^1}|w-\tilde I_h(w)|_{H^1}.$$
	% Using \eqref{eq:interpol tilde Ih} and \eqref{eq:H1 mult} for $q=1$,
Using \eqref{eq:H1 mult} and \eqref{eq:interpol tilde Ih} for $q=1$ to the term in the right hand side,
	\begin{equation*}
		\|u-u_h^{\times}\|_{L^2}^2
		\leqslant C_1 C_q \dfrac{\gamma^2}{\alpha} \|u_{\theta,M}\|_{W^{1,\infty}} \left|\frac{w}{u_{\theta,M}}\right|_{H^{2}} C_{\text{\rm gain},H^1}^{\times,M} h^{q+1}| u |_{H^{q+1}}.
	\end{equation*}
   Moreover
    $$\left|\frac{w}{u_{\theta,M}}\right|_{H^{2}}\leqslant C_{\theta,M}\|w\|_{H^{2}},$$
    with $C_{\theta,M}$ given in \eqref{eq:CthetaM}.
 Thanks to the elliptic regularity, we obtain
    \begin{equation*}
        \| u-u_h^\times\|_{L^2} \leqslant C_e C_1 C_q\dfrac{\gamma^2}{\alpha} C_{\text{\rm gain},L^2}^{\times,M} \, h^{q+1} |u|_{H^{q+1}},
    \end{equation*}
with $C_{\text{\rm gain},L^2}^{\times,M}$ defined in \eqref{eq:gain_mulL2}.
\end{proof}

\begin{remark}
	\label{rmk:C_gain_multiplicatif}
	We note that the gain constants $C_{\text{\rm gain},H^1}^{\times,M}$ and $C_{\text{\rm gain},L^2}^{\times,M}$
	are similar to the constant $C_\text{\rm gain}^+$
	introduced in \cref{sec:additive_prior},
	in that, it depends on high-order derivatives of the prior.
	Hence, a high-quality prior will necessarily involve
	a good approximation of the derivatives of the exact solution,
	and \cref{rmk:C_gain_additif} also applies in the present context.
	The major difference with the additive approach lies
	in the choice of the lifting constant $M$.
	To better understand this dependency in $M$,
	the following section provides a
	study of the behaviour of our two gain constants
	when $M$ goes to infinity.
	Moreover, the actual choice of $M$ will be
	numerically investigated in \cref{sec:numerical_results}.
\end{remark}



\section{Comparison of the two enriched methods}
\label{sec:comparison_add_mul}

This section aims to compare the two approaches proposed to improve the classical finite element method, the additive approach presented in \cref{sec:additive_prior} and the multiplicative approach proposed in this \cref{sec:multiplicative_prior}. Recall that the constant $M$ is chosen in the multiplicative approach so that $u_M>0$.
Let $u$ be the solution of problem~\eqref{eq:weakform} and $u_{\theta}\in H^{q+1}(\Omega)\cap H^1_0(\Omega)$ be a prior on $u$, with $1\le q\le k$ ($k$ the polynomial degree of the finite element method). Let us first recall the two approaches considered in the present paper:

\paragraph*{Additive approach.} We consider $u_h^+\in V_h^+$ as the solution to the finite element method associated to problem~\eqref{eq:approachform_add} with $V_h^+$ the modified trial space defined in \eqref{eq:Vh_add}, considering $\mathbb{P}_k$ polynomials. Using \cref{lem:error_estimation_add}, we have for $1\le q\le k$,
\begin{equation}
	\label{eq:error_add_comp}
	| u-u_h^+|_{H^1} \leq C_q\dfrac{\gamma}{\alpha} C_\text{\rm gain}^+ \, h^{q} |u|_{H^{q+1}}
\end{equation}
and
\begin{equation}
    \label{eq:error_addL2_comp}
    \| u-u_h^+\|_{L^2} \leqslant C_e C_1 C_q\dfrac{\gamma^2}{\alpha} C_\text{\rm gain}^+ \, h^{q+1} |u|_{H^{q+1}},
\end{equation}
with $C_e$, $C_1$, $C_q$, $\gamma$, $\alpha$ defined in \cref{sec:FEM} and
\begin{equation}
	\label{eq:C_gain_add_in_comparison}
	C_\text{\rm gain}^+= \frac{| u-u_{\theta} |_{H^{q+1}}}{| u |_{H^{q+1}}}.
\end{equation}

\paragraph*{Multiplicative approach.} Let $u_M=u+M$ be the solution of the enhanced problem \eqref{eq:weakform2} and $u_{\theta,M}=u_\theta+M$ be a prior on $u_M$. We consider $u_{h,M}^\times\in V_h^\times$ the solution to the finite element  problem~\eqref{eq:approachform_mul} with $\smash{V_h^\times}$ the modified trial space defined in \eqref{eq:Vh_mul}, considering $\mathbb{P}_k$ polynomials. We then recover the solution of the initial problem by taking $u_h^\times=u_{h,M}^\times-M$.
Using \cref{lem:error_estimate_multiplicative}, we have for $1\le q\le k$,
\begin{equation}
	\label{eq:error_mul_comp}
	| u-u_h^\times|_{H^1} \leq C_q\dfrac{\gamma}{\alpha} C_{\text{\rm gain},H^1}^{\times,M} h^{q}| u |_{H^{q+1}}
\end{equation}
and
\begin{equation}
    \label{eq:error_mulL2_comp}
    \| u-u_h^\times\|_{L^2} \leqslant C_e C_1 C_q\dfrac{\gamma^2}{\alpha} C_{\text{\rm gain},L^2}^{\times,M} \, h^{q+1} |u|_{H^{q+1}},
\end{equation}
with $C_e$, $C_1$, $C_q$, $\gamma$, $\alpha$ defined in \cref{sec:FEM}, and
\begin{equation}
    \label{eq:C_gain_mul_in_comparison}
    C_{\text{\rm gain},H^1}^{\times,M} = \left| \frac{u_M}{u_{\theta,M}} \right|_{H^{q+1}} \frac{\| u_{\theta,M}\|_{W^{1,\infty}}}{| u |_{H^{q+1}}},
\end{equation}
and
\begin{equation}
    \label{eq:gain_mulL2_in_comparison}
    C_{\text{\rm gain},L^2}^{\times,M} =
    C_{\theta,M}\left| \frac{u_M}{u_{\theta,M}} \right|_{H^{q+1}} \frac{\| u_{\theta,M}\|_{W^{1,\infty}}^2}{| u |_{H^{q+1}}},
\end{equation}
with $C_{\theta,M}$ given in \eqref{eq:CthetaM}.

\paragraph*{Comparison of the two approaches.} The following result proves that the upper bound in \eqref{eq:error_mul_comp} and in \eqref{eq:error_mulL2_comp} converges to the one in \eqref{eq:error_add_comp} when $M$ goes to infinity. In other words, the multiplicative gain constants defined in \eqref{eq:C_gain_mul_in_comparison} and \eqref{eq:gain_mulL2_in_comparison} converge to the additive gain constant defined in \eqref{eq:C_gain_add_in_comparison}.

\begin{theorem}\label{thm:comparison_add_mul}
	We have
	\begin{equation}\label{eq:convCH1}
        C_{\text{\rm gain},H^1}^{\times,M}
		\underset{M\rightarrow\infty}{\longrightarrow}
		C^{+}_{\text{\rm gain}}
	\end{equation}
	and
	\begin{equation*}\label{eq:convCL2}
		C_{\text{\rm gain},L^2}^{\times,M}
		\underset{M\rightarrow\infty}{\longrightarrow}
		C^{+}_{\text{\rm gain}}.
	\end{equation*}
\end{theorem}


\begin{proof}
\textbf{Convergence in $H^1$ theoretical gain:}
	According to the expressions \eqref{eq:C_gain_add_in_comparison} and \eqref{eq:C_gain_mul_in_comparison}
	of the gain constants, the objective of the proof is to show that
	\begin{equation*}
		\|u_{\theta,M}\|_{W^{1,\infty}} \left|\frac{u_M}{u_{\theta,M}}\right|_{H^{q+1}} \underset{M\rightarrow\infty}{\longrightarrow} |u-u_\theta|_{H^{q+1}}.
	\end{equation*}
	Denoting by
	\begin{equation*}
		\label{eq:prior_error_approximation}
		E_\theta=u - u_{\theta},
	\end{equation*}
	the error made by the prior $u_{\theta}$
	when approximaing the solution $u$, we have
	\begin{equation*}
		|u-u_\theta|_{H^{q+1}} = |E_\theta|_{H^{q+1}}.
	\end{equation*}
	On the one hand, we have that,
	\begin{equation*}
		\left\|u_{\theta,M}\right\|_{W^{1,\infty}}
		=
		\left\|u_\theta + M \right\|_{W^{1,\infty}}
		=
		M
		\left\|1 + \frac {u_\theta} M \right\|_{W^{1,\infty}}.
	\end{equation*}
	On the other hand, we have,
	\begin{equation*}
		\left|\frac{u_M}{u_{\theta,M}}\right|_{H^{q+1}}
		=
		\left|\frac{u + M}{u_\theta + M}\right|_{H^{q+1}}
		=
		\left|\frac{u - u_\theta + u_\theta + M}{u_\theta + M}\right|_{H^{q+1}}
		=
		\left|1 + \frac{u - u_\theta}{u_\theta + M}\right|_{H^{q+1}}
		=
		\dfrac{1}{M}\left| \frac{E_\theta}{1 + \frac {u_\theta} M}\right|_{H^{q+1}}.
	\end{equation*}
	Multiplying these expressions, we obtain
	\begin{equation}
		\label{eq:error_mul_q2}
		\|u_{\theta,M}\|_{W^{1,\infty}} \left|\frac{u_M}{u_{\theta,M}}\right|_{H^{q+1}}= \underbrace{\left\|1+\frac{u_\theta}{M}\right\|_{W^{1,\infty}}}_{\MakeUppercase{\romannumeral 1}} \;
		\underbrace{\left|\frac{E_\theta}{1+\frac{u_\theta}{M}}\right|_{H^{q+1}}}_{\MakeUppercase{\romannumeral 2}}.
	\end{equation}

	We now estimate term by term the right-hand side of the
	above equality \eqref{eq:error_mul_q2}, looking at their
	limits when $M$ goes to infinity.

	\textbf{Term \MakeUppercase{\romannumeral 1}:} By decomposing the first term, we obtain
	\[
		(\MakeUppercase{\romannumeral 1})=\left\|1+\frac{u_\theta}{M}\right\|_{W^{1,\infty}}=\left\|1+\frac{u_\theta}{M}\right\|_{L^\infty} + \frac{1}{M}\|\nabla u_\theta\|_{L^\infty}\underset{M\to\infty}{\longrightarrow} 1,
	\]
	since
	\begin{equation}\label{eq:plus_d_idee_de_label}
		1-\frac{\|u_\theta\|_{\infty}}{M}
		\le
		\left\|1+\frac{u_\theta}{M}\right\|_{L^\infty}
		\le
		1+\frac{\|u_\theta\|_{\infty}}{M}.
	\end{equation}


	\textbf{Term \MakeUppercase{\romannumeral 2}:}
	Let us prove that
	\begin{equation*}
		%\left|M + \frac{E_\theta}{1+\frac{u_\theta}{M}}\right|_{H^{q+1}}
		%=
	(\MakeUppercase{\romannumeral 2})=	\left|\frac{E_\theta}{1+\frac{u_\theta}{M}}\right|_{H^{q+1}}\underset{M\rightarrow\infty}{\longrightarrow} |E_\theta|_{H^{q+1}}.
	\end{equation*}
	We have
	\begin{align*}
		\left|\left|\frac{E_\theta}{1+\frac{u_\theta}{M}}\right|_{H^{q+1}}^2-|E_\theta|_{H^{q+1}}^2\right| & \le \left|\frac{E_\theta}{1+\frac{u_\theta}{M}}-E_\theta\right|_{H^{q+1}}^2                                    \\
		& = \left\|\nabla^{q+1}\left(\frac{E_\theta}{1+\frac{u_\theta}{M}}-E_\theta\right)\right\|_{L^2}^2
		\leqslant |\Omega| \left\|\nabla^{q+1}\left(\frac{E_\theta}{1+\frac{u_\theta}{M}}-E_\theta\right)\right\|_{L^\infty}^2.
	\end{align*}
	Then, using the general Leibniz rule, we have,


    \begin{align*}
		\left\|\nabla^{q+1}\left(\frac{E_\theta}{1+\frac{u_\theta}{M}}-E_\theta\right)\right\|_{L^\infty} & \leqslant
		\left\|\frac{\nabla^{q+1}E_\theta}{1+\frac{u_\theta}{M}}-\nabla^{q+1}E_\theta\right\|_{L^\infty}
		+\sum_{s=1}^{q+1} \begin{pmatrix} q+1 \\ s \end{pmatrix}
		\left\|\nabla^{q+1-s}E_\theta\right\|_{L^{\infty}}\left\|\nabla^{s}\left(\frac{1}{1+\frac{u_\theta}{M}}\right)\right\|_{L^\infty} \\
		& \leqslant
		\underbrace{ \left\|\frac{\nabla^{q+1}E_\theta}{1+\frac{u_\theta}{M}}-\nabla^{q+1}E_\theta\right\|_{L^\infty} }_{(1)}
		+\|E_\theta\|_{W^{q+1,{\infty}}} \sum_{s=1}^{q+1} \begin{pmatrix} q+1 \\ s \end{pmatrix}
		\underbrace{ \left\|\nabla^{s}\left(\frac{1}{1+\frac{u_\theta}{M}}\right)\right\|_{L^\infty} }_{(2)} .
	\end{align*}

	We now estimate terms (1) and (2) in the right-hand side of the above inequality.


	\textbf{Term (1):}
	Taking $M> \|u_{\theta}\|_{L^\infty}$, we obtain
	\begin{align*}
		%\left\|\frac{\nabla^{q+1}E_\theta}{1+\frac{u_\theta}{M}}-\nabla^{q+1}E_\theta\right\|_{L^\infty}
		(1)\le	\left\|\frac{1}{1+\frac{u_\theta}{M}}-1\right\|_{L^\infty}\|\nabla^{q+1}E_\theta\|_{L^\infty}
		& \le \frac{1}{M} \frac{\|u_\theta\|_{L^\infty}}{1-\frac{\|u_\theta\|_{L^\infty}}{M}} \|\nabla^{q+1}E_\theta\|_{L^\infty} \underset{M\to\infty}{\longrightarrow}0.
	\end{align*}

	\textbf{Term (2):} Using the Fa\`{a} di Bruno formula and \eqref{eq:plus_d_idee_de_label}, there exists a constant $C>0$ such that, for any $M> \|u_{\theta}\|_{L^\infty}$ and $1\le s\le q+1$,
	the following estimate holds
	\begin{equation*}
	%	\left\|\nabla^{s}\left(\frac{1}{1+\frac{u_\theta}{M}}\right)\right\|_{L^\infty}
	(2)	\leqslant C\frac{\prod_{m=0}^s\|\nabla^m\left(1+\frac{u_\theta}{M}\right)\|_{L^{\infty}}}{\left\|1+\frac{u_\theta}{M}\right\|_{\infty}^{s+1}}
\leqslant C\dfrac{1}{M^{s}}\frac{\left(1+\frac{\|u_\theta\|_{\infty}}{M}\right)\prod_{m=1}^s\|\nabla^m u_\theta\|_{L^{\infty}}^{s}}{\left(1-\frac{\|u_\theta\|_{\infty}}{M}\right)^{s+1}}
		\underset{M\to\infty}{\longrightarrow}0,
	\end{equation*}
	which leads to \eqref{eq:convCH1}.

\textbf{Convergence in $L^2$ theoretical gain:} Using the convergence of the gain for the $H^1$ semi-norm, we only lead to prove that
$$C_{\theta,M}\|u_{\theta,M}\|_{W^{1,\infty}}
\underset{M\rightarrow\infty}{\longrightarrow}1,$$
with $C_{\theta,M}$ given by
$$C_{\theta,M}=\|u_{\theta,M}^{-1}\|_{L^{\infty}}
		+2|u_{\theta,M}^{-1}|_{W^{1,\infty}}
		+|u_{\theta,M}^{-1}|_{W^{2,\infty}}.$$
Since
	$$\dfrac{1}{M}\|u_{\theta,M}\|_{W^{1,\infty}}\underset{M\rightarrow\infty}{\longrightarrow}1,$$
	let us only prove that
$$MC_{\theta,M}
\underset{M\rightarrow\infty}{\longrightarrow}1.$$
% 	We have
% $$M\|u_{\theta,M}^{-1}\|_{L^{\infty}}
% =\left\|\dfrac{M}{u_{\theta}+M}\right\|_{L^{\infty}}
% \leqslant 1+\left\|\dfrac{u_{\theta}}{u_{\theta}+M}\right\|_{L^{\infty}}
% \underset{M\rightarrow\infty}{\longrightarrow}1.$$
Considering $M> \|u_{\theta}\|_{L^\infty}$, we have
$$M\|u_{\theta,M}^{-1}\|_{L^{\infty}}
=\left\|\dfrac{1}{1+\dfrac{u_\theta}{M}}\right\|_{L^{\infty}}
\leqslant \dfrac{1}{1-\dfrac{\|u_\theta\|_{L^\infty}}{M}}
\underset{M\rightarrow\infty}{\longrightarrow}1.$$
Moreover
$$2M|u_{\theta,M}^{-1}|_{W^{1,\infty}}
=2M\left\|\dfrac{\nabla u_{\theta}}{(u_{\theta}+M)^2}\right\|_{L^{\infty}}
\underset{M\rightarrow\infty}{\longrightarrow}0.$$
Similarly
$$M|u_{\theta,M}^{-1}|_{W^{2,\infty}}
%=\left\|\dfrac{\nabla u_{\theta}}{(u_{\theta}+M)}\right\|_{L^{\infty}}
\underset{M\rightarrow\infty}{\longrightarrow}0,$$
which leads to the conclusion.

\end{proof}


\section{Prior construction using parametric PINNs}
\label{sec:prior_construction}


We have introduced new finite element approximation spaces in \cref{sec:additive_prior,sec:multiplicative_prior} depending on the construction of priors.
Physics-Informed Neural Networks (PINNs) are a good choice to build such priors.
Indeed, since PINNs minimize the PDE residual, they inherently give a good approximation of the derivative of the solution, in addition to the solution itself (see e.g. \cite{RAISSI2019686}).
This section is therefore dedicated to introducing PINNs in \cref{sec:PINNs_parametric_PDE}, and then to show how to improve them in \cref{sec:improve_PINNs}.

\subsection{Physics-Informed Neural Networks for parametric PDEs}
\label{sec:PINNs_parametric_PDE}

Physics-Informed Neural Networks, or PINNs, were introduced by \cite{RAISSI2019686} %\cite{raissi2017physicsinformeddeeplearning}
for solving a PDE with Neural Networks.
The main idea is to recast a PDE as an optimization problem.
We illustrate the method on our problem~\eqref{eq:ob_pde}, which we extend, in this section, to non-homogeneous Dirichlet boundary conditions.
Moreover, we introduce a dependency on some physical parameters, making the problem of interest a parametric PDE.
Unlike classical PINNs, which train for a specific case of boundary conditions or physical parameters, parametric PINNs seek to learn a generalized solution covering a range of parameters.
They incorporate these parameters as additional inputs to the network, allowing greater flexibility in solving problems where physical conditions or properties vary.
Moreover, since they are based on a combination of neural networks and the Monte-Carlo method, PINNs are ideally suited to solving such higher-dimensional problems.

Considering $p$ parameters \smash{$\bm{\mu} = (\mu_1, \dots, \mu_{p}) \in \mathcal{M} \subset \mathbb{R}^{p}$}, with some parameter space $\mathcal{M}$, the parametric PDE reads
\begin{equation}
	\label{eq:parametric_PDE}
	\begin{dcases}
		\mathcal{L}\big(u(\bm{x},\bm{\mu});\bm{x},\bm{\mu}\big) = f(\bm{x},\bm{\mu}), & \bm{x}\in\Omega, \\
		u(\bm{x},\bm{\mu}) = g(\bm{x},\bm{\mu}),             & \bm{x}\in\partial \Omega, \\
	\end{dcases}
\end{equation}
with $g$ the trace of a $H^2$ function on $\partial \Omega$.
Note that the solution of the equation depends on the parameters~$\bm{\mu}$, as do the operator $\mathcal{L}$ and the boundary conditions.
This can be contrasted to the solution of \eqref{eq:ob_pde}, which only depended on the space variable $\bm{x}$.
We then denote $u_{\theta}(\cdot, \bm{\mu})$ the approximate PINN prediction for given parameters~$\bm{\mu}$.

The first idea of PINNs comes from the observation that, by construction, neural networks with smooth activation functions
are nothing but smooth functions of their weights and inputs.
Therefore, neural networks form natural candidates for approximating solutions to PDEs, especially with the advent of automatic differentiation tools.
In our case, a PINN is a neural network that takes $d+p$ inputs, where $d$ is the dimension of the space variable $\bm{x} \in \Omega$ and $p$ is the number of parameters $\bm{\mu} \in \mathcal{M}$. We denote by $u_{\theta}(\bm{x},\bm{\mu})$ the output, where $\theta$ are the learnable weights of the network.
Classically, this neural network is a coordinate-based neural network, such as a multi-layer perceptron (MLP).

\begin{remark}\label{rmk:PINN_notations}
	In \cref{sec:numerical_results}, we need some notations describing how this PINN has been constructed. In particular, we will note $\sigma$ the MLP activation function, and \textit{layers} will describe an integer sequence describing the number of neurons associated with each layer of the MLP. For training, we will note \textit{lr} for the learning rate and $n_\text{epochs}$ the number of epochs considered, as well as \textit{decay} the multiplicative factor of the learning rate decay considered every $20$ epochs thanks to Pytorch's StepLR scheduler. Where this is not specified, the batch size will correspond to the number of collocation points chosen. The Adam optimizer~\cite{KinBa2015} will be considered during training, but in some cases, we will switch to the LBFGS optimizer~\cite{nocedal_quasi_newton_2006} at the $n_\text{switch}$-th epoch.
\end{remark}

Once this network is defined,
solving the PDE can be rewritten as a minimization problem on $\theta$,
namely finding the optimal weights $\theta^\star$
that satisfy the following minimization problem:
\begin{equation}
	\label{eq:minimization_problem}
	\theta^\star = \argmin_{\theta}
	\big( \omega_r J_r(\theta) + \omega_b J_b(\theta) + \omega_\text{data} J_\text{data}(\theta) \big),
\end{equation}
with $\omega_r$, $\omega_b$ and $\omega_\text{data}$ some weights to balance the different terms of the loss function.

In \eqref{eq:minimization_problem}, the loss function owns three terms: the residual loss function
\begin{equation}
	\label{eq:residual_loss_parametric}
	J_r(\theta) =
	\int_{\mathcal{M}}\int_{\Omega}
	\big| \mathcal{L}\big(u_\theta(\bm{x},\bm{\mu});\bm{x},\bm{\mu}\big)-f(\bm{x},\bm{\mu}) \big|^2 d\bm{x} d\bm{\mu},
\end{equation}
the boundary loss function
\begin{equation}
	\label{eq:boundary_loss_parametric}
	J_b(\theta) =
	\int_{\mathcal{M}}\int_{\partial \Omega} \big| u_\theta(\bm{x},\bm{\mu}) - g(\bm{x},\bm{\mu}) \big|^2 d\bm{x} d\bm{\mu},
\end{equation}
and the data loss function
\begin{equation}
	\label{eq:data_loss_parametric}
	J_\text{data}(\theta) =
	\frac 1 {N_\text{data}} \sum_{i=1}^{N_\text{data}} \big| u_\theta\big(\bm{x}_\text{data}^{(i)},\bm{\mu}_\text{data}^{(i)}\big) - u_\text{data}^{(i)} \big|^2,
\end{equation}
where \smash{$\big(\bm{x}_\text{data}^{(i)}, \bm{\mu}_\text{data}^{(i)}, u_\text{data}^{(i)}\big)_{i=1,\dots,N_\text{data}}$} are $N_\text{data}$ known data points with $u_\text{data}^{(i)}$ a reference solution at point \smash{$\bm{x}_\text{data}^{(i)}$} and for parameters \smash{$\bm{\mu}_\text{data}^{(i)}$}.

\begin{remark}
	These reference solutions can be the exact solutions of the parametric PDE, in this case, defined by \smash{$u_\text{data}^{(i)}=u\big(\bm{x}_\text{data}^{(i)};\bm{\mu}_\text{data}^{(i)}\big)$}. They can also be an approximation produced by a numerical method, such as finite elements on a fine mesh.
\end{remark}

\begin{remark}
	In \cref{sec:numerical_results}, the focus is on PINNs trained only on residual loss (with boundary conditions imposed exactly as presented in \cref{sec:exact_imposition_of_BC}). We will only consider BC loss in \cref{sec:Lap2Dlowbc}. Furthermore, we will not use loss data in PINNs except in \cref{sec:Lap1D} where we will seek to compare a full PINN to a network trained only on data.
\end{remark}

Solving the minimization problem \eqref{eq:minimization_problem}
requires computing the gradient of the loss function with respect to~$\theta$, which involves calculating the integrals
in \eqref{eq:residual_loss_parametric} and \eqref{eq:boundary_loss_parametric}.
The most natural idea is to estimate them with a Monte-Carlo method, see e.g.~\cite{Caf1998}.
One could also use Gauss-type quadrature rules to evaluate integrals, as is done in Variational Physics-Informed
Neural Networks \cite{KhaZhaKar2021}, but the limitation is the impossibility of selecting an adequate
quadrature order due to the unknown properties of the Neural Network approximation.
For that purpose, we define so-called ``collocation points'' on $\Omega\times\mathcal{M}$ and its boundary $\partial\Omega\times \mathcal{M}$,
denoted respectively by \smash{$\big(\bm{x}_\text{col}^{(i)}, \bm{\mu}_\text{col}^{(i)}\big)_{i=1,\dots,N_\text{col}}$} and \smash{$\big(\bm{x}_\text{bc}^{(i)}, \bm{\mu}_\text{bc}^{(i)}\big)_{i=1,\dots,N_\text{bc}}$}.
Then, we approximate the residuals and boundary losses by
\begin{equation*}\label{eq:residual_loss_parametric_MC}
	J_r(\theta) \simeq
	\frac{1}{N_\text{col}} \sum_{i=1}^{N_\text{col}} \big| \mathcal{L}\big(u_\theta(\bm{x}_\text{col}^{(i)},\bm{\mu}_\text{col}^{(i)});\bm{x}_\text{col}^{(i)},\bm{\mu}_\text{col}^{(i)}\big)-f(\bm{x}_\text{col}^{(i)},\bm{\mu}_\text{col}^{(i)})  \big|^2
\end{equation*}
and
\[
	J_b(\theta) \simeq
	\frac{1}{N_\text{bc}} \sum_{i=1}^{N_\text{bc}} \big| u_\theta\big(\bm{x}_\text{bc}^{(i)},\bm{\mu}_\text{bc}^{(i)}\big) - g\big(\bm{x}_\text{bc}^{(i)},\bm{\mu}_\text{bc}^{(i)}\big) \big|^2.
\]

\begin{remark}
	The values of $N_{\mathrm{col}}$ and $N_\mathrm{bc}$
	are heuristically determined and should be large enough to ensure that the Monte-Carlo integration is accurate enough.
	The precise values of these parameters will be given in the numerical experiments.
\end{remark}

\begin{remark}
	In the case of complex geometries, one solution for obtaining a sample of points in the $\Omega$ domain is to use a level-set function, denoted $\varphi$. This function, which vanishes on the boundary of $\Omega$, can be obtained differently. The authors of \cite{Sukumar_2022} propose different approaches to obtain a level-set function analytically in the case of polygonal or curved geometries. Learning-based approaches have also been proposed, notably in e.g. \cite{park2019deepsdflearningcontinuoussigned,sitzmann2020implicitneuralrepresentationsperiodic}.
\end{remark}

Because of the minimization problem \eqref{eq:minimization_problem}, the PINN $u_\theta$ does not exactly satisfy the boundary conditions.
Moreover, loss functions compete, which may require fine-tuning the coefficients between $J_r$ and $J_b$.
In addition, classical PINNs do not include information on higher-order derivatives.
As highlighted in \cref{rmk:C_gain_additif,rmk:C_gain_multiplicatif}, for our purposes, a good prior should yield a good approximation of the derivatives of the solution.
%Lastly, it is well-known that MLPs have a spectral bias:
%learning high-frequency solutions requires large networks
%and long training times.
For these reasons, the following section recalls several improvements of classical PINNs in the literature.

\subsection{Improving PINN training and prediction}\label{sec:improve_PINNs}

This section focuses on several ways of improving PINNs:
exactly imposing the boundary conditions in \cref{sec:exact_imposition_of_BC},
adding a higher-order derivative term in the loss function in \cref{sec:sobolev_training},
and countering the spectral bias in \cref{sec:spectral_bias}.
Although these approaches are presented separately,
they can easily be combined with one another.

\subsubsection{Exact imposition of boundary conditions}\label{sec:exact_imposition_of_BC}

To avoid the issues of classical PINNs discussed in \cref{sec:PINNs_parametric_PDE},
the authors of \cite{LagLikFot1998,FraMicNav2024}
propose a method to enforce inhomogeneous Dirichlet boundary conditions exactly.
To that end, they search the approximation $u_{\theta}$ of solution to \eqref{eq:parametric_PDE} with the form:
for all $\bm{x} \in \Omega$ and $\bm{\mu} \in \mathcal{M}$
\[
	u_{\theta}(\bm{x},\bm{\mu}) = \varphi(\bm{x}) w_{\theta}(\bm{x},\bm{\mu}) + g(\bm{x},\bm{\mu}),
\]
where $\varphi$ and $w_\theta$ are, respectively, the level-set function and a neural network as defined in \cref{sec:PINNs_parametric_PDE}.
Thus $u_{\theta}$ will automatically satisfies the boundary conditions, since
$u_{\theta}(\bm{x},\bm{\mu}) = g(\bm{x},\bm{\mu})$ for all $\bm{x} \in \partial \Omega$ and $\bm{\mu} \in \mathcal{M}$.

% they apply some functions to the PINN solution $w_\theta$,
%resulting in a new approximation $u_{\theta}$
%of the solution to problem \eqref{eq:parametric_PDE},
%that automatically satisfies the boundary conditions.
%Usually, the prior $u_{\theta}$ may be defined for all $\bm{x} \in \Omega$ and $\bm{\mu} \in \mathcal{M}$ as
%\[
%	u_{\theta}(\bm{x},\bm{\mu}) = \varphi(\bm{x}) w_{\theta}(\bm{x},\bm{\mu}) + g(\bm{x},\bm{\mu}),
%\]
%where $\varphi$ and $w_\theta$ are the level-set function and the neural network defined in \cref{sec:PINNs_parametric_PDE}.

\begin{remark}
	Note that this level-set function can be used in a few different ways, firstly to generate a sample of points in $\Omega$, as shown in \cref{sec:PINNs_parametric_PDE}, and secondly to impose boundary conditions. However, to use it directly in the formulation of the prior, it will require a certain regularity. For example, the signed distance function is not a good candidate.
\end{remark}

%Then, $u_{\theta}$ obviously satisfies the boundary conditions since
%$u_{\theta}(\bm{x},\bm{\mu}) = g(\bm{x},\bm{\mu})$ for all $\bm{x} \in \partial \Omega$ and $\bm{\mu} \in \mathcal{M}$.
In this case, only the residual and data loss functions are minimized,
and the minimization problem~\eqref{eq:minimization_problem} becomes
\begin{equation*}
	\label{eq:minimization_problem_without_lossBC}
	\theta^\star = \argmin_{\theta}
	\big( \omega_r J_r(\theta) + \omega_\text{data} J_\text{data}(\theta) \big),
\end{equation*}
with $\omega_r$ and $\omega_\text{data}$ some weights to balance the terms of the loss function.

\begin{remark}
	Similar methods exist for Robin and Neumann conditions; see \cite{Sukumar_2022}.
\end{remark}


\subsubsection{Sobolev training for PINNs}\label{sec:sobolev_training}

As presented in \cref{sec:PINNs_parametric_PDE}, PINNs approximate the PDE solution by directly incorporating the equations into their training. Despite their effectiveness, these models can sometimes struggle to learn correctly, especially when the solution or its derivatives are complicated.
The authors of \cite{son2021sobolevtrainingphysicsinformed} have proposed an approach called Sobolev training to try and overcome these difficulties. This method simply imposes constraints not only on the solutions themselves, but also on their derivatives.
% By taking this additional information into account, the model learns more effectively to capture the complex behaviors of solutions and their variations.
% The authors have shown that this approach can be used to guide network learning, particularly in PDEs such as the heat equation, Burgers’ equation, the Fokker–Planck equation and high-dimensional Poisson equation.
In the context of solving the problem \eqref{eq:parametric_PDE} under consideration, Sobolev training is applied by adding a cost term $J_\text{\rm sob}$ to the initial minimization problem \eqref{eq:minimization_problem}:
\begin{equation}
	\label{eq:minimization_problem_sobolev}
	\theta^\star = \argmin_{\theta}
	\big( \omega_r J_r(\theta) + \omega_\text{\rm sob} J_\text{\rm sob}(\theta) + \omega_b J_b(\theta) + \omega_\text{data} J_\text{data}(\theta) \big),
\end{equation}
with $J_r$, $J_b$ and $J_\text{data}$ defined as in \eqref{eq:residual_loss_parametric}, \eqref{eq:boundary_loss_parametric} and \eqref{eq:data_loss_parametric} respectively and $\omega_r$, $\omega_\text{\rm sob}$, $\omega_b$ and $\omega_\text{data}$ the weights to balance the different terms of the loss function.
The Sobolev loss function $J_\text{\rm sob}$ in \eqref{eq:minimization_problem_sobolev} is defined by
\begin{equation}
	\label{eq:sobolev_loss}
	J_\text{\rm sob}(\theta) = \int_{\mathcal{M}}\int_{\Omega} |\nabla_{\bm{x}}\big( \mathcal{L}\big(u_\theta(\bm{x},\bm{\mu});\bm{x},\bm{\mu}\big)-f(\bm{x},\bm{\mu})\big)|^2%\cdot\nabla_{\bm{x}}\big( \mathcal{L}\big(u_\theta(\bm{x},\bm{\mu});\bm{x},\bm{\mu}\big)-f(\bm{x},\bm{\mu})\big)
	 \; d\bm{x} d\bm{\mu},
\end{equation}
where the integral is estimated by the Monte-Carlo method, similar to the other loss functions.

\subsubsection{Overcoming the spectral bias}
\label{sec:spectral_bias}

Multiple ways of overcoming the spectral bias of MLPs are available.
For instance, in \cite{TanSri2020}, the authors introduce Fourier features to improve the network, while the authors of \cite{DolHeiMisMos2024} rely on a domain decomposition-based approach.

In this work, when dealing with high-frequency solutions (i.e., solutions with more than three wavelengths propagating), we use the Fourier features from \cite{TanSri2020}.
It relies on modifying the input of the neural network.
Indeed, the prior $u_\theta$ is now defined, for all $\bm{x} \in \Omega$ and $\bm{\mu} \in \mathcal{M}$, as
\begin{equation*}
    u_\theta(\bm{x},\bm{\mu}) =
    w_\theta\big(
        \bm{x},\bm{\mu};
        \sin(\pi a_1 \bm{x}),
        \cos(\pi b_1 \bm{x}),
        \dots,
        \sin(\pi a_{n_f} \bm{x}),
        \cos(\pi b_{n_f} \bm{x})
    \big),
\end{equation*}
where the $2 n_f$ real numbers $(a_l)_{l \in \{1, \dots, n_f\}}$ and $(b_l)_{l \in \{1, \dots, n_f\}}$ are additional trainable parameters.
This makes it possible to learn higher-frequency solutions, by also learning the frequency itself.

\begin{remark}\label{rmk:PINN_notations_Fourier}
	This MLP with Fourier features (MLP w/ FF) needs the same parameters as the classical MLP (defined in \cref{rmk:PINN_notations}), but also the number of Fourier features $n_f$.
\end{remark}


\section{Implementation details}\label{sec:implementation_details}

Before moving on to numerical experiments,
we discuss some practical details
regarding the implementation of
the methods introduced in \cref{sec:additive_prior,sec:multiplicative_prior}.
In \cref{sec:using_PINN}, we first look at how to effectively plug the PINN prediction in the FEM solver.
Then, in \cref{sec:boundary_conditions}, we discuss the imposition of boundary conditions in the two proposed methods. %Finally, we will present how the errors in \cref{sec:error_computation} are computed.

% \paragraph*{Implementation tools.}

The tools used to implement the methods and obtain the results in \cref{sec:numerical_results} are, on the one hand, \texttt{PyTorch} \cite{paszke2019pytorchimperativestylehighperformance} and \texttt{ScimBa} for the construction of the prior, in particular the implementation of PINN, and, on the other hand, \texttt{FEniCS} for the implementation of the finite element methods. For mesh generation, use either \texttt{FEniCS} or \texttt{mshr} mesh generators.

\begin{remark}
    In section \cref{sec:numerical_results}, we will not specify the training times of the networks, independently for each test case. To give an order of importance, PINN training takes less than ten minutes on a laptop GPU. For cases using the LBFGs optimizer, training takes a little longer, but still less than an hour.
\end{remark}

\begin{remark}
    Note that, \texttt{FEniCS} defines the characteristic mesh size $h$ as the length of the longest edge, for $d\in\{1,2\}$.
\end{remark}

\subsection{Using PINN prediction effectively}\label{sec:using_PINN}

To be effective, our methods will in practice depend on the quality of the approximation of the prior's derivatives, computed from automatic differentiation, and its precise integration on the domain.

\paragraph*{Automatic differentiation.}

% The first one is that
It is important to use the automatic differentiation offered by neural networks, enabling exact (in the sense of machine precision) derivative computation without having to manipulate complex symbolic expressions. In particular, in the context of PINNs, automatic differentiation will play a fundamental role in integrating the PDE under consideration. This automatic differentiation will enable the two improved finite element methods to use the exact derivatives of the prior $u_\theta$ and thus avoid introducing an additional error in the computation of the derivative.

\paragraph*{Analytical function integration.}

Analytical functions in the weak problem must be integrated with sufficient precision for these approaches to be effective.
Thus, in e.g. the additive approach, a quadrature rule with a higher degree than the traditional FEM has to be applied to discretize the term $l(v_h) - a(u_\theta,v_h)$ in \eqref{eq:approachform_add}.
This point, and the required degree of the quadrature rule,
will be studied in more detail in the first 2D test case considered in \cref{par:Lap2D_sup}.



\begin{remark}
In practice, the source term in the additive approach will be computed in the strong way. For instance, in the case of the Laplacian equation with $u_{\theta}=0$ on $\partial\Omega$, the term
$$l(v_h)-a(u_{\theta},v_h)=\int_{\Omega}f v_h-\int_{\Omega}\nabla u_\theta \cdot \nabla v_h$$
will be replaced by
$$\int_{\Omega}\big(f+\Delta u_{\theta}\big)v_h.$$
If $u_{\theta}$ is not equal to zero on $\partial\Omega$, one needs to include a boundary term.
\end{remark}

\subsection{Imposing boundary conditions}\label{sec:boundary_conditions}

In this section, we focus on the crucial question of imposing boundary conditions. We first look at this problem in the context of the additive approach presented in \cref{sec:additive_prior} and then in the context of the multiplicative approach presented in \cref{sec:multiplicative_prior}.

\begin{remark}
    For simplicity, this section focuses on (non-homogeneous) Dirichlet conditions.
    For our enhanced FEM, just like in classical FEM,
    these boundary conditions are imposed by manually eliminating essential dofs, see~\cite{Ern2004TheoryAP}, and more precisely by modifying the matrix and the right-hand side of the linear system. This approach is not needed for Neumann and Robin conditions.
\end{remark}

\subsubsection{Additive approach}\label{sec:additive_BC}

In this first approach, if our Dirichlet problem satisfies
$u=g$ on $\partial \Omega$,
then $p_h^+$ has to satisfy
\[
    p_h^+ = g - u_{\theta} \text{\quad on } \partial \Omega,
\]
with $u_\theta$ the PINN prior.
This non-homogeneous boundary condition becomes homogeneous as soon as $u_\theta$ is exact at the boundary, or, in other words, as soon as the boundary conditions are imposed exactly in the PINN, as presented in \cref{sec:exact_imposition_of_BC}.

\begin{remark}\label{rem:bconcurved}
    However, in the case of curved geometries (e.g. disks) where the meshes do not coincide with the boundary of the geometry, problems occur when $q>2$, and especially on coarse meshes. In the numerical results, to avoid this problem and check the error estimates, we assume that $g=u$ on $\partial\Omega_h$ where $\Omega_h$ is the domain covered by the mesh. Furthermore, we need to be careful because even if, in PINN, the conditions are imposed exactly (as shown in \cref{sec:exact_imposition_of_BC}), the prediction $u_\theta$ will not be exact on $\partial\Omega_h$. We made this choice here to simplify the problem, but in practice, solutions exist to improve the quality of the results.
    % These include border scraping methods or mesh refinement methods on the boundary.
\end{remark}

\subsubsection{Multiplicative approach}\label{sec:multiplicative_BC}

In this second method, the boundary conditions are a bit more complex to handle. In the \cref{sec:multiplicative_prior}, we have denoted by
\[
    u_{h,M}^\times = u_{\theta,M} \; p_h^\times
\]
the solution obtained by the multiplicative approach to the modified problem \eqref{eq:ob_pde_M} with the prior $u_{\theta,M}=u_\theta+M$. Therefore, we can recover the solution $u_h^\times$ of the original problem \eqref{eq:ob_pde} by setting $u_h^\times = u_{h,M}^\times - M$.

\paragraph*{Standard PINN.}

In the case where our prior $u_\theta$ is the prediction of a standard PINN, as presented in \cref{sec:PINNs_parametric_PDE}, the boundary conditions are not imposed exactly. Thus, if our problem satisfies $u=g$ on $\partial \Omega$, then $p_h^\times$ has to satisfy
\[
    p_h^\times = \frac{g+M}{u_{\theta,M}} \text{\quad on } \partial \Omega.
\]

\paragraph*{PINN with exact BC.}

We now focus on the case where we exactly impose the boundary conditions in the PINN, as presented in \cref{sec:exact_imposition_of_BC}.
Then, supposing that $M>0$, we have $u_{\theta,M}=g+M$ on $\partial \Omega$ and therefore~$p_h^\times$ has to satisfy
\[
    p_h^\times = 1 \text{\quad on } \partial \Omega.
\]
However, there is a specific case when this condition is not necessarily true. Indeed, if the boundary conditions are homogeneous, then $g=0$ and $u_{\theta,M}=M$ on $\partial \Omega$.
Considering that $u_\theta>0$ in $\Omega$,
$M=0$ is a possible choice.
In this case, \smash{$u_{\theta,M}=u_{\theta,0}=0$} on $\partial \Omega$,
and \smash{$u_{h,M}^\times=u_{h,0}^\times=u_h^\times$} automatically satisfies the boundary conditions.
Hence, imposing a boundary condition on $p_h^\times$ becomes unnecessary.

\begin{remark}
    % In the numerical results of \cref{sec:numerical_results}, one of these specific
    % cases is considered. Considering a 1D case in \cref{sec:Ell1D},
    In the numerical results of \cref{sec:Ell1D}, one of these specific
    cases is considered in 1D.
    We will see that leaving $p_h^\times$ free will give better results here than imposing $p_h^\times=1$ on $\partial \Omega$. Indeed, this approach leaves more freedom to capture the correct derivatives at the boundary.
\end{remark}

% A second possibility is not to impose boundary conditions in the matrix,
% but to impose them solely through a prior.
% This approach can only work for homogeneous Dirichlet conditions.
% To do this, we construct the basis functions associated with the edge nodes,
% and impose no conditions on the edge degrees of freedom.
% Since we multiply by the prior, which is zero at the edges,
% we impose the correct boundary conditions.
% This approach leaves more freedom to capture the right derivatives at the boundaries.

% \subsection{Computation of the error}\label{sec:error_computation}



\section{Numerical results}\label{sec:numerical_results}

This section is dedicated to validating the proposed method on several numerical experiments, which are instances of the problem \eqref{eq:ob_pde} presented in \cref{sec:introduction} with spatial dimension $d \in \{1,2\}$, and with increasing complexity.
The idea is to compare, in different ways, the additive and multiplicative approaches presented in \cref{sec:additive_prior} and \cref{sec:multiplicative_prior} to the standard finite element method presented in \cref{sec:FEM}.
The multiplicative approach will only be considered in dimension $d=1$, showing that only in rare cases, and with a good choice of the lifting constant $M$, does it provide better results than the additive one.
We will also show that the approach proposed in \cref{sec:prior_construction} for choosing the prior is more efficient than more classical ones, considering the 1D case.

In \cref{sec:setup}, we present the two tests that will be performed for each test case. We are interested in two 1D test cases ($d=1$): the Poisson problem with homogeneous Dirichlet conditions in \cref{sec:Lap1D} and a general elliptic system and convection-dominated regime problem in \cref{sec:Ell1D}. We then consider two 2D cases ($d=2$): We start with a Poisson problem with homogeneous Dirichlet conditions on a square domain in \cref{sec:Lap2D}. We then continue with a slightly more complicated elliptic problem, still with Dirichlet conditions and on a square domain, in \cref{sec:Ell2D}. In \cref{sec:Lap2DMixRing}, we return to the 2D Poisson problem, but this time considering mixed boundary conditions on a ring-shaped domain.



The implementation of the numerical results obtained on the various test cases is available at the following address:


\begin{center}
\url{https://github.com/flecourtier/EnrichedFEMUsingPINNs}.
\end{center}

\subsection{Setup of the numerical experiments}\label{sec:setup}

For each of the proposed test cases,
we consider a parametric problem, on which we train a PINN as presented in \cref{sec:PINNs_parametric_PDE},
to resolve it on a set of fixed parameters, denoted by $\mathcal{M}$.
Let $p=\dim(\mathcal{M})$ be the number of parameters,
and consider a sample $\mathcal{S}$ of $n_p$ parameters:
\begin{equation*}
	\mathcal{S}=\left\{\bm{\mu}^{(1)},\dots,\bm{\mu}^{(n_p)}\right\},
\end{equation*}
with, for $j=1,\dots,n_p$,
\begin{equation*}
	\bm{\mu}^{(j)}=\left(\mu_1^{(j)},\dots,\mu_{p}^{(j)}\right)\in \mathcal{M}.
\end{equation*}

In the following, we denote by $u^{(j)}$ a reference solution to problem \eqref{eq:parametric_PDE} for a given parameter $\bm{\mu}^{(j)}$ and by $u_h^{(j)}$ the solution obtained by the standard finite element method \eqref{eq:approachform} where $V_h$ is the $\mathbb{P}_k$ Lagrange space defined in \eqref{eq:Vh} and $h$ is the characteristic mesh size. We also denote by $u_\theta^{(j)}$ the solution obtained by the parametric PINN, and by $u_{h,+}^{(j)}$ the solution obtained by the additive approach \eqref{eq:approachform_add} with $V_h^+$ the $\mathbb{P}_k$ Lagrange space defined in \eqref{eq:Vh_add}. In some test cases, we also consider the solution  $u_{h, M}^{(j)}$ of the multiplicative approach \eqref{eq:approachform_mul} with $V_h^\times$ the $\mathbb{P}_k$ Lagrange space defined in \eqref{eq:Vh_mul}, depending on the lifting constant $M$.

\begin{remark}\label{rk:ref}
	In the following, to estimate the error, we consider the reference solution to be either an analytical solution or a solution obtained with a very fine mesh and a high polynomial degree. More precisely, we need the characteristic size $h_\text{ref}$ associated with the reference mesh to be much smaller than the size associated with the current mesh $h$, i.e. $h_\text{ref}\ll h$, and we will consider $k_\text{ref}=3$ the polynomial degree associated with the reference solution.
\end{remark}

In each test case, we investigate two aspects; the first in~\cref{sec:setup_error_estimates} involves verifying the error estimates and the second in~\cref{sec:setup_gains} is the evaluation of the gains achieved by the proposed methods compared with the standard one.

\subsubsection{Error estimates}\label{sec:setup_error_estimates}

Consider a small sample $\mathcal{S}$ with $n_p=2$ parameters. Given a fixed parameter $\bm{\mu}^{(j)}$, $j=1,2$, we start by testing the error estimates obtained in \cref{lem:error_estimation_add} for the additive approach. In the case $d=1$, we will also be interested in the error estimates in \cref{lem:error_estimate_multiplicative} for the multiplicative approach. By varying the mesh size~$h$, we then estimate the errors obtained with the two methods.
To evaluate these errors, we compare the approximations to the reference solution $u^{(j)}$ (see \cref{rk:ref}).
We then define by
\begin{equation}\label{eq:error_rel_FEM}
	e_h^{(j)}=\frac{||u^{(j)}-u_h^{(j)}||_{L^2}}{||u^{(j)}||_{L^2}}
	\text{ and }
	e_\theta^{(j)}=\frac{||u^{(j)}-u_\theta^{(j)}||_{L^2}}{||u^{(j)}||_{L^2}}
\end{equation}
the $L^2$ relative error obtained for the standard FEM and the PINN respectively. We further define,
\begin{equation}
    \label{eq:error_rel_add}
	e_{h,+}^{(j)}=\frac{||u^{(j)}-u_{h,+}^{(j)}||_{L^2}}{||u^{(j)}||_{L^2}}
	\text{ and }
	e_{h,M}^{(j)}=\frac{||u^{(j)}-u_{h,M}^{(j)}||_{L^2}}{||u^{(j)}||_{L^2}}
\end{equation}
the $L^2$ relative errors obtained for the additive and multiplicative approach (depending on the lifting constant $M$), respectively.

%\begin{remark}
%	In the numerical experiments, we consider only the relative $L^2$ error.
%\end{remark}

\subsubsection{Gains achieved with the enriched bases}\label{sec:setup_gains}

As we have trained the network to be parameter-dependent to predict a solution for a set of parameters, we are interested in the average gains we obtain with our enriched approaches compared to the PINN and the standard FEM. More precisely, for a fixed mesh size $h$ and a fixed polynomial degree $k$, for a sample $\mathcal{S}$ of $n_p$ parameters, the numerical gains obtained by the additive approach on PINN and standard FEM are respectively defined for $j=1,\dots,n_p$ by:
\begin{equation}
	G_{+,\theta}^{(j)}=\frac{e_\theta^{(j)}}{e_{h,+}^{(j)}}
    \text{\qquad and \qquad}
    G_+^{(j)}=\frac{e_h^{(j)}}{e_{h,+}^{(j)}}, \label{eq:gain_j}
\end{equation}
with $e_\theta^{(j)}$, $e_h^{(j)}$ and $e_{h,+}^{(j)}$ respectively the $L^2$ relative errors obtained with the PINN, the standard FEM and the additive approach, defined in \cref{sec:setup_error_estimates}.
Similarly, the theoretical gains obtained by the multiplicative approach (depending on the lifting constant $M$) on PINN and standard FEM are respectively defined for $j=1,\dots,n_p$ by:
\begin{equation}
	G_{M,\theta}^{(j)}=\frac{e_\theta^{(j)}}{e_{h,M}^{(j)}} \quad \text{and} \quad G_M^{(j)}=\frac{e_h^{(j)}}{e_{h,M}^{(j)}}, \label{eq:gain_j_mul}
\end{equation}
with $e_{h,M}^{(j)}$ the $L^2$ relative error obtained with the multiplicative approach (depending on the lifting constant $M$), defined in \cref{sec:setup_error_estimates}.
Therefore, we will be interested in the minimum, maximum, mean and standard deviation obtained on the following samples :
\begin{equation}
	G_{+,\theta}=\left\{G_{+,\theta}^{(1)},\dots,G_{+,\theta}^{(n_p)}\right\} \quad \text{and} \quad G_+=\left\{G_+^{(1)},\dots,G_+^{(n_p)}\right\}, \label{eq:gain_add_num}
\end{equation}
which respectively represent the gains obtained with our additive approach over PINNs and over standard FEM on the sample $\mathcal{S}$. In the same way, we define $G_{M,\theta}$ and $G_M$, which respectively represent the gains obtained with our multiplicative approach over PINN and over standard FEM on the sample $\mathcal{S}$ by:
\begin{equation}
	G_{M,\theta}=\left\{G_{M,\theta}^{(1)},\dots,G_{M,\theta}^{(n_p)}\right\} \quad \text{and} \quad G_M=\left\{G_M^{(1)},\dots,G_M^{(n_p)}\right\}. \label{eq:gain_mul_num}
\end{equation}


\subsection{1D Poisson problem}\label{sec:Lap1D}

In this section, we will consider the problem~\eqref{eq:ob_pde} in its most
simplified Poisson form, with homogeneous Dirichlet boundary conditions.
In the 1D case ($d=1$), we have,
\begin{equation}
	\left\{
	\begin{aligned}
		-\partial_{xx} u & = f, \; &  & \text{in } \; \Omega \times \mathcal{M}, \\
		u         & = 0, \;  &  & \text{on } \; \partial\Omega \times \mathcal{M},
	\end{aligned}
	\right.
	\label{eq:Lap1D}
\end{equation}
with $\Omega=[0,1]$, $\partial\Omega$ its boundary and $\mathcal{M} \subset \mathbb{R}^p$ the parameter space (with $p$ the number of parameters).
% Considering $\bm{x}=x\in\Omega$, we define the analytical solution by
{For this problem, we consider solutions defined with
$p=3$ parameters $\bm{\mu}=(\mu_1,\mu_2,\mu_3)\in\mathcal{M}=[0,1]^3$ defined by
}
\begin{equation*}
	u(x;\bm{\mu})=\mu_1\sin(2\pi x)+\mu_2\sin(4\pi x)+\mu_3\sin(6\pi x) \,.
\end{equation*}
% with $p=3$ parameters $\bm{\mu}=(\mu_1,\mu_2,\mu_3)\in\mathcal{M}=[0,1]^p$, and
{Note that the associated right-hand side $f$ in \eqref{eq:Lap1D}
also depends on $\bm{\mu}$.
This problem is thus four-dimensional: one dimension in space and three dimensions
for the parameters $\bm{\mu}$.}

For this first test case, we construct two priors,
as detailed in \cref{sec:Lap1D_priors}.
The first one, denoted by $u_\theta$,
is built from a PINN as presented in \cref{sec:prior_construction}.
The second one, denoted by $u_\theta^\text{data}$
is constructed only from data (obtained from the analytical solution).
The aim is to show that using physics-informed training to construct the prior
leads to better results than data-driven training.
In this test case, we also compare the additive and multiplicative approaches (presented for several values of $M$), but only with $k=1$ polynomial order to remain concise.

We first present the error estimates obtained with the PINN prior in \cref{sec:Lap1D_error_estimations}. First, we check the orders of convergence of the two enriched approaches. Then, we verify the results expected in \cref{sec:comparison_add_mul} by comparing the theoretical gain constants of the additive and multiplicative approaches. Afterwards, in \cref{sec:Lap1D_derivatives}, we compare, for a given parameter, the derivatives obtained with the two priors and analyze the associated gains. Finally, we evaluate the gains obtained with the two priors in \cref{sec:Lap1D_gains} on a sample of parameters.

\begin{remark}\label{rmk:N_nodes}
	In the following, the characteristic mesh size $h=\frac{1}{N-1}$, where $N$ represents the number of considered nodes.
\end{remark}

\subsubsection{Construction of the two priors}\label{sec:Lap1D_priors}

The hyperparameters used to construct the two priors are presented in \cref{tab:paramtest1_1D}. We discuss below the specific differences in training
both priors.

\begin{table}[htbp]
    \centering
    \begin{tabular}{cc}
        \toprule
        \multicolumn{2}{c}{\textbf{Network - MLP}} \\
        \midrule
        \textit{layers} & $20,80,80,80,20,10$ \\
        \cmidrule(lr){1-2}
        $\sigma$ & sine \\
        \bottomrule
    \end{tabular}
    \hspace{1cm}
    \begin{tabular}{cc}
        \toprule
        \multicolumn{2}{c}{\textbf{Training}} \\
        \midrule
        \textit{lr} & 9e-2 \\
        \cmidrule(lr){1-2}
        \textit{decay} & 0.99 \\
        \cmidrule(lr){1-2}
        $n_{epochs}$ & 10000 \\
        \cmidrule(lr){1-2}
        $N_\text{col}$ \text{or} $N_\text{data}$ & 5000 \\
        \bottomrule
    \end{tabular}
    \hspace{1cm}
    \begin{tabular}{cccccccc}
        \toprule
        \multicolumn{8}{c}{\textbf{Loss weights}} \\
        \midrule
        \multicolumn{4}{c}{PINN prior $u_\theta$} & \multicolumn{4}{c}{Data prior $u_\theta^\text{data}$} \\
        \cmidrule(lr){1-4} \cmidrule(lr){5-8}
        $\omega_r$ & 1 & $\omega_\text{data}$ & 0 & $\omega_r$ & 0 & $\omega_\text{data}$ & 1 \\
        \cmidrule(lr){1-2} \cmidrule(lr){3-4} \cmidrule(lr){5-6} \cmidrule(lr){7-8}
        $\omega_b$ & 0 & $\omega_\text{sob}$ & 0 & $\omega_b$ & 0 & $\omega_\text{sob}$ & 0 \\        
        \bottomrule
    \end{tabular}
    \caption{Network, training parameters (\cref{rmk:PINN_notations}) and loss weights for $u_\theta$ and $u_\theta^\text{data}$ in the \textit{1D Poisson problem}. Considering $N_\text{col}$ collocation points for the PINN prior and $N_\text{data}$ data for the data prior.}
    \label{tab:paramtest1_1D}
\end{table}


\paragraph*{Physics-informed training.}
For the first prior, we will consider a parametric PINN, depending on the problem parameters $\bm{\mu}$,
where we exactly impose the Dirichlet boundary conditions as presented in \cref{sec:exact_imposition_of_BC} and without using data in training.
% To do this,
We define the prior
\begin{equation}\label{eq:utheta}
	u_\theta(x,\bm{\mu}) = \varphi(x) w_\theta(x,\bm{\mu}) \,,
\end{equation}
where $w_\theta$ is the neural network under consideration and $\varphi$ is the level-set function defined by
\begin{equation}\label{eq:phii}
	\varphi(x)=x(x-1) \,,
\end{equation}
which vanishes exactly on $\partial\Omega$.
Since we impose the boundary conditions by using the level-set function, we will only consider the residual loss $J_r$ defined in \eqref{eq:residual_loss_parametric} in which the integrals are approached by a Monte-Carlo method, i.e.
\[
	J_r(\theta) \simeq
	\frac{1}{N_\text{col}} \sum_{i=1}^{N_\text{col}} \big| \partial_{xx}u_\theta(\bm{x}_\text{col}^{(i)};\bm{\mu}_\text{col}^{(i)}\big) + f\big(\bm{x}_\text{col}^{(i)};\bm{\mu}_\text{col}^{(i)}\big) \big|^2
\]
with the $N_\text{col}=5000$ collocation points \smash{$\big(\bm{x}_\text{col}^{(i)}, \bm{\mu}_\text{col}^{(i)}\big)_{i=1,\dots,N_\text{col}}$} uniformly chosen on $\Omega\times\mathcal{M}$. Thus, we seek to solve the following minimisation problem
\begin{equation*}
	\theta^\star = \argmin_\theta J_r(\theta).
\end{equation*}
In this first case,
we consider an MLP with $6$ layers and a sine activation function.
The hyperparameters are given in \cref{tab:paramtest1_1D};
we use the Adam optimizer~\cite{KinBa2015}.% and consider $N_\text{col}=5000$ collocation points, .




\paragraph*{Data-driven training.}

For the second prior considered, noted $u_\theta^\text{data}$, a network is trained only on the data (constructed from the analytical solution). In this second case, we still consider the parameters defined in \cref{tab:paramtest1_1D}, which are the same as for the physics-informed training except that we take $N_\text{data}=5000$ instead of $N_\text{col}=5000$.
Considering, for each epoch, a set of parameters and points \smash{$\big(\bm{x}_\text{data}^{(i)}, \bm{\mu}_\text{data}^{(i)}\big)_{i=1,\dots,N_\text{data}}$},
% we can consider the data loss defined by
the data loss is defined by
\begin{equation*}
	J_\text{data}(\theta) =
    \frac{1}{N_\text{data}}
    \sum_{i=1}^{N_\text{data}} \big| u_\theta^\text{data}(\bm{x}_\text{data}^{(i)};\bm{\mu}_\text{data}^{(i)}) - u(\bm{x}_\text{data}^{(i)};\bm{\mu}_\text{data}^{(i)}) \big|^2,
\end{equation*}
and thus we seek to solve the following minimisation problem
\begin{equation*}
	\theta^\star = \argmin_\theta J_\text{data}(\theta).
\end{equation*}

\begin{remark}
	As for the PINN prior, we impose the boundary conditions exactly in the data-driven training.
\end{remark}

\subsubsection{Error estimates --- with the PINN prior}\label{sec:Lap1D_error_estimations}

In this section, we look at the theoretical results of the additive and multiplicative approaches, considering the PINN prior $u_\theta$. First, we check the orders of convergence of \cref{lem:error_estimation_add,lem:error_estimate_multiplicative} (in the $L^2$ norm), associated with both methods. Next, we numerically verify \cref{thm:comparison_add_mul}, showing that the multiplicative correction converges, for sufficiently large $M$, towards the additive correction (in both the $L^2$ norm and $H^1$ semi-norm).

\paragraph*{Convergence rate.} We test the error estimates of \cref{lem:error_estimation_add,lem:error_estimate_multiplicative} for the following two sets of parameters:
\begin{equation} \label{equation:lap1d:mu-choices}
	\bm{\mu}^{(1)}=(0.3,0.2,0.1) \quad \text{and} \quad \bm{\mu}^{(2)}=(0.8,0.5,0.8) \,,
	% param 2 = ancien param 4
\end{equation}
with the PINN prior $u_\theta$.
For $j \in \{1, 2\}$, the aim is to compare, by varying the mesh size $h$, the $L^2$ relative errors \smash{$e_h^{(j)}$} obtained with the standard FEM, defined in~\eqref{eq:error_rel_FEM}, \smash{$e_{h,+}^{(j)}$} obtained with the additive approach, defined in~\eqref{eq:error_rel_add} and \smash{$e_{h,M}^{(j)}$} obtained with the multiplicative approach (taking $M=3$ and $M=100$), defined in~\eqref{eq:error_rel_add}. The results are presented in \cref{fig:case1_1D}
for a polynomial orders $k=1$ and $k=2$, with $h$ depending
on the number of nodes $N \in \{16,32,64,128,256\}$ as
presented in \cref{rmk:N_nodes}.


\begin{figure}[!ht]
	\centering
	\begin{subfigure}{0.48\linewidth}
		\centering
		\cvgFEMCorrMultOnedeg{fig_testcase1D_test1_cvg_FEM_case1_v1_param1_degree1.csv}{fig_testcase1D_test1_cvg_FEM_case1_v1_param1_degree2.csv}{fig_testcase1D_test1_cvg_Corr_case1_v1_param1_degree1.csv}{fig_testcase1D_test1_cvg_Mult_case1_v1_param1_degree1_M3.0.csv}{fig_testcase1D_test1_cvg_Mult_case1_v1_param1_degree1_M100.0.csv}{1e-5}
		\caption{Case of $\bm{\mu}^{(1)}$}
		\label{fig:case1param1_1D}
	\end{subfigure}
	\begin{subfigure}{0.48\linewidth}
		\centering
		\cvgFEMCorrMultOnedeg{fig_testcase1D_test1_cvg_FEM_case1_v1_param2_degree1.csv}{fig_testcase1D_test1_cvg_FEM_case1_v1_param2_degree2.csv}{fig_testcase1D_test1_cvg_Corr_case1_v1_param2_degree1.csv}{fig_testcase1D_test1_cvg_Mult_case1_v1_param2_degree1_M3.0.csv}{fig_testcase1D_test1_cvg_Mult_case1_v1_param2_degree1_M100.0.csv}{5e-6}
		\caption{Case of $\bm{\mu}^{(2)}$}
		\label{fig:case1param2_1D}
	\end{subfigure}
	\caption{Considering the \textit{1D Laplacian case} and the PINN prior $u_\theta$. Left -- Considering $\bm{\mu}^{(1)}$. $L^2$ error on $h$ obtained with standard FEM \smash{$e_h^{(1)}$} (solid lines) with $k=1$ and $k=2$, the additive approach \smash{$e_{h,+}^{(1)}$} (dashed lines) with $k=1$ and the multiplicative approach \smash{$e_{h,M}^{(1)}$} (dotted lines), with $k=1$ ($M=3$ and $M=100$). Right -- Same for $\bm{\mu}^{(2)}$.}
	\label{fig:case1_1D}
\end{figure}

The results of \cref{fig:case1_1D} show that all the enriched finite elements increase the accuracy of the method and that they also converge at the same rate as the classical approach (i.e., as for the polynomial approximation order $k=1$).
Furthermore, the theoretical analysis (which showed that the multiplicative correction has the same error as the additive one when $M \to \infty$) is confirmed for both sets of parameters.
In addition, \cref{fig:case1_1D} also shows that this multiplicative enrichment can be less efficient for small $M$ when the $(k+1)$\textsuperscript{th} derivative of the solution is large.
Indeed, for the first parameter considered in \cref{fig:case1param1_1D}, for which the second derivative takes lower values, we observe that the multiplicative approach with small $M$ is closer to the additive one than for the second set of parameters
considered in \cref{fig:case1param2_1D}, for which the derivatives are larger.
%Finally, in both cases, the gain given by the enriched FEM is significant, and, as expected, the gain is also greater at high frequencies. \textcolor{red}{à enlever ?} 
Moreover, it seems that we gain almost one order of interpolation with the additive approach: in \cref{fig:case1}, the additive method with polynomial order $k=1$ gives an error $L^2$ close to the original FEM method with $k=2$, although the rate of convergence is different.

\paragraph*{Gain constants.} We consider the first parameter $\bm{\mu}^{(1)}$ and the PINN prior $u_\theta$. We now evaluate the gain constants $C_\text{gain}^+$ and $C_\text{gain}^{\times,M}$ (for different values of $M$), are respectively defined in~\cref{rmk:gain_add,rmk:gain_mul} for the additive and multiplicative approaches. The idea is to check the convergence of the multiplicative gain constant towards the additive one, as proven in \cref{thm:comparison_add_mul}. The results are presented in \cref{fig:Lap1D_gain_constants} for $L^2$ and $H^1$ norms.

%\begin{remark}
%	Note that for $m=0$ and $m=1$, we have
%	\[
%		\frac{m!}{\lfloor \frac{m}{2} \rfloor!^2} = 1,
%	\]
%	which means that $C_\text{gain}^{\times,M}$ converges towards the same value in $L^2$ and $H^1$ norms.
%\end{remark}

\begin{figure}[ht!]
	\centering
    \begin{minipage}{0.55\linewidth}
		\includegraphics[scale=1]{fig_testcase1D_test1_comp_add_mult_standalone_m0.pdf}
		\includegraphics[scale=1]{fig_testcase1D_test1_comp_add_mult_standalone_m1.pdf}
    \end{minipage} \hfill
	% \begin{minipage}{0.42\linewidth}
	% 	\centering
	% 	\input{fig/testcase1D/test1/comp_add_mult/comp_mult_add_gain.tex}
	% \end{minipage}
	\caption{Considering the \textit{1D Laplacian case} with $\bm{\mu}^{(1)}$, $k=1$ and the PINN prior $u_\theta$. Left -- Convergence of \cref{thm:comparison_add_mul} with the $L^2$ error. Right -- Convergence of \cref{thm:comparison_add_mul} with the $H^1$ error. %Right -- Values of the gain constants for $m=0$ and $m=1$ (for different values of $M$).
	}
	\label{fig:Lap1D_gain_constants}
\end{figure}

\cref{fig:Lap1D_gain_constants} shows that the multiplicative gain constant converges to the additive gain constant when $M$ increases, as expected in  the theoretical results of \cref{thm:comparison_add_mul}.
 %Moreover, the results obtained here in $L^2$ norm seem to correspond to the numerical gains obtained in \cref{tab:case1_1D_both} (with the PINN prior $u_\theta$) of \cref{sec:Lap1D_derivatives}, \smash{$G_+^{(1)}$} for the additive approach defined in \eqref{eq:gain_j} and \smash{$G_M^{(1)}$} for the multiplicative approach (taking $M=3$ and $M=100$) defined in \eqref{eq:gain_j_mul}.

\subsubsection{Derivatives --- with both priors}\label{sec:Lap1D_derivatives}

To better explain the results of \cref{sec:Lap1D_error_estimations}, we compare the solution, the first- and second-order derivatives between the exact solution and the prediction of both priors, for selected parameter $\bm{\mu}^{(1)}$.
\cref{sinus1D_Pinns,sinus1D_nn} respectively present this comparison for the PINN prior $u_\theta$ and the data prior $u_\theta^\text{data}$.
We also compare the errors and gains obtained with these two priors for $N \in \{16,32\}$ in \cref{tab:case1_1D_both}.
More precisely, we evaluate the additive error \smash{$e_{h,+}^{(1)}$} and the additive gain on FEM \smash{$G_+^{(1)}$}, respectively defined in~\eqref{eq:error_rel_add} and~\eqref{eq:gain_j}, for both PINN and data priors.
We also evaluate the multiplicative error \smash{$e_{h,M}^{(1)}$} and the multiplicative gain on FEM \smash{$G_M^{(1)}$} defined in~\eqref{eq:error_rel_add} and~\eqref{eq:gain_j_mul}, for both priors, with $M=3$ and $M=100$.

\begin{figure}[ht!]
	\centering
	% \includegraphics[width=0.9\linewidth]{fig/testcase1D/test1/plots/derivatives_mu1_PINN.pdf}
	\includegraphics[scale=1]{fig_testcase1D_test1_plots_standalone_solutions_and_errors_PINN.pdf}
	\caption{Considering the \textit{1D Laplacian case} with $\bm{\mu}^{(1)}$ and the PINN prior $u_\theta$, comparison between analytical solution and network prediction.
    From left to right: solution; first derivative; second derivative; errors.}
	\label{sinus1D_Pinns}
\end{figure}

\begin{figure}[ht!]
	\centering
	% \includegraphics[width=0.9\linewidth]{fig/testcase1D/test1/plots/derivatives_mu1_NN.pdf}
	\includegraphics[scale=1]{fig_testcase1D_test1_plots_standalone_solutions_and_errors_NN.pdf}
	\caption{Considering the \textit{1D Laplacian case} with $\bm{\mu}^{(1)}$ and the data prior $u_\theta^\text{data}$,
    comparison between analytical solution and network prediction.
    From left to right: solution; first derivative; second derivative; errors.
	}
	\label{sinus1D_nn}
\end{figure}


\begin{table}[H]
	\centering
	\gainsbothNN{fig_testcase1D_test1_plots_FEM_param1.csv}{fig_testcase1D_test1_plots_compare_gains_param1.csv}
	\caption{Considering the \textit{1D Laplacian case} with $\bm{\mu}^{(1)}$, $k=1$ and $N=16,32$. Left -- $L^2$ relative error obtained with FEM. Right -- Considering the PINN prior $u_\theta$ and the data prior $u_\theta^\text{data}$, $L^2$ relative errors and gains with respect to FEM, obtained with our methods. Our methods : additive approach, multiplicative approach with $M=3$ and $M=100$.}
	\label{tab:case1_1D_both}
\end{table}

The results reported in \cref{sinus1D_Pinns,sinus1D_nn,tab:case1_1D_both} show that, even if the approach chosen to build the prior (physics-informed or data-driven training) gives a good approximation of the solution, the important point lies in the derivatives and mainly in the second-order derivatives, which are clearly better learned by PINNs.
Indeed, while the enriched FEM solution is more accurate using
the PINN prior (\cref{tab:case1_1D_both}), we see from
\cref{sinus1D_Pinns,sinus1D_nn,tab:case1_1D_both} that the raw PINN
approximates the solution $u$ less accurately than the raw data-driven
the solution, but the PINN better approximates the derivatives.
As the error of the enriched FEM is mainly due to the $k+1$\textsuperscript{th} derivatives of the network being close to the $k+1$\textsuperscript{th} derivatives of the solution, this explains why the enriched FEM with data prior does not perform as well as with a PINN prior.
Therefore, PINNs have two advantages: they do not require training data and give better results.
Their main shortcoming is that training takes longer.
However, we mention that if data are available for first- and second-order
derivatives, it could also be used to improve a purely data-driven prior.

\subsubsection{Gains achieved with the additive and multiplicative approaches -- with both priors}\label{sec:Lap1D_gains}

Considering a sample $\mathcal{S}$ of $n_p=100$ parameters, we now evaluate the gains $G_{+,\theta}$ and $G_+$ defined in~\eqref{eq:gain_add_num} with the PINN prior $u_\theta$ and with the data prior $u_\theta^\text{data}$.
We also compute $G_{M,\theta}$ and $G_M$, defined in~\eqref{eq:gain_mul_num}, similarly for both priors.
For fixed polynomial order $k=1$ and $N \in \{20,40\}$,
the results with the physics-informed prior $u_\theta$ and the data-driven prior $u_\theta^\text{data}$
are respectively presented in \cref{tab:case1_1D_PINNs,tab:case1_1D_data}.

\begin{table}[H] % u_theta
	\centering
	\gainstableMult{fig_testcase1D_test1_gains_Tab_stats_case1_v1_degree1.csv}
	\caption{Considering the \textit{1D Laplacian case} and the PINN prior $u_\theta$. Left -- Gains in $L^2$ error of our methods with respect to PINN by taking $k=1$. Right -- Gains in $L^2$ error of our methods with respect to FEM by taking $k=1$. Our methods : additive approach, multiplicative approach with $M=3$ and $M=100$.}
	\label{tab:case1_1D_PINNs}
\end{table}

\begin{table}[H] % u_theta^data
	\centering
	\gainstableMultData{fig_testcase1D_test1_gains_Tab_stats_case1_v2_degree1.csv}
	\caption{Considering the \textit{1D Laplacian case} and the Data prior $u_\theta^\text{data}$. Left -- Gains in $L^2$ error of our methods with respect to Data Network by taking $k=1$. Right -- Gains in $L^2$ error of our methods with respect to FEM by taking $k=1$. Our methods : additive approach, multiplicative approach with $M=3$ and $M=100$.}
	\label{tab:case1_1D_data}
\end{table}

The previous results indicate that the average gain provided by the enriched FE with the PINN prior is significant, particularly when using the additive approach. These findings also confirm the behaviour of the multiplicative prior method for varying values of $M$. In contrast, when applied with data-driven instead of physics-informed training, the same method does not yield similarly favourable results.
Consequently, in the experiments we perform below, we only employ the
PINN prior.


\subsection{1D general elliptic system and convection-dominated regime}\label{sec:Ell1D}

In this experiment, we consider the problem \eqref{eq:ob_pde}
in a more complex form, still in a 1D ($d=1$) configuration:
\begin{equation*}
	\left\{
	\begin{aligned}
		\partial_x u-\frac{1}{\peclet}\partial_{xx} u &= r, \; &  & \text{in } \; \Omega \times \mathcal{M}, \\
		u         & = 0, \;  &  & \text{on } \; \partial\Omega \times \mathcal{M},
	\end{aligned}
	\right.
	\label{eq:Ell1D}
\end{equation*}
with $\Omega=[0,1]$ and $\partial\Omega$ its boundary, $r$ the reaction constant term and $\peclet$ the Péclet number which describe the ratio between convective and diffusion term. We define an analytical solution for all $x\in\Omega$ by
\begin{equation}
    \label{eq:Ell1D_analytical}
	u(x;\bm{\mu})=r\left(x-\frac{e^{\peclet\, x}-1}{e^{\peclet}-1}\right) \,,
\end{equation}
with $p=2$ parameters $\bm{\mu}=(r,\peclet)\in\mathcal{M}=[1,2]\times[10,100]$.

\begin{remark}\label{rem:oscillations}
	In large Péclet regime, i.e., for convection-dominated flows,
    the classical finite element method may generate oscillations
    when no specific treatment is applied,
    see e.g. \cite{JohKnoNov2018}.
\end{remark}

In this % second 1D
test-case, we will construct only one prior, denoted $u_\theta$,
built from a PINN as presented in \cref{sec:prior_construction}.
We will also compare the additive and multiplicative approaches by considering
polynomial order $k=1$, and since the solution is positive in $\Omega$,
we consider $M=0$ {for the multiplicative approach}.
We start by evaluating the error in \cref{sec:Ell1D_error_estimations},
then we compare the derivatives of the PINN prior and compare the different
approaches in \cref{sec:Ell1D_comparison}. Finally, we evaluate the gains obtained
in \cref{sec:Ell1D_gains} on a sample of parameters.
As we are dealing with a specific case, we will compare two methods
for imposing boundary conditions, as presented in \cref{sec:multiplicative_BC}:
the strong and the weak approaches.

\begin{remark}\label{rmk:Ell1D_N_nodes}
	As in \cref{sec:Lap1D}, the characteristic mesh size $h=\frac{1}{N-1}$, where $N$ is the number of nodes considered.
\end{remark}

\paragraph*{Physics-informed training.}
We consider a parametric PINN, depending on the problem parameters $\bm{\mu}$,
where we exactly impose the Dirichlet boundary conditions as presented in \cref{sec:exact_imposition_of_BC}. We define the prior $u_{\theta}$ and the level set $\varphi$ as in \eqref{eq:utheta} and \eqref{eq:phii},
which vanishes exactly on $\partial\Omega$.

\begin{table}[htbp]
    \centering
    \begin{tabular}{cc}
        \toprule
        \multicolumn{2}{c}{\textbf{Network - MLP}} \\
        \midrule
        \textit{layers} & $40,40,40,40,40$ \\
        \cmidrule(lr){1-2}
        $\sigma$ & tanh \\
        \bottomrule
    \end{tabular}
    \hspace{1cm}
    \begin{tabular}{cc}
        \toprule
        \multicolumn{2}{c}{\textbf{Training}} \\
        \midrule
        \textit{lr} & 1e-3 \\
        \cmidrule(lr){1-2}
        \textit{decay} & 0.99 \\
        \cmidrule(lr){1-2}
        $n_{epochs}$ & 20000 \\
        \cmidrule(lr){1-2}
        $N_\text{col}$ & 5000 \\
        \bottomrule
    \end{tabular}
    \hspace{1cm}
    \begin{tabular}{cccc}
        \toprule
        \multicolumn{4}{c}{\textbf{Loss weights}} \\
        \midrule
        $\omega_r$ & 1 & $\omega_\text{data}$ & 0 \\
        \cmidrule(lr){1-2} \cmidrule(lr){3-4}
        $\omega_b$ & 0 & $\omega_\text{sob}$ & 0 \\        
        \bottomrule
    \end{tabular}
    \caption{Network, training parameters (\cref{rmk:PINN_notations}) and loss weights for $u_\theta$ in the \textit{1D Elliptic} case.}
    \label{tab:paramtest2_1D}
\end{table}


Since we impose the boundary conditions by using the level-set function, we will only consider the residual loss $J_r$ defined in \eqref{eq:residual_loss_parametric} in which the integrals are approached by a Monte-Carlo method, i.e.
\[
	J_r(\theta) \simeq
	\frac{1}{N_\text{col}} \sum_{i=1}^{N_\text{col}} \left| \partial_x u_\theta\big(\bm{x}_\text{col}^{(i)};\bm{\mu}_\text{col}^{(i)}\big)-\frac{1}{\peclet}\partial_{xx} u_\theta\big(\bm{x}_\text{col}^{(i)};\bm{\mu}_\text{col}^{(i)}\big) - r
	\right|^2 \,,
\]
with the $N_\text{col}=5000$ collocation points \smash{$\big(\bm{x}_\text{col}^{(i)}, \bm{\mu}_\text{col}^{(i)}\big)_{i=1,\dots,N_\text{col}}$} uniformly chosen on $\Omega\times\mathcal{M}$. Thus we seek to solve the following minimisation problem
\begin{equation*}
	\theta^\star = \argmin_\theta J_r(\theta).
\end{equation*}
In this case,
we consider a MLP with $5$ layers and a tanh activation function.
The hyperparameters are given in \cref{tab:paramtest2_1D};
we use the Adam optimizer~\cite{KinBa2015}.

\subsubsection{Error estimates}\label{sec:Ell1D_error_estimations}

We start by testing the error estimates (\cref{lem:error_estimation_add}) for the following two sets of parameters:
\begin{equation}\label{equation:ell1d:mu-choices}
	\bm{\mu}^{(1)}=(1.2,40) \quad \text{and} \quad \bm{\mu}^{(2)}=(1.5,90) \,,
\end{equation}
by considering the PINN prior $u_\theta$.
For $j \in \{1, 2\}$, the aim is to compare for different mesh sizes $h$,
the $L^2$ relative errors \smash{$e_h^{(j)}$} obtained with the standard FEM method, defined in \eqref{eq:error_rel_FEM}, \smash{$e_{h,+}^{(j)}$} obtained with the additive approach and \smash{$e_{h,M}^{(j)}$} obtained with the multiplicative approach (taking $M=0$), defined in \eqref{eq:error_rel_add}.  We will consider the two implementations of the boundary conditions for the multiplicative approach: the strong and the weak BC, as presented in \cref{sec:multiplicative_prior}.
The results are presented in \cref{fig:case2_1D} by varying the mesh size $h$.

\begin{figure}[!ht]
	\centering
	\begin{subfigure}{0.48\linewidth}
		\centering
		\cvgFEMCorrMultSWOnedeg{fig_testcase1D_test2_cvg_FEM_case2_v1_param1_degree1.csv}{fig_testcase1D_test2_cvg_FEM_case2_v1_param1_degree2.csv}{fig_testcase1D_test2_cvg_Corr_case2_v1_param1_degree1.csv}{fig_testcase1D_test2_cvg_Mult_case2_v1_param1_degree1_M0.0.csv}{fig_testcase1D_test2_cvg_Mult_case2_v1_param1_degree1_M0.0_weak.csv}{5e-6}
		\caption{Case of $\bm{\mu}^{(1)}$}
		\label{fig:case2param1_1D}
	\end{subfigure}
	\begin{subfigure}{0.48\linewidth}
		\centering
		\cvgFEMCorrMultSWOnedeg{fig_testcase1D_test2_cvg_FEM_case2_v1_param2_degree1.csv}{fig_testcase1D_test2_cvg_FEM_case2_v1_param2_degree2.csv}{fig_testcase1D_test2_cvg_Corr_case2_v1_param2_degree1.csv}{fig_testcase1D_test2_cvg_Mult_case2_v1_param2_degree1_M0.0.csv}{fig_testcase1D_test2_cvg_Mult_case2_v1_param2_degree1_M0.0_weak.csv}{6e-5}
		\caption{Case of $\bm{\mu}^{(2)}$}
		\label{fig:case2param2_1D}
	\end{subfigure}
	\caption{Considering the \textit{1D Elliptic case} and the PINN prior $u_\theta$. Left -- Considering $\bm{\mu}^{(1)}$. $L^2$ error on $h$ obtained with standard FEM \smash{$e_h^{(1)}$} (solid lines) with $k=1$ and $k=2$, the additive approach \smash{$e_{h,+}^{(1)}$} (dashed lines) with $k=1$ and the multiplicative approach \smash{$e_{h,M}^{(1)}$} (dotted lines) with $k=1$, considering strong and weak BC. Right -- Same for $\bm{\mu}^{(2)}$, \eqref{equation:ell1d:mu-choices}.}
	\label{fig:case2_1D}
\end{figure}

In \cref{fig:case2_1D}, we see that the enriched approaches seem to give better results than standard FEM except for the multiplicative approach with strong imposition of boundary conditions. Moreover, this approach, which imposes $p_h^\times=1$ on $\partial\Omega$, does not follow the expected convergence order.
The additive approach seems much less effective here than in the previous
experiment of \cref{sec:Lap1D}, whereas the multiplicative approach with
weak BC seems to significantly improve the results obtained with standard FEM.
A comparative study of the different methods is given in \cref{sec:Ell1D_comparison}. In addition, we see that the standard FEM method with polynomial order $k=2$ is
clearly less accurate than the multiplicative approach using weak BC applied
with polynomial order $k=1$.

\subsubsection{Comparison of different approaches}\label{sec:Ell1D_comparison}

We now focus on the second parameter $\bm{\mu}^{(2)}$. We first look at the PINN prediction for this parameter and its derivatives in \cref{fig:case2_1D_der}.
As in the previous section, we consider the following approaches: standard FEM, the additive approach and the multiplicative approach (with $M=0$) with strong or weak imposition of boundary conditions.
We compare the different methods in \cref{tab:case2_1D_comparison}, where we can see the different errors obtained with the considered methods for $k=1$ and $N\in\{16,32\}$ as well as the gains obtained in comparison with standard FEM. Next, we take a closer look at the solutions obtained with the different approaches in \cref{fig:case2_1D_plots};
for each method, we compare the solution obtained ($u_h$ for standard FEM, $u_h^+$ for the additive approach and $u_h^\times$ for the multiplicative approach, with strong or weak BC imposition) with the analytical solution $u$. For the enriched methods, using the PINN prior, we will also compare the proposed correction; namely, for the additive approach, we will compare $p_h^+$ with $u-u_\theta$ and for the multiplicative one $p_h^\times$ with $u/u_\theta$ (with $u_\theta>0$ in $\Omega$).

\begin{figure}[ht!]
	\centering
	\includegraphics[scale=1]{fig_testcase1D_test2_plots_standalone_solutions_and_errors_PINN.pdf}
	\caption{Considering the \textit{1D Ellipctic case} with $\bm{\mu}^{(2)}$ and the PINN prior $u_\theta$, comparison between analytical solution and network prediction.
    From left to right: solution; first derivative; second derivative; errors.}
	\label{fig:case2_1D_der}
\end{figure}

In \cref{fig:case2_1D_der}, we can see that PINN has difficulties to capture the solution and that the prediction it provides is far from the analytical solution. As for its derivatives, they seem to be relatively inaccurate compared to the analytical.
Indeed, since the PINN is a smooth function, it has trouble approximating functions with very sharp gradients such as the one of \eqref{eq:Ell1D_analytical}.

\begin{table}[H]
	\centering
	\gainsPINNsecond{fig_testcase1D_test2_plots_FEM.csv}{fig_testcase1D_test2_plots_compare_gains.csv}
	\caption{Considering the \textit{1D Elliptic case} with $\bm{\mu}^{(2)}$, $k=1$ and $N=16,32$. Left -- $L^2$ relative error obtained with FEM. Right -- Considering the PINN prior $u_\theta$, $L^2$ relative errors and gains with respect to FEM, obtained with our methods. Our methods : additive approach, multiplicative approach by taking $M=0$ (strong and weak BC).}
	\label{tab:case2_1D_comparison}
\end{table}

\begin{figure}[!ht]
	\centering
    \begin{subfigure}{0.48\linewidth} \centering
		\hspace{-12pt}\includegraphics[scale=0.84]{fig_testcase1D_test2_plots_standalone_solutions_FEM.pdf}
		\caption{FEM solutions with polynomials order approximation $k=1$ and $k=2$, and absolute errors.}
		\label{fig:case2_1D_plots_fem}
    \end{subfigure} \hfill
    \begin{subfigure}{0.48\linewidth} \centering
		\hspace{-12pt}\includegraphics[scale=0.84]{fig_testcase1D_test2_plots_standalone_solutions_allP1.pdf}
		\caption{Enriched solutions with polynomial order approximation $k=1$, and absolute errors.}
		\label{fig:case2_1D_plots_add}
    \end{subfigure}
	\caption{Considering the \textit{1D Elliptic case} with $\bm{\mu}^{(2)}$, $N=16$ and the PINN prior $u_\theta$. Comparison of the solution obtained with the different methods with the analytical solution. For each enriched method, comparison of the correction term with the analytical one. Different methods : standard FEM, additive approach, multiplicative approach by taking $M=0$ (strong and weak BC).}
	\label{fig:case2_1D_plots}
\end{figure}

In \cref{fig:case2_1D_plots,tab:case2_1D_comparison}, we present a comparison of the different approaches proposed. In \cref{fig:case2_1D_plots_fem}, we first notice the oscillations anticipated in \cref{rem:oscillations} for standard FEM at both polynomial
orders $k=1$ and $k=2$.
This behaviour is also seen in the additive enrichment (blue dashed line in \cref{fig:case2_1D_plots_add}), which does not seem to give better results than standard FEM due to the derivatives presented in \cref{fig:case2_1D_der}, and for the multiplicative approach with strong boundary conditions (green dotted line in \cref{fig:case2_1D_plots_add}).
However, weakly imposing the BC gives the appropriate results (red dotted line in \cref{fig:case2_1D_plots_add}).

\subsubsection{Gains achieved with the additive and the multiplicative approaches}\label{sec:Ell1D_gains}

Considering a sample $\mathcal{S}$ of $n_p=50$ parameters, we will evaluate the gains $G_{+,\theta}$ and $G_+$ defined in \eqref{eq:gain_add_num} with the PINN prior $u_\theta$. We will evaluate $G_{M,\theta}$ and $G_M$, defined in \eqref{eq:gain_mul_num}, in the same way with $M=0$ for the multiplicative approach, by considering the two implementations of the boundary conditions; strong and weak BC. The results are presented in \cref{tab:case2_1D_PINNs} for fixed $k=1$ and $N \in \{20,40\}$ fixed.

\begin{table}[H]
	\centering
	\gainstableMult{fig_testcase1D_test2_gains_Tab_stats_case2_v1_degree1.csv}
	\caption{Considering the \textit{1D Ellipctic case}, $k=1$ and the PINN prior $u_\theta$. Left -- Gains in $L^2$ relative error of our methods with respect to PINN. Right -- Gains in $L^2$ relative error of our methods with respect to FEM. Our methods : additive approach, multiplicative approach with $M=0$ (strong and weak BC).}
	\label{tab:case2_1D_PINNs}
\end{table}

\cref{tab:case2_1D_PINNs} confirms the above results. The multiplicative approach with weak BC seems to give the best results on our $\mathcal{S}$ parameter sample. The additive and multiplicative approaches with strong BC imposition do not appear to be very effective on this test case. In particular, even though the additive approach improves the standard FEM error by a factor of 3, we have seen in \cref{sec:Ell1D_comparison} that the solutions obtained do not correspond to the expected solution, whereas the multiplicative approach with low BCs does. In the following, we will only consider the additive approach, as it seems to be the most efficient one, except in special cases such as the one under consideration in this \cref{sec:Ell1D}. Indeed, the following test cases will not contain boundary layers or strong gradients.


\subsection{2D Poisson problem in a square domain} \label{sec:Lap2D}

We now consider the problem of \cref{sec:Lap1D}
but in two dimensions ($d=2$), with,
\begin{equation}
	\left\{
	\begin{aligned}
		-\Delta u & = f, \; &  & \text{in } \; \Omega \times \mathcal{M}, \\
		u         & =0, \;  &  & \text{on } \; \partial\Omega \times \mathcal{M},
	\end{aligned}
	\right.
	\label{eq:Lap2D}
\end{equation}
with $\Delta$ the Laplace operator on the domain
$\Omega=[-0.5 \pi, 0.5 \pi]^2$ with boundary $\partial\Omega$,
and $\mathcal{M} \subset \mathbb{R}^p$ the parameter space (with $p$ the number of parameters).
We define the right-hand side $f$ such that the solution is given by
\begin{equation}
    \label{eq:analytical_solution_Lap2D}
	u(\bm{x},\bm{\mu})=\exp\left(-\frac{(x-\mu_1)^2+(y-\mu_2)^2}{2}\right)\sin(\kappa x)\sin(\kappa y),
\end{equation}
with $\bm{x}=(x,y)\in\Omega$ and some parameters $\bm{\mu}=(\mu_1,\mu_2) \in \mathcal{M}=[-0.5,0.5]^p$, hence with $p=2$ parameters.
With an abuse of language as well,
we refer to the quantity $\kappa$ in \eqref{eq:analytical_solution_Lap2D}
as the frequency of the solution, in the sense that it characterizes
the number of oscillations in the solution.

We start with a ``low frequency'' case in \cref{sec:Lap2Dlow}, taking $\kappa = 2$ and considering a PINN where we impose the Dirichlet boundary conditions as presented in \cref{sec:exact_imposition_of_BC}., i.e. using a level-set function.
To further improve the prior quality, we introduce an augmented loss function in \cref{sec:Lap2Dlowaug} by using the Sobolev training presented in \cref{sec:sobolev_training}.
Afterwards, we test another loss in \cref{sec:Lap2Dlowbc} that includes the Dirichlet condition, or in other words, that does not use a level-set function.
Finally, we consider a ``higher frequency''
case in \cref{sec:Lap2Dhigh}, with $\kappa = 8$.

\begin{remark}\label{rmk:Lap2D_N_nodes}
	In the following, the characteristic mesh size $h=\frac{\pi\sqrt{2}}{N-1}$ is defined as a function of $N$, considering a cartesian mesh of $N^2$ nodes for our squared 2D domain of length $\pi$.
\end{remark}

\subsubsection{Low-frequency case}\label{sec:Lap2Dlow}

We consider a ``low-frequency'' problem, taking $\kappa = 2$.
In this section, we consider the additive approach, as presented in \cref{sec:additive_prior}, by considering the PINN prior $u_\theta$. We start by testing the error estimates in \cref{par:Lap2D_error_estimations} (in $L^2$ norm) with polynomial order
$k\in\{1,2,3\}$, then we compare the derivatives of the prior and compare the different approaches in \cref{par:Lap2D_comparison}. We evaluate the gains obtained in \cref{par:Lap2D_gains} on a sample of parameters. Then, we compare the numerical costs of the different methods in \cref{par:Lap2D_costs}. Finally we discuss the importance of integrating analytical functions in \cref{par:Lap2D_sup}, as presented in \cref{sec:using_PINN}.

\paragraph*{Physics-informed training.}
Since the problem under consideration is parametric
we deploy a parametric PINN,
which depends on both the space variable $\bm{x}=(x,y) \in \Omega$
and the parameters $\bm{\mu}=(\mu_1,\mu_2) \in \mathcal{M}$.
Moreover, we strongly impose the Dirichlet boundary conditions,
as presented in \cref{sec:PINNs_parametric_PDE}.
To do this, we define the prior
\begin{equation*}
	u_{\theta}(\bm{x},\bm{\mu}) = \varphi(\bm{x}) w_{\theta}(\bm{x},\bm{\mu}) \,,
\end{equation*}
where $w_\theta$ is the neural network under consideration and $\varphi$ is a level-set function defined by
\begin{equation*}
	\varphi(\bm{x})=(x+0.5\pi)(x-0.5\pi)(y+0.5\pi)(y-0.5\pi),
\end{equation*}
which exactly cancels out on $\partial\Omega$. Since we impose the boundary conditions by using the level-set function, we will only consider the residual loss $J_r$ defined in \eqref{eq:residual_loss_parametric} in which the integrals are approached by a Monte-Carlo method, i.e.
\[
	J_r(\theta) \simeq
	\frac{1}{N_\text{col}} \sum_{i=1}^{N_\text{col}} \big| \Delta u_\theta(\bm{x}_\text{col}^{(i)};\bm{\mu}_\text{col}^{(i)}\big) + f\big(\bm{x}_\text{col}^{(i)};\bm{\mu}_\text{col}^{(i)}\big) \big|^2
\]
with the $N_\text{col}=6000$ collocation points \smash{$\big(\bm{x}_\text{col}^{(i)}, \bm{\mu}_\text{col}^{(i)}\big)_{i=1,\dots,N_\text{col}}$} uniformly chosen on $\Omega\times\mathcal{M}$. Thus, we seek to solve the following minimization problem
\begin{equation*}
	\theta^\star = \argmin_\theta J_r(\theta).
\end{equation*}
The parametric PINN $w_\theta$ is defined as an MLP with the hyperparameters defined in \cref{tab:paramtest1_2D}; we use the Adam optimizer and then switch to the LBFGS optimizer after the $n_\text{switch}$-th epoch.

\begin{table}[htbp]
    \centering
    \begin{tabular}{cc}
        \toprule
        \multicolumn{2}{c}{\textbf{Network - MLP}} \\
        \midrule
        \textit{layers} & $40,60,60,60,40$ \\
        \cmidrule(lr){1-2}
        $\sigma$ & sine \\
        \bottomrule
    \end{tabular}
    \hspace{1cm}
    \begin{tabular}{cccc}
        \toprule
        \multicolumn{4}{c}{\textbf{Training - with LBFGS}} \\
        \midrule
        \textit{lr} & 1.7e-2 & $n_\text{epochs}$ & 5000 \\
        \cmidrule(lr){1-2} \cmidrule(lr){3-4}
        \textit{decay} & 0.99 & $n_\text{switch}$ & 1000 \\
        \cmidrule(lr){1-2} \cmidrule(lr){3-4}
        $N_\text{col}$ & 6000 \\
        \bottomrule
    \end{tabular}
    \hspace{1cm}
    \begin{tabular}{cccc}
        \toprule
        \multicolumn{4}{c}{\textbf{Loss weights}} \\
        \midrule
        $\omega_r$ & 1 & $\omega_\text{data}$ & 0 \\
        \cmidrule(lr){1-2} \cmidrule(lr){3-4}
        $\omega_b$ & 0 & $\omega_\text{sob}$ & 0 \\        
        \bottomrule
    \end{tabular}
    \caption{Network, training parameters (\cref{rmk:PINN_notations}) and loss weights for $u_\theta$ in the \textit{2D Laplacian} case.}
    \label{tab:paramtest1_2D}
\end{table}


\paragraph{Error estimates}\label{par:Lap2D_error_estimations}\mbox{} \\

We start by testing the error estimates of \cref{lem:error_estimation_add} for the following two sets of parameters, randomly selected in $\mathcal{M}$:
\begin{equation}\label{equation:lap2d:mu-choices}
	\bm{\mu}^{(1)}=(0.05,0.22) \quad \text{and} \quad \bm{\mu}^{(2)}=(0.1,0.04)
\end{equation}
by considering the PINN prior $u_\theta$. So, for $ j\in \{1,2\}$, the aim is to compare, by varying the mesh size $h$, the $L^2$ relative errors $e_h^{(j)}$ obtained with the standard FEM method, defined in \eqref{eq:error_rel_FEM}, and $e_{h,+}^{(j)}$ obtained with the additive approach, defined in \eqref{eq:error_rel_add}.
The results are presented in \cref{fig:case1} for fixed $k \in \{1,2,3\}$ with $h$ depending on $N\in\{16,32,64,128,256\}$ as presented in \cref{rmk:Lap2D_N_nodes}.

\begin{figure}[H]
	\centering
	\begin{subfigure}{0.48\linewidth}
		\centering
		\cvgFEMCorrAlldeg{fig_testcase2D_test1_cvg_FEM_case1_v1_param1.csv}{fig_testcase2D_test1_cvg_Corr_case1_v1_param1.csv}{1e-10}
		\caption{Case of $\bm{\mu}^{(1)}$}
		\label{fig:case1param1}
	\end{subfigure}
	\begin{subfigure}{0.48\linewidth}
		\centering
		\cvgFEMCorrAlldeg{fig_testcase2D_test1_cvg_FEM_case1_v1_param2.csv}{fig_testcase2D_test1_cvg_Corr_case1_v1_param2.csv}{1e-10}
		\caption{Case of $\bm{\mu}^{(2)}$}
		\label{fig:case1param2}
	\end{subfigure}
	\caption{Considering the \textit{2D low-frequency} case and the PINN prior $u_\theta$. Left -- $L^2$ relative error on $h$, obtained with the standard FEM $e_h^{(1)}$ (solid lines) and the additive approach $e_{h,+}^{(1)}$ (dashed lines) for $\bm{\mu}^{(1)}$, with $k \in \{1,2,3\}$. Right -- Same for $\bm{\mu}^{(2)}$, \eqref{equation:lap2d:mu-choices}.}
	\label{fig:case1}
\end{figure}

In \cref{fig:case1}, we observe the expected behaviour.
Indeed, the error decreases with the correct order of accuracy as the mesh size $h$ decreases.
This observation is valid for both the classical and enriched FEM.
Moreover, we observe that the error constant of the additive approach is significantly lower than that of the classical FEM. In \cref{par:Lap2D_comparison}, we will compare these different approaches in more detail. As noted in the 1D case, we can see that the additive enriched approach for $k=1$ (resp. $k=2$) seems to give the same results as standard FEM for $k=2$ (resp. $k=3$).

\paragraph{Comparison of different approaches}\label{par:Lap2D_comparison}\mbox{} \\

We now focus on the first parameter $\bm{\mu}^{(1)}$. We compare the standard FEM method with the additive approach, first in the \cref{tab:case1_2D_comparison} where we can see the different errors obtained with the different methods for $k=1$ fixed and $N\in\{16,32\}$ as well as the gains obtained in comparison with standard FEM. Next, we take a closer look at the solution obtained with the different approaches in \cref{fig:case1_2D_plots}; for each method, we compare the solution obtained ($u_h$ for standard FEM and $u_h^+$ for the additive approach) with the analytical solution $u$. For the enriched method, using the PINN prior $u_\theta$, we will also compare the proposed correction; namely, for the additive approach, we will compare $p_h^+$ with $u-u_\theta$.

\begin{table}[H]
	\centering
	\gainsPINNfirst{fig_testcase2D_test1_plots_FEM.csv}{fig_testcase2D_test1_plots_compare_gains.csv}
	\caption{Considering the \textit{2D low-frequency} case with $\bm{\mu}^{(1)}$, $k=1$ and $N\in\{16,32\}$. Left -- $L^2$ relative error obtained with FEM. Right -- Considering the PINN prior $u_\theta$, $L^2$ relative errors and gains with respect to FEM, obtained with the additive approach.}
	\label{tab:case1_2D_comparison}
\end{table}

\begin{figure}[!ht]
	\centering
    \includegraphics[scale=1]{fig_testcase2D_test1_plots_standalone_solutions.pdf}

    \includegraphics[scale=1]{fig_testcase2D_test1_plots_standalone_errors_v2.pdf}

	\caption{Considering the \textit{2D low-frequency} case with $\bm{\mu}^{(1)}$, $k=1$, $N=16$ and the PINN prior $u_\theta$. Comparison of the solution obtained with the standard FEM and the additive approach with the analytical solution. For the additive method, comparison of the correction term with the analytical one.}
	\label{fig:case1_2D_plots}
\end{figure}

In \cref{tab:case1_2D_comparison}, we observe that the additive approach significantly improves the error of the standard FEM, with gains of around $260$ for $N=16$ and $k=1$, which is equivalent to refining the mesh by a factor of $16$ for $\mathbb{P}_1$ elements. Indeed, in this case, our enriched approach gives much better results than standard FEM, notably on coarse meshes. In \cref{fig:case1_2D_plots}, we observe that the solution obtained with the additive approach is very close to the analytical solution, with a correction term that is also very close to the analytical one. This shows the effectiveness of the additive approach in this case.

\paragraph{Gains achieved with the additive approach}\label{par:Lap2D_gains}\mbox{} \\

Considering a sample $\mathcal{S}$ of $n_p=50$ parameters,
we now evaluate the gains $G_{+,\theta}$ and $G_+$ defined in \eqref{eq:gain_add_num}.
The results are presented in \cref{tab:case1_2D}
for $k \in \{1,2,3\}$ and $N \in \{20, 40\}$.

\begin{table}[H]
	\centering
	\gainstableallq{fig_testcase2D_test1_gains_Tab_stats_case1_v1.csv}
	\caption{Considering the \textit{2D low-frequency} case, $k\in\{1,2,3\}$ and the PINN prior $u_\theta$. Left -- Gains in $L^2$ relative error of the additive method with respect to PINN. Right -- Gains in $L^2$ relative error of our approach with respect to FEM.}
	\label{tab:case1_2D}
\end{table}

In \cref{tab:case1_2D}, we observe (left subtable) that our method
significantly improves the error of the PINN,
especially for large values of $k$,
where the enrichment is performed in a richer approximation space.
Moreover, we also observe (right subtable) significant
gains with respect to classical FEM.
For instance, as expected from the results of \cref{par:Lap2D_comparison}, the mean gains for $k=1$ are around $270$, which corresponds to refining the mesh approximately $16$ times for $\mathbb{P}_1$ elements. This means that our $\mathbb{P}_1$ enhanced bases capture the solution as accurately as classical $\mathbb{P}_1$ bases with a mesh four times finer.
For $k=2$ and $k=3$, the mean gains are around $134$ and $61$, respectively, which corresponds to refining the mesh approximately $5$ times for $\mathbb{P}_2$ elements and $2.8$ times for $\mathbb{P}_3$ elements. A natural follow-up question consists in assessing the impact of the PINN quality on our results. This will be the subject of the \cref{sec:Lap2Dlowaug}.

\paragraph{Costs of the different methods}\label{par:Lap2D_costs}\mbox{} \\

To more accurately assess the benefits of using the enriched methods, we look in this section at the costs of the different methods proposed, considering the parameter $\bm{\mu}^{(1)}$. Thus, we will consider that the cost of using the PINN prior $u_\theta$, corresponds to the total number of weights of the network considered. In this case, it is given as an MLP with the hyperparameters defined in \cref{tab:paramtest1_2D}, for a total of $N_\text{weights}=12,461$ weights.

For the different finite element methods, we will then try to determine, for a fixed polynomial degree $k$, the characteristic mesh size $h$ (depending on $N$ as described in \cref{rmk:Lap2D_N_nodes}) required to reach a fixed error $e$. In \cref{tab:case1_2D_costs}, we study, for $k\in\{1,2,3\}$, considering standard FEM and the additive approach, the $N$ required to achieve the same error $e$. More precisely, the characteristic mesh size required by standard FEM so that \smash{$e_h^{(1)}\approx e$} and the one required by the additive approach so that \smash{$e_{h,+}^{(1)}\approx e$}. Depending on the polynomial degree $k$, we can also determine the number of degrees of freedom $N_\text{dofs}$ associated with each case.

\begin{table}[H]
	\centering
	\coststableallq{fig_testcase2D_test1_costs_TabDoFs_case1_v1_param1.csv}
	\caption{Considering the \textit{2D low-frequency} case with $\bm{\mu}^{(1)}$, $q\in\{1,2,3\}$ and the PINN prior $u_\theta$. Left -- Characteristic $N$ (associated to the characteristic mesh size $h$) required to reach a fixed error $e$ for standard FEM and the additive approach. Right -- Number of degrees of freedom $N_\text{dofs}$ associated with each case.}
	\label{tab:case1_2D_costs}
\end{table}

In \cref{tab:case1_2D_costs}, we see that the additive approach proposed in \cref{sec:additive_prior} requires a much coarser mesh than standard FEM to achieve the same $e$ error. This is due to the error estimations of \cref{lem:error_estimation_add} which show that the error of the enhanced FEM is significantly lower than that of the classical FEM (depending on the quality of the prior). This is also reflected in the number of degrees of freedom required to achieve the same error $e$.

\begin{remark}
	The results in \cref{tab:case1_2D_costs} have been obtained by interpolating the convergence curves of \cref{fig:case1} for the different methods for a given $e$.
\end{remark}

However, the enriched approaches proposed require using the prior PINN $u_\theta$, which also includes its cost of use. For this reason, it will be interesting to look at these same costs on a set of parameters, say of size $n_p=100$. Since we are in the context of parametric PINN, we can estimate that the computational cost of solving \eqref{eq:Lap2D} on this sample of $n_p$ parameters corresponds, for the additive approach, to $n_p$ times its number of dofs plus the cost of using PINN (i.e. its total number of weights), thus $n_p\times N_\text{dofs}+N_\text{weights}$ (with $N_\text{dofs}$ the number of dofs associated to the additive approach). For standard FEM, this cost is equivalent to $n_p$ times its estimated number of dofs $n_p\times N_\text{dofs}$ (with $N_\text{dofs}$ the number of dofs associated with standard FEM). We will then compare these costs for a set of $n_p=100$ parameters, considering the same error $e$ to be achieved for both methods. The results are presented in \cref{tab:case1_2D_costs100}.

\begin{table}[H]
	\centering
	\coststableallqhundred{fig_testcase2D_test1_costs_TabDoFsParam_case1_v1_param1_nparams100.csv}
	\caption{Considering the \textit{2D low-frequency} case, $k\in\{1,2,3\}$ and the PINN prior $u_\theta$. Left -- Total costs of standard FEM and the additive approach to reach an error $e$ for a set of $n_p=1$ parameter. Right -- Same for a set of $n_p=100$ parameters.}
	\label{tab:case1_2D_costs100}
\end{table}

\begin{remark}
	Note that the \cref{tab:case1_2D_costs100} (right sub-table) is, in fact, only an estimate of the real cost of solving $n_p$ problems. In practice, the number of degrees of freedom $N_\text{dofs}$ associated with each method depends on the parameter itself. The error to be achieved $e$ will require more or less fine meshes for each parameter.
\end{remark}

In \cref{tab:case1_2D_costs100} for $n_p=1$ (left subtable), we can see that the cost of the additive method is generally lower than that of standard FEM, even though they are of the same range. However, it is important to note that these are not entirely comparable: in fact, a large part of the cost of the additive method lies in PINN prediction, which will be more or less well estimated depending on a number of hyper-parameters (number of epochs, learning rate, etc.). If we then take $n_p=100$ (right subtable), we can see that the cost of standard FEM becomes radically higher than with the additive approach. This is why the improved approach is particularly interesting for solving the \eqref{eq:Lap2D} problem on a set of parameters.

\paragraph{Integration of analytical functions}\label{par:Lap2D_sup}\mbox{} \\


This section aims to discuss one of the important points, presented in \cref{sec:using_PINN}, that enables PINN to be used effectively and can make our enriched methods more or less effective.
Indeed, according to \eqref{eq:approachform_add}, we have to integrate $f+\Delta u_\theta$ multiplied by the test function.
To perform this integration, we first interpolate this term on a polynomial space and then integrate it exactly.
The degree of this polynomial approximation is an important parameter to make our technique effective.

The goal here is simply to show that for enriched approaches to be effective, particularly the additive method, this polynomial approximation must be of a sufficiently high degree.
Consider the parameter $\bm{\mu}^{(1)}$, a polynomial degree $k=3$ and a number of nodes $N=128$.
In \cref{fig:case1_2D_highdeg}, we display the $L^2$ error of the additive approach with respect to the degree of polynomial approximation of $f+\Delta u_\theta$.

\begin{figure}[!ht]
	\centering
	\includegraphics[scale=0.8]{fig_testcase2D_test1_supp_standalone_high-degree.pdf}
	\caption{Considering the \textit{2D low-frequency} case with $\bm{\mu}^{(1)}$, $q=3$, $N=128$ and the PINN prior $u_\theta$. Considering the additive approach, $L^2$ error $e_{h,+}^{(1)}$ with respect to the degree of polynomial approximation of $f+\Delta u_\theta$.}
	\label{fig:case1_2D_highdeg}
\end{figure}

In \cref{fig:case1_2D_highdeg}, we observe that the error decreases as the degree of polynomial approximation increases. This shows the importance of properly interpolating analytical functions in the context of enriched methods.

\subsubsection{Low-frequency case --- Sobolev training}\label{sec:Lap2Dlowaug}

This section focuses on the same problem as in \cref{sec:Lap2Dlow}. The aim here is to show that the network quality has a non-negligible impact on the results obtained with our method and that if the network is better, our results will be, too. To this end, we defined a new prior $u_\theta^\text{sob}$ by using the Sobolev training presented in \cref{sec:sobolev_training} , where the derivatives of the solution should be better approximated than by a standard training and compared it with the PINN prior $u_\theta$ defined in \cref{sec:Lap2Dlow}. We start by testing the error estimation in \cref{par:Lap2Dlowaug_error_estimations} with $k\in\{1,2,3\}$ polynomial order and evaluate the gains obtained in \cref{par:Lap2Dlowaug_gains} on the same sample of parameters as in \cref{sec:Lap2Dlow}.

\paragraph*{Physics-informed training.} We deploy here a parametric PINN, denoted by $u_\theta^\text{sob}$, where we strongly impose the Dirichlet boundary conditions as in \cref{sec:Lap2Dlow}. The hyperparameters are defined in \cref{tab:paramtest1v7_2D};
we use the Adam optimizer and consider $N_\text{col}$ collocation points, uniformly chosen on $\Omega$.

\begin{remark}
	Adding the Sobolev loss can make training more difficult, so we only consider 3000 epochs but a batch size of 2000. This means that for each epoch, the weights will be updated 3 times (because $N_\text{col}=6000$).
\end{remark}

\begin{table}[htbp]
    \centering
    \begin{tabular}{cc}
        \toprule
        \multicolumn{2}{c}{\textbf{Network - MLP}} \\
        \midrule
        \textit{layers} & $40,60,60,60,40$ \\
        \cmidrule(lr){1-2}
        $\sigma$ & sine \\
        \bottomrule
    \end{tabular}
    \hspace{1cm}
    \begin{tabular}{cccc}
        \toprule
        \multicolumn{4}{c}{\textbf{Training - with LBFGS}} \\
        \midrule
        \textit{lr} & 1.7e-2 & $n_\text{epochs}$ & 3000 \\
        \cmidrule(lr){1-2} \cmidrule(lr){3-4}
        \textit{decay} & 0.99 & batch size & 2000 \\
        \cmidrule(lr){1-2} \cmidrule(lr){3-4}
        $N_\text{col}$ & 6000 \\
        \bottomrule
    \end{tabular}
    \hspace{1cm}
    \begin{tabular}{cccc}
        \toprule
        \multicolumn{4}{c}{\textbf{Loss weights}} \\
        \midrule
        $\omega_r$ & 1 & $\omega_\text{data}$ & 0 \\
        \cmidrule(lr){1-2} \cmidrule(lr){3-4}
        $\omega_b$ & 0 & $\omega_\text{sob}$ & 0.1 \\        
        \bottomrule
    \end{tabular}
    \caption{Network, training parameters (\cref{rmk:PINN_notations}) and loss weights for $u_\theta^\text{sob}$ in the \textit{2D Laplacian} case.}
    \label{tab:paramtest1v7_2D}
\end{table}


We consider the residual loss $J_r$ defined in \cref{sec:Lap2Dlow} and the Sobolev loss $J_\text{sob}$ defined in \eqref{eq:sobolev_loss}, whose integrals are both approximated by a Monte-Carlo method. We then seek to solve the following minimisation problem
\begin{equation*}
	\theta^\star = \argmin_\theta \omega_r J_r(\theta) + \omega_\text{sob} J_\text{sob}(\theta).
\end{equation*}

\paragraph{Error estimates}\label{par:Lap2Dlowaug_error_estimations}\mbox{} \\

For simplicity, we consider the first parameter $\bm{\mu}^{(1)}=(0.05,0.22)$ presented in \cref{sec:Lap2Dlow}. By varying the mesh size $h$, we compare the $L^2$ relative errors between the classical FEM (\smash{$e_h^{(1)}$} defined in \eqref{eq:error_rel_FEM}), the enhanced FEM with the prior $u_\theta$ from \cref{sec:Lap2Dlow}, and the enhanced FEM with the current prior $u_\theta^\text{sob}$ (\smash{$e_{h,+}^{(1)}$} defined in \eqref{eq:error_rel_add}). The results are presented in \cref{fig:case1v7} for fixed $k\in \{1,2,3\}$.

\begin{figure}[H]
	\centering
	\hspace{-1cm}
	\begin{subfigure}{0.32\linewidth}
		\centering
		\resizebox{1.1\linewidth}{!}{
			\cvgFEMCorrAugFirst{fig_testcase2D_test1_cvg_FEM_case1_v1_param1.csv}{fig_testcase2D_test1_cvg_Corr_case1_v1_param1.csv}{fig_testcase2D_test1_v7_cvg_Corr_case1_v7_param1.csv}{8e-7}{$u_\theta^\text{sob}$}
		}
		\caption{$k=1$}
		\label{fig:case1v7param1deg1}
	\end{subfigure}
	\begin{subfigure}{0.32\linewidth}
		\centering
		\resizebox{1.1\linewidth}{!}{
			\cvgFEMCorrAugSecond{fig_testcase2D_test1_cvg_FEM_case1_v1_param1.csv}{fig_testcase2D_test1_cvg_Corr_case1_v1_param1.csv}{fig_testcase2D_test1_v7_cvg_Corr_case1_v7_param1.csv}{4e-9}{$u_\theta^\text{sob}$}
		}
		\caption{$k=2$}
		\label{fig:case1v7param1deg2}
	\end{subfigure}
	\begin{subfigure}{0.32\linewidth}
		\centering
		\resizebox{1.1\linewidth}{!}{
			\cvgFEMCorrAugThird{fig_testcase2D_test1_cvg_FEM_case1_v1_param1.csv}{fig_testcase2D_test1_cvg_Corr_case1_v1_param1.csv}{fig_testcase2D_test1_v7_cvg_Corr_case1_v7_param1.csv}{4e-11}{$u_\theta^\text{sob}$}
		}
		\caption{$k=3$}
		\label{fig:case1v7param1deg3}
	\end{subfigure}
	\caption{Considering the \textit{2D low-frequency} case with $\bm{\mu}^{(1)}$. Left -- $L^2$ relative error on $h$, obtained with the standard FEM $e_h^{(1)}$ (solid line) and the additive approach $e_{h,+}^{(1)}$ (dashed lines), with $k=1$, by considering the PINN prior with standard training $u_\theta$ and Sobolev training $u_\theta^\text{sob}$. Middle -- Same with $k=2$. Right -- Same with $k=3$.}
	\label{fig:case1v7}
\end{figure}

We observe that the Sobolev training improves the results obtained with the $L^2$ training,
for $k\in\{1,2,3\}$.
This shows the impact of the quality of the network prediction on our method.
To further investigate this, we evaluate in \cref{par:Lap2Dlowaug_gains} the gains obtained with the Sobolev training on the same sample of parameters as in \cref{sec:Lap2Dlow}.

\paragraph{Gains achieved with the additive approach}\label{par:Lap2Dlowaug_gains}\mbox{} \\

Considering the same sample $\mathcal{S}$ of $n_p=50$ parameters as in \cref{sec:Lap2Dlow}, we now evaluate the gains $G_{+,\theta}$ and $G_+$ defined in \eqref{eq:gain_add_num} considering the PINN prior $u_\theta^\text{sob}$ using Sobolev training. The results are presented in \cref{tab:case1v7} for $k \in \{1,2,3\}$ and $N \in \{20,40\}$ (with $h$ depending on $N$ as described in \cref{rmk:Lap2D_N_nodes}).

\begin{table}[H]
	\centering
	\gainstableallq{fig_testcase2D_test1_v7_gains_Tab_stats_case1_v7.csv}
	\caption{Considering the \textit{2D low-frequency} case, $k\in\{1,2,3\}$ and the PINN prior $u_\theta^\text{sob}$ (Sobolev training). Left -- Gains in $L^2$ relative error of the additive method with respect to PINN. Right -- Gains in $L^2$ relative error of our approach with respect to FEM.}
	\label{tab:case1v7}
\end{table}

The gains reported in \cref{tab:case1v7} show that, compared to $L^2$ training,
Sobolev training increases the mean gains by a factor of about $3$.
This corresponds to almost half an additional mesh refinement for the $\mathbb{P}_1$ elements. We also note that this Sobolev training is particularly interesting for higher polynomial degrees, with standard $L^2$ training having lower gains than for $k=1$.
\subsubsection{Low-frequency case --- Boundary loss training}\label{sec:Lap2Dlowbc}

Now, we focus on the same problem as in \cref{sec:Lap2Dlow}and \cref{sec:Lap2Dlowaug}. We now turn to a standard PINN, denoted by $u_\theta^\text{bc}$, where we impose the boundary conditions in the loss (no longer with the level-set function). The aim here is to show that our enriched methods also work with priors that do not have exact boundary conditions. To this end, we start by testing the error estimation in \cref{par:Lap2Dlowbc_error_estimations} with $k\in\{1,2,3\}$ polynomial order and evaluate the gains obtained in \cref{par:Lap2Dlowbc_gains} on the same sample of parameters as in \cref{sec:Lap2Dlow}.

\paragraph*{Physics-informed training.} Since the problem under consideration is parametric, we deploy a parametric PINN and define the prior $u_{\theta}^\text{bc}$ as an MLP with the hyperparameters defined in \cref{tab:paramtest1v3_2D}; we use the Adam optimizer and then switch to the LBFGS optimizer after the $n_\text{switch}$-th epoch. We consider $N_\text{col}=6000$ collocation points, uniformly chosen on $\Omega$.

\begin{table}[htbp]
    \centering
    \begin{tabular}{cc}
        \toprule
        \multicolumn{2}{c}{\textbf{Network - MLP}} \\
        \midrule
        \textit{layers} & $40,60,60,60,40$ \\
        \cmidrule(lr){1-2}
        $\sigma$ & sine \\
        \bottomrule
    \end{tabular}
    \hspace{1cm}
    \begin{tabular}{cccc}
        \toprule
        \multicolumn{4}{c}{\textbf{Training - with LBFGS}} \\
        \midrule
        \textit{lr} & 1.7e-2 & $n_\text{epochs}$ & 5000 \\
        \cmidrule(lr){1-2} \cmidrule(lr){3-4}
        \textit{decay} & 0.99 & $n_\text{switch}$ & 1000 \\
        \cmidrule(lr){1-2} \cmidrule(lr){3-4}
        $N_\text{col}$ & 6000 & $N_\text{bc}$ & 2000 \\
        \bottomrule
    \end{tabular}
    \hspace{1cm}
    \begin{tabular}{cccc}
        \toprule
        \multicolumn{4}{c}{\textbf{Loss weights}} \\
        \midrule
        $\omega_r$ & 1 & $\omega_\text{data}$ & 0 \\
        \cmidrule(lr){1-2} \cmidrule(lr){3-4}
        $\omega_b$ & 30 & $\omega_\text{sob}$ & 0 \\        
        \bottomrule
    \end{tabular}
    \caption{Network, training parameters (\cref{rmk:PINN_notations}) and loss weights for $u_\theta^\text{bc}$ in the \textit{2D Laplacian} case.}
    \label{tab:paramtest1v3_2D}
\end{table}


We consider the same residual loss as in \cref{sec:Lap2Dlow} and the BC loss (where integral are approached by a Monte-Carlo method) is defined by
\[
	J_b(\theta) \simeq
	\frac{1}{N_\text{bc}} \sum_{i=1}^{N_\text{bc}} \big| u_\theta^\text{bc}\big(\bm{x}_\text{bc}^{(i)};\bm{\mu}_\text{bc}^{(i)}\big)\big|^2,
\]
with $N_\text{bc}=2000$ boundary collocation points \smash{$\big(\bm{x}_\text{bc}^{(i)}, \bm{\mu}_\text{bc}^{(i)}\big)_{i=1,\dots,N_\text{bc}}$}. Thus we seek to solve the following minimisation problem
\begin{equation*}
	\theta^\star = \argmin_\theta J_r(\theta) + \omega_b J_b(\theta).
\end{equation*}

\paragraph{Error estimates}\label{par:Lap2Dlowbc_error_estimations}\mbox{} \\

We consider the first parameter $\bm{\mu}^{(1)}=(0.05,0.22)$ presented in \cref{sec:Lap2Dlow} and by varying the mesh size $h$, we compare the $L^2$ relative errors between the classical FEM (\smash{$e_h^{(1)}$} defined in \eqref{eq:error_rel_FEM}), the enhanced FEM with the prior $u_\theta$ from \cref{sec:Lap2Dlow}, and the enhanced FEM with the current prior $u_\theta^\text{bc}$ (\smash{$e_{h,+}^{(1)}$} defined in \eqref{eq:error_rel_add}). The results are presented in \cref{fig:case1v3} for fixed $k\in \{1,2,3\}$.

\begin{figure}[H]
	\centering
	\hspace{-1cm}
	\begin{subfigure}{0.32\linewidth}
		\centering
		\resizebox{1.1\linewidth}{!}{
			\cvgFEMCorrAugFirst{fig_testcase2D_test1_cvg_FEM_case1_v1_param1.csv}{fig_testcase2D_test1_cvg_Corr_case1_v1_param1.csv}{fig_testcase2D_test1_v3_cvg_Corr_case1_v3_param1.csv}{3e-6}{$u_\theta^\text{bc}$}
		}
		\caption{$k=1$}
		\label{fig:case1v3param1deg1}
	\end{subfigure}
	\begin{subfigure}{0.32\linewidth}
		\centering
		\resizebox{1.1\linewidth}{!}{
			\cvgFEMCorrAugSecond{fig_testcase2D_test1_cvg_FEM_case1_v1_param1.csv}{fig_testcase2D_test1_cvg_Corr_case1_v1_param1.csv}{fig_testcase2D_test1_v3_cvg_Corr_case1_v3_param1.csv}{1e-8}{$u_\theta^\text{bc}$}
		}
		\caption{$k=2$}
		\label{fig:case1v3param1deg2}
	\end{subfigure}
	\begin{subfigure}{0.32\linewidth}
		\centering
		\resizebox{1.1\linewidth}{!}{
			\cvgFEMCorrAugThird{fig_testcase2D_test1_cvg_FEM_case1_v1_param1.csv}{fig_testcase2D_test1_cvg_Corr_case1_v1_param1.csv}{fig_testcase2D_test1_v3_cvg_Corr_case1_v3_param1.csv}{8e-11}{$u_\theta^\text{bc}$}
		}
		\caption{$k=3$}
		\label{fig:case1v3param1deg3}
	\end{subfigure}
	\caption{Considering the \textit{2D low-frequency} case with $\bm{\mu}^{(1)}$. Left -- $L^2$ relative error on $h$, obtained with the standard FEM $e_h^{(1)}$ (solid line) and the additive approach $e_{h,+}^{(1)}$ (dashed lines), with $k=1$, by considering the PINN prior with standard training $u_\theta$ and the BC loss training $u_\theta^\text{bc}$. Middle -- Same with $k=2$. Right -- Same with $k=3$.}
	\label{fig:case1v3}
\end{figure}

We can see in \cref{fig:case1v3} that the additive approach also works when the prior is not exact on the boundary, as here with $u_\theta^\text{bc}$. In particular, for $k\in\{1,2,3\}$ and the parameter $\bm{\mu}^{(1)}$, our enriched approach using the prior $u_\theta^\text{bc}$ seems to give very similar results to those obtained with $u_\theta$ even if the approach with level-set is, for this case, slightly better.

\paragraph{Gains achieved with the additive approach}\label{par:Lap2Dlowbc_gains}\mbox{} \\

Considering the same sample $\mathcal{S}$ of $n_p=50$ parameters as in \cref{sec:Lap2Dlow}, we now evaluate the gains $G_{+,\theta}$ and $G_+$ defined in \eqref{eq:gain_add_num} considering the PINN prior $u_\theta^\text{bc}$ using BC loss training. The results are presented in \cref{tab:case1v3} for $k \in \{1,2,3\}$ and $N \in \{20,40\}$.

\begin{table}[H]
	\centering
	\gainstableallq{fig_testcase2D_test1_v3_gains_Tab_stats_case1_v3.csv}
	\caption{Considering the \textit{2D low-frequency} case, $k\in\{1,2,3\}$ and the PINN prior $u_\theta^\text{bc}$ (BC loss training). Left -- Gains in $L^2$ relative error of the additive method with respect to PINN. Right -- Gains in $L^2$ relative error of our approach with respect to FEM.}
	\label{tab:case1v3}
\end{table}

The gains reported in \cref{tab:case1v3} show that for this test case, the use of the prior $u_\theta$ (using the level-set) in our enriched approach, seems to give better gains than those of \cref{tab:case1_2D}, considering the current prior $u_\theta^\text{bc}$. This may be due to the addition of the $\omega_b^\text{bc}$ hyperparameter for balancing losses in training, which may make training less efficient. However, the results obtained with the current prior $u_\theta^\text{bc}$ are still very good, and the gains are still significant compared to the standard FEM.

\subsubsection{High-frequency case}\label{sec:Lap2Dhigh}

To increase in complexity, we investigate a higher-frequency problem by taking $\kappa=8$ in \eqref{eq:analytical_solution_Lap2D}.
In this section, we start by testing the error estimates in \cref{par:Lap2Dhigh_error_estimations} with $k\in\{1,2,3\}$ polynomial order and evaluate the gains obtained in \cref{par:Lap2Dhigh_gains} on a sample of parameters.

\paragraph*{Physics-informed training.} This time, we use the Fourier features from \cite{TanSri2020} as presented in \cref{sec:spectral_bias} to construct the PINN prior $u_\theta$. The hyperparameters are defined in \cref{tab:paramtest2_2D};
we use the Adam optimizer and then switch to the LBFGS optimizer after the $n_\text{switch}$-th epoch. We consider $N_\text{col}=6000$ collocation points, uniformly chosen on $\Omega$. We impose the Dirichlet boundary conditions as in \cref{sec:Lap2Dlow} using the level-set function and the same residual loss.

\begin{table}[htbp]
    \centering
    \begin{tabular}{cc}
        \toprule
        \multicolumn{2}{c}{\textbf{Network - MLP w/ FF}} \\
        \midrule
        \textit{layers} & $40,60,60,60,40$ \\
        \cmidrule(lr){1-2}
        $\sigma$ & sine \\
        \cmidrule(lr){1-2}
        $n_f$ & 40 \\
        \bottomrule
    \end{tabular}
    \hspace{1cm}
    \begin{tabular}{cccc}
        \toprule
        \multicolumn{4}{c}{\textbf{Training - with LBFGS}} \\
        \midrule
        \textit{lr} & 1.7e-2 & $n_\text{epochs}$ & 20000 \\
        \cmidrule(lr){1-2} \cmidrule(lr){3-4}
        \textit{decay} & 0.99 & $n_\text{switch}$ & 1000 \\
        \cmidrule(lr){1-2} \cmidrule(lr){3-4}
        $N_\text{col}$ & 6000 \\
        \bottomrule
    \end{tabular}
    \hspace{1cm}
    \begin{tabular}{cccc}
        \toprule
        \multicolumn{4}{c}{\textbf{Loss weights}} \\
        \midrule
        $\omega_r$ & 1 & $\omega_\text{data}$ & 0 \\
        \cmidrule(lr){1-2} \cmidrule(lr){3-4}
        $\omega_b$ & 0 & $\omega_\text{sob}$ & 0 \\        
        \bottomrule
    \end{tabular}
    \caption{Network, training parameters (\cref{rmk:PINN_notations}) and loss weights for $u_\theta$ in the \textit{2D Laplacian} case.}
    \label{tab:paramtest2_2D}
\end{table}


\paragraph{Error estimates}\label{par:Lap2Dhigh_error_estimations}\mbox{} \\

We perform the same test as in \cref{sec:Lap2Dlow}, with a standard training (Sobolev training is not considered).

\begin{figure}[H]
	\centering
	\begin{subfigure}{0.48\linewidth}
		\centering
		\cvgFEMCorrAlldeg{fig_testcase2D_test2_cvg_FEM_case2_v1_param1.csv}{fig_testcase2D_test2_cvg_Corr_case2_v1_param1.csv}{3e-8}
		\caption{$\bm{\mu}^{(1)}$ parameter.}
		\label{fig:case2param1}
	\end{subfigure}
	\begin{subfigure}{0.48\linewidth}
		\centering
		\cvgFEMCorrAlldeg{fig_testcase2D_test2_cvg_FEM_case2_v1_param2.csv}{fig_testcase2D_test2_cvg_Corr_case2_v1_param2.csv}{3e-8}
		\caption{$\bm{\mu}^{(2)}$ parameter.}
		\label{fig:case2param2}
	\end{subfigure}
	\caption{Considering the \textit{2D high-frequency} case and the PINN prior $u_\theta$. Left -- $L^2$ relative error on $h$, obtained with the standard FEM $e_h^{(1)}$ (solid lines) and the additive approach $e_{h,+}^{(1)}$ (dashed lines) for $\bm{\mu}^{(1)}$, with $k \in \{1,2,3\}$. Right -- Same for $\bm{\mu}^{(2)}$.}
	\label{fig:case2}
\end{figure}

The results displayed in \cref{fig:case2},
where we observe the expected behavior.
Indeed, all schemes have the correct order of accuracy,
and the enhanced FEM has a significantly lower error constant than the classical FEM.

\paragraph{Comparison of different approaches}\label{par:Lap2Dhigh_comparison}\mbox{} \\

We perform the same comparison as in \cref{par:Lap2D_comparison} for the high-frequency case. We focus on the first parameter $\bm{\mu}^{(1)}$ and compare the standard FEM method with the additive approach, first in the \cref{tab:case2_2D_comparison} where we can see the different errors obtained with the different methods for $k=1$ fixed and $N\in\{16,32\}$ as well as the gains obtained in comparison with standard FEM. Next, we take a closer look at the solution obtained with the different approaches in \cref{fig:case2_2D_plots} considering $N=16$; for each method, we compare the solution obtained ($u_h$ for standard FEM and $u_h^+$ for the additive approach) with the analytical solution $u$. For the enriched method, using the PINN prior $u_\theta$, we will also compare the proposed correction; namely, for the additive approach, we will compare $p_h^+$ with $u-u_\theta$.

\begin{table}[H]
	\centering
	\gainsPINNfirst{fig_testcase2D_test2_plots_FEM.csv}{fig_testcase2D_test2_plots_compare_gains.csv}
	\caption{Considering the \textit{2D high-frequency} case with $\bm{\mu}^{(1)}$, $k=1$ and $N=16,32$. Left -- $L^2$ relative error obtained with FEM. Right -- Considering the PINN prior $u_\theta$, $L^2$ relative errors and gains with respect to FEM, obtained with the additive approach.}
	\label{tab:case2_2D_comparison}
\end{table}

\begin{figure}[!ht] \centering

    \includegraphics[scale=1]{fig_testcase2D_test2_plots_standalone_solutions.pdf}

    \includegraphics[scale=1]{fig_testcase2D_test2_plots_standalone_errors_v2.pdf}

	\caption{Considering the \textit{2D high-frequency} case with $\bm{\mu}^{(1)}$, $k=1$, $N=16$ and the PINN prior $u_\theta$. Comparison of the solution obtained with the standard FEM and the additive approach with the analytical solution. For the additive method, comparison of the correction term with the analytical one.}
	\label{fig:case2_2D_plots}
\end{figure}

We can see here that the gains obtained in \cref{tab:case2_2D_comparison} are much better than for the ``low frequency'' case presented in \cref{sec:Lap2Dlow}. This is, in fact, due to FEM's difficulty in approximating the solution for high frequencies, especially on coarse meshes. In fact, for the same choice of parameters, the FEM error on this high-frequency problem is 10 times worse than on the low-frequency one, which explains why our gains are so much greater. This also makes the use of the proposed enriched methods particularly interesting.
Moreover, we note that while FEM provides a reasonable approximation of the mean of the solution (as evidenced by the second figure on the top row of \cref{fig:case2_2D_plots}), it is unable to correctly resolve the small-scale oscillating behaviour of the solution.
The additive correction restores this ability, and the new solution (third figure on the top row of \cref{fig:case2_2D_plots}) is much better able to capture the oscillations.

\paragraph{Gains achieved with the additive approach}\label{par:Lap2Dhigh_gains}\mbox{} \\

We now evaluate the gains $G_{+,\theta}$ and $G_+$, defined in \eqref{eq:gain_add_num},
using the same sample $\mathcal{S}$ of $n_p=50$ parameters. The results are reported in \cref{tab:case2} for $k \in \{1,2,3\}$ and $N \in \{20, 40\}$.

\begin{table}[H]
	\centering
	\gainstableallq{fig_testcase2D_test2_gains_Tab_stats_case2_v1.csv}
	\caption{Considering the \textit{2D high-frequency} case, $k\in\{1,2,3\}$ and the PINN prior $u_\theta$. Left -- Gains in $L^2$ relative error of the additive method with respect to PINN. Right -- Gains in $L^2$ relative error of our approach with respect to FEM.}
	\label{tab:case2}
\end{table}

The same results can be observed in \cref{tab:case2} as for the $\bm{\mu}^{(1)}$ parameter. These could be improved by considering a Sobolev training, as in the ``low-frequency`` case presented in \cref{sec:Lap2Dlowaug}.



\subsection{2D anisotropic elliptic problem on a square}
\label{sec:Ell2D}

In this section, we will consider the \eqref{eq:ob_pde} problem in a more complex form than in \cref{sec:Lap2D}, by considering the following elliptic problem with homogeneous Dirichlet boundary conditions, in the 2D case ($d=2$),
\begin{equation}
	\left\{
	\begin{aligned}
		-\text{div}(D\nabla u) & = f, \; &  & \text{in } \; \Omega, \\
		u         & =0, \;  &  & \text{on } \; \partial\Omega,
	\end{aligned}
	\right.
	\label{eq:Ell2D}
\end{equation}
with $\Omega=[0,1]^2$, $\partial\Omega$ its boundary and $\mathcal{M} \subset \mathbb{R}^p$ the parameter space (with $p$ the number of parameters).
Considering $\bm{x}=(x,y)\in\Omega$, we define $p=4$ parameters $\bm{\mu}=(\mu_1,\mu_2,\epsilon,\sigma)\in\mathcal{M}=[0.4, 0.6]\times [0.4, 0.6]\times [0.01,1]\times [0.1,0.8]$. We will then define $D$, the diffusion matrix (symmetric and positive definite), by
\begin{equation*}
	D(\bm{x},\bm{\mu})=\begin{pmatrix}
		\epsilon x^2+y^2 & (\epsilon-1)xy \\
		(\epsilon-1)xy & x^2+\epsilon y^2
	\end{pmatrix}
\end{equation*}
and the right-hand side $f$ by
\begin{equation*}
	f(\bm{x},\bm{\mu})=\exp\left(-\frac{(x-\mu_1)^2+(y-\mu_2)^2}{0.025\sigma^2}\right).
\end{equation*}
Note that the matrix $D$ has eigenvalues $x^2 + y^2$
and $\epsilon(x^2 + y^2)$,
leading to a diffusion process whose anisotropy
increases as $\epsilon$ decreases.

\begin{remark}\label{rmk:Ell2D_N_nodes}
	In the following, the characteristic mesh size $h=\frac{\sqrt{2}}{N-1}$ is defined as a function of $N$, considering a Cartesian mesh of $N^2$ nodes.
\end{remark}

\paragraph*{Physics-informed training.} We then consider a parametric PINN where we exactly impose the Dirichlet boundary conditions as presented in \cref{sec:exact_imposition_of_BC}. To do this, we define the prior
\begin{equation*}
	u_{\theta}(\bm{x},\bm{\mu}) = \varphi(\bm{x}) w_{\theta}(\bm{x},\bm{\mu}),
\end{equation*}
where $w_\theta$ is the neural network under consideration and $\varphi$ is a level-set function defined by
\begin{equation*}
	\varphi(\bm{x})=x(x-1)y(y-1),
\end{equation*}
which vanishes exactly on $\partial\Omega$. The hyperparameters are defined in \cref{tab:paramtest3_2D}.

\begin{table}[htbp]
    \centering
    \begin{tabular}{cc}
        \toprule
        \multicolumn{2}{c}{\textbf{Network - MLP}} \\
        \midrule
        \textit{layers} & $40, 60, 60, 60, 40$ \\
        \cmidrule(lr){1-2}
        $\sigma$ & tanh \\
        \bottomrule
    \end{tabular}
    \hspace{1cm}
    \begin{tabular}{cc}
        \toprule
        \multicolumn{2}{c}{\textbf{Training}} \\
        \midrule
        \textit{lr} & 1.6e-2 \\
        \cmidrule(lr){1-2}
        \textit{decay} & 0.99 \\
        \cmidrule(lr){1-2}
        $n_{epochs}$ & 15000 \\
        \cmidrule(lr){1-2}
        $N_\text{col}$ & 8000 \\
        \bottomrule
    \end{tabular}
    \hspace{1cm}
    \begin{tabular}{cccc}
        \toprule
        \multicolumn{4}{c}{\textbf{Loss weights}} \\
        \midrule
        $\omega_r$ & 1 & $\omega_\text{data}$ & 0 \\
        \cmidrule(lr){1-2} \cmidrule(lr){3-4}
        $\omega_b$ & 0 & $\omega_\text{sob}$ & 0 \\        
        \bottomrule
    \end{tabular}
    \caption{Network, training parameters (\cref{rmk:PINN_notations}) and loss weights for $u_\theta$ in the \textit{2D Elliptic} case.}
    \label{tab:paramtest3_2D}
\end{table}

Since we impose the boundary conditions by using the level-set function, we will only consider the residual loss, with integral approached by a Monte-Carlo method, defined by
\[
	J_r(\theta) \simeq
	\frac{1}{N_\text{col}} \sum_{i=1}^{N_\text{col}} \big| div\big(D\big(\bm{x}_\text{col}^{(i)};\bm{\mu}_\text{col}^{(i)}\big) \nabla u_\theta(\bm{x}_\text{col}^{(i)};\bm{\mu}_\text{col}^{(i)}\big) \big) + f\big(\bm{x}_\text{col}^{(i)};\bm{\mu}_\text{col}^{(i)}\big) \big|^2,
\]
with the $N_\text{col}=8000$ collocation points \smash{$\big(\bm{x}_\text{col}^{(i)}, \bm{\mu}_\text{col}^{(i)}\big)_{i=1,\dots,N_\text{col}}$}. Thus we seek to solve the following minimisation problem
\begin{equation*}
	\theta^\star = \argmin_\theta J_r(\theta).
\end{equation*}

% Since we impose the boundary conditions by using the level-set function, we will only consider the residual loss defined by
% \begin{equation*}
% 	J_r(\theta)=\int_\mathcal{M}\int_\Omega \big(div(D(x,y;\mu)\nabla u_{\theta}(x,y;\mu))+f(x,y;\mu)\big)^2dxdyd\mu,
% \end{equation*}
% and thus we seek to solve the following minimisation problem
% \begin{equation*}
% 	\min_\theta J_r(\theta).
% \end{equation*}

\begin{remark}
	Here, we do not know the analytical solution associated with the problem under consideration. So, in order to analyze the results obtained, we will define $u$ as a reference solution $u_{\text{ref}}$ obtained from a FEM solver on an over-refined mesh of characteristic mesh size $h_\text{ref}$ and with $k_\text{ref}$ polynomial order. In this section, we will choose $N_{\text{ref}}=1000$ (and the associated characteristic mesh size $h_{\text{ref}}$, as defined in \cref{rmk:Ell2D_N_nodes}) and the degree $k_{\text{ref}}=3$.
\end{remark}

\subsubsection{Error estimates}

We will now test the error estimation (\cref{lem:error_estimation_add}) for the following two sets of parameters,
uniformly drawn from~$\mathcal{M}$:
\begin{equation*}
	\bm{\mu}^{(1)}=(0.51,0.54,0.52,0.55) \quad \text{and} \quad \bm{\mu}^{(2)}=(0.48,0.53,0.41,0.89).
\end{equation*}
So, for $j \in \{1,2\}$, the aim is to compare, by varying the mesh size $h$, the $L^2$ relative errors \smash{$e_h^{(j)}$} obtained with the standard FEM method, defined in \eqref{eq:error_rel_FEM}, and \smash{$e_{h,+}^{(j)}$} obtained with the additive approach, defined in \eqref{eq:error_rel_add}. The results are presented in \cref{fig:case3} for a fixed $k \in \{1,2,3\}$ with $N \in \{16,32,64,128,256\}$, as presented in \cref{rmk:Ell2D_N_nodes}.

\begin{figure}[H]
	\centering
	\begin{subfigure}{0.48\linewidth}
		\centering
		\cvgFEMCorrAlldeg{fig_testcase2D_test3_cvg_FEM_case3_v1_param1.csv}{fig_testcase2D_test3_cvg_Corr_case3_v1_param1.csv}{1e-9}
		\caption{$\mu^{(1)}$ parameter.}
		\label{fig:case3param1}
	\end{subfigure}
	\begin{subfigure}{0.48\linewidth}
		\centering
		\cvgFEMCorrAlldeg{fig_testcase2D_test3_cvg_FEM_case3_v1_param2.csv}{fig_testcase2D_test3_cvg_Corr_case3_v1_param2.csv}{5e-8}
		\caption{$\mu^{(2)}$ parameter.}
		\label{fig:case3param2}
	\end{subfigure}
	\caption{Considering the \textit{2D elliptic} case and the PINN prior $u_\theta$. Left -- $L^2$ relative error on $h$, obtained with the standard FEM $e_h^{(1)}$ (solid lines) and the additive approach $e_{h,+}^{(1)}$ (dashed lines) for $\bm{\mu}^{(1)}$, with $k \in \{1,2,3\}$. Right -- Same for $\bm{\mu}^{(2)}$.}
	\label{fig:case3}
\end{figure}

As in the other test cases, the two approaches tested appear to respect the correct slopes of \cref{thm:classical_error_estimate} and \cref{lem:error_estimation_add}. The additive approach seems to be more efficient than the standard FEM method for polynomial orders $k \in \{1,2,3\}$ and for the two sets of parameters considered.

\subsubsection{Comparison of different approaches}\label{sec:Ell2D_comparison}

We perform the same comparison as in \cref{par:Lap2D_comparison} for this elliptic case. We focus on the first parameter $\bm{\mu}^{(1)}$ by taking a closer look at the solution obtained with the different approaches in \cref{fig:case3_2D_plots} considering $N=16$ and $k=2$; for each method, we compare the solution obtained ($u_h$ for standard FEM and $u_h^+$ for the additive approach) with the analytical solution $u$. For the enriched method, using the PINN prior $u_\theta$, we will also compare the proposed correction; namely, for the additive approach, we will compare $p_h^+$ with $u-u_\theta$.

\begin{figure}[!ht] \centering

    \includegraphics[scale=1]{fig_testcase2D_test3_plots_standalone_solutions.pdf}

    \includegraphics[scale=1]{fig_testcase2D_test3_plots_standalone_errors.pdf}

	\caption{Considering the \textit{2D elliptic} case with $\bm{\mu}^{(1)}$, $k=2$, $N=16$ and the PINN prior $u_\theta$. Comparison of the solution obtained with the standard FEM and the additive approach with the analytical solution. For the additive method, comparison of the correction term with the analytical one.}
	\label{fig:case3_2D_plots}
\end{figure}

We observe that the enriched FEM provides a more accurate solution compared to the standard FEM one. The results indicate that the additive approach is particularly effective in capturing the solution's finer details. This demonstrates its potential in solving anisotropic problems with higher accuracy than standard methods.

\subsubsection{Gains achieved with the additive approach}

Considering a sample $\mathcal{S}$ of $n_p=50$ parameters, we will evaluate the gains $G_{+,\theta}$ and $G_+$ defined in \eqref{eq:gain_add_num}. The results are presented in \cref{tab:case3} for $k \in \{1,2,3\}$ fixed and $N \in \{20,40\}$ fixed.

\begin{table}[H]
	\centering
	\gainstableallq{fig_testcase2D_test3_gains_gains_table_case3.csv}
	\caption{Considering the \textit{2D elliptic} case, $k\in\{1,2,3\}$ and the PINN prior $u_\theta$. Left -- Gains in $L^2$ relative error of the additive method with respect to PINN. Right -- Gains in $L^2$ relative error of our approach with respect to FEM.}
	\label{tab:case3}
\end{table}

As in the previous test cases, the additive approach seems to be more efficient than the standard FEM method for the two polynomial orders $k \in \{1,2,3\}$ and for the two mesh sizes $N \in \{20,40\}$. However, the gains obtained are less significant than in the previous test cases. This is due to the fact that the problem under consideration is more complex, and thus, the prior $u_\theta$ is less accurate.


\subsection{2D Poisson problem on an annulus, with mixed boundary conditions} \label{sec:Lap2DMixRing}

This section concerns the problem \eqref{eq:ob_pde}, considering the Poisson problem with mixed (Dirichlet and Robin) boundary conditions defined in two space dimensions ($d=2$) by
\begin{equation*}
	\left\{
	\begin{aligned}
		-\Delta u & = f, \; &  & \text{in } \; \Omega \times \mathcal{M}, \\
		u         & = g, \;  &  & \text{on } \; \Gamma_E \times \mathcal{M}, \\
        \smash{\frac{\partial u}{\partial n}}+u  & = g_R, \;  &  & \text{on } \; \Gamma_I \times \mathcal{M},
	\end{aligned}
	\right.
	\label{eq:Lap2DMixed}
\end{equation*}
with $\mathcal{M} \subset \mathbb{R}^p$ the parameter space (with $p$ the number of parameters). We consider~$\Omega$ to be an annulus, defined by the unit circle (circle of radius $1$ and centre $(0,0)$) with a circular hole (of radius $0.25$ and centre $(0,0)$). We then define $\partial\Omega=\Gamma_I\cup\Gamma_E$ the boundary of $\Omega$, with $\Gamma_I$ the inner boundary (the hole) and $\Gamma_E$ the outer boundary (the unit circle). We consider the analytical solution defined for all $\bm{x}=(x,y)\in\Omega$ by
\begin{equation*}
	u(\bm{x};\bm{\mu})= 1 - \frac{\ln\big(\mu_1\sqrt{x^2+y^2}\big)}{\ln(4)},
\end{equation*}
with some parameters $\bm{\mu}=\mu_1\in[2.4, 2.6]$ ($p=1$ parameter), and the associated right-hand side $f=0$. The Dirichlet condition $g$ on $\Gamma_E$ is also defined by
\begin{equation*}
	g(\bm{x};\bm{\mu})=1 - \frac{\ln(\mu_1)}{\ln(4)}
\end{equation*}
and the Robin condition $g_R$ on $\Gamma_I$ is defined by
\begin{equation*}
    g_R(\bm{x};\bm{\mu})=2 + \frac{4-\ln(\mu_1)}{\ln(4)}.
\end{equation*}

In this section, we consider the additive approach, as presented in \cref{sec:additive_prior}, by considering the PINN prior $u_\theta$. We start by testing the error estimation in \cref{par:Lap2DAnn_error_estimations} with $k\in\{1,2,3\}$ polynomial order and evaluate the gains obtained in \cref{par:Lap2DAnn_gains} on a sample of parameters.

\begin{remark}
	To avoid geometric errors, we apply \cref{rem:bconcurved}, by considering that $g=u$ on $\Gamma_{E,h}$ and $g_R=\frac{\partial u}{\partial n}$ on~$\Gamma_{I,h}$, with $\Gamma_{E,h}$ and $\Gamma_{I,h}$ the respective outer and inner boundaries of $\Omega_h$, the domain covered by the mesh. Note also that $u_\theta$ is not exact on these approximate boundaries.
\end{remark}

\paragraph*{Physics-informed training.} Since the problem under consideration is parametric,
we deploy a parametric PINN,
which depends on both the space variable $\bm{x}=(x,y) \in \Omega$
and the parameters $\bm{\mu}=\mu_1 \in \mathcal{M}$. To improve the derivatives' quality, we consider the Sobolev training presented in \cref{sec:sobolev_training}. Moreover, we strongly impose the Dirichlet boundary conditions,
as explained in \cref{sec:exact_imposition_of_BC}, by using the formulation proposed in \cite{Sukumar_2022}.
To do this, we define the prior
\begin{equation}\label{eq:mixedformulation}
	u_{\theta} = \frac{\varphi_E}{\varphi_E+\varphi_I^2}\left[w_\theta+\varphi_I\big(w_\theta-\nabla\varphi_I\cdot\nabla w_\theta-h\big)\right] + \frac{\varphi_I^2}{\varphi_E+\varphi_I^2}g+\varphi_E\varphi_I^2w_\theta,
\end{equation}
where $w_\theta$ is the neural network under consideration and $\varphi_I$ and $\varphi_E$ are respectively the signed distance functions to $\Gamma_I$ and $\Gamma_E$ defined by
\begin{equation*}
	\varphi_I(\bm{x})=\sqrt{x^2+y^2}-0.25, \quad \varphi_E(\bm{x})=1-\sqrt{x^2+y^2},
\end{equation*}
which cancels out exactly on $\Gamma_I$ and $\Gamma_E$.

In this case, we consider an MLP with $5$ layers and a tanh activation function with the hyperparameters defined in \cref{tab:paramtest5_2D};
we use the Adam optimizer~\cite{KinBa2015} and consider $N_\text{col}=6000$ collocation points, uniformly chosen on $\Omega$.

\begin{table}[htbp]
    \centering
    \begin{tabular}{cc}
        \toprule
        \multicolumn{2}{c}{\textbf{Network - MLP}} \\
        \midrule
        \textit{layers} & $40,40,40,40,40$ \\
        \cmidrule(lr){1-2}
        $\sigma$ & tanh \\
        \bottomrule
    \end{tabular}
    \hspace{1cm}
    \begin{tabular}{cccc}
        \toprule
        \multicolumn{4}{c}{\textbf{Training - with LBFGS}} \\
        \midrule
        \textit{lr} & 1e-2 & $n_\text{epochs}$ & 4000 \\
        \cmidrule(lr){1-2} \cmidrule(lr){3-4}
        \textit{decay} & 0.99 & $n_\text{switch}$ & 3000 \\
        \cmidrule(lr){1-2} \cmidrule(lr){3-4}
        $N_\text{col}$ & 6000 \\
        \bottomrule
    \end{tabular}
    \hspace{1cm}
    \begin{tabular}{cccc}
        \toprule
        \multicolumn{4}{c}{\textbf{Loss weights}} \\
        \midrule
        $\omega_r$ & 1 & $\omega_\text{data}$ & 0 \\
        \cmidrule(lr){1-2} \cmidrule(lr){3-4}
        $\omega_b$ & 0 & $\omega_\text{sob}$ & 0 \\        
        \bottomrule
    \end{tabular}
    \caption{Network, training parameters (\cref{rmk:PINN_notations}) and loss weights for $u_\theta$ in the \textit{2D Laplacian} case on an Annulus.}
    \label{tab:paramtest5_2D}
\end{table}


\begin{remark}
	Note that the level sets considered are signed distance functions in this specific test case, which is not the case in the other test cases. In this test case, this is necessary because of the formulation proposed in \eqref{eq:mixedformulation} by \cite{Sukumar_2022}.
\end{remark}

Since we impose the boundary conditions by using the level-set function, we will only consider the residual loss, where the integral is approached by a Monte-Carlo method, defined by
\begin{equation*}
	J_r(\theta) \simeq
	\frac{1}{N_\text{col}} \sum_{i=1}^{N_\text{col}} \big| \Delta u_{\theta}(\bm{x}_\text{col}^{(i)};\bm{\mu}_\text{col}^{(i)}\big)+f(\bm{x}_\text{col}^{(i)};\bm{\mu}_\text{col}^{(i)}\big)\big|^2,
\end{equation*}
with the $N_\text{col}$ collocation points \smash{$\big(\bm{x}_\text{col}^{(i)}, \bm{\mu}_\text{col}^{(i)}\big)_{i=1,\dots,N_\text{col}}$}. Considering the Sobolev training, we seek to solve the following minimisation problem
\begin{equation*}
	\theta^\star = \omega_r\argmin_\theta J_r(\theta) + \omega_\text{\rm sob} J_\text{\rm sob}(\theta),
\end{equation*}
with $\omega_r=1$ and $\omega_\text{\rm sob}=0.1$ the weights associated with the residual and Sobolev losses, respectively.

\subsubsection{Error estimates}\label{par:Lap2DAnn_error_estimations}

We start by testing the error estimation of \cref{lem:error_estimation_add} for the following two sets of parameters,
uniformly selected from $\mathcal{M}$:
\begin{equation*}
	\bm{\mu}^{(1)}=(2.51) \quad \text{and} \quad \bm{\mu}^{(2)}=(2.54)
\end{equation*}
by considering the PINN prior $u_\theta$. So, for $ j\in \{1,2\}$, the aim is to compare, by varying the mesh size $h$, the $L^2$ relative errors \smash{$e_h^{(j)}$} obtained with the standard FEM method, defined in \eqref{eq:error_rel_FEM}, and \smash{$e_{h,+}^{(j)}$} obtained with the additive approach, defined in \eqref{eq:error_rel_add}.
The results are presented in \cref{fig:case1} for fixed $k \in \{1,2,3\}$ by varying the mesh size $h$.


\begin{figure}[H]
	\centering
	\begin{subfigure}{0.48\linewidth}
		\centering
		\cvgFEMCorrAlldeg{fig_testcase2D_test5_cvg_FEM_case5_v2_param1.csv}{fig_testcase2D_test5_cvg_Corr_case5_v2_param1.csv}{1e-10}
		\caption{Case of $\bm{\mu}^{(1)}$}
		\label{fig:case4param1}
	\end{subfigure}
	\begin{subfigure}{0.48\linewidth}
		\centering
		\cvgFEMCorrAlldeg{fig_testcase2D_test5_cvg_FEM_case5_v2_param2.csv}{fig_testcase2D_test5_cvg_Corr_case5_v2_param2.csv}{1e-10}
		\caption{Case of $\bm{\mu}^{(2)}$}
		\label{fig:case4param2}
	\end{subfigure}
	\caption{Considering the \textit{2D Laplacian} case on an Annulus and the PINN prior $u_\theta$. Left -- $L^2$ relative error on $h$, obtained with the standard FEM $e_h^{(1)}$ (solid lines) and the additive approach $e_{h,+}^{(1)}$ (dashed lines) for $\bm{\mu}^{(1)}$, with $k \in \{1,2,3\}$. Right -- Same for $\bm{\mu}^{(2)}$.}
	\label{fig:case5}
\end{figure}

As expected, we see in \cref{fig:case5} that the error estimates are confirmed by the numerical results obtained with the standard FEM and the additive approach. The error decreases with the correct order of convergence for these two methods. Furthermore, the enriched approach provides a better accuracy than the standard FEM, as expected.

\subsubsection{Comparison of different approaches}\label{sec:Lap2DAnn_comparison}

We perform the same comparison as in \cref{par:Lap2D_comparison} for this elliptic case. We focus on the first parameter $\bm{\mu}^{(1)}$ by taking a closer look at the solution obtained with the different approaches in \cref{fig:case5_2D_plots} considering $h\simeq 1.67\cdot 10^{-1}$ and $k=1$; for each method, we compare the solution obtained ($u_h$ for standard FEM and $u_h^+$ for the additive approach) with the analytical solution $u$. For the enriched method, using the PINN prior $u_\theta$, we will also compare the proposed correction; namely, for the additive approach, we will compare $p_h^+$ with $u-u_\theta$.

\begin{figure}[!ht] \centering

    \includegraphics[scale=1]{fig_testcase2D_test5_plots_standalone_solutions.pdf}

    \includegraphics[scale=1]{fig_testcase2D_test5_plots_standalone_errors.pdf}

	\caption{Considering the \textit{2D Laplacian} case on an Annulus with $\bm{\mu}^{(1)}$, $k=1$, $h\simeq 1.67\cdot 10^{-1}$ and the PINN prior $u_\theta$. Comparison of the solution obtained with the standard FEM and the additive approach with the analytical solution. For the additive method, comparison of the correction term with the analytical one.}
	\label{fig:case5_2D_plots}
\end{figure}

Once again, we observe that the enriched approach provides a significant improvement in accuracy compared to the standard FEM. This demonstrates the effectiveness of incorporating neural network priors in the case of mixed boundary conditions on more complex geometries than squares (here, on an annulus).

\subsubsection{Gains achieved with the additive approach} \label{par:Lap2DAnn_gains}

Considering a sample $\mathcal{S}$ of $n_p=50$ parameters,
we now evaluate the gains $G_{+,\theta}$ and $G_+$ defined in \eqref{eq:gain_add_num}.
The results are presented in \cref{tab:case4}
for $k \in \{1,2,3\}$ and $h \in \{1.33\cdot 10^{-1},6.90\cdot 10^{-2}\}$.

\begin{table}[H]
	\centering
	\gainstableallqh{fig_testcase2D_test5_gains_Tab_stats_case5_v2.csv}
	\caption{Considering the \textit{2D Laplacian} case on an Annulus, $k\in\{1,2,3\}$ and the PINN prior $u_\theta$. Left -- Gains in $L^2$ relative error of the additive method with respect to PINN. Right -- Gains in $L^2$ relative error of our approach with respect to FEM.}
	\label{tab:case4}
\end{table}

As in previous sections, the PINN-enriched approach seems to give better results than standard FEM. For $k=1$ we see a gain of $50$ on average for this test case, which is equivalent to refining the mesh by a factor of $7$ for $\mathbb{P}_1$ elements.



\section{Conclusion and future works}

In this work, we explored a new approach combining FEM and predictions from neural networks. The FEM scheme is used to enhance the prediction thanks to a correction. Two strategies were investigated: an additive correction and a multiplicative one.
For both approaches, we have proved a priori error estimates for both the $ H1$ semi-norm and the $ L2$ norm. 
We have also highlighted a link between these two techniques. 
Moreover, the constant appearing in these inequalities is compared with the case of classical FEM.
Numerical simulations on parametric problems in one and two dimensions confirm our theoretical analyses. The various numerical test cases have shown that PINNs are good candidates for our enriched methods due to their ability to approximate the derivatives of the solution, which is necessary for the quality of our techniques. Their ability to approximate the solution of the parametric PDE over a set of parameters also showed that the proposed approaches are much more interesting in terms of numerical costs than the standard method. Solutions to improve the quality of the prior and, thus, the quality of the results have also been highlighted, with Sobolev training in particular. We have also observed that the additive approach offers greater robustness and a more straightforward implementation than the multiplicative one. 


The present work opens up several perspectives. For instance, the additive and multiplicative can be easily adapted to non-linear equations. Moreover, the prediction could also be used to build an optimal mesh before the FEM resolution, for instance, via a posteriori error estimates.


\section{Acknowledgment}
This work was supported by the Agence Nationale de la Recherche, Project PhiFEM, under grant ANR-22-
CE46-0003-01.

\bibliographystyle{alpha}
\bibliography{bibliography}

\appendix
% \renewcommand{\thesection}{A}
% \setcounter{equation}{0}

\newpage
\section{Notations and definitions}\label{app:notations}

The aim of this section is to introduce the notations used throughout the paper. We first present the notations related to the parametric PDE (\cref{tab:notations_PDE}), to the neural network (\cref{tab:notations_PINN}), and to the finite element methods (\cref{tab:notations_FEM}).

\renewcommand{\arraystretch}{1.1}  % Augmente l'espacement des lignes

\begin{table}[ht!]
    \centering
    \begin{tabular}{c|c}
        \textbf{Notation} & \textbf{Definition} \\
        \hline
        $\Omega$ & Spatial domain \\
        $d$ & Spatial dimension \\
        $\bm{x}=(x_1,\dots,x_d)$ & Spatial coordinates \\
        \hline
        $\mathcal{M}$ & Parameter space \\
        $p$ & Number of parameters \\
        $\bm{\mu}=(\mu_1,\ldots,\mu_p)$ & Parameter vector \\
        \hline
        $M$ & Lifting constant \\
        $u$ & Solution of the problem \\
        $u_M$ & Solution of the lifted problem by $M$ \\
        $f$ & Right-hand side of the problem \\
        $\mathcal{L}$ & Parametric differential operator of the problem \\
        $R$ & Reaction coefficient \\
        $C$ & Convection coefficient \\
        $D$ & Diffusion matrix \\
        Pe & Péclet number \\
    \end{tabular}
    \caption{Notations introduced for the parametric PDE.}
    \label{tab:notations_PDE}
\end{table}

\begin{table}[ht!]
    \centering
    \begin{tabular}{c|c}
        \textbf{Notation} & \textbf{Description} \\
        \hline
        $u_\theta$ & Neural network prediction of $u$ \\
        $u_{\theta,M}$ & Neural network prediction of $u_M$ \\
        $\varphi$ & Level-set function used to impose BCs \\
        $\theta$ & Trainable parameters of the neural network \\
        $\theta^\star$ & Optimal parameters \\
        \hline
        $J_r$ & Residual loss \\
        $J_b$ & Boudary loss \\
        $J_\text{data}$ & Data loss \\
        $J_\text{sob} $ & Sobolev loss \\
    \end{tabular}
    \caption{Notations considered for the neural network.}
    \label{tab:notations_PINN}
\end{table}


\begin{table}[ht!]
    \centering
    \begin{tabular}{c|c|c}
        & \textbf{Notation} & \textbf{Description} \\
        \hline
        \multirow{5}{*}{\rotatebox{90}{\small Standard FEM}}
        & $V_h^0$ & Finite element approximation space \\
        & $u_h$ & Finite element approximation of $u$ \\
        & $h$ & Characteristic size of the mesh \\
        & $\mathcal{I}_h$ & Lagrange interpolation operator \\
        & $k$ & Polynomial degree of the finite element approximation \\
        \hline
        \multirow{4}{*}{\rotatebox{90}{\small Additive}}\;\multirow{4}{*}{\rotatebox{90}{\small enrichment}} & $V_h^+$ & Finite element approximation space enriched with additive prior \\
        & $u_h^+$ & Finite element approximation of $u$ in $V_h^+$ \\
        & $p_h^+$ & Finite element approximation of $u-u_\theta$ in $V_h^0$ \\
        & $C_\text{\rm gain}^+$ & Additive gain constant \\
        \hline
        \multirow{6}{*}{\rotatebox{90}{\small Multiplicative}}\;\multirow{6}{*}{\rotatebox{90}{\small enrichment}} & $V_h^\times$ & Finite element approximation space enriched with multiplicative prior \\
        & $u_h^\times$ & Finite element approximation of $u$ in $V_h^\times-M$ \\
        & $p_h^\times$ & Finite element approximation of $u_M/u_{\theta,M}$ in $1+V_h^0$ \\
        & $C_{\text{\rm gain},H^1}^\times$ & Multiplicative gain constant in $H^1$ semi-norm \\
        & $C_{\text{\rm gain},L^2}^\times$ & Multiplicative gain constant in $L^2$ norm \\
        & $\tilde{\mathcal{I}}_h$ & Modified Lagrange interpolation operator \\
    \end{tabular}
    \caption{Notations used in the various finite element methods.}
    \label{tab:notations_FEM}
\end{table}


\end{document}
