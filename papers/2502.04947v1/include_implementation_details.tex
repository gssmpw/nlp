\section{Implementation details}\label{sec:implementation_details}

Before moving on to numerical experiments,
we discuss some practical details
regarding the implementation of
the methods introduced in \cref{sec:additive_prior,sec:multiplicative_prior}.
In \cref{sec:using_PINN}, we first look at how to effectively plug the PINN prediction in the FEM solver.
Then, in \cref{sec:boundary_conditions}, we discuss the imposition of boundary conditions in the two proposed methods. %Finally, we will present how the errors in \cref{sec:error_computation} are computed.

% \paragraph*{Implementation tools.}

The tools used to implement the methods and obtain the results in \cref{sec:numerical_results} are, on the one hand, \texttt{PyTorch} \cite{paszke2019pytorchimperativestylehighperformance} and \texttt{ScimBa} for the construction of the prior, in particular the implementation of PINN, and, on the other hand, \texttt{FEniCS} for the implementation of the finite element methods. For mesh generation, use either \texttt{FEniCS} or \texttt{mshr} mesh generators.

\begin{remark}
    In section \cref{sec:numerical_results}, we will not specify the training times of the networks, independently for each test case. To give an order of importance, PINN training takes less than ten minutes on a laptop GPU. For cases using the LBFGs optimizer, training takes a little longer, but still less than an hour.
\end{remark}

\begin{remark}
    Note that, \texttt{FEniCS} defines the characteristic mesh size $h$ as the length of the longest edge, for $d\in\{1,2\}$.
\end{remark}

\subsection{Using PINN prediction effectively}\label{sec:using_PINN}

To be effective, our methods will in practice depend on the quality of the approximation of the prior's derivatives, computed from automatic differentiation, and its precise integration on the domain.

\paragraph*{Automatic differentiation.}

% The first one is that
It is important to use the automatic differentiation offered by neural networks, enabling exact (in the sense of machine precision) derivative computation without having to manipulate complex symbolic expressions. In particular, in the context of PINNs, automatic differentiation will play a fundamental role in integrating the PDE under consideration. This automatic differentiation will enable the two improved finite element methods to use the exact derivatives of the prior $u_\theta$ and thus avoid introducing an additional error in the computation of the derivative.

\paragraph*{Analytical function integration.}

Analytical functions in the weak problem must be integrated with sufficient precision for these approaches to be effective.
Thus, in e.g. the additive approach, a quadrature rule with a higher degree than the traditional FEM has to be applied to discretize the term $l(v_h) - a(u_\theta,v_h)$ in \eqref{eq:approachform_add}.
This point, and the required degree of the quadrature rule,
will be studied in more detail in the first 2D test case considered in \cref{par:Lap2D_sup}.



\begin{remark}
In practice, the source term in the additive approach will be computed in the strong way. For instance, in the case of the Laplacian equation with $u_{\theta}=0$ on $\partial\Omega$, the term
$$l(v_h)-a(u_{\theta},v_h)=\int_{\Omega}f v_h-\int_{\Omega}\nabla u_\theta \cdot \nabla v_h$$
will be replaced by
$$\int_{\Omega}\big(f+\Delta u_{\theta}\big)v_h.$$
If $u_{\theta}$ is not equal to zero on $\partial\Omega$, one needs to include a boundary term.
\end{remark}

\subsection{Imposing boundary conditions}\label{sec:boundary_conditions}

In this section, we focus on the crucial question of imposing boundary conditions. We first look at this problem in the context of the additive approach presented in \cref{sec:additive_prior} and then in the context of the multiplicative approach presented in \cref{sec:multiplicative_prior}.

\begin{remark}
    For simplicity, this section focuses on (non-homogeneous) Dirichlet conditions.
    For our enhanced FEM, just like in classical FEM,
    these boundary conditions are imposed by manually eliminating essential dofs, see~\cite{Ern2004TheoryAP}, and more precisely by modifying the matrix and the right-hand side of the linear system. This approach is not needed for Neumann and Robin conditions.
\end{remark}

\subsubsection{Additive approach}\label{sec:additive_BC}

In this first approach, if our Dirichlet problem satisfies
$u=g$ on $\partial \Omega$,
then $p_h^+$ has to satisfy
\[
    p_h^+ = g - u_{\theta} \text{\quad on } \partial \Omega,
\]
with $u_\theta$ the PINN prior.
This non-homogeneous boundary condition becomes homogeneous as soon as $u_\theta$ is exact at the boundary, or, in other words, as soon as the boundary conditions are imposed exactly in the PINN, as presented in \cref{sec:exact_imposition_of_BC}.

\begin{remark}\label{rem:bconcurved}
    However, in the case of curved geometries (e.g. disks) where the meshes do not coincide with the boundary of the geometry, problems occur when $q>2$, and especially on coarse meshes. In the numerical results, to avoid this problem and check the error estimates, we assume that $g=u$ on $\partial\Omega_h$ where $\Omega_h$ is the domain covered by the mesh. Furthermore, we need to be careful because even if, in PINN, the conditions are imposed exactly (as shown in \cref{sec:exact_imposition_of_BC}), the prediction $u_\theta$ will not be exact on $\partial\Omega_h$. We made this choice here to simplify the problem, but in practice, solutions exist to improve the quality of the results.
    % These include border scraping methods or mesh refinement methods on the boundary.
\end{remark}

\subsubsection{Multiplicative approach}\label{sec:multiplicative_BC}

In this second method, the boundary conditions are a bit more complex to handle. In the \cref{sec:multiplicative_prior}, we have denoted by
\[
    u_{h,M}^\times = u_{\theta,M} \; p_h^\times
\]
the solution obtained by the multiplicative approach to the modified problem \eqref{eq:ob_pde_M} with the prior $u_{\theta,M}=u_\theta+M$. Therefore, we can recover the solution $u_h^\times$ of the original problem \eqref{eq:ob_pde} by setting $u_h^\times = u_{h,M}^\times - M$.

\paragraph*{Standard PINN.}

In the case where our prior $u_\theta$ is the prediction of a standard PINN, as presented in \cref{sec:PINNs_parametric_PDE}, the boundary conditions are not imposed exactly. Thus, if our problem satisfies $u=g$ on $\partial \Omega$, then $p_h^\times$ has to satisfy
\[
    p_h^\times = \frac{g+M}{u_{\theta,M}} \text{\quad on } \partial \Omega.
\]

\paragraph*{PINN with exact BC.}

We now focus on the case where we exactly impose the boundary conditions in the PINN, as presented in \cref{sec:exact_imposition_of_BC}.
Then, supposing that $M>0$, we have $u_{\theta,M}=g+M$ on $\partial \Omega$ and therefore~$p_h^\times$ has to satisfy
\[
    p_h^\times = 1 \text{\quad on } \partial \Omega.
\]
However, there is a specific case when this condition is not necessarily true. Indeed, if the boundary conditions are homogeneous, then $g=0$ and $u_{\theta,M}=M$ on $\partial \Omega$.
Considering that $u_\theta>0$ in $\Omega$,
$M=0$ is a possible choice.
In this case, \smash{$u_{\theta,M}=u_{\theta,0}=0$} on $\partial \Omega$,
and \smash{$u_{h,M}^\times=u_{h,0}^\times=u_h^\times$} automatically satisfies the boundary conditions.
Hence, imposing a boundary condition on $p_h^\times$ becomes unnecessary.

\begin{remark}
    % In the numerical results of \cref{sec:numerical_results}, one of these specific
    % cases is considered. Considering a 1D case in \cref{sec:Ell1D},
    In the numerical results of \cref{sec:Ell1D}, one of these specific
    cases is considered in 1D.
    We will see that leaving $p_h^\times$ free will give better results here than imposing $p_h^\times=1$ on $\partial \Omega$. Indeed, this approach leaves more freedom to capture the correct derivatives at the boundary.
\end{remark}

% A second possibility is not to impose boundary conditions in the matrix,
% but to impose them solely through a prior.
% This approach can only work for homogeneous Dirichlet conditions.
% To do this, we construct the basis functions associated with the edge nodes,
% and impose no conditions on the edge degrees of freedom.
% Since we multiply by the prior, which is zero at the edges,
% we impose the correct boundary conditions.
% This approach leaves more freedom to capture the right derivatives at the boundaries.

% \subsection{Computation of the error}\label{sec:error_computation}

