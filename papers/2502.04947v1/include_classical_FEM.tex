\section{Continuous finite element method}\label{sec:FEM}

The goal of this section is to recall the classical FEM,
and to introduce the notation that will be used throughout the paper.
To solve the problem~\eqref{eq:ob_pde} under consideration for a fixed parameter $\bm{\mu}$
(which will be omitted for clarity)
with homogeneous Dirichlet boundary conditions using
the continuous FEM, we rewrite it as the following variational problem:
\begin{equation}
	\label{eq:weakform}
	\text{Find } u \in V^0 \text{ such that, } \; \forall v\in V^0, \; a(u,v)=l(v),
\end{equation}
where $V^0=H^1_0(\Omega)$,
and where the bilinear form $a$ is given by
\[
	a(u,v)=
	\frac{1}{\peclet} \int_{\Omega}D \nabla u \cdot  \nabla v+
	\int_{\Omega} R \, u \, v  +
	\int_{\Omega} v \, C \cdot \nabla u 	,
\]
while the linear form $l$ reads
\[
	l(v)=\int_{\Omega} f \, v .
\]

\begin{remark}
	Note that since $a$ is continuous on $V^0\times V^0$ and coercive and $l$ is continuous on $V^0$,
    the existence and uniqueness of the solution $u$ are ensured by the Lax-Milgram theorem.
\end{remark}


Let $\mathcal{T}_h$ be a mesh of the domain $\Omega$ composed of simplexes,
where $h$ denotes the characteristic size of the mesh, i.e.\ the biggest diameter of the simplexes. We suppose that $\mathcal{T}_h$ satisfies the Ciarlet condition (see e.g.\ \cite{Ern2004TheoryAP}) and that its boundary is exactly $\partial\Omega$.
Consider $V_h^0\subset V_h\subset V=H^1(\Omega)$  the two continuous Lagrange finite elements spaces of degree $k\geq 1$ defined by
\begin{equation}
	V_h = \left\{v_h\in C^0(\Omega),\; \forall K\in \mathcal{T}_h,\; v_h\vert_{K}\in\mathbb{P}_k\right\},
	\label{eq:Vh}
\end{equation}
and
\begin{equation*}
	V_h^0 = \left\{v_h\in C^0(\Omega),\; \forall K\in \mathcal{T}_h,\; v_h\vert_{K}\in\mathbb{P}_k,v_h\vert_{\partial\Omega}=0\right\},
	\label{eq:Vh0}
\end{equation*}
with $\mathbb{P}_k$ the space of polynomials with real coefficients of degree at most $k$.
The solution to \eqref{eq:weakform} will be approximated by the solution $u_h$ to
\begin{equation}
	\label{eq:approachform}
	\text{Find } u_h \in V_h^0 \text{ such that, } \; \forall v_h\in V_h^0, \; a(u_h,v_h)=l(v_h).
\end{equation}

Let us now some results used in the next sections.
We first introduce the Lagrange interpolation operator defined by
\begin{equation}
	\mathcal{I}_h  : C^0(\Omega) \ni v \mapsto \sum_{i=1}^{N_{\text{\rm dofs}}} v\big(\bm{x}^{(i)}\big) \psi_i\in V_h,
	\label{eq:Ih}
\end{equation}
with \smash{${(\bm{x}^{(i)})}_{i \in \{1,\ldots,N_{\text{\rm dofs}}\}}$}
the $N_{\text{\rm dofs}}$ degrees of freedom (dofs) associated to the mesh,
and \smash{${(\psi_i)}_{i \in \{1,\ldots,N_{\text{\rm dofs}}\}}$}
the associated Lagrange shape functions of degree $k$.

\begin{remark}
	In the whole manuscript, for a Sobolev space $H$, the notation $|\cdot|_{H}$ and $\|\cdot\|_{H}$ will represent respectively the semi-norm and the norm in $H$.
\end{remark}

The following result gives a bound of the interpolation error:
\begin{theorem}[see e.g.\ \cite{Ern2004TheoryAP}]\label{th:interpol}
There exists $C_q>0$ such that
for all $v\in H^{q+1}(\Omega)$ and $1\leqslant q\leqslant k$,
\begin{equation*}
    \|v-\mathcal{I}_h v\|_{H^1}\leqslant C_q h^q |v|_{H^{q+1}}.
\end{equation*}
\end{theorem}

The next estimate is associated to the elliptic regularity:

\begin{theorem}[see e.g. {\cite[Theorem 4, p. 317]{evans2022partial}}]\label{th:ellip}
There exists $C_e>0$, such that for all $f\in L^2(\Omega)$, the unique solution $w\in H^2(\Omega)$ to
$$\mathcal{L}^* w=\xi$$
with homogeneous Dirichlet boundary condition satisfies
\begin{equation*}
    \|w\|_{H^2}\leqslant C_{e} \|\xi\|_{L^2}.
\end{equation*}
Here $\mathcal{L}^*$ represents the adjoint of the operator $\mathcal{L}$.
\end{theorem}


\noindent These estimates, combined with Céa's Lemma, which uses the continuity and coercivity of $a$, give the following error estimate:
\begin{theorem}[see e.g.\ \cite{Ern2004TheoryAP}]\label{thm:classical_error_estimate}
	Let $u\in H^{q+1}(\Omega)$ and $u_h\in V_h^0$ the solutions to~\eqref{eq:weakform} and~\eqref{eq:approachform}.
	For all $1\leqslant q\leqslant k$, one has %$0\leqslant m\leq q$, one has
	\begin{equation*}
		\label{eq:classical_error_estimate}
		|u-u_h|_{H^1}\leqslant C_q\dfrac{\gamma}{\alpha}h^{q} |u|_{H^{q+1}}
	\end{equation*}
	and
	\begin{equation*}
		\|u-u_h\|_{L^2}\leqslant C_eC_1C_q\dfrac{\gamma^2}{\alpha}h^{q+1} |u|_{H^{q+1}},
	\end{equation*}
	where $\gamma$ and $\alpha$ are respectively
	the constants of continuity and coercivity of $a$.
\end{theorem}



For the sake of simplicity, we consider an elliptic boundary
value problem with homogeneous Dirichlet conditions.
Obviously, we can use more general boundary conditions as Robin-like
conditions depending on the parameters. This will be investigated
in the numerical experiments to show that the proposed methodology
applies to a larger class of boundary value problems, see \cref{sec:Lap2DMixRing}.
