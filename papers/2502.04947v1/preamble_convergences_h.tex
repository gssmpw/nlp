\usepackage{amssymb}
\usepackage{mathtools}
%\usepackage[scale=0.8]{geometry}
\usepackage{pgfplots}
\usepackage{pgfplotstable}
% \usepackage{filecontents}
\usepackage{datatool}
\usepackage{fp}

\pgfplotsset{
    compat=newest,
}
\pgfplotsset{
    smaller labels/.style={
        label style={font=\footnotesize},
        tick label style={font=\footnotesize}
    }
}
\tikzset{font=\small}
\usetikzlibrary{
    fpu,
    fixedpointarithmetic,
    babel,
    external,
    arrows.meta,
    plotmarks,
    positioning,
    angles,
    quotes,
    intersections,
    calc,
    spy,
    decorations.pathreplacing,
    matrix,
    fit,
}
\usepgfplotslibrary{fillbetween}

% Define colors
\definecolor{femcolor}{RGB}{51, 138, 55} %Green (27,158,119)
\definecolor{addcolor}{RGB}{217,95,2} %Orange
\definecolor{addsobcolor}{RGB}{199,39,34} %Red (sob or other)
\definecolor{multcolor3}{RGB}{117,112,179} %Purple 
%{RGB}{231,41,138} %Pink
\definecolor{multcolor100}{RGB}{0,0,0} %Black (+ empty marker)
\definecolor{multcolor0weak}{RGB}{49, 73, 181} %Blue
\definecolor{multcolor0strong}{RGB}{49, 181, 161} %Cyan

% Define line styles according to the method 
% FEM : solid
% Add : dashed
% Mult : dotted

% Define marker styles according to the degree
% P1 : square
% P2 : circle
% P3 : triangle

%________________ error lines (by Ricardo Costa) ________________

% argument 1: slopes (e.g. {4,6})
% argument 2: x position of the bottom left corner
% argument 3: y position of the bottom left corner
% argument 4: x length

\makeatletter

\newcommand{\printslopeinv}[4]{
    \tikzset{fixed point arithmetic}
    % get arguments
    \def\nero@printslope@orderlist{#1}
    \edef\nero@printslope@xpos{#2}
    \edef\nero@printslope@ypos{#3}
    \edef\nero@printslope@width{#4}
    % get points position
    \pgfmathparse{\nero@printslope@xpos+\nero@printslope@width}
    \edef\nero@printslope@px{\pgfmathresult}
    \edef\nero@printslope@py{\nero@printslope@ypos}
    \edef\nero@printslope@qx{\pgfmathresult}
    \edef\nero@printslope@ry{\nero@printslope@ypos}
    \foreach \nero@printslope@order in {#1}{
        \pgfmathparse{
        ((\nero@printslope@px/\nero@printslope@xpos)^(\nero@printslope@order))*\nero@printslope@ypos}
        \edef\nero@printslope@qy{\pgfmathresult}
            \edef\nero@aux1{\noexpand\draw[line width=0.6pt]
            (axis cs:\nero@printslope@xpos,\nero@printslope@ypos)
            -- (axis cs:\nero@printslope@qx,\nero@printslope@qy)
            -- (axis cs:\nero@printslope@px,\nero@printslope@py);}
        \nero@aux1
        % slope label
        \pgfmathparse{10^((ln(\nero@printslope@ry)+ln(\nero@printslope@qy))/(ln(10)*2))}
        \edef\nero@printslope@labelpos{\pgfmathresult}
        \edef\nero@aux2{\noexpand\node[anchor=west] at
            (axis cs:\nero@printslope@qx,\nero@printslope@labelpos)
            {\noexpand\tiny \nero@printslope@order};}
        \nero@aux2
        \global\edef\nero@printslope@ry{\nero@printslope@qy}
    }
    % base line
    \draw[line width=0.6pt] (axis cs:\nero@printslope@xpos,\nero@printslope@ypos)
        |- (axis cs:\nero@printslope@px,\nero@printslope@py);
    % label of base line
    \pgfmathparse{10^((ln(\nero@printslope@px)+ln(\nero@printslope@xpos))/(ln(10)*2))}
    \edef\nero@printslope@labelpos{\pgfmathresult}
    \node[anchor=north] at (axis cs:\nero@printslope@labelpos,\nero@printslope@ypos) {\tiny 1};
}

\makeatother

\newlength{\plotwidth}
\setlength{\plotwidth}{0.45\textwidth}
\newlength{\plotheight}
\setlength{\plotheight}{0.3\textwidth}

% \newcommand{\generatexticklabels}[2]{%
%     \pgfplotstableset{create on use/xlabels/.style={
%         create col/expr={##1}
%     }}%
%     \pgfplotstabletypeset[
%         col sep=comma,
%         string type,
%         columns={#2},
%         columns/xlabels/.style={column name={}, string type},
%         every head row/.style={output empty row}
%     ]{#1}%
%     \edef\xlabels{\pgfplotsretval}%
% }

% \newcommand{\generatexticklabels}[2]{%
%     % Number of rows
%     \pgfplotstablegetrowsof{#1}
%     \pgfmathtruncatemacro{\lastrow}{\pgfplotsretval-1}%

%     \edef\xlabels{}
%     \foreach \i in {0,...,\lastrow} {%
%         \pgfplotstablegetelem{\i}{#2}\of{#1}%
%         \edef\temp{\pgfplotsretval}%
%         \ifnum\i=0
%             \xappto\xlabels{\temp}%
%         \else
%             \xappto\xlabels{,\temp}%
%         \fi
%     }
%     \typeout{Xlabels: \xlabels}%
% }

\gdef\iterator{0}

\newenvironment{cvgh}[4]{
    \begin{tikzpicture}
        \edef\filename{#1}
        \edef\legendcolumns{#2}
        \edef\slopes{#3}
        \edef\ypos{#4}

        % Read the CSV file into a table
        \pgfplotstableread[col sep=comma]{\filename}\datatable

        % Add a new column with rounded values of "h"
        % \pgfplotstablecreatecol[
        %     create col/expr={
        %         round(\thisrow{h}*1000)/1000
        %     }
        % ]{h_rounded}{\datatable}

        % \generatexticklabels{\datatable}{h}
        
        % Obtenir le nombre de lignes
        % \pgfplotstablegetrowsof{\datatable}
        % \edef\Nrows{\pgfplotsretval}

        % Obtenir le second élément
        \pgfmathtruncatemacro{\secondrow}{1} % Index de la dernière ligne
        \pgfplotstablegetelem{\secondrow}{h}\of\datatable
        \pgfmathsetmacro{\second}{\pgfplotsretval} % Dernière valeur de h_rounded

        % Obtenir le premier élément
        \pgfmathtruncatemacro{\firstrow}{0} % Index de l'avant-dernière ligne
        \pgfplotstablegetelem{\firstrow}{h}\of\datatable
        \pgfmathsetmacro{\first}{\pgfplotsretval} % Avant-dernière valeur de h_rounded

        % Calculer la différence entre les deux
        \pgfmathsetmacro{\diff}{\first - \second}

        %update iterator
        \pgfmathtruncatemacro{\iterator}{\iterator+1}

        % \pgfkeys{/pgf/number format/.cd,sci,sci e,precision=2}

        \begin{loglogaxis}[
            smaller labels,
            name = left_plot,
            % axis lines
            axis lines = left,
            enlarge x limits={abs=10pt},
            enlarge y limits={abs=10pt},
            axis x line shift = -5pt,
            axis y line shift = -5pt,
            % labels
			xmode=log,
            xlabel = {$h$},
            ylabel = {\rotatebox{270}{$L^2$}},
            xlabel style={at={(ticklabel* cs:1.01)},anchor=west},
            ylabel style={at={(ticklabel* cs:1.01)},anchor=west},
            % ticks and labels
            xtick=data,
            xticklabels from table={\datatable}{h},
            % xticklabel=\pgfmathprintnumber{\tick}, % Conversion en notation scientifique
            % xticklabel style={/pgf/number format/fixed}, % Format fixe pour éviter l'approximation
			% xticklabels={%
            %     \pgfmathsetmacro\temp{\thisrow{h}}%
            %     \pgfmathprintnumber{\temp}%
            % }
            % xticklabels={%
            %     \pgfmathprintnumber[scientific, precision=2]{\thisrow{h_rounded}} 
            % },
            % xticklabel style={rotate=45, anchor=east},
            % size
            width=\plotwidth, height=\plotheight,
            % marks
            mark options={solid, scale=1},
            % grid
            grid = major,
            % legend
            % legend entries={\legendentries},
            legend columns=\legendcolumns,
            legend to name=leg:legendFEMCORR_\iterator,
            legend image post style={mark options={solid, scale=1}},
        ]
        \expandafter\printslopeinv\expandafter{\slopes}{\second}{\ypos}{\diff}
    }
    {
        \end{loglogaxis}
        \node[yshift=-20pt] at (left_plot.outer south) {\pgfplotslegendfromname{leg:legendFEMCORR_\iterator}};

    \end{tikzpicture}
}

\newcommand{\cvgFEMCorrAlldeg}[3]{
    \edef\fem{#1}
    \edef\add{#2}

    \begin{cvgh}{\fem}{3}{2,3,4}{#3}
        % Complete the legend
        \addlegendentry{\,FEM $\mathbb{P}_1$\;}
        \addlegendentry{\,FEM $\mathbb{P}_2$\;}
        \addlegendentry{\,FEM $\mathbb{P}_3$\;}
        \addlegendentry{\,Add $\mathbb{P}_1$\;}
        \addlegendentry{\,Add $\mathbb{P}_2$\;}
        \addlegendentry{\,Add $\mathbb{P}_3$\;}

        % Plot FEM
        \addplot [style={solid}, mark=square*, mark size=2, color=femcolor, line width=0.8pt ]
        table [x=h, y=P1, col sep=comma]
            {\fem};
        
        \addplot [style={solid}, mark=*, mark size=2, color=femcolor, line width=0.8pt ]
        table [x=h, y=P2, col sep=comma]
            {\fem};
        
        \addplot [style={solid}, mark=triangle*, mark size=2, color=femcolor, line width=0.8pt ]
        table [x=h, y=P3, col sep=comma]
            {\fem};

        % Plot Add
        \addplot [style={dashed}, mark=square*, mark size=2, color=addcolor, line width=0.8pt ]
        table [x=h, y=P1, col sep=comma]
            {\add};

        \addplot [style={dashed}, mark=*, mark size=2, color=addcolor, line width=0.8pt ]
        table [x=h, y=P2, col sep=comma]
            {\add};

        \addplot [style={dashed}, mark=triangle*, mark size=2, color=addcolor, line width=0.8pt ]
        table [x=h, y=P3, col sep=comma]
            {\add};

    \end{cvgh}
}


\newcommand{\cvgFEMCorrMultOnedeg}[6]{
    \edef\fem{#1}
    \edef\femsec{#2}
    \edef\add{#3}
    \edef\mult{#4}
    \edef\multHundred{#5}

    \begin{cvgh}{\fem}{3}{2,3}{#6}
        % Complete the legend
        \addlegendentry{\,FEM $\mathbb{P}_1$\;}
        \addlegendentry{\,Mult $\mathbb{P}_1$ (M=3)\;}
        \addlegendentry{\,Add $\mathbb{P}_1$\;}
        \addlegendentry{\,FEM $\mathbb{P}_2$\;}
        \addlegendentry{\,Mult $\mathbb{P}_1$ (M=100)\;}

        % Plot the data
        \addplot [style={solid}, mark=square*, mark size=2, color=femcolor, line width=0.8pt ]
        table [x=h, y=err, col sep=comma]
            {\fem};

        \addplot [style={dotted}, mark=square*, mark size=2, color=multcolor3, line width=1.0pt ]
        table [x=h, y=err, col sep=comma]
            {\mult};

        \addplot [style={dashed}, mark=square*, mark size=2, color=addcolor, line width=0.8pt ]
        table [x=h, y=err, col sep=comma]
            {\add};

        \addplot [style={solid}, mark=*, mark size=2, color=femcolor, line width=0.8pt ]
        table [x=h, y=err, col sep=comma]
            {\femsec};

        \addplot [style={dotted}, mark=square, mark size=2, color=multcolor100, line width=1.0pt ]
        table [x=h, y=err, col sep=comma]
            {\multHundred};
    \end{cvgh}
}


%convergence Ell1D on square (with mul strong and mul weak)
\newcommand{\cvgFEMCorrMultSWOnedeg}[6]{
    \edef\fem{#1}
    \edef\femsec{#2}
    \edef\add{#3}
    \edef\mults{#4}
    \edef\multw{#5}

    \begin{cvgh}{\fem}{3}{2,3}{#6}
        % Complete the legend
        \addlegendentry{\,FEM $\mathbb{P}_1$\;}
        \addlegendentry{\,Mult strong $\mathbb{P}_1$\;}
        \addlegendentry{\,Add $\mathbb{P}_1$\;}
        \addlegendentry{\,FEM $\mathbb{P}_2$\;}
        \addlegendentry{\,Mult weak $\mathbb{P}_1$\;}

        \addplot [style={solid}, mark=square*, mark size=2, color=femcolor, line width=0.8pt ]
        table [x=h, y=err, col sep=comma]
            {\fem};

        \addplot [style={dotted}, mark=square*, mark size=2, color=multcolor0strong, line width=0.8pt ]
        table [x=h, y=err, col sep=comma]
            {\mults};

        \addplot [style={dashed}, mark=square*, mark size=2, color=addcolor, line width=0.8pt ]
        table [x=h, y=err, col sep=comma]
            {\add};

        \addplot [style={solid}, mark=*, mark size=2, color=femcolor, line width=0.8pt ]
        table [x=h, y=err, col sep=comma]
            {\femsec};

        \addplot [style={dotted}, mark=square*, mark size=2, color=multcolor0weak, line width=0.8pt ]
        table [x=h, y=err, col sep=comma]
            {\multw};
    \end{cvgh}
}



% convergence Lap2D on square (Sobolev loss)
\newcommand{\cvgFEMCorrAugFirst}[5]{
    \edef\fem{#1}
    \edef\add{#2}
    \edef\addsob{#3}
    \begin{cvgh}{\fem}{3}{2}{#4}
        % Complete the legend
        \addlegendentry{\,FEM $\mathbb{P}_1$\;}
        \addlegendentry{\,Add $\mathbb{P}_1$ ($u_\theta$)\;}
        \addlegendentry{\,Add $\mathbb{P}_1$ (#5)\;}

        \addplot [style={solid}, mark=square*, mark size=2, color=femcolor, line width=0.8pt ]
            table [x=h, y=P1, col sep=comma]
                {\fem};

        \addplot [style={dashed}, mark=square*, mark size=2, color=addcolor, line width=0.8pt ]
            table [x=h, y=P1, col sep=comma]
                {\add};

        \addplot [style={dashed}, mark=square*, mark size=2, color=addsobcolor, line width=0.8pt ]
            table [x=h, y=P1, col sep=comma]
                {\addsob};
    \end{cvgh}
}

\newcommand{\cvgFEMCorrAugSecond}[5]{
    \edef\fem{#1}
    \edef\add{#2}
    \edef\addsob{#3}

    \begin{cvgh}{\fem}{3}{3}{#4}
        % Complete the legend
        \addlegendentry{\,FEM $\mathbb{P}_2$\;}
        \addlegendentry{\,Add $\mathbb{P}_2$ ($u_\theta$)\;}
        \addlegendentry{\,Add $\mathbb{P}_2$ (#5)\;}

        \addplot [style={solid}, mark=*, mark size=2, color=femcolor, line width=0.8pt ]
            table [x=h, y=P2, col sep=comma]
                {\fem};

        \addplot [style={dashed}, mark=*, mark size=2, color=addcolor, line width=0.8pt ]
            table [x=h, y=P2, col sep=comma]
                {\add};

        \addplot [style={dashed}, mark=*, mark size=2, color=addsobcolor, line width=0.8pt ]
            table [x=h, y=P2, col sep=comma]
                {\addsob};
    \end{cvgh}
}

\newcommand{\cvgFEMCorrAugThird}[5]{
    \edef\fem{#1}
    \edef\add{#2}
    \edef\addsob{#3}

    \begin{cvgh}{\fem}{3}{4}{#4}
        % Complete the legend
        \addlegendentry{\,FEM $\mathbb{P}_3$\;}
        \addlegendentry{\,Add $\mathbb{P}_3$ ($u_\theta$)\;}
        \addlegendentry{\,Add $\mathbb{P}_3$ (#5)\;}

        \addplot [style={solid}, mark=triangle*, mark size=2, color=femcolor, line width=0.8pt ]
            table [x=h, y=P3, col sep=comma]
                {\fem};

        \addplot [style={dashed}, mark=triangle*, mark size=2, color=addcolor, line width=0.8pt ]
            table [x=h, y=P3, col sep=comma]
                {\add};

        \addplot [style={dashed}, mark=triangle*, mark size=2, color=addsobcolor, line width=0.8pt ]
            table [x=h, y=P3, col sep=comma]
                {\addsob};
    \end{cvgh}
}