\subsection{Setup of the numerical experiments}\label{sec:setup}

For each of the proposed test cases,
we consider a parametric problem, on which we train a PINN as presented in \cref{sec:PINNs_parametric_PDE},
to resolve it on a set of fixed parameters, denoted by $\mathcal{M}$.
Let $p=\dim(\mathcal{M})$ be the number of parameters,
and consider a sample $\mathcal{S}$ of $n_p$ parameters:
\begin{equation*}
	\mathcal{S}=\left\{\bm{\mu}^{(1)},\dots,\bm{\mu}^{(n_p)}\right\},
\end{equation*}
with, for $j=1,\dots,n_p$,
\begin{equation*}
	\bm{\mu}^{(j)}=\left(\mu_1^{(j)},\dots,\mu_{p}^{(j)}\right)\in \mathcal{M}.
\end{equation*}

In the following, we denote by $u^{(j)}$ a reference solution to problem \eqref{eq:parametric_PDE} for a given parameter $\bm{\mu}^{(j)}$ and by $u_h^{(j)}$ the solution obtained by the standard finite element method \eqref{eq:approachform} where $V_h$ is the $\mathbb{P}_k$ Lagrange space defined in \eqref{eq:Vh} and $h$ is the characteristic mesh size. We also denote by $u_\theta^{(j)}$ the solution obtained by the parametric PINN, and by $u_{h,+}^{(j)}$ the solution obtained by the additive approach \eqref{eq:approachform_add} with $V_h^+$ the $\mathbb{P}_k$ Lagrange space defined in \eqref{eq:Vh_add}. In some test cases, we also consider the solution  $u_{h, M}^{(j)}$ of the multiplicative approach \eqref{eq:approachform_mul} with $V_h^\times$ the $\mathbb{P}_k$ Lagrange space defined in \eqref{eq:Vh_mul}, depending on the lifting constant $M$.

\begin{remark}\label{rk:ref}
	In the following, to estimate the error, we consider the reference solution to be either an analytical solution or a solution obtained with a very fine mesh and a high polynomial degree. More precisely, we need the characteristic size $h_\text{ref}$ associated with the reference mesh to be much smaller than the size associated with the current mesh $h$, i.e. $h_\text{ref}\ll h$, and we will consider $k_\text{ref}=3$ the polynomial degree associated with the reference solution.
\end{remark}

In each test case, we investigate two aspects; the first in~\cref{sec:setup_error_estimates} involves verifying the error estimates and the second in~\cref{sec:setup_gains} is the evaluation of the gains achieved by the proposed methods compared with the standard one.

\subsubsection{Error estimates}\label{sec:setup_error_estimates}

Consider a small sample $\mathcal{S}$ with $n_p=2$ parameters. Given a fixed parameter $\bm{\mu}^{(j)}$, $j=1,2$, we start by testing the error estimates obtained in \cref{lem:error_estimation_add} for the additive approach. In the case $d=1$, we will also be interested in the error estimates in \cref{lem:error_estimate_multiplicative} for the multiplicative approach. By varying the mesh size~$h$, we then estimate the errors obtained with the two methods.
To evaluate these errors, we compare the approximations to the reference solution $u^{(j)}$ (see \cref{rk:ref}).
We then define by
\begin{equation}\label{eq:error_rel_FEM}
	e_h^{(j)}=\frac{||u^{(j)}-u_h^{(j)}||_{L^2}}{||u^{(j)}||_{L^2}}
	\text{ and }
	e_\theta^{(j)}=\frac{||u^{(j)}-u_\theta^{(j)}||_{L^2}}{||u^{(j)}||_{L^2}}
\end{equation}
the $L^2$ relative error obtained for the standard FEM and the PINN respectively. We further define,
\begin{equation}
    \label{eq:error_rel_add}
	e_{h,+}^{(j)}=\frac{||u^{(j)}-u_{h,+}^{(j)}||_{L^2}}{||u^{(j)}||_{L^2}}
	\text{ and }
	e_{h,M}^{(j)}=\frac{||u^{(j)}-u_{h,M}^{(j)}||_{L^2}}{||u^{(j)}||_{L^2}}
\end{equation}
the $L^2$ relative errors obtained for the additive and multiplicative approach (depending on the lifting constant $M$), respectively.

%\begin{remark}
%	In the numerical experiments, we consider only the relative $L^2$ error.
%\end{remark}

\subsubsection{Gains achieved with the enriched bases}\label{sec:setup_gains}

As we have trained the network to be parameter-dependent to predict a solution for a set of parameters, we are interested in the average gains we obtain with our enriched approaches compared to the PINN and the standard FEM. More precisely, for a fixed mesh size $h$ and a fixed polynomial degree $k$, for a sample $\mathcal{S}$ of $n_p$ parameters, the numerical gains obtained by the additive approach on PINN and standard FEM are respectively defined for $j=1,\dots,n_p$ by:
\begin{equation}
	G_{+,\theta}^{(j)}=\frac{e_\theta^{(j)}}{e_{h,+}^{(j)}}
    \text{\qquad and \qquad}
    G_+^{(j)}=\frac{e_h^{(j)}}{e_{h,+}^{(j)}}, \label{eq:gain_j}
\end{equation}
with $e_\theta^{(j)}$, $e_h^{(j)}$ and $e_{h,+}^{(j)}$ respectively the $L^2$ relative errors obtained with the PINN, the standard FEM and the additive approach, defined in \cref{sec:setup_error_estimates}.
Similarly, the theoretical gains obtained by the multiplicative approach (depending on the lifting constant $M$) on PINN and standard FEM are respectively defined for $j=1,\dots,n_p$ by:
\begin{equation}
	G_{M,\theta}^{(j)}=\frac{e_\theta^{(j)}}{e_{h,M}^{(j)}} \quad \text{and} \quad G_M^{(j)}=\frac{e_h^{(j)}}{e_{h,M}^{(j)}}, \label{eq:gain_j_mul}
\end{equation}
with $e_{h,M}^{(j)}$ the $L^2$ relative error obtained with the multiplicative approach (depending on the lifting constant $M$), defined in \cref{sec:setup_error_estimates}.
Therefore, we will be interested in the minimum, maximum, mean and standard deviation obtained on the following samples :
\begin{equation}
	G_{+,\theta}=\left\{G_{+,\theta}^{(1)},\dots,G_{+,\theta}^{(n_p)}\right\} \quad \text{and} \quad G_+=\left\{G_+^{(1)},\dots,G_+^{(n_p)}\right\}, \label{eq:gain_add_num}
\end{equation}
which respectively represent the gains obtained with our additive approach over PINNs and over standard FEM on the sample $\mathcal{S}$. In the same way, we define $G_{M,\theta}$ and $G_M$, which respectively represent the gains obtained with our multiplicative approach over PINN and over standard FEM on the sample $\mathcal{S}$ by:
\begin{equation}
	G_{M,\theta}=\left\{G_{M,\theta}^{(1)},\dots,G_{M,\theta}^{(n_p)}\right\} \quad \text{and} \quad G_M=\left\{G_M^{(1)},\dots,G_M^{(n_p)}\right\}. \label{eq:gain_mul_num}
\end{equation}
