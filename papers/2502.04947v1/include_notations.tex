\newpage
\section{Notations and definitions}\label{app:notations}

The aim of this section is to introduce the notations used throughout the paper. We first present the notations related to the parametric PDE (\cref{tab:notations_PDE}), to the neural network (\cref{tab:notations_PINN}), and to the finite element methods (\cref{tab:notations_FEM}).

\renewcommand{\arraystretch}{1.1}  % Augmente l'espacement des lignes

\begin{table}[ht!]
    \centering
    \begin{tabular}{c|c}
        \textbf{Notation} & \textbf{Definition} \\
        \hline
        $\Omega$ & Spatial domain \\
        $d$ & Spatial dimension \\
        $\bm{x}=(x_1,\dots,x_d)$ & Spatial coordinates \\
        \hline
        $\mathcal{M}$ & Parameter space \\
        $p$ & Number of parameters \\
        $\bm{\mu}=(\mu_1,\ldots,\mu_p)$ & Parameter vector \\
        \hline
        $M$ & Lifting constant \\
        $u$ & Solution of the problem \\
        $u_M$ & Solution of the lifted problem by $M$ \\
        $f$ & Right-hand side of the problem \\
        $\mathcal{L}$ & Parametric differential operator of the problem \\
        $R$ & Reaction coefficient \\
        $C$ & Convection coefficient \\
        $D$ & Diffusion matrix \\
        Pe & Péclet number \\
    \end{tabular}
    \caption{Notations introduced for the parametric PDE.}
    \label{tab:notations_PDE}
\end{table}

\begin{table}[ht!]
    \centering
    \begin{tabular}{c|c}
        \textbf{Notation} & \textbf{Description} \\
        \hline
        $u_\theta$ & Neural network prediction of $u$ \\
        $u_{\theta,M}$ & Neural network prediction of $u_M$ \\
        $\varphi$ & Level-set function used to impose BCs \\
        $\theta$ & Trainable parameters of the neural network \\
        $\theta^\star$ & Optimal parameters \\
        \hline
        $J_r$ & Residual loss \\
        $J_b$ & Boudary loss \\
        $J_\text{data}$ & Data loss \\
        $J_\text{sob} $ & Sobolev loss \\
    \end{tabular}
    \caption{Notations considered for the neural network.}
    \label{tab:notations_PINN}
\end{table}


\begin{table}[ht!]
    \centering
    \begin{tabular}{c|c|c}
        & \textbf{Notation} & \textbf{Description} \\
        \hline
        \multirow{5}{*}{\rotatebox{90}{\small Standard FEM}}
        & $V_h^0$ & Finite element approximation space \\
        & $u_h$ & Finite element approximation of $u$ \\
        & $h$ & Characteristic size of the mesh \\
        & $\mathcal{I}_h$ & Lagrange interpolation operator \\
        & $k$ & Polynomial degree of the finite element approximation \\
        \hline
        \multirow{4}{*}{\rotatebox{90}{\small Additive}}\;\multirow{4}{*}{\rotatebox{90}{\small enrichment}} & $V_h^+$ & Finite element approximation space enriched with additive prior \\
        & $u_h^+$ & Finite element approximation of $u$ in $V_h^+$ \\
        & $p_h^+$ & Finite element approximation of $u-u_\theta$ in $V_h^0$ \\
        & $C_\text{\rm gain}^+$ & Additive gain constant \\
        \hline
        \multirow{6}{*}{\rotatebox{90}{\small Multiplicative}}\;\multirow{6}{*}{\rotatebox{90}{\small enrichment}} & $V_h^\times$ & Finite element approximation space enriched with multiplicative prior \\
        & $u_h^\times$ & Finite element approximation of $u$ in $V_h^\times-M$ \\
        & $p_h^\times$ & Finite element approximation of $u_M/u_{\theta,M}$ in $1+V_h^0$ \\
        & $C_{\text{\rm gain},H^1}^\times$ & Multiplicative gain constant in $H^1$ semi-norm \\
        & $C_{\text{\rm gain},L^2}^\times$ & Multiplicative gain constant in $L^2$ norm \\
        & $\tilde{\mathcal{I}}_h$ & Modified Lagrange interpolation operator \\
    \end{tabular}
    \caption{Notations used in the various finite element methods.}
    \label{tab:notations_FEM}
\end{table}
