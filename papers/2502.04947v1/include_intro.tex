\section{Introduction%: \flo{new version in progress, reorganizing paragraphs}
}\label{sec:introduction}



The finite element method (FEM, e.g., \cite{ciarlet2002finite,Ern2004TheoryAP,brenner2008mathematical})
is a widely used method for the numerical solution of PDEs.
It requires the construction of the mesh that discretizes the domain of interest into elements, typically simplexes or quadrilaterals/hexahedrals.
Basis functions are associated to each element to define the approximation space in which the numerical solution is sought.
This solution is defined from degrees of freedom, which depend on the elements and basis functions, and by the order of approximation
of the method.
In particular, the method converges according to the characteristic mesh size and, depending on the mesh size, it requires the inversion of a potentially large matrix. Hence, solving the problem thousands of times  (as in optimal control or uncertainty propagation) becomes expensive.
This has motivated intensive research on new finite element methods to combine accuracy and matrices of reduced orders.
We can, for instance, refer to the Trefftz method (e.g., \cite{hiptmair2013error,moiola2018space,ImbMoiSto2022}),
or the hybridizable discontinuous Galerkin method (e.g., \cite{Cockburn2008,hungria2017hdg,Pham2024stabilization})
where the resulting matrix uses degrees of freedom only on the skeleton of the mesh.
In addition to the computational cost, it is important to note that one needs a mesh of the discretized geometry to perform the computation.
This step is not always straightforward (e.g., in geosciences where Earth layers are not clearly localized), and it can be time-consuming.
That is why mesh-less strategies have also been investigated in the last decades, and isogeometric analysis has been employed, e.g.
 \cite{hughes2005isogeometric,Frambati2022practical}.


In recent years, learning-based alternatives have emerged, such as Physics-Informed Neural Networks (PINNs, \cite{RAISSI2019686})
or the Deep Ritz method~\cite{e2017deepritzmethoddeep}.
The idea is to approximate the solution of the PDE under consideration using a neural network trained by minimizing a loss function,
taking the underlying physics into account.
Unlike neural networks trained with more conventional data-driven loss functions, these methods share similarities with traditional
solvers: They require the same inputs, namely the PDE, physical parameters, boundary, and initial conditions.
In addition, the training phase requires approximating the PDE solution in a discrete set of points of the space domain.
Thus, these approaches have some advantages, notably the absence of meshing and their relative dimension-insensitivity.
Since they do not require data (reference solutions), they are particularly well suited to high-dimensional problems on complex domains.
However, at present, these learning-based techniques are not competitive with classical finite element methods (see~\cite{grossmann2023can}), mainly because network-based methods lack precision and convergence guarantees, see \cite{sikora2024comparison} for a comparison between PINNs and FEM.
While FEM has a better error/computation time ratio for a single resolution, PINNs are more advantageous for parametric systems where a multitude of resolutions is needed. We further refer to \cite{cuomo2022scientific} for some analysis on PINNs.


This paper aims to propose a new method that combines learning-based and finite element methods.
More precisely, the main idea is to use a parametric PINN to compute a large family of offline solutions.
This is followed by calculating an online solution for a single parameter using coarse finite elements, with the PINN solution used as a prior information.
The result is a method capable of rapidly predicting a PDE solution while guaranteeing convergence properties, thanks to the FEM framework.
Finite element resolution improves the prediction while remaining cheap as it is performed on a coarse mesh, benefiting from the network prediction.
This paper proposes two ways to enrich the FEM.
In both cases, the finite element error will be exhibited as a function of the network error with respect to the true solution.
These corrections will be called ``additive" and ``multiplicative" depending on how the prediction is incorporated in the FEM spaces.



% Other works use an a priori for FEM resolution. For example, in the $\varphi$-FEM method developed in \cite{duprez2020phi} (see also \cite{cotin2023phi,duprez2023new,DupLleLozVui2023,duprez2023phi} for different contexts), the a prior of the FEM is a level set function used to localized the boundary of the domain. In \cite{brunet2019physics}, the authors initialize Newton's algorithm when solving a hyperelastic equation with a prior derived from the prediction of a neural network. Such a prediction can also be used as a prior for discontinuous Galerkin methods (see \cite{FraMicNav2024}).

Some previous works combine FEM and the use of a prior as neural network prediction.
In \cite{brunet2019physics}, the authors initialize Newton's algorithm when solving a hyperelastic equation with a prior derived from the prediction of a neural network. Such a prediction can also be used as a prior for discontinuous Galerkin methods (see \cite{FraMicNav2024}).
In \cite{feng_hybrid_2024}, the authors solve PDE by using a neural network for the spatial resolution and a FEM scheme for the temporal one. It is also possible to include the shape function of the FEM in a PINN approach as in \cite{skardova_finite_2024}.
In the $\varphi$-FEM method developed in \cite{duprez2020phi} (see also \cite{cotin2023phi,duprez2023new,DupLleLozVui2023,duprez2023phi} for different contexts), the a prior of the FEM is a level set function used to localized the boundary of the domain.
In the FEM, the space of approximation can also be enrich to ensure stability as for instance the introduction of bubble function in mixed problems (see e.g. \cite{Ern2004TheoryAP}).

In this paper, we consider general parametric linear elliptic differential equations
defined on a smooth domain $\Omega \subset \R^d$
with $d$ space dimensions and $\partial\Omega$ the boundary of $\Omega$.
Let a parameter space $\mathcal{M}=\{\bm{\mu}=(\mu_1,\ldots,\mu_p)\in \mathbb{R}^p\}$.
The typical problem of interest is given by:
for one or several  $\bm{\mu}\in \mathcal{M}$, find $u: \Omega\to \R$ such that
\begin{equation}
    \label{eq:ob_pde}
		\mathcal{L}\big(u;\bm{x},\bm{\mu}\big) = f(\bm{x},\bm{\mu}),
\end{equation}
with $\bm{x}=(x_1,\dots,x_d)\in\Omega$ the space variable and where $\mathcal{L}$ is the parametric differential operator defined  by
\begin{equation*}
    \mathcal{L}(\cdot;\bm{x},\bm{\mu}) : u \mapsto R(\bm{x},\bm{\mu}) u + C(\bm{\mu}) \cdot \nabla u - \frac{1}{\peclet} \nabla \cdot (D(\bm{x},\bm{\mu}) \nabla u),
\end{equation*}
with $f(\bm{x},\bm{\mu})\in L^2(\Omega)$ the source term,
$R(\bm{x},\bm{\mu})\in L^{\infty}(\Omega)
$ the reaction coefficient,
$C(\bm{\mu}) \in \R^d%(L^{\infty}(\Omega))^d
$ the convection coefficient,
$D(\bm{x},\bm{\mu}) \in {(W^{1,\infty}(\Omega))}^{d\times d}
$ the diffusion matrix (symmetric positive definite)
and $\peclet \in \R_+^*$ the Péclet number represents the ratio between convection and diffusion.
The differential operator is considered with Dirichlet, Neumann or Robin boundary conditions, which can also depend on $\bm{\mu}$.

The pipeline associated with our approaches is presented in \cref{fig:pipeline}.

\begin{figure}[!ht]
    \centering
%    \begin{minipage}{0.48\textwidth}
%        \centering
%        \includegraphics[width=\linewidth]{fig/pipeline/offline.pdf}
%    \end{minipage}
%    \begin{minipage}{0.48\textwidth}
%        \centering
%        \includegraphics[width=\linewidth]{fig/pipeline/online.pdf}
%    \end{minipage}
    \begin{minipage}{0.48\textwidth}
        \centering
        \includegraphics[width=\linewidth]{fig_pipeline_offline_v2.pdf}
    \end{minipage}
    \begin{minipage}{0.48\textwidth}
        \centering
        \includegraphics[width=\linewidth]{fig_pipeline_online_v2.pdf}
    \end{minipage}
    \caption{Pipeline of the enriched method considered. Left: offline phase (PINN training). Right: online phase (Correction of the PINN prediction).}
    \label{fig:pipeline}
\end{figure}




The manuscript is organized as follows:
In \cref{sec:FEM}, we recall the used finite element method, allowing us to introduce the notations needed in the next sections.
In \cref{sec:additive_prior,sec:multiplicative_prior}, we present the two proposed approaches and prove error estimates.
They both rely on modifying the functions of the FEM approximation space, using information from prior knowledge of the solution.
This prior is introduced first in an additive way, then in a multiplicative way.
Both approaches are compared in \cref{sec:comparison_add_mul}.
\Cref{sec:prior_construction} is devoted to the construction of the prior, justifying the use of PINNs and recalling methods for improving their efficiency.
In \cref{sec:implementation_details}, we give details on the implementation.
Numerical simulations conclude this manuscript in \cref{sec:numerical_results} and show that the proposed methods can significantly reduce the computational cost of solving parametric problems.

In \cref{app:notations}, we introduce the main notations used throughout this manuscript.

% \fred{Références à ajouter (proposées par Emmanuel) : \cite{feng_hybrid_2024} \cite{skardova_finite_2024}}

