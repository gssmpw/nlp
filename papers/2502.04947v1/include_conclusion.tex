\section{Conclusion and future works}

In this work, we explored a new approach combining FEM and predictions from neural networks. The FEM scheme is used to enhance the prediction thanks to a correction. Two strategies were investigated: an additive correction and a multiplicative one.
For both approaches, we have proved a priori error estimates for both the $ H1$ semi-norm and the $ L2$ norm. 
We have also highlighted a link between these two techniques. 
Moreover, the constant appearing in these inequalities is compared with the case of classical FEM.
Numerical simulations on parametric problems in one and two dimensions confirm our theoretical analyses. The various numerical test cases have shown that PINNs are good candidates for our enriched methods due to their ability to approximate the derivatives of the solution, which is necessary for the quality of our techniques. Their ability to approximate the solution of the parametric PDE over a set of parameters also showed that the proposed approaches are much more interesting in terms of numerical costs than the standard method. Solutions to improve the quality of the prior and, thus, the quality of the results have also been highlighted, with Sobolev training in particular. We have also observed that the additive approach offers greater robustness and a more straightforward implementation than the multiplicative one. 


The present work opens up several perspectives. For instance, the additive and multiplicative can be easily adapted to non-linear equations. Moreover, the prediction could also be used to build an optimal mesh before the FEM resolution, for instance, via a posteriori error estimates.