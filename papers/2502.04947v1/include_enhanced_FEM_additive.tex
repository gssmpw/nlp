% !TeX root = ../main.tex

\section{Enriching the finite element method with additive priors}
\label{sec:additive_prior}

In this section, we assume that a prior knowledge of the solution to \eqref{eq:ob_pde} is available. In what follows, we call this information a ``prior''.
This prior is denoted by $\bm{x} \mapsto u_{\theta}(\bm{x})$ with parameters $\theta$, and we assume that it can be constructed with the desired regularity $u_{\theta} \in H^{q+1}(\Omega)\cap H_0^1(\Omega)$ for $1\leqslant q\leqslant k$, where $k$ is the polynomial degree of the enriched FEM.
In this section and the next two, the prior will be general, but up the \cref{sec:prior_construction}, it will be the prediction of a parametric PINN.
In \cref{sec:modified_problem_add}, we first show how to use this prior to enriching classical finite element spaces.
Then, in \cref{sec:error_estimates_add}, we prove a convergence estimate for the resulting method.

\subsection{Construction of the modified problem}
\label{sec:modified_problem_add}

In the general setting of FEM, we follow the Bobunov--Galerkin method \cite{Ern2004TheoryAP}, where the basis functions and the numerical solutions are in the same space (see \eqref{eq:approachform}, where both $u_h$ and $v_h$ are in $V_h^0$).
As we intend to enrich the classical approximation space, we exploit the idea formalized as Petrov--Galerkin method (e.g., \cite{j2005introduction,brenner2008mathematical,demkowicz2023mathematical}), where the test and trial functions belong to different spaces.
This approach is often used for convection-dominated problems, \cite{ALMEIDA1997291}.
We propose to enrich the trial space using the prior $u_\theta$ such that,
\begin{equation}
    \label{eq:Vh_add}
    V_h^+ = \left\{
    u_h^+= u_{\theta} + p_h^+, \quad p_h^+ \in V_h^0
    \right\},
\end{equation}
and we use the space $V_h^0$ for the test functions.
Since we have assumed that $u_{\theta} \in H^{q+1}(\Omega)\cap H^1_0(\Omega)$,~$V_h^+$ is also a subset of $V^0$, like $V_h^0$.
Plugging this new trial space into the approximate problem \eqref{eq:approachform}, we obtain the formulation
\begin{equation}\label{eq:add1}
    \text{Find } u_h^+ \in V_h^+ \text{ such that, } \; \forall v_h\in V_h^0, \; a(u_h,v_h)=l(v_h),
\end{equation}
which leads to the following approximation problem:
\begin{equation}\label{eq:approachform_add}
    \text{Find } p_h^+ \in V_h^0 \text{ such that, } \;
    \forall v_h \in V_h^0, \; a(p_h^+,v_h) = l(v_h) - a(u_{\theta},v_h).
\end{equation}
Therefore, we obtain a classical Galerkin approximation with a modified source term.
%Indeed, as shown in \eqref{eq:approachform_add}, using the space $V_h^+$ for the solution amounts to writing a classical finite element method approximating the residual between the solution and the prior.
%In practice, we solve the modified problem~\eqref{eq:approachform_add} with modified boundary conditions, as detailed in \cref{sec:boundary_conditions}.\textcolor{red}{clarifier cette phrase?}

\subsection{Convergence analysis}
\label{sec:error_estimates_add}

The objective is to prove that the FEM solution to problem \eqref{eq:approachform_add} converges, with an error depending on the quality of the prior.


\begin{theorem}\label{lem:error_estimation_add}
    Let $u\in H^{q+1}(\Omega)$ be the solution to problem \eqref{eq:weakform} and $u_{\theta}\in H^{q+1}(\Omega)\cap H_0^1(\Omega)$ be a prior on $u$.
    We consider $u_h^+\in V_h^+$ as the solution to the discrete problem \eqref{eq:approachform_add} with $V_h^+$ the modified trial space defined in \eqref{eq:Vh_add}.
    The following estimates hold.
    For all $1\leqslant q\leqslant k$,
    \begin{equation}
        \label{eq:error_add}
        | u-u_h^+|_{H^1} \leqslant C_q\dfrac{\gamma}{\alpha} C_\text{\rm gain}^+ \, h^{q} |u|_{H^{q+1}}
    \end{equation}
    and
    \begin{equation*}
        \label{eq:error_addL2}
        \| u-u_h^+\|_{L^2} \leqslant C_e C_1 C_q\dfrac{\gamma^2}{\alpha} C_\text{\rm gain}^+ \, h^{q+1} |u|_{H^{q+1}},
    \end{equation*}
    with $C_e$, $C_1$, $C_q$, $\gamma$, $\alpha$ defined in \cref{sec:FEM} and
    \begin{equation}
        \label{eq:gain_add}
        C_\text{\rm gain}^+= \frac{| u-u_{\theta} |_{H^{q+1}}}{| u |_{H^{q+1}}}.
    \end{equation}
\end{theorem}
\begin{remark}\label{rmk:gain_add}
    The constant \smash{$C_\text{\rm gain}^+$}
    represents the potential gain compared to the error of the classical FEM presented in \cref{thm:classical_error_estimate}. Note that this constant is the same in $L^2$ norm and $H^1$ semi-norm.
\end{remark}



\begin{proof}[Proof of \cref{lem:error_estimation_add}]
    \textbf{$H^1$-error:}   To prove \eqref{eq:error_add}, we adapt the proof of Céa's lemma to the additive prior case.
    Considering the trial space defined in \eqref{eq:Vh_add}, the numerical solution $u_h^+$ is given by
    \[
        u_h^+=u_{\theta}+p_h^+,
    \]
    with $p_h^+ \in V_h^0 \subset V$ solution to \eqref{eq:approachform_add}.
    We have
    \begin{alignat}{3}
        \notag
        a(u-u_h^+, u-u_h^+)
        % & =
        %a\big(u-u_h^+,(u-u_{\theta})-p_h^+\big) \\
        %\notag
         & =
        a\big(u-u_h^+,(u-u_{\theta})-p_h^+ - v_h+ v_h\big),
         &   & \quad \forall v_h \in V_h^0  \\
        \label{eq:proof_additive_1}
         & =
        a\big(u-u_h^+,(u-u_{\theta})-v_h\big)  +
        a\big(u-u_h^+, v_h-p_h^+\big),
         &   & \quad \forall v_h \in V_h^0.
    \end{alignat}

    Let us first estimate the second term on the right-hand side of \eqref{eq:proof_additive_1}.
    Using the fact that $V_h^0\subset V^0$, we have, by Galerkin orthogonality (difference of the continuous problem \eqref{eq:weakform} and discrete problem \eqref{eq:add1}),
    \begin{equation}\label{eq:orth-gal+}
        a(u-u_h^+, z_h)=0, \quad \forall z_h \in V_h^0.
    \end{equation}
    %   The above equality is valid for all $v_h \in V_h$, and $v_h-p_h^+ \in V_h$.
    Therefore, for  $z_h=v_h-p_h^+\in V_h^0$, we obtain
    \[
        a\big(u-u_h^+, v_h-p_h^+\big)=0, \quad \forall v_h \in V_h^0.
    \]
    Plugging this equality into \eqref{eq:proof_additive_1} yields
    \[
        a(u-u_h^+, u-u_h^+)=a\big(u-u_h^+,(u-u_{\theta})-v_h\big), \quad \forall v_h \in V_h^0.
    \]

    Denoting by $\alpha$ and $\gamma$ the
    coercivity and continuity constants of the bilinear form $a$, we have
    \begin{alignat*}{3}
        \alpha \big| u-u_h^+\big|_{H^1}^2 & \leq
        a\big(u-u_h^+,u-u_h^+\big) =
        a\big(u-u_h^+,(u-u_{\theta})-v_h\big),
                                          &      & \quad \forall v_h \in V_h^0, \\
                                          & \leq
        \gamma \big| u-u_h^+\big|_{H^1} \big| (u-u_{\theta})-v_h \big|_{H^1},
                                          &      & \quad \forall v_h \in V_h^0,
    \end{alignat*}
    which immediately leads to
    \[
        | u-u_h^+|_{H^1} \leq \frac{\gamma}{\alpha} \big| (u-u_{\theta})-v_h \big|_{H^1}, \quad \forall v_h \in V_h^0.
    \]
    Since the above relation is valid for all $v_h \in V_h^0$,
    we apply it to $v_h=\mathcal{I}_h(u-u_\theta) \in V_h^0$
    with $\mathcal{I}_h$ the Lagrange interpolation operator \eqref{eq:Ih} in $V_h$, it holds
    using interpolation estimate given in \cref{th:interpol},
    %classical interpolation results yields, see, e.g., \cite{Ern2004TheoryAP},
    \[
        |u-u_h^+|_{H^1} \leq C_q\frac{\gamma}{\alpha} h^{q} | u-u_{\theta} |_{H^{q+1}},
    \]
    with $C_q$ defined in \cref{sec:FEM}.

    %Rewriting the above expression and introducing the error associated with the classical FEM given in \cref{thm:classical_error_estimate}, we obtain
    The above expression can be rewritten as
    \begin{equation}\label{eq:trucbidule}
        | u-u_h^+|_{H^1} \leq C_q\frac{\gamma}{\alpha}  C_\text{\rm gain}^+ \, h^{q}|u|_{H^{q+1}} \,,
    \end{equation}
    with
    \[
        C_\text{\rm gain}^+ = \frac{| u-u_{\theta} |_{H^{q+1}}}{| u |_{H^{q+1}}},
    \]
    which completes the first part of the proof.

    \textbf{$L^2$-error:}
    We will follow the Aubin-Nitsche technique.
    Consider $w\in H^2(\Omega)$ the solution to
    $$\mathcal{L}^* w=u-u_h^+,$$
    with homogeneous Dirichlet boundary condition.
    Thanks to \cref{th:ellip}, one has
    \begin{equation}\label{eq:wH2}
        \|w\|_{H^2}\leqslant C_e\|u-u_h^+\|_{L^2}.
    \end{equation}
    Using the Galerkin orthogonality \eqref{eq:orth-gal+} and the continuity of the bilinear form $a$,
    $$\|u-u_h^+\|_{L^2}^2
        = a(u-u_h^+,w-I_hw)
        \leq \gamma |u-u_h^+|_{H_1}|w-I_hw|_{H_1}.$$
    Thanks to \cref{th:interpol}
    and \eqref{eq:wH2},
    $$|w-I_hw|_{H_1}\leq C_eC_1\|u-u_h^+\|_{L^2},$$
    which leads to the conclusion by using \eqref{eq:trucbidule}.
\end{proof}

\begin{remark}
    \label{rmk:C_gain_additif}
    The gain constant $C_\text{\rm gain}^+$ defined in \eqref{eq:gain_add} shows that the closer the prior is to the solution,
    the smaller is the error constant associated with the FEM while keeping the same order of accuracy.
    Therefore, as soon as \smash{$C_\text{\rm gain}^+ < 1$}, the FEM with additive prior will be more accurate than the classical one.
    While this gives us a particularly flexible constraint, our objective is to balance this gain by relaxing the contribution $h^q$, using a coarser grid and low-order polynomial, to reduce the computational cost of the FEM while maintaining accuracy.
    Nonetheless, the gain is related to the~$L^2$ error associated with the derivatives of $(q+1)$\textsuperscript{th} order (with $1\leqslant q\leqslant k$) of the prior.
    This shows that the prior must accurately approximate the derivatives of the solution in addition to the solution itself.
    This highlights that we need to build our prior by ensuring a good approximation of the derivatives of the solution.
    It also shows that the higher the order of the finite elements $k$ is, the better our prior should approximate the higher-order derivatives.
    Therefore, it is more appropriate to use only low-order FEM so that $k$ remains small.
\end{remark}
