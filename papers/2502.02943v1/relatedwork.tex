\section{Related Work}
\label{subsec:related-work}




Understanding human behavior to uncover the underlying reward mechanisms of decision processes gained significant attention after the seminal work of Ng and Russell~\cite{ng2000}. 
When combined with entropy regularization and deep learning, Inverse Reinforcement Learning (IRL) has evolved into a powerful tool for analyzing complex behavioral patterns, such as overtaking maneuvers in driving~\cite{you2019advanced} or identifying optimal NHL players for fantasy sports~\cite{ijcai2020p464}. 
While IRL has been extensively applied in vision-based domains, its application to social media has been more limited due to challenges in encoding the underlying data structure and ensuring sufficient data availability.

Early studies applying IRL to social media explored how feedback influences personal engagement on \reddit, showing that users tend to continue engaging based on the reception of their contributions~\cite{das2014effects}. 
Luceri et al.~\cite{luceri2020detecting} applied predictive modeling to detect troll behavior on X (formerly Twitter) by identifying key behavioral features. Geissler et al.~\cite{geissler2023analyzing} examined propaganda strategies following the Russian invasion of Ukraine using a comparable framework. 
On YouTube, Hoiles et al.~\cite{hoiles2020rationally} leveraged IRL to model and predict viewer commenting behavior, demonstrating how the rational inattention model~\cite{sims2003implications} can explain variations in user engagement. Among these platforms---X, YouTube, and \reddit---\reddit stands out for its highly hierarchical data structure, organized around nested discussions. Our framework uniquely adapts deep IRL by designing states, actions, and features that reflect \reddit's hierarchical conversation structures, considering platform-specific behaviors such as creating threads or root comments.

As our study focuses on homophily in social media behavior, it connects closely to research examining social media dynamics. Massachs et al.~\cite{massachs2020roots} investigated the roots of Trumpism on the subreddit \textit{r/The\_Donald} through the lens of homophily, social influence, and social feedback. In their study, homophily was measured through vector participation across different subreddits, which, while suitable for that case, lacks broader generalizability and behavioral detail. 
Monti et al.~\cite{monti2023evidence} evaluated homophily and heterophily among ideological and demographic groups in \reddit's \textit{r/news} community, finding that users tend to engage with opposite ideological sides, while demographic groups, particularly age and income, exhibit homophily. This challenges the echo chamber narrative and highlights the role of affective polarization in a divided society. Other studies~\cite{de2021no,efstratiou2023non} have challenged the echo chamber narrative on \reddit, showing that political interactions involve significant cross-cutting engagement, with polarization and hostility more prevalent within political groups or asymmetrically between supporters, rather than between opposing sides.

Our work builds upon these insights by extending the analysis to a diverse set of subreddits, each with unique conversational patterns. We offer a deeper understanding of homophily and user behavior across various communities on \reddit by introducing a novel behavioral homophily measure through our IRL framework. %