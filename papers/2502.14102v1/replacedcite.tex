\section{Related Work}
The grounded constraints in the explanations of X-DCOP bear some similarity to \emph{nogoods} in Constraint Satisfaction Problems (CSPs) and Constraint Optimization Problems (COPs) including their distributed versions (DCSPs and DCOPs)____. Nogoods are value assignments to subsets of variables that cannot lead to a solution because they violate one or more constraints. However, the role of nogoods differ substantially. Nogoods are used to speed up the search of solutions by pruning portions of the search space that can be ignored. However, grounded constraints are used as solutions themselves.

There is also a great deal of literature on Explainable CSP____. A common theme across many of these methods is their use of hitting sets to find minimal unsatisfiable sets (MUSes) to explain why a CSP is unsatisfiable. An MUS is a set of constraints that is unsatisfiable but any of its subsets are satisfiable. Therefore, similar to our goal of finding shortest explanations, MUSes are subset-minimal justifications for why a CSP is unsatisfiable. Our approach can thus be viewed as a generalization of MUSes to weighted constraints, where constraints are no longer Boolean, but have associated costs instead. The approach to use MUSes as explanations have also been explored in other explainable subareas, such as explainable planning and scheduling____, where MUSes correspond to sets of logical rules and constraints that must be satisfied by a plan or schedule. Finally, it is important to note that the existing explainable CSP, planning, and scheduling approaches discussed above are generally centralized approaches, while CEDAR is a distributed framework.

%\memo{Will will write about nogoods here. Possibly talk about explainable CSPs, which is mostly on MUS and MCS.\footnote{www.ijcai.org/proceedings/2021/0601.pdf} This will tie in nicely with logic-based explainable work that uses such concepts (e.g., Stelios' JAIR paper)}