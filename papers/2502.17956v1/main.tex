% This must be in the first 5 lines to tell arXiv to use pdfLaTeX, which is strongly recommended.
\pdfoutput=1
% In particular, the hyperref package requires pdfLaTeX in order to break URLs across lines.

\documentclass[11pt]{article}

\usepackage{array}
\newcolumntype{L}[1]{>{\raggedright\let\newline\\\arraybackslash\hspace{0pt}}m{#1}}
\newcolumntype{C}[1]{>{\centering\let\newline\\\arraybackslash\hspace{0pt}}m{#1}}
\newcolumntype{R}[1]{>{\raggedleft\let\newline\\\arraybackslash\hspace{0pt}}m{#1}}
% Change "review" to "final" to generate the final (sometimes called camera-ready) version.
% Changƒe to "preprint" to generate a non-anonymous version with page numbers.
% \usepackage[review]{acl}
\usepackage[preprint]{acl}

% Standard package includes
\usepackage{times}
\usepackage{latexsym}

% For proper rendering and hyphenation of words containing Latin characters (including in bib files)
\usepackage[T1]{fontenc}
% For Vietnamese characters
% \usepackage[T5]{fontenc}
% See https://www.latex-project.org/help/documentation/encguide.pdf for other character sets

% This assumes your files are encoded as UTF8
\usepackage[utf8]{inputenc}

% This is not strictly necessary, and may be commented out,
% but it will improve the layout of the manuscript,
% and will typically save some space.
\usepackage{microtype}

% This is also not strictly necessary, and may be commented out.
% However, it will improve the aesthetics of text in
% the typewriter font.
\usepackage{inconsolata}

%Including images in your LaTeX document requires adding
%additional package(s)
\usepackage{graphicx}
% \usepackage{arydshln} % for hdashline

% Customize
\usepackage{lipsum}
\usepackage{makecell}
\usepackage{booktabs}
\usepackage{multirow}

\usepackage{times}
\usepackage{latexsym}
\usepackage{amsmath}
\usepackage{amssymb}
\usepackage{physics}

\usepackage{graphicx}
\usepackage{caption}
\usepackage{subcaption}
\usepackage{hyperref}
\usepackage{enumerate}   

\usepackage{array}
\usepackage{float}
\restylefloat{table}
\usepackage{arydshln}
\usepackage{paralist}

\usepackage{mdframed}
\usepackage{enumitem}

\setlist{topsep=6pt,itemsep=4pt,partopsep=4pt, parsep=4pt}

\newcommand\blfootnote[1]{%
  \begingroup
  \renewcommand\thefootnote{}\footnote{#1}%
  \addtocounter{footnote}{-1}%
  \endgroup
}

\newcommand\dummy[1]{\textcolor{gray}{#1}}
\newcommand\todo[1]{\textcolor{red}{TODO: #1}}

% If the title and author information does not fit in the area allocated, uncomment the following
%
%\setlength\titlebox{<dim>}
%
% and set <dim> to something 5cm or larger.

% \title{Towards Better Understanding of Program-of-Thought \\ in Cross-Lingual and Multilingual Mathematical Reasoning}
\title{Towards Better Understanding of Program-of-Thought Reasoning \\ in Cross-Lingual and Multilingual Environments}

% Author information can be set in various styles:
% For several authors from the same institution:
% \author{Author 1 \and ... \and Author n \\
%         Address line \\ ... \\ Address line}
% if the names do not fit well on one line use
%         Author 1 \\ {\bf Author 2} \\ ... \\ {\bf Author n} \\
% For authors from different institutions:
% \author{Author 1 \\ Address line \\  ... \\ Address line
%         \And  ... \And
%         Author n \\ Address line \\ ... \\ Address line}
% To start a separate ``row'' of authors use \AND, as in
% \author{Author 1 \\ Address line \\  ... \\ Address line
%         \AND
%         Author 2 \\ Address line \\ ... \\ Address line \And
%         Author 3 \\ Address line \\ ... \\ Address line}

\author{
Patomporn Payoungkhamdee\textsuperscript{1}, 
Pume Tuchinda\textsuperscript{1},
Jinheon Baek\textsuperscript{2}
 \\
\textbf{Samuel Cahyawijaya}\textsuperscript{3},
\textbf{Can Udomcharoenchaikit}\textsuperscript{1},
\textbf{Potsawee Manakul}\textsuperscript{4}, \\
\textbf{Peerat Limkonchotiwat}\textsuperscript{5},
\textbf{Ekapol Chuangsuwanich}\textsuperscript{6}, \textbf{Sarana Nutanong}\textsuperscript{1}\\
  \textsuperscript{1}School of Information Science and Technology, VISTEC \hspace{1mm} 
  \textsuperscript{2}KAIST \hspace{1mm}
  \textsuperscript{3}Cohere \hspace{1mm} \\
  \textsuperscript{4}SCB 10X \hspace{1mm} 
  \textsuperscript{5}AI Singapore \hspace{1mm}
  \textsuperscript{6}Department of Computer Engineering,  Chulalongkorn University \\
  \texttt{\{patomporn.p\_s21, pumet\_pro,canu\_pro,snutanon\}@vistec.ac.th}
  \\
  \texttt{jinheon.baek@kaist.ac.kr, samuelcahyawijaya@cohere.com, potsawee@scb10x.com} \\
  \texttt{peerat@aisingapore.org, ekapolc@cp.eng.chula.ac.th}
}

%\author{
%  \textbf{First Author\textsuperscript{1}},
%  \textbf{Second Author\textsuperscript{1,2}},
%  \textbf{Third T. Author\textsuperscript{1}},
%  \textbf{Fourth Author\textsuperscript{1}},
%\\
%  \textbf{Fifth Author\textsuperscript{1,2}},
%  \textbf{Sixth Author\textsuperscript{1}},
%  \textbf{Seventh Author\textsuperscript{1}},
%  \textbf{Eighth Author \textsuperscript{1,2,3,4}},
%\\
%  \textbf{Ninth Author\textsuperscript{1}},
%  \textbf{Tenth Author\textsuperscript{1}},
%  \textbf{Eleventh E. Author\textsuperscript{1,2,3,4,5}},
%  \textbf{Twelfth Author\textsuperscript{1}},
%\\
%  \textbf{Thirteenth Author\textsuperscript{3}},
%  \textbf{Fourteenth F. Author\textsuperscript{2,4}},
%  \textbf{Fifteenth Author\textsuperscript{1}},
%  \textbf{Sixteenth Author\textsuperscript{1}},
%\\
%  \textbf{Seventeenth S. Author\textsuperscript{4,5}},
%  \textbf{Eighteenth Author\textsuperscript{3,4}},
%  \textbf{Nineteenth N. Author\textsuperscript{2,5}},
%  \textbf{Twentieth Author\textsuperscript{1}}
%\\
%\\
%  \textsuperscript{1}Affiliation 1,
%  \textsuperscript{2}Affiliation 2,
%  \textsuperscript{3}Affiliation 3,
%  \textsuperscript{4}Affiliation 4,
%  \textsuperscript{5}Affiliation 5
%\\
%  \small{
%    \textbf{Correspondence:} \href{mailto:email@domain}{email@domain}
%  }
%}

\begin{document}
\maketitle
\begin{abstract}
% \textcolor{red}{send the first draft (70\%) by 7 Jan for Sam, legal purpose}
% This document is a supplement to the general instructions for *ACL authors. It contains instructions for using the \LaTeX{} style files for ACL conferences.
% The document itself conforms to its own specifications, and is therefore an example of what your manuscript should look like.
% for reproducibility we release bla bla 

Multi-step reasoning is essential for large language models (LLMs), yet multilingual performance remains challenging. 
%
While Chain-of-Thought (CoT) prompting improves reasoning, it struggles with non-English languages due to the entanglement of reasoning and execution. 
%
Program-of-Thought (PoT) prompting separates reasoning from execution, offering a promising alternative but shifting the challenge to generating programs from non-English questions. 
%
We propose a framework to evaluate PoT by separating multilingual reasoning from code execution to examine (i) the impact of fine-tuning on question-reasoning alignment and (ii) how reasoning quality affects answer correctness.
%
Our findings demonstrate that PoT fine-tuning substantially enhances multilingual reasoning, outperforming CoT fine-tuned models. 
%
We further demonstrate a strong correlation between reasoning quality (measured through code quality) and answer accuracy, highlighting its potential as a test-time performance improvement heuristic. 


%Multi-step reasoning is essential for large language models (LLMs) in mathematical and symbolic tasks.
%
%While Chain-of-Thought (CoT) prompting enables step-by-step problem-solving, Program-of-Thought (PoT) prompting improves performance by decoupling natural language reasoning from computational execution.
%
%This distinction enhances accuracy in arithmetic and iterative reasoning but remains underexplored in multilingual settings, where LLMs struggle with non-English languages.
%
%This study investigates the effectiveness of PoT in cross-lingual and multilingual settings. We evaluate two approaches: (1) Cross-lingual fine-tuning, where models trained on English PoT data are tested zero-shot in other languages, and (2) Multilingual fine-tuning, where target languages are explicitly included. We also examine the link between PoT performance and code quality, using program correctness to enhance test-time inference.
%
%Experiments show that PoT outperforms CoT in cross-lingual settings, with further gains when inline comments are removed.
%
%In multilingual setups, translating inline comments improves alignment with target languages. Leveraging a code quality metric further boosts PoT performance, increasing low-resource accuracy from 11.3\% to 27.7\%. 
%
%Our findings highlight the potential of PoT for multilingual mathematical reasoning and better cross-lingual performance.


\end{abstract}



% These instructions are for authors submitting papers to *ACL conferences using \LaTeX. They are not self-contained. All authors must follow the general instructions for *ACL proceedings,\footnote{\url{http://acl-org.github.io/ACLPUB/formatting.html}} and this document contains additional instructions for the \LaTeX{} style files.

% The templates include the \LaTeX{} source of this document (\texttt{acl\_latex.tex}),
% the \LaTeX{} style file used to format it (\texttt{acl.sty}),
% an ACL bibliography style (\texttt{acl\_natbib.bst}),
% an example bibliography (\texttt{custom.bib}),
% and the bibliography for the ACL Anthology (\texttt{anthology.bib}).

\begin{figure}[ht]
    \centering
    \includegraphics[width=0.8\linewidth]{graphs/greater_than_naive.pdf}
    \vspace{0.5cm}
    \includegraphics[width=0.8\linewidth]{graphs/p1_bottom.png}
    \vspace{-5pt}
    \caption{\textcolor{positional}{Positional} vs.\ \textcolor{nonpositional}{non-positional} circuits. In a \textcolor{nonpositional}{non-positional} circuit, the same edges must be included at all positions. A \textcolor{positional}{positional} circuit can distinguish between the same edge at different positions. This specificity yields better trade-offs between circuit size and faithfulness. It can also increase both precision and recall.}
    \label{fig:p1}
    \vspace{-5pt}
\end{figure}

\section{Introduction}

\looseness=-1
A primary goal of interpretability research is to characterize the internal mechanisms in language models (LMs) and other NLP models. 
A core approach in this area is \textbf{circuit discovery}---identifying the minimal subgraph within the model's computation graph that performs a specific task \citep{olah2021framework,olah-mech}.
Typically, the nodes of a circuit represent model components (e.g., attention heads, neurons, or layers).
While manual circuit discovery methods can yield position-specific insights \citep{wanginterpretability,goldowskydill2023localizingmodelbehaviorpath}, \emph{automatic methods often overlook positional information}, treating components as uniformly relevant across all input token positions \citep{conmytowards,syed2023attribution}. 
For instance, if an attention head is included in a circuit, it is assumed to contribute equally to the computation for every position in the input sequence.
The assumption that circuits are position-invariant ignores the fact that different positions often require distinct computations.
By ignoring positions, current methods limit their ability to capture mechanisms that operate across positions, such as interactions between attention heads across positions.

In this study, we start by demonstrating that positional agnosticism is a significant limitation (\S\ref{sec:motivating}). Then, to address these limitations, we introduce a new approach: position-aware edge attribution patching (PEAP; \S\ref{sec:full_circ_discovery}; Figure~\ref{fig:p1}). Current approaches  assume that if an edge is in a circuit, then the same edge will be in the circuit at all positions, thus leading to low precision. It is also assumed that an edge's importance should be aggregated across positions before deciding whether it should be included in the circuit; this can lead to cancellation effects, and thus low recall. PEAP instead allows us to compute the importance of cross-positional edges, and separately evaluates edge importance at each position. We show that this leads to smaller and more accurate circuits; see Figure~\ref{fig:p1}.

Incorporating positional information into circuit discovery is straightforward when inputs have the same length and structure across examples.

However, realistic datasets are not nearly this templatic.
How, then, can we incorporate positional information into automatic circuit discovery?
To address this challenge, we propose \textbf{schemas} (\S\ref{sec:schema}). 
Schemas assign semantic labels to spans of tokens, enabling information aggregation across examples even when the spans differ in length.

For example, in the input ``The \textcolor{positional}{war} lasted from 1453 to 14\underline{\hspace{1em}},'' the span ``\textcolor{positional}{war}'' could be labeled as ``\emph{Subject}''.
This enables handling spans with varying lengths: the phrase ``\textcolor{positional}{Black Plague}'' in another example can be treated as a single positional span with the same role as ``\textcolor{positional}{war}''.
In experiments with two LMs and three tasks, we find that circuits discovered using schemas achieve a better trade-off between circuit size and faithfulness to the model's behavior than position-agnostic circuits.
Importantly, position-aware circuits offer a more precise representation of the underlying mechanisms, providing a more concise foundation for mechanistic explanations.

We also present a fully automated pipeline for schema generation and application (\S\ref{sec:schema-generation}) using large language models (LLMs). 
We evaluate the quality of the generated schemas and their utility in discovering position-aware circuits (\S\ref{sec:schema-eval}).
Notably, circuits derived using automatically generated and applied schemas achieve comparable faithfulness scores to circuits discovered with human-designed and manually applied schemas.

We summarize our contributions as follows:
\begin{itemize}[noitemsep,leftmargin=*,topsep=1pt,parsep=1pt]
    \item Introduce a position-aware circuit discovery method, which obtains better faithfulness than position-agnostic discovery.  
    \item Introduce dataset schemas,  facilitating positional circuit discovery in more naturalistic settings. 
    \item Develop an automated schema generation and application pipeline with LLMs, yielding schemas that are comparable to manually-annotated ones.
\end{itemize}

\section{Proposed Studies}


%\textcolor{blue}{restructure and shorten // dataset generation move to section 3 -- define what we can observe // jab Simplify -> Prof. Sarana revise}


% \section{Fine-tuning Decisions}
% \subsection{Cross-lingual Fine-tuning}
% \subsection{Multilingual Fine-tuning}
% \section{Test-time Scaling}
% \section{Conclusion}




% \textcolor{blue}{What makes others fail and why we do this way. Find a way to demonstrate.}

% \textcolor{red}{research question / hypothesis intro, then sequence section 3 -- To what extent does ICE correlate with performance at the sample level?}

% \subsection{Problem Formulation} 

%A dataset for the mathematical reasoning task can be represented as $\mathcal{D} = \{(\vb*{Q}_i, \vb*{R}_i, \vb*{A}_i)\}_{i=1}^N$, where $\vb*{Q}_i$ represents a math question, $\vb*{R}_i$ is a sequence of intermediate reasoning steps, $\vb*{A}_i$ is a numerical answer, and $N$ is the number of samples in the dataset. 
%
%Similar to \citet{mathoctopus}, we train the models to reason through the language modeling objective:
%\begin{equation}
%    \mathcal{L} = - \frac{1}{B}\sum_{i=1}^B\log p_\theta(\vb*{R}_i|\vb*{Q}_i)
%\label{eq:loss}
%\end{equation}
%where $B$ is the batch size.


%During inference, a trained language model generates multi-step reasoning, denoted as $\hat{\vb*{R}_i}\sim p_\theta(\vb*{Q}_i)$.
%
%In natural language reasoning, the final numerical answer is derived from the predicted chain of thought ($\hat{\vb*{C}_i}$) by applying regular expressions ($\mathsf{Re}$) for pattern matching, represented as  $\hat{\vb*{A}_i}=\mathsf{Re}(\hat{\vb*{C}_i})$.
% 
%In contrast, in programming-based reasoning, particularly within the PoT framework, the answer is obtained by executing ($\mathsf{Ex}$) the generated program-of-thought ($\hat{\vb*{R}_i}$) formulated as $\hat{\vb*{A}_i}=\mathsf{Ex}(\hat{\vb*{R}_i})$.



% \subsection{Fine-Tuning Experimental Design}
%\subsection{GSM8KPoT \& MGSM8KPoT}






%\subsection{Program-of-Thought Data}
\subsection{Fine-tuning for Q-R Alignment (P1)}

% \vspace{2mm}
% \noindent \textbf{Dataset Generation}:
% \label{section:synthesize-pot}
% 
To fairly compare PoT and CoT, we use the \emph{Grade School Math} (\texttt{GSM8K}) dataset \cite{cobbe2021gsm8k} and explore three prompting strategies for generating PoT with an Oracle LLM: (i) zero-shot PoT, (ii) few-shot PoT, and (iii) the proposed few-shot PoT with CoT guidance, 
%xw
as shown in Figure~\ref{fig:pipeline-cross}.
%
% Zero-shot PoT prompting generates Python solutions without examples, while few-shot PoT prompting improves upon this by providing two solved examples. The most effective approach, few-shot PoT prompting with CoT guidance, further enhances program generation by integrating CoT reasoning and PoT solutions within the prompt.
Zero-shot PoT generates Python solutions without examples. Few-shot PoT improves this with two solved examples while adding CoT guidance further enhances program generation.
%
This structured guidance substantially improves accuracy, achieving a 96.1\% correctness rate in PoT-generated samples and leading to the development of the \texttt{GSM8KPoT} dataset (details in Appendix~\ref{ap:pot-syn}).



\begin{figure}[!t]
    \includegraphics[width=\linewidth, trim={0 0 0 0.1cm}, clip]{latex/figures/pipeline-cross-v2.pdf}
    \caption{
    The generation pipeline for \texttt{GSM8KPoT}, in which a PoT answer ($\vb*{R}_i^\text{en}$) is synthesized using the Oracle LLM, with additional natural language reasoning ($\vb*{C}_i^\text{en}$) provided as guidance.
    }
    \label{fig:pipeline-cross}
\end{figure}








Examining PoT in multilingual settings is challenging due to the scarcity of datasets that align questions across multiple languages with structured reasoning steps.
%
To address this, we construct dataset variants (outlined in Table~\ref{tab:compare-metods}) to evaluate cross-lingual and multilingual fine-tuning.
%
% Table~\ref{tab:compare-metods} shows how we control for language effects by varying question and inline comment languages. This enables us to assess the impact of each fine-tuning strategy.
We control for language effects by varying the languages of questions and inline comments, allowing us to assess the impact of each fine-tuning strategy.

In the context of cross-lingual and multilingual PoT, inline comments can potentially play a crucial role. As established by \citet{mgsm}, the language used in multi-step reasoning processes, such as those in CoT reasoning, is a key design consideration.
%
We hypothesize that the design choices for inline comments in PoT function similarly to the language considerations in CoT. 
% Therefore, we conduct a comprehensive analysis to examine the implications of these choices.
Thus, we analyze its implications comprehensively.


%



% \begin{table}[h]
%   \centering
%   \resizebox{\columnwidth}{!}{
%   \begin{tabular}{C{0.8cm}|C{0.8cm}|C{1cm}|C{2.7cm}|C{2.7cm}}
%     \hline
%     \textbf{Setup} & Lang. of $Q$ & Lang. of Comm. in $R$ & Source Dataset & Resultant Dataset \\
%     \hline
%     Cross & En & En & \texttt{GSM8K} & \texttt{GSM8KPoT-en} \\
%     % \hdashline
%     \cdashline{2-5}
%     & En & -  & \texttt{GSM8K} &  \texttt{GSM8KPoT-nc} \\
%     \hline
%     Multi & En & Multi & \texttt{MGSM8kInstruct} + \texttt{GSM8KPoT-en} & \texttt{GSM8KPoT} \texttt{-cross-comment} \\
%      \cdashline{2-5}
%      & Multi & En & \texttt{MGSM8kInstruct} + \texttt{GSM8KPoT-en} & \makecell{\texttt{GSM8KPoT} \\ \texttt{-cross-question}} \\    
%     \cdashline{2-5}
%      & Multi & - & \texttt{MGSM8kInstruct} + \texttt{GSM8KPoT-nc}& \makecell{\texttt{GSM8KPoT} \\ \texttt{-nc}} \\
%     \cdashline{2-5}
%      & Multi & Multi & \texttt{MGSM8kInstruct} + \texttt{GSM8KPoT-en}& \makecell{\texttt{GSM8KPoT} \\ \texttt{-parallel}} \\
%     \hline
%   \end{tabular}
%   }
%   \caption{
%   The number of correct 
%   \textcolor{blue}{(ready to narrate in the paragraph)}
%   }
%   \label{tab:compare-metods}
% \end{table}















\subsubsection{Cross-lingual Setup}
% As shown in Table~\ref{tab:compare-metods},
The cross-lingual setup comprises two datasets: one with inline comments in the reasoning steps and one without, defined as follows.
%
\begin{table}[!t]
  \vspace{-2mm}

  \centering
  \renewcommand{\arraystretch}{1.4}

  \resizebox{\columnwidth}{!}{
  \begin{tabular}{C{1.0cm}|C{1cm}|C{1cm}|C{2.5cm}|C{0.3cm}}
    \hline
    \textbf{Setup} & Lang. of $Q$ & Lang. of Comm. in $R$ & Dataset & Eq. \\
    \hline
    Cross & En & En & $\mathcal{D}^\texttt{GSM8KPoT}_\texttt{en}$  & \ref{eq:gsm8kpot}\\
    % \hdashline
    \cdashline{2-5}
    & En & \emph{nc}  &  $\mathcal{D}^\texttt{GSM8KPoT}_\texttt{nc}$ & \ref{eq:gsm8kpot-nc} \\
    \hline
    Multi & En & Multi & $\mathcal{D}^\texttt{MGSM8KPoT}_\texttt{cross-comment}$ & \ref{eq:mgsm8kpot-cross-comment}\\
     \cdashline{2-5}
     & Multi & En & $\mathcal{D}^\texttt{MGSM8KPoT}_\texttt{cross-question}$ & \ref{eq:mgsm8kpot-cross-question}\\    

    \cdashline{2-5}
     & Multi & Multi & $\mathcal{D}^\texttt{MGSM8KPoT}_\texttt{parallel}$ & \ref{eq:mgsm8kpot-parallel} \\

    \cdashline{2-5}
     & Multi & \emph{nc} & $\mathcal{D}^\texttt{MGSM8KPoT}_\texttt{nc}$ & \ref{eq:mgsm8kpot-nc} \\
    \hline
  \end{tabular}
  }
  \vspace{-2mm}
  \caption{
  Our proposed study employs multiple approaches, leveraging the question-comment characteristics within the dataset to compare different best fine-tuning strategies. \emph{NC} stands for ``no comment''.
  }
  \label{tab:compare-metods}
\end{table}


\vspace{2mm}
\noindent
\underline{\emph{En-En}} --- 
% To construct the first dataset, we adopt 
% the \emph{Grade School Math (GSM)} dataset called, 
% $\texttt{GSM8K}$, as our source dataset.
%
% We convert their reasoning steps from natural language to Python code using the process described in a~\ref{section:synthesize-pot}. 
% 
% This dataset can be denoted as in the following equation.
We employ \texttt{GSM8KPoT} as the foundational dataset, which can be formally represented by the following equation.
\begin{equation}
\mathcal{D}^\texttt{GSM8KPoT}_\texttt{en} = \{(\vb*{Q}_i^\text{en}, \vb*{R}_i^\text{en})\}_{i=1}^N,
\label{eq:gsm8kpot}
\end{equation}
where the questions $\vb*{Q}_i^\text{en}$ are obtained from English GSM8K, and the synthesized intermediate reasoning in the programming language ($\vb*{R}_i^\text{en}$) include inline comments in English. Note that the superscript \text{en} in $\vb*{R}_i^\text{en}$ denotes the language of code comments. 
% 

\vspace{2mm}
\noindent
\underline{\emph{En-nc}} --- We also include a variant with all comments removed.
%
\begin{equation}
\mathcal{D}^\texttt{GSM8KPoT}_\texttt{nc} = \{(\vb*{Q}_i^\text{en}, \vb*{R}_i^\text{nc})\}_{i=1}^N,
\label{eq:gsm8kpot-nc}
\end{equation}
where the superscript ``$\text{nc}$'' in the reasoning steps $\vb*{R}_i^\text{nc}$ stands for ``no comment''. 


%These datasets are independently used to fine-tune language models following the supervised learning objective outlined in Equation~\ref{eq:loss}.
%
%Following the zero-shot cross-lingual approach described in \citet{xtreme}, each model is further evaluated on unseen languages using the test dataset, $\mathcal{D}^\text{test} = \{(\vb*{Q}_i^l, \vb*{A}_i^l) |l\in L_\text{tgt} \}_{i=1}^M$, where $L_\text{tgt}$ denotes the set of unseen target languages and $M$ represents the number of test samples.
% 
% The trained language models ($p_\theta$) generate sequences of reasoning steps $\hat{\vb*{R}}_i^l\sim p_\theta(\vb*{Q}_i^l)$.
% 



% The predicted answers are then obtained through a programmatic execution process ($\mathcal{E}$), expressed as $\hat{\vb*{A}}_i^l=\mathcal{E}(\hat{\vb*{R}}_i^l)$.
% Finally, the predicted answers are compared with the ground-truth answers to evaluate their accuracy.



\subsubsection{Multilingual Setup}

% To examine the effectiveness of PoT in a multilingual context, where explicit reasoning demonstrations are available in target languages, we conduct a comparative analysis with the CoT approach.
% We adopt MGSM8KInstruct~\cite{mathoctopus} as the reference dataset for CoT in multilingual settings.
% This dataset comprises question-reasoning pairs $(\vb*{R}_i$, $\vb*{Q}_i)$ with $\vb*{Q}_i$ expressed in English, along with translations in nine additional languages, enabling the alignment of reasoning capabilities across different languages.
% \citet{mathoctopus} introduced two training strategies:

% As shown in Table~\ref{tab:compare-metods},
The multilingual setup comprises four datasets.
%
Following the concept proposed in MGSM8KInstruct~\cite{mathoctopus}, we consider CoT cross and CoT parallel strategies, varying how the languages of questions and inline comments match or mismatch.
%
% \begin{compactenum}[(i)]
%     % \item \emph{CoT Cross}: Incorporates English questions with answers in the target language, promoting multilingual adaptability. 
%     \item \emph{CoT Cross}: One dataset pairs English questions with reasoning steps containing comments in the target language ($\mathcal{D}^\texttt{MGSM8KPoT}_\texttt{cross-comment}$), while the other pairs multilingual questions with reasoning steps containing English comments ($\mathcal{D}^\texttt{MGSM8KPoT}_\texttt{cross-question}$). 
%    
%     %where English questions have answers in the target language, promoting adaptability, and
%     % \item \emph{CoT Parallel}: Uses question-answer pairs in the same language to enhance the PoT capability within each target language (further details in Appendix~\ref{ap:mathoctopus}).
%     \item \emph{CoT Parallel}: A dataset where both questions and answers are in the same language ($\mathcal{D}^\texttt{MGSM8KPoT}_\texttt{parallel}$).
%     %(details in Appendix~\ref{ap:mathoctopus}).
% \end{compactenum}
%
% In addition, we include a no-comment variant where multilingual questions are paired with reasoning steps that exclude comments. 
We also include a no-comment variant, pairing multilingual questions with reasoning steps that exclude comments.
%
These four datasets are defined as follows. 
%

\vspace{2mm}
\noindent
\underline{\emph{En-Multi}} --- Following the CoT definition in CoT Cross, we translate English inline comments using machine translation (\texttt{MT}), producing program reasoning in target languages:
    \begin{equation}
        \mathcal{D}^\texttt{MGSM8KPoT}_\texttt{cross-comment} = \{(\vb*{Q}_i^\text{en}, \vb*{R}_i^l)|l\in L_\text{all}\}_{i=1}^N,
        \label{eq:mgsm8kpot-cross-comment}
    \end{equation}
    % 
where $L_\text{all}$ denotes the language set. 
%
The superscript $l$ in $\vb*{R}_i^l$ is a variable representing a language. 

\vspace{2mm}
\noindent
\underline{\emph{Multi-En}} --- This variant provides multilingual questions $\vb*{Q}_i^l$ by applying machine translation to $\vb*{Q}_i^\text{en}$, while keeping the inline comments in English.
    \begin{equation}
        \mathcal{D}^\texttt{MGSM8KPoT}_\texttt{cross-question} = \{(\vb*{Q}_i^l, \vb*{R}_i^\text{en})|l\in L_\text{all}\}_{i=1}^N.
        \label{eq:mgsm8kpot-cross-question}
    \end{equation}
    % where the PoT answers also came from $\mathcal{D}^\texttt{GSM8KPoT}_\texttt{en}$.

\vspace{2mm}
\noindent    
\underline{\emph{Multi-Multi}} --- Both questions and inline comments are in the same language $l$:
    \begin{equation}
        \mathcal{D}^\texttt{MGSM8KPoT}_\texttt{parallel} = \{(\vb*{Q}_i^l, \vb*{R}_i^l)|l\in L_\text{all}\}_{i=1}^N.
        \label{eq:mgsm8kpot-parallel}
    \end{equation}    
Note that in this case, the superscript $l$ in $\vb*{Q}_i^l$ and $\vb*{R}_i^l$ denotes the fact that both question and inline comments are in the same language. 

\vspace{2mm}
\noindent
\underline{\emph{Multi-nc}} --- Similar to $\mathcal{D}^\texttt{GSM8KPoT}_\texttt{nc}$, we also include a no-comment variant for this setup. 
    \begin{equation}
        \mathcal{D}^\texttt{MGSM8KPoT}_\texttt{nc} = \{(\vb*{Q}_i^l, \vb*{R}_i^\text{nc})|l\in L_\text{all}\}_{i=1}^N.
        \label{eq:mgsm8kpot-nc}
    \end{equation}

%This structured comparison enables an empirical evaluation of different translation strategies for multilingual PoT adaptation.


% -----To study the potential of PoT in multilingual setup, where the machine translation (MT), denotes as $\mathcal{T}$, is accessible to align the programming capability into target languages.
% We employ several strategies to incorporate the MT to translate the sample from the source language ($\mathcal{L}_\text{src}$), specifically English, into the target languages.

% \begin{compactenum}[(1)]
%     \item \textit{MathOctopus-PoT Cross}: ...
%    \item \textit{MathOctopus-PoT Parallel}: ...
%\end{compactenum}
% Formally, each sample $(\vb*{Q}_i^\text{en}, \vb*{A}_i^\text{en})$ is then translated into a set of samples $\{(\vb*{Q}_i^l, \vb*{A}_i^l)| l\in \mathcal{L}_\text{tgt} \}$.

% \dummy{\lipsum[1]}



\subsection{Code Quality Analysis (P2)} \label{subsec:code_analysis}

% \textcolor{blue}{do we care about other aspects? // justify why we are interested in functional correctness}

% motivate the program generation and execution
% After addressing the numerical problem within a multilingual context using PoT, the task is decomposed into two essential components: multi-step reasoning facilitated through program generation and execution performed via a Python interpreter for numerical computations.
%
% While the interpreter ensures precision in executing complex operations and arithmetic calculations, the primary challenge for models lies in generating programs that are both syntactically and logically sound.
%
% This approach differs from the step-by-step correctness required in CoT, where errors can occur not only in the logical reasoning process but also in arithmetic computations.
After addressing the multilingual problem with PoT, the task is split into two parts: multi-step reasoning via code generation and execution via a Python interpreter for numerical computations.
% While the interpreter ensures arithmetic accuracy, the challenge lies in generating syntactically and logically correct programs, which risks both logical and arithmetic errors -- unlike CoT.
While the interpreter ensures arithmetic accuracy, the challenge lies in generating syntactically and logically correct programs free of errors.


% in-depth investigation in parallel to CoT quality
% To this end, we conceptualize code quality as the intermediate correctness of PoT, evaluated using the \texttt{ICE-Score} \cite{ice-score}. This metric comprises two key components: usefulness, which measures the extent to which the generated code effectively addresses the user query,
% \textcolor{red}{and functional correctness, which is assessed through unit tests and reference code}
% and functional correctness, which is evaluated through intermediate-step validation of code snippets with the support of an Oracle LLM.
% 
% In this study, our primary focus is on the functional correctness aspect to analyze the intermediate validity of program generation.
%
% The \texttt{ICE-Score} is a numerical scale ranging from 0 to 4, where a score of 0 indicates that the code snippet is both incorrect and lacks substantive content, while a score of 4 signifies complete correctness.
We assess code quality using the \texttt{ICE-Score} \cite{ice-score}, which measures usefulness (how well the code addresses the query) and functional correctness (evaluated through intermediate validation with an Oracle LLM). Our focus is functional correctness, rating program validity from 0 (incorrect/incomplete) to 4 (fully correct).

% system wise + sample wise analysis
% Accordingly, we utilize this metric to compare the intermediate validity of candidate code generation with the correctness of the final answer produced by the trained models at two levels:
%
We use \texttt{ICE-Score} to assess whether improved alignment strategies enhance both accuracy and code quality. 
Furthermore, to compare code quality with final answer accuracy, we conduct two analyses:
% Furthermore, to explore the relationship between code quality and the accuracy of the final answer, we conduct two separate analyses.
% \textcolor{blue}{shorten 2 levels}
\begin{inparaenum}[(i)]
    \item \emph{System level}: Spearman correlation assesses whether higher-quality code improves overall model performance.
    \item \emph{Sample level}: AUC and t-test assess whether code validity can determine answer correctness.
\end{inparaenum}


% exploitation using soft self-consistency
% Furthermore, we investigate the potential of leveraging the \texttt{ICE-Score} to enhance model inference in test-time scaling scenarios.
% Based on the strong baseline approach 
% Building upon the strong baseline approach of Self-Consistency (\texttt{SC}) \cite{wang2023selfconsistency}, which generates multiple reasoning candidates and applies majority voting, we can substantially improve the performance of trained models on complex tasks. This method can be regarded as a form of hard voting.
% soft-sc
% We presume that by incorporating the code quality score alongside multiple PoT candidates, we can adopt the Soft Self-Consistency (\texttt{Soft-SC}) method \cite{soft-sc}, where each candidate answer is ranked by the average of interesting metric, for example, log-likelihood, etc.
% This approach extends the hard voting mechanism to promote a soft voting strategy, potentially leading to further performance enhancements.
% Soft Self-Consistency (\texttt{Soft-SC}) \cite{soft-sc} is introduced by integrating the code quality score with multiple PoT candidates. Specifically, each candidate answer is ranked based on the average of relevant metrics, with the \texttt{ICE-Score} serving as the primary criterion. This approach extends the conventional hard voting mechanism to a soft voting strategy, potentially leading to further performance improvements.
% 
\textbf{Test-time Scaling.} We investigate the use of \texttt{ICE-Score} to enhance model inference in test-time scaling. Building on Self-Consistency (\texttt{SC}) \cite{wang2023selfconsistency}, which generates multiple reasoning candidates and applies majority voting (hard voting), we extend this approach with Soft Self-Consistency (\texttt{Soft-SC}) \cite{soft-sc}.
\texttt{Soft-SC} refines this process by averaging the \texttt{ICE-Score} for each final answer candidate, ranking responses by overall code quality. This shift from hard to soft voting may improve performance.

\subsection{Discussions}



The six datasets enable us to examine how language alignment and inline comments impact cross-lingual and multilingual PoT reasoning. 
%
Inline comments act as alignment anchors between questions and reasoning steps expressed in a programming language. 
%
However, they can hinder cross-lingual generalization to unseen languages. 
%
In this respect, we aim to understand (i) how multilingual data availability influences PoT’s ability to generate accurate reasoning steps and (ii) how inline comments affect performance across language setups.
% 

Code quality analysis provides an intermediate observation linking these decisions to the accuracy of the final answer. 
%
By examining both aspects, we establish a structured understanding of how multilingual data and inference-time strategies interact to improve PoT performance, laying the groundwork for our experimental validation in Section~\ref{section:code-analysis-results}.

















\begin{table*}[ht]
\tiny
  \centering
  \resizebox{\textwidth}{!}{
  % \begin{tabular}{l|l|llllll|lll|l}
  \begin{tabular}{l|llllllllll|l}
    \hline
    % \multicolumn{1}{l|}{} & \multicolumn{1}{l|}{} &  \multicolumn{6}{c|}{\textbf{HRL}} & \multicolumn{3}{c|}{\textbf{URL}} & \multicolumn{1}{c}{\textbf{Avg.}} \\
    \textbf{Method} & en & de & fr & es & ru & zh & ja & th & sw & bn & All \\
    \hline
    \multicolumn{1}{l|}{\underline{Llama2-7B}} &     %     % \multicolumn{1}{c|}{} & \multicolumn{6}{c|}{} & \multicolumn{3}{c|}{} & \multicolumn{1}{c}{} \\
    \multicolumn{1}{c}{} & \multicolumn{6}{c}{} & \multicolumn{3}{c|}{} & \multicolumn{1}{c}{} \\    
    CoT & 43.6 & 32.4 & 30.4 & 30.4 & 26.4 & 25.2 & 15.2 & 4.8 & 2.0 & 5.6 & 21.6 \\
    PoT & \textbf{58.0} & \textbf{40.4} & \textbf{40.4} & \textbf{43.6} & \textbf{37.1} & \textbf{38.4} & \textbf{32.7} & \textbf{7.6} & \textbf{5.6} & \textbf{12.0} & \textbf{31.6} \\
    \hline
    \multicolumn{1}{l|}{\underline{CodeLlama 7B}} & 
    % \multicolumn{1}{c|}{} & \multicolumn{6}{c|}{} & \multicolumn{3}{c|}{} & \multicolumn{1}{c}{} \\
    \multicolumn{1}{c}{} & \multicolumn{6}{c}{} & \multicolumn{3}{c|}{} & \multicolumn{1}{c}{} \\  
    CoT & 43.2 & 33.2 & 32.8 & 39.6 & 26.8 & 27.2 & 18.8 & 16.4 & 3.2 & 9.2 & 25.0 \\
    PoT & \textbf{58.8} & \textbf{48.4} & \textbf{51.6} & \textbf{53.6} & \textbf{49.8} & \textbf{41.6} & \textbf{39.6} & \textbf{26.8} & \textbf{4.4} & \textbf{11.2} & \textbf{38.6} \\
    \hline    
    \multicolumn{1}{l|}{\underline{Llama2-13B}} &     %     % \multicolumn{1}{c|}{} & \multicolumn{6}{c|}{} & \multicolumn{3}{c|}{} & \multicolumn{1}{c}{} \\
    \multicolumn{1}{c}{} & \multicolumn{6}{c}{} & \multicolumn{3}{c|}{} & \multicolumn{1}{c}{} \\     
    CoT & 47.4 & 39.2 & 37.6 & 41.2 & 38.0 & 35.2 & 18.8 & 7.2 & \textbf{7.4} & 6.8 & 27.9 \\
    PoT & \textbf{64.0} & \textbf{52.4} & \textbf{54.4} & \textbf{55.6} & \textbf{51.2} & \textbf{44.0} & \textbf{40.0} & \textbf{13.9} & 7.2 & \textbf{13.6} & \textbf{39.6} \\
    \hline        
    \multicolumn{1}{l|}{\underline{Llama3-8B}} &    % \multicolumn{1}{c|}{} & \multicolumn{6}{c|}{} & \multicolumn{3}{c|}{} & \multicolumn{1}{c}{} \\
    \multicolumn{1}{c}{} & \multicolumn{6}{c}{} & \multicolumn{3}{c|}{} & \multicolumn{1}{c}{} \\ 
    CoT & 62.8 & 51.2 & 52.8 & 54.8 & 45.2 & 40.0 & 33.6 & 39.6 & 28.0 & 39.6 & 44.8 \\
    PoT & \textbf{68.4} & \textbf{62.2} & \textbf{59.2} & \textbf{62.4} & \textbf{60.4} & \textbf{52.4} & \textbf{45.4} & \textbf{43.6} & \textbf{34.8} & \textbf{46.0} & \textbf{53.5} \\    
    % \hline\hline
    % \multicolumn{12}{c}{\textit{Oracle}} \\
    % \hline
    % \multicolumn{1}{l|}{\underline{Llama3.1-405B}} &    % \multicolumn{1}{c|}{} & \multicolumn{6}{c|}{} & \multicolumn{3}{c|}{} & \multicolumn{1}{c}{} \\
    % \multicolumn{1}{c}{} & \multicolumn{6}{c}{} & \multicolumn{3}{c|}{} & \multicolumn{1}{c}{} \\ 
    % Few-Shot CoT & xx.x & xx.x & xx.x & xx.x & xx.x & xx.x & xx.x & xx.x & xx.x & xx.x & xx.x \\
    % Few-Shot PoT & 92.2	&89.1	&85.9	&92.1	&92.6	&87.9	&86.4	&64.8	&85.9	&59.6 & 83.6 \\
    % \hline
    % \multicolumn{1}{l|}{\underline{o1-mini [OpenAI]}}& \multicolumn{1}{c|}{} & \multicolumn{6}{c|}{} & \multicolumn{3}{c|}{} & \multicolumn{3}{c}{} \\
    % CoT & xx.x & xx.x & xx.x & xx.x & xx.x & xx.x & xx.x & xx.x & xx.x & xx.x & xx.x \\
    % PoT & xx.x & xx.x & xx.x & xx.x & xx.x & xx.x & xx.x & xx.x & xx.x & xx.x & xx.x \\
    \hline
  \end{tabular}
  }
  \caption{
  Accuracy (\%) on MGSM in \textbf{cross-lingual setting}.
  % \textcolor{blue}{separte result table page of cross-lingual and multilingual}
  }
  \label{tab:main-cross}

\end{table*}



\section{Experimental Setup}


\textbf{Base LLMs.} We conduct experiments with various base LLMs, using Llama2-7B \cite{Llama2} as the foundation for the following variants:
\begin{compactenum}[i)]
\item \emph{Code-specific variant}: CodeLlama-7B \cite{codeLlama}, optimized for code and programming-related tasks.
\item \emph{Size variant}: Llama2-13B \cite{Llama2}, a larger-scale version of Llama2.
% \item \emph{Version variant}: Llama3.1 8B \cite{Llama3}, a more recent version with better multilingual capabilities.
\item \emph{Version variant}: Llama3 8B \cite{Llama3}, a more recent iteration with enhanced multilingual capabilities. 
\end{compactenum}

\noindent\textbf{Oracle LLM.} To ensure reproducibility, we employ Llama3.1-405B Instruct \cite{Llama3} as our Oracle model for generating the PoT dataset and assessing the quality of the code.

% \textbf{Machine Translation} We employ the distilled version\footnote{\scriptsize \url{https://huggingface.co/facebook/nllb-200-distilled-600M}} of NLLB-200 \cite{nllb} for translating inline comments.



\vspace{2mm}
\noindent \textbf{Evaluation.}
We evaluate model performance by measuring accuracy on the MGSM \cite{mgsm} dataset in a zero-shot setting using greedy decoding.
%
The study includes the following languages: English (en), German (de), French (fr), Spanish (es), Russian (ru), Chinese (zh), Japanese (ja), Thai (th), Swahili (sw), and Bengali (bn).
%
For CoT evaluation, numerical outputs are extracted via regular expressions and compared to labels, following \citet{mathoctopus}.
For PoT evaluation, generated programs are executed in a Python interpreter, with outputs compared to labels for accuracy.





\vspace{2mm}
\noindent \textbf{Measures}: As outlined in Table~\ref{tab:compare-metods}, in cross-lingual setting, we finetune each LLM independently on GSM8K \cite{cobbe2021gsm8k} and GSM8KPoT, using both \(\mathcal{D}^\texttt{GSM8KPoT}_\texttt{en}\) and \(\mathcal{D}^\texttt{GSM8KPoT}_\texttt{nc}\) variants.
%
For multilingual CoT, we finetune each LLM separately on MGSM8K Instruct Parallel and Cross \cite{mathoctopus}.
%
For multilingual PoT, we utilize the generated answers from GSM8KPoT and map the questions for each language in MGSM8K Instruct to create MGSM8KPoT.
%
To study the effects of inline comments, we create versions of GSM8K and MGSM8KPoT without inline comments by removing them from the original datasets.
%
Additionally, we generate a variation of MGSM8KPoT by applying machine translation.
%
We utilize nllb-200-distilled-600M~\cite{nllb} for translating inline comments, ensuring coverage across all languages in this study.



% \textcolor{blue}{(Jab) shorten, highlight only 3 + move full to appendix // but keep Table 2}







% \section{GSM8KPoT}



% \begin{table*}[!t]
% \small
%   \centering
%   \resizebox{\textwidth}{!}{
%   % \begin{tabular}{l|l|llllll|lll|l}
%   \begin{tabular}{l|llllllllll|l}
%     \hline
%     % \multicolumn{1}{l|}{} & \multicolumn{1}{l|}{} &  \multicolumn{6}{c|}{\textbf{HRL}} & \multicolumn{3}{c|}{\textbf{URL}} & \multicolumn{1}{c}{\textbf{Avg.}} \\
%     \textbf{Method} & en & de & fr & es & ru & zh & ja & th & sw & bn & All \\
%     \hline
%     \multicolumn{1}{l|}{\underline{Llama2-7B}} &     %     % \multicolumn{1}{c|}{} & \multicolumn{6}{c|}{} & \multicolumn{3}{c|}{} & \multicolumn{1}{c}{} \\
%     \multicolumn{1}{c}{} & \multicolumn{6}{c}{} & \multicolumn{3}{c|}{} & \multicolumn{1}{c}{} \\    
%     CoT & 43.6 & 32.4 & 30.4 & 30.4 & 26.4 & 25.2 & 15.2 & 4.8 & 2.0 & 5.6 & 21.6 \\
%     PoT & 58.0 & 40.4 & 40.4 & 43.6 & 37.1 & 38.4 & 32.7 & 7.6 & 5.6 & 12.0 & 31.6 \\
%     \hline
%     \multicolumn{1}{l|}{\underline{CodeLlama 7B}} & 
%     % \multicolumn{1}{c|}{} & \multicolumn{6}{c|}{} & \multicolumn{3}{c|}{} & \multicolumn{1}{c}{} \\
%     \multicolumn{1}{c}{} & \multicolumn{6}{c}{} & \multicolumn{3}{c|}{} & \multicolumn{1}{c}{} \\  
%     CoT & 43.2 & 33.2 & 32.8 & 39.6 & 26.8 & 27.2 & 18.8 & 16.4 & 3.2 & 9.2 & 25.0 \\
%     PoT & 58.8 & 48.4 & 51.6 & 53.6 & 49.8 & 41.6 & 39.6 & 26.8 & 4.4 & 11.2 & 38.6 \\
%     \hline    
%     \multicolumn{1}{l|}{\underline{Llama2-13B}} &     %     % \multicolumn{1}{c|}{} & \multicolumn{6}{c|}{} & \multicolumn{3}{c|}{} & \multicolumn{1}{c}{} \\
%     \multicolumn{1}{c}{} & \multicolumn{6}{c}{} & \multicolumn{3}{c|}{} & \multicolumn{1}{c}{} \\     
%     CoT & 47.4 & 39.2 & 37.6 & 41.2 & 38.0 & 35.2 & 18.8 & 7.2 & 7.4 & 6.8 & 27.9 \\
%     PoT & 64.0 & 52.4 & 54.4 & 55.6 & 51.2 & 44.0 & 40.0 & 13.9 & 7.2 & 13.6 & 39.6 \\
%     \hline        
%     \multicolumn{1}{l|}{\underline{Llama3-8B}} &    % \multicolumn{1}{c|}{} & \multicolumn{6}{c|}{} & \multicolumn{3}{c|}{} & \multicolumn{1}{c}{} \\
%     \multicolumn{1}{c}{} & \multicolumn{6}{c}{} & \multicolumn{3}{c|}{} & \multicolumn{1}{c}{} \\ 
%     CoT & 62.8 & 51.2 & 52.8 & 54.8 & 45.2 & 40.0 & 33.6 & 39.6 & 28.0 & 39.6 & 44.8 \\
%     PoT & 68.4 & 62.2 & 59.2 & 62.4 & 60.4 & 52.4 & 45.4 & 43.6 & 34.8 & 46.0 & 53.5 \\    
%     \hline\hline
%     \multicolumn{12}{c}{\textit{Oracle}} \\
%     \hline
%     \multicolumn{1}{l|}{\underline{Llama 405B}} &    % \multicolumn{1}{c|}{} & \multicolumn{6}{c|}{} & \multicolumn{3}{c|}{} & \multicolumn{1}{c}{} \\
%     \multicolumn{1}{c}{} & \multicolumn{6}{c}{} & \multicolumn{3}{c|}{} & \multicolumn{1}{c}{} \\ 
%     CoT & xx.x & xx.x & xx.x & xx.x & xx.x & xx.x & xx.x & xx.x & xx.x & xx.x & xx.x \\
%     PoT & xx.x & xx.x & xx.x & xx.x & xx.x & xx.x & xx.x & xx.x & xx.x & xx.x & xx.x \\
%     \hline
%     % \multicolumn{1}{l|}{\underline{o1-mini [OpenAI]}}& \multicolumn{1}{c|}{} & \multicolumn{6}{c|}{} & \multicolumn{3}{c|}{} & \multicolumn{3}{c}{} \\
%     % CoT & xx.x & xx.x & xx.x & xx.x & xx.x & xx.x & xx.x & xx.x & xx.x & xx.x & xx.x \\
%     % PoT & xx.x & xx.x & xx.x & xx.x & xx.x & xx.x & xx.x & xx.x & xx.x & xx.x & xx.x \\
%     \hline
%   \end{tabular}
%   }
%   \caption{
%   Accuracy (\%) on MGSM in the cross-lingual setting.
%   }
%   \label{tab:main-cross}
% \end{table*}







\section{Experimental Results}




% \textcolor{blue}{mention QR and RA to section}

% \subsection{Results: Program-of-Thought Fine-tuning }
\subsection{Impact of Q-R Alignment Fine-tuning}
\label{sec:QR-Alignment}
\subsubsection{Cross-lingual Setting}
% intro
% trend in general PoT vs CoT
The experimental results presented in Table \ref{tab:main-cross} indicate that PoT consistently outperforms CoT across all languages and model classes, achieving superior results in 39 out of 40 cases. The only exception is Swahili in the Llama2-13B model, where PoT reached an accuracy of 7.2\%, compared to CoT's 7.4\%, showing only a slight difference.
%

When comparing models of the same size, CodeLlama-7B consistently outperforms Llama2-7B in most languages. 
%
The improvements are notable in non-English languages such as German (+8.0), French (+11.2), and Thai (+19.2), suggesting that the incorporation of code data during pretraining improves structured reasoning even in cross-lingual settings.
%
Scaling up to Llama2-13B leads to further improvements over both Llama2-7B and CodeLlama-7B. 
%
While model size remains an important factor in boosting overall accuracy, the strong performance of CodeLlama-7B relative to Llama2-7B indicates that increased code data during pretraining \cite{codeLlama} can enhance reasoning ability.
%
For models with enhanced multilingual capabilities, such as Llama3-8B, where the performance gap between languages is narrower, the results suggest that PoT remains more effective in cross-lingual settings, achieving superior accuracy across unseen languages.
%
%











In Table \ref{tab:pot-inline-comment}, we compare performance when fine-tuning between \(\mathcal{D}^\texttt{GSM8KPoT}_\texttt{en}\) and \(\mathcal{D}^\texttt{GSM8KPoT}_\texttt{nc}\).
%
Overall, training without comments tends to improve non-English accuracy across Llama2 models for both 7B and 13B variants, where omitting comments reduces English accuracy slightly but yields larger gains in non-English languages, like German and Bengali, boosting the overall score.
%
\begin{table}[htbp]
\tiny
  \centering
  \renewcommand{\arraystretch}{1.2}

  \resizebox{\columnwidth}{!}{
  \begin{tabular}{l|c|cc|c}
    \hline
    \textbf{Method} & en & de & bn & ALL \\
    \hline\hline
    \multicolumn{1}{l|}{\underline{Llama2-7B}}& \multicolumn{1}{c|}{} & \multicolumn{2}{c|}{} & \multicolumn{1}{c}{} \\
    With Comments & \textbf{58.3} & 37.9 & 9.9 & 30.0  \\
    Without Comments & 58.0 & \textbf{40.4} & \textbf{12.0} & \textbf{31.6}  \\   
    \hline     
    \multicolumn{1}{l|}{\underline{CodeLlama-7B}}& \multicolumn{1}{c|}{} & \multicolumn{2}{c|}{} & \multicolumn{1}{c}{} \\
    With Comments & \textbf{61.4}  & 45.2 & \textbf{15.6} & 36.6  \\
    Without Comments & 58.8  & \textbf{48.4} & 11.2 & \textbf{38.6}  \\
    \hline    
    \multicolumn{1}{l|}{\underline{Llama2-13B}}& \multicolumn{1}{c|}{} & \multicolumn{2}{c|}{} & \multicolumn{1}{c}{} \\
    With Comments & \textbf{67.3}  & 48.4 & 13.2 & 37.4  \\
    Without Comments & 64.0  & \textbf{52.4} & \textbf{13.6} & \textbf{39.6}  \\
    \hline
    \multicolumn{1}{l|}{\underline{Llama3-8B}}& \multicolumn{1}{c|}{} & \multicolumn{2}{c|}{} & \multicolumn{1}{c}{} \\
    With Comments & 46.4  & 48.2 & 37.5 & 40.6  \\
    Without Comments & \textbf{68.4}  & \textbf{62.2} & \textbf{46.0} & \textbf{53.5}  \\      
    \hline            
  \end{tabular}
  }
  \caption{The impact of code comments on accuracy across different models in cross-lingual setup. The ALL score is from Appendix~\ref{ap:full-main}.}
  \label{tab:pot-inline-comment}
\end{table}

Interestingly, CodeLlama-7B shows mixed results: including comments helps in English and Bengali, whereas excluding comments improves German and also leads to a higher overall score.
%
This may reflect the specialized training corpus for CodeLlama, which emphasizes code tokens and might interact differently with inline explanations.  

%
Finally, Llama3-8B shows the largest swing: removing comments substantially boosts performance for all languages (including English), suggesting that inline explanations can sometimes distract or misalign the Q–R.
%
Taken together, these findings indicate that, for most models, \(\mathcal{D}^\texttt{GSM8KPoT}_\texttt{nc}\) provides better cross-lingual generalization and more robust Q–R alignment.


\begin{table*}[htbp]
\tiny
\centering
\resizebox{\textwidth}{!}{
\begin{tabular}{l|llllllllll|l}
\hline
\textbf{Method} & en & de & fr & es & ru & zh & ja & th & sw & bn & All \\
\hline
\multicolumn{1}{l|}{\underline{Llama2-7B}} & & & & & & & & & & & \\
CoT Cross & 45.2 & 38.4 & 36.8 & 40.0 & 33.2 & \textbf{33.6} & 23.6 & 16.8 & \textbf{18.8} & 17.2 & 30.4 \\
PoT Cross Comment & \textbf{54.8} & \textbf{47.2} & \textbf{51.2} & \textbf{46.2} & \textbf{42.8} & 33.2 & \textbf{34.8} & \textbf{20.0} & 17.6 & \textbf{18.0} & \textbf{36.6} \\
\hdashline
CoT Parallel & 48.8 & 42.4 & 44.0 & 42.4 & 38.0 & 42.4 & 31.6 & 33.6 & 34.4 & 27.6 & 38.5 \\
PoT Parallel & \textbf{56.0} & \textbf{47.2} & \textbf{46.4} & \textbf{54.0} & \textbf{49.6} & \textbf{44.4} & \textbf{40.0} & \textbf{40.4} & \textbf{37.6} & \textbf{30.8} & \textbf{44.6} \\
\hline
\multicolumn{1}{l|}{\underline{CodeLlama2-7B}} & & & & & & & & & & & \\
CoT Cross & 47.6 & 38.8 & 33.2 & 38.8 & 35.2 & 31.6 & 28.8 & 23.6 & 17.2 & 20.4 & 31.5 \\
PoT Cross Comment & \textbf{58.0} & \textbf{47.2} & \textbf{51.4} & \textbf{52.4} & \textbf{48.0} & \textbf{44.2} & \textbf{38.0} & \textbf{28.8} & \textbf{20.4} & \textbf{22.4} & \textbf{41.1} \\
\hdashline
CoT Parallel & 46.0 & 40.0 & 38.8 & 44.0 & 43.2 & 41.2 & 35.6 & 41.6 & 30.8 & 32.0 & 39.3 \\
PoT Parallel & \textbf{61.9} & \textbf{52.8} & \textbf{54.4} & \textbf{52.4} & \textbf{53.6} & \textbf{50.4} & \textbf{44.8} & \textbf{44.8} & \textbf{39.6} & \textbf{35.6} & \textbf{49.0} \\
\hline
\multicolumn{1}{l|}{\underline{Llama2-13B}} & & & & & & & & & & & \\
CoT Cross & 58.4 & 50.4 & 46.4 & 49.6 & 43.6 & \textbf{43.2} & 33.6 & \textbf{25.6} & \textbf{23.6} & \textbf{24.4} & 39.9 \\
PoT Cross Comment & \textbf{62.0} & \textbf{53.6} & \textbf{52.4} & \textbf{54.8} & \textbf{50.0} & 42.0 & \textbf{39.2} & 21.6 & 23.2 & 23.2 & \textbf{42.2} \\
\hdashline
CoT Parallel & 60.8 & 53.6 & 52.0 & 54.4 & 52.8 & 53.6 & 45.2 & 43.6 & 41.2 & 38.0 & 49.5 \\
PoT Parallel & \textbf{63.5} & \textbf{56.4} & \textbf{59.2} & \textbf{59.2} & \textbf{55.2} & \textbf{54.0} & \textbf{51.6} & \textbf{50.0} & \textbf{52.8} & \textbf{44.4} & \textbf{54.6} \\
\hline
\multicolumn{1}{l|}{\underline{Llama3-8B}} & & & & & & & & & & & \\
CoT Cross & 69.2 & 58.0 & 54.8 & 58.0 & 57.2 & 50.0 & 44.4 & 40.4 & 40.4 & 42.0 & 51.4 \\
PoT Cross Comment & \textbf{72.8} & \textbf{62.4} & \textbf{66.4} & \textbf{67.2} & \textbf{63.6} & \textbf{52.0} & \textbf{49.6} & \textbf{52.0} & \textbf{46.2} & \textbf{51.2} & \textbf{58.3} \\
\hdashline
CoT Parallel & 66.8 & 53.6 & 57.2 & 60.8 & 62.4 & 60.0 & 50.8 & 57.6 & 53.6 & 54.8 & 57.8 \\
PoT Parallel & \textbf{76.5} & \textbf{64.4} & \textbf{63.2} & \textbf{66.4} & \textbf{64.0} & \textbf{63.2} & \textbf{56.4} & \textbf{57.6} & \textbf{59.6} & \textbf{55.2} & \textbf{62.6} \\
\hline
\end{tabular}
}
\caption{
Accuracy (\%) on MGSM in \textbf{multilingual setup}. 
% \textcolor{blue}{idea: if the results are already correct, move Llama3-8b to appendix and discuss in main then raise the issue, could be data leakage.}
%\textcolor{red}{If we fine-tune, we should consider the variance in some models (one or two Llama for 3 random seeds) in the appendix.}
}
\label{tab:main-multi}
\vspace{-3mm}
\end{table*}











\subsubsection{Multilingual Setting}
%
Table \ref{tab:main-multi} shows that PoT continues to outperform CoT in multilingual settings across all languages and model variants.
%
A few exceptions appear in some languages within the Llama 2 family trained on $\mathcal{D}^\texttt{MGSM8KPoT}_\texttt{cross-comment}$, yet overall results confirm that PoT provides consistent advantages over CoT.

As observed in the cross-lingual experiments, CodeLlama2-7B maintains its advantage over Llama2-7B across all languages in the multilingual setting.
%
This performance gap is particularly pronounced in French (+8.0), Chinese (+6.0), and English (+5.9), further suggesting that increasing code data during pretraining yields stronger reasoning capabilities.
%
Scaling to larger models continues to deliver gains, with Llama2-13B showing consistent improvements over both 7B variants.
%
However, the most dramatic improvements come from Llama3-8B, which achieves substantially higher accuracy across all languages, reaching 76.5\% in English while maintaining strong performance even in non-English languages like Thai (57.6\%) and Bengali (55.2\%).
% 
The stronger gains highlights the benefits of explicit multilingual training over multilingual transfer, emphasizing the role of scaling and adaptation in optimizing reasoning across languages.
%
% Interestingly, Llama3‑8B deviates from this trend by performing better with CoT than with PoT.
% %
% Upon further investigation of this unexpected behavior, we discovered that this performance gap stems primarily from compilation errors in the generated programs rather than logical reasoning failures.
% %
% For a comprehensive analysis of these findings, please refer to Appendix \ref{ap:Llama3}.











%
Finally, we investigate the most effective way to align Q and R in a multilingual context.
%
As shown in Table \ref{tab:multilingual-ablation}, translating inline comments into the target language consistently yields superior performance across all model variants. 
%
We hypothesize that this improvement comes from the enhanced semantic alignment between code and natural language when comments are presented in the target language during training.
%
In summary, these findings indicate that $\mathcal{D}^\texttt{MGSM8KPoT}_\texttt{parallel}$ provides the optimal Q-R alignment for multilingual settings.
%
\begin{table}[htbp]
\small
  \centering
  \resizebox{\columnwidth}{!}{
  \begin{tabular}{l|c|cc|c}
    \hline
    \textbf{Method} & en & de & bn & ALL \\
    \hline\hline
    \multicolumn{1}{l|}{\underline{Llama2-7B}}& \multicolumn{1}{c|}{} & \multicolumn{2}{c|}{} & \multicolumn{1}{c}{} \\
    PoT Cross Comment & \underline{54.8} & \textbf{47.2} & 18.0 & 36.6  \\
    PoT Cross Question & 46.0 & 37.6 & 28.8 & 37.7  \\ 
    PoT Parallel & \textbf{56.0} & \textbf{47.2} & \textbf{30.8} & \textbf{44.6} \\
    PoT No Comment & 53.6 & 41.6 & \underline{29.2} & \underline{40.6}  \\ 
    \hline
    \multicolumn{1}{l|}{\underline{CodeLlama2-7B}}& \multicolumn{1}{c|}{} & \multicolumn{2}{c|}{} & \multicolumn{1}{c}{} \\
    PoT Cross Comment & \underline{58.0} & 47.2 & 22.4 & 41.1  \\
    PoT Cross Question & 48.0 & 42.8 & 28.8 &  40.5 \\ 
    PoT Parallel & \textbf{61.9} & \textbf{52.8} & \textbf{35.6} & \textbf{49.0}  \\
    PoT No Comment & 56.8 & \underline{47.6} & \underline{35.2} & \underline{45.6}  \\ 
    \hline    
    \multicolumn{1}{l|}{\underline{Llama2-13B}}& \multicolumn{1}{c|}{} & \multicolumn{2}{c|}{} & \multicolumn{1}{c}{} \\
    PoT Cross Comment & \underline{62.0} & \underline{53.6} & 23.2 & 42.2  \\
    PoT Cross Question & 53.0 & 47.6 & \underline{35.9} & 45.1  \\ 
    PoT Parallel & \textbf{63.5} & \textbf{56.4} & \textbf{44.4} & \textbf{54.6}  \\
    PoT No Comment & 58.4 & 51.6 & \underline{35.2} & \underline{46.4}  \\ 
    \hline    
    \multicolumn{1}{l|}{\underline{Llama3-8B}}& \multicolumn{1}{c|}{} & \multicolumn{2}{c|}{} & \multicolumn{1}{c}{} \\
    PoT Cross Comment & \underline{72.8} & \underline{62.4} & \underline{51.2}	& \underline{58.3}  \\
    PoT Cross Question & 37.2 & 30.3 & 30.4 & 31.6  \\ 
    PoT Parallel & \textbf{76.5} & \textbf{64.4} & \textbf{55.2} & \textbf{62.6}  \\
    PoT No Comment &  65.2 & 60.0 & 48.4 & 56.5  \\ 
    \hline    
  \end{tabular}
  }
  \caption{
  The impact of various fine-tuning strategies is examined, where PoT Cross includes either comment-only or question-only translation. In contrast, the Parallel approach involves either the exclusion of comments or the inclusion of translated comments.
  % The complete table is available in Table~\ref{tab:}.
  }
  \label{tab:multilingual-ablation}
\end{table}

% % Intro
% In the multilingual setting, we fine-tune the models using data in both English and the target languages. This enables us to explore alignment strategies for improving reasoning capabilities across languages.
% %
% Similar to the cross-lingual setup, PoT consistently outperforms CoT across all languages and model variants in our multilingual setup, as presented in Table \ref{tab:main-multi}.
% %
% Moreover, the results indicate that when the model is exposed to explicit demonstrations of reasoning in target languages during fine-tuning, the performance gap across languages tends to diminish in most cases.

% Building on the findings of \citet{mathoctopus}, where CoT Cross and CoT Parallel were introduced, we conducted comparative experiments applying PoT with the same strategies. In this setup, the output answer translation serves as the inline comments. Our findings align with those of CoT, demonstrating that the parallel approach is generally more effective, as reflected in its higher accuracy.







% The ablation study aimed at identifying the most effective strategy for PoT alignment in a multilingual setup reveals that PoT Parallel is the most effective approach. This finding confirms that translating inline comments can further enhance performance across all languages, including English, as illustrated in Figure~\ref{tab:multilingual-ablation}.
% %
% The second-best performing approach is PoT Parallel No Comment, where the model is trained on translated questions without inline comments, in contrast to the cross-lingual setup.


% However, the PoT Cross Comment approach, which serves as a comparable alternative to CoT Cross, demonstrates the weakest performance among the four methods.
% %
% Finally, the PoT Cross Question strategy appears to perform slightly better than PoT Cross Comment, as it achieves higher overall accuracy in two out of three cases.








\subsection{R-A Relationship: Code Quality Analysis}
\label{section:code-analysis-results}


\textbf{Does Better Strategy Improve Code Quality?}
As discussed in Section~\ref{subsec:code_analysis}, we assess code quality across alignment strategies in cross-lingual and multilingual settings, focusing on Llama2-7B and CodeLlama-7B.
%
Table~\ref{tab:code-quality} shows that higher accuracy correlates with better code quality.
% 
Additionally, code quality in lower resource languages, like Bengali, is much lower than in English and German, which aligns with the accuracy trends.
%
This finding reflects the inherent challenges of generating code in low-resource languages, where model performance is typically more constrained.
%
\begin{table}[htbp]
 \tiny
  \centering
  \resizebox{\columnwidth}{!}{
  \begin{tabular}{l|c|cc|c}
    \hline
    \textbf{Method} & en & de & bn & ALL \\
    \hline\hline
    \multicolumn{5}{c}{\textit{Cross-lingual}} \\
    \hline    
    \multicolumn{1}{l|}{\underline{Llama2-7B}}& \multicolumn{1}{c|}{} & \multicolumn{2}{c|}{} & \multicolumn{1}{c}{} \\
    With Comments & \textbf{2.49} & \textbf{1.87} & 0.45 & 1.39  \\
    Without Comments & \textbf{2.49} & \textbf{1.87} & \textbf{0.49}	& \textbf{1.44}  \\ 
    \hline   
    \multicolumn{1}{l|}{\underline{CodeLlama2-7B}}& \multicolumn{1}{c|}{} & \multicolumn{2}{c|}{} & \multicolumn{1}{c}{} \\
    With Comments & \textbf{2.66} & 2.06 & \textbf{0.61}	&1.97  \\
    Without Comments & 2.55 & \textbf{2.13} & 0.54 & \textbf{2.02}  \\
    \hline\hline
    \multicolumn{5}{c}{\textit{Multilingual}} \\
    \hline
    \multicolumn{1}{l|}{\underline{Llama2-7B}}& \multicolumn{1}{c|}{} & \multicolumn{2}{c|}{} & \multicolumn{1}{c}{} \\
    PoT Cross Comment & \underline{2.56} & 2.41 & 1.26	&1.98  \\
    PoT Cross Question & 2.32 & 2.07 & 1.52	&2.03  \\
    PoT Parallel & \textbf{2.83} & \textbf{2.55} & \textbf{1.96} & \textbf{2.45}  \\
    PoT No Comment & 2.54 & \underline{2.16} & \underline{1.71}	& \underline{2.13}  \\
    
    \hline
    \multicolumn{1}{l|}{\underline{CodeLlama2-7B}}& \multicolumn{1}{c|}{} & \multicolumn{2}{c|}{} & \multicolumn{1}{c}{} \\
    PoT Cross Comment & \underline{2.84} & 2.40 & 1.34	& 2.15  \\
    PoT Cross Question & 2.45 & 2.23 & 1.54	&2.11  \\
    PoT Parallel & \textbf{2.88} & \textbf{2.68} & \textbf{2.04}	& \textbf{2.56}  \\
    PoT No Comment & 2.61 & \underline{2.41} & \underline{1.87} & \underline{2.28} \\
    % \hline\hline
    % \multicolumn{5}{c}{\textit{Oracle}} \\
    % \hline
    % \multicolumn{1}{l|}{\underline{Llama3.1-405B}}& \multicolumn{1}{c|}{} & \multicolumn{2}{c|}{} & \multicolumn{1}{c}{} \\
    % Few-Shot PoT & - & - & -	& -  \\
    \hline
  \end{tabular}
  }
  \caption{
  Code quality assessment with ICE-Score
  }
  \label{tab:code-quality}
\end{table}




% system level
% \begin{figure}[t]
% \centering
% \includegraphics[width=\columnwidth, trim={0 0 0 0.6cm}, clip]{latex/figures/corr-crosslingual-3.pdf}
%     \caption{
%     The relationship between code quality and answer accuracy in a cross-lingual setting exhibited a Spearman correlation of 0.91.
%     Each point corresponds to the accuracy and code quality for a given language, considering a specific system and model variation.
%     \textcolor{blue}{may put them next to each other, left and right with sharing axis + legend}
%     }
%     \label{fig:sys-level-assoc-cross}
% \end{figure}


% \begin{figure}[t]
% \centering
% \includegraphics[width=\columnwidth]{latex/figures/corr-multilingual-3.pdf}
%     \caption{
%     Similarly, in a multilingual setting, the observed trend between code quality and answer accuracy demonstrated a Spearman correlation of 0.76,
%     which is slightly lower than that observed in the cross-lingual scenario.
%     \textcolor{blue}{**make 2 and 3 subfigure and only one caption}
%     }
%     \label{fig:sys-level-assoc-multi}
% \end{figure}


\textbf{System Level Correlation.}  Figure~\ref{fig:sys-level-assoc} illustrates a strong system-level correlation between MGSM accuracy and code quality, as measured by \texttt{ICE-Score}.
%
Across all finetuning strategies, in both cross-lingual and multilingual, we observe a consistent trend where higher code quality positively correlates with improved accuracy.
%
This relationship is quantified by a Spearman rank correlation coefficient of 0.91 and 0.76 for cross-lingual and multilingual, respectively.
%
%
\begin{figure}[htbp]
    \centering
    \includegraphics[width=\columnwidth, trim={0 0 0 0.6cm}, clip]{latex/figures/corr-shared-y-axis-system-legend-5.pdf}
    \caption{The relationship between code quality and answer accuracy in cross-lingual and multilingual settings. Each point represents a given language, considering a specific system and model combination.
    % Both settings present a strong correlation.
    }
    \label{fig:sys-level-assoc}
\end{figure}


Notably, this correlation persists across different model architectures and code generation conditions, reinforcing the importance of alignment strategies in enhancing both code quality and accuracy. 
%
These insights highlight the broader impact of alignment and resource availability on code generation, supporting the necessity of assessing the quality of intermediate outputs.



% \begin{figure}[htbp]
%     \centering
%     \includegraphics[width=\columnwidth, trim={0 0 0 0.6cm}, clip]{latex/figures/sample-level-corr-combine-v0.pdf}
%     % \centering
%     % \begin{subfigure}{0.23\textwidth}
%     %     \centering
%     %     \includegraphics[width=\textwidth]{latex/figures/samplecorr-cross-2.pdf}
%     %     \caption{Cross-lingual setup using LLaMa2 7B.}
%     %     \label{fig:sam-level-assoc-cross}
%     % \end{subfigure}
%     % \hfill
%     % \begin{subfigure}{0.23\textwidth}
%     %     \centering
%     %     \includegraphics[width=\textwidth]{latex/figures/samplecorr-multi-2.pdf}
%     %     \caption{Multilingual setup using LLaMa2 7B.}
%     %     \label{fig:sam-level-assoc-multi}
%     % \end{subfigure}
%     \caption{
%     Distribution of code quality across answer candidates in both setups using LLaMa2 7B.
%     }
%     \label{fig:samplecorr-combined}
% \end{figure}






% \begin{figure}[htbp]
% \centering
% \includegraphics[width=\columnwidth]{latex/figures/samplecorr-cross-2.pdf}
%     \caption{
%     Distribution of code quality across answer candidates in the cross-lingual setup using LLaMa2 7B. \textcolor{blue}{two side histograms}
%     % \textcolor{blue}{remove log-scale in y // normalize red + green // wording - correct & incorrect // remove fill-in}
%     }
%     \label{fig:sam-level-assoc-cross}
% \end{figure}

% \begin{figure}[htbp]
% \centering
% \includegraphics[width=\columnwidth]{latex/figures/samplecorr-multi-2.pdf}
%     \caption{
%     Distribution of code quality across answer candidates in the multilingual setup using LLaMa2 7B.
%     }
%     \label{fig:sam-level-assoc-multi}
% \end{figure}




\textbf{Sample Level Association.}
Beyond system-level trends, we examined whether code quality can determine the correctness of individual solutions.
%
This relationship is demonstrated in Table \ref{tab:ice-score-distribution}, where the percentage distributions of \texttt{ICE-Score} for correct and incorrect answers show substantial differences across score ranges.
%
% This trend is evident in the histograms in Figures~\ref{fig:samplecorr-combined}, where the density distributions of correct and incorrect candidate answers are largely distinct.
% 
To further quantify this discriminative ability, we calculated the AUC for \texttt{ICE-Score} as a predictor of correctness, obtaining strong values of 0.94 and 0.96 for cross-lingual and multilingual settings, respectively.
% 
Additionally, a t-test reveals a statistically significant difference between the correct and incorrect groups.
A detailed language-wise analysis is provided in Appendix~\ref{ap:code-analysis}.
%
\begin{table}[htbp]
  \resizebox{\columnwidth}{!}{
    \centering
    \caption{ICE-Score distribution (\%) for correct and incorrect answers in cross- and multilingual settings.}
    \label{tab:ice-score-distribution}
    \begin{tabular}{l l C{0.7cm} C{0.7cm} C{0.7cm} C{0.7cm} C{0.7cm}}
        \toprule
        \textbf{Setting} & \textbf{Answer Type} & \textbf{0} & \textbf{1} & \textbf{2} & \textbf{3} & \textbf{4} \\
        \midrule
        \multirow{2}{*}{Cross} & Correct   &  3.4 & 1.5 & 3.9 & 3.8 & 87.3  \\
                                       & Incorrect & 75.2  & 14.2  & 8.4 & 0.7 & 1.5 \\
        \midrule
        \multirow{2}{*}{Multi}  & Correct   & 2.0 & 1.3 & 3.6 & 4.1 & 89.0 \\
                                       & Incorrect & 52.8 & 25.8 & 17.5 & 2.3 & 1.6 \\
        \bottomrule
    \end{tabular}
  }
\end{table}

% utilization
\textbf{Application in Test-Time Scaling.}
% 
We now explore the potential of applying the ICE-Score as a heuristic for test-time scaling. 
%
We evaluate three approaches as discussed in Section~\ref{subsec:code_analysis}: (i) baseline model predictions without scaling, (ii) Self-Consistency (\texttt{SC}), and (iii) Soft Self-Consistency (\texttt{Soft-SC}) guided by the \texttt{ICE-Score}.
%
As shown in Table~\ref{tab:test-time-scaling}, our results indicate that test-time scaling substantially improves reasoning accuracy across both cross-lingual and multilingual settings. 
%
Conventional \texttt{SC} provides moderate gains, but \texttt{Soft-SC} with \texttt{ICE-Score} further boosts performance by prioritizing high-quality reasoning steps. 
%
Notably, for Llama2-7B, \texttt{Soft-SC} improves \textbf{cross-lingual performance from 39.2 to 56.6} and \textbf{multilingual performance from 57.2 to 71.2}. 
%
Similarly, CodeLlama-7B shows strong gains in both setups, demonstrating the method's robustness across model architectures.
%
These findings underscore the benefit of intermediate quality assessment as a means to improve cross-lingual and multilingual PoT reasoning and overall performance.

\begin{table}[htbp]
\tiny
  \centering
  \resizebox{\columnwidth}{!}{
  \begin{tabular}{l|c|cc|c}
    \hline
    \textbf{Method} & en & de & bn & ALL \\
    \hline\hline
    \multicolumn{5}{c}{\textit{Cross-lingual}} \\
    \hline    
    \multicolumn{1}{l|}{\underline{Llama2-7B}}& \multicolumn{1}{c|}{} & \multicolumn{2}{c|}{} & \multicolumn{1}{c}{} \\
    Without Comments & 58.0 & 40.4 & 12.0	&31.6  \\
    + \texttt{SC} & 65.2 & 51.6 & 15.2 &39.2  \\
    + \texttt{Soft-SC} (\texttt{ICE-Score}) & \textbf{76.8} & \textbf{69.2} & \textbf{33.6}	& \textbf{56.6}  \\ 
    \hline   
    \multicolumn{1}{l|}{\underline{CodeLlama-7B}}& \multicolumn{1}{c|}{} & \multicolumn{2}{c|}{} & \multicolumn{1}{c}{} \\
    Without Comments & 58.8 & 48.4 & 11.2 & 38.6  \\
    + \texttt{SC} & 69.6 & 57.2 & 17.2 & 46.7  \\
    + \texttt{Soft-SC} (\texttt{ICE-Score}) & \textbf{75.7} & \textbf{71.2} & \textbf{33.6} & \textbf{61.1}  \\
    \hline\hline
    \multicolumn{5}{c}{\textit{Multilingual}} \\
    \hline
    \multicolumn{1}{l|}{\underline{Llama2-7B}}& \multicolumn{1}{c|}{} & \multicolumn{2}{c|}{} & \multicolumn{1}{c}{} \\
    PoT Parallel & 56.0 & 47.2 & 30.8	& 44.6   \\
    + \texttt{SC} & 64.8 & 58.0 & 47.6	& 57.2  \\
    + \texttt{Soft-SC} (\texttt{ICE-Score}) & \textbf{77.6} & \textbf{72.0} & \textbf{65.6}	& \textbf{71.2} \\
    \hline   
    \multicolumn{1}{l|}{\underline{CodeLlama-7B}}& \multicolumn{1}{c|}{} & \multicolumn{2}{c|}{} & \multicolumn{1}{c}{} \\
    PoT Parallel & 61.9 & 52.8 & 35.6 & 49.0  \\
    + \texttt{SC} & 68.8 & 66.4 & 53.6 & 62.8  \\
    + \texttt{Soft-SC} (\texttt{ICE-Score}) & \textbf{79.2} & \textbf{77.6} & \textbf{68.8} & \textbf{75.6}  \\     
    \hline
  \end{tabular}
  }
  \caption{
  A comparative analysis of performance when implementing conventional \texttt{SC} and the proposed \texttt{Soft-SC} with \texttt{ICE-Score} in an optimal framework for cross-lingual and multilingual configurations.
  }
  \label{tab:test-time-scaling}
\end{table}
























% \subsection{What attribute for marginal gains when dropping the comments in cross-lingual?}

% \textbf{Cross-lingual code quality result} \dummy{\lipsum[2]}




% \textbf{Cross-lingual code quality associate with the accuracy} \dummy{\lipsum[2]}




% \textcolor{red}{whatif we have the signal that is associated with the downstream accuracy. We can utilize them to further boost the downstream result -- point out that CodeBERTScore does not work in the appendix table. Scatter plot of CodeBERT and spearman and discuss that we should use ICE-Score}



% \textcolor{blue}{To what extent, does better alignment training strategy attribute to code quality? -- start with the key takeaways}
% \textbf{What does the optimal training strategy attribute to? Code Quality.} \textcolor{red}{How can we justify? Choice: i) Density plot between two disks ii) graph } \dummy{\lipsum[2]}






% \textbf{There is an association in code quality with accuracy where the optimal strategy provides the best of both.} \dummy{\lipsum[2]}

% \textbf{Cross-lingual consideration} \dummy{\lipsum[2]}

% \textcolor{red}{The color in the scatter plot is not meaningful. Language-wise comparison, e.g., t-test looks good. Maybe box-plot with minimax and sd.}


% \subsection{Test-time Improvement for PoT}

% i) system wise
% ii) Sample wise
% iii) language wise?  (can also appear in i and ii)

% \textcolor{blue}{Strong corr between code quality and acc. Then we investigate the potential of test-time improvement}

% \textcolor{red}{what if the test-time boost is poor? What can we learn about it? Or can we refine the program?}

% \textcolor{blue}{From the analysis it's a system level, not a sample level, how can we leverage it? An example level can be demonstrated through distribution. -- AUC and t-test at sample level}


% \textcolor{red}{modify the inference cost by reducing the beam size instead of varying K from fixed beam size}







% \subsection{Does PoT deliver better intermediate reasoning than CoT?}

% \textbf{Adopting the teacher PoT deliver better intermediate reasoning than CoT - Table~\ref{tab:intermediate-reason}} \lipsum[2]

% \textcolor{red}{change the 6.1 to last and put 6.3 to prior. Start from the main impact in the reasoning fields.}















%\subsection{Does PoT robust to the distributional shift?}

%\textbf{[cite GSM-Symbolic] shown that LLM can not generalize to (our best attempt)} \dummy{\lipsum[2]}







% \subsection{What Cause the Notable Advancement in Underrepresented Languages?}

% \textcolor{red}{May drop this one or just empirically demonstrated the alignment in multilingual and cross-lingual -- rethinking to phase this section}

% \textbf{Language generation in URL is more challenging than HRL} \lipsum[2]

%\textbf{PoT can bypass those challenge and conquer the problem} \dummy{\lipsum[2]}









%\subsection{(low priortity - may appear in the appendix) Specialized-Models Comparison}

%\textbf{Comparison between Llemma (math-specific) CoT vs CodeLlama (code-specific) PoT} \dummy{\lipsum[2]}









\section{Related Works}

\paragraph{Direct Alignment Algorithms} 
Many variants of direct alignment algorithms perform alignment on offline preference datasets without an external reward model. DPO~\cite{rafailov2023direct} performs alignment through preference modeling with the implicit reward derived from the optimal policy of reward maximization under the KL penalty objective. RRHF~\cite{yuan2023rrhf} performs alignment by training to maintain the likelihood margin between preference ranks. KTO~\cite{ethayarajh2024kto} changes the assumptions of the Bradley-Terry model~\cite{bradley1952rank} used by DPO and introduces Prospect Theory~\cite{kahneman2013prospect}, and IPO~\cite{azar2024general} converts to the root-finding problem for strengthening the KL constraint. SLiC-HF~\cite{zhao2023slic}, CPO~\cite{xu2024contrastive}, ORPO~\cite{hong2024orpo}, and SimPO~\cite{meng2024simpo} train without reference models utilizing behavior cloning, margin loss, contrastive loss, odds ratio loss, and fixed margin by replacing the implicit rewards.

\paragraph{Reward Over-optimization and KL Penalty}
Since RLHF~\cite{ziegler2020finetuning} utilizes a trained reward model, it amplifies the limitations of the reward model as it is optimized for an imperfect reward, according to Goodhart’s Law~\cite{hoskin1996awful}, and this is called reward over-optimization~\cite{gao2023scaling}. However, \citet{rafailov2024scaling} finds that direct alignment algorithms also experience similar reward over-optimization, regardless of the variant. Direct alignment algorithms commonly show humped curves of performance according to the increase of KL divergence from the reference model during training. TR-DPO~\cite{gorbatovski2024learn} argues that this is due to the Hessian of the loss landscape converging to zero as the implicit reward margin grows during training, so they update the reference model for mitigating this phenomenon. On the other hand, $\beta$-DPO~\cite{wu2024beta}, which also performs relaxation of KL penalty, claims that adaptively changing $\beta$ through the statistics of the implicit reward margin is required to reflect the quality of the preference pair.

\begin{figure}[!t]
  \includegraphics[width=\columnwidth]{figures/reward_margin_image.pdf}
  \caption{The average implicit reward margin of pairs showing logit monotonicity according to the perturbation of $\beta$ in policies trained with DPO under various $\beta$ using Antropic-HH. We can confirm that \method{} assigns higher $\beta$ for preference pairs reveling high confusion on preference labels based on the observation that preference pairs with monotonically increasing logits show low confidences on preference model $\mathbb{P}_{\theta, \beta}(y^w \succ y^l | x)$.}
  \label{fig:reward_margin}
\end{figure}

\paragraph{Combining Sampling Distribution}
Combining sampling distributions of language models can be utilized to estimate a new sampling distribution with specific characteristics. Contrastive Decoding~\cite{li2022contrastive} shows that the log-likelihood margins of the expert and amateur language models can enhance response diversity by penalizing incorrect response patterns favored by the amateur language model. \citet{sanchez2023stay} shows that classifier-free guidance~\cite{ho2022classifier} can enhance prompt relativity in language modeling by treating prompts as conditions and sharpening the conditional sampling distribution. Combining the change from instruction-tuning in a small language model with a large language model can approximate fine-tuning. \citet{liu2024tuning} utilizes the instruction-tuned small language model as the logit offset, and \citet{mitchell2023emulator} estimates the importance sampling ratio of the optimal distribution defined by the objective of RLHF from it. Inspired by the theoretical motivation of \citet{mitchell2023emulator}, \citet{liu2024decoding} shows that the sampling distribution of the policy trained under the near $\beta$ by DPO can be approximated by policy obtained under $\beta$ and the reference policy.

\begin{figure}[!t]
  \includegraphics[width=\columnwidth]{figures/kl_trade_off_image.pdf}
  \caption{Pareto Frontier between KL divergence and performance of DPO, \method{}, TR-DPO$^\tau$ and TR-DPO$^\alpha$. We measure the KL divergence and performance of the models trained with $\beta=[0.5, 0.1, 0.05, 0.01]$ using Antropic-HH. We can see that \method{} shows better best performance than DPO, simultaneously achieving better KL trade-off efficiency than TR-DPO.}
  \label{fig:kl_trade_off}
\end{figure}
% \qy{In this paper, we propose an efficient single-stage framework called \nickname{} for 3D object detection. Considering the task of object detection inherently focuses on the foreground points, we propose an instance-aware learning-based downsampling way to automatically select the sparse yet important instance points. In addition, a dedicated contextual centroid perception module is proposed to fully exploit the geometrical structure around the bounding boxes. Extensive experiments conducted on the KITTI detection benchmark demonstrated the superior efficiency and accuracy of the proposed \nickname{}. \revise{In future work, we will further tackle extreme cases such as overlapped bounding boxes.}}

%This paper presents a new point-based single-stage 3D object detection networks, named \nickname{}. With novel instance-aware downsampling strategy and centroid rally module, we can effectively and efficiently achieve muti-class 3D object detection in a bottom-up manner.  Our \nickname{} achieves the best results among pure point-based methods, and provides a state-of-the-art efficiency than existing LiDAR detectors. In the future, we will focus on designing an efficient network to achieve real-time and robust 3D detection in 360-degree LiDAR scenes.

\qy{In this paper, we propose an efficient solution termed \nickname{} for point-based 3D object detection in LiDAR point clouds. Considering the task of object detection inherently focuses on the foreground information, we propose an instance-aware learning-based downsampling way to automatically select the sparse yet important instance points. Additionally, a dedicated contextual centroid perception module is proposed to fully exploit the geometrical structure around the bounding boxes. Extensive experiments conducted on three detection benchmarks demonstrated the superior efficiency and accuracy of the proposed \nickname{}. 
}

\smallskip\noindent\textbf{Limitations.} Although the proposed \nickname{} can achieve remarkable efficiency in object detection of large-scale LiDAR points clouds, it also has limitations. \textit{e.g.,} the instance-aware sampling relies on the semantic prediction of each point, which is susceptible to class imbalances distribution. For future work, we will further explore advanced techniques to alleviate the imbalanced issue.






% \section{Preamble}

% The first line of the file must be
% \begin{quote}
% \begin{verbatim}
% \documentclass[11pt]{article}
% \end{verbatim}
% \end{quote}

% To load the style file in the review version:
% \begin{quote}
% \begin{verbatim}
% \usepackage[review]{acl}
% \end{verbatim}
% \end{quote}
% For the final version, omit the \verb|review| option:
% \begin{quote}
% \begin{verbatim}
% \usepackage{acl}
% \end{verbatim}
% \end{quote}

% To use Times Roman, put the following in the preamble:
% \begin{quote}
% \begin{verbatim}
% \usepackage{times}
% \end{verbatim}
% \end{quote}
% (Alternatives like txfonts or newtx are also acceptable.)

% Please see the \LaTeX{} source of this document for comments on other packages that may be useful.

% Set the title and author using \verb|\title| and \verb|\author|. Within the author list, format multiple authors using \verb|\and| and \verb|\And| and \verb|\AND|; please see the \LaTeX{} source for examples.

% By default, the box containing the title and author names is set to the minimum of 5 cm. If you need more space, include the following in the preamble:
% \begin{quote}
% \begin{verbatim}
% \setlength\titlebox{<dim>}
% \end{verbatim}
% \end{quote}
% where \verb|<dim>| is replaced with a length. Do not set this length smaller than 5 cm.

% \section{Document Body}

% \subsection{Footnotes}

% Footnotes are inserted with the \verb|\footnote| command.\footnote{This is a footnote.}

% \subsection{Tables and figures}

% See Table~\ref{tab:accents} for an example of a table and its caption.
% \textbf{Do not override the default caption sizes.}

% \begin{table}
%   \centering
%   \begin{tabular}{lc}
%     \hline
%     \textbf{Command} & \textbf{Output} \\
%     \hline
%     \verb|{\"a}|     & {\"a}           \\
%     \verb|{\^e}|     & {\^e}           \\
%     \verb|{\`i}|     & {\`i}           \\
%     \verb|{\.I}|     & {\.I}           \\
%     \verb|{\o}|      & {\o}            \\
%     \verb|{\'u}|     & {\'u}           \\
%     \verb|{\aa}|     & {\aa}           \\\hline
%   \end{tabular}
%   \begin{tabular}{lc}
%     \hline
%     \textbf{Command} & \textbf{Output} \\
%     \hline
%     \verb|{\c c}|    & {\c c}          \\
%     \verb|{\u g}|    & {\u g}          \\
%     \verb|{\l}|      & {\l}            \\
%     \verb|{\~n}|     & {\~n}           \\
%     \verb|{\H o}|    & {\H o}          \\
%     \verb|{\v r}|    & {\v r}          \\
%     \verb|{\ss}|     & {\ss}           \\
%     \hline
%   \end{tabular}
%   \caption{Example commands for accented characters, to be used in, \emph{e.g.}, Bib\TeX{} entries.}
%   \label{tab:accents}
% \end{table}

% As much as possible, fonts in figures should conform
% to the document fonts. See Figure~\ref{fig:experiments} for an example of a figure and its caption.

% Using the \verb|graphicx| package graphics files can be included within figure
% environment at an appropriate point within the text.
% The \verb|graphicx| package supports various optional arguments to control the
% appearance of the figure.
% You must include it explicitly in the \LaTeX{} preamble (after the
% \verb|\documentclass| declaration and before \verb|\begin{document}|) using
% \verb|\usepackage{graphicx}|.

% \begin{figure}[t]
%   \includegraphics[width=\columnwidth]{example-image-golden}
%   \caption{A figure with a caption that runs for more than one line.
%     Example image is usually available through the \texttt{mwe} package
%     without even mentioning it in the preamble.}
%   \label{fig:experiments}
% \end{figure}

% \begin{figure*}[t]
%   \includegraphics[width=0.48\linewidth]{example-image-a} \hfill
%   \includegraphics[width=0.48\linewidth]{example-image-b}
%   \caption {A minimal working example to demonstrate how to place
%     two images side-by-side.}
% \end{figure*}

% \subsection{Hyperlinks}

% Users of older versions of \LaTeX{} may encounter the following error during compilation:
% \begin{quote}
% \verb|\pdfendlink| ended up in different nesting level than \verb|\pdfstartlink|.
% \end{quote}
% This happens when pdf\LaTeX{} is used and a citation splits across a page boundary. The best way to fix this is to upgrade \LaTeX{} to 2018-12-01 or later.

% \subsection{Citations}

% \begin{table*}
%   \centering
%   \begin{tabular}{lll}
%     \hline
%     \textbf{Output}           & \textbf{natbib command} & \textbf{ACL only command} \\
%     \hline
%     \citep{Gusfield:97}       & \verb|\citep|           &                           \\
%     \citealp{Gusfield:97}     & \verb|\citealp|         &                           \\
%     \citet{Gusfield:97}       & \verb|\citet|           &                           \\
%     \citeyearpar{Gusfield:97} & \verb|\citeyearpar|     &                           \\
%     \citeposs{Gusfield:97}    &                         & \verb|\citeposs|          \\
%     \hline
%   \end{tabular}
%   \caption{\label{citation-guide}
%     Citation commands supported by the style file.
%     The style is based on the natbib package and supports all natbib citation commands.
%     It also supports commands defined in previous ACL style files for compatibility.
%   }
% \end{table*}

% Table~\ref{citation-guide} shows the syntax supported by the style files.
% We encourage you to use the natbib styles.
% You can use the command \verb|\citet| (cite in text) to get ``author (year)'' citations, like this citation to a paper by \citet{Gusfield:97}.
% You can use the command \verb|\citep| (cite in parentheses) to get ``(author, year)'' citations \citep{Gusfield:97}.
% You can use the command \verb|\citealp| (alternative cite without parentheses) to get ``author, year'' citations, which is useful for using citations within parentheses (e.g. \citealp{Gusfield:97}).

% A possessive citation can be made with the command \verb|\citeposs|.
% This is not a standard natbib command, so it is generally not compatible
% with other style files.

% \subsection{References}

% \nocite{Ando2005,andrew2007scalable,rasooli-tetrault-2015}

% The \LaTeX{} and Bib\TeX{} style files provided roughly follow the American Psychological Association format.
% If your own bib file is named \texttt{custom.bib}, then placing the following before any appendices in your \LaTeX{} file will generate the references section for you:
% \begin{quote}
% \begin{verbatim}
% \bibliography{custom}
% \end{verbatim}
% \end{quote}

% You can obtain the complete ACL Anthology as a Bib\TeX{} file from \url{https://aclweb.org/anthology/anthology.bib.gz}.
% To include both the Anthology and your own .bib file, use the following instead of the above.
% \begin{quote}
% \begin{verbatim}
% \bibliography{anthology,custom}
% \end{verbatim}
% \end{quote}

% Please see Section~\ref{sec:bibtex} for information on preparing Bib\TeX{} files.

% \subsection{Equations}

% An example equation is shown below:
% \begin{equation}
%   \label{eq:example}
%   A = \pi r^2
% \end{equation}

% Labels for equation numbers, sections, subsections, figures and tables
% are all defined with the \verb|\label{label}| command and cross references
% to them are made with the \verb|\ref{label}| command.

% This an example cross-reference to Equation~\ref{eq:example}.

% \subsection{Appendices}

% Use \verb|\appendix| before any appendix section to switch the section numbering over to letters. See Appendix~\ref{sec:appendix} for an example.

% \section{Bib\TeX{} Files}
% \label{sec:bibtex}

% Unicode cannot be used in Bib\TeX{} entries, and some ways of typing special characters can disrupt Bib\TeX's alphabetization. The recommended way of typing special characters is shown in Table~\ref{tab:accents}.

% Please ensure that Bib\TeX{} records contain DOIs or URLs when possible, and for all the ACL materials that you reference.
% Use the \verb|doi| field for DOIs and the \verb|url| field for URLs.
% If a Bib\TeX{} entry has a URL or DOI field, the paper title in the references section will appear as a hyperlink to the paper, using the hyperref \LaTeX{} package.

\section*{Limitations}

\textbf{GSM8K as a Reasoning Benchmark.}
The experimental setup of this study is grounded in grade school math problems from GSM8K; therefore, the results and key findings \emph{may not} generalize to other reasoning-intensive tasks. 
%
Furthermore, recent studies have raised concerns regarding potential data contamination~\cite{gsm-plus, gsm1k, mirzadeh2025gsmsymbolic}. 
%
Nonetheless, GSM8K remains the gold standard for assessing multi-step reasoning in the literature. 
%
We use this benchmark to ensure cross-comparability with existing work while emphasizing that our experimental framework is adaptable to any multi-step reasoning benchmark. 
%
In future work, we plan to extend our assessments to additional benchmarks to further validate our findings.

\textbf{ICE-Score Model Choice and Test-time Scaling.}
Our test-time scaling study presents a preliminary investigation into leveraging the ICE-Score as a Soft Self-Consistency (Soft-SC) heuristic. 
%
Prior work on ICE-Score~\cite{ice-score} suggests that stronger models yield better results.
%
This work prioritizes the evaluation accuracy of the code quality itself, verifying the correlation between the intermediate and end results. 
%
To this end, we employ the 405B variant of Llama3 for ICE-Score calculations in the correlation studies at the system and sample levels.
%
To maintain consistency, we continue to use this model for test-time scaling experiments.

Our findings indicate that incorporating ICE-Score into Soft-SC leads to performance improvements. 
%
However, the magnitude of these gains may depend on the specific ICE-Score model used. 
%
%A more comprehensive analysis of test-time scaling would require a deeper exploration of the ICE-Score model choice and its robustness across different reasoning tasks. 
%
%Future work should examine alternative ICE-Score configurations and their cost-benefit trade-off.
Future work should examine other ICE-Score configurations or alternative solutions, assessing their cost-benefit trade-off.

%Regarding the choice of base LLMs, our experiments primarily focus on the Llama family, and as such, the findings may not directly extend to other multilingual open-source LLMs.
%Additionally, in terms of test-time scaling applicability, while we employ the \texttt{ICE-Score} as a toolkit, its dependence on an Oracle LLM makes it computationally expensive at inference time.
%This highlights the need for further research into more cost-effective and reliable approaches for assessing the quality of math-like generated program candidates.

% \section{Ethical Consideration}




% \section*{Acknowledgments}

%\dummy{\lipsum[1]}

% Bibliography entries for the entire Anthology, followed by custom entries
%\bibliography{anthology,custom}
% Custom bibliography entries only
\bibliography{main}

\appendix

% \section{Example Appendix}
\section{MGSM8KInstruct}
\label{ap:mathoctopus}

We adopt MGSM8KInstruct~\cite{mathoctopus} as the reference dataset for CoT in multilingual settings.
% 
This dataset comprises question-reasoning pairs $(\vb*{R}_i$, $\vb*{Q}_i)$ with $\vb*{Q}_i$ expressed in English, along with translations in nine additional languages, enabling the alignment of reasoning capabilities across different languages.
% 
\citet{mathoctopus} introduced two training strategies:
% \textcolor{blue}{drop two dataset notations from CoT since it may get confused. (Probably move to the appendix.)}
\begin{inparaenum}[(i)]
    \item \emph{CoT Cross}: Incorporates English questions with answers in the target language, promoting multilingual adaptability. 
    %
    Formally, the dataset is represented as: \begin{equation*}
        \mathcal{D}^\texttt{MGSM8KInstruct}_\texttt{cross} = \{(\vb*{Q}_i^\text{en}, \vb*{C}_i^l)|l\in L_\text{all}\}_{i=1}^N
    \label{eq:d-mathoctopus-cross}
    \end{equation*}
    where $L_\text{all}$ includes both English and target languages.
    \item \emph{CoT Parallel}: Uses question-answer pairs in the same language to 
    % enhance accuracy within a language
    enhancing the PoT capability within each target language, denoted as: 
    \begin{equation*}
    \mathcal{D}^\texttt{MGSM8KInstruct}_\texttt{parallel} = \{(\vb*{Q}_i^l, \vb*{C}_i^l)|l\in L_\text{all}\}_{i=1}^N.
    \label{eq:d-mathoctopus-parallel}
    \end{equation*}
\end{inparaenum}










\section{PoT Generation Methods}
\label{ap:pot-syn}






% \begin{figure*}[ht]
%   \begin{subfigure}{0.45\textwidth}
%     \includegraphics[width=\linewidth]{latex/figures/pipeline-cross-v1.pdf}
%     \caption{Cross-lingual Setup}
%     \label{fig:pipeline-cross}
%   \end{subfigure}
%   \begin{subfigure}{0.55\textwidth}
%     \includegraphics[width=\linewidth]{latex/figures/pipeline-muti-v1.pdf}
%     \caption{Multilingual Setup}
%     \label{fig:pipeline-multi}
%   \end{subfigure}  
  
%     \caption{
%     Pipeline overview \textcolor{blue}{Idea: Put something in the main for eye catching}
%     }
%     \centering
%     \label{fig:pipeline-overview}
% \end{figure*}











To facilitate a fair comparison between PoT and CoT, we employ the GSM8K dataset, a collection of grade-school math problems that require 2-8 reasoning steps to solve, as the foundational benchmark.
%
As illustrated in Figure~\ref{fig:pipeline-cross}, we generate solutions in a programming language using an Oracle LLM through various methodologies:
\begin{compactenum}[1.]
% \begin{enumerate}
    \item \textit{Zero-shot PoT Prompting}: Following the zero-shot prompting framework from \citet{pot}, the model is instructed to generate the \texttt{solver()} function in Python using a prompt $\vb*{S}_\text{PoT}$ with no exemplars. 
    %
    Formally, the PoT synthesis from an Oracle LLM is represented as $\hat{\vb*{R}_i}\sim p_\text{Oracle}(\vb*{Q}_i|\vb*{S}_\text{PoT})$.
    
    \item \textit{Few-shot PoT Prompting}: Building on the methodologies of \citet{pot, pal}, $k$ in-context exemplars, $\vb*{E}_\text{FS}=\{(\vb*{Q}_1, \vb*{R}_1), ...,  (\vb*{Q}_k, \vb*{R}_k)\}$, are incorporated into the prompt to provide explicit guidance on desired outputs. 
    %
    The PoT synthesis is thus defined as $\hat{\vb*{R}_i}\sim p_\text{Oracle}(\vb*{Q}_i|\vb*{E}_\text{FS},\vb*{S}_\text{PoT})$.
    
    \item \textit{Few-shot PoT Prompting + CoT Guidance}: Based on initial observations that high-quality PoT outputs often align with structured CoT reasoning $\vb*{C})$, an additional CoT guidance mechanism is introduced to better direct program generation. 
    %
    In this setting, the examples $\vb*{E}_\text{FS-CoT}=\{(\vb*{Q}_1, \vb*{C}_1, \vb*{R}_1), ...,  (\vb*{Q}_k, \vb*{C}_k, \vb*{R}_k)\}$ include both CoT reasoning ($\vb*{C}_i$) and the corresponding PoT solution ($\vb*{R}_i$). 
    %
    The PoT synthesis is then formulated as $\hat{\vb*{R}_i}\sim p_\text{Oracle}(\vb*{Q}_i|\vb*{C}_i, \vb*{E}_\text{FS-CoT},\vb*{S}_\text{PoT})$.
\end{compactenum}
% \end{enumerate}


% To address the absense of PoT data, we introduce a simple yet effective approach to generate PoT training data from the existing GSM8K dataset.
%
We empirically tested three approaches to identify the most effective method for maximizing the match between program execution outputs and gold-standard answers, using Llama3.1 405B Instruct \cite{Llama3} as the Oracle LLM: zero-shot prompting, few-shot prompting, and few-shot prompting with CoT reasoning.
%
In zero-shot prompting, the model is given only the original GSM8K question and generates the corresponding Python code to solve it.
%
Few-shot prompting extends this by providing the model with two exemplars of correctly solved GSM8K questions along with their corresponding Python solutions. 
%
Few-shot prompting with CoT reasoning further builds upon this by incorporating both the original answer and its Chain-of-Thought (CoT) reasoning from GSM8K.
%
Our evaluation demonstrated that the few-shot + CoT approach consistently outperformed the other methods, achieving a correctness rate of 96.1\% in synthesizing PoT samples. In comparison, the few-shot prompting method yielded a correctness rate of 94.5\%, while the zero-shot approach resulted in a significantly lower accuracy of 58.7\%.
% 

% \begin{table}[htbp]
% \tiny
%   \centering
%   \resizebox{0.7\columnwidth}{!}{
%   \begin{tabular}{l|c}
%     \hline
%     \textbf{Method} & Correctness (\%) \\
%     \hline
%     a & b \\
%     \hline
%   \end{tabular}
%   }
%   \caption{
%   various pot syn.
%   }
%   \label{tab:compare-pot-syn}
% \end{table}








% \begin{figure*}
%   \begin{subfigure}{0.42\textwidth}
%     \includegraphics[width=\linewidth]{latex/figures/pipeline-cross-v0.pdf}
%     \caption{Cross-lingual Setup \textcolor{blue}{add symbols}}
%     \label{fig:pipeline-cross}
%   \end{subfigure}
%   \begin{subfigure}{0.58\textwidth}
%     \includegraphics[width=\linewidth]{latex/figures/pipeline-muti-v1.pdf}
%     \caption{Multilingual Setup}
%     \label{fig:pipeline-multi}
%   \end{subfigure}  
  
%     \caption{
%     % \textcolor{red}{illustrated or emphasize question map from En to L2 that they are the same} Overview of our proposed study.
%     Overview of cross-lingual and multilingual pipelines, detailing PoT dataset construction from existing CoT datasets \cite{cobbe2021gsm8k} and subsequent evaluations.
%     \textbf{(a)}  \emph{Cross-Lingual PoT}: A standardized program-of-thought dataset, \texttt{GSM8KPoT}, is derived from the English CoT dataset using few-shot PoT prompting with CoT guidance from an oracle LLM. This dataset is then employed for fine-tuning LLMs and assessing their performance across nine additional, previously unseen languages.
%     \textbf{(b)}  \emph{Multilingual PoT}: To establish a multilingual PoT dataset, English questions are mapped to their target-language counterparts using \texttt{MGSM8KInstruct} \cite{mathoctopus}, aligning model capabilities across multiple languages. Additionally, inline comments from the program reasoning in \texttt{GSM8KPoT} are translated into target languages. The resulting dataset is leveraged to fine-tune models and evaluate their performance across 10 languages, including both high-resource (HRL) and underrepresented languages (URL).
%     }
%     \centering
%     \label{fig:pipeline-overview}
% \end{figure*}



\begin{figure}[h]
    \small
    \centering
    \begin{mdframed}
    \textbf{System} \newline
    You are a helpful assistant. Answer the following question
    by implementing a solver() function in Python program
    step by step, and then return the answer.
    \newline
    Solve them in a step-by-step fashion and output a single option as the final answer in [language] language.
    \end{mdframed}
    \begin{mdframed}
    \textbf{User} \newline
    Question: [Question]
    \end{mdframed}    
    \caption{
    Zero-shot PoT prompt template for PoT synthesis, where [Question] serves as a placeholder for the problem statement.
    }
    \label{fig:zs-prompt}
\end{figure}



\begin{figure*}[h]
    \small
    \centering
    \begin{mdframed}
    \textbf{System} \newline
    You are a helpful assistant. Answer the following question
    by implementing a solver() function in Python program
    step by step, and then return the answer.
    \newline
    Solve them in a step-by-step fashion and output a single option as the final answer in [language] language.
    \end{mdframed}
    \begin{mdframed}
    \textbf{User} \newline
    Question: Roger has 5 tennis balls. He buys 2 more cans of tennis balls. Each can has 3 tennis balls. How many tennis balls does he have now?
    \end{mdframed}
    \begin{mdframed}
    \textbf{Assistant}\newline
    def solver():\newline\hspace*{6mm}    \# Roger started with 5 tennis balls.\newline\hspace*{6mm}    tennis\_balls = 5\newline\hspace*{6mm}    \# 2 cans of 3 tennis balls each is\newline\hspace*{6mm}    bought\_balls = 2 * 3\newline\hspace*{6mm}    \# tennis balls. The answer is\newline\hspace*{6mm}    answer = tennis\_balls + bought\_balls"\newline\hspace*{6mm}    return answer
    \end{mdframed}
    \begin{mdframed}
    \textbf{User}\newline
        Question: The bakers at the Beverly Hills Bakery baked 200 loaves of bread on Monday morning. They sold 93 loaves in the morning and 39 loaves in the afternoon. A grocery store returned 6 unsold loaves. How many loaves of bread did they have left?
    \end{mdframed}
    \begin{mdframed}
        \textbf{Assistant}\newline
        def solver():\newline\hspace*{6mm}    \# The bakers started with 200 loaves\newline\hspace*{6mm}    loaves\_baked = 200\newline\hspace*{6mm}    \# They sold 93 in the morning and 39 in the afternoon\newline\hspace*{6mm}    loaves\_sold\_morning = 93\newline\hspace*{6mm}    loaves\_sold\_afternoon = 39\newline\hspace*{6mm}    \# The grocery store returned 6 loaves.\newline\hspace*{6mm}    loaves\_returned = 6\newline\hspace*{6mm}    \# The answer is\newline\hspace*{6mm}    answer = loaves\_baked - loaves\_sold\_morning - loaves\_sold\_afternoon + loaves\_returned\newline\hspace*{6mm}    return answer
    \end{mdframed}
    \begin{mdframed}
        \textbf{User}\newline
        Question: [Question]
    \end{mdframed}
    \caption{
    Few-shot PoT prompt template for PoT synthesis, with exemplars adapted from \cite{pal, pot}.
    }
    \label{fig:fs-prompt}
\end{figure*}



\begin{figure*}[h]
    \small
    \centering
    \begin{mdframed}
    \textbf{System} \newline
    You are a helpful assistant. Answer the following question
    by implementing a solver() function in Python program
    step by step, and then return the answer.
    \newline
    Solve them in a step-by-step fashion and output a single option as the final answer in [language] language.
    \end{mdframed}
    \begin{mdframed}
    \textbf{User} \newline
    Question: Roger has 5 tennis balls. He buys 2 more cans of tennis balls. Each can has 3 tennis balls. How many tennis balls does he have now?\newline
    Chain-of-thought: Roger started with 5 tennis balls. 2 cans of 3 tennis balls each is 6 tennis balls. 5 + 6 = 11. The answer is 11.
    \end{mdframed}
    \begin{mdframed}
    \textbf{Assistant}\newline
    def solver():\newline\hspace*{6mm}    \# Roger started with 5 tennis balls.\newline\hspace*{6mm}    tennis\_balls = 5\newline\hspace*{6mm}    \# 2 cans of 3 tennis balls each is\newline\hspace*{6mm}    bought\_balls = 2 * 3\newline\hspace*{6mm}    \# tennis balls. The answer is\newline\hspace*{6mm}    answer = tennis\_balls + bought\_balls"\newline\hspace*{6mm}    return answer
    \end{mdframed}
    \begin{mdframed}
    \textbf{User}\newline
        Question: The bakers at the Beverly Hills Bakery baked 200 loaves of bread on Monday morning. They sold 93 loaves in the morning and 39 loaves in the afternoon. A grocery store returned 6 unsold loaves. How many loaves of bread did they have left?\newline
        Chain-of-thought: The bakers started with 200 loaves of bread. They sold 93 loaves in the morning and 39 loaves in the afternoon: 93 + 39 = 132 loaves sold. A grocery store returned 6 loaves, so they got 6 loaves back. 200 - 132 + 6 = 74 loaves left. The answer is 74.
    \end{mdframed}
    \begin{mdframed}
        \textbf{Assistant}\newline
        def solver():\newline\hspace*{6mm}    \# The bakers started with 200 loaves\newline\hspace*{6mm}    loaves\_baked = 200\newline\hspace*{6mm}    \# They sold 93 in the morning and 39 in the afternoon\newline\hspace*{6mm}    loaves\_sold\_morning = 93\newline\hspace*{6mm}    loaves\_sold\_afternoon = 39\newline\hspace*{6mm}    \# The grocery store returned 6 loaves.\newline\hspace*{6mm}    loaves\_returned = 6\newline\hspace*{6mm}    \# The answer is\newline\hspace*{6mm}    answer = loaves\_baked - loaves\_sold\_morning - loaves\_sold\_afternoon + loaves\_returned\newline\hspace*{6mm}    return answer
    \end{mdframed}
    \begin{mdframed}
        \textbf{User}\newline
        Question: [Question]\newline
        Chain-of-thought: [CoT]
    \end{mdframed}
    \caption{
    Few-shot PoT prompt template incorporating our proposed CoT-guided approach for PoT synthesis, where [CoT] serves as a placeholder for natural language reasoning.
    }
    \label{fig:fs-prompt}
\end{figure*}




% \section{Variation of Fine-Tuning Process}
% \textcolor{red}{Since there is swahil issue in llama2-13b then: run 3 seeds thingy for llama2-13b in cross and multilingual setup: CoT, PoT}



% \section{Why there are only native CoT and inline comments PoT reported}

% demonstrated that, in cross-lingual setup, it is challenging to enforce the model to generate native multi-step by a matrix of proportion,



% \begin{figure*}
% \caption{pot syn.}
% \centering
% \includegraphics[width=1.0\textwidth]{figures/potsyn_v01.pdf}
% \label{fig:comparesynpot}
% \end{figure*}





















\section{Training Setting}
Our code is primarily based on the MathOctopus codebase, with some minor modifications. The code will be made available.

%
\noindent\textbf{Prompt Template.} During training and testing, we consistently use the same prompt template from MathOctopus \cite{mathoctopus}.

%
\noindent\textbf{Setting.} We fully fintune all our models on a single 4xA100 node for three epochs with a maximum sequence length 1024.
%
For the Llama2 family and CodeLlama, we used a learning rate of 2e-5 and an effective batch size of 512.
%
However, we found that this setting caused the Llama3 8B model not to produce desirable results, which we discuss further in the next section.
%
Thus, we changed the effective batch size to 128 and the learning rate to 5e-6, following \cite{tulu3} for Llama 3 8B.
% 
To generate multiple candidate predictions, we set \( top_k=50 \) and a temperature of 0.7, selecting the top 40 sequences for the voting process.



\section{Computing Resources}

We trained LLaMA family models on 4× NVIDIA A100 (80GB) GPUs, completing the fine-tuning process within approximately one hour for cross-lingual settings and around eight hours for multilingual settings.

During inference, generating predictions in a greedy fashion requires only three minutes. However, when producing multiple answer candidates with K=40, the process takes approximately seven hours to complete.

For Oracle LLM inference, we utilize a separate dedicated setup with 4× NVIDIA A100 (80GB) GPUs to host the LLM service, which is responsible for constructing PoT answers and evaluating code quality. The quality assessment process requires approximately 45 minutes for a single prediction and extends to 32 hours when assessing 40 candidates across all languages for a given model configuration. Additionally, we employ 62 concurrent processes to maximize inference throughput.

In summary, our experiments required a total of 544 A100 GPU hours for fine-tuning, 52 hours for inference, and 146 hours for quality assessment.




\section{Comparison with Non-Fine-Tuned PoT}

We compare our test-time scaling experiments with state-of-the-art (SOTA) non-fine-tuned PoT prompting methods and observe that our product models from PoT parallel with \texttt{SC} outperform SCross-PAL from \citet{crosspal} by 0.9 percentage points.
Furthermore, our proposed \texttt{Soft-SC} with \texttt{ICE-Score} achieves a significant accuracy improvement, increasing from 57.2\% to 71.2\%.


\begin{table}
\tiny
  \centering
  \resizebox{0.7\columnwidth}{!}{
  \begin{tabular}{l|c}
    \hline
    \textbf{Method} & ALL \\
    \hline\hline
    \multicolumn{2}{c}{\textit{Cross-lingual}} \\
    \hline    
    \multicolumn{1}{l|}{\underline{Llama2-7B}} & \multicolumn{1}{c}{} \\
    Without Comments &39.2  \\
    + \texttt{Soft-SC} (\texttt{ICE-Score}) & \textbf{56.6}  \\ 
    \hline   
    \multicolumn{1}{l|}{\underline{CodeLlama-7B}}& \multicolumn{1}{c}{} \\
    Without Comments & 38.6  \\
    + \texttt{SC} & 46.7  \\
    + \texttt{Soft-SC} (\texttt{ICE-Score}) & \textbf{61.1}  \\
    \hline\hline
    \multicolumn{2}{c}{\textit{Multilingual}} \\
    \hline
    \multicolumn{1}{l|}{\underline{Llama2-7B}}& \multicolumn{1}{c}{} \\
    PoT Parallel & 44.6   \\
    + \texttt{SC} & 57.2  \\
    + \texttt{Soft-SC} (\texttt{ICE-Score}) & \textbf{71.2} \\
    \hline   
    \multicolumn{1}{l|}{\underline{CodeLlama-7B}}& \multicolumn{1}{c}{} \\
    PoT Parallel & 49.0  \\
    + \texttt{SC} & 62.8  \\
    + \texttt{Soft-SC} (\texttt{ICE-Score}) & \textbf{75.6}  \\ 
    \hline\hline
    \multicolumn{2}{c}{\textit{Non-Fine-Tuned PoT}} \\
    \hline
    \multicolumn{1}{l|}{\underline{Llama2-7B}}& \multicolumn{1}{c}{} \\
    CLP \cite{clp} & 48.3 \\
    SCLP \cite{clp} & 54.1 \\
    \hdashline
    Cross-PAL \cite{crosspal} & 49.9 \\
    SCross-PAL \cite{crosspal} & \textbf{56.3} \\
    \hline
  \end{tabular}
  }
  \caption{
  The comparison of our adopted test-time scaling approaches with SOTA non-fine-tuned PoT approaches. The results of non-fune-tuned PoT are taken from \citet{crosspal}.
  }
  \label{tab:test-time-scaling-compare-prompt-pot}
\end{table}


% \vspace{2mm}

\section{Sensitivity of Llama3-8B}
During our testing, we observed that Llama3-8B exhibited significant sensitivity to our hyperparameters and chat template configurations.
%
Notably, the model frequently failed to generate the \texttt{def solver():} function header at the beginning of its reasoning chain, which is critical for extracting and compiling the generated code correctly.
%
To mitigate this issue, we inserted a prefix in our prompt, as illustrated in Figure \ref{fig:llama3-prompt}.
%
Additionally, with our initial hyperparameters, Llama3-8B frequently generated code snippets that failed to compile. Specifically, 9.12\% of its outputs were non-compilable, a significantly higher rate compared to Llama2-7B (3.08\%), CodeLlama-7B (2.04\%), and Llama2-13B (1.84\%). 
%
However, after refining our hyperparameters based on the approach outlined by \cite{tulu3} and adjusting the chat template, we observed a substantial reduction in compilation errors, with the failure rate dropping to 1.68\%.
%

\begin{figure}[H]
    \small
    \centering
    \begin{mdframed}
    \textbf{User} \newline
    Below are instructions for a task. \newline
    Write a response that appropriately completes the request in [language]. Please answer in Python with inline comments in [language].\newline
    \#\#\# Instruction: \newline[Question]\newline
    \#\#\# Response:\newline
    \textcolor{red}{def solver():} 
    \end{mdframed}
    \caption{
    Updated prompt with an added prefix (\texttt{def solver():}) for Llama3-8B.
    }
    \label{fig:llama3-prompt}
\end{figure}



\section{Alternative Metric For Code Quality Assessment}
Alternatively, to \texttt{ICE-Score}, we evaluated code quality using \texttt{CodeBERT-Score} \cite{codebertscore}. 
%
However, we noticed that GSM8K \cite{cobbe2021gsm8k} primarily consists of short code snippets where errors often involved small numerical mistakes rather than large structural or semantic differences.
%
Many of the errors stemmed from minor computation mistakes, like using the wrong arithmetic expression or associating wrong counts with the subject.
%
Since \texttt{CodeBERT-Score} is designed to assess broader semantic similarity, it struggled to distinguish the minute differences between correct and incorrect code.
%
As shown in Table \ref{tab:ablation-multi-codebert}, the scores across different systems varied only slightly $(\pm ~1.0\%)$, failing to reflect the accuracy differences observed in Tables \ref{tab:ablation-cross}, \ref{tab:ablation-multi}.
%
This suggests that \texttt{CodeBERT-Score} may not be well-suited for evaluating correctness in GSM8K-style problems.


\begin{table}[!htp]
\centering
\resizebox{\textwidth}{!}{%
\begin{tabular}{lllllllll}
\toprule
Paper & Domain & Training Code? & Analysis Code? & Checkpoints? & Metric Scores? \\
\midrule
\cite{rosenfeld2019constructive} & Vision, LM & N & N & N & N \\
\cite{mikamiscaling} & Vision & N & Y & Y & Y \\
\cite{schaeffer2023emergent} & LM & N & N & N & N \\
\cite{sardana2023beyond} & LM & N & N & N & N \\
\cite{sorscher2022beyond} & Vision & N & N & N & Y \\
\cite{caballero2022broken} & LM & N & Y & N & Y \\
\cite{besiroglu2024chinchilla} & LM &  & Y & N & Y \\
\cite{gordon2021data} & NMT & Y & Y & Y & Y \\
\cite{bansal2022data} & NMT & N & N & N & N \\
\cite{hestness2017deep} & NMT, LM, Vision, Speech & N & N & N & N \\
\cite{bi2024deepseek} & LM & N & N & N & N \\
\cite{bahri2021explaining} & Vision & N & N & N & N \\
\cite{geiping2022much} & Vision & Y & Y & N & N \\
\cite{poli2024mechanistic} & LM & N & N & N & N \\
\cite{hu2024minicpm} & LM & Y & N & N & N \\
\cite{hashimoto2021model} & NLP & N & N & N & N \\
\cite{ruan2024observational} & LM & Y & Y & N & Y \\
\cite{anil2023palm} & LM & N & N & N & N \\
\cite{pearce2024reconciling} & LM & N & Y & N & N \\
\cite{cherti2023reproducible} & VLM & Y & Y & Y & Y \\
\cite{porian2024resolving} & LM & Y & Y & N & Y \\
\cite{alabdulmohsin2022revisiting} & LM, Vision & N & Y & Y & Y \\
\cite{gao2024scalingevaluatingsparseautoencoders} & NLP & Y & Y & Y & N \\
\cite{muennighoff2024scaling} & LM & Y & Y & Y & N \\
\cite{rae2021scaling} & LM & N & N & N & N \\
\cite{shin2023scaling} & RecSys & N & N & N & N \\
\cite{hernandez2022scaling} & LM & N & N & N & N \\
\cite{filipovich2022scaling} & LM & N & N & N & N \\
\cite{neumann2022scaling} & RL & Y & Y* & Y & N \\
\cite{droppo2021scaling} & Speech & N & N & N & N \\
\cite{henighan2020scaling} & LM, Vision, Video, VLM & N & N & N & N \\
\cite{goyal2024scaling} & LM, Vision, VLM & N & Y & N & Y \\
\cite{aghajanyan2023scaling} & Multimodal LM & N & N & N & N \\
\cite{kaplan2020scaling} & LM & N & N & N & N \\
\cite{ghorbani2021scaling} & NMT & N & Y & N & N \\
\cite{gao2023scaling} & RL/LM & N & N & N & N \\
\cite{hilton2023scaling} & RL & N & N & N & N \\
\cite{frantar2023scaling} & LM, Vision & N & N & N & N \\
\cite{prato2021scaling} & Vision & Y* & Y & Y & Y \\
\cite{covert2024scaling} & LM & Y & Y & N & N \\
\cite{hernandez2021scaling} & LM & N & N & N & N \\
\cite{ivgi2022scaling} & NLP & N & N & N & N \\
\cite{tay2022scaling} & LM & N & N & N & N \\
\cite{tao2024scaling} & LM & N & Y & N & Y \\
\cite{jones2021scaling} & RL & Y & Y & N & Y \\
\cite{zhai2022scaling} & Vision & Y & N & N & N \\
\cite{dettmers2023case} & LM & N & N & N & N \\
\cite{dubey2024llama} & LM & N & N & N & N \\
\cite{hoffmann2022training} & LM & N & N & N & N \\
\cite{ardalani2022understanding} & RecSys & N & N & N & N \\
\cite{clark2022unified} & LM & N & Y & N & Y \\
\bottomrule
\end{tabular}
}

\caption{Details on domain of experiments and availability of code by category for each paper surveyed.}
\label{tab:full-basic}
\end{table}

% LaTeX code for Dataframe 2:
\begin{table}[]
\centering
\resizebox{\textwidth}{!}{%
% LaTeX code for Dataframe (9, 12):
\begin{tabular}{lllll}
\toprule
Paper & Power Law Form & Purpose Of Power Law (E.G., Performance Prediction, Optimal Ratio) & \# Power Law Parameters & \# Of Scaling Laws \\
\midrule
\cite{rosenfeld2019constructive} & $\tilde{\epsilon}(m, n)=a n^{-\alpha}+b m^{-\beta}+c_{\infty}$ & Performance Prediction & 5-6 & 8 \\
\cite{mikamiscaling} & $L(n, s)=\delta\left(\gamma+n^{-\alpha}\right) s^{-\beta}$ & Performance Prediction & 4 & 3 \\
\cite{schaeffer2023emergent} & None & N/A & NA & NA \\
\cite{sardana2023beyond} & $L(N, D) = E + \frac{A}{N^\alpha} + \frac{B}{D^\beta} $; $N^*\left(\ell, D_{\text {inf }}\right), D_{\text {tr }}^*\left(\ell, D_{\text {inf }}\right)={\arg \min } _{N, D_{\mathrm{tr}} \mid L\left(N, D_{\mathrm{tr}}\right)=\ell}$ & Performance Prediction & 5 & 4 \\
\cite{sorscher2022beyond} & $c \cdot \alpha^{-\beta} ,  c \cdot \exp (-b \alpha)$ & Performance Prediction & 2 & 34 \\
\cite{caballero2022broken} & $y=a+\left(b x^{-c_0}\right) \prod_{i=1}^n\left(1+\left(\frac{x}{d_i}\right)^{1 / f_i}\right)^{-c_i * f_i}$ & Performance Prediction & 5+ & 100+ \\
\cite{besiroglu2024chinchilla} & $L(N, D) = E + \frac{A}{N^\alpha} + \frac{B}{D^\beta} $ & Performance Prediction & 5 & 1 \\
\cite{gordon2021data} & $L(N, D) = \left[ \left( \frac{N}{N_c}\right)^{\frac{\alpha_N}{\alpha_D}} + \frac{D}{D_c} \right]^{\alpha_D} $ & Performance Prediction & 4 & 3 \\
\cite{bansal2022data} & $L(D)=\alpha\left(D^{-1}+C\right)^p$ & Performance Prediction & 3 & 20 \\
\cite{hestness2017deep} & $\varepsilon(m) \sim \alpha m^{\beta_g}+\gamma$ & Performance Prediction & 3 & 17 \\
\cite{bi2024deepseek} & $\begin{aligned} M_{\mathrm{opt}} & =M_{\mathrm{base}} \cdot C^a \\ D_{\mathrm{opt}} & =D_{\mathrm{base}} \cdot C^b\end{aligned}$, $\begin{aligned} & \eta_{\mathrm{opt}}=0.3118 \cdot C^{-0.1250} \\ & B_{\mathrm{opt}}=0.2920 \cdot C^{0.3271}\end{aligned}$ & Optimal Ratio, Performance Prediction & 2 & 5 \\
\cite{bahri2021explaining} & $L(D) \propto D^{-\alpha_K}, \quad L(P) \propto P^{-\alpha_K}$ & Performance Prediction & 2 & 35 \\
\cite{geiping2022much} & $f(x)=a x^{-c}+b$, $v_{\text {Effective Extra Samples from Augmentations }}(x)=f_{\text {ref }}^{-1}\left(f_{\text {aug }}(x)\right)-x$ & Performance Prediction & 3 & ~50 \\
\cite{poli2024mechanistic} & $\log N^* \propto a \log C$ and $\log D^* \propto b \log C$ & Performance Prediction & 2 &  \\
\cite{hu2024minicpm} & $L(N, D)=C_N N^{-\alpha}+C_D D^{-\beta}+L_0$ & Performance Prediction & 5 & 6 \\
\cite{hashimoto2021model} & $\min _{\lambda, \alpha} \mathbb{E}_{\hat{q}, \hat{n}}\left[\left(\log (R(\hat{n}, \hat{q})-\epsilon)-\alpha \log (\hat{n})+\log \left(C_\lambda(\hat{q})\right)\right)^2\right]$ $R(\hat{n}, \hat{q})=\mathbb{E}\left[\ell\left(\hat{\theta}\left(p_{\hat{n}, \hat{q}}\right) ; x, y\right)\right]$ & Performance Prediction & 2+n(data mixes) & 4 \\
\cite{ruan2024observational} & $E_m \approx h \sigma\left(\beta^{\top} S_m+\alpha\right)$ & Performance Prediction & 3 &  \\
\cite{anil2023palm} & $N^{\star}(C) \approx N_0^{\star} \cdot C^a$ & Performance Prediction & 2 & 1 \\
\cite{pearce2024reconciling} & $N^*_{{\setminus E}} = b C_{{\setminus E}}^m$ $L = bC^m$ & Optimal Ratio, Performance Prediction & 2 & 1 \\
\cite{cherti2023reproducible} & $E=\beta C^{\alpha}$ & Performance Prediction & 2 & 8 \\
\cite{porian2024resolving} & $N^{\star}(C) \approx N_0^{\star} \cdot C^a$ & Optimal Ratio & 2 & 6 \\
\cite{alabdulmohsin2022revisiting} & $\varepsilon_x=\beta x^c$; $\varepsilon_x - \varepsilon_\infty=\beta x^c$; $\varepsilon_x=\beta (x^{-1} + \gamma)^{-c}$;   $\varepsilon_x=\gamma(x)(1+\gamma(x))^{-1} \varepsilon_0+(1+\gamma(x))^{-1} \varepsilon_{\infty}$ & Performance Prediction & 2-4 & ~600 \\
\cite{gao2024scalingevaluatingsparseautoencoders} & $L(n, k)=\exp \left(\alpha+\beta_k \log (k)+\beta_n \log (n)+\gamma \log (k) \log (n)\right)+\exp (\zeta+\eta \log (k))$ & Performance Prediction & 2-6 & 1 \\
\cite{muennighoff2024scaling} & $L\left(U_N, U_D, R_N, R_D\right)=\frac{A}{\left(U_N+U_N R_N^*\left(1-e^{\frac{-R_N}{R_N^*}}\right)\right)^\alpha}+\frac{B}{\left(U_D+U_D R_D^*\left(1-e^{\frac{-R_D}{R_D^*}}\right)\right)^\beta}+E$ & Performance Prediction & 2 (+4) & 1 \\
\cite{rae2021scaling} & None & Performance Prediction & N/A & N/A \\
\cite{shin2023scaling} & None & Scaling trend & NA & NA \\
\cite{hernandez2022scaling} & $E=k * N^\alpha$ & Optimal Ratio & 2 & 1 \\
\cite{filipovich2022scaling} & $\mathcal{L}(C)=\left(C_c C\right)^{\alpha_C}$ & Performance Prediction & 2 & 3 \\
\cite{neumann2022scaling} & $N_{\text {opt }}(C)=\left(\frac{C}{C_0}\right)^{\alpha_C^{o p t}}$, $E_i=\frac{1}{1+\left(N_j / N_i\right)^{\alpha_N}}$ & Performance Prediction & 2 & 3 * 2 \\
\cite{droppo2021scaling} & $L(N, D)=\left[\left(L_{\infty}\right)^{\frac{1}{\alpha}}+\left(\frac{N_C}{N}\right)^{\frac{\alpha_N}{\alpha}}+\left(\frac{D_C}{D}\right)^{\frac{\alpha_D}{\alpha}}\right]^\alpha$ & Performance Prediction & 6 & 3 \\
\cite{henighan2020scaling} & $L(x)=L_{\infty}+\left(\frac{x_0}{x}\right)^{\alpha_x}$ & Performance Prediction & 3 & 36 \\
\cite{goyal2024scaling} & $y_k=a \cdot n_1^{b_1} \prod_{j=2}^k\left(\frac{n_j}{n_{j-1}}\right)^{b_j}+d$ & Performance Prediction & 2+ 2*n(data mixes) & 1 \\
\cite{aghajanyan2023scaling} & $L(N, D_j)=E_j + \frac{A_j}{N^{\alpha_j}} + \frac{B_j}{|D_j|^{\beta_j}}$, $L(N, D_i, D_j) = [\frac{L(N, D_i) + L(N, D_j)}{2}] - C_{i,j} + \frac{A_{i,j}}{N^{\alpha_{i,j}}} + \frac{B_{i,j}}{|D_i|+|D_j|^{\beta_{i,j}}}$ & Performance Prediction & 5 & 14 \\
\cite{kaplan2020scaling} & $L(N, D) = \left[ \left( \frac{N}{N_c}\right)^{\frac{\alpha_N}{\alpha_D}} + \frac{D}{D_c} \right]^{\alpha_D}$     & Performance Prediction & 4 & ~7 \\
\cite{ghorbani2021scaling} & $\mathrm{BLEU}=c_B L^{-p_B}$, $\hat{L}_{o p t}(B)=\alpha^* B^{-\left(p_d+p_e\right)}+L_{\infty}, \quad \alpha^* \equiv \alpha\left(\frac{\bar{N}_e\left(p_e+p_d\right)}{p_e}\right)^{p_e}\left(\frac{\bar{N}_d\left(p_e+p_d\right)}{p_d}\right)^{p_d}$ & Optimal Ratio, Performance Prediction & 6 & ~8 \\
\cite{gao2023scaling} & $\begin{aligned} & R_{\mathrm{bo} n}(d)=d\left(\alpha_{\mathrm{bo} n}-\beta_{\mathrm{bo} n} d\right), \\ & R_{\mathrm{RL}}(d)=d\left(\alpha_{\mathrm{RL}}-\beta_{\mathrm{RL}} \log d\right)\end{aligned}$ & Performance Prediction & 2 & 2 \\
\cite{hilton2023scaling} & $I^{-\beta}=\left(\frac{N_c}{N}\right)^{\alpha_N}+\left(\frac{E_c}{E}\right)^{\alpha_E}$ & Optimal Ratio, Performance Prediction & 5 & 3 \\
\cite{frantar2023scaling} & $L(S, N, D)=\left(a_S(1-S)^{b_S}+c_S\right) \cdot\left(\frac{1}{N}\right)^{b_N}+\left(\frac{a_D}{D}\right)^{b_D}+c$ & Optimal Ratio, Performance Prediction & 7 & 2 \\
\cite{prato2021scaling} & $\begin{aligned} & \operatorname{Err}(N)=\operatorname{Err}_{\infty}+k N^\alpha, \\ & \operatorname{Err}(C)=\operatorname{Err}_{\infty}+k C^\alpha,\end{aligned}$ & Performance Prediction & 3 & 12 \\
\cite{covert2024scaling} & $\log \left|\psi_k(z)\right| \approx \log |c(z)|-\alpha(z) \log (k)$ & Performance Prediction & 2 & Many \\
\cite{hernandez2021scaling} & $L \approx\left[\left(\frac{N_C}{N}\right)^{\frac{\alpha_N}{\alpha_D}}+\frac{D_C}{k\left(D_F\right)^\alpha(N)^\beta}\right]^{\alpha_D}$ & Performance Prediction & 3 & 1 \\
\cite{ivgi2022scaling} & NS & Performance Prediction & NA & NA \\
\cite{tay2022scaling} & None & Scaling trend & NA & NA \\
\cite{tao2024scaling} & $N_{\mathrm{v}}^{\mathrm{opt}}=N_{\mathrm{v}}^0 *\left(\frac{N_{\mathrm{nv}}}{N_{\mathrm{nv}}^0}\right)^\gamma$, $\mathcal{L}_u=-E+\frac{A_1}{N_{\mathrm{nv}}^{\alpha_1}}+\frac{A_2}{N_{\mathrm{v}}^{\alpha_2}}+\frac{B}{D^\beta}$ & Optimal Ratio, Performance Prediction & 7 & 2 \\
\cite{jones2021scaling} & $\begin{aligned} \text { plateau } & =m_{\text {boardsize }}^{\text {plateau }} \cdot \text { boardsize }+c^{\text {plateau }} \\ \text { incline } & =m_{\text {boardsize }}^{\text {incline }} \cdot \text { boardsize }+m_{\text {flops }}^{\text {incline }} \cdot \log \text { flop }+c^{\text {incline }} \\ \text { elo } & =\text { incline.clamp }(\text { plateau }, 0)\end{aligned}$ & Performance Prediction & 5 & 1 \\
\cite{zhai2022scaling} & $E=\alpha+\beta(C+\gamma)^{-\mu}$ & Performance Prediction & 4 & 3 \\
\cite{dettmers2023case} & None & Scaling trend & NA & NA \\
\cite{dubey2024llama} & $N^{\star}(C)=A C^\alpha$. & Optimal Ratio  & 2 & 2 \\
\cite{hoffmann2022training} & A3: $L(N, D) = E + \frac{A}{N^\alpha} + \frac{B}{D^\beta} $ & Optimal Ratio, Performance Prediction & 5 & 3 \\
\cite{ardalani2022understanding} & $\left(\alpha x^{-\beta}+\gamma\right)$ & Performance Prediction & 3 & 3 \\
\cite{clark2022unified} & $\log L(N, E) \triangleq+a \log N+b \log E+c \log N \log E+d$ & Performance Prediction & 4 & 3 \\
\bottomrule
\end{tabular}
}

\caption{Details on power law for each paper surveyed.}
\label{tab:full-powerlaw}
\end{table}

% LaTeX code for Dataframe 2:
\begin{table}[]
\centering
\resizebox{\textwidth}{!}{%
% LaTeX code for Dataframe 3:
% LaTeX code for Dataframe (13, 19):
\begin{tabular}{llllllll}
\toprule
Paper & Training Runs / Law & Max. Training Flops & Max. Training Params & Max. Training Data & Data Described? & Hyperparameters Described? & How Are Model Params Counted \\
&&&&&&& (E.G., W/ Or W/Out Embeddings) \\
\midrule
\cite{rosenfeld2019constructive} & 42-49 &  & 0.7M-70M & 100M words / 1.2M images & Y & Y & Non-embedding \\
\cite{mikamiscaling} & 7 &  & ResNet-101 & 64k-1.28M images & Y & Y & NA \\
\cite{schaeffer2023emergent} & 4 &  & $10^{11}$ & NA & Y & NA & Non-embedding \\
\cite{sardana2023beyond} & 47 &  & 150M-6B & 1.5B-1.25T tokens & N & Y & NA \\
\cite{sorscher2022beyond} & ~60 &  & 86M (ViT) & 200 epochs & Y & Y & NA \\
\cite{caballero2022broken} & 3-40 &  & NS & NS & N & N & NS \\
\cite{besiroglu2024chinchilla} & NA & NA & NA & NA & Y & NA & Non-embedding \\
\cite{gordon2021data} & 45-55 &  & 56M & 28.3M-51.1M examples & Y & Y & Non-embedding \\
\cite{bansal2022data} & 10 &  & 170M-800M & 500K-512M sentences (28B tokens) & Y & Y & NS \\
\cite{hestness2017deep} & ~9 &  & upto 193M  & $2^{19}-2^{28}$ tokens, upto $2^9$ images, $2k$ audio hours & Y & Y & NS \\
\cite{bi2024deepseek} & 80 & $1e17-3e20$ &  &  & Y & Y & Non-embedding \\
\cite{bahri2021explaining} & 8-27 &  & 36.5M & upto 78k steps; 100 epochs & Y & Y & NS \\
\cite{geiping2022much} & 13 &  & ResNet-18 & upto 7.6M images & Y & Y & NS \\
\cite{poli2024mechanistic} & 500 total & 8.00E+19 & 70M-7B &  & Y & Y & Non-embedding  \\
\cite{hu2024minicpm} & 36 &  & 40M-2B & 400M-120B tokens & Y & Y & Non-embedding  \\
\cite{hashimoto2021model} &  &  &  & upto 600k sentences & Y & Y & NA \\
\cite{ruan2024observational} & 27* -77* &  & 70B-180B & 3T-6T tokens & N/A & N/A & N.S. \\
\cite{anil2023palm} & 12 & 1.00E+22 & 15B & 4.00E+11 & N & N & Non-embedding \\
\cite{pearce2024reconciling} & 20 (simulated), 25 (real) &  & 1.5B (simulated), 4.6M (real) & 23B (simulated), 500M (real) tokens & Y & Y & w/ Embedding and Non-embedding considered separately \\
\cite{cherti2023reproducible} & 3* - 29 &  & 214M & 34B (pretrain), 2B (finetune) examples  & Y & Y & N.S \\
\cite{porian2024resolving} & 16 & 2.00E+19 & 901M &  & Y & Y & w/ Embedding and Non-embedding considered separately \\
\cite{alabdulmohsin2022revisiting} & 1* &  & 110M-1B & 1e6-1e10 ex / 3e11 tokens & Mixed & N & N/A \\
\cite{gao2024scalingevaluatingsparseautoencoders} & N.S & N.S & N.S & N.S & N & N & N.S \\
\cite{muennighoff2024scaling} & 142 &  & 8,7B & 900B tokens & Y & Y & w/ embedding \\
\cite{rae2021scaling} & 4 & 6.31E+23 & 280B &  & Y & Y & Non-embedding \\
\cite{shin2023scaling} & 17 & ~0.1 PF Days & 160M & 500M-50B tokens & Y & Y & NA \\
\cite{hernandez2022scaling} & 56 &  & 1.5M-800M & 100B tokens & N & N & NS \\
\cite{filipovich2022scaling} & 4 &  & 57-509M & 30B token & Y & N & NS \\
\cite{neumann2022scaling} & 14 &  & ~$5*10^5$ & $10^4$ steps & Y & Y & NS \\
\cite{droppo2021scaling} & 5-21 &  & ~$10^7$ & 134-23k hrs speech & Y & Y & NS \\
\cite{henighan2020scaling} & 6-10 &  & ~$10^11$ & ~$10^12$ tokens & Y & Y & Non-embedding \\
\cite{goyal2024scaling} & 5 &  & CLIP L/14 - ~300M +63M & 32-640M samples & Y & Y & Embedding \\
\cite{aghajanyan2023scaling} & 21 &  & 8M-6.7B & 5-100B tokens & Y & Y & Non-embedding \\
\cite{kaplan2020scaling} & ~40-150 &  & 1.5B & 23B tokens & Y & Y & Non-embedding \\
\cite{ghorbani2021scaling} & 12-14 &  & 191-3B & NS & Y & Y & Non-embedding \\
\cite{gao2023scaling} & 9 &  & 3B & 120-90k & N & Y & NS \\
\cite{hilton2023scaling} & NS & $10^{20}$ &  &  & Y & Y & NS \\
\cite{frantar2023scaling} & 48 and 112 &  & 0.66M-85M  & 1.8B images, 65B tokens & Y & Y & Non-embedding \\
\cite{prato2021scaling} & 5 &  &  & $10^6$ samples & Y & N & NA \\
\cite{covert2024scaling} & 10 &  & NA & 1000 samples for IMDB & Y & Y & NA \\
\cite{hernandez2021scaling} & NS & $10^{21}$ & $10^8$ &  & Y & N & Non-embedding \\
\cite{ivgi2022scaling} & 5-8 &  & $10^4-10^8$ & varies; 500k steps PT & Y & Y & Non-embedding \\
\cite{tay2022scaling} &  &  & 16-30B & $2^19$ & Y & Y & NA \\
\cite{tao2024scaling} & 60 &  & 33M-1.13B NV + 4-96k V & 4.3B-509B Characters & Y & Y & Embedding and Non-embedding considered separately \\
\cite{jones2021scaling} & 200 & 1E+12-1E+17 &  & 4E+08-2E+09 & Y & Y & NA \\
\cite{zhai2022scaling} & 44 &  & 5.4M-1.8B & 1-13M images & Y & Y & NA \\
\cite{dettmers2023case} & 4 &  & 19M-176B & NA & NA & Y & NA \\
\cite{dubey2024llama} & NS & $6*10^{18}-10^22$ & 40M-16B &  & Y* & Y & NS \\
\cite{hoffmann2022training} & ~200-450 & $6*10^{18}-3*10^{21}$ & 16B & 5B-400B tokens & Y & Y & Non-embedding \\
\cite{ardalani2022understanding} & NS & $10^2$-$10^6$ TFlops &  & ~5M-5B samples & N & N & All are considered \\
\cite{clark2022unified} & 56 &  & 15M-1.3B & 130B tokens & Y & Y & Non-embedding \\
\bottomrule
\end{tabular}

}

\caption{Details on training setup for each paper surveyed.}
\label{tab:full-setup}
\end{table}

% LaTeX code for Dataframe 2:
\begin{table}[]
\centering
\resizebox{\textwidth}{!}{%

% LaTeX code for Dataframe (20, 24):
\begin{tabular}{lllll}
\toprule
Paper & Data Points Per Law? & Scaling Law Metric & Modification Of Final Metric? & Subsets Of Data Used \\
\midrule
\cite{rosenfeld2019constructive} & 42-49 & Loss / Top1 Error & N & N \\
\cite{mikamiscaling} & 7 & Error Rate & N & N \\
\cite{schaeffer2023emergent} & NA & Various downstream & NA & NA \\
\cite{sardana2023beyond} & NS & Loss & NS & NS \\
\cite{sorscher2022beyond} & ~60 & Error Rate & NA & NA \\
\cite{caballero2022broken} & 3-40 & FID, Loss, Error Rate, Elo Score & N & NS \\
\cite{besiroglu2024chinchilla} & 245 & Loss & N & N \\
\cite{gordon2021data} & 45-55 & Loss & N & N \\
\cite{bansal2022data} & NS & Loss, BLEU & NS & NS \\
\cite{hestness2017deep} & NS & Token Error, CER, Error Rate, Loss & Median min. validation error across multiple training runs with separate random seeds & NS \\
\cite{bi2024deepseek} & upto 80 & Validation bits-per-byte & NS & NS \\
\cite{bahri2021explaining} & upto 100 & Loss & NS & NS \\
\cite{geiping2022much} & ~50 & Effective Extra Samples & Interpolation & NS \\
\cite{poli2024mechanistic} & NS & Loss & NS & NS \\
\cite{hu2024minicpm} & NS & Loss & NS & NS \\
\cite{hashimoto2021model} & NS & Loss & NS & NS \\
\cite{ruan2024observational} &  & Various downstream & N & N \\
\cite{anil2023palm} & 12 & Loss & N & N \\
\cite{pearce2024reconciling} & 20, 5 & Loss & N & N \\
\cite{cherti2023reproducible} & 3-29 & Error Rate & N & N \\
\cite{porian2024resolving} & 12 & Loss & N & N \\
\cite{alabdulmohsin2022revisiting} & N.S. & Loss / Accuracy & N & N/A \\
\cite{gao2024scalingevaluatingsparseautoencoders} & N.S & MSE & N.S & N.S \\
\cite{muennighoff2024scaling} & 142 & Loss & N & Outliers removed \\
\cite{rae2021scaling} & 4 & Loss & N/A & N/A \\
\cite{shin2023scaling} & NA & Loss & NA & NA \\
\cite{hernandez2022scaling} & NS & Loss & N & N \\
\cite{filipovich2022scaling} & NS & Loss & N & N \\
\cite{neumann2022scaling} & 238 & Elo Score & N & N \\
\cite{droppo2021scaling} & NS & Loss & N & N \\
\cite{henighan2020scaling} & NS & Loss, Error Rate & NS & Drop smaller models \\
\cite{goyal2024scaling} & NS & Error Rate & N & N \\
\cite{aghajanyan2023scaling} & NS & Perplexity & N & N \\
\cite{kaplan2020scaling} & NS & Loss & NS & NS \\
\cite{ghorbani2021scaling} & NS & Loss, BLEU & Median of last 50k steps &  \\
\cite{gao2023scaling} & ~90 & RM Score & NS & NS \\
\cite{hilton2023scaling} & NS & Intrinsic Performance & Smoothing learning curve & Exclude early checkpoints \\
\cite{frantar2023scaling} & 48 and 112 & Loss & NS & NS \\
\cite{prato2021scaling} & 5 & Error Rate & NS & NS \\
\cite{covert2024scaling} & (1000-5000 )*10 & Expectation  & NS & N \\
\cite{hernandez2021scaling} & 40-120 & Loss & NS & NS \\
\cite{ivgi2022scaling} & 5-8 & Loss & N & [2.5, 97.5] percentile \\
\cite{tay2022scaling} & NA & Loss, Accuracy & NA & NA \\
\cite{tao2024scaling} & 20*60 & Loss & Interpolation & NS \\
\cite{jones2021scaling} & 2800 & Elo Score & NS & NS \\
\cite{zhai2022scaling} & NS & Accuracy & NS & NS \\
\cite{dettmers2023case} & NA & Accuracy & NA & NA \\
\cite{dubey2024llama} & ~150 & Loss, Accuracy & NS & NS \\
\cite{hoffmann2022training} & upto 1500 & Loss & N & Lowest loss model of a FLOP count, last 15\% of checkpoints \\
\cite{ardalani2022understanding} & ~130 & Loss & NS & NS \\
\cite{clark2022unified} & ~26*56 & Loss & Log & NS \\
\bottomrule
\end{tabular}

}

\caption{Details on data extraction for each paper surveyed.}
\label{tab:full-eval}
\end{table}




\begin{table}[]
\centering
\resizebox{\textwidth}{!}{%
% LaTeX code for Dataframe (25, 29):
\begin{tabular}{llllll}
\toprule
Paper & Curve-Fitting Method & Loss Objective & Hyperparameters Reported? & Initialization & Are Scaling Laws Validated? \\
\midrule
\cite{rosenfeld2019constructive} & Least Squares Regression & Custom error term & N/A & Random & Y \\
\cite{mikamiscaling} & Non-linear Least Squares in log-log space &  & N/A & N/A & Y \\
\cite{schaeffer2023emergent} & NA & NA & NA & NA & NA \\
\cite{sardana2023beyond} & L-BFGS & Huber Loss & Y & Grid Search & N \\
\cite{sorscher2022beyond} & NA & NA & NA & NA & NA \\
\cite{caballero2022broken} & Least Squares Regression & MSLE & N/A & Grid Search, optimize one & Y \\
\cite{besiroglu2024chinchilla} & L-BFGS & Huber Loss & Y & Grid Search & Y \\
\cite{gordon2021data} & Least Squares Regression &  & N/A & N.S. & N \\
\cite{bansal2022data} & NS & NS & N & NS & N \\
\cite{hestness2017deep} & NS & RMSE & N & NS & Y \\
\cite{bi2024deepseek} & NS & NS & N & NS & Y \\
\cite{bahri2021explaining} & NS & NS & N & NS & N \\
\cite{geiping2022much} & Non-linear Least Squares &  & NA & Non-augmented parameters & Y \\
\cite{poli2024mechanistic} & NS & NS & N & NS & N \\
\cite{hu2024minicpm} & scipy curvefit & NS & N & NS & N \\
\cite{hashimoto2021model} & Adagrad & Custom Loss & Y & Xavier & Y \\
\cite{ruan2024observational} & Linear Least Squares & Various & N/A & N/A & Y \\
\cite{anil2023palm} & Polynomial Regression (Quadratic) & N.S. & N & N.S. & Y \\
\cite{pearce2024reconciling} & Polynomial Least Squares & MSE on Log-loss & N/A & N/A & N \\
\cite{cherti2023reproducible} & Linear Least Squares & MSE & N/A & N/A & N \\
\cite{porian2024resolving} & Weighted Linear Regression & weighted SE on Log-loss & N/A & N/A & Y \\
\cite{alabdulmohsin2022revisiting} & Least Squares Regression & MSE & Y & N.S. & Y \\
\cite{gao2024scalingevaluatingsparseautoencoders} & N.S & N.S & N.S & N.S & N.S \\
\cite{muennighoff2024scaling} & L-BFGS & Huber on Log-loss & Y & Grid Search, optimize all & Y \\
\cite{rae2021scaling} & None & None & N/A & N/A & N \\
\cite{shin2023scaling} & NA & NA & NA & NA & NA \\
\cite{hernandez2022scaling} & NS & NS & NS & NS & NS \\
\cite{filipovich2022scaling} & NS & NS & NS & NS & NS \\
\cite{neumann2022scaling} & NS & NS & NS & NS & NS \\
\cite{droppo2021scaling} & NS & NS & NS & NS & NS \\
\cite{henighan2020scaling} & NS & NS & NS & NS & NS \\
\cite{goyal2024scaling} & Grid Search & L2 error & Y & NA & Y \\
\cite{aghajanyan2023scaling} & L-BFGS & Huber on Log-loss & Y & Grid Search, optimize all & Y \\
\cite{kaplan2020scaling} & NS & NS & NS & NS & N \\
\cite{ghorbani2021scaling} & Trust Region Reflective algorithm, Least Squares & Soft-L1 Loss & Y & Fixed & Y \\
\cite{gao2023scaling} & NS & NS & NS & NS & Y \\
\cite{hilton2023scaling} & CMA-ES+Linear Regression & L2 log loss & Y & Fixed & Y \\
\cite{frantar2023scaling} & BFGS & Huber on Log-loss & Y & N Random Trials & Y \\
\cite{prato2021scaling} & NS & NS & NS & NS & NS \\
\cite{covert2024scaling} & Adam & Custom Loss & Y & NS & Y \\
\cite{hernandez2021scaling} & NS & NS & NS & NS & Y \\
\cite{ivgi2022scaling} & Linear Least Squares in Log-Log space & MSE & NA & NS & Y \\
\cite{tay2022scaling} & NA & NA & NA & NA & NA \\
\cite{tao2024scaling} & L-BFGS, Least Squares & Huber on Log-loss & Y & N Random Trials from Grid & Y \\
\cite{jones2021scaling} & L-BFGS & NS & NS & NS & NS \\
\cite{zhai2022scaling} & NS & NS & NS & NS & NS \\
\cite{dettmers2023case} & NA & NA & NA & NA & NA \\
\cite{dubey2024llama} & NS & NS & NS & NS & Y \\
\cite{hoffmann2022training} & L-BFGS & Huber on Log-loss & Y & Grid Search, optimize all & Y \\
\cite{ardalani2022understanding} & NS & NS & NS & NS & NS \\
\cite{clark2022unified} & L-BFGS-B & L2 Loss & Y & Fixed & NS \\
\bottomrule
\end{tabular}

}

\caption{Details on optimization for each paper surveyed.}
\label{tab:full-opt}
\end{table}

\end{document}
