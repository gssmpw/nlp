\section{MGSM8KInstruct}
\label{ap:mathoctopus}

We adopt MGSM8KInstruct~\cite{mathoctopus} as the reference dataset for CoT in multilingual settings.
% 
This dataset comprises question-reasoning pairs $(\vb*{R}_i$, $\vb*{Q}_i)$ with $\vb*{Q}_i$ expressed in English, along with translations in nine additional languages, enabling the alignment of reasoning capabilities across different languages.
% 
\citet{mathoctopus} introduced two training strategies:
% \textcolor{blue}{drop two dataset notations from CoT since it may get confused. (Probably move to the appendix.)}
\begin{inparaenum}[(i)]
    \item \emph{CoT Cross}: Incorporates English questions with answers in the target language, promoting multilingual adaptability. 
    %
    Formally, the dataset is represented as: \begin{equation*}
        \mathcal{D}^\texttt{MGSM8KInstruct}_\texttt{cross} = \{(\vb*{Q}_i^\text{en}, \vb*{C}_i^l)|l\in L_\text{all}\}_{i=1}^N
    \label{eq:d-mathoctopus-cross}
    \end{equation*}
    where $L_\text{all}$ includes both English and target languages.
    \item \emph{CoT Parallel}: Uses question-answer pairs in the same language to 
    % enhance accuracy within a language
    enhancing the PoT capability within each target language, denoted as: 
    \begin{equation*}
    \mathcal{D}^\texttt{MGSM8KInstruct}_\texttt{parallel} = \{(\vb*{Q}_i^l, \vb*{C}_i^l)|l\in L_\text{all}\}_{i=1}^N.
    \label{eq:d-mathoctopus-parallel}
    \end{equation*}
\end{inparaenum}










\section{PoT Generation Methods}
\label{ap:pot-syn}






% \begin{figure*}[ht]
%   \begin{subfigure}{0.45\textwidth}
%     \includegraphics[width=\linewidth]{latex/figures/pipeline-cross-v1.pdf}
%     \caption{Cross-lingual Setup}
%     \label{fig:pipeline-cross}
%   \end{subfigure}
%   \begin{subfigure}{0.55\textwidth}
%     \includegraphics[width=\linewidth]{latex/figures/pipeline-muti-v1.pdf}
%     \caption{Multilingual Setup}
%     \label{fig:pipeline-multi}
%   \end{subfigure}  
  
%     \caption{
%     Pipeline overview \textcolor{blue}{Idea: Put something in the main for eye catching}
%     }
%     \centering
%     \label{fig:pipeline-overview}
% \end{figure*}











To facilitate a fair comparison between PoT and CoT, we employ the GSM8K dataset, a collection of grade-school math problems that require 2-8 reasoning steps to solve, as the foundational benchmark.
%
As illustrated in Figure~\ref{fig:pipeline-cross}, we generate solutions in a programming language using an Oracle LLM through various methodologies:
\begin{compactenum}[1.]
% \begin{enumerate}
    \item \textit{Zero-shot PoT Prompting}: Following the zero-shot prompting framework from \citet{pot}, the model is instructed to generate the \texttt{solver()} function in Python using a prompt $\vb*{S}_\text{PoT}$ with no exemplars. 
    %
    Formally, the PoT synthesis from an Oracle LLM is represented as $\hat{\vb*{R}_i}\sim p_\text{Oracle}(\vb*{Q}_i|\vb*{S}_\text{PoT})$.
    
    \item \textit{Few-shot PoT Prompting}: Building on the methodologies of \citet{pot, pal}, $k$ in-context exemplars, $\vb*{E}_\text{FS}=\{(\vb*{Q}_1, \vb*{R}_1), ...,  (\vb*{Q}_k, \vb*{R}_k)\}$, are incorporated into the prompt to provide explicit guidance on desired outputs. 
    %
    The PoT synthesis is thus defined as $\hat{\vb*{R}_i}\sim p_\text{Oracle}(\vb*{Q}_i|\vb*{E}_\text{FS},\vb*{S}_\text{PoT})$.
    
    \item \textit{Few-shot PoT Prompting + CoT Guidance}: Based on initial observations that high-quality PoT outputs often align with structured CoT reasoning $\vb*{C})$, an additional CoT guidance mechanism is introduced to better direct program generation. 
    %
    In this setting, the examples $\vb*{E}_\text{FS-CoT}=\{(\vb*{Q}_1, \vb*{C}_1, \vb*{R}_1), ...,  (\vb*{Q}_k, \vb*{C}_k, \vb*{R}_k)\}$ include both CoT reasoning ($\vb*{C}_i$) and the corresponding PoT solution ($\vb*{R}_i$). 
    %
    The PoT synthesis is then formulated as $\hat{\vb*{R}_i}\sim p_\text{Oracle}(\vb*{Q}_i|\vb*{C}_i, \vb*{E}_\text{FS-CoT},\vb*{S}_\text{PoT})$.
\end{compactenum}
% \end{enumerate}


% To address the absense of PoT data, we introduce a simple yet effective approach to generate PoT training data from the existing GSM8K dataset.
%
We empirically tested three approaches to identify the most effective method for maximizing the match between program execution outputs and gold-standard answers, using Llama3.1 405B Instruct \cite{Llama3} as the Oracle LLM: zero-shot prompting, few-shot prompting, and few-shot prompting with CoT reasoning.
%
In zero-shot prompting, the model is given only the original GSM8K question and generates the corresponding Python code to solve it.
%
Few-shot prompting extends this by providing the model with two exemplars of correctly solved GSM8K questions along with their corresponding Python solutions. 
%
Few-shot prompting with CoT reasoning further builds upon this by incorporating both the original answer and its Chain-of-Thought (CoT) reasoning from GSM8K.
%
Our evaluation demonstrated that the few-shot + CoT approach consistently outperformed the other methods, achieving a correctness rate of 96.1\% in synthesizing PoT samples. In comparison, the few-shot prompting method yielded a correctness rate of 94.5\%, while the zero-shot approach resulted in a significantly lower accuracy of 58.7\%.
% 

% \begin{table}[htbp]
% \tiny
%   \centering
%   \resizebox{0.7\columnwidth}{!}{
%   \begin{tabular}{l|c}
%     \hline
%     \textbf{Method} & Correctness (\%) \\
%     \hline
%     a & b \\
%     \hline
%   \end{tabular}
%   }
%   \caption{
%   various pot syn.
%   }
%   \label{tab:compare-pot-syn}
% \end{table}








% \begin{figure*}
%   \begin{subfigure}{0.42\textwidth}
%     \includegraphics[width=\linewidth]{latex/figures/pipeline-cross-v0.pdf}
%     \caption{Cross-lingual Setup \textcolor{blue}{add symbols}}
%     \label{fig:pipeline-cross}
%   \end{subfigure}
%   \begin{subfigure}{0.58\textwidth}
%     \includegraphics[width=\linewidth]{latex/figures/pipeline-muti-v1.pdf}
%     \caption{Multilingual Setup}
%     \label{fig:pipeline-multi}
%   \end{subfigure}  
  
%     \caption{
%     % \textcolor{red}{illustrated or emphasize question map from En to L2 that they are the same} Overview of our proposed study.
%     Overview of cross-lingual and multilingual pipelines, detailing PoT dataset construction from existing CoT datasets \cite{cobbe2021gsm8k} and subsequent evaluations.
%     \textbf{(a)}  \emph{Cross-Lingual PoT}: A standardized program-of-thought dataset, \texttt{GSM8KPoT}, is derived from the English CoT dataset using few-shot PoT prompting with CoT guidance from an oracle LLM. This dataset is then employed for fine-tuning LLMs and assessing their performance across nine additional, previously unseen languages.
%     \textbf{(b)}  \emph{Multilingual PoT}: To establish a multilingual PoT dataset, English questions are mapped to their target-language counterparts using \texttt{MGSM8KInstruct} \cite{mathoctopus}, aligning model capabilities across multiple languages. Additionally, inline comments from the program reasoning in \texttt{GSM8KPoT} are translated into target languages. The resulting dataset is leveraged to fine-tune models and evaluate their performance across 10 languages, including both high-resource (HRL) and underrepresented languages (URL).
%     }
%     \centering
%     \label{fig:pipeline-overview}
% \end{figure*}



\begin{figure}[h]
    \small
    \centering
    \begin{mdframed}
    \textbf{System} \newline
    You are a helpful assistant. Answer the following question
    by implementing a solver() function in Python program
    step by step, and then return the answer.
    \newline
    Solve them in a step-by-step fashion and output a single option as the final answer in [language] language.
    \end{mdframed}
    \begin{mdframed}
    \textbf{User} \newline
    Question: [Question]
    \end{mdframed}    
    \caption{
    Zero-shot PoT prompt template for PoT synthesis, where [Question] serves as a placeholder for the problem statement.
    }
    \label{fig:zs-prompt}
\end{figure}



\begin{figure*}[h]
    \small
    \centering
    \begin{mdframed}
    \textbf{System} \newline
    You are a helpful assistant. Answer the following question
    by implementing a solver() function in Python program
    step by step, and then return the answer.
    \newline
    Solve them in a step-by-step fashion and output a single option as the final answer in [language] language.
    \end{mdframed}
    \begin{mdframed}
    \textbf{User} \newline
    Question: Roger has 5 tennis balls. He buys 2 more cans of tennis balls. Each can has 3 tennis balls. How many tennis balls does he have now?
    \end{mdframed}
    \begin{mdframed}
    \textbf{Assistant}\newline
    def solver():\newline\hspace*{6mm}    \# Roger started with 5 tennis balls.\newline\hspace*{6mm}    tennis\_balls = 5\newline\hspace*{6mm}    \# 2 cans of 3 tennis balls each is\newline\hspace*{6mm}    bought\_balls = 2 * 3\newline\hspace*{6mm}    \# tennis balls. The answer is\newline\hspace*{6mm}    answer = tennis\_balls + bought\_balls"\newline\hspace*{6mm}    return answer
    \end{mdframed}
    \begin{mdframed}
    \textbf{User}\newline
        Question: The bakers at the Beverly Hills Bakery baked 200 loaves of bread on Monday morning. They sold 93 loaves in the morning and 39 loaves in the afternoon. A grocery store returned 6 unsold loaves. How many loaves of bread did they have left?
    \end{mdframed}
    \begin{mdframed}
        \textbf{Assistant}\newline
        def solver():\newline\hspace*{6mm}    \# The bakers started with 200 loaves\newline\hspace*{6mm}    loaves\_baked = 200\newline\hspace*{6mm}    \# They sold 93 in the morning and 39 in the afternoon\newline\hspace*{6mm}    loaves\_sold\_morning = 93\newline\hspace*{6mm}    loaves\_sold\_afternoon = 39\newline\hspace*{6mm}    \# The grocery store returned 6 loaves.\newline\hspace*{6mm}    loaves\_returned = 6\newline\hspace*{6mm}    \# The answer is\newline\hspace*{6mm}    answer = loaves\_baked - loaves\_sold\_morning - loaves\_sold\_afternoon + loaves\_returned\newline\hspace*{6mm}    return answer
    \end{mdframed}
    \begin{mdframed}
        \textbf{User}\newline
        Question: [Question]
    \end{mdframed}
    \caption{
    Few-shot PoT prompt template for PoT synthesis, with exemplars adapted from \cite{pal, pot}.
    }
    \label{fig:fs-prompt}
\end{figure*}



\begin{figure*}[h]
    \small
    \centering
    \begin{mdframed}
    \textbf{System} \newline
    You are a helpful assistant. Answer the following question
    by implementing a solver() function in Python program
    step by step, and then return the answer.
    \newline
    Solve them in a step-by-step fashion and output a single option as the final answer in [language] language.
    \end{mdframed}
    \begin{mdframed}
    \textbf{User} \newline
    Question: Roger has 5 tennis balls. He buys 2 more cans of tennis balls. Each can has 3 tennis balls. How many tennis balls does he have now?\newline
    Chain-of-thought: Roger started with 5 tennis balls. 2 cans of 3 tennis balls each is 6 tennis balls. 5 + 6 = 11. The answer is 11.
    \end{mdframed}
    \begin{mdframed}
    \textbf{Assistant}\newline
    def solver():\newline\hspace*{6mm}    \# Roger started with 5 tennis balls.\newline\hspace*{6mm}    tennis\_balls = 5\newline\hspace*{6mm}    \# 2 cans of 3 tennis balls each is\newline\hspace*{6mm}    bought\_balls = 2 * 3\newline\hspace*{6mm}    \# tennis balls. The answer is\newline\hspace*{6mm}    answer = tennis\_balls + bought\_balls"\newline\hspace*{6mm}    return answer
    \end{mdframed}
    \begin{mdframed}
    \textbf{User}\newline
        Question: The bakers at the Beverly Hills Bakery baked 200 loaves of bread on Monday morning. They sold 93 loaves in the morning and 39 loaves in the afternoon. A grocery store returned 6 unsold loaves. How many loaves of bread did they have left?\newline
        Chain-of-thought: The bakers started with 200 loaves of bread. They sold 93 loaves in the morning and 39 loaves in the afternoon: 93 + 39 = 132 loaves sold. A grocery store returned 6 loaves, so they got 6 loaves back. 200 - 132 + 6 = 74 loaves left. The answer is 74.
    \end{mdframed}
    \begin{mdframed}
        \textbf{Assistant}\newline
        def solver():\newline\hspace*{6mm}    \# The bakers started with 200 loaves\newline\hspace*{6mm}    loaves\_baked = 200\newline\hspace*{6mm}    \# They sold 93 in the morning and 39 in the afternoon\newline\hspace*{6mm}    loaves\_sold\_morning = 93\newline\hspace*{6mm}    loaves\_sold\_afternoon = 39\newline\hspace*{6mm}    \# The grocery store returned 6 loaves.\newline\hspace*{6mm}    loaves\_returned = 6\newline\hspace*{6mm}    \# The answer is\newline\hspace*{6mm}    answer = loaves\_baked - loaves\_sold\_morning - loaves\_sold\_afternoon + loaves\_returned\newline\hspace*{6mm}    return answer
    \end{mdframed}
    \begin{mdframed}
        \textbf{User}\newline
        Question: [Question]\newline
        Chain-of-thought: [CoT]
    \end{mdframed}
    \caption{
    Few-shot PoT prompt template incorporating our proposed CoT-guided approach for PoT synthesis, where [CoT] serves as a placeholder for natural language reasoning.
    }
    \label{fig:fs-prompt}
\end{figure*}




% \section{Variation of Fine-Tuning Process}
% \textcolor{red}{Since there is swahil issue in llama2-13b then: run 3 seeds thingy for llama2-13b in cross and multilingual setup: CoT, PoT}



% \section{Why there are only native CoT and inline comments PoT reported}

% demonstrated that, in cross-lingual setup, it is challenging to enforce the model to generate native multi-step by a matrix of proportion,



% \begin{figure*}
% \caption{pot syn.}
% \centering
% \includegraphics[width=1.0\textwidth]{figures/potsyn_v01.pdf}
% \label{fig:comparesynpot}
% \end{figure*}





















\section{Training Setting}
Our code is primarily based on the MathOctopus codebase, with some minor modifications. The code will be made available.

%
\noindent\textbf{Prompt Template.} During training and testing, we consistently use the same prompt template from MathOctopus \cite{mathoctopus}.

%
\noindent\textbf{Setting.} We fully fintune all our models on a single 4xA100 node for three epochs with a maximum sequence length 1024.
%
For the Llama2 family and CodeLlama, we used a learning rate of 2e-5 and an effective batch size of 512.
%
However, we found that this setting caused the Llama3 8B model not to produce desirable results, which we discuss further in the next section.
%
Thus, we changed the effective batch size to 128 and the learning rate to 5e-6, following \cite{tulu3} for Llama 3 8B.
% 
To generate multiple candidate predictions, we set \( top_k=50 \) and a temperature of 0.7, selecting the top 40 sequences for the voting process.



\section{Computing Resources}

We trained LLaMA family models on 4× NVIDIA A100 (80GB) GPUs, completing the fine-tuning process within approximately one hour for cross-lingual settings and around eight hours for multilingual settings.

During inference, generating predictions in a greedy fashion requires only three minutes. However, when producing multiple answer candidates with K=40, the process takes approximately seven hours to complete.

For Oracle LLM inference, we utilize a separate dedicated setup with 4× NVIDIA A100 (80GB) GPUs to host the LLM service, which is responsible for constructing PoT answers and evaluating code quality. The quality assessment process requires approximately 45 minutes for a single prediction and extends to 32 hours when assessing 40 candidates across all languages for a given model configuration. Additionally, we employ 62 concurrent processes to maximize inference throughput.

In summary, our experiments required a total of 544 A100 GPU hours for fine-tuning, 52 hours for inference, and 146 hours for quality assessment.




\section{Comparison with Non-Fine-Tuned PoT}

We compare our test-time scaling experiments with state-of-the-art (SOTA) non-fine-tuned PoT prompting methods and observe that our product models from PoT parallel with \texttt{SC} outperform SCross-PAL from \citet{crosspal} by 0.9 percentage points.
Furthermore, our proposed \texttt{Soft-SC} with \texttt{ICE-Score} achieves a significant accuracy improvement, increasing from 57.2\% to 71.2\%.


\begin{table}
\tiny
  \centering
  \resizebox{0.7\columnwidth}{!}{
  \begin{tabular}{l|c}
    \hline
    \textbf{Method} & ALL \\
    \hline\hline
    \multicolumn{2}{c}{\textit{Cross-lingual}} \\
    \hline    
    \multicolumn{1}{l|}{\underline{Llama2-7B}} & \multicolumn{1}{c}{} \\
    Without Comments &39.2  \\
    + \texttt{Soft-SC} (\texttt{ICE-Score}) & \textbf{56.6}  \\ 
    \hline   
    \multicolumn{1}{l|}{\underline{CodeLlama-7B}}& \multicolumn{1}{c}{} \\
    Without Comments & 38.6  \\
    + \texttt{SC} & 46.7  \\
    + \texttt{Soft-SC} (\texttt{ICE-Score}) & \textbf{61.1}  \\
    \hline\hline
    \multicolumn{2}{c}{\textit{Multilingual}} \\
    \hline
    \multicolumn{1}{l|}{\underline{Llama2-7B}}& \multicolumn{1}{c}{} \\
    PoT Parallel & 44.6   \\
    + \texttt{SC} & 57.2  \\
    + \texttt{Soft-SC} (\texttt{ICE-Score}) & \textbf{71.2} \\
    \hline   
    \multicolumn{1}{l|}{\underline{CodeLlama-7B}}& \multicolumn{1}{c}{} \\
    PoT Parallel & 49.0  \\
    + \texttt{SC} & 62.8  \\
    + \texttt{Soft-SC} (\texttt{ICE-Score}) & \textbf{75.6}  \\ 
    \hline\hline
    \multicolumn{2}{c}{\textit{Non-Fine-Tuned PoT}} \\
    \hline
    \multicolumn{1}{l|}{\underline{Llama2-7B}}& \multicolumn{1}{c}{} \\
    CLP \cite{clp} & 48.3 \\
    SCLP \cite{clp} & 54.1 \\
    \hdashline
    Cross-PAL \cite{crosspal} & 49.9 \\
    SCross-PAL \cite{crosspal} & \textbf{56.3} \\
    \hline
  \end{tabular}
  }
  \caption{
  The comparison of our adopted test-time scaling approaches with SOTA non-fine-tuned PoT approaches. The results of non-fune-tuned PoT are taken from \citet{crosspal}.
  }
  \label{tab:test-time-scaling-compare-prompt-pot}
\end{table}


% \vspace{2mm}

\section{Sensitivity of Llama3-8B}
During our testing, we observed that Llama3-8B exhibited significant sensitivity to our hyperparameters and chat template configurations.
%
Notably, the model frequently failed to generate the \texttt{def solver():} function header at the beginning of its reasoning chain, which is critical for extracting and compiling the generated code correctly.
%
To mitigate this issue, we inserted a prefix in our prompt, as illustrated in Figure \ref{fig:llama3-prompt}.
%
Additionally, with our initial hyperparameters, Llama3-8B frequently generated code snippets that failed to compile. Specifically, 9.12\% of its outputs were non-compilable, a significantly higher rate compared to Llama2-7B (3.08\%), CodeLlama-7B (2.04\%), and Llama2-13B (1.84\%). 
%
However, after refining our hyperparameters based on the approach outlined by \cite{tulu3} and adjusting the chat template, we observed a substantial reduction in compilation errors, with the failure rate dropping to 1.68\%.
%

\begin{figure}[H]
    \small
    \centering
    \begin{mdframed}
    \textbf{User} \newline
    Below are instructions for a task. \newline
    Write a response that appropriately completes the request in [language]. Please answer in Python with inline comments in [language].\newline
    \#\#\# Instruction: \newline[Question]\newline
    \#\#\# Response:\newline
    \textcolor{red}{def solver():} 
    \end{mdframed}
    \caption{
    Updated prompt with an added prefix (\texttt{def solver():}) for Llama3-8B.
    }
    \label{fig:llama3-prompt}
\end{figure}



\section{Alternative Metric For Code Quality Assessment}
Alternatively, to \texttt{ICE-Score}, we evaluated code quality using \texttt{CodeBERT-Score} \cite{codebertscore}. 
%
However, we noticed that GSM8K \cite{cobbe2021gsm8k} primarily consists of short code snippets where errors often involved small numerical mistakes rather than large structural or semantic differences.
%
Many of the errors stemmed from minor computation mistakes, like using the wrong arithmetic expression or associating wrong counts with the subject.
%
Since \texttt{CodeBERT-Score} is designed to assess broader semantic similarity, it struggled to distinguish the minute differences between correct and incorrect code.
%
As shown in Table \ref{tab:ablation-multi-codebert}, the scores across different systems varied only slightly $(\pm ~1.0\%)$, failing to reflect the accuracy differences observed in Tables \ref{tab:ablation-cross}, \ref{tab:ablation-multi}.
%
This suggests that \texttt{CodeBERT-Score} may not be well-suited for evaluating correctness in GSM8K-style problems.


\begin{table}[!htp]
\centering
\resizebox{\textwidth}{!}{%
\begin{tabular}{lllllllll}
\toprule
Paper & Domain & Training Code? & Analysis Code? & Checkpoints? & Metric Scores? \\
\midrule
\cite{rosenfeld2019constructive} & Vision, LM & N & N & N & N \\
\cite{mikamiscaling} & Vision & N & Y & Y & Y \\
\cite{schaeffer2023emergent} & LM & N & N & N & N \\
\cite{sardana2023beyond} & LM & N & N & N & N \\
\cite{sorscher2022beyond} & Vision & N & N & N & Y \\
\cite{caballero2022broken} & LM & N & Y & N & Y \\
\cite{besiroglu2024chinchilla} & LM &  & Y & N & Y \\
\cite{gordon2021data} & NMT & Y & Y & Y & Y \\
\cite{bansal2022data} & NMT & N & N & N & N \\
\cite{hestness2017deep} & NMT, LM, Vision, Speech & N & N & N & N \\
\cite{bi2024deepseek} & LM & N & N & N & N \\
\cite{bahri2021explaining} & Vision & N & N & N & N \\
\cite{geiping2022much} & Vision & Y & Y & N & N \\
\cite{poli2024mechanistic} & LM & N & N & N & N \\
\cite{hu2024minicpm} & LM & Y & N & N & N \\
\cite{hashimoto2021model} & NLP & N & N & N & N \\
\cite{ruan2024observational} & LM & Y & Y & N & Y \\
\cite{anil2023palm} & LM & N & N & N & N \\
\cite{pearce2024reconciling} & LM & N & Y & N & N \\
\cite{cherti2023reproducible} & VLM & Y & Y & Y & Y \\
\cite{porian2024resolving} & LM & Y & Y & N & Y \\
\cite{alabdulmohsin2022revisiting} & LM, Vision & N & Y & Y & Y \\
\cite{gao2024scalingevaluatingsparseautoencoders} & NLP & Y & Y & Y & N \\
\cite{muennighoff2024scaling} & LM & Y & Y & Y & N \\
\cite{rae2021scaling} & LM & N & N & N & N \\
\cite{shin2023scaling} & RecSys & N & N & N & N \\
\cite{hernandez2022scaling} & LM & N & N & N & N \\
\cite{filipovich2022scaling} & LM & N & N & N & N \\
\cite{neumann2022scaling} & RL & Y & Y* & Y & N \\
\cite{droppo2021scaling} & Speech & N & N & N & N \\
\cite{henighan2020scaling} & LM, Vision, Video, VLM & N & N & N & N \\
\cite{goyal2024scaling} & LM, Vision, VLM & N & Y & N & Y \\
\cite{aghajanyan2023scaling} & Multimodal LM & N & N & N & N \\
\cite{kaplan2020scaling} & LM & N & N & N & N \\
\cite{ghorbani2021scaling} & NMT & N & Y & N & N \\
\cite{gao2023scaling} & RL/LM & N & N & N & N \\
\cite{hilton2023scaling} & RL & N & N & N & N \\
\cite{frantar2023scaling} & LM, Vision & N & N & N & N \\
\cite{prato2021scaling} & Vision & Y* & Y & Y & Y \\
\cite{covert2024scaling} & LM & Y & Y & N & N \\
\cite{hernandez2021scaling} & LM & N & N & N & N \\
\cite{ivgi2022scaling} & NLP & N & N & N & N \\
\cite{tay2022scaling} & LM & N & N & N & N \\
\cite{tao2024scaling} & LM & N & Y & N & Y \\
\cite{jones2021scaling} & RL & Y & Y & N & Y \\
\cite{zhai2022scaling} & Vision & Y & N & N & N \\
\cite{dettmers2023case} & LM & N & N & N & N \\
\cite{dubey2024llama} & LM & N & N & N & N \\
\cite{hoffmann2022training} & LM & N & N & N & N \\
\cite{ardalani2022understanding} & RecSys & N & N & N & N \\
\cite{clark2022unified} & LM & N & Y & N & Y \\
\bottomrule
\end{tabular}
}

\caption{Details on domain of experiments and availability of code by category for each paper surveyed.}
\label{tab:full-basic}
\end{table}

% LaTeX code for Dataframe 2:
\begin{table}[]
\centering
\resizebox{\textwidth}{!}{%
% LaTeX code for Dataframe (9, 12):
\begin{tabular}{lllll}
\toprule
Paper & Power Law Form & Purpose Of Power Law (E.G., Performance Prediction, Optimal Ratio) & \# Power Law Parameters & \# Of Scaling Laws \\
\midrule
\cite{rosenfeld2019constructive} & $\tilde{\epsilon}(m, n)=a n^{-\alpha}+b m^{-\beta}+c_{\infty}$ & Performance Prediction & 5-6 & 8 \\
\cite{mikamiscaling} & $L(n, s)=\delta\left(\gamma+n^{-\alpha}\right) s^{-\beta}$ & Performance Prediction & 4 & 3 \\
\cite{schaeffer2023emergent} & None & N/A & NA & NA \\
\cite{sardana2023beyond} & $L(N, D) = E + \frac{A}{N^\alpha} + \frac{B}{D^\beta} $; $N^*\left(\ell, D_{\text {inf }}\right), D_{\text {tr }}^*\left(\ell, D_{\text {inf }}\right)={\arg \min } _{N, D_{\mathrm{tr}} \mid L\left(N, D_{\mathrm{tr}}\right)=\ell}$ & Performance Prediction & 5 & 4 \\
\cite{sorscher2022beyond} & $c \cdot \alpha^{-\beta} ,  c \cdot \exp (-b \alpha)$ & Performance Prediction & 2 & 34 \\
\cite{caballero2022broken} & $y=a+\left(b x^{-c_0}\right) \prod_{i=1}^n\left(1+\left(\frac{x}{d_i}\right)^{1 / f_i}\right)^{-c_i * f_i}$ & Performance Prediction & 5+ & 100+ \\
\cite{besiroglu2024chinchilla} & $L(N, D) = E + \frac{A}{N^\alpha} + \frac{B}{D^\beta} $ & Performance Prediction & 5 & 1 \\
\cite{gordon2021data} & $L(N, D) = \left[ \left( \frac{N}{N_c}\right)^{\frac{\alpha_N}{\alpha_D}} + \frac{D}{D_c} \right]^{\alpha_D} $ & Performance Prediction & 4 & 3 \\
\cite{bansal2022data} & $L(D)=\alpha\left(D^{-1}+C\right)^p$ & Performance Prediction & 3 & 20 \\
\cite{hestness2017deep} & $\varepsilon(m) \sim \alpha m^{\beta_g}+\gamma$ & Performance Prediction & 3 & 17 \\
\cite{bi2024deepseek} & $\begin{aligned} M_{\mathrm{opt}} & =M_{\mathrm{base}} \cdot C^a \\ D_{\mathrm{opt}} & =D_{\mathrm{base}} \cdot C^b\end{aligned}$, $\begin{aligned} & \eta_{\mathrm{opt}}=0.3118 \cdot C^{-0.1250} \\ & B_{\mathrm{opt}}=0.2920 \cdot C^{0.3271}\end{aligned}$ & Optimal Ratio, Performance Prediction & 2 & 5 \\
\cite{bahri2021explaining} & $L(D) \propto D^{-\alpha_K}, \quad L(P) \propto P^{-\alpha_K}$ & Performance Prediction & 2 & 35 \\
\cite{geiping2022much} & $f(x)=a x^{-c}+b$, $v_{\text {Effective Extra Samples from Augmentations }}(x)=f_{\text {ref }}^{-1}\left(f_{\text {aug }}(x)\right)-x$ & Performance Prediction & 3 & ~50 \\
\cite{poli2024mechanistic} & $\log N^* \propto a \log C$ and $\log D^* \propto b \log C$ & Performance Prediction & 2 &  \\
\cite{hu2024minicpm} & $L(N, D)=C_N N^{-\alpha}+C_D D^{-\beta}+L_0$ & Performance Prediction & 5 & 6 \\
\cite{hashimoto2021model} & $\min _{\lambda, \alpha} \mathbb{E}_{\hat{q}, \hat{n}}\left[\left(\log (R(\hat{n}, \hat{q})-\epsilon)-\alpha \log (\hat{n})+\log \left(C_\lambda(\hat{q})\right)\right)^2\right]$ $R(\hat{n}, \hat{q})=\mathbb{E}\left[\ell\left(\hat{\theta}\left(p_{\hat{n}, \hat{q}}\right) ; x, y\right)\right]$ & Performance Prediction & 2+n(data mixes) & 4 \\
\cite{ruan2024observational} & $E_m \approx h \sigma\left(\beta^{\top} S_m+\alpha\right)$ & Performance Prediction & 3 &  \\
\cite{anil2023palm} & $N^{\star}(C) \approx N_0^{\star} \cdot C^a$ & Performance Prediction & 2 & 1 \\
\cite{pearce2024reconciling} & $N^*_{{\setminus E}} = b C_{{\setminus E}}^m$ $L = bC^m$ & Optimal Ratio, Performance Prediction & 2 & 1 \\
\cite{cherti2023reproducible} & $E=\beta C^{\alpha}$ & Performance Prediction & 2 & 8 \\
\cite{porian2024resolving} & $N^{\star}(C) \approx N_0^{\star} \cdot C^a$ & Optimal Ratio & 2 & 6 \\
\cite{alabdulmohsin2022revisiting} & $\varepsilon_x=\beta x^c$; $\varepsilon_x - \varepsilon_\infty=\beta x^c$; $\varepsilon_x=\beta (x^{-1} + \gamma)^{-c}$;   $\varepsilon_x=\gamma(x)(1+\gamma(x))^{-1} \varepsilon_0+(1+\gamma(x))^{-1} \varepsilon_{\infty}$ & Performance Prediction & 2-4 & ~600 \\
\cite{gao2024scalingevaluatingsparseautoencoders} & $L(n, k)=\exp \left(\alpha+\beta_k \log (k)+\beta_n \log (n)+\gamma \log (k) \log (n)\right)+\exp (\zeta+\eta \log (k))$ & Performance Prediction & 2-6 & 1 \\
\cite{muennighoff2024scaling} & $L\left(U_N, U_D, R_N, R_D\right)=\frac{A}{\left(U_N+U_N R_N^*\left(1-e^{\frac{-R_N}{R_N^*}}\right)\right)^\alpha}+\frac{B}{\left(U_D+U_D R_D^*\left(1-e^{\frac{-R_D}{R_D^*}}\right)\right)^\beta}+E$ & Performance Prediction & 2 (+4) & 1 \\
\cite{rae2021scaling} & None & Performance Prediction & N/A & N/A \\
\cite{shin2023scaling} & None & Scaling trend & NA & NA \\
\cite{hernandez2022scaling} & $E=k * N^\alpha$ & Optimal Ratio & 2 & 1 \\
\cite{filipovich2022scaling} & $\mathcal{L}(C)=\left(C_c C\right)^{\alpha_C}$ & Performance Prediction & 2 & 3 \\
\cite{neumann2022scaling} & $N_{\text {opt }}(C)=\left(\frac{C}{C_0}\right)^{\alpha_C^{o p t}}$, $E_i=\frac{1}{1+\left(N_j / N_i\right)^{\alpha_N}}$ & Performance Prediction & 2 & 3 * 2 \\
\cite{droppo2021scaling} & $L(N, D)=\left[\left(L_{\infty}\right)^{\frac{1}{\alpha}}+\left(\frac{N_C}{N}\right)^{\frac{\alpha_N}{\alpha}}+\left(\frac{D_C}{D}\right)^{\frac{\alpha_D}{\alpha}}\right]^\alpha$ & Performance Prediction & 6 & 3 \\
\cite{henighan2020scaling} & $L(x)=L_{\infty}+\left(\frac{x_0}{x}\right)^{\alpha_x}$ & Performance Prediction & 3 & 36 \\
\cite{goyal2024scaling} & $y_k=a \cdot n_1^{b_1} \prod_{j=2}^k\left(\frac{n_j}{n_{j-1}}\right)^{b_j}+d$ & Performance Prediction & 2+ 2*n(data mixes) & 1 \\
\cite{aghajanyan2023scaling} & $L(N, D_j)=E_j + \frac{A_j}{N^{\alpha_j}} + \frac{B_j}{|D_j|^{\beta_j}}$, $L(N, D_i, D_j) = [\frac{L(N, D_i) + L(N, D_j)}{2}] - C_{i,j} + \frac{A_{i,j}}{N^{\alpha_{i,j}}} + \frac{B_{i,j}}{|D_i|+|D_j|^{\beta_{i,j}}}$ & Performance Prediction & 5 & 14 \\
\cite{kaplan2020scaling} & $L(N, D) = \left[ \left( \frac{N}{N_c}\right)^{\frac{\alpha_N}{\alpha_D}} + \frac{D}{D_c} \right]^{\alpha_D}$     & Performance Prediction & 4 & ~7 \\
\cite{ghorbani2021scaling} & $\mathrm{BLEU}=c_B L^{-p_B}$, $\hat{L}_{o p t}(B)=\alpha^* B^{-\left(p_d+p_e\right)}+L_{\infty}, \quad \alpha^* \equiv \alpha\left(\frac{\bar{N}_e\left(p_e+p_d\right)}{p_e}\right)^{p_e}\left(\frac{\bar{N}_d\left(p_e+p_d\right)}{p_d}\right)^{p_d}$ & Optimal Ratio, Performance Prediction & 6 & ~8 \\
\cite{gao2023scaling} & $\begin{aligned} & R_{\mathrm{bo} n}(d)=d\left(\alpha_{\mathrm{bo} n}-\beta_{\mathrm{bo} n} d\right), \\ & R_{\mathrm{RL}}(d)=d\left(\alpha_{\mathrm{RL}}-\beta_{\mathrm{RL}} \log d\right)\end{aligned}$ & Performance Prediction & 2 & 2 \\
\cite{hilton2023scaling} & $I^{-\beta}=\left(\frac{N_c}{N}\right)^{\alpha_N}+\left(\frac{E_c}{E}\right)^{\alpha_E}$ & Optimal Ratio, Performance Prediction & 5 & 3 \\
\cite{frantar2023scaling} & $L(S, N, D)=\left(a_S(1-S)^{b_S}+c_S\right) \cdot\left(\frac{1}{N}\right)^{b_N}+\left(\frac{a_D}{D}\right)^{b_D}+c$ & Optimal Ratio, Performance Prediction & 7 & 2 \\
\cite{prato2021scaling} & $\begin{aligned} & \operatorname{Err}(N)=\operatorname{Err}_{\infty}+k N^\alpha, \\ & \operatorname{Err}(C)=\operatorname{Err}_{\infty}+k C^\alpha,\end{aligned}$ & Performance Prediction & 3 & 12 \\
\cite{covert2024scaling} & $\log \left|\psi_k(z)\right| \approx \log |c(z)|-\alpha(z) \log (k)$ & Performance Prediction & 2 & Many \\
\cite{hernandez2021scaling} & $L \approx\left[\left(\frac{N_C}{N}\right)^{\frac{\alpha_N}{\alpha_D}}+\frac{D_C}{k\left(D_F\right)^\alpha(N)^\beta}\right]^{\alpha_D}$ & Performance Prediction & 3 & 1 \\
\cite{ivgi2022scaling} & NS & Performance Prediction & NA & NA \\
\cite{tay2022scaling} & None & Scaling trend & NA & NA \\
\cite{tao2024scaling} & $N_{\mathrm{v}}^{\mathrm{opt}}=N_{\mathrm{v}}^0 *\left(\frac{N_{\mathrm{nv}}}{N_{\mathrm{nv}}^0}\right)^\gamma$, $\mathcal{L}_u=-E+\frac{A_1}{N_{\mathrm{nv}}^{\alpha_1}}+\frac{A_2}{N_{\mathrm{v}}^{\alpha_2}}+\frac{B}{D^\beta}$ & Optimal Ratio, Performance Prediction & 7 & 2 \\
\cite{jones2021scaling} & $\begin{aligned} \text { plateau } & =m_{\text {boardsize }}^{\text {plateau }} \cdot \text { boardsize }+c^{\text {plateau }} \\ \text { incline } & =m_{\text {boardsize }}^{\text {incline }} \cdot \text { boardsize }+m_{\text {flops }}^{\text {incline }} \cdot \log \text { flop }+c^{\text {incline }} \\ \text { elo } & =\text { incline.clamp }(\text { plateau }, 0)\end{aligned}$ & Performance Prediction & 5 & 1 \\
\cite{zhai2022scaling} & $E=\alpha+\beta(C+\gamma)^{-\mu}$ & Performance Prediction & 4 & 3 \\
\cite{dettmers2023case} & None & Scaling trend & NA & NA \\
\cite{dubey2024llama} & $N^{\star}(C)=A C^\alpha$. & Optimal Ratio  & 2 & 2 \\
\cite{hoffmann2022training} & A3: $L(N, D) = E + \frac{A}{N^\alpha} + \frac{B}{D^\beta} $ & Optimal Ratio, Performance Prediction & 5 & 3 \\
\cite{ardalani2022understanding} & $\left(\alpha x^{-\beta}+\gamma\right)$ & Performance Prediction & 3 & 3 \\
\cite{clark2022unified} & $\log L(N, E) \triangleq+a \log N+b \log E+c \log N \log E+d$ & Performance Prediction & 4 & 3 \\
\bottomrule
\end{tabular}
}

\caption{Details on power law for each paper surveyed.}
\label{tab:full-powerlaw}
\end{table}

% LaTeX code for Dataframe 2:
\begin{table}[]
\centering
\resizebox{\textwidth}{!}{%
% LaTeX code for Dataframe 3:
% LaTeX code for Dataframe (13, 19):
\begin{tabular}{llllllll}
\toprule
Paper & Training Runs / Law & Max. Training Flops & Max. Training Params & Max. Training Data & Data Described? & Hyperparameters Described? & How Are Model Params Counted \\
&&&&&&& (E.G., W/ Or W/Out Embeddings) \\
\midrule
\cite{rosenfeld2019constructive} & 42-49 &  & 0.7M-70M & 100M words / 1.2M images & Y & Y & Non-embedding \\
\cite{mikamiscaling} & 7 &  & ResNet-101 & 64k-1.28M images & Y & Y & NA \\
\cite{schaeffer2023emergent} & 4 &  & $10^{11}$ & NA & Y & NA & Non-embedding \\
\cite{sardana2023beyond} & 47 &  & 150M-6B & 1.5B-1.25T tokens & N & Y & NA \\
\cite{sorscher2022beyond} & ~60 &  & 86M (ViT) & 200 epochs & Y & Y & NA \\
\cite{caballero2022broken} & 3-40 &  & NS & NS & N & N & NS \\
\cite{besiroglu2024chinchilla} & NA & NA & NA & NA & Y & NA & Non-embedding \\
\cite{gordon2021data} & 45-55 &  & 56M & 28.3M-51.1M examples & Y & Y & Non-embedding \\
\cite{bansal2022data} & 10 &  & 170M-800M & 500K-512M sentences (28B tokens) & Y & Y & NS \\
\cite{hestness2017deep} & ~9 &  & upto 193M  & $2^{19}-2^{28}$ tokens, upto $2^9$ images, $2k$ audio hours & Y & Y & NS \\
\cite{bi2024deepseek} & 80 & $1e17-3e20$ &  &  & Y & Y & Non-embedding \\
\cite{bahri2021explaining} & 8-27 &  & 36.5M & upto 78k steps; 100 epochs & Y & Y & NS \\
\cite{geiping2022much} & 13 &  & ResNet-18 & upto 7.6M images & Y & Y & NS \\
\cite{poli2024mechanistic} & 500 total & 8.00E+19 & 70M-7B &  & Y & Y & Non-embedding  \\
\cite{hu2024minicpm} & 36 &  & 40M-2B & 400M-120B tokens & Y & Y & Non-embedding  \\
\cite{hashimoto2021model} &  &  &  & upto 600k sentences & Y & Y & NA \\
\cite{ruan2024observational} & 27* -77* &  & 70B-180B & 3T-6T tokens & N/A & N/A & N.S. \\
\cite{anil2023palm} & 12 & 1.00E+22 & 15B & 4.00E+11 & N & N & Non-embedding \\
\cite{pearce2024reconciling} & 20 (simulated), 25 (real) &  & 1.5B (simulated), 4.6M (real) & 23B (simulated), 500M (real) tokens & Y & Y & w/ Embedding and Non-embedding considered separately \\
\cite{cherti2023reproducible} & 3* - 29 &  & 214M & 34B (pretrain), 2B (finetune) examples  & Y & Y & N.S \\
\cite{porian2024resolving} & 16 & 2.00E+19 & 901M &  & Y & Y & w/ Embedding and Non-embedding considered separately \\
\cite{alabdulmohsin2022revisiting} & 1* &  & 110M-1B & 1e6-1e10 ex / 3e11 tokens & Mixed & N & N/A \\
\cite{gao2024scalingevaluatingsparseautoencoders} & N.S & N.S & N.S & N.S & N & N & N.S \\
\cite{muennighoff2024scaling} & 142 &  & 8,7B & 900B tokens & Y & Y & w/ embedding \\
\cite{rae2021scaling} & 4 & 6.31E+23 & 280B &  & Y & Y & Non-embedding \\
\cite{shin2023scaling} & 17 & ~0.1 PF Days & 160M & 500M-50B tokens & Y & Y & NA \\
\cite{hernandez2022scaling} & 56 &  & 1.5M-800M & 100B tokens & N & N & NS \\
\cite{filipovich2022scaling} & 4 &  & 57-509M & 30B token & Y & N & NS \\
\cite{neumann2022scaling} & 14 &  & ~$5*10^5$ & $10^4$ steps & Y & Y & NS \\
\cite{droppo2021scaling} & 5-21 &  & ~$10^7$ & 134-23k hrs speech & Y & Y & NS \\
\cite{henighan2020scaling} & 6-10 &  & ~$10^11$ & ~$10^12$ tokens & Y & Y & Non-embedding \\
\cite{goyal2024scaling} & 5 &  & CLIP L/14 - ~300M +63M & 32-640M samples & Y & Y & Embedding \\
\cite{aghajanyan2023scaling} & 21 &  & 8M-6.7B & 5-100B tokens & Y & Y & Non-embedding \\
\cite{kaplan2020scaling} & ~40-150 &  & 1.5B & 23B tokens & Y & Y & Non-embedding \\
\cite{ghorbani2021scaling} & 12-14 &  & 191-3B & NS & Y & Y & Non-embedding \\
\cite{gao2023scaling} & 9 &  & 3B & 120-90k & N & Y & NS \\
\cite{hilton2023scaling} & NS & $10^{20}$ &  &  & Y & Y & NS \\
\cite{frantar2023scaling} & 48 and 112 &  & 0.66M-85M  & 1.8B images, 65B tokens & Y & Y & Non-embedding \\
\cite{prato2021scaling} & 5 &  &  & $10^6$ samples & Y & N & NA \\
\cite{covert2024scaling} & 10 &  & NA & 1000 samples for IMDB & Y & Y & NA \\
\cite{hernandez2021scaling} & NS & $10^{21}$ & $10^8$ &  & Y & N & Non-embedding \\
\cite{ivgi2022scaling} & 5-8 &  & $10^4-10^8$ & varies; 500k steps PT & Y & Y & Non-embedding \\
\cite{tay2022scaling} &  &  & 16-30B & $2^19$ & Y & Y & NA \\
\cite{tao2024scaling} & 60 &  & 33M-1.13B NV + 4-96k V & 4.3B-509B Characters & Y & Y & Embedding and Non-embedding considered separately \\
\cite{jones2021scaling} & 200 & 1E+12-1E+17 &  & 4E+08-2E+09 & Y & Y & NA \\
\cite{zhai2022scaling} & 44 &  & 5.4M-1.8B & 1-13M images & Y & Y & NA \\
\cite{dettmers2023case} & 4 &  & 19M-176B & NA & NA & Y & NA \\
\cite{dubey2024llama} & NS & $6*10^{18}-10^22$ & 40M-16B &  & Y* & Y & NS \\
\cite{hoffmann2022training} & ~200-450 & $6*10^{18}-3*10^{21}$ & 16B & 5B-400B tokens & Y & Y & Non-embedding \\
\cite{ardalani2022understanding} & NS & $10^2$-$10^6$ TFlops &  & ~5M-5B samples & N & N & All are considered \\
\cite{clark2022unified} & 56 &  & 15M-1.3B & 130B tokens & Y & Y & Non-embedding \\
\bottomrule
\end{tabular}

}

\caption{Details on training setup for each paper surveyed.}
\label{tab:full-setup}
\end{table}

% LaTeX code for Dataframe 2:
\begin{table}[]
\centering
\resizebox{\textwidth}{!}{%

% LaTeX code for Dataframe (20, 24):
\begin{tabular}{lllll}
\toprule
Paper & Data Points Per Law? & Scaling Law Metric & Modification Of Final Metric? & Subsets Of Data Used \\
\midrule
\cite{rosenfeld2019constructive} & 42-49 & Loss / Top1 Error & N & N \\
\cite{mikamiscaling} & 7 & Error Rate & N & N \\
\cite{schaeffer2023emergent} & NA & Various downstream & NA & NA \\
\cite{sardana2023beyond} & NS & Loss & NS & NS \\
\cite{sorscher2022beyond} & ~60 & Error Rate & NA & NA \\
\cite{caballero2022broken} & 3-40 & FID, Loss, Error Rate, Elo Score & N & NS \\
\cite{besiroglu2024chinchilla} & 245 & Loss & N & N \\
\cite{gordon2021data} & 45-55 & Loss & N & N \\
\cite{bansal2022data} & NS & Loss, BLEU & NS & NS \\
\cite{hestness2017deep} & NS & Token Error, CER, Error Rate, Loss & Median min. validation error across multiple training runs with separate random seeds & NS \\
\cite{bi2024deepseek} & upto 80 & Validation bits-per-byte & NS & NS \\
\cite{bahri2021explaining} & upto 100 & Loss & NS & NS \\
\cite{geiping2022much} & ~50 & Effective Extra Samples & Interpolation & NS \\
\cite{poli2024mechanistic} & NS & Loss & NS & NS \\
\cite{hu2024minicpm} & NS & Loss & NS & NS \\
\cite{hashimoto2021model} & NS & Loss & NS & NS \\
\cite{ruan2024observational} &  & Various downstream & N & N \\
\cite{anil2023palm} & 12 & Loss & N & N \\
\cite{pearce2024reconciling} & 20, 5 & Loss & N & N \\
\cite{cherti2023reproducible} & 3-29 & Error Rate & N & N \\
\cite{porian2024resolving} & 12 & Loss & N & N \\
\cite{alabdulmohsin2022revisiting} & N.S. & Loss / Accuracy & N & N/A \\
\cite{gao2024scalingevaluatingsparseautoencoders} & N.S & MSE & N.S & N.S \\
\cite{muennighoff2024scaling} & 142 & Loss & N & Outliers removed \\
\cite{rae2021scaling} & 4 & Loss & N/A & N/A \\
\cite{shin2023scaling} & NA & Loss & NA & NA \\
\cite{hernandez2022scaling} & NS & Loss & N & N \\
\cite{filipovich2022scaling} & NS & Loss & N & N \\
\cite{neumann2022scaling} & 238 & Elo Score & N & N \\
\cite{droppo2021scaling} & NS & Loss & N & N \\
\cite{henighan2020scaling} & NS & Loss, Error Rate & NS & Drop smaller models \\
\cite{goyal2024scaling} & NS & Error Rate & N & N \\
\cite{aghajanyan2023scaling} & NS & Perplexity & N & N \\
\cite{kaplan2020scaling} & NS & Loss & NS & NS \\
\cite{ghorbani2021scaling} & NS & Loss, BLEU & Median of last 50k steps &  \\
\cite{gao2023scaling} & ~90 & RM Score & NS & NS \\
\cite{hilton2023scaling} & NS & Intrinsic Performance & Smoothing learning curve & Exclude early checkpoints \\
\cite{frantar2023scaling} & 48 and 112 & Loss & NS & NS \\
\cite{prato2021scaling} & 5 & Error Rate & NS & NS \\
\cite{covert2024scaling} & (1000-5000 )*10 & Expectation  & NS & N \\
\cite{hernandez2021scaling} & 40-120 & Loss & NS & NS \\
\cite{ivgi2022scaling} & 5-8 & Loss & N & [2.5, 97.5] percentile \\
\cite{tay2022scaling} & NA & Loss, Accuracy & NA & NA \\
\cite{tao2024scaling} & 20*60 & Loss & Interpolation & NS \\
\cite{jones2021scaling} & 2800 & Elo Score & NS & NS \\
\cite{zhai2022scaling} & NS & Accuracy & NS & NS \\
\cite{dettmers2023case} & NA & Accuracy & NA & NA \\
\cite{dubey2024llama} & ~150 & Loss, Accuracy & NS & NS \\
\cite{hoffmann2022training} & upto 1500 & Loss & N & Lowest loss model of a FLOP count, last 15\% of checkpoints \\
\cite{ardalani2022understanding} & ~130 & Loss & NS & NS \\
\cite{clark2022unified} & ~26*56 & Loss & Log & NS \\
\bottomrule
\end{tabular}

}

\caption{Details on data extraction for each paper surveyed.}
\label{tab:full-eval}
\end{table}




\begin{table}[]
\centering
\resizebox{\textwidth}{!}{%
% LaTeX code for Dataframe (25, 29):
\begin{tabular}{llllll}
\toprule
Paper & Curve-Fitting Method & Loss Objective & Hyperparameters Reported? & Initialization & Are Scaling Laws Validated? \\
\midrule
\cite{rosenfeld2019constructive} & Least Squares Regression & Custom error term & N/A & Random & Y \\
\cite{mikamiscaling} & Non-linear Least Squares in log-log space &  & N/A & N/A & Y \\
\cite{schaeffer2023emergent} & NA & NA & NA & NA & NA \\
\cite{sardana2023beyond} & L-BFGS & Huber Loss & Y & Grid Search & N \\
\cite{sorscher2022beyond} & NA & NA & NA & NA & NA \\
\cite{caballero2022broken} & Least Squares Regression & MSLE & N/A & Grid Search, optimize one & Y \\
\cite{besiroglu2024chinchilla} & L-BFGS & Huber Loss & Y & Grid Search & Y \\
\cite{gordon2021data} & Least Squares Regression &  & N/A & N.S. & N \\
\cite{bansal2022data} & NS & NS & N & NS & N \\
\cite{hestness2017deep} & NS & RMSE & N & NS & Y \\
\cite{bi2024deepseek} & NS & NS & N & NS & Y \\
\cite{bahri2021explaining} & NS & NS & N & NS & N \\
\cite{geiping2022much} & Non-linear Least Squares &  & NA & Non-augmented parameters & Y \\
\cite{poli2024mechanistic} & NS & NS & N & NS & N \\
\cite{hu2024minicpm} & scipy curvefit & NS & N & NS & N \\
\cite{hashimoto2021model} & Adagrad & Custom Loss & Y & Xavier & Y \\
\cite{ruan2024observational} & Linear Least Squares & Various & N/A & N/A & Y \\
\cite{anil2023palm} & Polynomial Regression (Quadratic) & N.S. & N & N.S. & Y \\
\cite{pearce2024reconciling} & Polynomial Least Squares & MSE on Log-loss & N/A & N/A & N \\
\cite{cherti2023reproducible} & Linear Least Squares & MSE & N/A & N/A & N \\
\cite{porian2024resolving} & Weighted Linear Regression & weighted SE on Log-loss & N/A & N/A & Y \\
\cite{alabdulmohsin2022revisiting} & Least Squares Regression & MSE & Y & N.S. & Y \\
\cite{gao2024scalingevaluatingsparseautoencoders} & N.S & N.S & N.S & N.S & N.S \\
\cite{muennighoff2024scaling} & L-BFGS & Huber on Log-loss & Y & Grid Search, optimize all & Y \\
\cite{rae2021scaling} & None & None & N/A & N/A & N \\
\cite{shin2023scaling} & NA & NA & NA & NA & NA \\
\cite{hernandez2022scaling} & NS & NS & NS & NS & NS \\
\cite{filipovich2022scaling} & NS & NS & NS & NS & NS \\
\cite{neumann2022scaling} & NS & NS & NS & NS & NS \\
\cite{droppo2021scaling} & NS & NS & NS & NS & NS \\
\cite{henighan2020scaling} & NS & NS & NS & NS & NS \\
\cite{goyal2024scaling} & Grid Search & L2 error & Y & NA & Y \\
\cite{aghajanyan2023scaling} & L-BFGS & Huber on Log-loss & Y & Grid Search, optimize all & Y \\
\cite{kaplan2020scaling} & NS & NS & NS & NS & N \\
\cite{ghorbani2021scaling} & Trust Region Reflective algorithm, Least Squares & Soft-L1 Loss & Y & Fixed & Y \\
\cite{gao2023scaling} & NS & NS & NS & NS & Y \\
\cite{hilton2023scaling} & CMA-ES+Linear Regression & L2 log loss & Y & Fixed & Y \\
\cite{frantar2023scaling} & BFGS & Huber on Log-loss & Y & N Random Trials & Y \\
\cite{prato2021scaling} & NS & NS & NS & NS & NS \\
\cite{covert2024scaling} & Adam & Custom Loss & Y & NS & Y \\
\cite{hernandez2021scaling} & NS & NS & NS & NS & Y \\
\cite{ivgi2022scaling} & Linear Least Squares in Log-Log space & MSE & NA & NS & Y \\
\cite{tay2022scaling} & NA & NA & NA & NA & NA \\
\cite{tao2024scaling} & L-BFGS, Least Squares & Huber on Log-loss & Y & N Random Trials from Grid & Y \\
\cite{jones2021scaling} & L-BFGS & NS & NS & NS & NS \\
\cite{zhai2022scaling} & NS & NS & NS & NS & NS \\
\cite{dettmers2023case} & NA & NA & NA & NA & NA \\
\cite{dubey2024llama} & NS & NS & NS & NS & Y \\
\cite{hoffmann2022training} & L-BFGS & Huber on Log-loss & Y & Grid Search, optimize all & Y \\
\cite{ardalani2022understanding} & NS & NS & NS & NS & NS \\
\cite{clark2022unified} & L-BFGS-B & L2 Loss & Y & Fixed & NS \\
\bottomrule
\end{tabular}

}

\caption{Details on optimization for each paper surveyed.}
\label{tab:full-opt}
\end{table}