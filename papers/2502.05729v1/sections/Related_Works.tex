
\section{Related Works}
\label{sec:related_work}
% The development of Bangla TTS technology highlights the intricacies of synthesizing a language rich in morphology and phonetic diversity. Katha \cite{alam2007text}, the first Bangla TTS system, was introduced utilizing diphone concatenation within the Festival Framework. However, challenges were encountered by this initial model in achieving natural prosody and efficient runtime. Subsequent enhancements, such as the Subhachan system \cite{naser2010implementation}, aimed to improve these aspects but still grappled with similar limitations. The complexity of Bangla's diphthongs and consonant-vowel structures \cite{hossain2022lila,rashid2010text} necessitated innovative approaches. Deep Learning based models, including the use of LSTM-RNNs \cite{gutkin2016tts} and syllable-based  \cite{ahmed2019syllable} have shown promising results, indicating a pathway towards more effective synthesis. Furthermore, an end-to-end Bangla synthesizer  \cite{bhattacharjee2021end} was introduced that bypasses the need for grapheme-to-phoneme (G2P) conversion, simplifying the synthesis pipeline. 

% Some comprehensive work on Indian language includes Bangla for developing generic Indic-TTS \cite{Prakash2020}  using Tacotron2 for text to mel-spectrogram conversion and WaveGlow for  vocoder and  \cite{indictts2022} showed monolingual models with FastPitch and HiFi-GAN V1, trained jointly on male and female speakers to perform the best. But there is no recent work incorporating Bangla using LLM based framework and speaker adaptive model. For this resoon, we investigate the performance of speaker adpative and LLM based multilingual XTTS model 

% % \\
% % \\
% % \\

% The development of Bangla TTS technology shows the challenges of creating speech from a language with rich grammar and diverse sounds. Katha \cite{alam2007text}, the first Bangla TTS system, used diphone concatenation within the Festival Framework. However, this early model struggled with making speech sound natural and running efficiently. Later improvements, like the Subhachan system \cite{naser2010implementation}, tried to fix these issues but still faced similar problems. The complexity of Bangla's diphthongs and consonant-vowel patterns \cite{hossain2022lila, rashid2010text} required new and creative methods. 

% Some broader work on Indian languages, including Bangla, contributed to the development of generic Indic-TTS systems. For example, \citet{Prakash2020} used Tacotron2 for text-to-mel-spectrogram conversion and WaveGlow for the vocoder. Another study from \citet{indictts2022} showed that monolingual models with FastPitch and HiFi-GAN V1, trained on both male and female voices, performed better result. However, there is no recent work that incorporates Large Language Model (LLM) frameworks for Bangla speaker adaptation in low resource setting.

% For this reason, we explore the performance of the XTTS model on the Bangla language to develop the first Bangla TTS system in a low-resource setting.

The development of Bangla TTS technology presents unique challenges due to the language's rich morphology and phonetic diversity. The first Bangla TTS system, Katha \cite{alam2007text}, was developed using diphone concatenation within the Festival Framework. However, this approach struggled with natural prosody and efficient runtime. Later advancements, such as Subhachan \cite{naser2010implementation}, aimed to improve these aspects but still faced similar limitations. The introduction of LSTM-based models \cite{gutkin2016tts} showed promising results in Bangla speech synthesis. Beyond Bangla-specific TTS, broader efforts on Indian language synthesis have contributed to Indic-TTS systems. \citet{Prakash2020} employed Tacotron2 for text-to-mel-spectrogram conversion and WaveGlow as a vocoder. Another study \cite{indictts2022} demonstrated that monolingual models utilizing FastPitch and HiFi-GAN V1, trained on both male and female voices, outperformed previous approaches. However, these works supported a limited number of speakers and lacked speaker adaptability. To address this gap, we explore the LLM-based XTTS model for Bangla, developing the first Bangla TTS system designed for low-resource speaker adaptation.



% . \citet{fan2014tts} highlighted the potential of statistical parametric speech synthesis (SPSS) methods, demonstrating their advantages for low-resourced languages. Quantization-Aware training \cite{krishnamoorthi2018quantizing} has been used to generate high-quality Bangla speech \cite{shahjahanprecision}.



% The development of Bangla Text-To-Speech (TTS) technology has faced challenges due to the language’s morphological and phonetic complexity. Katha \cite{alam2007text}, the first Bangla TTS system based on diphone concatenation, struggled with natural prosody and efficient runtime. Later systems, such as Subhachan \cite{naser2010implementation}, aimed to improve these aspects but still encountered issues with diphthongs and consonant-vowel structures \cite{hossain2022lila,rashid2010text}. More recent approaches, including LSTM-RNNs \cite{gutkin2016tts} and syllable-based models \cite{ahmed2019syllable}, have shown improved synthesis performance.
% An end-to-end Bangla synthesizer \cite{bhattacharjee2021end} removed the need for G2P conversion, streamlining the process. Statistical parametric speech synthesis (SPSS) methods \cite{fan2014tts} have also demonstrated promise for low-resourced languages. However, challenges persist in achieving fast inference times with high-quality, natural-sounding audio. Recently, Quantization-Aware Training \cite{krishnamoorthi2018quantizing} has been applied to enhance Bangla speech quality \cite{shahjahanprecision}.