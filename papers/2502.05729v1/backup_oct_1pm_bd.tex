
\documentclass[11pt]{article}

% Change "review" to "final" to generate the final (sometimes called camera-ready) version.
% Change to "preprint" to generate a non-anonymous version with page numbers.
\usepackage[review]{acl}

% Standard package includes
\usepackage{times}
\usepackage{latexsym}


% For proper rendering and hyphenation of words containing Latin characters (including in bib files)
\usepackage[T1]{fontenc}
% For Vietnamese characters
% \usepackage[T5]{fontenc}
% See https://www.latex-project.org/help/documentation/encguide.pdf for other character sets

% This assumes your files are encoded as UTF8
\usepackage[utf8]{inputenc}

% This is not strictly necessary, and may be commented out,
% but it will improve the layout of the manuscript,
% and will typically save some space.
\usepackage{microtype}

% This is also not strictly necessary, and may be commented out.
% However, it will improve the aesthetics of text in
% the typewriter font.
\usepackage{inconsolata}

%Including images in your LaTeX document requires adding
%additional package(s)
\usepackage{graphicx}


\usepackage{algorithm}
\usepackage{algorithmicx}
\usepackage{algpseudocode}
% \usepackage[linesnumbered, ruled, vlined]{algorithm2e}
\usepackage{tabularray}
\usepackage{caption}
\usepackage{subcaption}
% \usepackage{lmodern}
\usepackage{amsmath}
\usepackage{multirow}

% ,amsmath,graphicx, spconf}
% \usepackage{minted}

% If the title and author information does not fit in the area allocated, uncomment the following
%
%\setlength\titlebox{<dim>}
%
% and set <dim> to something 5cm or larger.

\title{BnTTS: Few-Shot Speaker Adaptation in Low-Resource Setting}

% Author information can be set in various styles:
% For several authors from the same institution:
% \author{Author 1 \and ... \and Author n \\
%         Address line \\ ... \\ Address line}
% if the names do not fit well on one line use
%         Author 1 \\ {\bf Author 2} \\ ... \\ {\bf Author n} \\
% For authors from different institutions:
% \author{Author 1 \\ Address line \\  ... \\ Address line
%         \And  ... \And
%         Author n \\ Address line \\ ... \\ Address line}
% To start a separate ``row'' of authors use \AND, as in
% \author{Author 1 \\ Address line \\  ... \\ Address line
%         \AND
%         Author 2 \\ Address line \\ ... \\ Address line \And
%         Author 3 \\ Address line \\ ... \\ Address line}

\author{First Author \\
  Affiliation / Address line 1 \\
  Affiliation / Address line 2 \\
  Affiliation / Address line 3 \\
  \texttt{email@domain} \\\And
  Second Author \\
  Affiliation / Address line 1 \\
  Affiliation / Address line 2 \\
  Affiliation / Address line 3 \\
  \texttt{email@domain} \\}

%\author{
%  \textbf{First Author\textsuperscript{1}},
%  \textbf{Second Author\textsuperscript{1,2}},
%  \textbf{Third T. Author\textsuperscript{1}},
%  \textbf{Fourth Author\textsuperscript{1}},
%\\
%  \textbf{Fifth Author\textsuperscript{1,2}},
%  \textbf{Sixth Author\textsuperscript{1}},
%  \textbf{Seventh Author\textsuperscript{1}},
%  \textbf{Eighth Author \textsuperscript{1,2,3,4}},
%\\
%  \textbf{Ninth Author\textsuperscript{1}},
%  \textbf{Tenth Author\textsuperscript{1}},
%  \textbf{Eleventh E. Author\textsuperscript{1,2,3,4,5}},
%  \textbf{Twelfth Author\textsuperscript{1}},
%\\
%  \textbf{Thirteenth Author\textsuperscript{3}},
%  \textbf{Fourteenth F. Author\textsuperscript{2,4}},
%  \textbf{Fifteenth Author\textsuperscript{1}},
%  \textbf{Sixteenth Author\textsuperscript{1}},
%\\
%  \textbf{Seventeenth S. Author\textsuperscript{4,5}},
%  \textbf{Eighteenth Author\textsuperscript{3,4}},
%  \textbf{Nineteenth N. Author\textsuperscript{2,5}},
%  \textbf{Twentieth Author\textsuperscript{1}}
%\\
%\\
%  \textsuperscript{1}Affiliation 1,
%  \textsuperscript{2}Affiliation 2,
%  \textsuperscript{3}Affiliation 3,
%  \textsuperscript{4}Affiliation 4,
%  \textsuperscript{5}Affiliation 5
%\\
%  \small{
%    \textbf{Correspondence:} \href{mailto:email@domain}{email@domain}
%  }
%}

\begin{document}
\maketitle
\begin{abstract}
Text-to-Speech (TTS) systems have achieved remarkable success in generating high-quality speech, yet most advancements focus on high-resource languages, leaving low-resource languages like Bangla underserved. This paper presents BnTTS, the first open-source framework and the first to enable speaker adaptation for Bangla TTS, aiming to bridge the gap in Bangla speech synthesis with minimal training data. Our approach builds upon the XTTS architecture, integrating Bangla into a multilingual TTS pipeline with adjustments for the phonetic and linguistic nuances of the language. BnTTS supports zero-shot and few-shot speaker adaptation, significantly enhancing the naturalness, clarity, and speaker fidelity of synthesized Bangla speech. We evaluate BnTTS against state-of-the-art Bangla TTS systems, demonstrating superior performance in terms of speed, Subjective Mean Opinion Score (SMOS), and audio quality. This work marks a critical advancement towards inclusive TTS technology for low-resource languages, paving the way for broader linguistic accessibility in speech synthesis.
\end{abstract}



%In this paper, we present the BnTTS, an extension of the XTTS framework specifically focusing on synthesizing Bangla, a low-resource language. While XTTS demonstrates state-of-the-art performance across 16 high- and medium-resource languages, it lacks support for Bangla and other low-resource languages. To bridge this gap, we adapt the XTTS architecture to effectively model Bangla speech, enabling zero-shot multi-speaker text-to-speech (ZS-TTS) synthesis for the language. Our approach addresses the challenges associated with low-resource data by leveraging transfer-learning techniques and optimizing voice cloning for Bangla speakers. The results demonstrate significant improvements in naturalness and speaker similarity in Bangla, highlighting the potential for extending TTS capabilities to underrepresented languages.



% Recent advancements in text-to-speech (TTS) \cite{popov2021gradtts,kim2021conditional} have led to the development of speaker-adaptive models \cite{le2023voicebox,kim2023pflow} that can closely replicate target voices. These models are typically divided into zero-shot and one-shot approaches. Zero-shot TTS \cite{le2023voicebox,ju2024naturalspeech3} eliminates the need for additional training, but often requires large datasets and can struggle with out-of-distribution (OoD) voices. In contrast, one-shot TTS \cite{yan2021adaspeech2,wang2023neural} fine-tunes pre-trained models, offering improved adaptability to new speakers while reducing data and model size demands.
\section{Introduction \& Related Works}
Speaker adaptation in Text-to-Speech (TTS) technology has seen substantial advancements in recent years, particularly with speaker-adaptive models enhancing the naturalness and intelligibility of synthesized speech \cite{eren2023deep}. Notably, recent innovations have emphasized zero-shot and one-shot adaptation approaches \cite{kodirov2015unsupervised}. Zero-shot TTS models eliminate the need for speaker-specific training by generating speech from unseen speakers using reference audio samples \cite{min2021meta}. Despite this progress, zero-shot models often require large datasets and face challenges with out-of-distribution (OOD) voices, as they struggle to adapt effectively to novel speaker traits \cite{le2023voicebox, ju2024naturalspeech3}. Alternatively, one-shot adaptation fine-tunes pre-trained models using a single data instance, offering improved adaptability with reduced data and computational demands \cite{yan2021adaspeech2, wang2023neural}; however, the pretraining stage still necessitates substantial datasets \cite{zhang2021transfer}.

Recent works such as YourTTS \cite{bai2022yourtts} and VALL-E X \cite{xu2022vall} have made strides in cross-lingual zero-shot TTS, with YourTTS exploring English, French, and Portuguese, and VALL-E X incorporating language identification to extend support for a broader range of languages \cite{xu2022vall}. These advancements highlight the potential for multilingual TTS systems to achieve cross-lingual speech synthesis. Furthermore, the XTTS model \cite{casanova2024xtts} represents a significant leap by expanding zero-shot TTS capabilities across 16 languages. Based on the Tortoise model \cite{casanova2024xtts}, XTTS enhances voice cloning accuracy and naturalness but remains focused on high- and medium-resource languages, leaving low-resource languages such as Bangla underserved \cite{zhang2022universal, xu2023cross}. 

The scarcity of extensive datasets has hindered the adaptation of state-of-the-art (SOTA) TTS models for low-resource languages. Models like YourTTS \citet{bai2022yourtts}, VALL-E X \cite{baevski2022vall}, and Voicebox \cite{baevski2022voicebox} have demonstrated success in multilingual settings, yet their primary focus remains on languages with rich resources like English, Spanish, French, and Chinese. While a few Bangla TTS systems exist \cite{gutkin2016tts}, they often produce robotic tones \cite{hossain2018development} or are limited to a small set of static speakers \cite{gong2024initial}, lacking instant speaker adaptation capabilities and typically not being open-source.

To address these challenges, we propose the first open-source framework for few-shot speaker adaptation in Bangla TTS. Our approach integrates Bangla into the XTTS training pipeline, with minor architectural modifications to accommodate Bangla’s unique phonetic and linguistic features. This contribution marks a pivotal step for zero-shot TTS in Bangla, enabling natural-sounding speech generation with minimal training data. Building on XTTS’s foundational achievements, our model is optimized for effective few-shot voice cloning, addressing the needs of low-resource language settings.

By extending XTTS to support Bangla, we provide a valuable resource for developing TTS systems in underrepresented languages. Our model enhances voice cloning capabilities and sets the stage for expanding zero-shot TTS to other low-resource languages, thereby promoting greater linguistic inclusivity in speech synthesis technology. Our contributions are summarized as follows:
\begin{itemize}
    \item Introducing the first open-source framework and first Bangla few-shot speaker adaptation in TTS.
    \item Integrating Bangla into a multilingual TTS pipeline with architecture adjustments designed for low-resource language characteristics.
    \item Advancing few-shot TTS for low-resource languages, achieving superior speed, clarity, naturalness, and higher SMOS compared to existing Bangla TTS models.
    
\end{itemize}

% \section{Engines}

% To produce a PDF file, pdf\LaTeX{} is strongly recommended (over original \LaTeX{} plus dvips+ps2pdf or dvipdf). Xe\LaTeX{} also produces PDF files, and is especially suitable for text in non-Latin scripts.



\begin{figure}[h]
    \centering
    \includegraphics[width=1.001\linewidth]{resources/method2.png} 
    \caption{Overview of BnTTS Model} 
    \label{fig:xtts_train_diagram}
\end{figure}

\section{BnTTS}
\textbf{Preliminaries:} Given a text sequence with $N$ tokens, $\mathbf{T} = \{t_1, t_2, \dots, t_N\}$, and a speaker's mel-spectrogram $\mathbf{S} = \{s_1, s_2, \dots, s_L\}$, our objective is to generate speech $\hat{\mathbf{Y}}$ that matches the speaker's characteristics. Let $\mathbf{Y} = \{y_1, y_2, \dots, y_M\}$ denote the ground truth mel-spectrogram frames for the target speech. 

The framework aims to synthesize $\hat{\mathbf{Y}}$ directly from $\mathbf{T}$ and $\mathbf{S}$, such that:
\[
\hat{\mathbf{Y}} = \mathcal{F}(\mathbf{T}, \mathbf{S})
\]
where $\mathcal{F}$ is the model responsible for producing speech conditioned on both the text and the speaker's spectrogram.

\textbf{VQ-VAE: } Our method uses a Vector Quantized-Variational AutoEncoder (VQ-VAE) \cite{tortoise} to encode mel-spectrogram frames $\mathbf{Y}$ into discrete codes. The VQ-VAE encoder maps $\mathbf{Y}$ to a list of discrete codes with finite vocabulary.
% latent vectors $\mathbf{Y_z} \in \mathbb{R}^{M \times d}$, where $d$ is the dimensionality of the model.

\textbf{Conditioning Encoder \& Perceiver Resampler:}
The Conditioning Encoder consists of $l$ layers of $k$-head Scaled Dot-Product Attention, followed by a Perceiver Resampler. Each attention layer processes the input features to capture both local and global contexts.

Given the speaker spectrogram $\mathbf{S}$, it is first transformed into an intermediate representation $\mathbf{S_z} \in \mathbb{R}^{L \times d}$. Each attention layer applies a scaled dot-product attention mechanism:
\[
\text{Attention}(\mathbf{Q}, \mathbf{K}, \mathbf{V}) = \text{softmax}\left(\frac{\mathbf{Q}\mathbf{K}^\top}{\sqrt{d}}\right) \mathbf{V}
\]
where $\mathbf{Q}$, $\mathbf{K}$, and $\mathbf{V}$ are projections of $\mathbf{S_z}$, computed over $k$ attention heads. The output from the final attention layer is passed through a Perceiver Resampler to produce a fixed output dimensionality.

The Perceiver Resampler generates a fixed number of embeddings $\mathbf{S_p} \in \mathbb{R}^{d}$. This resampling step ensures that the final output, $\mathbf{R} \in \mathbb{R}^{P \times d}$, has a fixed size $P$, independent of the variable input length $L$.


\textbf{Text Encoder:} This step consists of a simple embedding layer that projects the input text tokens $\mathbf{T} = \{t_1, t_2, \dots, t_N\}$ into a continuous embedding space. The tokens are mapped to embeddings $\mathbf{T_e} \in \mathbb{R}^{N \times d}$, where $d$ is the embedding dimension.

\textbf{LLM:}
In this framework, we utilize the decoder portion of the transformer-based LLM. The  speaker embeddings $\mathbf{S_p}$, text embeddings $\mathbf{T_e}$, and ground truth spectrogram embeddings $\mathbf{Y_z}$ are concatenated to form the input during training. Let $\mathbf{X}$ denote this combined input:
\[
\mathbf{X} = \mathbf{S_p}  \oplus \mathbf{T_e} \oplus \mathbf{Y_z}
\]
where $\oplus$ represents concatenation, which results in $\mathbf{X} \in \mathbb{R}^{(N + P + M) \times d}$.

The LLM processes $\mathbf{X}$ through its layers, producing an output from the final layer denoted by $\mathbf{H} =  \{ h_1^S, h_2^S, \dots, h_M^S, h_1^T, h_2^T, \dots, h_N^T, h_1^Y, h_2^Y, \dots, h_P^Y \}$, where $\mathbf{H} \in \mathbb{R}^{(N + P + M) \times d}$. Each component of $\mathbf{H}$ corresponds to the hidden states for the text, speaker, and spectrogram embeddings.

During inference, only the text and speaker embeddings are concatenated, forming the input as:
\[
\mathbf{X} = \mathbf{S_p} \oplus \mathbf{T_e}
\]
The LLM then generates spectrogram embeddings $\{h_1^Y, h_2^Y, \dots, h_P^Y\}$, which serve as the predicted output speech representation.

\textbf{HiFi-GAN Decoder: }
The HiFi-GAN Decoder converts the LLM's output into realistic speech, preserving the speaker's unique characteristics. During training, it uses the ground truth spectrogram representation, whereas in inference, it uses the predicted spectrogram. Let $\mathbf{H}_\text{Y} = \{h_1^Y, h_2^Y, \dots, h_P^Y\} \in \mathbb{R}^{P \times d}$ represent spectrogram embedding.

The speaker embedding $\mathbf{S}$ is resized to match $\mathbf{H}_\text{Y}$, resulting in $\mathbf{S}' \in \mathbb{R}^{P \times d}$. We then combine these by adding them element-wise and feeding them into the HiFi-GAN decoder, which produces the final audio waveform $\mathbf{W}$:
\[
\mathbf{W} = g_\text{HiFi}(\mathbf{H}_\text{Y} + \mathbf{S}')
\]

In this way, HiFi-GAN decoder generates speech that reflects the input text while maintaining the speaker’s unique sound qualities.

\textbf{Training Objectives: } Our training process incorporates three different objectives. The first is the text generation objective, which helps the model predict future tokens based on the given text, improving its understanding of the sequence and aiding in advanced speech prediction. The second objective is audio generation, which aligns the synthesized audio with the input text. The third is the HiFi-GAN objective, which learns to transform the LLM's latents to speech signals via adversarial training. Details on each objective function can be found in Appendix E.


\section{Experiments}
\textbf{Model Architecture}
The BnTTS framework utilizes GPT-2 as the base model, selected due to resource constraints, although any LLM could be applied. The Conditioning Encoder consists of six attention blocks, each with 32 heads, to effectively capture contextual information. The Perceiver Resampler processes the output from the Conditioning Encoder to a fixed sequence length of 32. The model's dimensionality aligns with GPT-2, using a hidden size of 1024 and an intermediate layer size of 3072. The maximum sequence length is set to 384 tokens. Further details on the model architecture can be found in Appendix F.

\textbf{Training Dataset: }The BnTTS model was trained on approximately 3,860 hours of Bengali data, sourced from open-source datasets, pseudo-labeled data, synthetic datasets, and in-house data. Since Bengali is a low-resource language with limited large-scale, high-quality TTS data, we developed an automated in-house TTS Data Acquisition Framework. This framework collects high-quality speech data with aligned transcripts by utilizing neural speech processing models and custom algorithms that refine raw audio into word-aligned outputs. For further details about the training dataset, refer to Appendix A, and for more on the Data Acquisition Framework, see Appendix B.

\textbf{Evaluation Dataset: }To evaluate the performance of our TTS system, we used two curated datasets: BnStudioEval  and BnTTSTextEval. The BnStudioEval dataset, derived from in-house studio dataset, assesses the model’s ability to produce high-fidelity speech with accurate speaker impersonation. Conversly, the BnTTSTextEval dataset contains three subsets: BengaliStimuli, which tests the model's handling of diverse phonemes; BengaliNamedEntity1000, focused on conversational accuracy with real-world names and entities; and ShortText200, which evaluates performance in everyday dialogue scenarios. Detailed statistics and descriptions of these datasets are provided in Appendix C.

\textbf{Training setup: }For Bengali XTTS pretraining, we used the AdamW optimizer with betas 0.9 and 0.96, weight decay 0.01, and an initial learning rate of 2e-05 with a batch size equal to 12 with grad accumulation equal to 24 steps for each GPU. We also decayed the learning rate using using MultiStepLR by a gamma of 0.66 using the milestones equaling to completion of an epoch. We have pretrained the model for 15 days on a single A100 80GB GPU. The subsequent fine-tuning took around 30 minutes to 1 hour depending on duration of fine-tuning data available for speaker impersonation.



\textbf{Evaluation Metric: }We evaluated our TTS system using five criteria. The Subjective Mean Opinion Score (SMOS) rates the absolute quality of synthetic speech in isolation, providing insights into perceived audio quality.  To assess transcription accuracy, we used ASR-based Character Error Rate (CER), which compares the speech-to-text output of the synthesized speech with its ground truth transcription \cite{nandi-etal-2023-pseudo}. SpeechBERTScore evaluates the similarity between generated and reference speech by computing BERTScore using self-supervised dense speech features, accounting for variations in sequence length \cite{saeki2024spbertscore}. Finally, Speaker Encoder Cosine Similarity (SECS)\cite{casanova2021sc} objectively measures the resemblance between the speaker characteristics of synthesized and reference speech, ensuring fidelity in speaker identity (\citet{thienpondt2024ecapa2}). See Appendix D for the details of evaluation metrics. 

\textbf{Results and Discussion: }
\\
\textbf{Reference-aware Evaluation on BnStudioEval: }
Table \ref{tab:eval_on_studio_eval} shows the performance of various TTS systems on BnStudioEval dataset. AzureTTS performs best among synthetic methods, with the CER score, even surpassing the Ground Truth (GT) in transcription accuracy. However, GT scores highest on subjective metrics such as SMOS (4.671), Naturalness (4.625), and Clarity (4.9). The proposed BnTTS system closely follows GT, with competitive scores in SMOS (4.584), Naturalness (4.531), and Clarity (4.898).

For speaker similarity, GT achieves perfect SECS scores against reference audio and high scores against the speaker prompt. BnTTS also performs well, with SECS scores of 0.513 (reference) and 0.335 (prompt), just slightly behind GT. BnTTS achieves a SpeechBERTScore of 0.796, while GT maintains a perfect reference score of 1.0. Note that IndicTTS, GTTS, and AzureTTS do not support speaker impersonation, so SECS and SpeechBERTScore were not evaluated for these systems.

\textbf{Reference-independent Evaluation on BnTTSTextEval:} Table \ref{tab:eval_on_BnTTSTextEval} presents the comparative performance of various TTS systems evaluated on the BnTTSTextEval dataset. The AzureTTS and GTTS consistently achieve lower CER scores, with BnTTS following closely in third place, and IndicTTS trailing behind.

BnTTS performs strongly in subjective evaluations, excelling in SMOS, Naturalness, and Intelligibility across the BengaliStimuli53 and BengaliNamedEntity1000 subsets. However, it falls slightly behind AzureTTS in the ShortText200 subset. Despite this, BnTTS overall, remains the top-performing system in all subjective metrics, delivering the highest scores in SMOS(4.383), Naturalness(4.313), and Clarity(4.737).


\textbf{Effect of Sampling and Prompt Length on BnTTS Accuracy:} In this section, we investigate how different configurations of temperature ($T$), top-$k$ sampling, and prompt length (short or long) impact the CER and duration equality of the BnTTS model on the short texts(less than 30 characters) from BnStudioEval dataset. Table~4 provides a comparative performance analysis under four experimental settings, with the goal of understanding how these parameters contribute to transcription accuracy and duration similarity. The results indicate that higher temperatures (i.e., $T=1.0$) combined with lower Top-$k$ values (i.e., $k=2$) improve CER, especially when using short prompts. Short prompts consistently result in better transcription accuracy and duration equality, likely due to their reduced complexity and the model's ability to generate focused outputs. The combination of these parameters offers a promising approach for achieving both accurate and temporally aligned speech synthesis in low-resource languages such as Bengali.


\begin{table}[hbt!]
\centering
\footnotesize
\setlength{\tabcolsep}{2.9pt}
\resizebox{0.47\textwidth}{!}{%

\begin{tabular}{c|c c c c c}
\hline
\textbf{Method} & \textbf{GT} & \textbf{IndicTTS} & \textbf{GTTS} & \textbf{AzureTTS} & \textbf{BnTTS} \\ 
\hline
CER & \textit{0.037} & 0.067 & 0.022 & \textbf{0.019} & 0.036 \\ 
SMOS & \textit{4.671} & 3.038 & 3.788 & 4.083 & \textbf{4.584} \\ 
Naturalness & \textit{4.625} & 2.857 & 3.573 & 3.941 & \textbf{4.531} \\ 
Clarity & \textit{4.9} & 3.945 & 4.869 & 4.796 & \textbf{4.898} \\ 
SECS (Ref.) & \textit{1.0} & - & - & - & \textbf{0.513} \\ 
SECS (Prompt) & \textit{0.361} & - & - & - & \textbf{0.335} \\ 
SpeechBERT Score & \textit{1.0} & - & - & - & \textbf{0.796} \\ 
\hline
\end{tabular}}
\caption{Comparative performance analysis on Reference-aware BnStudioEval dataset}
\label{tab:eval_on_studio_eval}
\end{table}
\begin{table}[h!]
\centering
\footnotesize
\setlength{\tabcolsep}{2.9pt}
\resizebox{0.47\textwidth}{!}{%
\begin{tabular}{c|c|c c c c}
\hline
\textbf{Dataset} & \textbf{Method} & \textbf{CER} & \textbf{SMOS} & \textbf{Naturalness} & \textbf{Clarity} \\ 
\hline
Bengali-& IndicTTS & 0.115 & 2.618 & 2.469 & 3.360 \\ 
                                                      Stimulai-& GTTS     & \textbf{0.072} & 3.654 & 3.448 & 4.698 \\ 
                                                      53& Azure TTS & 0.077 & 3.845 & 3.712 & 4.507 \\ 
                                                      & BnTTS  & 0.090 & \textbf{4.485} & \textbf{4.417} & \textbf{4.825} \\ 
\hline
Bengali-& IndicTTS & 0.051 & 3.058 & 2.867 & 4.014 \\ 
                                                             Named-& GTTS     & 0.042 & 3.881 & 3.680 & 4.713 \\ 
                                                             Entity-& Azure TTS & \textbf{0.039} & 4.105 & 3.972 & 4.733 \\ 
                                                             1000& BnTTS  & 0.054 & \textbf{4.376} & \textbf{4.304} & \textbf{4.740} \\ 
\hline
Short-& IndicTTS & 0.600 & 2.486 & 2.387 & 2.983 \\ 
                                                 Text-& GTTS     & 0.328 & 4.039 & 3.882 & 4.881 \\ 
                                                 200& Azure TTS & \textbf{0.259} & \textbf{4.421} & \textbf{4.350} & \textbf{4.774} \\ 
                                                 & BnTTS  & 0.491 & 4.276 & 4.206 & 4.627 \\ 
\hline
Overall& IndicTTS & 0.145 & 2.806 & 2.649 & 3.593 \\ 
                         & GTTS     & 0.071 & 3.865 & 3.674 & 4.821 \\ 
                         & Azure TTS & \textbf{0.062} & 4.122 & 4.005 & 4.701 \\ 
                         & BnTTS  & 0.100 & \textbf{4.383} & \textbf{4.313} & \textbf{4.737} \\ 
\hline
\end{tabular}}
\caption{Comparative performance analysis on Reference-independent BnTTSTextEval dataset}
\label{tab:eval_on_BnTTSTextEval}
\end{table}

\begin{table}[hbt!]
\centering
\resizebox{0.4\textwidth}{!}{%
\setlength{\tabcolsep}{2.0pt}

\begin{tabular}{c|cccc} 
\hline
\textbf{Exp. No} & \textbf{T and TopK} & \begin{tabular}[c]{@{}c@{}}\textbf{Short}\\\textbf{Prompt}\end{tabular} & \begin{tabular}[c]{@{}c@{}}\textbf{Duration}\\\textbf{Equality}\end{tabular} & \textbf{CER}  \\ 
\hline
1                & T=0.85, TopK=50     & N                                                                       & 0.699                                                                        & 0.081         \\ 

2                & T=0.85, TopK=50     & Y                                                                       & 0.820                                                                        & 0.029         \\ 

3                & T=1.0, TopK=2       & N                                                                       & 0.701                                                                        & 0.023         \\ 

4                & T=1.0, TopK=2       & Y                                                                       & 0.827                                                                        & 0.015         \\
\hline
\end{tabular}}
\caption{Comparative performance analysis on Short-BnStudioEval Dataset}
\label{tab:eval_on_short_Studio_eval}
\end{table}


\section{Conclusion}

In this study, we introduce BnTTS, the first open-source speaker adaptation based TTS system, which improves speech generation for Bangla, a low-resource language. By adapting the XTTS pipeline to accommodate Bangla's unique phonetic characteristics, the model is able to produce natural, clear, and accurate speech with minimal training data, supporting both zero-shot and few-shot speaker adaptation. BnTTS outperforms existing Bangla TTS systems in terms of speed, sound quality, and clarity, as confirmed by listener ratings. 

\section{Limitations}

BnTTS has some limitations. It relies on a small and uniform dataset, which limits its ability to work well with the diverse dialects and accents in Bangla, potentially affecting the naturalness of the speech output for certain regional variations. The use of GPT-2, chosen due to limited resources, may also limit the system’s scalability and performance compared to newer models. Additionally, the system struggles with adapting to speakers with unique vocal traits, especially without prior training on their voices. Its focus on Bangla also restricts its usefulness for other low-resource languages. Furthermore, XTTS, the foundation for BnTTS, performs poorly on very short sequences. While adjustments to hyperparameters and generation settings (temperature and TopK) have mostly resolved this issue, instances remain where the model struggles to generate sequences under 2 words or 20 characters. Future improvements could involve a larger dataset, more advanced models, and multilingual support to boost its versatility and robustness.

\clearpage
\section{Ethical Considerations}

The development of BnTTS raises ethical concerns, particularly regarding the potential misuse for unauthorized voice impersonation, which could impact privacy and consent. Protections, such as requiring speaker approval and embedding markers in synthetic speech, are essential. Diverse training data is also crucial to reduce bias and reflect Bangla’s dialectal variety. Additionally, synthesized voices risk diminishing dialectal diversity. As an open-source tool, BnTTS requires clear guidelines for responsible use, ensuring adherence to ethical standards and positive community impact.


\bibliography{custom}



\newpage
\appendix
% \clearpage
% \section{Appendix}

% App A: Dataset
% App B: Data Aquization Framework
% C: Eval Data
% D: Eval Metrics
% E: Model Architecture
% F:Training Objective

\begin{table*}
\label{datasetdescription}
\centering
\caption{Overview of the evaluation dataset, where M denotes malware and B denotes benign applications.}
\begin{tabular}{c|c|c|c|c|c|c|c|c} 
\hline
           & \textbf{Time Interval} & \textbf{Sample Size} & \begin{tabular}[c]{@{}c@{}}\textbf{The number of}\\\textbf{Existing family}\end{tabular} & \begin{tabular}[c]{@{}c@{}}\textbf{The number of}\\\textbf{New family}\end{tabular} & \textbf{Packed} & \textbf{Malicious} & \textbf{Benign} & \textbf{M/(M+B)\%}  \\ 
\hline
Test set 1 & 2020.05 - 2021.01 & 3015  & 21                                                               & 24                                                          & 18     & 284       & 2731   & 9.42                       \\ 
\hline
Test set 2 & 2021.01 - 2021.12 & 3015      & 28                                                               & 32                                                          & 30     & 298       & 2717   & 9.88                      \\ 
\hline
Test set 3 & 2021.12 - 2023.12 & 3016      & 34                                                               & 36                                                          & 40     & 302       & 2714   & 10.01                       \\
\hline                  
\end{tabular}
\end{table*}

\section{TTS Data Acquisition Framework}
\label{sec:data_collection}

\begin{figure}[hbt!]
    \centering
    \includegraphics[width=0.8\linewidth]{resources/TTS_Data_Collection_Pipeline.png} 
    \caption{Overview of our TTS Data Acquisition Framework. The acquisition process involves using a Speech-to-Text model to obtain transcription, an LLM to restore transcription's punctuation, a noise suppression model to remove unwanted noise, and finally an audio superresolution model to enhance audio quality and loudness.}
    \label{fig:pseudo_labeled_dataset}
\end{figure}

Bangla is a low-resource language, and large-scale, high-quality TTS speech data are particularly scarce. To address this gap, we developed a TTS Data Acquisition Framework (Figure \ref{fig:pseudo_labeled_dataset}) designed to collect high-quality speech data with aligned transcripts. This framework leverages advanced speech processing models and carefully designed algorithms to process raw audio inputs and generate refined audio outputs with word-aligned transcripts. Below, we provide a detailed breakdown of the key components of the framework.


\textbf{1. Speech-to-Text (STT):} The audio files are first processed through an in-house our STT system, which transcribes the spoken content into text. The STT system used here is an enhanced version of the model proposed in \cite{nandi-etal-2023-pseudo}.

\textbf{2. Punctuation Restoration Using LLM:} Following transcription, a LLM is employed to restore appropriate punctuation \cite{openai2023gpt}. This step is crucial for improving grammatical accuracy and ensuring that the text is clear and coherent, aiding in further processing.

\textbf{3. Audio and Transcription Segmentation:} The audio and transcription are segmented based on terminal punctuation (full-stop, question mark, exclamatory mark, comma). This ensures that each audio segment aligns with a complete sentence, maintaining the speaker's prosody throughout.

\textbf{4. Noise and Music Suppression:} To improve audio quality, noise and music suppression techniques \cite{defossez2019music} are applied. This step ensures that the resulting audio is free of background disturbances, which could degrade TTS performance.

\textbf{5. Audio SuperResolution:} After noise suppression, the audio files undergo super-resolution processing to enhance audio fidelity \cite{liu2021voicefixer}. This ensures high-quality audio, crucial for producing natural-sounding TTS outputs.


This pipeline effectively enhances raw audio and corresponding transcription, resulting in a high-quality pseudo-labeled dataset. By combining ASR, LLM-based punctuation restoration, noise suppression, and super-resolution, the framework can generate very high-quality speech data suitable for training speech synthesis models.

\subsection{Dataset Filtering Criteria}
The pseudo-labeled data are further refined using the following criteria:

\begin{itemize}
    \item\textbf{Diarization:} Pyannote's Speaker Diarization v3.1  is employed to filter audio files by separating multi-speaker audios, ensuring that each instance contains only one speaker \cite{Plaquet23}, which is essential for effective TTS model training.

    \item \textbf{Audio Duration}: Audio segments shorter than 0.5 seconds are discarded, as they provide insufficient information for our model. Similarly, segments longer than 11 seconds are excluded to match the model’s sequence length.
    
    \item \textbf{Text Length}: Segments with transcriptions exceeding 200 characters are removed to ensure manageable input size during training.
    \item \textbf{Silence-based Filtering}: Audio files where over 35\% of the duration consists of silence are discarded, as they negatively impact model performance.
    \item \textbf{Text-to-Audio Ratio}: Based on our analysis, audio segments where the text-to-audio duration ratio falls outside (Figure \ref{fig:unprocessed_data}) the range of 6 to 25 are excluded (Figure \ref{fig:processed_data}), ensuring alignment with natural speech patterns observed in Pseudo-labeled data from Phase A (Figure \ref{fig:reviewed_data}).
\end{itemize}



\begin{figure}[hbt!]
    \centering
    % First Image: Unprocessed Data
    \begin{subfigure}[b]{0.45\textwidth}
        \centering
        \includegraphics[width=\textwidth]{resources/reviewd_data_100.png}
        \caption{The diagram illustrates the linear relationship between audio duration and character length in manually-reviewed Pseudo-labeled Data - Phase A.}
        \label{fig:reviewed_data}
    \end{subfigure}
    \hfill
    % Second Image: Processed Data
    \begin{subfigure}[b]{0.45\textwidth}
        \centering
        \includegraphics[width=\textwidth]{resources/700h_unprocessed.png}
        \caption{The diagram depicts the relationship between audio duration and character length in Pseudo-Labeled Data - Phase B.}
        \label{fig:unprocessed_data}
    \end{subfigure}
    \vskip\baselineskip
    % Third Image: Reviewd Data
    \begin{subfigure}[b]{0.45\textwidth}
        \centering
        \includegraphics[width=\textwidth]{resources/700h_processed.png}
        \caption{The diagram illustrates the audio duration vs. character length graph in Pseudo-Labeled Data - Phase B after filtering.}
        \label{fig:processed_data}
    \end{subfigure}
    \caption{These figures demonstrate how the ratio of text length to audio duration changes before and after processing the data.}
    \label{fig:audio_vs_length_grid}
\end{figure}


% \textcolor{red}{\subsection{Quality Control for Pseudo-Labeled data} These are already described in A1. Plese include phase A data review process by reviewer team} 
% Given the importance of data quality in the pseudo-labeling process, multiple manual and 
% automated verification steps are implemented to ensure accurate alignment between audio and 
% transcriptions. To address potential errors in automated methods, manual verification is conducted, focusing on key areas such as: 

% \noindent \textbf{Semantic Accuracy \& Punctuation Verification} This involves ensuring that the punctuation restoration process has preserved the intended meaning of the transcriptions and correcting any misinterpretations in punctuation placement.


% \noindent \textbf{Segmentation Accuracy}
% The process includes reviewing whether the sentence segmentation has been executed correctly, ensuring that transcriptions are chunked without breaking context. Additionally, manual adjustments are made to the segmentation whenever mis-segmentation errors are detected.


% \noindent \textbf{Speaker Diarization Validation} 
% This stage involves ensuring that each audio segment contains speech from only one speaker and does not contain overlapping speech. Additionally, any incorrectly diarized segments are identified and filtered out to maintain the clarity and accuracy of the speaker attribution in the dataset.

% \noindent \textbf{ASR Inference Correction} 
% The process includes checking if the ASR (Automatic Speech Recognition) model has misinterpreted words or phrases. Corrections are manually applied to the transcriptions for any inaccuracies identified, ensuring the accuracy of the transcribed text.


% \noindent \textbf{Audio Duration vs. Text Length Filtering}
% This involves applying a duration-to-text ratio filter to remove audio segments that are either too short or too long compared to their corresponding transcriptions. This step is critical to maintaining accurate audio-to-text alignment throughout the dataset.

\section{Human Guided Data Preparation}
\label{app:human_reviewed_data}
We curated approximately 82.39 hours of speech data through human-level observation, which we refer to as Pseudo-Labeled Data - Phase A (Table \ref{tab:dataset_info}). The audio samples, averaging 10 minutes in duration, are sourced from copyright-free audiobooks and podcasts, preferably featuring a single speaker in most cases.

Annotators were tasked with identifying prosodic sentences by segmenting the audio into meaningful chunks while simultaneously correcting ASR-generated transcriptions and restoring proper punctuation in the provided text. If a selected audio chunk contained multiple speakers, it was discarded to maintain dataset consistency. Additionally, background noise, mispronunciations, and unnatural speech patterns were carefully reviewed and eliminated to ensure the highest quality TTS training data.



\section{Evaluation Dataset}

For evaluating the performance of our TTS system, we curated two datasets: BnStudioEval and BnTTSTextEval, each serving distinct evaluation purposes.

\begin{itemize}
    \item \textbf{BnStudioEval}: This dataset comprises 100 high-quality instances (text and audio pair) taken from our in-house studio recordings. This dataset was selected to assess the model’s capability in replicating high-fidelity speech output with speaker impersonation. 
    
    \item \textbf{BnTTSTextEval}: The BnTTSTextEval dataset encompasses three subsets: \begin{itemize}
        \item \textbf{BengaliStimuli53}: A linguist-curated set of 53 instances, created to cover a comprehensive range of Bengali phonetic elements. This subset ensures that the model handles diverse phonemes.
        % Sample example: তোমাদের পড়াশুনা ধ্যানি ও গুরুমুখী নয়, বরং বাজারভিত্তিক।পূর্বসূরী অনেকের মত তোমরাও অনেকাংশে নিজের চিত্ত ও বোধকে গবেষণায় পূর্ণ করার চেয়ে ঘুষ খেয়ে ঘরে ফার্নিচারের মেলা বসাতে চাও, আর ঘৃত দিয়ে কিচেন।
        
        \item \textbf{BengaliNamedEntity1000}: A set of 1,000 instances focusing on proper nouns such as person, place, and organization names. This subset tests the model's handling of named entities, which is crucial for real-world conversational accuracy.
        % Sample Example: ময়মনসিংহ বিভাগের জেলাগুলো হচ্ছে শেরপুর, ময়মনসিংহ, জামালপুর, নেত্রকোণা।
        \item \textbf{ShortText200}: Composed of 200 instances, this subset includes short sentences  filler words, and common conversational phrases (less than three words) to evaluate the model’s performance in natural, day-to-day dialogue scenarios.
        % Sample Example: কি বলছেন?
    \end{itemize}  
\end{itemize}

The BnStudioEval dataset, with reference audio for each text, will be for reference-aware evaluation, while BnTTSTextEval supports reference-independent evaluation. Together, these datasets provide a comprehensive basis for evaluating various aspects of our TTS performance, including phonetic diversity, named entity pronunciation, and conversational fluency. 


% \section{Evaluation Metrics}
\label{app:eval_metrics}
We employed a combination of subjective and objective metrics to rigorously evaluate the performance of our TTS system, focusing on intelligibility, naturalness, speaker similarity, and transcription accuracy.

\noindent \textbf{Subjective Mean Opinion Score (SMOS):} SMOS is a perceptual evaluation where listeners rate synthesized speech on a Likert scale from 1 (poor) to 5 (excellent). It considers naturalness, clarity, and fluency, providing an absolute score for each sample. A higher SMOS indicates better overall speech quality.

\noindent \textbf{SpeechBERTScore:} SpeechBERTScore adapts BERTScore for speech, using self-supervised learning (SSL) models to compare dense representations of generated and reference speech. For generated speech waveform $\hat{X}$ and reference waveform $X$, the feature representations $\hat{Z}$ and $Z$ are extracted using a pretrained model. SpeechBERTScore is defined as the average maximum cosine similarity between feature vectors:
\[
\text{SpeechBERTScore} = \frac{1}{N_{\text{gen}}} \sum_{i=1}^{N_{\text{gen}}} \max_{j} \text{cos}(\hat{\mathbf{z}}_i, \mathbf{z}_j)
\]
where $\hat{\mathbf{z}}_i$ and $\mathbf{z}_j$ represent the SSL embeddings for generated and reference speech, respectively.

\noindent \textbf{Character Error Rate (CER):} CER measures transcription accuracy by calculating the ratio of errors (substitutions $S$, deletions $D$, and insertions $I$) in automatic speech recognition (ASR) transcriptions:
\[
CER = \frac{S + D + I}{N}
\]
where $N$ is the total number of characters in the reference transcription. A lower CER indicates better transcription accuracy.

\noindent \textbf{Speaker Encoder Cosine Similarity (SECS):} SECS evaluates speaker similarity by calculating the cosine similarity between speaker embeddings of the reference and synthesized speech:

\[
\text{SECS} = \frac{e_{\text{ref}} \cdot e_{\text{syn}}}{\|e_{\text{ref}}\| \|e_{\text{syn}}\|},
\]

where $e_{\text{ref}}$ and $e_{\text{syn}}$ are the speaker embeddings for reference and synthesized speech, respectively. SECS ranges from -1 (low similarity) to 1 (high similarity).

\label{sec:DurationEquality}
\noindent \textbf{Duration Equality Score:} This metric quantifies how closely the durations of the reference ($a$) and synthesized ($b$) speech match, with a score of 1 indicating identical durations:

\[
\text{DurationEquality}(a, b) = \frac{1}{\max\left(\frac{a}{b}, \frac{b}{a}\right)}.
\]

This score helps in assessing duration similarity between reference and generated audio, ensuring consistency in pacing.

Each metric provides a different perspective, allowing a holistic evaluation of the synthesized speech quality.


% \section{Evaluation Metrics}
% To rigorously evaluate the performance of the TTS system, a combination of subjective and objective metrics is employed. These metrics assess various dimensions of speech quality, including intelligibility, naturalness, speaker similarity, and transcription accuracy. The following describes the key evaluation metrics in detail:


% \textbf{Subjective Mean Opinion Score (SMOS)}: SMOS is a perceptual evaluation metric used to assess the overall quality of synthesized speech by human listeners. It is rated on a Likert scale from 1 (bad) to 5 (excellent), considering aspects such as naturalness, clarity, fluency, consistency, and emotional expressiveness. SMOS serves as an absolute rating, where each synthetic sample is evaluated independently, without reference to other samples. In addition to SMOS, separate scores for Naturalness and Clarity are reported for a comprehensive analysis.


% \textbf{SpeechBERTScore}:
% To evaluate the semantic consistency between the generated and reference speech in our proposed system, we employ the SpeechBERTScore metric, which extends the BERTScore framework, commonly used in text generation, to the speech domain by computing the similarity between dense speech representations derived from self-supervised learning (SSL) models. The metric aims to capture semantic congruence between the synthesized speech and a reference, accounting for differences in waveform length.

% Let the generated and reference speech waveforms be denoted as $\hat{X} = (\hat{x}_t \in \mathbb{R} \mid t = 1, \ldots, T_{\text{gen}})$ and $X = (x_t \in \mathbb{R} \mid t = 1, \ldots, T_{\text{ref}})$, respectively, where $T_{\text{gen}}$ and $T_{\text{ref}}$ represent the lengths of the generated and reference waveforms. To extract meaningful features from these waveforms, a pretrained SSL model is employed, which generates sequence representations $\hat{Z} = (\hat{\mathbf{z}}_n \in \mathbb{R}^D \mid n = 1, \ldots, N_{\text{gen}})$ and $Z = (\mathbf{z}_n \in \mathbb{R}^D \mid n = 1, \ldots, N_{\text{ref}})$ for the generated and reference speech, respectively:

% \[
% \hat{Z} = \text{Encoder}(\hat{X}; \theta), \quad Z = \text{Encoder}(X; \theta),
% \]

% where $\theta$ denotes the parameters of the pretrained encoder model, and $N_{\text{gen}}$ and $N_{\text{ref}}$ are determined by $T_{\text{gen}}$ and $T_{\text{ref}}$ based on the encoder's subsampling rate.

% The SpeechBERTScore is defined as the precision metric in the BERTScore framework, measuring the maximum cosine similarity between each feature vector in the generated speech and all feature vectors in the reference speech:

% \[
% \text{SpeechBERTScore} = \frac{1}{N_{\text{gen}}} \sum_{i=1}^{N_{\text{gen}}} \max_{j} \text{cos}(\hat{\mathbf{z}}_i, \mathbf{z}_j),
% \]

% where $\text{cos}(\hat{\mathbf{z}}_i, \mathbf{z}_j)$ is the cosine similarity between the SSL feature vectors $\hat{\mathbf{z}}_i$ from the generated speech and $\mathbf{z}_j$ from the reference speech.

% By leveraging pretrained SSL models, SpeechBERTScore captures high-level semantic information, making it suitable for evaluating synthesized speech's content and meaning. This metric is particularly advantageous for TTS evaluation, where semantic consistency and intelligibility are crucial, even in scenarios where the generated and reference audio lengths may differ.

% \textbf{Character Error Rate (CER)}: CER quantifies transcription accuracy by comparing the output of an automatic speech recognition (ASR) system on synthesized speech against a reference transcription. It is defined as:
% \[
% \text{CER} = \frac{S + D + I}{N}
% \]
% where $S$ is the number of substitutions, $D$ is the number of deletions, $I$ is the number of insertions, and $N$ is the total number of characters in the reference transcription. Lower CER values indicate higher transcription accuracy.

% \textbf{Speaker Encoder Cosine Similarity (SECS)}: SECS measures the speaker similarity between synthesized and reference speech by calculating the cosine similarity between their speaker embeddings:
% \[
% \text{SECS} = \frac{e_{\text{ref}} \cdot e_{\text{syn}}}{\|e_{\text{ref}}\| \|e_{\text{syn}}\|}
% \]
% where $e_{\text{ref}}$ and $e_{\text{syn}}$ are the speaker embeddings of the reference and synthesized speech, respectively. The similarity score ranges from -1 (low similarity) to 1 (high similarity), with higher values indicating closer resemblance in speaker characteristics.

% \textbf{DurationEquality Score}: DurationEquality quantifies the equality between two audio sample durations, \(a\) and \(b\), producing values between 0 and 1, where a score of 1 indicates identical durations. The metric is defined as:
% \begin{equation}
%     \text{DurationEquality}(a, b) = \frac{1}{\max\left(\frac{a}{b}, \frac{b}{a}\right)}
% \end{equation}
% This score approaches 1 as the durations of \(a\) and \(b\) become more equal, providing an effective measure of  discrepancy between duration of reference audio and synthesized audio .


\section{Subjective Evaluation}
For subjective evaluation of our system, we employ the Mean Opinion Score (MOS), a widely recognized metric primarily focusing on assessing the perceptual quality of audio outputs. To ensure the reliability and accuracy of our evaluations, we carefully select a panel of ten experts who are thoroughly trained in the intricacies of MOS scoring. These experts are equipped with the necessary skills and knowledge to critically assess and score the system, providing invaluable insights that help guide the refinement and enhancement of our technology. This structured approach guarantees that our evaluations are both comprehensive and precise, reflecting the true quality of the audio outputs under review.

\subsection{Evaluation Guideline}
For calculating MOS, we consider five essential evaluation criteria:
\begin{itemize} \item \textbf{Naturalness:} Evaluates how closely the TTS output resembles natural human speech. \item \textbf{Clarity:} Assesses the intelligibility and clear articulation of the spoken words. \item \textbf{Fluency:} Examines the smoothness of speech, including appropriate pacing, pausing, and intonation. \item \textbf{Consistency:} Checks the uniformity of voice quality across different texts. \item \textbf{Emotional Expressiveness:} Measures the ability of the TTS system to convey the intended emotion or tone. \end{itemize}

In the evaluation, we employ a five-point rating scale to meticulously assess performance based on specific criteria. This scale ranges from 1, denoting 'Bad' where the output has significant distortions, to 5, representing 'Excellent' where the output nearly replicates natural human speech and excels in all evaluation aspects. To capture more subtle nuances in the TTS output that might not perfectly fit into these whole-number categories, we also recommend using fractional scores. For example, a 1.5 indicates quality between 'Bad' and 'Poor,' a 2.5 signifies improvement over 'Poor' but not quite reaching 'Fair,' a 3.5 suggests better than 'Fair' but not up to 'Good,' and a 4.5 reflects performance that surpasses 'Good' but falls short of 'Excellent.' This fractional scoring allows for a more precise and detailed reflection of the system's quality, enhancing the accuracy and depth of the MOS evaluation.

\subsection{Evaluation Process}
We have developed an evaluation platform specifically designed for the subjective assessment of Text-to-Speech (TTS) systems. This platform features several key attributes that enhance the effectiveness and reliability of the evaluation process. Key features include anonymity of audio sources, ensuring that evaluators are unaware of whether the audio is synthetically generated or recorded from studio environment, or which TTS model, if any, was used. This promotes unbiased assessments based purely on audio quality. Comprehensive evaluation criteria allow evaluators to rate each audio sample on naturalness, clarity, fluency, consistency, and emotional expressiveness, ensuring a holistic review of speech synthesis quality. The user-centric interface is streamlined for ease of use, enabling efficient playback of audio samples and score entry, which reduces evaluator fatigue and maintains focus on the task. Finally, the structured data collection method systematically captures all ratings, facilitating precise analysis and enabling targeted improvements to TTS technologies. This platform is a vital tool for developers and researchers aiming to refine the effectiveness and naturalness of speech outputs in TTS systems.

\subsection{Evaluator Statistics}
For our evaluation process, we carefully selected 10 expert native speakers, achieving a balanced representation with 5 males and 5 females. The age range for these evaluators is between 20 to 28 years, ensuring a youthful perspective that aligns well with our target demographic. All evaluators are either currently enrolled as graduate students or have already completed their graduate studies. They hail from a variety of academic backgrounds, including economics, engineering, computer science, and social sciences, which provides a diverse range of insights and expertise. This careful selection of qualified individuals ensures a comprehensive and informed assessment process, suitable for our needs in evaluating advanced systems or processes where diverse, educated opinions are crucial.

\subsection{Subjective Evaluation Data Preparation} 
For reference-aware evaluation, we selected 20 audio samples from each of the four speakers, resulting in 80 Ground Truth (GT) audios. To facilitate comparison, we generated 400 synthetic samples (80 × 5) using the TTS systems examined in this study. Including the GT samples, the total dataset for this evaluation amounts to 480 audio files (400 + 80).

For the reference-independent evaluation, we utilized 453 text samples from BnTTSTextEval, comprising BengaliStimuli53 (53), BengaliNamedEntity1000 (200), and ShortText200 (200). Given the four speakers in both BnTTS-0 and BnTTS-n, this resulted in 3,624 audio samples (4 × 453 × 2). Additionally, IndicTTS, GTTS, and AzureTTS contributed 1,359 samples (3 × 453). IndicTTS samples were evenly distributed between two male and female speakers, while GTTS and AzureTTS used the "bn-IN-Wavenet-C" and "bn-IN-TanishaaNeural" voices, respectively.

In total, the reference-independent evaluation dataset comprised 5,436 audio samples. When combined with the 480 samples from the reference-aware evaluation, the overall dataset for subjective evaluation amounted to 5,916 audio files. These samples were randomly mixed and distributed to the reviewer team to ensure unbiased evaluations.

\section{Use of AI assistant}
\label{sec:use_of_ai_assistant}
We used AI assistants such as GPT-4o for spelling and grammar checking for the text of the paper.

\newpage

% \section{Potential Risks}
% \label{sec:use_of_potential risks}
% There are no potential risks associated with the outcomes of this research, as we do not utilize any sensitive information. Instead, this work will benefit the community by aiding the development of TTS systems for low-resource languages.



% 
% \section{Model Architecture}
% The model architecture consists of the following trainable components:

% % \textbf{VQ-VAE:}
% % The DiscreteVAE architecture consists of an encoder, decoder, and a quantization codebook. The encoder uses 2 Conv1d layers with strides of 2, reducing input dimensionality, followed by 3 residual blocks with 1024 channels, each with ReLU activations. The decoder mirrors the encoder, starting with a Conv1d layer, followed by 3 residual blocks and 2 upsampled convolution layers, reconstructing the original input. A Quantize layer is used for vector quantization. The architecture utilizes the DiscretizationLoss function for learning discrete latent representations. This module efficiently encodes and decodes spectrograms with a total parameter count of around 51 million.
% \textbf{Conditioning Encoder and Perceiver Resampler:}
% The Conditioning Encoder \cite{casanova2024xtts} consists of an initial Conv1d layer with 80 input channels and 1024 output channels, followed by 6 Attention blocks. Each Attention block includes a Group normalization layer (32 groups, 1024 dimensions), a Conv1d layer for query-key-value computation (1024 input channels, 3072 output channels), and a final projection Conv1d layer (1024 output channels). The attention mechanism utilizes QKV attention. Dropout with a probability of 0.1 is applied to facilitate regularization. The encoder outputs a sequence, which length is dependent on the input audio duration.

% The Conditioning Encoder is followed by the Perceiver Resampler, which produces a fixed number of embeddings by utilizing cross attention mechanism. The Perceiver Resampler is composed two attention blocks, each with 512-dimensional queries and 1024-dimensional keys and values. The module includes sequential layers with linear projections and GELU activations. For normalization, it uses RMS norm.
% The total number of parameters in Conditioning Encoder and Perceiver Resampler are approximately 25.29 million and 21 millions respectively.


% \textbf{LLM:}
% For LLM, we use a GPT-2 \cite{radford2019language} model with approximately 377.89 million parameters. The GPT-2 is consists of 30 transformer blocks, each with 16 attention heads and a hidden dimension of 1024. It uses layer normalization and attention mechanisms, with the MLP blocks containing two linear layers and a GELU activation.


% \textbf{HiFi-GAN Decoder:}
% The HiFi-GAN Decoder \cite{kong2020hifi} consists of a waveform generator with multiple convolutional layers and residual blocks. It includes 4 parametrized ConvTranspose1d layers for upsampling, followed by a series of residual blocks with various dilation rates. The total number of parameters is 25.86 million. This submodule is responsible for converting intermediate GPT-2 latent representations into high-quality waveform outputs


% \section{Objective Functions}
% \subsection{Language Modeling Loss}

% \textbf{Text Token Prediction} loss, denoted as $\mathcal{L}_{\text{text}}$, measures the discrepancy between the predicted text logits and the target text labels. Let $\hat{y}_{\text{text}}$ represent the predicted logits and $y_{\text{text}}$ the ground truth target labels. The text prediction loss is calculated as:

% \begin{equation}
% \mathcal{L}_{\text{text}} = \frac{1}{N} \sum_{i=1}^{N} \text{CE}(\hat{y}_{\text{text}}^{(i)}, y_{\text{text}}^{(i)}),
% \end{equation}

% where $\text{CE}$ denotes the cross-entropy loss, and $N$ is the number of training samples.

% \paragraph{Audio Token Prediction Loss}

% The second loss is Audio Token Prediction loss, $\mathcal{L}_{\text{mel}}$, evaluates the model's performance in generating acoustic Token that match the target VQ-VAE codes. It is defined as:

% \begin{equation}
% \mathcal{L}_{\text{audio}} = \frac{1}{N} \sum_{i=1}^{N} \text{CE}(\hat{y}_{\text{audio}}^{(i)}, y_{\text{audio}}^{(i)}),
% \end{equation}

% where $\hat{y}_{\text{audio}}$ represents the predicted logits for the audio token, and $y_{\text{audio}}$ are the corresponding target VQ-VAE tokens.


% The total loss used to train the model is a weighted sum of the text and audio losses:

% \begin{equation}
% \mathcal{L}_{\text{total}} = \alpha \mathcal{L}_{\text{text}} + \beta \mathcal{L}_{\text{audio}}
% \end{equation}

% where $\alpha$ and $\beta$ are scaling factors that control the relative importance of each loss term. This combined objective ensures that the model learns both the correct phonetic representations and acoustic features.

% $\alpha$ and $\beta$ are set 0.01  and 1.0 respectively.

% \subsection{Vocoder Loss}
% We used a HiFi-GAN-based vocoder \cite{kong2020hifi} that comprises multiple discriminators: the Multi-Period Discriminator, and Multi-Scale Discriminator. For the sake of clarity, we will refer to these discriminators as a single entity. The HiFi-GAN module is trained using a least squares loss rather than the conventional binary cross-entropy loss. The discriminator is tasked with classifying real audio samples as 1 and generated samples as 0, while the generator is optimized to produce audio that can deceive the discriminator into classifying it as close to 1. The adversarial losses for the generator \(G\) and the discriminator \(D\) are defined as follows:

% \begin{align}
%     \mathcal{L}_{\text{Adv}}(D; G) &= \mathbb{E}_{(x, s)} \left[(D(x) - 1)^2 + D(G(s))^2 \right], \\
%     \mathcal{L}_{\text{Adv}}(G; D) &= \mathbb{E}_{s} \left[(D(G(s)) - 1)^2 \right],
% \end{align}

% where \(x\) represents the real audio samples, and \(s\) denotes the input mel-spectrogram conditions.

% \paragraph{Mel-Spectrogram Loss}
% The model also employs L1 loss between the mel-spectrograms of the real and generated audio. This loss is formulated as:

% \begin{align}
%     \mathcal{L}_{\text{Mel}}(G) = \mathbb{E}_{(x, s)} \left[\left\| \phi(x) - \phi(G(s)) \right\|_{1}\right],
% \end{align}

% where \(\phi\) represents the transformation function that maps a waveform to its corresponding mel-spectrogram.

% \paragraph{Feature Matching Loss}
% The feature matching loss calculates the L1 distance between the intermediate features of the real and generated audio, as extracted from multiple layers of the discriminator. It is defined as:

% \begin{align}
%     \mathcal{L}_{\text{FM}}(G; D) = \mathbb{E}_{(x, s)} \left[\sum_{i=1}^{T} \frac{1}{N_i} \left\| D^i(x) - D^i(G(s)) \right\|_{1}\right],
% \end{align}

% where \(T\) denotes the number of discriminator layers, and \(D^i\) and \(N_i\) represent the features and number of features at the \(i\)-th layer, respectively.

% \paragraph{Final Loss}
% Given that the discriminator is composed of multiple sub-discriminators, the final objectives for training the generator and the discriminator are defined as follows::

% \begin{align}
%     \mathcal{L}_{G} &= \sum_{k=1}^{K} \left[\mathcal{L}_{\text{Adv}}(G; D_k) + \lambda_{\text{FM}} \mathcal{L}_{\text{FM}}(G; D_k)\right] + \lambda_{\text{Mel}} \mathcal{L}_{\text{Mel}}(G), \\
%     \mathcal{L}_{D} &= \sum_{k=1}^{K} \mathcal{L}_{\text{Adv}}(D_k; G),
% \end{align}

% where \(D_k\) denotes the \(k\)-th sub-discriminator and \(\lambda_{\text{FM}} = 2\), \(\lambda_{\text{Mel}} = 45\). 


\section{Training Objectives}
\label{app:training_objective}

Our BnTTS model is composed of two primary modules (GPT-2 and HiFi-GAN), which are trained separately. The GPT-2 module is trained using a Language Modeling objective, while the HiFi-GAN module is optimized using HiFi-GAN loss objective. This section provides an overview of the loss functions applied during training.

\subsection{Language Modeling Loss}
1. \textbf{Text Generation Loss}: Denoted as $\mathcal{L}_{\text{text}}$, it quantifies the difference between predicted logits and ground truth labels using cross-entropy. Let $\hat{y}_{\text{text}}$ represent the predicted logits and $y_{\text{text}}$ the ground truth target labels. For a sequence with $N$ text tokens, the Text Generation Loss is calculated as: 
   \begin{equation}
   \mathcal{L}_{\text{text}} = \frac{1}{N} \sum_{i=1}^{N} \text{CE}(\hat{y}_{\text{text}}^{(i)}, y_{\text{text}}^{(i)})
   \end{equation}
   
2. \textbf{Audio Generation Loss}: Denoted as $\mathcal{L}_{\text{audio}}$, it evaluates the accuracy of generated acoustic tokens against target VQ-VAE codes using cross-entropy loss:
   \begin{equation}
   \mathcal{L}_{\text{audio}} = \frac{1}{N} \sum_{i=1}^{N} \text{CE}(\hat{y}_{\text{audio}}^{(i)}, y_{\text{audio}}^{(i)})
   \end{equation}

where $\hat{y}_{\text{audio}}$ represents the predicted logits for the audio token, $y_{\text{audio}}$ are the corresponding target VQ-VAE tokens, and $N$ is the number of audio token in the sequence.
   
Total loss combines the text generation and audio generation losses with weighted factors:
   \begin{equation}
   \mathcal{L}_{\text{total}} = \alpha \mathcal{L}_{\text{text}} + \beta \mathcal{L}_{\text{audio}} \quad (\alpha = 0.01, \beta = 1.0)
   \end{equation}

where $\alpha$ and $\beta$ are scaling factors that control the relative importance of each loss term.




\subsection{HiFi-GAN Loss}
We used a HiFi-GAN-based vocoder \cite{kong2020hifi} that comprises multiple discriminators: the Multi-Period Discriminator, and Multi-Scale Discriminator. For the sake of clarity, we will refer to these discriminators as a single entity. The HiFi-GAN module is trained using multiple losses mentioned below:

1. \textbf{Adversarial Loss}: The adversarial losses for the generator \(G\) and the discriminator \(D\) are defined as follows:
\begin{align}
    \mathcal{L}_{\text{Adv}}(D; G) &= \mathbb{E}_{(x, s)} \left[(D(x) - 1)^2 + D(G(s))^2 \right] \\
    \mathcal{L}_{\text{Adv}}(G; D) &= \mathbb{E}_{s} \left[(D(G(s)) - 1)^2 \right]
\end{align}

where \(x\) represents the real audio samples, and \(s\) denotes the input conditions.

2. \textbf{Mel-Spectrogram Loss}: This loss calculates L1 distance between the mel-spectrograms of the real and generated audio. This loss is formulated as:
\begin{align}
    \mathcal{L}_{\text{Mel}}(G) = \mathbb{E}_{(x, s)} \left[\left\| \phi(x) - \phi(G(s)) \right\|_{1}\right]
\end{align}
where \(\phi\) represents the transformation function that maps a waveform to its corresponding mel-spectrogram.

3. \textbf{Feature Matching Loss}: The feature matching loss calculates the L1 distance between the intermediate features of the real and generated audio, as extracted from multiple layers of the discriminator. It is defined as:
% \begin{align}
%     \mathcal{L}_{\text{FM}}(G; D) = \mathbb{E}_{(x, s)} \left[\sum_{i=1}^{T} \frac{1}{N_i} \left\| D^i(x) - D^i(G(s)) \right\|_{1}\right]
% \end{align}

\begin{align}
    \mathcal{L}_{\text{FM}}(G; D) = \mathbb{E}_{(x, s)} \sum_{i=1}^{T} \frac{1}{N_i} \left\| D^i(x) - D^i(G(s)) \right\|_{1}
\end{align}

where \(T\) denotes the number of discriminator layers, and \(D^i\) and \(N_i\) represent the features and number of features at the \(i\)-th layer, respectively.


\paragraph{Final Loss:}
Given that the discriminator is composed of multiple sub-discriminators, the final objectives for training the generator and the discriminator are defined as follows:
% \begin{align}
%     \mathcal{L}_{G} &= \sum_{k=1}^{K} \left[\mathcal{L}_{\text{Adv}}(G; D_k) + \lambda_{\text{FM}} \mathcal{L}_{\text{FM}}(G; D_k)\right] + \lambda_{\text{Mel}} \mathcal{L}_{\text{Mel}}(G), \\
%     \mathcal{L}_{D} &= \sum_{k=1}^{K} \mathcal{L}_{\text{Adv}}(D_k; G),
% \end{align}
\begin{align}
    \mathcal{L}_{G} &= \sum_{k=1}^{K} \left[\mathcal{L}_{\text{Adv}}(G; D_k) + \lambda_{\text{FM}} \mathcal{L}_{\text{FM}}(G; D_k)\right] \notag \\
    &\quad + \lambda_{\text{Mel}} \mathcal{L}_{\text{Mel}}(G) \\
    \mathcal{L}_{D} &= \sum_{k=1}^{K} \mathcal{L}_{\text{Adv}}(D_k; G)
\end{align}

where \(D_k\) denotes the \(k\)-th sub-discriminator and \(\lambda_{\text{FM}} = 2\), \(\lambda_{\text{Mel}} = 45\). 




% 1. \textbf{Adversarial Losses}: For generator $G$ and discriminator $D$, using least squares instead of binary cross-entropy:
%    \begin{align}
%    \mathcal{L}_{\text{Adv}}(D; G) &= \mathbb{E}_{(x, s)} \left[(D(x) - 1)^2 + D(G(s))^2 \right], \\
%    \mathcal{L}_{\text{Adv}}(G; D) &= \mathbb{E}_{s} \left[(D(G(s)) - 1)^2 \right]
%    \end{align}
% 2. \textbf{Mel-Spectrogram Loss}: Measures the L1 distance between real and generated audio mel-spectrograms:
%    \begin{equation}
%    \mathcal{L}_{\text{Mel}}(G) = \mathbb{E}_{(x, s)} \left[\| \phi(x) - \phi(G(s)) \|_{1}\right]
%    \end{equation}

% 3. \textbf{Feature Matching Loss}: Compares intermediate features from real and generated audio across discriminator layers:
%    \begin{equation}
%    \mathcal{L}_{\text{FM}}(G; D) = \mathbb{E}_{(x, s)} \left[\sum_{i=1}^{T} \frac{1}{N_i} \| D^i(x) - D^i(G(s)) \|_{1}\right]
%    \end{equation}

% 4. \textbf{Final Loss Objectives}:
%    Generator Loss:
%    \begin{equation}
%    \mathcal{L}_{G} = \mathcal{L}_{\text{Adv}}(G; D) + \lambda_{\text{FM}} \mathcal{L}_{\text{FM}}(G; D) + \lambda_{\text{Mel}} \mathcal{L}_{\text{Mel}}(G)
%    \end{equation}
%    Discriminator Loss:
%    \begin{equation}
%    \mathcal{L}_{D} = \mathcal{L}_{\text{Adv}}(D; G)
%    \end{equation}

% This framework ensures effective training of the model, balancing text and audio prediction tasks, and optimizing for high-quality audio generation.


\section{Speech Generation}
\subsection{Synthesizing Short Sequences}

The generation of short audio sequences presents challenges in the BnTTS model, particularly for texts containing fewer than 30 characters when using the default generation settings (Temperature \(T = 0.85\) and TopK = 50). The primary issues observed are twofold: (1) the generated speech often lacks intelligibility, and (2) the output speech tends to be longer than expected.

To investigate these challenges, we curated a subset of 23 short text-speech pairs from the BnStudioEval dataset. For evaluation, we utilize the Character Error Rate (CER) metric to assess intelligibility, and we introduce the Audio Duration Equality metric to evaluate the alignment between the generated and reference audio durations. The Audio Duration Equality Score quantifies the equality between two audio sample durations, \(a\) and \(b\), producing values between 0 and 1, where a score of 1 indicates identical durations. The metric is defined as:

\begin{equation}
    \text{DurationEquality}(a, b) = \frac{1}{\max\left(\frac{a}{b}, \frac{b}{a}\right)}
\end{equation}



This score approaches 1 as the durations of \(a\) and \(b\) become more equal, providing an effective measure of  discrepancy between duration of reference audio and synthesized audio .


\paragraph{Effect of Short Prompt}
Under the default settings (Exp. 1 in Table X), the model achieves a Character Error Rate (CER) of 0.081 and a Duration Equality Score of 0.699. We hypothesize that the model's inability to accurately synthesize short speech stems from its training process. During training, the model reserves between 1 to 6 seconds of audio for speaker prompting. For audio shorter than 1 second, the model uses half of the audio as the prompt. This implies that the model is accustomed to short audio prompts for short sequences. By aligning the inference process with this training strategy and using short prompts, the generation performance improves markedly, as evidenced by a higher Duration Equality Score of 0.820 and a lower CER of 0.029 in Exp. 2.

\paragraph{Effect of Temperature and Top-K Sampling}
The default temperature (\(T = 0.85\)) and top-K value (50) were found to be sub-optimal for generating short sequences. By adjusting the temperature to \(T = 1.0\) and reducing the top-K value to 2, we observed an improvement in the Duration Equality Score from 0.699 to 0.701, accompanied by a substantial reduction in CER, from 0.081 to 0.023 (as shown in Exp. 3).

\paragraph{Effect of Both Short Prompts and Temperature, Top-K}
Combining short prompts with the adjusted temperature and top-K values yielded the best results. In this configuration, the Duration Equality Score improved to 0.827, with a CER of 0.015, demonstrating that both factors are crucial for accurate short sequence generation.

The ablation study demonstrates that employing short prompts in combination with fine-tuning temperature and top-K values is essential for optimizing short sequence generation in the BnTTS model.

\begin{table}[H]
\centering

    \begin{tabular}{c|l}
        \hline
        \textbf{Variable} & \textbf{Description} \\ \hline
        \( \mathbf{T} \) & Text sequence with \( N \) tokens \\ \hline
        \( N \) & Number of tokens in the text sequence \\ \hline
        \( \mathbf{S} \) & Speaker's mel-spectrogram with \( L \) frames \\ \hline
        \( \hat{\mathbf{Y}} \) & Generated speech that matches the speaker's characteristics \\ \hline
        \( \mathbf{Y} \) & Ground truth mel-spectrogram frames for the target speech \\ \hline
        \( \mathcal{F} \) & Model responsible for producing speech conditioned on both the text and the speaker's spectrogram \\ \hline
        \( \mathbf{z} \) & Discrete codes transformed from mel-spectrogram frames using VQ-VAE \\ \hline
        \( \mathcal{C} \) & Codebook of discrete codes from VQ-VAE \\ \hline
        \( l \) & Number of layers in the Conditioning Encoder \\ \hline
        \( k \) & Number of attention heads in Scaled Dot-Product Attention \\ \hline
        \( \mathbf{S_z} \) & Intermediate representation of speaker spectrogram in \( \mathbb{R}^{L \times d} \) \\ \hline
        \( d \) & Dimensionality of each token or embedding \\ \hline
        \( \mathbf{Q}, \mathbf{K}, \mathbf{V} \) & Projections of \( \mathbf{S_z} \) used in scaled dot-product attention \\ \hline
        \( P \) & Fixed number of sequences produced by the Perceiver Resampler \\ \hline
        \( \mathbf{R} \) & Fixed-size output from the Perceiver Resampler in \( \mathbb{R}^{P \times d} \) \\ \hline
        \( \mathbf{T_e} \) & Continuous embedding space of text tokens in \( \mathbb{R}^{N \times d} \) \\ \hline
        \( \mathbf{S_p} \) & Speaker embeddings \\ \hline
        \( \mathbf{Y_z} \) & Ground truth spectrogram embeddings \\ \hline
        \( \mathbf{X} \) & Combined input during training: concatenation of speaker, text, and spectrogram embeddings \\ \hline
        \( \oplus \) & Concatenation operation \\ \hline
        \( \mathbf{H} \) & Output from the LLM consisting of hidden states for text, speaker, and spectrogram embeddings \\ \hline
        \( \mathbf{H}_\text{Y} \) & Spectrogram embedding from LLM output used for HiFi-GAN \\ \hline
        \( \mathbf{S}' \) & Resized speaker embedding to match \( \mathbf{H}_\text{Y} \) \\ \hline
        \( \mathbf{W} \) & Final audio waveform produced by HiFi-GAN \\ \hline
        \( g_\text{HiFi} \) & HiFi-GAN function converting spectrogram embeddings to audio waveform \\ \hline
    \end{tabular}
    \label{tab:variables_descriptions}
    \caption{Table of Variables and Descriptions}
\end{table}

% \section{Results and Discussion}

% To evaluate the performance of our Bengali TTS system, we employed a combination of subjective and objective metrics across two datasets: BnStudioEval and BnTTSTextEval. The BnStudioEval dataset, consisting of high-quality recordings, was used for reference-aware evaluation, while BnTTSTextEval encompassed subsets focusing on phonetic diversity, named entity pronunciation, and conversational fluency for reference-independent evaluation. The metrics used include subjective measures such as SMOS and objective measures like CER, UTMOS, SECS, and SpeechBERT Precision.



% \subsection{Reference-aware Evaluation (BnStudioEval)}
% Table 1 presents the comparative performance of various TTS systems evaluated on the BnStudioEval dataset. Among the synthetic methods, AzureTTS exhibited the best performance with the lowest Character Error Rate (CER) and the highest UTMOS score, outperforming all other synthetic methods, including Ground Truths (GT) in terms of transcription accuracy. However, interestingly, the CER of the GT remains lower than that of BnTTS synthesized outputs. As expected, the GT, serving as a reference standard, outperforms all synthetic systems across key subjective metrics such as SMOS (4.671), Naturalness (4.625), and Clarity (4.9). In this context, the proposed BnTTS system closely follows, achieving competitive scores in SMOS (4.584), Naturalness (4.531), and Clarity (4.898).

% Regarding speaker similarity, the GT achieved SECS scores of 1.0 when compared to reference audios and 0.361 when compared to the speaker prompt. BnTTS also performed well, with an SECS(Ref.) score of 0.513 and an SECS(Prompt) score of 0.335, falling short of the ground truth by only 0.031 in the latter metric. Additionally, BnTTS received a SpeechBERT Precision score of 0.796, compared to the perfect score of 1.0 set by the ground truth.

% It should be noted that IndicTTS, GTTS, and AzureTTS lack speaker impersonation capabilities, rendering them inapplicable for reference-aware metrics such as SECS and SpeechBERT Precision. Consequently, these metrics were not calculated for these systems.


% \subsection{Reference-independent Evaluation (BnTTSTextEval)}
% Table 2 presents the comparative performance of various TTS systems evaluated on the BnTTSTextEval dataset, encompassing three distinct subsets: BengaliStimuli53, BengaliNamedEntity1000, and ShortText200. The trend observed in the BnStudioEval dataset persists here as well. AzureTTS and GTTS consistently trade leading positions in transcription accuracy (CER) and automated quality prediction (UTMOS), with BnTTS following closely in third place, and IndicTTS trailing behind.

% BnTTS performs strongly in subjective evaluations, excelling in SMOS, Naturalness, and Intelligibility across the BengaliStimuli53 and BengaliNamedEntity1000 subsets. However, it falls slightly behind AzureTTS in the ShortText200 subset, which focuses on model performance in short texts. Despite this, BnTTS overall, remains the top-performing system in all subjective metrics, delivering the highest average scores in SMOS(4.383), Naturalness(4.313), and Clarity(4.737).



\end{document}
