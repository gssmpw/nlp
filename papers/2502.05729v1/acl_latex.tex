
\documentclass[11pt]{article}

% Change "review" to "final" to generate the final (sometimes called camera-ready) version.
% Change to "preprint" to generate a non-anonymous version with page numbers.
\usepackage[final]{acl}

% Standard package includes
\usepackage{times}
\usepackage{latexsym}


% For proper rendering and hyphenation of words containing Latin characters (including in bib files)
\usepackage[T1]{fontenc}
% For Vietnamese characters
% \usepackage[T5]{fontenc}
% See https://www.latex-project.org/help/documentation/encguide.pdf for other character sets

% This assumes your files are encoded as UTF8
\usepackage[utf8]{inputenc}

% This is not strictly necessary, and may be commented out,
% but it will improve the layout of the manuscript,
% and will typically save some space.
\usepackage{microtype}

% This is also not strictly necessary, and may be commented out.
% However, it will improve the aesthetics of text in
% the typewriter font.
\usepackage{inconsolata}

%Including images in your LaTeX document requires adding
%additional package(s)
\usepackage{graphicx}
\usepackage{amssymb}  % Provides \mathbb among other symbols


\usepackage{algorithm}
\usepackage{algorithmicx}
\usepackage{algpseudocode}
% \usepackage[linesnumbered, ruled, vlined]{algorithm2e}
\usepackage{tabularray}
\usepackage{caption}
\usepackage{subcaption}
% \usepackage{lmodern}
\usepackage{amsmath}
\usepackage{multirow}

% ,amsmath,graphicx, spconf}
% \usepackage{minted}

% If the title and author information does not fit in the area allocated, uncomment the following
%
%\setlength\titlebox{<dim>}
%
% and set <dim> to something 5cm or larger.

\title{BnTTS: Few-Shot Speaker Adaptation in Low-Resource Setting}

% Author information can be set in various styles:
% For several authors from the same institution:
% \author{Author 1 \and ... \and Author n \\
%         Address line \\ ... \\ Address line}
% if the names do not fit well on one line use
%         Author 1 \\ {\bf Author 2} \\ ... \\ {\bf Author n} \\
% For authors from different institutions:
% \author{Author 1 \\ Address line \\  ... \\ Address line
%         \And  ... \And
%         Author n \\ Address line \\ ... \\ Address line}
% To start a separate ``row'' of authors use \AND, as in
% \author{Author 1 \\ Address line \\  ... \\ Address line
%         \AND
%         Author 2 \\ Address line \\ ... \\ Address line \And
%         Author 3 \\ Address line \\ ... \\ Address line}

% jahid version 
\author{
\textbf{Mohammad Jahid Ibna Basher}\textsuperscript{1},
\textbf{Md Kowsher}\textsuperscript{2},
\textbf{Md Saiful Islam}\textsuperscript{1},
\textbf{Rabindra Nath Nandi}\textsuperscript{1}, \\
\textbf{Nusrat Jahan Prottasha}\textsuperscript{2},
\textbf{Mehadi Hasan Menon}\textsuperscript{1},
\textbf{Tareq Al Muntasir}\textsuperscript{1}, \\
\textbf{Shammur Absar Chowdhury}\textsuperscript{3},
\textbf{Firoj Alam}\textsuperscript{3},
\textbf{Niloofar Yousefi}\textsuperscript{2}, 
\textbf{Ozlem Ozmen Garibay}\textsuperscript{2} \\
\textsuperscript{1}Hishab Singapore Pte. Ltd, Singapore, 
\textsuperscript{2}University of Central Florida, USA \\
\textsuperscript{3}Qatar Computing Research Institute, Qatar
}




%\author{
%  \textbf{First Author\textsuperscript{1}},
%  \textbf{Second Author\textsuperscript{1,2}},
%  \textbf{Third T. Author\textsuperscript{1}},
%  \textbf{Fourth Author\textsuperscript{1}},
%\\
%  \textbf{Fifth Author\textsuperscript{1,2}},
%  \textbf{Sixth Author\textsuperscript{1}},
%  \textbf{Seventh Author\textsuperscript{1}},
%  \textbf{Eighth Author \textsuperscript{1,2,3,4}},
%\\
%  \textbf{Ninth Author\textsuperscript{1}},
%  \textbf{Tenth Author\textsuperscript{1}},
%  \textbf{Eleventh E. Author\textsuperscript{1,2,3,4,5}},
%  \textbf{Twelfth Author\textsuperscript{1}},
%\\
%  \textbf{Thirteenth Author\textsuperscript{3}},
%  \textbf{Fourteenth F. Author\textsuperscript{2,4}},
%  \textbf{Fifteenth Author\textsuperscript{1}},
%  \textbf{Sixteenth Author\textsuperscript{1}},
%\\
%  \textbf{Seventeenth S. Author\textsuperscript{4,5}},
%  \textbf{Eighteenth Author\textsuperscript{3,4}},
%  \textbf{Nineteenth N. Author\textsuperscript{2,5}},
%  \textbf{Twentieth Author\textsuperscript{1}}
%\\
%\\
%  \textsuperscript{1}Affiliation 1,
%  \textsuperscript{2}Affiliation 2,
%  \textsuperscript{3}Affiliation 3,
%  \textsuperscript{4}Affiliation 4,
%  \textsuperscript{5}Affiliation 5
%\\
%  \small{
%    \textbf{Correspondence:} \href{mailto:email@domain}{email@domain}
%  }
%}

\begin{document}
\maketitle
\begin{abstract}

% \textcolor{green}{Text-to-speech (TTS) systems have made significant strides in generating high-quality speech, yet much of this progress has been concentrated on high-resource languages, leaving low-resource languages like Bangla underrepresented}. 
This paper introduces BnTTS (\textbf{B}a\textbf{n}gla \textbf{T}ext-\textbf{T}o-\textbf{S}peech), the first framework for Bangla speaker adaptation-based TTS, designed to bridge the gap in Bangla speech synthesis using minimal training data. Building upon the XTTS architecture, our approach integrates Bangla into a multilingual TTS pipeline, with modifications to account for the phonetic and linguistic characteristics of the language. We pretrain BnTTS on 3.85k hours of Bangla speech dataset with corresponding text labels and evaluate performance in both zero-shot and few-shot settings on our proposed test dataset. Empirical evaluations in few-shot settings show that BnTTS significantly improves the naturalness, intelligibility, and speaker fidelity of synthesized Bangla speech. Compared to state-of-the-art Bangla TTS systems, BnTTS exhibits superior performance in Subjective Mean Opinion Score (SMOS), Naturalness, and Clarity metrics.


 
% The open-source code is available at Anonymized.
\end{abstract}

\section{Introduction}

Large language models (LLMs) have achieved remarkable success in automated math problem solving, particularly through code-generation capabilities integrated with proof assistants~\citep{lean,isabelle,POT,autoformalization,MATH}. Although LLMs excel at generating solution steps and correct answers in algebra and calculus~\citep{math_solving}, their unimodal nature limits performance in plane geometry, where solution depends on both diagram and text~\citep{math_solving}. 

Specialized vision-language models (VLMs) have accordingly been developed for plane geometry problem solving (PGPS)~\citep{geoqa,unigeo,intergps,pgps,GOLD,LANS,geox}. Yet, it remains unclear whether these models genuinely leverage diagrams or rely almost exclusively on textual features. This ambiguity arises because existing PGPS datasets typically embed sufficient geometric details within problem statements, potentially making the vision encoder unnecessary~\citep{GOLD}. \cref{fig:pgps_examples} illustrates example questions from GeoQA and PGPS9K, where solutions can be derived without referencing the diagrams.

\begin{figure}
    \centering
    \begin{subfigure}[t]{.49\linewidth}
        \centering
        \includegraphics[width=\linewidth]{latex/figures/images/geoqa_example.pdf}
        \caption{GeoQA}
        \label{fig:geoqa_example}
    \end{subfigure}
    \begin{subfigure}[t]{.48\linewidth}
        \centering
        \includegraphics[width=\linewidth]{latex/figures/images/pgps_example.pdf}
        \caption{PGPS9K}
        \label{fig:pgps9k_example}
    \end{subfigure}
    \caption{
    Examples of diagram-caption pairs and their solution steps written in formal languages from GeoQA and PGPS9k datasets. In the problem description, the visual geometric premises and numerical variables are highlighted in green and red, respectively. A significant difference in the style of the diagram and formal language can be observable. %, along with the differences in formal languages supported by the corresponding datasets.
    \label{fig:pgps_examples}
    }
\end{figure}



We propose a new benchmark created via a synthetic data engine, which systematically evaluates the ability of VLM vision encoders to recognize geometric premises. Our empirical findings reveal that previously suggested self-supervised learning (SSL) approaches, e.g., vector quantized variataional auto-encoder (VQ-VAE)~\citep{unimath} and masked auto-encoder (MAE)~\citep{scagps,geox}, and widely adopted encoders, e.g., OpenCLIP~\citep{clip} and DinoV2~\citep{dinov2}, struggle to detect geometric features such as perpendicularity and degrees. 

To this end, we propose \geoclip{}, a model pre-trained on a large corpus of synthetic diagram–caption pairs. By varying diagram styles (e.g., color, font size, resolution, line width), \geoclip{} learns robust geometric representations and outperforms prior SSL-based methods on our benchmark. Building on \geoclip{}, we introduce a few-shot domain adaptation technique that efficiently transfers the recognition ability to real-world diagrams. We further combine this domain-adapted GeoCLIP with an LLM, forming a domain-agnostic VLM for solving PGPS tasks in MathVerse~\citep{mathverse}. 
%To accommodate diverse diagram styles and solution formats, we unify the solution program languages across multiple PGPS datasets, ensuring comprehensive evaluation. 

In our experiments on MathVerse~\citep{mathverse}, which encompasses diverse plane geometry tasks and diagram styles, our VLM with a domain-adapted \geoclip{} consistently outperforms both task-specific PGPS models and generalist VLMs. 
% In particular, it achieves higher accuracy on tasks requiring geometric-feature recognition, even when critical numerical measurements are moved from text to diagrams. 
Ablation studies confirm the effectiveness of our domain adaptation strategy, showing improvements in optical character recognition (OCR)-based tasks and robust diagram embeddings across different styles. 
% By unifying the solution program languages of existing datasets and incorporating OCR capability, we enable a single VLM, named \geovlm{}, to handle a broad class of plane geometry problems.

% Contributions
We summarize the contributions as follows:
We propose a novel benchmark for systematically assessing how well vision encoders recognize geometric premises in plane geometry diagrams~(\cref{sec:visual_feature}); We introduce \geoclip{}, a vision encoder capable of accurately detecting visual geometric premises~(\cref{sec:geoclip}), and a few-shot domain adaptation technique that efficiently transfers this capability across different diagram styles (\cref{sec:domain_adaptation});
We show that our VLM, incorporating domain-adapted GeoCLIP, surpasses existing specialized PGPS VLMs and generalist VLMs on the MathVerse benchmark~(\cref{sec:experiments}) and effectively interprets diverse diagram styles~(\cref{sec:abl}).

\iffalse
\begin{itemize}
    \item We propose a novel benchmark for systematically assessing how well vision encoders recognize geometric premises, e.g., perpendicularity and angle measures, in plane geometry diagrams.
	\item We introduce \geoclip{}, a vision encoder capable of accurately detecting visual geometric premises, and a few-shot domain adaptation technique that efficiently transfers this capability across different diagram styles.
	\item We show that our final VLM, incorporating GeoCLIP-DA, effectively interprets diverse diagram styles and achieves state-of-the-art performance on the MathVerse benchmark, surpassing existing specialized PGPS models and generalist VLM models.
\end{itemize}
\fi

\iffalse

Large language models (LLMs) have made significant strides in automated math word problem solving. In particular, their code-generation capabilities combined with proof assistants~\citep{lean,isabelle} help minimize computational errors~\citep{POT}, improve solution precision~\citep{autoformalization}, and offer rigorous feedback and evaluation~\citep{MATH}. Although LLMs excel in generating solution steps and correct answers for algebra and calculus~\citep{math_solving}, their uni-modal nature limits performance in domains like plane geometry, where both diagrams and text are vital.

Plane geometry problem solving (PGPS) tasks typically include diagrams and textual descriptions, requiring solvers to interpret premises from both sources. To facilitate automated solutions for these problems, several studies have introduced formal languages tailored for plane geometry to represent solution steps as a program with training datasets composed of diagrams, textual descriptions, and solution programs~\citep{geoqa,unigeo,intergps,pgps}. Building on these datasets, a number of PGPS specialized vision-language models (VLMs) have been developed so far~\citep{GOLD, LANS, geox}.

Most existing VLMs, however, fail to use diagrams when solving geometry problems. Well-known PGPS datasets such as GeoQA~\citep{geoqa}, UniGeo~\citep{unigeo}, and PGPS9K~\citep{pgps}, can be solved without accessing diagrams, as their problem descriptions often contain all geometric information. \cref{fig:pgps_examples} shows an example from GeoQA and PGPS9K datasets, where one can deduce the solution steps without knowing the diagrams. 
As a result, models trained on these datasets rely almost exclusively on textual information, leaving the vision encoder under-utilized~\citep{GOLD}. 
Consequently, the VLMs trained on these datasets cannot solve the plane geometry problem when necessary geometric properties or relations are excluded from the problem statement.

Some studies seek to enhance the recognition of geometric premises from a diagram by directly predicting the premises from the diagram~\citep{GOLD, intergps} or as an auxiliary task for vision encoders~\citep{geoqa,geoqa-plus}. However, these approaches remain highly domain-specific because the labels for training are difficult to obtain, thus limiting generalization across different domains. While self-supervised learning (SSL) methods that depend exclusively on geometric diagrams, e.g., vector quantized variational auto-encoder (VQ-VAE)~\citep{unimath} and masked auto-encoder (MAE)~\citep{scagps,geox}, have also been explored, the effectiveness of the SSL approaches on recognizing geometric features has not been thoroughly investigated.

We introduce a benchmark constructed with a synthetic data engine to evaluate the effectiveness of SSL approaches in recognizing geometric premises from diagrams. Our empirical results with the proposed benchmark show that the vision encoders trained with SSL methods fail to capture visual \geofeat{}s such as perpendicularity between two lines and angle measure.
Furthermore, we find that the pre-trained vision encoders often used in general-purpose VLMs, e.g., OpenCLIP~\citep{clip} and DinoV2~\citep{dinov2}, fail to recognize geometric premises from diagrams.

To improve the vision encoder for PGPS, we propose \geoclip{}, a model trained with a massive amount of diagram-caption pairs.
Since the amount of diagram-caption pairs in existing benchmarks is often limited, we develop a plane diagram generator that can randomly sample plane geometry problems with the help of existing proof assistant~\citep{alphageometry}.
To make \geoclip{} robust against different styles, we vary the visual properties of diagrams, such as color, font size, resolution, and line width.
We show that \geoclip{} performs better than the other SSL approaches and commonly used vision encoders on the newly proposed benchmark.

Another major challenge in PGPS is developing a domain-agnostic VLM capable of handling multiple PGPS benchmarks. As shown in \cref{fig:pgps_examples}, the main difficulties arise from variations in diagram styles. 
To address the issue, we propose a few-shot domain adaptation technique for \geoclip{} which transfers its visual \geofeat{} perception from the synthetic diagrams to the real-world diagrams efficiently. 

We study the efficacy of the domain adapted \geoclip{} on PGPS when equipped with the language model. To be specific, we compare the VLM with the previous PGPS models on MathVerse~\citep{mathverse}, which is designed to evaluate both the PGPS and visual \geofeat{} perception performance on various domains.
While previous PGPS models are inapplicable to certain types of MathVerse problems, we modify the prediction target and unify the solution program languages of the existing PGPS training data to make our VLM applicable to all types of MathVerse problems.
Results on MathVerse demonstrate that our VLM more effectively integrates diagrammatic information and remains robust under conditions of various diagram styles.

\begin{itemize}
    \item We propose a benchmark to measure the visual \geofeat{} recognition performance of different vision encoders.
    % \item \sh{We introduce geometric CLIP (\geoclip{} and train the VLM equipped with \geoclip{} to predict both solution steps and the numerical measurements of the problem.}
    \item We introduce \geoclip{}, a vision encoder which can accurately recognize visual \geofeat{}s and a few-shot domain adaptation technique which can transfer such ability to different domains efficiently. 
    % \item \sh{We develop our final PGPS model, \geovlm{}, by adapting \geoclip{} to different domains and training with unified languages of solution program data.}
    % We develop a domain-agnostic VLM, namely \geovlm{}, by applying a simple yet effective domain adaptation method to \geoclip{} and training on the refined training data.
    \item We demonstrate our VLM equipped with GeoCLIP-DA effectively interprets diverse diagram styles, achieving superior performance on MathVerse compared to the existing PGPS models.
\end{itemize}

\fi 

\section{BnTTS}
\begin{figure}[ht!]
    \raggedleft
    \includegraphics[width=0.90\linewidth]{resources/hqtts.png} 
    \caption{Overview of BnTTS Model.} 
    \label{fig:xtts_train_diagram}
    % \vspace{-.2cm}
\end{figure}


\textbf{Preliminaries:} Given a text sequence with \( N \) tokens, \( \mathbf{T} = \{t_1, t_2, \ldots, t_N\} \), and a speaker's mel-spectrogram \( \mathbf{S} = \{s_1, s_2, \ldots, s_L\} \), the objective is to generate speech \( \hat{\mathbf{Y}} \) that matches the speaker's characteristics. The ground truth mel-spectrogram frames for the target speech are denoted as \( \mathbf{Y} = \{y_1, y_2, \ldots, y_M\} \). The synthesis process can be described as:
% \vspace{-0.12cm}
\[
\hat{\mathbf{Y}} = \mathcal{F}(\mathbf{S}, \mathbf{T})
\]
% \vspace{-0.12cm}
where \( \mathcal{F} \) produces speech conditioned on both the text and the speaker's spectrogram.

\noindent \textbf{Audio Encoder:} A Vector Quantized-Variational AutoEncoder (VQ-VAE) \cite{tortoise} encodes mel-spectrogram frames \( \mathbf{Y} \) into discrete tokens \( M \in \mathcal{C} \), where $\mathcal{C}$ is vocab or codebook. An embedding layer then transforms these tokens into a \( d \)-dimensional vector: \( \mathbf{Y_e} \in \mathbb{R}^{M \times d} \).

\noindent \textbf{Conditioning Encoder \& Perceiver Resampler:} The Conditioning Encoder \cite{casanova2024xtts} consists of \( l \) layers of \( k \)-head Scaled Dot-Product Attention, followed by a Perceiver Resampler. The speaker spectrogram \( \mathbf{S} \) is transformed into an intermediate representation \( \mathbf{S_z} \in \mathbb{R}^{L \times d} \), where each attention layer applies a scaled dot-product attention mechanism. The Perceiver Resampler generates a fixed output dimensionality \( \mathbf{R} \in \mathbb{R}^{P \times d} \) from a variable input length \( L \).

\noindent \textbf{Text Encoder:} The text tokens \( \mathbf{T} = \{t_1, t_2, \ldots, t_N\} \) are projected into a continuous embedding space, yielding \( \mathbf{T_e} \in \mathbb{R}^{N \times d} \).

\noindent \textbf{Large Language Model (LLM):} The transformer-based LLM \cite{radford2019language} utilizes the decoder portion. Speaker embeddings \( \mathbf{S_p} \), text embeddings \( \mathbf{T_e} \), and ground truth spectrogram embeddings \( \mathbf{Y_e} \) are concatenated to form the input:
% \vspace{-0.2cm}
\[
\mathbf{X} = \mathbf{S_p} \oplus \mathbf{T_e} \oplus \mathbf{Y_e} \in \mathbb{R}^{(N + P + M) \times d}
\]
% \vspace{-0.12cm}
The LLM processes \( \mathbf{X} \), producing output \( \mathbf{H} \) with hidden states for the text, speaker, and spectrogram embeddings. During inference, only text and speaker embeddings are concatenated, generating spectrogram embeddings \( \{h_1^Y, h_2^Y, \ldots, h_P^Y\} \) as the output.

\noindent \textbf{HiFi-GAN Decoder:} The HiFi-GAN Decoder \cite{kong2020hifi} converts the LLM's output into realistic speech, preserving the speaker's characteristics. Specifically, it takes the LLM's speech head output \( \mathbf{H}_\text{Y} = \{h_1^Y, h_2^Y, \ldots, h_P^Y\} \). The speaker embedding \( \mathbf{S} \) is resized to match \( \mathbf{H}_\text{Y} \), resulting in \( \mathbf{S}' \in \mathbb{R}^{P \times d} \). The final audio waveform \( \mathbf{W} \) is then generated by:
% \vspace{-0.3cm}
\[
\mathbf{W} = g_\text{HiFi}(\mathbf{H}_\text{Y} + \mathbf{S}')
\]
% \vspace{-0.12cm}
Thus, the HiFi-GAN decoder produces speech that reflects the input text while maintaining the speaker's unique qualities.

\section{Experimental Analysis}
\label{sec:exp}
We now describe in detail our experimental analysis. The experimental section is organized as follows:
%\begin{enumerate}[noitemsep,topsep=0pt,parsep=0pt,partopsep=0pt,leftmargin=0.5cm]
%\item 

\noindent In {\bf 
Section~\ref{exp:setup}}, we introduce the datasets and methods to evaluate the previously defined accuracy measures.

%\item
\noindent In {\bf 
Section~\ref{exp:qual}}, we illustrate the limitations of existing measures with some selected qualitative examples.

%\item 
\noindent In {\bf 
Section~\ref{exp:quant}}, we continue by measuring quantitatively the benefits of our proposed measures in terms of {\it robustness} to lag, noise, and normal/abnormal ratio.

%\item 
\noindent In {\bf 
Section~\ref{exp:separability}}, we evaluate the {\it separability} degree of accurate and inaccurate methods, using the existing and our proposed approaches.

%\item
\noindent In {\bf 
Section~\ref{sec:entropy}}, we conduct a {\it consistency} evaluation, in which we analyze the variation of ranks that an AD method can have with an accuracy measures used.

%\item 
\noindent In {\bf 
Section~\ref{sec:exectime}}, we conduct an {\it execution time} evaluation, in which we analyze the impact of different parameters related to the accuracy measures and the time series characteristics. 
We focus especially on the comparison of the different VUS implementations.
%\end{enumerate}

\begin{table}[tb]
\caption{Summary characteristics (averaged per dataset) of the public datasets of TSB-UAD (S.: Size, Ano.: Anomalies, Ab.: Abnormal, Den.: Density)}
\label{table:charac}
%\vspace{-0.2cm}
\footnotesize
\begin{center}
\scalebox{0.82}{
\begin{tabular}{ |r|r|r|r|r|r|} 
 \hline
\textbf{\begin{tabular}[c]{@{}c@{}}Dataset \end{tabular}} & 
\textbf{\begin{tabular}[c]{@{}c@{}}S. \end{tabular}} & 
\textbf{\begin{tabular}[c]{c@{}} Len.\end{tabular}} & 
\textbf{\begin{tabular}[c]{c@{}} \# \\ Ano. \end{tabular}} &
\textbf{\begin{tabular}[c]{c@{}c@{}} \# \\ Ab. \\ Points\end{tabular}} &
\textbf{\begin{tabular}[c]{c@{}c@{}} Ab. \\ Den. \\ (\%)\end{tabular}} \\ \hline
Dodgers \cite{10.1145/1150402.1150428} & 1 & 50400   & 133.0     & 5612.0  &11.14 \\ \hline
SED \cite{doi:10.1177/1475921710395811}& 1 & 100000   & 75.0     & 3750.0  & 3.7\\ \hline
ECG \cite{goldberger_physiobank_2000}   & 52 & 230351  & 195.6     & 15634.0  &6.8 \\ \hline
IOPS \cite{IOPS}   & 58 & 102119  & 46.5     & 2312.3   &2.1 \\ \hline
KDD21 \cite{kdd} & 250 &77415   & 1      & 196.5   &0.56 \\ \hline
MGAB \cite{markus_thill_2020_3762385}   & 10 & 100000  & 10.0     & 200.0   &0.20 \\ \hline
NAB \cite{ahmad_unsupervised_2017}   & 58 & 6301   & 2.0      & 575.5   &8.8 \\ \hline
NASA-M. \cite{10.1145/3449726.3459411}   & 27 & 2730   & 1.33      & 286.3   &11.97 \\ \hline
NASA-S. \cite{10.1145/3449726.3459411}   & 54 & 8066   & 1.26      & 1032.4   &12.39 \\ \hline
SensorS. \cite{YAO20101059}   & 23 & 27038   & 11.2     & 6110.4   &22.5 \\ \hline
YAHOO \cite{yahoo}  & 367 & 1561   & 5.9      & 10.7   &0.70 \\ \hline 
\end{tabular}}
\end{center}
\end{table}











\subsection{Experimental Setup and Settings}
\label{exp:setup}
%\vspace{-0.1cm}

\begin{figure*}[tb]
  \centering
  \includegraphics[width=1\linewidth]{figures/quality.pdf}
  %\vspace{-0.7cm}
  \caption{Comparison of evaluation measures (proposed measures illustrated in subplots (b,c,d,e); all others summarized in subplots (f)) on two examples ((A)AE and OCSM applied on MBA(805) and (B) LOF and OCSVM applied on MBA(806)), illustrating the limitations of existing measures for scores with noise or containing a lag. }
  \label{fig:quality}
  %\vspace{-0.1cm}
\end{figure*}

We implemented the experimental scripts in Python 3.8 with the following main dependencies: sklearn 0.23.0, tensorflow 2.3.0, pandas 1.2.5, and networkx 2.6.3. In addition, we used implementations from our TSB-UAD benchmark suite.\footnote{\scriptsize \url{https://www.timeseries.org/TSB-UAD}} For reproducibility purposes, we make our datasets and code available.\footnote{\scriptsize \url{https://www.timeseries.org/VUS}}
\newline \textbf{Datasets: } For our evaluation purposes, we use the public datasets identified in our TSB-UAD benchmark. The latter corresponds to $10$ datasets proposed in the past decades in the literature containing $900$ time series with labeled anomalies. Specifically, each point in every time series is labeled as normal or abnormal. Table~\ref{table:charac} summarizes relevant characteristics of the datasets, including their size, length, and statistics about the anomalies. In more detail:

\begin{itemize}
    \item {\bf SED}~\cite{doi:10.1177/1475921710395811}, from the NASA Rotary Dynamics Laboratory, records disk revolutions measured over several runs (3K rpm speed).
	\item {\bf ECG}~\cite{goldberger_physiobank_2000} is a standard electrocardiogram dataset and the anomalies represent ventricular premature contractions. MBA(14046) is split to $47$ series.
	\item {\bf IOPS}~\cite{IOPS} is a dataset with performance indicators that reflect the scale, quality of web services, and health status of a machine.
	\item {\bf KDD21}~\cite{kdd} is a composite dataset released in a SIGKDD 2021 competition with 250 time series.
	\item {\bf MGAB}~\cite{markus_thill_2020_3762385} is composed of Mackey-Glass time series with non-trivial anomalies. Mackey-Glass data series exhibit chaotic behavior that is difficult for the human eye to distinguish.
	\item {\bf NAB}~\cite{ahmad_unsupervised_2017} is composed of labeled real-world and artificial time series including AWS server metrics, online advertisement clicking rates, real time traffic data, and a collection of Twitter mentions of large publicly-traded companies.
	\item {\bf NASA-SMAP} and {\bf NASA-MSL}~\cite{10.1145/3449726.3459411} are two real spacecraft telemetry data with anomalies from Soil Moisture Active Passive (SMAP) satellite and Curiosity Rover on Mars (MSL).
	\item {\bf SensorScope}~\cite{YAO20101059} is a collection of environmental data, such as temperature, humidity, and solar radiation, collected from a sensor measurement system.
	\item {\bf Yahoo}~\cite{yahoo} is a dataset consisting of real and synthetic time series based on the real production traffic to some of the Yahoo production systems.
\end{itemize}


\textbf{Anomaly Detection Methods: }  For the experimental evaluation, we consider the following baselines. 

\begin{itemize}
\item {\bf Isolation Forest (IForest)}~\cite{liu_isolation_2008} constructs binary trees based on random space splitting. The nodes (subsequences in our specific case) with shorter path lengths to the root (averaged over every random tree) are more likely to be anomalies. 
\item {\bf The Local Outlier Factor (LOF)}~\cite{breunig_lof_2000} computes the ratio of the neighbor density to the local density. 
\item {\bf Matrix Profile (MP)}~\cite{yeh_time_2018} detects as anomaly the subsequence with the most significant 1-NN distance. 
\item {\bf NormA}~\cite{boniol_unsupervised_2021} identifies the normal patterns based on clustering and calculates each point's distance to normal patterns weighted using statistical criteria. 
\item {\bf Principal Component Analysis (PCA)}~\cite{aggarwal_outlier_2017} projects data to a lower-dimensional hyperplane. Outliers are points with a large distance from this plane. 
\item {\bf Autoencoder (AE)} \cite{10.1145/2689746.2689747} projects data to a lower-dimensional space and reconstructs it. Outliers are expected to have larger reconstruction errors. 
\item {\bf LSTM-AD}~\cite{malhotra_long_2015} use an LSTM network that predicts future values from the current subsequence. The prediction error is used to identify anomalies.
\item {\bf Polynomial Approximation (POLY)} \cite{li_unifying_2007} fits a polynomial model that tries to predict the values of the data series from the previous subsequences. Outliers are detected with the prediction error. 
\item {\bf CNN} \cite{8581424} built, using a convolutional deep neural network, a correlation between current and previous subsequences, and outliers are detected by the deviation between the prediction and the actual value. 
\item {\bf One-class Support Vector Machines (OCSVM)} \cite{scholkopf_support_1999} is a support vector method that fits a training dataset and finds the normal data's boundary.
\end{itemize}

\subsection{Qualitative Analysis}
\label{exp:qual}



We first use two examples to demonstrate qualitatively the limitations of existing accuracy evaluation measures in the presence of lag and noise, and to motivate the need for a new approach. 
These two examples are depicted in Figure~\ref{fig:quality}. 
The first example, in Figure~\ref{fig:quality}(A), corresponds to OCSVM and AE on the MBA(805) dataset (named MBA\_ECG805\_data.out in the ECG dataset). 

We observe in Figure~\ref{fig:quality}(A)(a.1) and (a.2) that both scores identify most of the anomalies (highlighted in red). However, the OCSVM score points to more false positives (at the end of the time series) and only captures small sections of the anomalies. On the contrary, the AE score points to fewer false positives and captures all abnormal subsequences. Thus we can conclude that, visually, AE should obtain a better accuracy score than OCSVM. Nevertheless, we also observe that the AE score is lagged with the labels and contains more noise. The latter has a significant impact on the accuracy of evaluation measures. First, Figure~\ref{fig:quality}(A)(c) is showing that AUC-PR is better for OCSM (0.73) than for AE (0.57). This is contradictory with what is visually observed from Figure~\ref{fig:quality}(A)(a.1) and (a.2). However, when using our proposed measure R-AUC-PR, OCSVM obtains a lower score (0.83) than AE (0.89). This confirms that, in this example, a buffer region before the labels helps to capture the true value of an anomaly score. Overall, Figure~\ref{fig:quality}(A)(f) is showing in green and red the evolution of accuracy score for the 13 accuracy measures for AE and OCSVM, respectively. The latter shows that, in addition to Precision@k and Precision, our proposed approach captures the quality order between the two methods well.

We now present a second example, on a different time series, illustrated in Figure~\ref{fig:quality}(B). 
In this case, we demonstrate the anomaly score of OCSVM and LOF (depicted in Figure~\ref{fig:quality}(B)(a.1) and (a.2)) applied on the MBA(806) dataset (named MBA\_ECG806\_data.out in the ECG dataset). 
We observe that both methods produce the same level of noise. However, LOF points to fewer false positives and captures more sections of the abnormal subsequences than OCSVM. 
Nevertheless, the LOF score is slightly lagged with the labels such that the maximum values in the LOF score are slightly outside of the labeled sections. 
Thus, as illustrated in Figure~\ref{fig:quality}(B)(f), even though we can visually consider that LOF is performing better than OCSM, all usual measures (Precision, Recall, F, precision@k, and AUC-PR) are judging OCSM better than AE. On the contrary, measures that consider lag (Rprecision, Rrecall, RF) rank the methods correctly. 
However, due to threshold issues, these measures are very close for the two methods. Overall, only AUC-ROC and our proposed measures give a higher score for LOF than for OCSVM.

\subsection{Quantitative Analysis}
\label{exp:case}

\begin{figure}[t]
  \centering
  \includegraphics[width=1\linewidth]{figures/eval_case_study.pdf}
  %\vspace*{-0.7cm}
  \caption{\commentRed{
  Comparison of evaluation measures for synthetic data examples across various scenarios. S8 represents the oracle case, where predictions perfectly align with labeled anomalies. Problematic cases are highlighted in the red region.}}
  %\vspace*{-0.5cm}
  \label{fig:eval_case_study}
\end{figure}
\commentRed{
We present the evaluation results for different synthetic data scenarios, as shown in Figure~\ref{fig:eval_case_study}. These scenarios range from S1, where predictions occur before the ground truth anomaly, to S12, where predictions fall within the ground truth region. The red-shaded regions highlight problematic cases caused by a lack of adaptability to lags. For instance, in scenarios S1 and S2, a slight shift in the prediction leads to measures (e.g., AUC-PR, F score) that fail to account for lags, resulting in a zero score for S1 and a significant discrepancy between the results of S1 and S2. Thus, we observe that our proposed VUS effectively addresses these issues and provides robust evaluations results.}

%\subsection{Quantitative Analysis}
%\subsection{Sensitivity and Separability Analysis}
\subsection{Robustness Analysis}
\label{exp:quant}


\begin{figure}[tb]
  \centering
  \includegraphics[width=1\linewidth]{figures/lag_sensitivity_analysis.pdf}
  %\vspace*{-0.7cm}
  \caption{For each method, we compute the accuracy measures 10 times with random lag $\ell \in [-0.25*\ell,0.25*\ell]$ injected in the anomaly score. We center the accuracy average to 0.}
  %\vspace*{-0.5cm}
  \label{fig:lagsensitivity}
\end{figure}

We have illustrated with specific examples several of the limitations of current measures. 
We now evaluate quantitatively the robustness of the proposed measures when compared to the currently used measures. 
We first evaluate the robustness to noise, lag, and normal versus abnormal points ratio. We then measure their ability to separate accurate and inaccurate methods.
%\newline \textbf{Sensitivity Analysis: } 
We first analyze the robustness of different approaches quantitatively to different factors: (i) lag, (ii) noise, and (iii) normal/abnormal ratio. As already mentioned, these factors are realistic. For instance, lag can be either introduced by the anomaly detection methods (such as methods that produce a score per subsequences are only high at the beginning of abnormal subsequences) or by human labeling approximation. Furthermore, even though lag and noises are injected, an optimal evaluation metric should not vary significantly. Therefore, we aim to measure the variance of the evaluation measures when we vary the lag, noise, and normal/abnormal ratio. We proceed as follows:

\begin{enumerate}[noitemsep,topsep=0pt,parsep=0pt,partopsep=0pt,leftmargin=0.5cm]
\item For each anomaly detection method, we first compute the anomaly score on a given time series.
\item We then inject either lag $l$, noise $n$ or change the normal/abnormal ratio $r$. For 10 different values of $l \in [-0.25*\ell,0.25*\ell]$, $n \in [-0.05*(max(S_T)-min(S_T)),0.05*(max(S_T)-min(S_T))]$ and $r \in [0.01,0.2]$, we compute the 13 different measures.
\item For each evaluation measure, we compute the standard deviation of the ten different values. Figure~\ref{fig:lagsensitivity}(b) depicts the different lag values for six AD methods applied on a data series in the ECG dataset.
\item We compute the average standard deviation for the 13 different AD quality measures. For example, figure~\ref{fig:lagsensitivity}(a) depicts the average standard deviation for ten different lag values over the AD methods applied on the MBA(805) time series.
\item We compute the average standard deviation for the every time series in each dataset (as illustrated in Figure~\ref{fig:sensitivity_per_data}(b to j) for nine datasets of the benchmark.
\item We compute the average standard deviation for the every dataset (as illustrated in Figure~\ref{fig:sensitivity_per_data}(a.1) for lag, Figure~\ref{fig:sensitivity_per_data}(a.2) for noise and Figure~\ref{fig:sensitivity_per_data}(a.3) for normal/abnormal ratio).
\item We finally compute the Wilcoxon test~\cite{10.2307/3001968} and display the critical diagram over the average standard deviation for every time series (as illustrated in Figure~\ref{fig:sensitivity}(a.1) for lag, Figure~\ref{fig:sensitivity}(a.2) for noise and Figure~\ref{fig:sensitivity}(a.3) for normal/abnormal ratio).
\end{enumerate}

%height=8.5cm,

\begin{figure}[tb]
  \centering
  \includegraphics[width=\linewidth]{figures/sensitivity_per_data_long.pdf}
%  %\vspace*{-0.3cm}
  \caption{Robustness Analysis for nine datasets: we report, over the entire benchmark, the average standard deviation of the accuracy values of the measures, under varying (a.1) lag, (a.2) noise, and (a.3) normal/abnormal ratio. }
  \label{fig:sensitivity_per_data}
\end{figure}

\begin{figure*}[tb]
  \centering
  \includegraphics[width=\linewidth]{figures/sensitivity_analysis.pdf}
  %\vspace*{-0.7cm}
  \caption{Critical difference diagram computed using the signed-rank Wilkoxon test (with $\alpha=0.1$) for the robustness to (a.1) lag, (a.2) noise and (a.3) normal/abnormal ratio.}
  \label{fig:sensitivity}
\end{figure*}

The methods with the smallest standard deviation can be considered more robust to lag, noise, or normal/abnormal ratio from the above framework. 
First, as stated in the introduction, we observe that non-threshold-based measures (such as AUC-ROC and AUC-PR) are indeed robust to noise (see Figure~\ref{fig:sensitivity_per_data}(a.2)), but not to lag. Figure~\ref{fig:sensitivity}(a.1) demonstrates that our proposed measures VUS-ROC, VUS-PR, R-AUC-ROC, and R-AUC-PR are significantly more robust to lag. Similarly, Figure~\ref{fig:sensitivity}(a.2) confirms that our proposed measures are significantly more robust to noise. However, we observe that, among our proposed measures, only VUS-ROC and R-AUC-ROC are robust to the normal/abnormal ratio and not VUS-PR and R-AUC-PR. This is explained by the fact that Precision-based measures vary significantly when this ratio changes. This is confirmed by Figure~\ref{fig:sensitivity_per_data}(a.3), in which we observe that Precision and Rprecision have a high standard deviation. Overall, we observe that VUS-ROC is significantly more robust to lag, noise, and normal/abnormal ratio than other measures.




\subsection{Separability Analysis}
\label{exp:separability}

%\newline \textbf{Separability Analysis: } 
We now evaluate the separability capacities of the different evaluation metrics. 
\commentRed{The main objective is to measure the ability of accuracy measures to separate accurate methods from inaccurate ones. More precisely, an appropriate measure should return accuracy scores that are significantly higher for accurate anomaly scores than for inaccurate ones.}
We thus manually select accurate and inaccurate anomaly detection methods and verify if the accuracy evaluation scores are indeed higher for the accurate than for the inaccurate methods. Figure~\ref{fig:separability} depicts the latter separability analysis applied to the MBA(805) and the SED series. 
The accurate and inaccurate anomaly scores are plotted in green and red, respectively. 
We then consider 12 different pairs of accurate/inaccurate methods among the eight previously mentioned anomaly scores. 
We slightly modify each score 50 different times in which we inject lag and noises and compute the accuracy measures. 
Figure~\ref{fig:separability}(a.4) and Figure~\ref{fig:separability}(b.4) are divided into four different subplots corresponding to 4 pairs (selected among the twelve different pairs due to lack of space). 
Each subplot corresponds to two box plots per accuracy measure. 
The green and red box plots correspond to the 50 accuracy measures on the accurate and inaccurate methods. 
If the red and green box plots are well separated, we can conclude that the corresponding accuracy measures are separating the accurate and inaccurate methods well. 
We observe that some accuracy measures (such as VUS-ROC) are more separable than others (such as RF). We thus measure the separability of the two box-plots by computing the Z-test. 

\begin{figure*}[tb]
  \centering
  \includegraphics[width=1\linewidth]{figures/pairwise_comp_example_long.pdf}
  %\vspace*{-0.5cm}
  \caption{Separability analysis applied on 4 pairs of accurate (green) and inaccurate (red) methods on (a) the MBA(805) data series, and (b) the SED data series.}
  %\vspace*{-0.3cm}
  \label{fig:separability}
\end{figure*}

We now aggregate all the results and compute the average Z-test for all pairs of accurate/inaccurate datasets (examples are shown in Figures~\ref{fig:separability}(a.2) and (b.2) for accurate anomaly scores, and in Figures~\ref{fig:separability}(a.3) and (b.3) for inaccurate anomaly scores, for the MBA(805) and SED series, respectively). 
Next, we perform the same operation over three different data series: MBA (805), MBA(820), and SED. 
Then, we depict the average Z-test for these three datasets in Figure~\ref{fig:separability_agg}(a). 
Finally, we show the average Z-test for all datasets in Figure~\ref{fig:separability_agg}(b). 


We observe that our proposed VUS-based and Range-based measures are significantly more separable than other current accuracy measures (up to two times for AUC-ROC, the best measures of all current ones). Furthermore, when analyzed in detail in Figure~\ref{fig:separability} and Figure~\ref{fig:separability_agg}, we confirm that VUS-based and Range-based are more separable over all three datasets. 

\begin{figure}[tb]
  \centering
  \includegraphics[width=\linewidth]{figures/agregated_sep_analysis.pdf}
  %\vspace*{-0.5cm}
  \caption{Overall separability analysis (averaged z-test between the accuracy values distributions of accurate and inaccurate methods) applied on 36 pairs on 3 datasets.}
  \label{fig:separability_agg}
\end{figure}


\noindent \textbf{Global Analysis: } Overall, we observe that VUS-ROC is the most robust (cf. Figure~\ref{fig:sensitivity}) and separable (cf. Figure~\ref{fig:separability_agg}) measure. 
On the contrary, Precision and Rprecision are non-robust and non-separable. 
Among all previous accuracy measures, only AUC-ROC is robust and separable. 
Popular measures, such as, F, RF, AUC-ROC, and AUC-PR are robust but non-separable.

In order to visualize the global statistical analysis, we merge the robustness and the separability analysis into a single plot. Figure~\ref{fig:global} depicts one scatter point per accuracy measure. 
The x-axis represents the averaged standard deviation of lag and noise (averaged values from Figure~\ref{fig:sensitivity_per_data}(a.1) and (a.2)). The y-axis corresponds to the averaged Z-test (averaged value from Figure~\ref{fig:separability_agg}). 
Finally, the size of the points corresponds to the sensitivity to the normal/abnormal ratio (values from Figure~\ref{fig:sensitivity_per_data}(a.3)). 
Figure~\ref{fig:global} demonstrates that our proposed measures (located at the top left section of the plot) are both the most robust and the most separable. 
Among all previous accuracy measures, only AUC-ROC is on the top left section of the plot. 
Popular measures, such as, F, RF, AUC-ROC, AUC-PR are on the bottom left section of the plot. 
The latter underlines the fact that these measures are robust but non-separable.
Overall, Figure~\ref{fig:global} confirms the effectiveness and superiority of our proposed measures, especially of VUS-ROC and VUS-PR.


\begin{figure}[tb]
  \centering
  \includegraphics[width=\linewidth]{figures/final_result.pdf}
  \caption{Evaluation of all measures based on: (y-axis) their separability (avg. z-test), (x-axis) avg. standard deviation of the accuracy values when varying lag and noise, (circle size) avg. standard deviation of the accuracy values when varying the normal/abnormal ratio.}
  \label{fig:global}
\end{figure}




\subsection{Consistency Analysis}
\label{sec:entropy}

In this section, we analyze the accuracy of the anomaly detection methods provided by the 13 accuracy measures. The objective is to observe the changes in the global ranking of anomaly detection methods. For that purpose, we formulate the following assumptions. First, we assume that the data series in each benchmark dataset are similar (i.e., from the same domain and sharing some common characteristics). As a matter of fact, we can assume that an anomaly detection method should perform similarly on these data series of a given dataset. This is confirmed when observing that the best anomaly detection methods are not the same based on which dataset was analyzed. Thus the ranking of the anomaly detection methods should be different for different datasets, but similar for every data series in each dataset. 
Therefore, for a given method $A$ and a given dataset $D$ containing data series of the same type and domain, we assume that a good accuracy measure results in a consistent rank for the method $A$ across the dataset $D$. 
The consistency of a method's ranks over a dataset can be measured by computing the entropy of these ranks. 
For instance, a measure that returns a random score (and thus, a random rank for a method $A$) will result in a high entropy. 
On the contrary, a measure that always returns (approximately) the same ranks for a given method $A$ will result in a low entropy. 
Thus, for a given method $A$ and a given dataset $D$ containing data series of the same type and domain, we assume that a good accuracy measure results in a low entropy for the different ranks for method $A$ on dataset $D$.

\begin{figure*}[tb]
  \centering
  \includegraphics[width=\linewidth]{figures/entropy_long.pdf}
  %\vspace*{-0.5cm}
  \caption{Accuracy evaluation of the anomaly detection methods. (a) Overall average entropy per category of measures. Analysis of the (b) averaged rank and (c) averaged rank entropy for each method and each accuracy measure over the entire benchmark. Example of (b.1) average rank and (c.1) entropy on the YAHOO dataset, KDD21 dataset (b.2, c.2). }
  \label{fig:entropy}
\end{figure*}

We now compute the accuracy measures for the nine different methods (we compute the anomaly scores ten different times, and we use the average accuracy). 
Figures~\ref{fig:entropy}(b.1) and (b.2) report the average ranking of the anomaly detection methods obtained on the YAHOO and KDD21 datasets, respectively. 
The x-axis corresponds to the different accuracy measures. We first observe that the rankings are more separated using Range-AUC and VUS measures for these two datasets. Figure~\ref{fig:entropy}(b) depicts the average ranking over the entire benchmark. The latter confirms the previous observation that VUS measures provide more separated rankings than threshold-based and AUC-based measures. We also observe an interesting ranking evolution for the YAHOO dataset illustrated in Figure~\ref{fig:entropy}(b.1). We notice that both LOF and MatrixProfile (brown and pink curve) have a low rank (between 4 and 5) using threshold and AUC-based measures. However, we observe that their ranks increase significantly for range-based and VUS-based measures (between 2.5 and 3). As we noticed by looking at specific examples (see Figure~\ref{exp:qual}), LOF and MatrixProfile can suffer from a lag issue even though the anomalies are well-identified. Therefore, the range-based and VUS-based measures better evaluate these two methods' detection capability.


Overall, the ranking curves show that the ranks appear more chaotic for threshold-based than AUC-, Range-AUC-, and VUS-based measures. 
In order to quantify this observation, we compute the Shannon Entropy of the ranks of each anomaly detection method. 
In practice, we extract the ranks of methods across one dataset and compute Shannon's Entropy of the different ranks. 
Figures~\ref{fig:entropy}(c.1) and (c.2) depict the entropy of each of the nine methods for the YAHOO and KDD21 datasets, respectively. 
Figure~\ref{fig:entropy}(c) illustrates the averaged entropy for all datasets in the benchmark for each measure and method, while Figure~\ref{fig:entropy}(a) shows the averaged entropy for each category of measures.
We observe that both for the general case (Figure~\ref{fig:entropy}(a) and Figure~\ref{fig:entropy}(c)) and some specific cases (Figures~\ref{fig:entropy}(c.1) and (c.2)), the entropy is reducing when using AUC-, Range-AUC-, and VUS-based measures. 
We report the lowest entropy for VUS-based measures. 
Moreover, we notice a significant drop between threshold-based and AUC-based. 
This confirms that the ranks provided by AUC- and VUS-based measures are consistent for data series belonging to one specific dataset. 


Therefore, based on the assumption formulated at the beginning of the section, we can thus conclude that AUC, range-AUC, and VUS-based measures are providing more consistent rankings. Finally, as illustrated in Figure~\ref{fig:entropy}, we also observe that VUS-based measures result in the most ordered and similar rankings for data series from the same type and domain.










\subsection{Execution Time Analysis}
\label{sec:exectime}

In this section, we evaluate the execution time required to compute different evaluation measures. 
In Section~\ref{sec:synthetic_eval_time}, we first measure the influence of different time series characteristics and VUS parameters on the execution time. In Section~\ref{sec:TSB_eval_time}, we  measure the execution time of VUS (VUS-ROC and VUS-PR simultaneously), R-AUC (R-AUC-ROC and R-AUC-PR simultaneously), and AUC-based measures (AUC-ROC and AUC-PR simultaneously) on the TSB-UAD benchmark. \commentRed{As demonstrated in the previous section, threshold-based measures are not robust, have a low separability power, and are inconsistent. 
Such measures are not suitable for evaluating anomaly detection methods. Thus, in this section, we do not consider threshold-based measures.}


\subsubsection{Evaluation on Synthetic Time Series}\hfill\\
\label{sec:synthetic_eval_time}

We first analyze the impact that time series characteristics and parameters have on the computation time of VUS-based measures. 
to that effect, we generate synthetic time series and labels, where we vary the following parameters: (i) the number of anomalies {\bf$\alpha$} in the time series, (ii) the average \textbf{$\mu(\ell_a)$} and standard deviation $\sigma(\ell_a)$ of the anomalies lengths in the time series (all the anomalies can have different lengths), (iii) the length of the time series \textbf{$|T|$}, (iv) the maximum buffer length \textbf{$L$}, and (v) the number of thresholds \textbf{$N$}.


We also measure the influence on the execution time of the R-AUC- and AUC- related parameter, that is, the number of thresholds ($N$).
The default values and the range of variation of these parameters are listed in Table~\ref{tab:parameter_range_time}. 
For VUS-based measures, we evaluate the execution time of the initial VUS implementation, as well as the two optimized versions, VUS$_{opt}$ and VUS$_{opt}^{mem}$.

\begin{table}[tb]
    \centering
    \caption{Value ranges for the parameters: number of anomalies ($\alpha$), average and standard deviation anomaly length ($\mu(\ell_a)$,$\sigma(\ell_a)$), time series length ($|T|$), maximum buffer length ($L$), and number of thresholds ($N$).}
    \begin{tabular}{|c|c|c|c|c|c|c|} 
 \hline
 Param. & $\alpha$ & $\mu(\ell_a)$ & $\sigma(\ell_{a})$ & $|T|$ & $L$ & $N$ \\ [0.5ex] 
 \hline\hline
 \textbf{Default} & 10 & 10 & 0 & $10^5$ & 5 & 250\\ 
 \hline
 Min. & 0 & 0 & 0 & $10^3$ & 0 & 2 \\
 \hline
 Max. & $2*10^3$ & $10^3$ & $10$ & $10^5$ & $10^3$ & $10^3$ \\ [1ex] 
 \hline
\end{tabular}
    \label{tab:parameter_range_time}
\end{table}


Figure~\ref{fig:sythetic_exp_time} depicts the execution time (averaged over ten runs) for each parameter listed in Table~\ref{tab:parameter_range_time}. 
Overall, we observe that the execution time of AUC-based and R-AUC-based measures is significantly smaller than VUS-based measures.
In the following paragraph, we analyze the influence of each parameter and compare the experimental execution time evaluation to the theoretical complexity reported in Table~\ref{tab:complexity_summary}.

\vspace{0.2cm}
\noindent {\bf [Influence of $\alpha$]}:
In Figure~\ref{fig:sythetic_exp_time}(a), we observe that the VUS, VUS$_{opt}$, and VUS$_{opt}^{mem}$ execution times are linearly increasing with $\alpha$. 
The increase in execution time for VUS, VUS$_{opt}$, and VUS$_{opt}^{mem}$ is more pronounced when we vary $\alpha$, in contrast to $l_a$ (which nevertheless, has a similar effect on the overall complexity). 
We also observe that the VUS$_{opt}^{mem}$ execution time grows slower than $VUS_{opt}$ when $\alpha$ increases. 
This is explained by the use of 2-dimensional arrays for the storage of predictions, which use contiguous memory locations that allow for faster access, decreasing the dependency on $\alpha$.

\vspace{0.2cm}
\noindent {\bf [Influence of $\mu(\ell_a)$]}:
As shown in Figure~\ref{fig:sythetic_exp_time}(b), the execution time variation of VUS, VUS$_{opt}$, and VUS$_{opt}^{mem}$ caused by $\ell_a$ is rather insignificant. 
We also observe that the VUS$_{opt}$ and VUS$_{opt}^{mem}$ execution times are significantly lower when compared to VUS. 
This is explained by the smaller dependency of the complexity of these algorithms on the time series length $|T|$. 
Overall, the execution time for both VUS$_{opt}$ and VUS$_{opt}^{mem}$ is significantly lower than VUS, and follows a similar trend. 

\vspace{0.2cm}
\noindent {\bf [Influence of $\sigma(\ell_a)$]}: 
As depicted in Figure~\ref{fig:sythetic_exp_time}(d) and inferred from the theoretical complexities in Table~\ref{tab:complexity_summary}, none of the measures are affected by the standard deviation of the anomaly lengths.

\vspace{0.2cm}
\noindent {\bf [Influence of $|T|$]}:
For short time series (small values of $|T|$), we note that O($T_1$) becomes comparable to O($T_2$). 
Thus, the theoretical complexities approximate to $O(NL(T_1+T_2))$, $O(N*(T_1+T_2))+O(NLT_2)$ and $O(N(T_1+T_2))$ for VUS, VUS$_{opt}$, and VUS$_{opt}^{mem}$, respectively. 
Indeed, we observe in Figure~\ref{fig:sythetic_exp_time}(c) that the execution times of VUS, VUS$_{opt}$, and VUS$_{opt}^{mem}$ are similar for small values of $|T|$. However, for larger values of $|T|$, $O(T_1)$ is much higher compared to $O(T_2)$, thus resulting in an effective complexity of $O(NLT_1)$ for VUS, and $O(NT_1)$ for VUS$_{opt}$, and VUS$_{opt}^{mem}$. 
This translates to a significant improvement in execution time complexity for VUS$_{opt}$ and VUS$_{opt}^{mem}$ compared to VUS, which is confirmed by the results in Figure~\ref{fig:sythetic_exp_time}(c).

\vspace{0.2cm}
\noindent {\bf [Influence of $N$]}: 
Given the theoretical complexity depicted in Table~\ref{tab:complexity_summary}, it is evident that the number of thresholds affects all measures in a linear fashion.
Figure~\ref{fig:sythetic_exp_time}(e) demonstrates this point: the results of varying $N$ show a linear dependency for VUS, VUS$_{opt}$, and VUS$_{opt}^{mem}$ (i.e., a logarithmic trend with a log scale on the y axis). \commentRed{Moreover, we observe that the AUC and range-AUC execution time is almost constant regardless of the number of thresholds used. The latter is explained by the very efficient implementation of AUC measures. Therefore, the linear dependency on the number of thresholds is not visible in Figure~\ref{fig:sythetic_exp_time}(e).}

\vspace{0.2cm}
\noindent {\bf [Influence of $L$]}: Figure~\ref{fig:sythetic_exp_time}(f) depicts the influence of the maximum buffer length $L$ on the execution time of all measures. 
We observe that, as $L$ grows, the execution time of VUS$_{opt}$ and VUS$_{opt}^{mem}$ increases slower than VUS. 
We also observe that VUS$_{opt}^{mem}$ is more scalable with $L$ when compared to VUS$_{opt}$. 
This is consistent with the theoretical complexity (cf. Table~\ref{tab:complexity_summary}), which indicates that the dependence on $L$ decreases from $O(NL(T_1+T_2+\ell_a \alpha))$ for VUS to $O(NL(T_2+\ell_a \alpha)$ and $O(NL(\ell_a \alpha))$ for $VUS_{opt}$, and $VUS_{opt}^{mem}$.





\begin{figure*}[tb]
  \centering
  \includegraphics[width=\linewidth]{figures/synthetic_res.pdf}
  %\vspace*{-0.5cm}
  \caption{Execution time of VUS, R-AUC, AUC-based measures when we vary the parameters listed in Table~\ref{tab:parameter_range_time}. The solid lines correspond to the average execution time over 10 runs. The colored envelopes are to the standard deviation.}
  \label{fig:sythetic_exp_time}
\end{figure*}


\vspace{0.2cm}
In order to obtain a more accurate picture of the influence of each of the above parameters, we fit the execution time (as affected by the parameter values) using linear regression; we can then use the regression slope coefficient of each parameter to evaluate the influence of that parameter. 
In practice, we fit each parameter individually, and report the regression slope coefficient, as well as the coefficient of determination $R^2$.
Table~\ref{tab:parameter_linear_coeff} reports the coefficients mentioned above for each parameter associated with VUS, VUS$_{opt}$, and VUS$_{opt}^{mem}$.



\begin{table}[tb]
    \centering
    \caption{Linear regression slope coefficients ($C.$) for VUS execution times, for each parameter independently. }
    \begin{tabular}{|c|c|c|c|c|c|c|} 
 \hline
 Measure & Param. & $\alpha$ & $l_a$ & $|T|$ & $L$ & $N$\\ [0.5ex] 
 \hline\hline
 \multirow{2}{*}{$VUS$} & $C.$ & 21.9 & 0.02 & 2.13 & 212 & 6.24\\\cline{2-7}
 & {$R^2$} & 0.99 & 0.15 & 0.99 & 0.99 & 0.99 \\   
 \hline
  \multirow{2}{*}{$VUS_{opt}$} & $C.$ & 24.2  & 0.06 & 0.19 & 27.8 & 1.23\\\cline{2-7}
  & $R^2$& 0.99 & 0.86 & 0.99 & 0.99 & 0.99\\ 
 \hline
 \multirow{2}{*}{$VUS_{opt}^{mem}$} & $C.$ & 21.5 & 0.05 & 0.21 & 15.7 & 1.16\\\cline{2-7}
  & $R^2$ & 0.99 & 0.89 & 0.99 & 0.99 & 0.99\\[1ex] 
 \hline
\end{tabular}
    \label{tab:parameter_linear_coeff}
\end{table}

Table~\ref{tab:parameter_linear_coeff} shows that the linear regression between $\alpha$ and the execution time has a $R^2=0.99$. Thus, the dependence of execution time on $\alpha$ is linear. We also observe that VUS$_{opt}$ execution time is more dependent on $\alpha$ than VUS and VUS$_{opt}^{mem}$ execution time.
Moreover, the dependence of the execution time on the time series length ($|T|$) is higher for VUS than for VUS$_{opt}$ and VUS$_{opt}^{mem}$. 
More importantly, VUS$_{opt}$ and VUS$_{opt}^{mem}$ are significantly less dependent than VUS on the number of thresholds and the maximal buffer length. 







\subsubsection{Evaluation on TSB-UAD Time Series}\hfill\\
\label{sec:TSB_eval_time}

In this section, we verify the conclusions outlined in the previous section with real-world time series from the TSB-UAD benchmark. 
In this setting, the parameters $\alpha$, $\ell_a$, and $|T|$ are calculated from the series in the benchmark and cannot be changed. Moreover, $L$ and $N$ are parameters for the computation of VUS, regardless of the time series (synthetic or real). Thus, we do not consider these two parameters in this section.

\begin{figure*}[tb]
  \centering
  \includegraphics[width=\linewidth]{figures/TSB2.pdf}
  \caption{Execution time of VUS, R-AUC, AUC-based measures on the TSB-UAD benchmark, versus $\alpha$, $\ell_a$, and $|T|$.}
  \label{fig:TSB}
\end{figure*}

Figure~\ref{fig:TSB} depicts the execution time of AUC, R-AUC, and VUS-based measures versus $\alpha$, $\mu(\ell_a)$, and $|T|$.
We first confirm with Figure~\ref{fig:TSB}(a) the linear relationship between $\alpha$ and the execution time for VUS, VUS$_{opt}$ and VUS$_{opt}^{mem}$.
On further inspection, it is possible to see two separate lines for almost all the measures. 
These lines can be attributed to the time series length $|T|$. 
The convergence of VUS and $VUS_{opt}$ when $\alpha$ grows shows the stronger dependence that $VUS_{opt}$ execution time has on $\alpha$, as already observed with the synthetic data (cf. Section~\ref{sec:synthetic_eval_time}). 

In Figure~\ref{fig:TSB}(b), we observe that the variation of the execution time with $\ell_a$ is limited when compared to the two other parameters. We conclude that the variation of $\ell_a$ is not a key factor in determining the execution time of the measures.
Furthermore, as depicted in Figure~\ref{fig:TSB}(c), $VUS_{opt}$ and $VUS_{opt}^{mem}$ are more scalable than VUS when $|T|$ increases. 
We also confirm the linear dependence of execution time on the time series length for all the accuracy measures, which is consistent with the experiments on the synthetic data. 
The two abrupt jumps visible in Figure~\ref{fig:TSB}(c) are explained by significant increases of $\alpha$ in time series of the same length. 

\begin{table}[tb]
\centering
\caption{Linear regression slope coefficients ($C.$) for VUS execution time, for all time series parameters all-together.}
\begin{tabular}{|c|ccc|c|} 
 \hline
Measure & $\alpha$ & $|T|$ & $l_a$ & $R^2$ \\ [0.5ex] 
 \hline\hline
 \multirow{1}{*}{${VUS}$} & 7.87 & 13.5 & -0.08 & 0.99  \\ 
 %\cline{2-5} & $R^2$ & \multicolumn{3}{c|}{ 0.99}\\
 \hline
 \multirow{1}{*}{$VUS_{opt}$} & 10.2 & 1.70 & 0.09 & 0.96 \\
 %\cline{2-5} & $R^2$ & \multicolumn{3}{c|}{0.96}\\
\hline
 \multirow{1}{*}{$VUS_{opt}^{mem}$} & 9.27 & 1.60 & 0.11 & 0.96 \\
 %\cline{2-5} & $R^2$ & \multicolumn{3}{c|}{0.96} \\
 \hline
\end{tabular}
\label{tab:parameter_linear_coeff_TSB}
\end{table}



We now perform a linear regression between the execution time of VUS, VUS$_{opt}$ and VUS$_{opt}^{mem}$, and $\alpha$, $\ell_a$ and $|T|$.
We report in Table~\ref{tab:parameter_linear_coeff_TSB} the slope coefficient for each parameter, as well as the $R^2$.  
The latter shows that the VUS$_{opt}$ and VUS$_{opt}^{mem}$ execution times are impacted by $\alpha$ at a larger degree than $\alpha$ affects VUS. 
On the other hand, the VUS$_{opt}$ and VUS$_{opt}^{mem}$ execution times are impacted to a significantly smaller degree by the time series length when compared to VUS. 
We also confirm that the anomaly length does not impact the execution time of VUS, VUS$_{opt}$, or VUS$_{opt}^{mem}$.
Finally, our experiments show that our optimized implementations VUS$_{opt}$ and VUS$_{opt}^{mem}$ significantly speedup the execution of the VUS measures (i.e., they can be computed within the same order of magnitude as R-AUC), rendering them practical in the real world.











\subsection{Summary of Results}


Figure~\ref{fig:overalltable} depicts the ranking of the accuracy measures for the different tests performed in this paper. The robustness test is divided into three sub-categories (i.e., lag, noise, and Normal vs. abnormal ratio). We also show the overall average ranking of all accuracy measures (most right column of Figure~\ref{fig:overalltable}).
Overall, we see that VUS-ROC is always the best, and VUS-PR and Range-AUC-based measures are, on average, second, third, and fourth. We thus conclude that VUS-ROC is the overall winner of our experimental analysis.

\commentRed{In addition, our experimental evaluation shows that the optimized version of VUS accelerates the computation by a factor of two. Nevertheless, VUS execution time is still significantly slower than AUC-based approaches. However, it is important to mention that the efficiency of accuracy measures is an orthogonal problem with anomaly detection. In real-time applications, we do not have ground truth labels, and we do not use any of those measures to evaluate accuracy. Measuring accuracy is an offline step to help the community assess methods and improve wrong practices. Thus, execution time should not be the main criterion for selecting an evaluation measure.}

\section{Results}
\label{sec:Results}

In this section, we present various analysis results that demonstrate the adoption of code obfuscation in Google Play.

\subsection{Overall Obfuscation Trends} 
\label{sec:obstrend}

\subsubsection{Presence of obfuscation} Out of the 548,967 Google Play Store APKs analyzed, we identified 308,782 obfuscated apps, representing approximately 56.25\% of the total. In Figure~\ref{fig:obfuscated_percentage}, we show the year-wise percentage of obfuscated apps for 2016-2023. There is an overall obfuscation increase of 13\% between 2016 and 2023, and as can be seen, the percentage of obfuscated apps has been increasing in the last few years, barring 2019 and 2020. As explained in Section~\ref{subsec:dataset}, 2019 and 2020 contain apps that are more likely to be abandoned by developers, and as such, they may not use advanced development practices.

\begin{figure}[h!]
\centering
    \includegraphics[width=\linewidth]{Figures/Only_obfuscation_trendV2.pdf}
    \caption{Percentage of obfuscated apps by year} \vspace{-4mm}
    \label{fig:obfuscated_percentage}
\end{figure}


From 2016 to 2018, the obfuscation levels were relatively stable at around 50-55\%, while from 2021 to 2023, there was a marked rise, reaching approximately 66\% in 2023. This indicates a growing focus on app protection measures among developers, likely driven by heightened security and IP concerns and the availability of advanced obfuscation tools.


\subsubsection{Obfuscation tools} Among the obfuscated APKs, our tool detector identified that 40.92\% of the apps use Proguard, 36.64\% use Allatori, 1.01\% use DashO, and 21.43\% use other (i.e., unknown) tools. We show the yearly trends in Figure~\ref{fig:ofbuscated_tool}. Note that we omit results in 2019 and 2020 ({\bf cf.} Section~\ref{subsec:dataset}).

ProGuard and Allatori are the most consistently used obfuscation tools, with ProGuard showing a slight overall increase in popularity and Allatori demonstrating variability. This inclination could be attributed to ProGuard being the default obfuscator integrated into Android Studio, a widely used development environment for Android applications. Notably, ProGuard usage increased by 13\% from 2018 to 2021, likely due to the introduction of R8 in April 2019~\cite{release_note_android}, which further simplified ProGuard integration with Android apps.

\begin{figure}[h]
\centering
    \includegraphics[width=\linewidth]{Figures/Initial_Tool_Trend_2019_dropV2.pdf} 
    \caption{Yearly obfuscation tool usage}
    \label{fig:ofbuscated_tool}
\end{figure}


DashO consistently remains low in usage, likely due to its high cost. The use of other obfuscation tools decreased until 2018 but has shown a resurgence from 2021 to 2023. This suggests that developers might be using other or custom tools, or our detector might be predicting some apps obfuscated with Proguard or Allatori as `other.' To investigate, we manually checked a sample of apps from the `other' category and confirmed they are indeed obfuscated. However, we could not determine which obfuscation tools the developers used. We discuss this potential limitation further in Section~\ref{sec:limitations}.


\subsubsection{Obfuscation techniques} We show the year-wise breakdown of obfuscation technique usage in Figure~\ref{fig:obfuscated_tech}. Among the various obfuscation techniques, Identifier Renaming emerged as the most prevalent, with 99.62\% of obfuscated apps using it alone or in combination with other methods (Categories of Only IR, IR and CF, IR and SE, or All three). Furthermore, 81.04\% of obfuscated apps used Control Flow Modification, and 62.76\% used String Encryption. The pervasive use of Identifier Renaming (IR) can be attributed to the fact that all obfuscation tools support it ({\bf cf.} Table~\ref{tab:ob_tool_cap}). Similarly, lower adoption of Control Flow Modification and String Encryption can be attributed to Proguard not supporting it. 

\begin{figure}[h]
\centering
    \includegraphics[width=\linewidth]{Figures/Initial_Tech_Trend_2019_dropV2.pdf} 
    \caption{Yearly obfuscation technique usage}
    \label{fig:obfuscated_tech}
\end{figure}



Next, we investigate the adoption of obfuscation on Google Play Store from various perspectives. Same as earlier, due to the smaller dataset size and possible bias ({\bf cf.} Section~\ref{subsec:dataset}), we exclude the APKs from 2019 and 2020 from this analyses.


\subsection{App Genre}
\label{sec:app_genre}

First, we investigate whether the obfuscation practices vary according to the App genre. Initially, we analysed all the APKs together before separating them into two snapshots.


\begin{figure*}[h]
    \centering
    \includegraphics[width=\linewidth]{Figures/AppGenreObfuscationV3.pdf}
    \caption{Obfuscated app percentage by genre (overall)}
    \label{fig:app_genre_overall}
\end{figure*}

Figure~\ref{fig:app_genre_overall} shows the genre-wise obfuscated app percentage. We note that 19 genres have more than 60\% of the apps obfuscated, and almost all the genres have more than 40\% obfuscation percentage. \textit{Casino} genre has the highest obfuscation percentage rate at 80\%, and overall, game genres tend to be more obfuscated than the other genres. The higher obfuscation usage in casino apps is logical due to their nature. These apps often simulate or involve gambling activities and handle monetary transactions and sensitive data related to in-game purchases, making them attractive targets for reverse engineering and hacking. This necessitates robust security measures to prevent fraud and protect user data. 


\begin{figure}[h]
    \centering
    \includegraphics[width=\linewidth]{Figures/AppGenre2018_2023ComparisonV3.pdf}
    \caption{Percentage of obfuscated apps by genre (2018-2023)}
    \label{fig:app_genre_comparison}
\end{figure}



\subsubsection{Genre-wise obfuscation trends in the two snapshots} To investigate the adoption of obfuscation over time, we study the two snapshots of Google Play separately, i.e., APKs from 2016-2018 as one group and APKs from 2021-2023 as another. 

Figure~\ref{fig:app_genre_comparison} illustrates the change in obfuscation levels by app genre between 2016-2018 to 2021-2023. Notably, app categories such as Education, Weather, and Parenting, which had obfuscation levels below the 2018 average, have increased to above the 2023 average by 2023. One possible reason for this in Education and Parenting apps can be the increase in online education activities during and after COVID-19 and the developers identifying the need for app hardening.

There are some genres, such as Casino and Action, for which the percentage of obfuscated apps didn't change across the two snapshots (i.e., purple and orange circles are close together in Figure~\ref{fig:app_genre_comparison}). This is because these genres are highly obfuscated from the beginning. Finally, the four genres, including Simulation and Role Playing, have a lower percentage of obfuscated apps in the 2021-2023 dataset. Our manual analysis didn't result in a conclusion as to why.


\begin{figure}[!h]
    \centering
    \includegraphics[width=\linewidth]{Figures/AppGenreTechAllV5.pdf}
    \caption{Obfuscation technique usage by genre (overall)}
    \label{fig:app_genre_all_tech}
\end{figure}


\subsubsection{Obfuscation techniques in different app genres} In Figure~\ref{fig:app_genre_all_tech}, we show the prevalence of key obfuscation techniques among various genres. As expected, almost all obfuscated apps in all genres used  Identifier Renaming. Also, it can be noted that genres with more obfuscated app percentages tend to use all three obfuscation techniques. Notably, more than 85\% of \textit{Casino} genre apps employ multiple obfuscation techniques

\subsubsection{Obfuscation tool usage in different app genres} We also investigated whether specific obfuscation tools are favoured by developers in different genres. However, apart from the expected observation that  ProGuard and Allatori being the most used tools, we didn't find any other interesting patterns. Therefore, we haven't included those measurement results.

\subsection{App Developers}
Next, we investigate individual developer-wise code obfuscation practices. From the pool of analyzed APKs, we identified the number of apps associated with each developer. Subsequently, we sorted the developers according to the number of apps they had created and selected the top 100 developers with the highest number of APKs for the 2016-2018 and 2021-2023 datasets. For the 2018 snapshot, we had 8,349 apps among the top 100 developers, while for the 2023 snapshot, we had 11,338 apps among the top 100 developers.

We then proceeded to detect whether or not these developers obfuscate their apps and, if so, what kind of tools and techniques they use. We present our results in five levels; developer obfuscating over 80\% of their apps, 60\%--80\% of apps, 40\%--60\% of apps, less than 40\%, and no obfuscation.

Figure~\ref{fig:developer_trend_my_apps_all} compares the two datasets in terms of developer obfuscation adoption. It shows that more developers have moved to obfuscate more than 80\% of their apps in the 2021-2023 dataset (76\%) compared to the 2016-2018 dataset (48\%).

We also found that among developers who obfuscate more than 80\% of their apps, 73\% in 2018 and 93\% in 2023 used the same obfuscation tool. Additionally, these top developers employ Control Flow Modification (CF) and String Encryption (SE) above the average values discussed in Section~\ref{sec:obstrend}. Specifically, in 2018, top developers used CF in 81.3\% of cases and SE in 66.7\%, while in 2023, these figures increased to 88.2\% and 78.9\%. This results in two insights: 1) Most top developers obfuscate all their apps with advanced techniques, possibly due to concerns about IP and security, and 2) Developers stick to a single tool, possibly due to specialized knowledge or because they bought a commercial licence.

\begin{figure}[]
    \centering
    \includegraphics[width=\linewidth]{Figures/Developer_Analysed_Comparison.pdf}
    \caption{Obfuscation usage (Top-100 developers)}
    \label{fig:developer_trend_my_apps_all}
\end{figure}


Finally, we investigate the obfuscation practices of developers with only one app in Table~\ref{tab:my-table}. According to the table, from those developers, 45.5\% of them obfuscated their apps in the 2016-2018 dataset and 57.2\% obfuscated their apps in the 2021-2023 dataset, showing a clear increase. However, these percentages are approximately 10\% lower than the average obfuscation rate in both cohorts discussed in Section~\ref{sec:obstrend}. This indicates that single-app developers may be less aware or concerned about code protection.


\begin{table}[]
\caption{Developers with only one app}
\label{tab:my-table}
\resizebox{\columnwidth}{!}{%
\begin{tabular}{cccccc}
\hline
\textbf{Year} & \textbf{\begin{tabular}[c]{@{}c@{}}Non\\ Obfuscated\end{tabular}} & \multicolumn{4}{c}{\textbf{Obfuscated}} \\ \hline
\multirow{3}{*}{\textbf{\begin{tabular}[c]{@{}c@{}}2018 \\ Snapshot\end{tabular}}} & \multirow{3}{*}{\begin{tabular}[c]{@{}c@{}}26,581 \\ (54.5\%)\end{tabular}} & \multicolumn{4}{c}{\begin{tabular}[c]{@{}c@{}}22,214 (45.5\%)\end{tabular}} \\ \cline{3-6} 
 &  & \textbf{ProGuard} & \textbf{Allatori} & \textbf{DashO} & \textbf{Other} \\ \cline{3-6} 
 &  & 6,131 & 8,050 & 658 & 7,375 \\ \hline
\multirow{3}{*}{\textbf{\begin{tabular}[c]{@{}c@{}}2023 \\ Snapshot\end{tabular}}} & \multirow{3}{*}{\begin{tabular}[c]{@{}c@{}}19,510 \\ (42.8\%)\end{tabular}} & \multicolumn{4}{c}{\begin{tabular}[c]{@{}c@{}}26,084 (57.2\%)\end{tabular}} \\ \cline{3-6} 
 &  & \textbf{ProGuard} & \textbf{Allatori} & \textbf{DashO} & \textbf{Other} \\ \cline{3-6} 
 &  & 12,697 & 9,672 & 234 & 3,581 \\ \hline
\end{tabular}%
}
\end{table}

\subsection{Top-k Apps}

Next, we investigate the obfuscation practices of top apps in Google Play Store. First, we rank the apps using the same criterion used by our previous work~\cite{rajasegaran2019multi, karunanayake2020multi, seneviratne2015early}. That is, we sort the apps in descending order of number of downloads, average rating, and rating count, with the intuition that top apps have high download numbers and high ratings, even when reviewed by a large number of users. Then, we investigated the percentage of obfuscated apps and obfuscation tools and technique usage as summarized in Table~\ref{tab:top_k_apps_2018_2023}.

When considering the highly ranked applications (i.e., top-1,000), the obfuscation percentage is notably higher, at around 93\%, in both datasets, which is significantly higher than the average percentage of obfuscation we observed in Section~\ref{sec:obstrend}. Top-ranked apps, likely due to their higher visibility and potential revenue, invest more in obfuscation to safeguard their intellectual property and enhance security. 

The obfuscation percentage decreases when going from the top 1,000 apps to the top 30,000 apps. Nonetheless, the obfuscation percentage in both datasets remains around similar values until the top 30,000 (e.g., $\sim$74\% for top-30,000). This indicates that the major increase in obfuscation in the 2021-2023 dataset comes from apps beyond the top 30,000.

When observing the tools used, the usage of ProGuard increases as we move from top to lower-ranked apps in both datasets. This may be because ProGuard is free and the default in Android Studio, while commercial tools like Allatori and DashO are expensive. There is a notable increase in the use of Allatori among the top apps in the 2021-2023 dataset. Regarding obfuscation techniques, the top 1,000 apps utilize all three techniques more frequently than lower-ranked apps in both snapshots. This indicates that the top 1,000 apps are more heavily protected compared to lower-ranked ones.

\begin{table*}[]
\caption{Summary of analysis results for Top-k apps in 2018 and 2023}
\label{tab:top_k_apps_2018_2023}
\resizebox{\textwidth}{!}{%
\begin{tabular}{lccccccccc}
\hline
\multicolumn{1}{c}{\begin{tabular}[c]{@{}c@{}}Top k apps - \\ Year\end{tabular}} & \begin{tabular}[c]{@{}c@{}}Total \\ Apps\end{tabular} & \begin{tabular}[c]{@{}c@{}}Obfuscation\\ Percentage\end{tabular} & \begin{tabular}[c]{@{}c@{}}ProGuard\\ Percentage\end{tabular} & \begin{tabular}[c]{@{}c@{}}Allatori\\ Percentage\end{tabular} & \begin{tabular}[c]{@{}c@{}}DashO\\ Percentage\end{tabular} & \begin{tabular}[c]{@{}c@{}}Other\\ Percentage\end{tabular} & \begin{tabular}[c]{@{}c@{}}IR\\ Percentage\end{tabular} & \begin{tabular}[c]{@{}c@{}}CF\\ Percentage\end{tabular} & \begin{tabular}[c]{@{}c@{}}SE\\ Percentage\end{tabular} \\ \hline
1k (2018) & 1,000 & 93.40 & 29.98 & 28.48 & 0.64 & 40.90 & 99.90 & 88.76 & 65.42 \\
10k (2018) & 10,000 & 85.19 & 25.55 & 35.32 & 0.47 & 38.65 & 99.90 & 88.76 & 71.91 \\
20k (2018) & 20,000 & 78.42 & 26.31 & 36.76 & 0.57 & 36.36 & 99.87 & 87.37 & 71.49 \\
30k (2018) & 30,000 & 74.40 & 27.30 & 37.71 & 0.64 & 34.36 & 99.82 & 86.75 & 71.11 \\
30k+ (2018) & 314,568 & 53.36 & 36.72 & 34.70 & 1.33 & 27.24 & 99.34 & 83.54 & 63.11 \\ \hline
1k (2023) & 1,000 & 92.50 & 24.00 & 51.89 & 1.95 & 22.16 & 100.0 & 92.54 & 83.68 \\
10k (2023) & 10,000 & 81.88 & 26.03 & 56.20 & 1.03 & 16.74 & 99.89 & 89.40 & 82.01 \\
20k (2023) & 20,000 & 76.62 & 30.48 & 52.92 & 0.96 & 15.64 & 99.93 & 85.80 & 78.01 \\
30k (2023) & 30,000 & 73.72 & 33.87 & 50.34 & 0.89 & 14.90 & 99.95 & 83.31 & 75.34 \\
30k+ (2023) & 206,216 & 61.90 & 46.56 & 38.21 & 0.64 & 14.59 & 99.97 & 77.51 & 62.50 \\ \hline
\end{tabular}%
}
\end{table*}


\section{Related Works}
The intersection of AI and network management has prompted several innovative approaches, each aimed at enhancing the adaptability and efficiency of network systems. 
%This section discusses key contributions and methodologies from recent literature that align closely with the themes of leveraging AI to improve network operations.
NetGPT \cite{chen2024netgpt} has been developed as an AI-native network architecture that strategically deploys LLMs both at the edge and cloud to optimize personalization and efficiency. The architecture highlights improvements in network management and user intent inference by integrating communications and computing resources more deeply \cite{tong2023ten}. Similarly, NetLM \cite{wang2023network} introduces an AI-driven architecture to enhance autonomous capabilities in network management, notably in complex 6G environments. The system leverages multi-modal representation learning to integrate diverse network data, aiming to refine network intents and autonomously manage network operations.
ABC (Automatic Bottom-up Construction) \cite{ding2023abc} revolutionizes the configuration knowledge base for multi-vendor networks by automating the alignment and generation of configuration templates through natural language processing and active learning, significantly reducing the manual effort typically required.
CONFPILOT \cite{zhao2023confpilot} employs a retrieval-augmented generation framework to translate natural language intents into precise network configuration commands. This system not only accelerates configuration processes but also enhances accuracy with its innovative use of a retrained BERT model and a parameter description-enhanced BM25 algorithm, which together improve the retrieval and matching of network commands.
NetCR \cite{guo2023netcr} utilizes a knowledge graph to facilitate manual network configurations, providing adaptive recommendations that enhance the efficiency and accuracy of network operations across various devices. This tool underscores the potential of using structured knowledge to streamline network management tasks in multi-vendor environments. To the best of our knowledge, Text2Net is the first initiative that directly integrates AI, specifically NLP, into network simulation for educational purposes and beyond. While prior works have explored the use of AI to enhance network management and configuration, Text2Net uniquely applies these technologies to simplify and democratize the learning and execution processes in network simulations. 
\begin{figure}[t!]
    \centering
    \includegraphics[width=\columnwidth]{Figures/system_model5_svg-raw.pdf}
    \caption{Text2Net system model and pipeline}
    \vspace{-4mm}
    \label{fig: system model}
\end{figure}
\section{Conclusion}
In this work, we introduced BnTTS, the first speaker-adaptive TTS system for Bangla, capable of generating natural and clear speech with minimal training data. Built on the XTTS pipeline, BnTTS effectively supports zero-shot and few-shot speaker adaptation, outperforming existing Bangla TTS systems in sound quality, naturalness, and clarity. Despite its strengths, BnTTS faces challenges in handling diverse dialects and short-sequence generation. Future work will focus on training BnTTS from scratch, developing medium and small model variants, and exploring knowledge distillation to optimize inference speed for real-time applications.

% In this work, we introduce BnTTS, the first speaker-adaptive TTS system for Bangla, achieving natural and clear speech synthesis with minimal training data. By adopting the XTTS pipeline, BnTTS supports zero-shot and few-shot speaker adaptation, outperforming existing Bangla TTS systems in sound quality, naturalness, and clarity. However, it faces limitations with diverse dialects, and struggles with short sequences. Future improvements will focus on training BnTTS from scratch, with medium and small variants and also distrilling base model into smaller model for faster inference in realtime infeerence.
% expanding datasets, using more advanced models, and incorporating multilingual support to enhance its versatility.
% \section{Conclusion}

% In this study, we introduce BnTTS, the first open-source speaker adaptation based TTS system, which improves speech generation for Bangla, a low-resource language. By adapting the XTTS pipeline to accommodate Bangla's unique phonetic characteristics, the model is able to produce natural, clear, and accurate speech with minimal training data, supporting both zero-shot and few-shot speaker adaptation. BnTTS outperforms existing Bangla TTS systems in terms of speed, sound quality, and clarity, as confirmed by listener ratings. 

\section{Limitations}
Despite the significant performance of BnTTS, the system has several limitations. It struggles to adapt to speakers with unique vocal traits, especially without prior training on their voices, limiting its effectiveness in speaker adaptation tasks. We found poor performance on short text due to pre-existing issues in the XTTS foundation model. Although we improved performance by modifying generation settings and incorporating additional training with Complete Audio Prompting, the model still fails to generate sequences under two words or 20 characters in some cases. We did not investigate the performance of the XTTS model by training from scratch; instead, we used continual pretraining due to resource constraints, which may have yielded better results.

\section{Acknowledgments}
We are grateful to HISHAB\footnote{\url{https://www.verbex.ai/}}  for providing us with
all the necessary working facilities, computational
resources, and an appropriate environment through-
out our entire work.
% \textcolor{green}{
% \begin{itemize}
%     \item training from scratch is not possible
%     \item Subjective evaluation may change across the experiment
%     \item generated speech may change across inference
%     \item BengaliNamedEntity1000 only 200 are selected
%     5*1000 + 5*1000 + 1000+1000+1000
%     200 * 
% \end{itemize}
% }

% Despite the significant performances of BnTTS, the system has several limitations. It struggles to adapt to speakers with unique vocal traits, especially without prior training on their voices, limiting its effectiveness in speaker adaptation tasks. We found poor performance on short text due to pre-existing issues in the XTTS foundation model. Although we improved performance by modifying generation settings and incorporating additional training with full audio-text prompting, the model still fails to generate sequences under two words or 20 characters in some cases. We don't investigate the performance from XXTS model by training from strach instead we use continul pretraining due to resource contraints , this may provide better result. 
% As we did continual pretraining on 

% Future works include scratch from training using ba

% We found poor performance on short text becasure the prior issues existing in XXTS foundation model. Although, we improve the performance using some modification is generation setting and using an additional training with full audio text in prompting, still it fails to generated sequence under 2 words or 20 characters in some cases. 

% XTTS, the foundation for BnTTS, performs poorly on very short sequences. 



% While adjustments to hyperparameters and generation settings (temperature and TopK) have mostly resolved this issue, instances remain where the model struggles to generate sequences under 2 words or 20 characters. 

% - the system struggles with adapting to speakers with unique vocal traits, especially without prior training on their voices.

% - Its focus on Bangla also restricts its usefulness for other low-resource languages.

% - XTTS, the foundation for BnTTS, performs poorly on very short sequences. While adjustments to hyperparameters and generation settings (temperature and TopK) have mostly resolved this issue, instances remain where the model struggles to generate sequences under 2 words or 20 characters.

% - training from scratch is not possible
% - Subjective evaluations are subject to change because the speech generated by models may exhibit variations across different inference sessions.
% -  Evaluation is time-consuming, so we have to select 200 sentences from the BengaliNamedEntity1000 eval dataset with 1000 sentences. 



% It relies on a small and uniform dataset, which limits its ability to work well with the diverse dialects and accents in Bangla, potentially affecting the naturalness of the speech output for certain regional variations. 


% Some key observations include that training from scratch the original XTTS does not converge for our dataset, which may be due to insufficient amounts of training data. Subjective evaluations are subject to change because the speech generated by models may exhibit variations across different inference sessions. This variability could lead to differences in results and affect the reliability of the evaluations. Evaluation is time-consuming, so we have to select 200 sentences from the BengaliNamedEntity1000 eval dataset with 1000 sentences. 

% BnTTS has some limitations. It relies on a small and uniform dataset, which limits its ability to work well with the diverse dialects and accents in Bangla, potentially affecting the naturalness of the speech output for certain regional variations. The use of GPT-2, chosen due to limited resources, may also limit the system’s scalability and performance compared to newer models. Additionally, the system struggles with adapting to speakers with unique vocal traits, especially without prior training on their voices. Its focus on Bangla also restricts its usefulness for other low-resource languages. Furthermore, XTTS, the foundation for BnTTS, performs poorly on very short sequences. While adjustments to hyperparameters and generation settings (temperature and TopK) have mostly resolved this issue, instances remain where the model struggles to generate sequences under 2 words or 20 characters. Future improvements could involve a larger dataset, more advanced models, and multilingual support to boost its versatility and robustness.

\section{Ethical Considerations}

The development of BnTTS raises ethical concerns, particularly regarding the potential misuse for unauthorized voice impersonation, which could impact privacy and consent. Protections, such as requiring speaker approval and embedding markers in synthetic speech, are essential. Diverse training data is also crucial to reduce bias and reflect Bangla’s dialectal variety. Additionally, synthesized voices risk diminishing dialectal diversity. As an open-source tool, BnTTS requires clear guidelines for responsible use, ensuring adherence to ethical standards and positive community impact.


\bibliography{custom}



\newpage
\appendix
% \clearpage
% \section{Appendix}

% App A: Dataset
% App B: Data Aquization Framework
% C: Eval Data
% D: Eval Metrics
% E: Model Architecture
% F:Training Objective

% \input{sections/Related Works}

\section{TTS Data Acquisition Framework}
\label{sec:data_collection}

\begin{figure}[hbt!]
    \centering
    \includegraphics[width=0.8\linewidth]{resources/TTS_Data_Collection_Pipeline.png} 
    \caption{Overview of our TTS Data Acquisition Framework. The acquisition process involves using a Speech-to-Text model to obtain transcription, an LLM to restore transcription's punctuation, a noise suppression model to remove unwanted noise, and finally an audio superresolution model to enhance audio quality and loudness.}
    \label{fig:pseudo_labeled_dataset}
\end{figure}

Bangla is a low-resource language, and large-scale, high-quality TTS speech data are particularly scarce. To address this gap, we developed a TTS Data Acquisition Framework (Figure \ref{fig:pseudo_labeled_dataset}) designed to collect high-quality speech data with aligned transcripts. This framework leverages advanced speech processing models and carefully designed algorithms to process raw audio inputs and generate refined audio outputs with word-aligned transcripts. Below, we provide a detailed breakdown of the key components of the framework.


\textbf{1. Speech-to-Text (STT):} The audio files are first processed through an in-house our STT system, which transcribes the spoken content into text. The STT system used here is an enhanced version of the model proposed in \cite{nandi-etal-2023-pseudo}.

\textbf{2. Punctuation Restoration Using LLM:} Following transcription, a LLM is employed to restore appropriate punctuation \cite{openai2023gpt}. This step is crucial for improving grammatical accuracy and ensuring that the text is clear and coherent, aiding in further processing.

\textbf{3. Audio and Transcription Segmentation:} The audio and transcription are segmented based on terminal punctuation (full-stop, question mark, exclamatory mark, comma). This ensures that each audio segment aligns with a complete sentence, maintaining the speaker's prosody throughout.

\textbf{4. Noise and Music Suppression:} To improve audio quality, noise and music suppression techniques \cite{defossez2019music} are applied. This step ensures that the resulting audio is free of background disturbances, which could degrade TTS performance.

\textbf{5. Audio SuperResolution:} After noise suppression, the audio files undergo super-resolution processing to enhance audio fidelity \cite{liu2021voicefixer}. This ensures high-quality audio, crucial for producing natural-sounding TTS outputs.


This pipeline effectively enhances raw audio and corresponding transcription, resulting in a high-quality pseudo-labeled dataset. By combining ASR, LLM-based punctuation restoration, noise suppression, and super-resolution, the framework can generate very high-quality speech data suitable for training speech synthesis models.

\subsection{Dataset Filtering Criteria}
The pseudo-labeled data are further refined using the following criteria:

\begin{itemize}
    \item\textbf{Diarization:} Pyannote's Speaker Diarization v3.1  is employed to filter audio files by separating multi-speaker audios, ensuring that each instance contains only one speaker \cite{Plaquet23}, which is essential for effective TTS model training.

    \item \textbf{Audio Duration}: Audio segments shorter than 0.5 seconds are discarded, as they provide insufficient information for our model. Similarly, segments longer than 11 seconds are excluded to match the model’s sequence length.
    
    \item \textbf{Text Length}: Segments with transcriptions exceeding 200 characters are removed to ensure manageable input size during training.
    \item \textbf{Silence-based Filtering}: Audio files where over 35\% of the duration consists of silence are discarded, as they negatively impact model performance.
    \item \textbf{Text-to-Audio Ratio}: Based on our analysis, audio segments where the text-to-audio duration ratio falls outside (Figure \ref{fig:unprocessed_data}) the range of 6 to 25 are excluded (Figure \ref{fig:processed_data}), ensuring alignment with natural speech patterns observed in Pseudo-labeled data from Phase A (Figure \ref{fig:reviewed_data}).
\end{itemize}



\begin{figure}[hbt!]
    \centering
    % First Image: Unprocessed Data
    \begin{subfigure}[b]{0.45\textwidth}
        \centering
        \includegraphics[width=\textwidth]{resources/reviewd_data_100.png}
        \caption{The diagram illustrates the linear relationship between audio duration and character length in manually-reviewed Pseudo-labeled Data - Phase A.}
        \label{fig:reviewed_data}
    \end{subfigure}
    \hfill
    % Second Image: Processed Data
    \begin{subfigure}[b]{0.45\textwidth}
        \centering
        \includegraphics[width=\textwidth]{resources/700h_unprocessed.png}
        \caption{The diagram depicts the relationship between audio duration and character length in Pseudo-Labeled Data - Phase B.}
        \label{fig:unprocessed_data}
    \end{subfigure}
    \vskip\baselineskip
    % Third Image: Reviewd Data
    \begin{subfigure}[b]{0.45\textwidth}
        \centering
        \includegraphics[width=\textwidth]{resources/700h_processed.png}
        \caption{The diagram illustrates the audio duration vs. character length graph in Pseudo-Labeled Data - Phase B after filtering.}
        \label{fig:processed_data}
    \end{subfigure}
    \caption{These figures demonstrate how the ratio of text length to audio duration changes before and after processing the data.}
    \label{fig:audio_vs_length_grid}
\end{figure}


% \textcolor{red}{\subsection{Quality Control for Pseudo-Labeled data} These are already described in A1. Plese include phase A data review process by reviewer team} 
% Given the importance of data quality in the pseudo-labeling process, multiple manual and 
% automated verification steps are implemented to ensure accurate alignment between audio and 
% transcriptions. To address potential errors in automated methods, manual verification is conducted, focusing on key areas such as: 

% \noindent \textbf{Semantic Accuracy \& Punctuation Verification} This involves ensuring that the punctuation restoration process has preserved the intended meaning of the transcriptions and correcting any misinterpretations in punctuation placement.


% \noindent \textbf{Segmentation Accuracy}
% The process includes reviewing whether the sentence segmentation has been executed correctly, ensuring that transcriptions are chunked without breaking context. Additionally, manual adjustments are made to the segmentation whenever mis-segmentation errors are detected.


% \noindent \textbf{Speaker Diarization Validation} 
% This stage involves ensuring that each audio segment contains speech from only one speaker and does not contain overlapping speech. Additionally, any incorrectly diarized segments are identified and filtered out to maintain the clarity and accuracy of the speaker attribution in the dataset.

% \noindent \textbf{ASR Inference Correction} 
% The process includes checking if the ASR (Automatic Speech Recognition) model has misinterpreted words or phrases. Corrections are manually applied to the transcriptions for any inaccuracies identified, ensuring the accuracy of the transcribed text.


% \noindent \textbf{Audio Duration vs. Text Length Filtering}
% This involves applying a duration-to-text ratio filter to remove audio segments that are either too short or too long compared to their corresponding transcriptions. This step is critical to maintaining accurate audio-to-text alignment throughout the dataset.

\section{Human Guided Data Preparation}
\label{app:human_reviewed_data}
We curated approximately 82.39 hours of speech data through human-level observation, which we refer to as Pseudo-Labeled Data - Phase A (Table \ref{tab:dataset_info}). The audio samples, averaging 10 minutes in duration, are sourced from copyright-free audiobooks and podcasts, preferably featuring a single speaker in most cases.

Annotators were tasked with identifying prosodic sentences by segmenting the audio into meaningful chunks while simultaneously correcting ASR-generated transcriptions and restoring proper punctuation in the provided text. If a selected audio chunk contained multiple speakers, it was discarded to maintain dataset consistency. Additionally, background noise, mispronunciations, and unnatural speech patterns were carefully reviewed and eliminated to ensure the highest quality TTS training data.

\begin{table}[t]
  \centering
    \caption{Dataset statistics}
    \resizebox{0.49\textwidth}{!}{
        % \begin{threeparttable}

\begin{tabular}{lrrrrr}
    \toprule
    \textbf{ DATASET } & \textbf{ \#Users } & \textbf{ \#Items } & \textbf{\#Interactions}  & \textbf{Avg.Inter.} & \textbf{Sparsity}\\
    \midrule
     Gowalla  & 29,858 & 40,981& 1,027,370 & 34.4 & 99.92\% \\
     Yelp2018  & 31,668 & 38,048 & 1,561,406 & 49.3 & 99.88\% \\ 
     MIND  & 38,441 & 38,000 & 1,210,953 & 31.5 & 99.92\% \\ 
     MIND-Large  & 111,664 & 54,367 & 3,294,424 & 29.5 & 99.95\% \\
    \bottomrule
    \end{tabular}
        % \end{threeparttable}
    }
  \label{tab:datasets}%
\end{table}%
% 

\section{Evaluation Dataset}

For evaluating the performance of our TTS system, we curated two datasets: BnStudioEval and BnTTSTextEval, each serving distinct evaluation purposes.

\begin{itemize}
    \item \textbf{BnStudioEval}: This dataset comprises 100 high-quality instances (text and audio pair) taken from our in-house studio recordings. This dataset was selected to assess the model’s capability in replicating high-fidelity speech output with speaker impersonation. 
    
    \item \textbf{BnTTSTextEval}: The BnTTSTextEval dataset encompasses three subsets: \begin{itemize}
        \item \textbf{BengaliStimuli53}: A linguist-curated set of 53 instances, created to cover a comprehensive range of Bengali phonetic elements. This subset ensures that the model handles diverse phonemes.
        % Sample example: তোমাদের পড়াশুনা ধ্যানি ও গুরুমুখী নয়, বরং বাজারভিত্তিক।পূর্বসূরী অনেকের মত তোমরাও অনেকাংশে নিজের চিত্ত ও বোধকে গবেষণায় পূর্ণ করার চেয়ে ঘুষ খেয়ে ঘরে ফার্নিচারের মেলা বসাতে চাও, আর ঘৃত দিয়ে কিচেন।
        
        \item \textbf{BengaliNamedEntity1000}: A set of 1,000 instances focusing on proper nouns such as person, place, and organization names. This subset tests the model's handling of named entities, which is crucial for real-world conversational accuracy.
        % Sample Example: ময়মনসিংহ বিভাগের জেলাগুলো হচ্ছে শেরপুর, ময়মনসিংহ, জামালপুর, নেত্রকোণা।
        \item \textbf{ShortText200}: Composed of 200 instances, this subset includes short sentences  filler words, and common conversational phrases (less than three words) to evaluate the model’s performance in natural, day-to-day dialogue scenarios.
        % Sample Example: কি বলছেন?
    \end{itemize}  
\end{itemize}

The BnStudioEval dataset, with reference audio for each text, will be for reference-aware evaluation, while BnTTSTextEval supports reference-independent evaluation. Together, these datasets provide a comprehensive basis for evaluating various aspects of our TTS performance, including phonetic diversity, named entity pronunciation, and conversational fluency. 



% \section{Model Architecture}
% The model architecture consists of the following trainable components:

% % \textbf{VQ-VAE:}
% % The DiscreteVAE architecture consists of an encoder, decoder, and a quantization codebook. The encoder uses 2 Conv1d layers with strides of 2, reducing input dimensionality, followed by 3 residual blocks with 1024 channels, each with ReLU activations. The decoder mirrors the encoder, starting with a Conv1d layer, followed by 3 residual blocks and 2 upsampled convolution layers, reconstructing the original input. A Quantize layer is used for vector quantization. The architecture utilizes the DiscretizationLoss function for learning discrete latent representations. This module efficiently encodes and decodes spectrograms with a total parameter count of around 51 million.
% \textbf{Conditioning Encoder and Perceiver Resampler:}
% The Conditioning Encoder \cite{casanova2024xtts} consists of an initial Conv1d layer with 80 input channels and 1024 output channels, followed by 6 Attention blocks. Each Attention block includes a Group normalization layer (32 groups, 1024 dimensions), a Conv1d layer for query-key-value computation (1024 input channels, 3072 output channels), and a final projection Conv1d layer (1024 output channels). The attention mechanism utilizes QKV attention. Dropout with a probability of 0.1 is applied to facilitate regularization. The encoder outputs a sequence, which length is dependent on the input audio duration.

% The Conditioning Encoder is followed by the Perceiver Resampler, which produces a fixed number of embeddings by utilizing cross attention mechanism. The Perceiver Resampler is composed two attention blocks, each with 512-dimensional queries and 1024-dimensional keys and values. The module includes sequential layers with linear projections and GELU activations. For normalization, it uses RMS norm.
% The total number of parameters in Conditioning Encoder and Perceiver Resampler are approximately 25.29 million and 21 millions respectively.


% \textbf{LLM:}
% For LLM, we use a GPT-2 \cite{radford2019language} model with approximately 377.89 million parameters. The GPT-2 is consists of 30 transformer blocks, each with 16 attention heads and a hidden dimension of 1024. It uses layer normalization and attention mechanisms, with the MLP blocks containing two linear layers and a GELU activation.


% \textbf{HiFi-GAN Decoder:}
% The HiFi-GAN Decoder \cite{kong2020hifi} consists of a waveform generator with multiple convolutional layers and residual blocks. It includes 4 parametrized ConvTranspose1d layers for upsampling, followed by a series of residual blocks with various dilation rates. The total number of parameters is 25.86 million. This submodule is responsible for converting intermediate GPT-2 latent representations into high-quality waveform outputs


% \section{Objective Functions}
% \subsection{Language Modeling Loss}

% \textbf{Text Token Prediction} loss, denoted as $\mathcal{L}_{\text{text}}$, measures the discrepancy between the predicted text logits and the target text labels. Let $\hat{y}_{\text{text}}$ represent the predicted logits and $y_{\text{text}}$ the ground truth target labels. The text prediction loss is calculated as:

% \begin{equation}
% \mathcal{L}_{\text{text}} = \frac{1}{N} \sum_{i=1}^{N} \text{CE}(\hat{y}_{\text{text}}^{(i)}, y_{\text{text}}^{(i)}),
% \end{equation}

% where $\text{CE}$ denotes the cross-entropy loss, and $N$ is the number of training samples.

% \paragraph{Audio Token Prediction Loss}

% The second loss is Audio Token Prediction loss, $\mathcal{L}_{\text{mel}}$, evaluates the model's performance in generating acoustic Token that match the target VQ-VAE codes. It is defined as:

% \begin{equation}
% \mathcal{L}_{\text{audio}} = \frac{1}{N} \sum_{i=1}^{N} \text{CE}(\hat{y}_{\text{audio}}^{(i)}, y_{\text{audio}}^{(i)}),
% \end{equation}

% where $\hat{y}_{\text{audio}}$ represents the predicted logits for the audio token, and $y_{\text{audio}}$ are the corresponding target VQ-VAE tokens.


% The total loss used to train the model is a weighted sum of the text and audio losses:

% \begin{equation}
% \mathcal{L}_{\text{total}} = \alpha \mathcal{L}_{\text{text}} + \beta \mathcal{L}_{\text{audio}}
% \end{equation}

% where $\alpha$ and $\beta$ are scaling factors that control the relative importance of each loss term. This combined objective ensures that the model learns both the correct phonetic representations and acoustic features.

% $\alpha$ and $\beta$ are set 0.01  and 1.0 respectively.

% \subsection{Vocoder Loss}
% We used a HiFi-GAN-based vocoder \cite{kong2020hifi} that comprises multiple discriminators: the Multi-Period Discriminator, and Multi-Scale Discriminator. For the sake of clarity, we will refer to these discriminators as a single entity. The HiFi-GAN module is trained using a least squares loss rather than the conventional binary cross-entropy loss. The discriminator is tasked with classifying real audio samples as 1 and generated samples as 0, while the generator is optimized to produce audio that can deceive the discriminator into classifying it as close to 1. The adversarial losses for the generator \(G\) and the discriminator \(D\) are defined as follows:

% \begin{align}
%     \mathcal{L}_{\text{Adv}}(D; G) &= \mathbb{E}_{(x, s)} \left[(D(x) - 1)^2 + D(G(s))^2 \right], \\
%     \mathcal{L}_{\text{Adv}}(G; D) &= \mathbb{E}_{s} \left[(D(G(s)) - 1)^2 \right],
% \end{align}

% where \(x\) represents the real audio samples, and \(s\) denotes the input mel-spectrogram conditions.

% \paragraph{Mel-Spectrogram Loss}
% The model also employs L1 loss between the mel-spectrograms of the real and generated audio. This loss is formulated as:

% \begin{align}
%     \mathcal{L}_{\text{Mel}}(G) = \mathbb{E}_{(x, s)} \left[\left\| \phi(x) - \phi(G(s)) \right\|_{1}\right],
% \end{align}

% where \(\phi\) represents the transformation function that maps a waveform to its corresponding mel-spectrogram.

% \paragraph{Feature Matching Loss}
% The feature matching loss calculates the L1 distance between the intermediate features of the real and generated audio, as extracted from multiple layers of the discriminator. It is defined as:

% \begin{align}
%     \mathcal{L}_{\text{FM}}(G; D) = \mathbb{E}_{(x, s)} \left[\sum_{i=1}^{T} \frac{1}{N_i} \left\| D^i(x) - D^i(G(s)) \right\|_{1}\right],
% \end{align}

% where \(T\) denotes the number of discriminator layers, and \(D^i\) and \(N_i\) represent the features and number of features at the \(i\)-th layer, respectively.

% \paragraph{Final Loss}
% Given that the discriminator is composed of multiple sub-discriminators, the final objectives for training the generator and the discriminator are defined as follows::

% \begin{align}
%     \mathcal{L}_{G} &= \sum_{k=1}^{K} \left[\mathcal{L}_{\text{Adv}}(G; D_k) + \lambda_{\text{FM}} \mathcal{L}_{\text{FM}}(G; D_k)\right] + \lambda_{\text{Mel}} \mathcal{L}_{\text{Mel}}(G), \\
%     \mathcal{L}_{D} &= \sum_{k=1}^{K} \mathcal{L}_{\text{Adv}}(D_k; G),
% \end{align}

% where \(D_k\) denotes the \(k\)-th sub-discriminator and \(\lambda_{\text{FM}} = 2\), \(\lambda_{\text{Mel}} = 45\). 


\section{Training Objectives}
\label{app:training_objective}

Our BnTTS model is composed of two primary modules (GPT-2 and HiFi-GAN), which are trained separately. The GPT-2 module is trained using a Language Modeling objective, while the HiFi-GAN module is optimized using HiFi-GAN loss objective. This section provides an overview of the loss functions applied during training.

\subsection{Language Modeling Loss}
1. \textbf{Text Generation Loss}: Denoted as $\mathcal{L}_{\text{text}}$, it quantifies the difference between predicted logits and ground truth labels using cross-entropy. Let $\hat{y}_{\text{text}}$ represent the predicted logits and $y_{\text{text}}$ the ground truth target labels. For a sequence with $N$ text tokens, the Text Generation Loss is calculated as: 
   \begin{equation}
   \mathcal{L}_{\text{text}} = \frac{1}{N} \sum_{i=1}^{N} \text{CE}(\hat{y}_{\text{text}}^{(i)}, y_{\text{text}}^{(i)})
   \end{equation}
   
2. \textbf{Audio Generation Loss}: Denoted as $\mathcal{L}_{\text{audio}}$, it evaluates the accuracy of generated acoustic tokens against target VQ-VAE codes using cross-entropy loss:
   \begin{equation}
   \mathcal{L}_{\text{audio}} = \frac{1}{N} \sum_{i=1}^{N} \text{CE}(\hat{y}_{\text{audio}}^{(i)}, y_{\text{audio}}^{(i)})
   \end{equation}

where $\hat{y}_{\text{audio}}$ represents the predicted logits for the audio token, $y_{\text{audio}}$ are the corresponding target VQ-VAE tokens, and $N$ is the number of audio token in the sequence.
   
Total loss combines the text generation and audio generation losses with weighted factors:
   \begin{equation}
   \mathcal{L}_{\text{total}} = \alpha \mathcal{L}_{\text{text}} + \beta \mathcal{L}_{\text{audio}} \quad (\alpha = 0.01, \beta = 1.0)
   \end{equation}

where $\alpha$ and $\beta$ are scaling factors that control the relative importance of each loss term.




\subsection{HiFi-GAN Loss}
We used a HiFi-GAN-based vocoder \cite{kong2020hifi} that comprises multiple discriminators: the Multi-Period Discriminator, and Multi-Scale Discriminator. For the sake of clarity, we will refer to these discriminators as a single entity. The HiFi-GAN module is trained using multiple losses mentioned below:

1. \textbf{Adversarial Loss}: The adversarial losses for the generator \(G\) and the discriminator \(D\) are defined as follows:
\begin{align}
    \mathcal{L}_{\text{Adv}}(D; G) &= \mathbb{E}_{(x, s)} \left[(D(x) - 1)^2 + D(G(s))^2 \right] \\
    \mathcal{L}_{\text{Adv}}(G; D) &= \mathbb{E}_{s} \left[(D(G(s)) - 1)^2 \right]
\end{align}

where \(x\) represents the real audio samples, and \(s\) denotes the input conditions.

2. \textbf{Mel-Spectrogram Loss}: This loss calculates L1 distance between the mel-spectrograms of the real and generated audio. This loss is formulated as:
\begin{align}
    \mathcal{L}_{\text{Mel}}(G) = \mathbb{E}_{(x, s)} \left[\left\| \phi(x) - \phi(G(s)) \right\|_{1}\right]
\end{align}
where \(\phi\) represents the transformation function that maps a waveform to its corresponding mel-spectrogram.

3. \textbf{Feature Matching Loss}: The feature matching loss calculates the L1 distance between the intermediate features of the real and generated audio, as extracted from multiple layers of the discriminator. It is defined as:
% \begin{align}
%     \mathcal{L}_{\text{FM}}(G; D) = \mathbb{E}_{(x, s)} \left[\sum_{i=1}^{T} \frac{1}{N_i} \left\| D^i(x) - D^i(G(s)) \right\|_{1}\right]
% \end{align}

\begin{align}
    \mathcal{L}_{\text{FM}}(G; D) = \mathbb{E}_{(x, s)} \sum_{i=1}^{T} \frac{1}{N_i} \left\| D^i(x) - D^i(G(s)) \right\|_{1}
\end{align}

where \(T\) denotes the number of discriminator layers, and \(D^i\) and \(N_i\) represent the features and number of features at the \(i\)-th layer, respectively.


\paragraph{Final Loss:}
Given that the discriminator is composed of multiple sub-discriminators, the final objectives for training the generator and the discriminator are defined as follows:
% \begin{align}
%     \mathcal{L}_{G} &= \sum_{k=1}^{K} \left[\mathcal{L}_{\text{Adv}}(G; D_k) + \lambda_{\text{FM}} \mathcal{L}_{\text{FM}}(G; D_k)\right] + \lambda_{\text{Mel}} \mathcal{L}_{\text{Mel}}(G), \\
%     \mathcal{L}_{D} &= \sum_{k=1}^{K} \mathcal{L}_{\text{Adv}}(D_k; G),
% \end{align}
\begin{align}
    \mathcal{L}_{G} &= \sum_{k=1}^{K} \left[\mathcal{L}_{\text{Adv}}(G; D_k) + \lambda_{\text{FM}} \mathcal{L}_{\text{FM}}(G; D_k)\right] \notag \\
    &\quad + \lambda_{\text{Mel}} \mathcal{L}_{\text{Mel}}(G) \\
    \mathcal{L}_{D} &= \sum_{k=1}^{K} \mathcal{L}_{\text{Adv}}(D_k; G)
\end{align}

where \(D_k\) denotes the \(k\)-th sub-discriminator and \(\lambda_{\text{FM}} = 2\), \(\lambda_{\text{Mel}} = 45\). 




% 1. \textbf{Adversarial Losses}: For generator $G$ and discriminator $D$, using least squares instead of binary cross-entropy:
%    \begin{align}
%    \mathcal{L}_{\text{Adv}}(D; G) &= \mathbb{E}_{(x, s)} \left[(D(x) - 1)^2 + D(G(s))^2 \right], \\
%    \mathcal{L}_{\text{Adv}}(G; D) &= \mathbb{E}_{s} \left[(D(G(s)) - 1)^2 \right]
%    \end{align}
% 2. \textbf{Mel-Spectrogram Loss}: Measures the L1 distance between real and generated audio mel-spectrograms:
%    \begin{equation}
%    \mathcal{L}_{\text{Mel}}(G) = \mathbb{E}_{(x, s)} \left[\| \phi(x) - \phi(G(s)) \|_{1}\right]
%    \end{equation}

% 3. \textbf{Feature Matching Loss}: Compares intermediate features from real and generated audio across discriminator layers:
%    \begin{equation}
%    \mathcal{L}_{\text{FM}}(G; D) = \mathbb{E}_{(x, s)} \left[\sum_{i=1}^{T} \frac{1}{N_i} \| D^i(x) - D^i(G(s)) \|_{1}\right]
%    \end{equation}

% 4. \textbf{Final Loss Objectives}:
%    Generator Loss:
%    \begin{equation}
%    \mathcal{L}_{G} = \mathcal{L}_{\text{Adv}}(G; D) + \lambda_{\text{FM}} \mathcal{L}_{\text{FM}}(G; D) + \lambda_{\text{Mel}} \mathcal{L}_{\text{Mel}}(G)
%    \end{equation}
%    Discriminator Loss:
%    \begin{equation}
%    \mathcal{L}_{D} = \mathcal{L}_{\text{Adv}}(D; G)
%    \end{equation}

% This framework ensures effective training of the model, balancing text and audio prediction tasks, and optimizing for high-quality audio generation.


\section{Evaluation Metrics}
\label{app:eval_metrics}
We employed a combination of subjective and objective metrics to rigorously evaluate the performance of our TTS system, focusing on intelligibility, naturalness, speaker similarity, and transcription accuracy.

\noindent \textbf{Subjective Mean Opinion Score (SMOS):} SMOS is a perceptual evaluation where listeners rate synthesized speech on a Likert scale from 1 (poor) to 5 (excellent). It considers naturalness, clarity, and fluency, providing an absolute score for each sample. A higher SMOS indicates better overall speech quality.

\noindent \textbf{SpeechBERTScore:} SpeechBERTScore adapts BERTScore for speech, using self-supervised learning (SSL) models to compare dense representations of generated and reference speech. For generated speech waveform $\hat{X}$ and reference waveform $X$, the feature representations $\hat{Z}$ and $Z$ are extracted using a pretrained model. SpeechBERTScore is defined as the average maximum cosine similarity between feature vectors:
\[
\text{SpeechBERTScore} = \frac{1}{N_{\text{gen}}} \sum_{i=1}^{N_{\text{gen}}} \max_{j} \text{cos}(\hat{\mathbf{z}}_i, \mathbf{z}_j)
\]
where $\hat{\mathbf{z}}_i$ and $\mathbf{z}_j$ represent the SSL embeddings for generated and reference speech, respectively.

\noindent \textbf{Character Error Rate (CER):} CER measures transcription accuracy by calculating the ratio of errors (substitutions $S$, deletions $D$, and insertions $I$) in automatic speech recognition (ASR) transcriptions:
\[
CER = \frac{S + D + I}{N}
\]
where $N$ is the total number of characters in the reference transcription. A lower CER indicates better transcription accuracy.

\noindent \textbf{Speaker Encoder Cosine Similarity (SECS):} SECS evaluates speaker similarity by calculating the cosine similarity between speaker embeddings of the reference and synthesized speech:

\[
\text{SECS} = \frac{e_{\text{ref}} \cdot e_{\text{syn}}}{\|e_{\text{ref}}\| \|e_{\text{syn}}\|},
\]

where $e_{\text{ref}}$ and $e_{\text{syn}}$ are the speaker embeddings for reference and synthesized speech, respectively. SECS ranges from -1 (low similarity) to 1 (high similarity).

\label{sec:DurationEquality}
\noindent \textbf{Duration Equality Score:} This metric quantifies how closely the durations of the reference ($a$) and synthesized ($b$) speech match, with a score of 1 indicating identical durations:

\[
\text{DurationEquality}(a, b) = \frac{1}{\max\left(\frac{a}{b}, \frac{b}{a}\right)}.
\]

This score helps in assessing duration similarity between reference and generated audio, ensuring consistency in pacing.

Each metric provides a different perspective, allowing a holistic evaluation of the synthesized speech quality.


% \section{Evaluation Metrics}
% To rigorously evaluate the performance of the TTS system, a combination of subjective and objective metrics is employed. These metrics assess various dimensions of speech quality, including intelligibility, naturalness, speaker similarity, and transcription accuracy. The following describes the key evaluation metrics in detail:


% \textbf{Subjective Mean Opinion Score (SMOS)}: SMOS is a perceptual evaluation metric used to assess the overall quality of synthesized speech by human listeners. It is rated on a Likert scale from 1 (bad) to 5 (excellent), considering aspects such as naturalness, clarity, fluency, consistency, and emotional expressiveness. SMOS serves as an absolute rating, where each synthetic sample is evaluated independently, without reference to other samples. In addition to SMOS, separate scores for Naturalness and Clarity are reported for a comprehensive analysis.


% \textbf{SpeechBERTScore}:
% To evaluate the semantic consistency between the generated and reference speech in our proposed system, we employ the SpeechBERTScore metric, which extends the BERTScore framework, commonly used in text generation, to the speech domain by computing the similarity between dense speech representations derived from self-supervised learning (SSL) models. The metric aims to capture semantic congruence between the synthesized speech and a reference, accounting for differences in waveform length.

% Let the generated and reference speech waveforms be denoted as $\hat{X} = (\hat{x}_t \in \mathbb{R} \mid t = 1, \ldots, T_{\text{gen}})$ and $X = (x_t \in \mathbb{R} \mid t = 1, \ldots, T_{\text{ref}})$, respectively, where $T_{\text{gen}}$ and $T_{\text{ref}}$ represent the lengths of the generated and reference waveforms. To extract meaningful features from these waveforms, a pretrained SSL model is employed, which generates sequence representations $\hat{Z} = (\hat{\mathbf{z}}_n \in \mathbb{R}^D \mid n = 1, \ldots, N_{\text{gen}})$ and $Z = (\mathbf{z}_n \in \mathbb{R}^D \mid n = 1, \ldots, N_{\text{ref}})$ for the generated and reference speech, respectively:

% \[
% \hat{Z} = \text{Encoder}(\hat{X}; \theta), \quad Z = \text{Encoder}(X; \theta),
% \]

% where $\theta$ denotes the parameters of the pretrained encoder model, and $N_{\text{gen}}$ and $N_{\text{ref}}$ are determined by $T_{\text{gen}}$ and $T_{\text{ref}}$ based on the encoder's subsampling rate.

% The SpeechBERTScore is defined as the precision metric in the BERTScore framework, measuring the maximum cosine similarity between each feature vector in the generated speech and all feature vectors in the reference speech:

% \[
% \text{SpeechBERTScore} = \frac{1}{N_{\text{gen}}} \sum_{i=1}^{N_{\text{gen}}} \max_{j} \text{cos}(\hat{\mathbf{z}}_i, \mathbf{z}_j),
% \]

% where $\text{cos}(\hat{\mathbf{z}}_i, \mathbf{z}_j)$ is the cosine similarity between the SSL feature vectors $\hat{\mathbf{z}}_i$ from the generated speech and $\mathbf{z}_j$ from the reference speech.

% By leveraging pretrained SSL models, SpeechBERTScore captures high-level semantic information, making it suitable for evaluating synthesized speech's content and meaning. This metric is particularly advantageous for TTS evaluation, where semantic consistency and intelligibility are crucial, even in scenarios where the generated and reference audio lengths may differ.

% \textbf{Character Error Rate (CER)}: CER quantifies transcription accuracy by comparing the output of an automatic speech recognition (ASR) system on synthesized speech against a reference transcription. It is defined as:
% \[
% \text{CER} = \frac{S + D + I}{N}
% \]
% where $S$ is the number of substitutions, $D$ is the number of deletions, $I$ is the number of insertions, and $N$ is the total number of characters in the reference transcription. Lower CER values indicate higher transcription accuracy.

% \textbf{Speaker Encoder Cosine Similarity (SECS)}: SECS measures the speaker similarity between synthesized and reference speech by calculating the cosine similarity between their speaker embeddings:
% \[
% \text{SECS} = \frac{e_{\text{ref}} \cdot e_{\text{syn}}}{\|e_{\text{ref}}\| \|e_{\text{syn}}\|}
% \]
% where $e_{\text{ref}}$ and $e_{\text{syn}}$ are the speaker embeddings of the reference and synthesized speech, respectively. The similarity score ranges from -1 (low similarity) to 1 (high similarity), with higher values indicating closer resemblance in speaker characteristics.

% \textbf{DurationEquality Score}: DurationEquality quantifies the equality between two audio sample durations, \(a\) and \(b\), producing values between 0 and 1, where a score of 1 indicates identical durations. The metric is defined as:
% \begin{equation}
%     \text{DurationEquality}(a, b) = \frac{1}{\max\left(\frac{a}{b}, \frac{b}{a}\right)}
% \end{equation}
% This score approaches 1 as the durations of \(a\) and \(b\) become more equal, providing an effective measure of  discrepancy between duration of reference audio and synthesized audio .


\section{Subjective Evaluation}
For subjective evaluation of our system, we employ the Mean Opinion Score (MOS), a widely recognized metric primarily focusing on assessing the perceptual quality of audio outputs. To ensure the reliability and accuracy of our evaluations, we carefully select a panel of ten experts who are thoroughly trained in the intricacies of MOS scoring. These experts are equipped with the necessary skills and knowledge to critically assess and score the system, providing invaluable insights that help guide the refinement and enhancement of our technology. This structured approach guarantees that our evaluations are both comprehensive and precise, reflecting the true quality of the audio outputs under review.

\subsection{Evaluation Guideline}
For calculating MOS, we consider five essential evaluation criteria:
\begin{itemize} \item \textbf{Naturalness:} Evaluates how closely the TTS output resembles natural human speech. \item \textbf{Clarity:} Assesses the intelligibility and clear articulation of the spoken words. \item \textbf{Fluency:} Examines the smoothness of speech, including appropriate pacing, pausing, and intonation. \item \textbf{Consistency:} Checks the uniformity of voice quality across different texts. \item \textbf{Emotional Expressiveness:} Measures the ability of the TTS system to convey the intended emotion or tone. \end{itemize}

In the evaluation, we employ a five-point rating scale to meticulously assess performance based on specific criteria. This scale ranges from 1, denoting 'Bad' where the output has significant distortions, to 5, representing 'Excellent' where the output nearly replicates natural human speech and excels in all evaluation aspects. To capture more subtle nuances in the TTS output that might not perfectly fit into these whole-number categories, we also recommend using fractional scores. For example, a 1.5 indicates quality between 'Bad' and 'Poor,' a 2.5 signifies improvement over 'Poor' but not quite reaching 'Fair,' a 3.5 suggests better than 'Fair' but not up to 'Good,' and a 4.5 reflects performance that surpasses 'Good' but falls short of 'Excellent.' This fractional scoring allows for a more precise and detailed reflection of the system's quality, enhancing the accuracy and depth of the MOS evaluation.

\subsection{Evaluation Process}
We have developed an evaluation platform specifically designed for the subjective assessment of Text-to-Speech (TTS) systems. This platform features several key attributes that enhance the effectiveness and reliability of the evaluation process. Key features include anonymity of audio sources, ensuring that evaluators are unaware of whether the audio is synthetically generated or recorded from studio environment, or which TTS model, if any, was used. This promotes unbiased assessments based purely on audio quality. Comprehensive evaluation criteria allow evaluators to rate each audio sample on naturalness, clarity, fluency, consistency, and emotional expressiveness, ensuring a holistic review of speech synthesis quality. The user-centric interface is streamlined for ease of use, enabling efficient playback of audio samples and score entry, which reduces evaluator fatigue and maintains focus on the task. Finally, the structured data collection method systematically captures all ratings, facilitating precise analysis and enabling targeted improvements to TTS technologies. This platform is a vital tool for developers and researchers aiming to refine the effectiveness and naturalness of speech outputs in TTS systems.

\subsection{Evaluator Statistics}
For our evaluation process, we carefully selected 10 expert native speakers, achieving a balanced representation with 5 males and 5 females. The age range for these evaluators is between 20 to 28 years, ensuring a youthful perspective that aligns well with our target demographic. All evaluators are either currently enrolled as graduate students or have already completed their graduate studies. They hail from a variety of academic backgrounds, including economics, engineering, computer science, and social sciences, which provides a diverse range of insights and expertise. This careful selection of qualified individuals ensures a comprehensive and informed assessment process, suitable for our needs in evaluating advanced systems or processes where diverse, educated opinions are crucial.

\subsection{Subjective Evaluation Data Preparation} 
For reference-aware evaluation, we selected 20 audio samples from each of the four speakers, resulting in 80 Ground Truth (GT) audios. To facilitate comparison, we generated 400 synthetic samples (80 × 5) using the TTS systems examined in this study. Including the GT samples, the total dataset for this evaluation amounts to 480 audio files (400 + 80).

For the reference-independent evaluation, we utilized 453 text samples from BnTTSTextEval, comprising BengaliStimuli53 (53), BengaliNamedEntity1000 (200), and ShortText200 (200). Given the four speakers in both BnTTS-0 and BnTTS-n, this resulted in 3,624 audio samples (4 × 453 × 2). Additionally, IndicTTS, GTTS, and AzureTTS contributed 1,359 samples (3 × 453). IndicTTS samples were evenly distributed between two male and female speakers, while GTTS and AzureTTS used the "bn-IN-Wavenet-C" and "bn-IN-TanishaaNeural" voices, respectively.

In total, the reference-independent evaluation dataset comprised 5,436 audio samples. When combined with the 480 samples from the reference-aware evaluation, the overall dataset for subjective evaluation amounted to 5,916 audio files. These samples were randomly mixed and distributed to the reviewer team to ensure unbiased evaluations.

\section{Use of AI assistant}
\label{sec:use_of_ai_assistant}
We used AI assistants such as GPT-4o for spelling and grammar checking for the text of the paper.

\newpage

% \section{Potential Risks}
% \label{sec:use_of_potential risks}
% There are no potential risks associated with the outcomes of this research, as we do not utilize any sensitive information. Instead, this work will benefit the community by aiding the development of TTS systems for low-resource languages.




% Speech_Generation had been added to main paper 
% \section{Speech Generation}
\subsection{Synthesizing Short Sequences}

The generation of short audio sequences presents challenges in the BnTTS model, particularly for texts containing fewer than 30 characters when using the default generation settings (Temperature \(T = 0.85\) and TopK = 50). The primary issues observed are twofold: (1) the generated speech often lacks intelligibility, and (2) the output speech tends to be longer than expected.

To investigate these challenges, we curated a subset of 23 short text-speech pairs from the BnStudioEval dataset. For evaluation, we utilize the Character Error Rate (CER) metric to assess intelligibility, and we introduce the Audio Duration Equality metric to evaluate the alignment between the generated and reference audio durations. The Audio Duration Equality Score quantifies the equality between two audio sample durations, \(a\) and \(b\), producing values between 0 and 1, where a score of 1 indicates identical durations. The metric is defined as:

\begin{equation}
    \text{DurationEquality}(a, b) = \frac{1}{\max\left(\frac{a}{b}, \frac{b}{a}\right)}
\end{equation}



This score approaches 1 as the durations of \(a\) and \(b\) become more equal, providing an effective measure of  discrepancy between duration of reference audio and synthesized audio .


\paragraph{Effect of Short Prompt}
Under the default settings (Exp. 1 in Table X), the model achieves a Character Error Rate (CER) of 0.081 and a Duration Equality Score of 0.699. We hypothesize that the model's inability to accurately synthesize short speech stems from its training process. During training, the model reserves between 1 to 6 seconds of audio for speaker prompting. For audio shorter than 1 second, the model uses half of the audio as the prompt. This implies that the model is accustomed to short audio prompts for short sequences. By aligning the inference process with this training strategy and using short prompts, the generation performance improves markedly, as evidenced by a higher Duration Equality Score of 0.820 and a lower CER of 0.029 in Exp. 2.

\paragraph{Effect of Temperature and Top-K Sampling}
The default temperature (\(T = 0.85\)) and top-K value (50) were found to be sub-optimal for generating short sequences. By adjusting the temperature to \(T = 1.0\) and reducing the top-K value to 2, we observed an improvement in the Duration Equality Score from 0.699 to 0.701, accompanied by a substantial reduction in CER, from 0.081 to 0.023 (as shown in Exp. 3).

\paragraph{Effect of Both Short Prompts and Temperature, Top-K}
Combining short prompts with the adjusted temperature and top-K values yielded the best results. In this configuration, the Duration Equality Score improved to 0.827, with a CER of 0.015, demonstrating that both factors are crucial for accurate short sequence generation.

The ablation study demonstrates that employing short prompts in combination with fine-tuning temperature and top-K values is essential for optimizing short sequence generation in the BnTTS model.

\begin{table}[H]
\centering

    \begin{tabular}{c|l}
        \hline
        \textbf{Variable} & \textbf{Description} \\ \hline
        \( \mathbf{T} \) & Text sequence with \( N \) tokens \\ \hline
        \( N \) & Number of tokens in the text sequence \\ \hline
        \( \mathbf{S} \) & Speaker's mel-spectrogram with \( L \) frames \\ \hline
        \( \hat{\mathbf{Y}} \) & Generated speech that matches the speaker's characteristics \\ \hline
        \( \mathbf{Y} \) & Ground truth mel-spectrogram frames for the target speech \\ \hline
        \( \mathcal{F} \) & Model responsible for producing speech conditioned on both the text and the speaker's spectrogram \\ \hline
        \( \mathbf{z} \) & Discrete codes transformed from mel-spectrogram frames using VQ-VAE \\ \hline
        \( \mathcal{C} \) & Codebook of discrete codes from VQ-VAE \\ \hline
        \( l \) & Number of layers in the Conditioning Encoder \\ \hline
        \( k \) & Number of attention heads in Scaled Dot-Product Attention \\ \hline
        \( \mathbf{S_z} \) & Intermediate representation of speaker spectrogram in \( \mathbb{R}^{L \times d} \) \\ \hline
        \( d \) & Dimensionality of each token or embedding \\ \hline
        \( \mathbf{Q}, \mathbf{K}, \mathbf{V} \) & Projections of \( \mathbf{S_z} \) used in scaled dot-product attention \\ \hline
        \( P \) & Fixed number of sequences produced by the Perceiver Resampler \\ \hline
        \( \mathbf{R} \) & Fixed-size output from the Perceiver Resampler in \( \mathbb{R}^{P \times d} \) \\ \hline
        \( \mathbf{T_e} \) & Continuous embedding space of text tokens in \( \mathbb{R}^{N \times d} \) \\ \hline
        \( \mathbf{S_p} \) & Speaker embeddings \\ \hline
        \( \mathbf{Y_z} \) & Ground truth spectrogram embeddings \\ \hline
        \( \mathbf{X} \) & Combined input during training: concatenation of speaker, text, and spectrogram embeddings \\ \hline
        \( \oplus \) & Concatenation operation \\ \hline
        \( \mathbf{H} \) & Output from the LLM consisting of hidden states for text, speaker, and spectrogram embeddings \\ \hline
        \( \mathbf{H}_\text{Y} \) & Spectrogram embedding from LLM output used for HiFi-GAN \\ \hline
        \( \mathbf{S}' \) & Resized speaker embedding to match \( \mathbf{H}_\text{Y} \) \\ \hline
        \( \mathbf{W} \) & Final audio waveform produced by HiFi-GAN \\ \hline
        \( g_\text{HiFi} \) & HiFi-GAN function converting spectrogram embeddings to audio waveform \\ \hline
    \end{tabular}
    \label{tab:variables_descriptions}
    \caption{Table of Variables and Descriptions}
\end{table}

% \section{Results and Discussion}

% To evaluate the performance of our Bengali TTS system, we employed a combination of subjective and objective metrics across two datasets: BnStudioEval and BnTTSTextEval. The BnStudioEval dataset, consisting of high-quality recordings, was used for reference-aware evaluation, while BnTTSTextEval encompassed subsets focusing on phonetic diversity, named entity pronunciation, and conversational fluency for reference-independent evaluation. The metrics used include subjective measures such as SMOS and objective measures like CER, UTMOS, SECS, and SpeechBERT Precision.



% \subsection{Reference-aware Evaluation (BnStudioEval)}
% Table 1 presents the comparative performance of various TTS systems evaluated on the BnStudioEval dataset. Among the synthetic methods, AzureTTS exhibited the best performance with the lowest Character Error Rate (CER) and the highest UTMOS score, outperforming all other synthetic methods, including Ground Truths (GT) in terms of transcription accuracy. However, interestingly, the CER of the GT remains lower than that of BnTTS synthesized outputs. As expected, the GT, serving as a reference standard, outperforms all synthetic systems across key subjective metrics such as SMOS (4.671), Naturalness (4.625), and Clarity (4.9). In this context, the proposed BnTTS system closely follows, achieving competitive scores in SMOS (4.584), Naturalness (4.531), and Clarity (4.898).

% Regarding speaker similarity, the GT achieved SECS scores of 1.0 when compared to reference audios and 0.361 when compared to the speaker prompt. BnTTS also performed well, with an SECS(Ref.) score of 0.513 and an SECS(Prompt) score of 0.335, falling short of the ground truth by only 0.031 in the latter metric. Additionally, BnTTS received a SpeechBERT Precision score of 0.796, compared to the perfect score of 1.0 set by the ground truth.

% It should be noted that IndicTTS, GTTS, and AzureTTS lack speaker impersonation capabilities, rendering them inapplicable for reference-aware metrics such as SECS and SpeechBERT Precision. Consequently, these metrics were not calculated for these systems.


% \subsection{Reference-independent Evaluation (BnTTSTextEval)}
% Table 2 presents the comparative performance of various TTS systems evaluated on the BnTTSTextEval dataset, encompassing three distinct subsets: BengaliStimuli53, BengaliNamedEntity1000, and ShortText200. The trend observed in the BnStudioEval dataset persists here as well. AzureTTS and GTTS consistently trade leading positions in transcription accuracy (CER) and automated quality prediction (UTMOS), with BnTTS following closely in third place, and IndicTTS trailing behind.

% BnTTS performs strongly in subjective evaluations, excelling in SMOS, Naturalness, and Intelligibility across the BengaliStimuli53 and BengaliNamedEntity1000 subsets. However, it falls slightly behind AzureTTS in the ShortText200 subset, which focuses on model performance in short texts. Despite this, BnTTS overall, remains the top-performing system in all subjective metrics, delivering the highest average scores in SMOS(4.383), Naturalness(4.313), and Clarity(4.737).


\section{Symbols and Notations}
\label{sec:notation}

\begin{table}[H]
\centering
\setlength{\tabcolsep}{2.9pt}
\resizebox{0.45\textwidth}{!}{%

    \begin{tabular}{c|l}
        \hline
        \textbf{Variable} & \textbf{Description} \\ \hline
        \( \mathbf{T} \) & Text sequence with \( N \) tokens \\ \hline
        \( N \) & Number of tokens in the text sequence \\ \hline
        \( \mathbf{S} \) & Speaker's mel-spectrogram with \( L \) frames \\ \hline
        \( \hat{\mathbf{Y}} \) & Generated speech  \\ \hline
        \( \mathbf{Y} \) & Ground truth mel-spectrogram  \\ \hline
        \( \mathcal{F} \) & LLM Model \\ \hline
        \( \mathbf{z} \) & Discrete codes \\ \hline
        \( \mathcal{C} \) & Codebook of discrete codes \\ \hline
        \( l \) & Number of layers \\ \hline
        \( k \) & Number of attention heads i \\ \hline
        \( \mathbf{S_z} \) & speaker spectrogram embd.\( \mathbb{R}^{L \times d} \) \\ \hline
        \( d \) & Embedding \\ \hline
        \( \mathbf{Q}, \mathbf{K}, \mathbf{V} \) & Query, Key, Value \\ \hline
        \( P \) & Perceiver Resampler \\ \hline
        \( \mathbf{R} \) & Fixed-size output in \( \mathbb{R}^{P \times d} \) \\ \hline
        \( \mathbf{T_e} \) & Continuous embedding  \( \mathbb{R}^{N \times d} \) \\ \hline
        \( \mathbf{S_p} \) & Speaker embeddings \\ \hline
        \( \mathbf{Y_z} \) & Ground truth  \\ \hline
        \( \mathbf{X} \) & Input of LLM   \\ \hline
        \( \oplus \) & Concatenation operation \\ \hline
        \( \mathbf{H} \) & Output from the LLM  \\ \hline
        \( \mathbf{H}_\text{Y} \) & Spectrogram embedding  \\ \hline
        \( \mathbf{S}' \) & Resized embedding \( \mathbf{H}_\text{Y} \) \\ \hline
        \( \mathbf{W} \) & Final audio waveform \\ \hline
        \( g_\text{HiFi} \) & HiFi-GAN function \\ \hline
    \end{tabular}}
    \label{tab:variables_descriptions}
    \caption{Table of Variables and Descriptions}
\end{table}


\end{document}
