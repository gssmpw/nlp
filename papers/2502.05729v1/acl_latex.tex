
\documentclass[11pt]{article}

% Change "review" to "final" to generate the final (sometimes called camera-ready) version.
% Change to "preprint" to generate a non-anonymous version with page numbers.
\usepackage[final]{acl}

% Standard package includes
\usepackage{times}
\usepackage{latexsym}


% For proper rendering and hyphenation of words containing Latin characters (including in bib files)
\usepackage[T1]{fontenc}
% For Vietnamese characters
% \usepackage[T5]{fontenc}
% See https://www.latex-project.org/help/documentation/encguide.pdf for other character sets

% This assumes your files are encoded as UTF8
\usepackage[utf8]{inputenc}

% This is not strictly necessary, and may be commented out,
% but it will improve the layout of the manuscript,
% and will typically save some space.
\usepackage{microtype}

% This is also not strictly necessary, and may be commented out.
% However, it will improve the aesthetics of text in
% the typewriter font.
\usepackage{inconsolata}

%Including images in your LaTeX document requires adding
%additional package(s)
\usepackage{graphicx}
\usepackage{amssymb}  % Provides \mathbb among other symbols


\usepackage{algorithm}
\usepackage{algorithmicx}
\usepackage{algpseudocode}
% \usepackage[linesnumbered, ruled, vlined]{algorithm2e}
\usepackage{tabularray}
\usepackage{caption}
\usepackage{subcaption}
% \usepackage{lmodern}
\usepackage{amsmath}
\usepackage{multirow}

% ,amsmath,graphicx, spconf}
% \usepackage{minted}

% If the title and author information does not fit in the area allocated, uncomment the following
%
%\setlength\titlebox{<dim>}
%
% and set <dim> to something 5cm or larger.

\title{BnTTS: Few-Shot Speaker Adaptation in Low-Resource Setting}

% Author information can be set in various styles:
% For several authors from the same institution:
% \author{Author 1 \and ... \and Author n \\
%         Address line \\ ... \\ Address line}
% if the names do not fit well on one line use
%         Author 1 \\ {\bf Author 2} \\ ... \\ {\bf Author n} \\
% For authors from different institutions:
% \author{Author 1 \\ Address line \\  ... \\ Address line
%         \And  ... \And
%         Author n \\ Address line \\ ... \\ Address line}
% To start a separate ``row'' of authors use \AND, as in
% \author{Author 1 \\ Address line \\  ... \\ Address line
%         \AND
%         Author 2 \\ Address line \\ ... \\ Address line \And
%         Author 3 \\ Address line \\ ... \\ Address line}

% jahid version 
\author{
\textbf{Mohammad Jahid Ibna Basher}\textsuperscript{1},
\textbf{Md Kowsher}\textsuperscript{2},
\textbf{Md Saiful Islam}\textsuperscript{1},
\textbf{Rabindra Nath Nandi}\textsuperscript{1}, \\
\textbf{Nusrat Jahan Prottasha}\textsuperscript{2},
\textbf{Mehadi Hasan Menon}\textsuperscript{1},
\textbf{Tareq Al Muntasir}\textsuperscript{1}, \\
\textbf{Shammur Absar Chowdhury}\textsuperscript{3},
\textbf{Firoj Alam}\textsuperscript{3},
\textbf{Niloofar Yousefi}\textsuperscript{2}, 
\textbf{Ozlem Ozmen Garibay}\textsuperscript{2} \\
\textsuperscript{1}Hishab Singapore Pte. Ltd, Singapore, 
\textsuperscript{2}University of Central Florida, USA \\
\textsuperscript{3}Qatar Computing Research Institute, Qatar
}




%\author{
%  \textbf{First Author\textsuperscript{1}},
%  \textbf{Second Author\textsuperscript{1,2}},
%  \textbf{Third T. Author\textsuperscript{1}},
%  \textbf{Fourth Author\textsuperscript{1}},
%\\
%  \textbf{Fifth Author\textsuperscript{1,2}},
%  \textbf{Sixth Author\textsuperscript{1}},
%  \textbf{Seventh Author\textsuperscript{1}},
%  \textbf{Eighth Author \textsuperscript{1,2,3,4}},
%\\
%  \textbf{Ninth Author\textsuperscript{1}},
%  \textbf{Tenth Author\textsuperscript{1}},
%  \textbf{Eleventh E. Author\textsuperscript{1,2,3,4,5}},
%  \textbf{Twelfth Author\textsuperscript{1}},
%\\
%  \textbf{Thirteenth Author\textsuperscript{3}},
%  \textbf{Fourteenth F. Author\textsuperscript{2,4}},
%  \textbf{Fifteenth Author\textsuperscript{1}},
%  \textbf{Sixteenth Author\textsuperscript{1}},
%\\
%  \textbf{Seventeenth S. Author\textsuperscript{4,5}},
%  \textbf{Eighteenth Author\textsuperscript{3,4}},
%  \textbf{Nineteenth N. Author\textsuperscript{2,5}},
%  \textbf{Twentieth Author\textsuperscript{1}}
%\\
%\\
%  \textsuperscript{1}Affiliation 1,
%  \textsuperscript{2}Affiliation 2,
%  \textsuperscript{3}Affiliation 3,
%  \textsuperscript{4}Affiliation 4,
%  \textsuperscript{5}Affiliation 5
%\\
%  \small{
%    \textbf{Correspondence:} \href{mailto:email@domain}{email@domain}
%  }
%}

\begin{document}
\maketitle
\begin{abstract}

% \textcolor{green}{Text-to-speech (TTS) systems have made significant strides in generating high-quality speech, yet much of this progress has been concentrated on high-resource languages, leaving low-resource languages like Bangla underrepresented}. 
This paper introduces BnTTS (\textbf{B}a\textbf{n}gla \textbf{T}ext-\textbf{T}o-\textbf{S}peech), the first framework for Bangla speaker adaptation-based TTS, designed to bridge the gap in Bangla speech synthesis using minimal training data. Building upon the XTTS architecture, our approach integrates Bangla into a multilingual TTS pipeline, with modifications to account for the phonetic and linguistic characteristics of the language. We pretrain BnTTS on 3.85k hours of Bangla speech dataset with corresponding text labels and evaluate performance in both zero-shot and few-shot settings on our proposed test dataset. Empirical evaluations in few-shot settings show that BnTTS significantly improves the naturalness, intelligibility, and speaker fidelity of synthesized Bangla speech. Compared to state-of-the-art Bangla TTS systems, BnTTS exhibits superior performance in Subjective Mean Opinion Score (SMOS), Naturalness, and Clarity metrics.


 
% The open-source code is available at Anonymized.
\end{abstract}

% 
% 
The widespread integration of communication networks and smart devices in modern control systems has increased the vulnerability of industrial systems to online cyber-attacks, e.g., Industroyer, Blackenergy, etc \citep{osti_1505628}.
% Modern control systems have seen a large push to include communication networks and smart devices to increase performance, made possible by improvements in communication device cost and energy consumption. This trend has been coupled with the usage of open-standard communication protocols among industrial control systems, making them vulnerable to online cyber-attacks such as Industroyer, Blackenergy, etc \citep{osti_1505628}. 
To counter this, methods have been developed to improve security by achieving attack detection, mitigation, and monitoring, among others \citep{sandberg2022secure}. This paper focuses on active attack diagnosis to mitigate stealthy attacks. 
%
%\subsection{Literature review}

Active diagnosis techniques rely on the inclusion of additional moduli to control systems
% inclusion within the control system of additional moduli 
to alter the behavior of the system compared to information known by the attacker. 
For instance, the concept of additive watermarking was introduced in \cite{mo2015physical}, where noise signals of known mean and variance are added at the plant and compensated for it at the controller. 
This compensation, however, is not exact, causing some performance degradation. Thus, trade-offs between performance and detectability  are necessary \citep{zhu2023detection}.
% A later work \citep{zhu2023detection} designs the watermark signal by trading performance for detection. Thus, although additive watermarking serves as a good detection scheme, they endure performance losses even in the nominal case. 

In encrypted control \citep{darup2021encrypted}, the sensor data is encrypted, sent to the controller, and then operated on directly. Encrypted input signals are sent back to the plant for decryption. Although encryption is widespread in IT security, in control systems it presents some concerns, such as the introduction of time delays \citep{stabile2024verifiable}, while it may present inherent weaknesses \citep{alisic2023model}.
% they are not preferred as they introduce time delays \citep{stabile2024verifiable} which can cause instability, and some encryption schemes can be very weak  \citep{alisic2023model}. 

In moving target defense \citep{griffioen2020moving}, the plant is augmented with fictitious dynamics, known to the controller. The plant output is transmitted to the controller along with the fictitious states over a network under attack. 
The additional measurements then aide in the detection of attacks. 
This comes at the cost of higher communication bandwidth needs, which increases rapidly with the dimension of the augmented systems.
% Since the dynamics of the fictitious dynamics are exactly known to the controller, the attack is detected easily. However, when the scale of the system increases, the communication bandwidth used by moving the target defense approach increases rapidly. 

Other recently proposed works include two-way coding \citep{fang2019two}, a weak encryuption technique, and dynamic masking \citep{abdalmoaty2023privacy}, which enhances privacy as well as security, have been shown to be effective against zero-dynamics attacks.
% Two-way coding \citep{fang2019two} and dynamic masking \citep{abdalmoaty2023privacy} are other recently proposed approaches. Two-way coding is another form of weak encryption technique whilst dynamic masking proposes an architecture that enhances both privacy and security. These schemes are shown to be effective against zero dynamics attacks but remain to be studied for other classes of attacks. 
% Recent extensions include \citep{mukherjee2021secure,ramos2024privacy}.
% Some other works which are related are \citep{mukherjee2021secure}, an extension of \cite{fang2019two}. The work \citep{ramos2024privacy} is an extension of moving target defense for multi-agent systems. 
Furthermore, filtering techniques for attack detection are proposed by \cite{murguia2020security,hashemi2022codesign,escudero2023safety}, while not focusing on stealthy attacks.
% The works \citep{murguia2020security,hashemi2022codesign,escudero2023safety} develop filtering techniques to guarantee safety, without being focused on stealthy covert attacks.

Multiplicative watermarking (mWM) has been proposed by the authors as a diagnosis technique \citep{ferrari2020switching}. mWM consists of a pair of filters on each communication channel between the plant and its controller; the scheme is affine to weak encryption, whereby ``encoding'' and ``decoding'' are done by changing signals' dynamic characteristics through inverse pairs of filters. This enables original signals to be recovered exactly, and thus does not lead to performance degradation.
% A multiplicative watermark is an affine to a weak encryption technique, through which the signal is ``encoded'' by a filter, changing its dynamic behavior. The use of inverse pairs means that the original signal can be recovered, through ``decoding'' via an inverse filter. As such, differently to techniques based on additive watermarking, no performance is lost due to the injection of noise, and there are no bandwidth limitations.

%\subsection{Contributions}
One of the critical features of multiplicative watermarking is that to detect stealthy attacks, the mWM filter parameters must be switched over time. In this paper, an algorithm to optimally design the mWM parameters after a switching event is presented, enhancing detection performance, without changing the switching time.
% This is done without changing the switching time, which is taken as given.

\textcolor{black}{
To formalize the filter design problem, we suppose the defender is interested in optimal performance against adversaries injecting covert attacks with matched system parameters \citep{smith2015covert}, including the mWM parameters prior to the switch. This scenario represents a worst case where malicious agents can take full control of the system while remaining undetected.
Thus, the attack strategy is explicitly included within the formulation of the closed-loop system, and the mWM filters are chosen by solving an optimization problem minimizing the attack-energy-constrained output-to-output gain (AEC-OOG) \citep{anand2023risk}, a variation of the output-to-output gain proposed in  \cite{teixeira2015strategic}.
}
The main contributions of this paper are:
% We consider an adversary injecting a covert attack with matched system parameters \citep{smith2015covert}, i.e., an attacker with full knowledge of the control system parameters, including those of the mWM filters before the switch. This scenario is taken as a worst case, as it has been shown that this class of attacks can be made stealthy. To quantitatively define a cost, the output-to-output gain (OOG) \citep{teixeira2015strategic} is leveraged,
% a metric introduced to evaluate the impact of an additive attack in a control system. %Specifically, OOG evaluates the worst-case performance loss that an attacker injecting an undetectable attack can obtain. 
% Here, the maximum performance loss caused by a stealthy adversary with limited energy is taken, the attack-energy-constrained OOG (AEC-OOG) \citep{anand2023risk}. The main contributions of this paper are:
\begin{enumerate}
%[label=\alph*.]
\item The problem of optimally designing the switching mWM filters is formulated as an optimization problem, with the AEC-OOG is taken as the objective;%where the AEC-OOG is taken as the impact metric; 
\item The worst-case scenario of a covert attack with exact knowledge of plant and mWM filter parameters is embedded within the design problem;
% The optimization problem is defined to incorporate the worst-case scenario of a covert attack with exact knowledge of plant and mWM filter parameters;
\item The feasibility of the optimization problem is shown to be dependent only on stability conditions; 
\item A solution scheme is proposed to promote randomization of the mWM filter parameters such that an eavesdropping adversary cannot remain stealthy.
\end{enumerate} 

This builds on the results of \cite{ferrari2020switching}, where the focus was on the design of the switching protocols, rather than the parameters themselves.
Compared to previous work \citep{gallo2021design}, this paper introduces an optimization problem which is always feasible (thanks to the use of AEC-OOG in the objective), while also considering a more sophisticated class of covert attacks, where the presence of watermark is known to the adversary. 
Moreover, this paper poses a different objective than \citep{zhang2023hybrid}; indeed, while \citep{zhang2023hybrid} provided a design strategy to ensure certain privacy properties, in this paper we address the problem of optimal parameter design following a switching event.


%\subsection{Organization}
The rest of the paper is organized as follows. 
After formulating the problem in Section~\ref{sec:PF}, we propose our design algorithm in Section~\ref{sec:main}, and analyze its properties. It is then evaluated through a numerical example in Section~\ref{sec:NE}, and concluding remarks are given Section~\ref{sec:Con}.
% We provide the problem background in Section~\ref{sec:PF}. We formulate the design problem in Section~\ref{sec:main}, together with an analysis of its properties. The proposed algorithm is evaluated through a numerical example in Section \ref{sec:NE}. Concluding remarks are offered in Section \ref{sec:Con}.
\section{BnTTS}
\begin{figure}[ht!]
    \raggedleft
    \includegraphics[width=0.90\linewidth]{resources/hqtts.png} 
    \caption{Overview of BnTTS Model.} 
    \label{fig:xtts_train_diagram}
    % \vspace{-.2cm}
\end{figure}


\textbf{Preliminaries:} Given a text sequence with \( N \) tokens, \( \mathbf{T} = \{t_1, t_2, \ldots, t_N\} \), and a speaker's mel-spectrogram \( \mathbf{S} = \{s_1, s_2, \ldots, s_L\} \), the objective is to generate speech \( \hat{\mathbf{Y}} \) that matches the speaker's characteristics. The ground truth mel-spectrogram frames for the target speech are denoted as \( \mathbf{Y} = \{y_1, y_2, \ldots, y_M\} \). The synthesis process can be described as:
% \vspace{-0.12cm}
\[
\hat{\mathbf{Y}} = \mathcal{F}(\mathbf{S}, \mathbf{T})
\]
% \vspace{-0.12cm}
where \( \mathcal{F} \) produces speech conditioned on both the text and the speaker's spectrogram.

\noindent \textbf{Audio Encoder:} A Vector Quantized-Variational AutoEncoder (VQ-VAE) \cite{tortoise} encodes mel-spectrogram frames \( \mathbf{Y} \) into discrete tokens \( M \in \mathcal{C} \), where $\mathcal{C}$ is vocab or codebook. An embedding layer then transforms these tokens into a \( d \)-dimensional vector: \( \mathbf{Y_e} \in \mathbb{R}^{M \times d} \).

\noindent \textbf{Conditioning Encoder \& Perceiver Resampler:} The Conditioning Encoder \cite{casanova2024xtts} consists of \( l \) layers of \( k \)-head Scaled Dot-Product Attention, followed by a Perceiver Resampler. The speaker spectrogram \( \mathbf{S} \) is transformed into an intermediate representation \( \mathbf{S_z} \in \mathbb{R}^{L \times d} \), where each attention layer applies a scaled dot-product attention mechanism. The Perceiver Resampler generates a fixed output dimensionality \( \mathbf{R} \in \mathbb{R}^{P \times d} \) from a variable input length \( L \).

\noindent \textbf{Text Encoder:} The text tokens \( \mathbf{T} = \{t_1, t_2, \ldots, t_N\} \) are projected into a continuous embedding space, yielding \( \mathbf{T_e} \in \mathbb{R}^{N \times d} \).

\noindent \textbf{Large Language Model (LLM):} The transformer-based LLM \cite{radford2019language} utilizes the decoder portion. Speaker embeddings \( \mathbf{S_p} \), text embeddings \( \mathbf{T_e} \), and ground truth spectrogram embeddings \( \mathbf{Y_e} \) are concatenated to form the input:
% \vspace{-0.2cm}
\[
\mathbf{X} = \mathbf{S_p} \oplus \mathbf{T_e} \oplus \mathbf{Y_e} \in \mathbb{R}^{(N + P + M) \times d}
\]
% \vspace{-0.12cm}
The LLM processes \( \mathbf{X} \), producing output \( \mathbf{H} \) with hidden states for the text, speaker, and spectrogram embeddings. During inference, only text and speaker embeddings are concatenated, generating spectrogram embeddings \( \{h_1^Y, h_2^Y, \ldots, h_P^Y\} \) as the output.

\noindent \textbf{HiFi-GAN Decoder:} The HiFi-GAN Decoder \cite{kong2020hifi} converts the LLM's output into realistic speech, preserving the speaker's characteristics. Specifically, it takes the LLM's speech head output \( \mathbf{H}_\text{Y} = \{h_1^Y, h_2^Y, \ldots, h_P^Y\} \). The speaker embedding \( \mathbf{S} \) is resized to match \( \mathbf{H}_\text{Y} \), resulting in \( \mathbf{S}' \in \mathbb{R}^{P \times d} \). The final audio waveform \( \mathbf{W} \) is then generated by:
% \vspace{-0.3cm}
\[
\mathbf{W} = g_\text{HiFi}(\mathbf{H}_\text{Y} + \mathbf{S}')
\]
% \vspace{-0.12cm}
Thus, the HiFi-GAN decoder produces speech that reflects the input text while maintaining the speaker's unique qualities.

In this section, we empirically compare the proposed algorithm on both sequence windows and time windows with existing methods.
\paragraph{Datasets} For the sequence-based model, we used two synthetic datasets and two cross-language datasets. The statistics of the datasets are provided in Table \ref{table:statistics}:

\begin{table}[t]
    \centering
    \caption{The statistics of the datasets. The datasets satisfy $1 \leq \|\vx\|\|\vy\| \leq R $.}
    \label{table:statistics}
    \begin{tabular}{|c|c|c|c|c|c|}
    \hline
        Dataset & $n$ & $m_x$ & $m_y$ & $N$ & $R$ \\ \hline
        SYNTHETIC(1) & 100,000 & 1,000 & 2,000 & 50,000 & 65 \\ \hline
        SYNTHETIC(2) & 100,000 & 1,000 & 2,000 & 50,000 & 724 \\ \hline
        APR & 23,235 & 28,017 & 42,833 & 10,000 & 773 \\ \hline
        PAN11 & 88,977 & 5,121 & 9,959 & 10,000 & 5,548 \\ \hline
        EURO & 475,834 & 7,247 & 8,768 & 100,000 & 107,840 \\ \hline
    \end{tabular}
\end{table}

\begin{itemize}
    \item Synthetic: The elements of the two synthetic datasets are initially uniformly sampled from the range (0,1), then multiplied by a coefficient to adjust the maximum column squared norm $R$. The X matrix has 1,000 rows, and the Y matrix has 2,000 rows, each with 100,000 columns. The window size is set to 50,000.
    \item APR: The Amazon Product Reviews (APR) dataset is a publicly available collection containing product reviews and related information from the Amazon website. This dataset consists of millions of sentences in both English and French. We structured it into a review matrix where the X matrix has 28,017 rows, and the Y matrix has 42,833 rows, with both matrices sharing 23,235 columns. The window size is 10,000.
    \item PAN11: PANPC-11 (PAN11) is a dataset designed for text analysis, particularly for tasks such as plagiarism detection, author identification, and near-duplicate detection. The dataset includes texts in English and French. The X and Y matrices contain 5,121 and 9,959 rows, respectively, with both matrices having 88,977 columns. The window size is 10,000.
\end{itemize}
We evaluate the time-based model on another real-world dataset:
\begin{itemize}
    \item EURO: The Europarl (EURO) dataset is a widely used multilingual parallel corpus, comprising the proceedings of the European Parliament. We selected a subset of its English and French text portions. The X and Y matrices contain 7,247 and 8,768 rows, respectively, and both matrices share 475,834 columns. Timestamps are generated using the $Poisson$ $Arrival$ $Process$ with a rate parameter of $\lambda=2$. The window size is set to 100,000, with approximately 30,000 columns of data on average in each window.
\end{itemize}

\paragraph{Setup} For the sequence-based model, we compare the proposed hDS-COD and  aDS-COD with EH-COD~\cite{yao2024approximate} and DI-COD~\cite{yao2024approximate}. We do not consider the Sampling algorithm as a baseline, as its performance is inferior to that of EH-COD and DI-CID, as demonstrated in \cite{yao2024approximate}. %The hDS-COD is adjusted by the parameter $\ell$ and the maximum number of levels $L = \log{R}$, where $R$ is the prior estimate of the maximum squared column norm of the dataset. DI-COD similarly requires a prior estimate of $R$ to limit the maximum number of levels $L = \log{(R/\varepsilon})$. In contrast, aDS-COD and EH-COD do not require an estimate of $R$; their error-space balance is controlled by the parameter $\ell = \frac{1}{\varepsilon}$. 
For the time-based model, we compare the proposed hDS-COD and  aDS-COD with EH-COD and the Sampling algorithm since DI-COD cannot be applied to time-based sliding window model. To achieve the same error bound, the maximum number of levels for hDS-COD is set to $L = \log{(\varepsilon NR)}$, and the initial threshold for aDS-COD is set to $1$.

Our experiments aim to illustrate the trade-offs between space and approximation errors. The x-axis represents two metrics for space: final sketch size and total space cost. The final sketch size refers to the number of columns in the result sketches $\mA$ and $\mB$ generated by the algorithm, representing a compression ratio. The total space cost refers to the maximum space required during the algorithm's execution, measured by the number of columns.We evaluate the approximation performance of all algorithms based on correlation errors $\operatorname{corr-err}(\mathbf{X}_W \mathbf{Y}_W^\top, \mathbf{A} \mathbf{B}^\top)$, which is reflected on the y-axis. Every 1,000 iterations, all algorithms query the window and record the average and maximum errors across all sampled windows.

The experiments for all algorithms were conducted using MATLAB (R2023a), with all algorithms running on a Windows server equipped with 32GB of memory and a single processor of Intel i9-13900K.

\paragraph{Performance} Figure \ref{fig:error vs l} and Figure \ref{fig:error vs space} illustrate the space efficiency comparison of the algorithms on sequence-based datasets. Panels (a-d) show the average errors across all sampled windows, while panels (e-h) display the maximum errors.

Figure \ref{fig:error vs l} evaluates the compression effect of the final sketch. The hDS-COD, aDS-COD, and EH-COD show similar compression performances. But the DS series is more stable, particularly on the synthetic datasets, where they significantly outperform EH-COD and DI-COD. The performance of hDS-COD and aDS-COD is nearly the same, indicating that the adaptive threshold trick in aDS-COD does not have a noticeable negative impact on it, maintaining the same error as hDS-COD.

Figure \ref{fig:error vs space} measures the total space cost of the algorithms. hDS-COD and aDS-COD show a significant advantage over existing methods, as they can achieve the  $\varepsilon$-approximation error with much less space. For the same space cost, the correlation errors of hDS-COD and aDS-COD are much smaller than those of EH-COD and DI-COD. Also, aDS-COD has better space efficiency than hDS-COD because aDS only uses a single-level structure while hDS requires $\log R+1$ levels. We find that hDS-COD requires more space on  SYNTHETIC(2) dataset compared to SYNTHETIC(1) dataset. This phenomenon occurs because SYNTHETIC(2) dataset has a larger $R$, which confirms the dependence on $R$ as stated in Theorem~\ref{thm:hds}. 

Figure \ref{fig:time-based} compares the performance of algorithms on time-based windows. Panels (a) and (b) present the error against the final sketch size, which show that our aDS-COD and hDS-COD algorithms enjoy similar performance as EH-COD and significantly outperform the sampling algorithm. On the other hand, as shown in panels (c) and (d), our methods outperform baselines in terms of total space cost.


% \begin{figure*}[htpb!]
% \label{}
% \centering

%     {{\label{ROCIowaCedar} \includegraphics[width=\textwidth/3]{figures/IowaCedar_roc.png}}}%
%     \qquad
%     {{\label{ROCIowaDesMoines} \includegraphics[width=\textwidth/3]{figures/IowaDesMoines_roc.png} }%
%   \captionsetup{justification=centering}
%   \caption{\Acf{ROC} curves for \acf{RW} Iowa (CR) and  \acf{RW} Iowa (DM) dataset. Dummy model here represents a model whose output is solely a ``no Flood'' for all pixels.}
%   \label{fig:RW_ROC_Curves}%
% \end{figure*}



\section{Results and Discussions}
\label{sec:Results}

In this section, we aim to answer three main questions. First, we want to validate our hypothesis that \ac{SYN} data is a viable proxy for \ac{RW} data when training ML models for downscaling. Secondly, we seek to assess how much more skillful ML-based downscaling is compared to classical, non-data-driven techniques, such as our baseline methods, \textit{i.e.}, thresholded bicubic and Lanczos interpolation. Finally, we would like to appraise the extent to which data-driven models like ours are transferable (in terms of usefulness) to other regions without major performance degradations.  
To assess the quality of the models, we conduct a multiple comparison test --namely the Holm-Bonferroni procedure \cite{HolmBonferroni1979} -- that is designed to control the \ac{FWER}. We notice that, with a \ac{FWER} of $10^{-3}$, all the differences in model performance are significant. The only exception to this trend was observed in \ac{RW}-GH for whom the pairwise differences between \ac{RCAN} and \ac{ESRT}, Lanczos and Bicubic were not significant with the aforementioned \ac{FWER}. 

%Finally, we aim to find out the factors influencing the transferability of our models from one region to another.

\subsection{Potential of using SYN Data for RW downscaling}

In order to evaluate the utility of synthetic data for training, we compare performances of our candidate models on both \ac{SYN} and \ac{RW} Iowa data whose results are presented in Table \ref{tab:IowaResults}. We notice that 
\textbf{(i)} For the Iowa datasets, there is a drop in performance of all the models when going from \ac{SYN} to \ac{RW} datasets, 
\textbf{(ii)} for the \ac{RW}-IA (CR) as well as \ac{RW}-IA (DM) datasets, both bicubic and Lanczos interpolation have accuracies and MCC up to 70.89\% and 0.42 respectively while the deep learning models have accuracies and MCC up to 73.34\% and 0.46 respectively, 
\textbf{(iii)} There is a roughly 6\% accuracy improvement for the \ac{SYN} data for the deep learning models compared to the bicubic and lanczos models and this improvement drops to about 3\% for \ac{RW} data,  
\textbf{(iv)} the performance of all the models remain consistent across both \ac{RW}-IA datasets and \textbf{(v)} in \figref{fig:RW_ROC_Curves}, we observe that there is a high degree of overlap among the \ac{ROC} curves for the data-driven models.

From (i) and (iv) we can conclude that \ac{SYN} data is more intricate than \ac{RW} data. This implies that the benefits yielded by training with \ac{SYN} dataset, while significant, is not as prominent in the \ac{RW} Iowa datasets. 
% This may be due to sensor noise prevalent in the \ac{RW} Landsat-8 data that can be harder to reproduce in the synthetically generated examples. 
(i), (iii) and (v) implies that while \ac{SYN} data is not an exact replacement for \ac{RW} data, it provides a rather significant edge, which is all the more important when there is insufficient \ac{RW} for training. From (ii) we can conclude that the three proposed data driven models outperform classical super-resolution techniques such as bicubic and lanczos, conclusion supported by the \ac{ROC} curves in Figure \ref{fig:RW_ROC_Curves} for whom the data-driven models, in general, lie above the non-data-driven alternatives. Observation (iv) shows that  for the climatically similar \ac{RW}-Iowa(CR) and \ac{RW}-Iowa(DM) regions, training on \ac{SYN} Iowa data does indeed provide an edge. 

% have similar climate. 

\begin{figure*}[t!]
    \centering
    \begin{subfigure}[t]{0.5\textwidth}
        \centering
        \includegraphics[width=\textwidth/2]{figures/IowaCedar_roc.png}
        \caption{}
    \end{subfigure}%
    ~ 
    \begin{subfigure}[t]{0.5\textwidth}
        \centering
        \includegraphics[width=\textwidth/2]{figures/IowaDesMoines_roc.png}
        \caption{}
    \end{subfigure}
    \vspace*{0.5cm}
    \caption{    \label{fig:RW_ROC_Curves} \Acf{ROC} curves for (a) RW-IA (CR) and (b) RW-IA (DM) dataset. Na\"ive model here represents a model whose output is solely a ``no Flood'' for all pixels. Star here represents the pixel-wise classifier with a threshold of 0.5.}
\end{figure*}


\subsection{Effectiveness of data-driven approaches}

In order to evaluate the effectiveness of ML models in the downscaling task, we compare performances of our candidate models to Lanczos and bicubic interpolation methods by looking at figures of some sample predictions from Iowa (Figure \ref{fig:RWIowaDesMoines}), performance comparison in the region of Iowa in Table \ref{tab:IowaResults} and the ROC curves in Figure \ref{fig:RW_ROC_Curves} for \ac{RW} data. We notice that 
\textbf{(vi)} For RW-IA (DM) samples, the deep learning models maintain a higher degree of spatial continuity in the predicted \ac{FIM}, 
\textbf{(vii)} We observe that  bicubic and Lanczos interpolation produces over-smoothed \ac{FIM} reconstructions, while the plain \ac{RDN}, \ac{RCAN} and \ac{ESRT} models are more detail-inclusive. Similar conclusions can be drawn upon inspecting the \ac{ROC} curves in Figure \ref{fig:RW_ROC_Curves} and 
\textbf{(viii)} For RW-IA (CR), the ML models show a performance improvement of 3.06\% when comparing the best ML model and non-data-driven method and, while for RW-IA (DM) there is a performance improvement of 2.45\%.


Figures \ref{fig:EUSamples} and \ref{fig:RWIowaDesMoines} show the spatial disparity among the models whose details are often obscured in aggregated metrics such as accuracy. (vi) This implies that these data-driven models are better are recognizing an underlying stream network geometry than the classical methods. However, when it comes to narrow river streams, all the models struggle capturing the nuances of the \ac{FIM} resultant from localized high elevation features such as small islands within rivers or man-made structures. (vii) shows a clear advantage of our data-driven approaches over the non-data-driven alternatives. (viii) indicates the benefits of the data-driven models when evaluated over Iowa. 



\subsection{Applicability of our models to external regions}

To evaluate how transferable our models are, we draw conclusions from figures of the sample predictions from Western Europe (Figure \ref{fig:EUSamples}) and Ghana (Figure \ref{fig:GhanaSamples}) as well as the performance comparison in Table \ref{tab:ExternalResults}. We notice that 
\textbf{(ix)} for Ghana all of the models fail to adequately inundate the pixels over separated areas on account of several disconnected regions of inundation in the chosen area,
\textbf{(x)} the ML models outperform non-data driven methods for RW-EU, 
\textbf{(xi)} for the RW-EU dataset, there is an improvement of 4.89\% when comparing the accuracy of the best data- and non-data-driven methods, 
\textbf{(xii)} For RW-RR and RW-GH, there is marginal improvement (up to 0.77\% in accuracy) of the ML methods over the non-data driven methods and 
\textbf{(xiii)} For RW-EU, we notice that the ML models produce more connected streams over the non-data-driven models. 

(x) and (xi) implies that the models are transferable when considering hydroclimaticalogically similar regions since Iowa and the Meuse river in Europe lie within mid temperate zones. Similar to the observation (vi) for RW-IA (DM), (xiii) implies that the benefits of the ML model in identifying underlying network streams is also transferable to hydroclimatologically similar regions. In contrast, (xii) and (ix) both imply that the trained ML models struggle to generalize to RW-RR \& RW-GH. We speculate that this may be due to the significant differences in geography and climate when compared to Iowa. 

% More specifically, we notice that Ghana has a lot of disconnected regions when compared to Iowa and Western Europe, possibly indicating a geomorphological dissimilarity. Additionally, in the case of Red River and Ghana, we also speculate that they include drivers to flood inundation that are different from Iowa and Western Europe, which lie within mild temperate zones. Ghana on the other hand has a tropical (dry and hot) climate.

Our study directly implies that good quality synthetic data can be useful surrogates for downscaling low-resolution \acp{WFM} to high-resolution \acp{FIM} in regions, where such data are hard to come by, even when downscaling by a factor of 10. We noticed that such models were readily transferable to climatically similar regions as the region of training. However, Such derived ML models did not feature significantly different transferability when evaluated over hydroclimatologically dissimilar regions, which we attribute to different flood inundation characteristics, primarily at finer scales. A possible avenue to circumvent such issues is to explore additional training approaches that fall under the general area of domain adaptation. Nevertheless, data-driven models are still advantageous (and, hence, preferable) over non-data-driven alternatives in transfer scenarios like the one we considered here. 


%%%%%%%%%%%%%%%%%%%%%%%%%%%%%%% unused text %%%%%%%%%%%%%%%%%%%%%%%%%%%%%%%%%%%%%%%



% \tabref{tab:AccuracyResults} depicts test accuracies obtained by our models on both \ac{SYN} and \ac{RW} data. For Iowan floods, a comparison of \ac{SYN} and \ac{RW} results shows \textbf{(i)} bicubic and Lanczos interpolations remarkably gaining about $3\%$ in accuracy, as well as \textbf{(ii)} \ac{RDN} and \ac{RCAN} remaining relatively stable, while \textbf{(iii)} topography-aware models loosing $2.7\%$ in performance. From (i) one can conclude that \ac{SYN} data are morphologically slightly more intricate than \ac{RW} data. Also, (i) and (ii) likely imply that \ac{SYN} data, excluding topography, can serve as satisfactory surrogates of \ac{RW} data. However, as implied by (iii), our topography-dependent models seems to be particularly sensitive to distributional shifts of their combined inputs (\acp{WFM} and topographic features). More specifically, the topography-informed models' performance edge, while still statistically significant, is extremely marginal, even when compared to our non-data-driven approaches. Next, when comparing results between the cases of Iowan and Ghanaian \ac{RW} data, one observes that \textbf{(iv)} the accuracy of bicubic and Lanczos interpolations drops by almost $5\%$ due to over-smoothing. This may imply that Ghanaian \acp{FIM} bare a more complex morphology, when compared to Iowan \acp{FIM}. Also, \textbf{(v)} our topography-agnostic, data-driven models' performance degrades more gracefully (by about $2\%$), while \textbf{(vi)} our topography-aware models perform, virtually, as bad as our non-data-driven approaches. Hence, the differences in the data populations of the two regions we considered are significant enough to render our topography-dependent models noncompetitive. 




\section{Related Work}
% Goal-oriented dialogue requires agents to complete a specific task through multi-round dialogue~\cite{bordes2016learning,rajendran2018learning,williams2007partially}. 

% Although goal-oriented spoken and text-based dialogues have been studied for many years in the field of Natural Language Processing\cite{bordes2016learning,rajendran2018learning,williams2007partially}, goal-oriented visual dialogue moves the scene into a more realistic visual environment, making it a relatively more practical and challenging field. 

% The goal of GuessWhat?!~\cite{de2017guesswhat} is to distinguish a defined object in an image through dialogue, while the goal of GuessWhich~\cite{das2017learning} is to identify the correct image from a series of images. 

% There are usually two dialogue agents, Questioner and Oracle. The Questioner keeps asking questions to find the defined but undisclosed target, and the Oracle defines the target object in advance and answers questions accordingly.
% In a dialogue, there are typically two agent types, {\it i.e.}, the Questioner and the Oracle. The Questioner consists of two sub-models, QGen and Guesser. 
% They all involve QGen, Guesser and Oracle. 
% Our main focus is on the QGen. Please refer to the supplementary materials for more details about Oracle and Guesser.



% \subsection{Oracle}

% In the initial work of GuessWhat?!, a baseline Oracle was proposed, which concatenates the question encoding and the spatial and category information of the target object together and inputs them into the MLP layer to predict the final answer. However, without the introduction of visual information, the baseline Oracle may have difficulty understanding questions that involve color, shape, and object relations. Tu et al.\cite{tu2021learning} introduced visual features predicted by object detection models such as Faster-RCNN\cite{ren2015faster} into Oracle's decision-making process, but the way did not effectively help Oracle understand questions that involve information such as object relations or color.

% \subsection{Guesser}

% Guesser not only needs to perform referring expression comprehension for dialogue describing visual objects but also needs to perform reasoning. The initial work proposed a model that combines the encoding of the entire dialogue history with each object category and spatial information to predict the target object\cite{de2017guesswhat,strub2017end}. Later work\cite{shukla2019should,lu202012,deng2018visual} treated the entire dialogue history as a whole. However, the Guesser model does not encode any visual information. Considering that the lack of turn-level visual grounding can cause the Guesser to confuse the object referred to in each question, some methods\cite{simonyan2014very,pang2020guessing} introduced features such as VGG and Faster-RCNN into the Guesser model. Considering the dynamic characteristic of multi-turn dialogue reasoning, Pang et al.\cite{pang2020guessing} proposed to decompose the dialogue into turn-level and use state tracking to dynamically update the guessing confidence, demonstrating a significant performance improvement. Recent work\cite{tu2021learning} introduced a Visual-Linguistic pre-trained model, giving the agent more visual language shared representations and prior knowledge, which has achieved good results.


% \subsection{Question Generator}

%CHANGED-0614
\subsection{Question Generator (QGen)}
% \textbf{QGen.} 
The QGen plays a core role in the goal-oriented visual dialogue, as it not only needs to ask questions that can acquire certain information gain but also guides the dialogue towards the direction of the target.  
De Vries et al.~\shortcite{de2017guesswhat} propose the first QGen model with an encoder-decoder structure, in which the dialogue history is encoded by a Hierarchical Recurrent Encoder-Decoder (HRED)~\cite{serban2015hierarchical}, and the image is conditionally encoded as VGG features~\cite{simonyan2014very}.
Strub et al.~\shortcite{strub2017end} introduce the approach of RL and provide a 0-1 reward, where 1 indicates successful finding of the target in the dialogue. Built upon this approach, Zhang et al.~\shortcite{zhang2018goal} propose intermediate rewards from three dimensions to improve the model performance. 
Shekhar et al.~\shortcite{shekhar2018beyond} introduce a shared dialogue state encoder for Guesser and QGen, in which the visual encoder is based on ResNet~\cite{he2016deep}, and the language encoder is based on LSTM~\cite{hochreiter1997long}. Pang et al.~\shortcite{pang2020visual} introduce a turn-level object state tracking mechanism to QGen. Tu et al.~\shortcite{tu2021learning} introduce a Visual-Linguistic pre-trained model to QGen, which makes the object's semantic coverage more comprehensive and better.
Our main focus is on how to train QGen. 
The fundamental difference between TSADE and prior work lies in its clever use of a non-goal-oriented questioning strategy~(NGOQS) to find target, whereas prior works~\cite{zhang2018goal,shukla2019should,testoni2021looking} utilize a goal-oriented questioning strategy~(GOQS). 
We experimentally prove that flexibly using NGOQS is more useful than simply using GOQS, and GOQS can benefit from NGOQS.



%Please refer to the supplementary materials for the difference between our method and prior work, as well as for more details about Oracle and Guesser.
% Please refer to the supplementary materials for more details about Oracle and Guesser.



\subsection{Answer Distribution Estimator (ADE)}
% \textbf{Answer Distribution Estimator (ADE).}
Given a question, ADE actually employs an internal Oracle to answer all objects in the image to obtain an answer distribution. Lee et al.~\shortcite{lee2018answerer} first introduce the ADE module to propose an Answerer in Questioner’s Mind (AQM) algorithm to obtain question in each round.
In this work, ADE refers to an approximated model of the original Oracle explicitly trained by AQM's Questioner. 
It abandons the paradigm of deep learning, and uses mathematics and the approximated model to directly calculate information gain to select question from training data in each round. 
% However, this paradigm of selecting question from training data has great limitations. The fixed training data usually can't cover the huge actual scenes in life. 
% And the information gain of all training data must be calculated in each round, making the calculation cost very high.
% Different from AQM, TSADE is a paradigm based on question generation, which has stronger generalization and lower computational cost. TSADE employs the answer distribution to dynamically update the real-time candidate objects and calculate reward score for the quality of each question. 
% Then the reward score is put into RL to optimize question generation.
Zhang et al.~\shortcite{zhang2018goal} propose three intermediate rewards to optimize the model in RL. 
% It explicitly obtains higher rewards with fewer rounds. 
Based on the goal-oriented way, it hope that the probability of ground truth (target) will progressively increase during the whole process. It uses ADE to avoid useless questions based on answer distribution. However, it does not consider what kind of questions are most useful. The difference is that TSADE takes the issue into account and uses ADE to achieve the same final goal in a non-goal-oriented way, without paying attention to which target is during the whole process.
Testoni and Bernardi \shortcite{testoni2021looking} propose the ``confirm-it'' strategy to select question that can gradually increase the probability of the target from the candidate questions. It uses an internal Oracle to provide answers specific to the target for a set of candidate questions. These answers are then used by the Guesser to compute a probability distribution over candidate objects. 
% In contrast, TSADE uses the internal Oracle to obtain an answer distribution over the candidate objects. The former's internal Oracle responds to target based on a set of questions, while the latter's internal Oracle responds to candidate objects based on a single question.
%We can see that existing methods do not have an efficient and intuitive strategy to guide question generation. Previous research\cite{strub2017end,shukla2019should,zhang2018goal,zhao2018improving} has used Reinforcement Learning methods to learn the Questioner/Guesser model by designing different rewards, such as end-game success or information gain from question generation. However, the question-generation strategy under these methods is fuzzy, uninterpretable, and inefficient. This paper proposes an Answer Distribution Estimator (ADE) that explicitly uses a binary search strategy to guide question generation, further integrates the state distributions of different agents, and enhances the fusion of visual and textual information.

% \begin{figure}[h]
%   \centering
%   \includegraphics[width=0.8\linewidth]{images/fig1_emnlp.pdf}
%   \caption{It shows an example of the GuessWhat?! game that describes the process of attention transfer in dialogue based on the Tree-structured strategy. The excluded objects are in the lower-right candidate box. The target object is highlighted in green box.}
%   \label{fig:example of strategy}
% \end{figure}
\section{Conclusion}
In this work, we introduced BnTTS, the first speaker-adaptive TTS system for Bangla, capable of generating natural and clear speech with minimal training data. Built on the XTTS pipeline, BnTTS effectively supports zero-shot and few-shot speaker adaptation, outperforming existing Bangla TTS systems in sound quality, naturalness, and clarity. Despite its strengths, BnTTS faces challenges in handling diverse dialects and short-sequence generation. Future work will focus on training BnTTS from scratch, developing medium and small model variants, and exploring knowledge distillation to optimize inference speed for real-time applications.

% In this work, we introduce BnTTS, the first speaker-adaptive TTS system for Bangla, achieving natural and clear speech synthesis with minimal training data. By adopting the XTTS pipeline, BnTTS supports zero-shot and few-shot speaker adaptation, outperforming existing Bangla TTS systems in sound quality, naturalness, and clarity. However, it faces limitations with diverse dialects, and struggles with short sequences. Future improvements will focus on training BnTTS from scratch, with medium and small variants and also distrilling base model into smaller model for faster inference in realtime infeerence.
% expanding datasets, using more advanced models, and incorporating multilingual support to enhance its versatility.
% \section{Conclusion}

% In this study, we introduce BnTTS, the first open-source speaker adaptation based TTS system, which improves speech generation for Bangla, a low-resource language. By adapting the XTTS pipeline to accommodate Bangla's unique phonetic characteristics, the model is able to produce natural, clear, and accurate speech with minimal training data, supporting both zero-shot and few-shot speaker adaptation. BnTTS outperforms existing Bangla TTS systems in terms of speed, sound quality, and clarity, as confirmed by listener ratings. 

\section{Limitations}
Despite the significant performance of BnTTS, the system has several limitations. It struggles to adapt to speakers with unique vocal traits, especially without prior training on their voices, limiting its effectiveness in speaker adaptation tasks. We found poor performance on short text due to pre-existing issues in the XTTS foundation model. Although we improved performance by modifying generation settings and incorporating additional training with Complete Audio Prompting, the model still fails to generate sequences under two words or 20 characters in some cases. We did not investigate the performance of the XTTS model by training from scratch; instead, we used continual pretraining due to resource constraints, which may have yielded better results.

\section{Acknowledgments}
We are grateful to HISHAB\footnote{\url{https://www.verbex.ai/}}  for providing us with
all the necessary working facilities, computational
resources, and an appropriate environment through-
out our entire work.
% \textcolor{green}{
% \begin{itemize}
%     \item training from scratch is not possible
%     \item Subjective evaluation may change across the experiment
%     \item generated speech may change across inference
%     \item BengaliNamedEntity1000 only 200 are selected
%     5*1000 + 5*1000 + 1000+1000+1000
%     200 * 
% \end{itemize}
% }

% Despite the significant performances of BnTTS, the system has several limitations. It struggles to adapt to speakers with unique vocal traits, especially without prior training on their voices, limiting its effectiveness in speaker adaptation tasks. We found poor performance on short text due to pre-existing issues in the XTTS foundation model. Although we improved performance by modifying generation settings and incorporating additional training with full audio-text prompting, the model still fails to generate sequences under two words or 20 characters in some cases. We don't investigate the performance from XXTS model by training from strach instead we use continul pretraining due to resource contraints , this may provide better result. 
% As we did continual pretraining on 

% Future works include scratch from training using ba

% We found poor performance on short text becasure the prior issues existing in XXTS foundation model. Although, we improve the performance using some modification is generation setting and using an additional training with full audio text in prompting, still it fails to generated sequence under 2 words or 20 characters in some cases. 

% XTTS, the foundation for BnTTS, performs poorly on very short sequences. 



% While adjustments to hyperparameters and generation settings (temperature and TopK) have mostly resolved this issue, instances remain where the model struggles to generate sequences under 2 words or 20 characters. 

% - the system struggles with adapting to speakers with unique vocal traits, especially without prior training on their voices.

% - Its focus on Bangla also restricts its usefulness for other low-resource languages.

% - XTTS, the foundation for BnTTS, performs poorly on very short sequences. While adjustments to hyperparameters and generation settings (temperature and TopK) have mostly resolved this issue, instances remain where the model struggles to generate sequences under 2 words or 20 characters.

% - training from scratch is not possible
% - Subjective evaluations are subject to change because the speech generated by models may exhibit variations across different inference sessions.
% -  Evaluation is time-consuming, so we have to select 200 sentences from the BengaliNamedEntity1000 eval dataset with 1000 sentences. 



% It relies on a small and uniform dataset, which limits its ability to work well with the diverse dialects and accents in Bangla, potentially affecting the naturalness of the speech output for certain regional variations. 


% Some key observations include that training from scratch the original XTTS does not converge for our dataset, which may be due to insufficient amounts of training data. Subjective evaluations are subject to change because the speech generated by models may exhibit variations across different inference sessions. This variability could lead to differences in results and affect the reliability of the evaluations. Evaluation is time-consuming, so we have to select 200 sentences from the BengaliNamedEntity1000 eval dataset with 1000 sentences. 

% BnTTS has some limitations. It relies on a small and uniform dataset, which limits its ability to work well with the diverse dialects and accents in Bangla, potentially affecting the naturalness of the speech output for certain regional variations. The use of GPT-2, chosen due to limited resources, may also limit the system’s scalability and performance compared to newer models. Additionally, the system struggles with adapting to speakers with unique vocal traits, especially without prior training on their voices. Its focus on Bangla also restricts its usefulness for other low-resource languages. Furthermore, XTTS, the foundation for BnTTS, performs poorly on very short sequences. While adjustments to hyperparameters and generation settings (temperature and TopK) have mostly resolved this issue, instances remain where the model struggles to generate sequences under 2 words or 20 characters. Future improvements could involve a larger dataset, more advanced models, and multilingual support to boost its versatility and robustness.

\section{Ethical Considerations}

The development of BnTTS raises ethical concerns, particularly regarding the potential misuse for unauthorized voice impersonation, which could impact privacy and consent. Protections, such as requiring speaker approval and embedding markers in synthetic speech, are essential. Diverse training data is also crucial to reduce bias and reflect Bangla’s dialectal variety. Additionally, synthesized voices risk diminishing dialectal diversity. As an open-source tool, BnTTS requires clear guidelines for responsible use, ensuring adherence to ethical standards and positive community impact.


\bibliography{custom}



\newpage
\appendix
% \clearpage
% \section{Appendix}

% App A: Dataset
% App B: Data Aquization Framework
% C: Eval Data
% D: Eval Metrics
% E: Model Architecture
% F:Training Objective

% \input{sections/Related Works}

\section{TTS Data Acquisition Framework}
\label{sec:data_collection}

\begin{figure}[hbt!]
    \centering
    \includegraphics[width=0.8\linewidth]{resources/TTS_Data_Collection_Pipeline.png} 
    \caption{Overview of our TTS Data Acquisition Framework. The acquisition process involves using a Speech-to-Text model to obtain transcription, an LLM to restore transcription's punctuation, a noise suppression model to remove unwanted noise, and finally an audio superresolution model to enhance audio quality and loudness.}
    \label{fig:pseudo_labeled_dataset}
\end{figure}

Bangla is a low-resource language, and large-scale, high-quality TTS speech data are particularly scarce. To address this gap, we developed a TTS Data Acquisition Framework (Figure \ref{fig:pseudo_labeled_dataset}) designed to collect high-quality speech data with aligned transcripts. This framework leverages advanced speech processing models and carefully designed algorithms to process raw audio inputs and generate refined audio outputs with word-aligned transcripts. Below, we provide a detailed breakdown of the key components of the framework.


\textbf{1. Speech-to-Text (STT):} The audio files are first processed through an in-house our STT system, which transcribes the spoken content into text. The STT system used here is an enhanced version of the model proposed in \cite{nandi-etal-2023-pseudo}.

\textbf{2. Punctuation Restoration Using LLM:} Following transcription, a LLM is employed to restore appropriate punctuation \cite{openai2023gpt}. This step is crucial for improving grammatical accuracy and ensuring that the text is clear and coherent, aiding in further processing.

\textbf{3. Audio and Transcription Segmentation:} The audio and transcription are segmented based on terminal punctuation (full-stop, question mark, exclamatory mark, comma). This ensures that each audio segment aligns with a complete sentence, maintaining the speaker's prosody throughout.

\textbf{4. Noise and Music Suppression:} To improve audio quality, noise and music suppression techniques \cite{defossez2019music} are applied. This step ensures that the resulting audio is free of background disturbances, which could degrade TTS performance.

\textbf{5. Audio SuperResolution:} After noise suppression, the audio files undergo super-resolution processing to enhance audio fidelity \cite{liu2021voicefixer}. This ensures high-quality audio, crucial for producing natural-sounding TTS outputs.


This pipeline effectively enhances raw audio and corresponding transcription, resulting in a high-quality pseudo-labeled dataset. By combining ASR, LLM-based punctuation restoration, noise suppression, and super-resolution, the framework can generate very high-quality speech data suitable for training speech synthesis models.

\subsection{Dataset Filtering Criteria}
The pseudo-labeled data are further refined using the following criteria:

\begin{itemize}
    \item\textbf{Diarization:} Pyannote's Speaker Diarization v3.1  is employed to filter audio files by separating multi-speaker audios, ensuring that each instance contains only one speaker \cite{Plaquet23}, which is essential for effective TTS model training.

    \item \textbf{Audio Duration}: Audio segments shorter than 0.5 seconds are discarded, as they provide insufficient information for our model. Similarly, segments longer than 11 seconds are excluded to match the model’s sequence length.
    
    \item \textbf{Text Length}: Segments with transcriptions exceeding 200 characters are removed to ensure manageable input size during training.
    \item \textbf{Silence-based Filtering}: Audio files where over 35\% of the duration consists of silence are discarded, as they negatively impact model performance.
    \item \textbf{Text-to-Audio Ratio}: Based on our analysis, audio segments where the text-to-audio duration ratio falls outside (Figure \ref{fig:unprocessed_data}) the range of 6 to 25 are excluded (Figure \ref{fig:processed_data}), ensuring alignment with natural speech patterns observed in Pseudo-labeled data from Phase A (Figure \ref{fig:reviewed_data}).
\end{itemize}



\begin{figure}[hbt!]
    \centering
    % First Image: Unprocessed Data
    \begin{subfigure}[b]{0.45\textwidth}
        \centering
        \includegraphics[width=\textwidth]{resources/reviewd_data_100.png}
        \caption{The diagram illustrates the linear relationship between audio duration and character length in manually-reviewed Pseudo-labeled Data - Phase A.}
        \label{fig:reviewed_data}
    \end{subfigure}
    \hfill
    % Second Image: Processed Data
    \begin{subfigure}[b]{0.45\textwidth}
        \centering
        \includegraphics[width=\textwidth]{resources/700h_unprocessed.png}
        \caption{The diagram depicts the relationship between audio duration and character length in Pseudo-Labeled Data - Phase B.}
        \label{fig:unprocessed_data}
    \end{subfigure}
    \vskip\baselineskip
    % Third Image: Reviewd Data
    \begin{subfigure}[b]{0.45\textwidth}
        \centering
        \includegraphics[width=\textwidth]{resources/700h_processed.png}
        \caption{The diagram illustrates the audio duration vs. character length graph in Pseudo-Labeled Data - Phase B after filtering.}
        \label{fig:processed_data}
    \end{subfigure}
    \caption{These figures demonstrate how the ratio of text length to audio duration changes before and after processing the data.}
    \label{fig:audio_vs_length_grid}
\end{figure}


% \textcolor{red}{\subsection{Quality Control for Pseudo-Labeled data} These are already described in A1. Plese include phase A data review process by reviewer team} 
% Given the importance of data quality in the pseudo-labeling process, multiple manual and 
% automated verification steps are implemented to ensure accurate alignment between audio and 
% transcriptions. To address potential errors in automated methods, manual verification is conducted, focusing on key areas such as: 

% \noindent \textbf{Semantic Accuracy \& Punctuation Verification} This involves ensuring that the punctuation restoration process has preserved the intended meaning of the transcriptions and correcting any misinterpretations in punctuation placement.


% \noindent \textbf{Segmentation Accuracy}
% The process includes reviewing whether the sentence segmentation has been executed correctly, ensuring that transcriptions are chunked without breaking context. Additionally, manual adjustments are made to the segmentation whenever mis-segmentation errors are detected.


% \noindent \textbf{Speaker Diarization Validation} 
% This stage involves ensuring that each audio segment contains speech from only one speaker and does not contain overlapping speech. Additionally, any incorrectly diarized segments are identified and filtered out to maintain the clarity and accuracy of the speaker attribution in the dataset.

% \noindent \textbf{ASR Inference Correction} 
% The process includes checking if the ASR (Automatic Speech Recognition) model has misinterpreted words or phrases. Corrections are manually applied to the transcriptions for any inaccuracies identified, ensuring the accuracy of the transcribed text.


% \noindent \textbf{Audio Duration vs. Text Length Filtering}
% This involves applying a duration-to-text ratio filter to remove audio segments that are either too short or too long compared to their corresponding transcriptions. This step is critical to maintaining accurate audio-to-text alignment throughout the dataset.

\section{Human Guided Data Preparation}
\label{app:human_reviewed_data}
We curated approximately 82.39 hours of speech data through human-level observation, which we refer to as Pseudo-Labeled Data - Phase A (Table \ref{tab:dataset_info}). The audio samples, averaging 10 minutes in duration, are sourced from copyright-free audiobooks and podcasts, preferably featuring a single speaker in most cases.

Annotators were tasked with identifying prosodic sentences by segmenting the audio into meaningful chunks while simultaneously correcting ASR-generated transcriptions and restoring proper punctuation in the provided text. If a selected audio chunk contained multiple speakers, it was discarded to maintain dataset consistency. Additionally, background noise, mispronunciations, and unnatural speech patterns were carefully reviewed and eliminated to ensure the highest quality TTS training data.

\begin{table*}
\label{datasetdescription}
\centering
\caption{Overview of the evaluation dataset, where M denotes malware and B denotes benign applications.}
\begin{tabular}{c|c|c|c|c|c|c|c|c} 
\hline
           & \textbf{Time Interval} & \textbf{Sample Size} & \begin{tabular}[c]{@{}c@{}}\textbf{The number of}\\\textbf{Existing family}\end{tabular} & \begin{tabular}[c]{@{}c@{}}\textbf{The number of}\\\textbf{New family}\end{tabular} & \textbf{Packed} & \textbf{Malicious} & \textbf{Benign} & \textbf{M/(M+B)\%}  \\ 
\hline
Test set 1 & 2020.05 - 2021.01 & 3015  & 21                                                               & 24                                                          & 18     & 284       & 2731   & 9.42                       \\ 
\hline
Test set 2 & 2021.01 - 2021.12 & 3015      & 28                                                               & 32                                                          & 30     & 298       & 2717   & 9.88                      \\ 
\hline
Test set 3 & 2021.12 - 2023.12 & 3016      & 34                                                               & 36                                                          & 40     & 302       & 2714   & 10.01                       \\
\hline                  
\end{tabular}
\end{table*}
% 

\section{Evaluation Dataset}

For evaluating the performance of our TTS system, we curated two datasets: BnStudioEval and BnTTSTextEval, each serving distinct evaluation purposes.

\begin{itemize}
    \item \textbf{BnStudioEval}: This dataset comprises 100 high-quality instances (text and audio pair) taken from our in-house studio recordings. This dataset was selected to assess the model’s capability in replicating high-fidelity speech output with speaker impersonation. 
    
    \item \textbf{BnTTSTextEval}: The BnTTSTextEval dataset encompasses three subsets: \begin{itemize}
        \item \textbf{BengaliStimuli53}: A linguist-curated set of 53 instances, created to cover a comprehensive range of Bengali phonetic elements. This subset ensures that the model handles diverse phonemes.
        % Sample example: তোমাদের পড়াশুনা ধ্যানি ও গুরুমুখী নয়, বরং বাজারভিত্তিক।পূর্বসূরী অনেকের মত তোমরাও অনেকাংশে নিজের চিত্ত ও বোধকে গবেষণায় পূর্ণ করার চেয়ে ঘুষ খেয়ে ঘরে ফার্নিচারের মেলা বসাতে চাও, আর ঘৃত দিয়ে কিচেন।
        
        \item \textbf{BengaliNamedEntity1000}: A set of 1,000 instances focusing on proper nouns such as person, place, and organization names. This subset tests the model's handling of named entities, which is crucial for real-world conversational accuracy.
        % Sample Example: ময়মনসিংহ বিভাগের জেলাগুলো হচ্ছে শেরপুর, ময়মনসিংহ, জামালপুর, নেত্রকোণা।
        \item \textbf{ShortText200}: Composed of 200 instances, this subset includes short sentences  filler words, and common conversational phrases (less than three words) to evaluate the model’s performance in natural, day-to-day dialogue scenarios.
        % Sample Example: কি বলছেন?
    \end{itemize}  
\end{itemize}

The BnStudioEval dataset, with reference audio for each text, will be for reference-aware evaluation, while BnTTSTextEval supports reference-independent evaluation. Together, these datasets provide a comprehensive basis for evaluating various aspects of our TTS performance, including phonetic diversity, named entity pronunciation, and conversational fluency. 



% \section{Model Architecture}
% The model architecture consists of the following trainable components:

% % \textbf{VQ-VAE:}
% % The DiscreteVAE architecture consists of an encoder, decoder, and a quantization codebook. The encoder uses 2 Conv1d layers with strides of 2, reducing input dimensionality, followed by 3 residual blocks with 1024 channels, each with ReLU activations. The decoder mirrors the encoder, starting with a Conv1d layer, followed by 3 residual blocks and 2 upsampled convolution layers, reconstructing the original input. A Quantize layer is used for vector quantization. The architecture utilizes the DiscretizationLoss function for learning discrete latent representations. This module efficiently encodes and decodes spectrograms with a total parameter count of around 51 million.
% \textbf{Conditioning Encoder and Perceiver Resampler:}
% The Conditioning Encoder \cite{casanova2024xtts} consists of an initial Conv1d layer with 80 input channels and 1024 output channels, followed by 6 Attention blocks. Each Attention block includes a Group normalization layer (32 groups, 1024 dimensions), a Conv1d layer for query-key-value computation (1024 input channels, 3072 output channels), and a final projection Conv1d layer (1024 output channels). The attention mechanism utilizes QKV attention. Dropout with a probability of 0.1 is applied to facilitate regularization. The encoder outputs a sequence, which length is dependent on the input audio duration.

% The Conditioning Encoder is followed by the Perceiver Resampler, which produces a fixed number of embeddings by utilizing cross attention mechanism. The Perceiver Resampler is composed two attention blocks, each with 512-dimensional queries and 1024-dimensional keys and values. The module includes sequential layers with linear projections and GELU activations. For normalization, it uses RMS norm.
% The total number of parameters in Conditioning Encoder and Perceiver Resampler are approximately 25.29 million and 21 millions respectively.


% \textbf{LLM:}
% For LLM, we use a GPT-2 \cite{radford2019language} model with approximately 377.89 million parameters. The GPT-2 is consists of 30 transformer blocks, each with 16 attention heads and a hidden dimension of 1024. It uses layer normalization and attention mechanisms, with the MLP blocks containing two linear layers and a GELU activation.


% \textbf{HiFi-GAN Decoder:}
% The HiFi-GAN Decoder \cite{kong2020hifi} consists of a waveform generator with multiple convolutional layers and residual blocks. It includes 4 parametrized ConvTranspose1d layers for upsampling, followed by a series of residual blocks with various dilation rates. The total number of parameters is 25.86 million. This submodule is responsible for converting intermediate GPT-2 latent representations into high-quality waveform outputs


% \section{Objective Functions}
% \subsection{Language Modeling Loss}

% \textbf{Text Token Prediction} loss, denoted as $\mathcal{L}_{\text{text}}$, measures the discrepancy between the predicted text logits and the target text labels. Let $\hat{y}_{\text{text}}$ represent the predicted logits and $y_{\text{text}}$ the ground truth target labels. The text prediction loss is calculated as:

% \begin{equation}
% \mathcal{L}_{\text{text}} = \frac{1}{N} \sum_{i=1}^{N} \text{CE}(\hat{y}_{\text{text}}^{(i)}, y_{\text{text}}^{(i)}),
% \end{equation}

% where $\text{CE}$ denotes the cross-entropy loss, and $N$ is the number of training samples.

% \paragraph{Audio Token Prediction Loss}

% The second loss is Audio Token Prediction loss, $\mathcal{L}_{\text{mel}}$, evaluates the model's performance in generating acoustic Token that match the target VQ-VAE codes. It is defined as:

% \begin{equation}
% \mathcal{L}_{\text{audio}} = \frac{1}{N} \sum_{i=1}^{N} \text{CE}(\hat{y}_{\text{audio}}^{(i)}, y_{\text{audio}}^{(i)}),
% \end{equation}

% where $\hat{y}_{\text{audio}}$ represents the predicted logits for the audio token, and $y_{\text{audio}}$ are the corresponding target VQ-VAE tokens.


% The total loss used to train the model is a weighted sum of the text and audio losses:

% \begin{equation}
% \mathcal{L}_{\text{total}} = \alpha \mathcal{L}_{\text{text}} + \beta \mathcal{L}_{\text{audio}}
% \end{equation}

% where $\alpha$ and $\beta$ are scaling factors that control the relative importance of each loss term. This combined objective ensures that the model learns both the correct phonetic representations and acoustic features.

% $\alpha$ and $\beta$ are set 0.01  and 1.0 respectively.

% \subsection{Vocoder Loss}
% We used a HiFi-GAN-based vocoder \cite{kong2020hifi} that comprises multiple discriminators: the Multi-Period Discriminator, and Multi-Scale Discriminator. For the sake of clarity, we will refer to these discriminators as a single entity. The HiFi-GAN module is trained using a least squares loss rather than the conventional binary cross-entropy loss. The discriminator is tasked with classifying real audio samples as 1 and generated samples as 0, while the generator is optimized to produce audio that can deceive the discriminator into classifying it as close to 1. The adversarial losses for the generator \(G\) and the discriminator \(D\) are defined as follows:

% \begin{align}
%     \mathcal{L}_{\text{Adv}}(D; G) &= \mathbb{E}_{(x, s)} \left[(D(x) - 1)^2 + D(G(s))^2 \right], \\
%     \mathcal{L}_{\text{Adv}}(G; D) &= \mathbb{E}_{s} \left[(D(G(s)) - 1)^2 \right],
% \end{align}

% where \(x\) represents the real audio samples, and \(s\) denotes the input mel-spectrogram conditions.

% \paragraph{Mel-Spectrogram Loss}
% The model also employs L1 loss between the mel-spectrograms of the real and generated audio. This loss is formulated as:

% \begin{align}
%     \mathcal{L}_{\text{Mel}}(G) = \mathbb{E}_{(x, s)} \left[\left\| \phi(x) - \phi(G(s)) \right\|_{1}\right],
% \end{align}

% where \(\phi\) represents the transformation function that maps a waveform to its corresponding mel-spectrogram.

% \paragraph{Feature Matching Loss}
% The feature matching loss calculates the L1 distance between the intermediate features of the real and generated audio, as extracted from multiple layers of the discriminator. It is defined as:

% \begin{align}
%     \mathcal{L}_{\text{FM}}(G; D) = \mathbb{E}_{(x, s)} \left[\sum_{i=1}^{T} \frac{1}{N_i} \left\| D^i(x) - D^i(G(s)) \right\|_{1}\right],
% \end{align}

% where \(T\) denotes the number of discriminator layers, and \(D^i\) and \(N_i\) represent the features and number of features at the \(i\)-th layer, respectively.

% \paragraph{Final Loss}
% Given that the discriminator is composed of multiple sub-discriminators, the final objectives for training the generator and the discriminator are defined as follows::

% \begin{align}
%     \mathcal{L}_{G} &= \sum_{k=1}^{K} \left[\mathcal{L}_{\text{Adv}}(G; D_k) + \lambda_{\text{FM}} \mathcal{L}_{\text{FM}}(G; D_k)\right] + \lambda_{\text{Mel}} \mathcal{L}_{\text{Mel}}(G), \\
%     \mathcal{L}_{D} &= \sum_{k=1}^{K} \mathcal{L}_{\text{Adv}}(D_k; G),
% \end{align}

% where \(D_k\) denotes the \(k\)-th sub-discriminator and \(\lambda_{\text{FM}} = 2\), \(\lambda_{\text{Mel}} = 45\). 


\section{Training Objectives}
\label{app:training_objective}

Our BnTTS model is composed of two primary modules (GPT-2 and HiFi-GAN), which are trained separately. The GPT-2 module is trained using a Language Modeling objective, while the HiFi-GAN module is optimized using HiFi-GAN loss objective. This section provides an overview of the loss functions applied during training.

\subsection{Language Modeling Loss}
1. \textbf{Text Generation Loss}: Denoted as $\mathcal{L}_{\text{text}}$, it quantifies the difference between predicted logits and ground truth labels using cross-entropy. Let $\hat{y}_{\text{text}}$ represent the predicted logits and $y_{\text{text}}$ the ground truth target labels. For a sequence with $N$ text tokens, the Text Generation Loss is calculated as: 
   \begin{equation}
   \mathcal{L}_{\text{text}} = \frac{1}{N} \sum_{i=1}^{N} \text{CE}(\hat{y}_{\text{text}}^{(i)}, y_{\text{text}}^{(i)})
   \end{equation}
   
2. \textbf{Audio Generation Loss}: Denoted as $\mathcal{L}_{\text{audio}}$, it evaluates the accuracy of generated acoustic tokens against target VQ-VAE codes using cross-entropy loss:
   \begin{equation}
   \mathcal{L}_{\text{audio}} = \frac{1}{N} \sum_{i=1}^{N} \text{CE}(\hat{y}_{\text{audio}}^{(i)}, y_{\text{audio}}^{(i)})
   \end{equation}

where $\hat{y}_{\text{audio}}$ represents the predicted logits for the audio token, $y_{\text{audio}}$ are the corresponding target VQ-VAE tokens, and $N$ is the number of audio token in the sequence.
   
Total loss combines the text generation and audio generation losses with weighted factors:
   \begin{equation}
   \mathcal{L}_{\text{total}} = \alpha \mathcal{L}_{\text{text}} + \beta \mathcal{L}_{\text{audio}} \quad (\alpha = 0.01, \beta = 1.0)
   \end{equation}

where $\alpha$ and $\beta$ are scaling factors that control the relative importance of each loss term.




\subsection{HiFi-GAN Loss}
We used a HiFi-GAN-based vocoder \cite{kong2020hifi} that comprises multiple discriminators: the Multi-Period Discriminator, and Multi-Scale Discriminator. For the sake of clarity, we will refer to these discriminators as a single entity. The HiFi-GAN module is trained using multiple losses mentioned below:

1. \textbf{Adversarial Loss}: The adversarial losses for the generator \(G\) and the discriminator \(D\) are defined as follows:
\begin{align}
    \mathcal{L}_{\text{Adv}}(D; G) &= \mathbb{E}_{(x, s)} \left[(D(x) - 1)^2 + D(G(s))^2 \right] \\
    \mathcal{L}_{\text{Adv}}(G; D) &= \mathbb{E}_{s} \left[(D(G(s)) - 1)^2 \right]
\end{align}

where \(x\) represents the real audio samples, and \(s\) denotes the input conditions.

2. \textbf{Mel-Spectrogram Loss}: This loss calculates L1 distance between the mel-spectrograms of the real and generated audio. This loss is formulated as:
\begin{align}
    \mathcal{L}_{\text{Mel}}(G) = \mathbb{E}_{(x, s)} \left[\left\| \phi(x) - \phi(G(s)) \right\|_{1}\right]
\end{align}
where \(\phi\) represents the transformation function that maps a waveform to its corresponding mel-spectrogram.

3. \textbf{Feature Matching Loss}: The feature matching loss calculates the L1 distance between the intermediate features of the real and generated audio, as extracted from multiple layers of the discriminator. It is defined as:
% \begin{align}
%     \mathcal{L}_{\text{FM}}(G; D) = \mathbb{E}_{(x, s)} \left[\sum_{i=1}^{T} \frac{1}{N_i} \left\| D^i(x) - D^i(G(s)) \right\|_{1}\right]
% \end{align}

\begin{align}
    \mathcal{L}_{\text{FM}}(G; D) = \mathbb{E}_{(x, s)} \sum_{i=1}^{T} \frac{1}{N_i} \left\| D^i(x) - D^i(G(s)) \right\|_{1}
\end{align}

where \(T\) denotes the number of discriminator layers, and \(D^i\) and \(N_i\) represent the features and number of features at the \(i\)-th layer, respectively.


\paragraph{Final Loss:}
Given that the discriminator is composed of multiple sub-discriminators, the final objectives for training the generator and the discriminator are defined as follows:
% \begin{align}
%     \mathcal{L}_{G} &= \sum_{k=1}^{K} \left[\mathcal{L}_{\text{Adv}}(G; D_k) + \lambda_{\text{FM}} \mathcal{L}_{\text{FM}}(G; D_k)\right] + \lambda_{\text{Mel}} \mathcal{L}_{\text{Mel}}(G), \\
%     \mathcal{L}_{D} &= \sum_{k=1}^{K} \mathcal{L}_{\text{Adv}}(D_k; G),
% \end{align}
\begin{align}
    \mathcal{L}_{G} &= \sum_{k=1}^{K} \left[\mathcal{L}_{\text{Adv}}(G; D_k) + \lambda_{\text{FM}} \mathcal{L}_{\text{FM}}(G; D_k)\right] \notag \\
    &\quad + \lambda_{\text{Mel}} \mathcal{L}_{\text{Mel}}(G) \\
    \mathcal{L}_{D} &= \sum_{k=1}^{K} \mathcal{L}_{\text{Adv}}(D_k; G)
\end{align}

where \(D_k\) denotes the \(k\)-th sub-discriminator and \(\lambda_{\text{FM}} = 2\), \(\lambda_{\text{Mel}} = 45\). 




% 1. \textbf{Adversarial Losses}: For generator $G$ and discriminator $D$, using least squares instead of binary cross-entropy:
%    \begin{align}
%    \mathcal{L}_{\text{Adv}}(D; G) &= \mathbb{E}_{(x, s)} \left[(D(x) - 1)^2 + D(G(s))^2 \right], \\
%    \mathcal{L}_{\text{Adv}}(G; D) &= \mathbb{E}_{s} \left[(D(G(s)) - 1)^2 \right]
%    \end{align}
% 2. \textbf{Mel-Spectrogram Loss}: Measures the L1 distance between real and generated audio mel-spectrograms:
%    \begin{equation}
%    \mathcal{L}_{\text{Mel}}(G) = \mathbb{E}_{(x, s)} \left[\| \phi(x) - \phi(G(s)) \|_{1}\right]
%    \end{equation}

% 3. \textbf{Feature Matching Loss}: Compares intermediate features from real and generated audio across discriminator layers:
%    \begin{equation}
%    \mathcal{L}_{\text{FM}}(G; D) = \mathbb{E}_{(x, s)} \left[\sum_{i=1}^{T} \frac{1}{N_i} \| D^i(x) - D^i(G(s)) \|_{1}\right]
%    \end{equation}

% 4. \textbf{Final Loss Objectives}:
%    Generator Loss:
%    \begin{equation}
%    \mathcal{L}_{G} = \mathcal{L}_{\text{Adv}}(G; D) + \lambda_{\text{FM}} \mathcal{L}_{\text{FM}}(G; D) + \lambda_{\text{Mel}} \mathcal{L}_{\text{Mel}}(G)
%    \end{equation}
%    Discriminator Loss:
%    \begin{equation}
%    \mathcal{L}_{D} = \mathcal{L}_{\text{Adv}}(D; G)
%    \end{equation}

% This framework ensures effective training of the model, balancing text and audio prediction tasks, and optimizing for high-quality audio generation.


\section{Evaluation Metrics}
\label{app:eval_metrics}
We employed a combination of subjective and objective metrics to rigorously evaluate the performance of our TTS system, focusing on intelligibility, naturalness, speaker similarity, and transcription accuracy.

\noindent \textbf{Subjective Mean Opinion Score (SMOS):} SMOS is a perceptual evaluation where listeners rate synthesized speech on a Likert scale from 1 (poor) to 5 (excellent). It considers naturalness, clarity, and fluency, providing an absolute score for each sample. A higher SMOS indicates better overall speech quality.

\noindent \textbf{SpeechBERTScore:} SpeechBERTScore adapts BERTScore for speech, using self-supervised learning (SSL) models to compare dense representations of generated and reference speech. For generated speech waveform $\hat{X}$ and reference waveform $X$, the feature representations $\hat{Z}$ and $Z$ are extracted using a pretrained model. SpeechBERTScore is defined as the average maximum cosine similarity between feature vectors:
\[
\text{SpeechBERTScore} = \frac{1}{N_{\text{gen}}} \sum_{i=1}^{N_{\text{gen}}} \max_{j} \text{cos}(\hat{\mathbf{z}}_i, \mathbf{z}_j)
\]
where $\hat{\mathbf{z}}_i$ and $\mathbf{z}_j$ represent the SSL embeddings for generated and reference speech, respectively.

\noindent \textbf{Character Error Rate (CER):} CER measures transcription accuracy by calculating the ratio of errors (substitutions $S$, deletions $D$, and insertions $I$) in automatic speech recognition (ASR) transcriptions:
\[
CER = \frac{S + D + I}{N}
\]
where $N$ is the total number of characters in the reference transcription. A lower CER indicates better transcription accuracy.

\noindent \textbf{Speaker Encoder Cosine Similarity (SECS):} SECS evaluates speaker similarity by calculating the cosine similarity between speaker embeddings of the reference and synthesized speech:

\[
\text{SECS} = \frac{e_{\text{ref}} \cdot e_{\text{syn}}}{\|e_{\text{ref}}\| \|e_{\text{syn}}\|},
\]

where $e_{\text{ref}}$ and $e_{\text{syn}}$ are the speaker embeddings for reference and synthesized speech, respectively. SECS ranges from -1 (low similarity) to 1 (high similarity).

\label{sec:DurationEquality}
\noindent \textbf{Duration Equality Score:} This metric quantifies how closely the durations of the reference ($a$) and synthesized ($b$) speech match, with a score of 1 indicating identical durations:

\[
\text{DurationEquality}(a, b) = \frac{1}{\max\left(\frac{a}{b}, \frac{b}{a}\right)}.
\]

This score helps in assessing duration similarity between reference and generated audio, ensuring consistency in pacing.

Each metric provides a different perspective, allowing a holistic evaluation of the synthesized speech quality.


% \section{Evaluation Metrics}
% To rigorously evaluate the performance of the TTS system, a combination of subjective and objective metrics is employed. These metrics assess various dimensions of speech quality, including intelligibility, naturalness, speaker similarity, and transcription accuracy. The following describes the key evaluation metrics in detail:


% \textbf{Subjective Mean Opinion Score (SMOS)}: SMOS is a perceptual evaluation metric used to assess the overall quality of synthesized speech by human listeners. It is rated on a Likert scale from 1 (bad) to 5 (excellent), considering aspects such as naturalness, clarity, fluency, consistency, and emotional expressiveness. SMOS serves as an absolute rating, where each synthetic sample is evaluated independently, without reference to other samples. In addition to SMOS, separate scores for Naturalness and Clarity are reported for a comprehensive analysis.


% \textbf{SpeechBERTScore}:
% To evaluate the semantic consistency between the generated and reference speech in our proposed system, we employ the SpeechBERTScore metric, which extends the BERTScore framework, commonly used in text generation, to the speech domain by computing the similarity between dense speech representations derived from self-supervised learning (SSL) models. The metric aims to capture semantic congruence between the synthesized speech and a reference, accounting for differences in waveform length.

% Let the generated and reference speech waveforms be denoted as $\hat{X} = (\hat{x}_t \in \mathbb{R} \mid t = 1, \ldots, T_{\text{gen}})$ and $X = (x_t \in \mathbb{R} \mid t = 1, \ldots, T_{\text{ref}})$, respectively, where $T_{\text{gen}}$ and $T_{\text{ref}}$ represent the lengths of the generated and reference waveforms. To extract meaningful features from these waveforms, a pretrained SSL model is employed, which generates sequence representations $\hat{Z} = (\hat{\mathbf{z}}_n \in \mathbb{R}^D \mid n = 1, \ldots, N_{\text{gen}})$ and $Z = (\mathbf{z}_n \in \mathbb{R}^D \mid n = 1, \ldots, N_{\text{ref}})$ for the generated and reference speech, respectively:

% \[
% \hat{Z} = \text{Encoder}(\hat{X}; \theta), \quad Z = \text{Encoder}(X; \theta),
% \]

% where $\theta$ denotes the parameters of the pretrained encoder model, and $N_{\text{gen}}$ and $N_{\text{ref}}$ are determined by $T_{\text{gen}}$ and $T_{\text{ref}}$ based on the encoder's subsampling rate.

% The SpeechBERTScore is defined as the precision metric in the BERTScore framework, measuring the maximum cosine similarity between each feature vector in the generated speech and all feature vectors in the reference speech:

% \[
% \text{SpeechBERTScore} = \frac{1}{N_{\text{gen}}} \sum_{i=1}^{N_{\text{gen}}} \max_{j} \text{cos}(\hat{\mathbf{z}}_i, \mathbf{z}_j),
% \]

% where $\text{cos}(\hat{\mathbf{z}}_i, \mathbf{z}_j)$ is the cosine similarity between the SSL feature vectors $\hat{\mathbf{z}}_i$ from the generated speech and $\mathbf{z}_j$ from the reference speech.

% By leveraging pretrained SSL models, SpeechBERTScore captures high-level semantic information, making it suitable for evaluating synthesized speech's content and meaning. This metric is particularly advantageous for TTS evaluation, where semantic consistency and intelligibility are crucial, even in scenarios where the generated and reference audio lengths may differ.

% \textbf{Character Error Rate (CER)}: CER quantifies transcription accuracy by comparing the output of an automatic speech recognition (ASR) system on synthesized speech against a reference transcription. It is defined as:
% \[
% \text{CER} = \frac{S + D + I}{N}
% \]
% where $S$ is the number of substitutions, $D$ is the number of deletions, $I$ is the number of insertions, and $N$ is the total number of characters in the reference transcription. Lower CER values indicate higher transcription accuracy.

% \textbf{Speaker Encoder Cosine Similarity (SECS)}: SECS measures the speaker similarity between synthesized and reference speech by calculating the cosine similarity between their speaker embeddings:
% \[
% \text{SECS} = \frac{e_{\text{ref}} \cdot e_{\text{syn}}}{\|e_{\text{ref}}\| \|e_{\text{syn}}\|}
% \]
% where $e_{\text{ref}}$ and $e_{\text{syn}}$ are the speaker embeddings of the reference and synthesized speech, respectively. The similarity score ranges from -1 (low similarity) to 1 (high similarity), with higher values indicating closer resemblance in speaker characteristics.

% \textbf{DurationEquality Score}: DurationEquality quantifies the equality between two audio sample durations, \(a\) and \(b\), producing values between 0 and 1, where a score of 1 indicates identical durations. The metric is defined as:
% \begin{equation}
%     \text{DurationEquality}(a, b) = \frac{1}{\max\left(\frac{a}{b}, \frac{b}{a}\right)}
% \end{equation}
% This score approaches 1 as the durations of \(a\) and \(b\) become more equal, providing an effective measure of  discrepancy between duration of reference audio and synthesized audio .


\section{Subjective Evaluation}
For subjective evaluation of our system, we employ the Mean Opinion Score (MOS), a widely recognized metric primarily focusing on assessing the perceptual quality of audio outputs. To ensure the reliability and accuracy of our evaluations, we carefully select a panel of ten experts who are thoroughly trained in the intricacies of MOS scoring. These experts are equipped with the necessary skills and knowledge to critically assess and score the system, providing invaluable insights that help guide the refinement and enhancement of our technology. This structured approach guarantees that our evaluations are both comprehensive and precise, reflecting the true quality of the audio outputs under review.

\subsection{Evaluation Guideline}
For calculating MOS, we consider five essential evaluation criteria:
\begin{itemize} \item \textbf{Naturalness:} Evaluates how closely the TTS output resembles natural human speech. \item \textbf{Clarity:} Assesses the intelligibility and clear articulation of the spoken words. \item \textbf{Fluency:} Examines the smoothness of speech, including appropriate pacing, pausing, and intonation. \item \textbf{Consistency:} Checks the uniformity of voice quality across different texts. \item \textbf{Emotional Expressiveness:} Measures the ability of the TTS system to convey the intended emotion or tone. \end{itemize}

In the evaluation, we employ a five-point rating scale to meticulously assess performance based on specific criteria. This scale ranges from 1, denoting 'Bad' where the output has significant distortions, to 5, representing 'Excellent' where the output nearly replicates natural human speech and excels in all evaluation aspects. To capture more subtle nuances in the TTS output that might not perfectly fit into these whole-number categories, we also recommend using fractional scores. For example, a 1.5 indicates quality between 'Bad' and 'Poor,' a 2.5 signifies improvement over 'Poor' but not quite reaching 'Fair,' a 3.5 suggests better than 'Fair' but not up to 'Good,' and a 4.5 reflects performance that surpasses 'Good' but falls short of 'Excellent.' This fractional scoring allows for a more precise and detailed reflection of the system's quality, enhancing the accuracy and depth of the MOS evaluation.

\subsection{Evaluation Process}
We have developed an evaluation platform specifically designed for the subjective assessment of Text-to-Speech (TTS) systems. This platform features several key attributes that enhance the effectiveness and reliability of the evaluation process. Key features include anonymity of audio sources, ensuring that evaluators are unaware of whether the audio is synthetically generated or recorded from studio environment, or which TTS model, if any, was used. This promotes unbiased assessments based purely on audio quality. Comprehensive evaluation criteria allow evaluators to rate each audio sample on naturalness, clarity, fluency, consistency, and emotional expressiveness, ensuring a holistic review of speech synthesis quality. The user-centric interface is streamlined for ease of use, enabling efficient playback of audio samples and score entry, which reduces evaluator fatigue and maintains focus on the task. Finally, the structured data collection method systematically captures all ratings, facilitating precise analysis and enabling targeted improvements to TTS technologies. This platform is a vital tool for developers and researchers aiming to refine the effectiveness and naturalness of speech outputs in TTS systems.

\subsection{Evaluator Statistics}
For our evaluation process, we carefully selected 10 expert native speakers, achieving a balanced representation with 5 males and 5 females. The age range for these evaluators is between 20 to 28 years, ensuring a youthful perspective that aligns well with our target demographic. All evaluators are either currently enrolled as graduate students or have already completed their graduate studies. They hail from a variety of academic backgrounds, including economics, engineering, computer science, and social sciences, which provides a diverse range of insights and expertise. This careful selection of qualified individuals ensures a comprehensive and informed assessment process, suitable for our needs in evaluating advanced systems or processes where diverse, educated opinions are crucial.

\subsection{Subjective Evaluation Data Preparation} 
For reference-aware evaluation, we selected 20 audio samples from each of the four speakers, resulting in 80 Ground Truth (GT) audios. To facilitate comparison, we generated 400 synthetic samples (80 × 5) using the TTS systems examined in this study. Including the GT samples, the total dataset for this evaluation amounts to 480 audio files (400 + 80).

For the reference-independent evaluation, we utilized 453 text samples from BnTTSTextEval, comprising BengaliStimuli53 (53), BengaliNamedEntity1000 (200), and ShortText200 (200). Given the four speakers in both BnTTS-0 and BnTTS-n, this resulted in 3,624 audio samples (4 × 453 × 2). Additionally, IndicTTS, GTTS, and AzureTTS contributed 1,359 samples (3 × 453). IndicTTS samples were evenly distributed between two male and female speakers, while GTTS and AzureTTS used the "bn-IN-Wavenet-C" and "bn-IN-TanishaaNeural" voices, respectively.

In total, the reference-independent evaluation dataset comprised 5,436 audio samples. When combined with the 480 samples from the reference-aware evaluation, the overall dataset for subjective evaluation amounted to 5,916 audio files. These samples were randomly mixed and distributed to the reviewer team to ensure unbiased evaluations.

\section{Use of AI assistant}
\label{sec:use_of_ai_assistant}
We used AI assistants such as GPT-4o for spelling and grammar checking for the text of the paper.

\newpage

% \section{Potential Risks}
% \label{sec:use_of_potential risks}
% There are no potential risks associated with the outcomes of this research, as we do not utilize any sensitive information. Instead, this work will benefit the community by aiding the development of TTS systems for low-resource languages.




% Speech_Generation had been added to main paper 
% \section{Speech Generation}
\subsection{Synthesizing Short Sequences}

The generation of short audio sequences presents challenges in the BnTTS model, particularly for texts containing fewer than 30 characters when using the default generation settings (Temperature \(T = 0.85\) and TopK = 50). The primary issues observed are twofold: (1) the generated speech often lacks intelligibility, and (2) the output speech tends to be longer than expected.

To investigate these challenges, we curated a subset of 23 short text-speech pairs from the BnStudioEval dataset. For evaluation, we utilize the Character Error Rate (CER) metric to assess intelligibility, and we introduce the Audio Duration Equality metric to evaluate the alignment between the generated and reference audio durations. The Audio Duration Equality Score quantifies the equality between two audio sample durations, \(a\) and \(b\), producing values between 0 and 1, where a score of 1 indicates identical durations. The metric is defined as:

\begin{equation}
    \text{DurationEquality}(a, b) = \frac{1}{\max\left(\frac{a}{b}, \frac{b}{a}\right)}
\end{equation}



This score approaches 1 as the durations of \(a\) and \(b\) become more equal, providing an effective measure of  discrepancy between duration of reference audio and synthesized audio .


\paragraph{Effect of Short Prompt}
Under the default settings (Exp. 1 in Table X), the model achieves a Character Error Rate (CER) of 0.081 and a Duration Equality Score of 0.699. We hypothesize that the model's inability to accurately synthesize short speech stems from its training process. During training, the model reserves between 1 to 6 seconds of audio for speaker prompting. For audio shorter than 1 second, the model uses half of the audio as the prompt. This implies that the model is accustomed to short audio prompts for short sequences. By aligning the inference process with this training strategy and using short prompts, the generation performance improves markedly, as evidenced by a higher Duration Equality Score of 0.820 and a lower CER of 0.029 in Exp. 2.

\paragraph{Effect of Temperature and Top-K Sampling}
The default temperature (\(T = 0.85\)) and top-K value (50) were found to be sub-optimal for generating short sequences. By adjusting the temperature to \(T = 1.0\) and reducing the top-K value to 2, we observed an improvement in the Duration Equality Score from 0.699 to 0.701, accompanied by a substantial reduction in CER, from 0.081 to 0.023 (as shown in Exp. 3).

\paragraph{Effect of Both Short Prompts and Temperature, Top-K}
Combining short prompts with the adjusted temperature and top-K values yielded the best results. In this configuration, the Duration Equality Score improved to 0.827, with a CER of 0.015, demonstrating that both factors are crucial for accurate short sequence generation.

The ablation study demonstrates that employing short prompts in combination with fine-tuning temperature and top-K values is essential for optimizing short sequence generation in the BnTTS model.

\begin{table}[H]
\centering

    \begin{tabular}{c|l}
        \hline
        \textbf{Variable} & \textbf{Description} \\ \hline
        \( \mathbf{T} \) & Text sequence with \( N \) tokens \\ \hline
        \( N \) & Number of tokens in the text sequence \\ \hline
        \( \mathbf{S} \) & Speaker's mel-spectrogram with \( L \) frames \\ \hline
        \( \hat{\mathbf{Y}} \) & Generated speech that matches the speaker's characteristics \\ \hline
        \( \mathbf{Y} \) & Ground truth mel-spectrogram frames for the target speech \\ \hline
        \( \mathcal{F} \) & Model responsible for producing speech conditioned on both the text and the speaker's spectrogram \\ \hline
        \( \mathbf{z} \) & Discrete codes transformed from mel-spectrogram frames using VQ-VAE \\ \hline
        \( \mathcal{C} \) & Codebook of discrete codes from VQ-VAE \\ \hline
        \( l \) & Number of layers in the Conditioning Encoder \\ \hline
        \( k \) & Number of attention heads in Scaled Dot-Product Attention \\ \hline
        \( \mathbf{S_z} \) & Intermediate representation of speaker spectrogram in \( \mathbb{R}^{L \times d} \) \\ \hline
        \( d \) & Dimensionality of each token or embedding \\ \hline
        \( \mathbf{Q}, \mathbf{K}, \mathbf{V} \) & Projections of \( \mathbf{S_z} \) used in scaled dot-product attention \\ \hline
        \( P \) & Fixed number of sequences produced by the Perceiver Resampler \\ \hline
        \( \mathbf{R} \) & Fixed-size output from the Perceiver Resampler in \( \mathbb{R}^{P \times d} \) \\ \hline
        \( \mathbf{T_e} \) & Continuous embedding space of text tokens in \( \mathbb{R}^{N \times d} \) \\ \hline
        \( \mathbf{S_p} \) & Speaker embeddings \\ \hline
        \( \mathbf{Y_z} \) & Ground truth spectrogram embeddings \\ \hline
        \( \mathbf{X} \) & Combined input during training: concatenation of speaker, text, and spectrogram embeddings \\ \hline
        \( \oplus \) & Concatenation operation \\ \hline
        \( \mathbf{H} \) & Output from the LLM consisting of hidden states for text, speaker, and spectrogram embeddings \\ \hline
        \( \mathbf{H}_\text{Y} \) & Spectrogram embedding from LLM output used for HiFi-GAN \\ \hline
        \( \mathbf{S}' \) & Resized speaker embedding to match \( \mathbf{H}_\text{Y} \) \\ \hline
        \( \mathbf{W} \) & Final audio waveform produced by HiFi-GAN \\ \hline
        \( g_\text{HiFi} \) & HiFi-GAN function converting spectrogram embeddings to audio waveform \\ \hline
    \end{tabular}
    \label{tab:variables_descriptions}
    \caption{Table of Variables and Descriptions}
\end{table}

% \section{Results and Discussion}

% To evaluate the performance of our Bengali TTS system, we employed a combination of subjective and objective metrics across two datasets: BnStudioEval and BnTTSTextEval. The BnStudioEval dataset, consisting of high-quality recordings, was used for reference-aware evaluation, while BnTTSTextEval encompassed subsets focusing on phonetic diversity, named entity pronunciation, and conversational fluency for reference-independent evaluation. The metrics used include subjective measures such as SMOS and objective measures like CER, UTMOS, SECS, and SpeechBERT Precision.



% \subsection{Reference-aware Evaluation (BnStudioEval)}
% Table 1 presents the comparative performance of various TTS systems evaluated on the BnStudioEval dataset. Among the synthetic methods, AzureTTS exhibited the best performance with the lowest Character Error Rate (CER) and the highest UTMOS score, outperforming all other synthetic methods, including Ground Truths (GT) in terms of transcription accuracy. However, interestingly, the CER of the GT remains lower than that of BnTTS synthesized outputs. As expected, the GT, serving as a reference standard, outperforms all synthetic systems across key subjective metrics such as SMOS (4.671), Naturalness (4.625), and Clarity (4.9). In this context, the proposed BnTTS system closely follows, achieving competitive scores in SMOS (4.584), Naturalness (4.531), and Clarity (4.898).

% Regarding speaker similarity, the GT achieved SECS scores of 1.0 when compared to reference audios and 0.361 when compared to the speaker prompt. BnTTS also performed well, with an SECS(Ref.) score of 0.513 and an SECS(Prompt) score of 0.335, falling short of the ground truth by only 0.031 in the latter metric. Additionally, BnTTS received a SpeechBERT Precision score of 0.796, compared to the perfect score of 1.0 set by the ground truth.

% It should be noted that IndicTTS, GTTS, and AzureTTS lack speaker impersonation capabilities, rendering them inapplicable for reference-aware metrics such as SECS and SpeechBERT Precision. Consequently, these metrics were not calculated for these systems.


% \subsection{Reference-independent Evaluation (BnTTSTextEval)}
% Table 2 presents the comparative performance of various TTS systems evaluated on the BnTTSTextEval dataset, encompassing three distinct subsets: BengaliStimuli53, BengaliNamedEntity1000, and ShortText200. The trend observed in the BnStudioEval dataset persists here as well. AzureTTS and GTTS consistently trade leading positions in transcription accuracy (CER) and automated quality prediction (UTMOS), with BnTTS following closely in third place, and IndicTTS trailing behind.

% BnTTS performs strongly in subjective evaluations, excelling in SMOS, Naturalness, and Intelligibility across the BengaliStimuli53 and BengaliNamedEntity1000 subsets. However, it falls slightly behind AzureTTS in the ShortText200 subset, which focuses on model performance in short texts. Despite this, BnTTS overall, remains the top-performing system in all subjective metrics, delivering the highest average scores in SMOS(4.383), Naturalness(4.313), and Clarity(4.737).


\section{Symbols and Notations}
\label{sec:notation}

\begin{table}[H]
\centering
\setlength{\tabcolsep}{2.9pt}
\resizebox{0.45\textwidth}{!}{%

    \begin{tabular}{c|l}
        \hline
        \textbf{Variable} & \textbf{Description} \\ \hline
        \( \mathbf{T} \) & Text sequence with \( N \) tokens \\ \hline
        \( N \) & Number of tokens in the text sequence \\ \hline
        \( \mathbf{S} \) & Speaker's mel-spectrogram with \( L \) frames \\ \hline
        \( \hat{\mathbf{Y}} \) & Generated speech  \\ \hline
        \( \mathbf{Y} \) & Ground truth mel-spectrogram  \\ \hline
        \( \mathcal{F} \) & LLM Model \\ \hline
        \( \mathbf{z} \) & Discrete codes \\ \hline
        \( \mathcal{C} \) & Codebook of discrete codes \\ \hline
        \( l \) & Number of layers \\ \hline
        \( k \) & Number of attention heads i \\ \hline
        \( \mathbf{S_z} \) & speaker spectrogram embd.\( \mathbb{R}^{L \times d} \) \\ \hline
        \( d \) & Embedding \\ \hline
        \( \mathbf{Q}, \mathbf{K}, \mathbf{V} \) & Query, Key, Value \\ \hline
        \( P \) & Perceiver Resampler \\ \hline
        \( \mathbf{R} \) & Fixed-size output in \( \mathbb{R}^{P \times d} \) \\ \hline
        \( \mathbf{T_e} \) & Continuous embedding  \( \mathbb{R}^{N \times d} \) \\ \hline
        \( \mathbf{S_p} \) & Speaker embeddings \\ \hline
        \( \mathbf{Y_z} \) & Ground truth  \\ \hline
        \( \mathbf{X} \) & Input of LLM   \\ \hline
        \( \oplus \) & Concatenation operation \\ \hline
        \( \mathbf{H} \) & Output from the LLM  \\ \hline
        \( \mathbf{H}_\text{Y} \) & Spectrogram embedding  \\ \hline
        \( \mathbf{S}' \) & Resized embedding \( \mathbf{H}_\text{Y} \) \\ \hline
        \( \mathbf{W} \) & Final audio waveform \\ \hline
        \( g_\text{HiFi} \) & HiFi-GAN function \\ \hline
    \end{tabular}}
    \label{tab:variables_descriptions}
    \caption{Table of Variables and Descriptions}
\end{table}


\end{document}
