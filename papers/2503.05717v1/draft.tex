\documentclass[12pt, a4paper]{article}
\usepackage[left=1in,right=1in,top=1.0in,bottom=1in]{geometry}

\usepackage{amsmath,amssymb,amsthm,amsfonts,afterpage,hyperref}
\usepackage{amscd,subeqnarray}


\usepackage{graphicx}
\usepackage{caption}
\usepackage{subcaption}
\usepackage{authblk}
\usepackage{tikz}

\usepackage{array}
\usepackage{wrapfig}
\usepackage{color}
\usepackage{ragged2e}
%\usepackage{subfig}
\usepackage{stmaryrd}
\usepackage{siunitx}
\usepackage{multirow}
\usepackage{xcolor,cancel}
\usepackage{boldfonts}
\usepackage{symbols}
\usepackage{footmisc}
\usepackage{float}
\usepackage{setspace}
\usepackage{bm}
\usepackage{booktabs}
\usepackage{tikz, xcolor}
\usepackage{soul}
\usepackage{pdfpages} 

\usepackage[T1]{fontenc}
\usepackage[utf8]{inputenc}

\usepackage{cite}



\usepackage[linesnumbered,ruled,vlined]{algorithm2e}
\SetKwInput{KwInput}{Input}                
\SetKwInput{KwOutput}{Output} 

\usepackage[numbers,sort&compress]{natbib}
\setcitestyle{square}

\usetikzlibrary{shapes,arrows, positioning,calc}	
\usetikzlibrary{arrows,snakes,backgrounds}


%%%%%%%%%%%%%%%%%%%%%%%%%%%%%%%%%%%%%%%%%%%%%%%%%%
\newcommand{\bfa}[1]{\boldsymbol{#1}} 			
\newcommand{\sfa}[1]{\mathcal{#1}}			
\newcommand{\bfeps}{\boldsymbol{\epsilon}}		
\newcommand{\bfeta}{\boldsymbol{\eta}}		
\newcommand{\bfnu}{\boldsymbol{\nu}}			
\newcommand{\bfsigma}{\boldsymbol{\sigma}}	
\newcommand{\bfmu}{\boldsymbol{\mu}}		
\newcommand{\bfzeta}{\boldsymbol{\zeta}}		
\newcommand{\bftau}{\boldsymbol{\tau}}		
\newcommand{\bfgam}{\boldsymbol{\gamma}}	%
\newcommand{\bflam}{\boldsymbol{\lambda}}	%
\newcommand{\bfxi}{\boldsymbol{\xi}}			%
\newcommand{\tfa}[1]{\tilde{#1}}  				%
\newcommand{\wtfa}[1]{\widetilde{#1}}  			% 
\newcommand{\hfa}[1]{\hat{#1}}    				%
\newcommand{\whfa}[1]{\widehat{#1}}    			%
\newcommand{\bba}[1]{\mathbb{#1}} 			%
\newcommand{\mca}[1]{\mathcal{#1}}			%
\newcommand{\Span}{\text{Span}}     				%
\newcommand{\Sym}{\text{Sym}}   			%
\newcommand{\Div}{\text{Div}}     				%
\newcommand{\ddiv}{\text{div}}     				%
\newcommand{\Curl}{\text{Curl}}   				%
\newcommand{\curl}{\text{curl}}   				%
\newcommand{\Grad}{\text{Grad}}   			%
\newcommand{\grad}{\text{grad}}   				%
\newcommand{\tr}{\text{tr}}       				%
\newcommand{\sgn}{\text{sgn}}     				%
\newcommand{\diag}{\text{diag}}   				%
\newcommand{\del}{\triangle}     				%
\newcommand{\dens}{\rho_{i}(\bfa{x}_{0},t,\tau)}	%
\newcommand{\inprod}[2]{\langle #1,#2\rangle} 

\DeclareMathAlphabet{\mathpzc}{OT1}{pzc}{m}{it}

\DeclareMathOperator*{\OperatorCommand}{OperatorName}
\DeclareMathOperator*{\essinf}{ess\,inf}
\DeclareMathOperator*{\esssup}{ess\,sup}
\DeclareMathOperator*{\argmin}{arg\, min}
%---------------------------------------------------------
\newcommand{\bfn}{\boldsymbol{n}}	
\newcommand{\bfh}{\boldsymbol{h}}	
\newcommand{\bfu}{\boldsymbol{u}}	
\newcommand{\bfb}{\boldsymbol{b}}	
\newcommand{\bfv}{\boldsymbol{v}}	
\newcommand{\bfw}{\boldsymbol{w}}
\newcommand{\bfF}{\boldsymbol{F}}	
\newcommand{\bfE}{\boldsymbol{E}}	
\newcommand{\bfC}{\boldsymbol{C}}
\newcommand{\bfB}{\boldsymbol{B}}	
\newcommand{\bfx}{\boldsymbol{x}}	
\newcommand{\bfX}{\boldsymbol{X}}	
\newcommand{\bfS}{\boldsymbol{S}}	
\newcommand{\bfT}{\boldsymbol{T}}		
\newcommand{\bfI}{\boldsymbol{I}}	 % for second order Identity tensor
\newcommand{\bfzero}{\boldsymbol{0}}	

\newcommand{\bfD}{\boldsymbol{D}}
\newcommand{\bfM}{\boldsymbol{M}}
\newcommand{\bfN}{\boldsymbol{N}}

\newcommand{\bfA}{\boldsymbol{A}}
\newcommand{\bfP}{\boldsymbol{P}}

\newcommand{\bff}{\boldsymbol{f}}	
\newcommand{\bfg}{\boldsymbol{g}}	
%-------------------------------------------------------
%%%%%%%%%%%%%%%%%%%%%%%%%%%%%%%%%%%%%%%%%%%%%%%%%%


\newtheorem{definition}{Definition}
\newtheorem{theorem}{Theorem}[section]
\newtheorem{corollary}{Corollary}[theorem]
\newtheorem{lemma}[theorem]{Lemma}
\newtheorem{remark}{Remark}

\newtheorem{Algorithm}{Algorithm}[section]
\newtheorem{Theorem}{Theorem}[section]
\newtheorem{formulation}{Formulation}

\newtheorem*{cf}{Strong formulation}
\newtheorem*{cwf}{Continuous weak formulation}
\newtheorem*{dwf}{Discrete weak formulation}
\newtheorem*{tdiscretization}{Time discretization}
\newtheorem*{sdiscretization}{Weak Formulation}
\newtheorem*{dfep}{Discrete Finite Element Problem}

\newtheorem*{pia}{Picard's Iterative Algorithm}

\newcommand{\red}[1]{{\textcolor{red}{#1}}}
\newcommand{\blue}[1]{{\textcolor{blue}{#1}}}
\newcommand{\rv}[1]{{\textcolor{blue}{#1}}}



\providecommand{\keywords}[1]
{
  \small	
  \textbf{\textit{Keywords---}} #1
}


\title{{A computational model for crack-tip fields in a 3-D porous elastic solid with material moduli dependent on density}}

\author[1]{Kun Gou\thanks{kun.gou@tamusa.edu}}
\author[2]{S. M. Mallikarjunaiah\thanks{m.muddamallappa@tamucc.edu}\thanks{corresponding author}}
\affil[1]{Department of Computational, Engineering, and Mathematical Sciences,
Texas A\&M University-San Antonio,
San Antonio, TX 78224, USA}
\affil[2]{Department of Mathematics \& Statistics,
Texas A\&M University-Corpus Christi,
Corpus Christi, TX 78412, USA}
\date{}


\begin{document}

\maketitle
	    
\begin{abstract}
A mathematical model for crack-tip fields is proposed in this paper for the response of a three-dimensional (3-D) porous elastic solid whose material moduli are dependent on the density. Such a description wherein the generalized Lam\`e coefficients are nonlinear functions of material stiffness is more realistic because most engineering materials are porous, and their material properties depend on porosity and density. The governing boundary value problem for the static equilibrium state in a 3-D, homogeneous, isotropic material is obtained as a second-order, quasilinear partial-differential-equation system with a classical traction-free crack-surface boundary condition. The numerical solution is obtained from a continuous trilinear Galerkin-type finite element discretization. A Picard-type linearization is utilized to handle the nonlinearities in the discrete problem. The proposed model can describe the state of stress and strain in various materials, including recovering the classical singularities in the linearized model. The role of \textit{tensile stress},  \textit{stress intensity factor} (SIF), and \textit{strain energy density} are examined. The results indicate that the maximum values of all these quantities occur directly before the crack-tip, consistent with the observation made in the canonical problem for the linearized elastic fracture mechanics. One can use the same classical local fracture criterion, like the maximum of SIF,  to study crack tips' quasi-static and dynamic evolution within the framework described in this article. 
 \end{abstract}
 
 

\noindent \keywords{Implicit constitutive relation, Porous solid, Density-dependent moduli, Finite element method, Crack-tip fields, Tensile stress, Stress intensity factor, Strain energy density   }



\section{Introduction}\label{sec:Intro}

In the classical theory of elasticity, most engineering materials are modeled as compressible. Materials with \textit{Poisson's ratio} equal to $0.5$ do not endure volume changes under any applied deformation. Such materials have no density change; therefore, they must be approximated as incompressible with their constant density. Yet there are many compressible materials, such as polymers, in which density affects the material properties such as strength, hardness, and brittleness. Therefore, one needs to take material moduli dependence on the density to describe the elasticity of polymer networks and the response of such materials to mechanical loading \cite{rubinstein2002elasticity}.  A correction to the classical elasticity model is needed to model the bulk behavior of many porous materials. In the literature, the response of many polymeric materials undergoing infinitesimal deformation is modeled via the classical linearized elastic constitutive relationship in which Young's modulus, Poisson's ratio, and Lam\`e constants are considered constants. In a study on the strength of silica gels \cite{leventis2002nanoengineering}, a nonlinear response is reported between load and strain (see Figure~3 in \cite{leventis2002nanoengineering}) for a three-point bending of aerogels and Young's modulus is computed accordingly. Other studies also demonstrate that the material moduli depend on the density \cite{chandrasekaran2017carbon,anjos2010tensile}. In most solids, the porosity at the macro-level depends on the local density, and the porosity and density are inversely proportional to each other\cite{ALAGAPPAN2023100162}. 

In a series of articles \cite{rajagopal2003,rajagopal2007elasticity,rajagopal2014nonlinear,rajagopal2007response,rajagopal2009class},  Rajagopal generalized the framework of elastic bodies and showed that, through a rigorous linearization, one can obtain constitutive relations that are implicitly related between the linearized strain and Cauchy stress. Such structures are ideal for describing the response of the elastic bodies incapable of converting mechanical work into thermal energy. More importantly, these generalizations are shown to include both Cauchy and Green elastic bodies as a particular subclass of elastic material models \cite{bridges2015implicit,Mallikarjunaiah2015,MalliPhD2015,ortiz2014numerical,ortiz2012}.   The aforementioned nonlinear behavior can be modeled within the implicit theory of elasticity \cite{rajagopal2014nonlinear,muliana2018determining,kowalczyk2019finite}.  Rocks \cite{bustamante2020novel} and many rubber-like materials \cite{bustamante2021new} can be modeled using a particular subclass of models within the larger class of nonlinear relationships from \cite{rajagopal2003,rajagopal2007elasticity}. In \cite{bustamante2021new},  a constitutive model for rubber is proposed using the principal stress as the main variables, wherein a better corroboration with experimental data can be obtained than that for Ogden's model \cite{ogden1972large}. 


The nonlinear relations proposed in \cite{rajagopal2003,rajagopal2007elasticity} between the strain and stress cannot be obtained from Cauchy or Green elasticity by applying the same standard assumption that the displacement gradients are infinitesimal. From the experimental viewpoint, the classical linear elasticity can't adequately explain the nonlinear behavior between the stress and strain reported in \cite{saito2003multifunctional}. During the cold swagging process,  for the metal alloys Ti-12Ta-9Nb-3V-6Zr-O and 23Nb-0.7Ta-2Zr-O, the nonlinear behavior is observed well within a 2.5\% order of strain. A similar nonlinear behavior between stress and strain has been reported in other experimental studies on titanium alloys \cite{withey2008deformation}, gum metals \cite{zhang2009fatigue}, and rocks \cite{johnson1996manifestation}.  Several materials such as gum metal \cite{kulvait2019state}, titanium alloys (e.g., the one used in the orthopedic applications \cite{tian2015nonlinear,devendiran2017thermodynamically}), and other alloys (e.g., Ti-30Nb-10Ta-5Zr (TNTZ-30)) are known to exhibit an apparent nonlinear behavior within about $1\%$ strain \cite{hao2005super,saito2003multifunctional}. Such behavior is traditionally modeled using a linear relationship between the infinitesimal strain and Cauchy stress under the classical linearized elasticity construct, which is inadequate because a linear relationship can't fit the experimentally observed nonlinear relations.




The new theory of elasticity proposed by Rajagopal  \cite{rajagopal2003,rajagopal2007elasticity} has been used to revisit the classical problems in fracture mechanics, such as plane-strain and anti-plane static cracks in \cite{rajagopal2011modeling,gou2015modeling,HCY_SMM_MMS2022}, plane-strain cracks in a transversely isotropic elastic body \cite{Mallikarjunaiah2015}, quasi-static crack propagation in a new class of material models \cite{yoon2021quasi,lee2022finite}, static cracks in the implicit thermo-elastic body \cite{yoon2022finite}, and static cracks in 3-D materials body \cite{gou2023MMS,gou2023computational}. One of the significant advantages of the nonlinear constitutive theory in \cite{rajagopal2003,rajagopal2007elasticity} to study cracks and fractures is that the strain values are consistent with these from the linearization procedure used to derive the relationship, contrary to the classical square-root singularity predicated by the canonical boundary value problem (BVP) within the linear elasticity. However, one can still find the stress concentration within the framework proposed in \cite{rajagopal2003,rajagopal2007elasticity} suggesting using the same local fracture criterion as in the linear elastic fracture mechanics.  


A subclass of nonlinear models in \cite{rajagopal2003,rajagopal2007elasticity} for the response of elastic materials such as rocks and concrete wherein the material moduli dependent upon the density are obtained in \cite{rajagopal2021b,rajagopal2022implicit}. In such models, both linearized strain and Cauchy stress appear linearly \cite{itou2022investigation,itou2021implicit,itou2023generalization,pruuvsa2022pure,pruuvsa2023mechanical}. The material response of the models in \cite{rajagopal2021b,rajagopal2022implicit} is not linear as the constitutive relations contain the bilateral product of Cauchy stress and linearized strain. However, the underlying structure of such models allows one to describe the linear homogeneous material with the inherent properties depending upon the density. In \cite{HYCSMM2023}, a two-dimensional model for materials with properties dependent upon the density is proposed, and the crack-tip fields along with mechanical stiffness for the anisotropic material are studied. A variety of phenomena were captured for both tensile and in-plane shear loading. Several engaging failure scenarios were reported, which were impossible to obtain under the classical linear elasticity models. 

In this paper, we formulate a model for a 3-D plate with an edge crack, i.e., a v-notch with a slight opening angle. A BVP is developed for the v-shaped crack geometry within the setting of nonlinear constitutive relations with both Cauchy stress and strain appearing linearly. A convergent numerical method based on the continuous Galerkin-type finite elements is proposed for the quasi-linear elliptic partial differential equation system. Several postprocessing items, including the crack-tip stress and strain,  stress intensity factor, and strain energy density, are presented along a line leading to the v-notch tip. Our study is a basis for the future development of fracture theory for brittle materials with density-dependent moduli that play a crucial role in incorporating the material response to mechanical and thermal stimuli. 



\section{Constitutive modeling}
\subsection{Basic notations}
Here, we briefly present the geometrical description of the elastic body under investigation. We assume that the body contains bulk material around a crack tip with the ultimate goal of resolving the crack-tip fields.  Let ${\Omega}$ be a region in the 3-D Euclidean space $\mathbb{R}^3$ occupied by the body as the reference configuration. The Lipschitz continuous boundary of $\Omega$ is decomposed into a disjoint union of a Neumann boundary $\Gamma_N$ and a nonempty Dirichlet boundary $\Gamma_D$ (the boundary measure of $\Gamma_D >0$), i.e. $\partial \Omega = \overline{\Gamma_N} \cup \overline{\Gamma_D}$. Also, let $\bfa{n} = \left( n_1, \; n_2, \; n_3 \right)$ be the outward unit normal. We suppose that $\Gamma_c$ represents the {V-shaped} surface completely inside $\overline{\Omega}$. Let $\bfX = \left(X_1, \,  X_2, \, X_3 \right)$  and  $\bfx = \left(x_1, \, x_2, \, x_3 \right)$ denote typical points in the reference and deformed configurations of the body, respectively.  Let $\bfu \colon \Omega \to \mathbb{R}^3$ denote the displacement field satisfying
 \begin{equation}
 \bfu = \bfx - \bfX.  
 \end{equation} 
Let $\bfx = \chi(\bfX, \; t)$ be the smooth deformation of the body ${\Omega}$, where $t$ denotes the time. The displacement gradients $\dfrac{\partial \bfu}{\partial \bfX}$ and $\dfrac{\partial \bfu}{\partial \bfx}$ are given by
\begin{equation}
\dfrac{\partial \bfu}{\partial \bfX} = \nabla_{\bfX} \bfu = \bfa{F} - \bfa{I}, \quad \mbox{and} \quad \dfrac{\partial \bfu}{\partial \bfx} = \nabla_{\bfx} \bfu =  \bfa{I} - \bfa{F}^{-1},
\end{equation}
where $\bfF \colon \Omega \to \mathbb{R}^{3 \times 3}$ denotes the deformation gradient.  

Let $\Sym(\mathbb{R}^{3 \times 3})$ be the space of $3 \times 3$ symmetric tensors equipped with the inner product $\bfa{M} \colon \bfa{N} = \sum_{i, \, j=1}^3 \, \bfa{M}_{ij} \, \bfa{N}_{ij}$ for all $\bfa{M} $ and $\bfa{N} $ in $\Sym(\mathbb{R}^{3 \times 3})$, and with  {the associated norm}  {$\| \bfa{M} \| = \sqrt{\bfa{M} \colon \bfa{M}}$}. 

The \textit{right Cauchy-Green stretch tensor} $\bfC \colon \Omega \to \mathbb{R}^{3 \times 3}$, the \textit{left Cauchy-Green stretch tensor} $\bfB \colon \Omega \to \mathbb{R}^{3 \times 3}$, the \textit{Lagrange strain} $\bfE \colon \Omega \to \mathbb{R}^{3 \times 3}$, and the \textit{ linearized strain tensor} $\bfeps \colon \Omega \to  \Sym(\mathbb{R}^{3 \times 3})$ are defined by
\begin{subequations}
\begin{align}
 \bfB &:=\bfF\bfF^{\mathrm{T}}, \; \bfC :=\bfF^{\mathrm{T}}\bfF,  \; \bfE := \dfrac{1}{2} \left( \bfC - \bfI \right),\\
\bfeps &:= \dfrac{1}{2} \left( \nabla \bfu + \nabla \bfu^{\mathrm{T}}\right),
\end{align}
\end{subequations}d
where $\left( \cdot \right)^{\mathrm{T}}$ denotes the \textit{transpose} operator for tensors, and $\bfI$ is the
3-D identity tensor. Our interest in this paper is to derive the mathematical models for the response of porous elastic solids undergoing only small displacement gradients, and thus, it is required that 
\begin{equation}\label{small_grad}
\max_{\bfX \in \Omega} \| \nabla \bfu \|  \ll \mathcal{O}(\delta), \quad \delta \ll 1.
\end{equation}
Under the above assumption \eqref{small_grad}, one gets 
\begin{subequations}\label{lin_results}
\begin{align}
&\bfB \approx \bfI + 2 \bfeps + \mathcal{O}(\delta^2), \;\;  \bfC \approx \bfI + 2 \bfeps + \mathcal{O}(\delta^2), \;\;  \bfE \approx  \bfeps,  \\
&\det \bfF = 1 + \tr(\bfeps). \label{detF_linear}
\end{align}
\end{subequations}
Let $\bfT \colon \Omega \to  \Sym(\mathbb{R}^{3 \times 3})$ be the \textit{Cauchy stress tensor} in the current configuration and for the most general case of an isotropic homogeneous compressible elastic solid, the Cauchy stress takes the following form
\begin{equation}\label{Tequation}
\bfT = \widehat{\beta_0} \bfI + \widehat{\beta_1} \bfB +  \widehat{\beta_2} \bfB^2,
\end{equation}
where the material moduli  $\widehat{\beta_i}, \; i=0, \; 1, \; 2$ are functions of $\tr \, \bfB$, $\quad \dfrac{1}{2} \left[ \left(   \tr \, \bfB   \right)^2 - \tr \, \bfB^2  \right]$ and $\det \bfB$. The quantity $\widehat{\beta_0}$ can be the spherical stress due to the constraint of incompressibility.  The Cauchy stress $\bfT$ satisfies the linear momentum balance and the angular momentum balance, respectively, as below
\begin{subequations}\label{balance_equations}
\begin{align} 
\rho \, {\ddot{\bfu}} &= \ddiv \, \bfT + \rho \, \bfb, \\
\bfT &= \bfT^{\mathrm{T}}.
\end{align}
\end{subequations}
Here, $\rho$ is the density of the material in the current configuration, and $\bfb \colon \Omega \to \mathbb{R}^3$ is the body force density. 


In this paper, we only consider the case of infinitesimal displacement gradients. Hence, we are well within the realm of linearized elasticity. However, we will work with algebraically nonlinear constitutive relationships to describe the response of geometrically linear material.  In the following subsection, we briefly describe the derivation of nonlinear relations between linearized strain and Cauchy stress, but the material parameters depend on density. 


\subsection{Implicit constitutive relations}\label{implirela}
In this study, we aim to characterize the behavior of 3-D porous solids whose material moduli depend on the density. To this end, we closely follow the work of Rajagopal in \cite{rajagopal2003,rajagopal2007elasticity,rajagopal2021a,rajagopal2021b} where an implicit relation among the stress $\bfT$, deformation gradient $\bfF$, and density $\rho$ is used to describe the response of solids, satisfying
\begin{equation}\label{eq:implicit_relation}
\bff(\rho, \, \bfT,\,\bfB) =\bf0,
\end{equation}
where the tensor-valued function $\bff$ is assumed to be \textit{isotropic}. The above relationship in its most general representation form can  be written as \cite{gokulnath2017representations}
\begin{align} 
0 &= \beta_0 \, \bfI + \beta_1 \, \bfT + \beta_2 \, \bfB +  \beta_3 \, \bfT^2 + \beta_4 \, \bfB^2 +  \beta_5 \, ( \bfT \bfB + \bfB \bfT) + \beta_6 \, ( \bfT^2 \bfB + \bfB \bfT^2)  \notag \\
& + \beta_7 \, ( \bfB^2 \bfT + \bfT \bfB^2) +  \beta_8 \, ( \bfT^2 \bfB^2 + \bfB^2 \bfT^2),  
\end{align}
where the scalar-valued functions $\beta_i$, for $i=0,\, 1, \,  \ldots, 8$, are material moduli dependent on the density and  invariants of $\bfT$ and $\bfB$ as below
\begin{align*}
\big\{ &\rho, \, \tr (\bfT), \, \tr(\bfB), \, \tr(\bfT^2), \, \tr(\bfB^2), \,\tr(\bfT^3), \, \tr(\bfB^3), \,\tr(\bfT \, \bfB), \, \tr(\bfT^2 \, \bfB) \notag \\
&\tr(\bfT \, \bfB^2), \, \tr(\bfT^2 \, \bfB^2)   \big\}.
\end{align*}
A special sub-class of the above relations is given by \cite{murru2021stress,murru2021ZAMM,vajipeyajula2023stress,itou2021implicit,itou2022investigation}  
\begin{equation}\label{imp_model}
\widetilde{\delta}_0 \, \bfI + \widetilde{\delta}_1 \, \bfeps + \widetilde{\delta}_2 \, \bfT + \widetilde{\delta}_3 \, \bfT^2 + \widetilde{\delta}_4 \, \left( \bfeps \, \bfT + \bfT \, \bfeps \right) + \widetilde{\delta}_5 \, \left( \bfeps \, \bfT^2 + \bfT^2 \,\bfeps \right)    =0,
\end{equation}
where the functions  $\widetilde{\delta}_i, \; i=0, \, 2, \, 3$ depend linearly upon the invariants of $\bfeps$ and arbitrarily upon the invariants of  $\bfT$. While, the functions  $\widetilde{\delta}_i, \; i=1, \, 4, \, 5$ depends on the invariants of $\bfT$. The above sub-class of relations have been studied extensively in the literature for a variety of BVPs \cite{HYCSMM2023,yoon2022finite,HCY_SMM_MMS2022,yoon2021quasi,lee2022finite,ALAGAPPAN2023100162,murru2022density,gou2015modeling}.  

The density is considered globally constant in all the studies concerning the development of mathematical models for the response of solid bodies containing cracks and fractures. The balance of mass (or continuity equation) in the material description yields 
\begin{equation}\label{mass_balance1}
\rho_0 = \rho \, \det \bfF,
\end{equation}
where $\rho_0$ is the reference density. By \eqref{detF_linear},  \eqref{mass_balance1} reduces to 
\begin{equation}\label{mass_balance2}
\rho_0 = \rho \, \left( 1 + \tr(\bfeps)  \right).
\end{equation}
An implicit constitutive relation describing the small strain response of the porous elastic solid  is given by a special subclass of \eqref{imp_model}, \cite{itou2021implicit,HYCSMM2023}
\begin{equation} \label{spe_model1}
 \bfeps = C_1 \, \left( 1 + \xi_1 \, \tr \, \bfeps  \right) \; \bfT + C_2 \, \left( 1 + \xi_2 \, \tr \, \bfeps  \right) \;  \left(  \tr \bfT \right)\; \bfI, 
\end{equation}
where $\xi_1, \, \xi_2, \, C_1, \, C_2$ are all constants. {Notice that the classical linearized elasticity model is a special case of the above relation \eqref{spe_model1} for $\xi_1\to 0$ and $\xi_2\to 0$. Hence, the model \eqref{spe_model1} can generate broader classes of constitutive relations for different values of the parameters.} The constants $C_1$ and $C_2$ are shown to be
\begin{equation}\label{eq:material_coeff}
C_1 = \dfrac{1 + \nu}{E} = \dfrac{1}{2 \, \mu} >0, \quad C_2 = - \dfrac{ \nu}{E}  <0,
\end{equation}
where $E$ is the Young's modulus, $\mu$ is the shear modulus (one of the two Lam\`e constants), and $\nu$ is the Poisson's ratio. For linearized elasticity, the Lam\`e constants $\lambda$ and $\mu$ are related to Young's modulus  and  Poisson's ratio via
\begin{equation}\label{eq:linear_lame}
\lambda =  \dfrac{ E \, \nu}{(1 + \nu) (1 - 2 \nu)}, \quad \mu = \dfrac{ E }{2 (1 + \nu)}.
\end{equation}
The above constitutive class (\ref{spe_model1}) provides a relationship between the linearized strain and the Cauchy stress. 

\begin{remark}
The expression $\left( 1 + \xi_1 \, \tr \, \bfeps  \right)$ can be recognized as density-dependent moduli since  $\tr \, \bfeps$ can be expressed in terms of densities of two configurations. Thus, model \eqref{spe_model1} can describe the response of porous materials in which the density changes predominantly, and also, the material moduli of such solids generally depend on the density. On the contrary, the material moduli are density-independent under the linearized elastic description of solids. 
\end{remark}

\begin{remark}
A large body of literature is devoted to the BVP formulated within the realm of the linearized theory of elasticity \eqref{Tequation}-\eqref{balance_equations} for the brittle elastic bodies containing cracks and fractures. Based on the choice of the invariants in  \eqref{Tequation} and under the valid assumption of linearized elasticity \eqref{small_grad}, the stress and strain may be related linearly to study quasi-static evolution of crack-tip \cite{knees2015quasilinear,negri2017quasi,kuttler2006quasistatic}.   As in problems of cracks with classical traction-free crack-surface boundary conditions, the strain becomes very large in the neighborhood of crack-tip, thereby contradicting the fundamental doctrine under which the theory is built \cite{rajagopal2014nonlinear}. However, there have been a lot of modifications proposed to the canonical BVP within the context of a linearized theory \cite{sendova2010new,walton2012note,ferguson2015numerical,sinclair2004stress,kim2011analysis}. The implicit theory of elasticity proposed by Rajagopal  \cite{rajagopal2003} attempts to build in new algebraic nonlinear relations for the elastic solids, and the BVPs within this new paradigm have been shown to predict bounded strain in the neighborhood of the crack-tip \cite{rajagopal2011modeling,gou2015modeling}. 
\end{remark}

In the next section, using the novel constitutive relations, we develop a mathematical model to describe the state of stress and strain in porous elastic materials. Subsequently, we present a stable discretization of the partial differential equation systems and demonstrate the difference between the numerical results from the linearized elasticity model and the novel implicit model. 

\section{BVP for 3-D porous elastic solid}\label{sec:BVP}
In this section, we develop a BVP for the response of a 3-D porous elastic solid described by the special constitutive relation in Sec. \ref{implirela}. The main objective is to resolve the crack-tip fields within the new relationship wherein the stress and linearized strain appear linearly, and the material moduli of the porous material depend on the density. We assume the material bulk body under investigation is homogeneous, isotropic, and initially unstrained and unstressed.  To describe the state of stress-strain in the new class of models, we consider a static problem. The balance of linear momentum in the absence of body force reduces to 
\begin{equation}
 - \nabla \cdot \bfT  = \boldsymbol{0} \quad \text{in} \quad \Omega.  \label{eq_blm}
\end{equation}
To model the stress response of the material whose material moduli are dependent upon the density, we utilize the constitutive relationship described in \eqref{spe_model1} under 
\[
\xi_1=\xi_2=\beta,  \quad \beta \in \mathbb{R}. 
\]
Such a constitutive relationship is invertible to express the Cauchy stress as a function of the strain as
\begin{equation}\label{def-T}
 \bfT=\frac{\mathbb{E}[\bfeps]}{1 + \beta \, \tr(\bfeps)},
\end{equation}
 where the fourth-order elasticity tensor $\mathbb{E}[\bfeps]$ satisfies
\begin{equation}
\mathbb{E}[\bfeps] := \dfrac{E}{(1+ \nu)} \bfeps + \dfrac{ \nu \, E}{(1 -2\nu)  (1 + \nu) } \tr \bfeps \; \bfI.
\end{equation}
In the new description of Cauchy stress for porous elastic materials, the generalized Lam\`e coefficients for the nonlinear constitutive model are
\begin{equation}\label{eq:c1_c2_dd_model}
\lambda := \dfrac{\nu \, E}{(1 + \nu)(1-2\nu)(1+ \beta \, \tr(\bfeps))}, \quad \mu := \dfrac{E}{2(1 + \nu)(1+ \beta \, \tr(\bfeps))}.
\end{equation}
Notice that the stress function defined in \eqref{def-T} is nonlinearly dependent upon the strain due to the function itself and the nonlinearity of the Lam\`e coefficients. 
%------------------------- remark --------------------
\begin{remark}
It is clear from the constitutive relation \eqref{spe_model1} that when $\bfT =0$, $\bfeps=0$, and vice versa,  both as expected. Furthermore,  $ \| \bfT \| \to +\infty$ implies $\tr \, \bfeps \to -1/\beta$, generating a limited strain behavior. The strain limit needs to be positive for tensile stress; hence, for stress concentration problems such as cracks and fractures under tension, the constant $\beta$ should be negative.  Conversely, for the compressive stresses, the above model shares the same ``negative'' effects as the classical linearized model regarding the strain singularity near the crack tips. Hence, one must ensure that the linearized strain is appropriately small for the compressive stress problems. Overall, the model proposed in  \eqref{spe_model1} is the most generalized one, and more importantly, the strain is limited in tension. Still, the stress has to be appropriately small so that the strain is infinitesimal in compression. 
\end{remark}
%------------------------- remark --------------------

Utilizing  \eqref{spe_model1},  \eqref{eq_blm}, and \eqref{def-T}  we obtain the following BVP:
\begin{cf}
Given all the material parameters, find $\bfu = (u_1, \, u_2, \, u_3)$ such that 
\begin{subequations}\label{f1}
\begin{align}
- & \, \nabla \cdot \left[    \frac{\overline{c}_1/2  \, \left( \nabla \bfu + \nabla \bfu^T \right) + \overline{c}_2\nu \, \nabla \cdot \bfu  \, \bfI}{1 + \beta \;  \nabla \cdot \bfu}         \right]  = \bff, \; \mbox{in} \;\; \Omega, \;\;  \label{f1_1} \\
 \bfeps &= \dfrac{(1+\nu)(1+ \beta \, \tr(\bfeps))}{E} \, \bfT - \dfrac{\nu(1+ \beta \, \tr(\bfeps))}{E} \, \tr(\bfT) \,  \bfI  \;\; \text{in} \;\; \Omega,  \\
 \bfu &= \widehat{\bfu}, \quad  \mbox{on} \quad \Gamma_D \\
 \bfT  \bfn &= \bfg, \;\;  \mbox{on} \;\; \Gamma_N,
\end{align}
\end{subequations}
where  $\overline{c}_1$ and $\overline{c}_2$ are given by
\begin{equation}
\overline{c}_1 = \dfrac{E}{(1+\nu)}, \quad \mbox{and} \quad \overline{c}_2= \dfrac{ E}{(1+\nu)(1 - 2 \nu)},
\end{equation}
$\bfg \colon \Gamma_N \to \mathbb{R}^3$ is the given traction, and $\widetilde{\bfu} \colon \Gamma_D \to \mathbb{R}^3$ is the given boundary displacement.
\end{cf}


The above BVP  \eqref{f1} is nonlinear; currently, no analytical tools are available to obtain its closed-form solution. Therefore, we resort to a convergent numerical method, such as the finite element method, to find the approximate solution.  In the next section, we propose a stable finite-element discretization for (\ref{f1}) coupled with a Picard-type iteration algorithm to obtain the numerical solution.

%------------------------- Finite element discretization --------------------
\section{Finite element discretization}
We aim to study the response of the 3-D porous elastic solid whose material moduli depend upon the density. The constitutive relationship appears linear in the stress and linearized strain \cite{murru2021stress,itou2021implicit,rajagopal2021b,HYCSMM2023}.  Suppose that the domain $\Omega$ is bounded and connected. Let $L^p(\Omega)$ and $W^{k,p}(\Omega)$ denote the standard Lebesgue and Sobolev spaces in 1-D, respectively.  The corresponding spaces for the 3-D vector-valued functions and symmetric $3\times3$ tensor-valued functions are denoted by  $(L^p(\Omega))^3$ and $(L^p(\Omega))^{3\times3}_{\Sym}$ for the Lebesgue space, and $(W^{k,p}(\Omega))^3$ and $(W^{k,p}(\Omega))^{3\times3}_{\Sym}$ for the Sobolev space. To approximate the displacement that vanishes on the boundary $\Gamma_D$, we consider the Soboleve space $(W^{1,p}_0(\Omega))^3$, where $W_0^{1,p} $  means the closure of the set of infinitely differentiable functions with compact support in $\Omega$ and its norm defined as for the space    $W^{1,2}(\Omega)$, satisfying
\begin{equation}
W_0^{1,p} := \overline{C_{c}^{\infty}(\Omega)}^{ \| \cdot \|_{1,p}} \label{def-H01}.
\end{equation}
We also define the following subspaces of $\left( W^{1,\,2}(\Omega)\right)^3$:
\begin{subequations}
\begin{align}
V_{\bfzero} &:= \left\{ \bfu \in \left( W^{1,\,2}(\Omega)\right)^3 \colon \; \bfu=\bfzero \quad  \mbox{on} \;\;\Gamma_D\right\}, \label{test_V0}\\
V_{\widehat{\bfu}} &:= \left\{ \bfu \in \left( W^{1,\,2}(\Omega)\right)^3 \colon \; \bfu=\widehat{\bfu} \quad \mbox{on} \;\; \Gamma_D\right\}. \label{test_Vu0}
\end{align}
\end{subequations}
To obtain the weak formulation of the proposed BVP, we multiply  \eqref{f1_1} by a test function $\bfv \in {V}_{\bfzero}$ and integrate by parts. Then, using Green's formula along with appropriate boundary conditions, we obtain the following weak formulation. 
\begin{cwf}
Find  $\bfu \in {V}_{\widehat{\bfu}}$, such that 
\begin{equation}\label{eq:weak_formulation}
    a(\bfu, \, \bfv) = l(\bfv), \;  \forall\, \bfv \in {V}_{\bfzero},
\end{equation}
where the bilinear term $a(\bfu, \, \bfv)$ and the linear term $l(\bfv)$ are defined by
\begin{subequations}\label{def:A-L}
\begin{align}
    a(\bfu, \, \bfv) &= \int_{\Omega} \left[    \frac{\overline{c}_1/2  \, \left( \nabla \bfu + \nabla \bfu^T \right) + \overline{c}_2\nu \, \nabla \cdot \bfu  \, \bfI}{1 + \beta \;  \nabla \cdot \bfu}         \right] \colon \bfeps( {\bfv}) \;   d\bfx\, ,  \label{eq42a}\\
 l (\bfv) &= \int_{\Omega} \bff \cdot \, \bfv \; d\bfx  +  \int_{\Gamma_N} \bfa{g} \cdot \bfa{v} \; ds.  
\end{align} 
\end{subequations} 
\end{cwf}
The above continuous weak formulation is nonlinear, and hence, obtaining a sequence of linear problems for numerical simulations to facilitate investigating the stress-strain concentration near the crack tip in a 3-D body would be helpful. We construct a \textit{Picard's type iterative algorithm} at the continuous level and then discretize the subsequent linear problems using the stable Galerkin-type finite-element method by the following algorithm.
\begin{pia}
For $n=0, \, 1, \, 2, \, \ldots$, and given $\bfu^{0} \in {V}_{\widehat{\bfu}}$,  find  $\bfu^{n+1} \in {V}_{\widehat{\bfu}}$, such that 
\begin{equation}\label{eq_pia}
    a(\bfu^{n}; \; \bfu^{n+1}, \, \bfv) = l(\bfv), \;  \forall\, \bfv \in {V}_{\bfzero},
\end{equation}
where
\begin{equation}\label{eq42a}
%\begin{align}
    a(\bfu^{n}; \; \bfu^{n+1}, \, \bfv) = \int_{\Omega} \left[    \frac{\overline{c}_1/2  \, \left( \nabla \bfu^{n+1} + \nabla (\bfu^{n+1})^T \right) + \overline{c}_2\nu \, \nabla \cdot \bfu^{n+1}  \, \bfI}{1 + \beta \;  \nabla \cdot \bfu^{n}}         \right] \colon \bfeps( {\bfv}) \;   d\bfx.
% l (\bfv) &= \int_{\Omega} \bff \cdot \, \bfv \; d\bfx  +  \int_{\Gamma_N} \bfa{g} \cdot \bfa{v} \; ds.  
%\end{align} 
\end{equation} 
\end{pia}

%In the following subsection, we present a stable finite-element discretization using bilinear finite elements. 

% ------------------------- Finite element space and discrete problem ---------------
%---------------------------------------------------------------------------------------------------

\subsection{Finite element space and discrete problem}
We briefly describe the finite element discrete problem corresponding to the defined continuous weak formulation \eqref{eq_pia}-\eqref{eq42a}. We assume that the 3-D material domain $\Omega$ is a non-convex polyhedral domain and that there exists a mesh $\mathcal{T}_h$ with $h>0$ being the mesh size and such a mesh being either quasi-uniform or a priori refined mesh. The finite element discretization of the domain $\Omega$ is assumed to be 
conforming and shape-regular in the sense of  Ciarlet \cite{ciarlet2002finite}. The discretization is done in such a way that for any two elements $K_1, K_2 \in \mathcal{T}_h$, $\overline{K}_1 \cap \overline{K}_2$ can be one of several cases including a null set, a vertex, an edge, and even $\overline{K}_1$ (or $\overline{K}_2$),  and $\bigcup\limits_{K \in \mathcal{T}_h} \overline{K} = \overline{\Omega}$. 

 For approximating the  displacement field  $\bfu$,  we define the following space,
\begin{equation}
S_h = \left\{  \bfu_h \in  \left( C(\overline{\Omega})\right)^3 \colon \left. \bfu_h\right|_K \in \mathbb{Q}_k^d, \; \forall K \in \mathcal{T}_h \right\},
\end{equation}
where $\mathbb{Q}_k^d$ is a set of tensor-products of polynomials up to an order of $k$ over the reference cell $\widehat{K}$. Then the discrete  approximation space is
\begin{equation}\label{app-spaces}
\widehat{V}_h = S_h \, \cap  \, V_{\widehat{\bfu}}. 
\end{equation}
The discrete finite element problem is given below.
\begin{dfep}
Given $\beta$, the Dirichlet boundary data $\bfu^{0}_h \in \widehat{V}_h$, and the $n^{th}$ iteration solution $\bfu^n_h \in \widehat{V}_h$, for $n=0, 1, 2, \cdots $, find $\bfu^{n+1}_h  \in \widehat{V}_h$ such that 
\begin{equation}\label{discrete-wf}
   a(\bfu_h^n; \, \bfu^{n+1}_h,\, \bfv_h) = l(\bfv_h), \forall\, \bfv_h \in \widehat{V}_h,  
\end{equation}
where the bilinear and linear terms are given by
\begin{subequations}\label{A-L-Def}
\begin{align}
a(\bfu_h^n; \, \bfu^{n+1}_h,\, \bfv_h) &=\int_{\Omega} \left[    \frac{\overline{c}_1/2  \, \left( \nabla \bfu_h^{n+1} + \nabla (\bfu_h^{n+1})^T \right) + \overline{c}_2 \nu \, \nabla \cdot \bfu_h^{n+1}  \, \bfI}{1 + \beta \;  \nabla \cdot \bfu_h^{n}}         \right] \colon \bfeps( {\bfv_h}) \;   d\bfx\, ,  \label{disc_A} \\
l(\bfv_h) &= \int_{\Omega} \bff \cdot \, \bfv_h \; d\bfx  +  \int_{\Gamma_N} \bfa{g} \cdot \bfv_h \; ds.  \label{disc_L}  
\end{align}
\end{subequations}
\end{dfep}

\begin{remark}
A proper initial guess $\bfu^{0}_h$ is required for fast convergence of the above iterative algorithm. In our implementation, we solved the linear problem first (obtained from \eqref{discrete-wf} by taking $\beta=0$) and subsequently used the computed solution as an initial guess for Picard's iterations.  For multiple values of $\beta$, all our numerical simulations converged within a reasonable number of iterations.  
\end{remark}

\subsection{Algorithm for computation}
 The following algorithm depicts the overall discrete finite element computational procedure to obtain the numerical solution of the BVP.
% developed in this paper. We reiterate that this work aims to characterize the crack-tip fields in the three-dimensional porous elastic solid.\\
 
\begin{algorithm}[H]
\SetAlgoLined
\KwInput{Choose the parameters: $\beta, \, E, \, \nu$, $I_{Max}$ (maximum of the iteration number), $Tol$}
Start with a sufficiently refined mesh; \\
Solve the linear problem with $\beta =0$ in \eqref{discrete-wf} to obtain the initial guess $\bfu^{0}_h \in \widehat{V}_h$ for subsequent Picard's iteration algorithm; \\
\For {$n=0, \,1, \, 2, \, \ldots$}{
 \While{$[\text{Iteration Number} < \text{$I_{Max}$}]$ AND $[\text{Residual} > \text{$Tol$}]$}{
  Assemble Equations~\eqref{disc_A} and \eqref{disc_L} using test functions from the discrete finite element space $S_h$ \;
 Use a \textit{direct solver} to solve for $ \bfu_h^{n+1}$\;
  }
  }  
 Save the final converged solution $\bfu_h$ to output files for post-processing\;
 Compute the crack-tip fields (e.g., stress and strain) for visualization.
 \caption{Finite element algorithm for the nonlinear BVP}
 \label{algo001}
\end{algorithm}
 Two important post-processing items  need to be computed  to characterize the crack-tip fields in the porous elastic solid using the final converged solution $\bfa{u}_h$: 
\begin{subequations}
\begin{align}
\bfa{T}_h &=\frac{ \overline{c}_1/2 \, \left( \nabla \bfa{u}_h + (\nabla \bfa{u}_h)^T \right)  + \overline{c}_2\nu \, \nabla \cdot \bfa{u}_h \, \bfa{I} }{1 + \beta \;  \nabla \cdot \bfu_h}, \\
\bfeps_h &= \dfrac{(1+\nu)(1+ \beta \,  \nabla \cdot \bfu_h)}{E} \, \bfT - \dfrac{\nu(1+ \beta \, \nabla \cdot \bfu_h )}{E} \, \tr(\bfT) \,  \bfa{I}.
\end{align}
\end{subequations}
Another important crack-tip field variable that symbolizes the strength of the material and quantifies the crack-tip evolution is the \textit{strain energy density} which is obtained by taking the inner tensor product of the stress $\bfa{T}_h$ and the strain $\bfeps_h$. The current work is the first attempt to present the entire behavior of the 3-D porous elastic solid whose material properties depend upon the density. 

% ------------------------- Numerical results ---------------
%---------------------------------------------------------------------------------------------------

\section{{Computational results}}\label{sec:NumExp}

As shown in Fig. \ref{plategeo}, a geometry used in \cite{gou2023MMS} for studying a strain-limited model, we consider a crack embedded in a thin square plate. The crack bears an angle of $2^o$, and the crack tip is on the center of the square plate. Each side of the square is 100 mm long, and the plate is 10 mm thick. The parameter values of the material are $E=10^4$ Pa and $\nu=0.3$. 

\begin{figure}[H]
	\centering
	\includegraphics[width=0.5\textwidth]{./plate}
	\caption{\footnotesize{The geometry of the crack. The crack is on a square plate. The crack angle is $2^o$. The $xyz$ rectangular coordinate system is displayed on the geometry.} }
	\label{plategeo}
\end{figure}

 Tensile displacement boundary conditions are applied to the plate as shown in Fig. \ref{tensile}. Over the two side boundaries, the displacement magnitude in the $y$-direction is 1 mm. All the other sides are traction-free. We mainly study computational results near the crack tip to understand how the nonlinear model impacts the crack's strain and stress distribution, the stress intensity factor (SIF), and strain energy. In Fig. \ref{tensile}, a blue arrowed line with coordinate $r$ starting from the middle of the edge of the crack tip and perpendicular to the plate's back is created. At the crack tip, $r=0$, and $r$ increases when further away from the crack tip.  Part of the computational results are demonstrated near the crack tip along the line for a pronounced view. Another benefit of choosing the middle line is to minimize the effect of boundary conditions on the computational results. 
 
  In our computations, the \textit{tetrahedral meshes} are prescribed over the whole geometry (Fig. \ref{meshes}). The results near the crack tip are essential to studying the nonlinear model effect. Thus, highly fine meshes are generated in the neighborhood of the crack tip. There are 38052 tetrahedral meshes with 7853 mesh vertices.  Over the boundaries, there are 5718 triangle meshes. The quality of the meshes is measured by skewness, i.e.,  the angular measure of mesh quality compared to the angles of ideal mesh types. The minimum mesh quality is 0.2298, and the average is 0.6754. The shape function of the finite element is quadratic Lagrange. The number of degrees of freedom solved for is 169848.
  The computations are implemented in the Intel(R) Core(TM) i9-9980HK CPU at 2.40 GHz of a Macbook Pro laptop. The physical memory used is 3.69 GB, with a virtual memory of 42.62 GB. We demonstrate the computational results for both positive and negative $\beta$ values at $\beta=-30, \, -20, \,-10, \,0, \,10, \,20, \,30$. {The average computational time for each $\beta$ value is about 38 seconds.} $\beta=0$ generates results for the linear-elasticity fracture model, and the results from $\beta=0$ are compared with results for other  $\beta$ values to find how the nonlinear model is distinctive compared with the counterpart.  
\begin{figure}[H]
	\centering
	\subfloat[\footnotesize{Tensile-displacement boundary conditions.}]{\label{tensile}\includegraphics[width=0.5\textwidth]{./displacement}} 
	\subfloat[\footnotesize{Tetrahedral meshes. }]{\label{meshes}\includegraphics[width=0.35\textwidth]{./mesh}} 
	\caption{{\footnotesize{The tensile displacement boundary conditions are applied on two sides of the plate parallel to the $x$-axis. Other boundaries are traction-free. Tetrahedral meshes are prescribed over the cracked plate. To produce more accurate computational results, finer meshes are created near the crack tip.}}}
	\label{D3Stress22Bound1}
\end{figure}


\subsection{Strain $\epsilon_{22}$}

 The general pattern for $\epsilon_{22}$ results is similar for all $\beta$ values used in the computation. We only illustrate the results for $\beta=-30$ as a representation. The 3-D results in the whole geometry, including the interior and surface, are impossible to display in one image. Instead, we show the distribution over the surface and discrete plane slices oriented in different directions (Fig. \ref{epsi22sli}). The $\epsilon_{22}$ distribution over the plate surface is displayed in Fig.  \ref{surffa}. In particular, the distribution over the top surface can be viewed.  The bottom surface displays similar results as the top surface but cannot be considered here. The results in the top surface show two branches of concentration of $\epsilon_{22}$ in red near the crack tip. Such concentration fades away towards the two back corners. Overall, over the top surface, the $\epsilon_{22}$ value is more significant in the back behind the crack tip than in the part in front of the crack tip. 

Figure \ref{vslice} shows the $\epsilon_{22}$ distribution over the slices parallel to the $yz$ plane.  The concentration of $\epsilon_{22}$ can be viewed in bright blue in the middle of the slice crossing the crack tip. Over each slice, the result shows an apparent left-right symmetry. If the slice is in front of the crack tip, $\epsilon_{22}$ is more diminutive, given by a darker color. On the contrary, $\epsilon_{22}$ is greater in slices in the back of the crack tip provided by a brighter color. Such results correspond to the outcomes in the top surface of the plate in Fig. \ref{surffa}. Figure \ref{xzslice} shows the $\epsilon_{22}$ distribution in the slices parallel to the $xz$ plane. Each slice except the middle one shows two distinct parts: dark and bright, transitioning through the crack tip. Lastly, Figure \ref{xyslice} shows the $\epsilon_{22}$ distribution over a single slice parallel to the $xy$ plane in the middle of the plate. The outcome is similar to the one on the top surface of the plate in Fig.  \ref{surffa} but bears different values. In summary, the four panels in Fig. \ref{epsi22sli} assist in viewing the 3-D $\epsilon_{22}$ distribution from different perspectives. Each panel's color scale shows a transparent color gradient for easy viewing of the $\epsilon_{22}$ distribution. 

\begin{figure}[H]
	\centering
	\subfloat[\footnotesize{$\epsilon_{22}$ over the whole  surface.}]{\label{surffa}\includegraphics[width=0.5\textwidth]{./strain22Beta-30_surface}} 
	\subfloat[\footnotesize{$\epsilon_{22}$ over the slices parallel to the $yz$ plane. }]{\label{vslice}\includegraphics[width=0.5\textwidth]{./strain22Beta-30vertical_slice}} \\
		\subfloat[\footnotesize{$\epsilon_{22}$ over the slices parallel to the $xz$ plane.}]{\label{xzslice}\includegraphics[width=0.5\textwidth]{./strain22Beta-30in_out_slice_2}} 
	\subfloat[\footnotesize{$\epsilon_{22}$ over the  slice parallel to the $xy$ plane through the middle of the plate. }]{\label{xyslice}\includegraphics[width=0.5\textwidth]{./strain22Beta-30midplane}} 
	\caption{{\footnotesize{Illustration of $\epsilon_{22}$ over the whole geometrical surface and discrete slices in different directions. All the panels complement each other to facilitate understanding of the $\epsilon_{22}$ distribution. The concentration of $\epsilon_{22}$ forming two branches near the crack tip can be viewed particularly in Figs. \ref{surffa} and \ref{xyslice}.}}}
	\label{epsi22sli}
\end{figure}

The relation between $\epsilon_{22}$ and $r$ is demonstrated in Fig. \ref{strainMidl}. The results for $\beta=0$ are for the linear-elasticity fracture model.  Smaller $\epsilon_{22}$ at the crack tip ($r=0$) than $\epsilon_{22}$ for $\beta=0$ is deemed to realize the strain-limiting effect. The negative and positive $\beta$ values results are shown in Fig. \ref{negabsmid} and Fig. \ref{posbsmid}, respectively.  Fig. \ref{negabsmid} shows that as $\beta$ is negative and decreases, $\epsilon_{22}$ at the crack-tip ($r=0$) decreases, reflecting the effect of the nonlinear model in reducing the strain near the crack tip. Fig. \ref{posbsmid} shows that as $\beta$ is positive and increases, greater $\beta$ generates greater $\epsilon_{22}$ at the crack-tip. The outcomes show that only negative $\beta$ values realize the nonlinear, strain-limiting effect and that positive $\beta$ values produce a counter-effect in this aspect. Furthermore, the slightly wavy behavior in the curves shows the impact of density-dependent moduli. 


\begin{figure}[H]
	\centering
	\subfloat[\footnotesize{For negative $\beta$ values.}]{\label{negabsmid}\includegraphics[width=0.5\textwidth]{./strain22negativeBeta}} 
	\subfloat[\footnotesize{For positive $\beta$ values.}]{\label{posbsmid}\includegraphics[width=0.5\textwidth]{./strain22PositiveBeta}} 
	\caption{{\footnotesize{$\epsilon_{22}$ vs. $r$ for different $\beta$ values. For negative $\beta$, greater $|\beta|$ produces smaller $\epsilon_{22}$ at the crack-tip ($r=0$), realizing the strain-limiting effect. On the contrary, for positive $\beta$, greater $|\beta|$ produces greater $\epsilon_{22}$ at the crack-tip ($r=0$), failing to realize the strain-limit effect. All curves for both positive and negative $\beta$ values show slightly wavy behaviors.}}}
	\label{strainMidl}
\end{figure}



\subsection{Stress $T_{22}$}

The $T_{22}$ distribution over the plate is similar to the one for $\epsilon_{22}$ in Fig. \ref{surffa}.   We exhibit a transparent surface view of $T_{22}$ distribution at $\beta=-30$ in Fig. \ref{T22beta_30} as a representative for other $\beta$ values.  We skip showing the $T_{22}$ distribution over slices in different directions for brevity. In Fig. \ref{T22beta_30}, the concentration of $T_{22}$ can be viewed near the crack tip and on the rear half of the left and right lateral sides. Such concentration also spreads through a large area in the rear part of the plate. The distribution on the top and bottom surfaces is identical.  Such $T_{22}$ concentration is more intense than $\epsilon_{22}$ concentration, suggesting the stress is more sensitive than the strain under the impact of the parameter $\beta$.

\begin{figure}[H]
	\centering
	\includegraphics[width=0.5\textwidth]{./stress22Beta-30midplane}
	\caption{\footnotesize{A transparent surface view of $T_{22}$ distribution (unit: $10^4$ Pa) at $\beta=-30$. The concentration of $T_{22}$ can be viewed near the crack tip and on the rear part of the left and right lateral sides. The general pattern is similar to the one for $\epsilon_{22}$ in Fig. \ref{surffa}.} }
	\label{T22beta_30}
\end{figure}

The function of $T_{22}$ vs. $r$ is displayed in Fig. \ref{T22midl}. In Fig. \ref{T22NegBeta}, it shows negative $\beta$  fails to reduce $T_{22}$ at the crack tip for the linear-elasticity fracture model ($\beta=0$). On the contrary, negative $\beta$  increases $T_{22}$ at the crack tip, but a smaller negative $\beta$ value may not necessarily produce a larger increase for crack-tip $T_{22}$. This phenomenon is contrary to the $\beta$ effect on $\epsilon_{22}$ in Fig. \ref{negabsmid} for negative $\beta$. However, as shown in Fig. \ref{T22PosBeta},  positive $\beta$  is capable of reducing $T_{22}$ at the crack tip for $\beta=0$. The greater the positive $\beta$ is, the higher the reduction is. Similar to Fig. \ref{strainMidl}, all curves show slightly wavy patterns due to the impact of the density-dependent moduli.

\begin{figure}[H]
	\centering
	\subfloat[\footnotesize{For negative $\beta$ values.}]{\label{T22NegBeta}\includegraphics[width=0.5\textwidth]{./T22NegativeBetaMidline}} 
	\subfloat[\footnotesize{For positive $\beta$ values.}]{\label{T22PosBeta}\includegraphics[width=0.5\textwidth]{./T22PositiveBetaMidline}} 
	\caption{{\footnotesize{$T_{22}$ vs. $r$ ($T_{22}$ unit: $10^4$ Pa). In the first panel, at $r=0$ (the crack-tip), negative $\beta$ does not reduce $T_{22}$ compared with $T_{22}$ at $\beta=0$. Generally, a more negative value does not necessarily generate greater $T_{22}$ at the crack tip. In the second panel for $\beta>0$, positive $\beta$ generates smaller $T_{22}$ value compared with $T_{22}$ at $\beta=0$. Greater $\beta$ creates smaller $T_{22}$  at $r=0$. Slightly wavy patterns also appear in all the curves due to the effect of the density-dependent moduli.}}}
	\label{T22midl}
\end{figure}

\subsection{SIF}

{For the current problem, we do not have an asymptotic or analytical solution. Therefore, an explicit description of crack-tip SIF is not available. However, one can utilize the crack-tip SIF defined for the linear elastic fracture mechanics model and use the finite element solution for the nonlinear model to glean some vital physical insight into SIF.} \\

The SIF $K_I$ in the context of linear elasticity  defined as
\begin{equation}\label{SIF_M1}
	K_I = \lim_{r \to 0^+} \sqrt{2 \pi r} T_{22}.
\end{equation}
It is challenging to analytically compute $K_I$ for complicated models such as the ones in 3-D that we are investigating in this paper. We compute $K_I$ numerically for illustrative purposes to understand its value as a function of the nonlinear modeling parameter $\beta$. The function $\sqrt{2 \pi r} T_{22}$ vs. $r$ is shown in Fig. \ref{SIFcur}. For $\beta=0, \,-10, \,-20, \,-30$, the $K_I$ values are, respectively, 0.0110, 0.0185, 0.0251, 0.0178; For $\beta=0, \,10, \,20, \,30$, the $K_I$ values are,  respectively, 0.0110, 0.0060, 0.0037, 0.0026 (Unit: $10^4\text{mm}^{1/2}\text{Pa}$). All the $K_I$ values are very close, implying an identical theoretical $K_I$ value for all $\beta$  is highly likely. Such numerical approximation for $K_I$ shows that the nonlinear model with density-dependent moduli is designed appropriately under the common crack criteria used for the linear-elasticity fracture model. Also, in the two panels of Fig. \ref{SIFcur}, the curve shapes for $\beta<0$ and $\beta>0$ are slightly distinct. The $\beta<0$ curves show a greater curvature, while those for $\beta>0$ show flatter behaviors. Such difference is caused by the sign of $\beta$, disclosing how negative and positive $\beta$ values impact the SIF differently.

\begin{figure}[H]
	\centering
	\subfloat[\footnotesize{For negative $\beta$ values.}]{\label{negBetSIF}\includegraphics[width=0.5\textwidth]{./SIFNegBeta}} 
	\subfloat[\footnotesize{For positive $\beta$ values.}]{\label{PosBetSIF}\includegraphics[width=0.5\textwidth]{./SIFpositiveBeta}} 
	\caption{{\footnotesize{Exhibition of $\sqrt{2 \pi r} T_{22}$ vs. $r$ for studying the SIF (unit of $\sqrt{2 \pi r} T_{22}$: $10^4\text{mm}^{1/2}\text{Pa}$). All the curves approximately approach the same limit as $r$ approaches 0, implying a common SIF for all cases.}}}
	\label{SIFcur}
\end{figure}

It is clear from the above Fig~\ref{SIFcur} that the SIF values near the crack-tip are higher for the negative $\beta$ values; hence, there could be stress concentration only for negative $\beta$ values. 

\subsection{Strain energy}

The strain energy is defined to be $$(\mathbf{\epsilon}:\mathbf{T})=\sum_{i,j=1,2,3}\epsilon_{ij}T_{ij}.$$ We plot the function of strain energy vs. $r$ in Fig. \ref{energytip}. In Fig. \ref{eneNegBe}, $\beta=-30$ gives the smallest strain energy and particularly generates energy at the crack tip ($r=0$) smaller than that for $\beta=0$. However, both $\beta=-10$ and $\beta=-20$ generate greater energy than that for $\beta=0$ at $r=0$. Such outcomes suggest that not all negative $\beta$ values create smaller energy than that for the linear-elasticity fracture model.  Fig. \ref{enePosBe} shows that all positive $\beta$ values generate energy at $r=0$ greater than that for $\beta=0$. The pattern for energy under positive $\beta$ is that the greater the $\beta$ value is, the greater the energy at the crack tip is.

\begin{figure}[H]
	\centering
	\subfloat[\footnotesize{For negative $\beta$ values.}]{\label{eneNegBe}\includegraphics[width=0.5\textwidth]{./energyNegativeBeta}} 
	\subfloat[\footnotesize{For positive $\beta$ values.}]{\label{enePosBe}\includegraphics[width=0.5\textwidth]{./energyPositiveBeta}} 
	\caption{{\footnotesize{Energy vs. $r$ for both positive and negative $\beta$ values (unit of energy: $10^4\text{mm}^{1/2}\text{Pa}$). In the left panel, $\beta=-30$ gives the energy at the crack tip ($r=0$) smaller than that for $\beta=0$, and $\beta=-10$ and $\beta=-20$ generate greater energy than that for $\beta=0$ at $r=0$. In the right panel, all positive $\beta$ values generate energy at $r=0$ greater than that for $\beta=0$, and the greater $\beta$ is, the greater the energy at the crack tip is.}}}
	\label{energytip}
\end{figure}

An important observation about the strain energy is that it gets its most significant value directly ahead of the crack tip, which is consistent with the observation made in the linearized elasticity model. Therefore, one can utilize the classical fracture criterion to study the quasi-static and dynamic propagation of crack tips within the framework of the models proposed in this paper. One can also employ the crack-regularization techniques such as the one proposed in \cite{yoon2021quasi,lee2022finite} to study the evolution problems in the context of nonlinear models.    




\section{Conclusions and discussions}
In the last few years, there have been many theoretical investigations on developing algebraically nonlinear models for the description of geometrically linear as well as nonlinear elastic solids \cite{rajagopal2003,rajagopal2007elasticity,mai2015monotonicity,mai2015strong}. The ultimate goal of such studies is to formulate well-posed mathematical models for the bulk material behavior,  which can predict bounded crack-tip strain fields and preclude the necessity of using \textit{ad hoc} modeling augmentations such as cohesive or process zones to correct the erroneous predictions from linear elasticity. Perhaps the most crucial applicability of the modeling paradigm introduced in \cite{rajagopal2011modeling,gou2015modeling,Mallikarjunaiah2015,MalliPhD2015} is that it can pave a way to develop new fracture theories applicable to a wide variety of brittle materials.  This paper is the first part of the extended research program to develop theoretical and computational models for cracks and fractures in a new class of models describing the response of 3-D porous elastic solids. Using the particular subclass of constitutive relationship, we derive a quasilinear elliptic BVP for the static v-notch domain.  



The 3-D problem is discretized using a continuous Galerkin-type finite element method coupled with the Picard-type linearization, meaning that the problem is linear at each iteration.  Hence, a unique solution is guaranteed, and a direct solver can be used to find the nodal values of the computational results. A few remarks about the numerical solution and its interpretation are as follows.
\begin{itemize}
\item[1.] We observe that the finite element discretization of the 3-D BVP is convergent, and Picard's iterations take a reasonable number of iterations to meet the stopping criterion. The relative tolerance 0.001 is employed as the stopping criterion with a maximum number of iterations of 1000.The initial guess is obtained by solving the BVP from the linearized elasticity, i.e., by taking $\beta=0$ in the nonlinear model. We notice that the numerical solution to the proposed nonlinear BVP converges to the corresponding one from the classical linear elasticity model as $\beta \to 0$.  
\item[2. ] For the tensile loading, with negative $\beta$ values, the nonlinear model predicts that the crack-tip strain growth is much smaller than the classical linear model. Our model displays a strain-limiting effect for negative $\beta$ values. However, a clear opposite pattern for the crack-tip strain values is observed for the positive $\beta$ values.  Such results are expected because the stiffness parameter $(1 + \beta \, \tr \, \bfeps)$ changes in the nonlinear model, different from the constant stiffness values in the classical linearized model. 
\item[3. ] The stress concentration increases if the negative $\beta$ value decreases; hence, the proposed model has the same effect on stress concentration as in the classical model. This implies that the proposed material model exhibits a stronger brittle effect for smaller negative  $\beta$ values. The largest value of $\bfT_{22}$ appears directly before the crack-tip, another important feature in the classical linear model. However, the stress concentration decreases in the nonlinear model with the increasing positive $\beta$ values. 
\item[4. ] As pointed out earlier, we currently do not have an asymptotic crack-tip solution for the proposed model; hence, the classical definition of SIF from linear elasticity is utilized to glean some vital information about the singularities. The finite element solution is used in the SIF formula to characterize the nature of SIF for the current model. The term $\sqrt{2\pi r} \bfT_{22}$ for the SIF approximately reaches a constant ahead of the crack-tip, and the values are higher for negative $\beta$ values, which indicates the possibility of stress singularity. For the model studied in this paper, the important open problem is to characterize the singularity via asymptotic analysis similar to \cite{gou2015modeling,rajagopal2011modeling}. 
\item[5. ]  The strain energy density also obtains its maximum before the crack tip. These results are consistent with those from the classical linear elasticity model. Hence, one can use the same local fracture criterion near the crack tip to study evolution problems within the framework of the nonlinear theory presented in this paper. The phase-field regularization, similar to the one done for strain-limiting models in \cite{yoon2021quasi,lee2022finite}, can also be used for the current framework to study quasi-static crack propagation. 
\item[6. ] The present study may incorporate surface mechanics as described in \cite{ferguson2015numerical}, thereby enabling the elimination of unphysical stress singularities. Consequently, a new local criterion predicated on cleavage stress or the maximum stress at the crack tip may be employed to examine crack-tip evolution issues. Another significant future direction is to examine the error analysis of the local discontinuous Galerkin-type finite element discretization \cite{manohar2024hp}. 
\end{itemize}

%\section{Declarations}
%
%\subsection{Ethics approval and consent to participate:} Both authors consent to all ethical standards of the journal.  
%
%\subsection{Consent for publication:} Both authors consent to this publication. 
%
%\subsection{Availability of data and materials} All the data sets generated are provided in the articles as graphs or tables. 
%
%\subsection{Competing interest} The authors have no competing interest to declare in this research work.
%
%\subsection{Funding} No funding was received for this research. 
%
%\subsection{Authors' contributions} Both authors contributed equally to this research.  
%
%\subsection{Acknowledgements:} Authors acknowledge the support of their department's computing facilities at their respective universities. 
 


 

\bibliographystyle{unsrtnat}
\bibliography{references}





\end{document}