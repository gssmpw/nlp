\section{The Cabinets-Prophets Problem} \label{sec:non-pandora}

Recall that we require some intermediate results before we are ready for fully-general Markov Search Processes.
In this section, we consider a simple version of our problem with just one decision to be made per arrival and no search costs at all.
These results will be necessary to solve more general problems and will capture some key challenges for prophets when each arrival involves decisions that affect its value.

\subsection{Setting and tools}

In the \emph{Cabinets-Prophets$(\F)$ Problem}, there are $n$ \emph{cabinets}.
Each cabinet $i$ has $m_i \geq 1$ drawers, and each drawer $j$ contains a hidden random variable $X_i^j$.
In each cabinet $i$, at most one drawer may ever be opened.
The variables in $i$'s drawers, $(X_i^j : j \in [m_i])$, may be arbitrarily correlated with each other, but each cabinet is mutually independent of the other cabinets.

A fully general algorithm $\OPT$ first chooses some cabinet $i$, picks a drawer $j^*(i) \in [m_i]$, and opens it to observe $X_i^{j^*(i)}$.
Then, it may pick another cabinet and open one of its drawers, and so on.
However, it may never open a second drawer on any cabinet.
Eventually, $\OPT$ halts and claims a subset $F^* \subseteq 2^{[n]}$ of the opened cabinets, obtaining welfare $\Welf_{\OPT} = \sum_{i \in F^*} X_i^{j^*(i)}$.
We may require $\OPT$ to be either ex-ante feasible (Definition \ref{def:ex-ante}) or ex-post feasible with respect to $\F$.
This is a stochastic probing problem for which efficient constant-factor approximation algorithms are known \citep{asadpour2016maximizing, beyhaghi2019pandora}, but we do not know of a treatment in the prophet inequalities framework.

An online algorithm $\ALG$ processes the cabinets one at a time in order $i=1,\dots,n$, building a solution set $F$ that must be ex-post feasible, i.e. $F \in \F$.
When cabinet $i$ arrives, the algorithm picks a single drawer $j(i)$ and opens it to reveal $X_i^{j(i)}$.
$\ALG$ must then either claim $i$ and or discard it.
We have $\Welfof{\ALG} = \sum_{i \in F} X_i^{j(i)}$.



It will be useful to define a basic condition on algorithms in the cabinets setting: monotonicity (e.g. \citet{kleinberg2012matroid}).
We add another basic condition, locality.
\begin{definition}[Local, monotone] \label{def:monotone}
  An online algorithm for Cabinets-Prophets$(\F)$ is \emph{monotone} if, for all fixed realizations $(X_i^j : i \in [n], j \in [m_i])$ of all variables, increasing the value of any variable $X_i^j$ weakly increases the probability that the algorithm claims cabinet $i$.
  The algorithm is \emph{local} if the decision to claim $i$ depends only on the current solution set $F$ and the values $\{X_i^{j(i)} : i \in F\}$.
\end{definition}

\begin{observation} \label{obs:monotone}
  Without loss of generality, any online algorithm for Cabinets-Prophets is local monotone, and furthermore has the following form: when cabinet $i$ arrives and it opens drawer $j$, the algorithm defines a threshold $\tau_i^j$ based on the current solution set $F$ and $\{X_{i'}^{j(i)} : i' \in F\}$; it then claims $i$ if $X_i^j > \tau_i^j$, also claiming with some fixed probability if $X_i^j = \tau_i^j$.
\end{observation}
\begin{proof}
  (Monotonicity) Given any online algorithm $\ALG$, whenever it opens a drawer $j$ of cabinet $i$, the value $X_i^j$ it observes is independent of the entire history observed by the algorithm so far.
  It then claims $i$ with some probability $q$.
  Without changing the rest of the behavior of the algorithm, we can move probability mass in this case so that $i$ is claimed if $X_i^j$ exceeds the threshold $\tau_i^j$, or with a given probability when $X_i^j = \tau_i^j$, such that the probability of claiming remains $q$.
  The algorithm is now monotone and its welfare has only weakly increased.
  (Locality) Suppose the algorithm is nonlocal, i.e. on arrival $i$, its decision depends on values $X_{i'}^j$ for $i' < i, i \not\in F$.
  We independently re-draw these values from their prior distributions conditioned on not being taken, i.e. not being above their respective thresholds.
  The algorithm has the same distribution over decisions, but is now local.
\end{proof}


\subsection{Results}

Our first result in this setting is the following.
\begin{theorem}[Classic Prophets to Cabinets-Prophets] \label{thm:classic-cp-dc}
  For any downward-closed $\F$, if there exists an ex-ante classic prophet inequality $\alpha$ for $\F$, then there exists an ex-ante prophet inequality $\alpha$ for Cabinets-Prophets$(\F)$.
  If the former is efficient and if $\PF$ has an efficient separation oracle, as in the case of matroids, the latter is as well.
\end{theorem}
The proof relies on Lemma \ref{lemma:cabinets-nonadaptive-opt} (Appendix \ref{app:non-pandora}).
The lemma asserts that the ex-ante optimal is actually computable in polynomial time via convex programming, similar to an analysis in \citet{feldman2016online}.
This is a pleasant result, as the ex-post feasible optimum appears hard to compute.~
Furthermore, the ex-ante relaxation allows the optimal decision on each arrival $i$ to be uncorrelated, which enables our reduction.
\begin{proof}[Proof of Theorem \ref{thm:classic-cp-dc}]
  By Lemma \ref{lemma:cabinets-nonadaptive-opt}, WLOG, the ex-ante optimal offline algorithm opens a drawer $j^*(i)$ sampled from some fixed distribution $\lambda_i \in \Delta_{[m_i]}$ independently, for each cabinet $i$.
  Therefore, given the instance, we consider just the class of online algorithms $\ALG$ that also independently open $j^*(i)$ sampled from $\lambda_i$ at each arrival $i$.
  In other words, $\ALG$ and $\OPT$ both face WLOG a list of independent variables $Y_1,\dots,Y_n$ where $Y_i$ is distributed as $X_i^{j^*(i)}$, with randomness over both $(X_i^j : j \in [m_i])$ and $j^*(i) \sim \lambda_i$.
  Here, $\ALG$ can guarantee an approximation ratio $\alpha$ if there exists an ex-ante prophet inequality $\alpha$ for $\F$ in the classic prophets setting with arrivals $\{Y_i\}$.

  By Lemma \ref{lemma:cabinets-nonadaptive-opt}, $(\lambda_i : i \in [n])$ is efficiently computable with a separation oracle for $\PF$ (Definition \ref{def:ex-ante}), so $\ALG$ is efficient if the algorithm for classic prophets is efficient.
  In particular, any matroid constraint has such an oracle~\citep{cunningham1984testing}. 
\end{proof}

Theorem \ref{thm:classic-cp-dc} produces a \emph{monotone} algorithm, but it does not produce an \emph{incentive-compatible} posted-price mechanism.
If an agent $i$ arrives with a search process modeled as a cabinet, and faces some posted price for claiming any discovered reward, that agent will not generally maximize utility by opening drawers randomly according to a prescribed distribution $\lambda_i$.
To capture incentive compatibility, we make the following definition.

\begin{definition}[\SAUP{} in the Cabinets-Prophets setting] \label{def:csaup}
	The \emph{Cabinets-Prophets Single-Agent Utility Problem} \SAUP$(i,\tau)$, given a cabinet $i$ and threshold $\tau$, is to find $j(i) = \argmax_{j \in [m_i]} \E (X_i^j - \tau)^+$, open drawer $j(i)$, and claim $i$ iff $X_i^{j(i)} \geq \tau$.
	The \emph{greedy policy} directly computes the argmax and is efficient.
	A \emph{\SAUP{}-based algorithm} for Cabinets-Prophets$(\F)$ is one that computes prior to each arrival $i$ a threshold $T_i$, then solves $\SAUP{}(i,T_i)$.
\end{definition}

In the Cabinets setting, \SAUP{}-based algorithms appear to come with loss of generality.
In particular, they seem incompatible with approaches that impose constraints on the probability of claiming a given cabinet $i$.
However, a \SAUP{}-based prophet inequality of $\alpha$ immediately implies a posted-price algorithm with Price of Anarchy $\alpha$ for a setting where agents' value for selection depends on a single choice they may make (modeled as a drawer of their cabinet).

Not being able to rely on Theorem \ref{thm:classic-cp-dc} to produce \SAUP{}-based algorithms, we construct one explicitly.

\begin{theorem}[Matroids, \SAUP-based] \label{thm:classic-cp-saup}
  If $\F$ is a matroid, then there is an efficient, \SAUP-based ex-ante prophet inequality of $\frac{1}{2}$ for Cabinets-Prophets$(\F)$.
\end{theorem}
The full proof appears in Appendix \ref{app:non-pandora}; we sketch the ideas.
In the classic matroid prophets setting, \citet{kleinberg2012matroid} give a prophet inequality of $\frac{1}{2}$ to the ex-post feasible benchmark.
This algorithm can be made \SAUP-compatible and extended to Cabinets-Prophets, but that involves computing the ex-post optimal for Cabinets-Prophets, which we do not know how to do efficiently.
Furthermore, we will actually require the ex-ante benchmark later in the paper, so we turn to it now.
\citet{lee2018optimal} extend the proof to obtain the same $\frac{1}{2}$ factor against the ex-ante benchmark.
There, the thresholds are based on the ex-ante optimal algorithm, something we can compute for Cabinets-Prophets by Lemma \ref{lemma:cabinets-nonadaptive-opt}.
However, it is not immediately obvious how to extend \citet{lee2018optimal} to a \SAUP-based algorithm for Cabinets-Prophets: it relies on a generic reduction to algorithms for the Bernoulli case, and as in Theorem \ref{thm:classic-cp-dc}, the generic reduction cannot automatically be made \SAUP-based.

We show how to modify both the algorithm and proof of \citet{lee2018optimal}.
The first change is that we give a direct proof that does not go through a Bernoulli reduction, but instead relies on Bernoullis as a proof technique.
The second change is to extend the proof to the cabinets setting.
The interaction with cabinets is mediated through a simple application of the \SAUP{} property.
But to do so, we must carefully introduce the Bernoullis at just the right step in the proof.

Because we need it later in the paper, we give one additional tweak of independent interest: any $(1-\epsilon)$-approximation $\OPT'$ to ex-ante $\OPT$ yields an efficient $\tfrac{1-\epsilon}{2}$ prophet inequality via our algorithm (Lemma \ref{lemma:matroid-cabinet}).
This property does not obviously hold for the results of \citet{kleinberg2012matroid} and \citet{lee2018optimal}, because the thresholds are set based on marginal differences on a given item, and an approximation oracle for $\Welfof{\OPT}$ does not guarantee an approximation to the marginal differences.
We circumvent this obstacle by utilizing, not just an approximation to the value of $\OPT$, but the actual ex-ante feasible solution of an approximation algorithm $\OPT'$.
Roughly, proof techniques of \citet{kleinberg2012matroid} extend because our proof's ``re-randomization'' constructs Bernoulli instances relative to which $\OPT'$ is actually optimal.
The full proof is in Appendix \ref{app:non-pandora}.


