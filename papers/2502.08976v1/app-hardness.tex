\section{Cabinets and Complexity} \label{app:complexity}

In this appendix, we collect some reductions between cabinets problems and others.

\subsection{A $1-\frac{1}{e}-\epsilon$-approximate polynomial-time, non-adaptive algorithm for Cabinets}

While our \SAUP{}-based prophet algorithm for the Cabinets problem yields a $\frac{1}{2}$-approximation,
we can obtain a $1-\frac{1}{e}-\epsilon$-approximate polynomial-time non-adaptive algorithm for single-item selection for any $\epsilon > 0$ by an extension of results from
\citet{asadpour2016maximizing}, inspired by \citet{beyhaghi2019pandora}'s use of their results to analyze non-obligatory inspection.
\begin{observation}
The Cabinets problem is a special case of the Stochastic Monotone Submodular Optimization problem described in \citet{asadpour2016maximizing}.
\end{observation}
\begin{proof}
Consider a Cabinets instance $\mathcal{I}$ with cabinets $(X_1^1,\ldots,X_1^{m_i}),\ldots,(X_n^1,\ldots,X_n^{m_n})$.
Let $\mathcal{F}$ be the \emph{partition matroid} on $(X_1^1,\ldots,x_n^{m_n})$, of which the maximal independent sets are all sets of the form
$(X_1^{j_1},X_2^{j_2},\ldots,X_n^{j_n})$, i.e., containing a single variable from each cabinet. Observe that the problem of choosing $S\in\mathcal{F}$
maximizing $\E[\max_{X_i^j\in S} X_i^j]$ is exactly the cabinets problem; the optimal algorithm on a given Cabinets instance selects exactly one variable from each cabinet in such a way that it maximizes the expected maximum realization.
This is also a special case of the SMSM problem presented in \citet{asadpour2016maximizing}, in which the agent selects a feasible set of random variables from some ground set subject to an arbitrary matroid constraint, with the goal of maximizing the maximum realization.
In the SMSM problem, agents can either select variables \emph{adaptively}, observing each realization before making the next selection, or \emph{non-adaptively}, selecting every variable before observing any realizations.
\citet{asadpour2016maximizing} prove that the adaptivity gap of this problem is $1-\frac{1}{e}$, and devise a PTAS for the problem with a $1-\frac{1}{e}-\epsilon$ approximation factor for any $\epsilon > 0$.
As the Cabinets problem is a special case of SMSM in which the matroid constraint is the partition matroid described above, it follows that there is a polynomial-time non-adaptive algorithm for Cabinets which achieves a $1-\frac{1}{e}-\epsilon$ approximation for arbitrarily small $\epsilon$.
\end{proof}


\subsection{Reduction of Non-Obligatory Pandora's Box to Pandora's Cabinets}  \label{app:hardness}

We prove that Pandora's Cabinets and Combinatorial Markov Search are both NP-Hard.
\citet{fu2023pandora} showed that the problem of Pandora's Box with non-obligatory inspection,
described below, is NP-hard.
We reduce Non-Obligatory Pandora's Box to Pandora's Cabinets, and subsequently reduce Pandora's Cabinets to Combinatorial Markov Search.

\begin{observation}
Non-Obligatory Pandora's Box reduces to Pandora's Cabinets.
\end{observation}
\begin{proof}
Recall the original Pandora's Box problem: an agent is presented with $n$ boxes $\{1,\ldots,n\}$.
Box $i$ contains a value $v_i$ distributed according to a distribution $D_i$, and has an associated cost $c_i$ to open.
The agent sequentially opens boxes, paying cost $c_i$ to open box $i$ and drawing $v_i$ from $D_i$, and at any point may stop and select one of the revealed values,
with the objective of maximizing the selected value minus the sum of inspection costs paid.
In the non-obligatory setting, the agent has the additional option of selecting any unopened box $i$ ``without inspection:''
they obtain a value $v_i$ drawn from $D_i$ without paying the associated inspection cost and immediately halt.
Note that because the objective is to maximize utility in expectation, we can act as if taking a box without inspection yields a deterministic value $\mu_i := \E_{v_i \sim D_i}[v_i]$ rather than a random value $v_i \sim D_i$.


Given an instance $\calI$ of Non-Obligatory Pandora's Box, we construct an instance $\calI'$ of Pandora's Cabinets as follows.
For every box $i$ in $\calI$, we define a cabinet $i$ with 2 drawers in $\calI'$.
Drawer 1, which represents inspection of box $i$, contains a bandit $\M_i$ with a start state $s_1^*$ and, for every value $x$ in the support of $D_i$, an associated sink state $s_{1,x}$.
Advancing in state $s^*$ has cost $C_i^1(s_1^*)=c_i$, and for every $x\in\mathrm{supp}(D_i)$, $P_i^1(s_{1,x};a,s)$ is the probability of drawing $x$ from $D_i$,
and $V_i^1(s_{1,x})=x$.
Drawer 2, which represents taking the box without inspection, has a single state $s_2^*$ with value $V_i^2(s_2^*)=\mu_i$.
We set our selection constraint to be $\mathcal{F}=\{\{1\},\{2\},\ldots,\{2\}\}$, that is, the rank-one matroid constraint, i.e. selection of a single box.
\end{proof}


\subsection{Reduction of Pandora's Cabinets to Combinatorial Markov Search}

Pandora's Cabinets is not technically a special case of Combinatorial Markov Search because it includes decisions that have zero cost (which drawer to open).
Here, we show that it does reduce to Combinatorial Markov Search directly.

\begin{proposition}
Pandora's Cabinets reduces to Combinatorial Markov Search.
\end{proposition}
Pandora's Cabinets can almost be modeled directly as a special case of Combinatorial Markov Search,
by constructing an MSP for each cabinet that has a 0-cost action for each drawer of that cabinet that deterministically transitions to the start state of the corresponding bandit.
However, Markov Search Processes do not allow for 0-cost transitions.
Nevertheless, we can still reduce Pandora's Cabinets to Combinatorial Markov Search.
\begin{proof}
Given an instance $\calI$ of Pandora's Cabinets, construct an instance $\calI'$ of Combinatorial Markov Search as follows.
Let $c_{min}$ be the minimum cost of the first transition of any bandit in $\calI$, and choose $\epsilon < c_{min}$.
For each cabinet $i$, construct an MSP $\M_i$ with an action $a_i^j$ for every drawer $j$, and let $P(s_i^j;s_i^*,a_i^j)=1$ where $s_i^*$ is the start state of $\M_i$
and $s_i^j$ is the start state of the bandit contained in drawer $j$ of cabinet $i$. If $s_i^j$ is a sink, then add $\epsilon$ to $V(s_i^j)$, and otherwise, subtract $\epsilon$
from the cost $C(s_i^j)$ of advancing from state $s_i^j$.
Now the combined utility of selecting a drawer $j$ in a cabinet $i$ and advancing it once in $\calI'$ is
$-\epsilon+[V(s_i^j)+\epsilon]=V(s_i^j)$ if $s_i^j$ is a sink, and $-\epsilon-[C(s_i^j)-\epsilon]=C(s_i^j)$ otherwise --
equal to the utility of selecting the corresponding drawer $j$ from cabinet $i$ in $\calI$.
Note that in Pandora's Cabinets, it is never optimal to select a drawer without immediately advancing its bandit, as no information is revealed merely by selecting a cabinet.
Then because the utility of selecting a drawer and advancing it once is the same in both $\calI$ and $\calI'$, an optimal algorithm for $\calI'$ is optimal for $\calI$.
\end{proof}
