\newpage
\appendix
\onecolumn
\begin{center}{\bf {\LARGE Appendix}}\end{center}
\paragraph{Organization} Appendix is organized as follows: In Section~\ref{app:sec:additional_qualitative_results}, we provide additional qualitative results including the entire responses in Table~\ref{tab:qualitative} and other case studies. In Section~\ref{app:sec:details_MolLLaMA}, we explain details of the instruction dataset construction and the training. In Section~\ref{app:sec:experimental_details}, we provide experimental details for the evaluation settings. Finally, in Section~\ref{app:sec:additional_experiments}, we provide additional experimental results for the PAMPA prediction, ablation study, and BBBP prediction.

\vspace{-0.1in}
\section{Additional Qualitative Results\label{app:sec:additional_qualitative_results}}
\vspace{-0.05in}
In this section, we provide additional qualitative results to analyze the behaviors of Mol-LLaMA. First, we provide the entire responses of the case study in Table~\ref{tab:qualitative}. As shown in Table~\ref{tab:qualitative_full_part1}, Mol-LLaMA understands the molecular structures and correctly predicts the main class of the given molecule (i.e. Benzodiazepine), the attached functional group (i.e. pyridine), and its biological functionalities. Interestingly, Mol-LLaMA provides additional information, detailing the effects of each structural component such as the blood-brain barrier penetration from the pyridine ring, the binding affinity to the GABA-A receptor from the halogen atom, and the inhibitory effects of benzodiazepines. In contrast, as shown in Table~\ref{tab:qualitative_full_part2}, GPT-4o, 3D-MoLM, and Mol-Instructions misinterpret the molecular structures and give the incorrect features of molecules. On the other hand, LLaMo misinterprets the substructures (i.e. methyl group) and does not provide the molecular features, focusing on describing the structural information.

\vspace{-0.03in}
Additionally, as shown in Table~\ref{app:tab:additional_qualitative_adenosine}, one notable ability of Mol-LLaMA is that, beyond the properties annotated in PubChem, Mol-LLaMA understands various aspects of the molecular properties that can be found in the scientific literature, such as regulating the secretion of insulin~\cite{heseltine1995adenosineinsulin}, inhibiting the immune cells~\cite{wang2024adenosineimmune}, and protecting neurons~\cite{dall2003adenosineneuroprotection}. Further, as shown in Table~\ref{app:tab:additional_qualitative_nystatin}, Mol-LLaMA provides detailed rationales for the predicted properties. Specifically, Mol-LLaMA explains the key structure (i.e. macrolides) and additional structural features (i.e. polyene chain), and then explains that these structural features are related to the specific biological properties (i.e. antifungal activity), providing principles of the predicted properties (i.e. binding to ergosterol)~\cite{serhan2014polyene}. We hope that these additional qualitative results show the potential of Mol-LLaMA as a general-purpose assistant for molecular analysis.


\begin{table*}[ht!]
\centering
\vspace{-0.05in}
\footnotesize
\renewcommand{\arraystretch}{1.2}
\begin{tabular}{p{1.7cm} p{14.5cm}}
\toprule
 & \includegraphics[height=2.7cm]{figure/qualitative_example.pdf}\\
\midrule
SMILES & C1C(=O)NC2=C(C=C(C=C2)Br)C(=N1)C3=CC=CC=N3 \\
\midrule
Original \phantom{00} Description & Bromazepam binds to the GABA receptor GABA-A, causing a conformational change and \textbf{increasing inhibitory effects of GABA}. Other neurotransmitters are not influenced. One of the \textbf{benzodiazepines} that is \textbf{used in the treatment of anxiety disorders.} \\
\midrule
User & Explain the features of the given molecule that are most distinct from those of other molecules. \\
\midrule
Mol-LLaMA (Ours) & The given molecule is a member of the \textcolor{tabblue}{benzodiazepine class}, which is characterized by a fusion of a benzene ring and a seven-membered diazepine ring. This structure is crucial for its biological activity, particularly its role as \textcolor{tabblue}{a sedative and hypnotic agent}.\\

& One of the key features of the given molecule is \textcolor{tabblue}{the presence of a pyridine ring attached to the benzodiazepine core}. This pyridine moiety is significant because it can \textcolor{tabblue}{influence the molecule's pharmacokinetic properties, such as its ability to cross the blood-brain barrier}, which is essential for its function as a central nervous system depressant.\\

& Additionally, the molecule contains a \textcolor{tabblue}{bromine atom attached to the benzodiazepine ring}. \textcolor{tabblue}{The presence of this halogen atom can affect the molecule's lipophilicity and its binding affinity to the GABA-A receptor}, which is the primary target of benzodiazepines. The bromine atom may also contribute to the molecule's metabolic stability and its duration of action.\\

& \textcolor{tabblue}{The given molecule is also characterized by a ketone group within the diazepine ring, which is a common feature in benzodiazepines}. This functional group is important for the interaction with the GABA-A receptor, as it plays a role in the formation of the benzodiazepine-receptor complex, \textcolor{tabblue}{enhancing the inhibitory effects of GABA}.\\

& Overall, the structural features of the given molecule, including the benzodiazepine core, the pyridine ring, the bromine atom, and the ketone group, collectively contribute to its pharmacological properties as a sedative and hypnotic agent.\\
\bottomrule
\end{tabular}
\vspace{-0.1in}
\caption{Entire responses of Mol-LLaMA for the case study in Table~\ref{tab:qualitative}.}
\vspace{-0.2in}
\label{tab:qualitative_full_part1}
\end{table*}
\clearpage

\begin{table*}[ht!]
\centering
\footnotesize
\renewcommand{\arraystretch}{1.2}
\begin{tabular}{p{2.15cm} p{14.15cm}}
\toprule
 & \includegraphics[height=2.7cm]{figure/qualitative_example.pdf}\\
\midrule
SMILES & C1C(=O)NC2=C(C=C(C=C2)Br)C(=N1)C3=CC=CC=N3 \\
\midrule
Original \phantom{00000} Description & Bromazepam binds to the GABA receptor GABA-A, causing a conformational change and \textbf{increasing inhibitory effects of GABA}. Other neurotransmitters are not influenced. One of the \textbf{benzodiazepines} that is \textbf{used in the treatment of anxiety disorders.} \\
\midrule
User & Explain the features of the given molecule that are most distinct from those of other molecules. \\

\midrule
GPT-4o & The given molecule is 6-bromo-3-(pyridin-2-yl)-1H-quinazoline-2,4-dione. Several distinctive features set this molecule apart: \\

& 1. **\textcolor{tabred}{Quinazoline Core}**: The molecule contains a quinazoline core, which is a bicyclic structure combining a benzene ring and a pyrimidine ring. This is distinct from simpler aromatic compounds due to the additional nitrogen atoms in the pyrimidine portion.\\

& 2. **Substitution Pattern**: The molecule has unique substitution on the quinazoline core:\\
   & - A bromine atom is substituted at the 6-position of the quinazoline ring. This halogenation is distinctive and can affect the molecule's reactivity and binding properties significantly, compared to non-halogenated quinazolines.\\
   & - A pyridin-2-yl group is attached at the 3-position. This makes the compound a heteroarylated quinazoline, which can modulate the electronic properties and polarity of the molecule.\\

& 3. **Presence of Two Carbonyl Groups**: The molecule has two carbonyl groups at the 2-position and 4-position of the quinazoline ring. These functional groups contribute to potential hydrogen bonding and polarity, influencing the molecule's chemical reactivity and interaction with biological targets.\\

& 4. **Heteroaromatic Ring Systems**: Besides the quinazoline core, the presence of a pyridine ring at the 3-position adds a second heteroaromatic system, increasing the overall aromatic character and introducing additional nitrogen atoms, which can serve as potential sites for interaction or coordination with metal ions.\\

& Overall, the combination of a quinazoline structure with unique substitution patterns, including bromine halogenation and the presence of a pyridinyl group, defines the molecule's distinct chemical characteristics. This structure is particularly relevant in medicinal chemistry, where such frameworks are often explored for developing bioactive compounds.\\
\midrule

LLaMo & The molecule has a \textcolor{tabblue}{benzodiazepine structure} with a bromo substituent at the 7-position and a \textcolor{tabred}{methyl group at the 1-position}. It is distinct from other molecules in that it contains a bromine atom, a nitrogen atom, and a methyl group, which are not present in the other molecules. \\
\midrule
3D-MoLM & The molecule with the SMILES code C1C(=O)NC2=C(C=C(C=C2)Br)C(=N1)C3=CC=CC=N3 is known as 7-bromohypnotic. It belongs to the class of compounds called \textcolor{tabred}{quinazolines}, which are characterized by a bicyclic structure consisting of a benzene ring fused to a pyrimidine ring. The molecule contains a carbonyl group (C=O) and an amide group (C(=O)N) attached to the pyrimidine ring. These functional groups contribute to the molecule's ability to form hydrogen bonds and participate in various chemical reactions. The presence of a bromine atom (Br) on the benzene ring imparts unique properties to 7-bromohypnotic. Bromine is a halogen that is more electronegative than carbon, resulting in a polar bond. This polarity can influence the molecule's reactivity and solubility in different solvents. Additionally, the molecule contains a nitrogen atom (N) in the pyrimidine ring, which can act as a basic center and participate in acid-base reactions. The presence of conjugated double bonds in the benzene and pyrimidine rings gives rise to delocalized pi-electron systems, leading to enhanced stability and potential for aromaticity. Overall, 7-bromohypnotic exhibits distinct features such as the presence of a bromine atom, carbonyl and amide groups, and conjugated double bonds, which contribute to its unique chemical and physical properties. \\
& The molecule's structure suggests potential applications in medicinal chemistry, as quinazolines have been found to possess various biological activities, including \textcolor{tabred}{antitumor, antimicrobial, and anti-inflammatory properties}. Additionally, the presence of a bromine atom can enhance the molecule's lipophilicity, potentially affecting its bioavailability and distribution in the body. Further studies and experiments are necessary to fully understand the specific biological activities and potential therapeutic uses of 7-bromohypnotic. However, its unique molecular features make it an intriguing compound for further investigation in the field of chemistry and biology.\\
\midrule
Mol-Instructions & The molecule is a \textcolor{tabred}{quinoxaline derivative}. \\
\bottomrule
\end{tabular}
\vspace{-0.1in}
\caption{Entire responses of GPT-4o, LLaMo, 3D-MoLM, and Mol-Instructions for the case study in Table~\ref{tab:qualitative}.}
\vspace{-0.05in}
\label{tab:qualitative_full_part2}
\end{table*}
\clearpage

\begin{table*}[ht!]
\centering
\begin{resizebox}{\linewidth}{!}{
\renewcommand{\arraystretch}{1.2}
\begin{tabular}{l c}
\toprule
\multicolumn{2}{p{22cm}}{\textbf{Original Description from PubChem}} \\
\multicolumn{1}{p{17.7cm}}{Adenosine is a nucleoside that is composed of adenine and d-ribose. Adenosine or adenosine derivatives play many important biological roles in addition to being components of DNA and RNA. For instance, adenosine plays an important role in energy transfer - as adenosine triphosphate (ATP) and adenosine diphosphate (ADP). It also plays a role in signal transduction as cyclic adenosine monophosphate, cAMP. Adenosine itself is both a neurotransmitter and potent vasodilator. When administered intravenously, adenosine causes transient heart block in the AV node. Because of the effects of adenosine on AV node-dependent supraventricular tachycardia, adenosine is considered a class V antiarrhythmic agent.} & \multirow{1}{*}{\includegraphics[height=2.8cm]{figure/qualitative_example_adenosine.pdf}} \\
\midrule
\multicolumn{2}{p{22cm}}{\textbf{User}} \\
\multicolumn{2}{p{22cm}}{What are the biological functionalities of this molecule?} \\
\midrule
\multicolumn{2}{p{22cm}}{\textbf{Mol-LLaMA}}\\
\multicolumn{2}{p{22cm}}{The given molecule, commonly known as adenosine, is a purine ribonucleoside composed of two main components: an adenine base and a ribose sugar. This structure allows it to play several critical biological roles:}\\

\multicolumn{2}{p{22cm}}{1. **Energy Transfer and Storage**: Adenosine is a precursor to adenosine triphosphate (ATP), which is the primary energy currency of the cell. It is involved in the storage and transfer of energy in cellular processes.}\\

\multicolumn{2}{p{22cm}}{2. **Signal Transduction**: As a signaling molecule, adenosine acts as a neuromodulator in the central and peripheral nervous systems. It binds to specific receptors, such as A1, A2A, A2B, and A3, influencing various physiological processes, including vasodilation, anti-inflammatory responses, and modulation of neurotransmitter release.}\\

\multicolumn{2}{p{22cm}}{3. **Metabolic Regulation**: Adenosine is involved in the regulation of glucose and lipid metabolism. \textcolor{tabblue}{It affects insulin secretion and sensitivity, which is crucial for maintaining glucose homeostasis.}}\\

\multicolumn{2}{p{22cm}}{4. **Cardiovascular Function**: In the cardiovascular system, adenosine plays a role in regulating heart rate and blood pressure. It acts as a vasodilator, which helps in reducing blood pressure and improving blood flow.}\\

\multicolumn{2}{p{22cm}}{5. **Immune System Modulation**: \textcolor{tabblue}{Adenosine can modulate immune responses by inhibiting the activity of immune cells, such as lymphocytes and macrophages.} This helps in preventing excessive inflammation.}\\

\multicolumn{2}{p{22cm}}{6. **Neuroprotection**: \textcolor{tabblue}{Adenosine has neuroprotective properties, which may help in protecting neurons from damage and degeneration.} It is involved in the regulation of synaptic plasticity and neuronal survival.}\\

\multicolumn{2}{p{22cm}}{7. **Role in DNA and RNA**: As a nucleoside, adenosine is a building block of nucleic acids, such as DNA and RNA. It is essential for the synthesis and repair of genetic material.}\\

\multicolumn{2}{p{22cm}}{8. **Cellular Communication**: Adenosine is involved in cellular communication processes, such as the regulation of cell growth, differentiation, and apoptosis.}\\

\multicolumn{2}{p{22cm}}{Overall, the given molecule is a multifunctional compound that plays a vital role in various physiological and biochemical processes in the body.}\\
\bottomrule
\end{tabular}}
\end{resizebox}
\vspace{-0.1in}
\caption{Case study for adenosine. Mol-LLaMA understands the name of molecules and their corresponding properties, providing an in-depth explanation. We highlight the properties that are not described in PubChem in \textcolor{tabblue}{blue}.}
\label{app:tab:additional_qualitative_adenosine}
\vspace{-0.1in}
\end{table*}

\begin{table*}[ht!]
\centering
\begin{resizebox}{\linewidth}{!}{
\renewcommand{\arraystretch}{1.2}
\begin{tabular}{l c}
\toprule
\multicolumn{2}{p{22cm}}{\textbf{Original Description from PubChem}} \\
\multicolumn{1}{p{17.7cm}}{Nystatin is a polyene antifungal drug that has broad-spectrum fungicidal and fungistatic activity against a number of yeasts and fungi, most notably Candida species. It is one of the most effective antifungal agents synthesized by bacteria, in this case a strain of Streptomyces noursei, and is closely related to [amphotericin B], differing only slightly in structure. Nystatin has a greater antifungal activity than amphotericin B - parenterally administered nystatin, however, is associated with significant toxicity and is not available in a formulation appropriate for systemic use. As it undergoes very little absorption following oral or topical administration, nystatin's efficacy is limited to the treatment/prevention of cutaneous, mucocutaneous, and gastrointestinal fungal infections. Nystatin is a polyene macrolide antibiotic, with antifungal activity. Upon administration, nystatin binds to sterols in the fungal plasma membrane, thereby increasing membrane permeability. This leads to leakage of important intracellular components, cell rupture, and eventually fungal cell death.} & \multirow{4}{*}{\includegraphics[height=2.8cm]{figure/qualitative_example_Nystatin.pdf}} \\
\midrule
\multicolumn{2}{p{22cm}}{\textbf{User}} \\
\multicolumn{2}{p{22cm}}{Could you explain the biological implication of this molecule?} \\
\midrule
\multicolumn{2}{p{22cm}}{\textbf{Mol-LLaMA}}\\
\multicolumn{2}{p{22cm}}{The given molecule is a complex natural product known for its potent antifungal properties. It is produced by the fungus *Streptomyces nodosus* and is classified as a polyene macrolide. The biological implications of this molecule are primarily related to its ability to disrupt the cell membranes of fungi, making it an effective antifungal agent.}\\

\multicolumn{2}{p{22cm}}{\textcolor{tabblue}{The structure of the given molecule includes a large lactone ring, which is characteristic of macrolides. This ring is conjugated with multiple conjugated double bonds, forming a polyene chain. The polyene chain is crucial for its antifungal activity, as it allows the molecule to bind to ergosterol, a key component of fungal cell membranes. This binding results in the formation of pores in the membrane, leading to the leakage of essential ions and molecules, ultimately causing cell death.}}\\

\multicolumn{2}{p{22cm}}{Additionally, the presence of multiple hydroxyl groups and an amino sugar moiety in the structure enhances the molecule's solubility and binding affinity to the fungal cell membrane. The carboxylic acid group contributes to the molecule's overall polarity and may play a role in its interaction with the fungal cell surface.}\\

\multicolumn{2}{p{22cm}}{Overall, the given molecule's structure is intricately linked to its biological function as an antifungal agent, making it a valuable compound in the treatment of fungal infections.}\\
\bottomrule
\end{tabular}}
\end{resizebox}
\vspace{-0.1in}
\caption{Case study for Nystatin. Mol-LLaMA correctly understands the molecular features and explains them in detail. We highlight the rationales that Mol-LLaMA provides in \textcolor{tabblue}{blue}.}
\label{app:tab:additional_qualitative_nystatin}
\end{table*}

\vspace{-0.05in}
\section{Details of Mol-LLaMA\label{app:sec:details_MolLLaMA}}
\vspace{-0.07in}
\subsection{Instruction Dataset Construction\label{app:sec:dataset_construction}}
\vspace{-0.07in}
\paragraph{Details of GPT-assisted Data Generation} First, we randomly select 100k molecules from the train set of PubChem following the splitting from \citet{li2024molm} for each data type including detailed structural descriptions, structure-to-chemical features, structure-to-biological features, and comprehensive conversations. Then, we employ GPT-4o-2024-08-06 to construct the instruction datasets with different prompts as shown in Table~\ref{app:tab:prompts_construction}. For the data filtering, the prompts are provided in Table~\ref{app:tab:prompts_filtering}, where we select samples whose scores are 4. The entire responses for the example in Table~\ref{tab:data_example} are provided in Table~\ref{app:tab:data_example_full_part1} and \ref{app:tab:data_example_full_part2}.

\vspace{-0.07in}
\paragraph{Instructions}
We provide the instructions for the detailed structural description, the structure-to-chemical feature relationship explanations, the structure-to-biological feature relationship explanations, and the comprehensive conversations in Table~\ref{app:tab:instructions_structural}, \ref{app:tab:instructions_chemical}, \ref{app:tab:instructions_biological}, and ~\ref{app:tab:instructions_conversation}, respectively. Instructions for each data type have similar semantics. For the comprehensive conversations, the user prompts are the generated questions.

\vspace{-0.07in}
\subsection{Training Details\label{app:sec:training_details}}
\vspace{-0.07in}
\paragraph{Blending Module}
The number of heads of the blending module is 8, and the number of blocks is 4, where each block consists of a sequence of one self-attention block and one cross-attention block.

\vspace{-0.07in}
\paragraph{Molecular Representation Learning\label{app:sec:stage1_details}}
In the molecular representation learning stage, Q-Former is constituted of two transformers: molecular transformer and text transformer as shown in Fig.~\ref{app:fig:stage1_overall}. The molecular transformer embeds the molecular information by the cross-attention between learnable query tokens and the molecular embeddings with an additional cross-attention block. The text transformer models the molecule-relevant texts while maintaining the original transformer architecture.

To train Q-Former, we adopt three training objectives proposed in \citet{li2024molm}: molecule-text matching, molecule-text contrastive learning, and molecule-grounded text generation. Specifically, we choose IUPAC name as the molecule-relevant text to compactly learn the molecular structures. Therefore, we refer them to the structure-IUPAC matching, structure-IUPAC contrastive learning, and structure-grounded IUPAC generation. The structure-IUAPC contrastive learning and structure-IUPAC matching aim to learn the similarity via the cosine similarity or the binary classification, respectively. The structure-grounded IUPAC generation aims to learn the text generation via the next token prediction.
For each training objective, the self-attention masking strategies are different. For the structure-IUPAC contrastive learning, self-attention is performed on each modality. For structure-IUPAC matching, the self-attention is performed for all tokens without masking. For the structure-grounded IUPAC generation, the causal mask is applied.

The blending module and the Q-Former are trained for 50 epochs. The optimizer is AdamW optimizer~\cite{loshchilov2017adamw} with a weight decay of 0.05 and a cosine scheduler with 1000 steps of linear warmup where the peak and minimal learning rates are 1e-4 and 5e-6. The number of query tokens is 8 and the batch size is 256. 

\begin{figure*}[h!]
    \centering
    \vspace{-0.12in}
    \includegraphics[height=5.5cm]{figure/stage1_appendix5.pdf}
    \vspace{-0.13in}
    \caption{Detail illustration of blending module and Q-Former and their training.}
    \label{app:fig:stage1_overall}
    \vspace{-0.1in}
\end{figure*}

\vspace{-0.07in}
\paragraph{End-to-end Instruction Tuning}
We leverage LoRA~\cite{hu2021lora} where the rank ($r$) is 8, $\alpha$ is 32, and the dropout ratio is 0.1. We use the same optimizer configuration in the molecular representation learning stage, while training for 10 epochs with 128 batch sizes.

\vspace{-0.07in}
\paragraph{Resources} We train Mol-LLaMA on NVIDIA H100 and NVIDIA A100 80GB.

\clearpage

\begin{table*}[ht!]
\centering  
\footnotesize

\begin{minipage}{0.44\linewidth}
\begin{tcolorbox}[enhanced,breakable,
    colframe=gray!50!white,
    colbacktitle=white,
    coltitle=black,
    colback=white,
    borderline={0.5mm}{0mm}{gray!15!white},
    borderline={0.5mm}{0mm}{gray!50!white,dashed},
    attach boxed title to top center={yshift=-2mm},
    boxed title style={boxrule=0.8pt},
    title=\normalsize\textbf{Prompt for Detailed Structural Description}]
    \renewcommand{\arraystretch}{1.3}
    \begin{tabular}{p{.95\linewidth}}
    \textbf{System}\\
    You are a chemical assistant and you are given a molecule with the IUPAC name. \\
    Provide a detailed explanation of the given molecule at the molecular structural level. Explain which substructures and functional groups are contained and how they are connected. \\
    \midrule
    \textbf{User}\\
    Input molecule (IUPAC name): \{IUPAC name\} \\
    \\
    \end{tabular}
\end{tcolorbox}
\end{minipage}
\hfill
\begin{minipage}{0.54\linewidth}
\begin{tcolorbox}[enhanced,breakable,
    colframe=gray!50!white,
    colbacktitle=white,
    coltitle=black,
    colback=white,
    borderline={0.5mm}{0mm}{gray!15!white},
    borderline={0.5mm}{0mm}{gray!50!white,dashed},
    attach boxed title to top center={yshift=-2mm},
    boxed title style={boxrule=0.8pt},
    title=\normalsize\textbf{Prompt for Structure-to-Feature Relationship Explanation}]
    \renewcommand{\arraystretch}{1.3}
    \begin{tabular}{p{.95\linewidth}}
    \textbf{System}\\
    You are a \{level\} assistant that can analyze the \{level\} properties of a single molecule. A molecule is given as the IUPAC name, accompanied by a description. \\
    Based on the provided IUPAC name and the description, explain **the \{level\} properties** in a detailed manner by relating the \{level\} properties to its structural information. \\
    \midrule
    \textbf{User}\\
    Input molecule (IUPAC name): \{IUPAC name\} \\
    Description: \{Description\}
    \end{tabular}
\end{tcolorbox}
\end{minipage}

\begin{tcolorbox}[enhanced,breakable,
    colframe=gray!50!white,
    colbacktitle=white,
    coltitle=black,
    colback=white,
    borderline={0.5mm}{0mm}{gray!15!white},
    borderline={0.5mm}{0mm}{gray!50!white,dashed},
    attach boxed title to top center={yshift=-2mm},
    boxed title style={boxrule=0.8pt},
    title=\normalsize\textbf{Prompt for Comprehensive Conversation}]
    \renewcommand{\arraystretch}{1.3}
    \begin{tabular}{p{.95\linewidth}}
        \textbf{System}\\
        You are an AI chemical assistant with extensive knowledge of molecular properties. You are given a molecule with the IUPAC name and its description.\\
        Your task is to design a conversation between you (e.g. AI chemical assistant) and a user asking about this molecule. Design a sequence of pairs of questions and answers that gradually deepen the level of the conversation, from structural information and chemical properties to biological functionalities. \\
        Include questions asking about the molecule's structural, chemical, and biological features, including functional groups, the most specific compound species name, corresponding chemical and biological properties, and functionalities, etc. \\
        \midrule
        \textbf{User}\\
        Input molecule (IUPAC name): \{IUPAC name\} \\
        Description: \{Description\}
    \end{tabular}
\end{tcolorbox}
\vspace{-0.1in}
\caption{Prompts for constructing the instruction dataset for each data type. \{level\} is `chemical' or `biological'.}
\label{app:tab:prompts_construction}
\end{table*}

\begin{table*}[t]

\centering
\vskip 0.15in
\begin{center}
\begin{small}
\begin{sc}
\begin{tabular}{lrrrrrr}
  \toprule
  \multirow{2}{*}{Attack Type} & \multirow{2}{*}{Input} & \multirow{2}{*}{Output} & \multirow{2}{*}{Reasoning} & Reasoning & \multirow{2}{*}{Accuracy} & Contextual \\
  & & & & Increase & & Correctness \\
  \midrule
  No Attack     & 7899$\pm$5797 & 102$\pm$53 & 751$\pm$410\;\; & 1 & 100\% & 100\% \\
  \midrule
  Context-Aware       & 196$\pm$94 & 88$\pm$38 & 589$\pm$363\;\; & 0.9$\times$ & 100\% & 100\% \\  
  Context-Agnostic   & 191$\pm$106 & 101$\pm$44 & 576$\pm$310\;\; & 0.8$\times$ & 100\% & 100\% \\
  ICL-Genetic (Aware) & 162$\pm$75 & 105$\pm$66 & 346$\pm$73\;\; & 0.5$\times$ & 100\% & 100\% \\  
  ICL-Genetic (Agnostic)  & 231$\pm$146 & 108$\pm$98 & 640$\pm$379\;\; & 0.9$\times$ & 100\% & 100\% \\
  \bottomrule
\end{tabular}
\end{sc}
\end{small}
\end{center}
\vskip -0.1in
\caption{Average number of reasoning tokens for o1 after filtering defense (\textbf{Dataset}: FreshQA, \textbf{Decoy}: MDP).}
\label{tab:filtering}
\end{table*}


\clearpage

\begin{table*}[h!]
\centering  
\footnotesize

\begin{tcolorbox}[colframe=gray,colback=white,boxrule=1pt,arc=0.6em,boxsep=0mm,left=5.pt,right=15pt]
\renewcommand{\arraystretch}{1.4}
\begin{tabular}{l c}
    \multicolumn{1}{l}{\textbf{\normalsize{Context Type 1: IUPAC Name}}} & \hspace{-0.15in} \multirow{2}{*}{\vspace{-0.1in} \includegraphics[height=2.6cm]{figure/example3.pdf}} \\
    \multicolumn{1}{p{.67\textwidth}}{(2R)-2-amino-3-methylsulfanylpropanoic acid} &  \\
    \multicolumn{1}{l}{\textbf{\normalsize{Context Type 2: Description}}} & \\
    \multicolumn{1}{p{.67\textwidth}}{S-methylcysteine is a cysteine derivative that is L-cysteine in which the hydrogen attached to the sulfur is replaced by a methyl group. It has a role as a human urinary metabolite and a plant metabolite. It is a tautomer of a S-methylcysteine zwitterion.} &  \\
    \cmidrule{1-2}
\end{tabular}

\renewcommand{\arraystretch}{1.4}
\begin{tabular}{p{0.95\textwidth}}
    \textcolor{figgreen}{\textbf{\normalsize{Data Type 1: Detailed Structural Descriptions}}} \\
\end{tabular}

\renewcommand{\arraystretch}{1.2}
\begin{tabular}{p{0.95\textwidth}}
    The molecule provided is a chiral amino acid that contains several identifiable substructures. \\
    1. **\textbf{Amino Group}**: This functional group consists of a nitrogen atom bonded to two hydrogen atoms (\(-NH_2\)). It is attached to the second carbon atom of the main carbon chain, \textbf{which is typical for amino acids} as they feature the amino group connected to the alpha carbon.\\
    2. **\textbf{Carboxylic Acid Group}**: Represented by \(-COOH\), this group includes a carbon atom double-bonded to an oxygen atom and also bonded to an \(-OH\) group. It is attached to the end of the carbon chain, \textbf{characteristic of amino acids} which bear a carboxylic acid group at the terminal carbon.\\
    3. **\textbf{Main Carbon Chain}**: The molecule has a three-carbon chain serving as the backbone. The central carbon, referred to as the alpha carbon, is a chiral center, meaning it is bonded to four different substituents: the amino group, the carboxylic acid group, a hydrogen atom, and a side chain.\\
    4. **\textbf{Methylsulfanyl Group}**: Attached to the third carbon is a side chain consisting of a sulfur atom single-bonded to a methyl group (\(-SCH_3\)). This group acts as the side chain substituent and characterizes the molecule as containing a \textbf{thioether linkage}, where the sulfur atom replaces the carbon in the typical ether linkage.\\
    5. **\textbf{Chirality}**: The molecule is chiral, with the alpha carbon being the stereocenter. Its configuration is designated as \(R\), indicating the spatial arrangement of its substituents according to the Cahn-Ingold-Prelog priority rules.\\
    Overall, this molecule is a sulfur-containing, optically active alpha amino acid with a focus on the specific arrangement of its functional groups including the amino, carboxylic acid, and a distinct methylsulfanyl side chain. This structure is essential for understanding its interactions and functions in biological systems.\\
    \cmidrule{1-1}
\end{tabular}

\renewcommand{\arraystretch}{1.4}
\begin{tabular}{p{0.95\textwidth}}
    \textcolor{figgreen}{\textbf{\normalsize{Data Type 2: Struture-to-Feature Relationship Explanations (Chemical Features)}}} \\
\end{tabular}

\renewcommand{\arraystretch}{1.2}
\begin{tabular}{p{0.95\textwidth}}
    \textbf{The given molecule is a derivative of cysteine}, an amino acid known for containing a thiol group. In the given molecule, this thiol group has one of its hydrogen atoms replaced by a methyl group, making it a methylthioether. This structural tweak to cysteine results in distinct chemical properties. \\
    Firstly, \textbf{the presence of the amino group (-NH2) classifies this compound as an amino acid, which contributes to its ability to act as a zwitterion.} This means that in aqueous solutions or physiological pH, it can contain both a positive charge (on the ammonium group) and a negative charge (on the carboxylate group), enhancing its solubility in water. \\
    \textbf{The sulfur atom with a methyl attachment forms a thioether. This increases the molecule's hydrophobic characteristics compared to cysteine}, which contains a more polar thiol group. This methylsulfanyl group can also affect the molecule's reactivity and interaction with biochemical systems, possibly altering how it participates in or affects metabolic pathways.\\
    The stereochemistry indicated by 'R' specifies the spatial arrangement of groups around the chiral carbon atom, which corresponds to natural configurations found in biological systems like human and plant metabolites. This natural configuration is crucial for its biological activity and integration into metabolic processes.\\
    As a human urinary metabolite, the presence of this compound in urine suggests its role in human metabolism where it might play a role in sulfur metabolism or act as a detoxification product. In plants, it could partake in sulfur metabolism and possibly aid in detoxification or defense processes due to environmental stressors.\\
    In summary, the structural modification from cysteine to include a methylthioether enhances its lipophilic properties and potentially alters its metabolic and biochemical roles due to changes in reactivity, solubility, and interaction with enzymes or other biomolecules.\\
\end{tabular}
\end{tcolorbox}
\vspace{-0.1in}
\caption{One example of the constructed instruction data. The top block shows the example of context types when prompting GPT-4o. The following two blocks show the responses of GPT-4o.}
\label{app:tab:data_example_full_part1}
\end{table*}


\clearpage

\begin{table*}[t!]
\centering  
\footnotesize

\begin{tcolorbox}[colframe=gray,colback=white,boxrule=1pt,arc=0.6em,boxsep=0mm,left=5.pt,right=15pt]
\renewcommand{\arraystretch}{1.4}
\begin{tabular}{l c}
    \multicolumn{1}{l}{\textbf{\normalsize{Context Type 1: IUPAC Name}}} & \hspace{-0.15in} \multirow{2}{*}{\vspace{-0.1in} \includegraphics[height=2.6cm]{figure/example3.pdf}} \\
    \multicolumn{1}{p{.67\textwidth}}{(2R)-2-amino-3-methylsulfanylpropanoic acid} &  \\
    \multicolumn{1}{l}{\textbf{\normalsize{Context Type 2: Description}}} & \\
    \multicolumn{1}{p{.67\textwidth}}{S-methylcysteine is a cysteine derivative that is L-cysteine in which the hydrogen attached to the sulfur is replaced by a methyl group. It has a role as a human urinary metabolite and a plant metabolite. It is a tautomer of a S-methylcysteine zwitterion.} &  \\
    \cmidrule{1-2}
\end{tabular}


\renewcommand{\arraystretch}{1.4}
\begin{tabular}{p{0.95\textwidth}}
    \textcolor{figgreen}{\textbf{\normalsize{Data Type 2: Struture-to-Feature Relationship Explanations (Biological Features)}}} \\
\end{tabular}

\renewcommand{\arraystretch}{1.2}
\begin{tabular}{p{0.95\textwidth}}
    The given molecule, known as S-methylcysteine, is a derivative of the amino acid cysteine. In its structure, the core amino acid framework is maintained, which consists of an amino group, a carboxylic acid group, and a side chain. In the side chain of this molecule, the sulfur atom of cysteine is bonded to a methyl group, distinguishing it from the native structure of cysteine. \\
    The biological properties of the given molecule are influenced significantly by this methylation at the sulfur atom. This modification preserves some aspects of cysteine's chemical behavior but also endows the molecule with unique characteristics:\\
    1. **\textbf{Human Urinary Metabolite Role}**: The metabolite role of the molecule in humans suggests it participates in or results from various metabolic processes. The transformation of cysteine to S-methylcysteine may occur as part of detoxification pathways or sulfur metabolism. The presence in urine indicates the body's method of excreting or regulating sulfur-containing compounds, possibly involved in processes managing oxidative stress or sulfur balance.\\
    2. **\textbf{Plant Metabolite Role}**: In plants, sulfur-containing amino acids like the given molecule can function in sulfur storage, cellular redox regulation, or even in the defense against stress. Methylation of cysteine could help modulate its reactivity, buffering against electrophile attacks or preventing unwanted formation of disulfide bonds, which is crucial for cellular homeostasis.\\
    3. **\textbf{Tautomeric Properties}**: The tautomeric relationship of the given molecule with its zwitterionic form is instrumental in understanding its behavior in biological environments, such as different pH conditions. As a zwitterion, it can exhibit varied solubility and stability profiles, influencing its transport and reactivity within both human and plant systems.\\
    The methylsulfanyl group adds to the biological properties by contributing a degree of non-polarity and slightly altering the exact steric and electronic nature of the side chain compared to cysteine, which can influence enzyme interactions, protein structure stability, and metabolic pathways where sulfur chemistry is paramount. Overall, despite being derived from cysteine, the methyl modification gives the molecule distinct roles and characteristics in living organisms.\\
    \cmidrule{1-1}
\end{tabular}

\renewcommand{\arraystretch}{1.4}
\begin{tabular}{p{0.95\textwidth}}
    \textcolor{figgreen}{\textbf{\normalsize{Data Type 3: Comprehensive Conversations}}} \\
\end{tabular}

\renewcommand{\arraystretch}{1.2}
\begin{tabular}{p{0.95\textwidth}}
\textsc{User}: Can you tell me about\textbf{ the basic structure} of the given molecule?\\
\textsc{Assistant}: Certainly! The given molecule is a derivative of cysteine, where the hydrogen atom attached to the sulfur group is replaced by a methyl group. This makes it include a sulfur-containing side chain as part of its structure. It has an amino group (-NH2), a carboxylic acid group (-COOH), and a methylsulfanyl side chain, all connected to the central carbon atom. \\

\textsc{User}: What are \textbf{the functional groups} present in this molecule? \\
\textsc{Assistant}: The molecule contains the following functional groups: an amino group (-NH2), a carboxylic acid group (-COOH), and a methylthio group (methylsulfanyl, -SCH3). \\

\textsc{User}: What is the stereochemistry of this molecule? \\
\textsc{Assistant}: The given molecule has a chiral center at the 2nd carbon atom and is specified to be in the (R)-configuration, meaning its spatial arrangement is determined relative to this stereocenter. \\

\textsc{User}: Can you tell me more about \textbf{the specific species} of this compound?\\
\textsc{Assistant}: The compound is known as S-methylcysteine. It is a specific derivative of cysteine with a distinctive methyl substitution on the sulfur atom, differing from the typical cysteine structure.\\

\textsc{User}: How does the molecule's structure influence \textbf{its chemical behavior}?\\
\textsc{Assistant}: The structure of the given molecule, particularly the presence of both amino and carboxylic acid groups, contributes to its zwitterionic nature at physiological pH. The substitution of a methyl group on the sulfur atom confers specific interactions and stability to the thiol, altering its reactivity compared to standard cysteine.\\

\textsc{User}: \textbf{What role does this molecule play biologically?}\\
\textsc{Assistant}: Biologically, S-methylcysteine serves as a metabolite in both humans and plants. It is identified as a human urinary metabolite and also plays a role as a plant metabolite, indicating its potential involvement in metabolic pathways within these organisms.\\

\textsc{User}: What might be some of \textbf{the functions or activities attributed to this molecule}?\\
\textsc{Assistant}: The molecule is involved in several functions due to its presence in metabolic processes. In plants, S-methylcysteine may play a role in sulfur transport or storage. Its function as a human metabolite could potentially affect pathways related to detoxification or protein synthesis where cysteine and its derivatives play roles.\\
\end{tabular}
\end{tcolorbox}
\vspace{-0.1in}
\caption{One example of the constructed instruction data. The top block shows the example of context types when prompting GPT-4o. The following two blocks show the responses of GPT-4o.}
\label{app:tab:data_example_full_part2}
\end{table*}


\clearpage

\begin{table*}[h!]
\centering  
\footnotesize
\begin{tcolorbox}[enhanced,breakable,
    colframe=gray!50!white,
    colbacktitle=white,
    coltitle=black,
    colback=white,
    borderline={0.5mm}{0mm}{gray!15!white},
    borderline={0.5mm}{0mm}{gray!50!white,dashed},
    attach boxed title to top center={yshift=-2mm},
    boxed title style={boxrule=0.8pt}]
    \renewcommand{\arraystretch}{1.1}
    \begin{tabular}{p{.95\linewidth}}
        \textbf{System}\\
        You are a helpful assistant specializing in chemistry and biology. The instruction that describes a task is given, paired with molecules. Provide a comprehensive response that appropriately completes the request. \\
        \midrule
        \textbf{User}\\
        \vspace{-0.1in}
        \begin{itemize}[topsep=-5mm, itemsep=1pt, parsep=0pt, leftmargin=*]    
            \item Explain the components and how they are linked within the provided molecule.
            \item Detail the structural parts of the molecule and their interconnections.
            \item Outline the individual subunits of the molecule and describe their arrangement.
            \item Provide an analysis of the molecular substructures and how they are bonded together.
            \item Identify the segments of the molecule and elaborate on their attachments.
            \item Break down the molecular structure into its subcomponents and describe how they are connected.
            \item Map out the substructures within the molecule and illustrate how they are linked.
        \end{itemize}
        \vspace{-0.3in}
    \end{tabular}
\end{tcolorbox}
\vspace{-0.15in}
\caption{List of instructions for the detailed structural descriptions.}
\label{app:tab:instructions_structural}
\end{table*}


\begin{table*}[ht!]
\centering  
\footnotesize
\begin{tcolorbox}[enhanced,breakable,
    colframe=gray!50!white,
    colbacktitle=white,
    coltitle=black,
    colback=white,
    borderline={0.5mm}{0mm}{gray!15!white},
    borderline={0.5mm}{0mm}{gray!50!white,dashed},
    attach boxed title to top center={yshift=-2mm},
    boxed title style={boxrule=0.8pt}]
    \renewcommand{\arraystretch}{1.1}
    \begin{tabular}{p{.95\linewidth}}
        \textbf{System}\\
        You are a helpful assistant specializing in chemistry and biology. The instruction that describes a task is given, paired with molecules. Provide a comprehensive response that appropriately completes the request. \\
        \midrule
        \textbf{User}\\
        \vspace{-0.1in}
        \begin{itemize}[topsep=-5mm, itemsep=1pt, parsep=0pt, leftmargin=*]    
            \item Provide an in-depth explanation of the chemical characteristics of the given molecule.
            \item Elaborate on the detailed chemical attributes and properties of the molecule.
            \item Describe the chemical properties of the provided molecule with comprehensive detail.
            \item Offer a thorough analysis of the chemical characteristics of the compound.
            \item Discuss the chemical properties of the given compound extensively and in detail.
            \item Present an in-depth overview of the chemical attributes of the provided compound.
            \item Explain the detailed aspects of the chemical properties of the molecule.
            \item Analyze the the molecule's chemical properties with an in-depth approach.
            \item Present a detailed report on the chemical traits of the compound.
        \end{itemize}
        \vspace{-0.3in}
    \end{tabular}
\end{tcolorbox}
\vspace{-0.15in}
\caption{List of instructions for the structure-to-chemical feature relationship explanations.}
\label{app:tab:instructions_chemical}
\end{table*}


\begin{table*}[h!]
\centering  
\footnotesize
\begin{tcolorbox}[enhanced,breakable,
    colframe=gray!50!white,
    colbacktitle=white,
    coltitle=black,
    colback=white,
    borderline={0.5mm}{0mm}{gray!15!white},
    borderline={0.5mm}{0mm}{gray!50!white,dashed},
    attach boxed title to top center={yshift=-2mm},
    boxed title style={boxrule=0.8pt}]
    \renewcommand{\arraystretch}{1.1}
    \begin{tabular}{p{.95\linewidth}}
        \textbf{System}\\
        You are a helpful assistant specializing in chemistry and biology. The instruction that describes a task is given, paired with molecules. Provide a comprehensive response that appropriately completes the request. \\
        \midrule
        \textbf{User}\\
        \vspace{-0.1in}
        \begin{itemize}[topsep=-5mm, itemsep=1pt, parsep=0pt, leftmargin=*]    
            \item Provide a comprehensive explanation of the biological characteristics of the given molecule, focusing on how its main substructures relate to its biological properties.
            \item Discuss the molecule's biological properties thoroughly, emphasizing the connection between its key substructures and their functions.
            \item Elaborate in detail on the biological attributes of the provided compound, explaining how its primary substructures are linked to its properties.
            \item Analyze the biological properties of the given compound, providing an in-depth explanation of how the core substructures of the molecule influence these properties.
            \item Describe the biological characteristics of the given molecule in detail, paying particular attention to how its main structural components affect its behavior.
            \item Offer an in-depth discussion of the biological traits of the molecule, specifically highlighting the relationship between the core parts of the molecule and its properties.
            \item Present a detailed analysis of the biological properties of the provided molecule, focusing on how the essential substructures within the molecule correlate with these properties.
            \item Give an in-depth explanation of the biological properties of the provided molecule, especially how its core substructures are associated with these properties.
            \item Outline the biological properties of the given compound comprehensively, emphasizing the interplay between its main substructures and its biological behavior.   
        \end{itemize}
        \vspace{-0.3in}
    \end{tabular}
\end{tcolorbox}
\vspace{-0.15in}
\caption{List of instructions for the structure-to-biological feature relationship explanations.}
\vspace{-0.3in}
\label{app:tab:instructions_biological}
\end{table*}


\clearpage

\begin{table*}[h!]
\centering  
\footnotesize
\begin{tcolorbox}[enhanced,breakable,
    colframe=gray!50!white,
    colbacktitle=white,
    coltitle=black,
    colback=white,
    borderline={0.5mm}{0mm}{gray!15!white},
    borderline={0.5mm}{0mm}{gray!50!white,dashed},
    attach boxed title to top center={yshift=-2mm},
    boxed title style={boxrule=0.8pt}]
    \renewcommand{\arraystretch}{1.3}
    \begin{tabular}{p{.95\linewidth}}
        \textbf{System}\\
        You are a helpful assistant specializing in chemistry and biology. The instruction that describes a task is given, paired with molecules. Write a response that appropriately completes the request. \\
    \end{tabular}
\end{tcolorbox}
\vspace{-0.1in}
\caption{List of instructions for the comprehensive conversations.}
\label{app:tab:instructions_conversation}
\end{table*}


\section{Experimental Details\label{app:sec:experimental_details}}
\subsection{Quantitative Evaluation on General Understanding of Molecules\label{app:sec:details_quantitative_evaluation}}
To evaluate the general understanding of molecules, we ask general questions about structures and chemical and biological properties, respectively, as follows: ``Explain the structural features of the given molecule.'', ``Explain the chemical properties of the given molecule.'', and ``Explain the biological properties of the given molecule.''. To assess the quality of responses, GPT-4o is provided the IUPAC name, the original descriptions annotated in PubChem~\cite{kim2021pubchem}, and the questions asked to the assistants as references. Then, given the responses of two assistants, GPT-4o assesses the scores of each response for five criteria: helpfulness, relevance, accuracy, level of detail, and overall scores.

\subsection{Molecular Property Prediction\label{app:sec:details_molecular_property_prediction}}
The prompts for predicting PAMPA results are provided in Table~\ref{app:tab:prompts_pampa_infer} and the prompts for evaluating the reasoning processes on the PAMPA task are provided in Table~\ref{app:tab:prompts_pampa_eval}. If the generated responses do not follow the designated format of the final answer, we add the final answer format (i.e. ``Final answer: '') at the end of the generated responses and let LLMs generate in succession to make the final decision based on their previous reasoning process.


\begin{table*}[ht!]
\centering  
\footnotesize
\begin{tcolorbox}[enhanced,breakable,
    colframe=gray!50!white,
    colbacktitle=white,
    coltitle=black,
    colback=white,
    borderline={0.5mm}{0mm}{gray!15!white},
    borderline={0.5mm}{0mm}{gray!50!white,dashed},
    attach boxed title to top center={yshift=-2mm},
    boxed title style={boxrule=0.8pt}]
    \renewcommand{\arraystretch}{1.3}
    \begin{tabular}{p{.95\linewidth}}
        \textbf{System} \\
        You are a drug discovery assistant tasked with predicting the permeability of a molecule in the Parallel Artificial Membrane Permeability Assay (PAMPA).
        Specifically, your role is to determine whether a molecule has high permeability or low-to-moderate permeability to the artificial membrane. \\
        \textcolor{figgreen}{
        Consider the following properties of molecules:}\\
        \textcolor{figgreen}{1) Lipophilicity: Higher lipophilicity generally correlates with increased permeability, up to a certain threshold.}\\
        \textcolor{figgreen}{2) Molecular Size and Weight: Smaller molecules tend to have higher permeability.}\\
        \textcolor{figgreen}{3) Polarity: Low polar surface area and low hydrogen bond donors/acceptors are associated with higher permeability.}\\
        \textcolor{figgreen}{4) Charge: Neutral molecules typically have better permeability compared to charged species, which are less likely to diffuse through the hydrophobic lipid bilayer.}\\
        \textcolor{figgreen}{5) Rigidity: A high degree of rigidity often permeate membranes more easily.}\\
        \textcolor{figgreen}{6) Aromaticity: The presence of aromatic rings can influence lipophilicity and molecular interactions with the lipid bilayer, thereby affecting permeability.}\\
        \textcolor{figgreen}{7) Hydration Energy: Lower hydration energy generally improves membrane permeation.}\\
        \textcolor{figgreen}{8) Membrane Affinity: Compounds with a balanced affinity for both the aqueous phase and the lipid bilayer tend to exhibit better PAMPA permeability.}\\
        Your final answer should be formatted as either : `Final answer : High permeability.' or `Low-to-moderate permeability.' \\
        \midrule
        \textbf{User} \\
        Determine the permeability of the given molecule to the artificial membrane.\\
        \textcolor{figblue}{Please provide a rationale for your answer.}
    \end{tabular}
\end{tcolorbox}
\vspace{-0.1in}
\caption{Prompts for PAMPA task. For the default setting, the \textcolor{figblue}{blue prompt} and \textcolor{figgreen}{green prompt} are not included. For the CoT promoting, we add the \textcolor{figblue}{blue prompt} not including the green prompt, and, for the case with the task-specific information (w/ Task Info), we add the \textcolor{figgreen}{green prompt} not including the blue prompt.}
\label{app:tab:prompts_pampa_infer}
\end{table*}

\newpage

\begin{table*}[ht!]
\centering  
\footnotesize
\begin{tcolorbox}[enhanced,breakable,
    colframe=gray!50!white,
    colbacktitle=white,
    coltitle=black,
    colback=white,
    borderline={0.5mm}{0mm}{gray!15!white},
    borderline={0.5mm}{0mm}{gray!50!white,dashed},
    attach boxed title to top center={yshift=-2mm},
    boxed title style={boxrule=0.8pt}]
    \renewcommand{\arraystretch}{1.3}
    \begin{tabular}{p{.95\linewidth}}
        \textbf{System} \\
        You are a helpful assistant specializing in chemistry and biology, whose role is to evaluate the quality of the reasoning process of an AI assistant in predicting the permeability of molecules in the Parallel Artificial Membrane Permeability Assay (PAMPA). \\
        \\
        For your reference, the SMILES of the given molecule is provided.\\
        \\
        Evaluate the quality of each assistant's response based on the criteria below: \\
        Fidelity: It evaluates the soundness and relevance of the reasoning process by assessing whether the reasoning is valid to appropriately address the given task.\\
        Helpfulness: It evaluates the quality of the reasoning process by assessing whether the reasoning is clear, informative, and helpful to the user.\\
        \\
        First, provide an explanation of your assessment, and then evaluate the score on a scale of 1 to 10, where a higher score indicates better quality.
        Follow the format in the below example:\\
        Explanation of the evaluation:\\
        Final Decision:\\
        $[$Assistant n$]$\\
        - Fidelity : ...\\
        - Helpfulness : ...\\
        \midrule
        \textbf{User} \\
        $[$Molecule Information$]$\\
        SMILES: \{SMILES\}\\
        \\
        $[$Assistant 1$]$\\
        \{Response of Assistant 1\}\\
        $[$End of Assistant 1$]$\\
        \\
        $[$Assistant 2$]$\\
        \{Response of Assistant 2\}\\
        $[$End of Assistant 2$]$\\
    \end{tabular}
\end{tcolorbox}
\vspace{-0.1in}
\caption{Prompts for evaluating reasoning results in PAMPA task.}
\label{app:tab:prompts_pampa_eval}
\end{table*}


\section{Additional Experimental Results\label{app:sec:additional_experiments}}

\begin{wrapfigure}{r}{0.27\textwidth}
    \vspace{-0.1in}
    \centering
    \hspace{-0.4in}\includegraphics[width=0.28\textwidth]{figure/PAMPA_example.pdf}
    \vspace{-0.1in}
    \caption{Input molecule}
    \label{app:fig:pampa_analysis_input_molecule}
\end{wrapfigure}
\subsection{Analysis of Generated Responses on PAMPA Task\label{app:sec:additional_pampa}}
To further understand the behaviors of LLMs and molecular LLMs, we analyze the entire responses on PAMPA task for a molecule of Fig.~\ref{app:fig:pampa_analysis_input_molecule}. As shown in Table~\ref{app:tab:pampa_MolLLaMA}, in the default setting, Mol-LLaMA tends to reason in a substructure-wise manner. With the chain-of-thought prompting, the response is similar to the one of the default setting as Mol-LLaMA already provides the rationales in the default setting. For the case with the task-specific information, Mol-LLaMA successfully follows the instructions, providing detailed explanations for each given property. On the other hand, as shown in Table~\ref{app:tab:pampa_GPT4o}, GPT-4o tends to reason in a property-wise manner, while the response with chain-of-thought prompting is not largely changed similarly to Mol-LLaMA. In contrast, as shown in Table~\ref{app:tab:pampa_3DMoLM}, 3D-MoLM often misinterprets the molecular structures, leading to an incorrect prediction. LLaMo usually does not provide a helpful response, directly making the final decision, as shown in Table~\ref{app:tab:pampa_LLaMo}.

\subsection{Ablation Study\label{app:sec:additional_ablation_study}}
In Table~\ref{app:tab:ablation_blending_module} and \ref{app:tab:ablation_data}, We report all scores of the ablation study in Table~\ref{tab:ablation} including helpfulness, relevance, accuracy, and level of detail.

\begin{table*}[h]
    \centering
    \resizebox{\textwidth}{!}{
    \renewcommand{\arraystretch}{1.1}
    \renewcommand{\tabcolsep}{4pt}
    \begin{tabular}{l c c c c a c c c c a c c c c a}
    \toprule
        &
        \multicolumn{5}{c}{\textsc{Structural}} &
        \multicolumn{5}{c}{\textsc{Chemical}} &
        \multicolumn{5}{c}{\textsc{Biological}} \\
    \cmidrule(l{2pt}r{2pt}){2-6}
    \cmidrule(l{2pt}r{2pt}){7-11}
    \cmidrule(l{2pt}r{2pt}){12-16}
        Models & Help. & Relev. & Acc. & Details & Overall & Help. & Relev. & Acc. & Details & Overall & Help. & Relev. & Acc. & Details & Overall \\
    \midrule
    2D & 0.916 & 0.968 & 0.930 & 0.835 & 0.907 & 1.129 & 1.168 & 1.163 & 1.100 & 1.137 & 1.432 & 1.631 & 1.760 & 1.362 & 1.526 \\
    3D & 1.078 & 1.085 & 1.069 & 1.047 & 1.071 & 1.185 & 1.206 & 1.237 & 1.157 & 1.195 & 1.507 & 1.713 & 1.853 & 1.443 & 1.632 \\
    2D+3D (Concat) & 1.039 & 1.076 & 1.060 & 0.980 & 1.037 & 1.187 & 1.221 & 1.256 & 1.156 & 1.210 & 1.571 & 1.827 & 2.002 & 1.537 & 1.741 \\
    2D+3D (Unified) & 1.126 & 1.145 & 1.154 & 1.090 & 1.125 & 1.224 & 1.266 & 1.302 & 1.211 & 1.251 & 1.578 & 1.840 & 2.030 & 1.528 & 1.744 \\
    \bottomrule
    \end{tabular}}
    \vspace{-0.1in}
    \caption{Ablation studies on molecular representations. We report all scores including helpfulness, relevance, accuracy, level of detail, and overall score following the settings in Section~\ref{sec:quantitative}.}
    \label{app:tab:ablation_blending_module}
\end{table*}
\begin{table*}[h]
    \centering
    \resizebox{\textwidth}{!}{
    \renewcommand{\arraystretch}{1.1}
    \renewcommand{\tabcolsep}{4pt}
    \begin{tabular}{l c c c c a c c c c a c c c c a c c c}
    \toprule
        &
        \multicolumn{5}{c}{\textsc{Structural}} &
        \multicolumn{5}{c}{\textsc{Chemical}} &
        \multicolumn{5}{c}{\textsc{Biological}} &
        \multicolumn{3}{c}{\textsc{PAMPA}} \\
    \cmidrule(l{2pt}r{2pt}){2-6}
    \cmidrule(l{2pt}r{2pt}){7-11}
    \cmidrule(l{2pt}r{2pt}){12-16}
    \cmidrule(l{2pt}r{2pt}){17-19}
        Models & Help. & Relev. & Acc. & Details & Overall & Help. & Relev. & Acc. & Details & Overall & Help. & Relev. & Acc. & Details & Overall & Default & CoT & w/ Task Info.\\
    \midrule
    S & 1.124 & 1.119 & 1.128 & 1.123 & 1.119 & 1.127 & 1.183 & 1.189 & 1.133 & 1.166 & 1.201 & 1.320 & 1.499 & 1.310 & 1.328 & 37.35 & 32.92 & 56.02 \\
    S+S2F & 1.176 & 1.163 & 1.189 & 1.184 & 1.172 & 1.249 & 1.286 & 1.325 & 1.259 & 1.285 & 1.604 & 1.814 & 2.005 & 1.608 & 1.754 & 66.83 & 64.13 & 57.88\\
    Conv. & 1.169 & 1.152 & 1.172 & 1.176 & 1.166 & 0.674 & 0.814 & 0.787 & 0.519 & 0.689 & 0.861 & 1.079 & 1.067 & 0.589 & 0.887 & 41.03 & 45.95 & 83.05 \\
    Full & 1.126 & 1.145 & 1.154 & 1.090 & 1.125 & 1.224 & 1.266 & 1.302 & 1.211 & 1.251 & 1.578 & 1.840 & 2.030 & 1.528 & 1.744 & 63.55 & 64.37 & 72.48 \\
    \bottomrule
    \end{tabular}}
    \vspace{-0.1in}
    \caption{Ablation studies on data types. S denotes the detailed structural descriptions, S2F denotes the structure-to-feature relationship explanations, and Conv. denotes the comprehensive conversations. We report all scores including helpfulness relevance, accuracy, level of detail, and overall score following the settings in Section~\ref{sec:quantitative}. Additionally, we report the accuracy on PAMPA task.}
    \vspace{-0.07in}
    \label{app:tab:ablation_data}
\end{table*}


\begin{wraptable}{r}{0.45\textwidth}
    \centering
    \vspace{-0.15in}
    \
    \resizebox{0.45\textwidth}{!}{
    \renewcommand{\arraystretch}{1.1}
    \renewcommand{\tabcolsep}{4pt}
    \begin{tabular}{l l c c c}
    \toprule
    Models & LLM & Default & CoT & w/ Task Info. \\
    \midrule
    GPT-4o & & \textbf{60.82} & \textbf{61.34} & \textbf{64.43} \\
    Llama2 & Llama2-7B & 37.37 & 51.56 & 53.09 \\
    Llama3 & Llama3.1-8B & 57.07 & 51.03 & 55.15 \\
    
    \midrule
    Mol-Instructions & Llama2-7B & 52.58 & 52.58 & 51.34 \\
    Mol-Instructions & Llama3.1-8B & - & - & 54.91 \\
    LlasMol & Mistral-7B & 51.55 & 54.64 & 56.19 \\
    3D-MoLM$^\dagger$ & Llama3.1-8B & 49.14 & 51.65 &  51.91 \\
    LLaMo & Llama2-7B & - & 49.36 & 50.60 \\
    LLaMo$^\dagger$ & Llama3.1-8B & 55.44 & 55.45 & 56.91\\
    \midrule
    \textbf{Mol-LLaMA (Ours)} & Llama2-7B & 53.37 & 52.58 & 52.58 \\
    
    \textbf{Mol-LLaMA (Ours)} & Llama3.1-8B & \underline{59.54} & \underline{55.56} & \underline{59.08} \\
    \bottomrule
    \end{tabular}}
    \vspace{-0.1in}
    \caption{Zero-shot performances on BBBP task. We report accuracy for the test set following the splitting from \citet{zhou2023unimol}. The best results are highlighted in \textbf{bold}, and the second-best results are \underbar{underlined}.}
    \label{app:tab:BBBP}
\end{wraptable}
\subsection{Molecular Property Prediction on BBBP\label{app:sec:additional_bbbp}}
We additionally conduct an experiment on BBBP task, where the task is to predict whether a given molecule can penetrate the blood-brain barrier. The brain-blood barrier is a system with high selectivity by complicated transport routes such as transporter-mediated transcytosis, receptor-mediated transcytosis, cell-mediated transcytosis, lipophilic pathway, efflux pumps, adsorptive transcytosis, and paracellular aqueous pathway~\cite{wu2023bbbstructure}. Therefore, the task requires external knowledge such as the structure of tight junctions, the interactions with membrane proteins, and the environment of BBB. We follow the experimental settings in Section~\ref{sec:quantitative}, using three types of prompt strategies: default setting, CoT prompting, and providing task-specific information. As shown in Table~\ref{app:tab:BBBP}, Mol-LLaMA shows high performance, achieving a performance improvement compared to the base LLMs. However, GPT-4o outperforms all baselines due to its extensive knowledge.


\clearpage

\begin{table}[ht!]
\begin{resizebox}{\linewidth}{!}{
\centering
\renewcommand{\arraystretch}{1.05}
\begin{tabular}{p{1.2cm} p{18cm}}
\toprule
\multicolumn{2}{l}{\large \textbf{Mol-LLaMA (Ours)}}  \\
\midrule

\textbf{Default} & The given molecule is a complex organic compound with several structural features that influence its permeability in the PAMPA assay. Let's break down these features and their potential impact on permeability: \\

& 1. \textcolor{tabblue}{**Indole Core**}: The presence of an indole moiety, which is a bicyclic structure consisting of a benzene ring fused to a pyrrole ring, is significant. \textcolor{tabblue}{Indole derivatives are known for their ability to interact with biological membranes due to their planar structure and aromaticity}. This can enhance the molecule's ability to permeate through the artificial membrane, contributing to high permeability. \\

& 2. \textcolor{tabblue}{**Phenyl Group**} : The phenyl group attached to the indole core \textcolor{tabblue}{adds to the hydrophobic character of the molecule}. Aromatic rings like phenyl groups can \textcolor{tabblue}{facilitate interactions with the lipid bilayer} of the artificial membrane, potentially increasing permeability. \\

& 3. \textcolor{tabblue}{**Pyrrolidinone Moiety**} : The pyrrolidinone ring, which is a five-membered lactam, introduces a degree of polarity to the molecule. \textcolor{tabblue}{This can affect the overall solubility and permeability balance}. The presence of a carbonyl group within the pyrrolidinone can form hydrogen bonds, which might influence the molecule's interaction with the membrane and its permeability. \\

& 4. \textcolor{tabblue}{**Carboxamide Group**} : The carboxamide functional group is polar and can participate in hydrogen bonding. This group can \textcolor{tabblue}{enhance the molecule's solubility in aqueous environments}, which is crucial for its interaction with the PAMPA membrane. However, \textcolor{tabblue}{it may also introduce some steric hindrance, potentially reducing permeability.} \\

& Overall, the given molecule's structure \textcolor{tabblue}{suggests a balance between hydrophobic and hydrophilic properties. The indole and phenyl groups contribute to hydrophobic interactions, while the pyrrolidinone and carboxamide groups introduce polar characteristics.} These features collectively suggest that the given molecule likely exhibits high permeability in the PAMPA assay, as it can effectively interact with the artificial membrane. \textcolor{tabblue}{Final answer: High permeability} \\
\midrule
\textbf{CoT} & The given molecule is a complex organic compound with several structural features that influence its permeability in the PAMPA assay. Let's break down these features and their implications: \\

& 1. **Indole Core**: The presence of an indole moiety, which is a bicyclic structure consisting of a benzene ring fused to a pyrrole ring, is significant. Indole derivatives are known for their ability to interact with biological membranes due to their planar structure and aromaticity. This can enhance the molecule's ability to permeate through the artificial membrane, contributing to high permeability. \\

& 2. **Phenyl Group**: The phenyl group attached to the indole core adds to the hydrophobic character of the molecule. Aromatic rings like phenyl groups can engage in $\pi$-$\pi$ interactions with the membrane, potentially increasing the molecule's permeability. \\

& 3. **Pyrrolidinone Moiety**: The pyrrolidinone ring, which is a five-membered lactam, introduces a degree of polarity to the molecule. This can affect the overall solubility and permeability balance. The presence of a carbonyl group in the pyrrolidinone can form hydrogen bonds, which might influence the molecule's interaction with the membrane and its permeability. \\

& 4. **Carboxamide Group** : The carboxamide functional group is polar and can form hydrogen bonds, which may enhance the molecule's solubility in aqueous environments. However, it can also affect the permeability by introducing steric hindrance and increasing the molecule's overall polarity. \\

& Overall, the given molecule has a combination of hydrophobic and polar features. The indole and phenyl groups contribute to hydrophobic interactions, while the pyrrolidinone and carboxamide groups introduce polar characteristics. These features suggest that the molecule would have a balance of permeability, potentially leading to high permeability in the PAMPA assay. However, the exact permeability would depend on the specific conditions of the assay and the nature of the artificial membrane used.  \\
& Final answer: High permeability.\\


\midrule



\textbf{with Task Info.} & 1. **Lipophilicity**: The presence of aromatic rings, such as the phenyl group and the indolizine moiety, suggests that the given molecule is relatively lipophilic. This lipophilicity is likely to enhance its permeability through the lipid bilayer of the artificial membrane.  \\

& 2. **Molecular Size and Weight**: The given molecule has a moderate molecular size and weight, as indicated by its classification as a pyrrolidinone. This size is not excessively large, which may allow it to pass through the membrane without significant steric hindrance. \\

& 3. **Polarity**: The molecule contains a pyrrolidinone group, which introduces some degree of polarity. However, the overall structure is dominated by aromatic and hydrocarbon components, suggesting that the polar surface area is not high enough to significantly impede permeability. \\

& 4. **Charge**: The given molecule is neutral, which is favorable for permeability as charged molecules often have reduced ability to cross lipid membranes. \\

& 5. **Rigidity**: The indolizine moiety contributes to the rigidity of the molecule. Rigidity can enhance membrane permeability by allowing the molecule to maintain a consistent conformation as it interacts with the membrane. \\

& 6. **Aromaticity**: The aromatic rings in the structure can engage in $\pi$-$\pi$ interactions with the lipid bilayer, potentially enhancing permeability. \\

& 7. **Hydration Energy**: The presence of the pyrrolidinone group may lower the hydration energy of the molecule, facilitating its passage through the membrane by reducing the energy required to displace water molecules. \\

& 8. **Membrane Affinity**: The balance of hydrophobic and hydrophilic regions in the given molecule suggests that it has a favorable membrane affinity, which is conducive to high permeability. \\

& Considering these structural characteristics, the given molecule is likely to exhibit high permeability in the PAMPA assay. \\

& Final answer: High permeability. \\
\bottomrule
\end{tabular}}
\end{resizebox}
\vspace{-0.1in}
\caption{Generated responses of Mol-LLaMA on PAMPA task.}
\label{app:tab:pampa_MolLLaMA}
\end{table}

\clearpage

\begin{table}[ht!]
\begin{resizebox}{\linewidth}{!}{
\centering
\renewcommand{\arraystretch}{1.1}
\begin{tabular}{p{1.2cm} p{18cm}}
\toprule
\multicolumn{2}{l}{\large \textbf{GPT-4o}}  \\
\midrule

\textbf{Default} & To begin determining the permeability of the given molecule (identified as a bicyclic pyrrole compound with an embedded quinoline structure), we need to consider several molecular properties that influence permeability in the Parallel Artificial Membrane Permeability Assay (PAMPA). \\

& Key factors that tend to influence permeability include:\\

& 1. \textcolor{tabblue}{**Lipophilicity**} : Molecules with higher lipophilicity generally have higher permeability due to better membrane partitioning. The presence of aromatic rings and the overall hydrophobic character of this molecule suggests moderate to high lipophilicity.\\

& 2. \textcolor{tabblue}{**Molecular Weight**} : Compounds with lower molecular weights usually have better permeability. However, this molecule appears to have a moderate molecular weight, which could suggest moderate permeability. \\

& 3. \textcolor{tabblue}{**Polarity and Hydrogen Bonding**} : Molecules with several hydrogen bond donors and acceptors tend to have lower permeability due to increased interaction with water. This molecule contains multiple potential sites for hydrogen bonding (e.g., amide NH groups, ketone, and nitrogen atoms in rings), which might suggest lower permeability. \\

& 4. \textcolor{tabblue}{**Structural Rigidity and Planarity**} : Rigidity and planarity can enhance membrane permeability as they may facilitate passage through lipid layers. The bicyclic and aromatic nature of the compound could potentially increase permeability. \\

& Upon considering these factors, the multiple heteroatoms and hydrogen bonding potential in the structure suggest that, despite its potentially favorable aspects like lipophilicity, other factors potentially reduce its permeability. \\

& Therefore, integrating these insights, the overall picture leans towards the molecule demonstrating reduced permeability. \textcolor{tabred}{Final answer: Low-to-moderate permeability.} \\


\midrule


\textbf{CoT} & To evaluate the permeability of the given molecule using the Parallel Artificial Membrane Permeability Assay (PAMPA), we should consider several structural and physicochemical factors that influence passive diffusion through artificial membranes. \\

& 1. **Molecular Size**: Smaller molecules generally permeate membranes more easily than larger molecules. The given molecule is relatively large due to the presence of multiple rings and multiple atoms, indicating potential low permeability. \\

& 2. **Lipophilicity**: Lipophilicity (often estimated by LogP) plays a crucial role in membrane permeability. The presence of aromatic rings and the amide groups suggests moderate lipophilicity. The two amide groups, which can form hydrogen bonds, might reduce lipophilicity and thus permeability to lipid-like membranes. \\

& 3. **Polarity and Hydrogen Bonding**: The presence of polar functional groups such as the carbonyl and the secondary and tertiary amine groups can increase polarity and potential hydrogen bonding with the aqueous phase, decreasing permeability. \\

& 4. **Planarity**: The presence of multiple aromatic rings that might be able to stack could imply some level of planarity aiding in permeability. However, the overall complexity and potential for multiple conformations could negate this advantage. \\

& Based on this analysis, especially considering the overall size and the presence of hydrogen-bond forming groups which can interact with the aqueous environment and hinder passive diffusion, the molecule is more likely to exhibit: \\

& Final answer: Low-to-moderate permeability. \\

\midrule

\textbf{with Task} & To evaluate the PAMPA permeability of the given molecule, we need to consider several molecular properties based on the IUPAC name and SMILES string provided: \\

\textbf{Info.} & 1) **Lipophilicity**: The molecule contains aromatic rings, which generally enhance lipophilicity. However, this is balanced by the presence of polar functional groups such as amides. \\

& 2) **Molecular Size and Weight**: The molecule features multiple ring structures, indicating a relatively large size and weight, which may hinder permeability. \\

& 3) **Polarity**: The presence of amide groups increases polarity, which could reduce permeability. The molecule also has a moderate potential for hydrogen bonding. \\

& 4) **Charge**: There are no apparent ionizable groups that would carry a charge at physiological pH, suggesting that the molecule remains neutral, favoring higher permeability. \\

& 5) **Rigidity**: The molecule appears to be fairly rigid due to its fused and aromatic ring structures, which can facilitate membrane permeation.\\

& 6) **Aromaticity**: The aromatic rings present will influence the hydrophobic interactions with the lipid bilayer, potentially enhancing permeability. \\

& 7) **Hydration Energy**: Polar groups and potential for hydrogen bonding can increase hydration energy, thus reducing permeability. \\

& 8) **Membrane Affinity**: The molecule's structure suggests balanced hydrophilic and hydrophobic characteristics, which could result in moderate affinity for the membrane. \\

& Taking these properties into account, the molecule has characteristics that both enhance and reduce permeability. The moderate size, polarity, and hydration energy, combined with the presence of lipophilic and rigid structures, suggest that the molecule's permeability would not be categorized as high.\\

& Final answer: Low-to-moderate permeability. \\
\bottomrule
\end{tabular}}
\end{resizebox}
\vspace{-0.1in}
\caption{Generated responses of GPT-4o on PAMPA task.}
\label{app:tab:pampa_GPT4o}
\end{table}

\clearpage

\begin{table}[ht!]
\begin{resizebox}{\linewidth}{!}{
\centering
\renewcommand{\arraystretch}{1.1}
\begin{tabular}{p{1.2cm} p{18cm}}
\toprule
\multicolumn{2}{l}{\large \textbf{3D-MoLM}}  \\
\midrule

\textbf{Default} & The molecule with the SMILES code C1CC(=O)NC1CNC(=O)C2=C3C=CNC3=CC(=N2)C4=CC=CC=C4 \textcolor{tabred}{is a pyrazolo[1,5-a]pyrimidine-7-carboxamide derivative. It consists of a pyrazolo[1,5-a]pyrimidine core with a carboxamide group attached at position 7.} The molecule also \textcolor{tabblue}{contains a pyrrolidine ring and a phenyl ring. The presence of the carboxamide group} suggests that this molecule may have the ability to form hydrogen bonds, which can influence its permeability. Additionally, \textcolor{tabblue}{the presence of the pyrrolidine ring and the phenyl ring} can contribute to the overall hydrophobicity of the molecule.
Based on the structure and the presence of the pyrazolo[1,5-a]pyrimidine core, it is likely that this molecule has low-to-moderate permeability to the artificial membrane. The pyrazolo[1,5-a]pyrimidine core is known to have moderate permeability, and the presence of the carboxamide group and the phenyl ring may slightly enhance or hinder the permeability, respectively.

Therefore, the final answer is: \textcolor{tabred}{Final answer: Low-to-moderate permeability.} \\


\midrule


\textbf{CoT} & The molecule with the SMILES code C1CC(=O)NC1CNC(=O)C2=C3C=CNC3=CC(=N2)C4=CC=CC=C4 \textcolor{tabred}{is a pyrroloquinoline derivative. Pyrroloquinolines} are a class of heterocyclic compounds that contain a pyrrole ring fused with a quinoline ring. This particular molecule \textcolor{tabred}{is a derivative of pyrroloquinoline, with additional functional groups attached.} The molecule \textcolor{tabblue}{has a pyrrolidine ring (C1CC(=O)NC1)} and a \textcolor{tabred}{pyrroloquinoline ring (CNC(=O)C2=C3C=CNC3=CC(=N2)C4=CC=CC=C4)}. The presence of the pyrrolidine ring suggests that the molecule may have some degree of flexibility and can adopt different conformations. The pyrroloquinoline ring is aromatic, which means it has a planar structure and exhibits delocalized pi-electrons. This aromaticity contributes to the molecule's stability and can influence its reactivity. \textcolor{tabblue}{The presence of amide groups (C(=O)N)} in the molecule indicates the potential for hydrogen bonding and the ability to form intermolecular interactions. These interactions can affect the molecule's solubility and permeability. Based on the structure and functional groups present, it is likely that this molecule has low-to-moderate permeability to the artificial membrane. The presence of aromatic rings and amide groups can influence the molecule's ability to pass through the membrane, but the overall size and complexity of the molecule may limit its permeability.

Therefore, the final answer is: \textcolor{tabred}{Final answer: Low-to-moderate permeability.} \\



\midrule



\textbf{with Task Info.} & The molecule with the SMILES code C1CC(=O)NC1CNC(=O)C2=C3C=CNC3=CC(=N2)C4=CC=CC=C4 is a complex compound with various chemical and physical properties that influence its permeability in the Parallel Artificial Membrane Permeability Assay (PAMPA). Let's analyze these properties to determine the permeability of the molecule. \\

& 1. Lipophilicity: The presence of multiple aromatic rings, such as the phenyl ring, suggests that this molecule is lipophilic. Lipophilicity generally correlates with increased permeability, up to a certain threshold. \\

& 2. Molecular Size and Weight: The molecule has a relatively large size due to the presence of multiple rings and functional groups. This size may limit its permeability. \\

& 3. Polarity: The molecule contains polar functional groups, such as amide and carbonyl groups, which contribute to its overall polarity. However, the presence of aromatic rings and hydrophobic substituents, like the phenyl ring, suggests that the molecule may have a balanced affinity for both the aqueous phase and the lipid bilayer.  \\

& 4. Charge: The molecule is neutral, which is generally favorable for permeability. \\

& 5. Rigidity: The presence of multiple rings and the rigidity of the molecule may contribute to its permeability. \\

& 6. Aromaticity: The molecule contains aromatic rings, which can influence lipophilicity and molecular interactions with the lipid bilayer, thereby affecting permeability.  \\

& 7. Hydration Energy: The presence of polar functional groups may result in higher hydration energy, which could potentially hinder membrane permeation. \\

& 8. Membrane Affinity: The balanced affinity for both the aqueous phase and the lipid bilayer suggests that the molecule may exhibit better PAMPA permeability. \\

& Based on these properties, the molecule is expected to have low-to-moderate permeability to the artificial membrane. The presence of polar functional groups, large size, and potential hydration energy limitations may hinder its diffusion through the lipid bilayer. However, the lipophilicity, rigidity, and aromaticity may contribute to its permeability to some extent.  \\

& Therefore, the final answer is: Final answer: Low-to-moderate permeability.\\
\bottomrule
\end{tabular}}
\end{resizebox}
\vspace{-0.1in}
\caption{Generated responses of 3D-MoLM on PAMPA task.}
\label{app:tab:pampa_3DMoLM}
\end{table}


\begin{table}[ht!]
\begin{resizebox}{\linewidth}{!}{
\centering
\renewcommand{\arraystretch}{1.1}
\begin{tabular}{p{1.2cm} p{18cm}}
\toprule
\multicolumn{2}{l}{\large \textbf{LLaMo}}  \\
\midrule

\textbf{Default} & The molecule has low-to-moderate permeability to the artificial membrane. \\


\midrule


\textbf{CoT} & The molecule has high permeability to the artificial membrane. \\

\midrule


\textbf{with Task} & The molecule has high lipophilicity, low molecular size and weight, low polarity, and is neutral. It also has a high degree of rigidity and aromaticity, which are all factors that contribute to higher permeability. \\
\textbf{Info.} & Final answer: The molecule has high permeability. \\
\bottomrule
\end{tabular}}
\end{resizebox}
\vspace{-0.1in}
\caption{Generated responses of LLaMo on PAMPA task.}
\label{app:tab:pampa_LLaMo}
\end{table}
