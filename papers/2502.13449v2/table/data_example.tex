
\begin{table*}[t!]
\centering  
\footnotesize

\begin{tcolorbox}[colframe=gray,colback=white,boxrule=1pt,arc=0.6em,boxsep=0mm,left=5.pt,right=15pt]
\renewcommand{\arraystretch}{1.4}
\begin{tabular}{l c}
    \multicolumn{1}{l}{\textbf{\normalsize{Context Type 1: IUPAC Name}}} & \hspace{-0.15in} \multirow{2}{*}{\vspace{-0.1in} \includegraphics[height=2.6cm]{figure/example3.pdf}} \\
    \multicolumn{1}{p{.67\textwidth}}{(2R)-2-amino-3-methylsulfanylpropanoic acid} &  \\
    \multicolumn{1}{l}{\textbf{\normalsize{Context Type 2: Description}}} & \\
    \multicolumn{1}{p{.67\textwidth}}{S-methylcysteine is a cysteine derivative that is L-cysteine in which the hydrogen attached to the sulfur is replaced by a methyl group. It has a role as a human urinary metabolite and a plant metabolite. It is a tautomer of a S-methylcysteine zwitterion.} &  \\
    \cmidrule{1-2}
\end{tabular}

\renewcommand{\arraystretch}{1.4}
\begin{tabular}{p{0.95\textwidth}}
    \textcolor{figgreen}{\textbf{\normalsize{Data Type 1: Detailed Structural Descriptions}}} \\
\end{tabular}

\renewcommand{\arraystretch}{1.2}
\begin{tabular}{p{0.95\textwidth}}
    Instruction to GPT-4o: \textbf{Explain which substructures and functional groups are contained and how they are connected.} \\
    \cdashlinelr{1-1}
    The molecule provided is a chiral amino acid that contains several identifiable substructures. \\
    1. **\textbf{Amino Group}**: This functional group consists of a nitrogen atom bonded to two hydrogen atoms (-NH$_\text{2}$). It is attached to the second carbon atom ...$<$omitted$>$...\\
    2. **\textbf{Carboxylic Acid Group}**: This group includes a carbon atom double-bonded to an oxygen atom and also bonded to an (-OH) group. It is attached to the end of the carbon chain, characteristic of amino acids ...$<$omitted$>$... \\   
    \cmidrule{1-1}
\end{tabular}

\renewcommand{\arraystretch}{1.4}
\begin{tabular}{p{0.95\textwidth}}
    \textcolor{figgreen}{\textbf{\normalsize{Data Type 2: Struture-to-Feature Relationship Explanations}}} \\
\end{tabular}

\renewcommand{\arraystretch}{1.2}
\begin{tabular}{p{0.95\textwidth}}
    Instruction to GPT-4o: Explain chemical properties \textbf{by relating chemical properties to its structural information.} \\
    \cdashlinelr{1-1}
    Firstly, \textbf{the presence of the amino group (-NH2) classifies this compound as an amino acid, which contributes to its ability to act as a zwitterion}. This means that it can contain both a positive charge and a negative charge ...$<$omitted$>$... \\
    \textbf{The sulfur atom with a methyl attachment forms a thioether. This increases the molecule's hydrophobic characteristics compared to cysteine}, which contains a more polar thiol group.  ...$<$omitted$>$... \\
    \cmidrule{1-1}
\end{tabular}

\renewcommand{\arraystretch}{1.4}
\begin{tabular}{p{0.95\textwidth}}
    \textcolor{figgreen}{\textbf{\normalsize{Data Type 3: Comprehensive Conversations}}} \\
\end{tabular}

\renewcommand{\arraystretch}{1.2}
\begin{tabular}{p{0.95\textwidth}}
    Instruction to GPT-4o: Design a sequence of pairs of questions and answers that \textbf{gradually deepen the level of the conversation, from structural information and chemical properties to biological functionalities.} \\
    \cdashlinelr{1-1}
    \textsc{User}: Can you tell me about \textbf{the basic structure} of the given molecule? \\
    \textsc{Assistant}: The given molecule is a derivative of cysteine, where the hydrogen atom attached to the sulfur group is replaced by a methyl group. ...$<$omitted$>$...\\
    \textsc{User}: \textbf{How does the molecule's structure influence its chemical behavior?} \\
    \textsc{Assistant}: The structure of the given molecule, particularly the presence of both amino and carboxylic acid groups, contributes to its zwitterionic nature at physiological pH. ...$<$omitted$>$...\\ 
    \textsc{User}: \textbf{What role does this molecule play biologically?} \\
    \textsc{Assistant}: Biologically, S-methylcysteine serves as a metabolite in both humans and plants. ...$<$omitted$>$...\\ 
\end{tabular}
\end{tcolorbox}
\vspace{-0.17in}
\caption{\small One example of the constructed instruction data. We use the IUPAC name and description from PubChem as contexts for prompting GPT-4o as shown in the first block. The following three blocks show the instructions and corresponding responses of GPT-4o for each data type. Entire responses of the given example are provided in Table~\ref{app:tab:data_example_full_part1} and \ref{app:tab:data_example_full_part2} of Appendix~\ref{app:sec:dataset_construction}.}
\label{tab:data_example}
\vspace{-0.12in}
\end{table*}
