
\begin{table}[ht!]
\begin{resizebox}{\linewidth}{!}{
\centering
\renewcommand{\arraystretch}{1.1}
\begin{tabular}{p{1.2cm} p{18cm}}
\toprule
\multicolumn{2}{l}{\large \textbf{GPT-4o}}  \\
\midrule

\textbf{Default} & To begin determining the permeability of the given molecule (identified as a bicyclic pyrrole compound with an embedded quinoline structure), we need to consider several molecular properties that influence permeability in the Parallel Artificial Membrane Permeability Assay (PAMPA). \\

& Key factors that tend to influence permeability include:\\

& 1. \textcolor{tabblue}{**Lipophilicity**} : Molecules with higher lipophilicity generally have higher permeability due to better membrane partitioning. The presence of aromatic rings and the overall hydrophobic character of this molecule suggests moderate to high lipophilicity.\\

& 2. \textcolor{tabblue}{**Molecular Weight**} : Compounds with lower molecular weights usually have better permeability. However, this molecule appears to have a moderate molecular weight, which could suggest moderate permeability. \\

& 3. \textcolor{tabblue}{**Polarity and Hydrogen Bonding**} : Molecules with several hydrogen bond donors and acceptors tend to have lower permeability due to increased interaction with water. This molecule contains multiple potential sites for hydrogen bonding (e.g., amide NH groups, ketone, and nitrogen atoms in rings), which might suggest lower permeability. \\

& 4. \textcolor{tabblue}{**Structural Rigidity and Planarity**} : Rigidity and planarity can enhance membrane permeability as they may facilitate passage through lipid layers. The bicyclic and aromatic nature of the compound could potentially increase permeability. \\

& Upon considering these factors, the multiple heteroatoms and hydrogen bonding potential in the structure suggest that, despite its potentially favorable aspects like lipophilicity, other factors potentially reduce its permeability. \\

& Therefore, integrating these insights, the overall picture leans towards the molecule demonstrating reduced permeability. \textcolor{tabred}{Final answer: Low-to-moderate permeability.} \\


\midrule


\textbf{CoT} & To evaluate the permeability of the given molecule using the Parallel Artificial Membrane Permeability Assay (PAMPA), we should consider several structural and physicochemical factors that influence passive diffusion through artificial membranes. \\

& 1. **Molecular Size**: Smaller molecules generally permeate membranes more easily than larger molecules. The given molecule is relatively large due to the presence of multiple rings and multiple atoms, indicating potential low permeability. \\

& 2. **Lipophilicity**: Lipophilicity (often estimated by LogP) plays a crucial role in membrane permeability. The presence of aromatic rings and the amide groups suggests moderate lipophilicity. The two amide groups, which can form hydrogen bonds, might reduce lipophilicity and thus permeability to lipid-like membranes. \\

& 3. **Polarity and Hydrogen Bonding**: The presence of polar functional groups such as the carbonyl and the secondary and tertiary amine groups can increase polarity and potential hydrogen bonding with the aqueous phase, decreasing permeability. \\

& 4. **Planarity**: The presence of multiple aromatic rings that might be able to stack could imply some level of planarity aiding in permeability. However, the overall complexity and potential for multiple conformations could negate this advantage. \\

& Based on this analysis, especially considering the overall size and the presence of hydrogen-bond forming groups which can interact with the aqueous environment and hinder passive diffusion, the molecule is more likely to exhibit: \\

& Final answer: Low-to-moderate permeability. \\

\midrule

\textbf{with Task} & To evaluate the PAMPA permeability of the given molecule, we need to consider several molecular properties based on the IUPAC name and SMILES string provided: \\

\textbf{Info.} & 1) **Lipophilicity**: The molecule contains aromatic rings, which generally enhance lipophilicity. However, this is balanced by the presence of polar functional groups such as amides. \\

& 2) **Molecular Size and Weight**: The molecule features multiple ring structures, indicating a relatively large size and weight, which may hinder permeability. \\

& 3) **Polarity**: The presence of amide groups increases polarity, which could reduce permeability. The molecule also has a moderate potential for hydrogen bonding. \\

& 4) **Charge**: There are no apparent ionizable groups that would carry a charge at physiological pH, suggesting that the molecule remains neutral, favoring higher permeability. \\

& 5) **Rigidity**: The molecule appears to be fairly rigid due to its fused and aromatic ring structures, which can facilitate membrane permeation.\\

& 6) **Aromaticity**: The aromatic rings present will influence the hydrophobic interactions with the lipid bilayer, potentially enhancing permeability. \\

& 7) **Hydration Energy**: Polar groups and potential for hydrogen bonding can increase hydration energy, thus reducing permeability. \\

& 8) **Membrane Affinity**: The molecule's structure suggests balanced hydrophilic and hydrophobic characteristics, which could result in moderate affinity for the membrane. \\

& Taking these properties into account, the molecule has characteristics that both enhance and reduce permeability. The moderate size, polarity, and hydration energy, combined with the presence of lipophilic and rigid structures, suggest that the molecule's permeability would not be categorized as high.\\

& Final answer: Low-to-moderate permeability. \\
\bottomrule
\end{tabular}}
\end{resizebox}
\vspace{-0.1in}
\caption{Generated responses of GPT-4o on PAMPA task.}
\label{app:tab:pampa_GPT4o}
\end{table}
