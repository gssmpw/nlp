
\begin{table}[ht!]
\begin{resizebox}{\linewidth}{!}{
\centering
\renewcommand{\arraystretch}{1.05}
\begin{tabular}{p{1.2cm} p{18cm}}
\toprule
\multicolumn{2}{l}{\large \textbf{Mol-LLaMA (Ours)}}  \\
\midrule

\textbf{Default} & The given molecule is a complex organic compound with several structural features that influence its permeability in the PAMPA assay. Let's break down these features and their potential impact on permeability: \\

& 1. \textcolor{tabblue}{**Indole Core**}: The presence of an indole moiety, which is a bicyclic structure consisting of a benzene ring fused to a pyrrole ring, is significant. \textcolor{tabblue}{Indole derivatives are known for their ability to interact with biological membranes due to their planar structure and aromaticity}. This can enhance the molecule's ability to permeate through the artificial membrane, contributing to high permeability. \\

& 2. \textcolor{tabblue}{**Phenyl Group**} : The phenyl group attached to the indole core \textcolor{tabblue}{adds to the hydrophobic character of the molecule}. Aromatic rings like phenyl groups can \textcolor{tabblue}{facilitate interactions with the lipid bilayer} of the artificial membrane, potentially increasing permeability. \\

& 3. \textcolor{tabblue}{**Pyrrolidinone Moiety**} : The pyrrolidinone ring, which is a five-membered lactam, introduces a degree of polarity to the molecule. \textcolor{tabblue}{This can affect the overall solubility and permeability balance}. The presence of a carbonyl group within the pyrrolidinone can form hydrogen bonds, which might influence the molecule's interaction with the membrane and its permeability. \\

& 4. \textcolor{tabblue}{**Carboxamide Group**} : The carboxamide functional group is polar and can participate in hydrogen bonding. This group can \textcolor{tabblue}{enhance the molecule's solubility in aqueous environments}, which is crucial for its interaction with the PAMPA membrane. However, \textcolor{tabblue}{it may also introduce some steric hindrance, potentially reducing permeability.} \\

& Overall, the given molecule's structure \textcolor{tabblue}{suggests a balance between hydrophobic and hydrophilic properties. The indole and phenyl groups contribute to hydrophobic interactions, while the pyrrolidinone and carboxamide groups introduce polar characteristics.} These features collectively suggest that the given molecule likely exhibits high permeability in the PAMPA assay, as it can effectively interact with the artificial membrane. \textcolor{tabblue}{Final answer: High permeability} \\
\midrule
\textbf{CoT} & The given molecule is a complex organic compound with several structural features that influence its permeability in the PAMPA assay. Let's break down these features and their implications: \\

& 1. **Indole Core**: The presence of an indole moiety, which is a bicyclic structure consisting of a benzene ring fused to a pyrrole ring, is significant. Indole derivatives are known for their ability to interact with biological membranes due to their planar structure and aromaticity. This can enhance the molecule's ability to permeate through the artificial membrane, contributing to high permeability. \\

& 2. **Phenyl Group**: The phenyl group attached to the indole core adds to the hydrophobic character of the molecule. Aromatic rings like phenyl groups can engage in $\pi$-$\pi$ interactions with the membrane, potentially increasing the molecule's permeability. \\

& 3. **Pyrrolidinone Moiety**: The pyrrolidinone ring, which is a five-membered lactam, introduces a degree of polarity to the molecule. This can affect the overall solubility and permeability balance. The presence of a carbonyl group in the pyrrolidinone can form hydrogen bonds, which might influence the molecule's interaction with the membrane and its permeability. \\

& 4. **Carboxamide Group** : The carboxamide functional group is polar and can form hydrogen bonds, which may enhance the molecule's solubility in aqueous environments. However, it can also affect the permeability by introducing steric hindrance and increasing the molecule's overall polarity. \\

& Overall, the given molecule has a combination of hydrophobic and polar features. The indole and phenyl groups contribute to hydrophobic interactions, while the pyrrolidinone and carboxamide groups introduce polar characteristics. These features suggest that the molecule would have a balance of permeability, potentially leading to high permeability in the PAMPA assay. However, the exact permeability would depend on the specific conditions of the assay and the nature of the artificial membrane used.  \\
& Final answer: High permeability.\\


\midrule



\textbf{with Task Info.} & 1. **Lipophilicity**: The presence of aromatic rings, such as the phenyl group and the indolizine moiety, suggests that the given molecule is relatively lipophilic. This lipophilicity is likely to enhance its permeability through the lipid bilayer of the artificial membrane.  \\

& 2. **Molecular Size and Weight**: The given molecule has a moderate molecular size and weight, as indicated by its classification as a pyrrolidinone. This size is not excessively large, which may allow it to pass through the membrane without significant steric hindrance. \\

& 3. **Polarity**: The molecule contains a pyrrolidinone group, which introduces some degree of polarity. However, the overall structure is dominated by aromatic and hydrocarbon components, suggesting that the polar surface area is not high enough to significantly impede permeability. \\

& 4. **Charge**: The given molecule is neutral, which is favorable for permeability as charged molecules often have reduced ability to cross lipid membranes. \\

& 5. **Rigidity**: The indolizine moiety contributes to the rigidity of the molecule. Rigidity can enhance membrane permeability by allowing the molecule to maintain a consistent conformation as it interacts with the membrane. \\

& 6. **Aromaticity**: The aromatic rings in the structure can engage in $\pi$-$\pi$ interactions with the lipid bilayer, potentially enhancing permeability. \\

& 7. **Hydration Energy**: The presence of the pyrrolidinone group may lower the hydration energy of the molecule, facilitating its passage through the membrane by reducing the energy required to displace water molecules. \\

& 8. **Membrane Affinity**: The balance of hydrophobic and hydrophilic regions in the given molecule suggests that it has a favorable membrane affinity, which is conducive to high permeability. \\

& Considering these structural characteristics, the given molecule is likely to exhibit high permeability in the PAMPA assay. \\

& Final answer: High permeability. \\
\bottomrule
\end{tabular}}
\end{resizebox}
\vspace{-0.1in}
\caption{Generated responses of Mol-LLaMA on PAMPA task.}
\label{app:tab:pampa_MolLLaMA}
\end{table}
