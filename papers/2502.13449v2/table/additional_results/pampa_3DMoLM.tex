
\begin{table}[ht!]
\begin{resizebox}{\linewidth}{!}{
\centering
\renewcommand{\arraystretch}{1.1}
\begin{tabular}{p{1.2cm} p{18cm}}
\toprule
\multicolumn{2}{l}{\large \textbf{3D-MoLM}}  \\
\midrule

\textbf{Default} & The molecule with the SMILES code C1CC(=O)NC1CNC(=O)C2=C3C=CNC3=CC(=N2)C4=CC=CC=C4 \textcolor{tabred}{is a pyrazolo[1,5-a]pyrimidine-7-carboxamide derivative. It consists of a pyrazolo[1,5-a]pyrimidine core with a carboxamide group attached at position 7.} The molecule also \textcolor{tabblue}{contains a pyrrolidine ring and a phenyl ring. The presence of the carboxamide group} suggests that this molecule may have the ability to form hydrogen bonds, which can influence its permeability. Additionally, \textcolor{tabblue}{the presence of the pyrrolidine ring and the phenyl ring} can contribute to the overall hydrophobicity of the molecule.
Based on the structure and the presence of the pyrazolo[1,5-a]pyrimidine core, it is likely that this molecule has low-to-moderate permeability to the artificial membrane. The pyrazolo[1,5-a]pyrimidine core is known to have moderate permeability, and the presence of the carboxamide group and the phenyl ring may slightly enhance or hinder the permeability, respectively.

Therefore, the final answer is: \textcolor{tabred}{Final answer: Low-to-moderate permeability.} \\


\midrule


\textbf{CoT} & The molecule with the SMILES code C1CC(=O)NC1CNC(=O)C2=C3C=CNC3=CC(=N2)C4=CC=CC=C4 \textcolor{tabred}{is a pyrroloquinoline derivative. Pyrroloquinolines} are a class of heterocyclic compounds that contain a pyrrole ring fused with a quinoline ring. This particular molecule \textcolor{tabred}{is a derivative of pyrroloquinoline, with additional functional groups attached.} The molecule \textcolor{tabblue}{has a pyrrolidine ring (C1CC(=O)NC1)} and a \textcolor{tabred}{pyrroloquinoline ring (CNC(=O)C2=C3C=CNC3=CC(=N2)C4=CC=CC=C4)}. The presence of the pyrrolidine ring suggests that the molecule may have some degree of flexibility and can adopt different conformations. The pyrroloquinoline ring is aromatic, which means it has a planar structure and exhibits delocalized pi-electrons. This aromaticity contributes to the molecule's stability and can influence its reactivity. \textcolor{tabblue}{The presence of amide groups (C(=O)N)} in the molecule indicates the potential for hydrogen bonding and the ability to form intermolecular interactions. These interactions can affect the molecule's solubility and permeability. Based on the structure and functional groups present, it is likely that this molecule has low-to-moderate permeability to the artificial membrane. The presence of aromatic rings and amide groups can influence the molecule's ability to pass through the membrane, but the overall size and complexity of the molecule may limit its permeability.

Therefore, the final answer is: \textcolor{tabred}{Final answer: Low-to-moderate permeability.} \\



\midrule



\textbf{with Task Info.} & The molecule with the SMILES code C1CC(=O)NC1CNC(=O)C2=C3C=CNC3=CC(=N2)C4=CC=CC=C4 is a complex compound with various chemical and physical properties that influence its permeability in the Parallel Artificial Membrane Permeability Assay (PAMPA). Let's analyze these properties to determine the permeability of the molecule. \\

& 1. Lipophilicity: The presence of multiple aromatic rings, such as the phenyl ring, suggests that this molecule is lipophilic. Lipophilicity generally correlates with increased permeability, up to a certain threshold. \\

& 2. Molecular Size and Weight: The molecule has a relatively large size due to the presence of multiple rings and functional groups. This size may limit its permeability. \\

& 3. Polarity: The molecule contains polar functional groups, such as amide and carbonyl groups, which contribute to its overall polarity. However, the presence of aromatic rings and hydrophobic substituents, like the phenyl ring, suggests that the molecule may have a balanced affinity for both the aqueous phase and the lipid bilayer.  \\

& 4. Charge: The molecule is neutral, which is generally favorable for permeability. \\

& 5. Rigidity: The presence of multiple rings and the rigidity of the molecule may contribute to its permeability. \\

& 6. Aromaticity: The molecule contains aromatic rings, which can influence lipophilicity and molecular interactions with the lipid bilayer, thereby affecting permeability.  \\

& 7. Hydration Energy: The presence of polar functional groups may result in higher hydration energy, which could potentially hinder membrane permeation. \\

& 8. Membrane Affinity: The balanced affinity for both the aqueous phase and the lipid bilayer suggests that the molecule may exhibit better PAMPA permeability. \\

& Based on these properties, the molecule is expected to have low-to-moderate permeability to the artificial membrane. The presence of polar functional groups, large size, and potential hydration energy limitations may hinder its diffusion through the lipid bilayer. However, the lipophilicity, rigidity, and aromaticity may contribute to its permeability to some extent.  \\

& Therefore, the final answer is: Final answer: Low-to-moderate permeability.\\
\bottomrule
\end{tabular}}
\end{resizebox}
\vspace{-0.1in}
\caption{Generated responses of 3D-MoLM on PAMPA task.}
\label{app:tab:pampa_3DMoLM}
\end{table}
