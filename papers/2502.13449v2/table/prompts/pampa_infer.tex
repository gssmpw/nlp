
\begin{table*}[ht!]
\centering  
\footnotesize
\begin{tcolorbox}[enhanced,breakable,
    colframe=gray!50!white,
    colbacktitle=white,
    coltitle=black,
    colback=white,
    borderline={0.5mm}{0mm}{gray!15!white},
    borderline={0.5mm}{0mm}{gray!50!white,dashed},
    attach boxed title to top center={yshift=-2mm},
    boxed title style={boxrule=0.8pt}]
    \renewcommand{\arraystretch}{1.3}
    \begin{tabular}{p{.95\linewidth}}
        \textbf{System} \\
        You are a drug discovery assistant tasked with predicting the permeability of a molecule in the Parallel Artificial Membrane Permeability Assay (PAMPA).
        Specifically, your role is to determine whether a molecule has high permeability or low-to-moderate permeability to the artificial membrane. \\
        \textcolor{figgreen}{
        Consider the following properties of molecules:}\\
        \textcolor{figgreen}{1) Lipophilicity: Higher lipophilicity generally correlates with increased permeability, up to a certain threshold.}\\
        \textcolor{figgreen}{2) Molecular Size and Weight: Smaller molecules tend to have higher permeability.}\\
        \textcolor{figgreen}{3) Polarity: Low polar surface area and low hydrogen bond donors/acceptors are associated with higher permeability.}\\
        \textcolor{figgreen}{4) Charge: Neutral molecules typically have better permeability compared to charged species, which are less likely to diffuse through the hydrophobic lipid bilayer.}\\
        \textcolor{figgreen}{5) Rigidity: A high degree of rigidity often permeate membranes more easily.}\\
        \textcolor{figgreen}{6) Aromaticity: The presence of aromatic rings can influence lipophilicity and molecular interactions with the lipid bilayer, thereby affecting permeability.}\\
        \textcolor{figgreen}{7) Hydration Energy: Lower hydration energy generally improves membrane permeation.}\\
        \textcolor{figgreen}{8) Membrane Affinity: Compounds with a balanced affinity for both the aqueous phase and the lipid bilayer tend to exhibit better PAMPA permeability.}\\
        Your final answer should be formatted as either : `Final answer : High permeability.' or `Low-to-moderate permeability.' \\
        \midrule
        \textbf{User} \\
        Determine the permeability of the given molecule to the artificial membrane.\\
        \textcolor{figblue}{Please provide a rationale for your answer.}
    \end{tabular}
\end{tcolorbox}
\vspace{-0.1in}
\caption{Prompts for PAMPA task. For the default setting, the \textcolor{figblue}{blue prompt} and \textcolor{figgreen}{green prompt} are not included. For the CoT promoting, we add the \textcolor{figblue}{blue prompt} not including the green prompt, and, for the case with the task-specific information (w/ Task Info), we add the \textcolor{figgreen}{green prompt} not including the blue prompt.}
\label{app:tab:prompts_pampa_infer}
\end{table*}
