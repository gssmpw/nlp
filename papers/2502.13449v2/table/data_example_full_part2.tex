
\begin{table*}[t!]
\centering  
\footnotesize

\begin{tcolorbox}[colframe=gray,colback=white,boxrule=1pt,arc=0.6em,boxsep=0mm,left=5.pt,right=15pt]
\renewcommand{\arraystretch}{1.4}
\begin{tabular}{l c}
    \multicolumn{1}{l}{\textbf{\normalsize{Context Type 1: IUPAC Name}}} & \hspace{-0.15in} \multirow{2}{*}{\vspace{-0.1in} \includegraphics[height=2.6cm]{figure/example3.pdf}} \\
    \multicolumn{1}{p{.67\textwidth}}{(2R)-2-amino-3-methylsulfanylpropanoic acid} &  \\
    \multicolumn{1}{l}{\textbf{\normalsize{Context Type 2: Description}}} & \\
    \multicolumn{1}{p{.67\textwidth}}{S-methylcysteine is a cysteine derivative that is L-cysteine in which the hydrogen attached to the sulfur is replaced by a methyl group. It has a role as a human urinary metabolite and a plant metabolite. It is a tautomer of a S-methylcysteine zwitterion.} &  \\
    \cmidrule{1-2}
\end{tabular}


\renewcommand{\arraystretch}{1.4}
\begin{tabular}{p{0.95\textwidth}}
    \textcolor{figgreen}{\textbf{\normalsize{Data Type 2: Struture-to-Feature Relationship Explanations (Biological Features)}}} \\
\end{tabular}

\renewcommand{\arraystretch}{1.2}
\begin{tabular}{p{0.95\textwidth}}
    The given molecule, known as S-methylcysteine, is a derivative of the amino acid cysteine. In its structure, the core amino acid framework is maintained, which consists of an amino group, a carboxylic acid group, and a side chain. In the side chain of this molecule, the sulfur atom of cysteine is bonded to a methyl group, distinguishing it from the native structure of cysteine. \\
    The biological properties of the given molecule are influenced significantly by this methylation at the sulfur atom. This modification preserves some aspects of cysteine's chemical behavior but also endows the molecule with unique characteristics:\\
    1. **\textbf{Human Urinary Metabolite Role}**: The metabolite role of the molecule in humans suggests it participates in or results from various metabolic processes. The transformation of cysteine to S-methylcysteine may occur as part of detoxification pathways or sulfur metabolism. The presence in urine indicates the body's method of excreting or regulating sulfur-containing compounds, possibly involved in processes managing oxidative stress or sulfur balance.\\
    2. **\textbf{Plant Metabolite Role}**: In plants, sulfur-containing amino acids like the given molecule can function in sulfur storage, cellular redox regulation, or even in the defense against stress. Methylation of cysteine could help modulate its reactivity, buffering against electrophile attacks or preventing unwanted formation of disulfide bonds, which is crucial for cellular homeostasis.\\
    3. **\textbf{Tautomeric Properties}**: The tautomeric relationship of the given molecule with its zwitterionic form is instrumental in understanding its behavior in biological environments, such as different pH conditions. As a zwitterion, it can exhibit varied solubility and stability profiles, influencing its transport and reactivity within both human and plant systems.\\
    The methylsulfanyl group adds to the biological properties by contributing a degree of non-polarity and slightly altering the exact steric and electronic nature of the side chain compared to cysteine, which can influence enzyme interactions, protein structure stability, and metabolic pathways where sulfur chemistry is paramount. Overall, despite being derived from cysteine, the methyl modification gives the molecule distinct roles and characteristics in living organisms.\\
    \cmidrule{1-1}
\end{tabular}

\renewcommand{\arraystretch}{1.4}
\begin{tabular}{p{0.95\textwidth}}
    \textcolor{figgreen}{\textbf{\normalsize{Data Type 3: Comprehensive Conversations}}} \\
\end{tabular}

\renewcommand{\arraystretch}{1.2}
\begin{tabular}{p{0.95\textwidth}}
\textsc{User}: Can you tell me about\textbf{ the basic structure} of the given molecule?\\
\textsc{Assistant}: Certainly! The given molecule is a derivative of cysteine, where the hydrogen atom attached to the sulfur group is replaced by a methyl group. This makes it include a sulfur-containing side chain as part of its structure. It has an amino group (-NH2), a carboxylic acid group (-COOH), and a methylsulfanyl side chain, all connected to the central carbon atom. \\

\textsc{User}: What are \textbf{the functional groups} present in this molecule? \\
\textsc{Assistant}: The molecule contains the following functional groups: an amino group (-NH2), a carboxylic acid group (-COOH), and a methylthio group (methylsulfanyl, -SCH3). \\

\textsc{User}: What is the stereochemistry of this molecule? \\
\textsc{Assistant}: The given molecule has a chiral center at the 2nd carbon atom and is specified to be in the (R)-configuration, meaning its spatial arrangement is determined relative to this stereocenter. \\

\textsc{User}: Can you tell me more about \textbf{the specific species} of this compound?\\
\textsc{Assistant}: The compound is known as S-methylcysteine. It is a specific derivative of cysteine with a distinctive methyl substitution on the sulfur atom, differing from the typical cysteine structure.\\

\textsc{User}: How does the molecule's structure influence \textbf{its chemical behavior}?\\
\textsc{Assistant}: The structure of the given molecule, particularly the presence of both amino and carboxylic acid groups, contributes to its zwitterionic nature at physiological pH. The substitution of a methyl group on the sulfur atom confers specific interactions and stability to the thiol, altering its reactivity compared to standard cysteine.\\

\textsc{User}: \textbf{What role does this molecule play biologically?}\\
\textsc{Assistant}: Biologically, S-methylcysteine serves as a metabolite in both humans and plants. It is identified as a human urinary metabolite and also plays a role as a plant metabolite, indicating its potential involvement in metabolic pathways within these organisms.\\

\textsc{User}: What might be some of \textbf{the functions or activities attributed to this molecule}?\\
\textsc{Assistant}: The molecule is involved in several functions due to its presence in metabolic processes. In plants, S-methylcysteine may play a role in sulfur transport or storage. Its function as a human metabolite could potentially affect pathways related to detoxification or protein synthesis where cysteine and its derivatives play roles.\\
\end{tabular}
\end{tcolorbox}
\vspace{-0.1in}
\caption{One example of the constructed instruction data. The top block shows the example of context types when prompting GPT-4o. The following two blocks show the responses of GPT-4o.}
\label{app:tab:data_example_full_part2}
\end{table*}
