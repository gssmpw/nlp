
\begin{table*}[ht!]
\centering
\footnotesize
\renewcommand{\arraystretch}{1.2}
\begin{tabular}{p{2.15cm} p{14.15cm}}
\toprule
 & \includegraphics[height=2.7cm]{figure/qualitative_example.pdf}\\
\midrule
SMILES & C1C(=O)NC2=C(C=C(C=C2)Br)C(=N1)C3=CC=CC=N3 \\
\midrule
Original \phantom{00000} Description & Bromazepam binds to the GABA receptor GABA-A, causing a conformational change and \textbf{increasing inhibitory effects of GABA}. Other neurotransmitters are not influenced. One of the \textbf{benzodiazepines} that is \textbf{used in the treatment of anxiety disorders.} \\
\midrule
User & Explain the features of the given molecule that are most distinct from those of other molecules. \\

\midrule
GPT-4o & The given molecule is 6-bromo-3-(pyridin-2-yl)-1H-quinazoline-2,4-dione. Several distinctive features set this molecule apart: \\

& 1. **\textcolor{tabred}{Quinazoline Core}**: The molecule contains a quinazoline core, which is a bicyclic structure combining a benzene ring and a pyrimidine ring. This is distinct from simpler aromatic compounds due to the additional nitrogen atoms in the pyrimidine portion.\\

& 2. **Substitution Pattern**: The molecule has unique substitution on the quinazoline core:\\
   & - A bromine atom is substituted at the 6-position of the quinazoline ring. This halogenation is distinctive and can affect the molecule's reactivity and binding properties significantly, compared to non-halogenated quinazolines.\\
   & - A pyridin-2-yl group is attached at the 3-position. This makes the compound a heteroarylated quinazoline, which can modulate the electronic properties and polarity of the molecule.\\

& 3. **Presence of Two Carbonyl Groups**: The molecule has two carbonyl groups at the 2-position and 4-position of the quinazoline ring. These functional groups contribute to potential hydrogen bonding and polarity, influencing the molecule's chemical reactivity and interaction with biological targets.\\

& 4. **Heteroaromatic Ring Systems**: Besides the quinazoline core, the presence of a pyridine ring at the 3-position adds a second heteroaromatic system, increasing the overall aromatic character and introducing additional nitrogen atoms, which can serve as potential sites for interaction or coordination with metal ions.\\

& Overall, the combination of a quinazoline structure with unique substitution patterns, including bromine halogenation and the presence of a pyridinyl group, defines the molecule's distinct chemical characteristics. This structure is particularly relevant in medicinal chemistry, where such frameworks are often explored for developing bioactive compounds.\\
\midrule

LLaMo & The molecule has a \textcolor{tabblue}{benzodiazepine structure} with a bromo substituent at the 7-position and a \textcolor{tabred}{methyl group at the 1-position}. It is distinct from other molecules in that it contains a bromine atom, a nitrogen atom, and a methyl group, which are not present in the other molecules. \\
\midrule
3D-MoLM & The molecule with the SMILES code C1C(=O)NC2=C(C=C(C=C2)Br)C(=N1)C3=CC=CC=N3 is known as 7-bromohypnotic. It belongs to the class of compounds called \textcolor{tabred}{quinazolines}, which are characterized by a bicyclic structure consisting of a benzene ring fused to a pyrimidine ring. The molecule contains a carbonyl group (C=O) and an amide group (C(=O)N) attached to the pyrimidine ring. These functional groups contribute to the molecule's ability to form hydrogen bonds and participate in various chemical reactions. The presence of a bromine atom (Br) on the benzene ring imparts unique properties to 7-bromohypnotic. Bromine is a halogen that is more electronegative than carbon, resulting in a polar bond. This polarity can influence the molecule's reactivity and solubility in different solvents. Additionally, the molecule contains a nitrogen atom (N) in the pyrimidine ring, which can act as a basic center and participate in acid-base reactions. The presence of conjugated double bonds in the benzene and pyrimidine rings gives rise to delocalized pi-electron systems, leading to enhanced stability and potential for aromaticity. Overall, 7-bromohypnotic exhibits distinct features such as the presence of a bromine atom, carbonyl and amide groups, and conjugated double bonds, which contribute to its unique chemical and physical properties. \\
& The molecule's structure suggests potential applications in medicinal chemistry, as quinazolines have been found to possess various biological activities, including \textcolor{tabred}{antitumor, antimicrobial, and anti-inflammatory properties}. Additionally, the presence of a bromine atom can enhance the molecule's lipophilicity, potentially affecting its bioavailability and distribution in the body. Further studies and experiments are necessary to fully understand the specific biological activities and potential therapeutic uses of 7-bromohypnotic. However, its unique molecular features make it an intriguing compound for further investigation in the field of chemistry and biology.\\
\midrule
Mol-Instructions & The molecule is a \textcolor{tabred}{quinoxaline derivative}. \\
\bottomrule
\end{tabular}
\vspace{-0.1in}
\caption{Entire responses of GPT-4o, LLaMo, 3D-MoLM, and Mol-Instructions for the case study in Table~\ref{tab:qualitative}.}
\vspace{-0.05in}
\label{tab:qualitative_full_part2}
\end{table*}