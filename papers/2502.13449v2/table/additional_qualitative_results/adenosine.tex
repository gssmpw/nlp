
\begin{table*}[ht!]
\centering
\begin{resizebox}{\linewidth}{!}{
\renewcommand{\arraystretch}{1.2}
\begin{tabular}{l c}
\toprule
\multicolumn{2}{p{22cm}}{\textbf{Original Description from PubChem}} \\
\multicolumn{1}{p{17.7cm}}{Adenosine is a nucleoside that is composed of adenine and d-ribose. Adenosine or adenosine derivatives play many important biological roles in addition to being components of DNA and RNA. For instance, adenosine plays an important role in energy transfer - as adenosine triphosphate (ATP) and adenosine diphosphate (ADP). It also plays a role in signal transduction as cyclic adenosine monophosphate, cAMP. Adenosine itself is both a neurotransmitter and potent vasodilator. When administered intravenously, adenosine causes transient heart block in the AV node. Because of the effects of adenosine on AV node-dependent supraventricular tachycardia, adenosine is considered a class V antiarrhythmic agent.} & \multirow{1}{*}{\includegraphics[height=2.8cm]{figure/qualitative_example_adenosine.pdf}} \\
\midrule
\multicolumn{2}{p{22cm}}{\textbf{User}} \\
\multicolumn{2}{p{22cm}}{What are the biological functionalities of this molecule?} \\
\midrule
\multicolumn{2}{p{22cm}}{\textbf{Mol-LLaMA}}\\
\multicolumn{2}{p{22cm}}{The given molecule, commonly known as adenosine, is a purine ribonucleoside composed of two main components: an adenine base and a ribose sugar. This structure allows it to play several critical biological roles:}\\

\multicolumn{2}{p{22cm}}{1. **Energy Transfer and Storage**: Adenosine is a precursor to adenosine triphosphate (ATP), which is the primary energy currency of the cell. It is involved in the storage and transfer of energy in cellular processes.}\\

\multicolumn{2}{p{22cm}}{2. **Signal Transduction**: As a signaling molecule, adenosine acts as a neuromodulator in the central and peripheral nervous systems. It binds to specific receptors, such as A1, A2A, A2B, and A3, influencing various physiological processes, including vasodilation, anti-inflammatory responses, and modulation of neurotransmitter release.}\\

\multicolumn{2}{p{22cm}}{3. **Metabolic Regulation**: Adenosine is involved in the regulation of glucose and lipid metabolism. \textcolor{tabblue}{It affects insulin secretion and sensitivity, which is crucial for maintaining glucose homeostasis.}}\\

\multicolumn{2}{p{22cm}}{4. **Cardiovascular Function**: In the cardiovascular system, adenosine plays a role in regulating heart rate and blood pressure. It acts as a vasodilator, which helps in reducing blood pressure and improving blood flow.}\\

\multicolumn{2}{p{22cm}}{5. **Immune System Modulation**: \textcolor{tabblue}{Adenosine can modulate immune responses by inhibiting the activity of immune cells, such as lymphocytes and macrophages.} This helps in preventing excessive inflammation.}\\

\multicolumn{2}{p{22cm}}{6. **Neuroprotection**: \textcolor{tabblue}{Adenosine has neuroprotective properties, which may help in protecting neurons from damage and degeneration.} It is involved in the regulation of synaptic plasticity and neuronal survival.}\\

\multicolumn{2}{p{22cm}}{7. **Role in DNA and RNA**: As a nucleoside, adenosine is a building block of nucleic acids, such as DNA and RNA. It is essential for the synthesis and repair of genetic material.}\\

\multicolumn{2}{p{22cm}}{8. **Cellular Communication**: Adenosine is involved in cellular communication processes, such as the regulation of cell growth, differentiation, and apoptosis.}\\

\multicolumn{2}{p{22cm}}{Overall, the given molecule is a multifunctional compound that plays a vital role in various physiological and biochemical processes in the body.}\\
\bottomrule
\end{tabular}}
\end{resizebox}
\vspace{-0.1in}
\caption{Case study for adenosine. Mol-LLaMA understands the name of molecules and their corresponding properties, providing an in-depth explanation. We highlight the properties that are not described in PubChem in \textcolor{tabblue}{blue}.}
\label{app:tab:additional_qualitative_adenosine}
\vspace{-0.1in}
\end{table*}