
\begin{table*}[ht!]
\centering
\begin{resizebox}{\linewidth}{!}{
\renewcommand{\arraystretch}{1.2}
\begin{tabular}{l c}
\toprule
\multicolumn{2}{p{22cm}}{\textbf{Original Description from PubChem}} \\
\multicolumn{1}{p{17.7cm}}{Nystatin is a polyene antifungal drug that has broad-spectrum fungicidal and fungistatic activity against a number of yeasts and fungi, most notably Candida species. It is one of the most effective antifungal agents synthesized by bacteria, in this case a strain of Streptomyces noursei, and is closely related to [amphotericin B], differing only slightly in structure. Nystatin has a greater antifungal activity than amphotericin B - parenterally administered nystatin, however, is associated with significant toxicity and is not available in a formulation appropriate for systemic use. As it undergoes very little absorption following oral or topical administration, nystatin's efficacy is limited to the treatment/prevention of cutaneous, mucocutaneous, and gastrointestinal fungal infections. Nystatin is a polyene macrolide antibiotic, with antifungal activity. Upon administration, nystatin binds to sterols in the fungal plasma membrane, thereby increasing membrane permeability. This leads to leakage of important intracellular components, cell rupture, and eventually fungal cell death.} & \multirow{4}{*}{\includegraphics[height=2.8cm]{figure/qualitative_example_Nystatin.pdf}} \\
\midrule
\multicolumn{2}{p{22cm}}{\textbf{User}} \\
\multicolumn{2}{p{22cm}}{Could you explain the biological implication of this molecule?} \\
\midrule
\multicolumn{2}{p{22cm}}{\textbf{Mol-LLaMA}}\\
\multicolumn{2}{p{22cm}}{The given molecule is a complex natural product known for its potent antifungal properties. It is produced by the fungus *Streptomyces nodosus* and is classified as a polyene macrolide. The biological implications of this molecule are primarily related to its ability to disrupt the cell membranes of fungi, making it an effective antifungal agent.}\\

\multicolumn{2}{p{22cm}}{\textcolor{tabblue}{The structure of the given molecule includes a large lactone ring, which is characteristic of macrolides. This ring is conjugated with multiple conjugated double bonds, forming a polyene chain. The polyene chain is crucial for its antifungal activity, as it allows the molecule to bind to ergosterol, a key component of fungal cell membranes. This binding results in the formation of pores in the membrane, leading to the leakage of essential ions and molecules, ultimately causing cell death.}}\\

\multicolumn{2}{p{22cm}}{Additionally, the presence of multiple hydroxyl groups and an amino sugar moiety in the structure enhances the molecule's solubility and binding affinity to the fungal cell membrane. The carboxylic acid group contributes to the molecule's overall polarity and may play a role in its interaction with the fungal cell surface.}\\

\multicolumn{2}{p{22cm}}{Overall, the given molecule's structure is intricately linked to its biological function as an antifungal agent, making it a valuable compound in the treatment of fungal infections.}\\
\bottomrule
\end{tabular}}
\end{resizebox}
\vspace{-0.1in}
\caption{Case study for Nystatin. Mol-LLaMA correctly understands the molecular features and explains them in detail. We highlight the rationales that Mol-LLaMA provides in \textcolor{tabblue}{blue}.}
\label{app:tab:additional_qualitative_nystatin}
\end{table*}