
\begin{table*}[ht!]
\centering
\vspace{-0.05in}
\footnotesize
\renewcommand{\arraystretch}{1.2}
\begin{tabular}{p{1.7cm} p{14.5cm}}
\toprule
 & \includegraphics[height=2.7cm]{figure/qualitative_example.pdf}\\
\midrule
SMILES & C1C(=O)NC2=C(C=C(C=C2)Br)C(=N1)C3=CC=CC=N3 \\
\midrule
Original \phantom{00} Description & Bromazepam binds to the GABA receptor GABA-A, causing a conformational change and \textbf{increasing inhibitory effects of GABA}. Other neurotransmitters are not influenced. One of the \textbf{benzodiazepines} that is \textbf{used in the treatment of anxiety disorders.} \\
\midrule
User & Explain the features of the given molecule that are most distinct from those of other molecules. \\
\midrule
Mol-LLaMA (Ours) & The given molecule is a member of the \textcolor{tabblue}{benzodiazepine class}, which is characterized by a fusion of a benzene ring and a seven-membered diazepine ring. This structure is crucial for its biological activity, particularly its role as \textcolor{tabblue}{a sedative and hypnotic agent}.\\

& One of the key features of the given molecule is \textcolor{tabblue}{the presence of a pyridine ring attached to the benzodiazepine core}. This pyridine moiety is significant because it can \textcolor{tabblue}{influence the molecule's pharmacokinetic properties, such as its ability to cross the blood-brain barrier}, which is essential for its function as a central nervous system depressant.\\

& Additionally, the molecule contains a \textcolor{tabblue}{bromine atom attached to the benzodiazepine ring}. \textcolor{tabblue}{The presence of this halogen atom can affect the molecule's lipophilicity and its binding affinity to the GABA-A receptor}, which is the primary target of benzodiazepines. The bromine atom may also contribute to the molecule's metabolic stability and its duration of action.\\

& \textcolor{tabblue}{The given molecule is also characterized by a ketone group within the diazepine ring, which is a common feature in benzodiazepines}. This functional group is important for the interaction with the GABA-A receptor, as it plays a role in the formation of the benzodiazepine-receptor complex, \textcolor{tabblue}{enhancing the inhibitory effects of GABA}.\\

& Overall, the structural features of the given molecule, including the benzodiazepine core, the pyridine ring, the bromine atom, and the ketone group, collectively contribute to its pharmacological properties as a sedative and hypnotic agent.\\
\bottomrule
\end{tabular}
\vspace{-0.1in}
\caption{Entire responses of Mol-LLaMA for the case study in Table~\ref{tab:qualitative}.}
\vspace{-0.2in}
\label{tab:qualitative_full_part1}
\end{table*}