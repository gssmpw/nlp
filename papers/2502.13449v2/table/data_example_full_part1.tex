
\begin{table*}[h!]
\centering  
\footnotesize

\begin{tcolorbox}[colframe=gray,colback=white,boxrule=1pt,arc=0.6em,boxsep=0mm,left=5.pt,right=15pt]
\renewcommand{\arraystretch}{1.4}
\begin{tabular}{l c}
    \multicolumn{1}{l}{\textbf{\normalsize{Context Type 1: IUPAC Name}}} & \hspace{-0.15in} \multirow{2}{*}{\vspace{-0.1in} \includegraphics[height=2.6cm]{figure/example3.pdf}} \\
    \multicolumn{1}{p{.67\textwidth}}{(2R)-2-amino-3-methylsulfanylpropanoic acid} &  \\
    \multicolumn{1}{l}{\textbf{\normalsize{Context Type 2: Description}}} & \\
    \multicolumn{1}{p{.67\textwidth}}{S-methylcysteine is a cysteine derivative that is L-cysteine in which the hydrogen attached to the sulfur is replaced by a methyl group. It has a role as a human urinary metabolite and a plant metabolite. It is a tautomer of a S-methylcysteine zwitterion.} &  \\
    \cmidrule{1-2}
\end{tabular}

\renewcommand{\arraystretch}{1.4}
\begin{tabular}{p{0.95\textwidth}}
    \textcolor{figgreen}{\textbf{\normalsize{Data Type 1: Detailed Structural Descriptions}}} \\
\end{tabular}

\renewcommand{\arraystretch}{1.2}
\begin{tabular}{p{0.95\textwidth}}
    The molecule provided is a chiral amino acid that contains several identifiable substructures. \\
    1. **\textbf{Amino Group}**: This functional group consists of a nitrogen atom bonded to two hydrogen atoms (\(-NH_2\)). It is attached to the second carbon atom of the main carbon chain, \textbf{which is typical for amino acids} as they feature the amino group connected to the alpha carbon.\\
    2. **\textbf{Carboxylic Acid Group}**: Represented by \(-COOH\), this group includes a carbon atom double-bonded to an oxygen atom and also bonded to an \(-OH\) group. It is attached to the end of the carbon chain, \textbf{characteristic of amino acids} which bear a carboxylic acid group at the terminal carbon.\\
    3. **\textbf{Main Carbon Chain}**: The molecule has a three-carbon chain serving as the backbone. The central carbon, referred to as the alpha carbon, is a chiral center, meaning it is bonded to four different substituents: the amino group, the carboxylic acid group, a hydrogen atom, and a side chain.\\
    4. **\textbf{Methylsulfanyl Group}**: Attached to the third carbon is a side chain consisting of a sulfur atom single-bonded to a methyl group (\(-SCH_3\)). This group acts as the side chain substituent and characterizes the molecule as containing a \textbf{thioether linkage}, where the sulfur atom replaces the carbon in the typical ether linkage.\\
    5. **\textbf{Chirality}**: The molecule is chiral, with the alpha carbon being the stereocenter. Its configuration is designated as \(R\), indicating the spatial arrangement of its substituents according to the Cahn-Ingold-Prelog priority rules.\\
    Overall, this molecule is a sulfur-containing, optically active alpha amino acid with a focus on the specific arrangement of its functional groups including the amino, carboxylic acid, and a distinct methylsulfanyl side chain. This structure is essential for understanding its interactions and functions in biological systems.\\
    \cmidrule{1-1}
\end{tabular}

\renewcommand{\arraystretch}{1.4}
\begin{tabular}{p{0.95\textwidth}}
    \textcolor{figgreen}{\textbf{\normalsize{Data Type 2: Struture-to-Feature Relationship Explanations (Chemical Features)}}} \\
\end{tabular}

\renewcommand{\arraystretch}{1.2}
\begin{tabular}{p{0.95\textwidth}}
    \textbf{The given molecule is a derivative of cysteine}, an amino acid known for containing a thiol group. In the given molecule, this thiol group has one of its hydrogen atoms replaced by a methyl group, making it a methylthioether. This structural tweak to cysteine results in distinct chemical properties. \\
    Firstly, \textbf{the presence of the amino group (-NH2) classifies this compound as an amino acid, which contributes to its ability to act as a zwitterion.} This means that in aqueous solutions or physiological pH, it can contain both a positive charge (on the ammonium group) and a negative charge (on the carboxylate group), enhancing its solubility in water. \\
    \textbf{The sulfur atom with a methyl attachment forms a thioether. This increases the molecule's hydrophobic characteristics compared to cysteine}, which contains a more polar thiol group. This methylsulfanyl group can also affect the molecule's reactivity and interaction with biochemical systems, possibly altering how it participates in or affects metabolic pathways.\\
    The stereochemistry indicated by 'R' specifies the spatial arrangement of groups around the chiral carbon atom, which corresponds to natural configurations found in biological systems like human and plant metabolites. This natural configuration is crucial for its biological activity and integration into metabolic processes.\\
    As a human urinary metabolite, the presence of this compound in urine suggests its role in human metabolism where it might play a role in sulfur metabolism or act as a detoxification product. In plants, it could partake in sulfur metabolism and possibly aid in detoxification or defense processes due to environmental stressors.\\
    In summary, the structural modification from cysteine to include a methylthioether enhances its lipophilic properties and potentially alters its metabolic and biochemical roles due to changes in reactivity, solubility, and interaction with enzymes or other biomolecules.\\
\end{tabular}
\end{tcolorbox}
\vspace{-0.1in}
\caption{One example of the constructed instruction data. The top block shows the example of context types when prompting GPT-4o. The following two blocks show the responses of GPT-4o.}
\label{app:tab:data_example_full_part1}
\end{table*}
