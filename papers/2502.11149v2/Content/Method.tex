\section{Method}
In this section, we first introduce the preliminaries related to geometric modeling in \cref{sec:3.1}. Next, in \cref{sec:3.2}, we present the proposed framework EquiLLM. Finally, in \cref{sec:3.3}, we describe how the EquiLLM framework is applied to two representative tasks (\emph{e.g.} dynamic simulation and antibody design). The overview of our EquiLLM is illustrated in \cref{fig:architecture}.

\subsection{Preliminaries, Notations and Definitions}
\label{sec:3.1}
Physical systems (such as molecules) can be naturally modeled with geometric graphs. We represent each static system as a geometric graph $\gG = (\gV,\gE)$, where each node $v_i$ in $\gV$ is associated with an invariant feature $\vh_{i}\in\R^c$ (\emph{e.g.} atom type) and an 3D equivariant vector $\vec{\vx}_{i}\in\R^3$ (\emph{e.g.} atom coordinates); each edge (\emph{e.g.} chemical bonds) denotes the connectivity between nodes. Apart from modeling static systems, we explore dynamic systems, focusing on constructing geometric graphs across different time steps. The details will be thoroughly discussed in \cref{sec:3.3.1}. In the following sections, we use the matrices $\Vec{\mX}\in\R^{N \times 3}$ and $\mH\in\R^{N \times c}$ to denote the sets of node coordinates and invariant features of the geometric graph $\gG$.

\textbf{Task Formulation.} Here, we provide a general form of our task and will elaborate specific applications including dynamic simulation and antibody design in \cref{sec:3.3}. Given the input geometric graph $\gG^{\text{in}}$, our goal is to find a function $\phi$ to predict the output $\gG^{\text{out}}$. This process can be formally delineated as:
\begin{equation}\label{eq:task_general}
\begin{aligned}
    &\gG^{\text{out}} = \phi(\gG^{\text{in}}). \\
\end{aligned}
\end{equation}
Meanwhile, since we introduce LLMs into our framework, we will further construct task-specific prompts to guide the extraction of relevant domain knowledge from LLMs, recasting our task as:
\begin{equation}\label{eq:dynamic2}
\begin{aligned}
    &\gG^{\text{out}} = \phi(\gG^{\text{in}}, \mP), \\
\end{aligned}
\end{equation}
where $\mP$ denotes the prompt.

\textbf{Equivariance.} It is crucial to emphasize that in the tasks above, the function $\phi$ must satisfy $\mathrm{E}(3)$ symmetries of physical laws. Specifically, if arbitrary translations, reflections, or rotations are applied to the input coordinate matrix $\Vec{\mX}^{\text{in}}$, the output coordinate matrix $\Vec{\mX}^{\text{out}}$ should undergo the corresponding transformation.

\subsection{Large Language Geometric Model}
\label{sec:3.2}
In this section, we provide a meticulous description of our model EquiLLM, which consists of three main components: Equivariant Encoder, LLM, and Equivariant Adapter. Unlike existing works~\citep{gruver2024finetuned} that applys an LLM to predict the 3D coordinates directly, EquiLLM leverages an LLM to acquire broader scientific domain knowledge while employing geometric GNNs for precise modeling of 3D structures. These two components are seamlessly integrated through an equivariant adapter, achieving superior predictive performance without compromising $\mathrm{E}(3)$-equivariance.


\textbf{Equivariant Encoder.} The Equivariant Encoder is a domain-specific equivariant model, which can be any suitable equivariant model from the relevant field. The model takes the graph $\gG^{\text{in}} = (\gV^{\text{in}},\gE^{\text{in}})$ as input, performing initial encoding and embedding of geometric information, and outputs a processed geometric graph $\gG' = (\gV',\gE')$, This process can be formally defined as:
\begin{equation}\label{eq:equ_encoder}
\begin{aligned}
    &\gG' = \phi_{e} (\gG^{\text{in}}), \\
\end{aligned}
\end{equation}
where $\phi_{e}$ can be any equivariant model, used to jointly model the geometric relationships between $\Vec{\mX}^{\text{in}}$ and $\mH^{\text{in}}$ features across different nodes, resulting in processed features $\Vec{\mX}^{'}$ and $\mH^{'}$.

Since LLMs are not naturally equivariant, directly feeding $\mX'$ into an LLM would likely undermine the intrinsic equivariance of the overall architecture. Thus, in contrast to existing works, we convey the invariant features $\mH'$ to the LLM, but pass the equivariant matrix $\mX'$ to the subsequent Equivariant Adapter via a skip connection. Before feeding $\mH'$ to the LLM, we first conduct a projector on $\mH'$ to align its dimension with the input space of the LLM. This process can be formally characterized as:
\begin{equation}\label{eq:equ_encoder_proj}
\begin{aligned}
    \mH^{\text{proj}} = \phi_{\text{proj}} (\mH'), \\
\end{aligned}
\end{equation}
where $\phi_{\text{proj}}$ is implemented as a linear layer in EquiLLM.

\textbf{Geometric-aware Prompt.} One may directly input the aligned features $\mH^{\text{proj}}$ into the LLM to make the final predictions. However, this approach overlooks the pivotal role of the prompt, as it does not utilize the linguistic form of the prompt to effectively harness the LLM's comprehension and articulation of the specific task at hand. Therefore, in the EquiLLM framework, we carefully design task-specific prompts for different tasks to unleash domain-specific knowledge. 

The prompt content for all tasks can be broadly divided into three key components: (1) \emph{task description}, (2) \emph{object feature description}, and (3) \emph{object statistical information}.

\textbf{$\triangleright$ Task description.} The task description consists of two parts: <Task> and <Requirement>. <Task> appears at the beginning of the prompt, providing a succinct description of the task to help the LLM quickly identify the task's objective. <Requirement> is located in the main body of the prompt and elaborates on the input-output requirements and constraints of the task, ensuring a comprehensive understanding of the task by the LLM. 

\textbf{$\triangleright$ Object feature description.} The feature description of the input object begins with <Object> and primarily outlines the composition information as well as the structural characteristics of the input object. 

\textbf{$\triangleright$ Object statistical information.} This components starts with <Statistics>, encapsulating detailed metrics pertaining to the distribution of the object's coordinates in 3D space, including the maximum, minimum, and mean values. It is crucial to note that, unlike conventional tasks, directly incorporating absolute coordinate values into the prompt is not recommended in 3D spatial modeling tasks. This is due to the fact that transformations such as translation, reflection, or rotation applied to the input object will invariably alter the corresponding coordinate distribution, thereby violating the principle that the prompting process must remain $\mathrm{E}(3)$-invariant. Consequently, we represent the coordinate distribution of the input object indirectly by computing statistical metrics related to distances. 


\textbf{Large Language Model (LLM).} After designing the prompt, we employ the tokenizer and embedding layer of the LLM to obtain the corresponding word embedding features, denoted as $\mP$. Subsequently, depending on the specific task, we concatenate $\mP$ with the invariant features $\mH^{\text{proj}}$ in an appropriate way. The concatenation strategies for different tasks will be discussed in detail in \cref{sec:3.3}. Next, the concatenated features are input into the LLM with the aim of unlocking and leveraging the scientific knowledge embedded within the LLM to enhance the model's understanding and reasoning capabilities for the relevant tasks. Unlike previous works like CrystalLLM~\citep{gruver2024finetuned} and UniST~\citep{yuan2024unist} that require fine-tuning some layers within the LLM, resulting in significant computational costs and time expenditure, the EquiLLM framework freezes all parameters of the LLM, eliminating the need for additional training. This process can be roughly represented as follows:
\begin{equation}\label{eq:llm}
\begin{aligned}
    &[\mH^{\text{llm}}, \mP^{\text{llm}}] = \mathbf{LLM} (\texttt{Concat} (\mH^{\text{proj}}, \mP)), \\
\end{aligned}
\end{equation}

\textbf{Equivariant Adapter.} Upon obtaining the output from the LLM, we extract the part corresponding to the invariant features, denoted as $\mH^{\text{llm}}$. While directly utilizing it for final predictions may be viable for invariant tasks (\emph{e.g.} predicting the energy of a molecular system), it is inadequate for equivariant tasks, where the core objective is to predict the 3D coordinates of objects. To address this challenge, we propose the Equivariant Adapter, which leverages one-layer EGNN to process $\mH^{\text{llm}}$ while minimizing the introduction of excessive additional parameters. Specifically, we first employ a projection layer to re-project $\mH^{\text{llm}}$ back into the space corresponding to the invariant features $\mH'$ and add it with $\mH'$, yielding the refined feature representation $\mH^{r}$. Then, both the equivariant coordinate features $\Vec{\mX'}$ from Equivariant Encoder and the refined invariant features $\mH^{r}$ are transmitted to the EGNN, yielding the output $\Vec{\mX}^{\text{out}}$ and $\mH^{\text{out}}$. The whole process is formally expressed as:
\begin{equation}\label{eq:EA}
\begin{aligned}
    &\vm_{ij} = \varphi_m \left(\vh_{i}^{r},\, \vh_{j}^{r},\, \left\| \vec{\vx}'_{i} - \vec{\vx}'_{j} \right\|\right), \\
    &\vh_{i}^{\text{out}}  = \textstyle\vh_{i}^{r} + \varphi_{h}\left(\vh_{i}^{r},\sum_{j \in \mathcal{N}(i)}\vm_{ij}\right), \\
    &\vec{\vx}_{i}^{\text{out}} = \textstyle\vec{\vx}'_{i}+\frac{1}{|\mathcal{N}(i)|}\sum_{j \in \mathcal{N}(i)}\varphi_{x} \left(\vm_{ij}\right)\cdot\left(\vec{\vx}'_{i}-\vec{\vx}'_{j}\right), \\ 
\end{aligned}
\end{equation}
where $\varphi_{m}$, $\varphi_{x}$, and $\varphi_{h}$ denote Multi-Layer Perceptrons (MLPs), and $\mathcal{N}(i)$ refers to the set of neighboring nodes associated with the $i$-th node. Specifically, $\vm_{ij}$ represents an $\mathrm{E}(3)$-invariant message transmitted from node $j$ to node $i$, which is utilized to aggregate and refine the feature vector $\vh_{i}^{r}$ via the function $\varphi_{h}$. Regarding the update of $\vec{\vx}'_{i}$, the function $\varphi_{x}$ is employed to compute a scalar $\varphi_{x} (\vm_{ij})$, which is subsequently multiplied by the difference $\vec{\vx}'_{i} - \vec{\vx}'_{j}$ to retain directional information, while incorporating residual connections to ensure translation equivariance.


Our EquiLLM framework guarantees that the overall architecture preserves the critical property of $\mathrm{E}(3)$-equivariance in 3D space while also avoiding the introduction of lengthy text context due to direct 3D coordinate input, which could severely affect the efficiency of training and inference. Moreover, compared with domain-specific Equivariant Encoder, EquiLLM introduces only two projection layers and a one-layer EGNN network, significantly reducing the additional training parameters in comparison to the existing literature~\citep{jin2024timellm, yuan2024unist}. Finally, our EquiLLM framework demonstrates exceptional flexibility and can be applied to various geometric modeling tasks, showcasing its robustness and generalizability.

\subsection{Applications on Dynamic Simulation and Antibody Design}
\label{sec:3.3}
In this section, we will present a detailed discussion on the application of our EquiLLM in dynamic simulation and antibody design.
\subsubsection{Dynamic Simulation}
\label{sec:3.3.1}
While our model is applicable to the simulations of both molecular dynamics and human motions, we illustrate molecular dynamics here as an example. Given the 3D coordinate trajectory of a physical system (\emph{e.g.,} molecules) over $T$ frames, namely, $\Vec{\mX}\in\mathbb{R}^{T \times N \times 3}$, along with the invariant features $\mH\in\mathbb{R}^{N \times c_{a}}$ encoded by atomic numbers, the model aims to infer future trajectories $\Vec{\mX}\in\mathbb{R}^{F \times N \times 3}$ for $F$ subsequent frames.

\textbf{Geometric-aware Prompt.}
Here, we will provide a general overview of the contents encompassed within the prompt.

\textbf{$\triangleright$ Task description.} The model is tasked with predicting the 3D coordinates $(x, y, z)$ of heavy atoms for the next $F$ frames based on the information from the previous $T$ frames.

\textbf{$\triangleright$ Object feature description.} For molecular systems, the emphasis is on compositional information and structural characteristics.


\textbf{Task Formulation and Training Objective. }
With $\gG_{1:T}\coloneqq\{\gG_{t} = (\Vec{\mX}_{t}, \mH_{t}, \gE)\}_{t=1}^{T}$ and $\gG_{T+1:T+F}\coloneqq\{\gG_{t} = (\Vec{\mX}_{t}, \mH_{t}, \gE)\}_{t=T+1}^{T+F}$, we provide the entire process as follows:
\begin{equation}\label{eq:task_dynamic}
\begin{aligned}
    \gG_{T+1:T+F} = \phi (\gG_{1:T}, \mP). \\
\end{aligned}
\end{equation}
Let $\Vec{\mX}^{\text{gt}}_{T+f}$ denote the ground-truth 3D coordinates for the time period from $T+1$ to $T+F$, we define the object function as $\textstyle\mathcal{L} = \frac{1}{|F|}\sum^{F}_{f=1}\ell_{\mathrm{mse}}(\Vec{\mX}^{\text{out}}_{T+f}, \Vec{\mX}^{\text{gt}}_{T+f})$ refers to the mean squared error (MSE).

\subsubsection{Antibody Design}
\label{sec:3.3.2}
Antibodies are Y-shaped proteins primarily responsible for recognizing and binding to specific antigens. Current research predominantly focuses on the variable region. The variable region is present in both the heavy and light chains of the antibody and can be further subdivided into the framework region and three Complementarity-Determining Regions (CDRs). These six CDRs are critical in determining the affinity between the antibody and antigen, with the CDR-H3 region on the heavy chain exhibiting the most pronounced variability. Consequently, the primary objective of this paper is to predict the amino acid sequence and the 3D coordinates of the CDR-H3 region, given the antibody-antigen complexes excluding the CDR-H3 region. In antibody design task, each node in $\gV$ associates with a trainable feature $\vh_i\in\R^{c_r}$ encoded by amino acid type and a matrix of 3D coordinates $\Vec{\mZ}_i \in \R^{4 \times 3}$. We choose 4 backbone atoms $\{\text{N}, \text{C}_\alpha, \text{C}, \text{O}\}$ to constitute $\Vec{\mZ}_i$.

\textbf{Geometric-aware Prompt.}
Here, we will provide a general overview of the contents encompassed within the prompt.

\textbf{$\triangleright$ Task description.} The model is tasked with predicting both the 1D sequence and 3D coordinates of CDR-H3 region. 

\textbf{$\triangleright$ Object feature description.} The structural features of the light chain, heavy chain, and antigen, which are described individually.



\textbf{Task Formulation and Training Objective.}
With $\gG = (\Vec{\mX}, \mH, \gE)$, where $(\Vec{\mX}$, $\mH )\coloneqq \{ (\Vec{\mZ_{i}}, \vh_{i}) \} ^{N}_{i=1}$, the entire process is delineated as follows:
\begin{equation}\label{eq:task_antibody}
\begin{aligned}
    &\mH^{\text{out}}, \Vec{\mX}^{\text{out}} = \phi_{r} (\gG, \mP_{r}), \\
    &\vy^{\text{out}} = \texttt{Softmax}(\mH^{\text{out}}), \\
\end{aligned}
\end{equation}
where $N$ denotes the number of residues in CDR-H3 region; $\vy^{\text{out}}$ and $\vy^{\text{gt}}$ denote the predicted distribution over all amino acid categories and the ground truth amino acid type; $\Vec{\mX}^{\text{out}}$ and $\Vec{\mX}^{\text{gt}}$ denote the predicted 3D structure and the ground truth 3D structure of the CDR-H3 region, respectively. The loss function is definited as $\mathcal{L} = \mathcal{L}_{\mathrm{seq}} + \lambda \mathcal{L}_{\mathrm{struct}}$, where $\mathcal{L} _{\mathrm{ce}}= \frac{1}{N}\ell_{ce}(\vy^{\text{out}}, \vy^{\text{gt}})$ denotes the cross entropy and $\mathcal{L} _{\mathrm{huber}}\frac{1}{N}\ell_{\mathrm{huber}}(\Vec{\mX}^{\text{out}}, \Vec{\mX}^{\text{gt}})$ denotes the Huber loss~\citep{huber1992robust}; the $\lambda$ is used to balance the two losses.
