\section{Discussion}

\sssec{Adaptation to heterogeneous environments}. \sysname's ability to adapt to dynamic and heterogeneous GPU clusters provides significant advantages for real-world deployments. In contrast to other frameworks that may be optimized for specific hardware setups, \sysname can adjust its strategies based on the available resources, making it particularly suited for cloud-based or distributed environments where hardware configurations may vary.

\sssec{Scalability and future directions}. The scalability demonstrated by \sysname, particularly in handling large models like Llama-70B, is a testament to its robustness. However, as models and GPU configurations continue to grow in complexity, further optimizations in the simulation phase could be crucial for maintaining performance. This suggests future work may focus on improving the efficiency of the simulation component to prevent bottlenecks as the search space expands.

%\sssec{Limited adaptability to novel hardware}. \sysname's performance depends heavily on the existing GPU configurations and cost models predefined in the system. While it supports a range of popular GPUs (e.g., NVIDIA A100, V100), its adaptability to emerging hardware or less common configurations might be limited. As AI hardware rapidly evolves, \sysname could face challenges in staying up-to-date with new architectures or specialized accelerators like TPUs or proprietary chips from newer vendors, which may not yet be fully integrated into its cost models and parallelization strategies.

%\sssec{Complexity in rule-based filtering}. The rule-based filter system in \sysname provides flexibility, allowing users to impose specific constraints on the search space. However, this approach can introduce complexity, especially for non-expert users who struggle to define the appropriate rules for their parallelization strategy. Crafting these rules requires a solid understanding of the training architecture and hardware environment, potentially reintroducing a level of manual effort that \sysname aims to eliminate. Simplifying this process or providing more intuitive guidance could enhance usability.