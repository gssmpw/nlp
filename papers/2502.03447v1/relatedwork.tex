\section{Related Work}
% As noted in the introduction, the ``\verb|acmart|'' document class can
% be used to prepare many different kinds of documentation --- a
% double-blind initial submission of a full-length technical paper, a
% two-page SIGGRAPH Emerging Technologies abstract, a ``camera-ready''
% journal article, a SIGCHI Extended Abstract, and more --- all by
% selecting the appropriate {\itshape template style} and {\itshape
%   template parameters}.

% This document will explain the major features of the document
% class. For further information, the {\itshape \LaTeX\ User's Guide} is
% available from
% \url{https://www.acm.org/publications/proceedings-template}.

\subsection{Affordance and autistic children}
Affordances refer to the opportunities for perception and action that the environment offers to animals~\cite{gibson1977theory}. The term, coined by Gibson, evolved when Donald Norman introduced it to design, distinguishing between perceived and actual affordances~\cite{norman2013design, norman1988psychology} and highlighting potential everyday inconsistencies between them.
Loveland et al. categorize environmental affordances into three levels: physical transactions, cultural preferences, and social signals reflecting others' meanings. Autistic children often struggle with the latter two types~\cite{loveland1991social}. Difficulties with cultural affordances affect tool usage and may link to neurological issues~\cite{osiurak2017affordance}. Studies have explored interventions like electrical stimulation to enhance this understanding~\cite{lopes2015affordance++}. Understanding social signals involves grasping others' expectations, and research on helping autistic children in this area remains limited~\cite{ramstead2016cultural}.
Perceptual issues with affordances negatively impact social participation of autistic children and contribute to adverse behaviors, reducing learning opportunities~\cite{hellendoorn2014understanding}. While emotional signal transmission is crucial for interpersonal coordination~\cite{schmidt2008dynamics, hale2020you, hobson1991against}, autistic children's difficulties with social affordances~\cite{loveland1991social, hobson1993emotional, hobson1989sharing, hobson1990acquiring} lead to challenges in social adaptation and independent living~\cite{frith1994autism, schreibman1973overselective, kanner1972far, carter2005social}.
% Affordances refer to the opportunities for perception and action that the environment offers to animals~\cite{gibson1977theory}. The term, coined by Gibson, evolved when Donald Norman introduced it to design, distinguishing between perceived and actual affordances~\cite{norman2013design, norman1988psychology} and highlighting potential everyday inconsistencies between them.
% Loveland et al. categorize environmental affordances into three levels: physical transactions, cultural preferences, and social signals reflecting others' meanings. Autistic children often struggle with the latter two types~\cite{loveland1991social}. Difficulties with cultural affordances affect tool usage and may link to neurological issues~\cite{osiurak2017affordance}. Studies have explored interventions like electrical stimulation to enhance this understanding~\cite{lopes2015affordance++}. Understanding social signals involves grasping others' expectations, and research on helping autistic children in this area remains limited~\cite{ramstead2016cultural}.
% Perceptual issues with affordances negatively impact social participation of children with autism spectrum disorder and contribute to adverse behaviors, reducing learning opportunities~\cite{hellendoorn2014understanding}. While emotional signal transmission is crucial for interpersonal coordination~\cite{schmidt2008dynamics, hale2020you, hobson1991against}, autistic children's difficulties with social affordances~\cite{loveland1991social, hobson1993emotional, hobson1989sharing, hobson1990acquiring} lead to challenges in social adaptation and independent living~\cite{frith1994autism, schreibman1973overselective, kanner1972far, carter2005social}.
% Loveland et al. the affordances of human environment affordances into three levels: physical transactions, cultural preferences, and socithat al sign the meanings of otherrs' meanings. Autistic children often struggle with the latte of affordancesr two types~\cite{loveland1991social}. Diin understandingulties with cultural acan ffordances affect tool usabe ge aednd may link to neurological issues~\cite{osiurak2017afPrevious sordance}. Studies have explored in, such asntions like electrical ,stimulation to enhance this understanding~\cite{lopes2015affordance++}.
% Affordances refer to the opportunities for perception and action that the environment offers to animals~\cite{gibson1977theory}. The term, coined has by Gibsoover time.volved when Donald Norman 'affordance' to design, suggesting that affordances indicate how artifacts should be used. He distinguishedtinguishing between perceived and actual affordances~\cite{norman2013design, norman1988,hology} and highlighting potential everyday inconsistencies bet% ween them.


% lopes2015affoThe comprehension ofderstanding social signals involve gring others' e of others, and deficits in this area can hinder social integration. Studying how people understand social and cultural contexts is challenging, with limited research focused on aidrch on helping autistic childrn garains limited~\cite{ramstead2016% cultural}.
% Perceptual issues with affordances negativthe ely impact social participation of children with autism spectrum disorder and contribute to advers in everyday environmentse behaviors, reducing learning opportunities~\cite{hellendoorn2014underThe transmission of emotional signals and corresponding responsesransmission is crucial for interpersonal coordination~\cite{schmidt2008dynamics, hale2020you, hobson19 but91against}, autistic children's difficulties with social  disrupt this processaffordances~\cite{loveland1991social, hobson1993emotional, hobson1989sharing, hobson199,0acquingiring} lead to challenges in social adaptation and independent living~\cite{frith1994autism, schreibman1973overselective, kanner1972far, carter2005social}.

% The primary parameter given to the ``\verb|acmart|'' document class is
% the {\itshape template style} which corresponds to the kind of publication
% or SIG publishing the work. This parameter is enclosed in square
% brackets and is a part of the {\verb|documentclass|} command:
% \begin{verbatim}
%   \documentclass[STYLE]{acmart}
% \end{verbatim}

% Journals use one of three template styles. All but three ACM journals
% use the {\verb|acmsmall|} template style:
% \begin{itemize}
% \item {\verb|acmsmall|}: The default journal template style.
% \item {\verb|acmlarge|}: Used by JOCCH and TAP.
% \item {\verb|acmtog|}: Used by TOG.
% \end{itemize}

% The majority of conference proceedings documentation will use the {\verb|acmconf|} template style.
% \begin{itemize}
% \item {\verb|acmconf|}: The default proceedings template style.
% \item{\verb|sigchi|}: Used for SIGCHI conference articles.
% \item{\verb|sigchi-a|}: Used for SIGCHI ``Extended Abstract'' articles.
% \item{\verb|sigplan|}: Used for SIGPLAN conference articles.
% \end{itemize}

\subsection{Immersive environment for education}
Autistic children often experience significant deficits in social understanding and skills. Immersive systems provide effective learning environments and support mechanisms to help address these challenges ~\cite{cheng2015using}. Studies have shown these technologies successfully improve social behaviors and enhance communication and emotional skills ~\cite{halabi2017immersive,bekele2016multimodal,lorenzo2016design}. Through embodied interactions in virtual environments, children can safely explore new behavioral opportunities independently ~\cite{halabi2017immersive,matsentidou2014immersive}. These immersive settings provide a safe and inclusive space that reduces the hazards and unpredictability of real-life situations.
Immersive environments are typically delivered through VR systems with Head Mounted Displays (HMDs) or projection-based systems like Cave Automatic Virtual Environments (CAVE) ~\cite{bozgeyikli2017survey,burdea2003virtual}. While VR systems face limitations with autistic children often resisting headsets ~\cite{liu2017technology,soltiyeva2023my}, CAVE environments have proven effective for teaching safety skills, such as crossing streets or avoiding vehicles ~\cite{tzanavari2015effectiveness}. The current landscape of immersive environment systems presents significant research opportunities, particularly in integrating LLMs to enhance them with intelligent capabilities for memory, planning, and execution, though research in this area remains limited.
% Autistic children often experience significant deficits in social understanding and skills. Immersive systems have been shown to alleviate these deficits by providing effective learning environments and support mechanisms ~\cite{cheng2015using}. Previous studies have explored the use of immersive technologies as therapeutic tools for autism, confirming their success in improving social behaviors and enhancing communication and emotional skills ~\cite{halabi2017immersive,bekele2016multimodal,lorenzo2016design}. These environments engage and satisfy autistic children, guiding them through a series of social tasks via embodied interactions. In the controlled stimuli of virtual environments, children can explore new behavioral opportunities independently ~\cite{halabi2017immersive,matsentidou2014immersive}. Importantly, these immersive settings provide a safe and inclusive space that reduces the hazards and unpredictability of real-life situations, thereby shielding children from potential harm.


% Immersive environments can primarily be delivered through two types of systems: virtual reality (VR) systems that utilize wearable Head Mounted Displays (HMDs) to create immersive experiences, and projection-based systems, such as Cave Automatic Virtual Environments (CAVE) or desktop systems that project virtual environments into real-world settings ~\cite{bozgeyikli2017survey,burdea2003virtual}. However, VR systems have limitations, as many autistic children may resist wearing head-mounted devices ~\cite{liu2017technology,soltiyeva2023my}. In contrast, projection-based CAVE environments are effective for teaching children skills needed to navigate unsafe situations, such as crossing streets or avoiding vehicles. Tzanavari et al. have demonstrated the effectiveness of immersive virtual environments in teaching autistic children how to safely cross pedestrian crosswalks ~\cite{tzanavari2015effectiveness}. Nonetheless, the current landscape of immersive environment systems presents significant research opportunities that remain largely unexplored. Integrating LLMs into these environments could enhance them into intelligent systems capable of memory, planning, and execution; however, research in this area is still limited.







% In addition to specifying the {\itshape template style} to be used in
% formatting your work, there are a number of {\itshape template parameters}
% which modify some part of the applied template style. A complete list
% of these parameters can be found in the {\itshape \LaTeX\ User's Guide.}

% Frequently-used parameters, or combinations of parameters, include:
% \begin{itemize}
% \item {\verb|anonymous,review|}: Suitable for a ``double-blind''
%   conference submission. Anonymizes the work and includes line
%   numbers. Use with the \verb|\acmSubmissionID| command to print the
%   submission's unique ID on each page of the work.
% \item{\verb|authorversion|}: Produces a version of the work suitable
%   for posting by the author.
% \item{\verb|screen|}: Produces colored hyperlinks.
% \end{itemize}

% This document uses the following string as the first command in the
% source file:
% \begin{verbatim}
% \documentclass[sigconf,authordraft]{acmart}
% \end{verbatim}

\subsection{LLM-simulated social interaction}
Large Language Models like ChatGPT has demonstrated remarkable ability in generating human-like responses \cite{brown2020language}. 
LLMs excel in fundamental tasks like translation \cite{susnjak2022chatgpt}, conversation generation, and code writing, and have made significant advances in more complex domains such as autonomous decision-making and role-playing, as demonstrated by applications such as AutoGPT \cite{yang2023autogpt} and HuggingGPT \cite{shen2023hugginggpt} in task planning and execution.
These advances have enabled practical applications across various domains - from creating character-aligned dialogues and simulating human behaviors in role-playing scenarios \cite{Shanahan2023, 9980408, park2023generative}, to serving specialized functions like educational teaching assistance \cite{Celik2022} and psychological counseling for individuals with high-functioning autism \cite{cho2023evaluating}.
Recent research has developed innovative LLM applications. A project created a simulated job fair environment for training generative agents with enhanced communication capabilities \cite{li2023metaagents}, while another advanced social network simulation by modeling agents with emotional and interactive capabilities \cite{gao2023s3}. Previous research introduced an alignment learning approach that leverages simulated society interactions, providing collective ratings and iterative feedback \cite{liu2023training}. These developments showcase LLMs' potential in mimicking human social interactions, suggesting a future where AI agents can participate in sophisticated social behaviors.
% LLMs have revolutionized NLP, enabling applications far beyond basic text generation. ChatGPT has demonstrated remarkable ability in generating human-like responses \cite{brown2020language}. LLMs excel in tasks like translation \cite{susnjak2022chatgpt}, conversation generation, and code writing, while also advancing into decision-making and role-playing. AutoGPT \cite{yang2023autogpt} and HuggingGPT \cite{shen2023hugginggpt} demonstrate LLMs' capability in task planning and execution. In role-playing, LLMs create character-aligned dialogues \cite{Shanahan2023, 9980408} and simulate human behaviors \cite{park2023generative}. They serve as teaching assistants in classrooms \cite{Celik2022} and provide psychological counseling for individuals with high-functioning autism \cite{cho2023evaluating}.
% Recent research has developed innovative LLM applications. A project created a simulated job fair environment for training generative agents with enhanced communication capabilities \cite{li2023metaagents}, while another advanced social network simulation by modeling agents with emotional and interactive capabilities \cite{gao2023s3}. The SANDBOX project introduced an alignment learning approach using simulated society interactions, providing collective ratings and iterative feedback \cite{liu2023training}. These developments showcase LLMs' potential in mimicking human social interactions, suggesting a future where AI agents can participate in sophisticated social behaviors.


% The advent of LLMs has revolutionized NLP, paving the way for applications that extend far beyond basic text generation. Models such as ChatGPT have demonstrated the capacity to generate human-like responses from the context of the input text \cite{brown2020language}. LLMs have performed excellently in various language tasks, such as translation \cite{susnjak2022chatgpt}, spurring advancements in conversation generation, and even facilitating the writing of code. Beyond text generation, LLMs have stepped into the realms of decision-making and role-playing, where they not only respond to queries but also simulate complex human-like behaviors. AutoGPT \cite{yang2023autogpt} and HuggingGPT \cite{shen2023hugginggpt} use LLMs to plan, select, execute tasks, and generate operations. In role-playing simulations, LLMs can understand character backgrounds and produce dialogues aligned with characters' motivations \cite{Shanahan2023, 9980408}. Some projects proved that ChatGPT can observe, plan, and reflect to simulate human behavior and social interactions, creating computational software agents with believable human actions \cite{park2023generative}. In the educational sector, the versatility of LLMs enables them to serve as teaching assistants or conversational partners, enhancing interactivity in classrooms \cite{Celik2022}. Additionally, research has shown that LLMs can play a valuable role in providing psychological counseling for individuals with high-functioning autism \cite{cho2023evaluating}. 

% Building upon these capabilities, recent research initiatives have developed innovative applications for LLMs. One such project introduced a novel simulated job fair environment to train generative agents with enhanced communication and collaboration capabilities, underpinned by a framework comprising perception, memory, reasoning, and execution modules \cite{li2023metaagents}. Following this innovative approach, another significant advancement in the field is the simulation of social networks. One research advanced social network simulation by modeling agents with nuanced emotional, attitudinal, and interactive capabilities, validated through a two-tiered evaluation using real-world data, indicating a significant stride in LLM-driven agent-based modeling \cite{gao2023s3}. Moreover, a study presented a new alignment learning approach for language models, utilizing a simulated society named SANDBOX with recorded agent interactions, which provides a richer dataset including collective ratings and iterative feedback, shifting supervision to autonomous agents rather than traditional reward modeling \cite{liu2023training}. The groundbreaking research conducted in these studies showcases the impact of LLMs in mimicking human social interactions, suggesting a near future where AI-driven agents may contribute to them with a level of sophistication previously exclusive to human intelligence.

\begin{figure}[http]
  \includegraphics[width=0.4\textwidth]{fig/Review2.png}
  \caption{The literature screening procedure in this study involved two rounds of evaluation. This process resulted in the retention of 74 articles as the basis for design considerations.}
  \Description{This figure illustrates the literature screening process for the study. The literature screening procedure in this study involved two rounds of evaluation. This process resulted in the retention of 74 articles as the basis for design considerations.}
  \label{fig:Fig.1 The publication review procedure}
\end{figure}