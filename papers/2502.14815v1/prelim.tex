\section{Compound AI Systems}\label{sec:deluxeagent:Prelim}
\begin{figure}[ht]
    \centering
    \includegraphics[width=0.99\linewidth]{figure/demo/example_demo_h.pdf}
    \caption{Examples of static compound AI systems. (a) self-refine system. (b)  multi-agent-debate system. The diamond and star represent the input and output modules, and the circles represent the LLM modules. Directed lines represent data flow, and we omit most query inputs for simplicity.}
    \label{fig:deluxeagent:demo}
\end{figure}
\paragraph{Static Compound AI systems.} As defined by ~\cite{compound-ai-blog}, compound AI systems address AI tasks by synthesizing multiple components that interact with each other. Here, we denote a static compound AI system by a directed acyclic graph $G\triangleq (V,E)$, where each node $v\in V$ denotes one module, and each directed edge $e\triangleq (u,v)\in E$ indicates that the output from module $u$ is sent to module $v$ as input. Without loss of generality, we assume a final output module that generates the final output without any output edges, and an input module representing the input query which receives no input edges. 

\paragraph{LLM modules.} An LLM module is a module that utilizes an LLM to process the inputs. It typically concatenates all inputs as a text snippet (via some prompt template), obtain an LLM's response to this snippet, and send the response as output (potentially after some postprocessing). Throughout this paper, all modules are LLM modules to simplify notations. In practice, if a  module is not an LLM module, one can either merge it into an LLM module (e.g., a module that postprocesses output from some LLM module), or convert it into an LLM module by conceptually ``adding'' an LLM to the module.

\paragraph{Examples.} Consider two examples of static compound AI systems, self-refine and multi-agent-debate. Self-refine, as shown in Figure \ref{fig:deluxeagent:demo}(a), consists of three modules: a generator, a critic, and a refiner. Given a query, the generator produces an initial answer. The critic provides feedback on the initial answer, and the refiner uses the feedback to improve the initial answer. Figure \ref{fig:deluxeagent:demo}(b) shows the architecture of a six-module system: multi-agent-debate. Here, three generators first give their initial answers to a question, then three debaters debate with each other based on these initial answers. Refinements and debates can be iterative, but we focus on only one iteration for simplicity.  

\paragraph{Notations.} Table \ref{tab:deluxeagent:notation} list our notations. We also use $f_{i\rightarrow k}$ to indicate a function that is the same as function $f$ except that the value $i$ is mapped to the value $k$. 


\begin{table}[ht!]
\centering
\caption{Notations. 
}
\begin{tabular}{@{}ll@{}}
\toprule
\textbf{Symbol} & \textbf{Description} \\ \midrule
$G = (V, E)$   & A compound AI system \\ \midrule
$|V|$            & Number of LLM modules \\ \midrule
$M$            & The set of LLMs \\ \midrule
$\allocation:V\mapsto M$          & A model allocation  \\ \midrule
$\query$          & One task \\ \midrule
$\Performance(f)$            & End-to-end performance \\ \midrule
$\performance(f,z)$            & End-to-end performance on $\query$\\ \midrule
$p_i(f, \query)$    & $i$th module's performance on $\query$ \\
\bottomrule
\end{tabular}
\label{tab:deluxeagent:notation}
\end{table}