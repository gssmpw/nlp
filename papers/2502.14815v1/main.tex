\documentclass{article}
\PassOptionsToPackage{table,xcdraw}{xcolor}
\usepackage{xcolor}
\usepackage{bbm}
\usepackage{multirow}
% Language setting
% Replace `english' with e.g. `spanish' to change the document language
\usepackage[english]{babel}

% Set page size and margins
% Replace `letterpaper' with `a4paper' for UK/EU standard size
\usepackage[letterpaper,top=2cm,bottom=2cm,left=3cm,right=3cm,marginparwidth=1.75cm]{geometry}

% Useful packages
\usepackage{amsmath}
\usepackage{graphicx}
\usepackage[colorlinks=true, allcolors=blue]{hyperref}
\usepackage{booktabs} % To thicken table lines
\usepackage{caption}
\newcommand{\deluxesystem}{\textsc{LLMSelector}}

\newcommand{\eat}[1]{}


\usepackage{graphicx}
\usepackage{amssymb}
\usepackage{multirow}
\usepackage{bigstrut,morefloats}



\usepackage[resetlabels,labeled]{multibib}
\usepackage{caption}
\usepackage{subcaption}
%\usepackage{subfigure}




\usepackage[most]{tcolorbox}
%\usepackage{mdframed}
%\newcites{App}{Additional Reference}


%\usepackage{tcolorbox}








\newcommand{\E}{\mathbb{E}}

\usepackage{amsmath}
\usepackage{amsthm}
\usepackage{thmtools}

\theoremstyle{definition}
%\newtheorem{theorem}{Theorem}

%\usepackage[breaklinks]{hyperref}
%\usepackage{breakurl}


\usepackage[boxruled,linesnumbered]{algorithm2e}

\usepackage{amsthm}


\usepackage[most]{tcolorbox} % loads the tcolorbox package with most common features


% Custom style for the contribution box
\newtcolorbox[auto counter, number within=section]{mybox}[2][]{
    colback=blue!5!white,    % background color
    colframe=blue!75!black,  % frame color
    fonttitle=\bfseries,     % title font
    title=Main Contributions, % box title
    #1                      % allow for additional options
}


\newtcolorbox[auto counter, number within=section]{LLMdiagnoser}[2][]{
    colback=blue!5!white,    % background color
    colframe=blue!75!black,  % frame color
    fonttitle=\bfseries,     % title font
    title=LLM diagnoser prompt, % box title
    #1                      % allow for additional options
}


\usepackage{multirow}
\usepackage{array}
\usepackage{booktabs} % For professional looking tables

%\usepackage[table,xcdraw]{xcolor} % Required for row color


\usepackage{graphicx}
\usepackage[authoryear]{natbib}

%usepackage{authblk}

\usepackage{amsmath}

% !TeX root = main.tex 


\newcommand{\lnote}{\textcolor[rgb]{1,0,0}{Lydia: }\textcolor[rgb]{0,0,1}}
\newcommand{\todo}{\textcolor[rgb]{1,0,0.5}{To do: }\textcolor[rgb]{0.5,0,1}}


\newcommand{\state}{S}
\newcommand{\meas}{M}
\newcommand{\out}{\mathrm{out}}
\newcommand{\piv}{\mathrm{piv}}
\newcommand{\pivotal}{\mathrm{pivotal}}
\newcommand{\isnot}{\mathrm{not}}
\newcommand{\pred}{^\mathrm{predict}}
\newcommand{\act}{^\mathrm{act}}
\newcommand{\pre}{^\mathrm{pre}}
\newcommand{\post}{^\mathrm{post}}
\newcommand{\calM}{\mathcal{M}}

\newcommand{\game}{\mathbf{V}}
\newcommand{\strategyspace}{S}
\newcommand{\payoff}[1]{V^{#1}}
\newcommand{\eff}[1]{E^{#1}}
\newcommand{\p}{\vect{p}}
\newcommand{\simplex}[1]{\Delta^{#1}}

\newcommand{\recdec}[1]{\bar{D}(\hat{Y}_{#1})}





\newcommand{\sphereone}{\calS^1}
\newcommand{\samplen}{S^n}
\newcommand{\wA}{w}%{w_{\mathfrak{a}}}
\newcommand{\Awa}{A_{\wA}}
\newcommand{\Ytil}{\widetilde{Y}}
\newcommand{\Xtil}{\widetilde{X}}
\newcommand{\wst}{w_*}
\newcommand{\wls}{\widehat{w}_{\mathrm{LS}}}
\newcommand{\dec}{^\mathrm{dec}}
\newcommand{\sub}{^\mathrm{sub}}

\newcommand{\calP}{\mathcal{P}}
\newcommand{\totspace}{\calZ}
\newcommand{\clspace}{\calX}
\newcommand{\attspace}{\calA}

\newcommand{\Ftil}{\widetilde{\calF}}

\newcommand{\totx}{Z}
\newcommand{\classx}{X}
\newcommand{\attx}{A}
\newcommand{\calL}{\mathcal{L}}



\newcommand{\defeq}{\mathrel{\mathop:}=}
\newcommand{\vect}[1]{\ensuremath{\mathbf{#1}}}
\newcommand{\mat}[1]{\ensuremath{\mathbf{#1}}}
\newcommand{\dd}{\mathrm{d}}
\newcommand{\grad}{\nabla}
\newcommand{\hess}{\nabla^2}
\newcommand{\argmin}{\mathop{\rm argmin}}
\newcommand{\argmax}{\mathop{\rm argmax}}
\newcommand{\Ind}[1]{\mathbf{1}\{#1\}}

\newcommand{\norm}[1]{\left\|{#1}\right\|}
\newcommand{\fnorm}[1]{\|{#1}\|_{\text{F}}}
\newcommand{\spnorm}[2]{\left\| {#1} \right\|_{\text{S}({#2})}}
\newcommand{\sigmin}{\sigma_{\min}}
\newcommand{\tr}{\text{tr}}
\renewcommand{\det}{\text{det}}
\newcommand{\rank}{\text{rank}}
\newcommand{\logdet}{\text{logdet}}
\newcommand{\trans}{^{\top}}
\newcommand{\poly}{\text{poly}}
\newcommand{\polylog}{\text{polylog}}
\newcommand{\st}{\text{s.t.~}}
\newcommand{\proj}{\mathcal{P}}
\newcommand{\projII}{\mathcal{P}_{\parallel}}
\newcommand{\projT}{\mathcal{P}_{\perp}}
\newcommand{\projX}{\mathcal{P}_{\mathcal{X}^\star}}
\newcommand{\inner}[1]{\langle #1 \rangle}

\renewcommand{\Pr}{\mathbb{P}}
\newcommand{\Z}{\mathbb{Z}}
\newcommand{\N}{\mathbb{N}}
\newcommand{\R}{\mathbb{R}}
\newcommand{\E}{\mathbb{E}}
\newcommand{\F}{\mathcal{F}}
\newcommand{\var}{\mathrm{var}}
\newcommand{\cov}{\mathrm{cov}}


\newcommand{\calN}{\mathcal{N}}

\newcommand{\jccomment}{\textcolor[rgb]{1,0,0}{C: }\textcolor[rgb]{1,0,1}}
\newcommand{\fracpar}[2]{\frac{\partial #1}{\partial  #2}}

\newcommand{\A}{\mathcal{A}}
\newcommand{\B}{\mat{B}}
%\newcommand{\C}{\mat{C}}

\newcommand{\I}{\mat{I}}
\newcommand{\M}{\mat{M}}
\newcommand{\D}{\mat{D}}
%\newcommand{\U}{\mat{U}}
\newcommand{\V}{\mat{V}}
\newcommand{\W}{\mat{W}}
\newcommand{\X}{\mat{X}}
\newcommand{\Y}{\mat{Y}}
\newcommand{\mSigma}{\mat{\Sigma}}
\newcommand{\mLambda}{\mat{\Lambda}}
\newcommand{\e}{\vect{e}}
\newcommand{\g}{\vect{g}}
\renewcommand{\u}{\vect{u}}
\newcommand{\w}{\vect{w}}
\newcommand{\x}{\vect{x}}
\newcommand{\y}{\vect{y}}
\newcommand{\z}{\vect{z}}
\newcommand{\fI}{\mathfrak{I}}
\newcommand{\fS}{\mathfrak{S}}
\newcommand{\fE}{\mathfrak{E}}
\newcommand{\fF}{\mathfrak{F}}

\newcommand{\Risk}{\mathcal{R}}

\renewcommand{\L}{\mathcal{L}}
\renewcommand{\H}{\mathcal{H}}

\newcommand{\cn}{\kappa}
\newcommand{\nn}{\nonumber}


\newcommand{\Hess}{\nabla^2}
\newcommand{\tlO}{\tilde{O}}
\newcommand{\tlOmega}{\tilde{\Omega}}

\newcommand{\calF}{\mathcal{F}}
\newcommand{\fhat}{\widehat{f}}
\newcommand{\calS}{\mathcal{S}}

\newcommand{\calX}{\mathcal{X}}
\newcommand{\calY}{\mathcal{Y}}
\newcommand{\calD}{\mathcal{D}}
\newcommand{\calZ}{\mathcal{Z}}
\newcommand{\calA}{\mathcal{A}}
\newcommand{\fbayes}{f^B}
\newcommand{\func}{f^U}


\newcommand{\bayscore}{\text{calibrated Bayes score}}
\newcommand{\bayrisk}{\text{calibrated Bayes risk}}

\newtheorem{example}{Example}[section]
\newtheorem{exc}{Exercise}[section]
%\newtheorem{rem}{Remark}[section]

\newtheorem{theorem}{Theorem}[section]
\newtheorem{definition}{Definition}
\newtheorem{proposition}[theorem]{Proposition}
\newtheorem{corollary}[theorem]{Corollary}

\newtheorem{remark}{Remark}[section]
\newtheorem{lemma}[theorem]{Lemma}
\newtheorem{claim}[theorem]{Claim}
\newtheorem{fact}[theorem]{Fact}
\newtheorem{assumption}{Assumption}

\newcommand{\iidsim}{\overset{\mathrm{i.i.d.}}{\sim}}
\newcommand{\unifsim}{\overset{\mathrm{unif}}{\sim}}
\newcommand{\sign}{\mathrm{sign}}
\newcommand{\wbar}{\overline{w}}
\newcommand{\what}{\widehat{w}}
\newcommand{\KL}{\mathrm{KL}}
\newcommand{\Bern}{\mathrm{Bernoulli}}
\newcommand{\ihat}{\widehat{i}}
\newcommand{\Dwst}{\calD^{w_*}}
\newcommand{\fls}{\widehat{f}_{n}}


\newcommand{\brpi}{\pi^{br}}
\newcommand{\brtheta}{\theta^{br}}

% \newcommand{\M}{\mat{M}}
% \newcommand\Mmh{\mat{M}^{-1/2}}
% \newcommand{\A}{\mat{A}}
% \newcommand{\B}{\mat{B}}
% \newcommand{\C}{\mat{C}}
% \newcommand{\Et}[1][t]{\mat{E_{#1}}}
% \newcommand{\Etp}{\Et[t+1]}
% \newcommand{\Errt}[1][t]{\mat{\bigtriangleup_{#1}}}
% \newcommand\cnM{\kappa}
% \newcommand{\cn}[1]{\kappa\left(#1\right)}
% \newcommand\X{\mat{X}}
% \newcommand\fstar{f_*}
% \newcommand\Xt[1][t]{\mat{X_{#1}}}
% \newcommand\ut[1][t]{{u_{#1}}}
% \newcommand\Xtinv{\inv{\Xt}}
% \newcommand\Xtp{\mat{X_{t+1}}}
% \newcommand\Xtpinv{\inv{\left(\mat{X_{t+1}}\right)}}
% \newcommand\U{\mat{U}}
% \newcommand\UTr{\trans{\mat{U}}}
% \newcommand{\Ut}[1][t]{\mat{U_{#1}}}
% \newcommand{\Utinv}{\inv{\Ut}}
% \newcommand{\UtTr}[1][t]{\trans{\mat{U_{#1}}}}
% \newcommand\Utp{\mat{U_{t+1}}}
% \newcommand\UtpTr{\trans{\mat{U}_{t+1}}}
% \newcommand\Utptild{\mat{\widetilde{U}_{t+1}}}
% \newcommand\Us{\mat{U^*}}
% \newcommand\UsTr{\trans{\mat{U^*}}}
% \newcommand{\Sigs}{\mat{\Sigma}}
% \newcommand{\Sigsmh}{\Sigs^{-1/2}}
% \newcommand{\eye}{\mat{I}}
% \newcommand{\twonormbound}{\left(4+\DPhi{\M}{\Xt[0]}\right)\twonorm{\M}}
% \newcommand{\lamj}{\lambda_j}

% \renewcommand\u{\vect{u}}
% \newcommand\uTr{\trans{\vect{u}}}
% \renewcommand\v{\vect{v}}
% \newcommand\vTr{\trans{\vect{v}}}
% \newcommand\w{\vect{w}}
% \newcommand\wTr{\trans{\vect{w}}}
% \newcommand\wperp{\vect{w}_{\perp}}
% \newcommand\wperpTr{\trans{\vect{w}_{\perp}}}
% \newcommand\wj{\vect{w_j}}
% \newcommand\vj{\vect{v_j}}
% \newcommand\wjTr{\trans{\vect{w_j}}}
% \newcommand\vjTr{\trans{\vect{v_j}}}

% \newcommand{\DPhi}[2]{\ensuremath{D_{\Phi}\left(#1,#2\right)}}
% \newcommand\matmult{{\omega}}

\usepackage{pifont}
\theoremstyle{plain}
\newtheorem{theorem}{Theorem}[section]
\newtheorem{proposition}[theorem]{Proposition}
\newtheorem{lemma}[theorem]{Lemma}
\newtheorem{corollary}[theorem]{Corollary}
\theoremstyle{definition}
\newtheorem{definition}[theorem]{Definition}
\newtheorem{assumption}[theorem]{Assumption}
\theoremstyle{remark}
\newtheorem{remark}[theorem]{Remark}

\title{Optimizing Model Selection for Compound AI Systems}
\author{Lingjiao Chen$^{\dagger,\circ}$, Jared Quincy Davis$^{\circ}$, Boris Hanin$^{\S}$\\\\ Peter Bailis$^\ddagger$, Matei Zaharia$^\ddagger$, James Zou$^{\circ}$, Ion Stoica$^\ddagger$\\\\
$^\dagger$Microsoft Research, $^\circ$Stanford University,\\ $^\S$Princeton University, $^\ddagger$University of California, Berkeley}
\date{}



% Save the original \cite as \cite_old
%\let\cite_old\cite

% Redefine \cite to behave like \citep
\renewcommand{\cite}[1]{\citep{#1}}


\begin{document}
\maketitle

\begin{abstract}
Compound AI systems that combine multiple LLM calls, such as self-refine and multi-agent-debate, achieve strong performance on many AI tasks. We address a core question in optimizing compound systems: for each LLM call or module in the system, how should one decide which LLM to use? We show that these LLM choices have a large effect on quality, but the search space is exponential. We propose \deluxesystem{}, an efficient framework for model selection in compound systems, which leverages two key empirical insights: (i) end-to-end performance is often monotonic in how well each module performs, with all other modules held fixed, and (ii) per-module performance can be estimated accurately by an LLM. Building upon these insights, \deluxesystem{} iteratively selects one module and allocates to it the model with the highest module-wise performance, as estimated by an LLM, until no further gain is possible. \deluxesystem{} is applicable to any compound system with a bounded number of modules, and its number of API calls scales linearly with the number of modules, achieving high-quality model allocation both empirically and theoretically. Experiments with popular compound systems such as multi-agent debate and self-refine using LLMs such as GPT-4o, Claude 3.5 Sonnet and Gemini 1.5  show that \deluxesystem{} confers 5\%-70\% accuracy gains compared to using the same LLM for all modules.
 
\end{abstract}


\section{Introduction}


\begin{figure}[t]
\centering
\includegraphics[width=0.6\columnwidth]{figures/evaluation_desiderata_V5.pdf}
\vspace{-0.5cm}
\caption{\systemName is a platform for conducting realistic evaluations of code LLMs, collecting human preferences of coding models with real users, real tasks, and in realistic environments, aimed at addressing the limitations of existing evaluations.
}
\label{fig:motivation}
\end{figure}

\begin{figure*}[t]
\centering
\includegraphics[width=\textwidth]{figures/system_design_v2.png}
\caption{We introduce \systemName, a VSCode extension to collect human preferences of code directly in a developer's IDE. \systemName enables developers to use code completions from various models. The system comprises a) the interface in the user's IDE which presents paired completions to users (left), b) a sampling strategy that picks model pairs to reduce latency (right, top), and c) a prompting scheme that allows diverse LLMs to perform code completions with high fidelity.
Users can select between the top completion (green box) using \texttt{tab} or the bottom completion (blue box) using \texttt{shift+tab}.}
\label{fig:overview}
\end{figure*}

As model capabilities improve, large language models (LLMs) are increasingly integrated into user environments and workflows.
For example, software developers code with AI in integrated developer environments (IDEs)~\citep{peng2023impact}, doctors rely on notes generated through ambient listening~\citep{oberst2024science}, and lawyers consider case evidence identified by electronic discovery systems~\citep{yang2024beyond}.
Increasing deployment of models in productivity tools demands evaluation that more closely reflects real-world circumstances~\citep{hutchinson2022evaluation, saxon2024benchmarks, kapoor2024ai}.
While newer benchmarks and live platforms incorporate human feedback to capture real-world usage, they almost exclusively focus on evaluating LLMs in chat conversations~\citep{zheng2023judging,dubois2023alpacafarm,chiang2024chatbot, kirk2024the}.
Model evaluation must move beyond chat-based interactions and into specialized user environments.



 

In this work, we focus on evaluating LLM-based coding assistants. 
Despite the popularity of these tools---millions of developers use Github Copilot~\citep{Copilot}---existing
evaluations of the coding capabilities of new models exhibit multiple limitations (Figure~\ref{fig:motivation}, bottom).
Traditional ML benchmarks evaluate LLM capabilities by measuring how well a model can complete static, interview-style coding tasks~\citep{chen2021evaluating,austin2021program,jain2024livecodebench, white2024livebench} and lack \emph{real users}. 
User studies recruit real users to evaluate the effectiveness of LLMs as coding assistants, but are often limited to simple programming tasks as opposed to \emph{real tasks}~\citep{vaithilingam2022expectation,ross2023programmer, mozannar2024realhumaneval}.
Recent efforts to collect human feedback such as Chatbot Arena~\citep{chiang2024chatbot} are still removed from a \emph{realistic environment}, resulting in users and data that deviate from typical software development processes.
We introduce \systemName to address these limitations (Figure~\ref{fig:motivation}, top), and we describe our three main contributions below.


\textbf{We deploy \systemName in-the-wild to collect human preferences on code.} 
\systemName is a Visual Studio Code extension, collecting preferences directly in a developer's IDE within their actual workflow (Figure~\ref{fig:overview}).
\systemName provides developers with code completions, akin to the type of support provided by Github Copilot~\citep{Copilot}. 
Over the past 3 months, \systemName has served over~\completions suggestions from 10 state-of-the-art LLMs, 
gathering \sampleCount~votes from \userCount~users.
To collect user preferences,
\systemName presents a novel interface that shows users paired code completions from two different LLMs, which are determined based on a sampling strategy that aims to 
mitigate latency while preserving coverage across model comparisons.
Additionally, we devise a prompting scheme that allows a diverse set of models to perform code completions with high fidelity.
See Section~\ref{sec:system} and Section~\ref{sec:deployment} for details about system design and deployment respectively.



\textbf{We construct a leaderboard of user preferences and find notable differences from existing static benchmarks and human preference leaderboards.}
In general, we observe that smaller models seem to overperform in static benchmarks compared to our leaderboard, while performance among larger models is mixed (Section~\ref{sec:leaderboard_calculation}).
We attribute these differences to the fact that \systemName is exposed to users and tasks that differ drastically from code evaluations in the past. 
Our data spans 103 programming languages and 24 natural languages as well as a variety of real-world applications and code structures, while static benchmarks tend to focus on a specific programming and natural language and task (e.g. coding competition problems).
Additionally, while all of \systemName interactions contain code contexts and the majority involve infilling tasks, a much smaller fraction of Chatbot Arena's coding tasks contain code context, with infilling tasks appearing even more rarely. 
We analyze our data in depth in Section~\ref{subsec:comparison}.



\textbf{We derive new insights into user preferences of code by analyzing \systemName's diverse and distinct data distribution.}
We compare user preferences across different stratifications of input data (e.g., common versus rare languages) and observe which affect observed preferences most (Section~\ref{sec:analysis}).
For example, while user preferences stay relatively consistent across various programming languages, they differ drastically between different task categories (e.g. frontend/backend versus algorithm design).
We also observe variations in user preference due to different features related to code structure 
(e.g., context length and completion patterns).
We open-source \systemName and release a curated subset of code contexts.
Altogether, our results highlight the necessity of model evaluation in realistic and domain-specific settings.





\section{Preliminaries}
\label{sec:prelim}
\label{sec:term}
We define the key terminologies used, primarily focusing on the hidden states (or activations) during the forward pass. 

\paragraph{Components in an attention layer.} We denote $\Res$ as the residual stream. We denote $\Val$ as Value (states), $\Qry$ as Query (states), and $\Key$ as Key (states) in one attention head. The \attlogit~represents the value before the softmax operation and can be understood as the inner product between  $\Qry$  and  $\Key$. We use \Attn~to denote the attention weights of applying the SoftMax function to \attlogit, and ``attention map'' to describe the visualization of the heat map of the attention weights. When referring to the \attlogit~from ``$\tokenB$'' to  ``$\tokenA$'', we indicate the inner product  $\langle\Qry(\tokenB), \Key(\tokenA)\rangle$, specifically the entry in the ``$\tokenB$'' row and ``$\tokenA$'' column of the attention map.

\paragraph{Logit lens.} We use the method of ``Logit Lens'' to interpret the hidden states and value states \citep{belrose2023eliciting}. We use \logit~to denote pre-SoftMax values of the next-token prediction for LLMs. Denote \readout~as the linear operator after the last layer of transformers that maps the hidden states to the \logit. 
The logit lens is defined as applying the readout matrix to residual or value states in middle layers. Through the logit lens, the transformed hidden states can be interpreted as their direct effect on the logits for next-token prediction. 

\paragraph{Terminologies in two-hop reasoning.} We refer to an input like “\Src$\to$\brga, \brgb$\to$\Ed” as a two-hop reasoning chain, or simply a chain. The source entity $\Src$ serves as the starting point or origin of the reasoning. The end entity $\Ed$ represents the endpoint or destination of the reasoning chain. The bridge entity $\Brg$ connects the source and end entities within the reasoning chain. We distinguish between two occurrences of $\Brg$: the bridge in the first premise is called $\brga$, while the bridge in the second premise that connects to $\Ed$ is called $\brgc$. Additionally, for any premise ``$\tokenA \to \tokenB$'', we define $\tokenA$ as the parent node and $\tokenB$ as the child node. Furthermore, if at the end of the sequence, the query token is ``$\tokenA$'', we define the chain ``$\tokenA \to \tokenB$, $\tokenB \to \tokenC$'' as the Target Chain, while all other chains present in the context are referred to as distraction chains. Figure~\ref{fig:data_illustration} provides an illustration of the terminologies.

\paragraph{Input format.}
Motivated by two-hop reasoning in real contexts, we consider input in the format $\bos, \text{context information}, \query, \answer$. A transformer model is trained to predict the correct $\answer$ given the query $\query$ and the context information. The context compromises of $K=5$ disjoint two-hop chains, each appearing once and containing two premises. Within the same chain, the relative order of two premises is fixed so that \Src$\to$\brga~always precedes \brgb$\to$\Ed. The orders of chains are randomly generated, and chains may interleave with each other. The labels for the entities are re-shuffled for every sequence, choosing from a vocabulary size $V=30$. Given the $\bos$ token, $K=5$ two-hop chains, \query, and the \answer~tokens, the total context length is $N=23$. Figure~\ref{fig:data_illustration} also illustrates the data format. 

\paragraph{Model structure and training.} We pre-train a three-layer transformer with a single head per layer. Unless otherwise specified, the model is trained using Adam for $10,000$ steps, achieving near-optimal prediction accuracy. Details are relegated to Appendix~\ref{app:sec_add_training_detail}.


% \RZ{Do we use source entity, target entity, and mediator entity? Or do we use original token, bridge token, end token?}





% \paragraph{Basic notations.} We use ... We use $\ve_i$ to denote one-hot vectors of which only the $i$-th entry equals one, and all other entries are zero. The dimension of $\ve_i$ are usually omitted and can be inferred from contexts. We use $\indicator\{\cdot\}$ to denote the indicator function.

% Let $V > 0$ be a fixed positive integer, and let $\vocab = [V] \defeq \{1, 2, \ldots, V\}$ be the vocabulary. A token $v \in \vocab$ is an integer in $[V]$ and the input studied in this paper is a sequence of tokens $s_{1:T} \defeq (s_1, s_2, \ldots, s_T) \in \vocab^T$ of length $T$. For any set $\mathcal{S}$, we use $\Delta(\mathcal{S})$ to denote the set of distributions over $\mathcal{S}$.

% % to a sequence of vectors $z_1, z_2, \ldots, z_T \in \real^{\dout}$ of dimension $\dout$ and length $T$.

% Let $\mU = [\vu_1, \vu_2, \ldots, \vu_V]^\transpose \in \real^{V\times d}$ denote the token embedding matrix, where the $i$-th row $\vu_i \in \real^d$ represents the $d$-dimensional embedding of token $i \in [V]$. Similarly, let $\mP = [\vp_1, \vp_2, \ldots, \vp_T]^\transpose \in \real^{T\times d}$ denote the positional embedding matrix, where the $i$-th row $\vp_i \in \real^d$ represents the $d$-dimensional embedding of position $i \in [T]$. Both $\mU$ and $\mP$ can be fixed or learnable.

% After receiving an input sequence of tokens $s_{1:T}$, a transformer will first process it using embedding matrices $\mU$ and $\mP$ to obtain a sequence of vectors $\mH = [\vh_1, \vh_2, \ldots, \vh_T] \in \real^{d\times T}$, where 
% \[
% \vh_i = \mU^\transpose\ve_{s_i} + \mP^\transpose\ve_{i} = \vu_{s_i} + \vp_i.
% \]

% We make the following definitions of basic operations in a transformer.

% \begin{definition}[Basic operations in transformers] 
% \label{defn:operators}
% Define the softmax function $\softmax(\cdot): \real^d \to \real^d$ over a vector $\vv \in \real^d$ as
% \[\softmax(\vv)_i = \frac{\exp(\vv_i)}{\sum_{j=1}^d \exp(\vv_j)} \]
% and define the softmax function $\softmax(\cdot): \real^{m\times n} \to \real^{m \times n}$ over a matrix $\mV \in \real^{m\times n}$ as a column-wise softmax operator. For a squared matrix $\mM \in \real^{m\times m}$, the causal mask operator $\mask(\cdot): \real^{m\times m} \to \real^{m\times m}$  is defined as $\mask(\mM)_{ij} = \mM_{ij}$ if $i \leq j$ and  $\mask(\mM)_{ij} = -\infty$ otherwise. For a vector $\vv \in \real^n$ where $n$ is the number of hidden neurons in a layer, we use $\layernorm(\cdot): \real^n \to \real^n$ to denote the layer normalization operator where
% \[
% \layernorm(\vv)_i = \frac{\vv_i-\mu}{\sigma}, \mu = \frac{1}{n}\sum_{j=1}^n \vv_j, \sigma = \sqrt{\frac{1}{n}\sum_{j=1}^n (\vv_j-\mu)^2}
% \]
% and use $\layernorm(\cdot): \real^{n\times m} \to \real^{n\times m}$ to denote the column-wise layer normalization on a matrix.
% We also use $\nonlin(\cdot)$ to denote element-wise nonlinearity such as $\relu(\cdot)$.
% \end{definition}

% The main components of a transformer are causal self-attention heads and MLP layers, which are defined as follows.

% \begin{definition}[Attentions and MLPs]
% \label{defn:attn_mlp} 
% A single-head causal self-attention $\attn(\mH;\mQ,\mK,\mV,\mO)$ parameterized by $\mQ,\mK,\mV \in \real^{{\dqkv\times \din}}$ and $\mO \in \real^{\dout\times\dqkv}$ maps an input matrix $\mH \in \real^{\din\times T}$ to
% \begin{align*}
% &\attn(\mH;\mQ,\mK,\mV,\mO) \\
% =&\mO\mV\layernorm(\mH)\softmax(\mask(\layernorm(\mH)^\transpose\mK^\transpose\mQ\layernorm(\mH))).
% \end{align*}
% Furthermore, a multi-head attention with $M$ heads parameterized by $\{(\mQ_m,\mK_m,\mV_m,\mO_m) \}_{m=1}^M$ is defined as 
% \begin{align*}
%     &\Attn(\mH; \{(\mQ_m,\mK_m,\mV_m,\mO_m) \}_{m\in[M]}) \\ =& \sum_{m=1}^M \attn(\mH;\mQ_m,\mK_m,\mV_m,\mO_m) \in \real^{\dout \times T}.
% \end{align*}
% An MLP layer $\mlp(\mH;\mW_1,\mW_2)$ parameterized by $\mW_1 \in \real^{\dhidden\times \din}$ and $\mW_2 \in \real^{\dout \times \dhidden}$ maps an input matrix $\mH = [\vh_1, \ldots, \vh_T] \in \real^{\din \times T}$ to
% \begin{align*}
%     &\mlp(\mH;\mW_1,\mW_2) = [\vy_1, \ldots, \vy_T], \\ \text{where } &\vy_i = \mW_2\nonlin(\mW_1\layernorm(\vh_i)), \forall i \in [T].
% \end{align*}

% \end{definition}

% In this paper, we assume $\din=\dout=d$ for all attention heads and MLPs to facilitate residual stream unless otherwise specified. Given \Cref{defn:operators,defn:attn_mlp}, we are now able to define a multi-layer transformer.

% \begin{definition}[Multi-layer transformers]
% \label{defn:transformer}
%     An $L$-layer transformer $\transformer(\cdot): \vocab^T \to \Delta(\vocab)$ parameterized by $\mP$, $\mU$, $\{(\mQ_m^{(l)},\mK_m^{(l)},\mV_m^{(l)},\mO_m^{(l)})\}_{m\in[M],l\in[L]}$,  $\{(\mW_1^{(l)},\mW_2^{(l)})\}_{l\in[L]}$ and $\Wreadout \in \real^{V \times d}$ receives a sequence of tokens $s_{1:T}$ as input and predict the next token by outputting a distribution over the vocabulary. The input is first mapped to embeddings $\mH = [\vh_1, \vh_2, \ldots, \vh_T] \in \real^{d\times T}$ by embedding matrices $\mP, \mU$ where 
%     \[
%     \vh_i = \mU^\transpose\ve_{s_i} + \mP^\transpose\ve_{i}, \forall i \in [T].
%     \]
%     For each layer $l \in [L]$, the output of layer $l$, $\mH^{(l)} \in \real^{d\times T}$, is obtained by 
%     \begin{align*}
%         &\mH^{(l)} =  \mH^{(l-1/2)} + \mlp(\mH^{(l-1/2)};\mW_1^{(l)},\mW_2^{(l)}), \\
%         & \mH^{(l-1/2)} = \mH^{(l-1)} + \\ & \quad \Attn(\mH^{(l-1)}; \{(\mQ_m^{(l)},\mK_m^{(l)},\mV_m^{(l)},\mO_m^{(l)}) \}_{m\in[M]}), 
%     \end{align*}
%     where the input $\mH^{(l-1)}$ is the output of the previous layer $l-1$ for $l > 1$ and the input of the first layer $\mH^{(0)} = \mH$. Finally, the output of the transformer is obtained by 
%     \begin{align*}
%         \transformer(s_{1:T}) = \softmax(\Wreadout\vh_T^{(L)})
%     \end{align*}
%     which is a $V$-dimensional vector after softmax representing a distribution over $\vocab$, and $\vh_T^{(L)}$ is the $T$-th column of the output of the last layer, $\mH^{(L)}$.
% \end{definition}



% For each token $v \in \vocab$, there is a corresponding $d_t$-dimensional token embedding vector $\embed(v) \in \mathbb{R}^{d_t}$. Assume the maximum length of the sequence studied in this paper does not exceed $T$. For each position $t \in [T]$, there is a corresponding positional embedding  







\section{Method}\label{sec:method}
\begin{figure}
    \centering
    \includegraphics[width=0.85\textwidth]{imgs/heatmap_acc.pdf}
    \caption{\textbf{Visualization of the proposed periodic Bayesian flow with mean parameter $\mu$ and accumulated accuracy parameter $c$ which corresponds to the entropy/uncertainty}. For $x = 0.3, \beta(1) = 1000$ and $\alpha_i$ defined in \cref{appd:bfn_cir}, this figure plots three colored stochastic parameter trajectories for receiver mean parameter $m$ and accumulated accuracy parameter $c$, superimposed on a log-scale heatmap of the Bayesian flow distribution $p_F(m|x,\senderacc)$ and $p_F(c|x,\senderacc)$. Note the \emph{non-monotonicity} and \emph{non-additive} property of $c$ which could inform the network the entropy of the mean parameter $m$ as a condition and the \emph{periodicity} of $m$. %\jj{Shrink the figures to save space}\hanlin{Do we need to make this figure one-column?}
    }
    \label{fig:vmbf_vis}
    \vskip -0.1in
\end{figure}
% \begin{wrapfigure}{r}{0.5\textwidth}
%     \centering
%     \includegraphics[width=0.49\textwidth]{imgs/heatmap_acc.pdf}
%     \caption{\textbf{Visualization of hyper-torus Bayesian flow based on von Mises Distribution}. For $x = 0.3, \beta(1) = 1000$ and $\alpha_i$ defined in \cref{appd:bfn_cir}, this figure plots three colored stochastic parameter trajectories for receiver mean parameter $m$ and accumulated accuracy parameter $c$, superimposed on a log-scale heatmap of the Bayesian flow distribution $p_F(m|x,\senderacc)$ and $p_F(c|x,\senderacc)$. Note the \emph{non-monotonicity} and \emph{non-additive} property of $c$. \jj{Shrink the figures to save space}}
%     \label{fig:vmbf_vis}
%     \vspace{-30pt}
% \end{wrapfigure}


In this section, we explain the detailed design of CrysBFN tackling theoretical and practical challenges. First, we describe how to derive our new formulation of Bayesian Flow Networks over hyper-torus $\mathbb{T}^{D}$ from scratch. Next, we illustrate the two key differences between \modelname and the original form of BFN: $1)$ a meticulously designed novel base distribution with different Bayesian update rules; and $2)$ different properties over the accuracy scheduling resulted from the periodicity and the new Bayesian update rules. Then, we present in detail the overall framework of \modelname over each manifold of the crystal space (\textit{i.e.} fractional coordinates, lattice vectors, atom types) respecting \textit{periodic E(3) invariance}. 

% In this section, we first demonstrate how to build Bayesian flow on hyper-torus $\mathbb{T}^{D}$ by overcoming theoretical and practical problems to provide a low-noise parameter-space approach to fractional atom coordinate generation. Next, we present how \modelname models each manifold of crystal space respecting \textit{periodic E(3) invariance}. 

\subsection{Periodic Bayesian Flow on Hyper-torus \texorpdfstring{$\mathbb{T}^{D}$}{}} 
For generative modeling of fractional coordinates in crystal, we first construct a periodic Bayesian flow on \texorpdfstring{$\mathbb{T}^{D}$}{} by designing every component of the totally new Bayesian update process which we demonstrate to be distinct from the original Bayesian flow (please see \cref{fig:non_add}). 
 %:) 
 
 The fractional atom coordinate system \citep{jiao2023crystal} inherently distributes over a hyper-torus support $\mathbb{T}^{3\times N}$. Hence, the normal distribution support on $\R$ used in the original \citep{bfn} is not suitable for this scenario. 
% The key problem of generative modeling for crystal is the periodicity of Cartesian atom coordinates $\vX$ requiring:
% \begin{equation}\label{eq:periodcity}
% p(\vA,\vL,\vX)=p(\vA,\vL,\vX+\vec{LK}),\text{where}~\vec{K}=\vec{k}\vec{1}_{1\times N},\forall\vec{k}\in\mathbb{Z}^{3\times1}
% \end{equation}
% However, there does not exist such a distribution supporting on $\R$ to model such property because the integration of such distribution over $\R$ will not be finite and equal to 1. Therefore, the normal distribution used in \citet{bfn} can not meet this condition.

To tackle this problem, the circular distribution~\citep{mardia2009directional} over the finite interval $[-\pi,\pi)$ is a natural choice as the base distribution for deriving the BFN on $\mathbb{T}^D$. 
% one natural choice is to 
% we would like to consider the circular distribution over the finite interval as the base 
% we find that circular distributions \citep{mardia2009directional} defined on a finite interval with lengths of $2\pi$ can be used as the instantiation of input distribution for the BFN on $\mathbb{T}^D$.
Specifically, circular distributions enjoy desirable periodic properties: $1)$ the integration over any interval length of $2\pi$ equals 1; $2)$ the probability distribution function is periodic with period $2\pi$.  Sharing the same intrinsic with fractional coordinates, such periodic property of circular distribution makes it suitable for the instantiation of BFN's input distribution, in parameterizing the belief towards ground truth $\x$ on $\mathbb{T}^D$. 
% \yuxuan{this is very complicated from my perspective.} \hanlin{But this property is exactly beautiful and perfectly fit into the BFN.}

\textbf{von Mises Distribution and its Bayesian Update} We choose von Mises distribution \citep{mardia2009directional} from various circular distributions as the form of input distribution, based on the appealing conjugacy property required in the derivation of the BFN framework.
% to leverage the Bayesian conjugacy property of von Mises distribution which is required by the BFN framework. 
That is, the posterior of a von Mises distribution parameterized likelihood is still in the family of von Mises distributions. The probability density function of von Mises distribution with mean direction parameter $m$ and concentration parameter $c$ (describing the entropy/uncertainty of $m$) is defined as: 
\begin{equation}
f(x|m,c)=vM(x|m,c)=\frac{\exp(c\cos(x-m))}{2\pi I_0(c)}
\end{equation}
where $I_0(c)$ is zeroth order modified Bessel function of the first kind as the normalizing constant. Given the last univariate belief parameterized by von Mises distribution with parameter $\theta_{i-1}=\{m_{i-1},\ c_{i-1}\}$ and the sample $y$ from sender distribution with unknown data sample $x$ and known accuracy $\alpha$ describing the entropy/uncertainty of $y$,  Bayesian update for the receiver is deducted as:
\begin{equation}
 h(\{m_{i-1},c_{i-1}\},y,\alpha)=\{m_i,c_i \}, \text{where}
\end{equation}
\begin{equation}\label{eq:h_m}
m_i=\text{atan2}(\alpha\sin y+c_{i-1}\sin m_{i-1}, {\alpha\cos y+c_{i-1}\cos m_{i-1}})
\end{equation}
\begin{equation}\label{eq:h_c}
c_i =\sqrt{\alpha^2+c_{i-1}^2+2\alpha c_{i-1}\cos(y-m_{i-1})}
\end{equation}
The proof of the above equations can be found in \cref{apdx:bayesian_update_function}. The atan2 function refers to  2-argument arctangent. Independently conducting  Bayesian update for each dimension, we can obtain the Bayesian update distribution by marginalizing $\y$:
\begin{equation}
p_U(\vtheta'|\vtheta,\bold{x};\alpha)=\mathbb{E}_{p_S(\bold{y}|\bold{x};\alpha)}\delta(\vtheta'-h(\vtheta,\bold{y},\alpha))=\mathbb{E}_{vM(\bold{y}|\bold{x},\alpha)}\delta(\vtheta'-h(\vtheta,\bold{y},\alpha))
\end{equation} 
\begin{figure}
    \centering
    \vskip -0.15in
    \includegraphics[width=0.95\linewidth]{imgs/non_add.pdf}
    \caption{An intuitive illustration of non-additive accuracy Bayesian update on the torus. The lengths of arrows represent the uncertainty/entropy of the belief (\emph{e.g.}~$1/\sigma^2$ for Gaussian and $c$ for von Mises). The directions of the arrows represent the believed location (\emph{e.g.}~ $\mu$ for Gaussian and $m$ for von Mises).}
    \label{fig:non_add}
    \vskip -0.15in
\end{figure}
\textbf{Non-additive Accuracy} 
The additive accuracy is a nice property held with the Gaussian-formed sender distribution of the original BFN expressed as:
\begin{align}
\label{eq:standard_id}
    \update(\parsn{}'' \mid \parsn{}, \x; \alpha_a+\alpha_b) = \E_{\update(\parsn{}' \mid \parsn{}, \x; \alpha_a)} \update(\parsn{}'' \mid \parsn{}', \x; \alpha_b)
\end{align}
Such property is mainly derived based on the standard identity of Gaussian variable:
\begin{equation}
X \sim \mathcal{N}\left(\mu_X, \sigma_X^2\right), Y \sim \mathcal{N}\left(\mu_Y, \sigma_Y^2\right) \Longrightarrow X+Y \sim \mathcal{N}\left(\mu_X+\mu_Y, \sigma_X^2+\sigma_Y^2\right)
\end{equation}
The additive accuracy property makes it feasible to derive the Bayesian flow distribution $
p_F(\boldsymbol{\theta} \mid \mathbf{x} ; i)=p_U\left(\boldsymbol{\theta} \mid \boldsymbol{\theta}_0, \mathbf{x}, \sum_{k=1}^{i} \alpha_i \right)
$ for the simulation-free training of \cref{eq:loss_n}.
It should be noted that the standard identity in \cref{eq:standard_id} does not hold in the von Mises distribution. Hence there exists an important difference between the original Bayesian flow defined on Euclidean space and the Bayesian flow of circular data on $\mathbb{T}^D$ based on von Mises distribution. With prior $\btheta = \{\bold{0},\bold{0}\}$, we could formally represent the non-additive accuracy issue as:
% The additive accuracy property implies the fact that the "confidence" for the data sample after observing a series of the noisy samples with accuracy ${\alpha_1, \cdots, \alpha_i}$ could be  as the accuracy sum  which could be  
% Here we 
% Here we emphasize the specific property of BFN based on von Mises distribution.
% Note that 
% \begin{equation}
% \update(\parsn'' \mid \parsn, \x; \alpha_a+\alpha_b) \ne \E_{\update(\parsn' \mid \parsn, \x; \alpha_a)} \update(\parsn'' \mid \parsn', \x; \alpha_b)
% \end{equation}
% \oyyw{please check whether the below equation is better}
% \yuxuan{I fill somehow confusing on what is the update distribution with $\alpha$. }
% \begin{equation}
% \update(\parsn{}'' \mid \parsn{}, \x; \alpha_a+\alpha_b) \ne \E_{\update(\parsn{}' \mid \parsn{}, \x; \alpha_a)} \update(\parsn{}'' \mid \parsn{}', \x; \alpha_b)
% \end{equation}
% We give an intuitive visualization of such difference in \cref{fig:non_add}. The untenability of this property can materialize by considering the following case: with prior $\btheta = \{\bold{0},\bold{0}\}$, check the two-step Bayesian update distribution with $\alpha_a,\alpha_b$ and one-step Bayesian update with $\alpha=\alpha_a+\alpha_b$:
\begin{align}
\label{eq:nonadd}
     &\update(c'' \mid \parsn, \x; \alpha_a+\alpha_b)  = \delta(c-\alpha_a-\alpha_b)
     \ne  \mathbb{E}_{p_U(\parsn' \mid \parsn, \x; \alpha_a)}\update(c'' \mid \parsn', \x; \alpha_b) \nonumber \\&= \mathbb{E}_{vM(\bold{y}_b|\bold{x},\alpha_a)}\mathbb{E}_{vM(\bold{y}_a|\bold{x},\alpha_b)}\delta(c-||[\alpha_a \cos\y_a+\alpha_b\cos \y_b,\alpha_a \sin\y_a+\alpha_b\sin \y_b]^T||_2)
\end{align}
A more intuitive visualization could be found in \cref{fig:non_add}. This fundamental difference between periodic Bayesian flow and that of \citet{bfn} presents both theoretical and practical challenges, which we will explain and address in the following contents.

% This makes constructing Bayesian flow based on von Mises distribution intrinsically different from previous Bayesian flows (\citet{bfn}).

% Thus, we must reformulate the framework of Bayesian flow networks  accordingly. % and do necessary reformulations of BFN. 

% \yuxuan{overall I feel this part is complicated by using the language of update distribution. I would like to suggest simply use bayesian update, to provide intuitive explantion.}\hanlin{See the illustration in \cref{fig:non_add}}

% That introduces a cascade of problems, and we investigate the following issues: $(1)$ Accuracies between sender and receiver are not synchronized and need to be differentiated. $(2)$ There is no tractable Bayesian flow distribution for a one-step sample conditioned on a given time step $i$, and naively simulating the Bayesian flow results in computational overhead. $(3)$ It is difficult to control the entropy of the Bayesian flow. $(4)$ Accuracy is no longer a function of $t$ and becomes a distribution conditioned on $t$, which can be different across dimensions.
%\jj{Edited till here}

\textbf{Entropy Conditioning} As a common practice in generative models~\citep{ddpm,flowmatching,bfn}, timestep $t$ is widely used to distinguish among generation states by feeding the timestep information into the networks. However, this paper shows that for periodic Bayesian flow, the accumulated accuracy $\vc_i$ is more effective than time-based conditioning by informing the network about the entropy and certainty of the states $\parsnt{i}$. This stems from the intrinsic non-additive accuracy which makes the receiver's accumulated accuracy $c$ not bijective function of $t$, but a distribution conditioned on accumulated accuracies $\vc_i$ instead. Therefore, the entropy parameter $\vc$ is taken logarithm and fed into the network to describe the entropy of the input corrupted structure. We verify this consideration in \cref{sec:exp_ablation}. 
% \yuxuan{implement variant. traditionally, the timestep is widely used to distinguish the different states by putting the timestep embedding into the networks. citation of FM, diffusion, BFN. However, we find that conditioned on time in periodic flow could not provide extra benefits. To further boost the performance, we introduce a simple yet effective modification term entropy conditional. This is based on that the accumulated accuracy which represents the current uncertainty or entropy could be a better indicator to distinguish different states. + Describe how you do this. }



\textbf{Reformulations of BFN}. Recall the original update function with Gaussian sender distribution, after receiving noisy samples $\y_1,\y_2,\dots,\y_i$ with accuracies $\senderacc$, the accumulated accuracies of the receiver side could be analytically obtained by the additive property and it is consistent with the sender side.
% Since observing sample $\y$ with $\alpha_i$ can not result in exact accuracy increment $\alpha_i$ for receiver, the accuracies between sender and receiver are not synchronized which need to be differentiated. 
However, as previously mentioned, this does not apply to periodic Bayesian flow, and some of the notations in original BFN~\citep{bfn} need to be adjusted accordingly. We maintain the notations of sender side's one-step accuracy $\alpha$ and added accuracy $\beta$, and alter the notation of receiver's accuracy parameter as $c$, which is needed to be simulated by cascade of Bayesian updates. We emphasize that the receiver's accumulated accuracy $c$ is no longer a function of $t$ (differently from the Gaussian case), and it becomes a distribution conditioned on received accuracies $\senderacc$ from the sender. Therefore, we represent the Bayesian flow distribution of von Mises distribution as $p_F(\btheta|\x;\alpha_1,\alpha_2,\dots,\alpha_i)$. And the original simulation-free training with Bayesian flow distribution is no longer applicable in this scenario.
% Different from previous BFNs where the accumulated accuracy $\rho$ is not explicitly modeled, the accumulated accuracy parameter $c$ (visualized in \cref{fig:vmbf_vis}) needs to be explicitly modeled by feeding it to the network to avoid information loss.
% the randomaccuracy parameter $c$ (visualized in \cref{fig:vmbf_vis}) implies that there exists information in $c$ from the sender just like $m$, meaning that $c$ also should be fed into the network to avoid information loss. 
% We ablate this consideration in  \cref{sec:exp_ablation}. 

\textbf{Fast Sampling from Equivalent Bayesian Flow Distribution} Based on the above reformulations, the Bayesian flow distribution of von Mises distribution is reframed as: 
\begin{equation}\label{eq:flow_frac}
p_F(\btheta_i|\x;\alpha_1,\alpha_2,\dots,\alpha_i)=\E_{\update(\parsnt{1} \mid \parsnt{0}, \x ; \alphat{1})}\dots\E_{\update(\parsn_{i-1} \mid \parsnt{i-2}, \x; \alphat{i-1})} \update(\parsnt{i} | \parsnt{i-1},\x;\alphat{i} )
\end{equation}
Naively sampling from \cref{eq:flow_frac} requires slow auto-regressive iterated simulation, making training unaffordable. Noticing the mathematical properties of \cref{eq:h_m,eq:h_c}, we  transform \cref{eq:flow_frac} to the equivalent form:
\begin{equation}\label{eq:cirflow_equiv}
p_F(\vec{m}_i|\x;\alpha_1,\alpha_2,\dots,\alpha_i)=\E_{vM(\y_1|\x,\alpha_1)\dots vM(\y_i|\x,\alpha_i)} \delta(\vec{m}_i-\text{atan2}(\sum_{j=1}^i \alpha_j \cos \y_j,\sum_{j=1}^i \alpha_j \sin \y_j))
\end{equation}
\begin{equation}\label{eq:cirflow_equiv2}
p_F(\vec{c}_i|\x;\alpha_1,\alpha_2,\dots,\alpha_i)=\E_{vM(\y_1|\x,\alpha_1)\dots vM(\y_i|\x,\alpha_i)}  \delta(\vec{c}_i-||[\sum_{j=1}^i \alpha_j \cos \y_j,\sum_{j=1}^i \alpha_j \sin \y_j]^T||_2)
\end{equation}
which bypasses the computation of intermediate variables and allows pure tensor operations, with negligible computational overhead.
\begin{restatable}{proposition}{cirflowequiv}
The probability density function of Bayesian flow distribution defined by \cref{eq:cirflow_equiv,eq:cirflow_equiv2} is equivalent to the original definition in \cref{eq:flow_frac}. 
\end{restatable}
\textbf{Numerical Determination of Linear Entropy Sender Accuracy Schedule} ~Original BFN designs the accuracy schedule $\beta(t)$ to make the entropy of input distribution linearly decrease. As for crystal generation task, to ensure information coherence between modalities, we choose a sender accuracy schedule $\senderacc$ that makes the receiver's belief entropy $H(t_i)=H(p_I(\cdot|\vtheta_i))=H(p_I(\cdot|\vc_i))$ linearly decrease \emph{w.r.t.} time $t_i$, given the initial and final accuracy parameter $c(0)$ and $c(1)$. Due to the intractability of \cref{eq:vm_entropy}, we first use numerical binary search in $[0,c(1)]$ to determine the receiver's $c(t_i)$ for $i=1,\dots, n$ by solving the equation $H(c(t_i))=(1-t_i)H(c(0))+tH(c(1))$. Next, with $c(t_i)$, we conduct numerical binary search for each $\alpha_i$ in $[0,c(1)]$ by solving the equations $\E_{y\sim vM(x,\alpha_i)}[\sqrt{\alpha_i^2+c_{i-1}^2+2\alpha_i c_{i-1}\cos(y-m_{i-1})}]=c(t_i)$ from $i=1$ to $i=n$ for arbitrarily selected $x\in[-\pi,\pi)$.

After tackling all those issues, we have now arrived at a new BFN architecture for effectively modeling crystals. Such BFN can also be adapted to other type of data located in hyper-torus $\mathbb{T}^{D}$.

\subsection{Equivariant Bayesian Flow for Crystal}
With the above Bayesian flow designed for generative modeling of fractional coordinate $\vF$, we are able to build equivariant Bayesian flow for each modality of crystal. In this section, we first give an overview of the general training and sampling algorithm of \modelname (visualized in \cref{fig:framework}). Then, we describe the details of the Bayesian flow of every modality. The training and sampling algorithm can be found in \cref{alg:train} and \cref{alg:sampling}.

\textbf{Overview} Operating in the parameter space $\bthetaM=\{\bthetaA,\bthetaL,\bthetaF\}$, \modelname generates high-fidelity crystals through a joint BFN sampling process on the parameter of  atom type $\bthetaA$, lattice parameter $\vec{\theta}^L=\{\bmuL,\brhoL\}$, and the parameter of fractional coordinate matrix $\bthetaF=\{\bmF,\bcF\}$. We index the $n$-steps of the generation process in a discrete manner $i$, and denote the corresponding continuous notation $t_i=i/n$ from prior parameter $\thetaM_0$ to a considerably low variance parameter $\thetaM_n$ (\emph{i.e.} large $\vrho^L,\bmF$, and centered $\bthetaA$).

At training time, \modelname samples time $i\sim U\{1,n\}$ and $\bthetaM_{i-1}$ from the Bayesian flow distribution of each modality, serving as the input to the network. The network $\net$ outputs $\net(\parsnt{i-1}^\mathcal{M},t_{i-1})=\net(\parsnt{i-1}^A,\parsnt{i-1}^F,\parsnt{i-1}^L,t_{i-1})$ and conducts gradient descents on loss function \cref{eq:loss_n} for each modality. After proper training, the sender distribution $p_S$ can be approximated by the receiver distribution $p_R$. 

At inference time, from predefined $\thetaM_0$, we conduct transitions from $\thetaM_{i-1}$ to $\thetaM_{i}$ by: $(1)$ sampling $\y_i\sim p_R(\bold{y}|\thetaM_{i-1};t_i,\alpha_i)$ according to network prediction $\predM{i-1}$; and $(2)$ performing Bayesian update $h(\thetaM_{i-1},\y^\calM_{i-1},\alpha_i)$ for each dimension. 

% Alternatively, we complete this transition using the flow-back technique by sampling 
% $\thetaM_{i}$ from Bayesian flow distribution $\flow(\btheta^M_{i}|\predM{i-1};t_{i-1})$. 

% The training objective of $\net$ is to minimize the KL divergence between sender distribution and receiver distribution for every modality as defined in \cref{eq:loss_n} which is equivalent to optimizing the negative variational lower bound $\calL^{VLB}$ as discussed in \cref{sec:preliminaries}. 

%In the following part, we will present the Bayesian flow of each modality in detail.

\textbf{Bayesian Flow of Fractional Coordinate $\vF$}~The distribution of the prior parameter $\bthetaF_0$ is defined as:
\begin{equation}\label{eq:prior_frac}
    p(\bthetaF_0) \defeq \{vM(\vm_0^F|\vec{0}_{3\times N},\vec{0}_{3\times N}),\delta(\vc_0^F-\vec{0}_{3\times N})\} = \{U(\vec{0},\vec{1}),\delta(\vc_0^F-\vec{0}_{3\times N})\}
\end{equation}
Note that this prior distribution of $\vm_0^F$ is uniform over $[\vec{0},\vec{1})$, ensuring the periodic translation invariance property in \cref{De:pi}. The training objective is minimizing the KL divergence between sender and receiver distribution (deduction can be found in \cref{appd:cir_loss}): 
%\oyyw{replace $\vF$ with $\x$?} \hanlin{notations follow Preliminary?}
\begin{align}\label{loss_frac}
\calL_F = n \E_{i \sim \ui{n}, \flow(\parsn{}^F \mid \vF ; \senderacc)} \alpha_i\frac{I_1(\alpha_i)}{I_0(\alpha_i)}(1-\cos(\vF-\predF{i-1}))
\end{align}
where $I_0(x)$ and $I_1(x)$ are the zeroth and the first order of modified Bessel functions. The transition from $\bthetaF_{i-1}$ to $\bthetaF_{i}$ is the Bayesian update distribution based on network prediction:
\begin{equation}\label{eq:transi_frac}
    p(\btheta^F_{i}|\parsnt{i-1}^\calM)=\mathbb{E}_{vM(\bold{y}|\predF{i-1},\alpha_i)}\delta(\btheta^F_{i}-h(\btheta^F_{i-1},\bold{y},\alpha_i))
\end{equation}
\begin{restatable}{proposition}{fracinv}
With $\net_{F}$ as a periodic translation equivariant function namely $\net_F(\parsnt{}^A,w(\parsnt{}^F+\vt),\parsnt{}^L,t)=w(\net_F(\parsnt{}^A,\parsnt{}^F,\parsnt{}^L,t)+\vt), \forall\vt\in\R^3$, the marginal distribution of $p(\vF_n)$ defined by \cref{eq:prior_frac,eq:transi_frac} is periodic translation invariant. 
\end{restatable}
\textbf{Bayesian Flow of Lattice Parameter \texorpdfstring{$\boldsymbol{L}$}{}}   
Noting the lattice parameter $\bm{L}$ located in Euclidean space, we set prior as the parameter of a isotropic multivariate normal distribution $\btheta^L_0\defeq\{\vmu_0^L,\vrho_0^L\}=\{\bm{0}_{3\times3},\bm{1}_{3\times3}\}$
% \begin{equation}\label{eq:lattice_prior}
% \btheta^L_0\defeq\{\vmu_0^L,\vrho_0^L\}=\{\bm{0}_{3\times3},\bm{1}_{3\times3}\}
% \end{equation}
such that the prior distribution of the Markov process on $\vmu^L$ is the Dirac distribution $\delta(\vec{\mu_0}-\vec{0})$ and $\delta(\vec{\rho_0}-\vec{1})$, 
% \begin{equation}
%     p_I^L(\boldsymbol{L}|\btheta_0^L)=\mathcal{N}(\bm{L}|\bm{0},\bm{I})
% \end{equation}
which ensures O(3)-invariance of prior distribution of $\vL$. By Eq. 77 from \citet{bfn}, the Bayesian flow distribution of the lattice parameter $\bm{L}$ is: 
\begin{align}% =p_U(\bmuL|\btheta_0^L,\bm{L},\beta(t))
p_F^L(\bmuL|\bm{L};t) &=\mathcal{N}(\bmuL|\gamma(t)\bm{L},\gamma(t)(1-\gamma(t))\bm{I}) 
\end{align}
where $\gamma(t) = 1 - \sigma_1^{2t}$ and $\sigma_1$ is the predefined hyper-parameter controlling the variance of input distribution at $t=1$ under linear entropy accuracy schedule. The variance parameter $\vrho$ does not need to be modeled and fed to the network, since it is deterministic given the accuracy schedule. After sampling $\bmuL_i$ from $p_F^L$, the training objective is defined as minimizing KL divergence between sender and receiver distribution (based on Eq. 96 in \citet{bfn}):
\begin{align}
\mathcal{L}_{L} = \frac{n}{2}\left(1-\sigma_1^{2/n}\right)\E_{i \sim \ui{n}}\E_{\flow(\bmuL_{i-1} |\vL ; t_{i-1})}  \frac{\left\|\vL -\predL{i-1}\right\|^2}{\sigma_1^{2i/n}},\label{eq:lattice_loss}
\end{align}
where the prediction term $\predL{i-1}$ is the lattice parameter part of network output. After training, the generation process is defined as the Bayesian update distribution given network prediction:
\begin{equation}\label{eq:lattice_sampling}
    p(\bmuL_{i}|\parsnt{i-1}^\calM)=\update^L(\bmuL_{i}|\predL{i-1},\bmuL_{i-1};t_{i-1})
\end{equation}
    

% The final prediction of the lattice parameter is given by $\bmuL_n = \predL{n-1}$.
% \begin{equation}\label{eq:final_lattice}
%     \bmuL_n = \predL{n-1}
% \end{equation}

\begin{restatable}{proposition}{latticeinv}\label{prop:latticeinv}
With $\net_{L}$ as  O(3)-equivariant function namely $\net_L(\parsnt{}^A,\parsnt{}^F,\vQ\parsnt{}^L,t)=\vQ\net_L(\parsnt{}^A,\parsnt{}^F,\parsnt{}^L,t),\forall\vQ^T\vQ=\vI$, the marginal distribution of $p(\bmuL_n)$ defined by \cref{eq:lattice_sampling} is O(3)-invariant. 
\end{restatable}


\textbf{Bayesian Flow of Atom Types \texorpdfstring{$\boldsymbol{A}$}{}} 
Given that atom types are discrete random variables located in a simplex $\calS^K$, the prior parameter of $\boldsymbol{A}$ is the discrete uniform distribution over the vocabulary $\parsnt{0}^A \defeq \frac{1}{K}\vec{1}_{1\times N}$. 
% \begin{align}\label{eq:disc_input_prior}
% \parsnt{0}^A \defeq \frac{1}{K}\vec{1}_{1\times N}
% \end{align}
% \begin{align}
%     (\oh{j}{K})_k \defeq \delta_{j k}, \text{where }\oh{j}{K}\in \R^{K},\oh{\vA}{KD} \defeq \left(\oh{a_1}{K},\dots,\oh{a_N}{K}\right) \in \R^{K\times N}
% \end{align}
With the notation of the projection from the class index $j$ to the length $K$ one-hot vector $ (\oh{j}{K})_k \defeq \delta_{j k}, \text{where }\oh{j}{K}\in \R^{K},\oh{\vA}{KD} \defeq \left(\oh{a_1}{K},\dots,\oh{a_N}{K}\right) \in \R^{K\times N}$, the Bayesian flow distribution of atom types $\vA$ is derived in \citet{bfn}:
\begin{align}
\flow^{A}(\parsn^A \mid \vA; t) &= \E_{\N{\y \mid \beta^A(t)\left(K \oh{\vA}{K\times N} - \vec{1}_{K\times N}\right)}{\beta^A(t) K \vec{I}_{K\times N \times N}}} \delta\left(\parsn^A - \frac{e^{\y}\parsnt{0}^A}{\sum_{k=1}^K e^{\y_k}(\parsnt{0})_{k}^A}\right).
\end{align}
where $\beta^A(t)$ is the predefined accuracy schedule for atom types. Sampling $\btheta_i^A$ from $p_F^A$ as the training signal, the training objective is the $n$-step discrete-time loss for discrete variable \citep{bfn}: 
% \oyyw{can we simplify the next equation? Such as remove $K \times N, K \times N \times N$}
% \begin{align}
% &\calL_A = n\E_{i \sim U\{1,n\},\flow^A(\parsn^A \mid \vA ; t_{i-1}),\N{\y \mid \alphat{i}\left(K \oh{\vA}{KD} - \vec{1}_{K\times N}\right)}{\alphat{i} K \vec{I}_{K\times N \times N}}} \ln \N{\y \mid \alphat{i}\left(K \oh{\vA}{K\times N} - \vec{1}_{K\times N}\right)}{\alphat{i} K \vec{I}_{K\times N \times N}}\nonumber\\
% &\qquad\qquad\qquad-\sum_{d=1}^N \ln \left(\sum_{k=1}^K \out^{(d)}(k \mid \parsn^A; t_{i-1}) \N{\ydd{d} \mid \alphat{i}\left(K\oh{k}{K}- \vec{1}_{K\times N}\right)}{\alphat{i} K \vec{I}_{K\times N \times N}}\right)\label{discdisc_t_loss_exp}
% \end{align}
\begin{align}
&\calL_A = n\E_{i \sim U\{1,n\},\flow^A(\parsn^A \mid \vA ; t_{i-1}),\N{\y \mid \alphat{i}\left(K \oh{\vA}{KD} - \vec{1}\right)}{\alphat{i} K \vec{I}}} \ln \N{\y \mid \alphat{i}\left(K \oh{\vA}{K\times N} - \vec{1}\right)}{\alphat{i} K \vec{I}}\nonumber\\
&\qquad\qquad\qquad-\sum_{d=1}^N \ln \left(\sum_{k=1}^K \out^{(d)}(k \mid \parsn^A; t_{i-1}) \N{\ydd{d} \mid \alphat{i}\left(K\oh{k}{K}- \vec{1}\right)}{\alphat{i} K \vec{I}}\right)\label{discdisc_t_loss_exp}
\end{align}
where $\vec{I}\in \R^{K\times N \times N}$ and $\vec{1}\in\R^{K\times D}$. When sampling, the transition from $\bthetaA_{i-1}$ to $\bthetaA_{i}$ is derived as:
\begin{equation}
    p(\btheta^A_{i}|\parsnt{i-1}^\calM)=\update^A(\btheta^A_{i}|\btheta^A_{i-1},\predA{i-1};t_{i-1})
\end{equation}

The detailed training and sampling algorithm could be found in \cref{alg:train} and \cref{alg:sampling}.




\section{Experiments}
\label{sec:exp}
Following the settings in Section \ref{sec:existing}, we evaluate \textit{NovelSum}'s correlation with the fine-tuned model performance across 53 IT datasets and compare it with previous diversity metrics. Additionally, we conduct a correlation analysis using Qwen-2.5-7B \cite{yang2024qwen2} as the backbone model, alongside previous LLaMA-3-8B experiments, to further demonstrate the metric's effectiveness across different scenarios. Qwen is used for both instruction tuning and deriving semantic embeddings. Due to resource constraints, we run each strategy on Qwen for two rounds, resulting in 25 datasets. 

\subsection{Main Results}

\begin{table*}[!t]
    \centering
    \resizebox{\linewidth}{!}{
    \begin{tabular}{lcccccccccc}
    \toprule
    \multirow{3}*{\textbf{Diversity Metrics}} & \multicolumn{10}{c}{\textbf{Data Selection Strategies}} \\
    \cmidrule(lr){2-11}
    & \multirow{2}*{\textbf{K-means}} & \multirow{2}*{\vtop{\hbox{\textbf{K-Center}}\vspace{1mm}\hbox{\textbf{-Greedy}}}}  & \multirow{2}*{\textbf{QDIT}} & \multirow{2}*{\vtop{\hbox{\textbf{Repr}}\vspace{1mm}\hbox{\textbf{Filter}}}} & \multicolumn{5}{c}{\textbf{Random}} & \multirow{2}{*}{\textbf{Duplicate}} \\ 
    \cmidrule(lr){6-10}
    & & & & & \textbf{$\mathcal{X}^{all}$} & ShareGPT & WizardLM & Alpaca & Dolly &  \\
    \midrule
    \rowcolor{gray!15} \multicolumn{11}{c}{\textit{LLaMA-3-8B}} \\
    Facility Loc. $_{\times10^5}$ & \cellcolor{BLUE!40} 2.99 & \cellcolor{ORANGE!10} 2.73 & \cellcolor{BLUE!40} 2.99 & \cellcolor{BLUE!20} 2.86 & \cellcolor{BLUE!40} 2.99 & \cellcolor{BLUE!0} 2.83 & \cellcolor{BLUE!30} 2.88 & \cellcolor{BLUE!0} 2.83 & \cellcolor{ORANGE!20} 2.59 & \cellcolor{ORANGE!30} 2.52 \\    
    DistSum$_{cosine}$  & \cellcolor{BLUE!30} 0.648 & \cellcolor{BLUE!60} 0.746 & \cellcolor{BLUE!0} 0.629 & \cellcolor{BLUE!50} 0.703 & \cellcolor{BLUE!10} 0.634 & \cellcolor{BLUE!40} 0.656 & \cellcolor{ORANGE!30} 0.578 & \cellcolor{ORANGE!10} 0.605 & \cellcolor{ORANGE!20} 0.603 & \cellcolor{BLUE!10} 0.634 \\
    Vendi Score $_{\times10^7}$ & \cellcolor{BLUE!30} 1.70 & \cellcolor{BLUE!60} 2.53 & \cellcolor{BLUE!10} 1.59 & \cellcolor{BLUE!50} 2.23 & \cellcolor{BLUE!20} 1.61 & \cellcolor{BLUE!30} 1.70 & \cellcolor{ORANGE!10} 1.44 & \cellcolor{ORANGE!20} 1.32 & \cellcolor{ORANGE!10} 1.44 & \cellcolor{ORANGE!30} 0.05 \\
    \textbf{NovelSum (Ours)} & \cellcolor{BLUE!60} 0.693 & \cellcolor{BLUE!50} 0.687 & \cellcolor{BLUE!30} 0.673 & \cellcolor{BLUE!20} 0.671 & \cellcolor{BLUE!40} 0.675 & \cellcolor{BLUE!10} 0.628 & \cellcolor{BLUE!0} 0.591 & \cellcolor{ORANGE!10} 0.572 & \cellcolor{ORANGE!20} 0.50 & \cellcolor{ORANGE!30} 0.461 \\
    \midrule    
    \textbf{Model Performance} & \cellcolor{BLUE!60}1.32 & \cellcolor{BLUE!50}1.31 & \cellcolor{BLUE!40}1.25 & \cellcolor{BLUE!30}1.05 & \cellcolor{BLUE!20}1.20 & \cellcolor{BLUE!10}0.83 & \cellcolor{BLUE!0}0.72 & \cellcolor{ORANGE!10}0.07 & \cellcolor{ORANGE!20}-0.14 & \cellcolor{ORANGE!30}-1.35 \\
    \midrule
    \midrule
    \rowcolor{gray!15} \multicolumn{11}{c}{\textit{Qwen-2.5-7B}} \\
    Facility Loc. $_{\times10^5}$ & \cellcolor{BLUE!40} 3.54 & \cellcolor{ORANGE!30} 3.42 & \cellcolor{BLUE!40} 3.54 & \cellcolor{ORANGE!20} 3.46 & \cellcolor{BLUE!40} 3.54 & \cellcolor{BLUE!30} 3.51 & \cellcolor{BLUE!10} 3.50 & \cellcolor{BLUE!10} 3.50 & \cellcolor{ORANGE!20} 3.46 & \cellcolor{BLUE!0} 3.48 \\ 
    DistSum$_{cosine}$ & \cellcolor{BLUE!30} 0.260 & \cellcolor{BLUE!60} 0.440 & \cellcolor{BLUE!0} 0.223 & \cellcolor{BLUE!50} 0.421 & \cellcolor{BLUE!10} 0.230 & \cellcolor{BLUE!40} 0.285 & \cellcolor{ORANGE!20} 0.211 & \cellcolor{ORANGE!30} 0.189 & \cellcolor{ORANGE!10} 0.221 & \cellcolor{BLUE!20} 0.243 \\
    Vendi Score $_{\times10^6}$ & \cellcolor{ORANGE!10} 1.60 & \cellcolor{BLUE!40} 3.09 & \cellcolor{BLUE!10} 2.60 & \cellcolor{BLUE!60} 7.15 & \cellcolor{ORANGE!20} 1.41 & \cellcolor{BLUE!50} 3.36 & \cellcolor{BLUE!20} 2.65 & \cellcolor{BLUE!0} 1.89 & \cellcolor{BLUE!30} 3.04 & \cellcolor{ORANGE!30} 0.20 \\
    \textbf{NovelSum (Ours)}  & \cellcolor{BLUE!40} 0.440 & \cellcolor{BLUE!60} 0.505 & \cellcolor{BLUE!20} 0.403 & \cellcolor{BLUE!50} 0.495 & \cellcolor{BLUE!30} 0.408 & \cellcolor{BLUE!10} 0.392 & \cellcolor{BLUE!0} 0.349 & \cellcolor{ORANGE!10} 0.336 & \cellcolor{ORANGE!20} 0.320 & \cellcolor{ORANGE!30} 0.309 \\
    \midrule
    \textbf{Model Performance} & \cellcolor{BLUE!30} 1.06 & \cellcolor{BLUE!60} 1.45 & \cellcolor{BLUE!40} 1.23 & \cellcolor{BLUE!50} 1.35 & \cellcolor{BLUE!20} 0.87 & \cellcolor{BLUE!10} 0.07 & \cellcolor{BLUE!0} -0.08 & \cellcolor{ORANGE!10} -0.38 & \cellcolor{ORANGE!30} -0.49 & \cellcolor{ORANGE!20} -0.43 \\
    \bottomrule
    \end{tabular}
    }
    \caption{Measuring the diversity of datasets selected by different strategies using \textit{NovelSum} and baseline metrics. Fine-tuned model performances (Eq. \ref{eq:perf}), based on MT-bench and AlpacaEval, are also included for cross reference. Darker \colorbox{BLUE!60}{blue} shades indicate higher values for each metric, while darker \colorbox{ORANGE!30}{orange} shades indicate lower values. While data selection strategies vary in performance on LLaMA-3-8B and Qwen-2.5-7B, \textit{NovelSum} consistently shows a stronger correlation with model performance than other metrics. More results are provided in Appendix \ref{app:results}.}
    \label{tbl:main}
    \vspace{-4mm}
\end{table*}


\begin{table}[t!]
\centering
\resizebox{\linewidth}{!}{
\begin{tabular}{lcccc}
\toprule
\multirow{2}*{\textbf{Diversity Metrics}} & \multicolumn{3}{c}{\textbf{LLaMA}} & \textbf{Qwen}\\
\cmidrule(lr){2-4} \cmidrule(lr){5-5} 
& \textbf{Pearson} & \textbf{Spearman} & \textbf{Avg.} & \textbf{Avg.} \\
\midrule
TTR & -0.38 & -0.16 & -0.27 & -0.30 \\
vocd-D & -0.43 & -0.17 & -0.30 & -0.31 \\
\midrule
Facility Loc. & 0.86 & 0.69 & 0.77 & 0.08 \\
Entropy & 0.93 & 0.80 & 0.86 & 0.63 \\
\midrule
LDD & 0.61 & 0.75 & 0.68 & 0.60 \\
KNN Distance & 0.59 & 0.80 & 0.70 & 0.67 \\
DistSum$_{cosine}$ & 0.85 & 0.67 & 0.76 & 0.51 \\
Vendi Score & 0.70 & 0.85 & 0.78 & 0.60 \\
DistSum$_{L2}$ & 0.86 & 0.76 & 0.81 & 0.51 \\
Cluster Inertia & 0.81 & 0.85 & 0.83 & 0.76 \\
Radius & 0.87 & 0.81 & 0.84 & 0.48 \\
\midrule
NovelSum & \textbf{0.98} & \textbf{0.95} & \textbf{0.97} & \textbf{0.90} \\
\bottomrule
\end{tabular}
}
\caption{Correlations between different metrics and model performance on LLaMA-3-8B and Qwen-2.5-7B.  “Avg.” denotes the average correlation (Eq. \ref{eq:cor}).}
\label{tbl:correlations}
\vspace{-2mm}
\end{table}

\paragraph{\textit{NovelSum} consistently achieves state-of-the-art correlation with model performance across various data selection strategies, backbone LLMs, and correlation measures.}
Table \ref{tbl:main} presents diversity measurement results on datasets constructed by mainstream data selection methods (based on $\mathcal{X}^{all}$), random selection from various sources, and duplicated samples (with only $m=100$ unique samples). 
Results from multiple runs are averaged for each strategy.
Although these strategies yield varying performance rankings across base models, \textit{NovelSum} consistently tracks changes in IT performance by accurately measuring dataset diversity. For instance, K-means achieves the best performance on LLaMA with the highest NovelSum score, while K-Center-Greedy excels on Qwen, also correlating with the highest NovelSum. Table \ref{tbl:correlations} shows the correlation coefficients between various metrics and model performance for both LLaMA and Qwen experiments, where \textit{NovelSum} achieves state-of-the-art correlation across different models and measures.

\paragraph{\textit{NovelSum} can provide valuable guidance for data engineering practices.}
As a reliable indicator of data diversity, \textit{NovelSum} can assess diversity at both the dataset and sample levels, directly guiding data selection and construction decisions. For example, Table \ref{tbl:main} shows that the combined data source $\mathcal{X}^{all}$ is a better choice for sampling diverse IT data than other sources. Moreover, \textit{NovelSum} can offer insights through comparative analyses, such as: (1) ShareGPT, which collects data from real internet users, exhibits greater diversity than Dolly, which relies on company employees, suggesting that IT samples from diverse sources enhance dataset diversity \cite{wang2024diversity-logD}; (2) In LLaMA experiments, random selection can outperform some mainstream strategies, aligning with prior work \cite{xia2024rethinking,diddee2024chasing}, highlighting gaps in current data selection methods for optimizing diversity.



\subsection{Ablation Study}


\textit{NovelSum} involves several flexible hyperparameters and variations. In our main experiments, \textit{NovelSum} uses cosine distance to compute $d(x_i, x_j)$ in Eq. \ref{eq:dad}. We set $\alpha = 1$, $\beta = 0.5$, and $K = 10$ nearest neighbors in Eq. \ref{eq:pws} and \ref{eq:dad}. Here, we conduct an ablation study to investigate the impact of these settings based on LLaMA-3-8B.

\begin{table}[ht!]
\centering
\resizebox{\linewidth}{!}{
\begin{tabular}{lccc}
\toprule
\textbf{Variants} & \textbf{Pearson} & \textbf{Spearman} & \textbf{Avg.} \\
\midrule
NovelSum & 0.98 & 0.96 & 0.97 \\
\midrule
\hspace{0.10cm} - Use $L2$ distance & 0.97 & 0.83 & 0.90\textsubscript{↓ 0.08} \\
\hspace{0.10cm} - $K=20$ & 0.98 & 0.96 & 0.97\textsubscript{↓ 0.00} \\
\hspace{0.10cm} - $\alpha=0$ (w/o proximity) & 0.79 & 0.31 & 0.55\textsubscript{↓ 0.42} \\
\hspace{0.10cm} - $\alpha=2$ & 0.73 & 0.88 & 0.81\textsubscript{↓ 0.16} \\
\hspace{0.10cm} - $\beta=0$ (w/o density) & 0.92 & 0.89 & 0.91\textsubscript{↓ 0.07} \\
\hspace{0.10cm} - $\beta=1$ & 0.90 & 0.62 & 0.76\textsubscript{↓ 0.21} \\
\bottomrule
\end{tabular}
}
\caption{Ablation Study for \textit{NovelSum}.}
\label{tbl:ablation}
\vspace{-2mm}
\end{table}

In Table \ref{tbl:ablation}, $\alpha=0$ removes the proximity weights, and $\beta=0$ eliminates the density multiplier. We observe that both $\alpha=0$ and $\beta=0$ significantly weaken the correlation, validating the benefits of the proximity-weighted sum and density-aware distance. Additionally, improper values for $\alpha$ and $\beta$ greatly reduce the metric's reliability, highlighting that \textit{NovelSum} strikes a delicate balance between distances and distribution. Replacing cosine distance with Euclidean distance and using more neighbors for density approximation have minimal impact, particularly on Pearson's correlation, demonstrating \textit{NovelSum}'s robustness to different distance measures.






\section{Conclusion}
In this work, we propose a simple yet effective approach, called SMILE, for graph few-shot learning with fewer tasks. Specifically, we introduce a novel dual-level mixup strategy, including within-task and across-task mixup, for enriching the diversity of nodes within each task and the diversity of tasks. Also, we incorporate the degree-based prior information to learn expressive node embeddings. Theoretically, we prove that SMILE effectively enhances the model's generalization performance. Empirically, we conduct extensive experiments on multiple benchmarks and the results suggest that SMILE significantly outperforms other baselines, including both in-domain and cross-domain few-shot settings.

\bibliography{reference}
\bibliographystyle{plainnat}
\newpage
\appendix
\onecolumn 
\section{Ill-conditioned Differential Operators Lead to Difficult Optimization Problems}
In this section, we state and prove the formal version of \cref{thm:informal_ill_cond}.
The overall structure of the proof is based on showing the conditioning of the Gauss-Newton matrix of the population PINN loss is controlled by the conditioning of the differential operator.
We then show the empirical Gauss-Newton matrix is close to its population counterpart by using matrix concentration techniques. 
Finally, as the conditioning of $H_L$ at a minimizer is controlled by the empirical Gauss-Newton matrix, we obtain the desired result. 
% Part of the first portion of the analysis is similar to \citet{de2023operator}, but we perform no linearization. 
% So the quantities 

\label{section:ill-cond-D}
\subsection{Preliminaries}
Similar to \citet{de2023operator}, we consider a general linear PDE with Dirichlet boundary conditions:
\[
\begin{array}{ll}
    & \Dc[u](x) = f(x),\quad x\in \Omega, \\
    & u(x) = g(x), \quad x\in \partial \Omega,
\end{array}
\]
where $u: \R^d \mapsto \R$, $f:\R^d \mapsto \R$ and $\Omega$ is a bounded subset of $\R^d$.
The ``population'' PINN objective for this PDE is
\[
L_\infty(w) = \frac{1}{2}\int_{\Omega}\left(\Dc[u(x;w)]-f(x)\right)^2d\mu(x)+\frac{\lambda}{2} \int_{\partial \Omega}\left(u(x; w)-g(x)\right)^2d\sigma(x).
\]
$\lambda$ can be any positive real number; we set $\lambda = 1$ in our experiments.
Here $\mu$ and $\sigma$ are probability measures on $\Omega$ and $\partial \Omega$ respectively, from which the data is sampled. 
The empirical PINN objective is given by
\[
L(w) = \frac{1}{2\nres}\sum_{i=1}^{\nres}\left(\Dc[u(x^i_r;w)]-f(x_i)\right)^2+\frac{\lambda}{2\nbc}\sum_{j=1}^{\nbc}\left(u(x_b^j;w)-g(x_j)\right)^2.
\]
Moreover, throughout this section we use the notation $\langle f,g\rangle_{L^{2}(\Omega)}$ to denote the standard $L^2$-inner product on $\Omega$:
\[
\langle f,g\rangle_{L^{2}(\Omega)} = \int_{\Omega}fg d\mu(x).
\]


\begin{lemma}
    The Hessian of the $L_\infty(w)$ is given by
    \begin{align*}
    H_{L_\infty}(w) & = \int_{\Omega}\Dc[\nabla_w u(x;w)]\Dc[\nabla_w u(x;w)]^{T}d\mu(x)+\int_{\Omega}\Dc[\nabla^2_w u(x;w)]\left(\Dc[\nabla_w u(x;w)]-f(x)\right)d\mu(x)\\
    & + \lambda\int_{\partial \Omega}\nabla_w u(x; w)\nabla_w u(x; w)^{T}d\sigma(x) + \lambda\int_{\partial \Omega}\nabla^2_w u(x; w)\left(u(x;w)-g(x)\right)d\sigma(x). 
    \end{align*}
    The Hessian of $L(w)$ is given by
    \begin{align}
       H_L(w) & = \frac{1}{n_{\textup{res}}}\sum^{n_{\textup{res}}}_{i=1}\Dc[\nabla_w u(x_r^i; w)]\Dc[\nabla_w u(x_r^i; w)]^{T}+\frac{1}{n_{\textup{res}}}\sum^{n_{\textup{res}}}_{i=1}\Dc[\nabla^2_w u(x^r_i;w)]\left(\Dc[\nabla_w u(x_r^i;w)]-f(x_r^i)\right)\\
       & +\frac{\lambda}{n_{\textup{bc}}}\sum_{j=1}^{n_{\textup{bc}}}\nabla_w u(x_b^j;w)\nabla_w u(x_b^j;w)^{T} + \frac{\lambda}{n_{\textup{bc}}}\sum^{n_{\textup{bc}}}_{j=1}\nabla^2_w u(x_b^j;w)\left(u(x_b^j;w)-g(x_j)\right). \nonumber
    \end{align}
    In particular, for $w_\star\in \Wstar$
    \begin{align*}
        H_L(w_\star) = G_r(w)+ G_b(w).
    \end{align*}
    Here 
    \[
    G_r(w) \coloneqq \frac{1}{n_{\textup{res}}}\sum^{n_{\textup{res}}}_{i=1}\Dc[\nabla_w u(x_i; w_\star)]\Dc[\nabla_w u(x_i; w_\star)]^{T},\quad G_b(w) = \frac{\lambda}{n_{\textup{bc}}}\sum_{j=1}^{n_{\textup{bc}}}\nabla_w u(x_b^j;w_\star)\nabla_w u(x_b^j;w_\star)^{T}.
    \]
\end{lemma}
Define the maps $\F_{\textup{res}}(w) = \begin{bmatrix}
    \Dc[u(x_r^1;w)] \\
    \vdots \\
    \Dc[u(x_r^{\nres};w)]
\end{bmatrix}$,
and $\F_{\textup{bc}}(w) = \begin{bmatrix}
    u(x_b^1;w) \\
    \vdots \\
    u(x_b^{\nbc};w)]
\end{bmatrix}$.
We have the following important lemma, which follows via routine calculation. 
\begin{lemma}
\label{lemma:jac_ntk}
    Let $n = \nres+\nbc$. Define the map $\mathcal F:\R^p\rightarrow \R^{n}$, by stacking $\F_{\textup{res}}(w), \F_{\textup{bc}}(w)$.
    Then, the Jacobian of $\F$ is given by
    \[
    J_\F(w) = \begin{bmatrix}
        J_{\F_{\textup{res}}}(w) \\
        J_{\F_{\textup{bc}}}(w).
    \end{bmatrix}
    \]
    Moreover, the tangent kernel $K_\F(w) = J_\F(w)J_\F(w)^{T}$ is given by
    \[ K_\F(w) = 
    \begin{bmatrix}
        J_{\F_{\textup{res}}}(w)J_{\F_{\textup{res}}}(w)^{T} & J_{\F_{\textup{res}}}(w)J_{\F_{\textup{bc}}}(w)^{T}  \\
        J_{\F_{\textup{bc}}}(w)J_{\F_{\textup{res}}}(w)^{T} & J_{\F_{\textup{bc}}}(w)J_{\F_{\textup{bc}}}(w)^{T} 
    \end{bmatrix} =
    \begin{bmatrix}
        K_{\F_{\textup{res}}}(w) & J_{\F_{\textup{res}}}(w)J_{\F_{\textup{bc}}}(w)^{T}  \\
        J_{\F_{\textup{bc}}}(w)J_{\F_{\textup{res}}}(w)^{T} & K_{\F_{\textup{bc}}}(w) 
    \end{bmatrix}.
    \]
\end{lemma}

\subsection{Relating $G_{\infty}(w)$ to $\mathcal D$}
Isolate the population Gauss-Newton matrix for the residual:
\[
G_{\infty}(w) = \int_{\Omega}\Dc[\nabla_w u(x;w)]\Dc[\nabla_w u(x;w)]^{T}d\mu(x).
\]
Analogous to \citet{de2023operator} we define the functions $\phi_i(x;w) = \partial_{w_i}u(x;w)$ for $i\in\{1\dots,p\}$.
From this and the definition of $G_{\infty}(w)$, it follows that $\left(G_\infty(w)\right)_{ij} = \langle \Dc[\phi_i], \Dc[\phi_j]\rangle_{L^2(\Omega)}$.

Similar to \citet{de2023operator} we can associate each $w\in\R^p$ with a space of functions $\mathcal H(w) = \textup{span}\left(\phi_1(x;w),\dots,\phi_p(x;w)\right)\subset L^2(\Omega).$
We also define two linear maps associated with $\mathcal H(w)$:
\[
T(w)v = \sum_{i=1}^{p}v_i\phi_i(x;w),
\]
\[
T^{*}(w)f = \left(\langle f,\phi_1\rangle_{L^2(\Omega)},\dots,\langle f,\phi_p\rangle_{L^2(\Omega)}\right).
\]
From these definitions, we establish the following lemma. 
\begin{lemma}[Characterizing $G_{\infty}(w)$]
\label{lemma:Pop-GN}
Define $\mathcal A = \Dc^{*}\Dc$. 
Then the matrix $G_{\infty}(w)$ satisfies
    \[
    G_{\infty}(w) = T^{*}(w)\mathcal A T(w). 
    \]
\end{lemma}
\begin{proof}
Let $e_i$ and $e_j$ denote the $ith$ and $jth$ standard basis vectors in $\R^p$. 
Then,
\begin{align*}
(G_{\infty}(w))_{ij} &= \langle \Dc[\phi_i](w), \Dc[\phi_j](w)\rangle_{L^2(\Omega)} = \langle \phi_i(w),\Dc^{*}\Dc[\phi_j(w)] \rangle_{L^2(\Omega)} = \langle Te_i, \Dc^{*}\Dc[Te_j]\rangle_{L^2(\Omega)} \\
&= \langle e_i, (T^{*}\Dc^{*}\Dc T)[e_j]\rangle_{L^2(\Omega)},
\end{align*}
where the second equality follows from the definition of the adjoint. 
Hence, using $\mathcal A = \Dc^{*}\Dc$, we conclude $G_{\infty}(w) = T^{*}(w)\mathcal A T(w)$.
\end{proof}

Define the kernel integral operator $\mathcal K_{\infty}(w):L^2(\Omega)\rightarrow \mathcal H$ by
\begin{equation}
\label{eq:kern_op}
\mathcal K_{\infty}(w)[f](x) = T(w)T^{*}(w)f = \sum_{i=1}^{p}\langle f,\phi_i(x;w)\rangle \phi_i(x;w),
\end{equation}
and the kernel matrix $A(w)$ with entries $A_{ij}(w) = \langle \phi_{i}(x;w),\phi_{j}(x;w)\rangle_{L^2(\Omega)}$. 

Using \cref{lemma:Pop-GN} and applying the same logic as in the proof of Theorem 2.4 in \citet{de2023operator},
we obtain the following theorem. 
% We define the following weighted-inner product on $L^{2}(\Omega)$:
% \[
% \langle f,g\rangle = \langle f, \Kc_{\infty}g\rangle_{L_2(\Omega)}. 
% \]
\begin{theorem}
\label{thm:pop-gn-eigvals}
    Suppose that the matrix $A(w)$ is invertible.
    Then the eigenvalues of $G_{\infty}(w)$ satisfy
    \[
    \lambda_j(G_\infty(w)) = \lambda_j(\mathcal A\circ \Kc_\infty(w)),\quad \text{for all}~j\in[p].
    \]
\end{theorem}
% \begin{proof}
% \begin{align*}
%     \lamMax(G_{\infty}) &= \sup_{\|v\|=1}v^{T}G_{\infty}(w)v = \sup_{f\in \mathcal H, \|T^{*}f\| = 1}(T^{*}f)^{T}G_{\infty}(w)(T^{*}f) \\
%     &= \sup_{f\in \mathcal H,\|T^{*}f\| = 1}(T^{*}f)^{T}(T^{*}\mathcal A T)(T^{*}f) = \sup_{f\in \mathcal H, \|T^{*}f\| = 1}\langle f, TT^{*}(\mathcal ATT^{*}f)\rangle_{L^2(\Omega)}\\
%     & = \sup_{f\in \mathcal H,\|f\|_{\mathcal K_{\infty}} = 1}\langle f, \mathcal A\circ TT^{*}f\rangle_{\mathcal K_{\infty}} = \lamMax(\mathcal A\circ\Kc_{\infty})
% \end{align*}
% \end{proof}

% \begin{align*}
%     \langle f, (\mathcal A\circ TT^{*})g\rangle &= \langle f, TT^{*}(\mathcal A \circ TT^{*})g\rangle_{L^{2}(\Omega)} = \langle \mathcal A\circ TT^{*}f, TT^{*}g\rangle_{L^{2}(\Omega)}\\
%     &= \langle  (\A\circ TT^{*})f,g \rangle
% \end{align*}
% \[
% \kappa(G_{\infty}) = \kappa(\mathcal A\circ\Kc_{\infty})
% \]

\subsection{$G_r(w)$ Concentrates Around $G_{\infty}(w)$}
In order to relate the conditioning of the population objective to the empirical objective, we must relate the population Gauss-Newton residual matrix to its empirical counterpart. 
We accomplish this by showing $G_r(w)$ concentrates around $G_{\infty}(w)$. 
%Here we assume $\mu$ is a probability measure on $\Omega$, from which the training set is sampled.
To this end, we recall the following variant of the intrinsic dimension matrix Bernstein inequality from \citet{tropp2015introduction}.
\begin{theorem}[Intrinsic Dimension Matrix Bernstein]
 \label{thm:int_bern}
    Let $\{X_i\}_{i\in [n]}$ be a sequence of independent mean zero random matrices of the same size. 
    Suppose that the following conditions hold:
    \begin{align*}
        &\|X_i\|  \leq B,~\sum^{n}_{i=1}\mathbb E[X_i X_i^{T}]\preceq V_1,~\sum^{n}_{i=1}\mathbb E[X_i^{T} X_i]\preceq V_2.
    \end{align*}
    Define 
    \[\mathcal V = 
    \begin{bmatrix}
        V_1 & 0 \\
        0   &  V_2
    \end{bmatrix},~ \varsigma^2 = \max\{\|V_1\|,\|V_2\|\},
    \]
    and the \emph{intrinsic dimension} $d_{\textup{int}} = \frac{\textup{trace}(\mathcal V)}{\|\mathcal V\|}$.
    \newline
    Then for all $t\geq \varsigma+\frac{B}{3}$, 
    \[
    \mathbb P\left(\left\|\sum^{n}_{i=1}X_i\right\|\geq t\right) \leq 4d_{\textup{int}}\exp\left(-\frac{3}{8}\min\left\{\frac{t^2}{\varsigma^2},\frac{t}{B}\right\}\right).
    \]   
\end{theorem}

Next, we recall two key concepts from the kernel ridge regression literature and approximation via sampling literature: $\gamma$-\emph{effective dimension} and $\gamma$-\emph{ridge leverage coherence} \cite{bach2013sharp,cohen2017input,rudi2017falkon}. 
\begin{definition}[$\gamma$-Effective dimension and $\gamma$-ridge leverage coherence]
    Let $\gamma>0$. 
    Then the $\gamma$-effective dimension of $G_{\infty}(w)$ is given by
    \[
    d^{\gamma}_{\textup{eff}}(G_{\infty}(w)) = \textup{trace}\left(G_{\infty}(w)(G_{\infty}(w)+\gamma I)^{-1}\right).
    \]
    The $\gamma$-ridge leverage coherence is given by
    \[
    \chi^\gamma(G_{\infty}(w)) = \sup_{x\in \Omega}\frac{\left\|(G_{\infty}(w)+\gamma I)^{-1/2}\Dc[\nabla_w u(x;w)]\right\|^2}{\mathbb E_{x\sim \mu}\left\|(G_{\infty}(w)+\gamma I)^{-1/2}\Dc[\nabla_w u(x;w)]\right\|^2} = \frac{\sup_{x\in \Omega}{\left\|(G_{\infty}(w)+\gamma I)^{-1/2}\Dc[\nabla_w u(x;w)]\right\|^2}}{{d^{\gamma}_{\textup{eff}}(G_{\infty}(w))}}.
    \]
\end{definition}
Observe that $d^{\gamma}_{\textup{eff}}(G_{\infty}(w))$ only depends upon $\gamma$ and $w$, while $\chi^\gamma(G_{\infty}(w)) $ only depends upon $\gamma, w,$ and $\Omega$. 
Moreover, $\chi^\gamma(G_{\infty}(w))<\infty$ as $\Omega$ is bounded. 

We prove the following lemma using the $\gamma$-effective dimension and $\gamma$-ridge leverage coherence in conjunction with \cref{thm:int_bern}.
\begin{lemma}[Finite-sample approximation]
\label{lemma:sampling}
Let $0<\gamma<\lambda_1(G_{\infty}(w))$. 
If $\nres\geq 40\chi^\gamma(G_{\infty}(w))d^{\gamma}_{\textup{eff}}(G_{\infty}(w))\log\left(\frac{8d^{\gamma}_{\textup{eff}}(G_{\infty}(w))}{\delta}\right)$, then with probability at least $1-\delta$
    \[
    \frac{1}{2}\left[G_\infty(w)-\gamma I\right] \preceq G_{r}(w)\preceq \frac{1}{2}\left[3 G_\infty(w)+\gamma I.\right]
    \]
\end{lemma}
\begin{proof}
    Let $x_i = (G_{\infty}(w)+\gamma I)^{-1/2}\Dc[\nabla_w u(x_i;w)]$, and $X_i = \frac{1}{\nres}\left(x_ix_i^{T}-D_\gamma\right)$, where $D_\gamma = G_{\infty}(w)\left(G_{\infty}(w)+\gamma I\right)^{-1}$.
    Clearly, $\mathbb E[X_i] = 0$. 
    Moreover, the $X_i$'s are bounded as
    \begin{align*}
    \|X_i\| & = \max\left\{\frac{\lamMax(X_i)}{\nres},-\frac{\lamMin(X_i)}{\nres}\right\} \leq \max\left\{\frac{\|x_i\|^2}{\nres}, \frac{\lamMax(-X_i)}{\nres}\right\} \leq \max\left\{\frac{\chi^{\gamma}(G_{\infty}(w))d^{\gamma}_{\textup{eff}}(G_{\infty}(w))}{\nres},\frac{1}{\nres}\right\} \\
    & = \frac{\chi^{\gamma}(G_{\infty}(w))d^{\gamma}_{\textup{eff}}(G_{\infty}(w))}{\nres}.
    \end{align*}
    Thus, it remains to verify the variance condition. 
    We have
    \begin{align*}
    \sum^{\nres}_{i=1}\mathbb E[X_iX_i^{T}] & = \nres\mathbb E[X_1^2] = \nres\times \frac{1}{\nres^2}\mathbb E[(x_1x_1^{T}-D_\gamma)^{2}]\preceq \frac{1}{\nres}\mathbb E[\|x_1\|^2 x_1 x_1^{T}] \\ 
    & \preceq \frac{\chi^{\gamma}(G_{\infty}(w))d^{\gamma}_{\textup{eff}}(G_{\infty}(w))}{\nres} D_\gamma. 
    \end{align*}
    Hence, the conditions of \cref{thm:int_bern} hold with $B = \frac{\chi^{\gamma}(G_{\infty}(w))d^{\gamma}_{\textup{eff}}(G_{\infty}(w))}{\nres}$ and $V_1 = V_2 = \frac{\chi^{\gamma}(G_{\infty}(w))d^{\gamma}_{\textup{eff}}(G_{\infty}(w))}{\nres} D_\gamma$.
    Now $1/2 \leq \|\mathcal V\|\leq 1$ as $\nres\geq \chi^{\gamma}(G_{\infty}(w))d^{\gamma}_{\textup{eff}}(G_{\infty}(w))$ and $\gamma\leq \lambda_1\left(G_\infty(w)\right)$.
    Moreover, as $V_1 = V_2$ we have $d_{\textup{int}} \leq 4 d^{\gamma}_{\textup{eff}}(G_{\infty}(w))$. 
    So, setting 
    \[
    t = \sqrt{\frac{8\chi^\gamma(G_{\infty}(w))d^{\gamma}_{\textup{eff}}(G_{\infty}(w))\log\left(\frac{8d^{\gamma}_{\textup{eff}}(G_{\infty}(w))}{\delta}\right)}{3\nres}}+\frac{8\chi^\gamma(G_{\infty}(w))d^{\gamma}_{\textup{eff}}(G_{\infty}(w))\log\left(\frac{8d^{\gamma}_{\textup{eff}}(G_{\infty}(w))}{\delta}\right)}{3\nres}
    \]
    and using $\nres\geq 40\chi^\gamma(G_{\infty}(w)) d^{\gamma}_{\textup{eff}}(G_{\infty}(w))\log\left(\frac{8d^{\gamma}_{\textup{eff}}(G_{\infty}(w))}{\delta}\right)$, we conclude
    \[\mathbb P\left(\left\|\sum_{i=1}^{\nres}X_i\right\|\geq \frac{1}{2}\right)\leq \delta.\]
    Now, $\left\|\sum_{i=1}^{\nres}X_i\right\|\leq \frac{1}{2}$ implies
    \[
    -\frac{1}{2}\left[G_{\infty}(w)+\gamma I\right]\preceq G_r(w)-G_{\infty}(w)\preceq \frac{1}{2}\left[G_{\infty}(w)+\gamma I\right].
    \]
    The claim now follows by rearrangement. 
\end{proof}

By combining \cref{thm:pop-gn-eigvals} and \cref{lemma:sampling}, we show that if the spectrum of $\A\circ \Kc_{\infty}(w)$ decays, then the spectrum of the empirical Gauss-Newton matrix also decays with high probability.  
\begin{proposition}[Spectrum of empirical Gauss-Newton matrix decays fast]
\label{prop:emp_gn_spectrum}
Suppose the eigenvalues of $\A\circ \Kc_{\infty}(w)$ satisfy $\lambda_j(\mathcal A\circ \Kc_{\infty}(w))\leq Cj^{-2\alpha}$, where $\alpha>1/2$ and $C>0$ is some absolute constant.
Then if $\sqrt{\nres}\geq 40C_1\chi^{\gamma}(G_{\infty}(w))\log\left(\frac{1}{\delta}\right)$, for some absolute constant $C_1$, it holds that
\[
  \lambda_{\nres}(G_r(w))\leq \nres^{-\alpha}
\]
with probability at least $1-\delta$.
         
\end{proposition}
\begin{proof} 
    The hypotheses on the decay of the eigenvalues implies $d^{\gamma}_{\textup{eff}}(G_{\infty}(w)) \leq C_1\gamma^{-\frac{1}{2\alpha}}$ (see Appendix C of \citet{bach2013sharp}).
    Consequently, given $\gamma = \nres^{-\alpha}$, we have $d^{\gamma}_{\textup{eff}}(G_{\infty}(w)) \leq C_1\nres^{\frac{1}{2}}$. 
    Combining this with our hypotheses on $\nres$, it follows $\nres\geq 40 C_1\chi^{\gamma}(G_{\infty}(w))d^{\gamma}_{\textup{eff}}(G_{\infty}(w))\log\left(\frac{8d^{\gamma}_{\textup{eff}}(G_{\infty}(w))}{\delta}\right)$.
    Hence \cref{lemma:sampling} implies with probability at least $1-\delta$ that 
    \[
    G_r(w)\preceq \frac{1}{2}\left(3 G_\infty(w)+\gamma I\right),
    \]
    which yields for any $1\leq r\leq n$
    \[
    \lambda_{\nres}(G_r(w))\leq \frac{1}{2}\left(3\lambda_r(G_\infty(w))+\gamma\right).
    \]
    Combining the last display with $\nres\geq 3d^{\gamma}_{\textup{eff}}(G_{\infty}(w))$, 
    Lemma 5.4 of \citet{frangella2023randomized} guarantees $\lambda_r(G_\infty(w))\leq \gamma/3$, and so 
    \[
    \lambda_{\nres}(G_r(w))\leq \frac{1}{2}\left(3\lambda_r(G_\infty(w))+\gamma\right)\leq \gamma \leq \nres^{-\alpha}.
    \]
\end{proof}

\subsection{Formal Statement of \cref{thm:informal_ill_cond} and Proof}
\begin{theorem}[An ill-conditioned differential operator leads to hard optimization]
    Fix $w_\star \in \Wstar$, and let $\mathcal S$ be a set containing $w_\star$ for which $\mathcal S$ is $\mu$-\PL.
    Let $\alpha>1/2$.
    If the eigenvalues of $\A\circ \Kc_{\infty}(w_\star)$ satisfy $\lambda_j(\mathcal A\circ \Kc_{\infty}(w_\star))\leq C j^{-2\alpha}$ and $\sqrt{\nres}\geq 40 C_1\chi^{\gamma}(G_{\infty}(w_\star))\log\left(\frac{1}{\delta}\right)$, then 
    \[
            \kappa_L(\mathcal S) \geq C_2\nres^{\alpha},
    \]
    with probability at least $1-\delta$.
    Here $C, C_1,$ and $C_2$ are absolute constants. 
        
\end{theorem}

\begin{proof}
    By the assumption on $\nres$, the conditions of \cref{prop:emp_gn_spectrum} are met, so, 
    \[
    \lambda_{\nres}(G_r(w_\star))\leq \nres^{-\alpha}.
    \] 
    with probability at least $1-\delta$.
    By definition $G_r(w_\star) = J_{\F_{\textup{res}}}(w_\star)^{T}J_{\F_{\textup{res}}}(w_\star)$, consequently,
    \[
    \lambda_{\nres}(K_{\F_{\textup{res}}}(w_\star)) = \lambda_{\nres}(G_r(w_\star)) \leq \nres^{-\alpha}.
    \] 
    Now, the \PL-constant for $\mathcal S$, satisfies $\mu = \inf_{w \in \mathcal S}\lambda_{n}(K_{\F}(w))$ \cite{liu2022loss}. 
    Combining this with the expression for $K_\F(w_\star)$ in \cref{lemma:jac_ntk}, we reach 
    \[
    \mu\leq \lambda_n(K_\F(w_\star))\leq \lambda_{\nres}(K_{\F_{\textup{res}}}(w_\star))\leq \nres^{-\alpha},
    \]
    where the second inequality follows from Cauchy's Interlacing theorem. 
    Recalling that $\kappa_L(\mathcal S) = \frac{\sup_{w \in \mathcal S}\|H_L(w)\|}{\mu}$, and $H_L(w_\star)$ is symmetric psd, we reach
    \begin{align*}
        \kappa_L(\mathcal S) \geq \frac{\lambda_1(H_L(w_\star))}{\mu}\overset{(1)}{\geq} \frac{\lambda_1(G_r(w_\star))+\lambda_p(G_b(w_\star))}{\mu} \overset{(2)}{=} \frac{\lambda_1(G_r(w_\star))}{\mu} \overset{(3)}{\geq} C_3\lambda_1(G_\infty(w_\star))\nres^{\alpha}. 
    \end{align*}
    Here $(1)$ uses $H_L(w_\star) = G_r(w_\star)+G_b(w_\star)$ and Weyl's inequalities, $(2)$ uses $p\geq \nres+\nbc$, so that $\lambda_p(G_b(w_\star)) = 0$.
    Inequality $(3)$ uses the upper bound on $\mu$ and the lower bound on $G_r(w)$ given in \cref{lemma:sampling}. 
    Hence, the claim follows with $C_2 = C_3\lambda_1(G_\infty(w_\star))$.
    %As $\mu = \inf_{w \in \mathcal S}\lambda_n(K(w))\leq n^{-\beta}$, and $\kappa_L(\mathcal S) = \sup_{w \in \mathcal S}\frac{\lambda_1(H_L(w))}{\mu}$, we have
    %\[
    %\kappa_L(\mathcal S)\geq \frac{\lambda_1(H_L(w_\star))}{\mu} = \frac{\lambda_1(K(w_\star))}{\mu} \geq C_{2}\nres^{\beta}.
    %\]
\end{proof}
\subsection{$\kappa$ Grows with the Number of Residual Points}
\label{subsec:kappa_grows}
\begin{figure*}
    \centering
    \includegraphics[scale=0.45]{figs/condition_number_bound.pdf}
    \caption{Estimated condition number after 41000 iterations of \al{} with different number of residual points from $255 \times 100$ grid on the interior. Here $\lambda_i$ denotes the $i$th largest eigenvalue of the Hessian. The model has $2$ layers and the hidden layer has width $32$. The plot shows $\kappa_L$ grows polynomially in the number of residual points.}
    \label{fig:condition_number_bound}
\end{figure*}
\Cref{fig:condition_number_bound} plots the ratio $\lambda_1(H_L)/\lambda_{129}(H_L)$ near a minimizer $w_\star$. This ratio is a lower bound for the condition number of $H_L$, and is computationally tractable to compute. 
We see that the estimate of the $\kappa$ grows polynomially with $\nres$, which provides empirical verification for \cref{thm:informal_ill_cond}.

% Suppose, $n\geq 4Cd^{\lambda}_{\textup{eff}}(G_{\infty}(w))\log\left(\frac{1}{\delta}\right)$
% which implies
% \[
% \frac{1}{2}\left[G_\infty(w)-\lambda I\right]\preceq G(w)\preceq \frac{1}{2}\left[3 G_\infty(w)+\lambda I\right].
% \]
% Thus, $\lambda_{r_\star}(G(w))\leq 2\lambda_{r_\star}(G_{\infty}(w)) = 2 \lambda_{r_\star}\left(\A\circ K_{\infty}\right)$.
% As $n\geq 4d^{\lambda}_{\textup{eff}}(G_{\infty}(w))$ it follows that
% \[
% \lambda_{n}(G(w))\leq \lambda.
% \]
% With $n \geq C\sqrt{n}\log\left(\frac{1}{\delta}\right)$, for fast poly-decay this becomes 
% \[
% \lambda_{n}(G(w))\leq n^{-\frac{\beta}{2}},
% \]
% while for exponential
% \[
% \lambda_{n}(G(w))\leq \exp(-\sqrt{n}).
% \]
% So GD will converge in $\mathcal O\left(n^{\beta/2}\log(1/\epsilon)\right)$, $\mathcal O\left(\exp{(\sqrt{n})}\log(1/\epsilon)\right)$

\section{Convergence of GDND (\cref{alg-GDND})}
\label{section:GDND_conv}
In this section, we provide the formal version of \cref{thm:GDND_informal} and its proof. 
However, this is delayed till \cref{subsec:GDND_conv}, as the theorem is a consequence of a series of results.
Before jumping to the theorem, we recommend reading the statements in the preceding subsections to understand the statement and corresponding proof. 
\subsection{Overview and Notation}
Recall, we are interested in minimizing the objective in \eqref{eq:pinn_prob_gen}:
\[
L(w) = \frac{1}{2\nres}\sum_{i=1}^{\nres}\left(\Dc[u(x_r^i;w)]\right)^2+\frac{1}{2\nbc}\sum_{j=1}^{\nbc}\left(\Bc[u(x_b^j;w)]\right)^2, 
\]
where $\Dc$ is the differential operator defining the PDE and $\Bc$ is the operator defining the boundary conditions. 
Define
\[
\F(w) = \begin{bmatrix}
    \frac{1}{\sqrt{\nres}}\Dc[u(x^1_r;w)] \\
    \vdots \\
    \frac{1}{\sqrt{\nres}}\Dc[u(x_r^{\nres};w)]\\
    \frac{1}{\sqrt{\nbc}}\Bc[u(x^1_b;w)] \\
    \vdots \\
    \frac{1}{\sqrt{\nbc}}\Bc[u(x_b^{\nbc};w)]
\end{bmatrix},~ y = 0
\]
Using the preceding definitions, our objective may be rewritten as:
\[
L(w) = \frac{1}{2}\|\F(w)-y\|^2.
\]
Throughout the appendix, we work with the condensed expression for the loss given above.
We denote the $(\nres+\nbc)\times p$ Jacobian matrix of $\mathcal F$ by $J_\F(w)$. 
%Note that $D\mathcal F(w) \in \mathbb{R}^{n\times p}$.
The tangent kernel at $w$ is given by the $n\times n$ matrix $K_\F(w) = J_\F(w) J_\F(w)^{T}$.
The closely related Gauss-Newton matrix is given by $G(w) = J_\F(w)^{T} J_\F(w)$.


\subsection{Global Behavior: Reaching a Small Ball About a Minimizer}
We begin by showing that under appropriate conditions, gradient descent outputs a point close to a minimizer after a fixed number of iterations.
We first start with the following assumption which is common in the neural network literature \cite{liu2022loss,liu2023aiming}.
\begin{assumption}
\label{assp:loss_reg}
     The mapping $\mathcal F(w)$ is $\mathcal L_\F$-Lipschitz, and the loss $L(w)$ is $\beta_{L}$-smooth.
\end{assumption}

% \begin{assumption}
%     Let $w_0$ denote the network weight at initialization, then such $L(w)$ is $\mu$-\PL in $B(w_0,2R)$, with $$.
% \end{assumption}

Under \Cref{assp:loss_reg} and a \PL-condition, we have the following theorem of \citet{liu2022loss}, which shows gradient descent converges linearly. 
\begin{theorem}
\label{thm:grad_dsct_conv}
    Let $w_0$ denote the network weights at initialization. 
    Suppose \Cref{assp:loss_reg} holds, and that $L(w)$ is $\mu$-P\L$^{\star}$ in $B(w_0,2R)$ with $R = \frac{2\sqrt{2\beta_{L}L(w_0)}}{\mu}$.
    Then the following statements hold:
    \begin{enumerate}
        \item The intersection $B(w_0,R)\cap\Wstar$ is non-empty.
        \item Gradient descent with step size $\eta = 1/\beta_L$ satisfies:
        \begin{align*}
        &w_{k+1} = w_k-\eta \nabla L(w_k)\in B(w_0,R)~ \text{for all } k\geq 0,\\
        &L(w_k)\leq \left(1-\frac{\mu}{\beta_L}\right)^kL(w_0).
        \end{align*}
    \end{enumerate}
\end{theorem}
For wide neural neural networks, it is known that the $\mu$-\PL condition in \cref{thm:grad_dsct_conv} hold with high probability, see \citet{liu2022loss} for details.

We also recall the following lemma from \citet{liu2023aiming}.
    \begin{lemma}[Descent Principle]
    \label{lemma:descent_principle}
        Let $L:\R^p\mapsto [0,\infty)$ be differentiable and $\mu$-\PL in the ball $B(w,r)$. 
        Suppose $L(w)<\frac{1}{2}\mu r^2$.
        Then the intersection $B(w,r)\cap \Wstar$ is non-empty, and
        \[
        \frac{\mu}{2}\dist^2(w,\Wstar)\leq L(w).
        \]
    \end{lemma}
        Let $\Lc_{H_L}$ be the Hessian Lipschitz constant in $B(w_0,2R)$, and $\Lc_{J_\F} = \sup_{w\in B(w_0,2R)}\|H_\F(w)\|$, where $\|H_\F(w)\| = \max_{i\in [n]}\|H_{\F_i}(w)\|$. 
        Define $M = \max\{\mathcal L_{\HL},\Lc_{J_\F},\mathcal \Lc_\F \Lc_{J_\F},1\}$,  $\epsLoc = \frac{\varepsilon \mu^{3/2}}{4M}$, where $\varepsilon\in (0,1)$.
        By combining \cref{thm:grad_dsct_conv} and \cref{lemma:descent_principle}, we are able to establish the following important corollary, which shows gradient descent outputs a point close to a minimizer.
    \begin{corollary}[Getting close to a minimizer]
        \label{corr:close_to_min}
        Set $\rho =  \min\left\{\frac{\epsLoc}{19\sqrt{\frac{\beta_L}{\mu}}},\sqrt{\mu}R,R\right\}$.
        Run gradient descent for $k = \frac{\beta_L}{\mu}\log\left(\frac{4\max\{2\beta_{L},1\}L(w_0)}{\mu\rho^2}\right)$ iterations, 
        gradient descent outputs a point $\wloc$ satisfying 
        \[
        L(\wloc) \leq \frac{\mu \rho^2}{4}\min\left\{1,\frac{1}{2\beta_L}\right\},
        \]
        \[
        \|\wloc-w_\star\|_{H_{L}(w_\star)+\mu I}\leq \rho,~ \text{for some}~ w_\star\in \Wstar.
        \]
    \end{corollary}
\begin{proof}
    The first claim about $L(\wloc)$ is an immediate consequence of \Cref{thm:grad_dsct_conv}.
    For the second claim, consider the ball $B(\wloc,\rho)$.
    Observe that $B(\wloc,\rho)\subset B(w_0,2R)$, so $L$ is $\mu$-\PL~in $B(\wloc,\rho)$.
    Combining this with $L(\wloc) \leq \frac{\mu \rho^2}{4}$, \Cref{lemma:descent_principle} guarantees the existence of $w_\star\in B(\wloc,\rho)\cap\Wstar$, with 
    $\|\wloc-w_\star\|\leq \sqrt{\frac{2}{\mu}L(\wloc)}$.
    Hence Cauchy-Schwarz yields 
    \begin{align*}
        \|\wloc-w_\star\|_{H_L(w_\star)+\mu I} & \leq \sqrt{\beta_L+\mu}\|\wloc-w_\star\| \leq \sqrt{2\beta_L}\|\wloc-w_\star\|\\
        & \leq 2\sqrt{\frac{\beta_L}{\mu}L(\wloc)} \leq 2 \times \sqrt{\frac{\beta_L}{\mu}\frac{\mu \rho^2}{8{\beta_L}}}\leq \rho,
    \end{align*}
   which proves the claim.
\end{proof}

% \begin{lemma}[Local quadratic growth and error bound]
%     Let $\wloc$ and $\bar w$ be as in blah, and suppose $B(\bar w,2\epsLoc)\subset B(w_0,R)$. 
%     Then the following statements hold:
%     \begin{enumerate}
%         %\item (Existence of local minimizer) The intersection $B(w_\star,\epsLoc)\cap \Wstar$ is non-empty.
%         \item (Local quadratic growth) $L(w)\geq \frac{\mu}{8}\dist^2(w,\Wstar)\quad \forall w\in B(\bar w,\epsLoc)$.
%         \item (Local error bound) $\|\nabla L(w)\|\geq \frac{\mu}{2}\dist(w,\Wstar)\quad \forall w\in B(\bar w,\epsLoc)$.
%     \end{enumerate}
% \end{lemma}


\subsection{Fast Local Convergence of Damped Newton's Method}
% Let $\delta\in (0,1)$.
% Define $g_k = \nabla L(\tilde w_k)$, $H_k = \nabla^2 L(\tilde w_k)$, $p_k = (H_k+\mu I+\|g_k\|^{\delta}I)^{-1}g_k$, and consider
% the iteration 
% \[\tilde w_{k+1} = \tilde w_k+p_k, \quad \tilde w_0 = \wloc.\]

% \begin{lemma}
%     For all $w\in B(w_\star, \epsLoc)$, it holds that
%     \[
%     \lambda_{\textup{min}}(H(w))\geq -\frac{\varepsilon\mu}{2}.
%     \]
%     Consequently, 
%     \[
%     (H(w)+\mu I+\|g(w)\|^{\delta})\succ 0, \quad \forall w\in B(w_\star,\epsLoc).
%     \]
% \end{lemma}

% \begin{lemma}
%     Suppose $\tilde w_k \in B(w_\star,\epsLoc/2)$. Then
%     \[
%     \|d_k\| \leq \lambda ~\dist(\tilde w_k,\Wstar),
%     \]
%     where $\lambda = \left(\max\{1,\varepsilon/(2-\varepsilon)\}+\frac{L_{\HL}}{\mu^{\delta}}\right).$
% \end{lemma}

% \begin{proof}
%     \begin{align*}
%         \|d_k\| & = \left\|(H_k+\mu I+\|g_k\|^{\delta})^{-1}g_k\right\| \\
%         &= \left\|(H_k+\mu I+\|g_k\|^{\delta})^{-1}\left[g_k-g(\bar w_k)+H_k(w_k-\bar w_k)-H_k(w_k-\bar w_k)\right]\right\|\\
%         &\leq \left\|(H_k+\mu I+\|g_k\|^{\delta})^{-1}\left[g_k-g(\bar w_k)+H_k(w_k-\bar w_k)\right]\right\|+\left\|(H_k+\mu I+\|g_k\|^{\delta})^{-1}H_k(w_k-\bar w_k)\right\|.
%     \end{align*}
%     For term $T_1$, observe that
%     \begin{align*}
%         &\left\|(H_k+\mu I+\|g_k\|^{\delta})^{-1}\left[g_k-g(\bar w_k)+H_k(w_k-\bar w_k)\right]\right\| \leq \|g_k\|^{-\delta}\frac{L_{\HL}}{2}\|w_k-\bar w_k\|^2\\
%         &= \|g_k\|^{-\delta}\frac{L_{\HL}}{2}\dist^2(w_k,\Wstar) \leq (2/\mu)^{\delta}\frac{L_{\HL}}{2}\dist^{2-\delta}(w_k,\Wstar) \leq \frac{L_{\HL}}{\mu^{\delta}}\dist(w_k,\Wstar).
%     \end{align*}
%     For term $T_2$, note that
%     \[
%     \left\|(H_k+\mu I+\|g_k\|^{\delta})^{-1}H_k\right\|
%     \]
% \end{proof}

% \begin{lemma}[Staying in the ball]
%     Suppose that $\tilde w_k \in B(\bar w,\epsLoc/(1+\lambda))$, then
%     \[
%     \tilde w_k+\eta p_k \in B(\bar w, \epsLoc),\quad \forall \eta \in [0,1]. 
%     \]
% \end{lemma}
% \begin{proof}
%     \begin{align*}
%         \|\tilde w_k+\eta p_k-\bar w\| &\leq \|\tilde w_k-\bar w\|+\|p_k\|\leq \|\tilde w_k-\bar w\|+\lambda \dist(\tilde w_k,W_\star) = (1+\lambda)\|\tilde w_k-\bar w\|
%         \\ &\leq \epsLoc.
%     \end{align*}
% \end{proof}

% \begin{lemma}[Sufficient descent]
    
% \end{lemma}
In this section, we show damped Newton's method with fixed stepsize exhibits fast linear convergence in an appropriate region about the minimizer $w_\star$ from \cref{corr:close_to_min}. 
Fix $\varepsilon \in (0,1)$, then the region of local convergence is given by:
\[
\Neps = \left\{w\in \R^p: \|w-w_\star\|_{H_L(w_\star)+\mu I}\leq \epsLoc\right\},
\]
where $\epsLoc = \frac{\varepsilon \mu^{3/2}}{4M}$ as above. 
Note that $\wloc \in \Neps$.

We now prove several lemmas, that are essential to the argument. 
We begin with the following elementary technical result, which shall be used repeatedly below.  
\begin{lemma}[Sandwich lemma]
\label{lemma:sandwich}
    Let $A$ be a symmetric matrix and $B$ be a symmetric positive-definite matrix.
    Suppose that $A$ and $B$ satisfy $\|A-B\|\leq \varepsilon \lambda_{\textup{min}}(B)$ where $\varepsilon \in (0,1)$.
    Then
    \[
    (1-\varepsilon)B\preceq A\preceq (1+\varepsilon) B. 
    \]
\end{lemma}
\begin{proof}
    By hypothesis, it holds that
    \[
    -\varepsilon \lambda_{\textup{min}}(B)I\preceq A-B \preceq \varepsilon\lambda_{\textup{min}}(B) I.
    \]
    So using $B\succeq \lambda_{\textup{min}}(B) I$, and adding $B$ to both sides, we reach
    \[
    (1-\varepsilon)B \preceq A\preceq (1+\varepsilon) B.
    \]
\end{proof}


%Define $P = \HL(\wloc)+\lambda I$, the following lemma shows $P$ is positive definite. 
The next result describes the behavior of the damped Hessian in $\Neps$.
\begin{lemma}[Damped Hessian in $\Neps$]
\label{lemma:local_hess}
Suppose that $\gamma \geq \mu$ and $\varepsilon\in (0,1)$. 
\begin{enumerate}
    \item (Positive-definiteness of damped Hessian in $\Neps$) For any $w\in \Neps$, 
    \[
    \HL(w)+\gamma I \succeq \left(1-\frac{\varepsilon}{4}\right)\gamma I.
    \]
    \item (Damped Hessians stay close in $\Neps$)
    For any $w,w' \in \Neps$,
    \[
    (1-\varepsilon)\left[\HL(w)+\gamma I\right] \preceq \HL(w')+\gamma I \preceq (1+\varepsilon) \left[\HL(w)+\gamma I\right].
    \]
\end{enumerate}
% Then it holds that 
% \[
% P\succ \frac{\mu}{2} I.
% \]
\end{lemma}
\begin{proof}
    We begin by observing that the damped Hessian at $w_\star$ satisfies
    \begin{align*}
        \HL(w_\star)+\gamma I & = G(w_\star)+\gamma I+\frac{1}{n}\sum_{i=1}^{n}\left[\F(w_\star)-y\right]_{i}H_{\mathcal F_i}(w_\star)\\
        &= G(w_\star)+\gamma I \succeq \gamma I.
    \end{align*}
    Thus, $\HL(w_\star)+\gamma I$ is positive definite. 
    Now, for any $w\in \Neps$, it follows from Lipschitzness of $\HL$ that
    \begin{align*}
        \left\|\left(\HL(w)+\gamma I\right)-\left(\HL(w_\star)+\gamma I\right)\right\|\leq \Lc_{\HL}\|w-w_\star\|\leq \frac{\Lc_{\HL}}{\sqrt{\gamma}}\|w-w_\star\|_{\HL(w_\star)+\gamma I} \leq \frac{\varepsilon \mu}{4}.
    \end{align*}
    As $\lamMin\left(\HL(w_\star)+\gamma I\right)\geq \gamma>\mu$, we may invoke \Cref{lemma:sandwich} to reach 
    \[
        \left(1-\frac{\varepsilon}{4}\right)\left[\HL(w_\star)+\gamma I\right]\preceq \HL(w)+\gamma I \preceq \left(1+\frac{\varepsilon}{4}\right)\left[\HL(w_\star)+\gamma I\right].
    \]
    This immediately yields 
    \[
    \lamMin\left(\HL(w)+\gamma I\right)\geq \left(1-\frac{\varepsilon}{4}\right)\gamma \geq \frac{3}{4}\gamma,
    \]
    which proves item 1. 
    To see the second claim, observe for any $w,w'\in \Neps$ the triangle inequality implies
    \[
    \left\|\left(\HL(w')+\gamma I\right)-\left(\HL(w)+\gamma I\right)\right\|\leq \frac{\varepsilon \mu}{2} \leq \frac{2}{3}\varepsilon\left(\frac{3}{4}\gamma\right).
    \]
    As $\lamMin\left(\HL(w)+\gamma I\right)\geq \frac{3}{4}\gamma $, it follows from \Cref{lemma:sandwich} that
    \[
    \left(1-\frac{2}{3}\varepsilon\right)\left[\HL(w)+\gamma I\right]\preceq \HL(w')+\gamma I \preceq \left(1+\frac{2}{3}\varepsilon\right)\left[\HL(w)+\gamma I\right],
    \]
    which establishes item 2. 
\end{proof}

The next result characterizes the behavior of the tangent kernel and Gauss-Newton matrix in $\Neps$.
\begin{lemma}[Tangent kernel and Gauss-Newton matrix in $\Neps$]
\label{lemma:local_gn}
    Let $\gamma \geq \mu$. Then for any $w,w'\in \Neps$, the following statements hold:
    \begin{enumerate}
        %\item $(1-\varepsilon)\left[G(\wloc)+\gamma I\right] \preceq G(w)+\gamma I\preceq (1+\varepsilon)\left[G(\wloc)+\gamma I\right] $.
        \item (Tangent kernels stay close) 
        \[
        \left(1-\frac{\varepsilon}{2}\right)K_\F(w_\star)\preceq K_\F(w) \preceq \left(1+\frac{\varepsilon}{2}\right) K_\F(w_\star)
        \]
        \item (Gauss-Newton matrices stay close)
        \[
         \left(1-\frac{\varepsilon}{2}\right)\left[G(w)+\gamma I\right]\preceq G(w_\star)+\gamma I \preceq \left(1+\frac{\varepsilon}{2}\right) \left[G(w)
        +\gamma I\right]\]
        \item (Damped Hessian is close to damped Gauss-Newton matrix) 
        \[
        (1-\varepsilon)\left[G(w)+\gamma I\right] \preceq \HL(w)+\gamma I \preceq (1+\varepsilon)\left[G(w)+\gamma I\right].
        \]
        \item (Jacobian has full row-rank) The Jacobian satisfies $\textup{rank}(J_{\F}(w)) = n$.
    \end{enumerate}
\end{lemma}
\begin{proof}
\begin{enumerate}
    \item Observe that
    \begin{align*}
        \|K_\F(w)-K_\F(w_\star)\| &= \|J_{\F}(w)J_{\F}(w)^{T}-J_{\F}(w_\star)J_{\F}(w_\star)^{T}\| \\
                  &= \left\|\left[J_{\F}(w)-J_{\F}(w_\star)\right]J_{\F}(w)^{T}+J_{\F}(w_\star)\left[J_{\F}(w)-J_{\F}(w_\star)\right]^{T}\right\| \\
                  &\leq 2 \Lc_\F \Lc_{J_\F}\|w-w_\star\| \leq \frac{2 \Lc_\F \Lc_{J_\F}}{\sqrt{\gamma}}\|w-w_\star\|_{H_L(w_\star)+\gamma I} \leq \frac{\varepsilon\mu^{3/2}}{\sqrt{\gamma}} \leq \frac{\varepsilon}{2} \mu,
    \end{align*}
    where in the first inequality we applied the fundamental theorem of calculus to reach 
    \[
    \|J_{\F}(w)-J_{\F}(w_\star)\|\leq \Lc_{J_{\F}}\|w-w_\star\|.
    \]
    Hence the claim follows from \Cref{lemma:sandwich}.
    \item By an analogous argument to item 1, we find
    \[
    \left\|\left(G(w)+\gamma I\right)-\left(G(w_\star)+\gamma I\right)\right\| \leq \frac{\varepsilon}{2}\mu,
    \]
    so the result again follows from \Cref{lemma:sandwich}.
    \item First observe $\HL(w_\star)+\gamma I = G(w_\star)+\gamma I$. Hence the proof of \Cref{lemma:local_hess} implies,
    \[
    \left(1-\frac{\varepsilon}{4}\right)\left[G(w_\star)
        +\gamma I\right]\preceq \HL(w)+\gamma I\preceq \left(1+\frac{\varepsilon}{4}\right)\left[G(w_\star)
        +\gamma I\right].
    \]
    Hence the claim now follows from combining the last display with item 2. 
    \item This last claim follows immediately from item 1, as for any $w\in \Neps$,
    \[
    \sigma_{n}\left(J_{\F}(w)\right) = \sqrt{\lamMin(K_\F(w))}\geq \sqrt{\left(1-\frac{\varepsilon}{2}\right)\mu}>0.
    \]   
    Here the last inequality uses $\lamMin(K_\F(w_\star))\geq \mu$, which follows as $w_\star\in B(w_0,2R)$.
\end{enumerate}

\end{proof}

The next lemma is essential to proving convergence. It shows in $\Neps$ that $L(w)$ is uniformly smooth with respect to the damped Hessian, with nice smoothness constant $(1+\varepsilon)$. 
Moreover, it establishes that the loss is uniformly \PL with respect to the damped Hessian in $\Neps$. 
\begin{lemma}[Preconditioned smoothness and \PL]
\label{lemma:local-sm-pl}
    Suppose $\gamma \geq \mu$. Then 
    for any $w,w',w''\in \Neps$, the following statements hold:
    \begin{enumerate}
        \item $L(w'')\leq L(w')+\langle \nabla L(w'),w''-w'\rangle +\frac{1+\varepsilon}{2}\|w''-w'\|_{H_L(w)+\gamma I}^2$.
        \item $\frac{\|\nabla L(w)\|_{(\HL(w)+\gamma I)^{-1}}^2}{2}\geq \frac{1}{1+\varepsilon}\frac{1}{\left(1+\gamma/\mu\right)}L(w)$.
    \end{enumerate}
\end{lemma}

\begin{proof}
    \begin{enumerate}
        \item By Taylor's theorem
        \begin{align*}
        L(w'') = L(w')+\langle \nabla L(w'),w''-w'\rangle+\int_{0}^{1}(1-t)\|w''-w'\|_{\HL(w'+t(w''-w'))}^2 dt
        \end{align*}
        Note $w'+t(w''-w')\in \Neps $ as $\Neps$ is convex.
        Thus we have,
        \begin{align*}
            L(w'') & \leq L(w')+\langle \nabla L(w'),w''-w'\rangle+\int_{0}^{1}(1-t)\|w''-w'\|_{\HL(w'+t(w''-w'))+\gamma I}^2dt \\
            & \leq L(w')+\langle \nabla L(w'),w''-w'\rangle+\int_{0}^{1}(1-t)(1+\varepsilon)\|w''-w'\|_{\HL(w)+\gamma I}^2dt \\
            & = L(w')+\langle \nabla L(w'),w''-w'\rangle+\frac{(1+\varepsilon)}{2}\|w''-w'\|_{\HL(w)+\gamma I}^2.  
        \end{align*}
    
        \item Observe that
        \begin{align*}
            \frac{\|\nabla L(w)\|_{(\HL(w)+\gamma I)^{-1}}^2}{2} = \frac{1}{2}(\F(w)-y)^{T}\left[J_{\F}(w)\left(\HL(w)+\gamma I\right)^{-1}J_{\F}(w)^{T}\right](\F(w)-y).
        \end{align*}
        Now,
        \begin{align*}
            %J_{\F}(w)P^{-1}J_{\F}(w)^{T} 
            J_{\F}(w)\left(\HL(w)+\gamma I\right)^{-1}J_{\F}(w)^{T} & \succeq \frac{1}{(1+\varepsilon)}J_{\F}(w)\left(G(w)+\gamma I\right)^{-1}J_{\F}(w)^{T}\\ 
             &= \frac{1}{(1+\varepsilon)}J_{\F}(w)\left(J_{\F}(w)^{T}J_{\F}(w)+\gamma I\right)^{-1}J_{\F}(w)^{T}\\ 
        \end{align*}
        \Cref{lemma:local_gn} guarantees $J_{\F}(w)$ has full row-rank, so the SVD yields
        \[
        J_{\F}(w)\left(J_{\F}(w)^{T}J_{\F}(w)+\gamma I\right)^{-1}J_{\F}(w)^{T} = U\Sigma^2(\Sigma^2+\gamma I)^{-1}U^{T}\succeq \frac{\mu}{\mu+\gamma} I.
        \]
        Hence
        \[
          \frac{\|\nabla L(w)\|_{(\HL(w)+\gamma I)^{-1}}^2}{2}\geq \frac{\mu}{(1+\varepsilon)(\mu+\gamma)}\frac{1}{2}\|\F(w)-y\|^2 = \frac{\mu}{(1+\varepsilon)(\mu+\gamma)}L(w).
        \]
    \end{enumerate}
\end{proof}

% \begin{lemma}
%     Let $w_k \in \Neps$. Then 
%     \[
%     \|p_k\|_{P}\leq
%     \]
% \end{lemma}
% \begin{proof}
%     \begin{align*}
%         \|p_k\|_{P} & = \|\nabla L(w_k)\|_{P^{-1}}\leq \|\nabla L(w_k)-\nabla L(\bar w_k)-\nabla^2L(\bar w_k)(w_k-\bar w_k)\|_{P^{-1}}+\|\nabla^2L(\bar w_k)(w_k-\bar w_k)\|_{P^{-1}} \\
%         &= \|\nabla L(w_k)-\nabla L(\bar w_k)-\nabla^2L(\bar w_k)(w_k-\bar w_k)\|_{P^{-1}}+\|\nabla^2L(\bar w_k)^{1/2}(w_k-\bar w_k)\|_{\nabla^2L(\bar w_k)^{1/2}P^{-1}\nabla^2L(\bar w_k)^{1/2}} \\
%         &\leq \frac{1}{\sqrt{1-\varepsilon}}\|\nabla L(w_k)-\nabla L(\bar w_k)-\nabla^2L(\bar w_k)(w_k-\bar w_k)\|_{(\nabla^2 L(\bar w_k)+\rho I)^{-1}}\\
%         &+\|\nabla^2L(\bar w_k)^{1/2}(w_k-\bar w_k)\|_{\nabla^2L(\bar w_k)^{1/2}P^{-1}\nabla^2L(\bar w_k)^{1/2}}. 
%     \end{align*}
%     Now,
%     \begin{align*}
%         & \|\nabla L(w_k)-\nabla L(\bar w_k)-\nabla^2L(\bar w_k)(w_k-\bar w_k)\|_{(\nabla^2 L(\bar w_k)+\rho I)^{-1}} = \\
%         & \left\|\int_{0}^{1}\left[\nabla^2 L(\bar w_k+t(w_k-\bar w_k))-\nabla^2L(\bar w_k)\right](w_k-\bar w_k)\right\|_{(\nabla^2 L(\bar w_k)+\rho I)^{-1}}
%     \end{align*}
% \end{proof}
\begin{lemma}[Local preconditioned-descent]
\label{lemma:local_descent}
    Run Phase II of \cref{alg-GDND} with $\eta_{\textup{DN}} = (1+\varepsilon)^{-1}$ and $\gamma = \mu$. 
    Suppose that $\tilde w_{k}, \tilde w_{k+1}\in \Neps$, then
    \[
     L(\tilde w_{k+1})\leq \left(1-\frac{1}{2(1+\varepsilon)^2}\right)L(\tilde w_k).
    \]
\end{lemma}
\begin{proof}
    As $\tilde w_k, \tilde w_{k+1}\in \Neps$, item 1 of \Cref{lemma:local-sm-pl} yields
    \[
    L(\tilde w_{k+1})\leq L(\tilde w_k)-\frac{\|\nabla L(\tilde w_k)\|^2_{(\HL(\tilde w_k)+\mu I)^{-1}}}{2(1+\varepsilon)}.
    \]
    Combining the last display with the preconditioned \PL condition, 
    we conclude
    \[
    L(\tilde w_{k+1})\leq \left(1-\frac{1}{2(1+\varepsilon)^2}\right)L(\tilde w_k).
    \]
\end{proof}

The following lemma describes how far an iterate moves after one-step of Phase II of \cref{alg-GDND}.
\begin{lemma}[1-step evolution]
\label{lemma:one_step_evol}
    Run Phase II of \cref{alg-GDND} with $\eta_{\textup{DN}} = (1+\varepsilon)^{-1}$ and $\gamma \geq \mu$.
    Suppose $\tilde w_k \in \N_{\frac{\varepsilon}{3}}(w_\star)$, then $\tilde w_{k+1}\in \Neps$.
\end{lemma}

\begin{proof}
    Let $P = H_L(\tilde w_k)+\gamma I$.
    We begin by observing that 
    \begin{align*}
        \|\tilde w_{k+1}-w_\star\|_{\HL(w_\star)+\mu I}\leq \sqrt{1+\varepsilon}\|\tilde w_{k+1}-w_\star\|_{P}.
    \end{align*}
    Now,
    \begin{align*}
        \|\tilde w_{k+1}-w_\star\|_P & = \frac{1}{1+\varepsilon}\|\nabla L(\tilde w_{k})-\nabla L(w_\star)-(1+\varepsilon)P(w_\star-\tilde w_k)\|_{P^{-1}} \\ &=
        \frac{1}{1+\varepsilon}\left\|\int_{0}^{1}\left[\nabla^2 L(w_\star+t(w_k-w_\star))-(1+\varepsilon)P\right]dt(w_\star-\tilde w_k)\right\|_{P^{-1}} \\
        & =   \frac{1}{1+\varepsilon}\left\|\int_{0}^{1}\left[P^{-1/2}\nabla^2 L(w_\star+t(w_k-w_\star))P^{-1/2}-(1+\varepsilon)I\right]dtP^{1/2}(w_\star-\tilde w_k)\right\|\\
        &\leq \frac{1}{1+\varepsilon}\int_{0}^{1}\left\|P^{-1/2}\nabla^2 L(w_\star+t(w_k-w_\star))P^{-1/2}-(1+\varepsilon)I\right\|dt\|\tilde w_k-w_\star\|_{P}
    \end{align*}
    We now analyze the matrix $P^{-1/2}\nabla^2 L(w_\star+t(w_k-w_\star))P^{-1/2}$. 
    Observe that
    \begin{align*}
        & P^{-1/2}\nabla^2 L(w_\star+t(w_k-w_\star))P^{-1/2} = P^{-1/2}(\nabla^2 L(w_\star+t(w_k-w_\star))+\gamma I-\gamma I)P^{-1/2} \\
        & = P^{-1/2}(\nabla^2 L(w_\star+t(w_k-w_\star))+\gamma I)P^{-1/2}-\gamma P^{-1} \succeq (1-\varepsilon)I-\gamma P^{-1} \succeq -\varepsilon I.
        %&= I-\rho P^{-1}+P^{-1/2}EP^{-1/2}\succeq -\|E\|P^{-1} \succeq -(\varepsilon \mu)P^{-1}\succeq -\varepsilon I.
    \end{align*}
    Moreover,
    \[
    P^{-1/2}\nabla^2 L(w_\star+t(w_k-w_\star))P^{-1/2}\preceq P^{-1/2}(\nabla^2 L(w_\star+t(w_k-w_\star))+\gamma I)P^{-1/2}\preceq (1+\varepsilon)I.
    \]
    Hence, 
    \[0 \preceq (1+\varepsilon)I-P^{-1/2}\nabla^2 L(w_\star+t(w_k-w_\star))P^{-1/2}\preceq (1+2\varepsilon)I,\] 
    and so
    \[
    \|\tilde w^{k+1}-w_\star\|_P\leq \frac{1+2\varepsilon}{1+\varepsilon}\|\tilde w_k-w_\star\|_{P}.
    \]
    Thus,
    \[
    \|\tilde w^{k+1}-w_\star\|_{\HL(w_\star)+\mu I}\leq \frac{1+2\varepsilon}{\sqrt{1+\varepsilon}}\|\tilde w_k-w_\star\|_P \leq(1+2\varepsilon)\|\tilde w_k-w_\star\|_{\HL(w_\star)+\mu I}\leq \epsLoc.
    \]
\end{proof}

The following lemma is key to establishing fast local convergence; it shows that the iterates produced by damped Newton's method remain in $\Neps$, the region of local convergence. 
\begin{lemma}[Staying in $\Neps$]
\label{lemma:trapped}
    Suppose that $\wloc \in \mathcal N_{\rho}(w_\star)$, where $\rho = \frac{\epsLoc}{19\sqrt{\beta_L/\mu}}$.
    Run Phase II of \cref{alg-GDND} with $\gamma = \mu$ and $\eta = (1+\varepsilon)^{-1}$, then $\tilde w_{k+1} \in \Neps$ for all $k\geq 1$.
\end{lemma}
\begin{proof}
  In the argument that follows $\kappa_P = 2(1+\varepsilon)^2$.
  The proof is via induction. 
  Observe that if $\wloc \in \mathcal N_{\varrho}(w_\star)$ then by \Cref{lemma:one_step_evol}, $\tilde w_1 \in \Neps$.  
  %\[
  %\|\tilde w_1-w_\star\|_{\HL(w_\star)+\mu I}\leq 3\|\wloc-w_\star\|_{\HL(w_\star)+\mu I} \leq \epsLoc,
  %\]
  Now assume $\tilde w_j \in \Neps$ for $j = 2,\dots, k$. 
  We shall show $\tilde w_{k+1}\in \Neps$.
  To this end, observe that
  \begin{align*}
  \|\tilde w_{k+1}-w_\star\|_{\HL(w_\star)+\mu I} & \leq \|\wloc-w_\star\|_{\HL(w_\star)+\mu I}+\frac{1}{1+\varepsilon}\sum_{j=1}^{k}\|\nabla L(w_j)\|_{\left(\HL(w_\star)+\mu I\right)^{-1}} \\
  % &\leq \varrho+\sqrt{\frac{2}{1+\varepsilon}}\sum_{j=1}^{k}\sqrt{L(\tilde w_{j})} \leq \varrho+\sqrt{\frac{2}{1+\varepsilon}}\sum_{j=1}^{k}\left(1-\frac{\mu_P}{L_P}\right)^{t/2}\sqrt{L(\tilde w_0)} \\
  % &\leq \varrho+\|\wloc - w_\star\|_{\HL(w_\star)+\mu I}\sum_{j=1}^{k}\left(1-\frac{\mu_P}{L_P}\right)^{t/2}\leq \left(1+\sum_{k=0}^{\infty}\left(1-\frac{\mu_P}{L_P}\right)^{t/2}\right)\varrho\\
  % & = \left(1+\frac{1}{{1-\sqrt{1-\frac{\mu_P}{L_P}}}}\right)\varrho\leq \epsLoc.
  \end{align*}
  Now,
  \begin{align*}
      \|\nabla L(w_j)\|_{\left(\HL(w_\star)+\mu I\right)^{-1}} &\leq \frac{1}{\sqrt{\mu}}\|\nabla L(w_j)\|_2 \leq \sqrt{\frac{2\beta_L}{\mu}L(w_j) }\\
      &\leq \sqrt{\frac{2\beta_L}{\mu}}\left(1-\frac{1}{\kappa_P}\right)^{j/2}\sqrt{L(\wloc)},
  \end{align*}
  Here the second inequality follows from $\|\nabla L(w)\| \leq \sqrt{2\beta_L L(w)}$, and the last inequality follows from \Cref{lemma:local_descent}, which is applicable as $\tilde w_{0},\dots,\tilde w_k \in \Neps$. 
  Thus,
  \begin{align*}
  \|\tilde w_{k+1}-w_\star\|_{\HL(w_\star)+\mu I} &\leq \rho+\sqrt{\frac{2\beta_L}{\mu}}\sum_{j=1}^{k}\left(1-\frac{1}{\kappa_P}\right)^{j/2}\sqrt{L(\tilde w_0)} \\
  &\leq \rho+\sqrt{\frac{(1+\varepsilon)\beta_L}{2\mu}}\|\wloc - w_\star\|_{\HL(w_\star)+\mu I}\sum_{j=1}^{k}\left(1-\frac{1}{\kappa_P}\right)^{j/2}\\
  &\leq \left(1+\sqrt{\frac{\beta_L}{\mu}}\sum_{j=0}^{\infty}\left(1-\frac{1}{\kappa_P}\right)^{j/2}\right)\rho\\
  & = \left(1+\frac{\sqrt{\beta_L/\mu}}{{1-\sqrt{1-\frac{1}{\kappa_P}}}}\right)\rho\leq \epsLoc.
  \end{align*}
  Here, in the second inequality we have used $L(\tilde w_0)\leq 2(1+\varepsilon)\|\wloc - w_\star\|^2_{\HL(w_\star)+\mu I}$, which is an immediate consequence of \cref{lemma:local-sm-pl}.
  Hence, $\tilde w_{k+1}\in \Neps$, and the desired claim follows by induction. 
\end{proof}

\begin{theorem}[Fast-local convergence of Damped Newton]
\label{thm:dn_fast_loc}
    Let $\wloc$ be as in \cref{corr:close_to_min}. 
    Consider the iteration 
    \[
    \tilde w_{k+1} = \tilde w_k-\frac{1}{1+\varepsilon}(\HL(\tilde w_k)+\mu I)^{-1}\nabla L(\tilde w_k),\quad \text{where}~\tilde w_0 = \wloc.\] 
    Then, after $k$ iterations, the loss satisfies
        \[
        L(\tilde w_k) \leq \left(1-\frac{1}{2(1+\varepsilon)^2}\right)^{k}L(\wloc).
        \]
        Thus after $k = \mathcal O\left(\log\left(\frac{1}{\epsilon}\right)\right)$ iterations
        \[
        L(\tilde w_k)\leq \epsilon.
        \]
    \begin{proof}
        \cref{lemma:trapped} ensure that $\tilde w^{k} \in \Neps$ for all $k$.
         Thus, we can apply item $1$ of \Cref{lemma:local-sm-pl} and the definition of $\tilde w^{k+1}$, to reach
         \[
         L(\tilde w_{k+1})\leq L(\tilde w_{k})-\frac{1}{2(1+\varepsilon)}\|\nabla L(\tilde w_k)\|_{P^{-1}}^2. 
         \]
         Now, using item $2$ of \Cref{lemma:local-sm-pl} and recursing yields  
         \[
         L(\tilde w_{k+1})\leq \left(1-\frac{1}{2(1+\varepsilon)^2}\right)L(\tilde w_k)\leq \left(1-\frac{1}{2(1+\varepsilon)^2}\right)^{k+1}L(\wloc).
         \]
         The remaining portion of the theorem now follows via a routine calculation.
    \end{proof}
\end{theorem}

% \subsection{Fast local-convergence of Gauss-Newton with Levenberg-Marquardt regularization}
\subsection{Formal Convergence of \cref{alg-GDND}}
\label{subsec:GDND_conv}
Here, we state and prove the formal convergence result for \cref{alg-GDND}.
\begin{theorem}
\label{thm:GDND}
    Suppose that \cref{assp:interpolation} and \cref{assp:loss_reg} hold, and that the loss is $\mu$-\PL in $B(w_0,2R)$, where $R = \frac{2\sqrt{2\beta_L L(w_0)}}{\mu}$.
    Let $\epsLoc$ and $\rho$ be as in \cref{corr:close_to_min}, and set $\varepsilon = 1/6$ in the definition of $\epsLoc$. 
    Run \cref{alg-GDND} with parameters: $\eta_{\textup{GD}} = 1/\beta_L, K_{\textup{GD}} = \frac{\beta_L}{\mu}\log\left(\frac{4\max\{2\beta_{L},1\}L(w_0)}{\mu\rho^2}\right), \eta_{\textup{DN}} = 5/6, \gamma = \mu$ and $K_{\textup{DN}}\geq 1$.
    Then Phase II of \cref{alg-GDND} satisfies
    \[
    L(\tilde w_{k})\leq \left(\frac{2}{3}\right)^{k}L(w_{K_{\textup{GD}}}).
    \]
    Hence after $K_{\textup{DN}} \geq 3\log\left(\frac{L(w_{K_{\textup{GD}}})}{\epsilon}\right)$ iterations, Phase II of \cref{alg-GDND} outputs a point satisfying
    \[
     L(\tilde w_{K_{\textup{DN}}})\leq \epsilon.
    \]
\end{theorem}

\begin{proof}
    By assumption the conditions of \cref{corr:close_to_min} are met, therefore $w_{K_{\textup{GD}}}$ satisfies $\|w_{K_{\textup{GD}}}-w_\star\|_{H_L(w_\star)+\mu I}\leq \rho$, for some $w_\star \in \Wstar$.
    Hence, we may invoke \cref{thm:dn_fast_loc} to conclude the desired result. 
\end{proof}

\section{Additional Experimental Results}
\label{appendix:exp}

\subsection{Categorizing Model Responses Across Problem Variations}
\label{appendix:category}
Recall that for each problem, we have a \SAME modification which can be solved using the same method as the original problem, and a \HARD modification which requires more difficult problem-solving skills. Therefore, there are 8 possible cases regarding the correctness of the model's responses to the three problems. Modulo the fluctuations of the model's correctness among the \SAME variations, we can summarize the model's responses into the following 4 cases:
\begin{itemize}[itemsep=1pt, parsep=1pt, topsep=1pt]
    \item \textbf{Case I}: at least one of the original problem and the \SAME modification is solved \textit{correctly}, and the \HARD modification is also solved \textit{correctly}.
    \item \textbf{Case II}: both the original problem and the \SAME modification are solved \textit{incorrectly}, and the \HARD modification is also solved \textit{incorrectly}.
    \item \textbf{Case III}: both the original problem and the \SAME modification are solved \textit{incorrectly}, but the \HARD modification is solved \textit{correctly}.
    \item \textbf{Case IV}: at least one of the original problem and the \SAME modification is solved \textit{correctly}, but the \HARD modification is solved \textit{incorrectly}.
\end{itemize}
For each of the models, we calculate the percentage of the responses in \cref{tab:cate}. As expected, stronger models have a higher percentage of Case I responses and a lower percentage of Case II responses. Interestingly, the percentages of Case III responses are small (less than 10\%) but non-zero, where the models cannot solve the easier variants but can solve the hard variant correctly. After manual inspection, we found that this is due to the misalignment between the models' capabilities and the annotators' perception of the difficulties of math problems. 


\begin{table*}[t]
\caption{Number and percentage of the models' responses that belong to each of the four categories.}
\centering
\resizebox{0.9\textwidth}{!}{
\begin{tabular}{lccccccc}
 \toprule
 Model &  Case I &  Case II & Case III & Case IV &\\ \midrule
Gemini-2.0-flash-thinking-exp &  212 (75.99 \%) & 5 (1.79 \%) & 6 (2.15 \%) & 56 (20.07 \%) &  \\ 
o1-preview &  194 (69.53 \%) & 10 (3.58 \%) & 8 (2.87 \%) & 67 (24.01 \%) &  \\ 
o1-mini &  218 (78.14 \%) & 4 (1.43 \%) & 1 (0.36 \%) & 56 (20.07 \%) &  \\ 
\midrule
Gemini-2.0-flash-exp &  176 (63.08 \%) & 11 (3.94 \%) & 11 (3.94 \%) & 81 (29.03 \%) &  \\ 
Gemini-1.5-pro &  145 (51.97 \%) & 28 (10.04 \%) & 13 (4.66 \%) & 93 (33.33 \%) &  \\ 
GPT-4o &  94 (33.69 \%) & 56 (20.07 \%) & 16 (5.73 \%) & 113 (40.50 \%) &  \\ 
GPT-4-turbo &  81 (29.03 \%) & 72 (25.81 \%) & 15 (5.38 \%) & 111 (39.78 \%) &  \\ 
Claude-3.5-Sonnet &  88 (31.54 \%) & 56 (20.07 \%) & 20 (7.17 \%) & 115 (41.22 \%) &  \\ 
Claude-3-Opus &  49 (17.56 \%) & 99 (35.48 \%) & 25 (8.96 \%) & 106 (37.99 \%) &  \\ 
\midrule
Llama-3.1-8B-Instruct &  21 (7.53 \%) & 137 (49.10 \%) & 7 (2.51 \%) & 114 (40.86 \%) &  \\ 
Gemma-2-9b-it &  22 (7.89 \%) & 164 (58.78 \%) & 11 (3.94 \%) & 82 (29.39 \%) &  \\ 
Phi-3.5-mini-instruct &  22 (7.89 \%) & 161 (57.71 \%) & 18 (6.45 \%) & 78 (27.96 \%) &  \\ 
\midrule
Deepseek-math-7b-rl &  25 (8.96 \%) & 138 (49.46 \%) & 13 (4.66 \%) & 103 (36.92 \%) &  \\ 
Qwen2.5-Math-7B-Instruct &  61 (21.86 \%) & 70 (25.09 \%) & 15 (5.38 \%) & 133 (47.67 \%) &  \\ 
Mathstral-7b-v0.1 &  28 (10.04 \%) & 136 (48.75 \%) & 13 (4.66 \%) & 102 (36.56 \%) &  \\ 
NuminaMath-7B-CoT &  39 (13.98 \%) & 118 (42.29 \%) & 9 (3.23 \%) & 113 (40.50 \%) &  \\ 
MetaMath-13B-V1.0 &  6 (2.15 \%) & 199 (71.33 \%) & 10 (3.58 \%) & 64 (22.94 \%) &  \\ 
MAmmoTH2-8B &  9 (3.23 \%) & 201 (72.04 \%) & 12 (4.30 \%) & 57 (20.43 \%) &  \\
\bottomrule
\end{tabular}
}
\label{tab:cate}
\end{table*}


\subsection{Is Mode Collapse a Problem?}
\label{appendix:naive:memorization}

We provide \cref{tab:naive_memorization} to support \cref{sec:naive:memorization}.

\begin{table*}[htbp]
\caption{The number of errors with answers that match the corresponding original answers. The edit distances are normalized by the length of the responses to the original problems.}
\centering
\resizebox{\textwidth}{!}{
\begin{tabular}{lcccccccccccc}
 \toprule
\multirow{3}{*}{\textbf{Model}}   & \multicolumn{6}{c}{\textbf{\SAME}} & \multicolumn{6}{c}{\textbf{\HARD}}   \\ 
\cmidrule(r){2-7}  \cmidrule(r){8-13}
&  \multicolumn{3}{c}{Num. Errors} & \multicolumn{3}{c}{Normalized Edit Distance} & \multicolumn{3}{c}{Num. Errors} & \multicolumn{3}{c}{Normalized Edit Distance}  \\
\cmidrule(r){2-4} \cmidrule(r){5-7} \cmidrule(r){8-10} \cmidrule(r){11-13}
& $n_{\text{same}}$ & $n_{\text{total}}$ & percentage & min. & avg. & max. & $n_{\text{same}}$ & $n_{\text{total}}$ & percentage & min. & avg. & max. \\
\midrule
 
Gemini-2.0-flash-thinking-exp & 2 & 25 & 8.00 & 0.553 & 0.611 & 0.668  & 10 & 61 & 16.39  &  0.508 & 0.679 & 0.976  \\ 
o1-preview & 1 & 34 & 2.94 & 0.652 & 0.652 & 0.652  & 5 & 77 & 6.49  &  0.729 & 1.07 & 1.89  \\ 
o1-mini & 0 & 14 & 0 & N/A & N/A & N/A  & 9 & 60 & 15.00  &  0.559 & 14.7 & 126.0  \\ 
\midrule
Gemini-2.0-flash-exp & 4 & 48 & 8.33 & 0.644 & 0.82 & 1.09  & 13 & 92 & 14.13  &  0.546 & 1.1 & 1.76  \\ 
Gemini-1.5-pro & 5 & 63 & 7.94 & 0.472 & 0.751 & 1.3  & 11 & 121 & 9.09  &  0.257 & 0.866 & 1.58  \\ 
GPT-4o & 4 & 106 & 3.77 & 0.709 & 0.773 & 0.937  & 14 & 169 & 8.28  &  0.489 & 0.777 & 1.2  \\ 
GPT-4-turbo & 5 & 125 & 4.00 & 0.621 & 0.74 & 0.855  & 17 & 183 & 9.29  &  0.636 & 0.932 & 1.61  \\ 
Claude-3.5-Sonnet & 6 & 116 & 5.17 & 0.509 & 0.729 & 0.83  & 13 & 171 & 7.60  &  0.461 & 0.741 & 1.92  \\ 
Claude-3-Opus & 3 & 162 & 1.85 & 0.355 & 0.485 & 0.614  & 15 & 205 & 7.32  &  0.463 & 0.841 & 1.54  \\ 
\midrule
Llama-3.1-8B-Instruct & 13 & 191 & 6.81 & 0.595 & 0.901 & 1.99  & 18 & 251 & 7.17  &  0.618 & 0.946 & 2.7  \\ 
Gemma-2-9b-it & 3 & 202 & 1.49 & 0.361 & 0.506 & 0.716  & 7 & 246 & 2.85  &  0 & 0.662 & 1.08  \\ 
Phi-3.5-mini-instruct & 8 & 199 & 4.02 & 0.427 & 0.61 & 0.832  & 12 & 239 & 5.02  &  0.289 & 0.754 & 1.69  \\ 
\midrule
Deepseek-math-7b-rl & 9 & 186 & 4.84 & 0.189 & 0.423 & 0.676  & 11 & 241 & 4.56  &  0.121 & 1.5 & 4.24  \\ 
Qwen2.5-Math-7B-Instruct & 6 & 135 & 4.44 & 0.376 & 0.591 & 0.813  & 10 & 203 & 4.93  &  0.273 & 1.01 & 4.91  \\ 
Mathstral-7b-v0.1 & 11 & 178 & 6.18 & 0.0989 & 0.645 & 0.964  & 13 & 238 & 5.46  &  0.105 & 0.586 & 0.984  \\ 
NuminaMath-7B-CoT & 12 & 167 & 7.19 & 0.241 & 0.743 & 1.62  & 14 & 231 & 6.06  &  0.204 & 1.04 & 2.22  \\ 
MetaMath-13B-V1.0 & 13 & 258 & 5.04 & 0.27 & 0.55 & 0.748  & 14 & 263 & 5.32  &  0.509 & 0.982 & 2.83  \\ 
MAmmoTH2-8B & 5 & 229 & 2.18 & 0.00214 & 0.666 & 1.25  & 9 & 258 & 3.49  &  0.708 & 0.822 & 1.04  \\

\bottomrule
\end{tabular}
}
\label{tab:naive_memorization}
\end{table*}
















\clearpage
\subsection{The Effect of In-Context Learning}
\label{appendix:ICL}

In \cref{tab:OIC}, we report the performance of in-context learning (ICL) with the corresponding original (unmodified) problem and solution as the in-context learning example. Furthermore, we decompose the influences on \HARD into the \textbf{ICL effect} and the \textbf{misleading effect} in \cref{tab:icl:breakdown} and visualize the influences for representative models in \cref{fig:icl:full}. Please refer to \cref{sec:icl} for the full discussion.


\begin{table*}[htbp]
\caption{Performance comparisons without and with the original problem and solution as the in-context learning example.}
\centering
\resizebox{\textwidth}{!}{
\begin{tabular}{lccccc}
 \toprule
\multirow{2}{*}{\textbf{Model}} & \multirow{2}{*}{\textbf{\Original} (0-shot)}   & \multicolumn{2}{c}{\textbf{\SAME}} & \multicolumn{2}{c}{\textbf{\HARD}}   \\ 
\cmidrule(r){3-4}  \cmidrule(r){5-6}
& & zero-shot & ICL w. original & zero-shot & ICL w. original\\ 
\midrule
Gemini-2.0-flash-thinking-exp & 92.47 & 91.04 & 94.62 & 78.14 & 79.21 \\ 
o1-preview & 87.81 & 87.81 & 91.40 & 72.40 & 74.19 \\ 
o1-mini & 94.27 & 94.98 & 94.98 & 78.49 & 78.49 \\ 
\midrule
Gemini-2.0-flash-exp & 88.17 & 82.80 & 89.96 & 67.03 & 67.38 \\ 
Gemini-1.5-pro & 77.78 & 77.42 & 88.17 & 56.63 & 60.57 \\ 
GPT-4o & 67.03 & 62.01 & 77.06 & 39.43 & 43.01 \\ 
GPT-4-turbo & 56.99 & 55.20 & 69.89 & 34.41 & 39.07 \\ 
Claude-3.5-Sonnet & 64.52 & 58.42 & 83.15 & 38.71 & 49.46 \\ 
Claude-3-Opus & 41.94 & 41.94 & 68.10 & 26.52 & 33.33 \\ 
\midrule
Llama-3.1-8B-Instruct & 36.56 & 31.54 & 36.56 & 10.04 & 10.75 \\ 
Gemma-2-9b-it & 27.60 & 27.60 & 42.65 & 11.83 & 14.34 \\ 
Phi-3.5-mini-instruct & 26.16 & 28.67 & 36.92 & 14.34 & 14.34 \\ 
\midrule
Deepseek-math-7b-rl & 37.28 & 33.33 & 45.52 & 13.62 & 15.41 \\ 
Qwen2.5-Math-7B-Instruct & 58.78 & 51.61 & 56.99 & 27.24 & 26.88 \\ 
Mathstral-7b-v0.1 & 36.56 & 36.20 & 48.39 & 14.70 & 16.49 \\ 
NuminaMath-7B-CoT & 43.73 & 40.14 & 47.31 & 17.20 & 17.20 \\ 
MetaMath-13B-V1.0 & 21.15 & 7.53 & 11.11 & 5.73 & 3.58 \\ 
MAmmoTH2-8B & 12.90 & 17.92 & 31.18 & 7.53 & 5.73 \\ 
\bottomrule
\end{tabular}
}
\label{tab:OIC}
\end{table*}





\begin{table*}[htbp]
\caption{Effects of in-context learning (ICL) with original example on \HARD. The percentages of $n(\text{correct} \to \text{wrong})$ are normalized by the number of errors with ICL, while the percentages of $n(\text{wrong} \to \text{correct})$ are by the number of errors without ICL. }
\centering
\resizebox{\textwidth}{!}{
\begin{tabular}{lcccc}
 \toprule
Model & num. errors (zero-shot) & num. errors (ICL w. original) & $n(\text{correct} \to \text{wrong})$ & $n(\text{wrong} \to \text{correct})$ \\ 
\midrule
Gemini-2.0-flash-thinking-exp & 61 (21.86 \%) & 58 (20.79 \%) & 17 (29.31 \%) & 20 (32.79 \%) \\ 
o1-preview & 77 (27.60 \%) & 72 (25.81 \%) & 21 (29.17 \%) & 26 (33.77 \%) \\ 
o1-mini & 60 (21.51 \%) & 60 (21.51 \%) & 24 (40.00 \%) & 24 (40.00 \%) \\ 
\midrule
Gemini-2.0-flash-exp & 92 (32.97 \%) & 91 (32.62 \%) & 30 (32.97 \%) & 31 (33.70 \%) \\ 
Gemini-1.5-pro & 121 (43.37 \%) & 110 (39.43 \%) & 27 (24.55 \%) & 38 (31.40 \%) \\ 
GPT-4o & 169 (60.57 \%) & 159 (56.99 \%) & 31 (19.50 \%) & 41 (24.26 \%) \\ 
GPT-4-turbo & 183 (65.59 \%) & 170 (60.93 \%) & 33 (19.41 \%) & 46 (25.14 \%) \\ 
Claude-3.5-Sonnet & 171 (61.29 \%) & 141 (50.54 \%) & 27 (19.15 \%) & 57 (33.33 \%) \\ 
Claude-3-Opus & 205 (73.48 \%) & 186 (66.67 \%) & 35 (18.82 \%) & 54 (26.34 \%) \\ 
\midrule
Llama-3.1-8B-Instruct & 251 (89.96 \%) & 249 (89.25 \%) & 18 (7.23 \%) & 20 (7.97 \%) \\ 
Gemma-2-9b-it & 246 (88.17 \%) & 239 (85.66 \%) & 14 (5.86 \%) & 21 (8.54 \%) \\ 
Phi-3.5-mini-instruct & 239 (85.66 \%) & 239 (85.66 \%) & 17 (7.11 \%) & 17 (7.11 \%) \\ 
\midrule
Deepseek-math-7b-rl & 241 (86.38 \%) & 236 (84.59 \%) & 19 (8.05 \%) & 24 (9.96 \%) \\ 
Qwen2.5-Math-7B-Instruct & 203 (72.76 \%) & 204 (73.12 \%) & 32 (15.69 \%) & 31 (15.27 \%) \\ 
Mathstral-7b-v0.1 & 238 (85.30 \%) & 233 (83.51 \%) & 19 (8.15 \%) & 24 (10.08 \%) \\ 
NuminaMath-7B-CoT & 231 (82.80 \%) & 231 (82.80 \%) & 23 (9.96 \%) & 23 (9.96 \%) \\ 
MetaMath-13B-V1.0 & 263 (94.27 \%) & 269 (96.42 \%) & 11 (4.09 \%) & 5 (1.90 \%) \\ 
MAmmoTH2-8B & 258 (92.47 \%) & 263 (94.27 \%) & 12 (4.56 \%) & 7 (2.71 \%) \\
\bottomrule
\end{tabular}
}
\label{tab:icl:breakdown}
\end{table*}


\begin{figure*}[htbp]
    \centering
    \includegraphics[width=0.33\linewidth]{figures/ICL/icl/Gemini-2.0-flash-thinking.pdf}
    \includegraphics[width=0.32\linewidth]{figures/ICL/icl/o1-mini.pdf}
    \includegraphics[width=0.32\linewidth]{figures/ICL/icl/GPT-4o.pdf}
    \includegraphics[width=0.33\linewidth]{figures/ICL/icl/Claude-3.5-Sonnet.pdf}
    \vspace{5mm}
    \includegraphics[width=0.32\linewidth]{figures/ICL/icl/Llama-3.1-8B-Instruct.pdf}
    \includegraphics[width=0.32\linewidth]{figures/ICL/icl/Gemma-2-9b-it.pdf}
    \includegraphics[width=0.34\linewidth]{figures/ICL/icl/Deepseek-math-7b-rl.pdf}
    \includegraphics[width=0.32\linewidth]{figures/ICL/icl/Qwen2.5-Math-7B-Instruct.pdf}
    
    \caption{The error rates (\%) of the models without and with the original problem and solution as the in-context learning (ICL) example. For \HARD, we decompose the influences of in-context learning into \textbf{ICL effect} (the down arrow $\textcolor{brown}{\boldsymbol{\downarrow}}$), which reduces the error rates, and \textbf{misleading effect} (the up arrow $\textcolor{brown}{\boldsymbol{\uparrow}}$), which increases the error rates.
    }
    \label{fig:icl:full}
\end{figure*}





\clearpage
\subsection{Ablation Study: In-Context Learning with the Original Example v.s. In-Context Learning with a Random Example}

In \cref{tab:oic_sic}, we compare (1) the performance of one-shot in-context learning with the corresponding \textbf{original} unmodified (problem, solution) with (2) the performance of ICL with a \textbf{random} problem and solution chosen from the same category as the query problem. We find that ICL with the \textbf{original} problem and solution consistently outperforms ICL with a \textbf{random} example except for only one case.

\begin{table*}[htbp]
\vspace{-3mm}
\caption{Performance comparisons without and with the original problem and solution as the in-context learning example.}
\centering
\resizebox{0.75\textwidth}{!}{
\begin{tabular}{lcccc}
 \toprule
\multirow{2}{*}{\textbf{Model}} & \multicolumn{2}{c}{\textbf{\SAME}} & \multicolumn{2}{c}{\textbf{\HARD}}   \\ 
\cmidrule(r){2-3}  \cmidrule(r){4-5}
& ICL w. original & ICL (random) & ICL w. original  & ICL (random)\\ 
\midrule
o1-mini & \textbf{94.98} & 92.83 & \textbf{78.49} & 75.99 \\ 
\midrule
Gemini-1.5-pro & \textbf{88.17} & 75.99 & \textbf{60.57} & 51.97 \\ 
GPT-4o & \textbf{77.06} & 63.08 & \textbf{43.01} & 37.28 \\ 
GPT-4-turbo & \textbf{69.89} & 57.71 &\textbf{ 39.07} & 32.62 \\ 
Claude-3.5-Sonnet & \textbf{83.15} & 62.37 & \textbf{49.46} & 40.86 \\ 
Claude-3-Opus & \textbf{68.10} & 45.52 & \textbf{33.33} & 23.66 \\ 
\midrule
Llama-3.1-8B-Instruct & \textbf{36.56} & 28.32 & \textbf{10.75} & 6.45 \\ 
Gemma-2-9b-it & \textbf{42.65} & 27.60 & \textbf{14.34} & 12.90 \\ 
Phi-3.5-mini-instruct & \textbf{36.92} & 20.07 & \textbf{14.34} & 10.39 \\ 
\midrule
Deepseek-math-7b-rl & \textbf{45.52} & 34.41 & \textbf{15.41} & 13.26 \\ 
Qwen2.5-Math-7B-Instruct & \textbf{56.99} & 55.20 & \textbf{26.88} & 26.16 \\ 
Mathstral-7b-v0.1 & \textbf{48.39} & 24.37 & \textbf{16.49} & 8.96 \\ 
NuminaMath-7B-CoT & \textbf{47.31 }& 24.73 & \textbf{17.20} & 10.04 \\ 
MetaMath-13B-V1.0 & \textbf{11.11} & 8.60 & 3.58 & \textbf{5.38} \\ 
MAmmoTH2-8B & \textbf{31.18} & 3.94 &\textbf{ 5.73 }& 2.15 \\ 
\bottomrule
\end{tabular}
}
\label{tab:oic_sic}
\end{table*}


\subsection{Inference-time Scaling Behaviors}
\label{sec:inference:scaling}

In this subsection, we investigate the inference-time scaling behaviors of LLMs on our benchmarks. 
We compute the pass@k metric following~\citet{chen2021codex}. Specifically, for each problem, we generate $N$ solutions independently, and compute the pass@k metric via the following formula for each $1\leq k\leq N$:
\[
    \mathrm{pass@k} = \mathbb{E}_{\mathrm{problem}} \left[ 1-\frac{ { N-c \choose k}}{ {N \choose k} } \right], \text{ where } c \text{ is the number of correct answers of the } n \text{ runs}.
\]
We also compute the performance of self-consistency~\citep{wang2022self}, a.k.a., majority voting, where for each $k$, we randomly sample $k$ responses from the $N$ runs and get the majority-voted answer. We report the average and standard deviation among 5 random draws.
We only evaluate three models: o1-mini, Llama-3.1-8B-Instruct, and Qwen2.5-Math-7B-Instruct. For Llama-3.1-8B-Instruct, and Qwen2.5-Math-7B-Instruct, we choose $N=64$, while for o1-mini we set $N=8$. The results are plotted in \cref{fig:inference:scaling}.







\begin{figure*}[ht]
    \centering
    \includegraphics[width=0.32\linewidth]{figures/inference_scaling_meta-llama-Llama-3.1-8B-Instruct.pdf}
    \includegraphics[width=0.32\linewidth]{figures/inference_scaling_Qwen-Qwen2.5-Math-7B-Instruct.pdf}
    \includegraphics[width=0.32\linewidth]{figures/inference_scaling_o1-mini.pdf}
    \caption{The effect of scaling up inference-time compute. We report pass@k and self-consistency (SC) accuracies for different numbers of solutions $k$.
    }
    \label{fig:inference:scaling}
\end{figure*}























 

\end{document}