\documentclass{article}
\PassOptionsToPackage{table,xcdraw}{xcolor}
\usepackage{xcolor}
\usepackage{bbm}
\usepackage{multirow}
% Language setting
% Replace `english' with e.g. `spanish' to change the document language
\usepackage[english]{babel}

% Set page size and margins
% Replace `letterpaper' with `a4paper' for UK/EU standard size
\usepackage[letterpaper,top=2cm,bottom=2cm,left=3cm,right=3cm,marginparwidth=1.75cm]{geometry}

% Useful packages
\usepackage{amsmath}
\usepackage{graphicx}
\usepackage[colorlinks=true, allcolors=blue]{hyperref}
\usepackage{booktabs} % To thicken table lines
\usepackage{caption}
\newcommand{\deluxesystem}{\textsc{LLMSelector}}

\newcommand{\eat}[1]{}


\usepackage{graphicx}
\usepackage{amssymb}
\usepackage{multirow}
\usepackage{bigstrut,morefloats}



\usepackage[resetlabels,labeled]{multibib}
\usepackage{caption}
\usepackage{subcaption}
%\usepackage{subfigure}




\usepackage[most]{tcolorbox}
%\usepackage{mdframed}
%\newcites{App}{Additional Reference}


%\usepackage{tcolorbox}








\newcommand{\E}{\mathbb{E}}

\usepackage{amsmath}
\usepackage{amsthm}
\usepackage{thmtools}

\theoremstyle{definition}
%\newtheorem{theorem}{Theorem}

%\usepackage[breaklinks]{hyperref}
%\usepackage{breakurl}


\usepackage[boxruled,linesnumbered]{algorithm2e}

\usepackage{amsthm}


\usepackage[most]{tcolorbox} % loads the tcolorbox package with most common features


% Custom style for the contribution box
\newtcolorbox[auto counter, number within=section]{mybox}[2][]{
    colback=blue!5!white,    % background color
    colframe=blue!75!black,  % frame color
    fonttitle=\bfseries,     % title font
    title=Main Contributions, % box title
    #1                      % allow for additional options
}


\newtcolorbox[auto counter, number within=section]{LLMdiagnoser}[2][]{
    colback=blue!5!white,    % background color
    colframe=blue!75!black,  % frame color
    fonttitle=\bfseries,     % title font
    title=LLM diagnoser prompt, % box title
    #1                      % allow for additional options
}


\usepackage{multirow}
\usepackage{array}
\usepackage{booktabs} % For professional looking tables

%\usepackage[table,xcdraw]{xcolor} % Required for row color


\usepackage{graphicx}
\usepackage[authoryear]{natbib}

%usepackage{authblk}

\usepackage{amsmath}

\newcommand{\ours}{$\text{Q}$LASS}
\usepackage{pifont}
\theoremstyle{plain}
\newtheorem{theorem}{Theorem}[section]
\newtheorem{proposition}[theorem]{Proposition}
\newtheorem{lemma}[theorem]{Lemma}
\newtheorem{corollary}[theorem]{Corollary}
\theoremstyle{definition}
\newtheorem{definition}[theorem]{Definition}
\newtheorem{assumption}[theorem]{Assumption}
\theoremstyle{remark}
\newtheorem{remark}[theorem]{Remark}

\title{Optimizing Model Selection for Compound AI Systems}
\author{Lingjiao Chen$^{\dagger,\circ}$, Jared Quincy Davis$^{\circ}$, Boris Hanin$^{\S}$\\\\ Peter Bailis$^\ddagger$, Matei Zaharia$^\ddagger$, James Zou$^{\circ}$, Ion Stoica$^\ddagger$\\\\
$^\dagger$Microsoft Research, $^\circ$Stanford University,\\ $^\S$Princeton University, $^\ddagger$University of California, Berkeley}
\date{}



% Save the original \cite as \cite_old
%\let\cite_old\cite

% Redefine \cite to behave like \citep
\renewcommand{\cite}[1]{\citep{#1}}


\begin{document}
\maketitle

\begin{abstract}
Compound AI systems that combine multiple LLM calls, such as self-refine and multi-agent-debate, achieve strong performance on many AI tasks. We address a core question in optimizing compound systems: for each LLM call or module in the system, how should one decide which LLM to use? We show that these LLM choices have a large effect on quality, but the search space is exponential. We propose \deluxesystem{}, an efficient framework for model selection in compound systems, which leverages two key empirical insights: (i) end-to-end performance is often monotonic in how well each module performs, with all other modules held fixed, and (ii) per-module performance can be estimated accurately by an LLM. Building upon these insights, \deluxesystem{} iteratively selects one module and allocates to it the model with the highest module-wise performance, as estimated by an LLM, until no further gain is possible. \deluxesystem{} is applicable to any compound system with a bounded number of modules, and its number of API calls scales linearly with the number of modules, achieving high-quality model allocation both empirically and theoretically. Experiments with popular compound systems such as multi-agent debate and self-refine using LLMs such as GPT-4o, Claude 3.5 Sonnet and Gemini 1.5  show that \deluxesystem{} confers 5\%-70\% accuracy gains compared to using the same LLM for all modules.
 
\end{abstract}


\section{Introduction}
\label{sec:intro}

\begin{figure*}[tb]
    \centering
    \includegraphics[width=0.848\linewidth]{figs/circuitnn.pdf} 
    \caption{Illustration of differentiable CircuitNN. CircuitNN is designed based on differentiable NAND gates. After DAS is guided by PI and PO pairs of the truth table, CircuitNN can get the precise circuit architecture logic equivalent to the truth table.}
    \label{fig:circuitnn}
\end{figure*}

% 1. Describe the importance of logic synthesis
% 2. Existing Problems
% (a) Neural Architecture Search: Unstable, Predefined Setting, etc.
% (b) Circuit Generation: Probabilistic Model, Logic Equivalence

With the rapid advancement of technology, the scale of integrated circuits (ICs) has expanded exponentially. 
This expansion has introduced significant challenges in chip manufacturing, particularly concerning power and area metrics.
A primary objective in IC design is achieving the same circuit function with fewer transistors, thereby reducing power usage and area occupancy.

Logic synthesis~\cite{hachtel2005logicsynth}, a critical step in electronic design automation (EDA), transforms behavioral-level circuit designs into optimized gate-level circuits, ultimately yielding the final IC layout. 
The primary goal of logic synthesis is to identify the physical implementation with the fewest gates for a given circuit function. 
This task constitutes a challenging NP-hard combinatorial optimization problem. 
Current logic synthesis tools~\cite{brayton2010abc, wolf2013yosys} rely on human-designed heuristics, often leading to sub-optimal outcomes.

Differentiable architecture search (DAS) techniques~\cite{liu2018darts, chu2020darts} offer novel perspectives on addressing challenges in this problem.
Circuit functions can be represented through truth tables, which map binary inputs to their corresponding outputs. 
Truth tables provide a precise representation of input-output relationships, ensuring the design of functionally equivalent circuits.
Inspired by this, researchers~\cite{deepmind2024ai4sys, wang2024tnet} have begun exploring the application of DAS to synthesize circuits directly from truth tables.
Specifically, \citet{deepmind2024ai4sys} proposed CircuitNN, a framework that learns differentiable connection structures with logic gates, enabling the automatic generation of logic circuits from truth tables.
This approach significantly reduces the complexity of traditional circuit generation. 
Building on this, \citet{wang2024tnet} introduced T-Net, a triangle-shaped variant of CircuitNN, incorporating regularization techniques to enhance the efficiency of DAS.

Despite these advancements, several challenges remain. 
The computational complexity of DAS grows quadratically with the number of gates, posing scalability issues.
Although triangle-shaped architecture~\cite{wang2024tnet} partially mitigates this problem, redundancy persists. 
%Additionally, DAS is susceptible to converging to local optima, limiting the ability to search architectures that satisfy the given truth tables~\cite{liu2018darts}. 
%Furthermore, hyperparameters (network depth and layer width) require extensive searches, introducing complexity and prolonging the synthesis process. 
Additionally, DAS is susceptible to converging to local optima~\cite{liu2018darts} and hyperparameters (network depth and layer width) require extensive searches. 
The challenges arise from the vast search space in DAS. 
% Even with predefined settings for CircuitNN, finding a configuration that meets the truth table requires extensive trial and error during the DAS process. 
Intuitively, limiting the search space through predefined parameters (network depth, gates per layer, and connection probabilities) can significantly reduce the complexity.

Recent advances~\cite{openai2023gpt4, abramson2024alphafold3, esser2024sd3, li2024mar} in conditional generative models have demonstrated remarkable performance across language, vision, and graph generation tasks. 
Motivated by these developments, we propose a novel approach to circuit generation that generates preliminary circuit structures to guide DAS in generating refined circuits matching specified truth tables. 
Firstly, we introduce CircuitVQ, a tokenizer with a discrete codebook for circuit tokenization. 
Built upon our Circuit AutoEncoder framework~\cite{hou2022graphmae,li2023maskgae,wu2025mgvga}, CircuitVQ is trained through a circuit reconstruction task. 
Specifically, the CircuitVQ encoder encodes input circuits into discrete tokens using a learnable codebook, while the decoder reconstructs the circuit adjacency matrix based on these tokens.
Subsequently, the CircuitVQ encoder serves as a circuit tokenizer for CircuitAR pretraining, which employs a masked autoregressive modeling paradigm~\cite{chang2022maskgit, li2023mage}. 
In this process, the discrete codes function as supervision signals. 
After training, CircuitAR can generate discrete tokens progressively, which can be decoded into initial circuit structures by the decoder of the CircuitVQ. 
These prior insights can guide DAS in producing refined circuits that match the target truth tables precisely.

Our key contributions can be summarized as follows:
\begin{itemize}
\item We introduce CircuitVQ, a circuit tokenizer that facilitates graph autoregressive modeling for circuit generation, based on our Circuit AutoEncoder framework;
\item Develop CircuitAR, a model trained using masked autoregressive modeling, which generates initial circuit structures conditioned on given truth tables;
\item Propose a refinement framework that integrates differentiable architecture search to produce functionally equivalent circuits guided by target truth tables;
\item Comprehensive experiments demonstrating the scalability and capability emergence of our CircuitAR and the superior performance of the proposed circuit generation approach.
\end{itemize}

% Motivation
% (a) Diffusion (Vision, Graph), Autoregressive (Language, Vision)
% (b) Circuit Generation for Predefined Setting
% (c) Neural Architecture Search for Strict Logic Equivalence

% Contribution
% (a) Circuit Tokenizer (new transformer arch, training strategy)
% (b) CircuitAR (train and gen strategies, post-ar strategy)
% (c) Extensive Evaluation including BitD (Bit Distance) for Scalability

\section{Preliminary}

\paragraph{Notation} Consider a sentence of $T$ tokens $\vx=\{\vx_1,\ldots, \vx_T\}\in\gX$, and let $P$ be the unknown target language distribution, $\tilde P(\vx)$ be the empirical distribution of the training data (which is an approximation of $P$), and $Q$ be the distribution of our model at hand. Since our paper is also closely related to RLHF, we will also use $\pi$ to represent the distributions. In particular, we sometimes write $\pi_\theta$ for a distribution that is parameterized by $\theta$, where $\theta$ is usually the set of trainable parameters of the LLM; we write $\pr$ for a reference distribution that should be clear given the context. The next token prediction loss is minimizing the forward-KL between $P$ and $Q$. 




% \begin{figure}
%     \centering
%     \includegraphics[width=0.5\linewidth]{Move_teaser.pdf}
%     \caption{Comparison of different dynamic compute approaches. length of arrow indicates residual transformation per token while width indicates velocity of transformation.}
%     \label{fig:enter-label}
% \end{figure}

\section{Method}
\label{sec:method}
Residual connections play a crucial role in shaping token representations, yet their dynamics remain underexplored in the context of efficient decoding. In this work, we delve deeper into transformer residual dynamics and investigate how modulating residual transformation velocity can improve inference efficiency in token-level processing, optimizing both dense and sparse MoE transformers.


\subsection{Residual Dynamics and Motivation for Multi-rate Residuals} \label{sec:motivation}

To analyze how hidden representations evolve across different layers of a transformer architecture, it's crucial to consider the effect of residual connections. Each transformer decoder layer typically has residual connections across attention and MLP submodules. As the residual stream $h_i$ traverses from interval $E_j$ to $E_{j+1}$, it undergoes a residual transformation given by:  
% \begin{equation}
% \label{eq:slow_residual_transformation}
% H_{E_{j+1}} = H_{E_j} \prod_{i=E_j}^{E_{j+1}} \left( I + \mathcal{A}_i \right) \left( I + \mathcal{M}_i \right) \quad \text{where} \quad \mathcal{A}_i = f(c_i, h_{i}), \mathcal{M}_i = g(h_i)
% \end{equation}

\begin{equation} \label{eq:slow_residual_transformation}
h_{E_{j+1}} = h_{E_j} + \sum_{i=E_j}^{E_{j+1}-1} \left( \mathcal{A}_i(h_i) + \mathcal{M}_i(h_i + \mathcal{A}_i(h_i)) \right) \quad \text{where} \quad \mathcal{A}_i = f(c_i, h_{i}), \mathcal{M}_i = g(h_i). 
\end{equation}

Here, \( \mathcal{A}_i \) denotes the non-linear transformation introduced by the multi-head attention mechanism at layer \( i \), while \( \mathcal{M}_i \) corresponds to the non-linear transformation of the MLP block at the same layer. These transformations depend on the input residual stream \( h_i \) and, in the case of \( \mathcal{A}_i \), the previous contextual representation \( c_i \).\footnote{Normalization layers are typically applied in practice but are omitted here for simplicity of the argument.}


% For easy tokens, the magnitude and direction of this delta transformation become progressively smaller with each successive layer as shown in \cref{fig:delta_transformation}. Consequently, it is feasible to predict these tokens after only a few residual connections, whereas harder tokens necessitate more extensive processing through additional layers.

\begin{figure}[ht]
    \centering
    \begin{subfigure}{0.48\textwidth}
        \centering
        \includegraphics[width=\textwidth]{sections/figures/residual_change.pdf}
        \caption{}
        \label{fig:residual_change}
    \end{subfigure}%
    \hfill
    \begin{subfigure}{0.48\textwidth}
        \centering
        \includegraphics[width=\textwidth]{sections/figures/alignment_wrt_dedicated_model.pdf}
        \caption{}
    \label{fig:alignment_wrt_dedicated_model}
    \end{subfigure}
    \caption{(a) As residual streams propagate through the model, the directional shifts in the residuals become progressively smaller. (b) A dedicated model with $k$ layers achieves a faster rate of change in residual streams and higher alignment than base model leveraging early exit mechanisms at layer $k$.}
    \label{fig}
\end{figure}


To examine whether residual transformations can be accelerated across layers, we conducted experiments using a diverse set of prompts on a pre-trained Phi3 model~\cite{phi3_report}. As illustrated in \cref{fig:residual_change}, we measured the directional shift in residual states as \( 1 - \mathcal{C}(h_{i-1}, h_i) \), where \(\mathcal{C}\) denotes normalized cosine similarity. This shift is notably higher in the initial layers, gradually decreasing in subsequent layers. This behavior allows traditional early exit approaches to effectively accelerate decoding by enabling earlier exits for simpler tokens. However, these approaches typically rely on a distance-based approximation, where the full residual transformation of the model is approximated by the residual transformations of the initial layers. To gain deeper insights into the distance versus velocity aspects of residual transformation, we conducted a comparative study. Specifically, we trained an early exit head at layer $k$ of the Phi3 model, which consists of 32 layers, restricting the distance traveled by each token. To accelerate the residual transformation relative to number of layers, we trained a smaller model consisting of only $k$ layers, while keeping all other hyperparameters consistent. We then compared the next-token prediction accuracy of the early exit head of the base model with that of the smaller model. To ensure an equal number of trainable parameters, we inserted low-rank adapters into the smaller model and trained only these adapters, whereas, in the distance-based approach, we trained solely the early exit head. In addition, to accelerate the residual transformation in smaller model, we distilled the residual streams from the larger model by incorporating a distillation loss ~\cite{sanh2019distilbert} between the residual state at layer \(i\) of the smaller model and the residual state at layer \(4 \times i\) of the larger model. As shown in ~\cref{fig:alignment_wrt_dedicated_model} the smaller model demonstrates a significantly faster rate of change in residual streams, leading to higher next token prediction accuracy after $k$ layers compared to the base model that employs traditional early exit mechanisms after $k$ layers \cite{schuster2022confident, chen2023eellm, varshney-etal-2024-investigating}. This experimental setup, which modifies only the rate of change in residual streams while keeping other factors constant, suggests that dense transformers, trained with a fixed number of layers, may inherently possess a slow residual transformation bias.

This observation raises an intriguing question: if the rate of change in residual streams could be accelerated relative to the number of layers, is it possible to facilitate earlier alignment for a greater proportion of tokens? Earlier alignment would be beneficial to not only facilitate dynamic computation but also for generating speculative tokens efficiently with high acceptance rates in speculative decoding setups ~\cite{leviathan2023fast, chen2023accelerating}. 

%thereby enhancing the efficiency of early exiting? 
 % This bias likely constrains the effectiveness of early exiting, particularly for easier tokens. By addressing this limitation through accelerated residual transformations, we hypothesize that it is possible to substantially improve the efficiency and accuracy of early exit strategies in transformer models.

\subsection{Multi-Rate Residual Transformation} \label{m2r2_method}

To address the slow residual transformation bias described in ~\cref{sec:motivation}, we introduce \textit{accelerated residual streams} that operate at rate $R$ relative to original slow residual stream. We pair slow residual stream, $h$ with an accelerated residual stream, $p$, which has an intrinsic bias towards earlier alignment. Relative to ~\cref{eq:slow_residual_transformation}, accelerated residual transformation from interval $E_j$ to $E_{j+1}$ can be represented as: 

% \begin{equation}
% \label{eq:fast_residual_transformation}
% P_{E_{j+1}} = P_{E_j} \prod_{i=E_j}^{E_{j+1}} \left( I + \hat{\mathcal{A}_i} \right) \left( I + \hat{\mathcal{M}_i} \right) \quad \text{where} \quad \hat{\mathcal{A}_i} = \hat{f}(c_i, P_{i}), \hat{\mathcal{M}_i} = \hat{g}(P_{i})
% \end{equation}


\begin{equation} \label{eq:fast_residual_transformation}
p_{E_{j+1}} = p_{E_j} + \sum_{i=E_j}^{E_{j+1}-1} \left( \hat{\mathcal{A}_i}(p_i) + \hat{\mathcal{M}_i}(p_i + \hat{\mathcal{A}_i}(p_i)) \right) \quad \text{where} \quad \hat{\mathcal{A}_i} = \hat{f}(c_i, p_{i}), \hat{\mathcal{M}_i} = \hat{g}(h_i), 
\end{equation}



where $\hat{\mathcal{A}_i}$ and $\hat{\mathcal{M}_i}$ denote non-linear transformation added by layer $i$ to previous accelerated residual $p_{i}$. Similar to $\mathcal{A}_i$, non-linear transformation $\hat{\mathcal{A}_i}$ attends to same context $c_i$ but uses a different transformation $\hat{f}$ for accelerating $p_{E_j}$ relative to $h_{E_j}$. 

We integrate accelerated residual transformation directly into the base network using parallel accelerator adapters such that rank of accelerator adapters $R_p << d$ where $d$ denotes base model hidden dimension. This setup allows the slow residual stream $h_{E_j}$ to pass through the base model layers while the accelerated residual stream $p_{E_j}$ utilizes these parallel adapters as shown in ~\cref{fig:m2r2_main}. Both slow and accelerated residuals are processed in same forward pass via attention masking and incur negligible additional inference latency in memory bound decoding setups, while in compute bound decoding setups where FLOPs optimization is essential, accelerated residual stream utilizes a fraction of attention heads that of slow residual (see ~\cref{sec:flops_optimization}). Additionally, to maximize the utility of accelerated residual transformations without introducing dedicated KV caches, we propose a shared caching mechanism between the slow and accelerated streams which minimally impact alignment benefits of our approach while offering substantial memory savings (see ~\cref{fig:koala_alignment}). Specifically, the attention operation on the slow residuals \( \text{MHA}(h_t, h_{\leq t}, h_{\leq t}) \) is redefined for accelerated residuals as 
\[
\hat{\mathcal{A}} = MHA(p_t, h_{<t} \oplus p_t, h_{<t} \oplus p_t),
\]
where the accelerated residual at time-step $t$, \( p_t \) attends to the slow residual’s KV cache, facilitating the reuse of contextual information across both residual streams without incurring additional caching costs. Here, \(MHA(q, k, v) \) represents multi-head attention between query \( q \), key \( k \), and value \( v \).

\begin{figure}
    \centering
    \includegraphics[width=0.8\linewidth]{sections//figures/m2r2_main2.pdf}
    \caption{Multi-rate Residuals Framework: Slow residual stream of base model is accompanied by a faster stream that operates at a $2-(J+1)\times$ rate relative to the slow stream, undergoing transformations via accelerator adapters as detailed in \cref{m2r2_method}, where J denotes number of early exit intervals. Colors within the slow and fast residual streams indicate similarity, with matching colors representing the most closely aligned residual states. At the beginning of the forward pass and at each exit point, the accelerated residual state is initialized from the corresponding slow residual state to avoid gradient conflict during training (see ~\cref{sec:grad_conflict}). Early exiting decisions are informed by the Accelerated Residual Latent Attention (ARLA) mechanism, described in \cref{method_arla}, which evaluates residual dynamics across consecutive exit gates.}
    \label{fig:m2r2_main}
\end{figure}

% Furthermore. to maximize the benefits of fast residual transformations without using dedicated KV caches, we propose sharing the fast network’s cache with the slow network. Formally speaking, We modify attention operation on slow residuals $MHA(H_t, H_{<=t}, H_{<=t})$ as $MHA(P_{t}, H_{<t} \oplus P_t, H_{<t}  \oplus P_t)$ such that accelerated residuals attend to previous slow context KV cache, where $MHA(q,k,v)$ denotes multi head attention between query, $q$, key $k$ and value $v$.


\subsection{Enhanced Early Residual Alignment}
Early residual alignment is instrumental in optimizing early exiting, speculative decoding, and Mixture-of-Experts (MoE) inference mechanisms. In this section, we provide a detailed analysis of how accelerated residuals enhance these inference setups.

% By aligning the residual states of intermediate layers with the final output representations, the model can maintain high prediction accuracy even when computations are truncated at earlier layers. This enables more reliable early exiting, reducing the overall computational cost while preserving performance. Additionally, in speculative decoding, early residual alignment allows the model to make confident predictions using faster, partial computations, thereby accelerating inference without sacrificing output quality.


\subsubsection{Early Exiting} \label{method_early_exiting}

A prevalent strategy for enabling early exiting at an intermediate layer $E_{j}$ involves approximating the residual transformation between $E_{j}$ and the final layer $N-1$ using a linear, context independent mapping, $\mathcal{T}$, such that $H_{N-1} \approx \mathcal{T}(H_{E_{j}})$. This approximation has been extensively employed in conventional approaches ~\cite{schuster2022confident, chen2023eellm, varshney-etal-2024-investigating}, providing a computationally efficient means to project the output of deeper layers from intermediate states. Specifically, residual state of layer $N-1$ with this approximation can be expressed as:


% \begin{equation}
% \label{eq: vanila_ea_assumption}
% \Phi(H_{E_{j}}) \sim H_{E_{j}} \prod_{i=E_{j}}^{N}\left( I + \mathcal{A}_i \right) \left( I + \mathcal{M}_i \right) \quad \text{where} \quad \Phi \perp C
% \end{equation}

\begin{equation} \label{eq:early_exiting}
h_{E_j} + \sum_{i=E_j}^{N-1} \left( \mathcal{A}_i(h_i) + \mathcal{M}_i(h_i + \mathcal{A}_i(h_i)) \right) \sim \mathcal{T}(h_{E_{j}})  \quad \text{where} \quad \mathcal{T} \perp c. 
\end{equation}


Here, $\mathcal{A}_i$ and $\mathcal{M}_i$ represent the residual contributions of the multi-head attention and MLP layers, respectively, while $\mathcal{T}$ remains independent of $c$, the preceding context.

This approach is inherently limited by two major factors: first, the assumption of linearity between $h_{E_{j}}$ and $h_{N-1}$ may not hold uniformly for all tokens, particularly when $E_j \ll N$. Second, the linear transformation $\mathcal{T}$ disregards the influence of the context $c$ and fails to account for the latent representations of previous contextual states. In contrast, M2R2 accelerated residual states mitigate both of these challenges by approximating the slow residual transformation of all layers via a faster residual transformation of fewer layers as:
% \begin{equation}
% H_{E_j} \prod_{i=E_j}^{N}\left( I + \mathcal{A}_i \right) \left( I + \mathcal{M}_i \right) \sim P_{E_j} \prod_{i=E_j}^{E_j+1}\left( I + \hat{\mathcal{A}_i} \right) \left( I + \hat{\mathcal{M}_i} \right)
% \end{equation}


\begin{equation} \label{eq:m2r2_approximating_ea}
h_{E_j} + \sum_{i=E_j}^{N-1} \left( \mathcal{A}_i(h_i) + \mathcal{M}_i(h_i + \mathcal{A}_i(h_i)) \right) \sim p_{E_j} + \sum_{i=E_j}^{E_{j+1}-1} \left( \hat{\mathcal{A}_i}(p_i) + \hat{\mathcal{M}_i}(p_i + \hat{\mathcal{A}_i}(p_i)) \right), 
\end{equation}

% \begin{equation} \label{eq:fast_residual_transformation}
% p_{E_{j+1}} = p_{E_j} + \sum_{i=E_j}^{E_{j+1}-1} \left( \hat{\mathcal{A}_i}(p_i) + \hat{\mathcal{M}_i}(p_i + \hat{\mathcal{A}_i}(p_i)) \right) \quad \text{where} \quad \hat{\mathcal{A}_i} = \hat{f}(c_i, p_{i}), \hat{\mathcal{M}_i} = \hat{g}(h_i) 
% \end{equation}






where $p_{E_j}$ is initialized from the slow residual state $h_{E_j}$ at each early exit interval $E_j$ using an identity transformation (see ~\cref{fig:m2r2_main}). As shown in ~\cref{fig:m2r2_residual_sim}, accelerated residuals offer a smoother, more consistent shift in residual direction across layers, in contrast to the abrupt changes typically seen at early exit points in standard early exit methods. Moreover, the normalized cosine similarity between accelerated states at early exit intervals and final residual states is substantially higher compared to traditional early exit techniques, highlighting improved alignment with final layer representations. Traditional adaptive compute methods are constrained by two principal factors: the number of tokens eligible for early exit at intermediate layers and the precision of early exit decision. If residual streams fail to saturate early, the majority of tokens remain ineligible for exit, thereby diminishing potential speedups. Additionally, imprecise delineations between tokens suitable for early exit can lead to underthinking (premature exits that adversely affect accuracy) or overthinking (unnecessary processing that compromises efficiency) ~\cite{zhou2020self, dai2020dynamic}. Enhanced early alignment using ~\cref{eq:m2r2_approximating_ea} helps to address  first issue. To address the second issue we introduce Accelerated Residual Latent Attention, which dynamically assesses the saturation of the residual stream, allowing for a more precise differentiation between tokens that can exit early and those requiring further processing.

% This results in uniform change in residual direction    
% % We keep $\mathcal{A} = \hat{\mathcal{A}}$, while $\hat{\mathcal{M}}$ is accelerated by a factor of $2 - (N_{E}+1)X$ relative to the slower residual transformation $\mathcal{M}$, where $N_E$ represents number of early exiting intervals.
% Figure~\cref{fig:rate_change_comparison} illustrates the comparative rate of change between these transformation streams.



% fig:rate_change_comparison
% - grid plot x axis -> layer id (0, 8) , y axis -> layer id -> dark color cell for max similarity , lighter for lower 
% 
-------------------------------------------------------
Let's consider residual stream $h_i$ traverses through interval $E_j$ to $E_{j+1}$ and undergoes residual transformation given by 
\begin{equation}
h_{E_{j+1}} = h_{E_j} \prod_{i=E_j}^{E_{j+1}} \left( 1 + \delta_i \right)    
\end{equation}

where $\delta_i$ denotes non-linear transformation added by layer $i$. Each non-linear transformation of layer $i$ is a function of previous contextual representation, $c_i$ and input residual stream $h_i-1$ as
$\delta_i = f(c_i, h_{i-1})$ 

One way to exit early at exit $E_j+1$ is to assume that residual transformation from $E_j+1$ to final layer $N-1$ can be approximated by a linear function $\phi$ as $h_{N-1} \sim \Phi(h_{E_j+1})$ and most conventional approaches such as \todo{cite EA papers} use this approach. In other words, 

\begin{equation}
\Phi(h_{E_j+1} \sim h_{E_j+1} \prod_{i=E_j+1}^{N} \left( 1 + \delta_i \right)   
\end{equation}

This approach suffers from two primary issues, linearity assumption from $h_E_j+1$ to $H_N-1$ if often incorrect, particularly when $E_j << N$. More importantly, linear transformation $\Phi$ doesn't consider effect of context $C_i$. M2R2  effectively addresses these issues as accelerated residual stream at interval $E_j+1$ can be represented as 

\begin{equation}
r_{E_{j+1}} = r_{E_j} \prod_{i=E_j}^{E_{j+1}} \left( 1 + \gamma_i \right)    
\end{equation}

where $\gamma_i$ denotes non-linear transformation added by layer $i$ to previous accelerated residual $r_i-1$. Similar to $\delta_i$, non-linear transformation $\gamma_i$ considers context $C_i$ as 
$\gamma_i = g(c_i, r_{i-1})$. So in summary, slow residual transformation is approximated by accelerated residual as: 

\begin{equation}
h_{E_j} \prod_{i=E_j}^{N} \left( 1 + \delta_i \right) \sim h_{E_j} \prod_{i=E_j}^{E_j+1} \left( 1 + \gamma_i \right)
\end{equation}

It's worth noting that accelerated residual $r_i$ and slow residual $h_i$ are processed concurrently at layer $i$ by constructing proper attention mask such as attention of slow residual is represented as 

$MHA(H_it, H_{i<=t}, H_{i<=t}$ while attention of fast residual is computed as 

$MHA(r_it, H_{i<=t}, H_{i<=t}$ where $MHA(q,k,v$ denotes multi head attention between query, $q$, key $k$ and value $v$.


------------------------------------------------------------------

Vertical latent attention on accelerated residual is computed as 
$MHA(S_mt, S(Ej<=i<=m)t, S(Ej<=i<=m)t)$ where $Smt$ denotes query/key/value projection in latent domain at layer $m$ at time $t$. 
------------------------------------------------------------------

Gradient conflict Avoidance: 

Let's consider $w_j$ is a trainable parameter that belongs to a layer between $E_j$ and $E_j+1$. Consider early exit loss at gate $E_j+1$, $L_j+1$, gradient propagation of $w_j$ at another trainable parameter $w_j-n$ can be gives as 

$\sum_{k=E_j-n}^{E_j} \beta_k \frac{\partial L_{E_k}}{\partial w_k}$

where $\beta_j$ denotes backward transformation coefficient for weight $w_j$ to reach gate $E_j$. 
 
On the other hand, gradient propagation in proposed approach can be represented as 

\[
\frac{\partial L_{E_j}}{\partial w_j} = 
\begin{cases} 
\beta_j \frac{\partial L_{E_j}}{\partial w_j} & \text{if } E_j \leq w_j \leq E_{j+1} \\
0 & \text{otherwise}
\end{cases}
\]







% \begin{figure}[ht]
%     \centering
%     \includegraphics[width=0.8\textwidth, height=5cm]{rate_change_comparison.png}
%     \caption{Rate of change comparison between fast and slow residual streams.}
%     \label{fig:rate_change_comparison}
% \end{figure}

%vary k and and plot EA accuracy for larger and smaller models. 

% \begin{figure}[ht]
%     \centering
%     \includegraphics[width=0.5\textwidth,height=5cm]{sections/figures/alignment_comparison_dialogsum.pdf}
%     \caption{Alignment of exited tokens for different early exit layers using traditional early exiting heads, dedicated faster networks, and faster residuals.}
%     \label{fig:small_model_early_exiting}
% \end{figure}


\textbf{Accelerated Residual Latent Attention} \label{method_arla}

In the context of residual streams, we observe that the decision to exit at a given layer can be more effectively informed by analyzing the dynamics of residual stream transformations, instead of solely relying on a classification head applied at the early exit interval $E_j$. To capture the subtle dynamics of residual acceleration, we propose a \textit{Accelerated Residual Latent Attention} (ARLA) mechanism. This approach involves making the exit decision at gate $E_j$ by attending to the residuals spanning from gate $E_{j-1}$ to $E_j$, rather than considering only the residual at gate $E_j$. To minimize the computational overhead associated with exit decision-making, the attention mechanism operates within the latent domain as depicted in ~\cref{fig:arla_arch}. Formally, for each interval $[E_j, E_{j+1}]$, the accelerated residuals are projected into Query ($Q^s_{E_j}, \ldots, Q^s_{E_{j+1}}$), Key ($K^s_{E_j}, \ldots, K^s_{E_{j+1}}$), and Value ($V^s_{E_j}, \ldots, V^s_{E_{j+1}}$) vectors, with latent dimension $d^s$ for $Q^s$, $K^s$, and $V^s$ being significantly smaller than hidden dimension of $p$.\footnote{We use $d^s = 64$ for experiments described in ~\cref{sec:experiments}.} Notably, when the router is allowed to make exit decisions at gate $E_j$ based on residual change dynamics, we observe that the attention is not confined to the residual state at $E_j$ but is distributed across residual states from $E_{j-1}$ to $E_j$, %as illustrated in Figure~\ref{fig:vertical_latent_attention_dynamics}. 
This broader focus on residual dynamics significantly reduces decision ambiguity in early exits, as demonstrated in Figure~\ref{fig:roc_arla}, which contrasts routers based on the last hidden state, and the proposed ARLA router.

%show R -> S transformation. 
%show parameter and flop overhead as compared to adapter on last hidden state.

% \begin{figure}[ht]
%     \centering
%     \includegraphics[width=0.5\textwidth,height=5cm]{sections/figures/roc_arla.pdf}
%     \caption{ROC curves of early exit decision strategies: confidence-based methods (CALM/LITE), routers based on the accelerated hidden state, and latent attention routers.}
%     \label{fig:decision_making_comparison}
% \end{figure}

% \begin{figure}[ht]
%     \centering
%     \includegraphics[width=0.5\textwidth,height=5cm]{vertical_latent_attention.png}
%     \caption{Vertical latent attention mechanism for optimizing early exit decisions by considering residuals from gate \(M\) through \(M-1\).}
%     \label{fig:vertical_latent_attention}
% \end{figure}

\begin{figure}[ht]
    \centering
    \begin{subfigure}{0.52\textwidth}
        \centering
        \includegraphics[width=\textwidth, height = 4cm]{sections/figures/arla_arch.pdf}
        \caption{Accelerated Residual Latent Attention (ARLA): Accelerated residuals between early exit gates are projected into latent domain and attention over residual states within the interval is computed to capture residual dynamics and exit decision is made based on residual saturation.}
        \label{fig:arla_arch}
    \end{subfigure}%
    \hfill
    \begin{subfigure}{0.45\textwidth}
        \centering
        \includegraphics[width=\textwidth, height = 4.5cm]{sections/figures/vla_roc.pdf}
        \caption{ROC classification curves of early exit decision strategies using a linear router used on last residual state ~\cite{schuster2022confident, varshney-etal-2024-investigating, chen2023eellm}  and using ARLA approach that considers residual dynamics. }
        \label{fig:roc_arla}
    \end{subfigure}
    \caption{Effectiveness of ARLA in capturing residual dynamics for early exiting decisions.}


\end{figure}



% \begin{figure}[ht]
%     \centering
%     \includegraphics[width=1\textwidth,height=5cm]{sections/figures/arla.pdf}
%     \caption{fig that plots 32 rows 2 cols heatmap showing attention at each gate}
%     \label{fig:vertical_latent_attention_dynamics}
% \end{figure}

\subsubsection{Self Speculative Decoding} \label{method_self_speculative_decoding}

An alternative means to exploit the early alignment properties of our approach is through the use of accelerated residual states for speculative token sampling to accelerate autoregressive decoding. Speculative decoding aims to speed up memory-bound transformer inference by employing a lightweight draft model to predict candidate tokens, while verifying speculated tokens in parallel and advancing token generation by more than one token per full model invocation \cite{leviathan2023fast, chen2023accelerating, xia2023speculative, miao2023specinfer}. Despite its effectiveness in accelerating large language models (LLMs), speculative decoding introduces substantial complexity in both deployment and training. A separate draft model must be specifically trained and aligned with the target model for each application, which increases the training load and operational complexity ~\cite{chen2023accelerating}. Additionally, this approach is resource-inefficient, as it requires both the draft and target models to be simultaneously maintained in memory during inference \cite{leviathan2023fast, chen2023accelerating}. 

One strategy to address this inefficiency is to leverage the initial layers of the target model itself to generate speculative candidates, as depicted in ~\cite{Tang2024}. While this method reduces the autoregressive overhead associated with speculation, it suffers from suboptimal acceptance rates. This occurs because the linear transformation employed for translating hidden states from layer $k$ to the final layer $N$ is typically a poor approximation, as discussed in ~\cref{sec:motivation} and ~\cref{method_early_exiting}. Our approach resolves this limitation by utilizing accelerated residuals, which demonstrate higher fidelity to their slower counterparts. By utilizing accelerated residuals operating at a rate of $N/k$, where $k$ denotes the number of layers used for candidate speculation, we are able to efficiently generate speculative tokens for decoding.\footnote{We typically set $k = 4$ to balance the trade-off between autoregressive drafting overhead and acceptance rate, as discussed in~\cref{sec:experiments}.}
 This technique not only obviates the need for multiple models during inference but also improves the overall efficiency and effectiveness of speculative decoding.

\begin{figure}
    \centering    \includegraphics[width=1\linewidth]{sections/figures/m2r2_aot_loading.pdf}
    \caption{Ahead-of-Time Expert Loading: M2R2 accelerated residual stream predicts experts required for future layers, reducing reliance on on-demand lazy loading. Speculative pre-loading is efficiently overlapped with computation of multi-head attention (MHA) and MLP transformations. Only incorrectly speculated experts are loaded lazily, resulting in faster inference steps and improved computational efficiency. Here, H indicates LBM Host while D indicates HBM Device.}
    \label{fig:moe_expert_aot_loading}
\end{figure}


\subsubsection{Ahead of Time Expert Loading:} \label{method_aot_expert_loading}

Recent advancements in sparse Mixture-of-Experts (MoE) architectures ~\cite{shazeer2017outrageously, fedus2022switch, artetxe2019massively, lepikhin2020gshard, zoph2022designing} have introduced a paradigm shift in token generation by dynamically activating only a subset of experts per input, achieving superior efficiency in comparison to dense models, particularly under memory-bound constraints of autoregressive decoding \cite{fedus2022switch, zoph2022designing}. This sparse activation approach enables MoE-based language models to generate tokens more swiftly, leveraging the efficiency of selective expert usage and avoiding the overhead of full dense layer invocation. In dense transformer models, pre-loading layers is a common strategy to enhance throughput, as computations of current layer can be overlapped with pre-loading of next layer parameters ~\cite{narayanan2021efficient, shoeybi2020megatron}. However, MoE models face a unique challenge: expert selection occurs dynamically based on previous layer’s output, making it infeasible to preload next layer’s experts in parallel. This limitation results in inherent latency, as expert loading becomes a sequential, on-demand process ~\cite{lepikhin2020gshard, fedus2022switch}.

To address this inefficiency, our method introduces a mechanism with \textit{accelerated residuals}, which not only captures key characteristics of base slower residual states but also exhibit high cosine similarity with their final counterparts (as illustrated in \cref{fig:m2r2_residual_sim}). By employing accelerated residual streams, we can effectively predict the necessary experts for future layers well in advance of their actual invocation. Specifically, using a $2\times$ accelerated residual, the experts needed for layers $2i+2$ and $2i+3$ can be identified while still computing in layer $i$, thus overcoming the bottleneck of sequential, on-demand expert selection and mitigating latency in the decoding pipeline, as shown in \cref{fig:moe_expert_aot_loading}. Note that, we use fixed set of accelerator adapters for transforming accelerated residuals (as discussed in ~\cref{m2r2_method}) while slow residual is transformed via expert routing mechanism. 

Furthermore, our approach integrates a Least Recently Used (LRU) caching strategy, which enhances memory efficiency by replacing the least recently used experts with speculated experts that are anticipated to be needed in upcoming layers. This hybrid approach of preemptive expert loading with LRU caching yields substantial improvements over traditional on-demand loading or standalone caching strategies. By minimizing cache misses and efficiently managing memory, this approach addresses both compute and memory bottlenecks, leading to faster, more resource-efficient token generation in MoE architectures. A comprehensive evaluation of this strategy, in relation to state-of-the-art methods, is provided in \cref{experiments_aot}, and the compute and memory traces on an A100 GPU are detailed in \cref{fig:moe_aot_cuda_trace}.



% Recent advancements in sparse Mixture-of-Experts (MoE) architectures have introduced the concept of utilizing distinct computational paths for different tokens \cite{shazeer2017outrageously}. This approach, wherein only a subset of experts are activated per input, enables MoE-based language models to generate tokens more swiftly compared to their dense counterparts due to memory-bound nature of auto-regressive decoding. In dense models, pre-loading layers in advance is a common strategy to enhance computational efficiency. However, this technique is not applicable to MoE models, where expert selection occurs dynamically based on the outputs of previous layers, preventing parallel pre-fetching of experts.

% Our proposed method addresses this inefficiency. Accelerated residuals, which are highly similar to their slower counterparts (see \cref{fig:similarity}), can reliably predict the necessary experts ahead of time. For instance, by utilizing $2X$ accelerated residual stream, we can predict the experts needed for the layer $2i+1$ and $2i+3$ while carrying out computation in layer $i$. This enables us to commence expert loading significantly earlier, as illustrated in \cref{expert_loading}, effectively mitigating the delays observed with the naive on-demand expert loading. Additionally, our method benefits from incorporating a Least Recently Used (LRU) strategy, where speculated experts replace those that are least recently utilized, resulting in improved performance compared to using either strategy alone. For a comprehensive evaluation, refer to \cref{moe_trace}, which provides a CUDA compute and memory trace of our approach executed on <>.



% A naive solution involves using the residual state of the previous layer along with the gating function of the next layer to predict which experts need to be loaded, and initiating the expert loading process in parallel with the attention computation of the next layer. Yet, as shown in \cref{fig:MOE_attn_vs_loading_time}, the attention computation for medium to long contexts is considerably faster than the expert loading time, making this approach inefficient.




\subsection{Training} \label{method_training}
% This approach is feasible due to the absence of gradient conflicts, as discussed in \cref{sec:grad_conflict}.

To accelerate residual streams, we employ parallel accelerator adapters as described in \cref{m2r2_method}.  For the early exiting use-case outlined in \cref{method_early_exiting}, we define the training objective for these adapters using the following loss function, which combines cross-entropy loss at each exit $E_j$ with distillation loss at each layer $i$. Loss weights coefficients $\alpha_0$ and $\alpha_1$ are employed to balance contribution of corresponding losses.

\begin{align} \label{eq:mr_loss}
L_{\text{m2r2}} = \underbrace{-\alpha_0 \sum_{j=1}^{J} \sum_{t=1}^{T} \log p_{\theta} \left( \hat{y}_t^{E_j} \mid y_{<t}, x \right)}_{\text{cross-entropy loss}} 
+ \underbrace{\alpha_1\sum_{i=1}^{E_{J-1}} \sum_{t=1}^{T} \| \mathbf{p}_{t}^{i} - \mathbf{h}_{t}^{((i - E_{j(i)}) \cdot R_i) + E_{j(i)})} \|^2}_{\text{distillation loss}}.
\end{align}

where $\hat{y}_t^{E_j}$ denotes the predictions from the accelerated residual stream at layer $E_j$ and time step $t$, $y_t$ represents the corresponding ground truth tokens, and $x$ indicates previous context tokens. The distillation loss at each layer $i$ is computed by comparing accelerated residuals at layer $i$ with slow residuals at layer $(i - E_{j(i)}) \cdot R_i + E_{j(i)}$, where $R_i$ denotes the rate of accelerated residuals at layer $i$ while $E_{j(i)}$ represents the most recent gate layer index such that $E_{j(i)} <= i$. \( J \) represents the total number of early exit gates, N denotes number of hidden layers and $E_j$ denotes layer index corresponding to gate index $j$ and \( T \) denotes the sequence length. 

In dynamic compute settings, after training of accelerator adapters, we optimize the query, key, and value parameters governing the ARLA routers (see ~\cref{method_arla}) across all exits in parallel on binary cross entropy loss between predicted decision and ground truth exiting decision. The ground truth labels for the router are determined based on whether the application of the final logit head on $\hat{y}_t^{E_j}$ yields the correct next-token prediction. 


% The objective for this optimization is defined by the following loss function:


%TODO are equations required ? 
% \begin{equation} \label{eq:arla_loss_combined}\small
%     L_{\text{arla}} = -\frac{1}{N} \sum_{t=1}^{T} \left( \sum_{j=1}^{E_n} \left[ O_t^{E_j} \log(\hat{O}_t^{E_j}) + (1 - O_t^{E_j}) \log(1 - \hat{O}_t^{E_j}) \right] \right), \quad \text{where} \quad 
%     O_t^{E_j} = \begin{cases} 
%     1, & \text{if } L(\hat{y}_t^{E_j}) = y_t^{E_j} \\
%     0, & \text{otherwise}
%     \end{cases}
% \end{equation}

% where $\hat{O}_t^{E_j}$ represents the binary predicted logits produced by the vertical latent attention router, as described in \cref{sec:arla}, at gate $E_j$ and time step $t$, and $O_t^{E_j}$ denotes the corresponding ground truth labels. The ground truth labels for the router are determined based on whether the application of the logit head on $\hat{y}_t^{E_j}$ yields the correct next-token prediction. The parameters controlling vertical latent attention are trained concurrently to ensure consistency and efficient use of computational resources.

For self-speculative decoding, as described in \cref{method_self_speculative_decoding}, the training objective remains the same as \cref{eq:mr_loss}, but with the number of intervals set to $J = 1$ and the rate of residual transformation set to $R_n = N/k$, where the first $k$ layers generate speculative candidate tokens. In the context of Ahead-of-Time Expert Loading for Mixture-of-Experts (MoE) models (see \cref{method_aot_expert_loading}), setting the rate of residual transformation to $R_n = 2$ typically offers a good trade-off between the accuracy of expert speculation and AoT pre-loading of experts. 

% Thus, we set $J = 1$ and $E_1 = 16$.


~\subsection{FLOPs Optimization} \label{sec:flops_optimization}

Naively implemented, M2R2 incurs higher FLOP overhead compared to traditional speculative decoding and early exiting approaches such as ~\cite{medusa, schuster2022confident, Tang2024}. However, modern accelerators demonstrate compute bandwidth that exceeds memory access bandwidth by an order of magnitude or more~\cite{databricksLLMInference2023, jouppi2021ten}, meaning increased FLOPs do not necessarily translate to increased decoding latency. Nevertheless, to ensure fair comparison and efficiency in compute bound scenarios, we introduce targeted optimizations.

~\textbf{Attention FLOPs Optimization} For medium-to-long context lengths, attention computation dominates FLOPs in the self-attention layer, surpassing the contribution from MLP layers. Specifically, matrix multiplications involving queries, cached keys, and cached values scale with $l_{kv} * l_{q}$ where $l_{kv}$ denotes previous context length and $l_q$ denotes current query length. Since M2R2 pairs accelerated residuals with slow residuals, a naive implementation results in twice the FLOPs consumption compared to a standard attention layer. To address this, we limit the attention of accelerated residual stream to selectively attend to the top-k most relevant tokens, identified by the slow residual stream based on top attention coefficients\footnote{We set to k = 64 and attend to top 64 tokens as identified by the slow residual stream.}. This is possible since slow and accelerated residual streams are processed in same forward pass and accelerated streams have access to attention coefficients of slow stream. Note that, the faster residual stream still retains the flexibility to assign distinct attention coefficients to these tokens. Furthermore, we design the faster residual stream to employ only 8 attention heads, compared to the 32 heads used in the slow residual stream of the Phi-3 model, reducing query, key, value, and output projection FLOPs by a factor of 1/4. ~\cref{fig:m2r2_num_heads_ablation} indicates effect of using a slicker stream on alignment. As depicted, using $\hat{n}_h = 8$ offers a good trade-off between alignment and FLOPs overhead. 

~\textbf{MLP FLOPs Optimization} The accelerator adapters operating on the accelerated residual stream are intentionally designed with lower rank than their counterparts in the base model. This reduces FLOP overhead by a factor proportional to $hiddenSize / rank$. Additionally, since the faster residual stream uses only 8 attention heads (compared to 32 in the slow residual stream of Phi-3), the subsequent MLP layers process a smaller set of activations, further reducing FLOPs by another factor of 1/4.

These optimizations significantly reduce the FLOP overhead per speculative draft generation, as illustrated in ~\cref{fig:flops_optmization}. Notably, while traditional early-exiting speculative approaches such as DEED require propagating the full slow residual state through the initial layers, incurring substantial computational costs, M2R2 achieves efficient token generation via slimmer, low-rank faster residual streams. In contrast, Medusa introduces considerable FLOP overhead due to per-head computations scaling with $d^2+dv$\footnote{Here $d$ denotes hidden state dimension while $v$ denotes vocab size.}, whereas M2R2 employs low-rank layers for both MLP and language modeling heads, maintaining computational efficiency. All experiments involving the M2R2 approach, as detailed in ~\cref{sec:experiments}, are conducted using these FLOPs optimizations.









% \[
% O_t^{E_j} = 
% \begin{cases} 
% 1, & \text{if } L(\hat{y}_t^{E_j}) = y_t^{E_j} \\
% 0, & \text{otherwise}
% \end{cases}
% \]




%add distillation
% We train accelerator adapters described in \cref{m2r2_method} to accelerate residual streams on next token prediction all in parallel since there are no gradient conflict issues as described in \cref{sec:grad_conflict}.

% \begin{align} \label{eq:mr_loss}
% L_{mr} =  & -\sum_{j = 1}^{E_n} (\sum_{t=1}^{T}\log p_{\theta} (\hat{y}_t^{E_j} | \hat{y}_{<t}, x)) \nonumber
% \end{align}

% where $\hat{y_t^{E_j}}$ denotes predicted logits obtained from accelerated residual stream at gate $E_j$ and time-step $t$ while $y_t^{E_j}$ denotes corresponding truth tokens. 

% Upon training of adapters responsible for accelerating residual streams, we train query, key, value parameters responsible for vertical latent attention of all gates in parallel as

% \begin{equation} \label{eq:arla_loss}
%     L_{arla} = -\frac{1}{N} (\sum_{t=1}^{T}(1\sum_{j=1}^{E_n} \left[ O_t^{E_j} \log(\hat{O}_t^{E_j}) + (1 - o_t^{E_j}) \log(1 - \hat{o_t}_{E_j}) \right]))
% \end{equation}

% where $\hat{O_t^{E_j}}$ denotes binary predicted logits obtained from vertical latent attention router described in \cref{sec:arla} at gate $E_j$ and timestep $t$ while $O_t^{E_j}$ denotes corresponding truth label. Truth labels for router are obtained by computing whether logit head application on $\hat{y}_t^j$ results in true next token prediction. Formally speaking, 

% $O_t^{E_j} = 1 if L(\hat{y_t^{E_j}}) == y_t^{E_j} , 0 otherwise$. 

% Parameters responsible for vertical latent attention are also trained in parallel as well. 

%todo: training slow and fast residuals together and distillation can be two training mdoes. 
%Distillation can be an ablation. 




% Although transformer decoding is memory bound on most mainstream accelerators, there could be scenarios where flop savings are crucial. For instance, on on-device settings power consumption is directly correlated with flops per decoding step and reducing flops does help with overall energy consumption. Vanilla early exiting methods help with flop reduction but suffer from mismatch between training and inference due to early exited tokens. If token at decoding step $t$, $T_t$ exited at layer $E_i$, while token $T_{t+k}$ exits at layer $E_j$ such that $E_i < E_j$, hidden state $H_{t+k}l$ does not have corresponding hidden state $H_tl$ to attend to where $E_i < l <= E_j$. One solution that's often used in literature is to rely on last hidden state available, $H_t{E_j}$, however it tends to be sub-optimal and does affect generation quality \cite{ref}.  To alleviate this mismatch while reducing flops, we train router such that attention mask between token $T_{t+k}$ and token $T_{<t+k}$ is given by: 

% \begin{equation}
%     a_{T_{{t+k}{T_{<t+k}}} = 1 if  E_{T_{<t+k}} >= E{T_{t+k}}
%     else 0
% \end{equation}

% This attention mask enables router to account for exited tokens and get trained accordingly. Since attention mechanism during decoding remains exactly same as that during training, impact on generation quality tends to be minimal as noted in \cref{fig:gen_auality_with_and_without_recompute_attention_show_flops}.  Although MoD does not suffer from training and inference mismatch, we observe that it suffers from discountinuity between pre-training and super-vised fine-tuning resulting in sub-optimal perplexity. On the other hand, our method doesn't not require pre-training , doesn't suffer from discountinuity, and achieves much better perplexity in super-vised fine-tuning and instruction tuning setups as shown in \cref{fig:Mod_vs_m2r2_loss_curves}.






% Our techniques are directly applicable in such scenarios.    




%expert loading with cuda streams in experiments
\section{Experiments}\label{sec_exp}
%\hp{Accelerating IM simulation~\cite{tang2015influence}}

% \begin{itemize}
%     \item 6.1. Problem setting of three COPs, including the general model and three specific CO problems 
%     \item 6.2. Experiment Setting (hyperparameters, details of training, evaluation, and test) 写在appendix里吧
%     \item 6.3. Performance analysis 这个要占半页
% \end{itemize}

%\hp{need to think of a way to compress these tables / visuals.} 

%\hp{\cancel{Baselines}; hyperparamters; \cancel{metrics}; etc.}

With theoretical guarantees on the existence and convergence of NE for ACCES games, we are also interested in how our proposed algorithm CCDO-RL works empirically. To evaluate this, we conduct experiments of CCDO-RL on three distinct ACCES game instances introduced in Section \ref{sub_exp_ins} and analyze the performance of CCDO-RL in Section \ref{sub_train_eval}. Section 6.2.1 aims to empirically demonstrate the convergence (Figures \ref{fig_exploit_20} and \ref{fig_exploit_50}) of the algorithm CCDO-RL over realistic CO problems, and show its consistency with Theorem \ref{CCDOA}. Section 6.2.2 intends to show the average reward (to seen training graphs) as well as the generalizability (to unseen test graphs) of the combinatorial player in real-world ACCES games (shown in Tables \ref{tab_aver}, and \ref{tab_gene}).

\subsection{Three Instances of ACCES Games} \label{sub_exp_ins}
% \hp{This para does not make much sense. Need to follow the framework in the Preliminaries section.}
% For combinatorial optimization problems in real-world applications, situations are more complicated and intractable due to changeable environmental or physical parameters. The form of parameter sets is very crucial because different types have different solvability and computation complexity. Forms of parameter sets mainly contain discrete sets, interval sets \cite{buchheim2018robust} like polyhedral and ellipsoid, probability distributions \cite{carlsson2018wasserstein}, and variable functions \cite{krause2008robust}.

% In reality, these parameters are often impacted by some common factors, such as conditions of weather, transportation, and individual personalities. \cite{kalimeris2019robust} proposed an assumption that real instances (e.g. demands in CVRP, coverages in CSP) 
%Considering affected or attacked COPs, the real instance $\{\theta_{i}\}$ always relied on the estimated value $\{\hat{\theta}_{i}$\} and the variation determined by independent factors $\{g_{i}\}$ and environment/physical parameters/attacker actions $\{\eta\}$. The concrete parameter influence model is stated as follows:

We consider a certain COP which is parameterized with $\{\theta_{i}\}$, where $i$ is the index of nodes (such as a target in security games) -- e.g., such parameters can be interpreted as attack probability of targets.
%coverage radius, customer's demands, or attack probability of targets. 
In real-world applications, we often need to estimate such parameters before solving the COPs. Unfortunately, the estimation $\{\hat{\theta}_{i}\}$ often bears a gap to the true value $\{\theta_{i}\}$, which derives from e.g. environment (aleatoric) uncertainty, model (epistemic) uncertainty, or an attacker trying to manipulate the defender's utility. We use a generic model to formulate this gap:
\begin{equation}\label{linrob}
    \theta_{i} = \hat{\theta}_{i} + y \cdot \tau_{i},
\end{equation}
where $y$ represents the strategy of the nature/attacker, $\tau_{i}$ is the environment factors like weather and transportation conditions, or human subjective factors like the preference of the attacker. 
Such abstraction can represent a wide range of ACCES games, such as facility location covering problems \cite{an2020battery, TIRKOLAEE2020340}, CVRP \cite{vehiclerouting.ch8,dinh2018exact, FLORIO20231081}, security patrolling (OP) \citep{xu2021robust}, and influence maximization problem \cite{kalimeris2019robust}. We describe three instances of ACCES games based on the model (\ref{linrob}).%Based on this model (\ref{linrob}), we focus on three combinatorial optimization problems with attacks or environmental/physical influence.

% \hp{Hard to follow. We should point out what are the two players, what are X, Y, u etc}

\textbf{Adversarial Covering Salesman Problem (ACSP):} In a map of cities, every city $i$ has a coverage $\theta_{i}$. A salesman finds the shortest path such that all cities are visited or covered, with $\theta_{i}$ influenced by physical factors $\tau_i$ and transportation parameters $y$ based on Eq.(\ref{linrob}). The salesman is Player 1 where $X$ consists of the feasible paths of the salesman. Nature is Player 2 with $Y$ = $[0, 1]^K \ni y, K \in \mathbb{N}$. The utility function of Player 1 $u$ is the opposite of the total traveling distance.

\textbf{Adversarial Capacitated Vehicle Routing Problem (ACVRP):} A vehicle with a constrained capacity of goods finds the shortest path under the worst case with the $i_{th}$ customer's demand $\theta_i$ changed by environmental factors $\tau_i$ and weather parameter $y$ on Eq.(\ref{linrob}). The vehicle is Player 1 where $X$ is the set of the feasible path $x$. Nature is Player 2 where $Y$ is $[0, 1]^K \ni y, K \in \mathbb{N}$. The utility function of Player 1  $u$ is the opposite of total delivery distance satisfying all the demands of customers.


\textbf{Patrolling Game (PG):} The patrolling game is described in the introduction.

For all the problem instances, we run our algorithm on two problem sizes: 20 nodes and 50 nodes. The detailed description and problem parameters of the three game instances are in Appendix \ref{app_ex_para_set}.

% Similarly, in the vehicle route problem (VRP), conditions with correlated parameters arouse broad attention from scholars \cite{vehiclerouting.ch8,dinh2018exact,FLORIO20231081}. \cite{dinh2018exact} considered the demand correlation by geographical proximity of nodes, described by some independent random variables in the fractional form. \cite{FLORIO20231081} utilized 'external factors' to stand for unknown covariates affecting all demands and presented a Bayesian model to learn correlations. Further more, about IM problems, \cite{kalimeris2019robust} combined node features and uncertain hyperparameters to fit the influence probability on each edge.

% \subsection{Training CCDO-RL}

% For all the problems, CCDO-RL adopts the REINFORCE algorithm with an attention-based encoder-decoder framework \cite{kool2018attention} (used as an inductive graph representation component) to learn a (generalizable) COP solver for one player (protagonist), and PPO \cite{schulman2017proximal} to train a policy for the other player (adversary) whose strategy space is continuous. CCDO-RL is trained with 50 epochs on a set of 10,000 graphs (with 20 or 50 nodes). The hyperparameters of CCDO-RL are specified in Appendix \ref{app_ex_para_set} (Table \ref{tab_hyper_ccdorl}). Our code is included as supplementary material for ease of reproduction. 
% % \hp{need to specify hyperparas}

\subsection{Performance of CCDO-RL}\label{sub_train_eval}

Two aspects are evaluated for the performance of CCDO-RL, i.e., i) Convergence to NE (Section \ref{sub_per_conver}) exploring whether CCDO-RL can compute the NE, and ii) Protagonist policy's average reward and generalizability (Section \ref{sub_per_rob}). Generalizability refers to the ability of RL models trained on previously seen graphs (problem instances), to perform well on a new set of unseen test graphs. The model’s usability is enhanced by generalizability, rather than focusing solely on the average reward, which is a critical motivation in the literature on RL for COPs \citep{khalil2017learning, kool2018attention}.

For all the problems, CCDO-RL adopts the REINFORCE algorithm with an attention-based encoder-decoder framework \citep{kool2018attention} (used as an inductive graph representation component) to learn a generalizable COP solver for Player 1 (protagonist), and PPO to train a policy for Player 2 (adversary) whose strategy space is continuous. CCDO-RL is trained on a set of 10,000 graphs (with 20 or 50 nodes). The hyperparameters of CCDO-RL are specified in Appendix \ref{app_ex_para_set} (Table \ref{tab_hyper_ccdorl}). Our code is included as supplementary material and will be open-sourced for ease of reproduction. 

% \textbf{Training.} For all the problems, CCDO-RL adopts the REINFORCE algorithm with attention-based encoder-decoder framework \cite{kool2018attention} (used as an inductive graph representation component) to learn a (generalizable) COP solver for one player (protagonist), and PPO \cite{schulman2017proximal} to train a policy for the other player (adversary) whose strategy space is continuous. CCDO-RL is trained with 50 epochs on a set of 10,000 graphs (with 20 or 50 nodes). 

% \hp{We should first present results about convergence as it is mostly aligned with the theory.}

\subsubsection{Convergence to NE} \label{sub_per_conver}

Exploitability is a common metric to describe the closeness to true NE by calculating the sum of performance distances between each new best response and subgame NE, i.e. $\sum_{i=1,2} U(\pi_{i,k}^{br}, \sigma_{-i,k}) - U(\sigma)$ in the general two-player game. Since our game is zero-sum, the calculation is as follows:
\begin{equation*}
   \text{Exploitability}(\sigma) = \max_{\pi_1 \in \Sigma_1} U(\pi_1, \sigma_{2}) - \min_{\pi_2 \in \Sigma_2} U(\sigma_1, \pi_2).
\end{equation*}
From Figure \ref{fig_exploit_20}, we can see that CCDO-RL can converge to approximate NE in 25 iterations or less (in the PG setting), reaching 0.05 in ACSP, 0.10 in ACVRP, and 0.03 in PG with 20 nodes. Similar results are observed in problems with 50 nodes (see Figure \ref{fig_exploit_50} in Appendix \ref{app_exp}). These results validate the effectiveness of CCDO-RL in finding the NE for various types of games.

%Similarly, the exploitability of three COPs in 50 nodes is provided in the appendix \ref{app_exp}.
\vspace{-\baselineskip}
\begin{figure}[htbp]
	\centering
    \subfigure[ACSP20]{
    \label{csp20_nashconv}
    \includegraphics[scale=0.20]{Figures/nashconv_log_csp20_sm_7.eps}
    }
    \subfigure[ACVRP20]{
    \label{cvrp20_nashconv}%文中引用该图片代号
    \includegraphics[scale=0.20]{Figures/nashconv_log_svrp20_sm_7.eps}
    }
    \subfigure[PG20]{
    \label{opsa20_nashconv}
    \includegraphics[scale=0.20]{Figures/nashconv_log_pg20_sm_7.eps}
    }
    \caption{Exploitability curve of CCDO-RL on three games of 20 nodes}
    \label{fig_exploit_20}
\end{figure}
\vspace{-\baselineskip}
\subsubsection{Average reward and Generalizability of Combinatorial player} \label{sub_per_rob}
% \subsubsection{Robustness and Generalizability of Protagonist Policy} \label{sub_per_rob}
%\hp{CCDO-RL being better in these following metrics is only kind of a by-product.}

% \textbf{Evaluation.} The learned policies are then tested on 200 graphs, where 100 of them are randomly selected from the 10,000 training graphs, and the other 100 are unseen graphs. 
% We use two metrics to evaluate the performance of different policies for the protagonist player: \textbf{Average proportional loss} $R-$ describes the policy overfitting degree \citep{lanctot2017unified}; \textbf{Reward} evaluates the performance of the protagonist with the adversary under three COPs.  
% \begin{eqnarray}
%         &R- = (\hat{D} - \hat{O}) / \hat{D}.
% \end{eqnarray}
% in which $\hat{D}$ is the mean value of the diagonals and $\hat{O}$ is the mean value of the off-diagonals in the payoff matrix provided in the Appendix \ref{app_exp}.

% Because the protagonist policy is trained against a powerful adversary under our ACCES game setting, the obtained policy is naturally robust against adversarial perturbations. This subsection sheds a bit of light on this perspective and quantifies the extent of robustness of CCDO-RL as well as the ability of RL to generalize to unseen test graphs.

\textbf{Evaluation.} The learned policies are tested on 200 graphs, with 100 being randomly selected from the 10,000 training graphs (to show the average reward), and the other 100 being unseen graphs (to test policy generalization). We evaluate the performance of the protagonist with the adversary under three COPs. For each COP, the performance is considered both on the 20-node and 50-node map.
% We use two metrics to evaluate the performance of different policies for the protagonist player: \textbf{Average proportional loss} $R-$ describes the policy overfitting degree \citep{lanctot2017unified}; \textbf{Reward} evaluates the performance of the protagonist with the adversary under three COPs.

\textbf{Baselines.} There are heuristic algorithms for each game instance (Heuristic in Table \ref{tab_aver} and \ref{tab_gene}) and a single-player RL algorithm. For ACVRP, we adopt the Tabu Search algorithm (Tabu) \citep{li2020improved} as the heuristic algorithm, which is widely applied in the routing problem. For ACSP, the common benchmark local search algorithm, LS2 \citep{golden2012generalized}, is used. For PG, we choose the greedy algorithm as the baseline. The "RL against Stoc" algorithm in Tables \ref{tab_aver} and \ref{tab_gene} is identical to the protagonist model in CCDO-RL but trained in environments with stochastic adversarial perturbations.

% \textbf{Baselines.} There are a heuristic algorithms for each game instance {\color{red} (Heuristic mentioned in the Table \ref{tab_aver} and \ref{tab_gene})} and a single-player RL algorithm. For ACVRP, we adopt the Clarke-Wright (CW) algorithm \citep{pichpibul2013heuristic} and the Tabu Search algorithm (Tabu) \citep{li2020improved} as heuristics, which are applied widely in the routing problem. For ACSP, two common benchmark local search algorithms, LS1 and LS2 \citep{golden2012generalized}, are used. For PG, we choose a local search algorithm \citep{vansteenwegen2009iterated} and the greedy algorithm as the heuristic baselines. {\color{red} The "RL  against Stoc" algorithm referred to Tables \ref{tab_aver} and \ref{tab_gene}} is identical to the protagonist model in CCDO-RL {\color{red} but trained on environments with stochastic adversarial perturbations.} 

\textbf{Average Reward.}  As illustrated in Table \ref{tab_aver}, our algorithm achieves a better average reward than baselines (10.08\% improvement on average of all settings against two baselines), regardless of CO instance or problem size, when confronting the adversary trained by CCDO-RL. In the setting of CSP-20 nodes, the average reward is improved by 46.98\% compared to the heuristic and by 7.14\% compared with the RL against Stoc. For the 50-node setting, the improvements are 45.91\% and 5.28\% respectively. Similarly, the improvements in contrast to Heuristic and RL against Stoc are as follows: 1.72\% and 3.01\%  for CVRP-20 nodes, 0.75\% and 4.46\% for CVRP-50 nodes, 4.17\% and 1.48\% for PG-20 nodes, and 10.60\% and 4.38\% for PG-50 nodes.

\textbf{Generalizability.} From Table \ref{tab_gene}, CCDO-RL continues to achieve a better average reward when facing the adversary, demonstrating that the learned RL policies generalize well to unseen graphs. Even though the non-RL baselines do have access to the graph structures and other problem information of the unseen problem instances, CCDO-RL can obtain comparable performances without re-training on the new problem instances. The improvements versus Heuristic and RL against Stoc are 46.61\% and 7.02\% for CSP-20 nodes, 42.24\% and 3.94\% for CSP-50 nodes, 1.12\% and 1.56\% for CVRP-20 nodes, 0.90\% and 5.05\% for CVRP-50 nodes, 5.35\% and 2.40\% for PG-20 nodes, and 12.17\% and 10.33\% for PG-50 nodes. Even when confronting the stochastic adversary, CCDO shows superior generalizability compared to two baselines across three COPs, with average improvements of 6.31\%, 3.42\%, and 3.95\% respectively. Detailed results are provided in Appendix \ref{app_exp} (Tables \ref{tab_csp_full_20} - \ref{tab_op_full_50}). 
% The model’s usability is enhanced by the ability to generalize rather than focusing solely on the average reward, which is a critical motivation of the RL for combinatorial optimization literature \citep{khalil2017learning, kool2018attention}.  

\begin{remark}
    In CO problems (or more broadly, operations research and economics), it is known that achieving solution quality improvements against strong baselines (e.g., the RL methods trained with a stochastic adversary) is very challenging, and the margins are usually small \citep{kool2018attention}, sometimes even less than 1\%. However, these “tiny” marginal improvements in profits keep small business owners in the real world alive. Last, the improvement depends a lot on the problem settings, and we show that sometimes the improvement can be much more significant.
\end{remark}
\vspace{-\baselineskip}
% \textbf{Performance analysis.} The robustness results of CCDO-RL for ACSP are shown in Table \ref{tab_csp}. We have the following observations: 1) On both of the 100 seen/unseen graphs, single-player RL performs better than heuristic algorithms no matter whether attacked or not. (2) When confronting the adversary trained by CCDO-RL, CCDO-RL exceeds RL by 0.25 and 0.24 on the training set, and by 0.25 and 0.18 on the test set, respectively under the 20-node and 50-node graphs. This demonstrates the robustness of CCDO-RL. 3) Compared to the performance of the training set with that of the test set, we can see that RL and CCDO-RL both maintain a certain degree of generalization. Similar results for ACVRP (Table \ref{tab_cvrp}) and SPG (Table \ref{tab_op}) are provided in Appendix \ref{app_exp}. 

\begin{table}[ht]
  \caption{Average reward against CCDO-RL's adversary (on seen graphs)}
  \vspace{\baselineskip}
  \label{tab_aver}
  \centering
  \small
  \begin{tabular}{lllllll}
    \toprule
    \multirow{2}{*}{method} & \multicolumn{2}{c}{ACSP (Mean$\pm$Std)} & \multicolumn{2}{c}{ACVRP (Mean$\pm$Std)} & \multicolumn{2}{c}{PG (Mean$\pm$Std)} \\
    \cmidrule(r){2-3} \cmidrule{4-5} \cmidrule(r){6-7}
                            & 20 nodes & 50 nodes & 20 nodes & 50 nodes & 20 nodes & 50 nodes\\
    \midrule
    Heuristic & 6.13$\pm$1.20 & 7.55$\pm$1.42 & 7.65$\pm$1.23  & 13.38$\pm$1.70 & 2.64$\pm$1.03 & 4.53$\pm$1.84   \\
    RL against Stoc    & 3.50$\pm$0.47  & 4.55$\pm$0.62  & 7.55$\pm$1.16  & 13.90$\pm$1.63 & 2.71$\pm$0.90 & 4.80$\pm$2.18   \\
    CCDO-RL   & $\pmb{3.25}$$\pm$0.42 & $\pmb{4.31}$$\pm$0.51  & $\pmb{7.42}$$\pm$1.21  & $\pmb{13.28}$$\pm$1.52 &  $\pmb{2.75}$$\pm$0.87 & $\pmb{5.01}$$\pm$1.91  \\
    \bottomrule
  \end{tabular}
\end{table}
\vspace{-\baselineskip}

\begin{table}[htp]
  \caption{Generalizability against CCDO-RL's adversary (on unseen graphs)}
  \vspace{\baselineskip}
  \label{tab_gene}
  \centering
  \small
  \begin{threeparttable}
  \begin{tabular}{lllllll}
    \toprule
    \multirow{2}{*}{method} & \multicolumn{2}{c}{ACSP (Mean$\pm$Std)} & \multicolumn{2}{c}{ACVRP (Mean$\pm$Std)} & \multicolumn{2}{c}{PG (Mean$\pm$Std)} \\
    \cmidrule(r){2-3} \cmidrule{4-5} \cmidrule(r){6-7}
                            & 20 nodes & 50 nodes & 20 nodes & 50 nodes & 20 nodes & 50 nodes\\
    \midrule
    Heuristic & 6.20$\pm$1.33 & 7.60$\pm$1.37   & 7.64$\pm$1.30  & 13.27$\pm$1.87 & 2.43$\pm$0.98 & 4.19$\pm$1.69    \\
    RL against Stoc  & 3.56$\pm$0.37  & 4.57$\pm$0.58  & 7.67$\pm$1.30  & 13.85$\pm$1.53 &  2.50$\pm$0.95 & 4.26$\pm$2.17 \\
    CCDO-RL   & $\pmb{3.31}$$\pm$0.35 & $\pmb{4.39}$$\pm$0.52  & $\pmb{7.55}$$\pm$1.28  & $\pmb{13.15}$$\pm$1.59 & $\pmb{2.56}$$\pm$0.92 & $\pmb{4.70}$$\pm$1.94\\

    \bottomrule
  \end{tabular}
  \begin{tablenotes}
      \footnotesize
      \item[1] For the average reward of ACSP and ACVRP, smaller is better while for that of PG larger is better.
  \end{tablenotes}
  \end{threeparttable}
\end{table}
\vspace{-\baselineskip}
% two heuristics and one RL
% \begin{table}[ht]
%   \caption{{\color{red} Average reward of CCDO-RL (on seen graphs). For the value of CSP and CVRP, larger is better while for that of PG smaller is better.}}
%   \label{tab_aver}
%   \centering
%   \small
%   \begin{tabular}{lllllll}
%     \toprule
%     \multirow{2}{*}{method} & \multicolumn{2}{c}{CSP (Mean$\pm$Std)} & \multicolumn{2}{c}{CVRP (Mean$\pm$Std)} & \multicolumn{2}{c}{PG (Mean$\pm$Std)} \\
%     \cmidrule(r){2-3} \cmidrule{4-5} \cmidrule(r){6-7}
%                             & 20 nodes & 50 nodes & 20 nodes & 50 nodes & 20 nodes & 50 nodes\\
%     \midrule
%     Baseline 1 & 4.52$\pm$0.71  & 5.98$\pm$0.94 & 7.64$\pm$1.56  & 13.49$\pm$2.10 & 2.71$\pm$1.10 & 1.82$\pm$1.40   \\
%     Baseline 2 & 6.13$\pm$1.20 & 7.55$\pm$1.42   & 7.65$\pm$1.23  & 13.38$\pm$1.70 & 2.64$\pm$1.03 & 1.47$\pm$0.99  \\
%     RL {\color{red}against Stoc}    & 3.50$\pm$0.47  & 4.55$\pm$0.62  & 7.55$\pm$1.16  & 13.90$\pm$1.63 & 2.71$\pm$0.90 & 1.54$\pm$1.03   \\
%     CCDO-RL   & $\pmb{3.25}$$\pm$0.42 & $\pmb{4.31}$$\pm$0.51  & $\pmb{7.42}$$\pm$1.21  & $\pmb{13.28}$$\pm$1.52 &  $\pmb{2.75}$$\pm$0.87 & $\pmb{1.87}$$\pm$1.22  \\
%     \bottomrule
%   \end{tabular}
% \end{table}


% \begin{table}[htp]
%   \caption{{\color{red}Generalizability of CCDO-RL (on unseen graphs)}}
%   \label{tab_gene}
%   \centering
%   \small
%   \begin{threeparttable}
%   \begin{tabular}{lllllll}
%     \toprule
%     \multirow{2}{*}{method} & \multicolumn{2}{c}{CSP (Mean$\pm$Std)} & \multicolumn{2}{c}{CVRP (Mean$\pm$Std)} & \multicolumn{2}{c}{PG (Mean$\pm$Std)} \\
%     \cmidrule(r){2-3} \cmidrule{4-5} \cmidrule(r){6-7}
%                             & 20 nodes & 50 nodes & 20 nodes & 50 nodes & 20 nodes & 50 nodes\\
%     \midrule
%     Baseline 1 & 4.53$\pm$0.79  & 5.95$\pm$0.96 & 7.55$\pm$1.39  & 13.35$\pm$2.04 & 2.52$\pm$1.08 & $\pmb{1.86}$$\pm$1.44  \\
%     Baseline 2 & 6.20$\pm$1.33 & 7.60$\pm$1.37   & 7.64$\pm$1.3  & 13.27$\pm$1.87 & 2.43$\pm$0.98 & 1.52$\pm$1.20    \\
%     RL {\color{red}against Stoc}  & 3.56$\pm$0.37  & 4.57$\pm$0.58  & 7.67$\pm$1.30  & 13.85$\pm$1.53 &  2.50$\pm$0.95 & 1.03$\pm$5.05 \\
%     CCDO-RL   & $\pmb{3.31}$$\pm$0.35 & $\pmb{4.39}$$\pm$0.52  & $\pmb{7.55}$$\pm$1.28  & $\pmb{13.15}$$\pm$1.59 & $\pmb{2.56}$$\pm$0.92 & 1.35$\pm$5.09\\

%     \bottomrule
%   \end{tabular}
%   \begin{tablenotes}
%       \footnotesize
%       \item[1] For the value of CSP and CVRP, larger is better while for that of PG smaller is better.
%   \end{tablenotes}
%   \end{threeparttable}
% \end{table}

\section*{Conclusion}
This paper aims to enhance our understanding of the computational complexity of computing various Shapley value variants. We found that for various ML models --- including decision trees, regression tree ensembles, weighted automata, and linear regression --- both local and global interventional and baseline SHAP can be computed in polynomial time under HMM modeled distributions. This extends popular algorithms, such as TreeSHAP, beyond their empirical distributional scope. We also establish strict complexity gaps between the various SHAP variants (baseline, interventional, and conditional) and prove the intractability of computing SHAP for tree ensembles and neural networks in simplified scenarios. Overall, we present SHAP as a versatile framework whose complexity depends on four key factors: \begin{inparaenum}[(i)] \item model type, \item SHAP variant, \item distribution modeling approach, \item and local vs. global explanations\end{inparaenum}. We believe this perspective provides deeper insight into the computational complexity of SHAP, paving the way for future work.




%We believe that our framework provides a more intricate understanding of SHAP computation complexity across different models, distributions, and variants, paving the way for further research.

Our work opens promising directions for future research. First, expanding our computational analysis to other SHAP-related metrics, such as asymmetric SHAP~\citep{frye20} and SAGE~\citep{covert2020understanding}, would be valuable. Additionally, we aim to explore more expressive distribution classes and relaxed assumptions beyond those in Section \ref{sec:tractable} while maintaining tractable SHAP computation. Finally, when exact computation is intractable (Section \ref{sec:intractable}), investigating the approximability of SHAP metrics through approximation and parameterized complexity theory~\citep{downey2012parameterized} is an important direction.

%Our work opens several promising avenues for future research on the computational properties of explainable AI methods, with a particular focus on SHAP. First, it would be interesting to broaden the computational analysis conducted in this work to include other popular SHAP-related metrics in the literature, such as asymmetric SHAP \cite{frye20} and SAGE \cite{covert2020understanding}. Also, in the future, we aim to explore more expressive distribution classes and relaxed distributional assumptions—extending beyond those examined in Section \ref{sec:tractable} —that still yield tractable SHAP computation. Finally, when exact computation proves intractable (Section \ref{sec:intractable}), it is worthwhile to theoretically investigate the question of the approximability of computing the SHAP metrics across various configurations, through the lens of approximation and parametrized complexity theory \cite{arora2009computational}.

%This paper aims to deepen our understanding of the computational complexity involved in obtaining different Shapley value variants. We found that for a variety of ML models, including decision trees, tree ensembles for regression, weighted automata, and linear regression models — computing both local and global interventional and baseline SHAP can be done in polynomial time when distributions are modeled by HMMs. This extends the distributional scope of popular algorithms like TreeSHAP, which is limited to empirical distributions. Additionally, we demonstrate a strict complexity gap between SHAP variants, showing that interventional and baseline SHAP can be strictly easier to compute than conditional SHAP. Despite these positive results, we uncovered intractability for various SHAP variants in neural networks and tree ensembles. Finally, we provided generalized complexity relations across SHAP variants. We believe that our framework offers a deeper understanding of the complexity involved in computing SHAP across various variants, models, distributions, as well as in both local and global computations, laying the groundwork for future research.

\bibliography{reference}
\bibliographystyle{plainnat}
\newpage
\appendix
\onecolumn 




\section{Complete Theoretical Derivations and Proofs}

Here, we provide the theoretical analysis that is deferred from the main text, including the following subsections:
Appendix \ref{app:posterior_sampling_and_energy_guided_sampling} includes proof of how energy-guided sampling from $p(x)e^{-J(x)} / Z$ is equivalent to conditional sampling from $p(x|y)$.
Appendix \ref{app:general_guidance} proves Theorem \ref{theorem:general_guidance} by showing $g_t+v_t$ is equal to $v'_t$ which generates the correctly guided probability path.
Appendix \ref{app:affine_gaussian_guidance_matching} Proof of that under uncoupled affine Gaussian paths, our general guidance $g_t$ is equivalent to the diffusion guidance $\nabla_{x_t} \log Z_t$.
Appendix \ref{app:proposition_match_anything} is proof of Proposition \ref{proposition:matching_anything}.
Appendix \ref{appendix:other_ways_to_learn_z} discusses other ways to obtain $Z_t$, including using contrastive learning and Monte Carlo estimation.
In appendix \ref{app:guidance_matching}, we propose three other training losses $\ell^{VGM}$, $\ell^{RGM}$, and $\ell^{MRGM}$ for $g_\phi$ and prove that the losses will produce the correct gradient.
Appendix \ref{app:independent_mc} includes a more detailed explanation of $g^{\text{MC}}$ and a variant of it under uncoupled paths.
Appendix \ref{app:localized_approximation} derives $g^{\text{local}}$ and proves its error bound.
Appendix \ref{appendix:x1_parameterization} shows how to estimate $\hat{x}_1$ using $x_t$ and $v_\theta(x_t,t)$ under the affine path assumption.
Appendix \ref{appendix:affine_path_cov_local_approx_guidance} proves that $g^{\text{local}}$ becomes $g^{\text{cov}}$ under affine paths.
Appendix \ref{appendix:jacobian_trick} includes the proof of the Jacobian trick (Proposition \ref{appendix:jacobian_trick}).
Appendix \ref{appendix:approximate_simple_posterior_pigdm_like} includes a proof of $g_t^{\text{sim-inv}}$ for image inverse problem, how to derive $g^{text{sum-inv-A}}$ and how to recover $\Pi$GDM under the uncoupled affine Gaussian path assumption.


\subsection{Energy Guided Sampling as Posterior Sampling}
\label{app:posterior_sampling_and_energy_guided_sampling}

There exists a $J(x)$ such that sampling from $\frac{1}{Z}p(x)e^{-J(x)}$ is equivalent to sampling from $p(x|y)$.

Simply take $J=-\log \frac{p(y|x)}{p(y)}$. Plug it in to get
\begin{equation}
    \frac{1}{Z}p(x)e^{-J(x)}=\frac{p(x)e^{\log \frac{p(y|x)}{p(y)}}}{\int p(x)e^{\log \frac{p(y|x)}{p(y)}}dx}=p(x|y).
\end{equation}

Similar approaches have been used in probability inference reinforcement learning \citep{levine_reinforcement_2018}, and Diffuser \citep{janner_planning_2022} uses this to convert return-conditioned sampling into energy-guided sampling.


\subsection{General Guidance}
\label{app:general_guidance}
We prove Theorem \ref{theorem:general_guidance} here.
\begin{theorem}
    Adding the guidance $g_t(x_t)$, where
    \begin{align}\label{eq:general_guidance_appendix}
        g_t(x_t) & = \int (\frac{e^{-J(x_1)}}{Z_t} - 1) v_{t|z}(x_t|z) \frac{p_t(x_t|z)p(z)}{p_t(x_t)} dz\\
        Z_t & = \int e^{-J(x_1)} p(z|x_t)dz
    \end{align}
    to $v_t(x_t)$ that generates $p_t(x_t) = \int p_t(x_t|z)p(z)dz$, will form a vector field $v'_t(x_t)$ that generates $p_t(x_t) = \int p_t(x_t|z)p'(z)dz$, where 
    $p'(z) = p(x_0|x_1) \frac{1}{Z}p(x_1)e^{-J(x_1)}$ has the same reverse coupling as in $p(z)$.
\end{theorem}

\textbf{Proof.}
We can subtract $v_t(x_t)$ from $v'_t(x_t)$ to construct $g_t(x_t)$:
\begin{align}
    g_t(x_t) & =  v'_t(x_t) - \int v_{t|z}(x_t|z) \frac{p_t(x_t|z)p(z)}{p_t(x_t)} dz.
\end{align}
One possible $v'_t(x_t)$ to generate $p_t(x_t) = \int p_t(x_t|z)p'(z)dz$ is
\begin{equation}
    v'_t(x_t) = \int v_{t|z}(x_t|z) \frac{p_t(x_t|z)p'(z)}{p'(x_t)} dz,
\end{equation}
where $p'(z) = p(x_0|x_1) \frac{1}{Z}p(x_1)e^{-J(x_1)}$, which follows from conditional flow matching, \emph{i.e.}, a VF marginalizing a conditional VF will generate the corresponding marginal probability path \citep{lipman_flow_2023,tong_improving_2024}.
Then, we have a possible form of $g_t(x_t)$
\begin{align}
    \nonumber g_t(x_t) & = \int v_{t|z}(x_t|z) (\frac{p_t(x_t|z)p'(z)}{p'_t(x_t)} - \frac{p_t(x_t|z)p(z)}{p_t(x_t)}) dz \\
    \nonumber& = \int v_{t|z}(x_t|z) (\frac{p_t(x_t|z)p(x_0|x_1)\frac{1}{Z}p(x_1)e^{-J(x_1)}}{p'_t(x_t)} - \frac{p_t(x_t|z)p(z)}{p_t(x_t)}) dz \\ 
    \nonumber& = \int v_{t|z}(x_t|z) (\frac{p_t(x_t|z)p(z)\frac{1}{Z}e^{-J(x_1)}}{p'_t(x_t)} - \frac{p_t(x_t|z)p(z)}{p_t(x_t)}) dz \\ 
    & = \int v_{t|z}(x_t|z) \frac{p_t(x_t|z)p(z)}{p_t(x_t)} (\frac{1}{Z}e^{-J(x_1)}\frac{p_t(x_t)}{p'_t(x_t)} - 1) dz, \label{eq:appendix:general_guidance_theorem_intermediate}
\end{align}
where $p_t(x_t) = \int p(x_t,z)dz$ and $p'(x_t) = \int p'(x_t,z)dz$.
Since
\begin{equation}
    \int p(x_t,z)dz = p(x_1) \int p(z|x_1)dz\label{eq:appendix:general_guidance_theorem_marginal_pt}
\end{equation}
and 
\begin{align}
     \nonumber\int p'(x_t,z)dz &= \int p'(x_t|z) p'(z)dz  \\
     \nonumber& = \int p(x_t|z) p(x_0|x_1) \frac{1}{Z} p(x_1) e^{-J(x_1)}dz \\ 
     \nonumber& = \int p(x_t|z) p(z) \frac{1}{Z} e^{-J(x_1)}dz \\ 
     \nonumber& = \int p(x_t,z) \frac{1}{Z} e^{-J(x_1)}dz \\ 
     & = p(x_t) \int p(z|x_t) \frac{1}{Z} e^{-J(x_1)}dz .\label{eq:appendix:general_guidance_theorem_marginal_ptprime}
\end{align}
Plugging Eq. \eqref{eq:appendix:general_guidance_theorem_marginal_pt} and Eq. \eqref{eq:appendix:general_guidance_theorem_marginal_ptprime} into Eq. \eqref{eq:appendix:general_guidance_theorem_intermediate}, we get 
\begin{align}
    \nonumber g_t(x_t) & = \int v_{t|z}(x_t|z) \frac{p_t(x_t|z)p(z)}{p_t(x_t)} (\frac{1}{Z}e^{-J(x_1)}\frac{p_t(x_t)}{p'_t(x_t)} - 1) dz \\ 
    \nonumber& = \int v_{t|z}(x_t|z) \frac{p_t(x_t|z)p(z)}{p_t(x_t)} (\cancel{\frac{1}{Z}}e^{-J(x_1)}\frac{\cancel{p(x_t)} \int p(z|x_t)dz}{\cancel{p(x_t)} \int p(z|x_t) \cancel{\frac{1}{Z}} e^{-J(x_1)}dz} - 1) dz \\ 
    \nonumber& = \int v_{t|z}(x_t|z) \frac{p_t(x_t|z)p(z)}{p_t(x_t)} (e^{-J(x_1)}\frac{1}{\int p(z|x_t) e^{-J(x_1)}dz}\underbrace{\int p(z|x_t)dz}_{=1} - 1) dz \\ 
    & = \int v_{t|z}(x_t|z) \frac{p_t(x_t|z)p(z)}{p_t(x_t)} (\frac{e^{-J(x_1)}}{\mathbb{E}_{z\sim p(z|x_t)}[e^{-J(x_1)}]} - 1) dz. 
\end{align}
Finally, denote $Z_t = \mathbb{E}_{z\sim p(z|x_t)}[e^{-J(x_1)}]$ to complete the proof.

\textit{Remark.} The theorem states that $v'_t = g_t + v_t$ not only generates the desired terminal distribution $\frac{1}{Z}p(x_1)e^{-J(x_1)}$ at time $t=1$, but also generates a probability path $p'_t(x_t)$ that is similar to the original one. Specifically, their "noising process" $p(x_t|x_1)$, and the conditional vector fields $v(x_t|x_1)$, and the reverse coupling $p(x_0|x_1)$ are the same. These are all hyperparameters of flow matching, as one can choose an arbitrary conditional vector field satisfying boundary conditions and the conditional vector field uniquely determines the conditional probability path; the reverse coupling, given target (dataset) distribution $p(x_1)$ or $\frac{1}{Z}p(x_1)e^{-J(x_1)}$, composes the data coupling $p(x_0,x_1) = p(z)$ for flow matching training.

It should be noted that the $g_t$ and $v'_t$ we construct here is only one of infinitely many \citep{lipman_flow_2023} possible vector fields to generate $\frac{1}{Z}p(x)e^{-J(x_1)}$ at $t=1$. It remains an interesting question whether there exists better $v'_t$ that, for example, simplifies $g_t$ or improves the vector field by straightening the flow.



\subsection{Uncoupled Affine Gaussian Guidance}
\label{app:affine_gaussian_guidance_matching}

Here we prove that $\nabla_{x_t} \log Z_t(x_t)$ is proportional to the guidance $g_t$ in Eq. \eqref{eq:general_guidance}. 
Note that the term $\nabla_{x_t} \log Z_t(x_t)$ is widely used as guidance in the diffusion model literature \citep{dhariwal_diffusion_2021,ho_classifier-free_2022,song_score-based_2021,song_loss-guided_2023,song_pseudoinverse-guided_2022,janner_planning_2022,ajay_is_2023}. Therefore, we prove that our general flow matching guidance exactly covers the original diffusion guidance under the affine Gaussian path assumption, \emph{i.e.}, when flow matching falls back to the diffusion model. Our proof here also elucidates how the gradient $\nabla_{x_t}$ emerges from the original expression of the general guidance for flow matching in Eq. \eqref{eq:general_guidance} where there is no apparent gradient.

We restate Eq. \eqref{eq:general_guidance} here:
\begin{align}\label{eq:appendix_restate:general_guidance}
    g_t(x_t) & = \int (\frac{e^{-J(x_1)}}{Z_t(x_t)} - 1) v_{t|z}(x_t|z) \frac{p_t(x_t|z)p(z)}{p_t(x_t)} dz \\
    Z_t(x_t) & =  \int e^{-J(x_1)} \frac{p_t(x_t|z)p(z)}{p_t(x_t)} dz
\end{align}

Assuming the flow matching to be of uncoupled affine path, we have
\begin{equation}
    x_t = \sigma_t x_0 + \alpha_t x_1,
\end{equation}
where $\sigma_t$ and $\alpha_t$ are schedulers satisfying boundary conditions $\sigma_0 = \alpha_1 = 1$, $\sigma_1 = \alpha_0 = 0$.
Thus, 
\begin{align}
\nonumber
    v_{t|z}(x_t|z) & = \dot\sigma_t x_0 + \dot\alpha_t x_1 
    \\
    \label{eq:appendix:general_guidance_and_diffusion_guidance_affine_path}
    & =  \underbrace{\frac{\dot\sigma_t}{\sigma_t}}_{a_t} x_t + \underbrace{\frac{1}{\sigma_t} (\dot\alpha_t \sigma_t - \dot\sigma_t \alpha_t)}_{b_t} x_1 
\end{align}
where $\dot f_t \coloneqq \frac{d f}{d t}$ denotes derivative to time $t$, and we define $a_t\coloneqq \frac{\dot\sigma_t}{\sigma_t}$, $b_t\coloneqq \frac{1}{\sigma_t} (\dot\alpha_t \sigma_t - \dot\sigma_t \alpha_t)$.

First, we demonstrate a useful technique for the proof later.
Inserting Eq. \eqref{eq:appendix:general_guidance_and_diffusion_guidance_affine_path} into $g_t(x_t)$ in Eq. \eqref{eq:appendix_restate:general_guidance} and we have
\begin{align}\label{eq:appendix:general_guidance_expanded_v}
    g_t(x_t) = \int (\frac{e^{-J(x_1)}}{Z_t} - 1) (a_t x_t + b_t x_1) \frac{p_t(x_t|z)p(z)}{p_t(x_t)} dz.
\end{align}
Since $Z_t = \mathbb{E}_{z\sim p(z|x_t)}[e^{-J(x_1)}]$, 
\begin{equation}
    \int (\frac{e^{-J(x_1)}}{Z_t} - 1) a_t x_t \frac{p_t(x_t|z)p(z)}{p_t(x_t)} dz = 0.
    \label{eq:appendix:general_guidance_and_diffusion_guidance_integrate_to_zero}
\end{equation}
That is to say, $x_t$ inside the integral of Eq. \eqref{eq:appendix:general_guidance_and_diffusion_guidance_integrate_to_zero} will integrate to zero, and we can freely remove or add it to construct desired terms.

Recall the assumption of uncoupled Gaussian path, \emph{i.e.} $p(x_0,x_1) = p(x_0) p(x_1)$, $p(x_0) = \mathcal{N}(x_0;0,I)$. We can utilize the important fact that the conditional probability path for affine Gaussian path flows satisfies $x_t \sim \mathcal{N}(x_t;\alpha_t x_1, \sigma_t^2 I)$, which allows us to connect the conditional score to $x_1$
\begin{equation}\label{eq:appendix:general_guidance_and_diffusion_guidance_score_and_affine_path}
    \nabla_{x_t} \log p(x_t|x_1) = -\frac{x_t - \alpha_t x_1}{\sigma_t ^2}.
\end{equation}
Using Eq. \eqref{eq:appendix:general_guidance_and_diffusion_guidance_integrate_to_zero} and Eq. \eqref{eq:appendix:general_guidance_and_diffusion_guidance_score_and_affine_path}, Eq. \eqref{eq:appendix:general_guidance_expanded_v} can be further converted to 
\begin{align}\label{eq:appendix:general_guidance_and_diffusion_guidance_convert_conditional_v_to_score}
    g_t(x_t)& = \int (\frac{e^{-J(x_1)}}{Z_t} - 1) (b_t x_1 \underbrace{- \frac{b_t}{\alpha_t} x_t}_{\text{Eq. \eqref{eq:appendix:general_guidance_and_diffusion_guidance_integrate_to_zero}}}) \frac{p_t(x_t|x_1)p(x_1)}{p_t(x_t)} dx_1 \\ \nonumber
    & = \frac{b_t \sigma_t^2}{\alpha_t} \int (\frac{e^{-J(x_1)}}{Z_t} - 1) \nabla_{x_t} \log p(x_t|x_1) \frac{p_t(x_t|x_1)p(x_1)}{p_t(x_t)} dx_1 \\\nonumber
    & = \frac{b_t \sigma_t^2}{\alpha_t} \int (\frac{e^{-J(x_1)}}{Z_t} - 1) \left(\underbrace{\nabla_{x_t} \log p(x_t) + \nabla_{x_t} \log p(x_1 | x_t)}_{\text{Bayes' rule}}\right) \frac{p_t(x_t|x_1)p(x_1)}{p_t(x_t)} dx_1
    \\
    & = \frac{b_t \sigma_t^2}{\alpha_t} \int (\frac{e^{-J(x_1)}}{Z_t} - 1) \left(\underbrace{\cancel{\nabla_{x_t} \log p(x_t)}}_{\text{Integrates to zero as in Eq. \eqref{eq:appendix:general_guidance_and_diffusion_guidance_integrate_to_zero}}} + \nabla_{x_t} \log p(x_1 | x_t)\right) \frac{p_t(x_t|x_1)p(x_1)}{p_t(x_t)} dx_1 \\
    & = \frac{b_t \sigma_t^2}{\alpha_t} \int (\frac{e^{-J(x_1)}}{Z_t} - 1)  \nabla_{x_t} \log p(x_1 | x_t) \frac{p_t(x_t|x_1)p(x_1)}{p_t(x_t)} dx_1. \label{eq:appendix:general_guidance_and_diffusion_guidance_samplified_score}
\end{align}
Notice in Eq. \eqref{eq:appendix:general_guidance_and_diffusion_guidance_samplified_score} that 
\begin{align}\label{eq:appendix:general_guidance_and_diffusion_guidance_canceling_expectation_of_score}
    \nonumber
    &\int \nabla_{x_t} \log p(x_1 | x_t) \frac{p_t(x_t|x_1)p(x_1)}{p_t(x_t)} dx_1 \\
    \nonumber= & \int \underbrace{p(x_1|x_t)  \nabla_{x_t} \log p(x_1 | x_t)}_{\text{Since } f \nabla  \log f = \nabla f}  dx_1 \\
    \nonumber= & \int \nabla_{x_t} p(x_1 | x_t)  dx_1 \\
    \nonumber= &\, \nabla_{x_t} \int p(x_1 | x_t)  dx_1 \\
    = &\,0.
\end{align}
Therefore 
\begin{align}
    \nonumber g_t(x_t) & = -\frac{b_t \sigma_t^2}{\alpha_t} \int (\frac{e^{-J(x_1)}}{Z_t} \cancel{-1})  \nabla_{x_t} \log p(x_1 | x_t) \frac{p_t(x_t|x_1)p(x_1)}{p_t(x_t)} dx_1 \\
    \nonumber& = \frac{b_t \sigma_t^2}{\alpha_t} \int \frac{e^{-J(x_1)}}{Z_t(x_t)} \underbrace{p_t(x_1|x_t) \nabla_{x_t} \log p(x_1 | x_t)}_{\text{Using again $f \nabla \log f = \nabla f$}}  dx_1 \\
    \nonumber& = \frac{b_t \sigma_t^2}{\alpha_t} \frac{1}{Z_t(x_t)} \int e^{-J(x_1)} \nabla_{x_t} p(x_1 | x_t)  dx_1 \\
    \nonumber& = \frac{b_t \sigma_t^2}{\alpha_t} \frac{1}{Z_t(x_t)}\underbrace{\nabla_{x_t} \int e^{-J(x_1)} p(x_1 | x_t)  dx_1}_{\text{Absorb $e^{-J(x_1)}$ for it is independent of $x_1$, and exchange with integral}} \\
    \nonumber& = \frac{b_t \sigma_t^2}{\alpha_t} \frac{1}{Z_t(x_t)} \nabla_{x_t} \underbrace{Z_t(x_t)}_{\text{$Z_t$'s definition in Eq. \eqref{eq:general_guidance}}} \\
    & = \frac{b_t \sigma_t^2}{a_t} \nabla_{x_t} \log Z_t(x_t).
\end{align}

Another possible way to derive this is to first prove the vector field in affine Gaussian path flow matching to be affine to the marginal score, and we direct interested readers to \citep{zheng_guided_2023}. 

\begin{remark}
    The above derivation opens the possibility of using diffusion guidance into affine Gaussian path flow matching, \emph{i.e.}, by multiplying a scheduler $-\frac{b_t \sigma_t^2}{\alpha_t} = \frac{\sigma_t(\dot\alpha_t\sigma_t - \dot \sigma_t\alpha_t)}{\alpha_t} $ to the diffusion classifier guidance. The most common scheduler for flow matching is $\sigma_t = 1 - t, \alpha_t = t$ \citep{lipman_flow_2023,pokle_training-free_2024,zheng_guided_2023,tong_improving_2024,liu_flow_2022,lipman_flow_2024}. In this case, the guidance scheduler is $\frac{(1-t)}{t}$. It should be noted that this scheduler explodes near $t=0$, thus being unstable. The flow matching schedule $\sigma_t$ and $\alpha_t$ can be chosen as other ways to avoid this instability.
\end{remark} 

\begin{remark}
    Note that this guidance cannot be applied to coupled paths. Central to the proof is that in uncoupled affine Gaussian paths, we can convert the conditional vector field to the conditional score.
    If we could do this in coupled paths, we would require (1) $p_t(x|z)$ is Gaussian $\mathcal{N}(x;\mu_t,\sigma_t I)$, such that $\nabla_{x_t} \log p_t(x_t|z) \propto x_t - \mu_t$.  and (2) $v_{t|z}=\dot\mu_t + \dot \sigma_t (x_t - \mu_t) \propto \mu_t$, such that in Eq. \eqref{eq:appendix:general_guidance_and_diffusion_guidance_convert_conditional_v_to_score} the conditional vector field can be converted to the conditional score. Therefore, the following equation must hold \emph{for any $x_t,x_0,x_1$} 
    \begin{equation}
        \dot\mu_t + \dot \sigma_t (x_t - \mu_t) \propto \mu_t
    \end{equation}
    inside the integral of Eq. \eqref{eq:appendix:general_guidance_and_diffusion_guidance_convert_conditional_v_to_score}
    where $\mu_t = \xi_t x_0 + \eta_t x_1$. This equivalent to that
    \begin{equation}
        (\dot\xi_t - \dot \sigma_t \xi) x_0 + (\dot\eta_t - \dot \sigma_t \eta_t) x_1 +  \dot \sigma_t x_t \propto \xi_t x_0 + \eta_t x_1
    \end{equation}
    must hold \emph{for any $x_t,x_0,x_1$} inside the integral of Eq. \eqref{eq:appendix:general_guidance_and_diffusion_guidance_convert_conditional_v_to_score}. According to Eq. \eqref{eq:appendix:general_guidance_and_diffusion_guidance_integrate_to_zero}, $x_t$ terms will integrate to zero, thus
    \begin{equation}
        \frac{\dot\xi_t - \dot \sigma_t \xi}{\dot\eta_t - \dot \sigma_t \eta_t} = \frac{\xi_t}{\eta_t}
    \end{equation}
    which cannot hold: because of the boundary conditions $\xi_0 = \eta_1=1$ and $\xi_1=\eta_0 = 0$, $\exists t\in(0,1),\frac{d \log \xi_t}{ dt} \neq \frac{d \log \eta_t}{ dt}$. %
    It can be observed that the reason why this guidance does not apply to coupled paths is that $x_0,x_1$, and $x_t$ are all independent variables here, preventing us from canceling two of $x_0$ to avoid matching the schedulers' ratio.
    
    
\end{remark}




\subsection{Proof of Proposition \ref{proposition:matching_anything}}
\label{app:proposition_match_anything}
We prove proposition \ref{proposition:matching_anything} here.
\begin{proposition}
    Any \emph{marginal variable} $f(x_t,t)\coloneqq \mathbb{E}_{z\sim p_t(z|x_t)}[f_{t|z}(x_t,z,t)],~z=(x_0,x_1)$ has an intractable \emph{marginal loss}
    \begin{equation}\label{eq:app_proposition_matching_anything_marginal_loss}
        \mathcal{L}=\mathbb{E}_{x_t\sim p(x_t)}\left[\left\|f_{{\theta}}(x_t,t) - \mathbb{E}_{z\sim p_t(z|x_t)}[f_{t|z}(x_t,z,t)]\right\|_2^2\right],
    \end{equation}
    whose gradient is identical to the tractable \emph{conditional loss}
    \begin{equation}
        \mathcal{L}_{t|z}=\mathbb{E}_{x_t,z\sim p(x_t,z)}\left[\left\|f_{\theta}(x_t,t) - f_{t|z}(x_t,z,t)\right\|_2^2\right].
    \end{equation}
\end{proposition}

\textbf{Proof.} Expand and take gradient w.r.t. Eq. \eqref{eq:app_proposition_matching_anything_marginal_loss} to get
\begin{align}
    \nonumber\nabla_\theta \mathcal{L}_t
    &=\nabla_\theta \mathbb{E}_{x_t\sim p(x_t)}\left[\left\|f_{{\theta}}(x_t,t) - \mathbb{E}_{z\sim p_t(z|x_t)}[f(x_t,z,t)]\right\|_2^2\right] \\
    \nonumber&=\mathbb{E}_{x_t\sim p(x_t)}\left[\nabla_\theta \left\|f_{{\theta}}(x_t,t) - \mathbb{E}_{z\sim p_t(z|x_t)}[f(x_t,z,t)]\right\|_2^2\right] \\
    \nonumber&=\int \nabla_\theta p(x_t) \left\|f_{{\theta}}(x_t,t) - \mathbb{E}_{z\sim p_t(z|x_t)}[f(x_t,z,t)]\right\|_2^2 dx_t \\
    \nonumber&=\int \nabla_\theta p(x_t) \left(\|f_{{\theta}}(x_t,t)\|^2 - 2 \langle f_{{\theta}}(x_t,t),  \int p_t(z|x_t) dz f(x_t,z,t) \rangle\right)  dx_t \\ 
    \nonumber&=\int \nabla_\theta  p_t(z|x_t) p(x_t) \left(\|f_{{\theta}}(x_t,t)\|^2 - 2 \langle f_{{\theta}}(x_t,t),  f(x_t,z,t) \rangle\right)  dx_t dz \\ 
    \nonumber&=\int  p_t(z|x_t) p(x_t) \nabla_\theta  \left(\|f_{{\theta}}(x_t,t)\|^2 - 2 \langle f_{{\theta}}(x_t,t),  f(x_t,z,t) \rangle\right)  dx_t dz \\ 
    \nonumber&=\mathbb{E}_{z,x_t\sim p_t(z|x_t) p(x_t)} \left[ \nabla_\theta \left\|f_{{\theta}}(x_t,t) - f(x_t,z,t) \right\|^2 \right]\\ 
    &=\nabla_\theta \underbrace{\mathbb{E}_{z,x_t\sim p_t(z|x_t) p(x_t)} \left[ \left\|f_{{\theta}}(x_t,t) - f(x_t,z,t) \right\|^2 \right]}_{\mathcal{L}_{1|z}}.
\end{align}
Thus, the gradient of the \emph{marginal loss} $\mathcal{L}_{t}$ is identical to the gradient of the \emph{conditional loss} $\mathcal{L}_{t|z}$.


\subsection{Other Ways to Obtain $Z_{\phi_Z}$}
\label{appendix:other_ways_to_learn_z}

\citet{lu_contrastive_nodate} proposed to use contrastive learning to train $Z_{\phi_Z}$. The proof already applies to any uncoupled path, and we show that $Z_{\phi_Z}$ does not depend on the coupling
\begin{align}
    Z_t =& \mathbb{E}_{x_1\sim p(x_0,x_1|x_t)}[e^{-J(x_1)}]= \int e^{-J(x_1)}p(x_0|x_1,x_t)p(x_1|x_t) dx_0 dx_1 \\
    =& \int e^{-J(x_1)}p(x_1|x_t) dx_1 = \mathbb{E}_{x_1\sim p(x_1|x_t)}[e^{-J(x_1)}].
\end{align}
That is, instead of actually sampling from $p(x_0,x_1|x_t)$, sampling from $p(x_1|x_t)$ will result in the same $Z_t$. In the case of the coupled path, the marginalized distribution is identical to the uncoupled path case. Therefore, the contrastive learning method can be readily applied to train $Z_{\phi_Z}$.

Besides training-based $Z_{\phi_Z}$, we can also use Monte Carlo estimation to obtain $Z_t$. Notice that by using importance sampling, we have
\begin{equation}
     Z_t = \mathbb{E}_{x_1\sim p(x_0,x_1|x_t)}[e^{-J(x_1)}] = \mathbb{E}_{x_1\sim p(x_0,x_1)}\left[\frac{p(x_t|x_0,x_1)}{p(x_t)}e^{-J(x_1)}\right].
\end{equation}
As long as $p(x_t|x_0,x_1)$ is known (which is often the case \citep{lipman_flow_2023,tong_improving_2024}), we can estimate $Z_t$ by sampling $N$ pairs of $x_0^i,x_1^i$ from $p(x_0,x_1)$ and estimate
\begin{equation}
    \tilde{Z}_t = \sum_i^N \left(\frac{p(x_0^i,x_1^i|x_t)}{\sum_j^N p(x_0^j,x_1^j|x_t)}e^{-J(x_1^i)}\right).
\end{equation}
A similar technique is used in Section \ref{sec:g_mc}.


\subsection{Guidance Matching Losses}
\label{app:guidance_matching}
Here, we prove that the loss in guidance matching is correct and show there are three other equivalent training losses $\ell^{\text{VGM}},\ell^{\text{RGM}},\ell^{\text{MRGM}}$.
The expressions of different losses are summarized below, and their proof follows.

\paragraph{VF-added Guidance Matching (VGM) Loss.} By utilizing the learned VF $v_\theta(x_t, t)$ into Eq. \eqref{eq:guidance_matching_loss_g_1}, we have 
\begin{equation}
    \ell_\phi^{\text{VGM}} = \left\|g_{\phi}(x_t,t) + v_\theta(x_t,t) - \frac{e^{-J(x_1)}}{Z_{\phi_Z,sg}(x_t,t)}v_{t|z}(x_t|z)\right\|_2^2.
\end{equation}

\paragraph{Reweight Guidance Matching (RGM) Loss.} $\ell_\phi^{\text{VGM}}$ can be further shown equivalent to 
\begin{equation}\label{eq:guidance_matching_loss_g_3_loss}
    \ell_\phi^{\text{RGM}}=\frac{e^{-J(x_1)}}{Z_{\phi_Z,sg}(x_t,t)} \left\|g_{\phi}(x_t,t) + v_\theta(x_t,t) - v_{t|z}(x_t|z)\right\|_2^2.
\end{equation}
This training loss steers $g_\phi$ to where $e^{-J(x_1)}$ is larger by assigning a large loss to steer $g_t$ towards high $e^{-J(x_1)}$ regions.

\paragraph{Marginalized Reweight Guidance Matching (MRGM) Loss.} The above loss can be re-assigned a weight, which will result in the same optimal $g_{\phi_2}(x_t,t)$ as in Eq. \eqref{eq:guidance_matching_loss_g_1}. Specifically, by changing $Z_{\phi_Z,sg}(x_t,t)$ to its expectation under $p_t(x_t)$, we have the following equivalent loss
\begin{equation}\label{eq:guidance_matching_loss_g_4_loss}
    \ell_\phi^{\text{MRGM}}=\frac{e^{-J(x_1)}}{Z} \|g_{\phi_2}(x_t,t) + v_\theta(x_t,t) - v_{t|z}(x_t|z)\|_2^2,
\end{equation}
where $Z = \mathbb{E}_{x_1\sim p(x_1)}[e^{-J(x_1)}]$.
$\ell_\phi^{\text{MRGM}}$ is identical to a newly proposed fine-tuning loss in \citet{anonymous2025energyweighted}. It can also be derived via importance sampling in Eq. \ref{eq:guidance_matching_loss_sum} and
similar reweighting-based fine-tuning losses have been studied in the literature of diffusion models \citep{fan_dpok_2023}.


\paragraph{(1) Guidance Matching Loss $\ell^{\text{GM}}$}
By using proposition \ref{proposition:matching_anything}, the following conditional loss
\begin{equation}\label{eq:appendix:guidance_matching_loss_g_1}
    \mathcal{L}_{\phi}^{\text{GM}} = \mathbb{E}_{t\sim\mathcal{U}(0,1),z\sim p(z),x_t \sim p(x_t|z)}\left[\underbrace{\left\|g_{\phi}(x_t,t) - (\frac{e^{-J(x_1)}}{Z_{\phi_Z,sg}(x_t,t)} - 1) v_{t|z}(x_t|z)\right\|_2^2}_{=\ell^{\text{GM}}}\right]
\end{equation}
has a gradient that is equivalent to the marginal loss
\begin{equation}
    \mathbb{E}_{t\sim\mathcal{U}(0,1),z\sim p(z)x_t \sim p(x_t|z)}\left[\left\|g_{\phi}(x_t,t) - \underbrace{\mathbb{E}_{z\sim p(z|x_t)}\left[(\frac{e^{-J(x_1)}}{Z_{\phi_Z,sg}(x_t,t)} - 1) v_{t|z}(x_t|z)\right]}_{=g_t(x_t)}\right\|_2^2\right].
\end{equation}
Therefore, using the loss $\mathcal{L}_{\phi_Z}$ we can train $g_\phi$ to matching $g_t$. Recall that $\mathcal{L}$ in Eq. \eqref{eq:guidance_matching_loss_g_1} is identical
to $\mathcal{L}_{\phi_Z}$, and we proved the validity of the guidance matching training.

\paragraph{(2) VF-added Guidance Matching Loss $\ell^{\text{VGM}}$.} By replacing the learned VF $v_\theta(x_t, t)$ into Eq. \eqref{eq:appendix:guidance_matching_loss_g_1}, we show that  
\begin{equation}
\label{eq:appendix:guidance_matching_loss_g_2}
    \mathcal{L}_{\phi}^{\text{VGM}} = \mathbb{E}_{t\sim\mathcal{U}(0,1),z\sim p(z),x_t \sim p(x_t|z)}\left[\left\|g_{\phi}(x_t,t) + v_\theta(x_t,t) - \frac{e^{-J(x_1)}}{Z_{\phi_Z,sg}(x_t,t)}v_{t|z}(x_t|z)\right\|_2^2\right],
\end{equation}
has a gradient equal to that of $\mathcal{L}_{\phi}$ in Eq. \eqref{eq:appendix:guidance_matching_loss_g_1}.

Expand Eq. \eqref{eq:appendix:guidance_matching_loss_g_1} to get
\begin{align}
    \nonumber\mathcal{L}_{\phi}^{\text{GM}} & = \mathbb{E}_{t\sim\mathcal{U}(0,1),z\sim p(z),x_t \sim p(x_t|z)}\left[\left\|g_{\phi}(x_t,t) - (\frac{e^{-J(x_1)}}{Z_{\phi_Z,sg}(x_t,t)} - 1) v_{t|z}(x_t|z)\right\|_2^2\right] \\
    \nonumber & = \mathbb{E}_{t\sim\mathcal{U}(0,1),z\sim p(z),x_t \sim p(x_t|z)}[\|\underbrace{g_{\phi}(x_t,t)\|_2^2}_{\text{dependent on }\phi} + \|\frac{e^{-J(x_1)}}{Z_{\phi_Z,sg}(x_t,t)}v_{t|z}(x_t|z)\|_2^2 + \|v_{t|z}(x_t|z)\|_2^2 \\
    \label{eq:appendix:guidance_matching_loss_g1_expanded}& \underbrace{-2\langle g_{\phi}(x_t,t), \frac{e^{-J(x_1)}}{Z_{\phi_Z,sg}(x_t,t)}v_{t|z}(x_t|z)\rangle}_{\text{dependent on }\phi} - 2\langle\frac{e^{-J(x_1)}}{Z_{\phi_Z,sg}(x_t,t)}v_{t|z}(x_t|z), v_{t|z}(x_t|z)\rangle \underbrace{- 2\langle v_{t|z}(x_t|z), g_{\phi}(x_t,t)\rangle}_{\text{dependent on }\phi} ]. 
\end{align}
After taking gradient w.r.t. $\phi$, only the terms 
\begin{equation}
    \nabla_{\phi}\mathbb{E}_{t\sim\mathcal{U}(0,1),z\sim p(z),x_t \sim p(x_t|z)}
    \left[
    \|g_{\phi}(x_t,t)\|_2^2 - 2\langle g_{\phi}(x_t,t), \frac{e^{-J(x_1)}}{Z_{\phi_Z,sg}(x_t,t)}v_{t|z}(x_t|z)\rangle
    - 2\langle v_{t|z}(x_t|z), g_{\phi}(x_t,t)\rangle
    \right]
\end{equation}
survive. Therefore, by assuming a perfectly learned $v_\theta(x_t,t)$, \emph{i.e.}, 
\begin{equation}
    v_\theta(x_t,t) = \mathbb{E}_{z\sim p(z|x_t)}\left[ v_{t|z}(x_t|z) \right],
\end{equation}
we have 
\begin{align}
\nonumber&\mathbb{E}_{t\sim\mathcal{U}(0,1),z\sim p(z),x_t \sim p(x_t|z)}
\left[
    \langle v_{t|z}(x_t|z), g_{\phi}(x_t,t)\rangle
\right]  \\
\nonumber= &
\mathbb{E}_{t\sim\mathcal{U}(0,1),z\sim p(z|x_t),x_t \sim p(x_t)}
\left[
    \langle v_{t|z}(x_t|z), g_{\phi}(x_t,t)\rangle
\right] \\
\nonumber= & \mathbb{E}_{\tilde{z}\sim p(\tilde{z}|x_t)}\left[\mathbb{E}_{t\sim\mathcal{U}(0,1),z\sim p(z|x_t),x_t \sim p(x_t)}
\left[
    \langle v_{t|z}(x_t|z), g_{\phi}(x_t,t)\rangle
\right]
\right] \\
\nonumber= & \mathbb{E}_{\tilde{z}\sim p(\tilde{z}|x_t)}\left[\mathbb{E}_{t\sim\mathcal{U}(0,1),x_t \sim p(x_t)}
\left[
    \langle \mathbb{E}_{z\sim p(z|x_t)} [v_{t|z}(x_t|z)], g_{\phi}(x_t,t)\rangle
\right]
\right] \\
\nonumber= & \mathbb{E}_{z\sim p(z|x_t)}\left[\mathbb{E}_{t\sim\mathcal{U}(0,1),x_t \sim p(x_t)}
\left[
    \langle \mathbb{E}_{\tilde{z}\sim p(\tilde{z}|x_t)} [v_{t|z}(x_t|z)], g_{\phi}(x_t,t)\rangle
\right]
\right] \\
\label{eq:appendix:guidance_matching_loss_g_2_v_term}
= & \mathbb{E}_{t\sim\mathcal{U}(0,1),z\sim p(z|x_t),x_t \sim p(x_t)}
\left[
    \langle v_\theta(x_t,t), g_{\phi}(x_t,t)\rangle
\right], 
\end{align}
so by adding back terms that the gradient is agnostic to, we can see that the new loss  $\mathcal{L}_{\phi}^{(1)}$ in Eq. \eqref{eq:appendix:guidance_matching_loss_g_2} is equivalent to $\mathcal{L}_{\phi}$ in Eq. \eqref{eq:appendix:guidance_matching_loss_g_1}
\begin{align}
    \nonumber\nabla_{\phi}{\mathcal{L}}_{\phi}^{\text{GM}} & = \nabla_{\phi}\mathbb{E}_{t\sim\mathcal{U}(0,1),z\sim p(z),x_t \sim p(x_t|z)}\left[\left\|g_{\phi}(x_t,t) - (\frac{e^{-J(x_1)}}{Z_{\phi_Z,sg}(x_t,t)} - 1) v_{t|z}(x_t|z)\right\|_2^2\right] \\
    \nonumber& = \nabla_{\phi}\mathbb{E}_{t\sim\mathcal{U}(0,1),z\sim p(z),x_t \sim p(x_t|z)}[\|{g_{\phi}(x_t,t)\|_2^2} + \|\frac{e^{-J(x_1)}}{Z_{\phi_Z,sg}(x_t,t)}v_{t|z}(x_t|z)\|_2^2 + \underbrace{\|v_{\theta}(x_t,t)\|_2^2}_{\text{Vanishes after $\nabla_\phi$}} \\
    & {-2\langle g_{\phi}(x_t,t), \frac{e^{-J(x_1)}}{Z_{\phi_Z,sg}(x_t,t)}v_{t|z}(x_t|z)\rangle} - 2\langle\frac{e^{-J(x_1)}}{Z_{\phi_Z,sg}(x_t,t)}v_{t|z}(x_t|z), v_{t|z}(x_t|z)\rangle \underbrace{- 2\langle v_\theta(x_t,t), g_{\phi}(x_t,t)\rangle}_{\text{Changed $v_{t|z}$ to $v_\theta$ using Eq. \eqref{eq:appendix:guidance_matching_loss_g_2_v_term}}} ]\\
    \nonumber& =\nabla_{\phi} \mathbb{E}_{t\sim\mathcal{U}(0,1),z\sim p(z),x_t \sim p(x_t|z)}\left[\underbrace{\left\|g_{\phi}(x_t,t) + v_\theta(x_t,t) - \frac{e^{-J(x_1)}}{Z_{\phi_Z,sg}(x_t,t)}v_{t|z}(x_t|z)\right\|_2^2}_{\coloneqq\ell^{\text{VGM}}}\right]\\
    & = \nabla_{\phi} {\mathcal{L}}_{\phi}^{\text{VGM}}.
\end{align}


\paragraph{(3) Reweighted Guidance Matching Loss $\ell^{\text{RGM}}$.} Eq. Replacing $\ell^{\text{GM}}$ in \eqref{eq:appendix:guidance_matching_loss_g_1} with $\ell^{\text{VGM}}$: 
\begin{equation}\label{eq:appendix:guidance_matching_loss_g_3_loss}
    \frac{e^{-J(x_1)}}{Z_{\phi_Z,sg}(x_t,t)} \|g_{\phi}(x_t,t) + v_\theta(x_t,t) - v_{t|z}(x_t|z)\|_2^2,
\end{equation}
and the loss $\mathcal{L}^{\text{GM}}_\phi$ becomes $\mathcal{L}^{\text{VGM}}_\phi$, which are shown equivalent in the following.

Starting from Eq. \eqref{eq:appendix:guidance_matching_loss_g1_expanded}, we can extract $\frac{e^{-J(x_1)}}{Z_{\phi_Z}}$ from the three terms depended on $\phi$, and thus showing the resulting loss is indeed $\ell^{\text{RGM}}$.
Notice that because $Z_{\phi_Z} = \mathbb{E}_{z\sim p(z|x_t)}[e^{-J(z)}]$,
\begin{equation}
    \mathbb{E}_{z\sim p(z|x_t)}[\frac{e^{-J(z)}}{Z_{\phi_Z}(x_t)}f(x_t,t)] = f(x_t,t).
\end{equation}
Thus, we have:
\begin{align}
    \nonumber \mathcal{L}^{\text{GM}}_\phi& = \mathbb{E}_{t\sim\mathcal{U}(0,1),z\sim p(z),x_t \sim p(x_t|z)}[\|\underbrace{g_{\phi}(x_t,t)\|_2^2}_{\text{dependent on }\phi} + \|\frac{e^{-J(x_1)}}{Z_{\phi_Z,sg}(x_t,t)}v_{t|z}(x_t|z)\|_2^2 + \|v_{t|z}(x_t|z)\|_2^2 \\
    \nonumber& \underbrace{-2\langle g_{\phi}(x_t,t), \frac{e^{-J(x_1)}}{Z_{\phi_Z,sg}(x_t,t)}v_{t|z}(x_t|z)\rangle}_{\text{dependent on }\phi} - 2\langle\frac{e^{-J(x_1)}}{Z_{\phi_Z,sg}(x_t,t)}v_{t|z}(x_t|z), v_{t|z}(x_t|z)\rangle \underbrace{- 2\langle v_{t|z}(x_t|z), g_{\phi}(x_t,t)\rangle}_{\text{dependent on }\phi} ] \\
   \nonumber & =\mathbb{E}_{t\sim\mathcal{U}(0,1),z\sim p(z),x_t \sim p(x_t|z)}[\|\frac{e^{-J(x_1)}}{Z_{\phi_Z}(x_t)}{g_{\phi}(x_t,t)\|_2^2} + \|\frac{e^{-J(x_1)}}{Z_{\phi_Z,sg}(x_t,t)}v_{t|z}(x_t|z)\|_2^2 + \|v_{t|z}(x_t|z)\|_2^2 \\
    \nonumber& {-2\langle g_{\phi}(x_t,t), \frac{e^{-J(x_1)}}{Z_{\phi_Z,sg}(x_t,t)}v_{t|z}(x_t|z)\rangle} - 2\langle\frac{e^{-J(x_1)}}{Z_{\phi_Z,sg}(x_t,t)}v_{t|z}(x_t|z), v_{t|z}(x_t|z)\rangle - 2\frac{e^{-J(x_1)}}{Z_{\phi_Z}(x_t)}\langle v_{t|z}(x_t|z), g_{\phi}(x_t,t)\rangle ] \\
    & = \mathbb{E}_{t\sim\mathcal{U}(0,1),z\sim p(z),x_t \sim p(x_t|z)} \left[\underbrace{\frac{e^{-J(x_1)}}{Z_{\phi_Z,sg}(x_t,t)} \|g_{\phi}(x_t,t) + v_\theta(x_t,t) - v_{t|z}(x_t|z)\|_2^2}_{=\ell^{\text{RGM}}}\right]\\
    &=\mathcal{L}_\phi^{\text{RGM}}.
\end{align}
Where we used the conclusion of Eq. \eqref{eq:appendix:guidance_matching_loss_g_2_v_term}, and inserted terms that vanish after $\nabla_{\phi}$ to make $\ell^{\text{RGM}}$.

\paragraph{(3) Marginalized Reweighted Guidance Matching Loss $\ell^{\text{MRGM}}$.} The above loss Eq. \eqref{eq:appendix:guidance_matching_loss_g_3_loss} can be re-assigned a weight, which will result in the same optimal $g_{\phi}(x_t,t)$ as in Eq. \eqref{eq:guidance_matching_loss_g_1}. Specifically, by changing $\frac{1}{Z_{\phi_Z,sg}(x_t,t)}$ to $\frac{1}{\mathbb{E}_{x_t\sim p(x_t)}[Z_{\phi_Z,sg}(x_t,t)]}=\frac{1}{\int p(x_t)Z_{\phi_Z,sg}(x_t,t) dx_t}$, we have 
\begin{equation}\label{eq:appendix:guidance_matching_loss_g_4_loss_appendix}
    \frac{e^{-J(x_1)}}{\int e^{-J(x_1)} p(z)dz} \|g_{\phi}(x_t,t) + v_\theta(x_t,t) - v_{t|z}(x_t|z)\|_2^2.
\end{equation}
We only need to prove that
\begin{equation}
    \int p(x_t)Z_{t}(x_t) dx_t = \int e^{-J(x_1)} p(z)dz.
\end{equation}
Recall  that 
\begin{equation}
    Z_{t}(x_t) = \int p(z|x_t) e^{-J(x_1)} dz,
\end{equation}
so 
\begin{align}
    &\int p(x_t)Z_{t}(x_t) dx_t
    = \int  p(x_t)\int p(z|x_t) e^{-J(x_1)} dx_t dz \\
    =& \int p(z)e^{-J(x_1)}dz = Z.
\end{align}

Eq. \eqref{eq:guidance_matching_loss_g_4_loss} can also be derived by applying importance sampling $\mathbb{E}_{z\sim \frac{1}{Z}p(z)e^{-J(x_1)}}\left[\|\cdot\|_2^2 \right]=\mathbb{E}_{z\sim p(z)}\left[ \frac{e^{-J(x_1)}}{Z} \|\cdot\|_2^2  \right]$ to the flow matching objective of the new VF for the new target distribution $p'(x_t) = \frac{1}{Z} p(x_t) e^{-J(x_t)}$.


\paragraph{Discussions}
The losses have the same expected gradient, but their performance may differ. Among the four losses, $\ell_\phi^{\text{MRGM}}$ is the only one that does not require the auxiliary model $Z_{\phi_Z}$. However, $\ell_\phi^{\text{RGM}}$ assigns loss weight dependent on $x_t$. The weight is emphasized when the expectation of $e^{-J(x_1)}$ under $p(x_1|x_t)$ is small. Compared to these two losses, $\ell_\phi^{\text{GM}},\ell_\phi^{\text{VGM}}$ do not reweight the loss. The variance of $\ell_\phi^{\text{GM}}$ will be smaller if $J$ is smooth, while $\ell_\phi^{\text{VGM}}$ is better when $v_t$ is more complex.

\subsection{Algorithm Details and Variants of $g^{\text{MC}}$}\label{app:independent_mc}
The pseudocode for computing $g^{\text{MC}}$ is as follows.
\begin{algorithm}[h]
\caption{Monte Carlo estimation of the guidance $g_t(x_t)$}
\label{alg:mc_estimation_on_g}
\begin{algorithmic}[1]
\REQUIRE Current $t$, $x_t$, known $p_t(x_t|z)$.

\STATE Sample $z_i \sim p(z)$, where $i=1,2,...,N$ 
{\color{gray} // Recall $z_i = (x_{1}^i,x_{0}^i)$}
\STATE $\tilde{p}_t(x_t) \gets \frac{1}{N} \sum_i p_t(x_t|z_i)$
\STATE $\tilde{Z}_t(x_t) \gets \frac{1}{N} \sum_i e^{-J(x_{1}^i)} \frac{p_t(x_t|z_i)}{\tilde{p}(x_t)}$
\STATE ${g}^{\text{MC}}_t(x_t) \gets \frac{1}{N} \sum_i (\frac{e^{-J(x_{1}^i)}}{\tilde{Z}_t(x_t)} - 1) v_{t|z}(x_t|z_i) \frac{p_t(x_t|z_i)}{\tilde{p}_t(x_t)}$
\STATE \textbf{return} ${g}^{\text{MC}}_t(x_t)$
\end{algorithmic}
\end{algorithm}

\paragraph{Independent Couplings.}
Although we introduced the flow matching using the condition $z=(x_0,x_1)$, it can also be chosen as $x_0$ or $x_1$ \citep{lipman_flow_2024}. When we $z\coloneqq x_1$, Algorithm \ref{alg:mc_estimation_on_g} can be readily adopted for the $x_0$ condition. This way, the Monte Carlo estimation reduces the integration region dimensionality to half of the original one, thus becoming more efficient. 

In the case where $z=(x_0,x_1)$ and the data coupling is independent $\pi(x_0|x_1) = p(x_0)$, we show here that the MC estimation can be simplified to the $x_1$-conditioned case that is more efficient:
\begin{align}
&    \bm{g_t}^{\textbf{MC-$x_1$}}(x_t) \defg \mathbb{E}_{x_1\sim p(x_1)} 
\left[
(\frac{e^{-J(x_1)}}{Z_t} - 1) v_{t|x_1}(x_t|x_1) \frac{p_t(x_t|x_1)}{p_t(x_t)}
\right],
\\
& Z_t^{\text{MC-$x_1$}}(x_t) = \mathbb{E}_{x_1\sim p(x_1)} 
\left[
e^{-J(x_1)} \frac{p_t(x_t|x_1)}{p_t(x_t)} 
\right].
\end{align}
Obviously, as $Z_t = \int p(x_0,x_1|x_t) e^{-J(x_1)}dx_0dx_1 = \int p(x_0|x_1,x_t)p(x_1|x_t) e^{-J(x_1)}dx_0dx_1 $, integrating out $x_0$ gives $Z_t = Z^{\text{MC-$x_1$}}_t$. Therefore, to prove the above simplification, we only need to prove that:
\begin{equation}
    \mathbb{E}_{x_0,x_1\sim p(x_0,x_1|x_t)}\left[(\frac{e^{-J(x_1)}}{Z_t} - 1)  v(x_t|x_0,x_1)\right] = \mathbb{E}_{x_1\sim p(x_1|x_t)}\left[(\frac{e^{-J(x_1)}}{Z_t} - 1)  v(x_t|x_1)\right].
\end{equation}
The proof is simply integrating out $x_0$:
\begin{align}
    \nonumber&\int p(x_0,x_1|x_t) (\frac{e^{-J(x_1)}}{Z_t} - 1)  v(x_t|x_0,x_1) dx_0 dx_1 \\
    \nonumber=& \int \underbrace{\int p(x_0|x_1,x_t)v(x_t|x_0,x_1)dx_0}_{\coloneqq v(x_t|x_1)} (\frac{e^{-J(x_1)}}{Z_t} - 1) p(x_1|x_t) dx_1 \\
    =& \int v(x_t|x_1) (\frac{e^{-J(x_1)}}{Z_t} - 1) p(x_1|x_t) dx_1.
\end{align}
It should be noted that $v(x_t|x_1)$ is defined to be generally different from $v(x_t|x_0,x_1)$, and to do MC estimation via importance sampling, we need the forward probability path $p(x_t|x_1)$ to have a known density. The variance-reducing variant of $g^{\text{MC}}$ is summarized in Algorithm \ref{alg:mc_uncoupled}.

\begin{algorithm}[H]
\caption{Monte Carlo estimation of the guidance $g_t(x_t)$}
\label{alg:mc_uncoupled}
\begin{algorithmic}[1]
\REQUIRE Current $t$, $x_t$, known $p_t(x_t|x_1)$.

\STATE Sample $x_1^i \sim p(x_1)$, where $i=1,2,...,N$
\STATE $\tilde{p}_t(x_t) \gets \frac{1}{N} \sum_i p_t(x_t|x_1^i)$
\STATE $\tilde{Z}_t(x_t) \gets \frac{1}{N} \sum_i e^{-J(x_{1}^i)} \frac{p_t(x_t|x_1^i)}{\tilde{p}(x_t)}$
\STATE $\tilde{g}_t(x_t) \gets \frac{1}{N} \sum_i (\frac{e^{-J(x_{1}^i)}}{\tilde{Z}_t(x_t)} - 1) v_{t|z}(x_t|x_1^i) \frac{p_t(x_t|x_1^i)}{\tilde{p}_t(x_t)}$
\STATE \textbf{return} $\tilde{g}_t(x_t)$
\end{algorithmic}
\end{algorithm}

\subsection{Localized Approximation}
\label{app:localized_approximation}

To get $g^{\text{local}}$, we presume $p(z|x_t)$ is localized, and we can use a point estimation to approximate $Z_t$:
\begin{align}
Z_t(x_t) = \int p(z|x_t) e^{-J(x_1)} dz  
\approx e^{-J(\hat{x}_1)}
\end{align}
where $\hat{x}_1 \coloneqq \mathbb{E}_{x_0,x_1\sim p(z|x_t)}[x_1]$,
and then expanding $g_t$ to the first order
\begin{align}\nonumber
    g_t(x_t) \approx g_t(x_t)^{\text{cov}} &= \mathbb{E}_{z \sim p(z|x_t)} 
    \left[
    (\frac{e^{-J(x_1)}}{e^{-J(\hat{x}_1)}} - 1)v_{t|z}(x_t|z)
    \right] \\
    \nonumber &\approx \mathbb{E}_{z \sim p(z|x_t)} \left[
    (\frac{e^{-J(\hat{x}_1)}(1 - \nabla_{\hat{x}_1} J(\hat{x}_1) (x_1 - \hat{x}_1))}{e^{-J(\hat{x}_1)}} - 1)v_{t|z}(x_t|z)
    \right] \\
    &= -\mathbb{E}_{x_1 \sim p(x_1|x_t)}
    \left[
     (x_1 - \hat{x}_1)v_{t|z}(x_t|z)
    \right] \nabla_{\hat{x}_1} J(\hat{x}_1).
\end{align}




To quantify the approximation error, we have
\begin{align}
    \nonumber \|\delta g\|^2&\coloneqq \|g_t - g^{\text{local}}\|^2_2 \\
    \nonumber&=\bigg{\|}\mathbb{E}_{z \sim p(z|x_t)} 
    \left[
    (\frac{e^{-J(x_1)}}{Z_t(x_t)} - 1)v_{t|z}(x_t|z)
    \right] \\
    \nonumber&- \mathbb{E}_{z \sim p(z|x_t)} 
    \left[
    (\frac{e^{-J(\hat{x}_1)}(1 - \nabla_{\hat{x}_1} J(\hat{x}_1) (x_1 - \hat{x}_1))}{e^{-J(\hat{x}_1)}} - 1)v_{t|z}(x_t|z)
    \right] \bigg{\|}_2^2\\
    &=\left\|\mathbb{E}_{z \sim p(z|x_t)} 
    \left[
    (\frac{e^{-J(x_1)}}{Z_t(x_t)}-\frac{e^{-J(\hat{x}_1)}(1 - \nabla_{\hat{x}_1} J(\hat{x}_1) (x_1 - \hat{x}_1))}{e^{-J(\hat{x}_1)}})v_{t|z}(x_t|z)
    \right] \right\|_2^2
\end{align}
where 
\begin{equation}
    Z_t(x_t) = \mathbb{E}_{z\sim p(z|x_t)} [e^{-J(x_1)}].
\end{equation}
We start by computing the error bound of approximating $Z_t$ with $e^{-J(\hat{x}_1)}$. Using Taylor expansion and the Taylor Remainder Theorem \footnote{The notations here neglect the order of vector/matrix products, but this does not matter as all of them will be scaled using the triangle inequality.}, 
\begin{align}\nonumber
    \left\|Z_t(x_t)-e^{-J(\hat{x}_1)}\right\|^2_2 &= \left\|\mathbb{E}_{z\sim p(z|x_t)}[\sum_{k=2} \frac{1}{k!}D_{x_1}^k e^{-J(x)}\big{|}_{x_1=\hat{x}_1} (x_1 - \hat{x}_1)^k] \right\|_2^2 \\
    &\le \mathbb{E}_{z\sim p(z|x_t)}
    \left
    [\left\|\frac{1}{2} (x_1 - \hat{x}_1)^T \underbrace{\nabla_{x_1} \nabla_{x_1} e^{-J(x)}\big{|}_{x_1=\hat{x}_1 + t (x_1 - \hat{x}_1)}}_{\coloneqq h^{(J)}_t} (x_1 - \hat{x}_1)
    \right\|_2^2
    \right] ,
\end{align}
where $t\in[0, 1]$.


If we set the L2 norm of the covariance matrix $\mathbb{E}_{z\sim p(z|x_t)}[ (x_1 - \hat{x}_1)(x_1 - \hat{x}_1)^T]$ as $\sigma_1$, and the largest eigenvalue of $\max_{t,x_1}h^{(J)}_t$ to be $\lambda_{h}$, we can show that  
\begin{align}
    \nonumber \left\|Z_t(x_t)-e^{-J(\hat{x}_1)}\right\|^2_2 \le &\mathbb{E}_{z\sim p(z|x_t)}[(x_1 - \hat{x}_1)^T h^{(J)}_t(x_1 - \hat{x}_1)]\\
    \nonumber \le &\mathbb{E}_{z\sim p(z|x_t)}[(x_1 - \hat{x}_1)^T \lambda_h(x_1 - \hat{x}_1)]\\
    \nonumber \le &\lambda_h\mathbb{E}_{z\sim p(z|x_t)}[ (x_1 - \hat{x}_1)^T(x_1 - \hat{x}_1)]\\
    \nonumber \le& \lambda_h \mathbf{tr}\Sigma_{11} \\
    \le&\lambda_h \sigma_1 d,
\end{align}
where $\Sigma_{11}$ is the covariance matrix of $p(x_1|x_t)$, $d$ is the dimensionality of $x \in \mathbb{R}^d$. The last inequality follows from $\mathbf{tr}A = \sum_i^n \lambda_i \le n \max_i\lambda_i $, and the L2 norm of a matrix is its largest singular value, \emph{i.e.}, for the covariance matrix, that is the largest eigenvalue.

Then, 
\begin{align}
    \nonumber \delta g &= \mathbb{E}_{z \sim p(z|x_t)} 
    \left[
    \left(\frac{e^{-J(x_1)}}{Z_t(x_t)}-\frac{e^{-J(\hat{x}_1)}(1 - \nabla_{\hat{x}_1} J(\hat{x}_1) (x_1 - \hat{x}_1))}{e^{-J(\hat{x}_1)}}\right)v_{t|z}
    \right] \\
    & = \mathbb{E}_{z \sim p(z|x_t)} \left[ 
    \left(\underbrace{\frac{e^{-J({x}_1)}}{Z_t(x_t)} - \frac{e^{-J(x_1)}}{e^{-J(\hat{x}_1)}}}_{\text{Using the error bound of } Z_t} + \frac{e^{-J(x_1)}}{e^{-J(\hat{x}_1)}} -\frac{e^{-J(\hat{x}_1)}(1 - \nabla_{\hat{x}_1} J(\hat{x}_1) (x_1 - \hat{x}_1))}{e^{-J(\hat{x}_1)}}\right)v_{t|z}
    \right].
\end{align}
Therefore,
\begin{align}
    \nonumber \|\delta g\|_2^2& \le \left\|-\frac{\lambda_h d}{Z_t(x_t)e^{-J(\hat{x}_1)}} {\mathbb{E}_{z \sim p(z|x_t)} [e^{-J(x_1)}v_{t|z}]}\sigma_1\right\|_2^2 \\
    \nonumber &+ \left\|\mathbb{E}_{z \sim p(z|x_t)} \left[ 
    \frac{e^{-J(x_1)} - e^{-J(\hat{x}_1)}(1 - \nabla_{\hat{x}_1} J(\hat{x}_1) (x_1 - \hat{x}_1))}{e^{-J(\hat{x}_1)}} v_{t|z}
    \right] \right\|_2^2\\
    \nonumber & = \bigg{\|} -\frac{\lambda_h d}{Z_t(x_t)e^{-J(\hat{x}_1)}} {\mathbb{E}_{z \sim p(z|x_t)} [e^{-J(x_1)}v_{t|z}]}\sigma_1\bigg{\|}_2^2 \\
    & + \bigg{\|}\mathbb{E}_{z \sim p(z|x_t)} \left[ 
    \frac{e^{-J(\hat{x}_1)}(1 - \nabla_{\hat{x}_1} J(\hat{x}_1) (x_1 - \hat{x}_1)) + R_2 - e^{-J(\hat{x}_1)}(1 - \nabla_{\hat{x}_1} J(\hat{x}_1) (x_1 - \hat{x}_1))}{e^{-J(\hat{x}_1)}} v_{t|z}
    \right]\bigg{\|}_2^2.
\end{align}
By using the Taylor Remainder Theorem again, we have 
\begin{equation}
R_2 = \frac{1}{2}(x_1 - \hat{x}_1)^T \underbrace{\nabla_{\xi}\nabla_{\xi} e^{-J(\xi)}\big{|}_{\xi=\hat{x}_1 + t (x_1 - \hat{x}_1)}}_{= h^{(J)}_t}(x_1 - \hat{x}_1).
\end{equation}
Thus,
\begin{align}
    \nonumber \|\delta g\|_2^2\le&\left\|  \frac{\mathbb{E}[e^{-J(x_1)}v(x_t|z)]\lambda_h\sigma_1 d}{\mathbb{E}[e^{-J(x_1)}]e^{-J(\hat{x}_1)}} \right\|_2^2 + \big{\|}\mathbb{E}_{z \sim p(z|x_t)}[\frac{1}{2}(x_1 - \hat{x}_1)^T h_t^{(J)}(x_1 - \hat{x}_1) 
    v_{t|z}
    ]\big{\|}_2^2\\
    \nonumber \le&\left\|  \frac{\mathbb{E}[e^{-J(x_1)}v(x_t|z)]\lambda_h\sigma_1 d}{\mathbb{E}[e^{-J(x_1)}]e^{-J(\hat{x}_1)}} \right\|_2^2 \\
    \nonumber +&
        \left[
            \mathbb{E}_{z \sim p(z|x_t)} 
            \left\|
            \frac{1}{2}(x_1 - \hat{x}_1)^T h_t^{(J)}(x_1 - \hat{x}_1)
            \right\|_2^2
        \right]
        \left[
            \mathbb{E}_{z \sim p(z|x_t)} 
            \left\|
            v(x_t|z)
            \right\|_2^2
        \right]\\
    \le & \left\|  \frac{\mathbb{E}[e^{-J(x_1)}v(x_t|z)]\lambda_h\sigma_1 d}{\mathbb{E}[e^{-J(x_1)}]e^{-J(\hat{x}_1)}} \right\|_2^2 + \left\|\frac{\lambda_h \sigma_1 d}{2} \right\|^2_2 \mathbb{E}\left[\|v(x_t|z)\|_2^2\right]
\end{align}

Then we have
\begin{align}
    \nonumber \|\delta g\|^2_2 &\le \left\|  \frac{\mathbb{E}[e^{-J(x_1)}v(x_t|z)]\lambda_h\sigma_1 d}{\mathbb{E}[e^{-J(x_1)}]e^{-J(\mathbb{E}[x_1])}} \right\|_2^2
    + \left\|\frac{\lambda_h \sigma_1 d}{2} \right\|^2_2 \mathbb{E}\left[\|v(x_t|z)\|_2^2\right] \\
    & = (\lambda_h \sigma_1 d)^2
    \left( 
    \underbrace{\left\|  \frac{\mathbb{E}[e^{-J(x_1)}v(x_t|z)]}{\mathbb{E}[e^{-J(x_1)}]e^{-J(\mathbb{E}[{x}_1])}} \right\|_2^2}_{C_1} + \underbrace{\mathbb{E}\left[\|v(x_t|z)\|_2^2\right]}_{C_2}
    \right),
\end{align}
where we omit $z\sim p(z|x_t)$ in $\mathbb{E}_{z\sim p(z|x_t)}$ for simplicity.
Therefore, the approximation error of $g^{\text{local}}$ is bounded by $(\lambda_h \sigma_1 d)^2 (C_1 + C_2)$, where $\lambda_h$ is the largest eigenvalue of $h_t^{(J)}$, the Hessian matrix of the objective function $e^{-J}$, $\sigma_1$ is the L2 norm of the covariance matrix, $d$ is the sample dimensionality, $C_1$ is a constant that has to do with the norm of the new VF, and $C_2$ is the variance of the original \emph{conditional} VF. Some intuitions can be emphasized:
\begin{enumerate}
    \item The error is small as $J$ is smooth, in which case the Hessian of $e^{-J}$ will approach zero. This corresponds to the mild guidance, where approximation-based $g^{\text{local}}$ works well.
    \item The error is small as $\sigma_1$ is small, \emph{i.e.} the covariance matrix $\Sigma_{11}$ has a small Frobenius norm. This is the case when the flow time $t\rightarrow 1$ (and $\sigma_t = 0$), where $x_t$ predicts $x_1$ well.
    \item The cases where $C_1$ and $C_2$ are small are of less practical usage since with a small norm of the VF, the error in the guidance VF will likely cause a larger deviation due to increased relative error.
\end{enumerate}


\subsection{Estimation of $\hat{x}_1$}
\label{appendix:x1_parameterization}
Under the \emph{affine path assumption} (Assumption \ref{assumption:affine_path}), we can estimate the expectation of $x_1$ under the distribution $p(z|x_t)$. This is a well-known trick \citep{lipman_flow_2024,pokle_training-free_2024}, but our analysis includes the dependent coupling case.

Since the flow matching model learns
\begin{equation}
    v_{\theta}(x_t,t)\approx v_t(x_t) = \mathbb{E}_{z\sim p(z|x_t)}\left[
    v(x_t|z)
    \right],
\end{equation}
using the \emph{affine path assumption} ($x_t = \alpha_t x_1 + \beta_t x_0 + \sigma_t\dot\sigma_t\varepsilon$), 
\begin{equation}
    v(x_t|z) = \frac{d}{dt}x_t  = (\dot\alpha x_1 + \dot\beta_t x_0 + \dot\sigma_t\varepsilon),
\end{equation}
so
\begin{equation}\label{eq:appendix:x1_parameterization_v}
    v_t(x_t) = \mathbb{E}_{z\sim p(z|x_t)}\left[
    \dot\alpha x_1 + \dot\beta_t x_0 +\dot \sigma_t\varepsilon
    \right].
\end{equation}

Meanwhile, taking the expectation of $x_t$ under $p(z|x_t)$ yields
\begin{align}\label{eq:appendix:x1_parameterization_x}
    \underbrace{\mathbb{E}_{z\sim p(z|x_t)}[x_t] = x_t}_{\text{because} \int z p(z|x_t) dz = 1} = \mathbb{E}_{z\sim p(z|x_t)}[\alpha_t x_1 + \beta_t x_0 + \sigma_t\varepsilon].
\end{align}

Then, by using Eq. \ref{eq:appendix:x1_parameterization_v} and \ref{eq:appendix:x1_parameterization_x}, we can eliminate either $\hat{x}_0$ or $\hat{x}_1$ in each other's expression:
\begin{align}
    \hat{x}_0 \coloneqq \mathbb{E}_{z\sim p(z|x_t)}[x_0] & = 
    \frac{\dot\alpha_t x_t - \alpha_t v_t(x_t)}{\beta_t\dot\alpha_t - \dot\beta_t\alpha_t}
    +
    \underbrace{\mathbb{E}_{z\sim p(z|x_t)}\left[
    \frac{-\dot\alpha_t\sigma_t\varepsilon + \alpha_t\dot\sigma_t\varepsilon}{\beta_t\dot\alpha_t - \dot\beta_t\alpha_t}
    \right]}_{\coloneqq \zeta^0_t}
    \\
    \hat{x}_1 \coloneqq \mathbb{E}_{z\sim p(z|x_t)}[x_1] & = 
    \frac{-\dot\beta_t x_t + \beta_t v_t(x_t)}{\beta_t\dot\alpha_t - \dot\beta_t\alpha_t} + \underbrace{\mathbb{E}_{z\sim p(z|x_t)}\left[
    \frac{\dot\beta_t\sigma_t\varepsilon - \beta_t\dot\sigma_t\varepsilon}{\beta_t\dot\alpha_t - \dot\beta_t\alpha_t}
    \right]}_{\coloneqq \zeta^1_t}.
\end{align}
It should be noted that we have assumed that $\sigma_t$ is small and thus $\zeta_t^0$ and $\zeta_t^1$ are also small in the \emph{affine path assumption} (Assumption \ref{assumption:affine_path}):
\begin{align}
    \nonumber \zeta_t^0 &= \frac{-\dot\alpha_t\sigma_t + \alpha_t\dot\sigma_t}{\beta_t\dot\alpha_t - \dot\beta_t\alpha_t}\mathbb{E}_{z\sim p(z|x_t)}[\varepsilon] \\
    \nonumber &=\frac{-\dot\alpha_t\sigma_t + \alpha_t\dot\sigma_t}{\beta_t\dot\alpha_t - \dot\beta_t\alpha_t}
    \int \frac{p(x_t|x_0,x_1)\pi(x_0,x_1)}{p(x_t)}\varepsilon dx_0dx_1 \\
    &=\frac{-\dot\alpha_t\sigma_t + \alpha_t\dot\sigma_t}{\beta_t\dot\alpha_t - \dot\beta_t\alpha_t}\int\frac{1}{p(x_t)}
    \mathbb{E}_{\varepsilon\sim p_{\varepsilon}(\varepsilon)} \left[
    \pi(x_0(x_t,x_1,\varepsilon),x_1)\varepsilon \right] dx_1.
\end{align}
Since $\pi(x_0(x_t,x_1,\varepsilon),x_1)$ is a probability distribution that is assumed to be bounded, we denote $\max_{\varepsilon}\|\pi(x_0(x_t,x_1,\varepsilon),x_1)\|\le \mathcal{M}(x_1,x_t)$, and thus 
\begin{equation}
    \lim_{\sigma_t\rightarrow 0, \dot\sigma_t\rightarrow 0}\zeta_t^0
    \le\lim_{\sigma_t\rightarrow 0, \dot\sigma_t\rightarrow 0}\left|\frac{-\dot\alpha_t\sigma_t + \alpha_t\dot\sigma_t}{\beta_t\dot\alpha_t - \dot\beta_t\alpha_t}\right|\cdot \int 
    \left\|
    \frac{1}{p(x_t)}
    \right\|_2^2
    \cdot
    \left\|
    \mathbb{E}_{\varepsilon\sim p_{\varepsilon}(\varepsilon)}
    \|\varepsilon\|_2
    \mathcal{M}(x_1,x_t)
    \right\|_2^2dx_1 = 0.
\end{equation}
Since everything in the integral is independent of $\varepsilon$, $x_0$, or $\sigma_t$, as $\sigma_t\rightarrow 0$ $\zeta^0$ simply converges to zero. A similar approach can prove that $\lim_{\sigma_t\rightarrow 0, \dot\sigma_t\rightarrow 0}\zeta_2^1$ is also zero.

Next, we explain why we specifically care about the case where the small $\sigma_t$ assumption holds.
In independent coupling flow matching, $\sigma_t\varepsilon$ is exactly zero since we can use two of $x_t,x_0$, and $x_1$ to express the third one. In dependent coupling flow matching, this assumption also holds for famous methods such as optimal transport conditional flow matching or Schrodinger Bridge conditional flow matching \cite{tong_improving_2024}, where $\varepsilon\sim\mathcal{N}(0,I)$ and $\sigma_t$ is set as a small constant. Therefore, the assumption that $\sigma_t\varepsilon$ is small in \emph{affine path assumption} is general and applies to many existing flow matching methods. Hence, by approximating $\zeta_t^0$ and $\zeta_t^1$ as zero, we have the final estimation of $\hat{x}_1$
\begin{align}
    &\hat{x}_0 \approx 
    \frac{\dot\alpha_t x_t - \alpha_t v_\theta(x_t,t)}{\beta_t\dot\alpha_t - \dot\beta_t\alpha_t} \\
    &\label{eq:appendix:local_approx_guidance_affine_path_estimate_x1}\hat{x}_1 \approx 
    \frac{-\dot\beta_t x_t + \beta_t v_t(x_t)}{\beta_t\dot\alpha_t - \dot\beta_t\alpha_t},
\end{align}
where Eq. \eqref{eq:appendix:local_approx_guidance_affine_path_estimate_x1} is just Eq. \eqref{eq:local_approx_guidance_affine_path_estimate_x1}. Note the approximations become exact under Assumption \ref{assumption:uncoupled_affine_gaussian_path}.


\subsection{Proof of $g^{\text{cov}}$}
\label{appendix:affine_path_cov_local_approx_guidance}
Here, we prove that under the affine path assumption (Assumption \ref{assumption:affine_path}), Eq. \eqref{eq:g_cov}
\begin{equation}\label{eq:restate_g_cov}
    g_t^{\text{local}}\approx g_t^{\text{cov}} = -\underbrace{\frac{\dot\alpha_t\beta_t - \dot\beta_t\alpha_t}{\beta_t}}_{\text{schedule}} \Sigma_{1|t} \nabla_{\hat{x}_1}J(\hat{x}_1),
\end{equation}
where 
\begin{equation}
    g_t^{\text{local}} = -\mathbb{E}_{z \sim p(z|x_t)}
    \left[
     (x_1 - \hat{x}_1)v_{1|t}(x_t|z)
    \right] \nabla_{\hat{x}_1} J(\hat{x}_1).
\end{equation}

Under the affine path $x_t = \alpha_t x_1 + \beta_t x_0 + \sigma_t \varepsilon$ the conditional vector field $v_{1|t}$ follows
\begin{equation}
    v_{1|t}(x_t) = \dot \alpha_t x_1 + \dot\beta_t x_0 + \dot\sigma_t \varepsilon.
\end{equation}
Plugging this into the definition of $g^{\text{local}}$ and we get
\begin{align}
    \nonumber g_t^{\text{local}} = & -\mathbb{E}_{z \sim p(z|x_t)}[\underbrace{(x_1 - \hat{x}_1)(\dot \alpha_t x_1 + \dot\beta_t x_0 + \dot\sigma_t\varepsilon}_{\text{substitute } x_0 \text{ with } x_1,~ \sigma_t\varepsilon, \text{ and } x_t})] \nabla_{\hat{x}_1} J(\hat{x}_1)\\
    \nonumber =& -\mathbb{E}_{z \sim p(z|x_t)}[(x_1 - \hat{x}_1)(\dot \alpha_t x_1 + \frac{\dot\beta_t}{\beta_t} (x_t - \alpha_t x_1 - \sigma_t \varepsilon) + \dot\sigma_t\varepsilon)] \nabla_{\hat{x}_1} J(\hat{x}_1)\\
    \nonumber =& -\mathbb{E}_{z \sim p(z|x_t)}\left[
    (x_1 - \hat{x}_1)\left(
    \left(\frac{\beta_t\dot \alpha_t - \alpha_t\dot\beta_t}{\beta_t}\right) x_1 +  \cancel{x_t}  + (\dot\sigma_t - \sigma_t)\varepsilon
    \right)
    \right] 
    \nabla_{\hat{x}_1} J(\hat{x}_1)\\
    = & -{\frac{\dot\alpha_t\beta_t - \dot\beta_t\alpha_t}{\beta_t}} \Sigma_{1|t} \nabla_{\hat{x}_1}J(\hat{x}_1)
    + \underbrace{(\sigma_t - \dot\sigma_t)\mathbb{E}_{z \sim p(z|x_t)}[(x_1-\hat{x}_1)\varepsilon]\nabla_{\hat{x}_1}J(\hat{x}_1)}_{\coloneqq\Upsilon,~\lim_{\sigma_t\rightarrow 0, \dot\sigma_t \rightarrow 0}\|\Upsilon\|_2^2=0 },
\end{align}
where the $x_t$ term is canceled out because $\int p(z|x_t) (x_1 - \mathbb{E}_{z\sim p(z|x_t)}[x_1])dz = \int p(z|x_t) x_1 dz -\mathbb{E}_{z\sim p(z|x_t)}[x_1] = 0$, $\Sigma_{1|t}\coloneqq \mathbb{E}_{z \sim p(z|x_t)}\left[
(x_1 - \hat{x}_1)(x_1 - \hat{x}_1)\right]$, and the residual term that characterizes the approximation error (denoted as $\|\Upsilon\|^2_2$) in Eq. \eqref{eq:g_cov} (restated in Eq. \eqref{eq:restate_g_cov}) is
\begin{align}
    \nonumber \Upsilon = &(\sigma_t - \dot\sigma_t)\mathbb{E}_{z \sim p(z|x_t)}[(x_1-\hat{x}_1)\varepsilon]\nabla_{\hat{x}_1}J(\hat{x}_1) \\
    \nonumber =&(\sigma_t - \dot\sigma_t)\nabla_{\hat{x}_1}J(\hat{x}_1)\int \frac{p(x_t|z)p(z)}{p(x_t)}(x_1-\hat{x}_1)\varepsilon dx_1dx_0 \\
    \nonumber =&(\sigma_t - \dot\sigma_t)\nabla_{\hat{x}_1}J(\hat{x}_1)\int \frac{p(\sigma_t\varepsilon)\pi(x_0|x_1)p(x_1)}{p(x_t)}(x_1-\hat{x}_1)\frac{1}{\sigma_t}(x_t - \alpha_t x_1 - \beta_t x_0) dx_1dx_0 \\
    \nonumber =&(\sigma_t - \dot\sigma_t)\nabla_{\hat{x}_1}J(\hat{x}_1)\int \frac{p(x_1)}{p(x_t)}(x_1-\hat{x}_1) dx_1 \int p(\sigma_t\varepsilon)\pi(x_0|x_1) \varepsilon dx_0 \\
    =&(\sigma_t - \dot\sigma_t)\nabla_{\hat{x}_1}J(\hat{x}_1)\int \frac{p(x_1)}{p(x_t)}(x_1-\hat{x}_1) 
    \mathbb{E}_{\varepsilon\sim p_\varepsilon(\varepsilon)}
    \left[{\pi\left(\frac{1}{\beta_t}(x_t - \alpha_t x_1 - \sigma_t \varepsilon)\mid x_1\right)} \varepsilon \right]dx_1 ,
\end{align}
where $p_\varepsilon(\varepsilon)$ is the marginal distribution of $\varepsilon$.
Suppose $\|\pi\left(\frac{1}{\beta_t}(x_t - \alpha_t x_1 - \sigma_t \varepsilon)\mid x_1\right)\|^2_2 = \|\pi\left(x_0 \mid x_1\right)\|^2_2\le \mathcal{M}(x_1,x_t)$ (which is a function independent of $\varepsilon$, then
\begin{align}
    \nonumber &\|\Upsilon\|^2_2 \\
    \nonumber \le& \left\|(\sigma_t - \dot\sigma_t)\nabla_{\hat{x}_1}J(\hat{x}_1)\right\|_2^2\cdot \left\|\int\frac{p(x_1)}{p(x_t)}(x_1-\hat{x}_1)\mathbb{E}_{\varepsilon\sim p_\varepsilon(\varepsilon)}
    \left[{\pi\left(\frac{1}{\beta_t}(x_t - \alpha_t x_1 - \sigma_t \varepsilon)\mid x_1\right)} \varepsilon \right]dx_1\right\|^2_2 \\
    \nonumber \le& |(\sigma_t - \dot\sigma_t)|\underbrace{\left\|\nabla_{\hat{x}_1}J(\hat{x}_1)\right\|_2^2}_{\coloneqq \mathcal{G}}\cdot \int
    \underbrace{\left\|\frac{p(x_1)}{p(x_t)}(x_1-\hat{x}_1)\right\|^2_2}_{\coloneqq\mathcal{Q}}
    \cdot
    \underbrace{\left\|\mathbb{E}_{\varepsilon\sim p_\varepsilon(\varepsilon)}
    \left[{\pi\left(\frac{1}{\beta_t}(x_t - \alpha_t x_1 - \sigma_t \varepsilon)\mid x_1\right)} \varepsilon \right]\right\|^2_2}_{\le \left(\mathcal{M}\mathbb{E}_{\varepsilon\sim p_\varepsilon(\varepsilon)}[\|\varepsilon\|_2]\right)^2 \le \mathcal{M}^2 \text{Var}_{p_\varepsilon}}dx_1 \\
    \le& |(\sigma_t - \dot\sigma_t)|\mathcal{G}(x_t)\int \mathcal{Q}(x_1,x_t))\mathcal{M}^2(x_1,x_t)\text{Var}_{p_\varepsilon} dx_1,
\end{align}
all of which are independent on $x_0$ or $\sigma_t$. Thus, 
\begin{equation}
    \lim_{\sigma\rightarrow 0,\sigma_t\rightarrow 0}\|\Upsilon\|^2_2 = 0.
\end{equation}


\subsection{The Jacobian Trick}\label{appendix:jacobian_trick}
We prove the Jacobian Trick here. 
\begin{proposition} The Jacobian trick. Under Assumption \ref{assumption:uncoupled_affine_gaussian_path}, the inverse covariance matrix of $p(x_1|x_t)$, $\Sigma_{1|t}$, is affine to the Jacobian of the VF $\frac{\partial v_t}{\partial x_t}$, and is proportional to the Jacobian $\frac{\partial \hat{x}_1}{\partial x_t}$:
\begin{align}\nonumber
     \Sigma_{1|t}= \frac{\beta_t^2}{\alpha_t(\dot\alpha_t\beta_t - \dot\beta_t\alpha_t)} (-\dot\beta_t+ \beta_t\frac{\partial v_t}{\partial x_t} )
     = \frac{\beta_t^2}{\alpha_t} \frac{\partial \hat{x}_1}{\partial x_t}.
\end{align}
\end{proposition}

\textbf{Proof.}


To begin with, we prove $\Sigma_{1|t}=\frac{\beta_t^2}{\alpha_t} \frac{\partial \hat{x}_1}{\partial x_t}$. A similar conclusion has been proved in \citet{ye_tfg_2024}. We generalize their proof to affine Gaussian path flow matching:

Recall from Eq. \eqref{eq:appendix:local_approx_guidance_affine_path_estimate_x1} that
\begin{equation}
    \hat{x}_1 = -\frac{\dot\beta_t}{\dot \alpha_t \beta_t - \dot \beta_t \alpha_t} x_t + \frac{\beta_t}{\dot \alpha_t \beta_t - \dot \beta_t \alpha_t} v_t
\end{equation}
and \begin{equation}
    \hat{x}_0 = \frac{\dot\alpha_t}{\dot \alpha_t \beta_t - \dot \beta_t \alpha_t} x_t - \frac{\alpha_t}{\dot \alpha_t \beta_t - \dot \beta_t \alpha_t} v_t.
\end{equation}
So we have the Jacobian trick
\begin{equation}
    \frac{\partial \hat{x}_1}{\partial x_t} = -\frac{\dot\beta_t}{\dot \alpha_t \beta_t - \dot \beta_t \alpha_t} + \frac{\beta_t}{\dot \alpha_t \beta_t - \dot \beta_t \alpha_t} \frac{\partial v_t(x_t)}{\partial x_t},
\end{equation}
and because the VF is associated with the score
\begin{equation}\label{eq:appendix:score_and_vector_field}
    v_t(x_t) = \frac{\beta_t(\dot\alpha_t\beta_t - \dot \beta_t\alpha_t)}{\alpha_t} \nabla_{x_t}\log p_t(x_t) + \frac{\dot\alpha_t}{\alpha_t}x_t.
\end{equation}

Next, we try to prove 
\begin{equation}\label{eq:appendix:score_derivative_and_covariance}
    \nabla_{x_t}\nabla_{x_t}\log p_t(x_t) = -\frac{1}{\beta_t^2} + \frac{\alpha_t^2}{\beta_t^4} \Sigma_{x_1x_1},
\end{equation}
which allows us to connect the derivative of the score $\nabla^2_{x_t}\log p(x_t)$\footnotemark with the covariance matrix $\Sigma_{1|t}$.
\footnotetext{We use $\nabla \nabla$ and $\nabla^2$ interchangeably, with a little abuse of notation. It should not cause confusion since the size of the terms in the equations must match.}

\begin{align}
    \nonumber \nabla_{x_t}^2\log p(x_t) = 
    \nonumber & \frac{\nabla_{x_t}^2 p(x_t) }{p(x_t)} - \nabla_{x_t}\log p(x_t) \nabla_{x_t}\log p(x_t)\\
    \nonumber =& \frac{1}{p(x_t)} \int p(x_1)
    \underbrace{\nabla_{x_t}^2 p(x_t|x_1)}_{\text{using }\nabla^2 p = p\nabla^2 \log p + p(\nabla \log p)^2}
    dx_1
    - \nabla_{x_t}\log p(x_t) \nabla_{x_t}\log p(x_t) \\
    \nonumber =& \frac{1}{p(x_t)} \int p(x_1) (p(x_t|x_1)\nabla_{x_t}^2\log p(x_t|x_1) + p(x_t|x_1)\nabla_{x_t}\log p(x_t|x_1)\nabla_{x_t}\log p(x_t|x_1)) dx_1\\
    \nonumber &
    - \nabla_{x_t}\log p(x_t) \nabla_{x_t}\log p(x_t)\\
    \nonumber =&\mathbb{E}_{x_1\sim p(x_1|x_t)}\left[
        \nabla_{x_t}^2\log p(x_t|x_1)+ \nabla_{x_t}\log p(x_t|x_1)\nabla_{x_t}\log p(x_t|x_1)
    \right]
    - \nabla_{x_t}\log p(x_t) \nabla_{x_t}\log p(x_t) \\
    \nonumber =&\mathbb{E}_{x_1\sim p(x_1|x_t)}\left[
        -\frac{1}{\beta_t^2}
        +\left(
            \frac{x_t - \alpha_t x_1}{\beta_t^2}
        \right)^2
    \right]
    -\left(
        \frac{x_t - \alpha_t \mathbb{E}_{x_1\sim p(x_1|x_t)}[x_1]}{\beta_t^2}
    \right)^2 \\
    \nonumber =&-\frac{1}{\beta_t^2} + \frac{\alpha_t^2}{\beta_t^4} 
    \left(\mathbb{E}[x_1 x_1^T] - \mathbb{E}[x_1]\mathbb{E}[x_1]^T\right) \\
    =& -\frac{1}{\beta_t^2} + \frac{\alpha_t^2}{\beta_t^4} \Sigma_{x_1x_1}.
\end{align}

Then by combining Eq. \eqref{eq:appendix:score_and_vector_field} and \eqref{eq:appendix:score_derivative_and_covariance} we have
\begin{align}
    \nonumber \frac{\partial \hat{x}_1}{\partial x_t} &=  -\frac{\dot\beta_t}{\dot \alpha_t \beta_t - \dot \beta_t \alpha_t} + \frac{\beta_t}{\dot \alpha_t \beta_t - \dot \beta_t \alpha_t} 
    \left(
    \frac{\beta_t(\dot\alpha_t\beta_t - \dot \beta_t\alpha_t)}{\alpha_t} \nabla_{x_t}\nabla_{x_t}\log p_t(x_t) + \frac{\dot\alpha_t}{\alpha_t}
    \right) \\
    \nonumber & = -\frac{\dot\beta_t}{\dot \alpha_t \beta_t - \dot \beta_t \alpha_t} + \frac{\beta_t}{\dot \alpha_t \beta_t - \dot \beta_t \alpha_t} 
    \left(
    \frac{\beta_t(\dot\alpha_t\beta_t - \dot \beta_t\alpha_t)}{\alpha_t} (-\frac{1}{\beta_t^2} + \frac{\alpha_t^2}{\beta_t^4} \Sigma_{x_1x_1}) +\frac{\dot\alpha_t}{\alpha_t}
    \right) \\
    &=\frac{\alpha_t}{\beta_t^2}\Sigma_{x_1x_1}.
\end{align}

Inserting back Eq. \eqref{eq:appendix:local_approx_guidance_affine_path_estimate_x1} and we prove 
\begin{equation}
    \Sigma_{1|t}= \frac{\beta_t^2}{\alpha_t(\dot\alpha_t\beta_t - \dot\beta_t\alpha_t)} (-\dot\beta_t+ \beta_t\frac{\partial v_t}{\partial x_t} ).
\end{equation}



\subsection{Proof for $g_t^{\text{sim-inv}}$}\label{appendix:approximate_simple_posterior_pigdm_like}
We begin with
\begin{align}
    &g^{\text{sim-inv}}_t(x_t) = \int \left(\frac{e^{-J(x_1)}}{\tilde{Z}_t} - 1\right) v_{t|z}(x_t|z) \tilde{p}(z|x_t) dz, \\
    &\text{where }\tilde{Z}_t = \int e^{-J(x_1)} \tilde{p}(z|x_t)dz,
\end{align}
and approximate $p(x_1|x_t)$ with $\mathcal{N}(x_1;\hat{x}_1, \Sigma_t)$ where $\hat{x}_1\coloneqq \mathbb{E}_{z\sim p(z|x_t)}[x_1]$ and $\Sigma_t$ is already known.

With Assumption \ref{assumption:affine_path} and assuming $e^{-J(x_1)} = \mathcal{N}(Hx_1;y, \sigma_y I)$, we have 
\begin{align}
    \tilde{Z}_t &= \int e^{-J(x_1)} \tilde{p}(z|x_t)dz \nonumber\\
    &=\int e^{-\frac{1}{2\sigma_y^2}\|y-Hx_1\|^2_2 - \frac{1}{2}(z-\hat{z})^T\Sigma_t^{-1}(z-\hat{z})} dz
\end{align}
where $\hat{z} = (\hat{x}_0, \hat{x}_1)$ is the expectation of $z$ under $p(z|x_t)$. Note $H$ operates on $x_1$ only, and we pad the blocks related to $x_0$ with zero in $H$.


Then, by inserting the Gaussian approximation
\begin{align}
    g^{\text{sim-inv}}_t(x_t) &\approx \int \left(\frac{e^{-J(x_1)}}{\tilde{Z}_t} - 1\right) (\dot\alpha_t x_1 + \dot \beta_t x_0) \underbrace{\tilde{p}(z|x_t)}_{\text{Gaussian}} dz \\
    & = \int \underbrace{\frac{1}{\tilde{Z}_t}\exp\left(-\frac{1}{2\sigma_y^2}\|y-Hx_1\|^2_2 - \frac{1}{2}(z-\hat{z})^T\Sigma_t^{-1}(z-\hat{z})\right)}_{\coloneqq \tilde{\tilde{p}}(z|x_t)} (\dot\alpha_t x_1 + \dot \beta_t x_0)  dz - v_t(x_t).\label{eq:appendix:gaussian_approx_affine_integral_of_g}
\end{align}

\begin{remark}
    Note that $\Sigma_t^{-1}$ couples $x_0$ and $x_1$. This is a fundamental feature of dependent couplings $\pi(x_0,x_1)$. However, it may seem tempting to further assume that $\Sigma_t^{-1}$ is diagonal or even a scalar. It should be noted that this assumption completely discards the dependency of $x_0$ and $x_1$ in the coupling, and thus, we try to avoid that in the dependent coupling case.
\end{remark}

For clarity, we need to express $\Sigma_t^{-1}$ with 
\begin{equation}
    \Sigma_t^{-1} \overset{\Delta}{=} \begin{pmatrix}
        \Xi_{00} & \Xi_{01} \\
        \Xi_{10} & \Xi_{11}
    \end{pmatrix}.
\end{equation}

Then, the distribution $\exp\left(-\frac{1}{2\sigma_y^2}\|y-Hx_1\|^2_2 - \frac{1}{2}(z-\hat{z})^T\Sigma_t^{-1}(z-\hat{z})\right)$ is still a Gaussian, and to estimate the expectation of $z=(x_0,x_1)$ we need to simply the probability density of this Guassian into a standard form.
\begin{align}
    \nonumber {\tilde{Z}_t}\tilde{\tilde{p}}(z|x_t) = &\exp\left(-\frac{1}{2\sigma_y^2}\|y-Hx_1\|^2_2 - \frac{1}{2}(z-\hat{z})^T\Sigma_t^{-1}(z-\hat{z})\right) \\
    \nonumber =&\exp\left(-\frac{1}{2\sigma_y^2}
    \left(
        \|y\|^2- 2\langle y, Hx_1 \rangle + \|Hx_1\|^2_2
    \right)
    -
    \frac{1}{2}z^T\Sigma_t^{-1}z - \frac{1}{2}\hat{z}^T{\Sigma_t^{-1}}\hat{z} + \underbrace{\langle z,\Sigma_t^{-1} \hat{z} \rangle}_{\text{since } \Sigma_t^{-1} = {\Sigma_t^{-1}}^T}\right) \\
    =&\exp\left(-\frac{1}{2}\left(
        z^T\left(\frac{H^TH}{\sigma_y^2} + \Sigma_t^{-1}\right)z
        -2\langle
            z, \frac{H^T}{\sigma_y^2}y + \Sigma_t^{-1}\hat{z}
        \rangle
        +\left(
            \frac{1}{2\sigma_y^2}\|y\|^2 + \frac{1}{2}\hat{z}^T{\Sigma_t^{-1}}\hat{z}
        \right)
        \right)
    \right),
\end{align}
It is obvious that the mean of this Gaussian is
\begin{equation}
    \mu = \Bigg{(}\underbrace{\frac{H^TH}{\sigma_y^2} + \Sigma_t^{-1}}_{\coloneqq P}\Bigg{)}^{-1}
    \left(
    \frac{H^T}{\sigma_y^2}y + \Sigma_t^{-1}\hat{z}
    \right),
\end{equation}
where we can find $P$'s blocks to be 
\begin{equation}
    P = \begin{pmatrix}
        \Xi_{00} & \Xi_{01} \\
        \Xi_{10} & \Xi_{11} + \frac{H^TH}{\sigma_y^2}
    \end{pmatrix}.
\end{equation}

Then, by computing $\mu = (\hat{\hat{x}}_0,\hat{\hat{x}}_1)$ we can compute $g_t^{\text{sim-inv}} + v_t$, where $\hat{\hat{x}}_0,\hat{\hat{x}}_1$ are the $x_0$ and $x_1$ term in the integral of Eq. \eqref{eq:appendix:gaussian_approx_affine_integral_of_g}, because
\begin{equation}
     g_t^{\text{sim-inv}} + v_t = \mathbb{E}_{z\sim\tilde{\tilde{p}}(z|x_t)} [\dot \alpha_t x_1 + \dot\beta x_0].
\end{equation}

To simplify, insert back $v_t$ to get
\begin{align}
     \nonumber g_t^{\text{sim-inv}} &= \mathbb{E}_{z\sim\tilde{\tilde{p}}(z|x_t)} [\dot \alpha_t x_1 + \dot\beta x_0] - \mathbb{E}_{z\sim \tilde{p}(z|x_t)}[\dot \alpha_t x_1 + \dot\beta x_0] \\
     \nonumber &=
     \begin{pmatrix}
     \dot\beta_t I & \dot\alpha_t I
     \end{pmatrix}
     P^{-1}
     \left(
    \begin{pmatrix}0 \\ \frac{H^T}{\sigma_y^2}y\end{pmatrix}
        +
     \left(
        \begin{pmatrix}
            \Xi_{00} & \Xi_{01} \\
            \Xi_{10} & \Xi_{11}
        \end{pmatrix}
        -
        P
    \right)
     \begin{pmatrix}\hat{x}_0 \\ \hat{x}_1 \end{pmatrix}
     \right)\\
     \nonumber &=
     \begin{pmatrix}
     \dot\beta_t I & \dot\alpha_t I
     \end{pmatrix}
     P^{-1}\left(
    \begin{pmatrix}0 \\ \frac{H^T}{\sigma_y^2}y\end{pmatrix}
        +
    \begin{pmatrix}
            0 & 0 \\
            0 & -\frac{H^TH}{\sigma_y^2}
    \end{pmatrix}
    \begin{pmatrix}\hat{x}_0 \\ \hat{x}_1 \end{pmatrix}
    \right) \\
    & = 
    \begin{pmatrix}
     \dot\beta_t I & \dot\alpha_t I
     \end{pmatrix}
     P^{-1}
    \begin{pmatrix}0 \\ \frac{H^T}{\sigma_y^2}y - \frac{H^TH}{\sigma_y^2} \hat{x}_1\end{pmatrix}.
\end{align}
Usually, $\begin{pmatrix}\dot\beta_t I & \dot\alpha_t I\end{pmatrix}P^{-1}$ is difficult to obtain:
\begin{equation}
    g_t^{\text{sim-inv}} = (\dot\beta_t P^{-1}_{01} + \dot\alpha_t P^{-1}_{11})\left(\frac{H^T}{\sigma_y^2}y - \frac{H^TH}{\sigma_y^2} \hat{x}_1\right), 
\end{equation}
where $P_{01}^{-1}$ and $P_{11}^{-1}$ requires computing the inversion of $P$ and thus in general intractable. 
Using block matrix inversion, we have
\begin{align}
    g_t^{\text{sim-inv}} = \Bigg{(}-\dot\beta_t \Xi_{11}^{-1}\Xi_{01} \Big{(}\Xi_{11} + \frac{H^TH}{\sigma_y^2}-\Xi_{10}\Xi_{11}^{-1}\Xi_{01}\Big{)}^{-1}
    +\dot\alpha_t \Big{(}\Xi_{11} + \frac{H^TH}{\sigma_y^2}-\Xi_{10}\Xi_{11}^{-1}\Xi_{01}\Big{)}^{-1}\Bigg{)}
    \left(\frac{H^T}{\sigma_y^2}y - \frac{H^TH}{\sigma_y^2} \hat{x}_1\right).
\end{align}

For general (possibly coupled) affine path flow matching, we can make approximations and set the \emph{blocks} in $\Sigma_t^{-1}$ to scalars. It should be noted that this Gaussian assumption can still capture some coupling between $x_0$ and $x_1$ since the off-diagonal blocks $\Xi_{01}$ and $\Xi_{10}$ are not set to zero.
Specifically, we have
\begin{equation}
    g_t^{\text{sim-inv-A}} = -\lambda_t \Big{(}\frac{\sigma_y^2}{r_t^2} + H^TH\Big{)}^{-1}
    H^T\left(y - {H} \hat{x}_1\right),
\end{equation}
where $\lambda_t$ and $r_t^2$ are hyperparameters. $\lambda_t$ approximates $\dot\alpha_t-\dot\beta_t \Xi_{11}^{-1}\Xi_{01}$, absorbing the flow schedule.

\textbf{Special case: the non-coupled affine Gaussian path}

Next, we prove that $g_t^{\text{sim-inv}}$ covers $\Pi$GDM \citep{song_pseudoinverse-guided_2022} and OT-ODE \citep{pokle_training-free_2024} as special cases.
Under the uncoupled affine Gaussian path assumption (Assumption \ref{assumption:uncoupled_affine_gaussian_path}), one may think that the covariance matrix is block diagonal, but it is false: $x_0$ and $x_1$ are still dependent on each other in the distribution $p(z|x_t) = p(x_0,x_1|x_t)$ even if the coupling is independent. In the uncoupled case, the probability graph is $x_0 \rightarrow x_t \leftarrow x_1$, so although $x_0$ and $x_1$ are marginally independent ($\pi(x_0,x_1)= p(x_0)p(x_1)$), their conditional can be dependent $p(x_0,x_1|x_t)\neq p(x_0|x_t)p(x_1|x_t)$.
Then, we notice the uncoupled path is 
\begin{equation}
    x_t = \alpha_t x_1 + \beta_t x_0,
\end{equation}
so we actually should not have approximated the distribution $p_{z|x_t}$ as a Gaussian in the uncoupled case. Fortunately, there is a workaround to make $x_0$ almost entirely dependent on $x_1$. We can set $x_0 = -\frac{\alpha_t}{\beta_t}x_1 + \frac{1}{\beta_t}x_t + \xi \epsilon$, where $\epsilon \sim \mathcal{N}(0, I)$, and setting $\xi \rightarrow 0$ gives our desired uncoupled path results. The covariance matrix of $x_0$ and $x_1$ to:
\begin{equation}
    \Sigma_t = \begin{pmatrix}
        \frac{\alpha_t^2}{\beta_t^2}\Sigma_{x_1 x_1} + \xi^2 I & -\frac{\alpha_t}{\beta_t} \Sigma_{x_1 x_1} \\
        -\frac{\alpha_t}{\beta_t} \Sigma_{x_1 x_1} & \Sigma_{x_1 x_1}
    \end{pmatrix}
\end{equation}
Note that $\Sigma_{x_1 x_1}^{-1} \neq \Xi_{11}$ as $\Xi_{11}$ is a block in the inversion of the larger matrix $\Sigma_t$. Next we compute $\Sigma_t^{-1}$:
\begin{align}
    \nonumber \Sigma_t^{-1} =& \begin{pmatrix}
    \frac{1}{\xi^2} I & -\frac{a}{\xi^2} I \\
    -\frac{a}{\xi^2} I  & \Big{(} \Sigma_{x_1 x_1} - a^2\Sigma_{x_1 x_1} (a^2\Sigma_{x_1 x_1} +\xi^2 I)^{-1} \Sigma_{x_1 x_1} \Big{)}^{-1}
    \end{pmatrix} \\
    = &\begin{pmatrix}
    \frac{1}{\xi^2} I & \frac{\alpha_t}{\beta_t\xi^2} I  \\
    \frac{\alpha_t}{\beta_t\xi^2} I  & \Big{(} \Sigma_{x_1 x_1} - \Sigma_{x_1 x_1} (\Sigma_{x_1 x_1} +\frac{\beta_t^2}{\alpha_t^2}\xi^2 I)^{-1} \Sigma_{x_1 x_1} \Big{)}^{-1}
    \end{pmatrix}.
\end{align}
where $a = -\frac{\alpha_t}{\beta_t}$. Therefore, $\Xi_{11} \rightarrow \infty$, $\Xi_{01}=\Xi_{10}=\frac{\alpha_t}{\beta_t\xi^2} I \rightarrow \infty$, and $\Xi_{00} = \frac{1}{\xi^2} I\rightarrow \infty $. Thus, we need more detailed calculations to get the result:

\begin{align}
    \nonumber g_t^{\text{sim-inv-diffusion}} =& \Bigg{(}-\dot\beta_t \underbrace{\Xi_{11}^{-1}\Xi_{01}}_{\rightarrow \frac{\alpha_t}{\beta_t} I } \Big{(}\underbrace{\Xi_{11}}_{\rightarrow \infty} + \frac{H^TH}{\sigma_y^2}-\underbrace{\Xi_{10}\Xi_{11}^{-1}\Xi_{01}}_{\rightarrow \infty}\Big{)}^{-1}
    +\dot\alpha_t \Big{(}\underbrace{\Xi_{11}}_{\rightarrow \infty} + \frac{H^TH}{\sigma_y^2}-\underbrace{\Xi_{10}\Xi_{11}^{-1}\Xi_{01}}_{\rightarrow \infty} \Big{)}^{-1}\Bigg{)}\\
    &\left(\frac{H^T}{\sigma_y^2}y - \frac{H^TH}{\sigma_y^2} \hat{x}_1\right).
\end{align}
Obviously we want to find the finite term left in $\Xi_{11} - \Xi_{10}\Xi_{11}^{-1}\Xi_{01}$:
\begin{align}
    \nonumber &\lim_{\xi\rightarrow 0}\Xi_{11} - \Xi_{10}\Xi_{11}^{-1}\Xi_{01} \\
    \nonumber =&\lim_{\xi\rightarrow 0}\Big{(} \Sigma_{x_1 x_1} - \Sigma_{x_1 x_1} (\Sigma_{x_1 x_1} +\frac{\beta_t^2}{\alpha_t^2}\xi^2 I)^{-1} \Sigma_{x_1 x_1} \Big{)}^{-1} - \Xi_{10}\Xi_{11}^{-1}\Xi_{01}\\
    \nonumber =&\lim_{\xi\rightarrow 0}\Big{(} \Sigma_{x_1 x_1} - \Sigma_{x_1 x_1} (\Sigma_{x_1 x_1} +\frac{\beta_t^2}{\alpha_t^2}\xi^2 I)^{-1} \Sigma_{x_1 x_1} \Big{)}^{-1} - \frac{\alpha_t^2}{\beta_t^2\xi^2} \\
    \nonumber =&\lim_{\xi\rightarrow 0}\left( \Sigma_{x_1 x_1}\left(
    \Sigma_{x_1 x_1} +\frac{\beta_t^2}{\alpha_t^2}\xi^2 I\right)^{-1}
    \left(
    (\cancel{\Sigma_{x_1 x_1}} +\frac{\beta_t^2}{\alpha_t^2}\xi^2 I)
    - \cancel{\Sigma_{x_1 x_1}}
    \right)
    \right)^{-1} - \frac{\alpha_t^2}{\beta_t^2\xi^2}\\
    \nonumber =&\lim_{\xi\rightarrow 0}
    \frac{\alpha_t^2}{\beta_t^2\xi^2}
    \left(
    \Sigma_{x_1 x_1}\left(
    \Sigma_{x_1 x_1} +\frac{\beta_t^2}{\alpha_t^2}\xi^2 I\right)^{-1}
    -1
    \right)
    \\
    =&\Sigma_{x_1 x_1}^{-1}. 
\end{align}

Now we have 
\begin{equation}
    g_t^{\text{sim-inv-diffusion}} = \frac{\dot \alpha_t \beta_t - \dot\beta_t\alpha_t}{\beta_t} \left(
    \Sigma_{x_1 x_1}^{-1} + \frac{H^TH}{\sigma_y^2}
    \right)^{-1}
    \left(
    \frac{H^T}{\sigma_y^2}y - \frac{H^TH}{\sigma_y^2} \hat{x}_1
    \right).
\end{equation}

This is essentially the same formulation as in $\Pi$GDM \citep{song_pseudoinverse-guided_2022} and OT-ODE \citep{pokle_training-free_2024}. Next, we will make some trivial conversions to cover the formulations exactly. 


In diffusion paths (Assumption \ref{proposition:jacobian_trick}) we proved that $\frac{\partial\hat{x}_t}{\partial x_t} = \frac{\alpha_t}{\beta_t^2}\Sigma_{1|t}$ where $\Sigma_{1|t}$ is just what we denote $\Sigma_{x_1 x_1}$ here. Equivalently, \begin{equation}
    \frac{\partial x_t}{\partial \hat{x}_1} = \frac{\beta_t^2}{\alpha_t}\Sigma_{x_1x_1}^{-1}.
\end{equation}

Thus,
\begin{align}
    \nonumber g_t^{\text{sim-inv-diffusion}} =& \frac{\dot \alpha_t \beta_t - \dot\beta_t\alpha_t}{\beta_t} \left(
    \Sigma_{x_1 x_1}^{-1} + \frac{H^TH}{\sigma_y^2}
    \right)^{-1}
    \left(
    \frac{H^T}{\sigma_y^2}y - \frac{H^TH}{\sigma_y^2} \hat{x}_1
    \right) \\
    \nonumber =&\frac{\dot \alpha_t \beta_t - \dot\beta_t\alpha_t}{\beta_t}
    \Sigma_{x_1 x_1}
    \left(
    {\sigma_y^2}I + \Sigma_{x_1 x_1} H^TH
    \right)^{-1}
    H^T
    \left(
    y - H \hat{x}_1
    \right) \\
    \nonumber =&\frac{\dot \alpha_t \beta_t - \dot\beta_t\alpha_t}{\beta_t}
    \frac{\beta_t^2}{\alpha_t}\frac{\partial \hat{x}_1}{\partial x_t}
    \left(
        {\sigma_y^2}I + \Sigma_{x_1 x_1}H^TH 
        \right)^{-1}
        \left(
        H^T
        y - H^TH \hat{x}_1
    \right) \\
    \nonumber =&\frac{\beta_t(\dot \alpha_t \beta_t - \dot\beta_t\alpha_t)}{\alpha_t}
    \left(
        \frac{\partial \hat{x}_1}{\partial x_t}
        \left(
        {\sigma_y^2}I + \Sigma_{x_1 x_1}H^TH 
        \right)^{-1}
        \left(
        H^T
        y - H^TH \hat{x}_1
        \right)
    \right)\\
    =&\frac{\beta_t(\dot \alpha_t \beta_t - \dot\beta_t\alpha_t)}{\alpha_t}
    \left(
        \left(
        y - H \hat{x}_1
        \right)^TH
        \left(
        {\sigma_y^2}I + \Sigma_{x_1 x_1} H^T H
        \right)^{-1}
        \frac{\partial \hat{x}_1}{\partial x_t}
    \right)^T.
\end{align}

Now we make the same approximation in $\Pi$GDM that $\Sigma_{x_1x_1} = r_t^2 I$. Then
by noticing that 
\begin{align}
    \nonumber \left(
        {\sigma_y^2}I + r_t^2  H H^T
    \right)H
    =&
    H\left(
        {\sigma_y^2}I + r_t^2  H^T H
    \right)
    \\
    H\left(
        {\sigma_y^2}I + r_t^2  H^T H
    \right)^{-1}
    =&
    \left(
        {\sigma_y^2}I + r_t^2  H H^T
    \right)^{-1}   
    H
\end{align}
We exactly cover 
\begin{equation}
    g_t^{\text{sim-inv-$\Pi$GDM}} = \frac{\beta_t(\dot \alpha_t \beta_t - \dot\beta_t\alpha_t)}{\alpha_t}
    \left(
        \left(
        y - H \hat{x}_1
        \right)^T
        \left(
        {\sigma_y^2}I + r_t^2 H^T H
        \right)^{-1}H
        \frac{\partial \hat{x}_1}{\partial x_t}
    \right)^T,
\end{equation}
and the scheduler $\frac{\beta_t(\dot \alpha_t \beta_t - \dot\beta_t\alpha_t)}{\alpha_t}$ in the path $\alpha_t=t,\beta_t=1-t$ becomes $\frac{1-t}{t}$, which exactly covers the schedule in OT-ODE which takes the same path.
In addition, we can also directly compute $\Sigma_{x_1x_1}$ using $\frac{\partial \hat{x}_1}{\partial x_t}$ instead of approximating it with $r_t$. This corresponds to the approach in \citet{boys_tweedie_2024}, which uses the Jacobian to acquire the covariance and then remove the approximation error in computing $\left(
{\sigma_y^2}I + \Sigma_{x_1 x_1} H^T H
\right)^{-1}$.

\begin{remark}
    Starting from the more general assumption of affine path flow matching, we derived the guidance compatible with dependent coupling flow matching, including OT-CFM. The fact that our guidance can exactly cover classical diffusion guidance like $\Pi$GDM and affine Gaussian path flow matching guidance like OT-ODE verifies the validity of our theory.
\end{remark}
\section{Additional Experimental Results}
\label{appendix:exp}

\subsection{Categorizing Model Responses Across Problem Variations}
\label{appendix:category}
Recall that for each problem, we have a \SAME modification which can be solved using the same method as the original problem, and a \HARD modification which requires more difficult problem-solving skills. Therefore, there are 8 possible cases regarding the correctness of the model's responses to the three problems. Modulo the fluctuations of the model's correctness among the \SAME variations, we can summarize the model's responses into the following 4 cases:
\begin{itemize}[itemsep=1pt, parsep=1pt, topsep=1pt]
    \item \textbf{Case I}: at least one of the original problem and the \SAME modification is solved \textit{correctly}, and the \HARD modification is also solved \textit{correctly}.
    \item \textbf{Case II}: both the original problem and the \SAME modification are solved \textit{incorrectly}, and the \HARD modification is also solved \textit{incorrectly}.
    \item \textbf{Case III}: both the original problem and the \SAME modification are solved \textit{incorrectly}, but the \HARD modification is solved \textit{correctly}.
    \item \textbf{Case IV}: at least one of the original problem and the \SAME modification is solved \textit{correctly}, but the \HARD modification is solved \textit{incorrectly}.
\end{itemize}
For each of the models, we calculate the percentage of the responses in \cref{tab:cate}. As expected, stronger models have a higher percentage of Case I responses and a lower percentage of Case II responses. Interestingly, the percentages of Case III responses are small (less than 10\%) but non-zero, where the models cannot solve the easier variants but can solve the hard variant correctly. After manual inspection, we found that this is due to the misalignment between the models' capabilities and the annotators' perception of the difficulties of math problems. 


\begin{table*}[t]
\caption{Number and percentage of the models' responses that belong to each of the four categories.}
\centering
\resizebox{0.9\textwidth}{!}{
\begin{tabular}{lccccccc}
 \toprule
 Model &  Case I &  Case II & Case III & Case IV &\\ \midrule
Gemini-2.0-flash-thinking-exp &  212 (75.99 \%) & 5 (1.79 \%) & 6 (2.15 \%) & 56 (20.07 \%) &  \\ 
o1-preview &  194 (69.53 \%) & 10 (3.58 \%) & 8 (2.87 \%) & 67 (24.01 \%) &  \\ 
o1-mini &  218 (78.14 \%) & 4 (1.43 \%) & 1 (0.36 \%) & 56 (20.07 \%) &  \\ 
\midrule
Gemini-2.0-flash-exp &  176 (63.08 \%) & 11 (3.94 \%) & 11 (3.94 \%) & 81 (29.03 \%) &  \\ 
Gemini-1.5-pro &  145 (51.97 \%) & 28 (10.04 \%) & 13 (4.66 \%) & 93 (33.33 \%) &  \\ 
GPT-4o &  94 (33.69 \%) & 56 (20.07 \%) & 16 (5.73 \%) & 113 (40.50 \%) &  \\ 
GPT-4-turbo &  81 (29.03 \%) & 72 (25.81 \%) & 15 (5.38 \%) & 111 (39.78 \%) &  \\ 
Claude-3.5-Sonnet &  88 (31.54 \%) & 56 (20.07 \%) & 20 (7.17 \%) & 115 (41.22 \%) &  \\ 
Claude-3-Opus &  49 (17.56 \%) & 99 (35.48 \%) & 25 (8.96 \%) & 106 (37.99 \%) &  \\ 
\midrule
Llama-3.1-8B-Instruct &  21 (7.53 \%) & 137 (49.10 \%) & 7 (2.51 \%) & 114 (40.86 \%) &  \\ 
Gemma-2-9b-it &  22 (7.89 \%) & 164 (58.78 \%) & 11 (3.94 \%) & 82 (29.39 \%) &  \\ 
Phi-3.5-mini-instruct &  22 (7.89 \%) & 161 (57.71 \%) & 18 (6.45 \%) & 78 (27.96 \%) &  \\ 
\midrule
Deepseek-math-7b-rl &  25 (8.96 \%) & 138 (49.46 \%) & 13 (4.66 \%) & 103 (36.92 \%) &  \\ 
Qwen2.5-Math-7B-Instruct &  61 (21.86 \%) & 70 (25.09 \%) & 15 (5.38 \%) & 133 (47.67 \%) &  \\ 
Mathstral-7b-v0.1 &  28 (10.04 \%) & 136 (48.75 \%) & 13 (4.66 \%) & 102 (36.56 \%) &  \\ 
NuminaMath-7B-CoT &  39 (13.98 \%) & 118 (42.29 \%) & 9 (3.23 \%) & 113 (40.50 \%) &  \\ 
MetaMath-13B-V1.0 &  6 (2.15 \%) & 199 (71.33 \%) & 10 (3.58 \%) & 64 (22.94 \%) &  \\ 
MAmmoTH2-8B &  9 (3.23 \%) & 201 (72.04 \%) & 12 (4.30 \%) & 57 (20.43 \%) &  \\
\bottomrule
\end{tabular}
}
\label{tab:cate}
\end{table*}


\subsection{Is Mode Collapse a Problem?}
\label{appendix:naive:memorization}

We provide \cref{tab:naive_memorization} to support \cref{sec:naive:memorization}.

\begin{table*}[htbp]
\caption{The number of errors with answers that match the corresponding original answers. The edit distances are normalized by the length of the responses to the original problems.}
\centering
\resizebox{\textwidth}{!}{
\begin{tabular}{lcccccccccccc}
 \toprule
\multirow{3}{*}{\textbf{Model}}   & \multicolumn{6}{c}{\textbf{\SAME}} & \multicolumn{6}{c}{\textbf{\HARD}}   \\ 
\cmidrule(r){2-7}  \cmidrule(r){8-13}
&  \multicolumn{3}{c}{Num. Errors} & \multicolumn{3}{c}{Normalized Edit Distance} & \multicolumn{3}{c}{Num. Errors} & \multicolumn{3}{c}{Normalized Edit Distance}  \\
\cmidrule(r){2-4} \cmidrule(r){5-7} \cmidrule(r){8-10} \cmidrule(r){11-13}
& $n_{\text{same}}$ & $n_{\text{total}}$ & percentage & min. & avg. & max. & $n_{\text{same}}$ & $n_{\text{total}}$ & percentage & min. & avg. & max. \\
\midrule
 
Gemini-2.0-flash-thinking-exp & 2 & 25 & 8.00 & 0.553 & 0.611 & 0.668  & 10 & 61 & 16.39  &  0.508 & 0.679 & 0.976  \\ 
o1-preview & 1 & 34 & 2.94 & 0.652 & 0.652 & 0.652  & 5 & 77 & 6.49  &  0.729 & 1.07 & 1.89  \\ 
o1-mini & 0 & 14 & 0 & N/A & N/A & N/A  & 9 & 60 & 15.00  &  0.559 & 14.7 & 126.0  \\ 
\midrule
Gemini-2.0-flash-exp & 4 & 48 & 8.33 & 0.644 & 0.82 & 1.09  & 13 & 92 & 14.13  &  0.546 & 1.1 & 1.76  \\ 
Gemini-1.5-pro & 5 & 63 & 7.94 & 0.472 & 0.751 & 1.3  & 11 & 121 & 9.09  &  0.257 & 0.866 & 1.58  \\ 
GPT-4o & 4 & 106 & 3.77 & 0.709 & 0.773 & 0.937  & 14 & 169 & 8.28  &  0.489 & 0.777 & 1.2  \\ 
GPT-4-turbo & 5 & 125 & 4.00 & 0.621 & 0.74 & 0.855  & 17 & 183 & 9.29  &  0.636 & 0.932 & 1.61  \\ 
Claude-3.5-Sonnet & 6 & 116 & 5.17 & 0.509 & 0.729 & 0.83  & 13 & 171 & 7.60  &  0.461 & 0.741 & 1.92  \\ 
Claude-3-Opus & 3 & 162 & 1.85 & 0.355 & 0.485 & 0.614  & 15 & 205 & 7.32  &  0.463 & 0.841 & 1.54  \\ 
\midrule
Llama-3.1-8B-Instruct & 13 & 191 & 6.81 & 0.595 & 0.901 & 1.99  & 18 & 251 & 7.17  &  0.618 & 0.946 & 2.7  \\ 
Gemma-2-9b-it & 3 & 202 & 1.49 & 0.361 & 0.506 & 0.716  & 7 & 246 & 2.85  &  0 & 0.662 & 1.08  \\ 
Phi-3.5-mini-instruct & 8 & 199 & 4.02 & 0.427 & 0.61 & 0.832  & 12 & 239 & 5.02  &  0.289 & 0.754 & 1.69  \\ 
\midrule
Deepseek-math-7b-rl & 9 & 186 & 4.84 & 0.189 & 0.423 & 0.676  & 11 & 241 & 4.56  &  0.121 & 1.5 & 4.24  \\ 
Qwen2.5-Math-7B-Instruct & 6 & 135 & 4.44 & 0.376 & 0.591 & 0.813  & 10 & 203 & 4.93  &  0.273 & 1.01 & 4.91  \\ 
Mathstral-7b-v0.1 & 11 & 178 & 6.18 & 0.0989 & 0.645 & 0.964  & 13 & 238 & 5.46  &  0.105 & 0.586 & 0.984  \\ 
NuminaMath-7B-CoT & 12 & 167 & 7.19 & 0.241 & 0.743 & 1.62  & 14 & 231 & 6.06  &  0.204 & 1.04 & 2.22  \\ 
MetaMath-13B-V1.0 & 13 & 258 & 5.04 & 0.27 & 0.55 & 0.748  & 14 & 263 & 5.32  &  0.509 & 0.982 & 2.83  \\ 
MAmmoTH2-8B & 5 & 229 & 2.18 & 0.00214 & 0.666 & 1.25  & 9 & 258 & 3.49  &  0.708 & 0.822 & 1.04  \\

\bottomrule
\end{tabular}
}
\label{tab:naive_memorization}
\end{table*}
















\clearpage
\subsection{The Effect of In-Context Learning}
\label{appendix:ICL}

In \cref{tab:OIC}, we report the performance of in-context learning (ICL) with the corresponding original (unmodified) problem and solution as the in-context learning example. Furthermore, we decompose the influences on \HARD into the \textbf{ICL effect} and the \textbf{misleading effect} in \cref{tab:icl:breakdown} and visualize the influences for representative models in \cref{fig:icl:full}. Please refer to \cref{sec:icl} for the full discussion.


\begin{table*}[htbp]
\caption{Performance comparisons without and with the original problem and solution as the in-context learning example.}
\centering
\resizebox{\textwidth}{!}{
\begin{tabular}{lccccc}
 \toprule
\multirow{2}{*}{\textbf{Model}} & \multirow{2}{*}{\textbf{\Original} (0-shot)}   & \multicolumn{2}{c}{\textbf{\SAME}} & \multicolumn{2}{c}{\textbf{\HARD}}   \\ 
\cmidrule(r){3-4}  \cmidrule(r){5-6}
& & zero-shot & ICL w. original & zero-shot & ICL w. original\\ 
\midrule
Gemini-2.0-flash-thinking-exp & 92.47 & 91.04 & 94.62 & 78.14 & 79.21 \\ 
o1-preview & 87.81 & 87.81 & 91.40 & 72.40 & 74.19 \\ 
o1-mini & 94.27 & 94.98 & 94.98 & 78.49 & 78.49 \\ 
\midrule
Gemini-2.0-flash-exp & 88.17 & 82.80 & 89.96 & 67.03 & 67.38 \\ 
Gemini-1.5-pro & 77.78 & 77.42 & 88.17 & 56.63 & 60.57 \\ 
GPT-4o & 67.03 & 62.01 & 77.06 & 39.43 & 43.01 \\ 
GPT-4-turbo & 56.99 & 55.20 & 69.89 & 34.41 & 39.07 \\ 
Claude-3.5-Sonnet & 64.52 & 58.42 & 83.15 & 38.71 & 49.46 \\ 
Claude-3-Opus & 41.94 & 41.94 & 68.10 & 26.52 & 33.33 \\ 
\midrule
Llama-3.1-8B-Instruct & 36.56 & 31.54 & 36.56 & 10.04 & 10.75 \\ 
Gemma-2-9b-it & 27.60 & 27.60 & 42.65 & 11.83 & 14.34 \\ 
Phi-3.5-mini-instruct & 26.16 & 28.67 & 36.92 & 14.34 & 14.34 \\ 
\midrule
Deepseek-math-7b-rl & 37.28 & 33.33 & 45.52 & 13.62 & 15.41 \\ 
Qwen2.5-Math-7B-Instruct & 58.78 & 51.61 & 56.99 & 27.24 & 26.88 \\ 
Mathstral-7b-v0.1 & 36.56 & 36.20 & 48.39 & 14.70 & 16.49 \\ 
NuminaMath-7B-CoT & 43.73 & 40.14 & 47.31 & 17.20 & 17.20 \\ 
MetaMath-13B-V1.0 & 21.15 & 7.53 & 11.11 & 5.73 & 3.58 \\ 
MAmmoTH2-8B & 12.90 & 17.92 & 31.18 & 7.53 & 5.73 \\ 
\bottomrule
\end{tabular}
}
\label{tab:OIC}
\end{table*}





\begin{table*}[htbp]
\caption{Effects of in-context learning (ICL) with original example on \HARD. The percentages of $n(\text{correct} \to \text{wrong})$ are normalized by the number of errors with ICL, while the percentages of $n(\text{wrong} \to \text{correct})$ are by the number of errors without ICL. }
\centering
\resizebox{\textwidth}{!}{
\begin{tabular}{lcccc}
 \toprule
Model & num. errors (zero-shot) & num. errors (ICL w. original) & $n(\text{correct} \to \text{wrong})$ & $n(\text{wrong} \to \text{correct})$ \\ 
\midrule
Gemini-2.0-flash-thinking-exp & 61 (21.86 \%) & 58 (20.79 \%) & 17 (29.31 \%) & 20 (32.79 \%) \\ 
o1-preview & 77 (27.60 \%) & 72 (25.81 \%) & 21 (29.17 \%) & 26 (33.77 \%) \\ 
o1-mini & 60 (21.51 \%) & 60 (21.51 \%) & 24 (40.00 \%) & 24 (40.00 \%) \\ 
\midrule
Gemini-2.0-flash-exp & 92 (32.97 \%) & 91 (32.62 \%) & 30 (32.97 \%) & 31 (33.70 \%) \\ 
Gemini-1.5-pro & 121 (43.37 \%) & 110 (39.43 \%) & 27 (24.55 \%) & 38 (31.40 \%) \\ 
GPT-4o & 169 (60.57 \%) & 159 (56.99 \%) & 31 (19.50 \%) & 41 (24.26 \%) \\ 
GPT-4-turbo & 183 (65.59 \%) & 170 (60.93 \%) & 33 (19.41 \%) & 46 (25.14 \%) \\ 
Claude-3.5-Sonnet & 171 (61.29 \%) & 141 (50.54 \%) & 27 (19.15 \%) & 57 (33.33 \%) \\ 
Claude-3-Opus & 205 (73.48 \%) & 186 (66.67 \%) & 35 (18.82 \%) & 54 (26.34 \%) \\ 
\midrule
Llama-3.1-8B-Instruct & 251 (89.96 \%) & 249 (89.25 \%) & 18 (7.23 \%) & 20 (7.97 \%) \\ 
Gemma-2-9b-it & 246 (88.17 \%) & 239 (85.66 \%) & 14 (5.86 \%) & 21 (8.54 \%) \\ 
Phi-3.5-mini-instruct & 239 (85.66 \%) & 239 (85.66 \%) & 17 (7.11 \%) & 17 (7.11 \%) \\ 
\midrule
Deepseek-math-7b-rl & 241 (86.38 \%) & 236 (84.59 \%) & 19 (8.05 \%) & 24 (9.96 \%) \\ 
Qwen2.5-Math-7B-Instruct & 203 (72.76 \%) & 204 (73.12 \%) & 32 (15.69 \%) & 31 (15.27 \%) \\ 
Mathstral-7b-v0.1 & 238 (85.30 \%) & 233 (83.51 \%) & 19 (8.15 \%) & 24 (10.08 \%) \\ 
NuminaMath-7B-CoT & 231 (82.80 \%) & 231 (82.80 \%) & 23 (9.96 \%) & 23 (9.96 \%) \\ 
MetaMath-13B-V1.0 & 263 (94.27 \%) & 269 (96.42 \%) & 11 (4.09 \%) & 5 (1.90 \%) \\ 
MAmmoTH2-8B & 258 (92.47 \%) & 263 (94.27 \%) & 12 (4.56 \%) & 7 (2.71 \%) \\
\bottomrule
\end{tabular}
}
\label{tab:icl:breakdown}
\end{table*}


\begin{figure*}[htbp]
    \centering
    \includegraphics[width=0.33\linewidth]{figures/ICL/icl/Gemini-2.0-flash-thinking.pdf}
    \includegraphics[width=0.32\linewidth]{figures/ICL/icl/o1-mini.pdf}
    \includegraphics[width=0.32\linewidth]{figures/ICL/icl/GPT-4o.pdf}
    \includegraphics[width=0.33\linewidth]{figures/ICL/icl/Claude-3.5-Sonnet.pdf}
    \vspace{5mm}
    \includegraphics[width=0.32\linewidth]{figures/ICL/icl/Llama-3.1-8B-Instruct.pdf}
    \includegraphics[width=0.32\linewidth]{figures/ICL/icl/Gemma-2-9b-it.pdf}
    \includegraphics[width=0.34\linewidth]{figures/ICL/icl/Deepseek-math-7b-rl.pdf}
    \includegraphics[width=0.32\linewidth]{figures/ICL/icl/Qwen2.5-Math-7B-Instruct.pdf}
    
    \caption{The error rates (\%) of the models without and with the original problem and solution as the in-context learning (ICL) example. For \HARD, we decompose the influences of in-context learning into \textbf{ICL effect} (the down arrow $\textcolor{brown}{\boldsymbol{\downarrow}}$), which reduces the error rates, and \textbf{misleading effect} (the up arrow $\textcolor{brown}{\boldsymbol{\uparrow}}$), which increases the error rates.
    }
    \label{fig:icl:full}
\end{figure*}





\clearpage
\subsection{Ablation Study: In-Context Learning with the Original Example v.s. In-Context Learning with a Random Example}

In \cref{tab:oic_sic}, we compare (1) the performance of one-shot in-context learning with the corresponding \textbf{original} unmodified (problem, solution) with (2) the performance of ICL with a \textbf{random} problem and solution chosen from the same category as the query problem. We find that ICL with the \textbf{original} problem and solution consistently outperforms ICL with a \textbf{random} example except for only one case.

\begin{table*}[htbp]
\vspace{-3mm}
\caption{Performance comparisons without and with the original problem and solution as the in-context learning example.}
\centering
\resizebox{0.75\textwidth}{!}{
\begin{tabular}{lcccc}
 \toprule
\multirow{2}{*}{\textbf{Model}} & \multicolumn{2}{c}{\textbf{\SAME}} & \multicolumn{2}{c}{\textbf{\HARD}}   \\ 
\cmidrule(r){2-3}  \cmidrule(r){4-5}
& ICL w. original & ICL (random) & ICL w. original  & ICL (random)\\ 
\midrule
o1-mini & \textbf{94.98} & 92.83 & \textbf{78.49} & 75.99 \\ 
\midrule
Gemini-1.5-pro & \textbf{88.17} & 75.99 & \textbf{60.57} & 51.97 \\ 
GPT-4o & \textbf{77.06} & 63.08 & \textbf{43.01} & 37.28 \\ 
GPT-4-turbo & \textbf{69.89} & 57.71 &\textbf{ 39.07} & 32.62 \\ 
Claude-3.5-Sonnet & \textbf{83.15} & 62.37 & \textbf{49.46} & 40.86 \\ 
Claude-3-Opus & \textbf{68.10} & 45.52 & \textbf{33.33} & 23.66 \\ 
\midrule
Llama-3.1-8B-Instruct & \textbf{36.56} & 28.32 & \textbf{10.75} & 6.45 \\ 
Gemma-2-9b-it & \textbf{42.65} & 27.60 & \textbf{14.34} & 12.90 \\ 
Phi-3.5-mini-instruct & \textbf{36.92} & 20.07 & \textbf{14.34} & 10.39 \\ 
\midrule
Deepseek-math-7b-rl & \textbf{45.52} & 34.41 & \textbf{15.41} & 13.26 \\ 
Qwen2.5-Math-7B-Instruct & \textbf{56.99} & 55.20 & \textbf{26.88} & 26.16 \\ 
Mathstral-7b-v0.1 & \textbf{48.39} & 24.37 & \textbf{16.49} & 8.96 \\ 
NuminaMath-7B-CoT & \textbf{47.31 }& 24.73 & \textbf{17.20} & 10.04 \\ 
MetaMath-13B-V1.0 & \textbf{11.11} & 8.60 & 3.58 & \textbf{5.38} \\ 
MAmmoTH2-8B & \textbf{31.18} & 3.94 &\textbf{ 5.73 }& 2.15 \\ 
\bottomrule
\end{tabular}
}
\label{tab:oic_sic}
\end{table*}


\subsection{Inference-time Scaling Behaviors}
\label{sec:inference:scaling}

In this subsection, we investigate the inference-time scaling behaviors of LLMs on our benchmarks. 
We compute the pass@k metric following~\citet{chen2021codex}. Specifically, for each problem, we generate $N$ solutions independently, and compute the pass@k metric via the following formula for each $1\leq k\leq N$:
\[
    \mathrm{pass@k} = \mathbb{E}_{\mathrm{problem}} \left[ 1-\frac{ { N-c \choose k}}{ {N \choose k} } \right], \text{ where } c \text{ is the number of correct answers of the } n \text{ runs}.
\]
We also compute the performance of self-consistency~\citep{wang2022self}, a.k.a., majority voting, where for each $k$, we randomly sample $k$ responses from the $N$ runs and get the majority-voted answer. We report the average and standard deviation among 5 random draws.
We only evaluate three models: o1-mini, Llama-3.1-8B-Instruct, and Qwen2.5-Math-7B-Instruct. For Llama-3.1-8B-Instruct, and Qwen2.5-Math-7B-Instruct, we choose $N=64$, while for o1-mini we set $N=8$. The results are plotted in \cref{fig:inference:scaling}.







\begin{figure*}[ht]
    \centering
    \includegraphics[width=0.32\linewidth]{figures/inference_scaling_meta-llama-Llama-3.1-8B-Instruct.pdf}
    \includegraphics[width=0.32\linewidth]{figures/inference_scaling_Qwen-Qwen2.5-Math-7B-Instruct.pdf}
    \includegraphics[width=0.32\linewidth]{figures/inference_scaling_o1-mini.pdf}
    \caption{The effect of scaling up inference-time compute. We report pass@k and self-consistency (SC) accuracies for different numbers of solutions $k$.
    }
    \label{fig:inference:scaling}
\end{figure*}























 

\end{document}