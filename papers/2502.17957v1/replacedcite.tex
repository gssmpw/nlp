\section{Related Work}
\paragraph{Generative retrieval modeling.}
Previous work has explored various aspects of generative retrieval. One line of research aims to find appropriate document identifiers for generation, such as numerical or atomic identifier____, N-grams____, titles or URLs____, keywords-based or summary-based semantic identifiers____, codebook____, and full passages themselves____. There are also efforts to combine the advantages of different identifiers____. Another line of work tackles the optimization of generative retrieval models, such as incorporating ranking losses____, or using auxiliary tasks to enhance training____. During retrieval, different constrained decoding methods have been explored to obtain valid identifiers, such as FM-Index____, Trie-based____, and set-based inference____.

\paragraph{Synthetic query generation.}
Alongside the progress in generative retrieval modeling and optimization, synthetic query generation has emerged as a pivotal technique for enhancing retrieval systems, particularly in domains with limited annotated data.
In dense retrieval, synthetic queries have been used extensively to improve cross-domain performance.
For instance, ____ generated synthetic questions for target-domain documents with a question generation model trained on general-domain data, thereby improving retrieval performance in zero-shot settings.
Similarly, ____ introduced generative pseudo labeling, which combines query generation with pseudo labeling using a cross-encoder to capture finer-grained ranking signals.
Further advancements include ____ and ____, which leverage large language models to generate synthetic queries in a few-shot manner, and then combine with top K documents ranked by the conditional question generation probability, to train a domain-specific reranker.

Despite these successes in dense retrieval, the potential of synthetic data for generative retrieval has been underexplored.
Existing studies typically rely on passage-level synthetic queries generated by docT5query ____, following the DSI-QG paradigm ____.
____ explores breaking documents into text fragments for query generation and memorization. However, there still lacks a comprehensive discussions on effective strategies for generating synthetic data tailored to domain-specific corpora, especially with LLMs.
This work investigates data strategies from multiple perspectives, including the generation of synthetic queries using multi-granularity contexts, incorporating search constraints, and exploring the impact of context data. For preference learning, ____ proposes using preference learning objectives for generative retrieval with specialized reward models, though acquiring such models in a domain-specific setting can be challenging.
In contrast, our proposed preference learning strategy directly uses the retrieval results to obtain the preference data, offering a more streamlined approach for domain-specific applications.