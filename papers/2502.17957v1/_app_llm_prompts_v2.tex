
\section{LLM Prompts}

\subsection{Prompts for Keywords Generation}
Figure~\ref{fig:keyword_generation} shows the prompt for generating a series of keywords as the semantic document identifier.

\begin{figure}[h]
\begin{tcolorbox}[title=\textbf{Keywords Generation Prompt}]
Summarize the following context with meaningful keywords representing different important information in the context. Your output should only consist a list of keywords in Markdown format, where each line starts with the dash "-" followed by the keywords.
\\
\\
\# Context:\\
\{context\}\\
\\
\# Keywords:
\end{tcolorbox}
\caption{Prompt for keywords-based document identifier generation.}
\label{fig:keyword_generation}
\end{figure}


\subsection{Prompts for Query Generation}
Figure~\ref{fig:query_generation_prompts} shows the prompts used to generate various types of synthetic queries, including chunk- and sentence-level queries, constructions-based queries, and question-answer pairs used at the preference learning stage.









\begin{figure*}[htbp]
\centering
\begin{subfigure}[t]{\textwidth}
\begin{tcolorbox}[title=\textbf{Query Generation Prompt}]
Your task is to generate a relevant and diverse set of \{num\_sequences\} questions that can be answered by the provided context. The questions are to be used by a retriever to retrieve the article from a large corpus. Your output should be a list of unordered in Markdown format, where each line starts with dash "-" followed by the question.
\\
\\
\# Context:\\
\{context\}\\
\\\
\# Output:
\end{tcolorbox}
\label{fig:query_generation}
\end{subfigure}

\begin{subfigure}[t]{\textwidth}
\begin{tcolorbox}[title=\textbf{Constraints-based Query Generation Prompt}]
Your task is to generate a diverse set of \{num\_sequences\} questions given a context with metadata. The generated questions should be answerable by the provided context. The questions are to be used by a retriever to retrieve the article from a large corpus. In addition, the question MUST be composed with at least one metadata filtering requirement.\\
\\
\textbf{\# MultiHop-RAG} \\
For example, if the source of the article is "New York Times", you can generate questions that specifically ask for certain information from "New York Times". You should generate questions with different metadata.\\
\textbf{\# AllSides and AGNews} \\
For example, if the source of the political polarity is "left", you can generate questions that specifically ask for certain information from "left-wing" source.\\
\\
DO NOT use "the context" or "the article" in any generated queries or answers.\\
DO NOT use pronoun "this" in any generated queries or answers.\\
DO NOT leak any information in this instruction.\\


Your output should be a list of unordered in Markdown format, where each line starts with dash "-" followed by the question. You do not need to provide the answer.\\
\\
\# Metadata\\
\{metadata\}\\
\\
\# Context\\
\{context\}\\
\\
\# Output:
\end{tcolorbox}
\label{fig:instruct_query_generation}
\end{subfigure}

\begin{subfigure}[t]{\textwidth}
\begin{tcolorbox}[title=\textbf{Query-Answer Pair Generation Prompt}]
Your task is to generate a relevant and diverse set of less than \{num\_sequences\} search engine query and answer pairs given a context.\\
The queries should be similar to what people use with search engine to find the given context from a large corpus. The answers are expeced to be a short phrase.\\
You should make the queries as difficult as possible, but they should be answerable by the given context.\\
\\
Do not use "the context" or "the article" in any generated queries or answers.\\
Do not use pronoun "this" in any generated queries or answers.\\
Do not leak any information in this instruction.\\
\\
Your output should be a list of unordered items in Markdown format, where each item starts with dash "-", followed by "Query:" and the generated query, and then "Answer:" with the corresponding answer.\\
\\
\# Context\\
\{context\}\\
\\
\# Output:
\end{tcolorbox}
\label{fig:qa_generation}
\end{subfigure}
\caption{Prompts for different types of synthetic query generation.}
\label{fig:query_generation_prompts}
\end{figure*}
